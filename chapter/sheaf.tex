\chapter{Sheaves and cohomology of sheaves}
\section{Sheaves of sets}
In this section, $X$ always denotes a topological space.
\subsection{Presheaves and sheaves}
\begin{definition}
A \textbf{presheaf} $\mathscr{F}$ on a topological space $X$ is the following data.
\begin{itemize}
\item To each open set $U\sub X$, we have a set $\mathscr{F}(U)$.
\item For each inclusion $U\sub V$ of open sets, we have a \textbf{restriction map} $\res^V_U:\mathscr{F}(V)\to\mathscr{F}(U)$.
\end{itemize}
The data is required to satisfy the following two conditions.
\begin{itemize}
\item The map $\res^U_U$ is the identity: $\res^U_U=\mathbf{1}_{\mathscr{F}(U)}$.
\item If $U\sub V\sub W$ are inclusions of open sets, then the restriction maps commute,
\[\begin{tikzcd}
\mathscr{F}(W)\ar[rr,"\res^W_U"]\ar[rd,swap,"\res^W_V"]&&\mathscr{F}(U)\\
&\mathscr{F}(V)\ar[ru,swap,"\res^V_U"]&
\end{tikzcd}\]
\end{itemize}
\end{definition}
\begin{definition}
A \textbf{morphism of presheaves} $\varphi:\mathscr{F}\to\mathscr{G}$ is a family of maps $\varphi(U):\mathscr{F}(U)\to\mathscr{G}(U)$ for all open subsets $U\sub X$ behaving well with respect to restriction: if $U\hookrightarrow V$ then
\begin{equation}\label{presheaf morphism def}
\begin{tikzcd}
\mathscr{F}(U)\ar[r,"\varphi(U)"]\ar[d,swap,"\res^U_V"]&\mathscr{G}(U)\ar[d,"\res^U_V"]\\
\mathscr{F}(V)\ar[r,"\varphi(V)"]&\mathscr{G}(V)
\end{tikzcd}
\end{equation}
commutes. Composition of morphisms $\varphi$ and $\psi$ of presheaves is defined in the obvious way: $(\varphi\circ\psi)_U:=\varphi_U\circ\psi_U$. We obtain the category $\mathbf{PSh}(X)$ of presheaves on $X$.
\end{definition}
If $U\sub V$ are open subset of $X$ and $s\in\mathscr{F}(V)$, we will usually write $s|_U$ instead of $\res^V_U(s)$. The elements of $\mathscr{F}(U)$ are called \textbf{sections of \boldmath$\mathscr{F}$ over $\bm{U}$}. By convention, if the $U$ is omitted, it is implicitly taken to be $X$: sections of $\mathscr{F}$ means sections of $\mathscr{F}$ over $X$. These are also called \textbf{global sections}. Very often we will also write $\Gamma(U,\mathscr{F})$ instead of $\mathscr{F}(U)$.
\begin{remark}
Given any topological space $X$, we have a category of open sets $\mathcal{T}_X$, where the objects are the open sets and the morphisms are set inclusions. A presheaf is a contravariant funtor from this category to $\mathbf{Set}$:
\[\mathscr{F}:(\mathcal{T}_X)^{op}\to\mathbf{Set}\] 
and a morphism of presheaves is a natural transforml of presheaves. With this observation, for any category $\mathcal{C}$ we can define a \textbf{presheaf $\mathscr{F}$ with values in $\mathcal{C}$} to be a contravariant funtor $(\mathcal{T}_X)^{op}\to\mathcal{C}$. A morphism $\mathscr{F}\to\mathscr{G}$ of presheaves with values in $\mathcal{C}$ is then simply a morphism of functors.
\end{remark}
\begin{example}[\textbf{presheaf of functions}]
Let $E$ be a set. For each open subset $U$ of $X$, let $\mathrm{Map}_E(U)$ be the set of all maps $U\to E$. For an open subset $V\sub U$ we define $\res^U_V:\mathrm{Map}_E(U)\to\mathrm{Map}_E(V)$ as the usual restriction of maps. Then $ \mathrm{Map}_E$ is a presheaf on $X$.\par
More generally, a family $\mathscr{F}$ of subsets $\mathscr{F}(U)\sub\mathrm{Map}_E(U)$, where $U$ runs through the open subsets of $X$, is called a \textbf{presheaf of $\bm{E}$-valued functions on $\bm{X}$}, if it is stable under restriction, i.e., for all open sets $V\sub U$ and all $f\in\mathscr{F}(U)$ one has $f|_V\in\mathscr{F}(V)$. Then $\mathscr{F}$ together with the restriction maps is a presheaf of sets.\par 
If $E$ is a group (respectively an $R$-module for some ring $R$, respectively an $A$-algebra for some commutative ring $A$), then $\mathscr{F}$ is a presheaf of groups (respective of $R$-modules, respective of $A$-algebras).
\end{example}
\begin{example}[\textbf{Examples of presheaf of functions}]
\mbox{}
\begin{itemize}
\item[(a)] Let $Y$ be a topological space. For open subset $U\sub X$, define
\[C(U,Y):=\{f:U\to Y\mid\text{$f$ is continuous}\}.\]
Then $C(-,Y)$ is a presheaf of $Y$-valued functions on $X$. If $Y=\R$, then $C(-,Y)$ a presheaf of $\R$-algebras.
\item[(b)] Let $V$ be a finite-dimensional $R$-vector spaces and let $X$ be an open subspace of $V$. Let $0\leq n\leq\infty$. For an open subset $U$ of $X$, define
\[C^n(U):=\{f:U\to W, \text{$f$ is a $C^n$-map}\}.\]
Then $C^n$ is a presheaf of functions on $X$. It is a presheaf of $\R$-algebras.
\end{itemize}
\end{example}
\begin{definition}
A presheaf is a \textbf{sheaf} if it satisfies two more axioms.
\begin{itemize}
\item \textbf{Identity axiom}. If $\{U_i\}_{i\in I}$ is an open cover of $U$, and $s_1,s_2\in\mathscr{F}(U)$, and $s_1|_{U_i}=s_2|_{U_i}$ for all $i$, then $s_1=s_2$.
\item \textbf{Gluability axiom}. If $\{U_i\}_{i\in I}$ is an open cover of $U$, then given $s_i\in\mathscr{F}(U_i)$ for all $i$, such that 
\[s_i|_{U_i\cap U_j}=s_j|_{U_i\cap U_j}\]
for all $i$ and $j$, then there is some $s\in\mathscr{F}(U)$ such that $s|_{U_i}=s_i$ for all $i$.
\end{itemize}
A \textbf{morphism of sheaves} is a morphism of presheaves, and we obtain the category of sheaves on the topological space $X$, which we denote by $\mathbf{Shv}(X)$.
\end{definition}
\begin{remark}
If $\mathscr{F}$ is a sheaf on $X$, then by using the covering of the empty set $U=\emp$ with empty index set $I=\emp$, we see $\mathscr{F}(\emp)$ is a set consisting of one element. In general, if $U,V\sub X$ are disjoint open subsets, then
\[\mathscr{F}(U\cup V)=\mathscr{F}(U)\times\mathscr{F}(V).\]
\end{remark}
\begin{example}[\textbf{Sheaf of $\bm{E}$-valued functions}]\label{sheaf of E-valued function eg}
Let $E$ be a set and let $\mathscr{F}$ be a presheaf of $E$-valued functions on $X$. Then $\mathscr{F}$ is a sheaf if and only if the following condition holds:
\begin{equation*}
\parbox{\dimexpr\linewidth-6em}
{\strut
For every open subset $U$ of $X$, for every open covering $\{U_i\}_{i\in I}$ of $U$ and for every map $f:U\to E$ such that $f|_{U_i}\in\mathscr{F}(U_i)$ for all $i$, one has $f\in\mathscr{F}(U)$.
\strut}
\end{equation*}
In particular, the presheaves $C(-,Y)$ and $C^n$ are in fact sheaves.
\end{example}
\begin{example}[\textbf{Constant sheaf}]
The presheaf of constant functions with values in some set is in general not a sheaf: if $U_1$ and $U_2$ are disjoint non-empty open subsets and if $f_1:U_1\to E$ and $f_2:U_2\to E$ are constant maps that take different values, then there does not exist a constant map $f$ on $U=U_1\cup U_2$ whose restriction to $U_i$ is $f_i$ for $i=1,2$.\par 
If one takes instead the sheaf of locally constant functions with values in some set $E$, then this is a sheaf. This comes from the simple observation: endow $E$ with the discrete topology, then locally constant maps are exactly continuous maps from $U$ to $E$. This is called the \textbf{constant sheaf associated to $\bm{E}$}. We denote this sheaf $E_X$.
\end{example}
\subsection{Stalks of presheaves and sheaves}
Let $X$ be a topological space, $\mathscr{F}$ be a presheaf on $X$, and let $x\in X$ be a point. The \textbf{stalk} of $\mathscr{F}$ in $x$ is defined by the direct limit
\begin{align*}
\mathscr{F}_x:=\rlim_{U\ni x}\mathscr{F}(U).
\end{align*}
More concretely, one has
\[\mathscr{F}_x=\{(U,s):\text{$U$ is an open neighborhood of $x$ and $s\in\mathscr{F}(U)$}\}/\sim.\]
where two pairs $(U_1,s_1)\sim(U_2,s_2)$ are equivalent if there exists an open neighborhood $V$ of $x$ with $V\sub U_1\cap U_2$ such that $s_1|_V=s_2|_V$. Then for each open neighborhood $U$ of $x$ we have a canonical map
\[\mathscr{F}(U)\mapsto\mathscr{F}_x,\quad s\mapsto s_x,\]
which sends $s\in\mathscr{F}(U)$ to the class of $(U,s)$ in $\mathscr{F}_x$. We call $s_x$ the \textbf{germ} of $s$ in $x$.
\begin{remark}
If $\mathscr{F}$ is a presheaf on $X$ with values in $\mathcal{C}$, where $\mathcal{C}$ is any category in which filtered colimits exist, then the stalk $\mathscr{F}_x$ is an object in $\mathcal{C}$ and we obtain a functor $\mathscr{F}\to\mathscr{F}_x$ from the category of presheaves on $X$ with values in $\mathcal{C}$ to the category $\mathcal{C}$.
\end{remark}
\begin{example}[\textbf{Stalk of the sheaf of continuous functions}]
Let $X$ be a topological space, let $C_X$ be the sheaf of continuous $\R$-valued functions on $X$, and let $x\in X$. Then
\[C_{X,x}=\{(U,f):\text{$x\in U\sub X$ open, $f:U\to\R$ is continuous}\}/\sim,\]
where $(U,f)\sim(V,g)$ if there exists $x\in W\sub U\cap V$ open such that $f|_W=g|_W$. In particular we see that for $x\in V\sub U$ open and $f:U\to\R$ continuous one has $(U,f)\sim(V,f|_V)$. As $C_X$ is a sheaf of $\R$-algebras, $C_{X,x}$ is an $\R$-algebra.\par
If the germ $s\in C_{X,x}$ of a continuous function at $x$ is represented by $(U,f)$, then $s(x):=f(x)\in\R$ is independent of the choice of the representative $(U,f)$. We obtain an $\R$-algebra homomorphism 
\[\mathrm{ev}_x:C_{X,x}\to\R,\quad s\mapsto s(x),\]
which is surjective because $C_{X,x}$ contains in particular the germs of all constant functions. Let $\m_x=\ker\mathrm{ev}_x=\{s\in C_{X,x}:s(x)=0\}$. Then $\m_x$ is a maximal ideal because $C_{X,x}/\m_x\cong\R$ is a field. Let $s\in C_{X,x}\setminus\m_x$ be represented by $(U,f)$. Then $f(x)\neq 0$. By shrinking $U$ we may assume that $f(y)\neq0$ for all $y\in Y$ because $f$ is continuous. Hence $1/f$ exists and $s$ is a unit in $C_{X,x}$. Therefore the complement of $\m_x$ consists of units of $C_{X,x}$. This shows that $C_{X,x}$ is a local ring with maximal ideal $\m_x$. The same argument shows that for every open subspace $X$ of a finite-dimensional $\R$-vector space the stalk $C_{X,x}^n$ is a local ring with residue field $\R$.
\end{example}
Now let $\varphi:\mathscr{F}\to\mathscr{G}$ be a morphism of presheaves on $X$, we then have an induced map $\varphi_x:\mathscr{F}_x\to\mathscr{G}_x$ defined by
\[\varphi_x:=\rlim_{U\ni x}\varphi_U,\]
which sends the equivalence class of $(U,f)$ in $\mathscr{F}_x$ to the class of $(U,\varphi_U(f))$ in $\mathscr{G}_x$. This defines a functor $\mathscr{F}\mapsto\mathscr{F}_x$ from the category of presheaves on $X$ to the category of sets such that for every open neighborhood $U$ of $X$ one has a commutative diagram
\begin{equation}\label{presheaf morphism induced on stalk}
\begin{tikzcd}
\mathscr{F}(U)\ar[r,"f\mapsto f_x"]\ar[d,swap,"\varphi_U"]&\mathscr{F}_x\ar[d,"\varphi_x"]\\
\mathscr{G}(U)\ar[r,"g\mapsto g_x"]&\mathscr{G}_x
\end{tikzcd}
\end{equation}
Now we introduct a construction frequently apprears in the theory of sheaves. The \textbf{Godement construction} 
\[\mathrm{God}:\mathbf{Psh}(X)\to\mathbf{Shv}(X)\]
is the functor given by sending a presheaf $\mathscr{F}\in\mathbf{PSh}(X)$ to the sheaf $\mathrm{God}(\mathscr{F})$ defined by sending an open set $U$ to
\[\mathrm{God}(\mathscr{F})(U):=\prod_{x\in U}\mathscr{F}_x,\]
with the restriction morphisms given by product projections (it is easy to see this defines a sheaf on $X$). The assignment of $\mathrm{God}$ on morphisms is given by sending a morphism of presheaves $\phi:\mathscr{F}\to\mathscr{G}$ to the morphism
\[\mathrm{God}(\phi)_U:\prod_{x\in U}\mathscr{F}_x\to\prod_{x\in U}\mathscr{G}_x.\]
\begin{proposition}\label{Goement construction image of sheaf char}
Let $\mathscr{F}$ be a sheaf of sets. For every open $U\sub X$ the map
\[\mathscr{F}(U)\to\prod_{x\in U}\mathscr{F}_x,\quad s\mapsto (s_x)_{x\in U}.\]
identifies $\mathscr{F}(U)$ with the elements $(s_x)_{x\in U}$ with the property
\begin{equation*}
\parbox{\dimexpr\linewidth-6em}
{\strut
For any $x\in U$ there exists a open subset $V$ containing $x$ and a section $s\in\mathscr{F}(V)$ such that for all $y\in V$ we have $s_y=(V,s)$ in $\mathscr{F}_y$.
\strut}
\end{equation*}
If an element $(s_x)$ satisfies the condition above, we say it \textbf{consists of compatible germs}. Thus the Godement construction identifies $\mathscr{F}$ as a subsheaf of $\mathrm{God}(\mathscr{F})$.
\end{proposition}
\begin{proof}
Let $s,t\in\mathscr{F}(U)$ such that $s_x=t_x$ for all $x\in U$. Then for all $x\in U$ there exists an open neighborhood $V_x\sub U$ of $x$ such that $s|_{V_x}=t|_{V_x}$. Clearly, $U=\bigcup_{x\in U}V_x$ and therefore $s=t$ by identity axiom.\par
Clearly any section $s$ of $\mathscr{F}$ over $U$ gives a choice of compatible germs for $U$. Conversely, if $(s_x)_{x\in U}$ consists of compatible germs, that is, there is an open cover $\{U_i\}$ of $U$, and sections $s_i\in\mathscr{F}(U_i)$, such that $(s_x)_{x\in U_i}$ is given by $(U_i,s_i)$. Then by gluability there is a section $\sigma\in\mathscr{F}(U)$ such that $\sigma|_{U_i}=s_i$, and it is clear from the condition that $\sigma_x=s_x$.
\end{proof}
The importance of stalks is contained in the following result, which says a morphism between sheaves is determined by its value on stalks.
\begin{proposition}\label{sheaf morphism prop iff stalk prop}
Let $X$ be a topological space, $\mathscr{F}$ and $\mathscr{G}$ presheaves on $X$, and let $\varphi,\psi:\mathscr{F}\to\mathscr{G}$ be two morphisms of presheaves.
\begin{itemize}
\item[(a)] If $\mathscr{F}$ is a sheaf, the induced maps on stalks $\varphi_x:\mathscr{F}_x\to\mathscr{G}_x$ are injective for all $x\in X$ if and only of $\varphi_U:\mathscr{F}(U)\to\mathscr{G}(U)$ is injective for all open subsets $U\sub X$.
\item[(b)] If $\mathscr{F}$ and $\mathscr{G}$ are both sheaves, the maps $\varphi_x:\mathscr{F}_x\to\mathscr{G}_x$ are bijective for all $x\in X$ if and only of $\varphi_U:\mathscr{F}(U)\to\mathscr{G}(U)$ is injective for all open subsets $U\sub X$.
\item[(c)] If $\mathscr{F}$ and $\mathscr{G}$ are both sheaves, the morphism $\varphi$ and $\psi$ are equal if and only of $\varphi_x=\psi_x$ for all $x\in X$.
\end{itemize}
\end{proposition}
\begin{proof}
First we show that taking stalks preserves injectivity and surjectivity. Assume that $\varphi_U$ is injective for all $U$. Let $s_0,t_0\in\mathscr{F}_x$ such that $\varphi_x(s_0)=\varphi_x(t_0)$. Let $s_0$ be represented by $(s,U)$ and $t_0$ by $(t,V)$. By shrinking $U$ and $V$ we may assume $U=V$. From diagram (\ref{presheaf morphism induced on stalk}), we see
\[\varphi_U(s)_x=\varphi_x(s_0)=\varphi_x(t_0)=\varphi_U(t)_x,\]
so there exists an open neighborhood $W\sub U$ containing such that
\[\varphi_W(s|_W)=(\varphi_U(s))|_W=(\varphi_U(t))|_W=\varphi_W(t|_W).\]
As $\varphi_W$ is injective, we find $s|_W=t|_W$ and hence $s_0=s_x=t_x=t_0$. Thus $\varphi_x$ is injective. If on the other hand $\varphi_U$ is surjective for all $U\sub X$, let $t_0$ be any elememt in $\mathscr{G}_x$, which is represented by $(t,U)$. Then there is a $s\in\mathscr{F}(U)$ such that $\varphi_U(s)=t$, and $(\ref{presheaf morphism induced on stalk})$ implies
\[\varphi_x(s_x)=\big(\varphi_U(s)\big)_x=t_x.\]
so $\varphi_x$ is surjective.\par
For (a), assume that $\mathscr{F}$ is a sheaf, and consider the commutative diagram
\[\begin{tikzcd}
\mathscr{F}(U)\ar[r]\ar[d,"\varphi_U"]&\prod_{x\in U}\mathscr{F}_x\ar[d,"\prod_{x\in U}\varphi_x"]\\
\mathscr{G}(U)\ar[r]&\prod_{x\in U}\mathscr{G}_x
\end{tikzcd}\]
By Proposition~\ref{Goement construction image of sheaf char} the map $\mathscr{F}(U)\to\prod_{x\in U}\mathscr{F}_x$ is injective, thus if $\varphi_x$ are all injective, so is $\varphi_U$. Assertion (c) can also be derived from this diagram.\par
Then we prove (b). Assume $\varphi_x$ are bijective, we show that $\varphi_U$ is surjective. Let $t\in\mathscr{G}(U)$. For any $x\in U$, since $\varphi_x$ is surjective, there exist $s_x\in\mathscr{F}_x$ such that $\varphi_x(s_x)=t_x$. Let $s_x$ be represented by a section $s(x)$ on a neighborhood $V_x$ of $x$. Then by $(\ref{presheaf morphism induced on stalk})$,
\[\big(\varphi_{V_x}(s(x))\big)_x=\varphi_x(s_x)=(t|_{V_x})_x.\]
By shrinking $V_x$ we may assume that $\varphi_{V_x}(s(x))=t|_{V_x}$.\par 
Now $U$ is covered by such open sets $V_x$, and on each $V_x$ we have a section $s(x)\in\mathscr{F}(V_x)$. For two distinct points $x,y\in U$, $s(x)|_{V_x\cap V_y}$ and $s(y)|_{V_x\cap V_y}$ are both sent to $t|_{V_x\cap V_y}$ by $\varphi$, so by the injectivity of $\varphi$ we just proved, $s(x)|_{V_x\cap V_y}=s(y)|_{V_x\cap V_y}$. Therefore, the gulability axiom produces a section $s\in\mathscr{F}(U)$ such that $s|_{V_x}=s(x)$, and from diagram $(\ref{presheaf morphism def})$ and the construction, we have
\[\big(\varphi_{U}(s)\big)|_{V_x}=\varphi_{V_x}(s|_{V_x})=\varphi_{V_x}(s(x))=t|_{V_x},\]
so the identity axiom implies $\varphi_U(s)=t$.
\end{proof}
\begin{definition}
Let $\mathscr{F}$ be a sheaf of abelian groups on a topological space $X$, $U\sub X$ open and $s\in\mathscr{F}(U)$ a section. The \textbf{support} of $s$ is defined by
\[\supp(s)=\{x\in U:s_x\neq 0\}.\]
The \textbf{support} of $\mathscr{F}$ is defined to be
\[\supp(\mathscr{F})=\{x\in X:\mathscr{F}_x\neq 0\}.\]
\end{definition}
\begin{proposition}\label{sheaf section supp is closed}
Let $\mathscr{F}$ be a sheaf of abelian groups on a topological space, $U\sub X$ open, and $s\in\mathscr{F}(U)$ a section. Then $\supp(s)$ is closed in $U$.
\end{proposition}
\begin{proof}
For $x\in U\setminus\supp(s)$ we have $s_x=0$, so there exists an open subset $V\sub U$ with $s|_V=0$. This implies $s_{y}=0$ for every $y\in V$ and therefore $V\sub U\setminus\supp(s)$. Hence $U\setminus\supp(s)$ is open.
\end{proof}
\begin{example}
Let $X$ be a topological space. Let $C_X$ be the sheaf of continuous $\R$-valued functions on $X$. Let $U\sub X$ be open and $s\in C_X(U)$ a continuous function $U\to\R$. In the proof of Proposition~\ref{sheaf section supp is closed} we have just seen that $U\setminus\supp(s)$ is the interior of $\{x\in U:s(x)=0\}$. Therefore we have
\[\supp(s)=\widebar{\{x\in U:s(x)\neq 0\}}\]
which coincides with usual definition of the support of a continuous function.
\end{example}
\subsection{Sheafification}
In this part, we give a functorial way to attach to a presheaf a sheaf. This can be seen as the left adjoint of the forgetful functor from $\mathbf{Shv}(X)$ to $\mathbf{Psh}(X)$.
\begin{definition}[\textbf{Universal property of sheafification}]
If $\mathscr{F}$ is a presheaf on $X$, then a morphism of presheaves $\iota_{\mathscr{F}}:\mathscr{F}\to\mathscr{F}^{\hash}$ on $X$ is called a \textbf{sheafification} of $\mathscr{F}$ if $\mathscr{F}^{\hash}$ is a sheaf, and for any sheaf $\mathscr{G}$, and any presheaf morphism $\varphi:\mathscr{F}\to\mathscr{G}$, there exists a unique morphism of sheaves $\tilde{\varphi}:\mathscr{F}^{\hash}\to\mathscr{G}$ making the diagram
\[\begin{tikzcd}
\mathscr{F}\ar[rd,swap,"\varphi"]\ar[r,"\iota_{\mathscr{F}}"]&\mathscr{F}^{\hash}\ar[d,"\tilde{\varphi}"]\\
&\mathscr{G}
\end{tikzcd}\]
commute.\par
As a universal construction, the sheafification functor $\mathscr{F}\to\mathscr{F}^{\hash}$ is left-adjoint to the forgetful functor from sheaves on $X$ to presheaves on $X$: there is a canonical isomorphism
\[\Mor_{\mathbf{Shv}(X)}(\mathscr{F}^{\hash},\mathscr{G})\cong\Mor_{\mathbf{Psh}(X)}(\mathscr{F},\mathscr{G})\]
for any presheaf $\mathscr{F}$ and sheaf $\mathscr{G}$.
\end{definition}
In Proposition~\ref{Goement construction image of sheaf char} we see that if $\mathscr{F}$ is a sheaf, then $\mathscr{F}(U)$ can be identified as elements in $\prod_{x\in U}\mathscr{F}_x$ consists of compatible germs. This turns out to be a appropriate way to define the sheafification.
\begin{proposition}\label{sheaf sheafification construction}
Let $\mathscr{F}$ be a presheaf on a topological space $X$. Then there exists a sheafification $\iota_{\mathscr{F}}:\mathscr{F}\to\mathscr{F}^{\hash}$ such that the following properties hold:
\begin{itemize}
\item[(a)] For all $x\in X$ the map on stalks $\iota_{\mathscr{F},x}:\mathscr{F}_x\to\mathscr{F}^{\hash}_x$ is bijective.
\item[(b)] For every presheaf $\mathscr{G}$ on $X$ and every morphism of presheaves $\varphi:\mathscr{F}\to\mathscr{G}$ there exists
a unique morphism $\tilde{\varphi}:\mathscr{F}^{\hash}\to\mathscr{G}^{\hash}$ making the diagram commutes:
\begin{equation}\label{sheafification-1}
\begin{tikzcd}
\mathscr{F}\ar[r,"\iota_{\mathscr{F}}"]\ar[d,swap,"\varphi"]&\mathscr{F}^{\hash}\ar[d,"\tilde{\varphi}"]\\
\mathscr{G}\ar[r,"\iota_{\mathscr{G}}"]&\mathscr{G}^{\hash}
\end{tikzcd}
\end{equation}
In particular, $\mathscr{F}\to\mathscr{F}^{\hash}$ is a functor from the category of presheaves on $X$ to the category of sheaves on $X$. The sheafification $\iota_{\mathscr{F}}:\mathscr{F}\to\mathscr{F}^{\hash}$ is unique up to isomorphism. 
\end{itemize}
\end{proposition}
\begin{proof}
Suppose $\mathscr{F}$ is a presheaf. Define $\mathscr{F}^{\hash}$ by declearing $\mathscr{F}^{\hash}(U)$ as the set of compatible germs of the presheaf $\mathscr{F}$ over $U$. Explicitly:
\begin{align*}
\mathscr{F}^{\hash}(U)&=\{(s_x)_{x\in U}\in\prod_{x\in U}\mathscr{F}_x:\text{$(s_x)_{x\in U}$ consists of compatible germs}\}\\
&=\{(s_x)_{x\in U}\in\prod_{x\in U}\mathscr{F}_x:\text{$\forall x\in U$, $\exists V\sub U$ and $\tilde{s}\in\mathscr{F}(V)$ s.t. $\tilde{s}_{y}=s_x$ for all $y\in V$}\}.
\end{align*}
For $U\sub V$ the restriction map $\mathscr{F}^{\hash}(V)\to\mathscr{F}^{\hash}(U)$ is induced by the natural projection. With this, for any covering $U=\bigcup_iU_i$ and sections $(s_x)_{x\in U_i}$ in $\mathscr{F}^{\hash}$, the condition
\[\big((s_x)_{x\in U_i}\big)|_{U_i\cap U_j}=\big((s_x)_{x\in U_j}\big)|_{U_i\cap U_j}\]
implies that $(s_x)_{x\in U_i}$ and $(s_x)_{x\in U_j}$ have common germs on their common domains. Thus we can construct a unique section $(s_x)_{x\in U}$ by just gathering their germs. It is clear that such a section $(s_x)_{x\in U}$ has compatible germs, hence belongs to $\mathscr{F}^{\hash}$. This shows $\mathscr{F}^{\hash}$ is a sheaf. For $U\sub X$ open, we define $\iota_{\mathscr{F},U}:\mathscr{F}(U)\to\mathscr{F}^{\hash}(U)$ by $s\mapsto(s_x)_{x\in U}$. The definition of $\mathscr{F}^{\hash}$ shows that, for $x\in X$, $\mathscr{F}^{\hash}_x=\mathscr{F}_x$ and that $\iota_{\mathscr{F},x}$ is the identity.\par
Now let $\mathscr{G}$ be a presheaf on $X$ and let $\varphi:\mathscr{F}\to\mathscr{G}$ be a morphism. Sending $(s_x)_{x\in U}$ to $\big(\varphi_x(s_x)\big)_{x\in U}$ defines a morphism $\mathscr{F}^{\hash}\to\mathscr{G}^{\hash}$. By Proposition~\ref{sheaf morphism prop iff stalk prop}(c) this is the unique morphism making the diagram $(\ref{sheafification-1})$ commutative.\par
If we assume in addition that $\mathscr{G}$ is a sheaf, then the morphism of sheaves $\iota_{\mathscr{G}}$, which is bijective on stalks, is an isomorphism by Proposition~\ref{sheaf morphism prop iff stalk prop}(b). Composing the morphism $\mathscr{F}^{\hash}\to\mathscr{G}^{\hash}$ with $\iota_{\mathscr{G}}^{-1}$, we obtain the morphism $\tilde{\varphi}:\mathscr{F}^{\hash}\to\mathscr{G}$. Finally, the uniqueness of $\iota_{\mathscr{F}}:\mathscr{F}\to\mathscr{F}^{\hash}$ is a formal consequence of universal property.
\end{proof}
\begin{proposition}\label{sheaf sheafification on morphism prop}
The sheafification of an injection (resp. surjection) of presheaves of sets is an injection (resp. surjection).
\end{proposition}
\begin{proof}
This follows from the fact that sheafification does not change the stalk.
\end{proof}
From Proposition~\ref{sheaf morphism prop iff stalk prop}, we get the following characterization of the sheafification.
\begin{proposition}\label{sheaf iso to sheafification iff stalk iso}
Let $\mathscr{F}$ be a presheaf and $\mathscr{G}$ be a sheaf. Then $\mathscr{G}$ is isomorphic to the sheafification of $\mathscr{F}$ if and only and if there exists a morphism $\iota:\mathscr{F}\to\mathscr{G}$ such that $\iota_x$ is bijective for all $x\in X$.
\end{proposition}
\begin{proof}
One direction is trivial, assume the converse. Then there is a $\tilde{\iota}:\mathscr{F}^{\hash}\to\mathscr{G}$ such that the following diagram commutes
\[\begin{tikzcd}
\mathscr{F}^{\hash}\ar[r,"\tilde{\iota}"]&\mathscr{G}\\
\mathscr{F}\ar[ru,swap,"\iota"]\ar[u]
\end{tikzcd}\]
By Proposition~\ref{sheaf morphism prop iff stalk prop} $\tilde{\iota}$ induced isomorphisms on stalks, hence is an isomorphism $\mathscr{F}^{\hash}\cong\mathscr{G}$.
\end{proof}
\begin{example}\label{sheafification presheaf of function}
Let $E$ be a set and let $\mathscr{F}$ be a presheaf of functions with values in $E$. Then its sheafification is given by
\[\mathscr{F}^{\hash}(U)=\{f:U\to E\mid\exists\text{ open covering $(U_i)_i$ of $U$ such that $f|_{U_i}\in\mathscr{F}(U_i)$ for all $i$}\}.\]
Indeed, this is a sheaf by Example~\ref{sheaf of E-valued function eg} and the inclusions $\mathscr{F}(U)\hookrightarrow\mathscr{F}^{\hash}(U)$ for $U$ open define a morphism of presheaves $\mathscr{F}\to\mathscr{F}^{\hash}$ that is bijective on stalks. Hence we can apply Proposition~\ref{sheaf iso to sheafification iff stalk iso}.
\end{example}
\begin{example}
From the previous example, let's consider the constant presheaves. Let $E_{X}$ be the sheaf of locally constant functions with values in $E$:
\[E_X=\{f:U\to E\mid \forall x\in U,\exists V\ni x\text{ open s.t. $f$ is constant on $V$}\}\]
then by Proposition~\ref{sheaf iso to sheafification iff stalk iso}, $E_X$ is the sheafification of the presheaf of constant functions with values in $E$.
\end{example}
\begin{example}[\textbf{Open subset as a sheaf}]\label{sheaf by open subset}
Let $U\sub X$ be an open subset, then we can define a presheaf $\mathscr{U}$
\[\mathscr{U}(V)=\begin{cases}
\{i_V\}&\text{if }V\sub U\\
\emp&\text{otherwise}
\end{cases}\]
where $i_V:V\hookrightarrow U$ is the inclusion map. Let $\mathscr{U}^{\hash}$ be its sheafification.\par
Let $\mathscr{F}$ be a sheaf and $s\in\mathscr{F}(U)$, then we can define a morphism $\varphi_s$ by setting $\varphi_{s,U}(\mathbf{1}_U)=s$ and others by restriction. Then we get a map
\[\mathscr{F}(U)\to\Mor_{\mathbf{Psh}(X)}(\mathscr{U},\mathscr{F})\]
which can be shown is an isomorphism. Together with the adjointness of sheafification, we get isomorphisms
\[\mathscr{F}(U)\cong\Mor_{\mathbf{Psh}(X)}(\mathscr{U},\mathscr{F})\cong\Mor_{\mathbf{Shv}(X)}(\mathscr{U}^{\hash},\mathscr{F}).\]
\end{example}
\subsection{Direct and inverse images of sheaves}
In this part $f:X\to Y$ denotes a continuous map of topological spaces. We will now see how to use $f$ in order to attach to a sheaf on $X$ a sheaf on $Y$ (direct image) and to a sheaf on $Y$ a sheaf an $X$ (inverse image).
\begin{definition}[\textbf{Direct image of a presheaf}]
Let $f:X\to Y$ be a continuous map. Let $\mathscr{F}$ be a presheaf on $X$. We define a presheaf $f_*\mathscr{F}$ on $Y$ by (for $V\sub Y$ open)
\[(f_*\mathscr{F})(V)=\mathscr{F}\big(f^{-1}(V)\big)\]
the restriction maps given by the restriction maps for $\mathscr{F}$. Then $f_*\mathscr{F}$ is called the \textbf{direct image} of $\mathscr{F}$ under $f$ or the \textbf{pushforward presheaf} of $\mathscr{F}$ by $f$. Whenever $\varphi:\mathscr{F}_1\to\mathscr{F}_2$ is a morphism of presheaves, the family of maps $f_*(\varphi)_V:=\varphi_{f^{-1}(V)}$ for $V\sub Y$ open is a morphism $f_*(\varphi):f_*\mathscr{F}_1\to f_*\mathscr{F}_2$. We thus obtain a functor $f_*$ from the category of presheaves on $X$ to the category of presheaves on $Y$.
\end{definition}
\begin{proposition}\label{sheaf direct image prop}
Let $f:X\to Y$ be a continuous map
\begin{itemize}
\item[(a)] If $\mathscr{F}$ is a sheaf on $X$, then $f_*\mathscr{F}$ is a sheaf on $Y$. Therefore $f_*$ also defines a functor $f_*:\mathbf{Shv}(Y)\to\mathbf{Shv}(X)$.
\item[(b)] If $g:Y\to Z$ is a second continuous map, then $(g\circ f)_*=g_*\circ f_*$.
\end{itemize}
\end{proposition}
\begin{proof}
This first statement immediately follows from the fact that if $V=\bigcup V_i$ is an open covering in $Y$, then $f^{-1}(V)=\bigcup_if^{-1}(V_i)$ is an open covering in $X$. The second claim is a computation:
\[(g\circ f)_*\mathscr{F}(W)=\mathscr{F}\big((g\circ f)^{-1}(W)\big)=\mathscr{F}\big(f^{-1}\circ g^{-1}(W)\big)=g_*\mathscr{F}\big(f^{-1}(W)\big)=g_*\circ f_*\mathscr{F}(W).\]
where $W\sub Y$ is open.
\end{proof}
We now define the inverse image of a sheaf. Let $\mathscr{G}$ be a presheaf of sets on $Y$. The \textbf{pullback presheaf} $f^{p}\mathscr{G}$ of a given presheaf $\mathscr{G}$ is defined as the left adjoint of the pushforward $f_*$ on presheaves. In other words, it should be a presheaf $f^{p}\mathscr{G}$ on $X$ such that
\[\Mor_{\mathbf{Psh}(X)}(f^{p}\mathscr{G},\mathscr{F})\cong\Mor_{\mathbf{Psh}(X)}(\mathscr{G},f_*\mathscr{F})\]
It turns out that this actually exists.
\begin{proposition}[\textbf{Inverse image of a presheaf}]\label{presheaf inverse image exist}
Let $f:X\to Y$ be a continuous map and let $\mathscr{G}$ be a presheaf on $Y$. There exists a functor $f^{p}:\mathbf{Psh}(Y)\to\mathbf{Psh}(X)$ which is left adjoint to $f_*$. For a presheaf $\mathscr{G}$ it is determined by
\[f^p\mathscr{G}(U)=\rlim_{V\sups f(U)\atop\text{$V\sub Y$ open}}\mathscr{G}(V).\]
the restriction maps being induced by the restriction maps of $\mathscr{G}$. 
\end{proposition}
\begin{proof}
The colimit is over the partially ordered set consisting of open subset $V\sub Y$ which contain $f(U)$ with ordering by reverse inclusion. This is a directed partially ordered set, and if $U_1\sub U_2$, then every open neighbourhood of $f(U_2)$ is an open neighbourhood of $f(U_1)$. Hence the system defining $f^{p}\mathscr{G}(U_2)$ is a subsystem of the one defining $f^{p}\mathscr{G}(U_1)$ and we obtain a restriction map. Note that the construction of the colimit is clearly functorial in $\mathscr{G}$, and similarly for the restriction mappings. Hence we have defined $f^{p}$ as a functor. Now we turn to the proof of the adjointness. For this, we need to define the unit map and the counit map, as follows.
\begin{itemize}
\item There exists a canonical map $\mathscr{G}(V)\to f^{p}\mathscr{G}\big(f^{-1}(V)\big)$ for any open subset $V\sub Y$, because the system of open neighbourhoods of $f\big(f^{-1}(V)\big)$ contains the element $V$:
\[\begin{tikzcd}
\rho_{\mathscr{G},V}:\mathscr{G}(V)\ar[r]&f^p\mathscr{G}(f^{-1}(V))=\rlim_{U\sups f(f^{-1}(V))}\mathscr{G}(U)
\end{tikzcd}\] 
This is compatible with restriction mappings, so there is a canonical map $\rho_{\mathscr{G}}:\mathscr{G}\to f_*f^{p}\mathscr{G}$.
\item There exists a canonical map $f^{p}f_*\mathscr{F}(U)\to\mathscr{F}(U)$ for any open subset $U\sub X$:
\[\begin{tikzcd}
\sigma_{\mathscr{F},U}:f^pf_*\mathscr{F}(U)=\rlim_{V\sups f(U)}\mathscr{F}(f^{-1}(V))\ar[r]&\mathscr{F}(U)
\end{tikzcd}\]
where the map is given by the restriction of $\mathscr{F}(f^{-1}(V))$ to $\mathscr{F}(U)$. One easily verifies that the maps are compatible with restriction maps and thus there is a canonical map $\sigma_{\mathscr{F}}:f^{p}f_*\mathscr{F}\to\mathscr{F}$.
\end{itemize}
The maps we get are illustrated in the following diagram:
\[\begin{tikzcd}
X\ar[d,"f"]&&f^{p}\mathscr{G}\ar[r,swap,"\psi"]\ar[ld,swap,dashed]\ar[rr,bend left=20pt,"f^{p}\varphi"]&\mathscr{F}\ar[d,dashed]&f^{p}f_*\mathscr{F}\ar[l,"\sigma_{\mathscr{F}}"]\\
Y&f_*f^{p}\mathscr{G}\ar[rr,bend right=20pt,swap,"f_*\psi"]&\mathscr{G}\ar[l,swap,"\rho_{\mathscr{G}}"]\ar[u,dashed]\ar[r,"\varphi"]&f_*\mathscr{F}\ar[ru,dashed]
\end{tikzcd}\]
Now let $\mathscr{F}$ be a presheaf of sets on $X$. Suppose that $\psi:f^{p}\mathscr{G}\to\mathscr{F}$ is a map of presheaves of sets. The corresponding map $\psi^{\flat}:\mathscr{G}\to f_*\mathscr{F}$ is the composition
\[\begin{tikzcd}
\psi^\flat:\mathscr{G}\ar[r,"\rho_{\mathscr{G}}"]&f_*f^{p}\mathscr{G}\ar[r,"f_*\psi"]&f_*\mathscr{F}
\end{tikzcd}\]
Suppose that $\varphi:\mathscr{G}\to f_*\mathscr{F}$ is a map of presheaves of sets. The map $\varphi^{\sharp}:f^{p}\mathscr{G}\to\mathscr{F}$ is then the composition
\[\begin{tikzcd}
\varphi^\sharp:f^{p}\mathscr{G}\ar[r,"f^{p}\varphi"]&f^{p}f_*\mathscr{F}\ar[r,"\sigma_{\mathscr{F}}"]&\mathscr{F}
\end{tikzcd}\]
It can be verified that these tow maps are inverse of each other. Let $U\sub X$, then the map $(\psi^\flat)^{\sharp}:f^p\mathscr{G}\to\mathscr{F}$ is given by
\[\begin{tikzcd}[column sep=10pt]
f^p\mathscr{G}(U)=\rlim_{V\sups f(U)}\mathscr{G}(V)\ar[r]&\rlim_{V\sups f(U)}\rlim_{V'\sups f(f^{-1}(V))}\mathscr{G}(V')\ar[r,"\psi_{f^{-1}(V)}"]&\rlim_{V\sups f(U)}\mathscr{F}(f^{-1}(V))\ar[r,"\res^*_{U}"]&\mathscr{F}(U)
\end{tikzcd}\]
Let $s\in f^p\mathscr{G}(U)$ be represented by $v\in\mathscr{G}(V)$ for some $V\sups f(U)$. Since $f^{-1}(V)\sups U$, from the diagram
\[\begin{tikzcd}
f^p\mathscr{G}(f^{-1}(V))\ar[r,"\psi_{f^{-1}(V)}"]\ar[d,swap,"\res^{f^{-1}(V)}_{U}"]&\mathscr{F}(f^{-1}(V))\ar[d,"\res^{f^{-1}(V)}_{U}"]\\
f^p\mathscr{G}(U)\ar[r,"\psi_{U}"]&\mathscr{F}(U)
\end{tikzcd}\]
we obtain
\begin{align*}
(\psi^\flat)^{\sharp}(s)&=\res^*_U\big[\psi_{f^{-1}(V)}\big([v]_{f^{-1}(V)}\big)\big]_U=\res^{f^{-1}(V)}_U\circ\psi_{f^{-1}(V)}\big([v]_{f^{-1}(V)}\big)\\
&=\psi_U\big(\res^{f^{-1}(V)}_U([v]_{f^{-1}(V)})\big)=\psi_U([v]_{U})=\psi_U(s).
\end{align*}
Thus $(\psi^\flat)^\sharp=\psi$. A similar argument gives $(\varphi^\sharp)^\flat=\varphi$, hence we are done.
\end{proof}
\begin{remark}
We will almost never use the concrete description of $f^{p}\mathscr{G}$ in the sequel. Very often we are given $f$, $\mathscr{F}$, and $\mathscr{G}$ as in the Proposition~\ref{presheaf inverse image exist}, and a morphism of sheaves $\varphi:\mathscr{G}\to f_*\mathscr{F}$. Then usually it will be sufficient to understand for each $x\in X$ the map
\[\varphi^{\sharp}_x:(f^{p}\mathscr{G})_x=\mathscr{G}_{f(x)}\to\mathscr{F}_x\]
induced by $\varphi^{\sharp}:f^{p}\mathscr{G}\to\mathscr{F}$ on stalks. The proof of Proposition~\ref{presheaf inverse image exist} shows that we can describe this map in terms of as follows. For every open neighborhood $V\sub Y$ of $f(x)$, we have maps
\[\begin{tikzcd}
\mathscr{G}(V)\ar[r,"\varphi_V"]&\mathscr{F}(f^{-1}(V))\ar[r]&\mathscr{F}_x
\end{tikzcd}\]
and taking the colimit over all $V$ we obtain the map $\varphi^{\sharp}_x:\mathscr{G}_{f(x)}\to\mathscr{F}_x$.
\end{remark}
\begin{proposition}\label{pull back stalk}
Let $f:X\to Y$ be a continuous map. Let $x\in X$. Let $\mathscr{G}$ be a presheaf of sets on $Y$. There is a canonical bijection of stalks $(f^{p}\mathscr{G})_x=\mathscr{G}_{f(x)}$.
\end{proposition}
\begin{proof}
This is obtained as follows
\begin{align*}
(f^{p}\mathscr{G})_x&=\rlim_{U\ni x}f^{p}\mathscr{G}(U)=\rlim_{U\ni x}\rlim_{V\sups f(U)}\mathscr{G}(V)=\rlim_{V\ni f(x)}\mathscr{G}(V)=\mathscr{G}_{f(x)}.
\end{align*}
Here we have used the fact that any $V$ open in $Y$ containing $f(x)$ occurs in the description above. The case for sheaves is obtained from Proposition~\ref{sheaf sheafification construction}.
\end{proof}
Let $\mathscr{G}$ be a sheaf of sets on $Y$. The \textbf{pullback sheaf} $f^{-1}\mathscr{G}$ is defined by the formula
\[f^{-1}\mathscr{G}=(f^{p}\mathscr{G})^{\hash}.\]
Sheafification is a left adjoint to the inclusion of sheaves in presheaves, and $f^{p}$ is a left adjoint to $f_*$ on presheaves. As a formal consequence we obtain that $f^{-1}$ is a left adjoint of pushforward on sheaves: for sheaves $\mathscr{F}$ and $\mathscr{G}$, we have
\begin{align*}
\Mor_{\mathbf{Shv}(X)}(f^{-1}\mathscr{G},\mathscr{F})&=\Mor_{\mathbf{Shv}(X)}\big((f^{p}\mathscr{G})^{\hash},\mathscr{F}\big)\cong\Mor_{\mathbf{Psh}(X)}\big(f^{p}\mathscr{G},\mathscr{F}\big)\\
&\cong\Mor_{\mathbf{Psh}(X)}(\mathscr{G},f_*\mathscr{F})=\Mor_{\mathbf{Shv}(X)}(\mathscr{G},f_*\mathscr{F})
\end{align*}
\begin{proposition}
There are canonical maps 
\[f^{-1}f_*\mathscr{F}\to\mathscr{F},\quad \mathscr{G}\to f_*f^{-1}\mathscr{G}\]
for sheaves $\mathscr{F}$ on $X$ and $\mathscr{G}$ on $Y$.
\end{proposition}
\begin{proof}
We already have maps
\[\sigma_{\mathscr{F}}:f^pf_*\mathscr{F}\to\mathscr{F},\quad\rho_{\mathscr{G}}:\mathscr{G}\to f^pf_*\mathscr{G}.\]
The map $f^{-1}f_*\mathscr{F}\to\mathscr{F}$ is given by the universal property of sheafification:
\[\begin{tikzcd}
f^{p}f_*\mathscr{F}\ar[r,"sh"]\ar[rd,swap,"\sigma_{\mathscr{F}}"]&f^{-1}f_*\mathscr{F}\ar[d,dashed]\\
&\mathscr{F}
\end{tikzcd}\]
and the map $\mathscr{G}\to f_*f^{-1}\mathscr{G}$ is given by the composition
\[\begin{tikzcd}
\mathscr{G}(V)\ar[r,"\rho_{\mathscr{G},V}"]&f^p\mathscr{G}(f^{-1}(V))\ar[r,"sh"]&f^{-1}\mathscr{G}(f^{-1}(V))
\end{tikzcd}\]
for $V\sub Y$ open.
\end{proof}
\begin{proposition}[\textbf{Inverse image and composition}]\label{presheaf inverse image composition prop}
Let $f:X\to Y$ and $g:Y\to Z$ be continuous maps of topological spaces. The functors $(g\circ f)^{-1}$ and $f^{-1}\circ g^{-1}$ are canonically isomorphic. Similarly $(g\circ f)^p\cong f^{p}\circ g^p$ on presheaves.
\end{proposition}
\begin{proof}
This comes from the formal consequence
\begin{align*}
\Mor_{\mathbf{Psh}(X)}\big((g\circ f)^p\mathscr{G},\mathscr{F}\big)&\cong\Mor_{\mathbf{Psh}(X)}\big(\mathscr{G},(g\circ f)_*\mathscr{F}\big)=\Mor_{\mathbf{Psh}(X)}\big(\mathscr{G},g_*\circ f_*\mathscr{F}\big)\\
&\cong\Mor_{\mathbf{Psh}(X)}\big(g^p\mathscr{G},f_*\mathscr{F}\big)\cong\Mor_{\mathbf{Psh}(X)}\big(f^{p}\circ g^p\mathscr{G},\mathscr{F}\big)
\end{align*}
By the uniqueness of adjoint functors, we obtain $(g\circ f)^p\cong f^{p}\circ g^p$. A similar computation holds for $(g\circ f)^{-1}$.
\end{proof}
To conclude this part, we use the direct image functor to produce an adjoint of the stalk functor. First, we need the following concept of a skyscraper sheaf.
\begin{definition}
Let $x\in X$ be a point. Denote $i_x:\{x\}\mapsto X$ the inclusion map. Let $A$ be a set and think of $A$ as a sheaf on the one point space $\{x\}$. We define the \textbf{skyscraper sheaf} at $x$ with value $A$ as the pushforward sheaf $i_{x,*}A$. Explicitly,
\[i_{x,*}A(U)=\begin{cases}
A&\text{if }x\in U,\\
\emp&\text{if }x\notin U.
\end{cases}\]
We say a sheaf $\mathscr{F}$ is a skyscraper sheaf if $\mathscr{F}\cong i_{x,*}A$ for some $x\in X$ and a set $A$.
\end{definition}
\begin{lemma}\label{stalk of skyscraper}
Let $X$ be a topological space, $x\in X$ a point, and $A$ a set. For any point $y\in X$ the stalk of the skyscraper sheaf at $x$ with value $A$ at $y$ is
\[(i_{x,*}A)_y=\begin{cases}
A&\text{if }y\in\widebar{\{x\}}\\
\{\ast\}&\text{otherwise}
\end{cases}\]
\end{lemma}
\begin{proof}
If $x\notin\widebar{\{x\}}$, then there exist arbitrarily small open neighbourhoods $U$ of $y$ which do not contain $x$. Because $\mathscr{F}$ is a sheaf we have $\mathscr{F}(i_x^{-1}(U))=\{\ast\}$ for any such $U$.
\end{proof}
\begin{proposition}\label{stalk adj skyscraper}
Let $X$ be a topological space, and let $x\in X$ be a point. The functors $\mathscr{F}\mapsto\mathscr{F}_x$ and $A\mapsto i_{x,*}A$ are adjoint. In a formula,
\[\Mor_{\mathbf{Set}}(\mathscr{F}_x,A)\cong\Mor_{\mathbf{Shv}(X)}(\mathscr{F},i_{x,*}A)\]
\end{proposition}
\begin{proof}
Consider the pull back functor of the map $i_x:\{x\}\to X$: for a sheaf $\mathscr{F}$ on $X$,
\[i_x^{-1}\mathscr{F}(\{x\})=\rlim_{U\ni x}\mathscr{F}(U),\]
which is exactly the stalk of $\mathscr{F}$ at $x$. Thus the adjointness comes from that of $i_x^{-1}$ and $i_{x,*}$.
\end{proof}
\subsection{Open immersions and closed immersions}
\paragraph{Open immersions and sheaves}
Let $j:U\hookrightarrow X$ be an open immersion (that is, an embedded of $U$ into an open subset of $X$). It turns out that there is a functor $j_!$ which is left adjoint to $j^{-1}$, so that we get a triple $(j_!,j^{-1},j_*)$ in which each consecutive pair is an adjunction and $j^{-1}$ is exact. But first let us point out that $j^{-1}$ has a particularly simple description in the case of an open immersion.
\begin{proposition}\label{sheaf inverse image of open immersion}
Let $X$ be a topological space. Let $j:U\to X$ be the inclusion of an open subset $U$ into $X$.
\begin{itemize}
\item[(a)] Let $\mathscr{G}$ be a presheaf of sets on $X$. The presheaf $j^p\mathscr{G}$ is given by the rule $V\mapsto\mathscr{G}(V)$ for $V\sub U$ open.
\item[(b)] Let $\mathscr{G}$ be a sheaf of sets on $X$. The sheaf $j^{-1}\mathscr{G}$ is given by the rule $V\mapsto\mathscr{G}(V)$ for $V\sub U$ open.
\item[(c)] On the category of presheaves of $U$ we have $j^pj_*=\mathbf{1}$, and on the category of sheaves of $U$ we have $j^{-1}j_*=\mathbf{1}$.
\end{itemize}
\end{proposition}
\begin{proof}
Note that $j^{-1}j_*\mathscr{F}(V)=j^{p}j_*\mathscr{F}(V)=\mathscr{F}(V)$ for open subsets $V\sub U$, so the claims follow immediately.
\end{proof}
\begin{definition}
Let $X$ be a topological space. Let $j:U\to X$ be the inclusion of an open subset.
\begin{itemize}
\item Let $\mathscr{G}$ be a presheaf of sets on $X$. The presheaf $j^p\mathscr{G}$ is called the \textbf{restriction} of $\mathscr{G}$ to $U$ and denoted $\mathscr{G}|_U$.
\item Let $\mathscr{G}$ be a sheaf of sets on $X$. The presheaf $j^{-1}\mathscr{G}$ is called the \textbf{restriction} of $\mathscr{G}$ to $U$ and denoted $\mathscr{G}|_U$.
\end{itemize}
\end{definition}
\begin{definition}
Let $X$ be a topological space. Let $j:U\hookrightarrow X$ be the inclusion
of an open subset.
\begin{itemize}
\item[(a)] Let $\mathscr{F}$ be a presheaf of sets on $U$. We define the \textbf{extension of $\mathscr{F}$ by the empty set} $j_{p!}\mathscr{F}$ to be the presheaf of sets on $X$ defined by the rule
\[j_{p!}=\begin{cases}
\mathscr{F}(V)&\text{if }V\sub U,\\
\emp&\text{otherwise}.
\end{cases}\]
with obvious restriction mappings.
\item[(b)] Let $\mathscr{F}$ be a sheaf of sets on $U$. We define the \textbf{extension of $\mathscr{F}$} by the empty set $j_!\mathscr{F}$ to be the sheafification of the presheaf $j_{p!}\mathscr{F}$.
\end{itemize}
\end{definition}
\begin{proposition}\label{sheaf open extension functor prop}
Let $X$ be a topological space. Let $j:U\to X$ be the inclusion of an open subset. 
\begin{itemize}
\item[(a)] The functor $j_{p!}$ is a left adjoint to the restriction functor $j^p$:
\[\Mor_{\mathbf{Psh}(X)}(j_{p!}\mathscr{F},\mathscr{G})\cong\Mor_{\mathbf{Psh}(U)}(\mathscr{F},j^p\mathscr{G})=\Mor_{\mathbf{Psh}(X)}(\mathscr{F},\mathscr{G}|_U).\]
\item[(b)] The functor $j_!$ is a left adjoint to restriction,
\[\Mor_{\mathbf{Shv}(X)}(j_{!}\mathscr{F},\mathscr{G})\cong\Mor_{\mathbf{Shv}(U)}(\mathscr{F},j^{-1}\mathscr{G})=\Mor_{\mathbf{Shv}(U)}(\mathscr{F},\mathscr{G}|_U).\]
\item[(c)] Let $\mathscr{F}$ be a sheaf of sets on $U$. The stalks of the sheaf $j_!\mathscr{F}$ are described as follows
\[(j_!\mathscr{F})_x=\begin{cases}
\mathscr{F}_x&\text{if }x\in U,\\
\emp&\text{otherwise}.
\end{cases}\]
Therefore the functor $j_!$ is exact.
\item[(e)] On the category of presheaves of $U$ we have $j^pj_{p!}=\mathbf{1}$, and on the category of sheaves of $U$ we have $j^{-1}j_!=\mathbf{1}$.
\end{itemize}
\end{proposition}
\begin{proof}
To map $j_{p!}\mathscr{F}$ into $\mathscr{G}$ it is enough to map $\mathscr{F}(V)\to\mathscr{G}(V)$ whenever $V\sub U$ compatibly with restriction mappings. And the same description holds for maps $\mathscr{F}\mapsto\mathscr{G}|_U$. The adjointness of $j_!$ and restriction follows from this and the properties of sheafification. The identification of stalks is obvious from the definition of the extension by the empty set and the definition of a stalk.\par
Finally, if $\mathscr{F}$ is a sheaf on $U$, consider the canonical maps
\[\mathscr{F}\mapsto j^pj_{p!}\mathscr{F},\quad\mathscr{F}\mapsto j^{-1}j_{!}\mathscr{F}.\]
Since the induced maps on stalks are isomorphisms, the claim follows.
\end{proof}
\begin{theorem}\label{sheaf open extension fully faithful essential image char}
Let $X$ be a topological space. Let $j:U\to X$ be the inclusion of an open subset. The functor
\[j_!:\mathbf{Shv}(U)\to\mathbf{Shv}(X)\]
is fully faithful. Its essential image consists exactly of those sheaves $\mathscr{G}$ whose support is contained in $U$.
\end{theorem}
\begin{proof}
Fully faithfulness follows formally from $j^{-1}j_!=\mathbf{1}$. We have seen that any sheaf in the image of the functor has the property on the stalks mentioned in the lemma. Conversely, suppose that $\mathscr{G}$ has the indicated property, then the canonical map
\[j_!j^{-1}\mathscr{G}\to\mathscr{G}\]
is an isomorphism on all stalks and hence an isomorphism.
\end{proof}
\paragraph{Closed immersions and sheaves}
Let $X$ be a topological space. Let $i:Z\to X$ be the inclusion of a closed subset $Z$ into $X$. In a similar fashion, we can extend the adjunction $(i^{-1},i_*)$. First we state an important result for the direct image $i_*$.
\begin{lemma}\label{sheaf closed pushforward stalk char}
Let $X$ be a topological space. Let $i:Z\hookrightarrow X$ be the inclusion of a closed subset $Z$ into $X$. Let $\mathscr{F}$ be a sheaf of sets on $Z$. The stalks of $i_*\mathscr{F}$ are described By
\[(i_*\mathscr{F})_x=\begin{cases}
\mathscr{F}_x&x\in Z,\\
\{\ast\}&\text{otherwise}.
\end{cases}\]
\end{lemma}
\begin{proof}
This follows from the definition of $i_*\mathscr{F}$, and the fact that $Z$ is closed.
\end{proof}
\begin{theorem}\label{sheaf closed extension functor prop}
Let $X$ be a topological space. Let $i:Z\to X$ be the inclusion of a closed subset. The functor
\[i_*:\mathbf{Shv}(Z)\to\mathbf{Shv}(X)\]
is fully faithful. Its essential image consists exactly of those sheaves $\mathscr{G}$ whose support is contained in $Z$. The functor $i^{-1}$ is a left inverse to $i_*$.
\end{theorem}
\begin{proof}
Fully faithfulness follows formally from $i^{-1}i_*=\mathbf{1}$. We have seen that any sheaf in the image of the functor has the property on the stalks mentioned in the lemma. Conversely, suppose that $\mathscr{G}$ has the indicated property. Then the map $\mathscr{G}\to i_*i^{-1}\mathscr{G}$ is an isomorphism on stalks, and hence an isomrophism.
\end{proof}
Now we define the functor $i^!$ for a closed immersion.
\begin{proposition}
Let $X$ be a topological space. Let $Z\sub X$ be a closed subset. Let $\mathscr{F}$ be a sheaf on $X$. Define a sheaf $\mathscr{H}_Z(\mathscr{F})$ by
\[\Gamma(U,\mathscr{H}_Z(\mathscr{F}))=\{s\in\mathscr{F}(U):\supp(s)\sub Z\cap U\}.\]
Then $\mathscr{H}_Z(\mathscr{F})$ is a subsheaf of $\mathscr{F}$. It is the largest subsheaf of $\mathscr{F}$ whose support is contained in $Z$. The construction $\mathscr{F}\mapsto\mathscr{H}_Z(\mathscr{F})$ is functorial in the sheaf $\mathscr{F}$.
\end{proposition}
\begin{proof}

\end{proof}
\begin{definition}
Let $i:Z\to X$ be the inclusion of a closed subset. For a sheaf $\mathscr{F}$ on $X$, define the \textbf{inverse image of $\mathscr{F}$ supported on $\bm{Z}$} to be the sheaf $i^!\mathscr{F}=i^{-1}\mathscr{H}_Z(\mathscr{F})$.
\end{definition}
\begin{proposition}
For a closed immersion $i:Z\hookrightarrow X$, the functor $i^!$ is right adjoint to $i_*$. In a formula
\[\Mor_{\mathbf{Shv}(X)}(i_*\mathscr{G},\mathscr{F})=\Mor_{\mathbf{Shv}(Z)}(\mathscr{G},i^!\mathscr{F})\]
\end{proposition}
\begin{proof}
Note that $i_*i^!\mathscr{F}=\mathscr{H}_Z(\mathscr{F})$ by Theorem~\ref{sheaf closed extension functor prop}. Since $i_*$ is fully faithful we are reduced to showing that
\[\Mor_{\mathbf{Shv}(X)}(i_*\mathscr{G},\mathscr{F})=\Mor_{\mathbf{Shv}(Z)}(i_*\mathscr{G},\mathscr{H}_Z(\mathscr{F})).\]
This follows since the support of the image via any homomorphism of a section of $i_*\mathscr{G}$ is contained in $Z$, by Theorem~\ref{sheaf closed extension functor prop}.
\end{proof}
\begin{proposition}
Let $i:Z\hookrightarrow X$ be a closed embedding, set $U=Z^c$ and let $j:U\hookrightarrow X$ be the corresponding open embedding. Then for any sheaf $\mathscr{F}$ on $X$, the sequence
\[\begin{tikzcd}
0\ar[r]&j_!j^{-1}\mathscr{F}\ar[r]&\mathscr{F}\ar[r]&j_*j^{-1}\mathscr{F}\ar[r]&0
\end{tikzcd}\]
is exact. Moreover, for any morphism $\varphi:\mathscr{F}\to\mathscr{G}$ of sheaves on $X$ we obtain a morphism of short exact sequences
\[\begin{tikzcd}
0\ar[r]&j_!j^{-1}\mathscr{F}\ar[r]\ar[d]&\mathscr{F}\ar[r]\ar[d]&j_*j^{-1}\mathscr{F}\ar[r]\ar[d]&0\\
0\ar[r]&j_!j^{-1}\mathscr{G}\ar[r]&\mathscr{G}\ar[r]&j_*j^{-1}\mathscr{G}\ar[r]&0
\end{tikzcd}\]
\end{proposition}
\begin{proof}
The functoriality of the short exact sequence is immediate from the naturality of the adjunction mappings. We may check exactness on stalks: for $x\in X$, the sequence is
\[\begin{cases}
0\to\mathscr{F}_x\to\mathscr{F}_x\to 0\to 0&x\in U,\\
0\to0\to\mathscr{F}_x\to \mathscr{F}_x\to 0&x\in Z.
\end{cases}\]
so the claim follows.
\end{proof}
\subsection{Glueing sheaves}
It is quite often that we want to glue a bunch of objects defined on the members of a covering of $X$ to create a new one. In this paragraph we will see how to do this for morphisms and sheaves.
\begin{proposition}\label{sheaf glue morphism of sheaf}
Let $X$ be a topological space. Let $X=\bigcup_iU_i$ be an open covering. Let $\mathscr{F},\mathscr{G}$ be sheaves of sets on $X$. Given a collection
\[\varphi_i:\mathscr{F}|_{U_i}\to\mathscr{G}|_{U_i}\]
of maps of sheaves such that for all $i,j\in I$ the maps $\varphi_i,\varphi_j$ restrict to the same map $\mathscr{F}|_{U_i\cap U_j}\to\mathscr{G}|_{U_i\cap U_j}$, there exists a unique map of sheaves
\[\varphi:\mathscr{F}\to\mathscr{G}\]
whose restriction to each $U_i$ agrees with $\varphi_i$.
\end{proposition}
\begin{proof}
For $V\sub X$ open, we have $V=\bigcup_i(V\cap U_i)$, so for $s\in\mathscr{F}(V)$ we define $\varphi_V(s)$ by the equations
\begin{align}\label{sheaf glue morphism of sheaf-1}
\big(\varphi_V(s)\big)|_{V\cap U_i}=(\varphi_{i})_{V\cap U_i}(s|_{V\cap U_i})\for i\in I.
\end{align}
By the condition of sheaf, this is well-defined, and indeed defines a morphism of sheaves. It is clear that $\varphi$ restrict to $\varphi_i$ on each $U_i$. To see the uniqueness, if $\psi:\mathscr{F}\to\mathscr{G}$ is another morphism restricting to $\varphi_i$ on each $i$, then for each $s\in\mathscr{F}(U)$, $\psi(s)$ also satisfies (\ref{sheaf glue morphism of sheaf-1}), so it must coincides with $\varphi(s)$ by the condition of sheaf. This shows $\varphi=\psi$, as desired.
\end{proof}
The previous proposition implies that given two sheaves $\mathscr{F},\mathscr{G}$ on the topological space $X$ the rule
\[U\mapsto\Mor_{\mathbf{Shv}(X)}(\mathscr{F}|_U,\mathscr{G}|_U)\]
defines a sheaf. This is a kind of \textbf{internal hom sheaf}. It is seldom used in the setting of sheaves of sets, and more usually in the setting of sheaves of modules.\par
Let $X$ be a topological space. Let $X=\bigcup_iU_i$ be an open covering. For each $i\in I$ let $\mathscr{F}_i$ be a sheaf of sets on $U_i$. For each pair $i,j\in I$, let 
\[\varphi_{ij}:\mathscr{F}_i|_{U_i\cap U_j}\to\mathscr{F}_j|_{U_i\cap U_j}\]
be an isomorphism of sheaves of sets. Assume in addition that for every triple of indices $i,j,k\in I$ the following diagram is commutative
\[\begin{tikzcd}
\mathscr{F}_i|_{U_i\cap U_j\cap U_k}\ar[rr,"\varphi_{ik}"]\ar[rd,"\varphi_{ij}"]&&\mathscr{F}_k|_{U_i\cap U_j\cap U_k}\\
&\mathscr{F}_j|_{U_i\cap U_j\cap U_k}\ar[ru,"\varphi_{jk}"]
\end{tikzcd}\]
We will call such a collection of data $(\mathscr{F}_i,\varphi_{ij})$ a \textbf{glueing data} for sheaves of sets with respect to the covering $X=\bigcup_iX_i$.
\begin{proposition}\label{sheaf glue construction}
Let $X$ be a topological space. Let $X=\bigcup_iU_i$ be an open covering. Given any glueing data $(\mathscr{F}_i,\varphi_{ij})$ for sheaves of sets with respect to the covering, there exists a sheaf of sets $\mathscr{F}$ on $X$ together with isomorphisms
\[\varphi_i:\mathscr{F}|_{U_i}\to\mathscr{F}_i\]
such that the diagrams
\[\begin{tikzcd}
\mathscr{F}|_{U_i\cap U_j}\ar[r,"\varphi_i"]\ar[d,equal]&\mathscr{F}_i|_{U_i\cap U_j}\ar[d,"\varphi_{ij}"]\\
\mathscr{F}|_{U_i\cap U_j}\ar[r,"\varphi_j"]&\mathscr{F}_j|_{U_i\cap U_j}
\end{tikzcd}\]
are commutative.
\end{proposition}
\begin{proof}
Actually we can write a formula for the set of sections of $\mathscr{F}$ over an open $W\sub X$. Namely, we define
\[\mathscr{F}(W)=\{(s_i)_{i\in I}\mid\text{$s_i\in\mathscr{F}_i(W\cap U_i)$ and $\varphi_{ij}(s_i|_{W\cap U_i\cap U_j})=s_j|_{W\cap U_i\cap U_j}$}\}.\]
Restriction mappings for $W'\sub W$ are defined by the restricting each of the $(s_i)$ to $W'\cap U_i$. The sheaf condition for $\mathscr{F}$ follows immediately from the sheaf condition for each of the $\mathscr{F}_i$.\par
We still have to prove that $\mathscr{F}|_{U_i}$ maps isomorphically to $\mathscr{F}_i$. Let $W\sub U_i$; then the commutativity of the diagrams in the definition of a glueing data assures that we may start with any section $s\in\mathscr{F}_i(W)$ and obtain a compatible collection by setting $s_i=s$ and $s_j=\varphi_{ij}(s_{i}|_{W\cap U_j})$. Thus the claim follows.
\end{proof}
\begin{corollary}\label{sheaf cat equivalent to glueing data}
Let $X$ be a topological space. Let $X=\bigcup_iU_i$ be an open covering. The functor which associates to a sheaf of sets $\mathscr{F}$ the following collection of glueing data
\[\big(\mathscr{F}|_{U_i},(\mathscr{F}_i)|_{U_i\cap U_j}\to(\mathscr{F}_j)|_{U_i\cap U_j}\big)\]
with respect to the covering $X=\bigcup_iU_i$ defines an equivalence of categories between $\mathbf{Shv}(X)$ and the category of glueing data.
\end{corollary}
This result means that if the sheaf $\mathscr{F}$ was constructed from the glueing data $(\mathscr{F})_i,\varphi_{ij})$ and if $\mathscr{G}$ is a sheaf on $X$, then a morphism $f:\mathscr{F}\to\mathscr{G}$ is given by a collection of morphisms of sheaves
\[f_i:\mathscr{F}_i\to\mathscr{G}\]
compatible with the glueing maps $\varphi_{ij}$. Similarly, to give a morphism of sheaves $g:\mathscr{G}\to\mathscr{F}$ is the same as giving a collection of morphisms of sheaves
\[g_i:\mathscr{G}|_{U_i}\to\mathscr{F}_i\]
compatible with the glueing maps $\varphi_{ij}$.
\subsection{Preheaves and sheaves over a basis}
Sometimes there exists a basis for the topology consisting of opens that are easier to work with than general opens. For convenience we give here some definitions and simple lemmas in order to facilitate working with (pre)sheaves in such a situation.\par
Let $X$ be a topological space. Let $\mathcal{B}$ be a basis for the topology on $X$. A \textbf{presheaf $\mathscr{F}$ of sets on $\mathcal{B}$} is a rule which assigns to each $U\in\mathcal{B}$ a set $\mathscr{F}(U)$ and to each inclusion $V\sub U$ of elements of $\mathcal{B}$ a map $\res^U_V:\mathscr{F}(U)\to\mathscr{F}(V)$ such that whenever $W\sub V\sub U$ in $\mathcal{B}$ we have $\res^U_V\circ\res^V_W=\res^U_W$. If $\mathscr{F}$ be a presheaf over the basis $\mathcal{B}$. We can associate $\mathscr{F}$ with a sheaf $\mathscr{F}_{\mathcal{B}}$ by defining
\begin{align}\label{sheaf on basis associate def}
\mathscr{F}_{\mathcal{B}}(U)=\{(s_V)\in\prod_{\substack{V\in\mathcal{B}\\V\sub U}}\mathscr{F}(V):\text{for all $W,V\in\mathcal{B}$, $W\sub V$, $s_V|_{W}=s_{W}$}\}=\llim_{\substack{V\in\mathcal{B}\\V\sub U}}\mathscr{F}(V).
\end{align}
for any open subset $U$ of $X$. If $U$ and $U'$ are two open sets of $X$ such that $U\sub U'$, we define $\res^{U'}_{U}$ as the inverse limit (for $V\sub U$) of the canonical morphisms $\mathscr{F}_{\mathcal{B}}(U')\to\mathscr{F}(V)$, in other words, the unique morphism $\mathscr{F}_{\mathcal{B}}(U')\to\mathscr{F}_{\mathcal{B}}(U)$ which, composed with the canonical morphisms $\mathscr{F}_{\mathcal{B}}(U)\to\mathscr{F}(V)$, gives the canonical morphisms $\mathscr{F}_{\mathcal{B}}(U')\to\mathscr{F}(V)$; it is then immediate that the transitivity holds. Moreover, if $U\in\mathcal{B}$, the canonical morphism $\mathscr{F}_{\mathcal{B}}(U)\to\mathscr{F}(U)$ is an isomorphism, allowing us to identify these two sets.
\begin{proposition}\label{sheaf on basis sheaf condition}
The following conditions are equivalent:
\begin{itemize}
\item[(\rmnum{1})] The presheaf $\mathscr{F}_{\mathcal{B}}$ is a sheaf on $X$.
\item[(\rmnum{2})] For any covering $(U_\alpha)$ of $U\in\mathcal{B}$ given by elements of $\mathcal{B}$, the set $\mathscr{F}(U)$ corresponds bijectively to $(s_\alpha)\in\prod_\alpha\mathscr{F}(U_\alpha)$ such that $s_\alpha|_V=s_\beta|_V$ for any $V\in\mathcal{B}$ and $V\sub U_\alpha\cap U_\beta$.
\item[(\rmnum{3})] For any covering $(U_\alpha)$ of $U\in\mathcal{B}$ given by elements of $\mathcal{B}$ and $(U_{\alpha\beta\gamma})$ of $U_{\alpha\beta}=U_\alpha\cap U_\beta$ by elements of $\mathcal{B}$, the set $\mathscr{F}(U)$ corresponds bijectively to $(s_\alpha)\in\prod_\alpha\mathscr{F}(U_\alpha)$ such that $s_\alpha|_{U_{\alpha\beta\gamma}}=s_\beta|_{U_{\alpha\beta\gamma}}$ for all $\gamma$.
\end{itemize}
\end{proposition}
\begin{proof}
It is clear that (\rmnum{1})$\Rightarrow$(\rmnum{3})$\Rightarrow$(\rmnum{2}). Now assume (\rmnum{2}) and let $(U_\alpha)$ be a covering of $U\in\mathcal{B}$ by elements of $\mathcal{B}$ and $(U_{\alpha\beta\gamma})$ a covering of $U_{\alpha\beta}$ by elements of $\mathcal{B}$. Let $(s_\alpha)\in\prod_\alpha\mathscr{F}(U_\alpha)$ be a family such that $s_\alpha|_{U_{\alpha\beta\gamma}}=s_\beta|_{U_{\alpha\beta\gamma}}$ for all $\gamma$. Let $V\sub U_\alpha\cap U_\beta$ be an basis element in $\mathcal{B}$. For each index $\gamma$, let $(V_{\mu\gamma})$ be a covering of $V\cap U_{\alpha\beta\gamma}$ by elements of $\mathcal{B}$, so that the family $(V_{\mu\gamma})_{\mu,\gamma}$ is a covering of $V$ by elements of $\mathcal{B}$. For each pair $(\mu,\gamma)$ of indices, set
\[t^\alpha_{\mu\gamma}=s_\alpha|_{V_{\mu\gamma}},\quad t^{\beta}_{\mu\gamma}=s_\beta|_{V_{\mu\gamma}}.\]
By hypothesis $s_\alpha|_{U_{\alpha\beta\gamma}}=s_\beta|_{U_{\alpha\beta\gamma}}$ and $V_{\mu\gamma}\sub U_{\alpha\beta\gamma}$, so we have $t^{\alpha}_{\mu\gamma}=t^{\beta}_{\mu\gamma}$. Moreover, for each $W\in\mathcal{B}$ and $W\sub V_{\mu\gamma}\cap V_{\tilde{\mu}\tilde{\gamma}}$, since $V_{\mu\gamma}$ and $V_{\tilde{\mu}\tilde{\gamma}}$ are both contained in $U_{\alpha\beta}$,
\[t^\alpha_{\mu\gamma}|_W=s_\alpha|_W=t^\alpha_{\tilde{\mu}\tilde{\gamma}}|_W,\quad t^\beta_{\mu\gamma}|_W=s_\beta|_W=t^\beta_{\tilde{\mu}\tilde{\gamma}}|_W.\]
Applying (\rmnum{2}) on the open set $V$ and the covering $(V_{\mu\gamma})$, we then conclude that $s_\alpha|_V=s_\beta|_V$, which again by condition (\rmnum{2}) implies that $(s_\alpha)$ corresponds to a section on $\mathscr{F}(U)$. This proves (\rmnum{2})$\Rightarrow$(\rmnum{3}).\par
Now we prove the implication (\rmnum{3})$\Rightarrow$(\rmnum{1}). Before this, we first note that, if (\rmnum{3}) holds and $\mathcal{B}'$ is a basis of $X$ contained in $\mathcal{B}$, then the presheaf $\mathscr{F}_{\mathcal{B}'}$ associated to the presheaf $(\mathscr{F}(U))_{U\in\mathcal{B}'}$ is canonically isomorphic to the presheaf $\mathscr{F}_{\mathcal{B}}$ associated to $(\mathscr{F}(U))_{U\in\mathcal{B}}$. Indeed, first of all the inverse limit (for $V\in\mathcal{B}'\sub\mathcal{B}$, $V\sub U$) of the canonical morphisms $\mathscr{F}_{\mathcal{B}}(U)\to\mathscr{F}(V)$ gives a morphism
\[\mathscr{F}_{\mathcal{B}}(U)\to\mathscr{F}_{\mathcal{B}'}(U)\]
for any open set $U$. We claim that this is an isomorphism if $U\in\mathcal{B}$. To see this, let $(U_\alpha)$ be a covering of $U$ by elements of $\mathcal{B}'$ and for each $(\alpha,\beta)$, choose a covering $(U_{\alpha\beta\gamma})$ of $U_{\alpha\beta}=U_\alpha\cap U_\beta$ by elements of $\mathcal{B}'$. For each $\gamma$, we have a commutative diagram
\[\begin{tikzcd}[row sep=15pt,column sep=12pt]
&&\mathscr{F}(U_\alpha)\ar[rd]\\
\mathscr{F}_{\mathcal{B}'}(U)\ar[rru,bend left=20pt]\ar[rrd,bend right=20pt]\ar[r,dashed]&\mathscr{F}(U)\ar[rd]\ar[ru]&&\mathscr{F}(U_{\alpha\beta\gamma})\\
&&\mathscr{F}(U_\beta)\ar[ru]
\end{tikzcd}\]
so condition (\rmnum{3}) shows that the canonical morphism $\mathscr{F}_{\mathcal{B}'}(U)\to\mathscr{F}(U_\alpha)$ factors through $\mathscr{F}(U)$. It is immediate that the morphisms $\mathscr{F}_{\mathcal{B}}(U)\to\mathscr{F}_{\mathcal{B}'}(U)$ and $\mathscr{F}_{\mathcal{B}'}(U)\to\mathscr{F}_{\mathcal{B}}(U)$ thus defined are inverses of each other. This being so, for all open set $U$ of $X$, the morphisms
\[\mathscr{F}_{\mathcal{B}'}(U)\to\mathscr{F}_{\mathcal{B}'}(W)=\mathscr{F}_{\mathcal{B}}(W)=\mathscr{F}(W)\]
for $W\in\mathcal{B}$, $W\sub U$ satisfy the conditions characterizing the inverse limit of the $\mathscr{F}(W)$'s, so our claim follows from the uniqueness of the inverse limit.\par
Now let $U$ be any open set of $X$, $(U_\alpha)$ a covering of $U$ by open sets contained in $U$, and let $\mathcal{B}'$ be the subfamily of $\mathcal{B}$ consisting of the sets of $\mathcal{B}$ contained in at least one $U_\alpha$. It is clear that $\mathcal{B}'$ is still a basis of the topology of $X$, so $\mathscr{F}_{\mathcal{B}}(U)$ (resp. $\mathscr{F}_{\mathcal{B}}(U_\alpha)$) is the inverse limit of the $\mathscr{F}(V)$ for $V\in\mathcal{B}'$ and $V\sub U$ (resp. $V\sub U_\alpha$); the sheaf axiom is then verified immediately by virtue of the definition of the inverse limit.
\end{proof}
We say $\mathscr{F}$ is a \textbf{sheaf on $\mathcal{B}$} if it satisfies the equivalent conditions in Proposition~\ref{sheaf on basis sheaf condition}.
\begin{corollary}\label{sheaf basis cofinal coro}
Let $X$ be a topological space. Let $\mathcal{B}$ be a basis for the topology on $X$. Assume that for every pair $U,U'\in\mathcal{B}$ we have $U\cap U'\in\mathcal{B}$. Let $\mathscr{F}$ be a presheaf of sets on $\mathcal{B}$. The following are equivalent:
\begin{itemize}
\item[(\rmnum{1})] The presheaf $\mathscr{F}$ is a sheaf on $\mathcal{B}$.
\item[(\rmnum{2})] For every $U\in\mathcal{B}$ and for every family of sections $s_i\in\mathscr{F}(U_i)$ such that $s_{i}|_{U_i\cap U_j}=s_j|_{U_i\cap U_j}$ there exists a unique section $s\in\mathscr{F}(U)$ which restricts to $s_i$ on $U_i$.
\end{itemize}
\end{corollary}
\begin{proof}
This is a reformulation of Proposition~\ref{sheaf on basis sheaf condition}, as we can take $V$ to be $U_\alpha\cap U_\beta$.
\end{proof}
Note that for any $x\in X$ we have $\mathscr{F}_x=(\mathscr{F}_{\mathcal{B}})_x$ in the situation of the proposition. This is so because the collection of elements of $\mathcal{B}$ containing $x$ forms a fundamental system of open neighbourhoods of $x$.\par
Let $\mathscr{F}$, $\mathscr{G}$ be two presheaves over $\mathcal{B}$. A morphism $\varphi:\mathscr{F}\to\mathscr{G}$ is defined to be a family $(\varphi_V)_{V\in\mathcal{B}}$ of morphisms $\varphi_V:\mathscr{F}(V)\to\mathscr{F}(V)$ satisfying the compatible conditions with restriction maps. By passing to inverse limits, we get a morphism $\varphi_\mathcal{B}:\mathscr{F}_\mathcal{B}\to\mathscr{G}_\mathcal{B}$ of presheaves (it is easy to verify the compatible conditions with restriction maps).
\begin{theorem}
Let $X$ be a topological space. Let $\mathcal{B}$ be a basis for the topology on $X$. Denote $\mathbf{Shv}(\mathcal{B})$ the category of sheaves on $\mathcal{B}$. There is an equivalence of categories
\[\mathbf{Shv}(X)\to\mathbf{Shv}(\mathcal{B})\]
which assigns to a sheaf on $X$ its restriction to the members of $\mathcal{B}$.
\end{theorem}
\begin{proof}
If $\mathscr{F}$ is a sheaf on $\mathcal{B}$, then the sheaf $\mathscr{F}_{\mathcal{B}}$ satisfies $\mathscr{F}(U)=\mathscr{F}_{\mathcal{B}}(U)$ for $U\in\mathcal{B}$, thus the restriction of $\mathscr{F}_{\mathcal{B}}$ equals $\mathscr{F}$. Conversely, if $\mathscr{F}$ is a sheaf on $X$, then the restriction $\mathscr{F}|_{\mathcal{B}}$ induces a sheaf $\mathscr{F}':=(\mathscr{F}|_{\mathcal{B}})_{\mathcal{B}}$ on $X$. Then $\mathscr{F}$ and $\mathscr{F}'$ has the same stalk, so we conclude $\mathscr{F}\cong\mathscr{F}'$. Moreover, by looking at stalks, we see the Hom sets are canonically identified, whence the claim.
\end{proof}
\subsection{The category of presheaves and sheaves}
In this subsection, we derive some result for the categories $\mathbf{Psh}(X)$ and $\mathbf{Shv}(X)$.
\paragraph{The category of presheaves} We first consider the category of presheaves, with morphisms are defined to be morphisms of presheaves. As we will see, this category behaves much like the base category $\mathbf{Set}$.
\begin{example}[\textbf{Final object of presheaves}]
Let $X$ be a topological space. Consider a rule $\mathscr{F}$ that associates to every open subset a singleton set. Since every set has a unique map into a
singleton set, there exist unique restriction maps $\res^U_V$. The resulting structure is a presheaf of sets. It is a final object in the category of presheaves of sets, by the property of singleton sets mentioned above. Hence it is also unique up to unique isomorphism. We will sometimes write $\ast$ for this presheaf.
\end{example}
\begin{proposition}
Let $\mathcal{I}$ be a small category and let $\mathscr{F}:\mathcal{I}\to\mathbf{Psh}(X)$ be an $\mathcal{I}$-diagram of presheaves on $X$. Then the limit and colimit of $\mathscr{F}$ both exist, whichi are given by
\[(\llim_i\mathscr{F}_i)(U):=\llim_i\mathscr{F}_i(U),\quad (\rlim_i\mathscr{F}_i)(U):=\rlim_i\mathscr{F}_i(U).\]
and the restriction maps are induced by that of the $\mathscr{F}_i$'s.
\end{proposition}
\begin{proof}
The given constructions are clearly meaningful and define presheaves, where the restriction map is given by the limit (or colimit) of that of $\mathscr{F}_i$. For the colimit, if we are given morphisms $\varphi_i:\mathscr{F}_i\to G$ such that the following diagram commutes
\[\begin{tikzcd}
\mathscr{F}_i\ar[rr]\ar[rd]&&\mathscr{F}_j\ar[ld]\\
&\mathscr{G}&
\end{tikzcd}\]
Then for each $U\sub X$ open we can take limit of the system $(\varphi_{i,U})$ to get a morphism $\varphi:\rlim_i\mathscr{F}_i(U)\to\mathscr{G}(U)$. Moreover, these morphisms are compatible with the restriction maps of $\mathscr{F}_i$, hence compatible with that of $\rlim_i\mathscr{F}_i$. This gives a morphism $\rlim_i\mathscr{F}_i\to\mathscr{G}$, so $\rlim_i\mathscr{F}_i$ satisfies the universal property of colimits. Similarly, we can show that $\llim_i\mathscr{F}_i$ satisfies the universal property of limits.
\end{proof}
\begin{definition}
Let $\varphi:\mathscr{F}\to\mathscr{G}$ be a morphism of presheaves of sets.
\begin{itemize}
\item[(a)] We say that $\varphi$ is \textbf{injective} if for every open subset $U\sub X$ the map $\varphi_U:\mathscr{F}(U)\to\mathscr{G}(U)$ is injective.
\item[(b)] We say that $\varphi$ is \textbf{surjective} if for every open subset $U\sub X$ the map $\varphi_U:\mathscr{F}(U)\to\mathscr{G}(U)$ is injective.
\end{itemize}
\end{definition}
We show that the injectivity and surjectivity gives monomorphisms and epimorphisms in the category of presheaves.
\begin{proposition}\label{presheaf mono epi iff injective surjective}
The injective (resp. surjective) maps defined above are exactly the monomorphisms (resp. epimorphisms) of $\mathbf{Psh}(X)$. A map is an isomorphism if and only if it is both injective and surjective.
\end{proposition}
\begin{proof}
We shall show that $\varphi:\mathscr{F}\to\mathscr{G}$ is injective if and only if it is an monomorphism of $\mathbf{Psh}(X)$. Indeed, the only if direction is straightforward, so let us show the if direction. If $\varphi$ is a monomorphism, let $U\sub X$ be an open subset; we are going to show that $\varphi_U$ is a monomorphism in the category $\mathbf{Set}$. For this, consider any two maps $f,g:A\to\mathscr{F}(U)$ such that $\varphi_U\circ f=\varphi_U\circ g$. We define a presheaf $\mathscr{A}$ by
\[\mathscr{A}(V)=\begin{cases}
A&V\sub U,\\
\emp&\text{otherwise}.
\end{cases}\]
Then we have induced morphism of presheaves $\psi_f$ and $\psi_g$ given by the diagram
\[\begin{tikzcd}
\mathscr{A}(U)\ar[r,bend left=10pt,"f"]\ar[r,bend right=10pt,swap,"g"]\ar[d,equal]&\mathscr{F}(U)\ar[d,"\res^U_V"]\\
\mathscr{A}(V)\ar[r,dashed,"\psi_V"]&\mathscr{F}(V)
\end{tikzcd}\]
Then we can see $\varphi\circ\psi_f=\varphi\circ\psi_g$, which implies $\psi_f=\psi_g$ since $\varphi$ is monic. This gives $f=g$ by our construction, so $\varphi_U$ is monic in the category of sets, hence injective.\par
Now we show that $\varphi:\mathscr{F}\to\mathscr{G}$ is surjective if and only if it is an epimorphism of $\mathbf{Psh}(X)$. Similarly, the only if direction is straightforward, so let us show the if direction. Assume that $\varphi$ is an epimorphism, and we show $\varphi_U$ is epic in $\mathbf{Set}$. For any maps $f,g:\mathscr{G}(U)\to B$ such that $f\circ\varphi_U=g\circ\varphi_U$, we define a presheaf $\mathscr{B}$ by
\[\mathscr{B}(V)=\begin{cases}
B&V\sups U,\\
\emp&\text{othewise}.
\end{cases}\]
Similarly, we can define morphism of presheaves by the diagram
\[\begin{tikzcd}
\mathscr{G}(V)\ar[r,dashed,"\psi_V"]\ar[d,"\res^U_V"]&\mathscr{B}(V)\ar[d,equal]\\
\mathscr{G}(U)\ar[r,bend left=10pt,"f"]\ar[r,bend right=10pt,swap,"g"]&\mathscr{B}(U)
\end{tikzcd}\]
and we have $\psi_f\circ\varphi_U=\psi_g\circ\varphi_U$, which implies $f=g$, so $\varphi_U$ is surjective.
\end{proof}
\paragraph{The category of sheaves}
We have already seen that limits and colimits exist in the category of presheaves. Now we consider them in the category of sheaves.
\begin{proposition}
Let $\mathcal{I}$ be a small category and let $\mathscr{F}:\mathcal{I}\to\mathbf{Psh}(X)$ be an $\mathcal{I}$-diagram of sheaves on $X$. Then the limit and colimit of $\mathscr{F}$ both exist, whichi are given by
\[\llim_{i,\mathbf{Shv}(X)}\mathscr{F}_i=\llim_{i,\mathbf{Psh}(X)}\mathscr{F}_i,\quad \rlim_{i,\mathbf{Shv}(X)}\mathscr{F}_i:=(\rlim_{i,\mathbf{Psh}(X)}\mathscr{F}_i)^{\hash}.\]
\end{proposition}
\begin{proof}
Since the limit and colimit exist in $\mathbf{Psh}(X)$, we can define
\[\llim_{i,\mathbf{Shv}(X)}\mathscr{F}_i:=(\llim_{i,\mathbf{Psh}(X)}\mathscr{F}_i)^{\hash},\quad \rlim_{i,\mathbf{Shv}(X)}\mathscr{F}_i:=(\rlim_{i,\mathbf{Psh}(X)}\mathscr{F}_i)^{\hash}.\]
Using the universal property of sheafification, we can show that the construction above indeed gives the colimits in the category of sheaves. Now we note that, since the forgetful functor $\iota:\mathbf{Shv}(X)\to\mathbf{Psh}(X)$ has a left adjoint given by the sheafification, it commutes with limits. In other words, if $\mathcal{I}$ is a small category and $\mathscr{F}:\mathcal{I}\to\mathbf{Psh}(X)$ be an $\mathcal{I}$-diagram of sheaves on $X$. Then the limit $\llim_i\mathscr{F}_i$ satisfies
\[\iota(\llim_{i,\mathbf{Shv}(X)}\mathscr{F}_i)=\llim_{i,\mathbf{Psh}(X)}\iota(\mathscr{F}_i).\]
But the forgetful functor does nothing actually, so the limit $\llim_i\mathscr{F}_i$ is in fact given by the limit in $\mathbf{Psh}(X)$, i.e.,
\[\llim_{i,\mathbf{Shv}(X)}\mathscr{F}_i=\llim_{i,\mathbf{Psh}(X)}\mathscr{F}_i.\]
However, since $\iota$ may not commutes with colimits, the colimit of $\mathscr{F}$ should be defined as the sheafification of that in $\mathbf{Psh}(X)$:
\[\rlim_{i,\mathbf{Shv}(X)}\mathscr{F}_i:=(\rlim_{i,\mathbf{Psh}(X)}\mathscr{F}_i)^{\hash}.\]
This completes the proof.
\end{proof}
\begin{proposition}
Let $x\in X$ be a point. Then there are maps
\[(\llim_i\mathscr{F}_i)_x\to\llim_i(\mathscr{F}_i)_x,\quad (\rlim_i\mathscr{F}_i)_x\cong\rlim_i(\mathscr{F}_i)_x\]
for limits and colimits of $($pre$)$sheaves. The second map is always bijective, and the first map becomes an bijection when $\mathcal{I}$ is finite,
\end{proposition}
\begin{proof}
First we consider (co)limit of presheaves. The maps $\mathscr{F}_i(U)\to(\mathscr{F}_i)_x$ yield maps \[\llim_i\mathscr{F}_i(U)\to\llim_i(\mathscr{F}_i)_x\And \rlim_i\mathscr{F}_i(U)\to\rlim_i(\mathscr{F}_i)_x.\] 
Taking the filtered colimit over the open neighborhoods of $x$ we obtain maps
\[(\llim_i\mathscr{F}_i)_x\to\llim_i(\mathscr{F}_i)_x,\quad (\rlim_i\mathscr{F}_i)_x\to\rlim_i(\mathscr{F}_i)_x\]
As filtered colimits commute with finite limits, the first map is an isomorphism if $\mathcal{I}$ is finite. And since colimits commute with each other, the second is always bijective.\par
In the case of sheaves, we need to take sheafification. Since the sheafification does not change stalk, the result is the same.
\end{proof}
\begin{proposition}\label{sheaf cat monomorphism iff}
Suppose $\varphi:\mathscr{F}\to\mathscr{G}$ is a morphism of sheaves of sets on a topological space $X$. Then the following are equivalent.
\begin{itemize}
\item[(a)] $\varphi$ is a monomorphism in the category of sheaves.
\item[(b)] $\varphi$ is injective on the level of stalks: $\varphi_x:\mathscr{F}_x\to\mathscr{G}_x$ is injective for all $x\in X$.
\item[(c)] $\varphi$ is injective on the level of open sets: $\varphi(U):\mathscr{F}(U)\to\mathscr{G}(U)$ is injective for all open $U\sub X$.
\end{itemize}
If these conditions hold, we say that $\mathscr{F}$ is a \textbf{subsheaf} of $\mathscr{G}$.
\end{proposition}
\begin{proof}
The last two statements are equivalent by Proposition~\ref{sheaf morphism prop iff stalk prop}, and the implication (c)$\Rightarrow$(a) is immediate, since a monomorphism in $\mathbf{Psh}(X)$ is clearly monic in $\mathbf{Shv}(X)$.\par 
Now assume that $\varphi$ is a monomorphism, and let $s,t\in\mathscr{F}(U)$ be such that $\varphi_U(s)=\varphi_U(t)$. Then by Example~\ref{sheaf by open subset} there are morphisms $\psi_s,\psi_t:\mathscr{U}^{\hash}\to\mathscr{F}$. From $\varphi_U(s)=\varphi_U(t)$ we see $\varphi\circ\psi_s=\varphi\circ\psi_t$, which implies $\psi_s=\psi_t$. By the isomorphism in Example~\ref{sheaf by open subset}, we conclude $s=t$, so $\varphi_U$ is injective.
\end{proof}
\begin{proposition}\label{sheaf cat epimorphism iff}
Suppose $\varphi:\mathscr{F}\to\mathscr{G}$ is a morphism of sheaves of sets on a topological space $X$. Then the following are equivalent.
\begin{itemize}
\item[(a)] $\varphi$ is a epimorphism in the category of sheaves.
\item[(b)] $\varphi$ is surjective on the level of stalks: $\varphi_x:\mathscr{F}_x\to\mathscr{G}_x$ is injective for all $x\in X$.
\item[(c)] For any open subsets $U\sub X$ and every $t\in\mathscr{G}(U)$ there exist an open covering $U=\bigcup_iU_i$ and sections $s_i\in\mathscr{F}(U_i)$ such that $\varphi_{U_i}(s_i)=t|_{U_i}$, i.e., we can locally find preimage of $t$.
\end{itemize}
If these conditions hold, we say that $\mathscr{G}$ is a \textbf{quotient sheaf} of $\mathscr{F}$.
\end{proposition}
\begin{proof}
If $\varphi_x:\mathscr{F}_x\to\mathscr{G}_x$ is surjective for all $x\in X$, then by Proposition~\ref{sheaf morphism prop iff stalk prop} and the diagram
\[\begin{tikzcd}
\mathscr{F}(U)\ar[r,"\varphi_U"]\ar[d,hook]&\mathscr{G}\ar[r,"\psi_U"]\ar[d,hook]&\mathscr{H}(U)\ar[d,hook]\\
\prod_{x\in U}\mathscr{F}_x\ar[r,"\varphi_x"]&\prod_{x\in U}\mathscr{G}_x\ar[r,"\psi_x"]&\prod_{x\in U}\mathscr{H}_x
\end{tikzcd}\]
we see $\varphi$ is an epimorphism.\par
Now assume that $\varphi$ is an epimorphism, and let $f,g:\mathscr{G}_x\to B$ be two maps such that $f\circ\varphi_x=g\circ\varphi_x$. Then by Proposition~\ref{stalk adj skyscraper} we have induced morphism of sheaves $\psi_f,\psi_g:\mathscr{G}\to i_{x,*}(B)$ such that $\psi_f\circ\varphi=\psi_g\circ\varphi$. Since $\varphi$ is an epic, this implies $f=g$, which means $\varphi_x$ is surjective.\par
Finally, the condition (b) clearly implies (c), and if (c) holds, let $t_x\in\mathscr{G}_x$ with $t_x=(t,V)$ for some $V\sub X$ open and $t\in\mathscr{G}_x$. Then there is a covering $V=\bigcup_iV_i$ and $s_i\in\mathscr{F}(U_i)$ such that $\varphi_{V_i}(s_i)=t|_{V_i}$. Choose a $x\in V_i$, then $\varphi_x\big((s_i)_x\big)=t_x$. Thus $\varphi_x$ is surjective.
\end{proof}
\begin{remark}
The condition for $\varphi$ in Proposition~\ref{sheaf cat epimorphism iff} does not imply that $\varphi_U$ is surjective for all open sets $U$ of $X$ as Example~\ref{sheaf cat epi not surjective on object} shows. In other words, being epic in the category of sheaves is a weaker that begin surjective on objects.
\end{remark}
\begin{example}\label{sheaf cat epi not surjective on object}
Let $\mathcal{O}_X$ be the sheaf of holomorphic functions on an open subset $X$ of $\C$. For every open subspace $U\sub X$ and $f\in\mathcal{O}_X(U)$ we let $D_U(f)=f'$ be the derivative. We obtain a morphism $D:\mathcal{O}_X\to\mathcal{O}_X$ of sheaves of $\C$-vector spaces. Then $D$ is an epimorphism, because locally every holomorphic function has a primitive. But there exist open subsets $U$ of $X$ and functions $f$ on $U$ that have no primitive., for instance $U=\B(z_0)\setminus\{z_0\}\sub\C$ contained in $X$ and $f=1/(z-z_0)$. More precisely, by complex analysis we know that $D_U$ is surjective if and only if every connected component of $U$ is simply connected. Thus $D$ is not surjective.
\end{example}
\section{Sheaf of modules}
\subsection{Presheaf of moudles}
\begin{definition}
Let $X$ be a topological space, and let $\mathscr{O}$ be a presheaf of rings
on $X$.
\begin{itemize}
\item A presheaf of $\mathscr{O}$-modules is given by an abelian presheaf $\mathscr{F}$ together with a map of presheaves of sets
\[\mathscr{O}\times\mathscr{F}\to\mathscr{F}\]
such that for every open $U\sub X$ the map $\mathscr{O}(U)\times\mathscr{F}(U)\to\mathscr{F}(U)$ defines the structure of an $\mathscr{O}(U)$-module structure on the abelian group $\mathscr{F}(U)$.
\item A morphism $\varphi:\mathscr{F}\to\mathscr{G}$ of presheaves of $\mathscr{O}$-modules is a morphism of abelian presheaves $\varphi:\mathscr{F}\to\mathscr{G}$ such that the diagram 
\[\begin{tikzcd}
\mathscr{O}\times\mathscr{F}\ar[r]\ar[d,swap,"\mathbf{1}\times\varphi"]&\mathscr{F}\ar[d,"\varphi"]\\
\mathscr{O}\times\mathscr{G}\ar[r]&\mathscr{G}
\end{tikzcd}\]
commutes.
\item The set of $\mathscr{O}$-module morphisms as above is denoted $\Hom_{\mathscr{O}}(\mathscr{F},\mathscr{G})$. The category of presheaves of $\mathscr{O}$-modules is denoted $\mathbf{PMod}(\mathscr{O})$.
\end{itemize}
\end{definition}
Suppose that $\mathscr{O}_1\to\mathscr{O}_2$ is a morphism of presheaves of rings on $X$. In this case, if $\mathscr{F}$ is a presheaf of $\mathscr{O}_2$-modules then we can think of $\mathscr{F}$ as a presheaf of $\mathscr{O}_2$-modules by using the composition
\[\mathscr{O}_1\times\mathscr{F}\to\mathscr{O}_2\times\mathscr{F}\to\mathscr{F}.\]
We sometimes denote this by $\mathscr{F}_{\mathscr{O}_1}$ to indicate the restriction of rings. We call this the \textbf{restriction of $\mathscr{F}$}. We obtain the restriction functor
\[\mathbf{PMod}(\mathscr{O}_2)\to\mathbf{PMod}(\mathscr{O}_1).\]
On the other hand, given a presheaf of $\mathscr{O}_1$-modules $\mathscr{G}$ we can construct a presheaf of $\mathscr{O}_2$-modules $\mathscr{O}_2\otimes_{p,\mathscr{O}_1}\mathscr{F}$ by the rule
\[(\mathscr{O}_2\otimes_{p,\mathscr{O}_1}\mathscr{F})(U)=\mathscr{O}_2(U)\otimes_{p,\mathscr{O}_1}\mathscr{F}(U).\]
The index $p$ stands for presheaf. This presheaf is called the \textbf{tensor product presheaf}. We obtain the change of rings functor
\[\mathbf{PMod}(\mathscr{O}_1)\to\mathbf{PMod}(\mathscr{O}_2).\]
\begin{proposition}\label{presheaf module restriction}
With $\mathscr{O}_1,\mathscr{O}_2$ and $\mathscr{F},\mathscr{G}$ as above there exists a canonical bijection
\[\Hom_{\mathscr{O}_1}(\mathscr{G},\mathscr{F}_{\mathscr{O}_1})=\Hom_{\mathscr{O}_2}(\mathscr{O}_2\otimes_{p,\mathscr{O}_1}\mathscr{G},\mathscr{F}).\]
In other words, the restriction and change of rings functors are adjoint to each other.
\end{proposition}
\begin{proof}
This follows from the fact that for a ring map $A\to B$ the restriction functor and the change of ring functor are adjoint to each other.
\end{proof}
The stalk of a sheaf of $\mathscr{O}$-module is defined in the same as that of sheaf of sets.
\begin{proposition}
Let $X$ be a topological space. Let $\mathscr{O}$ be a presheaf of rings on $X$. Let $\mathscr{F}$ be a presheaf of $\mathscr{O}$-modules. Let $x\in X$. The canonical map $\mathscr{O}_x\times\mathscr{F}_x\to\mathscr{F}_x$ coming from the multiplication map $\mathscr{O}\times\mathscr{F}\to\mathscr{F}$ defines a $\mathscr{O}_x$-module structure on the abelian group $\mathscr{F}_x$.
\end{proposition}
\begin{proposition}\label{ext module stalk pre}
Let $X$ be a topological space. Let $\mathscr{O}\to\mathscr{O}'$ be a morphism of presheaves of rings on $X$. Let $\mathscr{F}$ be a presheaf of $\mathscr{O}$-modules. Let $x\in X$. We have
\[\mathscr{F}_x\otimes_{\mathscr{O}_x}\mathscr{O}'_x=(\mathscr{F}\otimes_{p,\mathscr{O}}\mathscr{O}')_x.\]
as $\mathscr{O}'_x$-modules.
\end{proposition}
\begin{proof}
Tensor product is left-adjoint, so it commutes with colimit.
\end{proof}
\subsection{Sheaf of moudles}
\begin{definition}
Let $X$ be a topological space. Let $\mathscr{O}$ be a sheaf of rings on $X$.
\begin{itemize}
\item A sheaf of $\mathscr{O}$-modules is a presheaf of $\mathscr{O}$-modules $\mathscr{F}$, such that the underlying presheaf of abelian groups $\mathscr{F}$ is a sheaf.
\item A morphism of sheaves of $\mathscr{O}$-modules is a morphism of presheaves of $\mathscr{O}$-modules.
\item Given sheaves of $\mathscr{O}$-modules $\mathscr{F}$ and $\mathscr{G}$ we denote $\Hom_{\mathscr{O}}(\mathscr{F},\mathscr{G})$ the set of morphism of sheaves of $\mathscr{O}$-modules. The category of sheaves of $\mathscr{O}$-modules is denoted $\mathbf{Mod}(\mathscr{O})$.
\end{itemize}
\end{definition}
\subsection{Sheafification of presheaves of modules}
\begin{proposition}\label{sheafification module}
Let $X$ be a topological space. Let $\mathscr{O}$ be a presheaf of rings on $X$. Let $\mathscr{F}$ be a presheaf $\mathscr{O}$-modules. Let $\mathscr{O}^{\hash}$ be the sheafification of $\mathscr{O}$. Let $\mathscr{F}^{\hash}$ be the sheafification of $\mathscr{F}$ as a presheaf of abelian groups. There exists a map of sheaves of sets
\[\mathscr{O}^{\hash}\times\mathscr{F}^{\hash}\to\mathscr{F}^{\hash}.\]
which makes the diagram
\[\begin{tikzcd}
\mathscr{O}\times\mathscr{F}\ar[r]\ar[d]&\mathscr{F}\ar[d]\\
\mathscr{O}^{\hash}\times\mathscr{F}^{\hash}\ar[r]&\mathscr{F}^{\hash}
\end{tikzcd}\]
commute and which makes $\mathscr{F}^{\hash}$ into a sheaf of $\mathscr{O}^{\hash}$-modules. In addition, if $\mathscr{G}$ is a presheaf of $\mathscr{O}$-modules, then any morphism of presheaves of $\mathscr{O}$-modules $\mathscr{F}\to\mathscr{G}$ induced a unique morphism $\mathscr{F}^{\hash}\to\mathscr{G}^{\hash}$ of sheaf of $\mathscr{O}^{\hash}$-modules.
\end{proposition}
\begin{proof}
Since finite product and coproduct coincide in the category of modules, the sheafification commutes with both of them. Thus we can apply the universal property of sheafification on the map $\mathscr{O}\times\mathscr{F}\to\mathscr{F}$ to get a map
\[\mathscr{O}^{\hash}\times\mathscr{F}^{\hash}\to\mathscr{F}^{\hash}\]
Moreover, for a morphism of $\mathscr{O}$-modules $\mathscr{F}\to\mathscr{G}$, apply sheafification on the diagram
\[\begin{tikzcd}
\mathscr{O}\times\mathscr{F}\ar[r]\ar[d]&\mathscr{F}\ar[d]\\
\mathscr{O}\times\mathscr{G}\ar[r]&\mathscr{G}
\end{tikzcd}\]
we get the desired morphism.
\end{proof}
This actually means that the functor $\iota:\mathbf{Mod}(\mathscr{O}^{\hash})\to\mathbf{PMod}(\mathscr{O})$ and the sheafification functor of the lemma $\#:\mathbf{PMod}(\mathscr{O})\to\mathbf{Mod}(\mathscr{O}^{\hash})$ are adjoint. In a formula
\[\Hom_{\mathbf{Mod}(\mathscr{O}^{\hash})}(\mathscr{F}^{\hash},\mathscr{G})=\Hom_{\mathbf{PMod}(\mathscr{O})}(\mathscr{F},\iota\mathscr{G}).\]
Let $X$ be a topological space. Let $\mathscr{O}_1\to\mathscr{O}_2$ be a morphism of sheaves of rings on $X$. We defined a restriction functor and a change of rings functor on presheaves of modules associated to this situation. If $\mathscr{F}$ is a sheaf of $\mathscr{O}_2$-modules then the restriction $\mathscr{F}_{\mathscr{O}_2}$ of $\mathscr{F}$ is clearly a sheaf of $\mathscr{O}_1$-modules. We obtain the restriction functor
\[\mathbf{Mod}(\mathscr{O}_2)\to\mathbf{Mod}(\mathscr{O}_1).\]
On the other hand, given a sheaf of $\mathscr{O}_1$-modules $\mathscr{G}$ the presheaf of $\mathscr{O}_2$-modules $\mathscr{O}_2\otimes_{p,\mathscr{O}_1}\mathscr{G}$ is in general not a sheaf. Hence we define the tensor product sheaf $\mathscr{O}_2\otimes_{\mathscr{O}_1}\mathscr{G}$ by the formula
\[\mathscr{O}_2\otimes_{\mathscr{O}_1}\mathscr{G}=(\mathscr{O}_2\otimes_{p,\mathscr{O}_1}\mathscr{G})^{\hash}.\]
\begin{proposition}\label{sheaf module restrict adj}
With $X$, $\mathscr{O}_1,\mathscr{O}_2$, $\mathscr{F}$ and $\mathscr{G}$ as above there exists a canonical bijection
\[\Hom_{\mathscr{O}_1}(\mathscr{G},\mathscr{F}_{\mathscr{O}_1})=\Hom_{\mathscr{O}_2}(\mathscr{O}_2\otimes_{\mathscr{O}_1}\mathscr{G},\mathscr{F}).\]
In other words, the restriction and change of rings functors are adjoint to each other.
\end{proposition}
\begin{proof}
This follows from Proposition~\ref{presheaf module restriction} and the fact that \[\Hom_{\mathscr{O}_2}(\mathscr{O}_2\otimes_{\mathscr{O}_1}\mathscr{G},\mathscr{F})=\Hom_{\mathscr{O}_2}(\mathscr{O}_2\otimes_{p,\mathscr{O}_1}\mathscr{G},\mathscr{F},\mathscr{F})\]
by the property of sheafification.
\end{proof}
\begin{proposition}\label{ext module stalk}
Let $X$ be a topological space. Let $\mathscr{O}\to\mathscr{O}'$ be a morphism of sheaves of rings on $X$. Let $\mathscr{F}$ be a sheaf $\mathscr{O}$-modules. Let $x\in X$. We have
\[\mathscr{F}_x\otimes_{\mathscr{O}_x}\mathscr{O}'_x=(\mathscr{F}\otimes_{\mathscr{O}}\mathscr{O}')_x.\]
\end{proposition}
\subsection{Continuous maps}
The case of sheaves of modules is more complicated. First we state a few obvious lemmas.
\begin{lemma}\label{push presheaf module def}
Let $f:X\to Y$ be a continuous map of topological spaces. Let $\mathscr{O}$ be a presheaf of rings on $X$. Let $\mathscr{F}$ be a presheaf of $\mathscr{O}$-modules. There is a natural map of underlying presheaves of sets
\[f_*\mathscr{O}\times f_*\mathscr{F}\to f_*\mathscr{F}\]
which turns $f_*\mathscr{F}$ into a presheaf of $f_*\mathscr{O}$-modules. This construction is functorial in $\mathscr{F}$.
\end{lemma}
\begin{proof}
Let $V\sub Y$ is open. We define the map of the lemma to be the map
\[f_*\mathscr{O}(V)\times f_*\mathscr{F}(V)=\mathscr{O}(f^{-1}(V))\times\mathscr{F}(f^{-1}(V))\to\mathscr{F}(f^{-1}(V))=f_*\mathscr{F}(V).\]
Here the arrow in the middle is the multiplication map on $X$.
\end{proof}
\begin{lemma}\label{pull back presheaf module def}
Let $f:X\to Y$ be a continuous map of topological spaces. Let $\mathscr{O}$ be a presheaf of rings on $Y$. Let $G$ be a presheaf of $\mathscr{O}$-modules. There is a natural map of underlying presheaves of sets
\[f^p\mathscr{O}\times f^p\mathscr{G}\to\mathscr{G}\]
which turns $f^p\mathscr{G}$ into a presheaf of $f^p\mathscr{O}$-modules. This construction is functorial in $\mathscr{G}$.
\end{lemma}
\begin{proof}
Let $U\sub X$ is open. We define the map of the lemma to be the map
\begin{align*}
f^p\mathscr{O}(U)\times f^p\mathscr{G}(U)&=\rlim_{f(U)\sub V}\mathscr{O}(V)\times\rlim_{f(U)\sub V}\mathscr{G}(V)=\rlim_{f(U)\sub V}\big(\mathscr{O}(V)\times\mathscr{G}(V)\big)\\
&\to\rlim_{f(U)\sub V}\mathscr{G}(V)=f^p\mathscr{G}(V)
\end{align*}
Here the arrow in the middle is the multiplication map on $Y$. The second equality holds because directed colimits commute with finite limits.
\end{proof}
Let $f:X\to Y$ be a continuous map. Let $\mathscr{O}_X$ be a presheaf of rings on $X$ and let $\mathscr{O}_Y$ be a presheaf of rings on $Y$. So at the moment we have defined functors
\[f_*:\mathbf{PMod}(\mathscr{O}_X)\to\mathbf{PMod}(f_*\mathscr{O}_X),\quad f^p:\mathbf{PMod}(\mathscr{O}_Y)\to\mathbf{PMod}(f^p\mathscr{O}_Y).\]
These satisfy some compatibilities as follows.
\begin{proposition}\label{pull back presheaf module adj}
Let $f:X\to Y$ be a continuous map of topological spaces. Let $\mathscr{O}$ be a presheaf of rings on $Y$. Let $\mathscr{G}$ be a presheaf of $\mathscr{O}$-modules. Let $\mathscr{F}$ be a presheaf of $f^p\mathscr{O}$-modules. Then
\[\Hom_{\mathbf{PMod}(f^p\mathscr{O})}(f^p\mathscr{G},\mathscr{F})=\Hom_{\mathbf{PMod}(\mathscr{O})}(\mathscr{G},f_*\mathscr{F})\]
Here we think of $f_*\mathscr{F}$ as an $\mathscr{O}$-module via the map $\rho_{\mathscr{O}}:\mathscr{O}\to f_*f^p\mathscr{O}$.
\end{proposition}
\begin{proof}
Note that we have
\[\Mor_{\mathbf{PAb}(X)}(f^p\mathscr{G},\mathscr{F})=\Mor_{\mathbf{PAb}(Y)}(\mathscr{G},f_*\mathscr{F}).\]
So what we have to prove is that under this correspondence, the subsets of module maps correspond. In addition, the correspondence is determined by the rule
\[(\psi:f^p\mathscr{G}\to\mathscr{F})\to(f_*\psi\circ\rho_{\mathscr{G}}:\mathscr{G}\to f_*\mathscr{F})\]
and in the other direction by the rule
\[(\varphi:\mathscr{G}\to f_*\mathscr{F})\mapsto(\sigma_{\mathscr{F}}\circ f^p\varphi:f^p\mathscr{F}\to\mathscr{G})\]
Hence, using the functoriality of $f_*$ and $f^p$ we see that it suffices to check that the maps $\rho_{\mathscr{G}}:\mathscr{G}\to f_*f^p\mathscr{G}$ and $\sigma_{\mathscr{F}}:f^pf_*\mathscr{F}\to\mathscr{F}$ are compatible with module structures, which can be done by tracing definitions.
\end{proof}
\begin{proposition}\label{pushforward presheaf module adj}
Let $f:X\to Y$ be a continuous map of topological spaces. Let $\mathscr{O}$ be a presheaf of rings on $X$. Let $\mathscr{F}$ be a presheaf of $\mathscr{O}$-modules. Let $\mathscr{G}$ be a presheaf of $f_*\mathscr{O}$-modules. Then
\[\Hom_{\mathbf{PMod}(\mathscr{O})}(\mathscr{O}\otimes_{p,f^pf_*\mathscr{O}}f^p\mathscr{G},\mathscr{F})=\Hom_{\mathbf{PMod}(f_*\mathscr{O})}(\mathscr{G},f_*\mathscr{F}).\]
Here we use the map $\sigma_{\mathscr{O}}:f^pf_*\mathscr{O}\to\mathscr{O}$ in
the definition of the tensor product.
\end{proposition}
\begin{proof}
This follows from the equalities
\begin{align*}
\Hom_{\mathbf{PMod}(\mathscr{O})}(\mathscr{O}\otimes_{p,f^pf_*\mathscr{O}}f^p\mathscr{G},\mathscr{F})&=\Hom_{\mathbf{PMod}(f^pf_*\mathscr{O})}(f^p\mathscr{G},\mathscr{F}_{f^pf_*\mathscr{O}})\\
&=\Hom_{\mathbf{PMod}(f_*\mathscr{O})}\big(\mathscr{G},f_*(\mathscr{F}_{f^pf_*\mathscr{O}})\big)\\
&=\Hom_{\mathbf{PMod}(f_*\mathscr{O})}\big(\mathscr{G},f_*\mathscr{F}).
\end{align*}
For the third equality, note that $\mathbf{1}_{f_*\mathscr{O}}$ corresponds to $\sigma_{\mathscr{O}}$ under the adjunction described in the proof of Proposition~\ref{pull back presheaf module adj} and thus we have the equality $\mathbf{1}_{f_*\mathscr{O}}=f_*\sigma_{\mathscr{O}}\circ\rho_{f_*\mathscr{O}}$. Now consider the module structures:
\[\begin{tikzcd}[row sep=small]
\mathscr{F}_{f^pf_*\mathscr{O}}:&f^pf_*\mathscr{O}\times\mathscr{F}\ar[r,"\sigma_{\mathscr{O}}\times\mathbf{1}"]&\mathscr{O}\times\mathscr{F}\ar[r]&\mathscr{F}\\
f_*(\mathscr{F}_{f^pf_*\mathscr{O}}):&f_*\mathscr{O}\times f_*\mathscr{F}\ar[r,"\rho_{f_*\mathscr{O}}\times\mathbf{1}"]&f_*f^pf_*\mathscr{O}\times f_*\mathscr{F}\ar[r,"f_*\sigma_{\mathscr{O}}\times\mathbf{1}"]&f_*\mathscr{O}\times f_*\mathscr{F}\ar[r]&f_*\mathscr{F}
\end{tikzcd}\]
we conclude that $f_*(\mathscr{F}_{f^pf_*\mathscr{O}})=f_*\mathscr{F}$.
\end{proof}
Now we consider the case of sheaves.
\begin{lemma}\label{pushforward sheaf module def}
Let $f:X\to Y$ be a continuous map of topological spaces. Let $\mathscr{O}$ be a sheaf of rings on $X$. Let $\mathscr{F}$ be a sheaf of $\mathscr{O}$-modules. The pushforward $f_*\mathscr{F}$ is a sheaf of $f_*\mathscr{O}$-modules.
\end{lemma}
\begin{lemma}\label{pull back module def}
Let $f:X\to Y$ be a continuous map of topological spaces. Let $\mathscr{O}$ be a sheaf of rings on $Y$. Let $\mathscr{G}$ be a sheaf of $\mathscr{O}$-modules. There is a natural map of underlying presheaves of sets
\[f^{-1}\mathscr{O}\times f^{-1}\mathscr{G}\to f^{-1}\mathscr{G}\]
which turns $f^{-1}\mathscr{G}$ into a sheaf of $f^{-1}\mathscr{O}$-modules.
\end{lemma}
\begin{proof}
Recall that $f^{-1}$ is defined as the composition of the functor $f^p$ and sheafification. Thus the lemma is a combination of Lemma~\ref{pull back presheaf module def} and Proposition~\ref{sheafification module}.
\end{proof}
Let $f:X\to Y$ be a continuous map. Let $\mathscr{O}_X$ be a sheaf of rings on $X$ and let $\mathscr{O}_Y$ be a sheaf of rings on $Y$. So now we have defined functors
\[f_*:\mathbf{Mod}(\mathscr{O}_X)\to\mathbf{Mod}(f_*\mathscr{O}_X),\quad f^{-1}:\mathbf{Mod}(\mathscr{O}_Y)\to\mathbf{Mod}(f^{-1}\mathscr{O}_Y).\]
These satisfy some compatibilities as follows.
\begin{proposition}\label{pull back module sheaf adj}
Let $f:X\to Y$ be a continuous map of topological spaces. Let $\mathscr{O}$ be a sheaf of rings on $Y$. Let $\mathscr{G}$ be a sheaf of $\mathscr{O}$-modules. Let $\mathscr{F}$ be a sheaf of $f^{-1}\mathscr{O}$-modules. Then
\[\Hom_{\mathbf{Mod}(f^{-1}\mathscr{O})}(f^{-1}\mathscr{G},\mathscr{F})=\Hom_{\mathbf{Mod}(\mathscr{O})}(\mathscr{G},f_*\mathscr{F}).\]
Here we think of $f_*\mathscr{F}$ as an $\mathscr{O}$-module by restriction via $\mathscr{O}\to f_*f^{-1}\mathscr{O}$.
\end{proposition}
\begin{proof}
Argue by the equalities
\[\Hom_{\mathbf{Mod}(f^{-1}\mathscr{O})}(f^{-1}\mathscr{G},\mathscr{F})=\Hom_{\mathbf{Mod}(f^{p}\mathscr{O})}(f^{p}\mathscr{G},\mathscr{F})=\Hom_{\mathbf{Mod}(\mathscr{O})}(\mathscr{G},f_*\mathscr{F}).\]
which holds by the property of sheafification.
\end{proof}
\begin{proposition}\label{push module sheaf adj}
Let $f:X\to Y$ be a continuous map of topological spaces. Let $\mathscr{O}$ be a sheaf of rings on $X$. Let $\mathscr{F}$ be a sheaf of $\mathscr{O}$-modules. Let $\mathscr{G}$ be a sheaf of $f_*\mathscr{O}$-modules. Then
\[\Hom_{\mathbf{Mod}(\mathscr{O})}(\mathscr{O}\otimes_{f^{-1}f_*\mathscr{O}}f^{-1}\mathscr{G},\mathscr{F})=\Hom_{\mathbf{Mod}(f_*\mathscr{O})}(\mathscr{G},f_*\mathscr{F}).\]
Here we use the canonical map $f^{-1}f_*\mathscr{O}\to\mathscr{O}$ in the definition of the tensor product.
\end{proposition}
\begin{proof}
This follows from the equalities
\begin{align*}
\Hom_{\mathbf{Mod}(\mathscr{O})}(\mathscr{O}\otimes_{f^{-1}f_*\mathscr{O}}f^{-1}\mathscr{G},\mathscr{F})&=\Hom_{\mathbf{Mod}(f^{-1}f_*\mathscr{O})}(f^{-1}\mathscr{G},\mathscr{F}_{f^{-1}f_*\mathscr{O}})\\
&=\Hom_{\mathbf{Mod}(f_*\mathscr{O})}\big(\mathscr{G},f_*(\mathscr{F}_{f^{-1}f_*\mathscr{O}})\big)\\
&=\Hom_{\mathbf{Mod}(f_*\mathscr{O})}(\mathscr{G},f_*\mathscr{F}).
\end{align*}
where the last equality is obtained in Proposition~\ref{pushforward presheaf module adj}.
\end{proof}
\subsection{Supports of modules and sections}
\begin{definition}
Let $(X,\mathscr{O}_X)$ be a ringed space. Let $\mathscr{F}$ be a sheaf of $\mathscr{O}_X$-modules.
\begin{itemize}
\item The support of $\mathscr{F}$ is the set of points $x\in X$ such that $\mathscr{F}_x=0$. We denote it by $\supp(\mathscr{F})$.
\item Let $s\in\Gamma(X,\mathscr{F})$ be a global section. The support of $s$ is the set of points $x\in X$ such that the image $s_x\in\mathscr{F}_x$ of $s$ is not zero.
\end{itemize}
\end{definition}
Of course the support of a local section is then defined also since a local section is a global section of the restriction of $\mathscr{F}$.
\begin{lemma}
Let $(X,\mathscr{O}_X)$ be a ringed space. Let $\mathscr{F}$ be a sheaf of $\mathscr{O}_X$-modules. Let $U\sub X$ open.
\begin{itemize}
\item The support of $s\in\mathscr{F}(U)$ is closed in $U$.
\item The support of $fs$ is contained in the intersections of the supports of $f\in\mathscr{O}_X(U)$ and $s\in\mathscr{F}(U)$.
\item The support of $s+s'$ is contained in the union of the supports of $s,s'\in\mathscr{F}(U)$.
\item The support of $\mathscr{F}$ is the union of the supports of all local sections of $\mathscr{F}$.
\item If $\varphi:\mathscr{F}\to\mathscr{G}$ is a morphism of $\mathscr{O}_X$-modules, then the support of $\varphi(s)$ is contained in the support of $s\in\mathscr{F}(U)$. 
\end{itemize}
\end{lemma}
In general the support of a sheaf of modules is not closed. Namely, the sheaf could be an abelian sheaf on $\R$ (with the usual archimedean topology) which is the direct sum of infinitely many nonzero skyscraper sheaves each supported at a single point $p_i$ of $\R$. Then the support would be the set of points pi which may not be closed.\par
Another example is to consider the open immersion $j:U=(0,\infty)\to\R=X$,
and the abelian sheaf $j_!\Z_U$. By Proposition~\ref{sheaf open extension functor prop} the support of this sheaf is exactly $U$.
\begin{lemma}\label{sheaf ring supp closed}
Let $X$ be a topological space. The support of a sheaf of rings is closed.
\end{lemma}
\begin{proof}
This is true because a ring is $0$ if and only if $1=0$, and hence the support of a sheaf of rings is the support of the unit section.
\end{proof}
\subsection{Modules generated by sections}
Let $(X,\mathscr{O}_X)$ be a ringed space. Let $\mathscr{F}$ be a sheaf of $\mathscr{O}_X$-modules. Then there is an canonical identification $\Hom_{\mathscr{O}_X}(\mathscr{O}_{X},\mathscr{F})=\Gamma(X,\mathscr{F})$ which associate a global section $s\in\Gamma(X,\mathscr{F})$ with the unique homomorphism of $\mathscr{O}_X$-modules $\mathscr{O}_X\to\mathscr{F},f\mapsto fs$. That is, a local section $f$ of $\mathscr{O}_X$, i.e., a section $f$ over some open $U$, is mapped to the multiplication of $f$ with the restriction of $s$ to $U$. We say that $\mathscr{F}$ is \textbf{generated by global sections} if there exist a set $I$ and global sections $s_i\in\Gamma(X,\mathscr{F})$, $i\in I$ such that the homomorphism
\[\mathscr{O}_X^{\oplus I}\to\mathscr{F}\]
which is the homomorphism associated to $s_i$ on the summand corresponding to $i$, is surjective. In this case we say that the sections $s_i$ \textbf{generate} $\mathscr{F}$.
\begin{proposition}\label{sheaf of module global section generate iff}
Let $(X,\mathscr{O}_X)$ be a ringed space and $\mathscr{F}$ be a sheaf of $\mathscr{O}_X$-modules. Let $(s_i)_{i\in I}$ be a family of global sections of $\mathscr{F}$. Then the sections $s_i$ generate $\mathscr{F}$ if and only if for any point $x\in X$ the elements $s_{i,x}\in\mathscr{F}_x$ generate the $\mathscr{O}_{X,x}$-module $\mathscr{F}_x$.
\end{proposition}
\begin{proof}
The homomorphism $\mathscr{O}_X^{\oplus I}\to\mathscr{F}$ is surjective if and only if for each point $x\in X$ the homomorphism $(\mathscr{O}_{X}^{\oplus I})_x\to\mathscr{F}_x$ is surjective. Since taking stalk commutes with colimit, we have $(\mathscr{O}_X)^{\oplus I}_x\mathscr{O}_{X,x}^{\oplus I}$, which implies the claim.
\end{proof}
\begin{proposition}\label{sheaf of module local section generate subsheaf}
Let $(X,\mathscr{O}_X)$ be a ringed space and $\mathscr{F}$ be a sheaf of $\mathscr{O}_X$-modules. Let $(s_i)_{i\in I}$ be a collection of local sections of $\mathscr{F}$, i.e., $s_i\in\Gamma(U_i,\mathscr{F})$ for some open subset $U_i$ of $X$. Then there exists a unique smallest sub-$\mathscr{O}_X$-module $\mathscr{G}$ of $\mathscr{F}$ such that each $s_i$ corresponds to a local section of $\mathscr{G}$, which is called the \textbf{sub-$\mathscr{O}_X$-module generated by the $\bm{s_i}$}.
\end{proposition}
\begin{proof}
Consider the subpresheaf of $\mathscr{F}$ defined by the rule
\[U\mapsto\{\sum_{i\in J}f_i(s_i|_U):\text{$J\sub I$ is finite, $U\sub U_i$ for every $i\in J$ and $f_i\in\Gamma(U,\mathscr{O}_X)$}\}.\]
Let $\mathscr{G}$ be the sheafification of this subpresheaf. This is a subsheaf of $\mathscr{F}$ by Proposition~\ref{sheaf sheafification on morphism prop}. Since all the finite sums clearly have to be in $\mathscr{F}$ this is the smallest subsheaf as desired.
\end{proof}
\begin{proposition}
Let $(X,\mathscr{O}_X)$ be a ringed space and $\mathscr{F}$ be a sheaf of $\mathscr{O}_X$-modules. Let $(s_i)_{i\in I}$ be a family of local sections of $\mathscr{F}$ and $\mathscr{G}$ be the subsheaf generated by the $s_i$ and let $x\in X$. Then $\mathscr{G}_x$ is the $\mathscr{O}_{X,x}$-submodule of $\mathscr{F}_x$ generated by the elements $s_{i,x}$ for those $i$ such that $s_i$ is defined at $x$.
\end{proposition}
\begin{proof}
This is clear from the construction of $\mathscr{G}$ in the proof of Proposition~\ref{sheaf of module local section generate subsheaf}.
\end{proof}
\begin{example}\label{sheaf of module not generated by global section eg}
Consider the open immersion $j:U=(0,\infty)\to\R=X$, and the abelian sheaf $j_!(\Z_U)$. By Proposition~\ref{sheaf open extension functor prop} the stalk of $j_!(\Z_U)$ at $x=0$ is $0$. In fact the sections of this sheaf over any open interval containing $0$ are $0$. Thus there is no open neighbourhood of the point $0$ over which the sheaf can be generated by global sections.
\end{example}
Let $(X,\mathscr{O}_X)$ be a ringed space. Let $\mathscr{F}$ be a sheaf of $\mathscr{O}_X$-modules. We say that $\mathscr{F}$ is \textbf{locally generated by sections} if for every $x\in X$ there exists an open neighbourhood $U$ such that $\mathscr{F}|_U$ is globally generated as a sheaf of $\mathscr{O}_U$-modules. In other words there exists a set $I$ and for each $i$ a section $s_i\in\mathscr{F}(U)$ such that the associated map
\[\bigoplus_{i\in I}\mathscr{O}_U\to\mathscr{F}|_U\]
is surjective.
\begin{proposition}\label{pull back local generated section}
Let $(f,f^{\hash}):(X,\mathscr{O}_X)\to(Y,\mathscr{O}_Y)$ be a morphism of ringed spaces. Let $\mathscr{G}$ be a $\mathscr{O}_Y$-module. The pullback $f^*\mathscr{G}$ is locally generated by sections if $\mathscr{G}$ is locally generated by sections.
\end{proposition}
\begin{proof}
Given an open subspace $V$ of $Y$ we may consider the commutative diagram
of ringed spaces
\[\begin{tikzcd}
(f^{-1}(V),\mathscr{O}_{f^{-1}(V)})\ar[r,"i"]\ar[d,"\tilde{f}"]&(X,\mathscr{O}_X)\ar[d,"f"]\\
(V,\mathscr{O}_V)\ar[r,"j"]&(Y,\mathscr{O}_Y)
\end{tikzcd}\]
We know that $(f^*\mathscr{G})|_{f^{-1}(V)}\cong(\tilde{f})^*(\mathscr{G}|_V)$ by Proposition~\ref{pull push composition}. Thus we may assume that $\mathscr{G}$ is globally generated. We have seen that $f^*$ commutes with all colimits, and is right exact. Thus if we have a surjection $\mathscr{O}_Y^{\oplus I}\to\mathscr{G}\to 0$, then upon applying $f^*$ we obtain the surjection $\mathscr{O}_X^{\oplus I}\to f^*\mathscr{G}\to 0$, where we use the observation that
\[f^*\mathscr{O}_Y=\mathscr{O}_X\otimes_{f^{-1}\mathscr{O}_Y}f^{-1}\mathscr{O}_Y=\mathscr{O}_X.\]
This implies the assertion.
\end{proof}
\subsection{Tensor product}
Let $(X,\mathscr{O}_X)$ be a ringed space. Let $\mathscr{F},\mathscr{G}$ be $\mathscr{O}_X$-modules. We define first the tensor product presheaf
\[\mathscr{F}\otimes_{p,\mathscr{O}_X}\mathscr{G}\]
as the rule which assigns to $U\sub X$ open the $\mathscr{O}_X(U)$-module $\mathscr{F}(U)_{\mathscr{O}_X(U)}\mathscr{G}(U)$. Having defined this we dene the tensor product sheaf as the sheafification of the above:
\[\mathscr{F}\otimes_{\mathscr{O}_X}\mathscr{G}=(\mathscr{F}\otimes_{p,\mathscr{O}_X}\mathscr{G})^{\hash}\]
This can be characterized as the sheaf of $\mathscr{O}_X$-modules such that for any third sheaf of $\mathscr{O}_X$-modules $\mathscr{H}$ we have
\[\Hom_{\mathscr{O}_X}(\mathscr{F}\otimes_{\mathscr{O}_X}\mathscr{G},\mathscr{H})=\mathrm{Bilin}_{\mathscr{O}_X}(\mathscr{F}\times\mathscr{G},\mathscr{H}).\]
Here the right hand side indicates the set of bilinear maps of sheaves of $\mathscr{O}_X$-modules.\par
The tensor product of modules $M,N$ over a ring $R$ satisfies symmetry, hence the same holds for tensor products of sheaves of modules, i.e., we have
\[\mathscr{F}\otimes_{\mathscr{O}_X}\mathscr{G}=\mathscr{G}\otimes_{\mathscr{O}_X}\mathscr{F}\]
functorial in $\mathscr{F}$, $\mathscr{G}$. And since tensor product of modules satisfies associativity we also get canonical functorial isomorphisms
\[(\mathscr{F}\otimes_{\mathscr{O}_X}\mathscr{G})\otimes_{\mathscr{O}_X}\mathscr{H}=\mathscr{F}\otimes_{\mathscr{O}_X}(\mathscr{G}\otimes_{\mathscr{O}_X}\mathscr{H}).\]
functorial in $\mathscr{F},\mathscr{G}$, and $\mathscr{H}$.
\begin{proposition}
Let $(X,\mathscr{O}_X)$ be a ringed space. Let $\mathscr{F}$, $\mathscr{G}$ be $\mathscr{O}_X$-modules. Let $x\in X$. There is a canonical isomorphism of $\mathscr{O}_{X,x}$-modules
\[(\mathscr{F}\otimes_{\mathscr{O}_{X}}\mathscr{G})_x=\mathscr{F}_x\otimes_{\mathscr{O}_{X,x}}\mathscr{G}_x.\]
functorial in $\mathscr{F}$ and $\mathscr{G}$.
\end{proposition}
\begin{proposition}
Let $(X,\mathscr{O}_X)$ be a ringed space. Let $\mathscr{F}',\mathscr{G}'$ be presheaves of $\mathscr{O}_X$-modules with sheafifications $\mathscr{F},\mathscr{G}$. Then $\mathscr{F}\otimes_{\mathscr{O}_{X}}\mathscr{G}=(\mathscr{F}'\otimes_{p,\mathscr{O}_{X}}\mathscr{G}')^{\hash}$.
\end{proposition}
\begin{proof}
On stalks we have 
\[(\mathscr{F}\otimes_{\mathscr{O}_X}\mathscr{G})_x=\mathscr{F}_x\otimes_{\mathscr{O}_{X,x}}\mathscr{G}_x=\mathscr{F}'_x\otimes_{\mathscr{O}_{X,x}}\mathscr{G}'_x=\mathscr{F}'_x\otimes_{p,\mathscr{O}_{X,x}}\mathscr{G}'_x\]
Thus by Proposition~\ref{sheaf iso to sheafification iff stalk iso} we conclude the result.
\end{proof}
\begin{proposition}~\label{sheaf module tensor exact}
Let $(X,\mathscr{O}_X)$ be a ringed space. Let $\mathscr{G}$ be an $\mathscr{O}_X$-module. If $\mathscr{F}_1\to\mathscr{F}_2\to\mathscr{F}_3\to 0$ is an exact sequence of $\mathscr{O}_X$-modules then the induced sequence
\[\mathscr{F}_1\otimes_{\mathscr{O}_X}\mathscr{G}\to\mathscr{F}_2\otimes_{\mathscr{O}_X}\mathscr{G}\to\mathscr{F}_3\to 0\]
is exact.
\end{proposition}
\begin{proof}
This follows from the fact that exactness may be checked at stalks, the description of stalks and the corresponding result for tensor products of modules
\end{proof}
\begin{proposition}\label{sheaf tensor pull back}
Let $f:X\to Y$ be a morphism of ringed spaces. Let $\mathscr{F},\mathscr{G}$ be $\mathscr{O}_Y$-modules. Then $f^*(\mathscr{F}\otimes_{\mathscr{O}_Y}\mathscr{G})=f^*\mathscr{F}\otimes_{\mathscr{O}_X}f^*\mathscr{G}$ functorially in $\mathscr{F},\mathscr{G}$.
\end{proposition}
\begin{proof}
Let $x\in X$, we check that
\begin{align*}
\big(f^*(\mathscr{F}\otimes_{\mathscr{O}_Y}\mathscr{G})\big)_x&=\mathscr{O}_{X,x}\otimes_{\mathscr{O}_{Y,f(x)}}(\mathscr{F}\otimes_{\mathscr{O}_Y}\mathscr{G})_{f(x)}=\mathscr{O}_{X,x}\otimes_{\mathscr{O}_{Y,f(x)}}(\mathscr{F}_{f(x)}\otimes_{\mathscr{O}_{Y,f(x)}}\mathscr{G}_{f(x)})\\
&=\mathscr{O}_{X,x}\otimes_{\mathscr{O}_{X,x}}\mathscr{O}_{X,x}\otimes_{\mathscr{O}_{Y,f(x)}}\mathscr{F}_{f(x)}\otimes_{\mathscr{O}_{Y,f(x)}}\mathscr{G}_{f(x)}\\
&=\mathscr{O}_{X,x}\otimes_{\mathscr{O}_{Y,f(x)}}\mathscr{F}_{f(x)}\otimes_{\mathscr{O}_{X,x}}\mathscr{O}_{X,x}\otimes_{\mathscr{O}_{Y,f(x)}}\mathscr{G}_{f(x)}\\
&=(f^*\mathscr{F})_x\otimes_{X,x}(f^*\mathscr{G})_x.
\end{align*}
as desired.
\end{proof}
\begin{proposition}
Let $(X,\mathscr{O}_X)$ be a ringed space. Let $\mathscr{F},\mathscr{G}$ be $\mathscr{O}_X$-modules.
\begin{itemize}
\item[(\rmnum{1})] If $\mathscr{F},\mathscr{G}$ are locally generated by sections, so is $\mathscr{F}\otimes_{\mathscr{O}_X}\mathscr{G}$.
\item[(\rmnum{2})] If $\mathscr{F},\mathscr{G}$ are of finite type, so is $\mathscr{F}\otimes_{\mathscr{O}_X}\mathscr{G}$.
\item[(\rmnum{3})] If $\mathscr{F},\mathscr{G}$ are quasi-coherent, so is $\mathscr{F}\otimes_{\mathscr{O}_X}\mathscr{G}$.
\item[(\rmnum{4})] If $\mathscr{F},\mathscr{G}$ are of finite presentation, so is $\mathscr{F}\otimes_{\mathscr{O}_X}\mathscr{G}$.
\item[(\rmnum{5})] If $\mathscr{F}$ is of finite presentation and $\mathscr{G}$ is coherent, then $\mathscr{F}\otimes_{\mathscr{O}_X}\mathscr{G}$ is coherent.
\item[(\rmnum{6})] If $\mathscr{F},\mathscr{G}$ are coherent, so is $\mathscr{F}\otimes_{\mathscr{O}_X}\mathscr{G}$.
\item[(\rmnum{7})] If $\mathscr{F},\mathscr{G}$ are locally free, so is $\mathscr{F}\otimes_{\mathscr{O}_X}\mathscr{G}$.
\end{itemize}
\end{proposition}
\begin{proof}
We first prove that the tensor product of locally free $\mathscr{O}_X$-modules is locally free. This follows if we show that \[(\bigoplus_{i\in I}\mathscr{O}_X)\otimes_{\mathscr{O}_X}(\bigoplus_{j\in J}\mathscr{O}_X)\cong\bigoplus_{(i,j)\in I\times J}\mathscr{O}_X.\]
The sheaf $\bigoplus_{i\in I}\mathscr{O}_X$ is the sheaf associated to the presheaf $U\mapsto\bigoplus_{i\in I}\mathscr{O}_X(U)$. Hence the tensor product is the sheaf associated to the presheaf
\[U\mapsto\Big(\bigoplus_{i\in I}\mathscr{O}_X(U)\Big)\otimes_{\mathscr{O}_X(U)}\Big(\bigoplus_{j\in J}\mathscr{O}_X(U)\Big).\]
We deduce what we want since for any ring $R$ we have $(\bigoplus_{i\in I}R)\otimes_{R}(\bigoplus_{j\in J}R)\cong\bigoplus_{(i,j)\in I\times J}R$.\par
If $\mathscr{F}_2\to\mathscr{F}_1\to\mathscr{F}\to0$ is exact, then by Proposition~\ref{sheaf module tensor exact} the complex $\mathscr{F}_2\otimes\mathscr{G}\to\mathscr{F}_1\otimes\mathscr{G}\to\mathscr{F}\otimes\mathscr{G}\to 0$ is exact. Using this we can prove (\rmnum{5}). Namely, in this case there exists locally such an exact sequence with $\mathscr{F}_i,i=1,2$ finite free. Hence the two terms $\mathscr{F}_i\otimes\mathscr{G}$ are isomorphic to finite direct sums of $\mathscr{G}$. Since finite direct sums are coherent sheaves, these are coherent and so is the cokernel of the map.\par
If we also have another exact sequence $\mathscr{G}_2\to\mathscr{G}_1\to\mathscr{G}\to0$, then tensoring together we get an exact sequence
\[\begin{tikzcd}
(\mathscr{F}_2\otimes\mathscr{G}_1)\oplus(\mathscr{F}_1\otimes\mathscr{G}_2)\ar[r]&\mathscr{F}_1\otimes\mathscr{G}_1\ar[r]&\mathscr{F}\otimes\mathscr{G}\ar[r]&0
\end{tikzcd}\]
This can be used to prove $(\rmnum{1}),(\rmnum{2}),(\rmnum{3}),(\rmnum{4}),(\rmnum{6})$.
\end{proof}
\begin{proposition}\label{sheaf module tensor colim}
Let $(X,\mathscr{O}_X)$ be a ringed space. For any $\mathscr{O}_X$-module $\mathscr{F}$ the functor $\mathscr{F}\otimes_{\mathscr{O}_X}$ commutes with arbitrary colimits.
\end{proposition}
\begin{proof}
Let $I$ be a partially ordered set and let $\{\mathscr{G}_i\}$ be a system over $I$. Set $\mathscr{G}=\rlim_i\mathscr{G}_i$. Recall that $\mathscr{G}$ is the sheaf associated to the presheaf $\mathscr{G}':U\mapsto\rlim_i\mathscr{G}_i(U)$. By  the tensor product $\mathscr{F}\otimes_{\mathscr{O}_X}\mathscr{G}$ is the sheafification of the presheaf
\[U\mapsto\mathscr{F}(U)\otimes_{\mathscr{O}_X(U)}\rlim_i\mathscr{G}_i(U)=\rlim_i\mathscr{F}(U)\otimes_{\mathscr{O}_X(U)}\mathscr{G}_i(U)\]
where the equality sign follows from the property of tensor product of modules. Hence the lemma follows from the description of colimits in $\mathbf{Mod}(\mathscr{O}_X)$.
\end{proof}
\subsection{Internal Hom}
Let $(X,\mathscr{O}_X)$ be a ringed space. Let $\mathscr{F},\mathscr{G}$ be $\mathscr{O}_X$-modules. Consider the rule
\[U\mapsto\Hom_{\mathscr{O}_X|_U}(\mathscr{F}|_U,\mathscr{G}|_U)\]
It follows from Proposition~\ref{sheaf glue morphism of sheaf} that this is a sheaf of abelian groups. In addition, given an element $\varphi\in\Hom_{\mathscr{O}_X|_U}(\mathscr{F}|_U,\mathscr{G}|_U)$ and a section $f\in\mathscr{O}_X(U)$ then we can define $f\varphi\in\Hom_{\mathscr{O}_X|_U}(\mathscr{F}|_U,\mathscr{G}|_U)$ by either precomposing with multiplication by $f$ on $\mathscr{F}|_U$ or postcomposing with multiplication by $f$ on $\mathscr{G}|_U$ (it gives the same result). Hence we in fact get a sheaf of $\mathscr{O}_X$-modules. We will denote this sheaf $\mathcal{H}om_{\mathscr{O}_X}(\mathscr{F},\mathscr{G})$. For every $x\in X$ there is also a canonical morphism
\[\mathcal{H}\!om_{\mathscr{O}_X}(\mathscr{F},\mathscr{G})_x\to \Hom_{\mathscr{O}_{X,x}}(\mathscr{F}_x,\mathscr{G}_x)\]
which is rarely an isomorphism.
\begin{proposition}
Let $(X,\mathscr{O}_X)$ be a ringed space. Let $\mathscr{F}$, $\mathscr{G}$, $\mathscr{H}$ be $\mathscr{O}_X$-modules. There is a canonical isomorphism
\[\mathcal{H}om_{\mathscr{O}_X}(\mathscr{F}\otimes_{\mathscr{O}_X}\mathscr{G},\mathscr{H})\to \mathcal{H}om_{\mathscr{O}_X}(\mathscr{F},\mathcal{H}om_{\mathscr{O}_X}(\mathscr{G},\mathscr{H}))\]
which is functorial in all three entries. In particular, to give a morphism $\mathscr{F}\otimes_{\mathscr{O}_X}\mathscr{G}\to\mathscr{H}$ is the same as giving a morphism $\mathscr{F}\to\mathcal{H}om_{\mathscr{O}_X}(\mathscr{G},\mathscr{H})$.
\end{proposition}
\begin{proof}
This is the analogue of that for modules.
\end{proof}
Due to this proposition, the functors $\mathcal{H}om(-,\mathscr{G})$ and $\mathcal{H}om(\mathscr{F},-)$ are left-exact.
\begin{proposition}
Let $(X,\mathscr{O}_X)$ be a ringed space. Let $\mathscr{F}$, $\mathscr{G}$ be $\mathscr{O}_X$-modules.
\begin{itemize}
\item If $\mathscr{F}_2\to\mathscr{F}_1\to\mathscr{F}\to 0$ is an exact sequence of $\mathscr{O}_X$-modules, then
\[\begin{tikzcd}
0\ar[r]&\mathcal{H}om(\mathscr{F},\mathscr{G})\ar[r]&\mathcal{H}om(\mathscr{F}_1,\mathscr{G})\ar[r]&\mathcal{H}om(\mathscr{F}_2,\mathscr{G})
\end{tikzcd}\]
is exact.
\item If $0\to\mathscr{G}\to\mathscr{G}_1\to\mathscr{G}_2\to 0$ is an exact sequence of $\mathscr{O}_X$-modules, then
\[\begin{tikzcd}
0\ar[r]&\mathcal{H}om(\mathscr{F},\mathscr{G})\ar[r]&\mathcal{H}om(\mathscr{F},\mathscr{G}_1)\ar[r]&\mathcal{H}om(\mathscr{F},\mathscr{G}_2)
\end{tikzcd}\]
is exact.
\end{itemize}
\end{proposition}
\begin{proposition}\label{stalk hom f.pre}
Let $(X,\mathscr{O}_X)$ be a ringed space. Let $\mathscr{F},\mathscr{G}$ be $\mathscr{O}_X$-modules. If $\mathscr{F}$ is finitely presented then the canonical map
\[\mathcal{H}om_{\mathscr{O}_X}(\mathscr{F},\mathscr{G})_x\to \Hom_{\mathscr{O}_{X,x}}(\mathscr{F}_x,\mathscr{G}_x)\]
is an isomorphism.
\end{proposition}
\begin{proof}
By localizing on $X$ we may assume that $\mathscr{F}$ has a presentation
\[\begin{tikzcd}
\bigoplus_{j=1}^{m}\mathscr{O}_X\ar[r]&\bigoplus_{i=1}^{n}\mathscr{O}_X\ar[r]&\mathscr{F}\ar[r]&0
\end{tikzcd}\]
Then this gives an exact sequence
\[\begin{tikzcd}
0\ar[r]&\mathcal{H}om_{\mathscr{O}_X}(\mathscr{F},\mathscr{G})\ar[r]&\bigoplus_{i=1}^{n}\mathscr{G}\ar[r]&\bigoplus_{j=1}^{m}\mathscr{G}
\end{tikzcd}\]
Taking stalks we get an exact sequence
\[\begin{tikzcd}
0\ar[r]&\mathcal{H}om_{\mathscr{O}_X}(\mathscr{F},\mathscr{G})_x\ar[r]&\bigoplus_{i=1}^{n}\mathscr{G}_x\ar[r]&\bigoplus_{j=1}^{m}\mathscr{G}_x
\end{tikzcd}\]
The result now follows since $\mathscr{F}_x$ sits in an exact sequence
\[\begin{tikzcd}
\bigoplus_{j=1}^{m}\mathscr{O}_{X,x}\ar[r]&\bigoplus_{i=1}^{n}\mathscr{O}_{X,x}\ar[r]&\mathscr{F}_x\ar[r]&0
\end{tikzcd}\]
which induces the exact sequence
\[\begin{tikzcd}
0\ar[r]&\Hom_{\mathscr{O}_{X,x}}(\mathscr{F}_x,\mathscr{G}_x)\ar[r]&\bigoplus_{i=1}^{n}\mathscr{G}_x\ar[r]&\bigoplus_{j=1}^{m}\mathscr{G}_x
\end{tikzcd}\]
which is the
same as the one above.
\end{proof}
\begin{proposition}\label{pull back hom f.pre}
Let $f:X\to Y$ be a morphism of ringed spaces. Let $\mathscr{F}$, $\mathscr{G}$ be $\mathscr{O}_Y$-modules. If $\mathscr{F}$ is finitely presented then the canonical map
\[f^*\mathcal{H}om_{\mathscr{O}_Y}(\mathscr{F},\mathscr{G})\to\mathcal{H}om_{\mathscr{O}_X}(f^*\mathscr{F},f^*\mathscr{G})\]
is an isomorphism.
\end{proposition}
\begin{proof}
Note that $f^*\mathscr{F}$ is also finitely presented. Let $x\in X$ map
to $y\in Y$. Looking at the stalks at $x$ we get an isomorphism by Proposition~\ref{stalk hom f.pre} and that in this case $\Hom$ commutes with base change by $\mathscr{O}_{Y,y}\to\mathscr{O}_{X,x}$.
\end{proof}
\begin{proposition}
Let $(X,\mathscr{O}_X)$ be a ringed space. Let $\mathscr{F},\mathscr{G}$ be $\mathscr{O}_{X}$-modules. If $\mathscr{F}$ is finitely presented then the sheaf $\mathcal{H}om_{\mathscr{O}_X}(\mathscr{F},\mathscr{G})$ is locally a kernel of a map between finite direct sums of copies of $\mathscr{G}$. In particular, if $\mathscr{G}$ is coherent then $\mathcal{H}om_{\mathscr{O}_X}(\mathscr{F},\mathscr{G})$ is coherent too.
\end{proposition}
\begin{proof}
The first assertion we saw in the proof of Proposition~\ref{stalk hom f.pre}. And the result for coherent sheaves then follows from Proposition~\ref{sheaf of module coh prop}.
\end{proof}
\begin{proposition}
Let $X$ be a topological space. Let $\mathscr{O}_1\to\mathscr{O}_2$ be a homomorphism of sheaves of rings. Then we have
\[\Hom_{\mathscr{O}_1}(\mathscr{F}_{\mathscr{O}_1},\mathscr{G})=\Hom_{\mathscr{O}_2}(\mathscr{F},\mathcal{H}om_{\mathscr{O}_1}(\mathscr{O}_2,\mathscr{G}))\]
bifunctorially in $\mathscr{F}\in\mathbf{Mod}(\mathscr{O}_2)$ and $\mathscr{G}\in\mathbf{Mod}(\mathscr{O}_1)$.
\end{proposition}
\begin{proof}
This is the analogue of the result for modules.
\end{proof}
\subsection{The abelian category of sheaves of modules}
Let $(X,\mathscr{O}_X)$ be a ringed space. Let $\mathscr{F},\mathscr{G}$ be sheaves of $\mathscr{O}_X$-modules. Let $\varphi,\psi:\mathscr{F}\to\mathscr{G}$ be morphisms of sheaves of $\mathscr{O}_X$-modules. We define $\varphi+\psi:\mathscr{F}\to\mathscr{G}$ to be the map which on each open $U\sub X$ is the sum of the maps induced by $\varphi,\psi$. This is clearly again a map of sheaves of $\mathscr{O}_X$-modules. It is also clear that composition of maps of $\mathscr{O}_X$-modules is bilinear with respect to this addition. Thus $\mathbf{Mod}(\mathscr{O}_X)$ is a pre-additive category. We will denote $0$ the sheaf of $\mathscr{O}_X$-modules which has constant value $\{0\}$ for all open $U\sub X$. Clearly this is both a final and an initial object of $\mathbf{Mod}(\mathscr{O}_X)$. Moreover, given a pair $\mathscr{F},\mathscr{G}$ of sheaves of $\mathscr{O}_X$-modules we may define the direct sum as
\[\mathscr{F}\oplus\mathscr{G}=\mathscr{F}\times\mathscr{G}\]
Thus $\mathbf{Mod}(\mathscr{O}_X)$ is an additive category.\par
Let $\varphi:\mathscr{F}\to\mathscr{G}$ be a morphism of $\mathscr{O}_X$-modules. We may define the \textbf{presheaf kernel} $\ker\varphi$ and the \textbf{presheaf cokernel} to be
\[(\mathrm{ker}_p\varphi)(U)=\ker\varphi_U,\quad(\mathrm{coker}_p\varphi)(U)=\coker\varphi_U.\]
for open subsets $U\sub X$. We define $\coker\varphi$ be the sheafification of $\mathrm{coker}_p\varphi$.
\begin{proposition}
The presheaf kernel is a presheaf of $\mathscr{O}_X$-modules, so is the presheaf cokernel.
\end{proposition}
\begin{proof}
For $U\sub V$ open in $X$, consider the diagram
\begin{equation}\label{presheaf kernel restriction}
\begin{tikzcd}
0\ar[r]&\mathrm{ker}_p\varphi_V\ar[r,hook]\ar[d,dashed]&\mathscr{F}(V)\ar[r,"\varphi_V"]\ar[d,"\res^V_U"]&\mathscr{G}(V)\ar[d,"\res^V_U"]\\
0\ar[r]&\mathrm{ker}_p\varphi_U\ar[r,hook]&\mathscr{F}(U)\ar[r,"\varphi_U"]&\mathscr{G}(U)
\end{tikzcd}
\end{equation}
By the comutativity, there is a unique map $\res^V_U:\ker\varphi_V\to\ker\varphi_U$ fitting in the diagram. This makes $\mathrm{ker}_p\varphi$ into a presheaf of $\mathscr{O}_X$-modules, and essentially the same argument works for $\mathrm{coker}_p\varphi$.
\end{proof}
\begin{proposition}
The presheaf $($co$)$kernel satisfies the universal property of $($co$)$kernel in the category of presheaves of $\mathscr{O}_X$-modules.
\end{proposition}
\begin{proof}
Let $\psi:\mathscr{G}\to\mathscr{H}$ be another morphism of presheaves of $\mathscr{O}_X$-modules such that $(\psi\circ\varphi)_U=\psi_U\circ\varphi_U=0$ for any $U\sub X$. Then we have an induced map $\tilde{\psi}_U:\ker\varphi_U\to\mathscr{H}(U)$ for each $U$. For $U\sub V$ open in $X$, it is easy to verify that the following commutative diagram:
\[\begin{tikzcd}
\ker\varphi_V\ar[r,"\tilde{\psi}_V"]\ar[d,"\res^V_U"]&\mathscr{H}(V)\ar[d,"\res^V_U"]\\
\ker\varphi_U\ar[r,"\tilde{\psi}_U"]&\mathscr{H}(U)
\end{tikzcd}\]
so $\ker\varphi$ is the kernel in the category of presheaves. A similar verification works for cokernels.
\end{proof}
\begin{proposition}\label{presheaf ker and coker universal property}
Suppose $\varphi:\mathscr{F}\to\mathscr{G}$ is a morphism of sheaves. Then the presheaf kernel $\mathrm{ker}_p\varphi$ satisfies the universal property of kernels in $\mathbf{Mod}(\mathscr{O}_X)$, and the sheafification of $\mathrm{coker}_p\varphi$ satisfies the universal property of cokernels in $\mathbf{Mod}(\mathscr{O}_X)$.
\end{proposition}
\begin{proof}
Let $U=\bigcup_iU_i$ be an open covering in $X$, and $s_i\in\ker\varphi(U_i)$ such that $s_i|_{U_i\cap U_j}=s_j|_{U_i\cap U_j}$ for all $i,j$. Then by the sheaf condition of $\mathscr{F}$, there is a unique $s\in\mathscr{F}(U)$ such that $s|_{U_i}=s_i$. Moreover, from the diagram $(\ref{presheaf kernel restriction})$ we get
\[\big(\varphi_U(s)\big)_{U_i}=\varphi_{U_i}(s|_{U_i})=0.\]
Thus by the sheaf condition of $\mathscr{G}$, we conclude $\varphi_U(s)=0$. This implies $s\in\varphi_U$, so $\ker\varphi$ is a sheaf.\par
For the cokernel, given any sheaf $\mathscr{E}$ and a diagram
\[\begin{tikzcd}
\mathscr{F}\ar[r,"\varphi"]&\mathscr{G}\ar[d]\ar[r,"\psi"]&\mathscr{E}\\
&\mathrm{coker}_p\varphi\ar[r]\ar[ru]&(\mathrm{coker}_p\varphi)^{\hash}\ar[u,dashed,swap,"\tilde{\psi}"]
\end{tikzcd}\]
we construct the map $\tilde{\psi}$ by using the universal property of $\mathrm{coker}_p\varphi$ and that of sheafification.
\end{proof}
In view of Proposition~\ref{presheaf ker and coker universal property}, for a morphism $\varphi:\mathscr{F}\to\mathscr{G}$, we define the kernel and cokernel of $\varphi$ by
\[\ker\varphi=\mathrm{ker}_p\varphi,\quad \coker\varphi=(\mathrm{coker}_p\varphi)^{\hash}.\]
Since taking stalks commutes with taking sheafification, the following result is immediate.
\begin{proposition}\label{sheaf ker coker and stalk prop}
Suppose $\mathscr{F}\to\mathscr{G}$ is a morphism of sheaves of $\mathscr{O}_X$-modules, then for all $x\in X$, we have canonical isomorphisms
\[(\ker\varphi)_x\cong\ker\varphi_x,\quad(\coker\varphi)_x\cong\coker\varphi_x.\]
\end{proposition}
\begin{proof}
Let $\varphi:\mathscr{F}\to\mathscr{G}$ be a morphism of sheaves. We have the commutative diagram:
\[\begin{tikzcd}
\mathscr{F}(U)\ar[r,"\varphi_U"]\ar[d]&\mathscr{G}(U)\ar[d]\\
\mathscr{F}_x\ar[r,"\varphi_x"]&\mathscr{G}_x
\end{tikzcd}\]
Let $s\in\ker\varphi_U$, then its image $\bar{s}$ is mapped to zero by $\varphi_x$, so $\bar{s}\in\ker\varphi_x$. Conversely, if $\bar{s}\in\ker\varphi_x$ and $\bar{s}=(U,s)$ where $s\in\mathscr{F}(U)$, then the image of $t=\varphi(s)\in\mathscr{G}(U)$ is zero in $\mathscr{G}_x$. Therefore in some neighborhood, say $V\sub U$, $t|_V=0$. Then we have the commutative diagram
\[\begin{tikzcd}
\mathscr{F}(V)\ar[r,"\varphi_V"]\ar[d]&\mathscr{G}(V)\ar[d]\\
\mathscr{F}_x\ar[r,"\varphi_x"]&\mathscr{G}_x
\end{tikzcd}\]
where now $\varphi_V(s|_V)=0$. This shows $\bar{s}\in(\ker\varphi)_p$. The proof is similar for cokernels.
\end{proof}
Now that we kernel and cokernels, we can prove that $\mathbf{Mod}(\mathscr{O}_X)$ is an abelian category.
\begin{theorem}\label{sheaf of module abelian cat exactness char}
Let $(X,\mathscr{O}_X)$ be a ringed space. The category $\mathbf{Mod}(\mathscr{O}_X)$ is an abelian category. Moreover, a complex $\mathscr{F}\to\mathscr{G}\to\mathscr{H}$ is exact at $\mathscr{G}$ if and only if for all $x\in X$ the complex $\mathscr{F}_x\to\mathscr{G}_x\to\mathscr{H}_x$ is exact at $\mathscr{G}_x$.
\end{theorem}
\begin{proof}
We have to show that image and coimage agree. By Proposition~\ref{sheaf morphism prop iff stalk prop} it is enough to show that image and coimage have the same stalk at every $x\in X$. By the constructions of kernels and cokernels above these stalks are the coimage and image in the categories of $\mathscr{O}_{X,x}$-modules. Thus we get the result from the fact that the category of modules over a ring is abelian.
\end{proof}
Actually the category $\mathbf{Mod}(\mathscr{O}_X)$ has many more properties. Here are two constructions we can do.
\begin{itemize}
\item Given any set $I$ and for each $i\in I$ a $\mathscr{O}_X$-module we can form the product
\[\prod_{i\in I}\mathscr{F}_i\]
which is the sheaf that associates to each open $U$ the product of the modules $\mathscr{F}_i(U)$.
\item Given any set $I$ and for each $i\in I$ a $\mathscr{O}_X$-module we can form the direct sum
\[\bigoplus_{i\in I}\mathscr{F}_i\]
which is the sheafification of the presheaf that associates to each open $U$ the direct sum of the modules $\mathscr{F}_i(U)$.
\end{itemize}
Using these we conclude that all limits and colimits exist in $\mathbf{Mod}(\mathscr{O}_X)$.
\begin{proposition}\label{sheaf module limit}
Let $(X,\mathscr{O}_X)$ be a ringed space.
\begin{itemize}
\item[(a)] All limits exist in $\mathbf{Mod}(\mathscr{O}_X)$. Limits are the same as the corresponding limits of presheaves of $\mathscr{O}_X$-modules.
\item[(b)] All colimits exist in $\mathbf{Mod}(\mathscr{O}_X)$. Colimits are the sheafification of the corresponding colimit in the category of presheaves. Taking colimits commutes with taking stalks.
\item[(c)] Filtered colimits are exact.
\item[(d)] Finite direct sums are the same as the corresponding finite direct sums of presheaves of $\mathscr{O}_X$-modules.
\end{itemize}
\end{proposition}
\begin{proof}
As $\mathbf{Mod}(\mathscr{O}_X)$ is abelian it has all finite limits and colimits. Thus the existence of limits and colimits and their description follows from the existence of products and coproducts and their description. Since sheafification commutes with taking stalks we see that colimits commute with taking stalks. Part (c) signifies that given a system $0\to\mathscr{F}_i\to\mathscr{G}_i\to\mathscr{H}_i\to 0$ of exact sequences of $\mathscr{O}_X$-modules over a directed partially ordered set $I$ the sequence $0\to\rlim_i\mathscr{F}_i\to\rlim_i\mathscr{G}_i\to\rlim_i\mathscr{H}_i\to 0$ is exact as well. Since we can check exactness on stalks, this follows from the case of modules. Part (d) comes from the fact that finite direct sum coincides with finite product in an Abelian category.
\end{proof}
\begin{remark}
For an arbitrary direct sum $\bigoplus_{i\in I}\mathscr{O}_X$, by the construction of sheafification we see that an element $s\in\bigoplus_{i\in I}\mathscr{O}_X$ satisfies the following property: For any $x\in X$ there is a neighborhood of $x$ such that $s|_U$ is a finite sum $\sum_{i\in I'}f_i$ with $f_i\in\mathscr{O}_X(U)$.
\end{remark}
The existence of limits and colimits allows us to consider exactness properties of functors defined on the category of $\mathscr{O}_X$-modules in terms of limits and colimits.
\begin{proposition}\label{ringed space morphism pullback pushforward exactness}
Let $(f,f^{\hash}):(X,\mathscr{O}_X)\to(Y,\mathscr{O}_Y)$ be a morphism of ringed spaces.
\begin{itemize}
\item[(a)] The functor $f_*:\mathbf{Mod}(\mathscr{O}_X)\to\mathbf{Mod}(\mathscr{O}_Y)$ is left exact. In fact it commutes with all limits.
\item[(b)] The functor $f^*:\mathbf{Mod}(\mathscr{O}_X)\to\mathbf{Mod}(\mathscr{O}_Y)$ is right exact. In fact it commutes with all colimits.
\item[(c)] The functor $f^{-1}:\mathbf{Ab}(Y)\to\mathbf{Ab}(X)$ on abelian sheaves is exact.
\end{itemize}
\end{proposition}
\begin{proof}
Recall that $(f^*,f_*)$ is an adjoint pair of functors. The last part holds because exactness can be checked on stalks and the description of stalks of the pullback.
\end{proof}
\begin{proposition}
Let $j:U\to X$ be an open immersion of topological spaces. The functor $j_!:\mathbf{Ab}(U)\to\mathbf{Ab}(X)$ is exact.
\end{proposition}
\begin{proof}
This follows from the description of stalks given in Proposition~\ref{sheaf open extension functor prop}.
\end{proof}
\begin{proposition}\label{sheaf module quasi-compact sum}
Let $(X,\mathscr{O}_X)$ be a ringed space. Let $I$ be a set. For $i\in I$, let $\mathscr{F}_i$ be a sheaf of $\mathscr{O}_X$-modules. For $U\sub X$ quasi-compact open the map 
\[\bigoplus_{i\in I}\mathscr{F}_i(U)\to\Big(\bigoplus_{i\in I}\mathscr{F}_i\Big)(U)\]
is bijective.
\end{proposition}
\begin{proof}
If $s$ is an element of the right hand side, then there exists an open covering $U=\bigcup_{j\in J}U_j$ such that $s|_{U_j}$ is a finite sum $\sum_{i\in I_j}s_{ji}$ with $s_{ji}\in\mathscr{F}_i(U_j)$. Because $U$ is quasi-compact we may assume that the covering is finite, i.e., that $J$ is finite. Then $I'=\bigcup_{j\in J}I_j$ is a finite subset of $I$. Clearly, $s$ is a section of the subsheaf $\bigoplus_{i\in I'}\mathscr{F}_i$. The result follows from the fact that for a finite direct sum sheafification
is not needed.
\end{proof}
\section{Ringed spaces}
\subsection{Ringed spaces, \texorpdfstring{$\mathscr{A}$}{A}-modules, and \texorpdfstring{$\mathscr{A}$}{A}-algebras}
A \textbf{ringed space} (resp. \textbf{topologically ringed space}) is defiend to be a couple $(X,\mathscr{A})$ formed by a topological space $X$ and a sheaf of rings (resp. a sheaf of topological rings) $\mathscr{A}$ on $X$. We call $X$ the topological space underling the ringed space $(X,\mathscr{A})$, and $\mathscr{A}$ is the structural sheaf. We only denote by $\mathscr{O}_X$ the structural sheaf, and for $x\in X$, $\mathscr{O}_{X,x}$ denotes the stalk of $\mathscr{O}_X$ at $x$. If $\mathscr{A}$ is a sheaf of commutative rings, we say $(X,\mathscr{A})$ is a \textbf{commutative ringed space}. Without further specifications, we only consider commutative ringed spaces.\par
The ringed spaces (resp. topologically ringed spaces) form a category, if we define a morphism $(X,\mathscr{A})\to(Y,\mathscr{B})$ as a couple $(f,f^{\hash})$ formed by a continuous map $f:X\to Y$ and a $f$-morphism $f^{\hash}:\mathscr{B}\to\mathscr{A}$ (that is, a morphism from $\mathscr{B}$ to $f_*(\mathscr{A})$) of sheaf of rings (resp. sheaf of topological rings). As the category of rings admits inductive limits, for any $x\in X$ we have a homomorphism $f^{\hash}_x:\mathscr{B}_{f(x)}\to\mathscr{A}_x$. The composition of two morphisms $(f,f^{\hash}):(X,\mathscr{A})\to(Y,\mathscr{B})$ and $(g,g^{\hash}):(Y,\mathscr{B})\to(Z,\mathscr{C})$ is then defined to be the couple $(h,h^{\hash})$, where $h=g\circ f$ and $g^{\hash}$ is the composition of $f^{\hash}$ and $g^{\hash}$ (which equals to $g_*(f^{\hash})\circ g^{\hash}$). For any $x\in X$, we then have $h^{\hash}_x=g^{\hash}_{f(x)}\circ f^{\hash}_x$, so if the homomorphisms $f^{\hash}$ and $f^{\hash}$ are injective (resp. surjective), then so is $h^{\hash}$. We then verify that, if $f$ is a injective continuous map and $f^{\hash}$ is a surjective homomorphism of sheaf of rings, then the morphism $(f,f^{\hash})$ is a monomorphism of category of ringed spaces.\par
By abuse of language, we only replace $(f,f^{\hash})$ by $f$, and say that $f:(X,\mathscr{A})\to(Y,\mathscr{B})$ is a morphism of ringed spaces. In this case, it is understood that the homomorphism $f^{\hash}$ is also given.\par
For any subset $U$ of $X$, the couple $(U,\mathscr{A}|_U)$ is clearly a ringe space, called induced over $U$ by $(X,\mathscr{A})$, or the restriction of $(X,\mathscr{A})$ to $U$. If $j:U\to X$ is the injection and $j^{\hash}:\mathscr{A}\to j_*(\mathscr{A}|_U)$ is the canonical homomorphism, we then have a \textit{monomorphism} $(j,j^{\hash}):(U,\mathscr{A}|_U)\to(X,\mathscr{A})$ of ringed spaces, called the \textbf{canonical injection}. The composition of a morphism $f:(X,\mathscr{A})\to(Y,\mathscr{B})$ with this canonical injection is said to be the restriction of $f$ to $U$, and denoted by $f|_U$.
\begin{example}
Let $f:X\to Y$ be a continuous map of topological spaces. Consider the sheaves of continuous real valued functions $C^0_X$ on $X$ and $C^0_Y$ on $Y$. We claim that there is a natural $f$-map $f^{\hash}:C^0_Y\to C^0_X$ associated to $f$. Namely, we simply define it by the rule
\[C^0_Y(V)\to C^0_X(f^{-1}(V)),\quad h\mapsto h\circ f\]
Strictly speaking we should write $f^{\hash}(h)=h\circ f|_{f^{-1}(V)}$. It is clear that this is an $f$-map of sheaves of $\R$-algebras.\par
Of course there are lots of other situations where there is a canonical morphism of ringed spaces associated to a geometrical type of morphism. For example, if $M,N$ are $C^\infty$-manifolds and $f:M\to N$ is a smooth map, then $f$ induces a canonical morphism of ringed spaces $(M,C^\infty_M)\to(N,C^\infty_N)$.
\end{example}
We will not review the definition of the $\mathscr{A}$-modules for a ringed space $(X,\mathscr{A})$. If $\mathscr{A}$ is a sheaf of commutative rings, and we replace the module structure by the algebra structure in the definition of $\mathscr{A}$-modules, we obtain the definition of an $\mathscr{A}$-algebra over $X$. In other words, an $\mathscr{A}$-algebra (not necessarily commutative) is an $\mathscr{A}$-module $\mathscr{C}$ endowed with a homomorphism of $\mathscr{A}$-modules $\varphi:\mathscr{C}\otimes_{\mathscr{A}}\mathscr{C}\to\mathscr{C}$ and of a section $e$ above $X$, such that:
\begin{itemize}
\item[(\rmnum{1})] the diagram 
\[\begin{tikzcd}
\mathscr{C}\otimes_{\mathscr{A}}\mathscr{C}\otimes_{\mathscr{A}}\mathscr{C}\ar[d,swap,"1\otimes\varphi"]\ar[r,"\varphi\otimes 1"]&\mathscr{C}\otimes_{\mathscr{A}}\mathscr{C}\ar[d,"\varphi"]\\
\mathscr{C}\otimes_{\mathscr{A}}\mathscr{C}\ar[r,"\varphi"]&\mathscr{C}
\end{tikzcd}\]
is commutative;
\item[(\rmnum{2})] for any open subset $U\sub X$ and any section $s\in\Gamma(U,\mathscr{C})$, we have $\varphi((e|_U)\otimes s)=\varphi(s\otimes(e|_U))=s$. 
\end{itemize} 
Saying that $\mathscr{C}$ is a commutative $\mathscr{A}$-algebra amounts to the fact that the diagram
\[\begin{tikzcd}
\mathscr{C}\otimes_{\mathscr{A}}\mathscr{C}\ar[rd,swap,"\varphi"]\ar[rr,"\sigma"]&&\mathscr{C}\otimes_{\mathscr{A}}\mathscr{C}\ar[ld,"\varphi"]\\
&\mathscr{C}&
\end{tikzcd}\]
is commutative, where $\sigma$ is the canonical symmetry of the tensor producr $\mathscr{C}\otimes_{\mathscr{A}}\mathscr{C}$.\par
The homomorphism of $\mathscr{A}$ are defined just as that of $\mathscr{A}$-modules, where we replace the "modules" by "algebras". If $\mathscr{M}$ is a sub-$\mathscr{A}$-module of an $\mathscr{A}$-algebra $\mathscr{C}$, the sub-$\mathscr{A}$-algebra of $\mathscr{C}$ \textbf{generated by} $\mathscr{M}$ is the sum of images of the homomorphisms $\otimes^n\mathscr{M}\to\mathscr{C}$ (for $n\geq 0$). This is the sheaf associated with the presheaf $U\mapsto\mathscr{B}(U)$ of algebras, where $\mathscr{B}(U)$ is the sub-algebra of $\mathscr{C}(U)$ generated by the sub-module $\mathscr{M}(U)$.\par
We say a sheaf of rings $\mathscr{A}$ over a topological space $X$ is \textbf{reduced} (resp. \textbf{integral}) at a point $x$ of $X$ if the stalk $\mathscr{A}_x$ is a reduced ring (resp. integral ring). We say that $\mathscr{A}$ is reduced if it is reduced at every point of $X$. Recall that a ring $A$ is called regular if for any prime ideal $\p$ of $A$, the local ring $A_\p$ is a regular local ring. We say a sheaf of ring $\mathscr{A}$ over $X$ is \textbf{regular at a point $\bm{x}$} (resp. \textbf{regular}) if the stalk $\mathscr{A}_x$ is a regular local ring (resp. if $\mathscr{A}$ is regular at every point). Finally, we say that a sheaf of rings $\mathscr{A}$ over $X$ is \textbf{normal at a point $\bm{x}$} (resp. \textbf{normal}) if the stalk $\mathscr{A}_x$ is an integrally closed ring (resp. if $\mathscr{A}$ is normal at every point). We say the ringed space $(X,\mathscr{A})$ is reduced (resp. normal, regular) if the structural sheaf $\mathscr{A}$ satisfies this property.\par
A \textbf{sheaf of graded rings} is by definition a sheaf or rings $\mathscr{A}$ which is the direct sum of a family $(\mathscr{A}_n)_{n\in\Z}$ of sheaves of abelian groups which satisfies the condition $\mathscr{A}_n\mathscr{A}_m\sub\mathscr{A}_{m+n}$. A graded $\mathscr{A}$-module is an $\mathscr{A}$-module $\mathscr{F}$ which is a direct sum of a family $(\mathscr{F}_n)_{n\in\Z}$ of sheaves of abelian groups, satisfying $\mathscr{A}_m\mathscr{F}_n\sub\mathscr{F}_{m+n}$.\par
Given a ringed space $(X,\mathscr{A})$ (commutative, and we shall not specify this condition further), we recall that definition of the bifunctors $\mathscr{F}\otimes_{\mathscr{A}}\mathscr{G}$, $\sHom_{\mathscr{A}}(\mathscr{F},\mathscr{G})$, and $\Hom_{\mathscr{A}}(\mathscr{F},\mathscr{G})$ on the category of $\mathscr{A}$-modules, with values in the category of sheaves of abelian groups. the stalk $(\mathscr{F}\otimes_{\mathscr{A}}\mathscr{G})_x$ at any point $x\in X$ is identified canonically with $\mathscr{F}_x\otimes_{\mathscr{A}_x}\mathscr{G}_x$ and we define a functorial homomorphism $(\sHom_{\mathscr{A}}(\mathscr{F},\mathscr{G}))_x\mapsto\Hom_{\mathscr{A}_x}(\mathscr{F}_x,\mathscr{G}_x)$ which is, in general, neither injective nor surjective. These bifunctors are additive and, in particular, commutes with finite direct sums. The functor $\mathscr{F}\otimes_{\mathscr{A}}\mathscr{G}$ is right exact on $\mathscr{F}$ and $\mathscr{G}$, commutes with inductive limits, and $\mathscr{A}\otimes_{\mathscr{A}}\mathscr{G}$ (resp. $\mathscr{F}\otimes_{\mathscr{A}}\mathscr{A}$) is canonically identified with $\mathscr{G}$ (resp. $\mathscr{F}$). The functor $\sHom_{\mathscr{A}}(\mathscr{F},\mathscr{G})$ and $\Hom_{\mathscr{A}}(\mathscr{F},\mathscr{G})$ are left exact on $\mathscr{F}$ and $\mathscr{G}$ (but note that $\sHom_{\mathscr{A}}(-,\mathscr{G})$ and $\Hom_{\mathscr{A}}(\mathscr{F},\mathscr{G})$ are contravariant). Moreover, $\sHom_{\mathscr{A}}(\mathscr{A},\mathscr{G})$ is canonically identified with $\mathscr{G}$, and for open subset $U\sub X$, we have
\[\Gamma(U,\sHom_{\mathscr{A}}(\mathscr{F},\mathscr{G}))=\Hom_{\mathscr{A}|_U}(\mathscr{F}|_U,\mathscr{G}|_U).\]
In particular, $\Hom_{\mathscr{A}}(\mathscr{A},\mathscr{G})$ is identified with $\Gamma(X,\mathscr{G})$. For any $\mathscr{A}$-module $\mathscr{F}$, we denote by $\mathscr{F}^*$ the \textbf{dual} of $\mathscr{F}$, which is $\sHom_{\mathscr{A}}(\mathscr{F},\mathscr{A})$.\par
Finally, if $\mathscr{A}$ is a sheaf of rings and $\mathscr{F}$ is an $\mathscr{A}$-module, then $U\mapsto\bigw^p\Gamma(U,\mathscr{F})$ is a presheaf whose associated sheaf is an $\mathscr{A}$-module and is denoted by $\bigw^p\mathscr{F}$, called the \textbf{$p$-th exterior power} of $\mathscr{F}$. We can easily verify that the cannical map from the presheaf $U\mapsto\bigw^p\Gamma(U,\mathscr{F})$ to $\bigw^p\mathscr{F}$ is injective, and for $x\in X$, we have $(\bigw^p\mathscr{F})_x=\bigw^p(\mathscr{F}_x)$. It is clear that $\bigw^p\mathscr{F}$ is a covariant functor on $\mathscr{F}$. We can similarly define the functors $\bm{T}_p(\mathscr{F})$ and $\bm{S}_p(\mathscr{F})$, which are the \textbf{$p$-th tensor power} and \textbf{$p$-th symmetric power} of $\mathscr{F}$.\par
Let $\mathscr{I}$ be an ideal of $\mathscr{A}$ and $\mathscr{F}$ be an $\mathscr{A}$-module. Then we note that $\mathscr{I}\mathscr{F}$, the image of $\mathscr{I}\otimes_{\mathscr{A}}\mathscr{F}$ by the canonical map $\mathscr{I}\otimes_{\Z}\mathscr{F}\to\mathscr{F}$, is a sub-$\mathscr{A}$-module of $\mathscr{F}$. It is clear that for any $x\in X$, we have $(\mathscr{I}\mathscr{F})_x=\mathscr{I}_x\mathscr{F}_x$. It is immediate that $\mathscr{I}\mathscr{F}$ is also the $\mathscr{A}$-module associated sheaf of the presheaf $U\mapsto\Gamma(U,\mathscr{I})\Gamma(U,\mathscr{F})$. If $\mathscr{I}_1$, $\mathscr{I}_2$ are ideals of $\mathscr{A}$, we have $\mathscr{I}_1(\mathscr{I}_2\mathscr{F})=(\mathscr{I}_1\mathscr{I}_2)\mathscr{F}$.\par
Let $(X_\lambda,\mathscr{A}_\lambda)_{\lambda\in L}$ be a family of ringed spaces; for each couple $(\lambda,\mu)$, suppose that we are given an open subset $V_{\lambda\mu}\sub X_\lambda$, and an isomorphism $\varphi_{\lambda\mu}:(V_{\mu\lambda},\mathscr{A}_\mu|_{V_{\mu\lambda}})\to(V_{\lambda\mu},\mathscr{A}_{\lambda}|_{V_{\lambda\mu}})$ of ringed spaces, with $V_{\lambda\lambda}=X_\lambda$ and $\varphi_{\lambda\lambda}$ being the identity. Suppose moreover that, for any triple $(\lambda,\mu,\nu)$, we have $\varphi_{\lambda\nu}=\varphi_{\lambda\mu}\circ\varphi_{\mu\nu}$ on the open subset $V_{\lambda\mu}\cap V_{\lambda\nu}$ (glueing condition for $\varphi_{\lambda\mu}$). We can then consider the topological space obtained by glueing (via the morphism $\varphi_{\lambda\mu}$) the $X_\lambda$ along $V_{\lambda\mu}$. If we identify $X_\lambda$ with the corresponding open subset $X_\lambda'$ of $X$, the hypotheses implies that $V_{\lambda\mu}\cap V_{\lambda\nu}$, $V_{\mu\nu}\cap V_{\mu\lambda}$, $V_{\nu\lambda}\cap V_{\nu\mu}$ are identified with $X_\lambda'\cap X_\mu'\cap X_\nu'$. We an then transport the ringed space structure of $X_\lambda$ to $X_\lambda'$, and if $\mathscr{A}_\lambda'$ is the sheaf of rings transported by $\mathscr{A}_\lambda$, the $\mathscr{A}_\lambda'$ satisfies the glueing condition for sheaves and define a sheaf of rings $\mathscr{A}$ over $X$. We say that $(X,\mathscr{A})$ is the ringed space obtained by glueing $(X_\lambda,\mathscr{A}_\lambda)$ along $V_{\lambda\mu}$ via the morphisms $\varphi_{\lambda\mu}$.\par
Let $(X,\mathscr{O}_X)$ be a ringed space. For a section $s\in\Gamma(U,\mathscr{I}_X)$ over an open subset $U$ of $X$ to be \textbf{invertible}, it is necessary and sufficient that for any open cover $(U_\alpha)$ of $U$, the restriction of $s$ to $U_\alpha$ is invertible in $\Gamma(U_\alpha,\mathscr{O}_X)$, in view of the uniqueness of inverse element. There then exists a sub-sheaf of multiplication groups $\mathscr{O}_X^{\times}$ of $\mathscr{O}_X$ such that, for any open subset $U$ of $X$, $\Gamma(U,\mathscr{O}_X^{\times})$ is the group of invertible elements of the ring $\Gamma(U,\mathscr{O}_X)$. For any $x\in X$, the stalk $(\mathscr{O}_X^{\times})_x$ is the set of invertible elements of the ring $\mathscr{O}_{X,x}$, because if $s_x\in\mathscr{O}_{X,x}$ admits an inverse $t_x$ in this ring, $s_x$ and $t_x$ are the germs of two sections $s,t$ of $\mathscr{O}_X$ over a neighborhood $V$ of $x$, and the relation $(st)_x=1_x$ implies $st|_W=1$ over a smaller neighborhood $W\sub V$ of $x$.\par
On the other hand, with the same notations, if $s$ is a regular element of $\Gamma(U,\mathscr{O}_X)$ and $V$ is an open subset of $U$, $s|_V$ is not necessarily regular in $\Gamma(V,\mathscr{O}_X)$, because if $(s|_V)t=0$ for a section $t\in\Gamma(V,\mathscr{O}_X)$, $t$ does not necessarily admits an extension to $U$. We denote by $\mathscr{S}(\mathscr{O}_X)$ the presheaf of sets such that $\mathscr{S}(\mathscr{O}_X)(U)$, for each open subset $U$ of $X$, is the set of sections $s\in\Gamma(U,\mathscr{O}_X)$ whose restriction to \textit{any} open subset $V\sub U$ is a regular element of the ring $\Gamma(V,\mathscr{O}_X)$. From this definition, it is clear that $\mathscr{S}(\mathscr{O}_X)$ is a sheaf, since if $s$ is a section of $\mathscr{O}_X$ over $U$ and $s|_V$ is regular in $\Gamma(V,\mathscr{O}_X)$ is regular for some open subset $V\sub U$, then $s$ is regular in $\Gamma(U,\mathscr{O}_X)$. For a section $s\in\Gamma(U,\mathscr{O}_X)$, to be in $\mathscr{S}(\mathscr{O}_X)(U)$ amount to saying that for any open subset $V\sub U$, the map $t\mapsto(s|_V)$ on $\Gamma(V,\mathscr{O}_X)$ is injective (recall that we always assume that $\mathscr{O}_X$ is commutative). In this case, by the exactness of the functor $\rlim$ on the category of modules, it follows that for any $x\in U$, the germ $s_x$ is regular in $\mathscr{O}_{X,x}$. Conversely, if this does not hold, then for some $x\in U$, then there exists a section $t\in\Gamma(V,\mathscr{O}_X)$ for some open subset $V\sub U$ such that $(s|_V)t=0$, so $s\notin\mathscr{S}(\mathscr{O}_X)(U)$. We then say that $\mathscr{S}(\mathscr{O}_X)$ is the sheaf of sets defined by the condition that $\Gamma(U,\mathscr{O}_X)$ is the set of sections of $\Gamma(U,\mathscr{O}_X)$ whose germ at every point $x\in U$ is regular. We also note that the stalk $(\mathscr{S}(\mathscr{O}_X))_x$, which is contained in the set of regular elements of $\mathscr{O}_{X,x}$, does not necessarily equal to this set.\par
We say a ringed space $(X,\mathscr{O}_X)$ is \textbf{local} if for each $x\in X$, the stalk $\mathscr{O}_{X,x}$ is a local ring. In this case, we say $(X,\mathscr{O}_X)$ is a \textbf{locally ringed space}. We denote by $\m_x$ the maximal idea lof $\mathscr{O}_{X,x}$ and $\kappa(x)=\mathscr{O}_{X,x}/\m_x$ the residue field. For any $\mathscr{O}_X$-module $\mathscr{F}$, any open subset $U\sub X$, any point $x\in U$ and any section $f\in\Gamma(U,\mathscr{F})$, we denote by $f(x)\in\kappa(x)$ the class of the germ $f_x\in\mathscr{F}_x$ modulo $\m_x\mathscr{F}_x$, and we say $f(x)$ is the \textbf{value} of $f$ at $x$. The relation $f(x)=0$ then signifies $f_x\in\m_x\mathscr{F}_x$ (do not confuse with the condition $f_x=0_x$). We denote by $U_f$ the set of $x\in U$ such that $f(x)\neq 0$ (or equivalently $f_x\notin\m_x\mathscr{F}_x$). We note that if $f\in\Gamma(X,\mathscr{F})$, then $X_f$ is contained in $\supp(\mathscr{F})$.
\begin{proposition}\label{ringed space local residue of tensor char}
Let $(X,\mathscr{O}_X)$ be a locally ringed space, $\mathscr{F}$ and $\mathscr{F}$ be two $\mathscr{O}_X$-module. For any $x\in X$, $(\mathscr{F}\otimes_{\mathscr{O}_X}\mathscr{G})_x/\m_x(\mathscr{F}\otimes_{\mathscr{O}_X}\mathscr{G})_x$ is canonically identified with $(\mathscr{F}_x/\m_x\mathscr{F}_x)\otimes_{\kappa(x)}(\mathscr{G}_x/\m_x\mathscr{G}_x)$. If $s$ (resp. $t$) is a section of $\mathscr{F}$ (resp. $\mathscr{G}$) over $X$, $(s\otimes t)(x)$ is identified with $s(x)\otimes t(x)$ and we have $X_{s\otimes t}=X_s\cap X_t$.
\end{proposition}
\begin{proof}
In fact $(\mathscr{F}\otimes_{\mathscr{O}_X}\mathscr{G})_x=\mathscr{F}_x\otimes_{\mathscr{O}_{X,x}}\mathscr{G}_x$, and $(\mathscr{F}_x\otimes_{\mathscr{O}_{X,x}}\mathscr{G}_x)\otimes_{\mathscr{O}_{X,x}}(\mathscr{O}_{X,x}/\m_x)$ is canonically isomorphic to $(\mathscr{F}\otimes_{\mathscr{O}_X}\mathscr{G})_x/\m_x(\mathscr{F}\otimes_{\mathscr{O}_X}\mathscr{G})_x$, hence to $(\mathscr{F}_x/\m_x\mathscr{F}_x)\otimes_{\kappa(x)}(\mathscr{G}_x/\m_x\mathscr{G}_x)$. The relation $X_{s\otimes t}=X_s\cap X_t$ then follows from the fact that a product $a\otimes b$ in a tensor product of vector spaces is nonzero if and only if $a$ and $b$ are both nonzero.
\end{proof}
\begin{proposition}\label{ringed space local clopen and idempotents}
Let $(X,\mathscr{O}_X)$ be a locally ringed space. The set of idempotents of the ring $\Gamma(X,\mathscr{O}_X)$ corresponds to the set of clopen subsets of $X$. In particular, for $X$ to be connected, it is necessary and sufficent that $\Gamma(X,\mathscr{O}_X)$ has no nontrivial idempotents.
\end{proposition}
\begin{proof}
Let $U$ be a clopen subset of $X$. Then it corresponds to the section $s\in\Gamma(X,\mathscr{O}_X)$ such that $s|_V=1$ for any open subset $V\sub U$ and $s|_V=0$ for any open subset $V\sub X-U$; these open subsets by hypotheses form a base for the topology of $X$, so we define a section $s=e_U$ of $\mathscr{O}_X$ over $X$, which is an idempotent of $\Gamma(X,\mathscr{O}_X)$. Conversely, if $s$ is an idempotent, for any $x\in X$, $s_x$ is an idempotent of $\mathscr{O}_{X,x}$, hence equal to $0_x$ or $1_x$ because $\mathscr{O}_{X,x}$ is a local ring ($s_x(1-s_x)=0$, and if $s_x\in\m_x$, then $1-s_x$ is invertible so $s_x=0$). It is clear that the set $U$ of $x\in X$ such that $s_x=1_x$ is open, and so is the set $X-U$ of $x\in X$ such that $s_x=0_x$, so $U$ is a clopen subset of $X$ and we have $s=e_U$. 
\end{proof}
A morphism of locally ringed space $f:(X,\mathscr{O}_X)\to(Y,\mathscr{O}_Y)$ is defined to be a morphism of ringed spaces such that for each $x\in X$ the homomorphism $f_{x}^{\hash}:\mathscr{O}_{Y,f(x)}\to\mathscr{O}_{X,x}$ is local. Note that by this definition, the category of locally ringed spaces is not a full subcategory of that of ringed spaces.
\begin{proposition}\label{ringed space local residue of pullback char}
Let $f:X\to Y$ be a morphism of locally ringed spaces, $x$ be a point of $X$, and $y=f(x)$. For any $\mathscr{O}_Y$-module $\mathscr{G}$, $(f^*(\mathscr{G}))_x/\m_x(f^*(\mathscr{G}))_x$ is canonically identified with $(\mathscr{G}_y/\m_y\mathscr{G}_y)\otimes_{\kappa(y)}\kappa(x)$. If $t$ is a section of $\mathscr{G}$ over $Y$ and $s=\rho_{\mathscr{G}}(t)$ is the corresponding section of $f^*(\mathscr{G})$ over $X$, then $s(x)$ is identified with $t(y)\otimes 1$ and we have $X_s=f^{-1}(Y_t)$.
\end{proposition}
\begin{proof}
We have $(f^*(\mathscr{G}))_x=\mathscr{G}_y\otimes_{\mathscr{O}_{Y,y}}\mathscr{O}_X$, so $(f^*(\mathscr{G}))_x/\m_x(f^*(\mathscr{G}))_x$ is identified with $\mathscr{G}_y\otimes_{\mathscr{O}_{Y,y}}\kappa(x)$. The homomorphism $\mathscr{O}_{Y,y}\to\mathscr{O}_{X,x}$ is local by hypotheses, so $\m_y$ annihilates $\kappa(x)$ and $\mathscr{G}_y\otimes_{\mathscr{O}_{Y,y}}\kappa(x)$ is isomorphic to $(\mathscr{G}_y/\m_y\mathscr{G}_y)\otimes_{\kappa(y)}\kappa(x)$. The last assertion follows from the fact that $t(y)\otimes 1=0$ is equivalent to $t(y)=0$.
\end{proof}
\subsection{Direct image of \texorpdfstring{$\mathscr{A}$}{A}-modules}
Let $(X,\mathscr{A})$ and $(Y,\mathscr{B})$ be ringed spaces, $f$ be a morphism $(X,\mathscr{A})\to(Y,\mathscr{B})$. Then $f_*(\mathscr{A})$ is a sheaf of rings over $Y$, and $f^{\hash}$ is a homomorphism $\mathscr{B}\to f_*(\mathscr{A})$ of sheaf of rings. Let $\mathscr{F}$ be an $\mathscr{A}$-module; the direct image $f_*(\mathscr{F})$ is then a sheaf of abelian groups over $Y$. Moreover, for any open subset $U\sub Y$,
\[\Gamma(U,f_*(\mathscr{F}))=\Gamma(f^{-1}(U),\mathscr{F})\]
is endowed with a module structure over the ring $\Gamma(U,f_*(\mathscr{A}))=\Gamma(f^{-1}(U),\mathscr{A})$. These bilinear maps are compatible with restrictions, so $f_*(\mathscr{F})$ becomes an $f_*(\mathscr{A})$-module. The homomorphism $f^{\hash}:\mathscr{B}\to f_*(\mathscr{A})$ then makes $f_*(\mathscr{F})$ a $\mathscr{B}$-module. We say that this $\mathscr{B}$-module is the direct image of $\mathscr{F}$ under the morphism $f$, still denoted by $f_*(\mathscr{F})$. If $\mathscr{F}_1$, $\mathscr{F}_2$ are two $\mathscr{A}$-modules over $X$ and $u:\mathscr{F}_1\to\mathscr{F}_2$ is an $\mathscr{A}$-homomorphism, it is immediate that $f_*(u)$ is an $f_*(\mathscr{A})$-homomorphism $f_*(\mathscr{F}_1)\to f_*(\mathscr{F}_2)$, and a fortiori a $\mathscr{B}$-homomorphism, also denoted by $f_*(u)$. We then see that $f_*$ is a covariant funtor from the category of $\mathscr{A}$-modules to that of $\mathscr{B}$-modules. Moreover, it is immediate that this functor is left exact.\par
Over $f_*(\mathscr{A})$, the $\mathscr{B}$-module structure and the sheaf of rings structure define a structure of $\mathscr{B}$-algebras; we denote by $f_*(\mathscr{A})$ this $\mathscr{B}$-algebra. Let $(Z,\mathscr{C})$ be a third ringed space, $g:(Y,\mathscr{B})\to(Z,\mathscr{C})$ be a morphism; if $h=g\circ f$ is the composition morphism, then we have $h_*=g_*\circ f_*$.\par
Let $\mathscr{M},\mathscr{N}$ be two $\mathscr{A}$-modules. For any open subset $U$ of $Y$, we have a canonical map
\[\Gamma(f^{-1}(U),\mathscr{M})\times\Gamma(f^{-1}(U),\mathscr{N})\to\Gamma(f^{-1}(U),\mathscr{M}\otimes_{\mathscr{A}}\mathscr{N})\]
which is bilinear over the ring $\Gamma(f^{-1}(U),\mathscr{A})=\Gamma(U,f_*(\mathscr{A}))$, and a fortiori over $\Gamma(U,\mathscr{B})$; this defines a homomorphism
\[\Gamma(U,f_*(\mathscr{M}))\otimes_{\Gamma(U,\mathscr{B})}\Gamma(U,f_*(\mathscr{N}))\to\Gamma(U,f_*(\mathscr{M}\otimes_{\mathscr{A}}\mathscr{N}))\]
and as we can verify that these homomorphisms are compatible with restrictions, we obtain a canonical homomorphism of $\mathscr{B}$-modules
\begin{align}\label{ringed space direct image and tensor-1}
f_*(\mathscr{M})\otimes_{\mathscr{B}}f_*(\mathscr{N})\to f_*(\mathscr{M}\otimes_{\mathscr{A}}\mathscr{N})
\end{align}
which in general is neither injective nor surjective. If $\mathscr{P}$ is a third $\mathscr{A}$-module, we verify that the diagram
\[\begin{tikzcd}
f_*(\mathscr{M})\otimes_{\mathscr{B}}f_*(\mathscr{N})\otimes_{\mathscr{B}}f_*(\mathscr{P})\ar[r]\ar[d]&f_*(\mathscr{M}\otimes_{\mathscr{A}}\mathscr{N})\otimes_{\mathscr{B}}f_*(\mathscr{P})\ar[d]\\
f_*(\mathscr{M})\otimes_{\mathscr{B}}f_*(\mathscr{N}\otimes_{\mathscr{A}}\mathscr{P})\ar[r]&f_*(\mathscr{M}\otimes_{\mathscr{A}}\mathscr{N}\otimes_{\mathscr{A}}\mathscr{P})
\end{tikzcd}\]
is commutative.\par
Let $\mathscr{M},\mathscr{N}$ be two $\mathscr{A}$-modules. For any open $U\sub Y$, we have by definition
\[\Gamma(f^{-1}(U),\sHom_{\mathscr{A}}(\mathscr{M},\mathscr{N}))=\Hom_{\mathscr{A}|_V}(\mathscr{M}|_V,\mathscr{N}|_V),\]
where we put $V=f^{-1}(U)$. The map $u\mapsto f_*(u)$ is a homomorphism
\[\Hom_{\mathscr{A}|_V}(\mathscr{M}|_V,\mathscr{N}|_V)\to\Hom_{\mathscr{B}|_U}(f_*(\mathscr{M})|_U,f_*(\mathscr{N})|_U)\]
for the structure of $\Gamma(U,\mathscr{B})$-modules. These homomorphisms are compatible with restrictions, so define a canonical homomorphism of $\mathscr{B}$-modules
\[f_*(\sHom_{\mathscr{A}}(\mathscr{M},\mathscr{N}))\to\sHom_{\mathscr{B}}(f_*(\mathscr{M}),f_*(\mathscr{N})).\]

If $\mathscr{C}$ is an $\mathscr{A}$-algebra, the composition homomorphism
\[f_*(\mathscr{C})\otimes_{\mathscr{B}}f_*(\mathscr{C})\to f_*(\mathscr{C}\otimes_{\mathscr{A}}\mathscr{C})\to f_*(\mathscr{C})\]
defines on $f_*(\mathscr{C})$ a $\mathscr{B}$-algebra structure. We see that if $\mathscr{M}$ is a $\mathscr{C}$-module, $f_*(\mathscr{M})$ is an $f_*(\mathscr{C})$-module.\par
Consider in particular the case where $X$ is a closed subspace of $Y$ and $f$ is the canonical injection $j:X\to Y$. If $\mathscr{B}'=\mathscr{B}|_X=j^{-1}(\mathscr{B})$ is the restriction of $\mathscr{B}$ to $X$, an $\mathscr{A}$-module $\mathscr{M}$ can be considered as a $\mathscr{B}'$-module via the homomorphism $f^{\sharp}:\mathscr{B}'\to\mathscr{A}$; $f_*(\mathscr{M})$ is then the $\mathscr{B}$-module inducing $\mathscr{M}$ on $X$ and $0$ elsewhere. If $\mathscr{N}$ is a second $\mathscr{A}$-module, $f_*(\mathscr{M})\otimes_{\mathscr{B}}f_*(\mathscr{N})$ is identified with $f_*(\mathscr{M}\otimes_{\mathscr{B}'}\mathscr{N})$ and $\sHom_{\mathscr{B}}(f_*(\mathscr{M}),f_*(\mathscr{N}))$ with $f_*(\sHom_{\mathscr{B}'}(\mathscr{M},\mathscr{N}))$.\par
Let $(S,\mathscr{O}_S)$ be a ringed space and let $f:(X,\mathscr{O}_X)\to(S,\mathscr{O}_S)$ be a morphism of ringed spaces. If we fix $S$, we denote by $\mathscr{A}(X)$ (if there is no confusion) the direct image $f_*(\mathscr{O}_X)$, which is an $\mathscr{O}_S$-algebra. For any open subset $U$ of $S$, we have $\mathscr{A}(f^{-1}(U))=\mathscr{A}(X)|_U$. Simialrly, for any $\mathscr{O}_X$-module $\mathscr{F}$ (resp. any $\mathscr{O}_X$-algebra $\mathscr{B}$), we denote by $\mathscr{A}(\mathscr{F})$ (resp. $\mathscr{A}(\mathscr{B})$) the direct image $f_*(\mathscr{F})$ (resp. $f_*(\mathscr{B})$), which is an $\mathscr{A}(X)$-module (resp. an $\mathscr{A}(X)$-algebra) and also an $\mathscr{O}_S$-module (resp. $\mathscr{O}_S$-algebra).\par
We can define $\mathscr{A}(X)$ as a contravariant functor on $X$, from the category of $S$-ringed spaces to the category of $\mathscr{O}_S$-algebras. In fact, consider two morphisms $f:X\to S$, $g:Y\to S$ and let $h:X\to Y$ be a morphism, such that the diagram
\[\begin{tikzcd}
X\ar[rd,swap,"f"]\ar[rr,"h"]&&Y\ar[ld,"g"]\\
&S
\end{tikzcd}\]
is commutative. Then by definition $h^{\hash}:\mathscr{O}_Y\to h_*(\mathscr{O}_X)$ is a homomorphism of sheaves of rings; we also deduce a homomorphism $g_*(h^{\hash}):g_*(\mathscr{O}_Y)\to g_*(h_*(\mathscr{O}_X))=f_*(\mathscr{O}_X)$ of $\mathscr{O}_S$-algebras, which is a homomorphism $\mathscr{A}(Y)\to\mathscr{A}(X)$, denoted by $\mathscr{A}(h)$. If $h':Y\to Z$ is a second $S$-morphism, then we have $\mathscr{A}(h'\circ h)=\mathscr{A}(h)\circ\mathscr{A}(h')$.\par
Now let $\mathscr{F}$ be an $\mathscr{O}_X$-module, $\mathscr{G}$ be an $\mathscr{O}_Y$-module, and $u:\mathscr{G}\to h_*(\mathscr{F})$ be a homomorphism of $\mathscr{O}_Y$-modules. Then $g_*(u):g_*(\mathscr{G})\to g_*(h_*(\mathscr{F}))=f_*(\mathscr{F})$ is a homomorphism $\mathscr{A}(\mathscr{G})\to\mathscr{A}(\mathscr{F})$ of $\mathscr{O}_S$-modules, which we denote by $\mathscr{A}(u)$. Moreover, the couple $(\mathscr{A}(h),\mathscr{A}(u))$ is a bi-homomorphism from $\mathscr{A}(Y)$-module $\mathscr{A}(\mathscr{G})$ to the $\mathscr{A}(X)$-module $\mathscr{A}(\mathscr{F})$. The ringed space $S$ being fixed, we can consider the couples $(X,\mathscr{F})$, where $X$ is a $S$-ringed space and $\mathscr{F}$ is an $\mathscr{O}_X$-module, which form a category, and define a morphism $(X,\mathscr{F})\to(Y,\mathscr{G})$ to be a couple $(h,u)$, where $h:X\to Y$ is an $S$-morphism and $u:\mathscr{G}\to h_*(\mathscr{F})$ is a homomorphism of $\mathscr{O}_Y$-modules. We can then say that $(X,\mathscr{F})\mapsto(\mathscr{A}(X),\mathscr{A}(\mathscr{F}))$ is a contravariant functor from this category to the cateogry of couples formed by an $\mathscr{O}_S$-algebra and a module of this algebra.
\subsection{Inverse image of a \texorpdfstring{$\mathscr{B}$}{B}-module}
Let $f:(X,\mathscr{A})\to(Y,\mathscr{B})$ be a morphism of ringed spaces. Let $\mathscr{G}$ be a $\mathscr{B}$-module and $f^{-1}(\mathscr{G})$ be the inverse image of $\mathscr{G}$, which is a sheaf of abelian groups over $X$. The definition of sections of $f^{-1}(\mathscr{G})$ and of $f^{-1}(\mathscr{B})$ shows that $f^{-1}(\mathscr{G})$ is canonically endowed with an $f^{-1}(\mathscr{B})$-module structure. On the other hand, the homomorphism $f^{\sharp}:f^{-1}(\mathscr{B})\to\mathscr{A}$ endows $\mathscr{A}$ with an $f^{-1}(\mathscr{B})$-module structure. The tensor product $f^{-1}(\mathscr{G})\otimes_{f^{-1}(\mathscr{B})}\mathscr{A}$ is then an $\mathscr{A}$-module, called the inverse image of $\mathscr{G}$ under the morphism $(f,f^{\hash})$ and denoted by $f^*(\mathscr{G})$. If $\mathscr{G}_1$, $\mathscr{G}_2$ are two $\mathscr{B}$-modules over $Y$ and $v:\mathscr{G}_1\to\mathscr{G}_2$ is a $\mathscr{B}$-homomorphism, $f^{-1}(v)$ is then an $f^{-1}(\mathscr{B})$-homomorphism from $f^{-1}(\mathscr{G}_1)$ to $f^{-1}(\mathscr{G}_2)$; therefore $f^{-1}(v)\otimes 1_{\mathscr{A}}$ is an $\mathscr{A}$-homomorphism $f^*(\mathscr{G}_1)\to f^*(\mathscr{G}_2)$, which we denote by $f^*(v)$. We then define $f^*$ as a covariant functor from the category of $\mathscr{B}$-modules to that of $\mathscr{A}$-modules. Note that this functor (contrary to $f^{-1}$) is not in general exact, and is only right exact, since the tensor product with $\mathscr{A}$ is only right exact. We will see that $f^*$ is the left adjoint of the functor $f_*$. For any $x\in X$, we have $(f^*(\mathscr{G}))_x=\mathscr{G}_{f(x)}\otimes_{\mathscr{B}_{f(x)}}\mathscr{A}_x$ in view of the formula for the stalk of tensor products. The support of $f^*(\mathscr{G})$ is tens contained in $f^{-1}(\supp(\mathscr{G}))$.\par
Let $(Z,\mathscr{C})$ be a third ringed space and $g:(Y,\mathscr{B})\to(Z,\mathscr{C})$ be a morphism. Then if $h=g\circ f$ is the composition morphism, then it follows from the definitions that $h^*=f^*\circ g^*$.\par
Let $(\mathscr{G}_\lambda)$ be a inductive system of $\mathscr{B}$-modules, and $\mathscr{G}=\rlim\mathscr{G}_\lambda$ be the inductive limit. The canonical homomorphisms $\mathscr{G}_\lambda\to\mathscr{G}$ define an $f^{-1}(\mathscr{B})$-homomorphism $f^{-1}(\mathscr{G}_\lambda)\to f^{-1}(\mathscr{G})$, which gives a canonical homomorphism $\rlim f^{-1}(\mathscr{G}_\lambda)\to f^{-1}(\mathscr{G})$. As taking stalk commutes with inductive limits, this canonical homomorphism is bijective. Moreover, tensor product also commutes with inductive limits, and we then have a canonical functorial isomorphism $\rlim f^*(\mathscr{G}_\lambda)\cong f^*(\rlim\mathscr{G}_\lambda)$ of $\mathscr{A}$-modules.\par
On the other hand, for a finite direct sum $\bigoplus_i\mathscr{G}_i$ of $\mathscr{B}$-modules, it is clear that
\[f^*(\bigoplus_i\mathscr{G}_i)=\bigoplus_if^*(\mathscr{G}_i).\]
By passing to inductive limits, we then deduce that, in view of the preceding, that this equality holds for arbitrary direct sums.\par
Let $\mathscr{G}_1,\mathscr{G}_2$ be two $\mathscr{B}$-modules; from the definition of inverse images of sheaves of abelian groups we deduce a canonical homomorphism
\[f^{-1}(\mathscr{G}_1)\otimes_{f^{-1}(\mathscr{B})}f^{-1}(\mathscr{G}_2)\to f^{-1}(\mathscr{G}_1\otimes_{\mathscr{B}}\mathscr{G}_2)\]
of $f^{-1}(\mathscr{B})$, and since the stalk of a tensor product is the rensor product of stalks, this homomorphism is an isomorphism. By tensoring with $\mathscr{A}$, we then deduce a canonical isomorphism 
\begin{align}\label{ringed space inverse image and tensor}
f^*(\mathscr{G}_1)\otimes_{\mathscr{A}}f^*(\mathscr{G}_2)\stackrel{\sim}{\to}f^*(\mathscr{G}_1\otimes\mathscr{G}_2).
\end{align}
Let $\mathscr{C}$ be a $\mathscr{B}$-algebra. The algebra structure over $\mathscr{C}$ is given by a $\mathscr{B}$-homomorphism $\mathscr{C}\otimes_{\mathscr{B}}\mathscr{C}\to\mathscr{C}$ satisfying the associativity and commutativity conditions (which can be verified on stalks); the isomorphism (\ref{ringed space inverse image and tensor}) then provides a homomorphism $f^*(\mathscr{C})\otimes_{\mathscr{A}}f^*(\mathscr{C})\to f^*(\mathscr{C})$ satisfying the same conditions, whence $f^*(\mathscr{C})$ is endowed with an $\mathscr{A}$-algebra structure. In particular, it follows from the definitions that the $\mathscr{A}$-algebra $f^*(\mathscr{B})$ is equal to $\mathscr{A}$. If $\mathscr{C}_1,\mathscr{C}_2$ are two $\mathscr{B}$-algebras and $v:\mathscr{C}_1\to\mathscr{C}_2$ is a homomorphism of $\mathscr{B}$-algebras, then $f^*(v):f^*(\mathscr{C}_1)\to f^*(\mathscr{C}_2)$ is a homomorphism of $\mathscr{A}$-algebras.\par
Similarly, if $\mathscr{M}$ is a $\mathscr{C}$-module, then the structure of a $\mathscr{B}$-module is given by a $\mathscr{B}$-homomorphism $\mathscr{C}\otimes_{\mathscr{B}}\mathscr{M}\to\mathscr{M}$ satisfying the associativity condition; by transporting structure, we see that $f^*(\mathscr{C})$ is endowed with a $f^*(\mathscr{C})$-module structure.\par
Let $\mathscr{I}$ be an ideal of $\mathscr{B}$; as the functor $f^{-1}$ is exact, the $f^{-1}(\mathscr{B})$-module $f^{-1}(\mathscr{I})$ is canonically identified an ideal of $f^{-1}(\mathscr{B})$; the canonical injection $f^{-1}(\mathscr{I})\to f^{-1}(\mathscr{B})$ then gives a homomorphism of $\mathscr{A}$-modules
\[f^*(\mathscr{I})=f^{-1}(\mathscr{I})\otimes_{f^{-1}(\mathscr{B})}\mathscr{A}\to\mathscr{A};\]
we denote by $f^*(\mathscr{I})\mathscr{A}$, or simply $\mathscr{I}\mathscr{A}$ if there is no confusion, the image of $f^*(\mathscr{I})$ under this hommorphism. We then have $\mathscr{I}\mathscr{A}=f^{\sharp}(f^{-1}(\mathscr{I}))\mathscr{A}$ and in particular, for any $x\in X$, $(\mathscr{I}\mathscr{A})_x=f^{\sharp}_x(\mathscr{I}_{f(x)})\mathscr{A}_x$, in view of the canonical identification of the stalk $f^{-1}(\mathscr{I})$. If $\mathscr{I}_1,\mathscr{I}_2$ are two ideals of $\mathscr{B}$, we have $(\mathscr{I}_1\mathscr{I}_2)\mathscr{A}=\mathscr{I}_1(\mathscr{I}_2\mathscr{A})=(\mathscr{I}_1\mathscr{A})(\mathscr{I}_2\mathscr{A})$. If $\mathscr{F}$ is an $\mathscr{A}$-module, we put $\mathscr{I}\mathscr{F}=(\mathscr{I}\mathscr{A})\mathscr{F}$.
\begin{remark}
Over any topological space, we can define a canonical sheaf of rings $\Z_X$, which is the constant sheaf associated the presheaf $U\mapsto\Z$. It is clear thet the sheaves of abelian groups over $X$ are identified with the $\Z_X$-modules over the ringed space $(X,\Z_X)$, and we can in particular consider the tensor product $\mathscr{F}\otimes_{\Z_X}\mathscr{G}$ of two sheaves of rings $\mathscr{F},\mathscr{G}$ over $X$. If $f:X\to Y$ is a continuous map, for any open subset $V$ of $Y$, we have a canonical homomorphism $\Z\to\Gamma(f^{-1}(V),\Z_X)$ of rings and this defines a homomorphism $f^{\hash}:\Z_Y\to f_*(\Z_X)$ of sheaves of rings over $Y$. We then obtain a morphism $(f,f^{\hash}):(X,\Z_X)\to(Y,\Z_Y)$ of ringed spaces. If $\mathscr{F}$ is a sheaf of abelian groups over $Y$, $f^*(\mathscr{F})$ is canonically identified with $f^{-1}(\mathscr{F})$; if $\mathscr{F}$, $\mathscr{G}$ are two sheves of abelian groups over $Y$, we then have a canonical isomorphism $f^{-1}(\mathscr{F}\otimes_{\Z_Y}\mathscr{G})=f^{-1}(\mathscr{F})\otimes_{\Z_X}f^{-1}(\mathscr{G})$. We then deduce that if $f$ is a quasi-homeomorphism, $f^{-1}$ is not only an equivalence from the category of sheaf of abelian groups over $Y$ to that of sheaf of abelian groups over $X$, but also an equivalence from the category of sheaf of rings over $Y$ to that of sheaf of rings over $X$.\par
Given two ringed spaces $(X,\mathscr{O}_X)$, $(Y,\mathscr{O}_Y)$, we say a morphism $f:(X,\mathscr{O}_X)\to(Y,\mathscr{O}_Y)$ is a \textbf{quasi-isomorphism} if $f$ is a quasi-homeomorphism and $f^{\sharp}:f^{-1}(\mathscr{O}_Y)\to\mathscr{O}_X$ is an isomorphism of sheaf of rings. If this is the case, the ringed space $(X,\mathscr{O}_X)$ is entirely determined up to isomorphism, by $(Y,\mathscr{O}_Y)$, the space $X$, and the quasi-homeomorphism.\par
If $f$ is a quasi-isomorphism of ringed spaces, the functor $\mathscr{F}\to f^*(\mathscr{F})$ is an equivalence from category of $\mathscr{O}_Y$-modules to the category of $\mathscr{O}_X$-modules, since $f^*(\mathscr{F})$ is identified with $f^{-1}(\mathscr{F})$. We then conclude for example the isomorphisms of bi-$\partial$-functors
\[\Ext^i_{\mathscr{O}_Y}(\mathscr{F},\mathscr{G})\stackrel{\sim}{\to}\Ext^i_{\mathscr{O}_X}(f^*(\mathscr{F}),f^*(\mathscr{G})).\]
In general, we can say that the usual constructions of the sheaf theory and homological algebra on the ringed space $Y$ or $X$, are equivalent.
\end{remark}
\subsection{Relations of direct images and inverse images}
Again we let $f:(X,\mathscr{O}_X)\to (Y,\mathscr{O}_Y)$ be a morphism of ringed spaces. By definition, a homomorphism $u:\mathscr{G}\to f_*(\mathscr{F})$ of $\mathscr{B}$-modules is called an \textbf{$f$-morphism} from $\mathscr{G}$ to $\mathscr{F}$ and we denote it by $u:\mathscr{G}\to\mathscr{F}$ if there is no confusion. Given such a homomorphism, for any couple $(U,V)$ where $U$ is an open subset of $X$ and $V$ is an open subset of $Y$ such that $f(U)\sub V$, a \textbf{homomorphism} $u_{U,V}:\Gamma(V,\mathscr{G})\to\Gamma(U,\mathscr{F})$ of $\Gamma(V,\mathscr{B})$ modules, where $\Gamma(U,\mathscr{F})$ is considered as a $\Gamma(V,\mathscr{B})$-module via the ring homomorphism $f^{\sharp}_{U,V}:\Gamma(V,\mathscr{B})\to\Gamma(U,\mathscr{A})$. The homomorphisms $u_{U,V}$ fits into the commutative diagram
\[\begin{tikzcd}
\mathscr{G}(V)\ar[r,"u_{U,V}"]\ar[d]&\mathscr{F}(U)\ar[d]\\
\mathscr{G}(V')\ar[r,"u_{U',V'}"]&\mathscr{F}(U')
\end{tikzcd}\]
for $U'\sub U$, $V'\sub V$, $f(U')\sub V'$. Moreover, for the homomorphism $u$, it suffices to define $u_{U,V}$ for $U$ (resp. $V$) in a base $\mathcal{B}$ (resp. $\mathcal{B}$') of the topology of $X$ (resp. $Y$). Let $g:(Y,\mathscr{B})\to(Z,\mathscr{C})$ be another morphism and $h=g\circ f$. Let $\mathscr{H}$ be a $\mathscr{C}$-module, and $v:\mathscr{H}\to g_*(\mathscr{G})$ be a $g$-morphism; then
\[w:\mathscr{H}\stackrel{v}{\longrightarrow}g_*(\mathscr{G})\stackrel{g_*(u)}{\longrightarrow}g_*(f_*(\mathscr{F}))\]
is an $h$-morphism which is called the \textbf{composition} of $u$ and $v$.\par
We will now see that there is a canonical isomorphism of bifunctors on $\mathscr{F}$ and $\mathscr{G}$
\begin{align}\label{ringed space direct and inverse adjoint prop}
\Hom_{\mathscr{A}}(f^*(\mathscr{G}),\mathscr{F})\stackrel{\sim}{\to}\Hom_{\mathscr{B}}(\mathscr{G},f_*(\mathscr{F}))
\end{align}
which we denote by $v\mapsto v^{\flat}$, and the inverse of this isomorphism is denoted by $u\mapsto u^{\sharp}$. The definition is the folowing: for an $\mathscr{A}$-homomorphism $v:f^*(\mathscr{G})\to\mathscr{F}$, by composing with the canonical homomorphism $f^{-1}(\mathscr{G})\to f^*(\mathscr{G})$, we obtain a hommorphism $\tilde{v}:f^{-1}(\mathscr{G})\to\mathscr{F}$ of sheaves of abelian groups, which is also a homomorphism of $f^{-1}(\mathscr{B})$-modules. We then deduce a homomorphism $\tilde{v}^{\flat}:\mathscr{G}\to f_*(\mathscr{F})$ by the adjointness of $f_*$ and $f^{-1}$, which is a homomorphism of $\mathscr{B}$-modules, and we denote by $v^{\flat}$. Similarly, for a $\mathscr{B}$-homomorphism $u:\mathscr{G}\to f_*(\mathscr{F})$, we deduce a homomorphism $u^{\sharp}:f^{-1}(\mathscr{G})\to\mathscr{F}$ of $f^{-1}(\mathscr{B})$-modules, whence by tensoring with $\mathscr{A}$ a homomorphism of $\mathscr{A}$-modules $f^*(\mathscr{G})\to\mathscr{F}$, still denoted by $u^{\sharp}$. It is immediate that $(u^{\sharp})^{\flat}=u$ and $(v^{\flat})^{\sharp}=v$, as well as the functoriality in $\mathscr{F}$ of the isomorphism $v\mapsto v^{\flat}$. The functoriality in $\mathscr{G}$ of $u\mapsto u^{\sharp}$ can be then deduced formally and we then see that $f^*$ is left adjoint to $f_*$.\par
If we choose $v$ to be the identify homomorphism of $f^*(\mathscr{G})$, then $v^{\flat}$ is a homomorphism
\[\rho_{\mathscr{G}}:\mathscr{G}\to f_*(f^*(\mathscr{G}));\]
if we choose $u$ to be the identify homomorphism on $f_*(\mathscr{F})$, then $u^{\sharp}$ is a homomorphism
\[\sigma_{\mathscr{F}}:f^*(f_*(\mathscr{F}))\to\mathscr{F};\]
these homomorphism are called canonical, and are in general neither injective nor surjective. As always, for a homomorphism $v:\mathscr{G}\to f_*(\mathscr{F})$, we have a canonical factorization
\begin{align}\label{ringed space v^flat factorization by rho}
v^{\flat}:\mathscr{G}\stackrel{\rho_{\mathscr{G}}}{\to}f_*(f^*(\mathscr{G}))\stackrel{f_*(v)}{\to}f_*(\mathscr{F})
\end{align}
and for a homomorphism $u:\mathscr{G}\to f_*(\mathscr{F})$, we have a canonical factorization
\begin{align}\label{ringed space u^sharp factorization by sigma}
u^{\sharp}:f^*(\mathscr{G})\stackrel{f^*(u)}{\to}f^*(f_*(\mathscr{F}))\stackrel{\sigma_{\mathscr{F}}}{\to}\mathscr{F}
\end{align}
Now let $g:(Y,\mathscr{B})\to(Z,\mathscr{C})$ be another morphism and $h=g\circ f$ be the composition. Let $\mathscr{H}$ be a $\mathscr{C}$-module, $u:\mathscr{G}\to\mathscr{F}$ and $v:\mathscr{H}\to\mathscr{G}$ be homomorphisms, and $w=g_*(u)\circ v$ be the composition of $u$ and $v$. Then $w^{\sharp}$ is the composition homomorphism
\begin{align}\label{ringed space sharp of a composition}
w^{\sharp}:f^*(g^*(\mathscr{H}))\stackrel{f^*(v^{\sharp})}{\to}f^*(\mathscr{G})\stackrel{u^{\sharp}}{\to}\mathscr{F}.  
\end{align}
To verify this, we use the description of $w^{\sharp}$ in (\ref{ringed space u^sharp factorization by sigma}) to obtain that $w^{\sharp}$ is given by the following diagram
\[\begin{tikzcd}[column sep=15mm,row sep=15mm]
f^*(g^*(\mathscr{H}))\ar[rr,bend left=20pt,"f^*(g^*(w))"]\ar[rd,swap,"f^*(v^{\sharp})"]\ar[r,"f^*(g^*(v))"]&f^*(g^*(g_*(\mathscr{G})))\ar[d,swap,"f^*(\sigma_{\mathscr{G}})"]\ar[r,"f^*(g^*(g_*(u)))"]&f^*(g^*(g_*(f_*(\mathscr{F}))))\ar[d,"f^*(\sigma_{f_*(\mathscr{F})})"]\ar[rd,"\sigma_{\mathscr{F}}"]&\\
&f^*(\mathscr{G})\ar[rr,bend right=20pt,"u^{\sharp}"]\ar[r,"f^*(u)"]&f^*(f_*(\mathscr{F}))\ar[r,"\sigma_{\mathscr{F}}"]&\mathscr{F}
\end{tikzcd}\]
The central square is immediately verified to be commutative by the naturality of $\sigma$, whence the assertion.\par
We note that if $s$ is a section of $\mathscr{G}$ over an open subset $V$ of $Y$, $\rho_{\mathscr{G}}(s)$ is the section $s'\otimes 1$ of $f^*(\mathscr{G})$ over $f^{-1}(V)$, where $s'$ is the section such that $s'_x=s_{f(x)}$ for any $x\in f^{-1}(V)$. We say that $\rho_{\mathscr{G}}(s)$ is the \textbf{inverse image} of $s$ under $f$. Note also that if $u:\mathscr{G}\to f_*(\mathscr{F})$ is a homomorphism, it defines for each $x\in X$ a homomorphism $u_x:\mathscr{G}_{f(x)}\to\mathscr{F}_x$ over stalks, obtained by composing $(u^{\sharp})_x:(f^*(\mathscr{G}))_x\to\mathscr{F}_x$ and the canonical homomorphism $s_x\mapsto s_x\otimes 1$ from $\mathscr{G}_{f(x)}$ to $(f^*(\mathscr{G}))_x=\mathscr{G}_{f(x)}\otimes_{\mathscr{B}_{f(x)}}\mathscr{A}_x$. The homomorphism $u_x$ is also obtained by taking inductive limit of the homomorphisms $\Gamma(V,\mathscr{G})\stackrel{u}{\to}\Gamma(f^{-1}(V),\mathscr{F})\to\mathscr{F}_x$, where $V$ runs through neighborhood of $f(x)$.\par
Let $\mathscr{F}_1,\mathscr{F}_2$ be $\mathscr{A}$-modules, $\mathscr{G}_1,\mathscr{G}_2$ be $\mathscr{B}$-modules, $u_i$ ($i=1,2$) be a homomorphism from $\mathscr{G}_i$ to $\mathscr{F}_i$. We denote by $u_1\otimes u_2$ the homomorphism $u:\mathscr{G}_1\otimes_{\mathscr{B}}\mathscr{G}_2\to\mathscr{F}_1\otimes_{\mathscr{A}}\mathscr{F}_2$ such that $u^{\sharp}=(u_1)^{\sharp}\otimes(u_2)^{\sharp}$; we verify that $u$ is also the composition
\[\mathscr{G}_1\otimes_{\mathscr{B}}\mathscr{G}_2\to f_*(\mathscr{F}_1)\otimes_{\mathscr{B}}f_*(\mathscr{F}_2)\to f_*(\mathscr{F}_1\otimes_{\mathscr{A}}\mathscr{F}_2),\]
where the first morphism is the ordinary tensor product $u_1\otimes_{\mathscr{B}}u_2$ and the second homomorphism is the canonical homomorphism (\ref{ringed space direct image and tensor-1}).\par
Let $(\mathscr{G}_\lambda)_{\lambda\in L}$ be an inductive system of $\mathscr{B}$-modules, and, for $\lambda\in L$, let $u_\lambda$ be a homomorphism $\mathscr{G}_\lambda\to f_*(\mathscr{F})$, which form an inductive system; put $\mathscr{G}=\rlim\mathscr{G}_\lambda$ and $u=\rlim u_\lambda$. Then $(u^\lambda)^{\sharp}$ form an inductive system of homomorphisms $f^*(\mathscr{G}_\lambda)\to\mathscr{F}$, and the inductive limit of this system is just $u^{\sharp}$, since taking tensor products commutes with inductive limits.\par
Let $\mathscr{M},\mathscr{N}$ be two $\mathscr{B}$-modules, $V$ be an open subset of $Y$, and $U=f^{-1}(V)$. The map $v\mapsto f^*(v)$ is a homomorphism
\[\Hom_{\mathscr{B}|_V}(\mathscr{M}|_V,\mathscr{N}|_V)\to\Hom_{\mathscr{A}|_U}(f^*(\mathscr{M})|_U,f^*(\mathscr{N})|_U)\]
for the structure of $\Gamma(V,\mathscr{B})$-modules: $\Hom_{\mathscr{A}|_U}(f^*(\mathscr{M})|_U,f^*(\mathscr{N})|_U)$ is endowed with a $\Gamma(U,f^{-1}(\mathscr{B}))$-module structure, and thanks to the canonical homomorphism $\Gamma(V,\mathscr{B})\to\Gamma(U,f^{-1}(\mathscr{B}))$ obtained from the definition of $f^{-1}$, this is then a $\Gamma(V,\mathscr{B})$-module. We also verify that these homomorphisms are compatible with restrictions, and therefore define a canonical homomorphism
\[\gamma:\sHom_{\mathscr{B}}(\mathscr{M},\mathscr{N})\to f_*\big(\sHom_{\mathscr{A}}(f^*(\mathscr{M}),f^*(\mathscr{N}))\big)\]
which corresponds to a homomorphism
\begin{align}\label{ringed space inverse image and sheaf Hom}
\gamma^{\sharp}:f^*(\sHom_{\mathscr{B}}(\mathscr{M},\mathscr{N}))\to\sHom_{\mathscr{A}}(f^*(\mathscr{M}),f^*(\mathscr{N}))
\end{align}
and this canonical homomorphisms are functorial on $\mathscr{M}$ and $\mathscr{N}$.\par
Suppose that $\mathscr{F}$ (resp. $\mathscr{G}$) is an $\mathscr{A}$-algebra (resp. a $\mathscr{B}$-algebra). If $u:\mathscr{G}\to f_*(\mathscr{F})$ is a homomorphisms of $\mathscr{B}$-algebras, $u^{\sharp}$  is a homomorphism $f^*(\mathscr{G})\to\mathscr{F}$ of $\mathscr{A}$-algebras; this follows from the diagram
\[\begin{tikzcd}
\mathscr{G}\otimes_{\mathscr{B}}\mathscr{G}\ar[r]\ar[d]&\mathscr{G}\ar[d,"u"]&\\
f_*(\mathscr{F}\otimes_{\mathscr{A}}\mathscr{F})\ar[r]&f_*(\mathscr{F})
\end{tikzcd}\]
Similarly, if $v:f^*(\mathscr{G})\to\mathscr{F}$ is a homormorphism of $\mathscr{A}$-algebras, then $v^\flat:\mathscr{G}\to f_*(\mathscr{F})$ is a homomorphisms of $\mathscr{B}$-algebras. We then get an isomorphism of bifunctors
\[\Hom_{\mathscr{A}-\mathbf{alg}}(f^*(\mathscr{G}),\mathscr{F})\stackrel{\sim}{\to}\Hom_{\mathscr{B}-\mathbf{alg}}(\mathscr{G},f_*(\mathscr{F})).\]
We can then say that $f^*$ is the left adjoint of the functor $f^*$ from the category of $\mathscr{A}$-algebras to that of $\mathscr{B}$-algebras.
\subsection{Open immersions and representable natrual transformations}
The set of open immersions in the category $\mathbf{Rsp}$ is closed under composition and fiber product, so we can speak of natrual transformations $F\to G$ (where $F$ and $G$ are contravariant functors from $\mathbf{Rsp}$ to $\mathbf{Set}$) which are \textbf{representable by open immersions}. The same is true when if we consider the category $\mathbf{Rsp}_{/S}$, where $S$ is a base ringed space.
\begin{proposition}\label{ringed space representable functor via subfunctor}
Let $S$ be a ringed space, $F:(\mathbf{Rsp}_{/S})^{\circ}\to\mathbf{Set}$ a contravariant functor, and $(F_i)_{i\in I}$ be a family of subfunctors of $F$. Suppose that the following conditions are satisfied:
\begin{itemize}
\item[(\rmnum{1})] For each $i$, the canonical natrual transformation $u_i:F_i\to F$ are representable by an open immersion.
\item[(\rmnum{2})] For any $S$-ringed space $X$, the map $U\mapsto F(U)$, where $U$ is an open subset of $X$, is a sheaf of sets over $X$ (i.e., $F$ is a sheaf over $\mathbf{Rsp}_{/S}$).
\item[(\rmnum{3})] For any $S$-ringed space $Z$ and any natrual transformation $h_Z\to F$, if $Z_i$ is the $S$-ringed space representating the functor $F_i\times_Fh_Z$ and $U_i$ is the image of morphism $Z_i\to Z$, then $(U_i)$ form an open covering of $Z$. 
\item[(\rmnum{4})] For each $i$, the functor $F_i$ is representable by an $S$-ringed space $X_i$.
\end{itemize}
Then the functor $F$ is representable by an $S$-ringed space $X$, and the images of $X_i$ under the morphism $X_i\to X$ (which is open by condition (\rmnum{1})) form an open covering of $X$.
\end{proposition}
\section{Quasi-coherent sheaves and coherent sheaves}
\subsection{Quasi-coherent sheaves}
In this subsection we introduce an abstract notion of quasi-coherent $\mathscr{O}_X$-module. This notion is very useful in algebraic geometry, since quasi-coherent modules on a scheme have a good description on any affine open. However, in the general setting of locally ringed spaces this notion is not well behaved at all. The category of quasi-coherent sheaves is not abelian in general, infinite direct sums of quasi-coherent sheaves aren't quasi-coherent, etc, etc.\par
Let $(X,\mathscr{O}_X)$ be a ringed space. Let $\mathscr{F}$ be a sheaf of $\mathscr{O}_X$-modules. We say that $\mathscr{F}$ is a \textbf{quasi-coherent sheaf} of $\mathscr{O}_X$-modules if for each $x\in X$, there exists an open neighbourhood $U$ of $x$ such that $\mathscr{F}|_U$ is isomorphic to the cokernel of a homomorphism of the form $\mathscr{O}_X^{\oplus J}|_U\to\mathscr{O}_X^{\oplus I}|_U$, where $I$ and $J$ are arbitrary index sets. It is clear that $\mathscr{O}_X$ itself is a quasi-coherent $\mathscr{O}_X$-module, and a finite direct sum of quasi-coherent modules is quasi-coherent. The category of quasi-coherent $\mathscr{O}_X$-modules is denoted $\mathbf{QCoh}(\mathscr{O}_X)$. We say an $\mathscr{O}_X$-lagebra $\mathscr{A}$ is quasi-coherent if it is a quasi-coherent $\mathscr{O}_X$-module.\par
The definition of quasi-coherence amounts to saying that locally the sheaf $\mathscr{F}$ admits a \textit{presentation} by the structural sheaf $\mathscr{O}_X$. This definition is inspired by following ideal: for any module $M$ over a ring $A$, $M$ admits a presentation of the form
\[\begin{tikzcd}
A^{\oplus J}\ar[r]&A^{\oplus I}\ar[r]&M\ar[r]&0
\end{tikzcd}\]
where $I$ and $J$ are arbitrary index sets (and in particular may not be finite). Therefore, quasi-coherent sheaves can be seen as a "real module" over the ringed space $(X,\mathscr{O}_X)$, and this idea really makes sense in the realm of algebraic geometry.
\begin{proposition}\label{sheaf of module qcoh under inverse image}
Let $(f,f^{\hash}):(X,\mathscr{O}_X)\to(Y,\mathscr{O}_Y)$ be a morphism of ringed spaces. Then the pullback $f^*(\mathscr{G})$ of a quasi-coherent $\mathscr{O}_Y$-module $\mathscr{G}$ is quasi-coherent.
\end{proposition}
\begin{proof}
Since the question is local, we may assume that $\mathscr{G}$ has a global presentation by $\mathscr{O}_Y$. We have seen that $f^*$ commutes with all colimits, and is right exact, so if we have an exact sequence
\[\begin{tikzcd}
\mathscr{O}_Y^{\oplus J}\ar[r]&\mathscr{O}_Y^{\oplus I}\ar[r]&\mathscr{G}\ar[r]&0
\end{tikzcd}\]
then upon applying $f^*$ we obtain the exact sequence
\[\begin{tikzcd}
\mathscr{O}_X^{\oplus J}\ar[r]&\mathscr{O}_X^{\oplus I}\ar[r]&f^*(\mathscr{G})\ar[r]&0
\end{tikzcd}\]
This implies the assertion.
\end{proof}
\begin{proposition}\label{qcoh module associated with a module}
Let $(X,\mathscr{O}_X)$ be ringed space. Let $\rho:A\to\Gamma(X,\mathscr{O}_X)$ be a ring homomorphism from a ring $A$ into the ring of global sections on $X$ and $M$ be an $A$-module. Then the following three constructions give canonically isomorphic $\mathscr{O}_X$-modules:
\begin{itemize}
\item[(\rmnum{1})] Let $(\pi,\pi^{\hash}):(X,\mathscr{O}_X)\to(\{\ast\},A)$ be the morphism of ringed spaces where $\pi:X\to\{\ast\}$ is the unique map and $\pi^{\hash}$ is the given homomorphism $\rho:A\to\Gamma(X,\mathscr{O}_X)$. Set $\mathscr{F}_1=\pi^*(M)$.
\item[(\rmnum{2})] Choose a presentation $A^{\oplus J}\to A^{\oplus I}\to M\to 0$ and set
\[\mathscr{F}_2=\coker(\mathscr{O}_X^{\oplus J}\to\mathscr{O}_X^{\oplus I})\]
where the homomorphism is induced by $\rho$ and the matrix coefficients of the homomorphism in the presentation of $M$.
\item[(\rmnum{3})] Let $\mathscr{F}_3$ be the sheaf associated to the presheaf $U\mapsto\Gamma(U,\mathscr{O}_X)\otimes_AM$, where the homomorphism $A\to\Gamma(X,\mathscr{O}_X)$ is the composition of $\rho$ with the restriction map $\Gamma(X,\mathscr{O}_X)\to\Gamma(U,\mathscr{O}_X)$.
\end{itemize}
This construction has the following properties:
\begin{itemize}
\item[(a)] The resulting $\mathscr{O}_X$-modules $\mathscr{F}_M=\mathscr{F}_1=\mathscr{F}_2=\mathscr{F}_3$ is quasi-coherent.
\item[(b)] The construction gives a functor from the category of $A$-modules to the category of quasi-coherent sheaves on $X$ which commutes with arbitrary colimits.
\item[(c)] For any point $x\in X$ we have $\mathscr{F}_{M,x}=\mathscr{O}_{X,x}\otimes_AM$ which is functorial in $M$.
\item[(d)] For any $\mathscr{O}_X$-module $\mathscr{G}$ we have
\[\Hom_{\mathscr{O}_X}(\mathscr{F}_M,\mathscr{G})=\Hom_A(M,\Gamma(X,\mathscr{G}))\]
where the $A$-module structure on $\Gamma(X,\mathscr{G})$ is induced from the $\Gamma(X,\mathscr{O}_X)$-module structure via $\alpha$.
\end{itemize}
\end{proposition}
\begin{proof}
The isomorphism between $\mathscr{F}_1$ and $\mathscr{F}_3$ comes from the fact that $\pi^*$ is defined as the sheafification of the presheaf in (\rmnum{3}). The isomorphism between the constructions in (\rmnum{2}) and (\rmnum{1}) comes from the fact that the functor $\pi^*$ is right exact, so the sequence
\[\begin{tikzcd}
\pi^*(A^{\oplus I})\ar[r]&\pi^*(A^{\oplus J})\ar[r]&\pi^*(M)\to 0
\end{tikzcd}\] 
is exact, that $\pi^*$ commutes with arbitrary direct sums, and the fact that $\pi^*(A)=\mathscr{O}_X$.\par
Now assertion (a) is clear from construction (\rmnum{2}), so is (b) since $\pi^*$ has these properties. Assertion (c) follows from the description of stalks of pullback sheaves, and (d) follows from adjointness of $\pi^*$ and $\pi_*$.
\end{proof}
In the situation of Proposition~\ref{qcoh module associated with a module} we say $\mathscr{F}_M$ is the \textbf{sheaf associated to the module $\bm{M}$ and the ring map $\bm{\rho}$}. If $A=\Gamma(X,\mathscr{O}_X)$ and $\rho=1_A$, we simply say that $\mathscr{F}_M$ is the sheaf associated to the $A$-module $M$.
\begin{proposition}\label{qcoh module associated with a module and pull back}
Let $(X,\mathscr{O}_X)$ be a ringed space and $A=\Gamma(X,\mathscr{O}_X)$. Let $M$ be an $A$-module and $\mathscr{F}_M$ be the quasi-coherent sheaf of $\mathscr{O}_X$-modules associated to $M$. If $(f,f^{\hash}):(Y,\mathscr{O}_Y)\to(X,\mathscr{O}_X)$ is a morphism of ringed spaces, then $f^*(\mathscr{F}_M)$ is the sheaf associated to the $\Gamma(Y,\mathscr{O}_Y)$-module $\Gamma(Y,\mathscr{O}_Y)\otimes_AM$.
\end{proposition}
\begin{proof}
In view of the following diagram of ringed spaces
\begin{equation*}
\begin{tikzcd}
(Y,\mathscr{O}_Y)\ar[r,"\pi"]\ar[d,swap,"f"]&(\{\ast\},\Gamma(Y,\mathscr{O}_Y))\ar[d,"\text{induced by $f^{\hash}$}"]\\
(X,\mathscr{O}_X)\ar[r,"\pi"]&(\{\ast\},\Gamma(X,\mathscr{O}_X))
\end{tikzcd}
\end{equation*}
the assertion follows from the first description of $\mathscr{F}_M$ in Proposition~\ref{qcoh module associated with a module} as $\pi^*(M)$.
\end{proof}
To conclude this part, we prove an important result which will be used when we consider quasi-coherent sheaf on affine schemes. We state it in a general manner.
\begin{proposition}\label{quasi-coherent sheaf locally module on qc nbhd}
Let $(X,\mathscr{O}_X)$ be a ringed space and $x$ be a point of $X$. suppose that $x$ has a fundamental system of quasi-compact neighbourhoods, then for any quasi-coherent $\mathscr{O}_X$-module $\mathscr{F}$, there exists an open neighbourhood $U$ of $x$ such that $\mathscr{F}|_U$ is isomorphic to the sheaf of modules $\mathscr{F}_M$ on $(U,\mathscr{O}_U)$ associated to a $\Gamma(U,\mathscr{O}_U)$-module $M$.
\end{proposition}
\begin{proof}
Since $\mathscr{F}$ is quasi-coherent, we may replace $X$ by an open neighbourhood of $x$ and assume that $\mathscr{F}$ is isomorphic to the cokernel of a map $\mathscr{O}_X^{\oplus J}\to\mathscr{O}_X^{\oplus I}$. The problem is that this map may not be given by a matrix, because the global sections of a direct sum is in general different from the direct sum of the global sections.\par
Let $U$ be a quasi-compact neighbourhood of $x$. We proceed as in the proof of Proposition~\ref{sheaf module quasi-compact sum}. For each $j\in J$ denote $s_j\in\Gamma(X,\bigoplus_{i\in I}\mathscr{O}_X)$ the image of the section $1$ in the summand $\mathscr{O}_X$ corresponding to $j$. There exists a finite collection of opens $U_{jk},k\in K_j$ such that $U=\bigcup_{k\in K_j}U_{jk}$ and such that each restriction $s_j|_{U_{jk}}$ is a finite sum $\sum_{i\in I_{jk}}f_{jki}$ with $I_{jk}\sub I$. Let $I_j=\bigcup_{k\in K_j}I_{jk}$. This is a finite set since there are finitely many $U_{jk}$ and each $I_{jk}$ is finite. Since $U=\bigcup_{k\in K_j}U_{jk}$ the section $s_j|_U$ is a section of the finite direct sum $\bigoplus_{i\in I_j}\mathscr{O}_X$. Then by Proposition~\ref{sheaf module limit} we see that actually $s_j|_U$ is a sum $\sum_{i\in I_j}f_{ij}$ with $f_{ij}\in\mathscr{O}_X(U)=\Gamma(U,\mathscr{O}_U)$. At this point we can define a module $M$ as the cokernel of the map
\[\bigoplus_{j\in J}\Gamma(U,\mathscr{O}_U)\to\bigoplus_{i\in I}\Gamma(U,\mathscr{O}_U).\]
with matrix given by the $(f_{ij})$. By construction (\rmnum{2}) of Proposition~\ref{qcoh module associated with a module} we see that $\mathscr{F}_M$ has the same presentation as $\mathscr{F}|_U$ and therefore $\mathscr{F}_M\cong\mathscr{F}|_U$.
\end{proof}
\subsection{Sheaves of finite type}
Let $(X,\mathscr{O}_X)$ be a ringed space and $\mathscr{F}$ be an $\mathscr{O}_X$-modules. We say that $\mathscr{F}$ is \textbf{of finite type} if for every point $x\in X$ there exists an open neighbourhood $U$ of $x$ such that $\mathscr{F}|_U$ is generated by a finite family of sections over $U$, whence isomorphic to a quotient sheaf of a sheaf of the form $\mathscr{O}_X^n|_U$. It is clear that any quotient of a sheaf of finite type is of finite type, and a finite direct sum of sheaves of finite type is of finite type. 
\begin{proposition}\label{sheaf of module ft inverse image is ft}
Let $(f,f^{\hash}):(X,\mathscr{O}_X)\to(Y,\mathscr{O}_Y)$ be a morphism of ringed spaces. The pullback $f^*\mathscr{G}$ of a finite type $\mathscr{O}_Y$-module is a finite type $\mathscr{O}_X$-module.
\end{proposition}
\begin{proof}
Since the question is local, we may assume $\mathscr{G}$ is globally generated by finitely many sections. We have seen that $f^*$ commutes with all colimits, and is right exact, so if we have a surjection $\bigoplus_{i=1}^{n}\mathscr{O}_Y\to\mathscr{G}\to 0$, then by applying $f^*$ we obtain the surjection $\bigoplus_{i=1}^{n}\mathscr{O}_X\to f^*\mathscr{G}\to 0$.
\end{proof}
\begin{proposition}\label{sheaf of module ft local prop}
Let $\mathscr{F}$ be an $\mathscr{O}_X$-module of finite type.
\begin{itemize}
\item[(\rmnum{1})] If $(s_i)_{1\leq i\leq n}$ are sections of $\mathscr{F}$ over an open neighborhood $U$ of a point $x$ and if the $s_{i,x}$ generate $\mathscr{F}$, then there exists an open neighborhood $V\sub U$ of $x$ such that the $s_{i,y}$ generate $\mathscr{F}_y$ for any $y\in V$.
\item[(\rmnum{2})] If $\varphi:\mathscr{F}\to\mathscr{G}$ is a homomorphism such that $\varphi_x=0$, there exists an open neighborhood $U$ of $x$ such that $\varphi|_U=0$.
\item[(\rmnum{3})] If $\psi:\mathscr{G}\to\mathscr{F}$ is a homomorphism such that $\psi_x$ is surjective, then there exists an open neighborhood $V$ of $x$ such that $\psi|_V$ is surjective.
\item[(\rmnum{4})] The support of $\mathscr{F}$ is closed. 
\end{itemize}
\end{proposition}
\begin{proof}
We first prove (\rmnum{1}), so let $(t_j)_{1\leq j\leq m}$ be a family of sections of $\mathscr{F}$ over an open neighborhood $U'\sub U$ of $x$ generating $\mathscr{F}|_{U'}$. Since $(s_{i,x})$ generates $\mathscr{F}_x$, there exists sections $a_{ij}$ of $\mathscr{O}_X$ over an open neighborhood $U''\sub U$ of $x$ such that $t_{j,x}=\sum_{i=1}^{n}a_{ij,x}s_{i,x}$ for each $j$. We then conclude that there is an open neighborhood $V\sub U''$ of $x$ such that for each $y\in V$, we have $t_{j,y}=\sum_{i=1}^{n}a_{ij,y}s_{i,y}$, so $(s_{i,y})$ generates $\mathscr{F}_y$ for $y\in V$.\par
Assertion (\rmnum{4}) follows from (\rmnum{1}) by taking $n=1$ and $s_1=0$; also, (\rmnum{3}) follows from (\rmnum{4}) by considering $\coker\psi$, which is of finite type. Finally, (\rmnum{2}) follows from (\rmnum{4}) by considering $\im\varphi$, which is also of finite type. 
\end{proof}
\begin{corollary}\label{sheaf of module ft over lrs supp char by maximal ideal}
Let $(X,\mathscr{O}_X)$ be a locally ringed space and $\mathscr{F}$ be an $\mathscr{O}_X$-module of finite type. Then $\supp(\mathscr{F})$ is the set of $x\in X$ such that $\mathscr{F}_x/\m_x\mathscr{F}_x\neq 0$.
\end{corollary}
\begin{proof}
In fact, $\mathscr{F}_x$ is a finitely generated $\mathscr{O}_{X,x}$, so $\mathscr{F}_x=0$ if and only if $\m_x\mathscr{F}_x=\mathscr{F}_x$ by Nakayama's lemma.
\end{proof}
\begin{corollary}\label{sheaf of module ft over lrs supp of tensor}
Let $(X,\mathscr{O}_X)$ be a locally ringed space and $\mathscr{F}$, $\mathscr{G}$ be $\mathscr{O}_X$-modules of finite type. Then
\[\supp(\mathscr{F}\otimes_{\mathscr{O}_X}\mathscr{G})=\supp(\mathscr{F})\cap\supp(\mathscr{G}).\]
\end{corollary}
\begin{proof}
As the tensor product of two $\kappa(x)$-vector spaces is nonzero if both of them are nonzero, this follows from Corollary~\ref{sheaf of module ft over lrs supp char by maximal ideal}.
\end{proof}
\begin{corollary}
Let $f:X\to Y$ be a morphism of locally ringed spaces. For any $\mathscr{O}_Y$-module $\mathscr{G}$ of finite type, we have
\[\supp(f^*(\mathscr{G}))=f^{-1}(\supp(\mathscr{G})).\]
\end{corollary}
\begin{proof}
This follows from Corollary~\ref{sheaf of module ft over lrs supp char by maximal ideal} and Proposition~\ref{}.
\end{proof}
\begin{proposition}\label{sheaf of module ft over qc surjective homomorphism prop}
Suppose that $X$ is quasi-compact and let $\mathscr{F},\mathscr{G}$ be two $\mathscr{O}_X$-modules such that $\mathscr{G}$ is of finite type and $\mathscr{F}$ is the filtered limit $(\mathscr{F}_\lambda)$ of $\mathscr{O}_X$-modules. Let $\varphi:\mathscr{F}\to\mathscr{G}$ be a surjective homomorphism, then there exists an index $\lambda$ such that the homomorphism $\mathscr{F}_\lambda\to\mathscr{G}$ is surjective.
\end{proposition}
\begin{proof}
For any $x\in X$, there exists a finite system of sections $s_i$ of $\mathscr{G}$ over an open neighborhood $U(x)$ of $x$ such that $s_{i,y}$ generates $\mathscr{G}_y$ for any $y\in U(x)$. Then there is an open neighborhood $V(x)\sub U(x)$ of $x$ and sections $t_i$ of $\mathscr{F}$ over $V(x)$ such that $s_i|_{V(x)}=\varphi(t_i)$ for each $i$. We can then suppose that the $t_i$ are the images of sections of a single sheaf $\mathscr{F}_{\lambda(x)}$ over $V(x)$. Since $X$ is quasi-compact, it can be covered by finitely many $V(x_k)$, and let $\lambda$ be the supremum of the $\lambda(x_k)$ the assertion then follows.
\end{proof}
\begin{corollary}\label{sheaf of module ft over qc global generated then finite}
Suppose that $X$ is quasi-compact and let $\mathscr{F}$ be an $\mathscr{O}_X$-module of finite type that is generated by global sections. Then $\mathscr{F}$ is generated by finitely many global sections.
\end{corollary}
\begin{proof}
It suffices to cover $X$ be finitely many open neighborhoods $U_k$ such that for each $k$, there exists finitely many sections $s_{ik}$ of $\mathscr{F}$ over $X$ whose restrictions to $U_k$ generate $\mathscr{F}|_{U_k}$. It is clear that the $s_{ik}$ then generate $\mathscr{F}$.
\end{proof}
\begin{proposition}\label{sheaf of module ft exact sequence}
Let $X$ be a ringed space. Let 
\[\begin{tikzcd}
0\ar[r]&\mathscr{F}\ar[r,"\varphi"]&\mathscr{G}\ar[r,"\psi"]&\mathscr{H}\ar[r]&0
\end{tikzcd}\] 
be a short exact sequence of $\mathscr{O}_X$-modules. If $\mathscr{F}$ and $\mathscr{H}$ are of finite type, so is $\mathscr{G}$.
\end{proposition}
\begin{proof}
Since the question is local, we may assume that $\mathscr{F}$ (resp. $\mathscr{H}$) is generated by finitely many global sections $(s_i)_{1\leq i\leq n}$ (resp. $(t_j)_{1\leq j\leq m}$), and there are sections $(t_j')_{1\leq j\leq m}$ of $\mathscr{G}$ over $X$ such that $t_j=\psi(t_j')$ for each $j$. It is then clear that $\mathscr{G}$ is generated by the sections $\varphi(s_i)$ and $t_j'$.
\end{proof}
Let $(X,\mathscr{O}_X)$ be a ringed space. Let $\mathscr{F}$ be a sheaf of $\mathscr{O}_X$-modules. We say that $\mathscr{F}$ is \textbf{of finite presentation} if for every point $x\in X$ there exists an open neighbourhood $x\in U\sub X$, and $n,m\in\N$ such that $\mathscr{F}|_U$ is isomorphic to the cokernel of a homomorphism $\bigoplus_{j=1}^{m}\mathscr{O}_U\to \bigoplus_{j=1}^{n}\mathscr{O}_U$. This means that $X$ is covered by open sets $U$ such that $\mathscr{F}|_U$ has a presentation of the form
\[\begin{tikzcd}
\bigoplus_{j=1}^{m}\mathscr{O}_U\ar[r]&\bigoplus_{j=1}^{n}\mathscr{O}_U\ar[r]&\mathscr{F}|_U\ar[r]&0
\end{tikzcd}\]
As in the case of $\mathscr{O}_X$-modules, the pullback of a $\mathscr{O}_X$-module of finite presentation is of finite presentation. We also note that any $\mathscr{O}_X$-module of finite presentation is in particular quasi-coherent.
\begin{proposition}\label{sheaf of module fp prop}
Let $(X,\mathscr{O}_X)$ be a ringed space. Let $\mathscr{F}$ be a $\mathscr{O}_X$-module of finite presentation.
\begin{itemize}
\item[(a)] If $\psi:\mathscr{O}_X^{\oplus r}\to\mathscr{F}$ is a surjective homomorphism, then $\ker\psi$ is of finite type.
\item[(b)] If $\varphi:\mathscr{G}\to\mathscr{F}$ is a surjective homomorphism with $\mathscr{G}$ of finite type, then $\ker\theta$ is of finite type.
\end{itemize}
\end{proposition}
\begin{proof}

\end{proof}
\begin{proposition}\label{sheaf of module fp sheaf Hom bijective on stalk}
Let $(X,\mathscr{O}_X)$ be a ringed space and $\mathscr{F}$ be an $\mathscr{O}_X$-module of finite presentation. Then for any $\mathscr{O}_X$-module $\mathscr{H}$, the canonical homomorphism
\[(\sHom_{\mathscr{O}_X}(\mathscr{F},\mathscr{H}))_x\to\Hom_{\mathscr{O}_{X,x}}(\mathscr{F}_{x},\mathscr{G}_x)\]
is bijective.
\end{proposition}
\begin{proof}

\end{proof}
\begin{proposition}\label{sheaf of module fp isomorphic on stalk}
Let $(X,\mathscr{O}_X)$ be a ringed space and $\mathscr{F},\mathscr{G}$ be finitely presented $\mathscr{O}_X$-module. If for a point $x\in X$, $\mathscr{F}_x$ and $\mathscr{G}_x$ are isomorphic $\mathscr{O}_{X,x}$-modules, then there exists an open neighbourhood $U$ of $x$ such that $\mathscr{F}|_U\cong\mathscr{G}|_U$.
\end{proposition}
\begin{proof}
If $\varphi:\mathscr{F}_x\to\mathscr{G}_x$ and $\psi:\mathscr{G}_x\to\mathscr{F}_x$ are the isomorphisms, there exists, by Proposition~\ref{sheaf of module fp sheaf Hom bijective on stalk}, an open neighborhood $V$ of $x$ and a section $u$ (resp. $v$) of $\sHom_{\mathscr{O}_X}(\mathscr{F},\mathscr{G})$ (resp. $\sHom_{\mathscr{O}_X}(\mathscr{G}\circ\mathscr{F})$) over $V$ such that $u_x=\varphi$ (resp. $v_x=\psi$). As $(u\circ v)_x$ and $(v\circ u)_x$ are the identities, by Proposition~\ref{sheaf of module fp sheaf Hom bijective on stalk} again there exists an open neighborhood $U\sub V$ of $x$ such that $(u\circ v)|_U$ and $(v\circ u)|_U$ are idnetities, whence the proposition.
\end{proof}
\subsection{Coherent sheaves}
Let $(X,\mathscr{O}_X)$ be a ringed space. Let $\mathscr{F}$ be a sheaf of $\mathscr{O}_X$-modules. We say that $\mathscr{F}$ is a \textbf{coherent} $\mathscr{O}_X$-module if $\mathscr{F}$ is of finite type and for every open $U\sub X$ and every finite collection $s_1,\dots,s_n$ of sections of $\mathscr{F}$ over $U$, the kernel of the associated map $\bigoplus_{i=1}^{n}\mathscr{O}_U\to\mathscr{F}|_U$ is of finite type. The category of coherent $\mathscr{O}_X$-modules is denoted $\mathbf{Coh}(\mathscr{O}_X)$. This is a more reasonable object than the category of quasi-coherent sheaves, in the sense that it is at least an abelian subcategory of $\mathbf{Mod}(\mathscr{O}_X)$ no matter what $X$ is. However, the pullback of a coherent module is almost never coherent in the general setting of ringed spaces.
\begin{proposition}\label{sheaf of module coh is qcoh and fp}
Let $(X,\mathscr{O}_X)$ be a ringed space. Any coherent $\mathscr{O}_X$-module is of finite presentation and hence quasi-coherent.
\end{proposition}
\begin{proof}
Let $\mathscr{F}$ be a coherent sheaf on $X$ and let $x$ be a point of $X$. We may find an open neighbourhood $U$ and sections $(s_i)_{1\leq i\leq n}$ of $\mathscr{F}$ over $U$ such that the associated homomorphism $\varphi:\bigoplus_{i=1}^{n}\mathscr{O}_U\to\mathscr{F}|_U$ is surjective. Since $\ker\varphi$ is also of finite type, we may find an open neighbourhood $x\in V\sub U$ and sections $(t_j)_{1\leq j\leq m}$ of $\bigoplus_{i=1}^{m}\mathscr{O}_V$ which generate the kernel of $\varphi|_V$. Then over $V$ we get the presentation
\[\begin{tikzcd}
\bigoplus_{j=1}^{m}\mathscr{O}_V\ar[r]&\bigoplus_{i=1}^{n}\mathscr{O}_V\ar[r]&\mathscr{F}|_V\ar[r]&0
\end{tikzcd}\]
which shows that $\mathscr{F}$ is of finite presentation.
\end{proof}
\begin{example}
Suppose that $X$ is a point. In this case the definition above gives a notion for modules over rings. What does the definition of coherent mean? It is closely related to the notion of Noetherian, but it is not the same: namely, the ring $A=\C[x_1,x_2,x_3,\dots]$ is coherent as a module over itself but not Noetherian as a module over itself.
\end{example}
\begin{proposition}\label{sheaf of module coh prop}
Let $(X,\mathscr{O}_X)$ be a ringed space. 
\begin{itemize}
\item[(a)] Let $\varphi:\mathscr{F}\to\mathscr{G}$ be a homomorphism from an $\mathscr{O}_X$-module $\mathscr{F}$ of finite type to a coherent $\mathscr{O}_X$-module $\mathscr{G}$. Then $\ker\varphi$ is of finite type.
\item[(b)] Let $\varphi:\mathscr{F}\to\mathscr{G}$ be a homomorphism of coherent $\mathscr{O}_X$-modules. Then $\ker\varphi$ and $\coker\varphi$ are coherent.
\item[(c)] Given a short exact sequence of $\mathscr{O}_X$-modules $0 \to\mathscr{F}_1\to\mathscr{F}_2 \to\mathscr{F}_3\to 0$ if two out of three are coherent so is the third.
\item[(d)] The category of coherent $\mathscr{O}_X$-modules is abelian and the inclusion functor $\mathbf{Coh}(\mathscr{O}_X)\to\mathbf{Mod}(\mathscr{O}_X)$ is exact.
\end{itemize}
\end{proposition}
\begin{proof}
Let $\varphi:\mathscr{F}\to\mathscr{G}$ be a homomorphism where $\mathscr{F}$ is of finite type and $\mathscr{G}$ is coherent. Let us show that $\ker\varphi$ is of finite type. Pick $x\in X$ and choose an open neighbourhood $U$ of $x$ in $X$ such that $\mathscr{F}|_U$ is generated by $s_1,\dots,s_n$. By definition the kernel $\mathscr{K}$ of the induced map $\bigoplus_{i=1}^{n}\mathscr{O}_U\to\mathscr{G},e_i\mapsto\varphi(s_i)$ is of finite type. Hence $\ker\varphi$ which is the image of the composition $\mathscr{K}\to\bigoplus_{i=1}^{n}\mathscr{O}_U\to\mathscr{F}$ is of finite type.\par
Now consider the case of assertion (b). By assertion (a) the kernel of $\varphi$ is of finite type and hence is coherent as a subsheaf of $\mathscr{F}$. With the same hypotheses let us show that $\coker\varphi$ is coherent. Since $\mathscr{G}$ is of finite type so is $\coker\varphi$. Let $U\sub X$ be open and let $\widebar{s}_i\in\coker\varphi(U),1\leq i\leq n$ be sections. We have to show that the kernel of the associated morphism $\widebar{\psi}:\bigoplus_{i=1}^{n}\mathscr{O}_U\to\coker\varphi$ has finite type. There exists an open covering of $U$ such that on each open all the sections $\widebar{s}_i$ lift to sections $s_i$ of $\mathscr{G}$. Hence we may assume this is the case over $U$. Thus $\widebar{\psi}$ lifts to $\psi:\bigoplus_{i=1}^{n}\mathscr{O}_U\to\mathscr{G}$. Consider the following
diagram
\[\begin{tikzcd}
0\ar[r]&\ker\psi\ar[r]\ar[d]&\bigoplus_{i=1}^{n}\mathscr{O}_U\ar[d,equal]\ar[r]&\mathscr{G}\ar[r]\ar[d]&0\\
0\ar[r]&\ker\widebar{\psi}\ar[r]&\bigoplus_{i=1}^{n}\mathscr{O}_U\ar[r]&\coker\psi\ar[r]&0
\end{tikzcd}\]
By the snake lemma we have $\im\varphi\cong\coker(\ker\psi\to\ker\widebar{\psi})$, thus there is a short exact sequence $0\to\ker\psi\to\ker\widebar{\psi}\to\im\varphi\to 0$. Hence by Proposition~\ref{sheaf of module ft exact sequence} we see that $\ker\widebar{\psi}$ is of finite type.\par
Let $0 \to\mathscr{F}_1\to\mathscr{F}_2 \to\mathscr{F}_3\to 0$ be a short exact sequence of $\mathscr{O}_X$-modules. By part (b) it suffices to prove that if $\mathscr{F}_1$ and $\mathscr{F}_3$ are coherent so is $\mathscr{F}_2$. By Proposition~\ref{sheaf of module ft exact sequence} we see that $\mathscr{F}_2$ has finite type. Let $s_1,\dots,s_n$ be finitely many local sections of $\mathscr{F}_2$ defined over a common open $U$ of $X$. We have to show that the module of relations $\mathscr{K}$ between them is of finite type. Consider the following commutative diagram
\[\begin{tikzcd}
0\ar[r]&0\ar[r]\ar[d]&\bigoplus_{i=1}^{n}\mathscr{O}_U\ar[d]\ar[r]&\bigoplus_{i=1}^{n}\mathscr{O}_U\ar[r]\ar[d]&0\\
0\ar[r]&\mathscr{F}_1\ar[r]&\mathscr{F}_2\ar[r]&\mathscr{F}_3\ar[r]&0
\end{tikzcd}\]
with obvious notation. By the snake lemma we get a short exact sequence $0\to\mathscr{K}\to\mathscr{K}_3\to\mathscr{F}_1$ where $\mathscr{K}_3$ is the module of relations among the images of the sections $s_i$ in $\mathscr{F}_3$. Since $\mathscr{F}_3$ is coherent we see that $\mathscr{K}_3$ is finite type. Since $\mathscr{F}_1$ is coherent we see that the image $\mathscr{I}$ of $\mathscr{K}_3\to\mathscr{F}_1$ is coherent. Hence $\mathscr{K}$ is the kernel of the map $\mathscr{K}\to\mathscr{I}$ between a finite type sheaf and a coherent sheaves and hence finite type by (b).
\end{proof}
\begin{corollary}\label{sheaf of module coh 5 of 4}
Let $\begin{tikzcd}\mathscr{F}_1\ar[r,"u"]&\mathscr{F}_2\ar[r,"v"]&\mathscr{F}_3\ar[r,"w"]&\mathscr{F}_4\ar[r,"t"]&\mathscr{F}_5\end{tikzcd}$ be an exact sequence of $\mathscr{O}_X$-modules. If $\mathscr{F}_1,\mathscr{F}_2,\mathscr{F}_4$ and $\mathscr{F}_5$ are coherent, then $\mathscr{F}_3$ is coherent.
\end{corollary}
\begin{proof}
In fact, $\coker u=\mathscr{F}_2/\ker v$ and $\im w=\ker t$ are coherent, and it suffices to consider the exact sequence
\[\begin{tikzcd}
0\ar[r]&\coker u\ar[r]&\mathscr{F}_3\ar[r]&\im w\ar[r]&0
\end{tikzcd}\]
and apply Proposition~\ref{sheaf of module coh prop}.
\end{proof}
\begin{corollary}\label{sheaf of module coh sum and quotient}
Let $\mathscr{F}$ and $\mathscr{G}$ be two coherent subsheaves of a coherent sheaf $\mathscr{K}$. The sheaves $\mathscr{F}+\mathscr{G}$ and $\mathscr{F}/\mathscr{G}$ are coherent.
\end{corollary}
\begin{proof}
The sheaf $\mathscr{F}+\mathscr{G}$ is a subsheaf of $\mathscr{K}$ of fintie type, so it is coherent by definition. As for $\mathscr{F}\cap\mathscr{G}$, it is the kernel of $\mathscr{F}\to\mathscr{K}/\mathscr{G}$, so is coherent.
\end{proof}
\begin{corollary}\label{sheaf of module coh tensor and Hom}
If $\mathscr{F}$ and $\mathscr{G}$ are coherent $\mathscr{O}_X$-modules, so are $\mathscr{F}\otimes_{\mathscr{O}_X}\mathscr{G}$ and $\sHom_{\mathscr{O}_X}(\mathscr{F},\mathscr{G})$.
\end{corollary}
\begin{proof}
Since the question is local, we can assume that $\mathscr{F}$ is the cokernel of a homomorphism $\mathscr{O}_X^n\to\mathscr{O}_X^m$. Then $\mathscr{F}\otimes_{\mathscr{O}_X}\mathscr{G}$ is isomorphic to the cokernel of the homomorphism $\mathscr{O}_X^n\otimes_{\mathscr{O}_X}\mathscr{G}\to\mathscr{O}_X^m\otimes_{\mathscr{O}_X}\mathscr{G}$, which is identified with the cokernel of $\mathscr{G}^n\to\mathscr{G}^m$. Since $\mathscr{G}$ is coherent, we then see $\mathscr{F}\otimes_{\mathscr{O}_X}\mathscr{G}$ is also coherent.\par
Similarly, in view of Proposition~\ref{sheaf of module fp sheaf Hom bijective on stalk}, we have an exact sequence 
\end{proof}
\begin{corollary}
Let $\mathscr{F}$ be a coherent $\mathscr{O}_X$-module and $\mathscr{I}$ be an coherent ideal of $\mathscr{O}_X$. Then the $\mathscr{O}_X$-module $\mathscr{I}\mathscr{F}$ is coherent.
\end{corollary}
\begin{proof}
The image of the canonical homomorphism $\mathscr{I}\otimes_{\mathscr{O}_X}\mathscr{F}\to\mathscr{F}$ is $\mathscr{I}\mathscr{F}$, so this follows from Corollary~\ref{sheaf of module coh tensor and Hom}. 
\end{proof}
\begin{corollary}\label{sheaf of module coh injective on stalk prop}
Let $X$ be a ringed space and $\varphi:\mathscr{F}\to\mathscr{G}$ be a homomorphism of $\mathscr{O}_X$-modules. Assume $\mathscr{F}$ of finite type, $\mathscr{G}$ is coherent and the homomorphism $\varphi:\mathscr{F}_x\to\mathscr{G}_x$ is injective for a point $x\in X$. Then there exists an open neighbourhood $x\in U\sub X$ such that $\varphi|_U$ is injective.
\end{corollary}
\begin{proof}
Denote by $\mathscr{K}\sub\mathscr{F}$ the kernel of $\varphi$. By Proposition~\ref{sheaf of module coh prop} we see that $\mathscr{K}$ is a finite type $\mathscr{O}_X$-module. Our assumption is that $\mathscr{K}_x=0$, so by Proposition~\ref{sheaf of module ft local prop}(\rmnum{4}) there exists an open neighbourhood $U$ of $x$ such that $\mathscr{K}|_U=0$.
\end{proof}
We say an $\mathscr{O}_X$-algebra $\mathscr{A}$ is \textbf{coherent} if $\mathscr{A}$ is a coherent $\mathscr{O}_X$-module. In particular, $\mathscr{O}_X$ is a coherent $\mathscr{O}_X$-algebra if, for any open subset $U\sub X$ and any homomorphism $\mathscr{O}_X^n|_U\to\mathscr{O}_X|U$ of $\mathscr{O}_U$-modules, the kernel of this homomorphism is of finite type. We then say that $\mathscr{O}_X$ is a coherent sheaf of rings. If $\mathscr{O}_X$ is a coherent sheaf of rings, any $\mathscr{O}_X$-module of finite presentation is coherent in view of Proposition~\ref{sheaf of module coh prop}.
\begin{example}
The annihilator of a $\mathscr{O}_X$-module $\mathscr{F}$ is the kernel $\mathscr{I}$ of the homomorphism $\mathscr{O}_X\to\sHom_{\mathscr{O}_X}(\mathscr{F},\mathscr{F})$ which sends a section $s\in\Gamma(U,\mathscr{O}_X)$ to the multiplication by $s$ in $\Hom(\mathscr{F}|_U,\mathscr{F}|_U)$. If $\mathscr{O}_X$ is a coherent sheaf of rings and if $\mathscr{F}$ is a coherent $\mathscr{O}_X$-module, $\mathscr{I}$ is coherent by Proposition~\ref{sheaf of module coh prop}, and it follows from Proposition~\ref{sheaf of module fp sheaf Hom bijective on stalk} that for each $x\in X$, $\mathscr{I}_x$ is the annihilator of the $\mathscr{O}_{X,x}$-module $\mathscr{F}_x$.
\end{example}
\begin{proposition}\label{sheaf of module O/I coh iff}
Suppose that $\mathscr{O}_X$ is a coherent sheaf of rings and let $\mathscr{I}$ be a coherent ideal of $\mathscr{I}$. For an $(\mathscr{O}_X/\mathscr{I})$-module $\mathscr{F}$ to be coherent, it is necessary and sufficient that as an $\mathscr{O}_X$-module, $\mathscr{F}$ is coherent. In particular, $\mathscr{O}_X/\mathscr{I}$ is a coherent sheaf of rings.
\end{proposition}
\begin{proof}
We note that $\mathscr{O}_X/\mathscr{I}$ is a coherent $\mathscr{O}_X$-module. If $\mathscr{F}$ is a coherent $(\mathscr{O}_X/\mathscr{I})$, any point of $X$ admits an open neighborhood $U$ such that $\mathscr{F}|_U$ is the cokernel of a homomorphism $(\mathscr{O}_X/\mathscr{I})^m|_U\to (\mathscr{O}_X/\mathscr{I})^n|_U$, so $\mathscr{F}$ is a coherent $\mathscr{O}_X$-module.\par
Conversely, suppose that $\mathscr{F}$, as an $\mathscr{O}_X$-module, is coherent. First, since $\mathscr{F}$ is an $\mathscr{O}_X$-module of finite type, it is an $(\mathscr{O}_X/\mathscr{I})$-module of finite type. On ther other hand, let $U$ be an open subset of $X$ and $u:(\mathscr{O}_X/\mathscr{I})^n|_U\to\mathscr{F}|_U$ is an $(\mathscr{O}_X/\mathscr{I})$-homomorphism; by composing with the canonical homomorphism $v:\mathscr{O}_X^n|_U\to(\mathscr{O}_X/\mathscr{I})^n|_U$, we obtain a homomorphism $u\circ v:\mathscr{O}_X^n|_U\to\mathscr{F}|_U$, and $\ker(u\circ v)$ is by hypothesis of finite type. But since $v$ is surjective, $\ker u$ is the image of $\ker(u\circ v)$ by $v$, so is of finite type.
\end{proof}
\begin{proposition}\label{sheaf of module coh pullback if coh sheaf of ring}
Let $f:X\to Y$ be a morphism of ringed spaces and suppose that $\mathscr{O}_X$ is a coherent rings. Then for any coherent $\mathscr{O}_Y$-module $\mathscr{G}$, $f^*(\mathscr{G})$ is a coherent $\mathscr{O}_X$-module.
\end{proposition}
\begin{proof}
Let $V$ be an open subset of $Y$ such that $\mathscr{G}|_V$ is the cokernel of a homomorphism $v:\mathscr{O}_Y^n|_V\to\mathscr{O}_Y^n|_V$. As $f_U^*$ is right exact, we have $f^*(\mathscr{G})|_U=f_U^*(\mathscr{G}|_V)$ (where $U=f^{-1}(V)$) and the cokernel of the homomorphism $f_U^*(v):\mathscr{O}_X^m|_U\to\mathscr{O}_X^n|_U$ is coherent by Proposition~\ref{sheaf of module coh prop}. 
\end{proof}
\subsection{Locally free sheaves}
Let $(X,\mathscr{O}_X)$ be a ringed space. We say an $\mathscr{O}_X$-module $\mathscr{F}$ is \textbf{locally free} if, for any $x\in X$, there exists an open neighborhood $U$ of $x$ such that $\mathscr{F}|_U$ is isomorphic to an $(\mathscr{O}_X|_U)$-module of the form $\mathscr{O}_X^{\oplus I}|_U$. If for any open subset $U$ the set $I$ is finite, we say that $\mathscr{F}$ is \textbf{of finite rank}. If for any open subset $U$, the set $I$ has $n$ elements, we say that $\mathscr{F}$ is \textbf{of rank $\bm{n}$}. If $\mathscr{F}$ is a locally free $\mathscr{O}_X$-module of finite rank, for any point $x\in X$, $\mathscr{F}_x$ is a free $\mathscr{O}_X$-module of finite rank $n(x)$, and there exists an open neighborhood $U$ of $x$ such that $\mathscr{F}|_U$ is of rank $n(x)$. If $X$ is connected, we then see $n(x)$ is constant.\par
It is clear that any locally finite sheaf is quasi-coherent, and if $\mathscr{O}_X$ is a coherent sheaf of rings, any locally free $\mathscr{O}_X$-module of finite rank is coherent. If $\mathscr{E}$ is locally free, $\mathscr{E}\otimes_{\mathscr{O}_X}\mathscr{F}$ is an exact functor on the category of $\mathscr{O}_X$-modules.\par
We will mostly consider locally free $\mathscr{O}_X$-modules of finite rank, so when we mention the notation of a locally free $\mathscr{O}_X$-modules, it should be understood that we mean locally free $\mathscr{O}_X$-modules of finite rank.\par
\begin{example}
Let $\mathscr{F}$ be an $\mathscr{O}_X$-module of finite presentation. Then if for a point $x\in X$, $\mathscr{F}_x$ is a free module of rank $n$, there exists an open neighborhood $U$ of $x$ such that $\mathscr{F}|_U$ is locally free of rank $n$ (Proposition~\ref{sheaf of module fp isomorphic on stalk}).
\end{example}
\begin{proposition}\label{sheaf of module local free tensor with dual iso}
Let $\mathscr{E}$, $\mathscr{F}$ be $\mathscr{O}_X$-modules and consider the canonical functorial homomorphism
\[\mathscr{E}^*\otimes_{\mathscr{O}_X}\mathscr{F}=\sHom_{\mathscr{O}_X}(\mathscr{E},\mathscr{O}_X)\otimes_{\mathscr{O}_X}\mathscr{F}\to\sHom_{\mathscr{O}_X}(\mathscr{E},\mathscr{F})\]
If $\mathscr{F}$ or $\mathscr{E}$ is locally free of finite rank, then this homomorphism is bijective.
\end{proposition}
\begin{proof}
The homomorphism is defined by the following: for any open subset $U$ and a couple $(u,t)$, where $u\in\Gamma(U,\mathscr{E}^*)=\Hom(\mathscr{E}|_U,\mathscr{O}_X|_U)$ and $t\in\Gamma(U,\mathscr{F})$, we associate the element of $\Hom(\mathscr{E}|_U,\mathscr{F}|_U)$ whose stalk at each point $x\in U$ send $s_x\in\mathscr{E}_x$ to the element $u_x(s_x)t_s\in\mathscr{F}_x$. Since the question is local, we may assume that $\mathscr{E}=\mathscr{O}_X^n$ or $\mathscr{F}=\mathscr{O}_X^n$. As for any $\mathscr{O}_X$-module $\mathscr{G}$, $\sHom_{\mathscr{O}_X}(\mathscr{O}_X^n,\mathscr{G})$ is canonically isomorphic to $\mathscr{G}^n$ and $\sHom_{\mathscr{O}_X}(\mathscr{G},\mathscr{O}_X^n)$ is isomorphic to $(\mathscr{G}^n)^*$, we are reduced to the case $\mathscr{E}=\mathscr{F}=\mathscr{O}_X$, and the claim is then immediate.
\end{proof}
If $\mathscr{L}$ is locally free of rank $1$, so is its dual $\mathscr{L}^*=\sHom_{\mathscr{O}_X}(\mathscr{L},\mathscr{O}_X)$, because this is ture for $\mathscr{L}=\mathscr{O}_X$ and the question is local. Moreover, we have a canonical isomorphism
\[\sHom_{\mathscr{O}_X}(\mathscr{L},\mathscr{O}_X)\otimes_{\mathscr{O}_X}\mathscr{L}\cong\mathscr{O}_{X}.\]
In fact, it suffices to prove that the canonical homomorphism $\mathscr{O}_X\to\sHom_{\mathscr{O}_X}(\mathscr{L},\mathscr{L})$ is bijective, and for this we may assume that $\mathscr{L}=\mathscr{O}_X$, and then the claim is immediate. Due to this, we put $\mathscr{L}^{-1}=\sHom_{\mathscr{O}_X}(\mathscr{L},\mathscr{O}_X)$, and $\mathscr{L}^{-1}$ is called the inverse of $\mathscr{L}$.\par
If $\mathscr{L}$ and $\mathscr{L}'$ are two $\mathscr{O}_X$-modules locally free of rank $1$, so is their tensor product $\mathscr{L}\otimes_{\mathscr{O}_X}\mathscr{L}'$, since locally we have $\mathscr{O}_X\otimes_{\mathscr{O}_X}\mathscr{O}_X\cong\mathscr{O}_X$. For any integer $n\geq 1$, we denote by $\mathscr{L}^{\otimes n}$ the $n$-fold tensor product $\mathscr{L}$, which is also a locally free $\mathscr{O}_X$-module of rank $1$; by convention, we set $\mathscr{L}^{\otimes 0}=\mathscr{O}_X$ and $\mathscr{L}^{\otimes(-n)}=(\mathscr{L}^{-1})^{\otimes n}$. Then there exists a canonical isomorphism
\[\mathscr{L}^{\otimes m}\otimes_{\mathscr{O}_X}\mathscr{L}^{\otimes n}\cong\mathscr{L}^{\otimes(n+m)}\]
\begin{proposition}\label{sheaf of module local free inverse image}
Let $f:Y\to X$ be a morphism of ringed spaces. If $\mathscr{E}$ is a locally free $\mathscr{O}_X$-module (resp. locally free $\mathscr{O}_X$-module of rank $n$), then $f^*(\mathscr{E})$ is a locally free $\mathscr{O}_Y$-module (resp. locally free $\mathscr{O}_Y$-module of rank $n$). Moreover, we have a canonical isomorphism $f^*(\mathscr{E}^*)=(f^*(\mathscr{E}))^*$.
\end{proposition}
\begin{proof}
The first assertion from the fact that $f^*$ commutes with direct sums and $f^*(\mathscr{O}_X)=\mathscr{O}_Y$, the second assertin can be checked for the case $\mathscr{E}=\mathscr{O}_X$, since the question is local.
\end{proof}
Let $\mathscr{L}$ be a locally free $\mathscr{O}_X$-module of rank $1$. We denote by $\Gamma_*(X,\mathscr{L})$ or simple $\Gamma_*(\mathscr{L})$ the abelian group of the direct sum $\bigoplus_{n\in\Z}\Gamma(X,\mathscr{L}^{\otimes n})$. We endow it a graded ring structure by defining the product of a couple $(s_n,s_m)$, where $s_n\in\Gamma(X,\mathscr{L}^{\otimes n})$ and $s_m\in\Gamma(X,\mathscr{L}^{\otimes m})$, the section of $\mathscr{L}^{\otimes(n+m)}$ over $X$ corresponding to the section $s_n\otimes s_m$ of $\mathscr{L}^{\otimes n}\otimes_{\mathscr{O}_X}\mathscr{L}^{\otimes m}$. The associative of this multiplication is immediately verified, and it is clear that $\Gamma_*(X,\mathscr{L})$ is a covariant functor on $\mathscr{L}$ with values in the category of graded rings of type $\Z$.\par
If now $\mathscr{F}$ is an $\mathscr{O}_X$-module, we set
\[\Gamma_*(\mathscr{L},\mathscr{F})=\bigoplus_{n\in\Z}\Gamma(X,\mathscr{F}\otimes_{\mathscr{O}_X}\mathscr{L}^{\otimes n}).\]
We endow this abelian group a graded $\Gamma_*(\mathscr{L})$-module structure by the following: for any couple $(s_n,u_m)$, where $s_n\in\Gamma(X,\mathscr{L}^{\otimes n})$ and $u_m\in\Gamma(X,\mathscr{F}\otimes\mathscr{L}^{\otimes n})$, we associate the section of $\mathscr{F}\otimes\mathscr{L}^{\otimes(n+m)}$ that corresponds to $s_n\otimes u_m$; the verification that this defines a module structure is immediate. For fixed $X$ and $\mathscr{L}$, we see $\Gamma_*(\mathscr{L},\mathscr{F})$ is a covariant functor on $\mathscr{F}$ with values in the category of graded $\Gamma_*(\mathscr{L})$-modules. For $X$ and $\mathscr{F}$ fixed, this is a covariant functor on $\mathscr{L}$ iwth values in the category of abelian groups.\par
Let $f:Y\to X$ be a morphism of ringed spaces. The canonical homomorphism $\rho:\mathscr{L}^{\otimes n}\to f_*(f^*(\mathscr{L}^{\otimes n}))$ defines a homomorphism $\Gamma(X,\mathscr{L}^{\otimes n})\to\Gamma(Y,f^*(\mathscr{L}^{\otimes n}))$ of abelian groups, and as $f^*(\mathscr{L}^{\otimes n})=(f^*(\mathscr{L}))^{\otimes n}$, it gives a homomorphism $\Gamma_*(\mathscr{L})\to\Gamma_*(f^*(\mathscr{L}))$ of graded rings. Similarly, the canonical homomorphism $\mathscr{F}\otimes\mathscr{L}^{\otimes n}\to f_*(f^*(\mathscr{F}\otimes\mathscr{L}^{\otimes n}))$ defines a homomorphism $\Gamma(X,\mathscr{F}\otimes\mathscr{L}^{\otimes n})\to\Gamma(Y,f^*(\mathscr{F}\otimes\mathscr{L}^{\otimes n}))$ and as
\[f^*(\mathscr{F}\times\mathscr{L}^{\otimes n})=f^*(\mathscr{F})\otimes(f^*(\mathscr{L}))^{\otimes n}\]
these homomorphisms give rise to a bi-homomorphism $\Gamma_*(\mathscr{L},\mathscr{F})\to\Gamma_*(f^*(\mathscr{L}),f^*(\mathscr{F}))$ of graded modules.
\begin{proposition}\label{sheaf of module local free projection formula}
Let $f:X\to Y$ be a morphism of ringed spaces, $\mathscr{F}$ be an $\mathscr{O}_X$-module, and $\mathscr{E}$ be a locally free $\mathscr{O}_Y$-module of finite rank. Then there exists a canonical isomorphism
\[f_*(\mathscr{F})\otimes_{\mathscr{O}_Y}\mathscr{E}\cong f_*(\mathscr{F}\otimes_{\mathscr{O}_X}f^*(\mathscr{E})).\]
\end{proposition}
\begin{proof}
For any $\mathscr{O}_Y$-module $\mathscr{E}$, we have canonical homomorphisms
\[\begin{tikzcd}
f_*(\mathscr{F})\otimes_{\mathscr{O}_Y}\mathscr{E}\ar[r,"1\otimes\rho_{\mathscr{E}}"]&f_*(\mathscr{E})\otimes_{\mathscr{O}_Y}f_*(f^*(\mathscr{E}))\ar[r,"\alpha"]&f_*(\mathscr{F}\otimes_{\mathscr{O}_X}f^*(\mathscr{E}))
\end{tikzcd}\]
Since $\mathscr{E}$ is locally free and the question is local on $Y$, we may assume that $\mathscr{E}=\mathscr{O}_Y^n$; as $f_*$ and $f^*$ commutes with finite direct sums, we can also suppose that $n=1$, and in this case, the proposition follows directly from the definition and the relation $f^*(\mathscr{O}_Y)=\mathscr{O}_X$.
\end{proof}
\begin{proposition}\label{sheaf of module local free inverse image of Hom}
Let $f:X\to Y$ be a morphism of ringed spaces, $\mathscr{F},\mathscr{G}$ be two $\mathscr{O}_Y$-modules, and suppose that $\mathscr{F}$ is locally free of finite rank. Then the canonical homomorphism
\[f^*(\sHom_{\mathscr{O}_Y}(\mathscr{F},\mathscr{G}))\to\sHom_{\mathscr{O}_X}(f^*(\mathscr{F}),f^*(\mathscr{G}))\]
is an isomorphism.
\end{proposition}
\begin{proof}
Since the question is local on $Y$, we can assume that $\mathscr{F}=\mathscr{O}_Y^n$; then $\sHom_{\mathscr{O}_Y}(\mathscr{O}_Y,\mathscr{G})$ is identified with $\mathscr{G}^n$, $f^*(\mathscr{F})$ is identified with $\mathscr{O}_X^n$, and $\sHom_{\mathscr{O}_X}(f^*(\mathscr{F}),f^*(\mathscr{G}))$ is identified with $(f^*(\mathscr{G}))^n$, whence the assertion.
\end{proof}
Let $X$ be a ringed space. For any integer $n>0$, we prove that there is a \textit{set} $\mathfrak{M}_n$ (denoted also by $\mathfrak{M}_n(X)$) of locally free $\mathscr{O}_X$-module of rank $n$ such that any locally free $\mathscr{O}_X$-module of rank $n$ is isomorphic to an element of $\mathfrak{M}_n$. For this, we consider the set $\mathfrak{R}_n$ of couples $(\mathfrak{U},\Theta)$, where $\mathfrak{U}$ is an open covering of $X$ and $\Theta$ is a family $(\theta_{UV})_{(U,V)\in\mathfrak{U}\times\mathfrak{U}}$, where $\theta_{UV}$ is an automorphism of $\mathscr{O}_X^n|_{(U\cap V)}$, $\theta_{UU}$ is the identity automorphism of $\mathscr{O}_U^n$, and the family $(\theta_{UV})$ satisfies the cocycle conditions. Then any element $(\mathfrak{U},\Theta)$ of $\mathfrak{R}_n$ corresponds to a well-defined locally free $\mathscr{O}_X$-module of rank $n$. If we denote by $\mathfrak{L}_n$ the set of these $\mathscr{O}_X$-modules, then any locally free $\mathscr{O}_X$-module of rank $n$ is isomorphic to one of the elements in $\mathfrak{L}_n$; it then suffices to choose for $\mathfrak{M}_n$ a system of representatives of $\mathfrak{L}_n$ for the equivalence condition: $\mathscr{E}$ and $\mathscr{E}'$ are equivalent if they are isomorphic. For any locally free $\mathscr{O}_X$-module $\mathscr{E}$ of rank $n$, we denote by $\cl(\mathscr{E})$ the uique element of $\mathfrak{M}_n$ which is isomorphic to $\mathscr{E}$.\par
We can define a composition law on the set $\mathfrak{M}_1$ by associating two elements $\mathscr{L}$, $\mathscr{L}'$ of $\mathfrak{M}_1(X)$ the element $\cl(\mathscr{L}\otimes\mathscr{L}')$. It is clear that this law is associative and commutative and with identity element $\cl(\mathscr{O}_X)$. Moreover for any $\mathscr{L}\in\mathfrak{M}_1(X)$, $\cl(\mathscr{L}^{-1})$ is the inverse of $\mathscr{L}$ for this composition law. We then define a commutative group structure on $\mathfrak{M}_1(X)$, and this group is called the \textbf{Picard group} of the ringed space $X$ and denoted by $\Pic(X)$. We also note that there exists a canonical isomorphism
\[\varphi_X:H^1(X,\mathscr{O}_X^{\times})\stackrel{\sim}{\to}\Pic(X).\]
For this, we note that for any open subset $U$ of $X$ the multiplicative group $\Gamma(U,\mathscr{O}_X^{\times})$ is canonically identified with the grop of automorphisms of $\mathscr{O}_U$-module $\mathscr{O}_U$, which send each section $\eps$ of $\mathscr{O}_X^{\times}$ over $U$ to the automorphism $u:\mathscr{O}_U\to\mathscr{O}_U$ such that $u_x(s_x)=\eps_xs_x$ for any $x\in U$ and any $s_x\in\mathscr{O}_{X,x}$. Let $\mathfrak{U}=(U_\lambda)$ be an open covering of $X$; the datum that, given any couple $(\lambda,\mu)$ of indices, an automorphism $\theta_\lambda$ of $\mathscr{O}_X|_{U_\lambda\cap U_\mu}$ which satisfy the cocycle conditions, is then equivalent to giving a $1$-cochain of the covering $\mathfrak{U}$, with values in $\mathscr{O}_X^{\times}$. Similarly, the datum that, given an index $\lambda$, an automorphism $\omega_\lambda$ of $\mathscr{O}_{U_\lambda}$, is equivalent to giving a $0$-cochain of $\mathfrak{U}$ with values in $\mathscr{O}_X^{\times}$, and the coboundary of this cochain corresponds to the automorphisms $(\omega_\lambda|_{U_\lambda\cap U_\mu})\circ(\omega_\mu|_{U_\lambda\cap U_\mu})^{-1}$. We then corresponds any $1$-cocycle $(\theta_{\lambda\mu})$ of $\mathfrak{U}$ with values in $\mathscr{O}_X^{\times}$, the element $\cl(X)$ of $\Pic(X)$, where $\mathscr{L}$ is the locally free $\mathscr{O}_X$-module of rank $1$ defined by the family $\theta=(\theta_{\lambda\mu})$; two cohomologous cocycles then correspond to the same element of $\Pic(X)$, so we obtain a map $\varphi_{\mathfrak{U}}:H^1(\mathfrak{U},\mathscr{O}_X^{\times})\to\Pic(X)$. Moreover, if $\mathfrak{V}$ is a second open covering of $X$, which is a refinement of $\mathfrak{U}$, the diagram
\[\begin{tikzcd}
H^1(\mathfrak{U},\mathscr{O}_X^{\times})\ar[dd]\ar[rd,"\varphi_{\mathfrak{U}}"]&\\
&\Pic(X)\\
H^1(\mathfrak{V},\mathscr{O}_X^{\times})\ar[ru,"\varphi_{\mathfrak{V}}"]&
\end{tikzcd}\]
where the vertical arrow is the canonical homomorphism, is commutative. By passing to limit, we then obtain a map $\varphi_X:H^1(X,\mathscr{O}_X^{\times})\to\Pic(X)$, where the \v{C}ech cohomology group $\check{H}^1(X,\mathscr{O}_{X}^{\times})$ is canonically identified with $H^1(X,\mathscr{O}_X^{\times})$. The map $\varphi_X$ is clearly surjective, since any locally free sheaf of rank $1$ is defined by a $1$-cocycle. It is injective, because it suffices to show that the maps $\varphi_{\mathfrak{U}}$ are injective, and this follows from the definition of $H^1(\mathfrak{U},\mathscr{O}_X^{\times})$. It remains to prove that $\varphi_{\mathfrak{U}}$ is a homomorphism of groups. Let $\mathscr{L},\mathscr{L}'$ be two locally free $\mathscr{O}_X$-modules of rank $1$ such that, for any $\lambda$, $\mathscr{L}|_{U_\lambda}$ and $\mathscr{L}'|_{U_\lambda}$ are isomorphic to $\mathscr{O}_{U_\lambda}$. There then exists for each $\lambda$ an element $a_\lambda$ (resp. $a'_\lambda$) of $\Gamma(U_\lambda,\mathscr{L})$ (resp. $\Gamma(U_\lambda,\mathscr{L}')$) such that the elements of $\Gamma(U_\lambda,\mathscr{L})$ (resp. $\Gamma(U_\lambda,\mathscr{L}')$) are of the form $s_\lambda\cdot a_\lambda$ (resp. $s_\lambda\cdot a'_\lambda$) where $s_\lambda$ runs through $\Gamma(U_\lambda,\mathscr{O}_X)$. The corresponding cocycles $(\eps_{\lambda\mu})$, $(\eps_{\lambda\mu}')$ are such that the relation $s_\lambda\cdot a_\lambda=s_\mu\cdot a_\mu$ (resp. $s_\lambda\cdot a'_\lambda=s_\mu\cdot a'_\mu$) over $U_\lambda\cap U_\mu$ is equivalent to $s_\lambda=\eps_{\lambda\mu}s_\mu$ (resp. $s_\lambda=\eps'_{\lambda\mu}s_\mu$) over $U_\lambda\cap U_\mu$. As the sections of $\mathscr{L}\otimes\mathscr{L}'$ over $U_\lambda$ are the finite sums of $s_\lambda s'_\lambda\cdot(a_\lambda\otimes a'_\lambda)$ where $s_\lambda,s'_\lambda$ runs through $\Gamma(U_\lambda,\mathscr{O}_X)$, it is clear that the cocycle $(\eps_{\lambda\mu}\eps'_{\lambda\mu})$ corresponds to $\mathscr{L}\otimes\mathscr{L}'$, which completes the proof.\par
Let $f:X'\to X$ be a morphism of ringed spaces. If $\mathscr{L}_1$, $\mathscr{L}_2$ are two locally free $\mathscr{O}_X$-modules of rank $1$ and are isomorphic, the $\mathscr{O}_{X'}$-modules $f^*(\mathscr{L}_1)$ and $f^*(\mathscr{L}_2)$ are isomorphic. On the other hand, for any $\mathscr{O}_X$-modules $\mathscr{F}$, $\mathscr{G}$, we have $f^*(\mathscr{F}\otimes\mathscr{G})=f^*(\mathscr{F})\otimes f^*(\mathscr{G})$. We then conclude that the morphism $f$ defines a canonical homomorphism of abelian groups
\[\Pic(f):\Pic(X)\to\Pic(X').\]
On the other hand, we have a canonical homomorphism
\[H^1(f):H^1(X,\mathscr{O}_X^{\times})\to H^1(X',\mathscr{O}_{X'}^{\times})\]
corresponds the restriction of the homomorphism $f^{\hash}:\mathscr{O}_X\to f_*(\mathscr{O}_{X'})$ to $\mathscr{O}_X^{\times}$. We claim that the diagram
\[\begin{tikzcd}
H^1(X,\mathscr{O}_X^{\times})\ar[r,"H^1(f)"]\ar[d,swap,"\varphi_X","\sim"']&H^1(X',\mathscr{O}_{X'}^{\times})\ar[d,"\varphi_{X'}","\sim"']\\
\Pic(X)\ar[r,"\Pic(f)"]&\Pic(X')
\end{tikzcd}\]
is commutative. In fact, if $\mathscr{L}$ is given by the cocycle $(\eps_{\lambda\mu})$ of an open covering $(U_\lambda)$ of $X$, it suffices to prove that $f^*(\mathscr{L})$ is defined by a cocycle whose class is cohomologous to the image of $(\eps_{\lambda\mu})$ under $H^1(f)$. But if $\theta_{\lambda\mu}$ is the automorphism of $\mathscr{O}_X|_{U_\lambda\cap U_\mu}$ corresponding to $\eps_{\lambda\mu}$, it is clear that $f^*(\mathscr{L})$ is obtained by glueing $\mathscr{O}_{X'}|_{\psi^{-1}(U_\lambda)}$ with the automorphisms $g^*(\theta_{\lambda\mu})$, and it sufficies to verify that this corresponds to the cocycle $(f^{\hash}(\eps_{\lambda\mu}))$; this follows also from the definition by identify $\eps_{\lambda\mu}$ with its image under $\rho_{\mathscr{O}_X}$, which is a section of $\psi^{-1}(\mathscr{O}_X)$ over $\psi^{-1}(U_\lambda\cap U_\mu)$.
\subsection{Locally free sheaves over locally ringed spaces}
\begin{proposition}\label{sheaf of module local free section principal open}
Let $X$ be a locally ringed space and $\mathscr{E}$ be a locally free $\mathscr{O}_X$-module of finite rank. Then for any section $s$ of $\mathscr{E}$ over $X$, the set $X_s$ of $x\in X$ such that $s(x)\neq 0$ is open in $X$ and $s$ is invertible over $X_s$.
\end{proposition}
\begin{proof}
Since the question is local, we can assume that $\mathscr{E}=\mathscr{O}_X^n$. If $(s_j)_{1\leq j\leq n}$ is the projection of $s$ to the $n$-th component, we see $s(x)\neq 0$ if and only if $s_j(x)\neq 0$ for some $j$, so $X_s$ is the union of the $X_{s_j}$, and we are reduced to the case $n=1$. But for $s\in\Gamma(X,\mathscr{O}_X)$, to say that $s(x)\neq 0$ amounts to that $s_x\notin\m_x$, hence $s_x$ is inveritble in $\mathscr{O}_{X,x}$. By then there then existes an open neighborhood $U$ of $x$ in $X$ such that $s|_U$ is invertible in $\Gamma(U,\mathscr{O}_X)$, and therefore $s(y)\neq 0$ for any $y\in U$.
\end{proof}
\begin{corollary}\label{sheaf of module local free section li open}
Let $X$ be a locally ringed space and $\mathscr{E}$ be a locally free $\mathscr{O}_X$-module of finite rank. Let $s_1,\dots,s_p$ be sections of $\mathscr{E}$ over $X$. Then the set of $x\in X$ such that $s_1(x),\dots,s_p(x)$ are lineraly independent over $\mathscr{E}_x/\m_x\mathscr{E}_x$ is open in $X$.
\end{corollary}
\begin{proof}
In fact, $\bigw^p\mathscr{E}$ is a locally free $\mathscr{O}_X$-module of rank $\binom{n}{p}$. Moreover, for any $x\in X$, $(\bigw^p\mathscr{E})_x/\m_x(\bigw^p\mathscr{E})_x$ is identified with $\bigw^p(\mathscr{E}_x/\m_x\mathscr{E}_x)$. If $s=s_1\wedge\cdots\wedge s_p$, then $s(x)$ is identified with $s_1(x)\wedge\cdots\wedge s_p(x)$, and $s(x)\neq 0$ if and only if $s_1(x),\dots,s_p(x)$ are linearly independent, whence the assertion by Proposition~\ref{sheaf of module local free section principal open}. 
\end{proof}
\begin{corollary}\label{sheaf of module local free section define iso if li}
Let $X$ be a locally ringed space, $\mathscr{E}$ be a locally free $\mathscr{O}_X$-module of rank $n$, and $s_1,\dots,s_n$ be sections of $\mathscr{E}$ over $X$ such that, for any $x\in X$, $s_1(x),\dots,s_n(x)$ are lineraly independent. Then the homomorphism $u:\mathscr{O}_X^n\to\mathscr{E}$ defined by the sections $s_i$ is bijective.
\end{corollary}
\begin{proof}
Again we can assume that $\mathscr{E}=\mathscr{O}_X^n$ and canonically identity $\bigw^n\mathscr{E}$ with $\mathscr{O}_X$. The section $s=s_1\wedge\cdots\wedge s_n$ is then a section of $\mathscr{O}_X$ over $X$ such that $s(x)\neq 0$ for all $x\in X$, and thus invertible in $\Gamma(X,\mathscr{O}_X)$. We can then define a homomorphism inverse to $u$ by Cramer's rule.
\end{proof}
\begin{proposition}\label{sheaf of module homomorphism ft to local free prop}
Let $X$ be a locally ringed space, $\mathscr{F}$ be an $\mathscr{O}_X$-module of finite type, $\mathscr{E}$ a locally free $\mathscr{O}_X$-module of finite rank, and $u:\mathscr{F}\to\mathscr{E}$ be a homomorphism, and $x$ be a point of $X$. Then the following conditions are equivalent:
\begin{itemize}
\item[(\rmnum{1})] The homomorphism $u_x$ is left invertible (which means $u_x$ is injective and its image in $\mathscr{E}_x$ is a direct factor). 
\item[(\rmnum{2})] The homomorphism $u_x\otimes 1:\mathscr{F}_x/\m_x\mathscr{F}_x\to\mathscr{E}_x/\m_x\mathscr{E}_x$ of vector spaces induced by $u_x$ is injective.
\item[(\rmnum{3})] There exists an open neighborhood $U$ of $x$ such that $u|_U:\mathscr{F}|_U\to\mathscr{E}|_U$ is left invertible, the image $u(\mathscr{F})|_U$ is a locally free $\mathscr{O}_U$-module isomorphic to $\mathscr{F}|_U$, which admits a locally free complement in $\mathscr{E}|_U$.
\end{itemize}
Moreover, the set of $x\in X$ satisfying these equivalent conditions are open.
\end{proposition}
\begin{proof}
The equivalence of (\rmnum{1}) and (\rmnum{2}) is a general result in algebras, and it is clear that (\rmnum{3}) implies (\rmnum{1}). We now prove that (\rmnum{1}) implies (\rmnum{3}); there exists by hypothesis a homomorphism $w:\mathscr{E}_x\to\mathscr{F}_x$ such that $w\circ u_x$ is the identify automorphism on $\mathscr{F}_x$. As $\mathscr{E}$ is locally free of finite rank, hence of finite presentation, it follows from Proposition~\ref{sheaf of module fp sheaf Hom bijective on stalk} that there exists an open neigbourhood $U$ of $x$ and a homomorphism $v:\mathscr{E}|_U\to\mathscr{F}|_U$ such that $w=v_x$, hence $(v\circ(u|_U))_x=v_x\circ u_x$ is the identity automorphism. We then conclude that, by restricting $U$, we can suppose that $v\circ(u|_U)$ is the identity on $\mathscr{F}|_U$, which means $u|_U$ is left invertible. We then know that $p=(u|_U)\circ v$ is a projection from $\mathscr{E}|_U$ to $u(\mathscr{F})|_U$, and that $u|_U$ is an isomorphism from $\mathscr{F}|_U$ to $u(\mathscr{F})|_U$. We claim that (aftering shrinking $U$ if necessary), $u(\mathscr{E})_U$ and $\ker p$ are locally free $\mathscr{O}_U$-modules (supplementary in $\mathscr{F}|_U$). In fact, $p_x:\mathscr{F}_x\to(u(\mathscr{F}))_x$ is a projection, hence so is $p_x\otimes 1:\mathscr{E}_x/\m_x\mathscr{E}_x\to(u(\mathscr{F}))_x/\m_x(u(\mathscr{F}))_x$. There exists sections $s_j$ ($1\leq j\leq n$) of $\mathscr{F}$ over $U$ such that the first $m$ sections $s_j$ ($1\leq j\leq m$) are sections of $u(\mathscr{F})$, and the later $n-m$ are sections of $\ker p$, and that the $s_j(x)$ ($1\leq j\leq n$) form a base for $\mathscr{F}_x/\m_x\mathscr{F}_x$. In view of Corollary~\ref{sheaf of module local free section define iso if li}, by shrinking $U$, we can suppose that the $\mathscr{O}_X$-module $\mathscr{M}$ generated by $s_j$ for $1\leq j\leq n$ and the $\mathscr{O}_X$-module $\mathscr{N}$ generated by the $s_j$ for $m+1\leq j\leq n$ are free and supplementary in $\mathscr{E}|_U$. We have evidently $\mathscr{M}\sub u(\mathscr{F})|_U$ and $\mathscr{N}\sub\ker p$; on the other hand, if $i:\mathscr{M}\to u(\mathscr{F})|_U$ and $j:\mathscr{N}\to\ker p$ are the canonical injections, then the choice of $s_j$ implies that $i_x$ and $j_x$ are bijections. As $u(\mathscr{F})|_U$ and $\ker p$ are $\mathscr{O}_U$-modules of finite type (the second being the image of $\mathscr{E}|_U$ under $1-p$), we conclude from Proposition~\ref{sheaf of module ft local prop} (by shrinking $U$ if necessary) that $\mathscr{M}=u(\mathscr{E})|_U$ and $\mathscr{N}=\ker p$.
\end{proof}
\begin{corollary}\label{sheaf of module local free inverse image injective iff}
With the hypotheses in Proposition~\ref{sheaf of module homomorphism ft to local free prop}, the following conditions are equivalent:
\begin{itemize}
\item[(\rmnum{1})] For any morphism $g:X'\to X$ of locally ringed spaces, the homomorphism $g^*(u):g^*(\mathscr{F})\to g^*(\mathscr{E})$ is injective.
\item[(\rmnum{2})] For any $x\in X$, the homomorphism $u_x\otimes 1:\mathscr{F}_x/\m_x\mathscr{F}_x\to\mathscr{E}_x/\m_x\mathscr{E}_x$ is injective.
\item[(\rmnum{3})] For any $x\in X$, there exists an open neighborhood $U$ of $x$ such that $u|_U:\mathscr{F}|_U\to\mathscr{E}|_U$ is left invertible.
\end{itemize}
Moreover, if the conditions are satisfied, $\mathscr{F}$ is a locally free $\mathscr{O}_X$-module of finite rank.
\end{corollary}
\begin{proof}
The equivalence of (\rmnum{2}) and (\rmnum{3}) follows from Proposition~\ref{sheaf of module homomorphism ft to local free prop}, so does the last one. The fact that (\rmnum{3}) implies (\rmnum{1}) follows from follows from the fact that we can reduce ourselves to the case where $\mathscr{E}=\mathscr{O}_X^n$ and $\mathscr{F}=\mathscr{O}_X^m$ and that $g^*$ is left exact, since the question is local on $X$. Finally, we show that (\rmnum{2}) is a particular case of (\rmnum{1}): it suffices to consider the locally ringed space $X'$ reduced to a point $x$, with sheaf of rings $\kappa(x)$ (that is, $\Spec(\kappa(x))$). We let $g:X'\to X$ be the canonical morphism which maps $x$ to $x$ and $g^{\hash}:\mathscr{O}_X\to\kappa(x)$ is the cannical homomorphism $\Gamma(U,\mathscr{O}_X)\to\mathscr{O}_{X,x}\to\kappa(x)$ for any open neigbourhood $U$ of $x$. It is then easily verified that $g^*(u)$ is the homomorphism $u_x\otimes 1$. 
\end{proof}
\begin{remark}
If $u$ satisfies the conditions of Corollary~\ref{sheaf of module local free inverse image injective iff}, then we say that it is \textbf{universally injective}.
\end{remark}
\begin{corollary}\label{sheaf of module local free surjective iff}
Let $X$ be a locally ringed space, $\mathscr{F}$, $\mathscr{E}$ be two locally free $\mathscr{O}_X$-modules of finite rank, $u:\mathscr{F}\to\mathscr{E}$ be a homomorphism, and $x$ be a point of $X$. The following conditions are equivalent:
\begin{itemize}
\item[(\rmnum{1})] The homomorphism $u_x\otimes 1:\mathscr{F}_x/\m_x\mathscr{F}_x\to\mathscr{E}_x/\m_x\mathscr{E}_x$ is surjective.
\item[(\rmnum{2})] The homomorphism $u_x$ is surjective.
\item[(\rmnum{3})] The homomorphism $\bigw^mu_x:\bigw^m\mathscr{F}_x\to\bigw^m\mathscr{E}_x$ (where $m$ is the rank of $\mathscr{F}_x$) is surjective.
\item[(\rmnum{4})] The homomorphism $u_x^{t}:\mathscr{E}_x^*\to\mathscr{F}_x^*$ is left invertible.
\end{itemize}
Moreover, the set $S$ of $x\in X$ satisfying these conditions is open in $X$, $\ker(u)|_S$ is a locally free $\mathscr{O}_S$-module and any $x\in S$ admits an open neighborhood $U\sub S$ such that $\ker(u)|_U$ admits in $\mathscr{F}|_U$ a locally free complement (isomorphic to $\mathscr{E}|_U$). 
\end{corollary}
\begin{proof}
The equivalence of (\rmnum{1}) and (\rmnum{2}) follows from Nakayama's Lemma. Similarly, (\rmnum{3}) is equivalent to that
\[(\bigw^mu_x)\otimes 1:(\bigw^m\mathscr{F}_x)/\m_x(\bigw^m\mathscr{F}_x)\to(\bigw^m\mathscr{E}_x)/\m_x(\bigw^m\mathscr{E}_x)\]
is surjective; but $(\bigw^mu_x)\otimes 1$ is identified with
\[\bigw^m(u_x\otimes 1):\bigw^m(\mathscr{F}_x/\m_x\mathscr{F}_x)\to\bigw^m(\mathscr{E}_x/\m_x\mathscr{E}_x),\]
and as $\mathscr{E}_x/\m_x\mathscr{E}_x$ is a vector space of dimension $m$ over $\kappa(x)$, $\bigw^m(u_x\otimes 1)$ is surjective if $u_x\otimes 1$ is surjective, and is zero in the contrary case, whence the equivalence of (\rmnum{1}) and (\rmnum{3}). On the other hand, as $(\mathscr{F}^*)^*=\mathscr{F}$, $(\mathscr{E}^*)^*=\mathscr{E}$ and $(u^{t})^{t}=u$, it is the same to say that $u_x\otimes 1$ is surjective and that $(u_x\otimes 1)^t=(u_x)^t\otimes 1:\mathscr{E}_x^*/\m_x\mathscr{E}_x^*\to\mathscr{F}_x^*/\m_x\mathscr{F}_x^*$ is injective, whence the equivalence of (\rmnum{1}) and (\rmnum{4}) in view of Proposition~\ref{sheaf of module homomorphism ft to local free prop}. The fact that $S$ is open follows from Proposition~\ref{sheaf of module homomorphism ft to local free prop}. We can then reduce to the case where $S=X$, and the other assertions of the statement are deduced by transposing the conclusions of Proposition~\ref{sheaf of module homomorphism ft to local free prop} applied to $u^t$.
\end{proof}
\begin{corollary}\label{sheaf of module local free same rank inj iff bij}
With the notations of Corollary~\ref{sheaf of module local free surjective iff}, suppose moreover that $\mathscr{F}$ and $\mathscr{E}$ have the same rank at each point. Then, for any $x\in X$, the following conditions are equivalent:
\begin{itemize}
\item[(\rmnum{1})] $u_x$ is left invertible;
\item[(\rmnum{2})] $u_x$ is surjective;
\item[(\rmnum{3})] $u_x$ is bijective.
\end{itemize}
Moreover, the set of $x\in x$ satisfying these conditions is open in $X$.
\end{corollary}
\begin{corollary}\label{sheaf of module local free same rank sur of inverse image}
With the notations of Corollary~\ref{sheaf of module local free surjective iff}, let $f:X'\to X$ be a morphism of locally ringed spaces and put $\mathscr{F}'=f^*(\mathscr{F})$, $\mathscr{E}'=f^*(\mathscr{E})$, which are locally free $\mathscr{O}_{X'}$-modules of finite rank. Let $u'=f^*(u):\mathscr{F}'\to\mathscr{E}'$. Then for a point $x'\in X'$, $u'_{x'}$ is surjective (resp. left invertible) if and only if at the point $x=f(x')$, $u_x$ is surjective (resp. left invertible).
\end{corollary}
\begin{proof}
In fact, we have $\mathscr{F}'_{x'}=\mathscr{F}_x\otimes_{\mathscr{O}_{X,x}}\mathscr{O}_{X',x'}$, $\mathscr{E}'_{x'}=\mathscr{F}_x\otimes_{\mathscr{O}_{X,x}}\mathscr{O}_{X',x'}$, and $u'_{x'}$ is deduced from $u_x$ be base changing $\mathscr{O}_{X,x}$ to $\mathscr{O}_{X',x'}$. If $k$ and $k'$ are the residue fields of $x$ and $x'$, we then have $\mathscr{F}'_{x'}\otimes k'=(\mathscr{F}_x\otimes k)\otimes_kk'$, $\mathscr{F}'_{x'}\otimes k'=(\mathscr{F}_x\otimes k)\otimes_kk'$, and the homomorphism $u'_{x'}\otimes 1:\mathscr{F}'_{x'}\otimes k'\to\mathscr{E}'_{x'}\otimes k'$ is then deduced from $u_x\otimes 1_k:\mathscr{F}_x\otimes k\to\mathscr{E}_x\otimes k$ by base changing from $k$ to $k'$. The conclusion then follows from the fact that this base change is faithfully flat, the Nakayama lemma, and Proposition~\ref{sheaf of module homomorphism ft to local free prop}. 
\end{proof}
\begin{proposition}\label{sheaf of module local rank 1 iff invertible}
Let $X$ be a locally ringed space, $\mathscr{L}$ be an $\mathscr{O}_X$-module of finite type. For that $\mathscr{L}$ to be locally free of rank $1$, it is necessary and sufficient that there exists an $\mathscr{O}_X$-module $\mathscr{F}$ such that $\mathscr{L}\otimes_{\mathscr{O}_X}\mathscr{F}$ is isomorphic to $\mathscr{O}_X$. Moreover, any $\mathscr{O}_X$-module with this property is isomorphic to $\mathscr{L}^{-1}$.
\end{proposition}
\begin{proof}
We have seen that if $\mathscr{L}$ is locally of rank $1$ then $\mathscr{L}^{-1}\otimes_{\mathscr{O}_X}\mathscr{L}\cong\mathscr{O}_X$. Moreover, in this case $\mathscr{L}^{-1}$ is isomorphic to $\mathscr{L}^{-1}\otimes_{\mathscr{O}X}\mathscr{O}_X$, hence to $\mathscr{L}^{-1}\otimes_{\mathscr{O}_X}(\mathscr{L}\otimes_{\mathscr{O}_X}\mathscr{F})$ and therefore to $\mathscr{F}$. It then remains to prove that if $\mathscr{L}\otimes_{\mathscr{O}_X}\mathscr{F}$ is isomorphic to $\mathscr{O}_X$ then $\mathscr{L}$ is locally free of rank $1$. Let $x\in X$ and put $A=\mathscr{O}_{X,x}$ (which is a local ring with maximal ideal $\m$), $M=\mathscr{L}_x$, $N=\mathscr{F}_x$. The hypothesis implies that $M\otimes_AN$ is isomorphic to $A$, and as $(A/\m)\otimes_A(M\otimes_AN)$ is identified with $(M/\m M)\otimes_{A/\m}(N/\m N)$, this tensor product is isomorphic to $A/\m$ over the field $A/\m$, which shows that $M/\m M$ and $N/\m N$ are of dimension $1$ over $A/\m$. For any element $z\in M$ not belonging to $\m M$, we then have $M=Az+\m M$, which implies $M=Az$ by Nakayama's Lemma. Also, the annihilator of $z$ also annihilates $M\otimes_AN$, which is isomorphic to $A$, so this annihilator is zero and $M$ is isomorphic to $A$. There is then a section $s$ of $\mathscr{L}$ over an open neigbourhood $U$ of $x$ such that $t_x\mapsto t_xs_x$ is an isomorphism from $\mathscr{O}_{X,x}$ to $\mathscr{L}_x$. Since $\mathscr{L}$ is of finite type, we can, by shrinking $U$, suppose that $s$ generates $\mathscr{L}|_U$ (Proposition~\ref{sheaf of module ft local prop}), which means we have a surjective homomorphism $u:\mathscr{O}_U\to\mathscr{L}|_U$. Moreover, for any $y\in U$, the homomorphism $\mathscr{O}_{X,y}/\m_y\to\mathscr{L}_y/\m_y\mathscr{L}_y$ deduced from $u$ is bijective, hence so is $u$ (Proposition~\ref{localization map injective iff}).
\end{proof}
The $\mathscr{O}_X$-modules locally free of rank $1$ over a locally ringed space $X$ are then called the \textbf{invertible} $\mathscr{O}_X$-modules.
\begin{proposition}\label{sheaf of module invertible section zero iff}
Let $X$ be a locally ringed space, $\mathscr{L}$ be an invertible $\mathscr{O}_X$-module, and $f$ be a section of $\mathscr{L}$ over $X$. For any $x\in X$, the following conditions are equivalent:
\begin{itemize}
\item[(\rmnum{1})] $f_x$ generates the $\mathscr{O}_{X,x}$-module $\mathscr{L}_x$.
\item[(\rmnum{2})] $f_x\notin\m_x\mathscr{L}_x$ (that is, $f(x)\neq 0$).
\item[(\rmnum{3})] There exists a section $g$ of $\mathscr{L}^{-1}$ over an open neigbourhood $V$ of $x$ such that the canonical image of $(f|_V)\otimes g$ in $\Gamma(V,\mathscr{O}_X)$ is the unit element.
\end{itemize}
Moreover, the set $X_f$ of $x\in X$ satisfying these conditions is open in $X$.
\end{proposition}
\begin{proof}
The question is local on $X$ so we can assume that $\mathscr{L}=\mathscr{O}_X$, and the proposition then follows.
\end{proof}

\chapter{Cohomology group of sheaves}
In this chapter we consider the cohomology of shaves of modules. First we have a proposition.
\begin{proposition}
A sequence of sheaves of $\mathscr{O}_X$-modules on a space $X$
\[\begin{tikzcd}
0\ar[r]&\mathscr{F}\ar[r]&\mathscr{G}\ar[r]&\mathscr{H}\ar[r]&0
\end{tikzcd}\]
is exact if and only if for all points $x\in X$ the sequence of stalks is exact. This is equivalent to
\begin{itemize}
\item[(a)] For all open sets $U\sub X$ the sequence
\[\begin{tikzcd}
0\ar[r]&\mathscr{F}(U)\ar[r]&\mathscr{G}(U)\ar[r]&\mathscr{H}(U)
\end{tikzcd}\]
is exact.
\item[(b)] For any $s\in\mathscr{H}(U)$, we can find a covering $U=\bigcup_iU_i$ by open sets and $s_i\in\mathscr{F}(U_i)$ such that $s|_{U_i}=s_i$.
\end{itemize}
\end{proposition}
Applied to $U=X$ this tells us that the functor of global sections $\mathscr{F}\mapsto\mathscr{F}(X)$ is left-exact. It turns out that the category of sheaves of $\mathscr{O}_X$-modules has enough injectives, thus the right derived functors $\mathcal{R}^p\Gamma(X,-)$ exist, and for every sheaf $\mathscr{F}$ on $X$, the cohomology groups $\mathcal{R}^p\Gamma(X,-)(\mathscr{F})$ are defined. These groups denoted by $H^p(X,\mathscr{F})$ are called the \textbf{cohomology groups of the sheaf $\mathscr{F}$} or \textbf{the cohomology groups of $\bm{X}$ with values in $\mathscr{F}$}.
\section{Definition of sheaf cohomology}
We first show that the category $\mathbf{Mod}(\mathscr{O}_X)$ has enough injectives.
\begin{proposition}
Let $(X,\mathscr{O}_X)$ be a ringed space. Then the category $\mathbf{Mod}(\mathscr{O}_X)$ of sheaves of $\mathscr{O}_X$-modules has enough injectives.
\end{proposition}
\begin{proof}
Let $\mathscr{F}$ be a sheaf of $\mathscr{O}_X$-modules. For each point $x\in X$, the stalk $\mathscr{F}_x$ is an $\mathscr{O}_{X,x}$-module. Therefore there is an injection $\mathscr{F}_x\to I_x$, where $\mathscr{I}_x$ is an injective $\mathscr{O}_{X,x}$-module. For each point $x$, let $i_x$ denote the inclusion of the one-point space $\{x\}$ into $X$, and consider the sheaf $\mathscr{I}=\prod_{X\in X}i_{x,*}(I_x)$. Here we consider $I_x$ as a sheaf on the one-point space $\{x\}$.\par
Now for any sheaf $\mathscr{G}$ of $\mathscr{O}_X$-modules, we have
\[\Hom_{\mathscr{O}_{X}}(\mathscr{G},\mathscr{I})=\prod_{x\in X}\Hom_{\mathscr{O}_{X}}(\mathscr{G},i_{x,*}(I_x))\] 
by definition of the direct product. On the other hand, for each point $x\in X$, by Proposition~\ref{sheaf module stalk adj} we have
\[\Hom_{\mathscr{O}_{X}}(\mathscr{G},i_{x,*}(I_x))=\Hom_{\mathscr{O}_{X,x}}(\mathscr{G}_x,I_x)\]
Thus we conclude first that there is a natural morphism of sheaves of $\mathscr{O}_X$-modules $\mathscr{F}\to\mathscr{I}$ obtained from the local maps $\mathscr{F}_x\to\mathscr{I}_x$. It is clearly injective. Second, the functor $\Hom_{\mathscr{O}_{X}}(-,\mathscr{I})$ is the direct product over all $x\in X$ of the stalk functor $\mathscr{G}\to\mathscr{G}$ which is exact, followed by $\Hom_{\mathscr{O}_{X,x}}(-,I_x)$, which is exact, since $I_x$ is a injective. Hence $\Hom(-,\mathscr{I})$ is an exact functor, and therefore $\mathscr{I}$ is an injective $\mathscr{O}_X$-module.
\end{proof}
\begin{corollary}
If $X$ is any topological space, then the category $\mathbf{Ab}(X)$ of sheaves of abelian groups on $X$ has enough injectives.
\end{corollary}
\begin{definition}
Let $X$ be a topological space, and let $\Gamma(X,-):\mathbf{Ab}(X)\to\mathbf{Ab}(X)$ be the global section functor. The cohomology groups of the sheaf $\mathscr{F}$ or the cohomology groups of $X$ with values in $\mathscr{F}$, denoted by $H^i(X,\mathscr{F})$, are the groups $\mathcal{R}^i\Gamma(X,-)(\mathscr{F})$ induced by the right derived functor $\mathcal{R}^i\Gamma(X,-)$.
\end{definition}
Similarly, we can define cohomology groups of a $\mathscr{O}_X$-module to be the right-derived functor $\mathcal{R}^i\Gamma(X,-)$, where $\Gamma(X,-)$ is viewed as a functor from $\mathbf{Mod}(\mathcal{O}_X)$ to $\mathbf{Ab}(X)$. However, it turns out that this definition is unnecessary: the dericed functor of $\Gamma(X,-)$ in $\mathbf{Ab}(X)$ and $\mathbf{Mod}(\mathscr{O}_X)$ coincide.
\section{Flasque sheaves}
\begin{definition}
A sheaf $\mathscr{F}$ on a topological space $X$ is \textbf{flasque} if for every open subset $U$ of $X$ the restriction map $\mathscr{F}(X)\to\mathscr{F}(U)$ is surjective.
\end{definition}
We will see shortly that injective sheaves are flasque. Although this is not obvious from the definition, the notion of being flasque is local.
\begin{proposition}\label{flasque is local}
Let $\mathscr{F}$ be an $\mathscr{O}_X$-module. If $\mathscr{F}$ is flasque, so is $\mathscr{F}|_U$ for every open subset $U$ of $X$. Conversely, if for every $x\in X$, there is a neighborhood $U$ such that $\mathscr{F}|_U$ is flasque, then $\mathscr{F}$ is flasque.
\end{proposition}
\begin{proof}
The first statement is trivial, let us prove the converse. Given any open set $V$ of $X$, let $s$ be a section of $\mathscr{F}$ over $V$. Let $T$ be the set of all pairs $(U,\sigma)$, where $U$ is an open in $X$ containing $V$, and $\sigma$ is an extension of $s$ to $U$. Partially order $T$ by saying that $(U_1,\sigma_1)\leq(U_2,\sigma_2)$ if $U_1\sub U_2$ and $\sigma_2$ extends $\sigma_1$, and observe that $T$ is inductive, which means that every chain has an upper bound. Zorn's lemma provides us with a maximal extension of $s$ to a section $\sigma$ over an open set $U_0$. Were $U_0$ not $X$, there would exist an open set $W$ in $X$ not contained in $U_0$ such that $F|_W$ is flasque. Thus we could extend the section $\sigma|_{U_0\cap W}$ to a section $\sigma_0$ of $\mathscr{F}$. Since $\sigma$ and $\sigma_0$ agree on $U_0\cap W$ by construction, their common extension to $U_0\cup W$ extends $s$, a contradiction.
\end{proof}
\begin{proposition}
Every $\mathscr{O}_X$-module may be embedded in a canonical functorial way into a flasque $\mathscr{O}_X$-module. Consequently, every $\mathscr{O}_X$-module has a canonical flasque resolution.
\end{proposition}
\begin{proof}
Let $\mathscr{F}$ be an $\mathscr{O}_X$-module, and consider the Godement construction
\[U\mapsto\prod_{x\in U}\mathscr{F}_x\]
which we denote by $C^0(X,\mathscr{F})$. It is immediate that we have an injection of $\mathscr{O}_X$-modules $\mathscr{F}\to C^0(X,\mathscr{F})$ by Proposition~\ref{Goement construction image of sheaf char}. An element of $C^0(X,\mathscr{F})$ over any open set $U$ is a collection $(s_x)$ of elements indexed by $U$, each $s_x$ lying over the $\mathscr{O}_{X,x}$-module $\mathscr{F}_x$. Such a sheaf is flasque because every $U$-indexed sequence $s_x$ can be extended to an $X$-indexed sequence by assigning any arbitrary element of $\mathscr{F}_x$ to any $x\in X-U$. Hence $\mathbf{Mod}(\mathscr{O}_X)$ possesses enough flasque sheaves.\par
If $\mathscr{Z}_1$ is the cokernel of the canonical injection $\mathscr{F}\to C^0(X,\mathscr{F})$, we define $C^1(X,\mathscr{F})$ to be the flasque sheaf $C^0(X,\mathscr{Z}_1)$. In general, 
\[\mathscr{Z}_n=\coker\Big(\mathscr{Z}_{n-1}\hookrightarrow C^0(X,\mathscr{Z}_{n-1})\Big)\And C^n(X,\mathscr{F})=C^0(X,\mathscr{Z}_n).\] Putting all this information together, we obtain the desired flasque resolution of $\mathscr{F}$
\[\begin{tikzcd}
0\ar[r]&\mathscr{F}\ar[r]&C^0(X,\mathscr{F})\ar[r]&C^1(X,\mathscr{F})\ar[r]&C^2(X,\mathscr{F})\ar[r]&\cdots
\end{tikzcd}\]
as claimed.
\end{proof}
\begin{remark}
The resolution of $\mathscr{F}$ constructed above will be called the \textbf{canonical flasque resolution} of $\mathscr{F}$ or the \textbf{Godement resolution} of $\mathscr{F}$.
\end{remark}
Here is the principal property of flasque sheaves.
\begin{theorem}\label{flasque sheaf prop}
Let $0\to\mathscr{F}_1\to\mathscr{F}_2\to\mathscr{F}_3\to 0$ be an exact sequence of $\mathscr{O}_X$-modules, and assume $\mathscr{F}_1$ is flasque. Then this sequence is exact as a sequence of presheaves. If both $\mathscr{F}_1$ and $\mathscr{F}_2$ are flasque, so is $\mathscr{F}_3$. Finally, any direct summand of a flasque sheaf is flasque.
\end{theorem}
\begin{proof}
Given any open set $U$, we must prove that
\[\begin{tikzcd}
0\ar[r]&\mathscr{F}_1(U)\ar[r,"\varphi"]&\mathscr{F}_2(U)\ar[r,"\psi"]&\mathscr{F}_3(U)\ar[r]&0
\end{tikzcd}\]
is exact. By Proposition~\ref{sheaf cat monomorphism iff} and~\ref{sheaf cat epimorphism iff}, the sole problem is to prove that $\mathscr{F}_2(U)\to\mathscr{F}_3(U)$ is surjective. By restricting we only need to prove the case for $X$. Let $t$ be a global section of $\mathscr{F}_3$, then by Proposition~\ref{sheaf cat epimorphism iff}, locally $t$ may be lifted to sections of $\mathscr{F}$. Let $T$ be the family of all pairs $(U,\sigma)$ where $U$ is an open in $X$, and $\sigma$ is a section of $\mathscr{F}$ over $U$ whose image in $\mathscr{F}_3(U)$ equal $t|_{U}$. Partially order $T$ as in the proof of Proposition~\ref{flasque is local} and observe that $T$ is inductive. Zorn's lemma provides us with a maximal lifting of $t$ to a section $\sigma\in\mathscr{F}(U_0)$.\par
Were $U_0$ not $X$, there would exist $x\in X-U_0$, a neighborhood $V$ of $x$, and a section $\tau$ of $\mathscr{F}$ over $V$ which is a local lifting of $t|_V$. The sections $\sigma|_{V\cap U_0}$ and $\tau|_{V\cap U_0}$ have the same image in $\mathscr{F}_3(U_0\cap V)$ under the map $\psi$, so their difference maps to $0$. Since $\im\varphi=\ker\psi$, there is a section $s$ of $\mathscr{F}_1(U_0\cap V)$ such that
\[\sigma|_{U_0\cap V}=\tau|_{V\cap U_0}+\varphi(s).\]
Since $\mathscr{F}_1$ is flasque, the section $s$ is the restriction of a section $s_0\in\mathscr{F}_1(V)$. Upon replacing $\tau$ by $\tau+\varphi(t_0)$ (which does not affect the image in $\mathscr{F}_3(V)$), we may assume that $\sigma|_{V\cap U_0}=\tau|_{V\cap U_0}$; that is, $\tau$ and $\sigma$ agree on the overlap. Then we may extend $\sigma$ to $U_0\cup V$, contradicting the maximality of $(U_0,\sigma)$; hence, $U_0=X$.\par
Now suppose that $\mathscr{F}_1$ and $\mathscr{F}_2$ are flasque. If $t\in\mathscr{F}_3(U)$, then by the above, there is a section $s\in\mathscr{F}_2(U)$ mapping onto $t$. Since $\mathscr{F}_2$ is also flasque, we may lift $s$ to a global section $s_0$ of $\mathscr{F}$. The image $t_0$ of $s_0$ in $\mathscr{F}_3(X)$ is the required extension of $t$ to a global section of $\mathscr{F}_3$.\par
Finally, assume that $\mathscr{F}$ is flasque, and that $\mathscr{F}=\mathscr{F}_1\oplus\mathscr{F}_2$ for some sheaf $\mathscr{F}_1,\mathscr{F}_2$. For any open subset $U$ of $X$ and any section $s\in\mathscr{F}_1(U)$, we can make $s$ into a section $\tilde{s}\in\mathscr{F}(U)$ by setting the component of $\tilde{s}(U)$ in $\mathscr{F}_2(U)$ equal to the zero section. Since $\mathscr{F}$ is flasque, there is some section $t\in\mathscr{F}(X)$ such that $t|_U=\tilde{s}$. Write $t=t_1+t_2$ for some unique $t_1\in\mathscr{F}_1(X)$ and $t_2\in\mathscr{F}_2(X)$, then
\[s+0=\tilde{s}=t|_U=(t_1)|_U+(t_2)|_U\]
with $(t_1)|_U\in\mathscr{F}_1(U)$ and $(t_2)|_U\in\mathscr{F}_2(U)$, so $s=(t_1)|_U$ with $t_1\in\mathscr{F}_1(X)$, which shows that $\mathscr{F}_1$ is flasque.
\end{proof}
The following general proposition from Tohoku implies that flasque sheaves are $\Gamma(X,-)$-acyclic. It will also imply that soft sheaves are $\Gamma(X,-)$-acyclic. Since the only functor involved is the global section functor, it is customary to abbreviate $\Gamma(X,-)$-acyclic to acyclic.
\begin{theorem}\label{acyclic thm}
Let $\mathscr{F}$ be an additive functor from the abelian category $\mathcal{C}$ to the abelian category $\mathcal{D}$, and suppose that $\mathcal{C}$ has enough injectives. Let $\mathfrak{X}$ be a class of objects in $\mathcal{C}$ which satisfies the following conditions:
\begin{itemize}
\item $\mathcal{C}$ possesses enough $\mathfrak{X}$-objects.
\item If $0\to A_1\to A_2\to A_3\to 0$ is exact and if $A_1$ belongs to $\mathfrak{X}$, then $0\to\mathscr{F}(A_1)\to\mathscr{F}(A_2)\to\mathscr{F}(A_3)\to 0$ is exact.
\item If $0\to A_1\to A_2\to A_3\to 0$ is exact and $A_1,A_2$ belongs to $\mathfrak{X}$, then $A_3$ belongs to $\mathfrak{X}$.
\item If $A$ is an object of $\mathcal{C}$ and $A$ is a direct summand of some object in $\mathfrak{X}$, then $A$ belongs to $\mathfrak{X}$.
\end{itemize}
Then every injective object belongs to $\mathfrak{X}$, for each $M$ in $\mathfrak{X}$ we have $\mathcal{R}^i\mathscr{F}(M)=0$ for $i>0$, and finally the functors $\mathcal{R}^i\mathscr{F}$ may be computed by taking $\mathfrak{X}$-resolutions.
\end{theorem}
\begin{proof}
Let $I$ be an injective of $\mathcal{C}$. Then $I$ admits a monomorphism into some object $M$ of the class $\mathfrak{X}$. We have an exact sequence
\[\begin{tikzcd}
0\ar[r]&I\ar[r,"\varphi"]&M\ar[r]&\coker\varphi\ar[r]&0
\end{tikzcd}\]
and as $I$ is injective this sequence split. Thus $I$ is a direct summand of $M$ and thus belongs to $\mathfrak{X}$ be the condition.\par
Let us now show that $\mathcal{R}^n\mathscr{F}(M)=0$ for $i>0$ if $M$ lies in $\mathfrak{X}$. Now, $\mathcal{C}$ possesses enough injectives, so if we have an exact sequence
\[\begin{tikzcd}
0\ar[r]&M\ar[r]&I\ar[r]&K\ar[r]&0
\end{tikzcd}\]
where $I$ is injective. Then from the long exact sequence of $\mathcal{R}^\bullet\mathscr{F}$ and the fact $\mathcal{R}^i\mathscr{F}(I)=0$ for $i\geq 1$ we conclude
\[\mathcal{R}^1\mathscr{F}(M)=0\And \mathcal{R}^i\mathscr{F}(M)=\mathcal{R}^{i-1}\mathscr{F}(K)\text{ \textit{for} $i\geq 2$}.\]
Since $M$ and $I$ are both in $\mathfrak{X}$, $K$ is also in $\mathfrak{X}$. Therefore by applying the same argument on $K$ we also get $\mathcal{R}^1\mathscr{F}(K)=0$. Now the claim follows by an induction.
\end{proof}
This result tells us that flasque sheaves are acyclic for the functor $\Gamma(X,-)$. Hence we can calculate cohomology using flasque resolutions. 
\begin{proposition}\label{flasque is acyclic}
Flasque sheaves are acyclic, that is $H^i(X,\mathscr{F})=0$ for every flasque
sheaf $\mathscr{F}$ and all $i\geq1$, and the cohomology groups $H^i(X,\mathscr{F})$ of any arbitrary sheaf $\mathscr{F}$ can be computed using flasque resolutions.
\end{proposition}
\begin{proof}
Apply Theorem~\ref{acyclic thm} on the functor $\Gamma(X,-):\mathbf{Ab}(X)\to\mathbf{Ab}(X)$.
\end{proof}
In view of the proposition above, we also have the following result.
\begin{proposition}
Let $(X,\mathscr{O}_X)$ be a ringed space. Then the derived functors of the functor $\Gamma(X,-)$ from $\mathbf{Mod}(\mathscr{O}_X)$ to $\mathbf{Ab}(X)$ coincide with the cohomology functors $H^i(X,-)$.
\end{proposition}
\begin{proof}
Let $I$ be an injective $\mathscr{O}_X$-module, then $\mathscr{I}$ is a flasque $\mathscr{O}_X$-module by Theorem~\ref{acyclic thm}. Therefore an injective resolution in $\mathbf{Mod}(\mathscr{O}_X)$ is a flasque resolution in $\mathbf{Ab}(X)$, hence computes the sheaf cohomology.
\end{proof}
\section{Locality of cohomology}
we first sate a useful result in abelian categories.
\begin{theorem}\label{abelian cat adjoint exactness of one prop}
Let $(F,G)$ be an adjoint pair between abelian categories $\mathcal{C}$ and $\mathcal{D}$, in the sense that
\[\Hom_{\mathcal{D}}(F(X),Y)=\Hom_\mathcal{C}(X,G(Y)).\]
Assume that $F$ is exact, then $G$ maps injectives to injectives.
\end{theorem}
\begin{proof}
Let $I$ be injective in $\mathcal{D}$. Assume that we have an exact diagram in $\mathcal{C}$:
\[\begin{tikzcd}
0\ar[r]&X\ar[r]\ar[d]&Y\\
&G(I)
\end{tikzcd}\]
Then by applying the exact functor $F$ we get an exact diagram in $\mathcal{D}$:
\[\begin{tikzcd}
0\ar[r]&F(X)\ar[r]\ar[d]&F(Y)\ar[d,dashed]\\
&FG(I)\ar[r]&I
\end{tikzcd}\]
where $FG(I)\to I$ is the unit map of the adjunction $(F,G)$. Since $I$ is injective, this extends to a map $F(Y)\to I$. Applying $G$ again and compose the counit map $I\to GF(Y)$ we get the desired map $Y\to G(I)$, so $G(I)$ is injective.
\end{proof}
\begin{proposition}
Let $f:X\to Y$ be a morphism of ringed spaces. Assume $f$ is flat. Then $f_*\mathscr{I}$ is an injective $\mathscr{O}_Y$-module for any injective $\mathscr{O}_X$-module $\mathscr{I}$. In particular, the pushforward $f_*:\mathbf{Ab}(X)\to\mathbf{Ab}(Y)$ transforms injective abelian sheaves into injective abelian sheaves.
\end{proposition}
\begin{proof}
In this case $f^*$ is exact, and we have an adjoint pair $(f^*,f_*)$. Now apply Theorem~\ref{abelian cat adjoint exactness of one prop} we get the claim.
\end{proof}
The following lemma says there is no ambiguity in defining the cohomology of a sheaf $\mathscr{F}$ over an open.
\begin{lemma}
Let $X$ be a ringed space. Let $U\sub X$ be an open subspace.
\begin{itemize}
\item[(a)] If $\mathscr{I}$ is an injective $\mathcal{O}_X$-module then $\mathscr{I}|_U$ is an injective $\mathcal{U}$-module.
\item[(b)] For any sheaf of $\mathcal{O}_X$-modules $\mathscr{F}$ we have $H^p(U,\mathscr{F})=H^p(U,\mathscr{F}|_U)$.
\end{itemize}
\end{lemma}
\begin{proof}
Denote $j:U\to X$ the open immersion. Then $(j_!,j^{-1})$ satisfies the condition of Theorem~\ref{abelian cat adjoint exactness of one prop}. By definition $H^p(U,\mathscr{F})=H^p(\Gamma(U,\mathscr{I}^\bullet))$ where $\mathscr{F}\to\mathscr{I}^\bullet$ is an injective resolution in $\mathbf{Mod}(\mathcal{O}_X)$. By the above we see that $\mathscr{F}|_U\to\mathscr{I}|_U$ is an injective resolution in $\mathbf{Mod}(\mathcal{O}_U)$. Hence $H^p(U,\mathscr{F}|_U)$ is equal to $H^p(\Gamma(U,\mathscr{I}^\bullet|_U))$. Of course $\Gamma(U,\mathscr{F})=\Gamma(U,\mathscr{F}|_U)$ for any sheaf $\mathscr{F}$ on $X$. Hence the equality in (b).
\end{proof}
Let $f:X\to Y$ be a continuous. Since the functor $f_*$ is left-exact, for an injective resolution $0\to\mathscr{F}\to\mathscr{I}^\bullet$ we define 
\[\mathcal{R}^if_*\mathscr{F}=H^i(f_*\mathscr{I}^\bullet)\]
to be the \textbf{$\bm{i}$-th higher direct image of $\mathscr{F}$}.
\begin{proposition}\label{push derive char}
Let $f:X\to Y$ be a morphism of ringed spaces. Let $\mathscr{F}$ be a $\mathscr{O}_X$-module. The sheaves $\mathcal{R}^if_*\mathscr{F}$ are the sheaves associated to the presheaves
\[U\mapsto H^i(f^{-1}(U),\mathscr{F}).\]
\end{proposition}
\begin{proof}
Let $0\to\mathscr{F}\to\mathscr{I}^\bullet$ be an injective resolution. Then $\mathcal{R}^if_*\mathscr{F}$ is by definition the $i$-th cohomology sheaf of the complex
\[\begin{tikzcd}
f_*\mathscr{I}^0\ar[r]&f_*\mathscr{I}^1\ar[r]&f_*\mathscr{I}^2\ar[r]&\cdots
\end{tikzcd}\]
By definition of the abelian category structure on $\mathscr{O}_Y$-modules this cohomology sheaf is the sheaf associated to the presheaf
\[V\mapsto\frac{\ker(f_*\mathscr{I}^i(V)\to f_*\mathscr{I}^{i+1}(V))}{\im (f_*\mathscr{I}^{i-1}(V)\to f_*\mathscr{I}^{i}(V))}\]
and this is obviously equal to
\[\frac{\ker\big(\mathscr{I}^i(f^{-1}(V))\to\mathscr{I}^{i+1}(f^{-1}(V))\big)}{\im\big(\mathscr{I}^{i-1}(f^{-1}(V))\to\mathscr{I}^{i}(f^{-1}(V))\big)}\]
which is equal to $H^i(f^{-1}(V),\mathscr{F})$ and we win.
\end{proof}
\begin{corollary}
Let $f:X\to Y$ be a morphism of ringed spaces. Let $\mathscr{F}$ be a sheaf of $\mathscr{O}_X$-modules. If $\mathscr{F}$ is flasque, then $\mathcal{R}^pf_*\mathscr{F}=0$ for $p>0$.
\end{corollary}
\begin{proof}
This follows from Proposition~\ref{flasque is acyclic} and Proposition~\ref{push derive char}.
\end{proof}
\begin{proposition}
Let $f:X\to Y$ be a morphism of ringed spaces. Let $\mathscr{F}$ be an $\mathcal{O}_X$-module. Let $V\sub Y$ be an open subspace. Denote $g:f^{-1}(V)\to V$ the restriction of $f$. Then we have
\[\mathcal{R}^ig_*(\mathscr{F}|_{f^{-1}(V)})=(\mathcal{R}^if_*\mathscr{F})|_V.\]
There is a similar statement for the derived image $\mathcal{R}^if_*\mathscr{F}$ where $\mathscr{F}^\bullet$ is a bounded
below complex of $\mathcal{O}_X$-modules.
\end{proposition}
\begin{proof}
Choose an injective resolution $0\to\mathscr{F}\to\mathscr{I}^\bullet$ and use that $\mathscr{F}|_{f^{-1}(V)}\to\mathscr{I}|_{f^{-1}(V)}$ is an injective resolution also.
\end{proof}
\chapter{\v{C}ech cohomology}
\section{The \v{C}ech cohomology group with respect to a covering}
Let $X$ be a topological space and $\mathcal{U}=(U_i)_{i\in I}$ be an open covering of $X$. Given any finite sequence $(i_0,\dots,i_p)$ of elements of $I$, we let
\[U_{i_0\dots i_p}:=U_{i_0}\cap\cdots\cap U_{i_p}.\]
Also, we denote by $U_{i_0\dots\widehat{i_j}\dots i_p}$ the intersection
\[U_{i_0\dots\widehat{i_j}\dots i_p}=U_{i_0}\cap\cdots\cap\widehat{U}_{i_j}\cap\cdots\cap U_{i_p}.\]
Then we have $p+1$ inclusion maps
\[\delta^p_j:U_{i_0\dots i_p}\to U_{i_0\dots\widehat{i_j}\dots i_p}\for 0\leq j\leq p.\]
\begin{definition}
Given a topological space $X$, an open cover $\mathcal{U}=(U_i)_{i\in I}$ of $X$, and a abelian presheaf $\mathscr{F}$ on $X$, the \textbf{\v{C}ech $\bm{p}$-cochains} $C^p(\mathcal{U},\mathscr{F})$ is the set of all functions $f$ with domain $I^{p+1}$ such that $f(i_0,\dots,i_{p})\in\mathscr{F}(U_{i_0}\dots i_p)$; in other words,
\[C^p(\mathcal{U},\mathscr{F})=\prod_{(i_0,\dots,i_p)\in I^{p+1}}\mathscr{F}(U_{i_0\dots i_p})\]
the set of all $I^{p+1}$-indexed families $(f_{i_0\dots i_p})$ with $f_{i_0\dots i_p}\in\mathscr{F}(U_{i_0\dots i_p})$.
\end{definition}
\begin{remark}
Since $\mathscr{F}(\emp)=0$, for any cochain $f\in C^p(\mathcal{U},\mathscr{F})$, if $U_{i_0\dots i_p}=\emp$, then $f_{i_0\dots i_p}=0$. Therefore, we could define $C^p(\mathcal{U},\mathscr{F})$ as the set of families $f_{i_0\dots i_p}$ corresponding to tuples $(i_0,\dots,i_p)\in I^{p+1}$ such that $U_{i_0\dots i_p}\neq\emp$.
\end{remark}
Each inclusion map $\delta^p_j:U_{i_0\dots i_p}\to U_{i_0\dots\widehat{i_j}\dots i_p}$ induces a map
\[\mathscr{F}(\delta^p_j):\mathscr{F}(U_{i_0\dots\widehat{i_j}\dots i_p})\to\mathscr{F}(U_{i_0\dots i_p})\]
which is none other that the restriction map.
\begin{definition}
Given a topological space $X$, an open cover $\mathcal{U}=(U_i)_{i\in I}$ of $X$, and a abelian presheaf $\mathscr{F}$ on $X$, the coboundary maps $d:C^p(\mathcal{U},\mathscr{F})\to C^{p+1}(\mathcal{U},\mathscr{F})$ are given by
\[d=\sum_{j=0}^{p+1}(-1)^j\mathscr{F}(\delta^{p+1}_j).\]
More explicitly, for any $p$-cochain $f\in C^p(\mathcal{U},\mathscr{F})$, for any sequence $(i_0,\dots,i_{p+1})\in I^{p+2}$, we have
\[(df)_{i_0\dots i_{p+1}}=\sum_{j=0}^{p+1}(-1)^j(f_{i_0\dots\widehat{i_j}\dots i_{p+1}})|_{U_{i_0\dots i_{p+1}}}.\]
\end{definition}
\begin{proposition}
With the notations above, we have $d^2=0$. Thus we obtain a complex $C^\bullet(\mathcal{U},\mathscr{F})$.
\end{proposition}
\begin{proof}
This is a typical computation. Let $f\in C^p(\mathcal{U},\mathscr{F})$,
\begin{align*}
(d^2f)_{i_0\dots i_{p+2}}&=\sum_{k=0}^{p+2}(-1)^k\big((df)_{i_0\dots\widehat{i_k}\dots i_{p+2}}\big)|_{U_{i_0\dots i_{p+2}}}\\
&=\sum_{j<k}(-1)^k\big((-1)^j(f_{i_0\dots\widehat{i_j}\dots\widehat{i_k}\dots i_{p+1}})|_{U_{i_0\dots\widehat{i_k}\dots i_{p+2}}}\big)|_{U_{i_0\dots i_{p+2}}}\\
&+\sum_{j>k}(-1)^k\big((-1)^{j-1}(f_{i_0\dots\widehat{i_k}\dots\widehat{i_j}\dots i_{p+1}})|_{U_{i_0\dots\widehat{i_k}\dots i_{p+2}}}\big)|_{U_{i_0\dots i_{p+2}}}\\
&=\sum_{j<k}(-1)^k(-1)^j(f_{i_0\dots\widehat{i_j}\dots\widehat{i_k}\dots i_{p+1}})|_{U_{i_0\dots i_{p+2}}}\\
&-\sum_{j>k}(-1)^k(-1)^{j}(f_{i_0\dots\widehat{i_k}\dots\widehat{i_j}\dots i_{p+1}})|_{U_{i_0\dots i_{p+2}}}\\
&=0
\end{align*}
as desired.
\end{proof}
Therefore, we can form the \v{C}ech cohomology groups with respect to $\mathcal{U}$ as follows.
\begin{definition}
Given a topological space $X$, an open cover $\mathcal{U}=(U_{i})_{i\in I}$ of $X$, and a abelian presheaf $\mathscr{F}$ on $X$, we define
\[\check{H}^p(\mathcal{U},\mathscr{F})=H^p(C^\bullet(\mathcal{U},\mathscr{F}))\]
to be the \textbf{$\bm{p}$-th \v{C}ech-cohomology group with respect to the covering $\mathcal{U}$}.
\end{definition}
First of all, we note that $\check{H}^0(\mathcal{U},\mathscr{F})$ can be easily computed.
\begin{proposition}\label{Cech H^0}
Given a topological space $X$, an open cover $\mathcal{U}$ of $X$, and a sheaf $\mathscr{F}$ on $X$, then
\[\check{H}^0(\mathcal{U},\mathscr{F})=\mathscr{F}(X)=\Gamma(X,\mathscr{F}).\]
More generally, if $\mathscr{F}$ is an abelian presheaf, then the following
are equivalent
\begin{itemize}
\item $\mathscr{F}$ is a sheaf.
\item For every open covering $\mathcal{U}=(U_i)_{i\in I}$ the natural map
\[\mathscr{F}(U)\to\check{H}^0(\mathcal{U},\mathscr{F})\]
is bijective.
\end{itemize}
\end{proposition}
\begin{proof}
By definition, $\check{H}^0(\mathcal{U},\mathscr{F})=\ker(C^0(\mathcal{U},\mathscr{F})\to C^1(\mathcal{U},\mathscr{F}))$. If $f\in C^0(\mathcal{U},\mathscr{F})$ is given by $(f_i)$, then for each $(i,j)\in I^2$, $(df)_{i,j}=f_j-f_i$. So $df=0$ says the sections $f_i$ and $f_j$ agree on $U_i\cap U_j$. Thus it follows that $\ker d=\Gamma(X,\mathscr{F})$ if and only if $\mathscr{F}$ is a sheaf.
\end{proof}
An element of $C^p(\mathcal{U},\mathscr{F})$ is called a \textbf{$\bm{p}$-cochain}. We say that a $p$-cochain $f\in C^p(\mathcal{U},\mathscr{F})$ is \textbf{alternating} if
\begin{itemize}
\item $f_{i_0\dots i_p}=0$ whenever any two of the indices $i_0,\dots,i_p$ are equal.
\item For every permutation $\sigma$ of the indices, we have $f_{i_{\sigma(0)}\dots i_{\sigma(p)}}=(-1)^\sigma f_{i_0\dots i_p}$.
\end{itemize} 
It is clear that if $f\in C^p(\mathcal{U},\mathscr{F})$ is alternating, then $df$ is also alternating, hence we get a complex $C_{alt}^\bullet(\mathcal{U},\mathscr{F})$.
\begin{definition}
Let $X$ be a topological space. Let $\mathcal{U}=(U_i)_{i\in I}$ be an open covering. Let $\mathscr{F}$ be an abelian presheaf on $X$. The complex $C^\bullet_{alt}(\mathcal{U},\mathscr{F})$ is the \textbf{alternating \v{C}ech complex} associated to $\mathscr{F}$ and the open covering $\mathcal{U}$.
\end{definition}
Let us endow the set of indices $I$ with a total ordering and set
\[C_{ord}^{p}(\mathcal{U},\mathscr{F}):=\prod_{i_0<\cdots<i_p}\mathscr{F}(U_{i_0\dots i_p})\]
It can immediately be verified that the differential $d$ of $C^\bullet(\mathcal{U},\mathscr{F})$ induces a differential on $C_{ord}^\bullet(\mathcal{U},\mathscr{F})$, by restriction.
\begin{definition}
Let $X$ be a topological space. Let $\mathcal{U}=(U_i)_{i\in I}$ be an open covering. Let $\mathscr{F}$ be an abelian presheaf on $X$. Assume given a total ordering on $I$. The complex $C^\bullet_{ord}(\mathcal{U},\mathscr{F})$ is the \textbf{ordered \v{C}ech complex} associated to $\mathscr{F}$, the open covering $\mathcal{U}$ and the given total ordering on $I$.
\end{definition}
There is an obvious comparison map between the ordered \v{C}ech complex and the alternating \v{C}ech complex. Namely, consider the map
\[c:C^\bullet_{ord}(\mathcal{U},\mathscr{F})\to C^\bullet_{alt}(\mathcal{U},\mathscr{F})\]
given by the rule
\[c(s)_{i_0\dots i_p}=\begin{cases}
(-1)^\sigma s_{i_{\sigma(0)},\dots,i_{\sigma(p)}}&\text{if } i_{\sigma(0)}<\dots<i_{\sigma(p)},\\
0&\text{if $i_n=i_m$ for some $n\neq m$}.
\end{cases}\]
The alternating and ordered \v{C}ech complexes are often identified in the literature via the map $c$. Namely we have the following easy lemma.
\begin{lemma}
Let $X$ be a topological space. Let $\mathcal{U}$ be an open covering. Assume $I$ comes equipped with a total ordering. The map $c$ is a morphism of complexes. In fact it induces an isomorphism
\[c:C^\bullet_{ord}(\mathcal{U},\mathscr{F})\to C^\bullet_{alt}(\mathcal{U},\mathscr{F})\]
of complexes.
\end{lemma}
There is also a map $\pi:C^\bullet(\mathcal{U},\mathscr{F})\to C^\bullet_{ord}(\mathcal{U},\mathscr{F})$ which is described by the rule
\[\pi(s)_{i_0\dots i_p}=s_{i_0\dots i_p}\]
whenever $i_0<\dots<i_p$. The following result is immediate.
\begin{lemma}
Let $X$ be a topological space. Let $\mathcal{U}$ be an open covering. Assume $I$ comes equipped with a total ordering. The map 
$\pi:C^\bullet(\mathcal{U},\mathscr{F})\to C^\bullet_{ord}(\mathcal{U},\mathscr{F})$ is a morphism of complexes which is a left inverse to the morphism $c$. Moreover, 
it induces an isomorphism
\[\tilde{\pi}:C^\bullet_{alt}(\mathcal{U},\mathscr{F})\to C^\bullet_{ord}(\mathcal{U},\mathscr{F})\]
of complexes.
\end{lemma}
It turns out that the maps $\pi$ and $c$ give a homotopy equivalence between $C^\bullet(\mathcal{U},\mathscr{F})$ and $C^\bullet_{alt}(\mathcal{U},\mathscr{F})$. Therefore, we can 
use the alternating complex to compute the \v{C}ech cohomology.
\begin{theorem}
Let $X$ be a topological space. Let $\mathcal{U}$ be an open covering. Assume $I$ comes equipped with a total ordering. The map $c\circ\pi$ is homotopic to the identity on $C^\bullet(\mathcal{U},\mathscr{F})$. In particular the inclusion map $C^\bullet_{alt}(\mathcal{U},\mathscr{F})\to C^\bullet(\mathcal{U},\mathscr{F})$ is a homotopy equivalence.
\end{theorem}
\begin{corollary}
Given a total ordering on $I$, there exists a canonical isomorphism \[H^\bullet(C_{ord}^\bullet(\mathcal{U},\mathscr{F}))\cong H^\bullet(C^\bullet(\mathcal{U},\mathscr{F})).\]
\end{corollary}
\begin{corollary}
If $\mathcal{U}$ is made up of $n$ open subsets, then $\check{H}^p(\mathcal{U},\mathscr{F})=0$ for every $p\geq n$.
\end{corollary}
\begin{proof}
Indeed, if $p\geq n$, there does not exist any strictly increasing $(p+1)$-uple of indices $i_0,\dots,i_p$. Hence $C_{ord}^p(\mathcal{U},\mathscr{F})=0$, whence $\check{H}^p(\mathcal{U},\mathscr{F})=0$.
\end{proof}
\section{\v{C}ech cohomology as a functor on presheaves}
Warning: In this subsection we work almost exclusively with presheaves and categories of presheaves and the results are completely wrong in the setting of sheaves and categories of sheaves!\par
Let $X$ be a ringed space. Let $\mathcal{U}=(U_i)_{i\in I}$ be an open covering. Let $\mathscr{F}$ be a presheaf of $\mathcal{O}_X$-modules. We have the \v{C}ech complex $C^\bullet(\mathcal{U},\mathscr{F})$ of $\mathscr{F}$ just by thinking of $\mathscr{F}$ as a presheaf of abelian groups. However, each term $C^p(\mathcal{U},\mathscr{F})$ has a natural structure of a $\mathcal{O}_X(X)$-module and the differential is given by $\mathcal{O}_X(X)$-module maps. Moreover, it is clear that the construction
\[\mathscr{F}\to C^\bullet(\mathcal{U},\mathscr{F})\]
is functorial in $\mathscr{F}$. In fact, it is a functor
\[C^\bullet(\mathcal{U},-):\mathbf{PMod}(\mathscr{O}_X)\to\mathcal{C}^+(\mathbf{Mod}_{\mathscr{O}_X(X)})\]
\begin{proposition}\label{Cech complex exact presheaf}
The functor $C^\bullet(\mathcal{U},-)$ is an exact functor.
\end{proposition}
\begin{proof}
For any open $U\sub X$ the functor $\mathscr{F}\to\mathscr{F}(U)$ is an additive exact functor from $\mathbf{PMod}(\mathcal{O}_X)$ to $\mathbf{Mod}_{\mathscr{O}_X(X)}$. The terms $C^p(\mathcal{U},\mathscr{F})$ of the complex are products of these exact functors and hence exact. Moreover a sequence of complexes is exact if and only if the sequence of terms in a given degree is exact. Hence the lemma follows.
\end{proof}
\begin{theorem}\label{Cech presheaf delta functor}
Let $X$ be a ringed space. Let $\mathcal{U}=(U_i)_{i\in I}$ be an open covering. The functors $\check{H}^p(\mathcal{U},-)$ form a $\delta$-functor from the abelian category of presheaves of $\mathscr{O}_X$-modules to the category of $\mathscr{O}_X(X)$-modules
\end{theorem}
\begin{proof}
By Proposition~\ref{Cech complex exact presheaf} a short exact sequence of presheaves of $\mathscr{O}_X$-modules is turned into a short exact sequence of complexes of $\mathscr{O}_X(X)$-modules. Hence we can get a long exact sequence.
\end{proof}
\begin{proposition}
Let $X$ be a ringed space. Let $\mathcal{U}$ be an open covering of $X$. The \v{C}ech cohomology functors $\check{H}^p(\mathcal{U},-)$ are canonically isomorphic as a $\delta$-functor to the right derived functors of the functor 
\[\check{H}^0(\mathcal{U},-):\mathbf{PMod}(\mathscr{O}_X)\to\mathbf{Mod}_{\mathscr{O}_X(X)}.\]
Moreover, there is a functorial quasi-isomorphism
\[C^\bullet(\mathcal{U},\mathscr{F})\to\mathcal{R}\check{H}^0(\mathcal{U},\mathscr{F}).\]
\end{proposition}
\begin{proof}
This comes from the universal property of the $\delta$-functor.
\end{proof}
\section{The \v{C}ech cohomology groups}
Our next goal is to define \v{C}ech cohomology groups $\check{H}^p(X,\mathscr{F})$ that are independent of the open cover $\mathcal{U}$ chosen for $X$.\par
We say that a covering $\mathcal{V}=(V_j)_{j\in J}$ of $X$ is a \textbf{refinement} of another covering $\mathcal{U}=\{U_i\}_{i\in I}$ if there exists a map $\tau:J\to I$ such that $V_j\sub U_{\tau(j)}$ for every $j\in J$. We then have a homomorphism, which we also denote by $\tau$:
\[C^p(\mathcal{U},\mathscr{F})\to C^p(\mathcal{V},\mathscr{F})\]
defined by
\[\tau(f)_{j_0\dots j_p}=f_{\tau(j_0)\dots\tau(j_p)}|_{V_{j_0\dots j_p}}\]
This homomorphism commutes with the differentials and therefore induces a
homomorphism
\[\tau^*:\check{H}^p(\mathcal{U},\mathscr{F})\to\check{H}^p(\mathcal{V},\mathscr{F}).\]
\begin{proposition}\label{Cech refinement map}
The homomorphisms $\tau^*:\check{H}^p(\mathcal{U},\mathscr{F})\to\check{H}^p(\mathcal{V},\mathscr{F})$ depend only on $\mathcal{U}$ and $\mathcal{V}$ and not on the chosen mapping $\tau$.
\end{proposition}
\begin{proof}
Let $\tau_1$ and $\tau_2$ be two mappings from $I$ to $J$ such that $V_j\sub U_{\tau_1(j)}$ and $V_j\sub U_{\tau_2(j)}$, we have to show that $\tau_1^*=\tau_2^*$.\par
Let $f\in C^p(\mathcal{U},\mathscr{F})$, define a map $\kappa:C^p(\mathcal{U},\mathscr{F})\to C^{p-1}(\mathcal{V},\mathscr{F})$ by setting
\[(\kappa f)_{j_0\dots j_{p-1}}=\sum_{k=0}^{p-1}(-1)^k(f_{\tau_1(j_1)\dots\tau_1(j_k)\tau_2(j_k)\dots\tau_2(j_{p-1})})|_{V_{j_0\dots j_{p-1}}}.\]
Then, it can be verified that
\[\kappa(df)+d(\kappa f)=\tau_2(f)-\tau_1(f).\]
Thus $k$ defines a homotopy from $\tau_2^*$ to $\tau_1^*$, which implies the claim.
\end{proof}
\begin{corollary}
If $\mathcal{V}$ is a refinement of $\mathcal{U}$ and if $\mathcal{U}$ is a refinement of $\mathcal{V}$, then $\check{H}^p(\mathcal{U},\mathscr{F})\to\check{H}^p(\mathcal{V},\mathscr{F})$ is an isomorphism.
\end{corollary}
\begin{proof}
Let us keep the notation above. If $\mathcal{U}$ is a refinement of a covering $\mathcal{W}$ with a map $\sigma$ such that $U_i\sub W_{\sigma(i)}$, then $\mathcal{V}$ is a refinement of $\mathcal{W}$ because $V_j\sub W_{\sigma\circ\tau(j)}$. Moreover, $(\sigma\circ\tau)^*=\sigma^*\circ\tau^*$ in an obvious way. Let us now take $\mathcal{W}=\mathcal{V}$. Then $\sigma^*\circ\tau^*$ and $\tau^*\circ\sigma^*$ coincides with $\mathbf{1}^*$ by the result above. Hence $\tau^*$ is bijective.
\end{proof}
The relation $\mathcal{U}$ is a refinement of $\mathcal{V}$ (which we denote henceforth by $\mathcal{U}\prec\mathcal{V}$) is a relation of a preorder between coverings of $X$; moreover, this relation is filtered, since if $\mathcal{U}=\{U_i\}_{i\in I}$ and $\mathcal{V}=\{V_j\}_{j\in J}$ are two coverings, then the covering $\mathcal{W}=\{U_i\cap V_j\}_{(i,j)\in I\times J}$ is a refinement of $\mathcal{U}$ and $\mathcal{V}$. Consequently, it appears that the family $(\check{H}^p(\mathcal{U},\mathscr{F}))_{\mathcal{U}}$ is a direct mapping family of groups indexed by the directed set of open covers of $X$. However, there is a set-theoretic diffculty, which is that the family of open covers of $X$ is not a set because it allows arbitrary index sets.\par
A way to circumvent this difficulty is provided by Serre. The key observation is that any covering $(U_i)_{i\in I}$ is equivalent to a covering $(U'_{j})_{j\in L}$ whose index set $L$ is a subset of $2^X$. Indeed, we can take for $(U'_{j})_{j\in L}$ the set of all open subsets of $X$ that belong to the family $(U_i)_{i\in I}$.\par
As we noted earlier, if $\mathcal{U}$ and $\mathcal{V}$ are equivalent, then there is an isomorphism between $\check{H}^p(\mathcal{U},\mathscr{F})$ and $\check{H}^p(\mathcal{V},\mathscr{F})$, so we can define
\[\check{H}^p(X,\mathscr{F})=\rlim\check{H}^p(\mathcal{U},\mathscr{F})\]
with respect to coverings $\mathcal{U}$ whose index set is a subset of $2^X$. In summary, we have the following definition.
\begin{definition}
Given a topological space $X$ and a abelian presheaf $\mathscr{F}$ on $X$, the \textbf{\v{C}ech cohomology groups} $\check{H}^p(X,\mathscr{F})$ with values in $\mathscr{F}$ are defined by
\[\check{H}^p(X,\mathscr{F})=\rlim\check{H}^p(\mathcal{U},\mathscr{F})\]
with respect to coverings $\mathcal{U}$ whose index set is a subset of $2^X$.
\end{definition}
\section{Long exact sequence of \v{C}ech-cohomology}
\subsection{General case}
Let $0\to\mathscr{A}\to\mathscr{B}\to\mathscr{C}\to 0$ be an exact sequence of sheaves. If $\mathcal{U}$ is a covering of $X$, the sequence
\[\begin{tikzcd}
0\ar[r]&C^\bullet(\mathcal{U},\mathscr{A})\ar[r,"\alpha"]&C^\bullet(\mathcal{U},\mathscr{B})\ar[r,"\beta"]&C^\bullet(\mathcal{U},\mathscr{C})
\end{tikzcd}\]
is obviously exact, but the homomorphism $\beta$ need not be surjective in general. Denote by $C_0^\bullet(\mathcal{U},\mathscr{C})$ the image of this homomorphism; it is a subcomplex of $C^\bullet(\mathcal{U},\mathscr{C})$ whose cohomology groups will be denoted by $\check{H}^p_0(\mathcal{U},\mathscr{C})$. The exact sequence of complexes:
\[\begin{tikzcd}
0\ar[r]&C^\bullet(\mathcal{U},\mathscr{A})\ar[r,"\alpha"]&C^\bullet(\mathcal{U},\mathscr{B})\ar[r,"\beta"]&C_0^\bullet(\mathcal{U},\mathscr{C})\ar[r]&0
\end{tikzcd}\]
giving rise to a long exact sequence of cohomology
\[\begin{tikzcd}
\cdots\ar[r]&\check{H}^p(\mathcal{U},\mathscr{A})\ar[r]&\check{H}^p(\mathcal{U},\mathscr{B})\ar[r]&\check{H}^p_0(\mathcal{U},\mathscr{C})\ar[r,"\delta"]&\check{H}^{p+1}(\mathcal{U},\mathscr{A})\ar[r]&\cdots
\end{tikzcd}\]
where the coboundary operator $\delta$ is defined as usual.\par 
Now let $\mathcal{U}=(U_i)_{i\in I}$ and $\mathcal{V}=(V_j)_{j\in J}$ be two coverings and let $\tau:J\to I$ be such that $V_j\sub U_{\tau(j)}$; we thus have $\mathcal{V}\prec\mathcal{U}$. The commutative diagram:
\[\begin{tikzcd}
0\ar[r]&C^\bullet(\mathcal{U},\mathscr{A})\ar[r,"\alpha"]\ar[d,"\tau"]&C^\bullet(\mathcal{U},\mathscr{B})\ar[r,"\beta"]\ar[d,"\tau"]&C^\bullet(\mathcal{U},\mathscr{C})\ar[d,"\tau"]\\
0\ar[r]&C^\bullet(\mathcal{V},\mathscr{A})\ar[r,"\alpha"]&C^\bullet(\mathcal{V},\mathscr{B})\ar[r,"\beta"]&C^\bullet(\mathcal{V},\mathscr{C})
\end{tikzcd}\]
shows that $\tau$ maps $C_0^\bullet(\mathcal{U},\mathscr{C})$ into $C_0^\bullet(\mathcal{V},\mathscr{C})$, thus defining the homomorphisms
\[\tau^*:\check{H}^p_0(\mathcal{U},\mathscr{C})\to\check{H}^p_0(\mathcal{V},\mathscr{C}).\]
Moreover, the homomorphisms $\tau$ are independent of the choice of the mapping $\tau$: this follows from the fact that, if $f\in C^p_0(\mathcal{U},\mathscr{C})$, we have $\kappa f\in C^{p-1}_0(\mathcal{V},\mathscr{C})$, with the notations of Proposition~\ref{Cech refinement map}. We have thus obtained canonical homomorphisms $H^p_0(\mathcal{U},\mathscr{C})\to\check{H}^p_0(\mathcal{V},\mathscr{C})$; we might then define $\check{H}^p_0(X,\mathscr{C})$ as the inductive limit of the groups $\check{H}^p(X,\mathscr{C})$.\par
Because an inductive limit of exact sequences is an exact sequence, we obtain:
\begin{proposition}
The sequence
\[\begin{tikzcd}
\cdots\ar[r]&\check{H}^p(X,\mathscr{A})\ar[r,"\alpha^*"]&\check{H}^p(X,\mathscr{B})\ar[r,"\beta^*"]&\check{H}^p_0(X,\mathscr{C})\ar[r,"\delta"]&\check{H}^{p+1}(X,\mathscr{A})\ar[r]&\cdots
\end{tikzcd}\]
is exact.
\end{proposition}
To apply the preceding proposition, it is convenient to compare the groups
$\check{H}^p_0(X,\mathscr{C})$ and $\check{H}^p(X,\mathscr{C})$. The inclusion of $C_0^\bullet(\mathcal{U},\mathscr{C})$ in $C^\bullet(\mathcal{U},\mathscr{C})$ defines the homomorphisms
\[\check{H}_0^p(\mathcal{U},\mathscr{C})\to\check{H}^p(\mathcal{U},\mathscr{C}),\]
hence, by passing to the limit with $\mathcal{U}$, the homomorphisms:
\[\check{H}_0^p(X,\mathscr{C})\to\check{H}^p(X,\mathscr{C})\]
We first prove a lemma.
\begin{lemma}\label{Cech C^0_0 lem}
Let $\mathcal{U}=(U_i)_{i\in I}$ be a covering and let $f=(f_j)$ be an element of $C^0(\mathcal{U},\mathscr{C})$. There exists a covering $\mathcal{V}=(V_j)_{j\in J}$ and a mapping $\tau:J\to I$ such that $V_j\sub U_{\tau(j)}$ and $\tau(f)\in C^0_0(\mathcal{V},\mathscr{C})$.
\end{lemma}
\begin{proof}
For any $x\in X$, take a $\tau(x)\in I$ such that $x\in U_{\tau(x)}$. Since $f_{\tau(x)}$ is a section of $\mathscr{C}$ over $U_{\tau(x)}$, by the surjectivity of $\beta$ there exists an open neighborhood $V_x$ of $x$, contained in $U_{\tau(x)}$ and a section $b_x$ of $\mathscr{B}$ over $V_x$ such that $\beta(b_x)=f_{\tau(x)}|_{V_x}$ on $V_x$. The $(V_x)_{x\in X}$ form a covering $\mathcal{V}$ of $X$, and the $b_x$ form a $0$-chain $b$ of $\mathcal{V}$ with values in $\mathcal{U}$. From the construction we have $\tau(f)=\beta(b)$, so that $\tau(f)\in C^0_0(\mathcal{V},\mathscr{C})$.
\end{proof}
Now we can prove the following result.
\begin{theorem}\label{Cech map p=0,1}
The canonical homomorphism $\check{H}^p_0(X,\mathscr{C})\to\check{H}^p(X,\mathscr{C})$ is bijective for $p=0$ and injective for $p=1$.
\end{theorem}
\begin{proof}
By Lemma~\ref{Cech C^0_0 lem} the bijectivity for $p=0$ is immediate. We now show that
\[\check{H}^1_0(X,\mathscr{C})\to\check{H}^1(X,\mathscr{C})\]
is injective. Let $[z]$ be in the kernel of this map, which is represented by a $1$-cocycle $z=(z_{i_0i_1})\in C^1_0(\mathcal{U},\mathscr{C})$. Then since $[z]=0$ in $\check{H}^1(X,\mathscr{C})$, there exists an $f=(f_i)\in C^0(\mathcal{U},\mathscr{C})$ with $df=z$; applying Lemma~\ref{Cech C^0_0 lem} (and its notations) to $f$ yields a covering $\mathcal{V}$ such that $\tau(f)\in C^0_0(\mathcal{V},\mathscr{C})$, which shows that $\tau(z)$ is cohomologous to $0$ in $C^1_0(\mathcal{V},\mathscr{C})$, thus its image $[z]$ in $\check{H}^1_0(X,\mathscr{C})$ is $0$. This shows the claim.
\end{proof}
\begin{corollary}\label{Cech long exact seq general}
With notations above, we have an exact sequence:
\[\begin{tikzcd}[column sep=small]
0\ar[r]&\Gamma(X,\mathscr{A})\ar[r]&\Gamma(X,\mathscr{B})\ar[r]&\Gamma(X,\mathscr{C})\ar[r,"\delta"]&\check{H}^1(X,\mathscr{A})\ar[r]&\check{H}^1(X,\mathscr{B})\ar[r]&\check{H}^1(X,\mathscr{C})
\end{tikzcd}\]
\end{corollary}
\begin{corollary}\label{Cech global section exact}
If $\check{H}^1(X,\mathscr{A})=0$, then $\Gamma(X,\mathscr{B})\to\Gamma(X,\mathscr{C})$ is surjective.
\end{corollary}
\subsection{Paracompact space}
Recall that a space $X$ is said to be paracompact if any covering of $X$ admits a locally finite refinement. On paracompact spaces, we can extend Proposition~\ref{Cech map p=0,1} for all values of $p$:
\begin{theorem}\label{Cech paracompact thm}
If $X$ is paracompact, the canonical homomorphism
\[\check{H}^p_0(X,\mathscr{C})\to\check{H}^p(X,\mathscr{C})\]
is bijective for all $p\geq0$.
\end{theorem}
This Proposition is an immediate consequence of the following lemma, analogous to Lemma~\ref{Cech C^0_0 lem}:
\begin{lemma}
Let $X$ be a paracompact space. Let $\mathcal{U}=(U_i)_{i\in I}$ be a covering, and let $f=(f_{i_0\dots i_p})$ be an element of $C^p(\mathcal{U},\mathscr{C})$. Then there exists a covering $\mathcal{V}=(V_j)_{j\in J}$ and a mapping $\tau:J\to I$ such that $V_j\sub U_{\tau(j)}$ and $\tau(f)\in C_0^p(\mathcal{V},\mathscr{C})$.
\end{lemma}
\begin{proof}
Since $X$ is paracompact, we might assume that $\mathcal{U}$ is locally finite. For every $x\in X$, we can choose an open neighborhood $V_x$ of $x$ such that
\begin{itemize}
\item[(a)] If $x\in U_i$, then $V_x\sub U_i$.
\item[(b)] If $V_x\cap U_i\neq\emp$, then $V_x\sub U_i$.
\item[(c)] If $x\in U_{i_0\dots i_p}$, there exists a section $b$ of $\mathscr{B}$ over $V_x$ such that $\beta(b)=f_{i_0\dots i_p}|_{V_x}$.
\end{itemize}
The condition (c) can be satisfid due to the surjectivity of $\beta$ and to the fact that $x$ belongs to a finite number of sets $U_{i_0\dots i_p}$. Having (c) satisfied, it suffices to restrict $V_x$ to satisfy (a) and (b).\par
The family $(V_x)_{x\in X}$ forms a covering $\mathcal{V}$; for any $x\in X$, choose $\tau(x)\in I$ such that $x\in U_{\tau(x)}$. We now check that $\tau(f)$ belongs to $C^p_0(\mathcal{V},\mathscr{C})$. If $V_{x_0\dots x_p}$ is empty, this is obvious; if not, we have $V_{x_0}\cap U_{x_k}\neq\emp$ for $0\leq k\leq p$, and then
\[V_{x_0}\cap U_{\tau(x_k)}\neq\emp\for 0\leq k\leq p,\] 
which implies by (b) that $V_{x_0}\sub U_{\tau(x_k)}$ for all $k$, and hence $x_0\in U_{\tau(x_0)\dots\tau(x_p)}$. We then apply (c) to get a section $b$ of $\mathscr{B}$ over $V_{x_0}$ such that $\beta(b)=f_{\tau(x_0)\dots\tau(x_p)}|_{V_{x_0}}$. Thus $\tau(f)\in C^p_0(\mathcal{V},\mathscr{C})$, which completes the proof.
\end{proof}
\begin{corollary}\label{Cech long exact seq paracompact}
If $X$ is paracompact, we have an exact sequence:
\[\begin{tikzcd}
\cdots\ar[r]&\check{H}^p(X,\mathscr{A})\ar[r,"\alpha^*"]&\check{H}^p(X,\mathscr{B})\ar[r,"\beta^*"]&\check{H}^p(X,\mathscr{C})\ar[r,"\delta"]&\check{H}^{p+1}(X,\mathscr{A})\ar[r]&\cdots
\end{tikzcd}\]
\end{corollary}
The exact sequence mentioned above is called the long exact sequence of cohomology defined by a given exact sequence of sheaves $0\to\mathscr{A}\to\mathscr{B}\to\mathscr{C}\to 0$. More generally, it exists whenever we can show that $\check{H}^p_0(X,\mathscr{C})\to\check{H}^p(X,\mathscr{C})$ is bijective.
\section{\v{C}ech resolution and Leray Acyclic Theorem}
We will now compare the \v{C}ech cohomology and derived functor cohomology of
a sheaf $\mathscr{F}$ on a topological space $X$. We will see that in some cases we are able to conclude that these two cohomologies coincide. In order to compare \v{C}ech cohomology with derived functor cohomology, we
will need to consider first a sheafified version of the \v{C}ech complex.\par
Fix a space $X$, an open cover $\mathcal{U}=(U_i)_{i\in I}$ and a sheaf $\mathscr{F}$ on $X$. For every open set $U$ of $X$, let $j_U:U\hookrightarrow X$ denote the inclusion. Define a sheaf $\mathscr{C}^p(\mathcal{U},\mathscr{F})$ by
\[\mathscr{C}^p(\mathcal{U},\mathscr{F})=\prod_{(i_0,\dots,i_p)}(j_{U_{i_0\dots i_p}})_*(\mathscr{F}|_{U_{i_0\dots i_p}}).\]
Explicitly, for an open subset $V\sub X$ we have
\[\Gamma(V,\mathscr{C}^p(\mathcal{U},\mathscr{F}))=\prod_{(i_0,\dots,i_p)}\mathscr{F}(V\cap U_{i_0\dots i_p}).\]
For each inclusion $\delta^p_j:U_{i_0\dots i_p}\to U_{i_0\dots\widehat{i_j}\dots i_p}$, we also have a restriction map
\[\mathscr{C}(\delta^p_j):(j_{U_{i_0\dots\widehat{i_j}\dots i_p}})_*(\mathscr{F}|_{U_{i_0\dots\widehat{i_j}\dots i_p}})\to(j_{U_{i_0\dots i_p}})_*(\mathscr{F}|_{U_{i_0\dots i_p}})\]
so we can define the differential $d:\mathscr{C}^p(\mathcal{U},\mathscr{F})\to\mathscr{C}^{p+1}(\mathcal{U},\mathscr{F})$ by
\[d=\sum_{j=0}^{p+1}(-1)^j\mathscr{C}(\delta^{p+1}_j).\]
Thus we get complex $\mathscr{C}^\bullet(\mathcal{U},\mathscr{F})$. By definition, we have
\[\Gamma(X,\mathscr{C}^\bullet(\mathcal{U},\mathscr{F}))=C^\bullet(\mathcal{U},\mathscr{F})\]
Also, there is a product map $\epsilon:\mathscr{F}\to\mathscr{C}^0(\mathcal{U},\mathscr{F})$ defined by
\[\mathscr{F}\to\prod_i(j_{U_i})_*(j_{U_i})^{-1}\mathscr{F}=\prod_i(j_{U_i})_*(\mathscr{F}|_{U_i}).\]
\begin{proposition}\label{Cech resolution}
For an open cover $\mathcal{U}$ and a sheaf $\mathscr{F}$, there is an exact sequence of sheaves
\[\begin{tikzcd}
0\ar[r]&\mathscr{F}\ar[r,"\epsilon"]&\mathscr{C}^0(\mathcal{U},\mathscr{F})\ar[r]&\mathscr{C}^1(\mathcal{U},\mathscr{F})\ar[r]&\mathscr{C}^2(\mathcal{U},\mathscr{F})\ar[r]&\cdots
\end{tikzcd}\]
This is called the \textbf{\v{C}ech resolution} of $\mathscr{F}$ with respect to the cover $\mathcal{U}$.
\end{proposition}
\begin{proof}
The facts that $\epsilon$ is injective and that $\im\epsilon=\ker d^0$ follow directly from $\mathscr{F}$ being a sheaf.\par
It remains to be shown that the proposed sequence is exact in degrees $p>0$. For this, it suffices to work at the level of stalks.\par
To prove the exactness at $\mathscr{C}^p(\mathcal{U},\mathscr{F})$, we need to check the sequence of stalks $\mathscr{C}^\bullet(\mathcal{U},\mathscr{F})$ is exact for all $x\in X$. Since $\mathcal{U}$ covers $X$, we can choose an open subset $U_j$ containing $x$. Take $f_x\in\mathscr{C}^p(\mathcal{U},\mathscr{F})_x$ which is represented by $(V,f)$, where $V$ is a neighborhood of $x$ and $f\in\Gamma(V,\mathscr{C}^p(\mathcal{U},\mathscr{F}))$. We may assume $V\sub U_j$. Then we observe that for any index $(i_0,\dots,i_{p-1})\in I^p$ we have
\[V\cap U_{i_0\dots i_{p-1}}=V\cap U_{j,i_0\dots i_{p-1}}\]
Thus, we may define an element $\theta f\in\Gamma(V,\mathscr{C}^{p-1}(\mathcal{U},\mathscr{F}))$ by the formula
\[(\theta f)_{i_0\dots i_{p-1}}:=f_{j,i_0\dots i_{p-1}}\]
This gives a map \[\theta^p:\mathscr{C}^p(\mathcal{U},\mathscr{F})_x\to\mathscr{C}^{p-1}(\mathcal{U},\mathscr{F})_x,\quad f_x\mapsto (\theta f)_x\]
Now we check that
\[(\theta(df))_{i_0\dots i_p}=(df)_{j,i_0\dots i_p}=f_{i_0\dots i_p}-\sum_{k=0}^{p}(-1)^k(f_{j,i_0\dots\widehat{i_k}\dots i_p})|_{U_{i_0\dots i_p}}.\]
and
\[(d(\theta f))_{i_0\dots i_p}=\sum_{k=0}^{p}(-1)^k((\theta f)_{i_0\dots\widehat{i_k}\dots i_p})|_{U_{i_0\dots i_p}}=\sum_{k=0}^{p}(-1)^k(f_{j,i_0\dots\widehat{i_k}\dots i_p})|_{U_{i_0\dots i_p}}.\]
Therefore
\begin{align*}
\theta^{p+1}\big((df)_x\big)=(\theta(df))_x=\big(f-d(\theta f)\big)_x=f_x-d((\theta f)_x)=f_x-d(\theta^pf_x).
\end{align*}
Hence $\theta^p$ is a homotopy from the identity of $\mathscr{C}^p(\mathcal{U},\mathscr{F})_x$ to the zero map, which shows the sequence of stalks is exact.
\end{proof}
One of the motivitions of defining the \v{C}ech resolution is the following proposition.
\begin{proposition}
The \v{C}ech resolution computes the \v{C}ech cohomology. In the sense that the \v{C}ech cohomology groups can be derived from the \v{C}ech cohomology by appying the global section and taking cohomology.
\end{proposition}
\begin{proof}
Applying the global section on the complex $0\to\mathscr{F}\to\mathscr{C}^\bullet(\mathcal{U},\mathscr{F})$ gives the complex
\[\begin{tikzcd}
0\ar[r]&\mathscr{F}(X)\ar[r]&C^0(\mathcal{U},\mathscr{F})\ar[r]&C^1(\mathcal{U},\mathscr{F})\ar[r]&C^2(\mathcal{U},\mathscr{F})\ar[r]&\cdots
\end{tikzcd}\]
which is exactly the \v{C}ech complex. Thus its cohomology gives the \v{C}ech cohomology.
\end{proof}
\begin{corollary}
With $X$, $\mathcal{U}$, $\mathscr{F}$ as above, there is a canonical map
\[\check{H}^\bullet(\mathcal{U},\mathscr{F})\to H^\bullet(X,\mathscr{F}).\]
\end{corollary}
\begin{proof}
Let $0\to\mathscr{F}\to\mathscr{I}^\bullet$ be an injective resolution of $\mathscr{F}$. Then since $0\to\mathscr{F}\to\mathscr{C}^\bullet(\mathcal{U},\mathscr{F})$ is also a resolution of $\mathscr{F}$, we have a canonical induced map $\mathscr{C}^\bullet(\mathcal{U},\mathscr{F})\to\mathscr{I}^\bullet$ induced by the identity of $\mathscr{F}$. The induced map on cohomology is what we want.
\end{proof}
We will now state some results stating sufficient conditions for these
canonical morphisms to actually be isomorphisms, thus enabling us to calculate sheaf cohomology via \v{C}ech cohomology.
\begin{proposition}\label{Cech flasque}
If $\mathscr{F}$ is a flasque sheaf on $X$, then $\check{H}^p(\mathcal{U},\mathscr{F})=0$ for all open covers $\mathcal{U}$ of $X$ and all $p>0$.
\end{proposition}
\begin{proof}
Let $\mathscr{C}^\bullet(\mathcal{U},\mathscr{F})$ be the \v{C}ech resolution of $\mathscr{F}$ with respect to $\mathcal{U}$. Recall that $\mathscr{C}^p(\mathcal{U},\mathscr{F})$ is defined for any open $V\sub X$ by
\[\Gamma(V,\mathscr{C}^p(\mathcal{U},\mathscr{F}))=\prod\mathscr{F}(V\cap U_{i_0\dots i_p}).\]
For each of these $U_{i_0\dots i_p}$, the sheaf $V\mapsto\mathscr{F}(V\cap U_{i_0\dots i_p})$ is flasque and since products of flasque sheaves are flasque, the entire sheaf $\mathscr{C}^p(\mathcal{U},\mathscr{F})$ is flasque. Therefore $\mathscr{C}^p(\mathcal{U},\mathscr{F})$ is acyclic by Proposition~\ref{flasque is acyclic}, so the \v{C}ech resolution can used to compute sheaf cohomology:
\[\check{H}^p(\mathcal{U},\mathscr{F})=H^p(X,\mathscr{F})=0\]
as desired.
\end{proof}
\begin{corollary}\label{Cech H^1}
For any sheaf $\mathscr{F}$ on $X$, the map $\check{H}^1(X,\mathscr{F})\to H^1(X,\mathscr{F})$ is an isomorphism.
\end{corollary}
\begin{proof}
Let $\mathscr{G}$ be a flasque sheaf and $0\to\mathscr{F}\to\mathscr{G}\to\mathscr{Q}\to 0$ be a short exact sequence of sheaves on $X$. Then $\check{H}^1(X,\mathscr{G})=0$, and by Corollary~\ref{Cech long exact seq general} there is an exact sequence
\[\begin{tikzcd}
0\ar[r]&\Gamma(X,\mathscr{F})\ar[r]\ar[d,"\mathbf{1}"]&\Gamma(X,\mathscr{G})\ar[r]\ar[d,"\mathbf{1}"]&\Gamma(X,\mathscr{Q})\ar[r]\ar[d,"\mathbf{1}"]&\check{H}^1(X,\mathscr{F})\ar[r]\ar[d]&0\\
0\ar[r]&\Gamma(X,\mathscr{F})\ar[r]&\Gamma(X,\mathscr{G})\ar[r]&\Gamma(X,\mathscr{Q})\ar[r]&H^1(X,\mathscr{F})\ar[r]&0
\end{tikzcd}\]
Thus the map $\check{H}^1(X,\mathscr{F})\to H^1(X,\mathscr{F})$ is an isomorphism by the five lemma.
\end{proof}
\begin{proposition}
Let $X$ be a paracompact space, and $\mathscr{F}$ be a sheaf on $X$. Then the canonical maps
\[\check{H}^p(X,\mathscr{F})\to H^p(X,\mathscr{F})\]
are isomorphisms for $p\geq 0$.
\end{proposition}
\begin{proof}
This follows from Corollary~\ref{Cech long exact seq paracompact} and an induction.
\end{proof}
\begin{definition}
A sheaf $\mathscr{F}$ on $X$ is \textbf{acyclic for an open cover} $\mathcal{U}=(U_i)_{i\in I}$ if for all $p>0$ we have
\[H^p(U_{i_0\dots i_p},\mathscr{F}|_{U_{i_0\dots i_p}})=0.\]
\end{definition}
\begin{theorem}[\textbf{Leray}]\label{Leray acyclic thm}
If $\mathscr{F}$ is a sheaf on $X$ which is acyclic for an open cover $\mathcal{U}$, then the canonical maps
\[\check{H}^p(\mathcal{U},\mathscr{F})\to H^p(X,\mathscr{F})\]
are isomorphisms for all $p\geq 0$.
\end{theorem}
\begin{proof}
We proceed by induction on the degree $p$. For $p=0$ we already know that the result is true thanks to Proposition~\ref{Cech H^0}.\par
Now, assume $\check{H}^j(U,\mathscr{G})\to H^j(X,\mathscr{G})$ is an isomorphism for all $j\leq p$ and all sheaves $\mathscr{G}$ acyclic for $\mathcal{U}$. Embed $\mathscr{F}$ in a flasque sheaf $\mathscr{G}$ and let $\mathscr{H}$ be the quotient, so that we have an exact sequence
\[\begin{tikzcd}
0\ar[r]&\mathscr{F}\ar[r]&\mathscr{G}\ar[r]&\mathscr{H}\ar[r]&0
\end{tikzcd}\]
For each finite sequence $\sigma=(i_0,\dots,i_p)$, by hypothesis  we have $H^i(U_\sigma,\mathscr{F}|_{U_\sigma})=0$ and $H^i(U_\sigma,\mathscr{G}|_{U_\sigma})=0$. Therefore $H^i(U_\sigma,\mathscr{H}|_{U_\sigma})$ is zero by the long exact sequence of $H^i(U_\sigma.-)$, and by taking the product over all such $U_\sigma$, we conclude that $\mathscr{H}$ is also acyclic for $\mathcal{U}$.\par
Now by the argument above, the sequence
\[\begin{tikzcd}
0\ar[r]&\mathscr{F}(U_\sigma)\ar[r]&\mathscr{G}(U_\sigma)\ar[r]&\mathscr{H}(U_\sigma)\ar[r]&0
\end{tikzcd}\]
is exact. Hence by taking product the corresponding short sequence of \v{C}ech complexes
\[\begin{tikzcd}
0\ar[r]&\mathscr{C}^\bullet(\mathcal{U},\mathscr{F})\ar[r]&\mathscr{C}^\bullet(\mathcal{U},\mathscr{G})\ar[r]&\mathscr{C}^\bullet(\mathcal{U},\mathscr{H})\ar[r]&0
\end{tikzcd}\]
is exact, so that we get a long exact sequence in \v{C}ech cohomology
\[\begin{tikzcd}
\cdots\ar[r]&\check{H}^{p}(\mathcal{U},\mathscr{F})\ar[r]&\check{H}^{p}(\mathcal{U},\mathscr{G})\ar[r]&\check{H}^{p}(\mathcal{U},\mathscr{H})\ar[r]&\check{H}^{p+1}(\mathcal{U},\mathscr{F})\ar[r]&\cdots
\end{tikzcd}\]
Now since $\mathscr{G}$ is flasque, Proposition~\ref{Cech flasque} shows that $\check{H}^p(\mathcal{U},\mathscr{G})=0$. Therefore we have a diagram with exact rows
\[\begin{tikzcd}
0\ar[r]&\check{H}^{p}(\mathcal{U},\mathscr{H})\ar[r]\ar[d]&\check{H}^{p+1}(\mathcal{U},\mathscr{F})\ar[r]\ar[d]&0\\
0\ar[r]&H^{p}(\mathcal{U},\mathscr{H})\ar[r]&H^{p+1}(\mathcal{U},\mathscr{F})\ar[r]&0
\end{tikzcd}\]
By induction, the left column is an isomorphism, so $\check{H}^{p+1}(X,\mathscr{F})\to H^{p+1}(X,\mathscr{F})$ is also an isomorphism.
\end{proof}
\section{\v{C}ech vs. Sheaf cohomology}
\begin{proposition}\label{Cech exact section}
Let $X$ be a ringed space. Let
\[\begin{tikzcd}
0\ar[r]&\mathscr{F}\ar[r]&\mathscr{G}\ar[r]&\mathscr{H}\ar[r]&0
\end{tikzcd}\]
be a short exact sequence of $\mathscr{O}_X$-modules. Let $U\sub X$ be an open subset. If there exists a cofinal system of open coverings $\mathcal{U}$ of $U$ such that 
$\check{H}^1(\mathcal{U},\mathscr{F})=0$, then the map $\mathscr{G}(U)\to\mathscr{H}(U)$ is surjective.
\end{proposition}
\begin{proof}
Take an element $s\in\mathscr{H}(U)$. Choose an open covering $\mathcal{U}:U=(U_i)_{i\in I}$ such that 
\begin{itemize}
\item[(a)] $\check{H}^1(\mathcal{U},\mathscr{F})=0$.
\item[(b)] $s|_{U_i}$ is the image of a section $s_i\in\mathscr{G}(U_i)$.
\end{itemize} 
Since we can certainly find a covering such that (b) holds by the exactness of the sequence, it follows from the assumptions that we can find a covering such that (a) and (b) both hold. Consider the sections
\[s_{i_0i_1}=s_{i_1}|_{U_{i_0i_1}}-s_{i_0}|_{U_{i_0i_1}}\]
Since $s_i$ lifts $s$ we see that $s_{i_0i_1}\in\mathscr{F}(U_{i_0i_1})$. By the vanishing of $\check{H}^1(\mathcal{U},\mathscr{F})$ we can find sections $t_i\in\mathscr{F}(U_i)$ such that
\[s_{i_0i_1}=t_{i_1}|_{U_{i_0i_1}}-t_{i_0}|_{U_{i_0i_1}}\]
Then clearly the sections $s_i-t_i$ satisfy the sheaf condition and glue to a section of $\mathscr{G}$ over $U$ which maps to $s$. Hence we win.
\end{proof}
\begin{proposition}\label{Cech vanish H^p vanish}
Let $X$ be a ringed space. Let $\mathscr{F}$ be an $\mathcal{O}_X$-module such that
\[\check{H}^p(\mathcal{U},\mathscr{F})=0\]
for all $p>0$ and any open covering $\mathcal{U}:U=\bigcup_{i\in I}U_i$ of an open subset $U$ of $X$. Then $H^p(U,\mathscr{F})=0$ for all $p>0$ and any open $U\sub X$.
\end{proposition}
\begin{proof}
Choose an embedding $\mathscr{F}\to\mathscr{I}$ into an injective $\mathscr{O}_X$-module. By Proposition~\ref{Cech flasque} $\mathscr{I}$ has vanishing higher \v{C}ech cohomology for any open covering. Let $\mathscr{Q}=\mathscr{I}/\mathscr{F}$ so that we have a short exact sequence
\[\begin{tikzcd}
0\ar[r]&\mathscr{F}\ar[r]&\mathscr{I}\ar[r]&\mathscr{Q}\ar[r]&0
\end{tikzcd}\]
By Proposition\ref{Cech exact section} and our assumptions this sequence is actually exact as a sequence of presheaves! In particular we have a long exact sequence of \v{C}ech cohomology groups for any open covering $\mathcal{U}$, see Proposition~\ref{Cech presheaf delta functor} for example. This implies that $\mathscr{Q}$ is also an $\mathscr{O}_X$-module with vanishing higher \v{C}ech cohomology for all open coverings.\par
Next, we look at the long exact cohomology sequence
\[\begin{tikzcd}
\cdots\ar[r]&H^p(U,\mathscr{F})\ar[r]&H^p(U,\mathscr{I})\ar[r]&H^p(U,\mathscr{Q})\ar[r]&H^{p+1}(U,\mathscr{F})\ar[r]&\cdots
\end{tikzcd}\]
for any open $U\sub X$. Since $\mathscr{I}$ is injective we have $H^p(U,\mathscr{I})=0$ for $p>0$. By the above we see that $H^0(U,\mathscr{I})\to H^0(U,\mathscr{Q})$ is surjective, thus $H^1(U,\mathscr{F})=0$. Also, $H^p(U,\mathscr{Q})=H^{p+1}(U,\mathscr{F})$ for $p\geq 1$. Thus the claim follows by an induction.
\end{proof}
\begin{proposition}
Let $X$ be a ringed space. Let $\mathcal{B}$ be a basis for the topology on $X$. Let $\mathscr{F}$ be an $\mathscr{O}_X$-module. Assume there exists a set of open coverings $\mathrm{Cov}$ with the following properties:
\begin{itemize}
\item[(\rmnum{1})] For every $\mathcal{U}\in\mathrm{Cov}$ with $\mathcal{U}:U=\bigcup_{i\in I}U_i$ we have $U,U_i\in\mathcal{B}$ and every $U_{i_0\dots i_p}\in\mathcal{B}$.
\item[(\rmnum{2})] For every $U\in\mathcal{B}$ the open coverings of $U$ occurring in $\mathrm{Cov}$ is a cofinal system of open coverings of $U$.
\item[(\rmnum{3})] For every $\mathcal{U}\in\mathrm{Cov}$ we have $\check{H}^p(\mathcal{U},\mathscr{F})=0$ for all $p>0$.
\end{itemize}
Then $H^p(U,\mathscr{F})=0$ for all $p>0$ and any $U\in\mathcal{B}$.
\end{proposition}
\begin{proof}
Let $\mathscr{F}$ and $\mathrm{Cov}$ be as in the lemma. We will indicate this by saying $\mathscr{F}$ has vanishing higher \v{C}ech cohomology for any $\mathcal{U}\in\mathrm{Cov}$. Choose an embedding $\mathscr{F}\hookrightarrow\mathscr{I}$ into an injective $\mathscr{O}_X$-module. By Proposition~\ref{Cech flasque} has vanishing higher \v{C}ech cohomology for any $\mathcal{U}\in\mathrm{Cov}$. Let $\mathscr{Q}=\mathscr{I}/\mathscr{F}$ so that we have a short exact sequence
\[\begin{tikzcd}
0\ar[r]&\mathscr{F}\ar[r]&\mathscr{I}\ar[r]&\mathscr{Q}\ar[r]&0
\end{tikzcd}\]
By Proposition~\ref{Cech exact section} and our assumption (\rmnum{2}) this sequence gives rise to an exact sequence
\[\begin{tikzcd}
0\ar[r]&\mathscr{F}(U)\ar[r]&\mathscr{I}(U)\ar[r]&\mathscr{Q}(U)\ar[r]&0
\end{tikzcd}\]
for every $U\in\mathcal{B}$. Hence for any $\mathcal{U}\in\mathrm{Cov}$ we get a short exact sequence of \v{C}ech complexes
\[\begin{tikzcd}
0\ar[r]&\mathscr{C}^\bullet(\mathcal{U},\mathscr{F})\ar[r]&\mathscr{C}^\bullet(\mathcal{U},\mathscr{I})\ar[r]&\mathscr{C}^\bullet(\mathcal{U},\mathscr{Q})\ar[r]&0
\end{tikzcd}\]
since each term in the \v{C}ech complex is made up out of a product of values over elements of $\mathcal{B}$ by assumption (\rmnum{1}). In particular we have a long exact sequence of \v{C}ech cohomology groups for any open covering $\mathcal{U}\in\mathrm{Cov}$. This implies that $\mathscr{Q}$ is also an $\mathscr{O}_X$-module with vanishing higher \v{C}ech cohomology for all $\mathcal{U}\in\mathrm{Cov}$.\par
Next, we look at the long exact cohomology sequence
\[\begin{tikzcd}
\cdots\ar[r]&H^p(U,\mathscr{F})\ar[r]&H^p(U,\mathscr{I})\ar[r]&H^p(U,\mathscr{Q})\ar[r]&H^{p+1}(U,\mathscr{F})\ar[r]&\cdots
\end{tikzcd}\]
for any open $U\in\mathcal{B}$. Since $\mathscr{I}$ is injective we have $H^n(U,\mathscr{I})=0$ for $n>0$. By the above we see that $H^0(U,\mathscr{I})\to H^0(U,\mathscr{Q})$ is surjective and hence $H^1(U,\mathscr{F})=0$. Since $\mathscr{F}$ was an arbitrary $\mathscr{O}_X$-module with vanishing higher \v{C}ech cohomology for all $\mathcal{U}\in\mathrm{Cov}$ we conclude that also $H^1(U,\mathscr{Q})=0$ since $\mathscr{Q}$ is another of these sheaves (see above). Also, $H^p(U,\mathscr{Q})=H^{p+1}(U,\mathscr{F})$ for $p\geq 1$. Thus the claim follows by an induction.
\end{proof}
\begin{proposition}\label{push injective}
Let $f:X\to Y$ be a morphism of ringed spaces. Let $\mathscr{I}$ be an injective $\mathscr{O}_X$-module. Then
\begin{itemize}
\item $\check{H}^p(\mathcal{V},f_*\mathscr{I})=0$ for all $p>0$ and any open covering $\mathcal{V}=(V_j)_{j\in J}$ of $Y$.
\item $H^p(V,f_*\mathscr{I})=0$ for all $p>0$ and every open $V\sub Y$.
\end{itemize}
In other words, $f_*\mathscr{I}$ is right acyclic for $\Gamma(U,-)$ for any $U\sub X$ open.
\end{proposition}
\begin{proof}
Set $\mathcal{U}=f^{-1}(\mathcal{V})$. It is an open covering of $X$ and
\[C^\bullet(\mathcal{V},f_*\mathscr{I})=C^\bullet(\mathcal{U},\mathscr{I}).\]
This is true because
\[f_*\mathscr{I}(V_{j_0\dots j_p})=\mathscr{I}(f^{-1}(V_{j_0\dots j_p}))=\mathscr{I}\big(f^{-1}(V_{j_0})\cap\cdots\cap f^{-1}(V_{j_p})\big)=\mathscr{I}(U_{j_0\dots j_p}).\]
Thus the first statement of the lemma follows from Proposition~\ref{Cech flasque}. The second statement follows from the first and Proposition~\ref{Cech vanish H^p vanish}.
\end{proof}