\chapter{The language of \boldmath\texorpdfstring{$\infty$}{inf}-category}\label{simplicial set inf-cat language chapter}
\section{Simplicial sets}\label{simplicial set section}
For each integer $n\geq 0$, we define the \textbf{$n$-simplex}
\[|\Delta^n|=\{(t_0,\dots,t_n)\in[0,1]^{n+1}:t_0+\cdots+t_n=1\}.\]
as a topological simplex of dimension $n$. For any topological space $X$, a continuous map $\sigma:|\Delta^n|\to X$ is called a \textbf{singular $\bm{n}$-simplex} in $X$. Every singular $n$-simplex $\sigma$ determines a finite collection of singular $(n-1)$-simplices $\{d_i\sigma\}_{0\leq i\leq n}$, called the faces of $\sigma$, which are given explicitly by the formula
\[(d_i\sigma)(t_0,\dots,t_{n-1})=\sigma(t_0,t_1,\dots,t_{i-1},0,t_i,\dots,t_{n-1}).\]
Let $\Sing_n(X)=\Hom_{\mathsf{Top}}(|\Delta^n|,X)$ denote the set of singular $n$-simplices of $X$. Many important algebraic invariants of $X$ can be directly extracted from the sets $\{\Sing_n(X)\}_{n\geq 0}$ and the face maps $\{d_i:\Sing_n(X)\to\Sing_{n-1}(X)\}_{0\leq i\leq n}$.
\begin{example}[\textbf{Singular Homology}]\label{simplicial set singular homology from Sing}
For any topological space $X$, the singular homology groups $H_*(X,\Z)$ are defined as the homology groups of a chain complex
\[\begin{tikzcd}
\cdots\ar[r,"\partial"]&\Z[\Sing_2(X)]\ar[r,"\partial"]&\Z[\Sing_1(X)]\ar[r,"\partial"]&\Z[\Sing_0(X)]
\end{tikzcd}\]
where $\Z[\Sing_n(X)]$ is the free abelian group over $\Sing_n(X)$ and the boundary map $\partial$ is defined by
\[\partial(\sigma)=\sum_{i=0}^{n}(-1)^id_i\sigma.\]
\end{example}
For some other algebraic invariants, it is convenient to keep track of a bit more structure. A singular $n$-simplex $\sigma:|\Delta^n|\to X$ also determines a collection of singular $(n+1)$-simplices $\{s_i\sigma\}_{0\leq i\leq n}$, given by the formula
\[(s_i\sigma)(t_0,\dots,t_{n+1})=\sigma(t_0,t_1,\dots,t_{i-1},t_i+t_{i+1},t_{i+2},\dots,t_{n+1}).\]
The resulting constructions $s_i:\Sing_n(X)\to\Sing_{n+1}(X)$ are called \textbf{degeneracy maps}, because singular $(n+1)$-simplices of the form $s_i\sigma$ factor through the linear projection $|\Delta^{n+1}|\to|\Delta^n|$. For example, the map $s_0:\Sing_0(X)\to\Sing_1(X)$ carries each point $x\in X\cong\Sing_0(X)$ to the constant map $\uline{x}:\Delta^1\to X$ taking the value $x$. In general, one can view the map $s_i$ as obtained by "squashing" the simplex $|\Delta^{n+1}|$ along the direction of one of its face to obtain a $|\Delta^n|$.
\begin{example}[\textbf{Fundamental Group}]\label{simplicial set pi_1 from Sing}
Let $X$ be a topological space equipped with a base point $x\in X\cong\Sing_0(X)$. Then continuous paths $p:[0,1]\to X$ satisfying $p(0)=p(1)=x$ can be identified with elements of the set $\{\sigma\in\Sing_1(X):d_0\sigma=d_1\sigma=x\}$. The fundamental group $\pi_1(X,x)$ can then be described as the quotient
\[\{\sigma\in\Sing_1(X):d_0(\sigma)=d_1(\sigma)=x\}/\sim\]
where $\sim$ is the equivalence relation on $\Sing_1(X)$ described by $\sigma\sim\sigma'$ if and only if there exists $\tau\in\Sing_2(X)$ such that $d_0(\tau)=s_0(x)$, $d_1(\tau)=\sigma$ and $d_2(\tau)=\sigma'$. The datum of a $2$-simplex $\tau$ satisfying these conditions is equivalent to the datum of a continuous map $|\Delta^2|\to X$ with boundary behavior as indicated in the following diagram
\[\begin{tikzcd}
&x\ar[rd,"\uline{x}"]&\\
x\ar[ru,"\sigma'"]\ar[rr,"\sigma"]&&x
\end{tikzcd}\]
It is easy to see that such a map can be identified with a homotopy between the paths determined by $\sigma$ and $\sigma'$.
\end{example}
Motivated by the preceding examples, we can ask the following question: given a topological space $X$, what can we say about the collection of sets $\{\Sing_n(X)\}_{n\geq 0}$, together with the face and degeneracy maps
\[d_i:\Sing_n(X)\to\Sing_{n-1}(X),\quad s_i:\Sing_n(X)\to\Sing_{n+1}(X).\]
Eilenberg and Zilber supplied an answer to this by introducing what they called \textit{complete semi-simplicial complexes}, which are now more commonly known as \textit{simplicial sets}. Roughly speaking, a simplicial set $S_\bullet$ is a collection of sets $\{S_n\}_{n\geq 0}$ indexed by the nonnegative integers, equipped with face and degeneracy operators $\{d_i:S_n\to S_{n-1},s_i:S_n\to S_{n+1}\}_{0\leq i\leq n}$ satisfying a short list of identities. These identities can be summarized conveniently by saying that a simplicial set is a presheaf on the simplex category $\Delta$, which we will define later. Simplicial sets are connected to algebraic topology by two closely related constructions
\begin{itemize}
\item For every topological space $X$, the face and degeneracy operators defined above endow the collection $\{\Sing_n(X)\}_{n\geq 0}$ with the structure of a simplicial set. We denote this simplicial set by $\Sing_\bullet(X)$ and refer to it as the \textbf{singular simplicial set of $X$}. These simplicial sets tend to be quite large: in any nontrivial example, the sets $\Sing_n(X)$ will be uncountable for every nonnegative integer $n$.
\item Any simplicial set $S_\bullet$ can be regarded as a "blueprint" for constructing a topological space $|S_\bullet|$ called the \textbf{geometric realization} of $S_\bullet$, which can be obtained as a quotient of the disjoint union $\coprod_{n\geq 0}S_n\times|\Delta^n|$ by an equivalence relation determined by the face and degeneracy operators on $S_\bullet$. Many topological spaces of interest (for example, any space which admits a finite triangulation) can be realized as a geometric realization of a simplicial set $S_\bullet$ having only finitely many nondegenerate simplices.
\end{itemize}
These constructions determine adjoint functors
\[\begin{tikzcd}
\mathbf{Set}_{\Delta}\ar[r,shift left=1mm,"|\ |"]&\mathbf{Top}\ar[l,shift left=1mm,"\Sing_\bullet"]
\end{tikzcd}\]
relating the category $\mathbf{Set}_\Delta$ of simplicial sets to the category $\mathbf{Top}$ of topological spaces. For any (pointed) topological space $X$, \cref{simplicial set singular homology from Sing} and \cref{simplicial set pi_1 from Sing} the singular homology and fundamental group of $X$ can be recovered from the simplicial set $\Sing_\bullet(X)$. In fact, one can say more: under mild assumptions, the entire homotopy type of $X$ can be recovered from $\Sing_\bullet(X)$. More precisely, there is always a canonical map $|\Sing_\bullet(X)|\to X$ (given by the counit of the adjunction described above), and Giever proved that it is always a weak homotopy equivalence (hence a homotopy equivalence when $X$ has the homotopy type of a CW complex). Consequently, for the purpose of studying homotopy theory, nothing is lost by replacing $X$ by $\Sing_\bullet(X)$ and working in the setting of simplicial sets, rather than topological spaces. In fact, it is possible to develop the theory of algebraic topology in entirely combinatorial terms, using simplicial sets as surrogates for topological spaces. However, not every simplicial set $S_\bullet$ behaves like the singular complex of a space, and it is therefore necessary to single out a class of "good" simplicial sets to work with. Later we will introduce a special class of simplicial sets, called \textit{Kan complexes}. By a theorem of Milnor, the homotopy theory of Kan complexes is equivalent to the classical homotopy theory of CW complexes.
\subsection{Simplicial and cosimplicial objects}
For every nonnegative integer $n$, we let $[n]$ denote the linearly ordered $\{0,1,\dots,n\}$. We define a category $\Delta$ (called the \textbf{simplex category}) over the ordered sets $[n]$ for $n\geq 0$ as follows: a morphism from $[m]$ to $[n]$ in the category $\Delta$ is a \textit{nondecreasing} function $\alpha:[m]\to[n]$. We note that the category $\Delta$ is equivalent to the category of all nonempty finite linearly ordered sets, with morphisms given by nondecreasing maps. In fact, we can say something even better: for every nonempty finite linearly ordered set $I$, there is a unique order-preserving bijection $I\cong[n]$, for some $n\geq 0$.\par
Now let $\mathcal{C}$ be an arbitray category. A \textbf{simplicial object} (resp. \textbf{cosimplicial object}) of $\mathcal{C}$ is defined to be a contravariant functor $\Delta^{\op}\to\mathcal{C}$ (resp. convariant functor $\Delta\to\mathcal{C}$). We often use the notation $C_\bullet$ to denote a simplicial object of $\mathcal{C}$. In this case, we use $C_n$ (resp. $C^n$) to denote the value of $C_\bullet$ (resp. $C^\bullet$) on $[n]\in\Delta^{\op}$.\par
As a variant of the category $\Delta$, we define $\Delta_{\inj}$ to be the category whose objects are sets of the form $[n]$ (where $n$ is a nonnegative integer) and whose morphisms are strictly increasing functions $\alpha:[m]\to[n]$. If $\mathcal{C}$ is a category, a contravariant functor $\Delta_{\inj}^{\op}\to\mathcal{C}$ (resp. covariant functor) as a \textbf{semi-simplicial object} (\textbf{semi-cosimplicial object}) of $\mathcal{C}$. We use the notation $C_\bullet$ to indicate a semi-simplicial object of $\mathcal{C}$, whose value on an object $[n]\in\Delta^{\op}_{\inj}$ we denote by $C_n$.
\begin{remark}
The category $\Delta_{\inj}$ can be regarded as a (non-full) subcategory of the category $\Delta$. Consequently, any simplicial object $C_\bullet$ of a category $\mathcal{C}$ determines a semisimplicial object of $\mathcal{C}$, given by the composition
\[\begin{tikzcd}
\Delta_{\inj}^{\op}\ar[r,hook]&\Delta^{\op}\ar[r,"C_\bullet"]&\mathcal{C}
\end{tikzcd}
\]
We will often abuse notation by identifying a simplicial object $C_\bullet$ of $\mathcal{C}$ with the underlying semisimplicial object of $\mathcal{C}$.
\end{remark}
To a first degree of approximation, a simplicial object $C_\bullet$ of a category $\mathcal{C}$ can be identified with the collection of objects $\{C_n\}_{n\geq 0}$. However, these objects are equipped with additional structure, arising from the morphisms in the simplex category $\Delta$. We now spell this out more concretely.\par
Let $n$ be a positive integer. For $0\leq i\leq n$, we let $\delta^i:[n-1]\to[n]$ be the unique strictly increasing function whose image does not contain the element $i$, given concretely by the formula
\[\delta^i(j)=\begin{cases}
j&\text{if $j<i$},\\
j+1&\text{if $j\geq i$}.
\end{cases}\]
If $C_\bullet$ is a (semi-)simplicial object of $\mathcal{C}$, then we can evaluate $C_\bullet$ on the morphism $\delta^i$ to obtain a morphism $d_i:C_n\to C_{n-1}$, called the $i$-th \textbf{face map} of $C_\bullet$. Dually, if $C^\bullet$ is a cosimplicial object of $\mathcal{C}$, then the evaluation of $C^\bullet$ on the morphism $d^i:C^{n-1}\to C^n$ determines a map $d^i:C^{n-1}\to C^n$, called the $i$-th \textbf{coface map}.\par
Smilarly, for $0\leq i\leq n$, let $\sigma^i:[n+1]\to[n]$ denote the function given by the formula
\[\sigma^i(j)=\begin{cases}
j&\text{if $j\leq i$},\\
j-1&\text{if $j>i$}.
\end{cases}\]
If $C_\bullet$ is a simplicial object of a category $\mathcal{C}$, then we can evaluate $C_\bullet$ on the morphism $\sigma_i$ to obtain a morphism $s_i:C_n\to C_{n+1}$, called the $i$-th \textbf{degeneracy map}. Dually, if $C^\bullet$ is a cosimplicial object of $\mathcal{C}$, then the evaluation on $C^\bullet$ on the morphism $\sigma_i$ determines a map $s^i:C^{n+1}\to C^n$, which is called the $i$-th \textbf{codegeneracy map}.
\begin{proposition}\label{simplicial set face degeneracy map char}
Let $C_\bullet$ be a simplicial object of a category $\mathcal{C}$. Then the face maps and degeneracy maps satisfy the following simplicial identities:
\begin{itemize}
\item[(S1)] For $0\leq i<j\leq n$, we have $d_i\circ d_j=d_{j-1}\circ d_i$.
\item[(S2)] For $0\leq i\leq j\leq n$, we have $s_i\circ s_j=s_{j+1}\circ s_i$.
\item[(S3)] For $0\leq i,j\leq n$, we have
\[d_i\circ s_j=\begin{cases}
s_{j-1}\circ d_i&\text{if $i<j$},\\
1_{C_n}&\text{if $i=j$ or $i=j+1$},\\
s_j\circ d_{i-1}&\text{if $i>j+1$}.
\end{cases}\] 
\end{itemize}
Conversely, any collection of objects $\{C_n\}_{n\geq 0}$ and morphisms $\{d_i:C_n\to C_{n-1},s_i:C_n\to C_{n+1}\}_{0\leq i\leq n}$ satisfying these conditions determine a unique simplicial object of $\mathcal{C}$.
\end{proposition}
\begin{proof}
The above conditions in fact hold for the maps $\delta^i$ and $\sigma^i$, so the first assertion follows from the functoriality. Conversely, let $\{d_i:C_n\to C_{n-1},s_i:C_n\to C_{n+1}\}_{0\leq i\leq n}$ be a family of morphisms satisfying the simplicial identities. In order to define a contravariant functor $C_\bullet:\Delta\to\mathcal{C}$ such that $C_\bullet([n])=C_n$, we show that any nondecreasing map $\alpha:[m]\to[n]$ can be written as a composition $\alpha=\delta\circ\sigma$, where
\[\delta=\delta^{j_1}\circ\cdots\circ\delta^{j_s},\quad 0\leq j_s\leq\cdots\leq j_1\leq n\]
and
\[\sigma=\sigma^{i_1}\circ\cdots\circ\sigma^{j_t},\quad 0\leq i_1<\cdots<i_t<m.\]
For this, let $j_s<\cdots<j_1$ be the elements of $[n]$ not in the image of $\alpha$ and $i_1<\dots<i_t$ be the elements of $[m]$ such that $\alpha(i)=\alpha(i+1)$. Then if $p=m-t=n-s$, the map $\alpha$ factors into
\[\begin{tikzcd}
{[m]}\ar[r,two heads,"\sigma"]&{[p]}\ar[r,hook,"\delta"]&{[n]}
\end{tikzcd}\]
It then suffices to decompose the maps $\delta$ and $\sigma$, which is straightforward.
\end{proof}
\begin{corollary}\label{semi-simplicial set face map char}
Let $C_\bullet$ be a semi-simplicial object of a category $\mathcal{C}$. Then the face maps satisfy the condition (S1). Conversely, any collection of objects $\{C_n\}_{n\geq 0}$ and morphisms $\{d_i:C_n\to C_{n-1}\}_{0\leq i\leq n}$ satisfying (S1) determine a unique semisimplicial object of $\mathcal{C}$.
\end{corollary}
\begin{proof}
This follows from the fact that any increasing map $\alpha:[m]\to[n]$ can be written as a composition of the $d_i$.
\end{proof}
Let $\mathbf{Set}$ denote the category of sets. A \textbf{simplicial set} is then by definition a simplicial object of $\mathbf{Set}$: that is, a contravariant functor $\Delta^{\op}\to\mathbf{Set}$. Similarly, a \textbf{semi-simplicial set} is a convariant functor $\Delta\to\mathbf{Set}$. If $S_\bullet$ is a (semi-)simplicial set, then the elements of $S_n$ are called the \textbf{$\bm{n}$-simplices} of $S_\bullet$. The elements of $S_0$ and $S_1$ are also called the \textbf{vertices} and \textbf{edges} of $S_\bullet$ of $S_\bullet$, respectively. The category of contravariant functors from $\Delta$ to $\mathbf{Set}$ is denoted by $\mathbf{Set}_\Delta=\Fun(\Delta^{\op},\mathbf{Set})$ and called the \textbf{category of simplicial sets}.
\begin{remark}
Since the category of sets has all (small) limits and colimits, the category of (semi-)simplicial sets also has all (small) limits and colimits. Moreover, these limits and colimits are computed levelwise: for any functor
\[S_\bullet:\mathcal{C}\to\mathbf{Set}_{\Delta},\quad \mathcal{C}\ni C\mapsto S_\bullet(C)\]
and any nonnegative integer $n$, we have canonical bijections
\[(\rlim_{C\in\mathcal{C}}S(C))_n=\rlim_{C\in\mathcal{C}}S_n(C),\quad (\llim_{C\in\mathcal{C}}S(C))_n=\llim_{C\in\mathcal{C}}S_n(C).\]
\end{remark}
\begin{example}
Let $S_\bullet$ be a simplicial set. Suppose that, for every integer $n\geq 0$, we are given a subset $T_n\sub S_n$, and that the face and degeneracy maps $d_i:S_n\to S_{n-1}$, $s_i:S_n\to S_{n+1}$ send $T_n$ into $T_{n-1}$ and $T_{n+1}$, respectively. Then by \cref{simplicial set face degeneracy map char}, the collection $\{T_n\}_{n\geq 0}$ inherits the structure of a simplicial set $T_\bullet$. In this case, we say that $T_\bullet$ is a simplicial subset of $S_\bullet$ and write $T_\bullet\sub S_\bullet$.
\end{example}
We now consider some elementary examples of simplicial sets. First, for any integer $n\geq 0$, the object $[n]$ itself determines a simplicial set $\Delta^n=h^{[n]}$: that is, the simplicial set which send $[m]\in\Delta^{\op}$ to $\Hom_\Delta([m],[n])$. The simplicial set $\Delta^n$ is called the \textbf{standard $\bm{n}$-simplex}, and by convention, we set $\Delta^{-1}=\emp$ for $n=-1$.\par
Since any set can be mapped onto $[0]$ in a unique way, it is clear that the standard $0$-simplex $\Delta^0$ is a final object of the category of simplicial sets; note that $\Delta^0$ carries each $[n]$ to a set having a single element.\par
For each $n\geq 0$, the standard $n$-simplex $\Delta^n$ is characterized by the following universal property: for every simplicial set $S_\bullet$, Yoneda's lemma supplies a bijection
\[\Hom_{\mathbf{Set}_{\Delta}}(\Delta^n,S_\bullet)\cong S_n.\]
We often invoke this bijection implicitly to identify $n$-simplices of $S_\bullet$ with maps of simplicial sets $\sigma:\Delta^n\to S_\bullet$.
\begin{example}
Let $S_\bullet$ be a simplicial set and let $v$ be a vertex of $S_\bullet$. Then $v$ can be identified with a map of simplicial sets $\Delta^0\to S_\bullet$. This map is automatically a monomorphism (note that $\Delta^0$ has only a single $n$-simplex for every $n\geq 0$), whose image is a simplicial subset of $S_\bullet$. It will often be convenient to denote this simplicial subset by $\{v\}$. For example, since $\Delta^n([0])=\Hom_{\Delta}([0],[n])$, we can identify the vertices of the standard $n$-simplex $\Delta^n$ with the set $[n]$: every integer $0\leq i\leq n$ determines a simplicial subset $\{i\}\sub\Delta^n$ whose $k$-simplices are the constant maps $[k]\to[n]$ taking the value $i$.
\end{example}
Let $n\geq 0$ be an integer. We define a simplicial set $(\partial\Delta^n)$ by the formula
\[(\partial\Delta^n)([m])=\{\alpha\in\Hom_{\Delta}([m],[n]):\text{$\alpha$ is not surjective}\}.\]
Note that we can regard $\partial\Delta^n$ as a simplicial subset of the standard $n$-simplex $\Delta^n$. It is called the \textbf{boundary} of $\partial\Delta^n$. 
\begin{example}
We can think of the boundary $\partial\Delta^n$ as obtained from the $n$-simplex $\Delta^n$ by removing its interior. It is clear that $\partial\Delta^0=\emp$. Now for $0\leq i\leq n$, the map $\delta^i:[n-1]\to[n]$ determines a map of simplicial sets $\Delta^{n-1}\to\Delta^n$ which factors through the simplicial subset $\partial\Delta^n\sub\Delta^n$. We therefore obtain a map of simplicial sets $\Delta^{n-1}\to\partial\Delta^n$, which is also denoted by $\delta^i$. Now let $S_\bullet$ be a simplicial set, and consider the injective map
\[\Hom_{\mathbf{Set}_\Delta}(\partial\Delta^n,S_\bullet)\to\prod_{i\in[n]}S_{n-1},\quad (f:\partial\Delta^n\to S_\bullet)\mapsto\{f\circ\delta^i\}_{0\leq i\leq n}\]
where we use Yoneda's lemma to identify $\Hom_{\mathbf{Set}_\Delta}(\Delta^{n-1},S_\bullet)$ with $S_{n-1}$. The image of this map is the collection of sequences of $(n-1)$-simplicies $(\sigma_0,\sigma_1,\dots,\sigma_n)$ satisfying the identities $d_i(\sigma_j)=d_{j-1}(\sigma_i)$ for $0\leq i<j\leq n$.
\end{example}
Suppose we are given integers $0\leq i\leq n$. We define a simplicial set $\Lambda_i^n$ by the formula
\[(\Lambda_i^n)([m])=\{\alpha\in\Hom_{\Delta}([m],[n]):\alpha([m])\cup\{i\}\neq[n]\}.\]
We can consider $\Lambda_i^n$ as a simplicial subset of the boundary $\partial\Delta^n\sub\Delta^n$. It is called the $i$-th \textbf{horn} in $\Delta^n$. We say that $\Lambda_i^n$ is an \textbf{inner horn} if $0<i<n$, and an \textbf{outer horn} if $i=0$ or $n$.
\begin{example}\label{simplicial set horn description}
Roughly speaking, one can think of the horn $\Lambda_i^n$ as obtained from the $n$-simplex $\Delta^n$ by removing its interior together with the face opposite its $i$-th vertex. For example, the horns contained in $\Delta^2$ are depicted in the following diagram
\[\begin{tikzcd}[row sep=12mm,column sep=8mm]
&\{1\}\ar[d,phantom,"\Lambda_0^2"]\ar[rd,dashed]&\\
\{0\}\ar[ru]\ar[rr]&{}&\{2\}
\end{tikzcd}\quad \begin{tikzcd}[row sep=12mm,column sep=8mm]
&\{1\}\ar[d,phantom,"\Lambda_1^2"]\ar[rd]&\\
\{0\}\ar[ru]\ar[rr,dashed]&{}&\{2\}
\end{tikzcd}\quad \begin{tikzcd}[row sep=12mm,column sep=8mm]
&\{1\}\ar[d,phantom,"\Lambda_2^2"]\ar[rd]&\\
\{0\}\ar[ru,dashed]\ar[rr]&{}&\{2\}
\end{tikzcd}\]
Here the dotted arrows indicate edges of $\Delta^2$ which are not contained in the corresponding horn.\par
Now let $0\leq i\leq n$ be integers. For $j\in[n]\setminus\{i\}$, the map $\delta^j$ can be regarded as a map of simplicial sets from $\Delta^{n-1}$ to $\Lambda_i^n\sub\Delta^n$. For any simplicial set $S_\bullet$, we can therefore consider the injective map
\[\Hom_{\mathbf{Set}_\Delta}(\Lambda_i^n,S_\bullet)\to\prod_{j\in[n]\setminus\{i\}}S_{n-1},\quad (f:\Lambda_i^n,S_\bullet)\mapsto\{f\circ\delta^j\}_{j\in[n]\setminus\{i\}}.\]
It is not hard to see that the image of this map is the collection of "imcomplete" sequences $(\sigma_0,\dots,\sigma_{i-1},\ast,\sigma_{i+1},\dots,\sigma_n)$ satisfying $d_j(\sigma_k)=d_{k-1}(\sigma_j)$ for $j,k\in[n]\setminus\{i\}$, $j<k$.
\end{example}
\subsection{The skeletal filtration}
Roughly speaking, one can think of the simplicial sets $\Delta^n$ as  elementary building blocks out of which more complicated simplicial sets can be constructed. In this subsection, we make this idea more precise by introducing the skeletal filtration of a simplicial set. This filtration allows us to write every simplicial set $S_\bullet$ as the union of an increasing sequence of simplicial subsets
\[\sk_0(S_\bullet)\sub \sk_1(S_\bullet)\sub\cdots\sub \sk_n(S_\bullet)\sub\cdots\]
where each $\sk_n(S_\bullet)$ is obtained from $\sk_{n-1}(S_\bullet)$ by attaching copies of $\Delta^n$. Before doing this, we need some preparations.
\begin{proposition}\label{simplicial set n-simplex image of n-1 iff}
Let $S_\bullet$ be a simplicial set and $\tau\in S_n$ be an $n$-simplex of $S_\bullet$ for $n>0$, which is also considered as a map $\tau:\Delta^n\to S_\bullet$. Then the following are equivalent:
\begin{itemize}
\item[(\rmnum{1})] The simplex $\tau$ belongs to the image of the degeneracy map $s_i:S_{n-1}\to S_n$ for some $0\leq i\leq n-1$.
\item[(\rmnum{2})] The map $\tau$ factors as a composition $\Delta^{n}\stackrel{f}{\to}\Delta^{n-1}\to S_\bullet$, where $f$ corresponds to a surjective map of linearly ordered sets $[n]\to[n-1]$.
\item[(\rmnum{3})] The map $\tau$ factors as a composition $\Delta^{n}\stackrel{f}{\to}\Delta^{m}\to S_\bullet$, where $f$ corresponds to a surjective map of linearly ordered sets $[n]\to[m]$.
\item[(\rmnum{4})] The map $\tau$ factors as a composition $\Delta^n\to\Delta^m\to S_\bullet$ where $m<n$.
\item[(\rmnum{5})] The map $\tau$ factors as a composition $\Delta^{n}\stackrel{\tau'}{\to}\Delta^{m}\to S_\bullet$, where $\tau'$ is not injective on vertices.
\end{itemize}
\end{proposition}
\begin{proof}
The implications (\rmnum{1})$\Leftrightarrow$(\rmnum{2})$\Rightarrow$(\rmnum{3})$\Rightarrow$(\rmnum{4})$\Rightarrow$(\rmnum{5}) are immediate. We now show that (\rmnum{5})$\Rightarrow$(\rmnum{1}). Assume that $\tau$ factors into $\Delta^{n}\stackrel{\tau'}{\to}\Delta^{m}\to S_\bullet$, where $\tau'$ is not injective on vertices. Then there exists some integer $0\leq i<n$ satisfying $\tau'(i)=\tau'(i+1)$. It follows that $\tau'$ factors through the map $\sigma^i:\Delta^n\to\Delta^{n-1}$ of, so that $\tau$ belongs to the image of the degeneracy map $s_i$.
\end{proof}
Let $S_\bullet$ be a simplicial set and let $\sigma:\Delta^n\to S_\bullet$ be an $n$-simplex of $S_\bullet$. We say that $\sigma$ is \textbf{degenerate} if $n>0$ and $\sigma$ satisfies the equivalent conditions of \cref{simplicial set n-simplex image of n-1 iff}; otherwise $\sigma$ is said to be \textbf{nondegenerate} (in particular, every $0$-simplex of $S_\bullet$ is nondegenerate).
\begin{example}
Let $f:S_\bullet\to T_\bullet$ be a map of simplicial sets. If $\sigma$ is a degenerate $n$-simplex of $S_\bullet$, then $f(\sigma)$ is a degenerate $n$-simplex of $T_\bullet$. The converse holds if $f$ is a monomorphism of simplicial sets (for example, if $S_\bullet$ is a simplicial subset of $T_\bullet$).
\end{example}
\begin{proposition}\label{simplicial set map from Delta^n factorization}
Let $\sigma:\Delta^n\to S_\bullet$ be a map of simplicial sets. Then $\sigma$ can be factored uniquely as a composition
\[\begin{tikzcd}
\Delta^n\ar[r,"\alpha"]&\Delta^m\ar[r,"\tau"]&S_\bullet
\end{tikzcd}\]
where $\alpha$ corresponds to a surjective map of linearly ordered sets $[n]\to[m]$ and $\tau$ is a nondegenerate $m$-simplex of $S_\bullet$.
\end{proposition}
\begin{proof}
Let $m$ be the smallest nonnegative integer for which $\sigma$ can be factored as a composition $\Delta^n\stackrel{\alpha}{\to}\Delta^m\stackrel{\tau}{\to}S_\bullet$. It follows from the minimality of $m$ that $\alpha$ must induce a surjection of linearly ordered sets $[n]\to[m]$ (otherwise, we could replace $[m]$ by the image of $\alpha$) and that the $m$-simplex $\tau$ is nondegenerate. This proves the existence of the desired factorization.\par
Suppose we are given another factorization of $\sigma$ as a
composition $\Delta^n\stackrel{\alpha'}{\to}\Delta^{m'}\stackrel{\tau'}{\to}S_\bullet$, and assume that $\alpha'$ induces a surjection $[n]\to[m']$. We first prove that for any pair of integers $0\leq i<j\leq n$ satisfying $\alpha'(i)=\alpha'(j)$, we also have $\alpha(i)=\alpha(j)$. Assume otherwise; then $\alpha$ admits a section $\beta:\Delta^m\hookrightarrow\Delta^n$ whose images include $i$ and $j$, and we have
\[\tau=\tau\circ\alpha\circ\beta=\sigma\circ\beta=\tau'\circ\alpha'\circ\beta.\]
Our assumption that $\alpha'(i)=\alpha'(j)$ guarantees that the map $(\alpha'\circ\beta):\Delta^m\to\Delta^{m'}$ is not injective on vertices, contradicting to the fact that $\tau$ is nondegenerate.\par
Now it follows from the preceding argument that $\alpha$ factors uniquely as a composition $\Delta^n\stackrel{\alpha'}{\to}\Delta^{m'}\stackrel{\eta}{\to}\Delta^m$, for some morphism $\eta:\Delta^{m'}\to\Delta^m$ (which is also surjective on vertices). Let $\beta'$ be a section of $\alpha'$, and note that we have
\[\tau'=\tau'\circ\alpha'\circ\beta'=\sigma\circ\beta'=\tau\circ\alpha\circ\beta'=\tau\circ\eta\circ\alpha'\circ\beta'=\tau\circ\eta.\]
Consequently, if the simplex $\tau'$ is nondegenerate, then $\eta$ must also be injective on vertices. It follows that $m'=m$ and $\eta$ is the identity map, so $\alpha=\alpha'$ and $\tau=\tau'$.
\end{proof}
Let $S_\bullet$ be a simplicial set, $k\geq-1$ be an integer, and $\sigma:\Delta^n\to S_\bullet$ be an $n$-simplex of $S_\bullet$. The proof of \cref{simplicial set map from Delta^n factorization} shows that the following conditions are equivalent:
\begin{itemize}
\item[(a)] Let $\Delta^n\stackrel{\alpha}{\to}\Delta^m\stackrel{\tau}{\to}S_\bullet$ be the factorization of \cref{simplicial set map from Delta^n factorization} (so that $\alpha$ induces a surjection $[n]\to[m]$, the map $\tau$ is nondegenerate, and $\sigma=\tau\circ\alpha$). Then $m\leq k$.
\item[(b)] There exists a factorization $\Delta^n\to\Delta^{m'}\to S_\bullet$ of $\sigma$ for which $m'\leq k$.
\end{itemize}
For each positive integer $n\geq 0$, we denote by $\sk_k(S_n)$ the subset of $S_n$ consisting of those $n$-simplices which satisfy the equivalent conditions above. In view of condition (b), the collection of subsets $\{\sk_k(S_n)\}_{n\geq 0}$ is stable under the face and degeneracy operators of $S_\bullet$, and therefore determine a simplicial subset of $S_\bullet$. This simplicial subset is denoted by $\sk_k(S_\bullet)$ and called the \textbf{$\bm{k}$-skeleton} of $S_\bullet$.\par
By definition, it is clear that $\sk_k(S_n)$ contains every $n$-simplex of $S_\bullet$ if $n\leq k$. In particular, $\bigcup_{k\geq 0}\sk_k(S_\bullet)$ is equal to $S_\bullet$, so the collection $\{\sk_k(S_\bullet)\}_{k\geq-1}$ is a filtration of $S_\bullet$. We also note that a nondegenerate $n$-simplex $\sigma$ of $S_\bullet$ is contained in a skeleton $\sk_k(S_\bullet)$ if and only if $n\leq k$.\par
We define the \textbf{dimension} of a simpliciat set $S_\bullet$ to be the smallest integer $k$ such that for every integer $n>k$, every $n$-simplex of $S_\bullet$ is degenerate. A simplicial set $S_\bullet$ is said to be \textbf{finite-dimensional} if it has dimension $\leq k$ for some $k\gg 0$. Using the dimension of $S_\bullet$, we can now prove the following universal property for skeletons of $S_\bullet$.
\begin{proposition}\label{simplicial set skeleton universal prop}
Let $S_\bullet$ be a simplicial set and $k\geq -1$ be an integer.
\begin{itemize}
\item[(a)] The simplicial set $\sk_k(S_\bullet)$ has dimension $\leq k$.
\item[(b)] For every simplicial set $T_\bullet$ of dimension $\leq k$, composition with the inclusion map $\sk_k(S_\bullet)\hookrightarrow S_\bullet$ induces a bijection
\[\Hom_{\mathbf{Set}_\Delta}(T_\bullet,\sk_k(S_\bullet))\to\Hom_{\mathbf{Set}_\Delta}(T_\bullet,S_\bullet).\]
In other words, any map of simplicial sets $T_\bullet\to S_\bullet$ factors through $\sk_k(S_\bullet)$.
\end{itemize}
\end{proposition}
\begin{proof}
The first assertion is clear. To prove (b), suppose that $f:T_\bullet\to S_\bullet$ is a map of simplicial sets, where $T_\bullet$ has dimension $\leq k$. We wish to show that $f$ carries every $n$-simplex $\sigma$ of $T_\bullet$ to an $n$-simplex of $\sk_k(S_\bullet)$. Using \cref{simplicial set map from Delta^n factorization}, we can reduce to the case where $\sigma$ is a nondegenerate $n$-simplex of $T_\bullet$. In this case, our assumption that $T_\bullet$ has dimension $\leq k$ guarantees that $n\leq k$, so that $f(\sigma)$ belongs to $\sk_k(S_\bullet)$ be the definition of $\sk_k(S_\bullet)$.
\end{proof}
\begin{proposition}\label{simplicial set dimension of product}
Let $S_\bullet$ and $S'_\bullet$ be simplicial sets of dimension $k$ and $k'$, respectively. Then the product $S_\bullet\times S'_\bullet$ is of dimension $k+k'$.
\end{proposition}
\begin{proof}
Let $(\sigma,\sigma')$ be a nondegenerate $n$-simplex of the product $S_\bullet\times S'_\bullet$. Using \cref{simplicial set map from Delta^n factorization}, we see that $\sigma$ and $\sigma'$ admit factorizations
\[\begin{tikzcd}
\Delta^n\ar[r,"\alpha"]&\Delta^{m}\ar[r,"\tau"]&S_\bullet
\end{tikzcd}\quad \begin{tikzcd}
\Delta^n\ar[r,"\alpha'"]&\Delta^{m'}\ar[r,"\tau'"]&S'_\bullet
\end{tikzcd}\]
where $\tau$ and $\tau'$ are nondegenerate, so that $m\leq k$, $m'\leq k'$. It follows that $(\sigma,\sigma')$ factors as a composition
\[\begin{tikzcd}
\Delta^n\ar[r,"{(\alpha,\alpha')}"]&\Delta^m\times\Delta^{m'}\ar[r,"{\tau\times\tau'}"]&S_\bullet\times S'_\bullet
\end{tikzcd}\]
The nondegeneracy of $\eta$ guarantees that the map of partially ordered sets $[n]\stackrel{(\alpha,\alpha')}{\to}[m]\times[m']$ is a monomorphism, so that $n\leq m+m'\leq k+k'$. Thus we have shown that the dimension of $S_\bullet\times S_\bullet'$ is $\leq k+k'$.\par
On the other hand, since $\dim(S_\bullet)=k$, $\dim(S_\bullet')=k'$, there exists nondegenerate simplicies $\sigma:\Delta^k\to S_\bullet$ and $\sigma':\Delta^{k'}\to S_\bullet'$. We consider the unique factorization
\[\begin{tikzcd}
\Delta^k\times\Delta^{k'}\ar[r,"\beta"]&\Delta^m\ar[r,"\gamma"]&S_\bullet\times S_\bullet'
\end{tikzcd}\]
of $\sigma\times\sigma'$, where $\gamma$ is nondegenerate. By projecting onto $S_\bullet$ and $S'_\bullet$, we obtain factorizations
\[\begin{tikzcd}
\Delta^k\ar[r,"\beta_1"]&\Delta^m\ar[r,"\gamma_1"]&S_\bullet
\end{tikzcd},\quad\begin{tikzcd}
\Delta^{k'}\ar[r,"\beta_2"]&\Delta^m\ar[r,"\gamma_2"]&S'_\bullet
\end{tikzcd}\]
of $\sigma$ and $\sigma'$, respectively. Since $\sigma$ and $\sigma'$ are nondegenerate, the maps $\beta_1$ and $\beta_2$ are necessarily monomorphisms, and thus so is their product $\beta:\Delta^k\times\Delta^{k'}\to\Delta^m$. This implies $m\geq k+k'$, and so $\sigma\times\sigma'$ is nondegenerate. We then conclude that $\dim(S_\bullet\times S'_\bullet)=k+k'$.
\end{proof}
Let $S_\bullet$ be a simplicial set. For each $k\geq 0$, we let $S_k^{\circ}$ denote the collection of all \textit{nondegenerate} $k$-simplices of $S_\bullet$. Every element $\sigma\in S_k^{\circ}$ determines a map of simplicial sets $\Delta^k\to\sk_k(S_\bullet)$. Since the boundary $\partial\Delta^k\sub\Delta^k$ has dimension $\leq k-1$, this map carries $\partial\Delta^k$ into the $(k-1)$-skeleton $\sk_{k-1}(S_\bullet)$.
\begin{proposition}\label{simplicial set skeleton pushout square}
Let $S_\bullet$ be a simplicial set and let $k\geq 0$. Then we have a pushout square
\[\begin{tikzcd}
\coprod_{\sigma\in S_k^{\circ}}\partial\Delta^k\ar[r]\ar[d]&\coprod_{\sigma\in S_k^{\circ}}\Delta^k\ar[d]\\
\sk_{k-1}(S_\bullet)\ar[r]&\sk_k(S_\bullet)
\end{tikzcd}\]
in the category $\mathbf{Set}_\Delta$ of simplicial sets.
\end{proposition}
\begin{proof}
Unwinding the definitions, we only need to prove the following: Let $\tau$ be an $n$-simplex of $\sk_k(S_\bullet)$ which is not contained in $\sk_{k-1}(S_\bullet)$, then $\tau$ factors uniquely as a composition
\[\begin{tikzcd}
\Delta^n\ar[r,"\alpha"]&\Delta^k\ar[r,"\sigma"]&S_\bullet
\end{tikzcd}\]
where $\sigma$ is a nondegenerate simplex of $S_\bullet$ and $\alpha$ does not factor through the boundary $\partial\Delta^k$ (in other words, $\alpha$ induces a surjection of linearly ordered sets $[n]\to[k]$). Now \cref{simplicial set map from Delta^n factorization} implies that any $n$-simplex of $S_\bullet$ admits a unique factorization $\Delta^n\stackrel{\alpha}{\to}\Delta^m\stackrel{\sigma}{\to}S_\bullet$, where $\alpha$ is surjective and $\sigma$ is nondegenerate. Our assumption that $\tau$ belongs to the $\sk_k(S_\bullet)$ guarantees that $m\leq k$, and that $\tau$ does not belong to $\sk_{k-1}(S_\bullet)$ implies that $m\geq k$.
\end{proof}
\subsection{Discrete simplicial sets}
Let $\mathcal{C}$ be a category. For each object $C\in\mathcal{C}$, we let $\uline{C}_\bullet$ denote the constant functor $\Delta\to\{C\}\hookrightarrow\mathcal{C}$ taking the value $C$ on objects and identity morphism $1_C$ on morphisms. We regard $C_\bullet$ as a simplicial object of $\mathcal{C}$, which is called the \textbf{constant simplicial object with value $\bm{C}$}. The constant simplicial object $\uline{C}_\bullet$ can be characterized by a universal property:
\begin{proposition}\label{simplicial set constant universal prop}
Let $\mathcal{C}$ be a category and let $C$ be an object of $\mathcal{C}$. For any simplicial object $S_\bullet$ of $\mathcal{C}$, the canonical map
\[\Hom_{\Fun(\Delta^{\op},\mathcal{C})}(\uline{C}_\bullet,S_\bullet)\to\Hom_{\mathcal{C}}(C,S_0)\]
is a bijection.
\end{proposition}
\begin{proof}
Let $f:C\to S_0$ be a morphism in $\mathcal{C}$; we want to show that $f$ can be extended uniquely to a map of simplicial sets $f_\bullet:\uline{C}_\bullet\to S_\bullet$. For this, we define $f_\bullet$ be the natural transformation whose value on an object $[n]\in\Delta^{\op}$ is given by the composite map
\[\begin{tikzcd}
\uline{C}_n=C\ar[r,"f"]&S_0\ar[r,"S_{\alpha(n)}"]&S_n
\end{tikzcd}\]
where $\alpha(n)$ denotes the unique morphism in $\Delta$ from $[n]$ to $[0]$. To prove the naturality of $f_\bullet$, we observe that for any nondecreasing map $\beta:[m]\to[n]$, we have a commutative diagram
\[\begin{tikzcd}
\uline{C}_n\ar[r,equal]\ar[d,"\uline{C}_\beta"]&C\ar[r,"f"]\ar[d,equal]&S_0\ar[r,"S_{\alpha(n)}"]\ar[d,equal]&S_n\ar[d,"S_\beta"]\\
\uline{C}_m\ar[r,equal]&C\ar[r,"f"]&S_0\ar[r,"S_{\alpha(m)}"]&S_m
\end{tikzcd}\]
where the commutativity of the square on the right follows from the observation that $\alpha(m)$ is equal to the composition $[m]\stackrel{\beta}{\to}[n]\stackrel{\alpha(n)}{\to}[0]$.
\end{proof}
\begin{corollary}\label{simplicial set limit of S_n isomorphic to S_0}
Let $\mathcal{C}$ be a category. Then for any simplicial object $S_\bullet$ of $\mathcal{C}$, the limit $\llim_{[n]\in\Delta^{\op}}S_n$ exists in the category $\mathcal{C}$ and is isomorphic to $S_0$.
\end{corollary}
\begin{proof}
This follows from \cref{simplicial set constant universal prop} by noting that a map $\uline{C}_\bullet\to S_\bullet$ consists of morphisms $C\to S_n$ satisfying compatible conditions. Alternatively, this can be deduced formally from the observation that $[0]$ is a final object of the category $\Delta$ (and therefore an initial object of the category $\Delta^{\op}$).
\end{proof}
\begin{corollary}\label{simplicial set evaluation adjoint to constant functor}
Let $\mathcal{C}$ be a category. Then the evaluation functor
\[\ev_0:\Fun(\Delta^{\op},\mathcal{C})\to\mathcal{C},\quad S_\bullet\to S_0\]
admits a left adjoint, given on objects by the formation of constant simplicial objects $C\mapsto\uline{C}_\bullet$.
\end{corollary}
\begin{corollary}\label{simplicial set constant functor is embedding}
Let $\mathcal{C}$ be a category. Then the construction $C\mapsto\uline{C}_\bullet$ determines a fully faithful embedding from $\mathcal{C}$ to the category $\Fun(\Delta^{\op},\mathcal{C})$ of simplicial objects of $\mathcal{C}$.
\end{corollary}
\begin{proof}
If $C$ and $D$ are objects of $\mathcal{C}$, we want to show that the canonical map
\[\theta:\Hom_{\mathcal{C}}(C,D)\to\Hom_{\Fun(\Delta^{\op},\mathcal{C})}(\uline{C}_\bullet,\uline{D}_\bullet)\]
is bijective. This is clear since $\theta$ is the right inverse to the evaluation map
\[\Hom_{\Fun(\Delta^{\op},\mathcal{C})}(\uline{C}_\bullet,\uline{D}_\bullet)\to\Hom_{\mathcal{C}}(C,D)\]
which is bijective in view of \cref{simplicial set constant universal prop}.
\end{proof}
In particular, if we take $\mathcal{C}=\mathbf{Set}$, then there is an embedding of $\mathbf{Set}$ to $\mathbf{Set}_{\Delta}$ given by $X\mapsto\uline{X}_\bullet$. Given this, we say that a simplicial set $S_\bullet$ is \textbf{discrete} if there exists a set $X$ such that $S_\bullet\cong\uline{X}_\bullet$. It is clear that the following corollary is true.
\begin{corollary}\label{simplicial set constant functor image is discrete}
The functor $X\mapsto\uline{X}_\bullet$ determines a fully faithful embedding $\mathbf{Set}\to\mathbf{Set}_{\Delta}$ whose essential image is the full subcategory of $\mathbf{Set}_\Delta$ spanned by the discrete simplicial sets.
\end{corollary}
Let $S$ be a set. We will often abuse notation by identifying $S$ with the constant simplicial set $\uline{S}_\bullet$. This abuse will occur most frequently in the special case where $S=\{v\}$ consists of a single vertex $v$ of some other simplicial set $X_\bullet$. In this case, we view $\{v\}$ as a simplicial subset of $X_\bullet$ which is abstractly isomorphic to $\Delta^0$.
\begin{remark}
The fully faithful embedding $\mathbf{Set}\to\mathbf{Set}_\Delta,S\mapsto\uline{S}_\bullet$ preserves (small) limits and colimits (since limits and colimits of simplicial sets are computed levelwise). It follows that the collection of discrete simplicial sets is closed under the formation of (small) limits and colimits in $\mathbf{Set}_\Delta$.
\end{remark}
\begin{proposition}\label{simplicial set discrete iff}
Let $S_\bullet$ be a simplicial set. The following conditions are equivalent:
\begin{itemize}
\item[(\rmnum{1})] $S_\bullet$ is discrete.
\item[(\rmnum{2})] For every morphism $\alpha:[m]\to[n]$ in the category $\Delta$, the induced map $S_n\to S_m$ is a bijection.
\item[(\rmnum{3})] For every positive integer $n$, the $0$-th face map $d_0:S_n\to S_{n-1}$ is a bijection.
\item[(\rmnum{4})] $S_\bullet$ is of zero dimension.   
\end{itemize}
\end{proposition}
\begin{proof}
The implication (\rmnum{1})$\Rightarrow$(\rmnum{2}) follows from the definition of $\uline{X}_\bullet$, and the implication (\rmnum{2})$\Rightarrow$(\rmnum{3}) is immediate. To prove that (\rmnum{3})$\Rightarrow$(\rmnum{4}), we observe that if the face map $d_0:S_n\to S_{n-1}$ is bijective, then the degeneracy operator $s_0:S_{n-1}\to S_n$ is also bijective (since it is a right inverse of $d_0$). In particular, $s_0$ is surjective, so every $n$-simplex of $S_\bullet$ is degenerate by \cref{simplicial set n-simplex image of n-1 iff}.\par
We complete the proof by showing that (\rmnum{4})$\Rightarrow$(\rmnum{1}). If $S_\bullet$ is a simplicial set of dimension $0$ and $X=S_0$ is the set of vertices of $S_\bullet$, then \cref{simplicial set skeleton pushout square} supplies an isomorphism of simplicial sets $\amalg_{v\in X}\Delta^0\cong S_\bullet$, whose domain can be identified with the constant simplicial set $\uline{X}_\bullet$.
\end{proof}
\begin{example}\label{simplicial set from adjoint pair}
Let $F:\mathcal{C}\to\mathcal{D}$ and $G:\mathcal{D}\to\mathcal{C}$ be functors between categories such that $(F,G)$ is an adjoint pair. In this case, we have two natural transformations
\[\eta:1_{\mathcal{C}}\to GF,\quad \eps:FG\to 1_{\mathcal{D}}.\]
We write $L=FG$ and $\delta=F\eta G$, so that $\eps:L\to 1$ is the counit map and $\delta:L\to L^2$ is the comultiplication. Now suppose that $D$ is any fixed object in $\mathcal{D}$ and set $S_n=L^{n+1}(D)$ for each positive integer $n\geq 0$. For any integer $0\leq i\leq n$, the counit map $\eps$ then gives face morphisms
\[d_i=L^{n-i}\eps L^i(D):L^{n+1}(D)\to L^n(D),\]
whilst the comultiplication gives degeneracy morphisms
\[s_i=L^{n-1-i}\delta L^i:L^n(D)\to L^{n+1}(D).\]
It can be verify that the $d_i$ and $s_i$ satisfy simplicial identities, so we obtain a simplicial object $S_\bullet$ of $\mathcal{D}$. Note that we have an agumentation map $S_\bullet\to\uline{D}_\bullet$ given by the counit map $\eps_D:L(D)\to D$.
\end{example}
We now generalize \cref{simplicial set discrete iff} to obtain a concrete description of simplicial sets of dimension $\leq 1$. But for this, let us recall that a directed graph $G$ is by definition the following data:
\begin{itemize}
\item A set $V(G)$ of \textbf{verticies} of $G$.
\item A set $E(G)$ of \textbf{edges} of $G$.
\item A pair of functions $s,t:E(G)\to V(G)$ which assign to each edge $e\in E(G)$ a pair of vertices $s(e),t(e)\in V(G)$, called the \textbf{source} and \textbf{target} of $e$, respectively.
\end{itemize}
Let $G$ and $G'$ be directed graphs. A \textbf{morphism} from $G$ to $G'$ is a function $f:V(G)\amalg E(G)\to V(G')\amalg E(G')$ which satisfies the following conditions:
\begin{itemize}
\item For each vertex $v\in V(G)$, the image $f(v)$ belongs to $V(G')$.
\item Let $e\in E(G)$ be an edge of $G$ with source $v=s(e)$ and target $w=t(e)$. Then the image $f(e)$ is either an edge of $G'$ having source $s(f(e))=f(v)$ and target $t(f(e))=f(w)$, or a vertex of $G'$ satisfying $f(v)=f(e)=f(w)$.
\end{itemize}
We let $\mathbf{Graph}$ denote the category whose objects are directed graphs and whose morphisms are morphisms of directed graphs (with composition defined in the evident way). We now construct a functor from $\mathbf{Set}_\Delta$ to $\mathbf{Graph}$. To every simplicial set $S_\bullet$, we can associate a directed graph $\Gamma(S_\bullet)$ as follows:
\begin{itemize}
\item The vertex set $V(\Gamma(S_\bullet))$ is the set of $0$-simplices of $S_\bullet$.
\item The edge set $E(\Gamma(S_\bullet))$ is the set of nondegenerate $1$-simplcies of $S_\bullet$.
\item For every edge $e\in E(\Gamma(S_\bullet))\sub S_1$, the source $s(e)$ is the vertex $d_1(e)$, and the target $t(e)$ is the vertex $d_0(e)$ (here $d_0,d_1$ are the face maps).
\end{itemize}
Note that the disjoint union $V(\Gamma(S_\bullet))\amalg E(\Gamma(S_\bullet))$ can be identified with the set $S_1$ of all $1$-simplices of $S_\bullet$ (where we identify $V(\Gamma(S_\bullet))$ with the collection of degenerate $1$-simplices via the degeneracy map $s_0:S_0\to S_1$).
\begin{proposition}\label{simplicial set graph functor def}
Let $f:S_\bullet\to T_\bullet$ be a map of simplicial sets. Then the induced map
\[\begin{tikzcd}
V(\Gamma(S_\bullet))\amalg E(\Gamma(S_\bullet))\cong S_1\ar[r,"f"]&T_1\cong V(\Gamma(T_\bullet))\amalg E(\Gamma(T_\bullet))
\end{tikzcd}\]
is a morphism of directed graphs from $\Gamma(S_\bullet)$ to $\Gamma(T_\bullet)$.
\end{proposition}
\begin{proof}
Since $f$ commutes with the degeneracy operator $s_0$, it carries degenerate $1$-simplices of $S_\bullet$ to degenerate $1$-simplices of $T_\bullet$, and therefore takes vertices to vertices. The second condition follows from the fact that $f$ commutes with the face operators $d_0$ and $d_1$.
\end{proof}
From Propotision~\ref{simplicial set graph functor def}, the construction $S_\bullet\mapsto\Gamma(S_\bullet)$ is a functor from $\mathbf{Set}_\Delta\to\mathbf{Graph}$. We shall use this functor to characterize simplicial sets of dimension $\leq 1$.
\begin{proposition}\label{simplicial set graph functor on dim 1 faithful}
Let $S_\bullet$ and $T_\bullet$ be simplicial sets. If $S_\bullet$ has dimension $\leq 1$, then the canonical map
\[\Hom_{\mathbf{Set}_\Delta}(S_\bullet,T_\bullet)\to\Hom_{\mathbf{Graph}}(\Gamma(S_\bullet),\Gamma(T_\bullet))\]
is bijective.
\end{proposition}
\begin{proof}
If $S_\bullet$ has dimension $\leq 1$, then \cref{simplicial set skeleton pushout square} provides a pushout diagram
\[\begin{tikzcd}
\coprod_{e\in E(\Gamma(S_\bullet))}\partial\Delta^1\ar[r]\ar[d]&\coprod_{e\in E(\Gamma(S_\bullet))}\ar[d]\\
\coprod_{v\in V(\Gamma(S_\bullet))}\Delta^0\ar[r]&S_\bullet
\end{tikzcd}\]
It then follows that, for any simplicial set $T_\bullet$, we can identify $\Hom_{\mathbf{Set}_\Delta}(S_\bullet,T_\bullet)$ with the fiber product
\[\Big(\prod_{e\in E(\Gamma(S_\bullet))}T_1\Big)\times_{\prod_{e\in E(\Gamma(S_\bullet))}(T_0\times T_0)}\Big(\prod_{v\in V(\Gamma(S_\bullet))}T_0\Big)\]
which is precisely the set of morphisms of directed graphs from $\Gamma(S_\bullet)$ to $\Gamma(T_\bullet)$.
\end{proof}
\begin{proposition}\label{simplicial set graph functor equivalence to dim 1}
Let $\mathbf{Set}^{\leq 1}_{\Delta}\sub\mathbf{Set}_\Delta$ denote the full subcategory spanned by the simplicial sets of dimension $\leq 1$. Then the functor $S_\bullet\mapsto\Gamma(S_\bullet)$ induces an equivalence of categories $\mathbf{Set}_\Delta^{\leq 1}\to\mathbf{Graph}$.
\end{proposition}
\begin{proof}
It follows from \cref{simplicial set graph functor on dim 1 faithful} that the functor $S_\bullet\to\Gamma(S_\bullet)$ is fully faithful when restricted to simplicial sets of dimension $\leq 1$. It will therefore suffice to show that it is essentially surjective. Let $G$ be any directed graph, and form a pushout diagram of simplicial sets
\[\begin{tikzcd}
\coprod_{e\in E(G)}\partial\Delta^1\ar[r]\ar[d]&\coprod_{e\in E(G)}\ar[d]\\
\coprod_{v\in V(G)}\Delta^0\ar[r]&S_\bullet
\end{tikzcd}\]
Then $S_\bullet$ is a simplicial set of dimension $\leq 1$, and the directed graph $\Gamma(S_\bullet)$ is isomorphic to $G$.
\end{proof}
\begin{remark}\label{simplicial set graph functor description}
The proof of \cref{simplicial set graph functor equivalence to dim 1} gives an explicit description of the inverse equivalence $\mathbf{Graph}\to\mathbf{Set}_\Delta$: it carries a directed graph $G$ to the $1$-dimensional simplicial set $G_\bullet$ given by the pushout
\[\Big(\coprod_{v\in V(G)}\Delta^0\Big)\coprod_{\coprod_{e\in E(G)}\partial\Delta^1}\Big(\coprod_{e\in E(G)}\Delta^1\Big)\]
\end{remark}
\begin{example}
Let $G$ be a directed graph and let $G_\bullet$ denote the associated simplicial set of dimension $\leq 1$ (by \cref{simplicial set graph functor equivalence to dim 1}). Then $G_\bullet$ has dimension $\leq 0$ if and only if the edge set $E(G)$ is empty. In this case, $G_\bullet$ can be identified with the constant simplicial set associated to the vertex set $V(G)$.
\end{example}
\subsection{Connected components of simplicial sets}
We now introduce the notion of a \textit{connected simplicial set}, and show that any simplicial set $S_\bullet$ can be decomposed into a disjoint union of connected simplicial subsets, indexed by a set $\pi_0(S_\bullet)$, which is called the \textit{set of connected components} of $S_\bullet$. Also, we will see that the functor $S_\bullet\to\pi_0(S_\bullet)$ is a left adjoint to the functor $X\mapsto\uline{X}_\bullet$.\par
Let $S_\bullet$ be a simplicial set and $S'_\bullet\sub S_\bullet$ be a simplicial subset of $S_\bullet$. We say that $S'_\bullet$ is a \textbf{summand} of $S_\bullet$ if $S_\bullet$ can be written into a coproduct $S'_\bullet\amalg S''_\bullet$ for some simplicial subset $S''_\bullet$ of $S_\bullet$. In this case, the complementray $S''_\bullet$ is uniquely determined: for each $n\geq 0$, we must have $S''_n=S_n\setminus S'_n$, so the condition that $S'_\bullet$ is a summand of $S_\bullet$ is equivalent to the condition that the map
\[([n]\in\Delta^{\op})\mapsto S_n\setminus S'_n\]
is functorial: that is, that the face and degeneracy operators for the simplicial set $S_\bullet$ preserve the subsets $S_n\setminus S'_n$.
\begin{remark}\label{simplicial set summand prop}
Let $S_\bullet$ be a simplicial set. Then it is clear that the collection of all summands of $S_\bullet$ is closed under the formation of unions and intersections. Moreover, if $S'_\bullet\sub S_\bullet$ is a summand of $S_\bullet$ and $S''_\bullet\sub S'_\bullet$ is a summand of $T_\bullet$, then $S''_\bullet$ is a summand of $S_\bullet$.\par
Finally, if $f:S_\bullet\to T_\bullet$ is a map of simplicial sets and $T'_\bullet\sub T_\bullet$ is a sumand, then the inverse image $f^{-1}(T_\bullet)\cong S_\bullet\times_{T_\bullet}T'_\bullet$ is a summand of $S_\bullet$.
\end{remark}
Let $S_\bullet$ be a simplicial set. We say that $S_\bullet$ is \textbf{connected} if it is nonempty and every summand of $S_\bullet$ is either empty or coincides with $S_\bullet$. We say a simplicial subset $S'_\bullet\sub S_\bullet$ is a connected component of $S_\bullet$ if $S'_\bullet$ is connected and is a summand of $S_\bullet$. We denote by $\pi_0(S_\bullet)$ the set of connected components of $S_\bullet$. It often will be convenient to view $I=\pi_0(S_\bullet)$ as an abstract index set which is equipped with a bijection
\[I\cong\{\text{connected components of $S_\bullet$}\},\quad I\ni i\mapsto S^{(i)}_\bullet\sub S_\bullet\]
rather than as the set of connected components itself.
\begin{example}\label{simplicial set constant pi_0 char}
Let $X$ be a set and $\uline{X}_\bullet$ be the constant simplicial set associated to $X$. Then the connected components of $\uline{X}_\bullet$ are exactly the simplicial subsets of the form $\{x\}=\uline{\{x\}}_\bullet$. In particular, we have a canonical bijection $X\cong\pi_0(\uline{X}_\bullet)$.
\end{example}
\begin{proposition}\label{simplicial set connected image is connected}
Let $f:S_\bullet\to T_\bullet$ be a map of simplicial sets, and suppose that $S_\bullet$ is connected. Then there exists a unique connected component $T'_\bullet\sub T_\bullet$ such that $f(S_\bullet)\sub T'_\bullet$.
\end{proposition}
\begin{proof}
Let $T'_\bullet$ be the smallest summand of $T_\bullet$ which contains the image of f (the existence of $T'_\bullet$ follows from \cref{simplicial set summand prop}: we can take $T'_\bullet$ to be the intersection of all those summands of $T_\bullet$ which contain the image of $f$). It then suffices to prove that $T'_\bullet$ is connected. Since $S_\bullet$ is nonempty, $T'_\bullet$ is also nonempty; now let $T''_\bullet\sub T'_\bullet$ be a summand. Note that $f^{-1}(T''_\bullet)$ is then a summand of $S_\bullet$, so $f^{-1}(T''_\bullet)$ is equal to $\emp$ or $S_\bullet$. By replacing $T''_\bullet$ with its complement in $T'_\bullet$ if necessary, we may assume that $f^{-1}(T''_\bullet)=S_\bullet$, so that $f$ factors through $T''_\bullet$. Since $T''_\bullet$ is a summand of $T_\bullet$, from the minimality of $T'_\bullet$ we must have $T''_\bullet=T_\bullet$.
\end{proof}
\begin{corollary}\label{simplicial set connected iff to constant}
Let $S_\bullet$ be a simplicial set. Then $S_\bullet$ is connected if and only if for every set $X$ the canonical map
\begin{align}\label{simplicial set connected iff to constant-1}
X\cong\Hom_{\mathbf{Set}_\Delta}(\Delta^0,\uline{X}_\bullet)\to\Hom_{\mathbf{Set}_\Delta}(S_\bullet,\uline{X}_\bullet).
\end{align}
is bijective.
\end{corollary}
\begin{proof}
One direction is clear from \cref{simplicial set connected image is connected} and \cref{simplicial set constant pi_0 char}. Conversely, suppose that (\ref{simplicial set connected iff to constant-1}) is bijective. Then by considering $X=\emp$, we conclude that there are no maps from $S_\bullet$ to the empty simplicial set, so that $S_\bullet$ is nonempty. If $S_\bullet$ is a disjoint union of simplicial subsets $S'_\bullet\amalg S'_\bullet$, then we obtain a map of simplicial sets
\[S_\bullet\cong S'_\bullet\amalg S''_\bullet\to\Delta^0\amalg\Delta^0\]
and the bijectivity of (\ref{simplicial set connected iff to constant-1}) guarantees that this map factors through one of the summands on the right hand side; it follows that either $S'_\bullet$ or $S''_\bullet$ is empty.
\end{proof}
\begin{proposition}\label{simplicial set disjoint union of connected component}
Let $S_\bullet$ be a simplicial set. Then $S_\bullet$ is a disjoint union of its connected components.
\end{proposition}
\begin{proof}
It is clear that different connected components of $S_\bullet$ are disjoint. Let $\sigma$ be an $n$-simplex of $S_\bullet$. We want to show that there exists a unique connected component of $S_\bullet$ which contains $\sigma$. This follows from \cref{simplicial set connected image is connected} applied to the map $\Delta^n\to S_\bullet$ classified by $\sigma$ (note that the standard $n$-simplex $\Delta^n$ is connected).
\end{proof}
\begin{corollary}\label{simplicial set connected iff pi_0 singleton}
Let $S_\bullet$ be a simplicial set. Then $S_\bullet$ is empty if and only if $\pi_0(S_\bullet)$ is empty, and is connected if and only if $\pi_0(S_\bullet)$ has exactly one element.
\end{corollary}
\begin{remark}
Let $S_\bullet$ be a simplicial set and $S'_\bullet\sub S_\bullet$ be a simplicial subset. If $S'_\bullet$ is a summand of $S_\bullet$, then it is clear from definition that any connected component $S''_\bullet$ of $S_\bullet$ intersecting $S'_\bullet$ must be contained in $S'_\bullet$ (otherwise $S'_\bullet\cap S''_\bullet$ is a nontrivial summand of $S''_\bullet$). Therefore $S'_\bullet$ is a disjoint union of connected components of $S'_\bullet$. Conversely, a disjoint union of connected components is clearly a summand of $S_\bullet$, so we obtain a canonical bijection between subsets of $\pi_0(S_\bullet)$ and summands of $S_\bullet$.
\end{remark}
\begin{remark}
Let $f:S_\bullet\to T_\bullet$ be a map of simplicial sets. It follows from \cref{simplicial set connected image is connected} that for each connected component $S'_\bullet\sub S_\bullet$, there is a unique connected component $T'_\bullet\sub T_\bullet$ such that $f(S'_\bullet)\sub T'_\bullet$. The map $S'_\bullet\mapsto T'_\bullet$ then determines a map of sets $\pi_0(f):\pi_0(S_\bullet)\to\pi_0(T_\bullet)$. This map is compatible
with composition, and therefore allows us to view $S_\bullet\to \pi_0(S_\bullet)$ as a functor $\pi_0:\mathbf{Set}_\Delta\to\mathbf{Set}$ from the category of simplicial sets to the category of sets.
\end{remark}
Let $S_\bullet$ be a simplicial set. For every $n$-simplex $\sigma$ of $S_\bullet$, \cref{simplicial set connected image is connected} implies that there is a unique connected component $S'_\bullet\sub S_\bullet$ which contains $\sigma$. The map $\sigma\mapsto S'_\bullet$ then determines a map of simplicial sets
\[u:S_\bullet\to\uline{\pi_0(S_\bullet)}_\bullet\]
where $\uline{\pi_0(S_\bullet)}_\bullet$ denotes the constant simplicial set associated to $\pi_0(S_\bullet)$. The map $u$ is called the \textbf{component map} of $S_\bullet$.
\begin{proposition}\label{simplicial set component map prop}
Let $S_\bullet$ be a simplicial set and $u:S_\bullet\to\uline{\pi_0(S_\bullet)}_\bullet$ be the component map of $S_\bullet$. Then for every set $X$, composition with $u$ induces a bijection
\[\Hom_{\mathbf{Set}}(\pi_0(S_\bullet),X)\to\Hom_{\mathbf{Set}_\Delta}(S_\bullet,\uline{X}_\bullet).\]
\end{proposition}
\begin{proof}
By decomposing $S_\bullet$ as the union of its connected components, we may assume that $S_\bullet$ is connected, and the desired result is then a reformulation of \cref{simplicial set connected iff to constant}.
\end{proof}
\begin{corollary}\label{simplicial set pi_0 is left adjoint to constant}
The connected component functor $\pi_0$ is left adjoint to the constant simplicial set functor $X\mapsto\uline{X}_\bullet$.
\end{corollary}
\begin{proposition}\label{simplicial set pi_0 is coequalizer of d_0 and d_1}
Let $S_\bullet$ be a simplicial set and $u:S_\bullet\to\uline{\pi_0(S_\bullet)}_\bullet$ be the component map. Then $u$ exhibits $\pi_0(S_\bullet)$ as the colimit of the diagram $\Delta^{\op}\to\mathbf{Set}$ given by $S_\bullet$. In fact, if $u_0:S_0\to \pi_0(S_\bullet)$ is the map given by the component map of $S_\bullet$, then $u_0$ exhibits $\pi_0(S_\bullet)$ as the coequalizer of the face maps $d_0,d_1:S_1\to S_0$:
\[\begin{tikzcd}
S_1\ar[r,shift left=2pt,"d_0"]\ar[r,shift right=2pt,swap,"d_1"]&S_0\ar[r,"u_0"]&\pi_0(S_\bullet).
\end{tikzcd}\]
\end{proposition}
\begin{proof}
The first assertion follows from the fact that $\Hom_{\mathbf{Set}_\Delta}(S_\bullet,\uline{X}_\bullet)$ is the set of maps from $S_n\to X$ satisfying compatibilities. In order to prove the second one, let $I$ be a set and $f:S_0\to X$ be a function satisfying $f\circ d_0=f\circ d_1$. Let $\sigma$ be an $n$-simpliex of $S_\bullet$, which we consider as a map $\sigma:\Delta^n\to S_\bullet$. For $0\leq i\leq n$, we may consider $\sigma(i)$ as a vertex of $S_\bullet$. Now if $0\leq i\leq j\leq n$, we then have $f(\sigma(i))=f(\sigma(j))$: we may assume that $i=0$ and $j=n=1$, and then this follows from the hypothesis $f\circ d_0=f\circ d_1$. Then there is a unique element $F(\sigma)\in I$ such that $F(\sigma)=f(\sigma(i))$ for each $0\leq i\leq n$. The map $\sigma\mapsto F(\sigma)$ then defines a map of simplicial sets $S_\bullet\to\uline{X}_\bullet$ which coincides with $f$ at degree zerom, and by \cref{simplicial set component map prop}, we conclude that $f$ factors through $u_0$.
\end{proof}
\begin{remark}\label{simplicial set pi_0 is quotient of d_0 and d_1 relation}
\cref{simplicial set pi_0 is coequalizer of d_0 and d_1} allows us to identify $\pi_0(S_\bullet)$ with the quotient of $S_0/\sim$, where $\sim$ is the equivalence relation generated by the set of edges of $S_\bullet$ (that is, the smallest equivalence relation with the property that $d_0(e)\sim d_1(e)$ for every edge $e\in S_1$). In particular, the set $\pi_0(S_\bullet)$ depends only on the $1$-skeleton of $S_\bullet$.
\end{remark}
\begin{remark}\label{simplicial set pi_0 is equivalent class of vertex}
Let $S_\bullet$ be a simplicial set. Then the set of connected components $\pi_0(S_\bullet)$ can also be described as the coequalizer of the pair of maps $d_0,d_1:S_1^{\circ}\to S_0$, where $S_1^{\circ}$ denotes the set of nondegenerate edges of $S_\bullet$ (since every degenerate edge $e\in S_1$ automatically satisfies $d_0(e)=d_1(e)$). We therefore have a coequalizer diagram of sets
\[\begin{tikzcd}
E(G)\ar[r,shift left=2pt,"d_0"]\ar[r,shift right=2pt,swap,"d_1"]&V(G)\ar[r,"u_0"]&\pi_0(S_\bullet).
\end{tikzcd}\]
where $G=\Gamma(S_\bullet)$ is the directed graph of $S_\bullet$. In other words, we can identify $\pi_0(S_\bullet)$ with the set of connected components of $G$, in the usual graph-theoretic sense. 
\end{remark}
\begin{proposition}\label{simplicial set connected closed under product}
The collection of connected simplicial sets is closed under finite products.
\end{proposition}
\begin{proof}
Since the final object $\Delta^0$ is connected, it suffices to show that the collection of connected simplicial sets is closed under pairwise products. Let $S_\bullet$ and $T_\bullet$ be connected simplicial sets; we wish to show that $S_\bullet\times T_\bullet$ is connected. Equivalently, we want to show that $\pi_0(S_\bullet\times T_\bullet)$ consists of a single element (\cref{simplicial set connected iff pi_0 singleton}). By \cref{simplicial set pi_0 is coequalizer of d_0 and d_1}, the component map gives a surjection
\[u_0:S_0\times T_0\twoheadrightarrow \pi_0(S_\bullet\times T_\bullet)\]
It will therefore suffice to show that every pair of vertices $(s,t),(s',t')\in S_0\times T_0$ belong to the same connected component of $S_\bullet\times T_\bullet$. Let $K_\bullet\sub S_\bullet\times T_\bullet$ be the connected component which contains the vertex $(s',t)$. Since $S_\bullet$ is connected, the map
\[S_\bullet\cong S_\bullet\times\{t\}\hookrightarrow S_\bullet\times T_\bullet\]
factors through a unique connected component of $S_\bullet\times T_\bullet$, which must be equal to $K_\bullet$. It follows that $K_\bullet$ contains the vertex $(s,t)$. A similar argument (with the roles of $S_\bullet$ and $T_\bullet$ reversed) also shows that $K_\bullet$ contains $(s',t')$.
\end{proof}
\begin{corollary}\label{simplicial set pi_0 preserves finite product}
The functor $\pi_0:\mathbf{Set}_\Delta\to\mathbf{Set}$ preserves finite products.
\end{corollary}
\begin{proof}
Since $\pi_0(\Delta^0)$ is a singleton, it suffices to show that for every pair of simplicial sets $S_\bullet$ and $T_\bullet$, the canonical map
\[\pi_0(S_\bullet\times T_\bullet)\to\pi_0(S_\bullet)\times \pi_0(T_\bullet)\]
is bijective. Writing $S_\bullet$ and $T_\bullet$ as a disjoint union of connected components (\cref{simplicial set disjoint union of connected component}), we can reduce to the case where $S_\bullet$ and $T_\bullet$ are connected, in which case the desired result follows from \cref{simplicial set connected closed under product}.
\end{proof}
\begin{example}
The collection of connected simplicial sets is not closed under infinite products (so the functor $\pi_0:\mathbf{Set}_\Delta\to\mathbf{Set}$ does not commute with infinite products). For example, let $G$ be the directed graph with vertex set $V(G)=E(G)=\N$, with source and target maps
\[s,t:E(G)\to V(G),\quad s(n)=n,t(n)=n+1.\]
More informally, $G$ is the directed graph depicted in the diagram
\[\begin{tikzcd}
0\ar[r]&1\ar[r]&2\ar[r]&3\ar[r]&\cdots
\end{tikzcd}\]
The associated $1$-dimensional simplicial set $G_\bullet$ is connected. However, the infinite product $S_\bullet=\prod_{n\in\N}G_\bullet$ is not connected. By definition, the vertices of $S_\bullet$ can be identified with functions $f:\N\to\N$. It is not difficult to see that two such functions $f,g$ belong to the same connected component of $S_\bullet$ if and only if the function $|f-g|$ is bounded. In particular, the identity function $1$ and the zero function $0$ do not belong to the same connected component of $S_\bullet$.
\end{example}
\subsection{Singular simplicial sets and geometric realizations}
Let $X$ be a topological space. We define a simplicial set $\Sing_\bullet(X)$ as follows:
\begin{itemize}
\item To each object $[n]\in\Delta^{\op}$, we let $\Sing_n(X)=\Hom_{\mathbf{Top}}(|\Delta^n|,X)$ be the set of singular $n$-simplicies of $X$.
\item To each non-decreasing map $\alpha:[m]\to[n]$, we let $\Sing_\bullet(\alpha):\Sing_n(X)\to\Sing_m(X)$ be the map given by composing with the continuous map
\[|\Delta^m|\to|\Delta^n|,\quad (t_0,\dots,t_m)\mapsto\Big(\sum_{\alpha(i)=0}t_i,\sum_{\alpha(i)=1}t_i,\dots,\sum_{\alpha(i)=n}t_i\Big).\]
\end{itemize}
In particular, by definition the vertices of $\Sing_\bullet(X)$ can be identified with points of $X$, and the edges of $\Sing_\bullet(X)$ can be identified with continuous paths $p:[0,1]\to X$. The simplicial set $\Sing_\bullet(X)$ is called the \textbf{singular simplicial} set of $X$, and we consider $X\mapsto\Sing_\bullet(X)$ as a functor $\Sing_\bullet:\mathbf{Top}\to\mathbf{Set}_\Delta$ from the category of topological spaces to the category of simplicial sets.
\begin{remark}
The functor $X\mapsto\Sing_\bullet(X)$ carries limits in the category of topological spaces to limits in the category of simplicial sets (in fact, the functor $\Sing_\bullet$ admits a left adjoint). It does not preserve colimits in general. However, it does carry coproducts of topological spaces to coproducts of simplicial sets: this follows from the observation that the topological $n$-simplex $|\Delta^n|$ is connected for every $n\geq 0$.
\end{remark}
Let $X$ be a topological space. We denote by $\pi_0(X)$ the set of path components of $X$: that is, the quotient of $X$ by the equivalence relation
\[(x\sim y)\Leftrightarrow(\exists p:[0,1]\to X)[p(0)=x,p(1)=y].\]
Then it follows from \cref{simplicial set pi_0 is equivalent class of vertex} that we have a canonical bijection $\pi_0(\Sing_\bullet(X))\cong\pi_0(X)$. That is, we can identify connected components of the simplicial set $\Sing_\bullet(X)$ with path components of the topological space $X$. In particular, the simplicial set $\Sing_\bullet(X)$ is connected if and only if $X$ is path connected. Since there exist topological spaces $X$ which are connected but not path connected, we see $X$ can be connnected without the singular simplicial set $\Sing_\bullet(X)$ being connected.\par
It will be convenient to consider a generalization $\Sing_\bullet$. Let $\mathcal{C}$ be any category and let $Q^\bullet$ be a cosimplicial object of $\mathcal{C}$, which we view as a covariant functor $Q:\Delta\to\mathcal{C}$. For every object $X\in\mathcal{C}$, the construction $\Delta\ni[n]\mapsto\Hom_{\mathcal{C}}(Q([n]),X)$ determines a functor from $\Delta^{\op}$ to $\mathbf{Set}$, which we can view as a simplicial set. We will denote this simplicial set by $\Sing_\bullet^Q(X)$, so that we have canonical bijections $\Sing_n^Q(X)\cong\Hom_{\mathcal{C}}(Q^n,X)$. The map $X\mapsto\Sing_\bullet^Q(X)$ then gives a functor from the category $\mathcal{C}$ to the category of simplicial sets, which we denote by $\Sing_\bullet^Q:\mathcal{C}\to\mathbf{Set}_\Delta$.\par
\begin{example}
The construction $[n]\mapsto|\Delta^n|$ determines a covariant functor from the simplex category $\Delta$ to the category $\mathbf{Top}$ of topological spaces, which assigns to each morphism $\alpha:[m]\to[n]$ the continuous map
\[|\Delta^m|\to|\Delta^n|,\quad (t_0,\dots,t_m)\mapsto\Big(\sum_{\alpha(i)=0}t_i,\sum_{\alpha(i)=1}t_i,\dots,\sum_{\alpha(i)=n}t_i\Big).\]
We then obtain a cosimplicial topological space, which we denote by $|\Delta^\bullet|$. The corresponding functor $\Sing_\bullet^{|\Delta|}:\mathbf{Top}\to\mathbf{Set}_\Delta$ is just the singular simplicial set functor $\Sing_\bullet$.
\end{example}
\begin{example}
The map $[n]\mapsto\Delta^n$ determines a functor from the simplex category $\Delta$ to the category $\mathbf{Set}_\Delta=\Fun(\Delta^{\op},\mathbf{Set})$ of simplicial sets (this is the Yoneda embedding for the simplex category $\Delta$). We can consider this functor as a cosimplicial object in $\mathbf{Set}_\Delta$, which we denote by $\Delta^\bullet$. Then we obtain a functor $\Sing_\bullet^{\Delta}$ from the category of simplicial sets to itself, which is canonically isomorphic to the identity functor $1_{\mathbf{Set}_\Delta}$, in view of the Yoneda's lemma.
\end{example}
Let $X$  be a topological space. By definition, the $n$-simplices of the simplicial set $\Sing_\bullet(X)$ are continuous maps $|\Delta^n|\to X$, so we have a bijection
\[\Hom_{\mathbf{Top}}(|\Delta^n|,X)\cong\Hom_{\mathbf{Set}_\Delta}(\Delta^n,\Sing_\bullet(X)).\]
We now consider a generalization of this construction, which can be applied to simplicial sets other than $\Delta^n$. Let $S_\bullet$ be a simplicial set and let $Y$ be a topological space. We say that a map of simplicial sets $u:S_\bullet\to\Sing_\bullet(Y)$ exhibits $Y$ as a \textbf{geometric realization} of $S_\bullet$ if, for every topological space $X$, the composite map
\[\Hom_{\mathbf{Top}}(Y,X)\to\Hom_{\mathbf{Set}_\Delta}(\Sing_\bullet(Y),\Sing_\bullet(X))\stackrel{\circ u}{\to}\Hom_{\mathbf{Set}_\Delta}(S_\bullet,\Sing_\bullet(X))\]
is bijective. For example, for each $n\geq 0$ the identity map $\id:|\Delta^n|\to|\Delta^n|$ determines an $n$-simplex of the simplical set $\Sing_\bullet(|\Delta^n|)$, which can be identified as a map of simplicial sets $\Delta^n\to\Sing_\bullet(|\Delta^n|)$ exhibiting $|\Delta^n|$ as a geometric realization of $\Delta^n$.\par
Let $S_\bullet$ be a simplicial set. It follows immediately from the definitions that if there exists a map $u:S_\bullet\to\Sing_\bullet(Y)$ which exhibits $Y$ as a geometric realization of $S_\bullet$, then the topological space $Y$ is determined up to homeomorphism and depends functorially on $S_\bullet$. We will emphasize this dependence by writing $|S_\bullet|$ to denote a geometric realization of $S_\bullet$. Later we will see that every simplicial set $S_\bullet$ admits a geometric realization.
\begin{lemma}\label{simplicial set limit geometric realization}
Let $\mathcal{J}$ be a small category equipped with a functor $F:\mathcal{J}\to\mathbf{Set}_\Delta$ given by $J\mapsto F(J)_\bullet$ and $S_\bullet=\rlim_{j\in\mathcal{J}} F(J)_\bullet$ be the colimit of $F$. If each of the simplicial sets $F(J)_\bullet$ admits a geometric realization $|F(J)_\bullet|$, then $S_\bullet$ admits a realization, given by the colimit $Y=\rlim_{J\in\mathcal{J}}|F(J)_\bullet|$.
\end{lemma}
\begin{proof}
For each $J\in\mathcal{J}$, let $u_J:F(J)_\bullet\to\Sing_\bullet(|F(J)|_\bullet)$ be a map which exhibits $|F(J)_\bullet|$ as a geometric realization of $F(J)_\bullet$. We can then amalgamate the composite maps
\[\begin{tikzcd}
F(J)_\bullet\ar[r,"u_J"]&\Sing_\bullet(|F(J)|_\bullet)\ar[r]&\Sing_\bullet(Y)
\end{tikzcd}\]
to a single map of simplicial sets $u:S_\bullet\to\Sing_\bullet(Y)$. Let $X$ be any topological space; we wish to show that the composite map
\[\Hom_{\mathbf{Top}}(Y,X)\to \Hom_{\mathbf{Set}_\Delta}(\Sing_\bullet(Y),\Sing_\bullet(X))\stackrel{\circ u}{\to}\Hom_{\mathbf{Set}_\Delta}(S_\bullet,\Sing_\bullet(X))\]
is bijective. This is clear, since this composite map can be written as an inverse limit of the bijections $\Hom_{\mathbf{Top}}(|F(J)_\bullet|,X)\cong\Hom_{\mathbf{Set}_\Delta}(F(J)_\bullet,\Sing_\bullet(X))$ determined by the $u_J$.
\end{proof}
It is now possible to deduce the existence of geometric realizations in a completely formal way from \cref{simplicial set limit geometric realization}, since every simplicial set can be presented as a colimit of simplices. However, we will instead give a less direct argument which yields some additional information about the structure of the topological spaces $|S_\bullet|$. We begin by studying simplicial subsets of the standard simplex $\Delta^n$.\par
Let $n\geq 0$ be an integer and let $\mathcal{U}$ be a collection of nonempty subsets of $[n]$. We will say that $\mathcal{U}$ is downward closed if $\emp\neq I\sub J\in\mathcal{U}$ implies that $I\in\mathcal{U}$. If this condition is satisfied, we denote by $\Delta^n_{\mathcal{U}}$ denote the simplicial subset of $\Delta^n$ whose $m$-simplices are nondecreasing maps $\alpha:[m]\to[n]$ for which the image of $\alpha$ is an element of $\mathcal{U}$. Similarly, we set
\[|\Delta^n|_{\mathcal{U}}=\{(t_0,\dots,t_n)\in|\Delta^n|:\{i\in[n]:t_i\neq 0\}\in\mathcal{U}\}.\]
\begin{example}\label{simplicial set boundary horn given downward set}
For each $n\geq 0$, the boundary $\partial\Delta^n$ is given by $\Delta^n_\mathcal{U}$ where $\mathcal{U}$ is the collection of all nonempty proper subsets of $[n]$. Similarly, for each $0\leq i\leq n$ the horn $\Lambda^n_i$ is given by $\Delta^n_\mathcal{U}$ where $\mathcal{U}$ is the (uniquely determined) downward closed collection of nonempty subsets $I$ of $[n]$ such that $I\cup\{i\}\neq[n]$. In fact, if $S_\bullet\sub\Delta^n$, then there exists a unique downward closed collection $\mathcal{U}$ of nonempty subsets of $[n]$ such that $S_\bullet=\Delta^n_\mathcal{U}$: just take $\mathcal{U}$ to be the image of all simplicies $\sigma:\Delta^k\to S_\bullet$.
\end{example}
\begin{proposition}\label{simplicial set Delta^n_U geometric realization}
Let $n$ be a nonnegative integer and let $\mathcal{U}$ be a downward closed collection of nonempty subsets of $[n]$. Then the canonical map $\Delta^n\to\Sing_\bullet(|\Delta^n|)$ restricts to a map of simplicial sets $f_\mathcal{U}:\Sing_\bullet(|\Delta^n|_{\mathcal{U}})$, which exhibits the topological space $|\Delta^n|_{\mathcal{U}}$ as a geometric realization of $\Delta^n_\mathcal{U}$.
\end{proposition}
\begin{proof}
We proceed by induction on the cardinality of $\mathcal{U}$. If $\mathcal{U}$ is empty, then the simplicial set $\Delta^n_\mathcal{U}$ and the topological space $|\Delta^n|_\mathcal{U}$ are both empty, in which case there is nothing to prove. We may therefore assume that $\mathcal{U}$ is nonempty. Choose some $I\in\mathcal{U}$ whose cardinality is as large as possible, and set
\[\mathcal{U}_0=\mathcal{U}\setminus\{I\},\quad \mathcal{U}_1=\{J\sub I:J\neq\emp\},\quad \mathcal{U}_{0,1}=\mathcal{U}_{0}\cap\mathcal{U}_1.\]
Our inductive hypothesis implies that the maps $f_{\mathcal{U}_0}$ and $f_{\mathcal{U}_{01}}$ exhibit $|\Delta^n|_{\mathcal{U}_0}$ and $|\Delta^n|_{\mathcal{U}_{01}}$ as geometric realizations of $\Delta^n_{\mathcal{U}_0}$ and $\Delta^n_{\mathcal{U}_{01}}$, respectively. Moreover, if $I=\{i_0<i_1<\cdots<i_k\}\sub[n]$, then we can identify $f_{\mathcal{U}_1}$ with the tautological map $\Delta^k\to\Sing_\bullet(|\Delta^k|)$, so that $f_{\mathcal{U}_1}$ exhibits $|\Delta^n|_{\mathcal{U}_1}$ as a geometric realization of $\Delta^n_{\mathcal{U}_1}$. Now, it follows immediately from the definitions that the diagram of simplicial sets
\[\begin{tikzcd}
\Delta^n_{\mathcal{U}_{01}}\ar[r]\ar[d]&\Delta^n_{\mathcal{U}_0}\ar[d]\\
\Delta^n_{\mathcal{U}_1}\ar[r]&\Delta^n_{\mathcal{U}}
\end{tikzcd}\]
is a pushout square, so by \cref{simplicial set limit geometric realization} we are reduced to proving that the diagram of topological spaces
\[\begin{tikzcd}
{|\Delta^n|}_{\mathcal{U}_{01}}\ar[r]\ar[d]&{|\Delta^n|}_{\mathcal{U}_0}\ar[d]\\
{|\Delta^n|}_{\mathcal{U}_1}\ar[r]&{|\Delta^n|}_{\mathcal{U}}
\end{tikzcd}\]
is also a pushout square. This is clear, since $|\Delta^n|_{\mathcal{U}_0}$ and $|\Delta^n|_{\mathcal{U}_1}$ are closed subsets of $|\Delta^n|$ whose union is $|\Delta^n|_{\mathcal{U}}$ and whose intersection is $|\Delta^n|_{\mathcal{U}_{01}}$.
\end{proof}
\begin{example}\label{simplicial set boundary horn geometric realization}
Let $n$ be a nonnegative integer. Combining \cref{simplicial set boundary horn given downward set} with \cref{simplicial set Delta^n_U geometric realization}, we see that the inclusion map $\partial\Delta^n\hookrightarrow\Delta^n$ induces a homeomorphism from $|\partial\Delta^n|$ to the boundary of the topological $n$-simplex $|\Delta^n|$, given by
\[\{(t_0,\dots,t_n)\in|\Delta^n|:\text{$t_i=0$ for some $i$}\}.\]
Similarly, for $0\leq i\leq n$, the inclusion map $\Lambda^n_i\hookrightarrow\Delta^n$ induces a homeomorphism from $|\Lambda^n_i|$ to the subset of $|\Delta^n|$ given by
\[\{(t_0,\dots,t_n)\in\Delta^n:\text{$t_j=0$ for some $j\neq i$}\}.\]
\end{example}
\begin{proposition}\label{simplicial set geometric realization exist}
For every simplicial set $S_\bullet$, there exists a topological space $Y$ and a map $u:S_\bullet\to\Sing_\bullet(Y)$ which exhibits $Y$ as a geometric realization of $S_\bullet$.
\end{proposition}
\begin{proof}
Let $S_\bullet$ be a simplicial set. We first show that for each $k\geq -1$, the skeleton $\sk_k(S_\bullet)$ admits a geometric realization. The proof proceeds by induction on $k$, the case $k=-1$ being trivial (since $\sk_{-1}(S_\bullet)$ is empty). Let $S_k^{\circ}$ denote the collection of nondegenerate $n$-simplices of $S_\bullet$. we note that \cref{simplicial set skeleton pushout square} provides a pushout diagram
\[\begin{tikzcd}
\coprod_{\sigma\in S_k^{\circ}}\partial\Delta^k\ar[d]\ar[r]&\coprod_{\sigma\in S^\circ_k}\Delta^k\ar[d]\\
\sk_{k-1}(S_\bullet)\ar[r]&\sk_k(S_\bullet)
\end{tikzcd}\]
Combining our inductive hypothesis, \cref{simplicial set boundary horn geometric realization} and \cref{simplicial set limit geometric realization}, we see that $\sk_k(S_\bullet)$ admits a geometric realization $|\sk_k(S_\bullet)|$ which fits into a pushout diagram of topological spaces
\[\begin{tikzcd}
\coprod_{\sigma\in S_k^{\circ}}{|\partial\Delta^k|}\ar[r]\ar[d]&\coprod_{\sigma\in S^\circ_k}{|\Delta^k|}\ar[d]\\
{|\sk_{k-1}(S_\bullet)|}\ar[r]&{|\sk_k(S_\bullet)|}
\end{tikzcd}\]
Since $S_\bullet=\bigcup_k\sk_k(S_\bullet)$, the simplicial set $S_\bullet$ also admits a geometric realization, given by the direct limit $\rlim_k|\sk_k(S_\bullet)|$.
\end{proof}
\begin{remark}
The proof of \cref{simplicial set geometric realization exist} shows that the geometric realization $|S_\bullet|$ of a simplicial set $S_\bullet$ has a canonical realization as a CW complex, having one cell of dimension $n$ for each nondegenerate $n$-simplex $\sigma$ of $S_\bullet$; this cell can be described explicitly as the image of the map
\[|\Delta^n|\setminus|\partial\Delta^n|\hookrightarrow|\Delta^n|\stackrel{\sigma}{\to}|S_\bullet|.\]
\end{remark}
\begin{proposition}\label{simplicial set full subcategory nonproper if}
Let $\mathcal{U}$ be a full subcategory of the category $\mathbf{Set}_\Delta$ of simplicial sets. Suppose that $\mathcal{U}$ satisfies the following three conditions:
\begin{itemize}
\item[(a)] Suppose we are given a pushout diagram of simplicial sets
\[\begin{tikzcd}
X_\bullet\ar[r,"f"]\ar[d]&Y_\bullet\ar[d]\\
X'_\bullet\ar[r]&Y'_\bullet
\end{tikzcd}\]
where $f$ is a monomorphism. If $X_\bullet$, $Y_\bullet$, and $X'_\bullet$ belong to $\mathcal{U}$, then $Y_\bullet$ belongs to $\mathcal{U}$.
\item[(b)] Suppose we are given a sequence of monomorphisms of simplicial sets
\[\begin{tikzcd}
X^{(0)}_\bullet\ar[r,hook]&X^{(1)}_\bullet\ar[r,hook]&X^{(2)}_\bullet\ar[r]&\cdots
\end{tikzcd}\]
If each $X^{(m)}_\bullet$ belongs to $\mathcal{U}$, then the sequentual colimit $\rlim_mX{(m)}_\bullet$ belongs to $\mathcal{U}$.
\item[(c)] For each $n\geq 0$ and every set $I$, the coproduct $\coprod_{i\in I}\Delta^n$ belongs to $\mathcal{U}$.
\end{itemize}
Then every simplicial set belongs to the category $\mathcal{U}$.
\end{proposition}
\begin{proof}
Set $S_\bullet$ be a simplicial set; we want to show that $S_\bullet$ belongs to $\mathcal{U}$. Since $S_\bullet=\bigcup_k\sk_k(S_\bullet)$, we can identify $S_\bullet$ with the colimit $\rlim_k\sk_k(S_\bullet)$. By condition (b), it will suffice to show that each skeleton $\sk_k(S_\bullet)$ belongs to $\mathcal{U}$. We may therefore assume without loss of generality that $S_\bullet$ has dimension $\leq k$, for some integer $n$, and we proceed by induction on $k$. In the case $k=-1$, the simplicial set $S_\bullet$ is empty, and the desired result is a special case of (c). To carry out the inductive step, we invoke \cref{simplicial set skeleton pushout square} to choose a pushout diagram
\[\begin{tikzcd}
\coprod_{\sigma\in S_k^{\circ}}\partial\Delta^k\ar[d]\ar[r]&\coprod_{\sigma\in S^\circ_k}\Delta^k\ar[d]\\
\sk_{k-1}(S_\bullet)\ar[r]&\sk_k(S_\bullet)
\end{tikzcd}\]
By condition (a), it will suffice to show that the simplicial sets $\sk_{n-1}(S_\bullet)$, $\coprod_{\sigma\in S_k^{\circ}}\partial\Delta^k$, and $\coprod_{\sigma\in S_k^{\circ}}\Delta^k$ belong to $\mathcal{U}$. In the first two cases, this follows from our inductive hypothesis, and the third follows from assumption (c).
\end{proof}
\begin{corollary}\label{simplicial set subcategory small colimits and Delta^n nonproper}
Let $\mathcal{U}$ be a full subcategory of the category $\mathbf{Set}_\Delta$ of simplicial sets. If $\mathcal{U}$ is closed under small colimits and contains the standard $n$-simplex $\Delta^n$ for each $n\leq 0$, then $\mathcal{U}=\mathbf{Set}_\Delta$.
\end{corollary}
\begin{proof}
If $\mathcal{U}$ is closed under small colimits and contains $\Delta^n$, then it satisfies conditions of \cref{simplicial set full subcategory nonproper if}.
\end{proof}
\begin{remark}
We can state \cref{simplicial set subcategory small colimits and Delta^n nonproper} more informally as follows: the category $\mathbf{Set}_\Delta$ of simplicial sets is generated, under small colimits, by objects of the form $\Delta^n$. In fact, one can say more: it is freely generated (under small colimits) by the essential image of the Yoneda embedding
\[\Delta\hookrightarrow\mathbf{Set}_\Delta,\quad [n]\mapsto\Delta^n.\]
In fact, this is a general fact about presheaf categories.
\end{remark}
We now sketch another proof of \cref{simplicial set subcategory small colimits and Delta^n nonproper}, which illustrates some ideas which will be useful later. Let $S_\bullet$ be a simplicial set, we define a category $\Delta_S$ as follows:
\begin{itemize}
\item The objects of $\Delta_S$ are pairs $([n],\sigma)$, where $[n]$ is an object of $\Delta$ and $\sigma$ is an $n$-simplex.
\item A morphism from $([n],\sigma)$ to $([m],\tau)$ in the category $\Delta_S$ is a nondecreasing function $f:[n]\to[m]$ with the property that the induced map $S_m\to S_n$ sends $\tau$ to $\sigma$.
\end{itemize}
The category $\Delta_S$ is called the \textbf{category of simplicies of \boldmath$S_\bullet$}. Via the Yoneda embedding $\Delta\hookrightarrow\mathbf{Set}_\Delta$, we can identify $\Delta_S$ with the category whose objects are simplicial sets of the form $\Delta^n$ (for some $n\neq 0$), which are equipped with a map of simplicial sets $\Delta^n\to S_\bullet$. In particular, we have a canonical map of simplicial sets $\rlim_{([n],\sigma)}\Delta^n\to S_\bullet$. To prove \cref{simplicial set subcategory small colimits and Delta^n nonproper}, it suffices to observe that this map is an isomorphism.\par
Now as a generalization of \cref{simplicial set geometric realization exist}, we prove the following theorem about the adjointness of $\Sing_\bullet$ functors.
\begin{proposition}\label{simplicial set Sing^Q functor adjoint}
Let $\mathcal{C}$ be a category, let $Q^\bullet$ be a cosimplicial object of $\mathcal{C}$, and let $\Sing_\bullet^Q:\mathcal{C}\to\mathbf{Set}_\Delta$ be the corresponding functor. If the category $\mathcal{C}$ admits small colimits, then the functor $\Sing_\bullet^Q$ admits a left adjoint $\mathbf{Set}_\Delta\to\mathcal{C}$, which is denoted by $S_\bullet\to|S_\bullet|^Q$.
\end{proposition}
\begin{proof}
We say a simplicial set $S_\bullet$ is good if the functor
\[\mathcal{C}\ni C\mapsto\Hom_{\mathbf{Set}_\Delta}(S_\bullet,\Sing_\bullet^Q(C))\]
is corepresentable by an object of the category $C$ (in which case we denote the corepresenting object by $|S_\bullet|^Q$). It follows from Yoneda's lemma that the standard $n$-simplex $\Delta^n$ is good for each $n\geq 0$, with $|\Delta^n|^Q=Q^n$. If $\mathcal{C}$ admits small colimits, then the proof of \cref{simplicial set limit geometric realization} shows that the collection of good simplicial sets is closed under small colimits. It now suffices to observe that every simplicial set $S_\bullet$ can be written as a small colimit of simplices (\cref{simplicial set subcategory small colimits and Delta^n nonproper}).
\end{proof}
\begin{example}\label{simplicial set pi_0 as geometric realization}
The functor $\pi_0:\mathbf{Set}_\Delta\to\mathbf{Set}$ can be regarded as special case of \cref{simplicial set Sing^Q functor adjoint}. In fact, if $Q^\bullet\to\mathbf{Set}$ is the constant functor whose value is a singleton set $\ast\in\mathbf{Set}_\Delta$, then $\Sing_\bullet^Q$ coincides with the constant simplicial set functor $X\mapsto\uline{X}_\bullet$, so the functor $|-|^Q$ agrees with $\pi_0$ in view of \cref{simplicial set component map prop}.
\end{example}
\begin{proposition}\label{simplicial set geometric realization connected iff}
Let $S_\bullet$ be a simplicial set. The following conditions are equivalent:
\begin{itemize}
\item[(\rmnum{1})] The geometric realization $|S_\bullet|$ is a path-connected topological space.
\item[(\rmnum{2})] The geometric realization $|S_\bullet|$ is a connected topological space.
\item[(\rmnum{3})] The simplicial set $S_\bullet$ is connected.
\end{itemize}
\end{proposition}
\begin{proof}
The implication (\rmnum{1})$\Rightarrow$(\rmnum{2}) holds for any topological space. To prove that (\rmnum{2})$\Rightarrow$(\rmnum{3}), we observe that any decomposition $S_\bullet\cong S'_\bullet\amalg S''_\bullet$ into disjoint nonempty simplicial subsets determines a homeomorphism $|S_\bullet|\cong|S'_\bullet|\amalg|S''_\bullet|$. We now complete the proof by showing that (\rmnum{3})$\Rightarrow$(\rmnum{1}). Note that we have a commutative diagram of sets
\[\begin{tikzcd}
\rlim_{\sigma:\Delta^n\to S_\bullet}{|\Delta^n|}\ar[r,"\sim"]\ar[d]&{|S_\bullet|}\ar[d]\\
\rlim_{\sigma:\Delta^n\to S_\bullet}\pi_0(|\Delta^n|)\ar[r]&\pi_0(|S_\bullet|)
\end{tikzcd}\]
where the upper horizontal map is bijective and the right vertical map is surjective. It follows that the lower horizontal map is also surjective. Since each of the topological spaces $|\Delta^n|$ is path connected, the colimit in the lower left can be identified with the set $\pi_0(S_\bullet)$ (\cref{simplicial set pi_0 as geometric realization}). If $S_\bullet$ is connected, the set $\pi_0(S_\bullet)$ consists of a single element, so that $\pi_0(|S_\bullet|)$ is also a singleton.
\end{proof}
\begin{remark}
Recall that we have remarked that there exist topological spaces $X$ which are connected but not path connected, so $X$ can be connnected without the singular simplicial set $\Sing_\bullet(X)$ being connected. However, \cref{simplicial set geometric realization connected iff} shows that the geometric realizations $|S_\bullet|$ rule out such pathological examples. This also suggests that the functor $S_\bullet\mapsto|S_\bullet|$ from $\mathbf{Set}_\Delta$ to $\mathbf{Top}$ is not essentially surjective, since connectedness and path-connectedness are topological properties.
\end{remark}
\begin{corollary}\label{simplicial set pi_0 and geometric realization}
For every simplicial set $S_\bullet$, we have a canonical bijection $\pi_0(S_\bullet)\cong\pi_0(|S_\bullet|)$.
\end{corollary}
\begin{proof}
Writing $S_\bullet$ as a disjoint union of connected components (\cref{simplicial set disjoint union of connected component}), we can reduce to the case where $S_\bullet$ is connected, in which case both sets have a single element.
\end{proof}
\subsection{Kan complexes}
Let $S_\bullet$ be a simplicial set, we say that $S_\bullet$ is a \textbf{Kan complex} if it satisfies the following condition:
\begin{enumerate}[leftmargin=40pt]
\item[(Kan)] For any positive integer $n>0$ and $0\leq i\leq n$, any map of simplicial sets $\sigma_0:\Lambda^n_i\to S_\bullet$ can be extended into a map $\sigma:\Delta^n\to S_\bullet$, where $\Lambda^n_i$ is the $i$-th horn.
\end{enumerate}
As we have remarked on the begining of this chapter, such properties hold for simplicial sets of the form $\Sing_\bullet(X)$: in other words, $\Sing_\bullet$ sends $\mathbf{Top}$ to the full subcategory of Kan complexes of $\mathbf{Set}_\Delta$.
\begin{example}\label{simplicial set dim 1 not Kan}
Let $S_\bullet$ be a simplicial set of dimension exactly $1$ (that is, a simplicial set $S_\bullet$ which arises from a directed graph with at least one edge), then $S_\bullet$ is not a Kan complex. In fact, let $e\in S_1$ be an edge of $S_\bullet$, $d_0(e)=x$, $d_1(e)=y$, and consider the map $\tau_0:\Lambda^2_0\to S_\bullet$ given by $(e,s_0(x),\ast)$, which we describe as a diagram
\[\begin{tikzcd}
&y\ar[rd,"e"]&\\
x\ar[ru,dashed]\ar[rr,"s_0(x)"]&&x
\end{tikzcd}\]
If $\tau:\Delta^2\to S_\bullet$ is an extension of $\tau_0$, then we claim that $\tau$ is nondegenerate, which contradicts the assumption that $\dim(S_\bullet)=1$. In fact, since $e$ is nondegenerate, any factorization of $\tau$ is of the form
\[\begin{tikzcd}
\Delta^2\ar[r]&\Delta^1\ar[r,"\sigma"]&S_\bullet
\end{tikzcd}\]
where the map $\Delta^2\to\Delta^1$ is given by corresponds to a surjection $\alpha:[2]\to[1]$. If $e'=d_2(\tau)$ is another edge of $\tau$, then we have
\begin{gather*}
x=d_0(e)=d_0(d_0(\tau))=\sigma(\alpha(2)),\quad y=d_1(e)=d_1(d_0(\tau))=\sigma(\alpha(1)),\\
y=d_0(e')=d_0(d_2(\tau))=\sigma(\alpha(1)),\quad x=d_1(e')=d_1(d_2(\tau))=\sigma(\alpha(0)).
\end{gather*}
But this contradicts to the fact that $\alpha$ is a nondecreasing map.
\end{example}
\begin{example}[\textbf{Products of Kan Complexes}]\label{simplicial set product of Kan}
Let $\{S_\bullet^\alpha\}_{\alpha\in I}$ be a collection of simplicial sets parametrized by a set $I$, and $S_\bullet=\prod_{\alpha}S_\bullet^\alpha$ be their product. If each $S_\bullet^\alpha$ is a Kan complex, then $S_\bullet$ is a Kan complex. The converse holds provided that each $S_\bullet^\alpha$ is nonempty.
\end{example}
\begin{example}[\textbf{Coproducts of Kan Complexes}]\label{simplicial set coproduct of Kan}
Let $\{S_\bullet^\alpha\}_{\alpha\in I}$ be a collection of simplicial sets parametrized by a set $I$, and $S_\bullet=\coprod_{\alpha}S_\bullet^\alpha$ be their coproduct. For each $0\leq i\leq n$, the restriction map
\[\theta:\Hom_{\mathbf{Set}_\Delta}(\Delta^n,S_\bullet)\to\Hom_{\mathbf{Set}_\Delta}(\Lambda_i^n,S_\bullet)\]
can be identified with the coproduct (formed in the arrow category $\Fun([1],\mathbf{Set}_\Delta))$ of restriction maps $\theta_\alpha:\Hom_{\mathbf{Set}_\Delta}(\Delta^n,S_\bullet^\alpha)\to\Hom_{\mathbf{Set}_\Delta}(\Lambda^n_i,S_\bullet^\alpha)$ (this follows from the observation that
the simplicial sets $\Delta^n$ and $\Lambda_i^n$ are connected). It follows that $\theta$ is surjective if and only if each $\theta_\alpha$ is surjective. Allowing $n$ and $i$ to vary, we conclude that $S_\bullet$ is a Kan complex if and only if each summand $S_\bullet^\alpha$ is a Kan complex. In particular, if $S_\bullet$ is a simplicial set, then $S_\bullet$ is a Kan complex if and only if each connected component of $S_\bullet$ is a Kan complex.
\end{example}
\begin{example}\label{simplicial set constant is Kan}
Let $S$ be a set and let $\uline{S}_\bullet$ denote the associated constant simplicial set. Then $\uline{S}_\bullet$ is a Kan complex: this follows from the fact that each connected component of $\uline{S}_\bullet$ is isomorphic to $\Delta^0$ and each horn $\Lambda^n_i$ is connected.
\end{example}
\begin{proposition}\label{simplicial set singular is Kan}
Let $X$ be a topological space. Then the singular simplicial set $\Sing_\bullet(X)$ is a Kan complex.
\end{proposition}
\begin{proof}
Let $\sigma_0:\Lambda_i^n\to\Sing_\bullet(X)$ be a map of simplicial sets for $n>0$. Using the geometric realization functor, we can identify $\sigma_0$ with a continuous map of topological spaces $f_0:|\Lambda^n_i|\to X$; we want to show that $f_0$ factors as a composition
\[\begin{tikzcd}
{|\Lambda^n_i|}\ar[r]&{|\Delta^n|}\ar[r,"f"]&X
\end{tikzcd}\]
Now using \cref{simplicial set boundary horn geometric realization}, $|\Lambda^n_i|$ is identified with the subset
\[\{(t_0,\dots,t_n)\in|\Delta^n:\text{$t_j=0$ for some $j\neq i$}\}\sub|\Delta^n|\]
so we can take $f$ to be the composition $f_0\circ\iota$, where $\iota$ is any continuous retraction of $|\Delta^n|$ onto the subset $|\Lambda^n_i|$.
\end{proof}
\begin{proposition}\label{simplicial set group is Kan}
Let $G_\bullet$ be a simplicial group (that is, a simplicial object of the category of groups). Then the underlying simplicial set of $G_\bullet$ is a Kan complex.
\end{proposition}
\begin{proof}
Let $n$ be a positive integer and $\sigma:\Lambda^n_i\to G_\bullet$ be a map of simplicial sets for some $0\leq i\leq n$, which we will identify with a tuple $(\sigma_0,\sigma_1,\dots,\sigma_{i-1},\ast,\sigma_{i+1},\dots,\sigma_n)$ of elements of the group $G_{n-1}$ (\cref{simplicial set horn description}). We want to show that there exists an element $\tau\in G_n$ such that $d_h\tau=\sigma_j$ for $j\neq i$. Now let $e$ be the identity of $G_{n-1}$; we first treat the special case where $\sigma_{i+1}=\cdots=\sigma_n=e$. If in addition we have $\sigma_0=\cdots=\sigma_{n-1}=e$, then we can take $\tau$ to be the identity element of $G_n$. Otherwise, there exists some smallest integer $j<i$ such that $\sigma_j\neq e$, and we proceed by descending induction on $j$. Let $\eta=s_j(\sigma_j)\in G_n$ and consider the map $\tilde{\sigma}:\Lambda_i^n\to G_n$ given by the tuple $(\tilde{\sigma}_0,\tilde{\sigma}_1,\dots,\tilde{\sigma}_{i-1},\ast,\tilde{\sigma}_{i+1},\dots,\tilde{\sigma}_n)$ where $\tilde{\sigma}_k=\sigma_k(d_k(\eta))^{-1}$. Since $d_j\circ s_j=\id$ (\cref{simplicial set face degeneracy map char}), we have $\tilde{\sigma}_0=\cdots=\tilde{\sigma}_j=e$ and $\tilde{\sigma}_{i+1}=\cdots=\tilde{\sigma}_n=e$, so by invoking our inductive hypothesis we conclude that there exists an element $\tilde{\tau}\in G_n$ satisfying $d_k\tilde{\tau}=\tilde{\sigma}_k$ for $k\neq i$. We can then complete the proof by taking $\tau$ to be the product $\tilde{\tau}\eta$.\par
If not all of the equalities $\sigma_{i+1}=\cdots=\sigma_n=e$ hold, then there exists some largest integer $j>i$ such that $\sigma_j\neq e$, and we proceed by ascending induction on $j$. Let $\eta=s_{j-1}(\sigma_j)$ and $\tilde{\sigma}:\Lambda^n_i\to G_\bullet$ be the map given by the tuple $(\tilde{\sigma}_0,\tilde{\sigma}_1,\dots,\tilde{\sigma}_{i-1},\ast,\tilde{\sigma}_{i+1},\dots,\tilde{\sigma}_n)$ with $\tilde{\sigma}_k=\sigma_k(d_k(\eta))^{-1}$, as above. Then $\tilde{\sigma}_{j}=\cdots=\tilde{\sigma}_n=e$ so the inductive hypothesis guarantees the existence of an element $\tilde{\tau}\in G_n$ such that $d_k(\tilde{\tau})=\tilde{\sigma}_k$ for $k\neq i$. As before, we complete the proof by setting $\tau=\tilde{\tau}\eta$.
\end{proof}
Let $S_\bullet$ be a simplicial set. According to \cref{simplicial set pi_0 is quotient of d_0 and d_1 relation}, we can identify the set of connected components $\pi_0(S_\bullet)$ with the quotient $S_0/\sim$, where $\sim$ is the equivalence relation generated by the image of the map $(d_0,d_1):S_1\to S_0\times S_0$. In the special case where $S_\bullet=\Sing_\bullet(X)$ is the singular simplicial set of a topological space $X$, this description simplifies: the image of the map $(d_0,d_1)$ is already an equivalence relation, and $\pi_0(S_\bullet)$ can be identified with the set of path components $\pi_0(X)$. This observation can be generalized to any Kan complex:
\begin{proposition}\label{simplicial set Kan same component iff connected by edge}
Let $S_\bullet$ be a Kan complex containing a pair of vertices $x,y\in S_0$. Then $x$ and $y$ belong to the same connected component of $S_\bullet$ if and only if there exists an edge $e\in S_1$ satisfying $d_0(e)=x$ and $d_1(e)=y$.
\end{proposition}
\begin{proof}
Let $R$ denote the image of the map $(d_0,d_1):S_1\to S_0\times S_0$. According to \cref{simplicial set pi_0 is quotient of d_0 and d_1 relation}, we can identify $\pi_0(S_\bullet)$ with the quotient of $S_0$ by the equivalence relation generated by $R$. It will therefore suffice to show that $R$ is already an equivalence relation on $S_0$. To prove this, we first note that for any $x\in S_0$, the map $(d_0,d_1)$ sends the degenerate edge $s_0(x)$ to the pair $(x,x)\in S_0\times S_0$, so $R$ is reflexive. Now suppose that $(x,y)\in R$, so that there exists an edge $e\in S_1$ with $d_0(e)=x$, $d_1(e)=y$. Then the tuple $(e,s_0(x),\ast)$ gives a map of simplicial sets $\sigma_0:\Lambda^2_2\to S_\bullet$, which we depict as a diagram
\[\begin{tikzcd}
&y\ar[rd,"e"]&\\
x\ar[ru,dashed]\ar[rr,"s_0(x)"]&&x
\end{tikzcd}\]
Since $S_\bullet$ is a Kan complex, we can complete this diagram to a $2$-simplex $\sigma:\Delta^2\to S_\bullet$. Then $e'=d_2(\sigma)$ is an edge of $S_\bullet$ satisfying $d_0(e')=y$ and $d_1(e')=x$, which proves that the pair $(y,x)$ belongs to $R$, so $R$ is symmetirc\par
Finally, suppose that we are given vertices $x,y,z$ in $S_0$ with $(x,y)\in R$ and $(y,z)\in R$; we want to show that $(x,z)\in R$. Choose edges $e,e'\in S_1$ satisfying $d_0(e)=x$, $d_1(e)=y=d_0(e')$, $d_1(e')=z$. Then the tuple $(e',\ast,e)$ determines a map of simplicial sets $\tau_0:\Lambda^2_1\to S_\bullet$, which we depict as a diagram
\[\begin{tikzcd}
&y\ar[rd,"e"]&\\
x\ar[ru,"e'"]\ar[rr,dashed]&&x
\end{tikzcd}\]
Our assumption that $S_\bullet$ is a Kan complex guarantees that we can extend $\tau_0$ to a $2$-simplex $\tau:\Delta^2\to S_\bullet$. Then $e''=d_1(\tau)$ is an edge of $S_\bullet$ satisfying $d_0(e'')=x$ and $d_1(e'')=z$, which proves that $(x,z)\in R$. In other words, $R$ is transitive.
\end{proof}
\begin{corollary}\label{simplicial set Kan product and pi_0}
Let $\{S_\bullet^\alpha\}_{\alpha\in I}$ be a collection of Kan complexes, and $S_\bullet=\prod_\alpha S_\bullet^\alpha$ be their product. Then the canonical map
\[\pi_0(S_\bullet)\to\prod_{\alpha\in I}\pi_0(S_\bullet^\alpha)\]
is bijective. In particular, $S_\bullet$ is connected if and only if each factor $S_\bullet^\alpha$ is connected.
\end{corollary}
\section{Nerves}
In \autoref{simplicial set section}, we introduced the theory of simplicial sets and discussed its relationship to the theory of topological spaces. Every topological space $X$ determines a simplicial set $\Sing_\bullet(X)$, and simplicial sets of the form $\Sing_\bullet(X)$ have a special property: they are Kan complexes. In this section, we will study a different class of simplicial sets, which arise instead from the theory of categories. We will associate to every category $\mathcal{C}$ a simplicial set $N_\bullet(\mathcal{C})$, called the nerve of $\mathcal{C}$. We will see that the construction $\mathcal{C}\mapsto N_\bullet(\mathcal{C})$ is fully faithful, and a simplicial set $S_\bullet$ belongs to the essential image of the functor $\mathcal{C}\mapsto N_\bullet(\mathcal{C})$ if and only if it satisfies a certain lifting condition. This lifting condition is similar to the Kan extension condition, but has a slight difference. Also, we will see that a simplicial set of the form $N_\bullet(\mathcal{C})$ is a Kan complex if and only if every morphism in $\mathcal{C}$ is invertible.
\subsection{Nerve of a category}
For each integer $n\geq 0$, let us view the linearly ordered set $[n]$ as a category (where there is a unique morphism from $i$ to $j$ if and only if $i\leq j$). For any category $\mathcal{C}$, we denote $N_n(\mathcal{C})$ the set of all functors from $[n]$ to $\mathcal{C}$. Note that any nondecreasing map $\alpha:[m]\to [n]$ can be also considered as a functor between two categories, so composition with $\alpha$ determines a map of sets $N_n(\mathcal{C})\to N_m(\mathcal{C})$. We can therefore view the construction $[n]\mapsto N_n(\mathcal{C})$ as a simplicial set, which is denoted by $N_\bullet(\mathcal{C})$ and called the \textbf{nerve} of $\mathcal{C}$. The geometric realization $|N_\bullet(\mathcal{C})|$ is called the \textbf{classifying space} of the category $\mathcal{C}$.\par
For each $n\geq 1$, from the definition of $N_n(\mathcal{C})$ we see that an element of $N_n(\mathcal{C})$ can be identified with a diagram
\[\begin{tikzcd}
C_0\ar[r,"f_1"]&C_1\ar[r,"f_2"]&\cdots\ar[r,"f_n"]&C_n
\end{tikzcd}\]
in the category $\mathcal{C}$. In other words, the elements of $N_n(\mathcal{C})$ are $n$-tuples $(f_1,\dots,f_n)$ of morphisms of $\mathcal{C}$ having the property that, for $0<i<n$, the source of $f_{i+1}$ coincides with the target of $f_i$.
\begin{example}
Let $\mathcal{C}$ be a category. Then vertices of the simplicial set $N_\bullet(\mathcal{C})$ can be identified with objects of the category $\mathcal{C}$, and edges of $N_\bullet(\mathcal{C})$ can be identified with morphisms in the category $\mathcal{C}$. Let $f:X\to Y$ be a morphism in $\mathcal{C}$, regarded as an edge of the simplicial set $N(\mathcal{C})$. Then the faces of $f$ are given by the target $d_0(f)=Y$ and the source $d_1(f)=X$, respectively. Conversely, if $X$ is an object of $\mathcal{C}$, which we regard as a vertex of the simplicial set $N_\bullet(\mathcal{C})$, then the degenerate edge $s_0(X)$ is the identity morphism $\id_X:X\to X$.
\end{example}
\begin{example}
Let $\mathcal{C}$ be a category. For every integer $n\geq 0$, we let $N_{\leq n}(\mathcal{C})$ denote the $n$-skeleton of the simplicial set $N_\bullet(\mathcal{C})$. In the special case $n=0$, this recovers the discrete simplicial set associated to the set of objects $\Ob(C)$.
\end{example}
\begin{example}[\textbf{Face Operators on \boldmath$N_\bullet(\mathcal{C})$}]\label{simplicial set nerve of cat face map char}
Let $\mathcal{C}$ be a category and suppose we are given an $n$-simplex $\sigma$ of the simplicial set $N_\bullet(\mathcal{C})$ for some $n>0$, which we identify with a diagram
\[\begin{tikzcd}
C_0\ar[r,"f_1"]&C_1\ar[r,"f_2"]&C_2\ar[r]&\cdots\ar[r,"f_n"]&C_n
\end{tikzcd}\]
Then the $0$-th face $d_0(\sigma)\in N_{n-1}(\mathcal{C})$ can be identified with the diagram
\[\begin{tikzcd}
C_1\ar[r,"f_2"]&C_2\ar[r]&C_3\ar[r]&\cdots\ar[r,"f_n"]&C_n
\end{tikzcd}\]
obtained from $\sigma$ by "deleting" the object $C_0$ (and the morphism $f_1$ with source $C_0$). Similarly, the $n$-th face $d_n(\sigma)\in N_{n-1}(\mathcal{C})$ can be identified with the diagram
\[\begin{tikzcd}
C_0\ar[r,"f_1"]&C_1\ar[r,"f_2"]&C_2\ar[r]&\cdots\ar[r,"f_{n-1}"]&C_{n-1}
\end{tikzcd}\]
obtained from $\sigma$ by "deleting" the object $C_n$ (and the morphism $f_n$ with target $C_n$). On the other hand, for $0<i<n$, the $i$-th face $d_i(\sigma)\in N_{n-1}(\mathcal{C})$ can be identified with the diagram
\[\begin{tikzcd}
C_0\ar[r,"f_1"]&C_1\ar[r,"f_2"]&\cdots\ar[r]&C_{i-1}\ar[r,"f_{i+1}\circ f_i"]&C_{i+1}\ar[r]&\cdots\ar[r,"f_n"]&C_n
\end{tikzcd}\]
obtained by "deleting" the object $C_i$ (and composing the morphisms $f_i$ and $f_{i+1}$).
\end{example}
\begin{example}[\textbf{Degeneracy Operators on \boldmath$N_\bullet(\mathcal{C})$}]\label{simplicial set nerve of cat degeneracy map char}
Let $\mathcal{C}$ be a category and suppose we are given an $n$-simplex $\sigma$ of the simplicial set $N_\bullet(\mathcal{C})$ which we identify with a diagram
\[\begin{tikzcd}
C_0\ar[r,"f_1"]&C_1\ar[r,"f_2"]&C_2\ar[r]&\cdots\ar[r,"f_n"]&C_n
\end{tikzcd}\]
Then, for $0\leq i\leq n$, we can identify $s_i(\sigma)\in N_{n+1}(\mathcal{C})$ with the diagram
\[\begin{tikzcd}
C_0\ar[r,"f_1"]&\cdots\ar[r]&C_{i-1}\ar[r,"f_i"]&C_i\ar[r,"\id_{C_i}"]&C_i\ar[r,"f_{i+1}"]&C_{i+1}\ar[r]&\cdots\ar[r,"f_n"]&C_n
\end{tikzcd}\]
obtained from $\sigma$ by "inserting" the identity morphism $\id_{C_i}$. From this, we also conclude that an $n$-simplex $\sigma$ is degenerate if and only if some $f_i$ is an identity of $\mathcal{C}$ (in which case we must have $C_{i-1}=C_i$).
\end{example}
\begin{example}[\textbf{Nerve of a partially ordered set}]
Let $I$ be a set equipped with a partial ordering $\preceq$. Then we can regard $I$ as a category whose objects are the elements of $I$, with morphisms given by
\[\Hom_I(i,j)=\begin{cases}
\ast&\text{if $i\preceq j$},\\
\emp&\text{otherwise}.
\end{cases}\]
The nerve of this category is denoted by $N_\bullet(I)$, and called the \textbf{nerve of the partially ordered set $I$}. For each $n\geq 0$, we can identify $n$-simplices of $N_\bullet(I)$ with monotone functions $[n]\to I$: that is, with nondecreasing sequences $(i_0,\dots,i_n)$ of elements of $I$. For example, if $n\geq 0$ is an integer, then the nerve $N_\bullet([n])$ can be identified with the standard $n$-simplex $\Delta^n$.
\end{example}
\begin{remark}\label{simplicial set nerve of cat is Sing^Q}
The map $\mathcal{C}\to N_\bullet(\mathcal{C})$ determines a functor $N_\bullet:\mathbf{Cat}\to\mathbf{Set}_\Delta$ from the category of (small) categories to the category $\mathbf{Set}_\Delta$ of simplicial sets. This is a special case of the functor $\Sing_\bullet^Q$. More precisely, we can identify $N_\bullet$ with the functor $\Sing_\bullet^Q$, where $Q:\Delta\to\mathbf{Cat}$ is the functor which carries each object $[n]\in\Delta$ to itself, regarded as a category.
\end{remark}
It is an important fact that passing from a category $\mathcal{C}$ to its nerve $N_\bullet(\mathcal{C})$ does not lose any information: one can think that $N_\bullet(\mathcal{C})$ is just obtained by organizing the datum of $\mathcal{C}$ in a very efficient way. Thanks to the following proposition, we often identify a category $\mathcal{C}$ with its nerve $N_\bullet(\mathcal{C})$.
\begin{proposition}\label{simplicial set nerve of cat fully faithful}
The nerve functor $N_\bullet:\mathbf{Cat}\to\mathbf{Set}_\Delta$ is fully faithful.
\end{proposition}
\begin{proof}
Let $\mathcal{C}$ and $\mathcal{D}$ be categories. We wish to show that the nerve functor $N_\bullet$ induces a bijection
\[\theta:\Hom_{\mathbf{Cat}}(\mathcal{C},\mathcal{D})\to\Hom_{\mathbf{Set}_\Delta}(N_\bullet(\mathcal{C}),N_\bullet(\mathcal{D})).\]
Here the source of $\theta$ is the set of all functors from $\mathcal{C}$ to $\mathcal{D}$. We first note that $\theta$ is injective: a functor $F:\mathcal{C}\to\mathcal{D}$ is completely determined by its behavior on the objects and morphisms of $\mathcal{C}$, and therefore by the behavior of $\theta(F)$ on the vertices and edges of the simplicial set $N_\bullet(\mathcal{C})$. Now let $f:N_\bullet(\mathcal{C})\to N_\bullet(\mathcal{D})$ be a morphism of simplicial sets; we use $f$ to construct a functor $F:\mathcal{C}\to\mathcal{D}$ such that $\theta(F)=f$. For each $n\geq 0$, the morphism $f$ determines a map of sets $N_n(\mathcal{C})\to N_n(\mathcal{D})$, which is also denoted by $f$. In the case $n=0$, this map sends each object $C\in\mathcal{C}$ to an object of $\mathcal{D}$, which we denote by $F(C)$. For every pair of objects $C,D\in\mathcal{C}$, the map $f$ carries each morphism $u:C\to D$ to a morphism $f(u)$ in the category $\mathcal{D}$. Since $f$ commutes with face maps, the morphism $f(u)$ has source $F(C)$ and target $F(D)$ (see \cref{simplicial set nerve of cat face map char}), and can therefore be regarded as an element of $\Hom_{\mathcal{D}}(F(C),F(D))$; we denote this element by $F(u)$. We then complete the proof by verifying the following:
\begin{itemize}
\item[(a)] The preceding construction determines a functor $F:\mathcal{C}\to\mathcal{D}$.
\item[(b)] We have an equality $f=\theta(F)$ of maps from $N_\bullet(\mathcal{C})$ to $N_\bullet(\mathcal{D})$.
\end{itemize}
To prove (a), we first note that the compatibility of $f$ with degeneracy maps implies that we have $F(\id_C)=\id_{F(C)}$ for each $C\in\mathcal{C}$ (see \cref{simplicial set nerve of cat degeneracy map char}). It therefore suffices to show that for every pair of composable morphisms $u:C\to D$ and $v:D\to E$ in the category $\mathcal{C}$, we have $F(v)\circ F(u)=F(v\circ u)$ as elements of the set $\Hom_{\mathcal{D}}(F(C),F(E))$. For this, we observe that the diagram
\[\begin{tikzcd}
C\ar[r,"u"]&D\ar[r,"v"]&E
\end{tikzcd}\]
can be identified with a $2$-simplex $\sigma$ of $N_\bullet(\mathcal{C})$. Using the equality $d_i(f(\sigma))=f(d_i(\sigma))$ for $i=0,2$, we see that $f(\sigma)$ corresponds to the diagram
\[\begin{tikzcd}
F(C)\ar[r,"F(u)"]&F(D)\ar[r,"F(v)"]&F(E)
\end{tikzcd}\]
in $\mathcal{D}$. Since $F(v)\circ F(u)=d_1(f(\sigma))$ and $F(v\circ u)=f(d_1(\sigma))$, condition (a) therefore follows. To prove (b), we must show that $f(\tau)=\theta(F)(\tau)$ for each $n$-simplex $\tau$ of $N_\bullet(\mathcal{C})$. This follows by construction in the case $n=1$, and follows in general since an $n$-simplex of $N_\bullet(\mathcal{D})$ is determined by its $1$-dimensional faces.
\end{proof}
We now describe the essential image of the functor $N_\bullet:\mathbf{Cat}\to\mathbf{Set}_\Delta$. As we shall see, this is characterized by the following conditions similar to (Kan):
\begin{enumerate}[leftmargin=40pt]
\item[(Ner)] For any pair of integers $0<i<n$, any map of simplicial sets $\sigma_0:\Lambda_i^n\to S_\bullet$ can be uniquely extended into a map $\sigma:\Delta^n\to S_\bullet$.
\end{enumerate}
The proof of this characterization will require some preliminaries. We begin by establishing the necessity.
\begin{lemma}\label{simplicial set nerve of cat satisfy condition Ner}
Let $\mathcal{C}$ be a category. Then the simplicial set N•(C) satisfies condition (Ner).
\end{lemma}
\begin{proof}
Choose integers $0<i<n$ together with a map of simplicial sets $\sigma_0:\Lambda_i^n\to N_\bullet(\mathcal{C})$; we want to show that $\sigma_0$ can be extended uniquely to a $n$-simplex of $N_\bullet(\mathcal{C})$. For $0\leq j\leq n$, let $C_j\in\mathcal{C}$ denote the image under $\sigma_0$ of the $j$-th vertex of $\Delta^n$ (which belongs to the horn $\Lambda^n_i$). We first consider the case where $n\geq 3$. In this case, $\Lambda^n_i$ contains every edge of $\Delta^n$. For $0\leq j\leq k\leq n$, let $f_{kj}:C_j\to C_k$ denote the $1$-simplex of $N_\bullet(\mathcal{C})$ obtained by evaluating $\sigma_0$ on the edge of $\Delta^n$ corresponding to the pair $(j,k)$. We claim that the map $j\mapsto C_j,(j\leq k)\mapsto f_{kj}$ determines a functor $[n]\to\mathcal{C}$, which can be identified with an $n$-simplex of $N_\bullet(\mathcal{C})$ having the desired properties. It is easy to see that $f_{jj}=\id_{C_j}$ for each $0\leq j\leq n$, so we only need to show that $f_{lk}\circ f_{kj}=f_{lj}$ for every triple $0\leq j\leq k\leq l\leq n$. Such a triple determines a $2$-simplex $\tau$ of $N_\bullet(\mathcal{C})$. If $\tau$ is contained in $\Lambda^n_i$, then $\tilde{\tau}=\sigma_0(\tau)$ is a $2$-simplex of $N_\bullet(\mathcal{C})$ satisfying
\[d_0(\tilde{\tau})=f_{lk},\quad d_1(\tilde{\tau})=f_{lj},\quad d_2(\tilde{\tau})=f_{kj},\]
so that $\tilde{\tau}$ "witnesses" the identity $f_{lk}\circ f_{kj}=f_{lj}$ (c.f. \cref{simplicial set nerve of cat face map char}). It therefore suffices to treat the case where the simplex $\tau$ does not belong to the $\Lambda^n_i$ (and therefore $\{j,k,l\}\cup\{i\}=[n]$). In this case, our assumption that $n\geq 3$ guarantees that $\{j,k,l\}=[n]\setminus\{i\}$, so either $i=1$ or $i=2$ (recall that $0<i<n$ by our hypothesis). We will treat the case $i=1$ (the case $i=2$ follows by a similar argument). Note that $\Lambda^3_1$ contains all of the nondegenerate $2$-simplices of $\Delta^3$ other than $\tau$; by applying the map $\sigma_0$, we obtain $2$-simplices of $N_\bullet(\mathcal{C})$ which witness the identities
\[f_{30}=f_{31}\circ f_{10},\quad f_{31}=f_{32}\circ f_{21},\quad f_{20}=f_{21}\circ f_{10}.\]
It then follows that
\[f_{30}=f_{31}\circ f_{10}=(f_{32}\circ f_{21})\circ f_{10}=f_{32}\circ(f_{21}\circ f_{10})=f_{32}\circ f_{20}\]
so that $f_{lj}=f_{lk}\circ f_{kj}$, as desired.\par
It remains to treat the case $n=2$, so that we must also have $i=1$. In this situation, the map $\sigma_0:\Lambda^n_i\to N_\bullet(\mathcal{C})$ determines a pair of composable morphisms $f_{1,0}:C_0\to C_1$ and $f_{21}:C_1\to C_2$. This data extends uniquely to a $2$-simplex $\sigma$ of $\mathcal{C}$ satisfying $d_1(\sigma)=f_{21}\circ f_{10}$.
\end{proof}
\begin{lemma}\label{simplicial set satisfy condition Ner map prop}
Let $f:S_\bullet\to T_\bullet$ be a map of simplicial sets. Assume that $f$ induces bijections $S_0\to T_0$ and $S_1\to T_1$, and that both $S_\bullet$ and $T_\bullet$ satisfy condition (Ner). Then $f$ is an isomorphism.
\end{lemma}
\begin{proof}
We claim that for every simplicial set $K_\bullet$, composing with $f$ induces a bijection
\[\theta_{K_\bullet}:\Hom_{\mathbf{Set}_\Delta}(K_\bullet,S_\bullet)\to\Hom_{\mathbf{Set}_\Delta}(K_\bullet,T_\bullet)\]
Writing $K_\bullet$ as a union of its skeleta $\sk_n(K_\bullet)$, we can reduce to the case where $K$ has dimension $\leq n$, for some integer $n\geq -1$. We now proceed by induction on $n$. The case $n=-1$ is trivial (since a simplicial set of dimension $\leq -1$ is empty). Let us therefore assume that $n\geq 0$, so that \cref{simplicial set skeleton pushout square} gives a pushout diagram of simplicial sets
\[\begin{tikzcd}
\coprod\partial\Delta^n\ar[r]\ar[d]&\coprod\Delta^n\ar[d]\\
\sk_{n-1}(K_\bullet)\ar[r]&K_\bullet
\end{tikzcd}\]
It follows from our inductive hypothesis that the maps $\theta_{\partial\Delta^n}$ and $\theta_{\sk_{n-1}(K_\bullet)}$ are bijective. Consequently, to show that $\theta_{K_\bullet}$ is bijective, it suffices to show that $\theta_{\Delta^n}$ is bijective: that is, $f$ induces a bijection $S_n\to T_n$. For $n=1$, this follows from our hypothesis. To handle the case $n\geq 2$, we observe that there is a commutative diagram
\[\begin{tikzcd}
\Hom_{\mathbf{Set}_\Delta}(\Delta^n,S_\bullet)\ar[r]\ar[d,swap,"\theta_{\Delta^n}"]&\Hom_{\mathbf{Set}_\Delta}(\Lambda^n_1,S_\bullet)\ar[d,"\theta_{\Lambda^n_1}"]\\
\Hom_{\mathbf{Set}_\Delta}(\Delta^n,T_\bullet)\ar[r]&\Hom_{\mathbf{Set}_\Delta}(\Lambda^n_1,T_\bullet)
\end{tikzcd}\]
Here the right vertical map is bijective by our inductive hypothesis, and the horizontal maps are bijective by our assumption that both $S_\bullet$ and $T_\bullet$ satisfy condition (Ner). It follows that the left vertical map is also bijective, as desired.
\end{proof}
\begin{proposition}\label{simplicial set isomorphic to nerve iff condition Ner}
Let $S_\bullet$ be a simplicial set. Then $S_\bullet$ is isomorphic to the nerve of a category if and only if it satisfies condition (Ner).
\end{proposition}
\begin{proof}
Let $S_\bullet$ be a simplicial set satisfying condition (Ner); we will show that there is a category $\mathcal{C}$ and an isomorphism of simplicial sets $u:S_\bullet\to N_\bullet(\mathcal{C})$ (the converse assertion follows from \cref{simplicial set nerve of cat satisfy condition Ner}). It follows from \cref{simplicial set nerve of cat fully faithful} that the category $\mathcal{C}$ is uniquely determined (up to isomorphism), and from the proof of \cref{simplicial set nerve of cat fully faithful} we can extract an explicit construction of $\mathcal{C}$:
\begin{itemize}
\item The objects of $\mathcal{C}$ are the vertices of $S_\bullet$.
\item Given a pair of objects $C,D\in\mathcal{C}$, we let $\Hom_{\mathcal{C}}(C,D)$ denote the collection of edges $e$ of $S_\bullet$ satisfying $d_0(e)=D$ and $d_1(e)=C$.
\item For each object $C\in\mathcal{C}$, we define the identity morphism $\id_C\in\Hom_{\mathcal{C}}(C,C)$ to be the degenerate edge $s_0(C)$.
\item Given a triple of objects $C,D,E\in\mathcal{C}$ and a pair of morphisms $f\in\Hom_{\mathcal{C}}(C,D)$ and $g\in\Hom_{\mathcal{C}}(D,E)$, we can apply condition (Ner) (in the special case $n=2$ and $i=1$) to conclude that there is a unique $2$-simplex $\sigma$ of $S_\bullet$ satisfying $d_2(\sigma)=f$ and $d_0(\sigma)=g$. We define the composition $g\circ f\in\Hom_{\mathcal{C}}(C,E)$ to be the edge $d_1(\sigma)$.
\end{itemize}
We claim that $\mathcal{C}$ is a category. For this, we must check the following:
\begin{itemize}
\item The composition law on $\mathcal{C}$ is unital: for every morphism $f:C\to D$ in $\mathcal{C}$, we have equalities
\[\id_D\circ f=f=f\circ\id_C.\]
We verify the first equality since the proof in the other case is similar. For this, we can construct a $2$-simplex $\sigma$ of $S_\bullet$ such that $d_0(\sigma)=\id_D$ and $d_1(\sigma)=d_2(\sigma)=f$. The degenerate $2$-simplex $s_1(f)$ has these properties.
\item The composition law on $\mathcal{C}$ is associative. That is, for every triple of composable morphisms
\[f:W\to X,\quad g:X\to Y,\quad h:Y\to Z\]
in $\mathcal{C}$, we have an identity $h\circ(g\circ f)=(h\circ g)\circ f$ in the category $\mathcal{C}$. Applying condition (Ner) repeatedly, we deduce the following:
\begin{itemize}
\item There is a unique $2$-simplex $\sigma_0$ of $\mathcal{C}$ satisfying $d_0(\sigma_0)=h$ and $d_2(\sigma_0)=g$ (it follows that $d_1(\sigma_0)=h\circ g$).
\item There is a unique $2$-simplex $\sigma_3$ of $\mathcal{C}$ satisfying $d_3(\sigma_0)=g$ and $d_2(\sigma_3)=f$ (it follows that $d_1(\sigma_3)=g\circ f$).
\item There is a unique $2$-simplex $\sigma_2$ of $\mathcal{C}$ satisfying $d_0(\sigma_2)=h\circ g$ and $d_2(\sigma_2)=f$ (it follows that $d_1(\sigma_2)=(h\circ g)\circ f$).
\item There is a unique $3$-simplex $\tau$ of $\mathcal{C}$ satisfying $d_0(\tau)=\sigma_0$, $d_2(\tau)=\sigma_2$, and $d_3(\tau)=\sigma_3$.
\end{itemize}
The $3$-simplex $\tau$ can be depicted in the following diagram
\[\begin{tikzcd}[row sep=15mm,column sep=15mm]
&X\ar[r,"g"]&Y\ar[rd,"h"]&\\
W\ar[ru,"f"]\ar[rru,"g\circ f"{anchor=south}]\ar[rrr,"(h\circ g)\circ f"]&&&Z\ar[llu,crossing over,leftarrow,"h\circ g"{anchor=south}]
\end{tikzcd}\]
Set $\sigma_1=d_1(\tau)$, then $\sigma_1$ is a $2$-simplex of $S_\bullet$ satisfying $d_0(\sigma_1)=h$, $d_1(\sigma_1)=(h\circ g)\circ f$, and $d_2(\sigma_1)=g\circ f$. It follows that $\sigma_1$ "witnesses" the identity $h\circ(g\circ f)=(h\circ g)\circ f$.
\end{itemize}
Note that every $n$-simplex $\sigma:\Delta^n\to S_\bullet$ determines a functor $[n]\to\mathcal{C}$, given on objects by the values of $\sigma$ on the vertices of $\Delta^n$ and on morphisms by the values of $\sigma$ on the edges of $\Delta^n$. This construction determines a map of simplicial sets $u:S_\bullet\to N_\bullet(\mathcal{C})$, which is clearly bijective on simplices of dimension $\leq 1$. Since the simplicial sets $S_\bullet$ and $N_\bullet(\mathcal{C})$ both satisfy condition (Ner), it follows from \cref{simplicial set satisfy condition Ner map prop} that $u$ is an isomorphism.
\end{proof}
\begin{remark}
The characterization of \cref{simplicial set isomorphic to nerve iff condition Ner} has many variants. For example, one can replace condition (Ner) by the following a priori weaker condition:
\begin{enumerate}[leftmargin=40pt]
\item[(Ner')] For every $n\geq 2$ and every map of simplicial sets $\sigma_0:\Lambda^n_1\to S_\bullet$, there exists a unique map $\sigma:\Delta^n\to S_\bullet$ satisfying $\sigma_0=\sigma|_{\Lambda^n_i}$.
\end{enumerate}
\end{remark}
According to \cref{simplicial set nerve of cat fully faithful}, every category $\mathcal{C}$ can be recovered, up to canonical isomorphism, from the nerve $N_\bullet(\mathcal{C})$. In particular, any isomorphism-invariant condition on a category $\mathcal{C}$ can be reformulated as a condition on the simplicial set $N_\bullet(\mathcal{C})$. We now illustrate this principle with a simple example, that is, the category $\mathcal{C}$ being a groupoid.
\begin{proposition}\label{simplicial set nerve of cat Kan iff groupoid}
Let $\mathcal{C}$ be a category. Then $\mathcal{C}$ is a groupoid if and only if $N_\bullet(\mathcal{C})$ is a Kan complex.
\end{proposition}
\begin{proof}
Suppose first that $N_\bullet(\mathcal{C})$ is a Kan complex. Let $f:C\to D$ be a morphism in $\mathcal{C}$; using the surjectivity of the map $\Hom_{\mathbf{Set}_\Delta}(\Delta^2,N_\bullet(\mathcal{C}))\to\Hom_{\mathbf{Set}_\Delta}(\Lambda^2_2,N_\bullet(\mathcal{C}))$, we see that there exists a $2$-simplex $\sigma$ of $N_\bullet(\mathcal{C})$ such that $d_0(\sigma)=f$ and $d_1(\sigma)=\id_D$. If we set $g=d_2(\sigma)$, then $f\circ g=\id_D$, so $g$ is a left inverse of $f$. Similarly, by considering $\Lambda^2_0$, we see there exists a map $h:D\to C$ such that $h\circ f=\id_C$. It then follows that $g=h$ and $f$ is an isomorphism, so $\mathcal{C}$ is a groupoid.
Now suppose that $\mathcal{C}$ is a groupoid. We show that for $0\leq i\leq n$, any map $\sigma_0:\Lambda_i^n\to N_\bullet(\mathcal{C})$ can be extended to an $n$-simplex $\sigma:\Delta^n\to N_\bullet(\mathcal{C})$. For $0<i<n$, this follows from \cref{simplicial set nerve of cat satisfy condition Ner} (and does not require the assumption that $\mathcal{C}$ is a groupoid). We will treat the case where $i=0$, since the other case $i=n$ follows by similar reasoning. We consider several cases: in the case $n=1$, the map $\sigma_0:\Lambda_0^n\to N_\bullet(\mathcal{C})$ can be identified with an object $C\in\mathcal{C}$. In this case, we can take $\sigma$ to be an edge of $N_\bullet(\mathcal{C})$ corresponding to any morphism with target $\mathcal{C}$ (for example, we can take $\sigma$ to be the identity map $\id_C$). In the case $n=2$, we can identify $\sigma_0$ with a pair of morphisms in $\mathcal{C}$ having the same source, which we can depict as a diagram
\[\begin{tikzcd}
&D\ar[rd,dashed]&\\
C\ar[ru,"f"]\ar[rr,"g"]&&E
\end{tikzcd}\]
Our assumption that $\mathcal{C}$ is a groupoid guarantees that we can extend this diagram to a $2$-simplex of $\mathcal{C}$, whose $0$-th face is given by the morphism $g\circ f^{-1}:D\to E$.\par
Finally, in the case $n\geq 3$, the map $\sigma_0$ determines a collection of objects $\{C_i\}_{0\leq i\leq n}$ and morphisms $f_{ji}:C_i\to C_j$ for $i\leq j$. We show that these morphisms determine a functor $[n]\to\mathcal{C}$ (which we can then identify with an $n$-simplex $\sigma$ of $N_\bullet(\mathcal{C})$ satisfying $\sigma_0=\sigma|_{\Lambda^n_0}$). For this, we must verify the identity $f_{kj}\circ f_{ji}=f_{ki}$ for $0\leq i\leq j\leq k\leq n$. Note that this identity is satisfied whenever the triple $(i,j,k)$ determines a $2$-simplex of $\Delta^n$ belonging to the horn $\Lambda^n_0$. This is automatic unless $n=3$ and $(i,j,k)=(1,2,3)$. To handle this exceptional case, we note that
\begin{align*}
(f_{32}\circ f_{21})\circ f_{10}&=f_{22}\circ (f_{21}\circ f_{10})=f_{32}\circ f_{20}=f_{30}.
\end{align*}
Since C is a groupoid, composing with $f_{10}^{-1}$ on the right yields the desired identity $f_{32}\circ f_{21}=f_{31}$.
\end{proof}
\begin{example}
Let $M$ be a monoid. We can then form a category $BM$ having a single object $X$, where $\Hom_{BM}(X,X)=M$ and the composition of morphisms in $BM$ is given by multiplication in $M$. We will denote the nerve of the category $BM$ by $B_\bullet M$. In the special case where $M=G$ is a group, the geometric realization $|B_\bullet G|$ is a topological space called the \textbf{classifying space} of $G$. It can be characterized (up to homotopy equivalence) by the fact that it is a CW complex with either of the following properties:
\begin{itemize}
\item The space $|B_\bullet G|$ is connected, and its homotopy groups (with respect to any choice of base point) are given by the formula
\[\pi_*(|B_\bullet G|)=\begin{cases}
G&\text{if $\ast=1$},\\
0&\text{if $\ast>1$}.
\end{cases}\]
\item For any paracompact topological space $X$, there is a canonical bijection
\[\{\text{continuous maps $f:X\to|B_\bullet G|$}\}/\text{homotopy}\stackrel{\sim}{\to}\{\text{$G$-torsors $P\to X$}\}/\text{isomorphism}.\]
\end{itemize}
\end{example}
\begin{example}
Let $\mathcal{C}$ be a category and $N_\bullet(\mathcal{C})$ be the nerve of $\mathcal{C}$. The category $\mathcal{C}$ is called a \textbf{setoid} if it is equivalence to a set. It is not hard to see that this is true if and only if the canonical functor $\mathcal{C}\to\pi_0(N_\bullet(\mathcal{C}))$ (where $\pi_0(N_\bullet(\mathcal{C}))$ is considered as a category) is an equivalence of categories.
\end{example}
\subsection{Homotopy category of a simplicial set}
Recall that the functor $N_\bullet$ can be identified with $\Sing_\bullet^Q$, where $Q:\Delta\to\mathbf{Cat}$ is the functor sending $[n]\in\Delta$ to itself, considered as a category. By \cref{simplicial set Sing^Q functor adjoint}, $N_\bullet$ then admits a left adjoint, which is the functor that sends a simplicial set to its homotopy category as we shall now define.\par
Let $\mathcal{C}$ be a category. We will say that a map of simplicial sets $u:S_\bullet\to N_\bullet(\mathcal{C})$ exhibits $\mathcal{C}$ as the \textbf{homotopy category} of $S_\bullet$ if, for every category $\mathcal{D}$, the composite map
\[\Hom_{\mathbf{Cat}}(\mathcal{C},\mathcal{D})\to\Hom_{\mathbf{Set}_\Delta}(N_\bullet(\mathcal{C}),N_\bullet(\mathcal{D}))\stackrel{\circ u}{\to}\Hom_{\mathbf{Set}_\Delta}(S_\bullet,N_\bullet(\mathcal{D}))\]
is bijective (note that since $N_\bullet$ is fully faithful, the first map is always bijective).
\begin{example}\label{simplicial set Sing(X) homotopy cat is Pi(X)}
Let $X$ be a category and $\Sing_\bullet(X)$ be the singular simplicial set of $X$. We consider the fundamental groupoid $\Pi(X)$ of $X$, which is defined in the begining of this chapter. Then there exists a unique map of simplicial sets $u:\Sing_\bullet(X)\to N_\bullet(\Pi(X))$ with teh following properties:
\begin{itemize}
\item On $0$-simplicies, $u$ sends each point $x\in X$ to itself, considered as an object of $\Pi(X)$.
\item On $1$-simplicies, $u$ sends a path $p:[0,1]\to X$ (considered as an edge of $\Sing_\bullet(X)$) to its homotopy class $[p]$ (as a morphism of $\Pi(X)$).
\item On $n$-simplicies, $u$ sends a continuous map $|\Delta^n|\to X$ to the functor that associates each element $i\in[n]$ with $\sigma(i)$, and each morphism $(i\leq j)$ with the homotopy class of the path $\sigma_{ij}$ defined by the restriction of $\sigma$ on the edge from $i$ to $j$ in $|\Delta^n|$.
\end{itemize}
Moreover, for any category $\mathcal{C}$, the map
\[\Hom_{\mathbf{Set}_\Delta}(N_\bullet(\Pi(X)),N_\bullet(\mathcal{C}))\to\Hom_{\mathbf{Set}_\Delta}(\Sing_\bullet(X),N_\bullet(\mathcal{C}))\]
is bijective. In fact, any homotopy of paths $\sigma,\sigma'$ in $X$ can be written as a continuous map $\tau\in\Sing_2(X)$ such that $d_0(\tau)=s_0(x)$, $d_1(\tau)=\sigma$ and $d_2(\tau)=\sigma'$. If $f:\Sing_\bullet(X)\to N_\bullet(\mathcal{C})$ is a map of simplicial sets, then $f$ sends $\tau$ to a $2$-simplex in $N_\bullet(\mathcal{C})$, which means
\[f(\sigma)=\id_{f(x)}\circ f(\sigma'),\]
so $f$ is constant on homotopy classes. This indicates that $f$ descents to a uniquely determined map $N_\bullet(\Pi(X))\to N_\bullet(\mathcal{C})$ of simplicial sets, so we therefore conclude that $u$ exhibits the fundamental groupoid $\Pi(X)$ as a homotopy category of the singular simplicial set $\Sing_\bullet(X)$.
\end{example}
Let $S_\bullet$ be a simplicial set. It follows immediately from the definition that if there exists a category $\mathcal{C}$ and a map $u:S_\bullet\to N_\bullet(\mathcal{C})$ which exhibits $\mathcal{C}$ as a homotopy category of $S_\bullet$, then the category $\mathcal{C}$ is unique up to isomorphism and depends functorially on $S_\bullet$. To emphasize this dependence, we will refer to $\mathcal{C}$ as the homotopy category of $S_\bullet$ and denote it by $\ho S_\bullet$.
\begin{proposition}\label{simplicial set homotopy cat exist}
Let $S_\bullet$ be a simplicial set. Then there exists a category $\mathcal{C}$ and a map of simplicial sets $u:S_\bullet\to N_\bullet(\mathcal{C})$ which exhibits $\mathcal{C}$ as a homotopy category of $S_\bullet$.
\end{proposition}
\begin{proof}
Let $Q^\bullet$ denote the cosimplicial object of $\mathbf{Cat}$ given by the inclusion $\Delta\hookrightarrow\mathbf{Cat}$. Unwinding the definitions, we see that a homotopy category of $S_\bullet$ can be identified with a realization $|S_\bullet|^Q$, whose existence is a special case of \cref{simplicial set Sing^Q functor adjoint}. Alternatively, we can give a direct construction of the homotopy category $\ho S_\bullet$:
\begin{itemize}
\item The objects of $\ho S_\bullet$ are the vertices of $S_\bullet$.
\item Every edge $e$ of $S_\bullet$ determines a morphism $[e]$ in $\ho S_\bullet$, whose source is the vertex $d_1(e)$ and whose target is the vertex $d_0(e)$.
\item The collection of morphisms in $\ho S_\bullet$ is generated under composition by morphisms of the form $[e]$, subject only to the relations
\begin{gather*}
[s_0(x)]=\id_x\for x\in S_0,\\
[d_1(\sigma)]=[d_0(\sigma)]\circ[d_2(\sigma)]\for \sigma\in S_2.
\end{gather*}
\end{itemize}
As in \cref{simplicial set Sing(X) homotopy cat is Pi(X)}, it is easy to verify that the defined class $\ho S_\bullet$ is indeed a category and the canonical map $u:S_\bullet\to N_\bullet(\ho S_\bullet)$ exhibits $\ho S_\bullet$ as a homotopy category of $S_\bullet$. 
\end{proof}
\begin{corollary}\label{simplicial set nerve and homotopy cat adjoint}
The nerve functor $N_\bullet:\mathbf{Cat}\to\mathbf{Set}_\Delta$ admits a left adjoint, given on $S_\bullet\to \ho S_\bullet$.
\end{corollary}
\begin{remark}
Let $\mathcal{C}$ be a category. Then the counit of the adjunction in \cref{simplicial set nerve and homotopy cat adjoint} induces an isomorphism of categories $\ho N_\bullet(\mathcal{C})\cong\mathcal{C}$ (this is a restatement of \cref{simplicial set nerve of cat fully faithful}). In other words, every category $\mathcal{C}$ can be recovered as the homotopy category of its nerve $N_\bullet(\mathcal{C})$.
\end{remark}
\begin{remark}
Let $S_\bullet$ be a simplicial set. Our proof of \cref{simplicial set homotopy cat exist} gives a construction of the homotopy category $\ho S_\bullet$ by generators and relations. The result of this construction is not easy to describe. If $x$ and $y$ are vertices of $S_\bullet$, then every morphism from $x$ to $y$ in $\ho S_\bullet$ can be represented by a composition
\[[e_n]\circ[e_{n-1}]\circ\cdots\circ[e_1]\]
where $\{e_i\}_{0\leq i\leq n}$ is a sequence of edges satisfying
\[d_1(e_1)=x,\quad d_0(e_i)=d_1(e_{i+1}),\quad d_0(e_n)=y.\]
In general, it can be difficult to determine whether or not two such compositions represent the same morphism of $\ho S_\bullet$ (even for finite simplicial sets, this question is algorithmically undecidable). However, there are two situations in which the homotopy category $\ho S_\bullet$ admits a simpler description:
\begin{itemize}
\item Let $S_\bullet$ be a simplicial set of dimension $\leq 1$, which we can identify with a directed graph $G$ (\cref{simplicial set graph functor on dim 1 faithful}). In this case, the homotopy category $\ho S_\bullet$ is generated freely by the vertices and edges of the graph $G$: that is, it can be identified with the \textit{path category} of $G$.
\item Let $S_\bullet$ be an $\infty$-category (we will define this later). In this case, every morphism in the homotopy category $C=\ho S_\bullet$ can be represented by a single edge of $S_\bullet$, rather than a composition of edges (in other words, the canonical map $u:S_\bullet\to N_\bullet(\mathcal{C})$ is surjective on edges), and two edges of $S_\bullet$ represent the same morphism in $\ho S_\bullet$ if and only if they are homotopic. This leads to a more explicit description of the homotopy category $\mathcal{C}$ which generalize \cref{simplicial set Sing(X) homotopy cat is Pi(X)}.
\end{itemize}
\end{remark}
\subsection{Path category of a directed graph}
Let $S_\bullet$ be a simplicial set of dimension $\leq 1$. In this paragraph, we will show that the homotopy category $\ho S_\bullet$ admits a concrete description, which can be conveniently described using the language of directed graphs.\par
Let $G$ be a directed graph. For each edge $e\in E(G)$, we let $s(e),t(e)\in V(G)$ denote the source and target of $e$, respectively. If $x$ and $y$ are vertices of $V(G)$, then a path from $x$ to $y$ is a sequence of edges $(e_n,e_{n-1},\dots,e_1)$ satisfying
\[s(e_1)=x,\quad t(e_i)=s(e_{i+1}),\quad t(e_n)=y.\]
 By convention, we regard the empty sequence of edges as a path from each vertex $x\in V(G)$. to itself. We define a category $P[G]$ as follows:
\begin{itemize}
\item The objects of $P[G]$ are the vertices of $G$.
\item For every pair of vertices $x,y\in V(G)$, we let $\Hom_{P[G]}(x,y)$ denote the set of all paths $(e_n,\dots,e_1)$ from $x$ to $y$.
\item For every vertex $x\in V(G)$, the identity morphism $\id_x$ in the category $P[G]$ is the empty path from $x$ to itself.
\item Let $x,y,z\in V(G)$. The composition law
\[\circ:\Hom_{P[G]}(y,z)\times\Hom_{P[G]}(x,y)\to\Hom_{P[G]}(x,z)\]
is described by the formula
\[(e_n,\dots,e_1)\circ(t_m,\dots,t_1)=(e_n,\dots,e_1,t_m,\dots,t_1).\]
In other words, composition in $P[G]$ is given by concatentation of paths.
\end{itemize}
The category $P[G]$ is called the \textbf{path category} of the directed graph $G$.
\begin{example}
Fix an integer $n\geq 0$ and let $G$ be the directed graph with vertex set $V(G)=\{v_0,v_1,\dots,v_n\}$, and edge set $E(G)=\{e_1,\dots,e_n\}$, where each edge $e_i$ has source $s(e_i)=v_{i-1}$ and target $t(e_i)=v_i$; we can represent $G$ graphically by the diagram
\[\begin{tikzcd}
v_0\ar[r,"e_1"]&v_1\ar[r,"e_2"]&\cdots\ar[r,"e_{n-1}"]&v_{n-1}\ar[r,"e_n"]&v_n.
\end{tikzcd}\]
Let $v_i$ and $v_j$ be a pair of vertices of $G$, then there is a (necessarily unique) path from $v_i$ to $v_j$ if and only if $i\leq j$, given by $(e_j,e_{j-1},\dots,e_{i+1})$. It follows that the path category $P[G]$ is isomorphic to the linearly ordered set $[n]$ (regarded as a category).
\end{example}
\begin{example}
Let $G$ be a directed graph having a single vertex $V(G)=\{x\}$. Then the path category $P[G]$ has a single object $x$, and can therefore be identified with the category $BM$ associated to the monoid $M=\End_{P[G]}(x)=\Hom_{P[G]}(x,x)$. Note that the elements of $M$ can be identified with (possibly empty) sequences of elements of the set $E(G)$, and that the multiplication on $M$ is given by concatenation of sequences. In other words, $M$ can be identified with the \textit{free monoid} generated by the set $E(M)$ (this identification is not completely tautological: it can be regarded as a special case of \cref{simplicial set graph homotopy cat is path cat} below).
\end{example}
\begin{example}
Let $G$ be a directed graph having a single vertex $V(G)=\{x\}$ and a single edge $E(G)=\{e\}$ (necessarily satisfying $s(e)=x=t(e)$). Then the path category $P[G]$ has a single object $x$ whose endomorphism monoid $\End_{P[G]}(x)$ can be identified with the set $\N$ of nonnegative integers (with monoid structure given by addition).
\end{example}
Let $G$ be a directed graph, and let $G_\bullet$ denote the associated $1$-dimensional simplicial set. Then there is an evident map of simplicial sets $u:G_\bullet\to N_\bullet(P[G])$, which carries each vertex $v\in V(G)$ to itself and each edge $e\in E(G)$ to the path consisting of the single edge $e$.
\begin{proposition}\label{simplicial set graph homotopy cat is path cat}
Let $G$ be a directed graph. Then the map $u:G_\bullet\to N_\bullet(P[G])$ exhibits $P[G]$ as the homotopy category of the simplicial set $G_\bullet$. In other words, for every category $\mathcal{C}$, the composite map
\[\Hom_{\mathbf{Cat}}(P[G],\mathcal{C})\to\Hom_{\mathbf{Set}_\Delta}(N_\bullet(P[G]),N_\bullet(\mathcal{C}))\stackrel{\circ u}{\to}\Hom_{\mathbf{Set}_\Delta}(G_\bullet,N_\bullet(\mathcal{C}))\]
is a bijection.
\end{proposition}
\begin{proof}
Let $f:G_\bullet\to N_\bullet(\mathcal{C})$ be a map of simplicial sets. We want to show that there is a unique functor $F:P[G]\to\mathcal{C}$ for which the composite map
\[\begin{tikzcd}
G_\bullet\ar[r,"u"]&N_\bullet(P[G])\ar[r,"N_\bullet(F)"]&N_\bullet(\mathcal{C})
\end{tikzcd}
\]
coincides with $f$. Unwinding the definitions, we see that this agreement imposes the following requirements on $F$:
\begin{itemize}
\item[(a)] For each vertex $v\in V(G)$, we have $F(x)=f(x)$ (as an object in $\mathcal{C}$).
\item[(b)] For each edge $e\in E(G)$ having $x=s(e)$ and target $y=t(e)$, the functor $F$ carries the path $(e)$ to the morphism $f(e):f(x)\to f(y)$ in $\mathcal{C}$.
\end{itemize}
The existence and uniqueness of the functor $F$ is now clear: it is determined on objects by property (a), and on morphisms by the formula
\begin{equation*}
F(e_n,e_{n-1},\dots,e_1)=f(e_n)\circ f(e_{n-1})\circ\cdots\circ f(e_1).\qedhere
\end{equation*}
\end{proof}
A category $\mathcal{C}$ is \textbf{free} if it is isomorphic to $P[G]$, for some directed graph $G$. We now give a characterization of those categories which are free.\par
Let $\mathcal{C}$ be any category. We say that a morphism $f:X\to Y$ in $\mathcal{C}$ is \textbf{indecomposable} if $f$ is not an identity morphism, and for every factorization $f=g\circ h$ have either $g=\id_Y$ (so $h=f$) or $h=\id_X$ (so $g=f$). We now define a directed graph $\Gamma_0(\mathcal{C})$ as follows:
\begin{itemize}
\item The vertices of$\Gamma_0(\mathcal{C})$ are the objects of $\mathcal{C}$.
\item The edges of $\Gamma_0(\mathcal{C})$ are the indecomposable morphisms of $\mathcal{C}$ (where an indecomposable morphism $f:X\to Y$ is regarded as an edge with source $s(f)=X$ and target $t(f)=Y$).
\end{itemize}
By definition, the graph $\Gamma_0(\mathcal{C})$ comes equipped with a canonical map $\Gamma_0(\mathcal{C})_\bullet\to N_\bullet(\mathcal{C})$, which we can identify (by means of \cref{simplicial set graph homotopy cat is path cat}) with a functor $F:P[\Gamma_0(\mathcal{C})]\to\mathcal{C}$.
\begin{proposition}
Let $\mathcal{C}$ be a category. The following conditions on $\mathcal{C}$ are equivalent:
\begin{itemize}
\item[(\rmnum{1})] The category $\mathcal{C}$ is free. That is, there exists a directed graph $G$ and an isomorphism of categories $C\cong P[G]$.
\item[(\rmnum{2})] The functor $F:P[\Gamma_0(\mathcal{C})]\to\mathcal{C}$ is an isomorphism of categories.
\item[(\rmnum{3})] The functor $F:P[\Gamma_0(\mathcal{C})]\to\mathcal{C}$ is an equivalence of categories.
\item[(\rmnum{4})] The functor $F:P[\Gamma_0(\mathcal{C})]\to\mathcal{C}$ is fully faithful.
\item[(\rmnum{5})] Every morphism $f$ in $\mathcal{C}$ admits a unique factorization $f=f_n\circ f_{n-1}\circ\cdots\circ f_1$, where each $f_i$ is an indecomposable morphism of $\mathcal{C}$.
\end{itemize}
\end{proposition}
\begin{proof}
The functor $F$ is bijective on objects, which shows that (\rmnum{2})$\Leftrightarrow$(\rmnum{3})$\Leftrightarrow$(\rmnum{4}). The equivalence of (\rmnum{4}) and (\rmnum{5}) follows from the definition of morphisms in the path category $P[\Gamma_0(\mathcal{C})]$. The implication (\rmnum{2})$\Rightarrow$(\rmnum{1}) is immediate, and the converse follows from the observation that a path $(e_n,e_{n-1},\dots,e_1)$ in $P[G]$ is indecomposable if and only if $n=1$.
\end{proof}
\section{\boldmath\texorpdfstring{$\infty$}{inf}-categories}
Up to now, we have encountered two closely related conditions on a simplicial set $S_\bullet$:
\begin{enumerate}[leftmargin=40pt]
\item[(Kan)] For any positive integer $n>0$ and $0\leq i\leq n$, any map of simplicial sets $\sigma_0:\Lambda^n_i\to S_\bullet$ can be extended into a map $\sigma:\Delta^n\to S_\bullet$, where $\Lambda^n_i$ is the $i$-th horn.
\item[(Ner)] For any pair of integers $0<i<n$, any map of simplicial sets $\sigma_0:\Lambda_i^n\to S_\bullet$ can be uniquely extended into a map $\sigma:\Delta^n\to S_\bullet$.
\end{enumerate}
Simplicial sets satisfying (Kan) are called Kan complexes and form the basis for a combinatorial approach to homotopy theory, while simplicial sets satisfying (Ner) can be identified with categories. These notions admit a common generalization, which we shall define now.\par
An \textbf{$\infty$-category} is a simplicial set $S_\bullet$ which satisfies the following weak Kan extension condition:
\begin{enumerate}[leftmargin=40pt]
\item[(Kan')] For any pair of integers $0<i<n$, any map of simplicial sets $\sigma_0:\Lambda^n_i\to S_\bullet$ can be extended into a map $\sigma:\Delta^n\to S_\bullet$, where $\Lambda^n_i$ is the $i$-th horn.
\end{enumerate}
It follows from this definition that every Kan complex is an $\infty$-category. In particular, if $X$ is a topological space, then the singular simplicial set $\Sing_\bullet(X)$ is an $\infty$-category. Also, for every category $\mathcal{C}$, the nerve $N_\bullet(\mathcal{C})$ is an $\infty$-category. 
\begin{example}[\textbf{Product of $\infty$-categories}]\label{simplicial set product of inf-cat}
Let $\{S_\bullet^\alpha\}_{\alpha\in I}$ be a collection of simplicial sets and $S_\bullet=\prod_{\alpha}S_\bullet^\alpha$ denote their product. If each $S_\bullet^\alpha$ is an $\infty$-category, then $S_\bullet$ is an 1-category. The converse holds provided that each $S_\bullet^\alpha$ is nonempty.
\end{example}
\begin{example}[\textbf{Coproducts of $\infty$-categories}]\label{simplicial set coproduct of inf-cat}
Let $\{S_\bullet^\alpha\}_{\alpha\in I}$ be a collection of simplicial sets parametrized by a set $I$, and $S_\bullet=\coprod_{\alpha}S_\bullet^\alpha$ be their coproduct. For each $0<i<n$, the restriction map
\[\theta:\Hom_{\mathbf{Set}_\Delta}(\Delta^n,S_\bullet)\to\Hom_{\mathbf{Set}_\Delta}(\Lambda_i^n,S_\bullet)\]
can be identified with the coproduct (formed in the arrow category $\Fun([1],\mathbf{Set}))$ of restriction maps $\theta_\alpha:\Hom_{\mathbf{Set}_\Delta}(\Delta^n,S_\bullet^\alpha)\to\Hom_{\mathbf{Set}_\Delta}(\Lambda^n_i,S_\bullet^\alpha)$ (this follows from the observation that
the simplicial sets $\Delta^n$ and $\Lambda_i^n$ are connected). It follows that $\theta$ is surjective if and only if each $\theta_\alpha$ is surjective. Allowing $n$ and $i$ to vary, we conclude that $S_\bullet$ is an $\infty$-category if and only if each summand $S_\bullet^\alpha$ is an $\infty$-category. In particular, if $S_\bullet$ is a simplicial set, then $S_\bullet$ is an $\infty$-category if and only if each connected component of $S_\bullet$ is an $\infty$-category.
\end{example}
\begin{remark}
Suppose that we are given a filtered diagram of simplicial sets $\{S(\alpha)_\bullet\}$ having colimit $S_\bullet=\rlim S(\alpha)_\bullet$. If each $S(\alpha)_\bullet$ is an $\infty$-category, then $S_\bullet$ is an $\infty$-category.
\end{remark}
We will generally use calligraphic letters (like $\mathcal{C}$, $\mathcal{D}$, and $\mathcal{E}$) to denote $\infty$-categories, and we will generally describe them using terminology borrowed from category theory. For example, if $C=S_\bullet$ is an $\infty$-category, then we will refer to vertices of the simplicial set $S_\bullet$ as objects of the $\infty$-category $\mathcal{C}$, and to edges of the simplicial set $S_\bullet$ as morphisms of the $\infty$-category $\mathcal{C}$. One of the central themes of this exposition is that $\infty$-categories behave much like ordinary categories. In particular, for any $\infty$-category $\mathcal{C}$, there is a notion of composition for morphisms of $\mathcal{C}$. Given a pair of morphisms $f:X\to Y$ and $g:Y\to X$ in $\mathcal{C}$ (corresponding to edges $f,g\in S_1$ satisfying $d_0(f)=d_1(g)$), the pair $(f,g)$ defines a map of simplicial sets $\sigma_0:\Lambda^2_1\to\mathcal{C}$. Applying condition (Kan'), we can extend $\sigma_0$ to a $2$-simplex $\sigma$ of $\mathcal{C}$, which we can think of heuristically as a commutative diagram
\[\begin{tikzcd}
&Y\ar[rd,"g"]&\\
X\ar[ru,"f"]\ar[rr,dashed,"h"]&&Z
\end{tikzcd}\]
In this case, the morphism $h=d_1(\sigma)$ is called a composition of $f$ and $g$. However, this comes with a caveat: the extension $\sigma$ is usually not unique (rather than the case in categories, where we have condition (Ner)), so the morphism $h$ is not completely determined by $f$ and $g$. However, we will show that it is unique up to a certain notion of homotopy, and we will apply this observation to give a concrete description of the homotopy category $\ho\mathcal{C}$ when $\mathcal{C}$ is an $\infty$-category.
\subsection{Objects and morphisms}
Let $\mathcal{C}=S_\bullet$ be an $\infty$-category. An \textbf{object} of $\mathcal{C}$ is a vertex of the simplicial set $S_\bullet$ (that is, an element of the set $S_0$). A morphism of $\mathcal{C}$ is an edge of the simplicial set $S_\bullet$ (that is, an element of $S_1$). If $f\in S_1$ is a morphism of $\mathcal{C}$, the objects $X=d_1(f)$ and $Y=d_0(f)$ are called the \textbf{source} and \textbf{target} of $f$, respectively. In this case, we say that $f$ is a morphism from $X$ to $Y$. For any object $X$ of $\mathcal{C}$, we can regard the degenerate edge $s_0(X)$ as a morphism from $X$ to itself; this is denoted by $\id_X$ and called the \textbf{identity morphism} of $X$. We often write $X\in\mathcal{C}$ to indicate that $X$ is an object of $\mathcal{C}$, and we use the phrase "$f:X\to Y$ is a morphism of $\mathcal{C}$" to indicate that $f$ is a morphism of $\mathcal{C}$ having source $X$ and target $Y$.
\begin{example}
Let $\mathcal{C}$ be an ordinary category, and regard the simplicial set $N_\bullet(\mathcal{C})$ as an $\infty$-category. Then the objects of the $\infty$-category $N_\bullet(\mathcal{C})$ are the objects of $\mathcal{C}$, and the morphisms of the $\infty$-category $N_\bullet(\mathcal{C})$ are the morphisms of $\mathcal{C}$. Moreover, the source and target of a morphism of $\mathcal{C}$ coincide with the source and target of the corresponding morphism in $N_\bullet(\mathcal{C})$. Also note that for every object $X\in\mathcal{C}$, the identity morphism $\id_X$ does not depend on whether we view $X$ as an object of the category $\mathcal{C}$ or the $\infty$-category $N_\bullet(\mathcal{C})$.
\end{example}
\begin{example}
Let $X$ be a topological space, and regard the simplicial set $\Sing_\bullet(X)$ as an $\infty$-category. Then the objects of $\Sing_\bullet(X)$ are the points of $X$, and the morphisms of $\Sing_\bullet(X)$ are continuous paths $f:[0,1]\to X$. The source of a morphism $f$ is the point $f(0)$, and the target is the point $f(1)$. Also, for every point $x\in X$, the identity morphism $\id_x$ is the constant path $[0,1]\to X$ taking the value $x$.
\end{example}
Let $\mathcal{C}$ be an ordinary category. Then we can construct a new category $\mathcal{C}^{\op}$, called the opposite category of $\mathcal{C}$, by the formula
\[\Hom_{\mathcal{C}^{\op}}(C,D)=\Hom_{\mathcal{C}}(D,C).\]
As we will see, the construction $\mathcal{C}\mapsto\mathcal{C}^{\op}$ admits a straightforward generalization to the setting of $\infty$-categories. In fact, it can be extended to arbitrary simplicial sets.\par
Let $\mathbf{Lin}$ denote the category whose objects are finite linearly ordered sets and whose morphisms are nondecreasing functions. Let $I$ be an object of $\mathbf{Lin}$, regarded as a set with a linear ordering $\preceq_I$. We let $I^{\op}$ denote the same set with the opposite ordering, so that
\[(i\preceq_{I^{\op}}j)\Leftrightarrow(j\preceq_I i).\]
Recall that the simplex category $\Delta$ is the full subcategory of $\mathbf{Lin}$ spanned by objects of the form $[n]$, and is equivalent to the full subcategory of $\mathbf{Lin}$ spanned by those linearly ordered sets which are finite and nonempty. There is a unique functor $\op:\Delta\to\Delta$ for which the diagram
\[\begin{tikzcd}
\Delta\ar[r]\ar[d,swap,"\op"]&\mathbf{Lin}\ar[d,"I\mapsto I^{\op}"]\\
\Delta\ar[r]&\mathbf{Lin}
\end{tikzcd}\]
commutes up to isomorphism, where the horizontal maps are given by the inclusion. The functor $\op$ can be described more concretely as follows:
\begin{itemize}
\item For each object $[n]\in\Delta$, we have $\op([n])=[n]$ (note that the maps $i\mapsto n-i$ determines an isomorphism of $[n]$ with the opposite linear ordering $[n]^{\op}$).
\item For each morphism $\alpha:[m]\to[n]$ in $\Delta$, the morphism $\op(\alpha):[m]\to[n]$ is given by $\op(\alpha)(i)=n-\alpha(m-i)$.
\end{itemize}
Now let $S_\bullet$ be a simplicial set, which we consider as a functor $\Delta^{\op}\to\mathbf{Set}$. We denote by $S_\bullet^\bullet$ the simplicial set given by the composition
\[\begin{tikzcd}
\Delta^{\op}\ar[r,"\op"]&\Delta^{\op}\ar[r,"S_\bullet"]&\mathbf{Set}
\end{tikzcd}\]
We say that $S_\bullet^{\op}$ is the \textbf{opposite} of the simplicial set $S_\bullet$. By definition, it is defined by the following datum:
\begin{itemize}
\item For each $n\geq 0$, we have $S_n^{\op}=S_n$.
\item The face and degeneracy maps of $S_\bullet^{\op}$ are given by
\[(d_i:S_n^{\op}\to S_{n-1}^{\op})=(d_{n-i}:S_n\to S_{n-1}),\quad (s_i:S_n^{\op}\to S_{n+1}^{\op})=(s_{n-i}:S_n\to S_{n+1}).\]
\end{itemize}
\begin{example}
Let $\mathcal{C}$ be a category. For each $n\geq 0$, we can identify $n$-simplices $\sigma$ of $N_\bullet(\mathcal{C})$ with diagrams
\[\begin{tikzcd}
C_0\ar[r,"f_1"]&C_1\ar[r,"f_2"]&\cdots\ar[r,"f_{n-1}"]&C_{n-1}\ar[r,"f_n"]&C_n
\end{tikzcd}\]
in the category $\mathcal{C}$. Then $\sigma$ determines an $n$-simplex $\sigma^{\op}$ of $N_\bullet(\mathcal{C}^{\op})$, given by the diagram
\[\begin{tikzcd}
C_n\ar[r,"f_n"]&C_{n-1}\ar[r,"f_{n-1}"]&\cdots\ar[r,"f_2"]&C_1\ar[r,"f_1"]&C_0
\end{tikzcd}\]
in the opposite category $\mathcal{C}^{\op}$. The construction $\sigma\mapsto\sigma^{\op}$ determines an isomorphism of simplicial sets $N_\bullet(\mathcal{C})^{\op}\cong N_\bullet(\mathcal{C}^{\op})$.
\end{example}
\begin{example}
Let $X$ be a topological space. Then there is a canonical isomorphism of simplicial set $\Sing_\bullet(X)\cong\Sing_\bullet(X)^{\op}$, which carries a singular $n$-simplex $\sigma:|\Delta^n|\to X$ to the composition map
\[\begin{tikzcd}
{|\Delta^n|}\ar[r,"r"]&{|\Delta^n|}\ar[r,"\sigma"]&X
\end{tikzcd}\]
where $r$ is the homeomorphism of $|\Delta^n|$ with itself given by $r(t_0,\dots,t_{n-1},t_n)=(t_n,\dots,t_1,t_0)$.
\end{example}
\begin{proposition}\label{simplicial set opposite of inf-cat is inf-cat}
Let $\mathcal{C}$ be an $\infty$-category, then the opposite category $\mathcal{C}^{\op}$ is also an $\infty$-category.
\end{proposition}
\begin{proof}
Let $\sigma_0:\Lambda_i^n\to\mathcal{C}^{\op}$ be a map of simplicial sets for $0<i<n$. Passing to opposite simplicial sets, we are reduced to showing that the map $\sigma_0^{\op}:(\Lambda_i^n)^{\op}\to\mathcal{C}$ can be extended to a map $(\Lambda_i^n)^{\op}\to\mathcal{C}$. This follows from our assumption that $\mathcal{C}$ is an $\infty$-category, since there is a canonical isomorphism $(\Delta^n)^{\op}\cong\Delta^n$ which carries the simplicial subset $(\Lambda^n_i)^{\op}$ to $\Lambda^n_{n-i}$.
\end{proof}
Let $\mathcal{C}$ be an $\infty$-category. The $\infty$-category of \cref{simplicial set opposite of inf-cat is inf-cat} is called the opposite of the $\infty$-category $\mathcal{C}$. Note that the objects of $\mathcal{C}^{\op}$ are the objects of $\mathcal{C}$, and given a pair of objects $X,Y\in\mathcal{C}$, the datum of a morphism from $X$ to $Y$ in $\mathcal{C}^{\op}$ is equivalent to the datum of a morphism from $Y$ to $X$ in $\mathcal{C}$.
\subsection{Homotopies of morphisms}
For any topological space $X$, we can view the singular simplicial set $\Sing_\bullet(X)$ as an $\infty$-category, where a morphism from a point $x\in X$ to a point $y\in X$ is given by a continuous path $f:[0,1]\to X$ satisfying $f(0)=x$ and $f(1)=y$. For many purposes (for example, in the study of the fundamental group $\pi_1(X,x)$), it is useful to work not with paths but with homotopy classes of paths (having fixed endpoints). This notion can be generalized to an arbitrary $\infty$-category: let $\mathcal{C}$ be an $\infty$-category and let $f,g:C\to D$ be a pair of morphisms in $\mathcal{C}$ having the same source and target. A \textbf{homotopy} from $f$ to $g$ is a $2$-simplex $\sigma$ of $\mathcal{C}$ satisfying $d_0(\sigma)=\id_D$, $d_1(\sigma)=g$, and $d_2(\sigma)=f$, as depicted in the diagram
\[\begin{tikzcd}
&D\ar[rd,"\id_D"]&\\
C\ar[ru,"f"]\ar[rr,"g"]&&D
\end{tikzcd}\]
We say that $f$ and $g$ are homotopic if there exists such a homotopy from $f$ to $g$.
\begin{example}
Let $\mathcal{C}$ be an ordinary category. Then a pair of morphisms $f,g:C\to D$ in $\mathcal{C}$ (having the same source and target) are homotopic as morphisms of the $\infty$-category $N_\bullet(\mathcal{C})$ if and only if $f=g$.
\end{example}
\begin{example}\label{simplicial set Sing(X) homotopy iff path homotopy}
Let $X$ be a topological space. Suppose we are given points $x,y\in X$ and a pair of continuous paths $f,g:[0,1]\to X$ satisfying $f(0)=x=g(0)$ and $f(1)=y=g(1)$. Then $f$ and $g$ are homotopic as morphisms of the $\infty$-category $\Sing_\bullet(X)$ if and only if the paths $f$ and $g$ are homotopic relative to their endpoints: that is, if and only if there exists a continuous function $H:[0,1]\times[0,1]\to X$ satisfying
\begin{align}\label{simplicial set Sing(X) homotopy iff path homotopy-1}
H(s,0)=f(s),\quad H(s,1)=g(s),\quad H(0,t)=x,\quad H(1,t)=y.
\end{align}
To see this, let $\pi:[0,1]\times[0,1]\to|\Delta^2|$ be the continuous map given by
\[\pi(s,t)=(1-s,(1-t)s,ts).\]
Intuitively, one can think that $\pi$ pinch the edge $\{0\}\times[0,1]$ into a single point $(1,0,0)$ to give a $2$-simplex $|\Delta^2|$. The map $\sigma\mapsto\sigma\circ\pi$ then determines a map from the set $\Sing_2(X)$ of singular $2$-simplices of $X$ to the set of all continuous functions $H:[0,1]\times[0,1]\to X$, and sends homotopies from $f$ to $g$ (in the category $\Sing_\bullet(X)$) to the set of continuous functions $H$ satisfying the requirements (\ref{simplicial set Sing(X) homotopy iff path homotopy-1}). Conversely, if $H:[0,1]\times[0,1]\to X$ is a continuous function satisfying (\ref{simplicial set Sing(X) homotopy iff path homotopy-1}), then we define a $2$-simplex $\sigma:|\Delta^2|\to X$ by the formula
\[\sigma(t_0,t_1,t_2)=H(\pi^{-1}(t_0,t_1,t_2)).\]
Note that the map $\pi$ is not injective only on $\{0\}\times[0,1]$, on which $H$ takes constant values, so the map $\sigma$ is well-defined. To see $\sigma$ is continuous, it is enough to note that $\pi$ sends an open subset $U$ of $[0,1]\times[0,1]$ intersecting $\{0\}\times[0,1]$ to an open subset of $|\Delta^2|$ containing $(1,0,0)$.
\end{example}
\begin{proposition}\label{simplicial set inf-cat homotopy is equivalence}
Let $\mathcal{C}$ be an $\infty$-category contiaining objects $X,Y\in\mathcal{C}$, and let $E$ be the collection of morphisms from $X$ to $Y$. Then homotopy is an equivalence relation on $E$.
\end{proposition}
\begin{proof}
We first observe that for any morphism $f:X\to Y$ in $\mathcal{C}$, the degenerate $2$-simplex $s_1(f)$ is a homotopy from $f$ to itself. It follows that homotopy is a reflexive relation on $E$. We will complete the proof by establishing the following:
\begin{equation}\label{simplicial set inf-cat homotopy is equivalence-1}
\parbox{\dimexpr\linewidth-6em}
{\strut
Let $f,g,h:X\to Y$ be three morphisms from $X$ to $Y$. If $f$ is homotopic to $g$ and $f$ is homotopic to $h$, then $g$ is homotopic to $h$.
\strut}
\end{equation}
Let us first observe that assertion (\ref{simplicial set inf-cat homotopy is equivalence-1}) implies \cref{simplicial set inf-cat homotopy is equivalence}. Note that in the special case $f=h$, (\ref{simplicial set inf-cat homotopy is equivalence-1}) asserts that if $f$ is homotopic to $g$, then $g$ is homotopic to $f$ (since $f$ is always homotopic to itself). That is, the relation of homotopy is symmetric. We can therefore replace the hypothesis that $f$ is homotopic to $g$ in assertion (\ref{simplicial set inf-cat homotopy is equivalence-1}) by the hypothesis that $g$ is homotopic to $f$, so that (\ref{simplicial set inf-cat homotopy is equivalence-1}) is equivalent to the transitivity of the relation of homotopy.\par
It remains to prove (\ref{simplicial set inf-cat homotopy is equivalence-1}). Let $\sigma_2$ and $\sigma_3$ be $2$-simplices of $\mathcal{C}$ which are homotopies from $f$ to $h$ and $f$ to $g$, respectively, and let $\sigma_0$ be the $2$-simplex given by the constant map $\Delta^2\to\Delta^0\stackrel{Y}{\to}\mathcal{C}$. Then the tuple $(\sigma_0,\ast,\sigma_2,\sigma_3)$ determines a map of simplicial sets $\tau_0:\Lambda_1^3\to\mathcal{C}$, depicted informally by the diagram
\[\begin{tikzcd}[row sep=15mm,column sep=15mm]
&Y\ar[r,"\id_Y"]&Y\ar[rd,dashed,"\id_Y"]&\\
X\ar[ru,"f"]\ar[rru,dashed,"g"{anchor=south}]\ar[rrr,dashed,"h"]&&&Y\ar[llu,crossing over,leftarrow,swap,"\id_Y"{anchor=south}]
\end{tikzcd}\]
where the dotted arrows indicate the boundary of the “missing” face of the horn $\Lambda^3_1$. Our hypothesis that $\mathcal{C}$ is an $\infty$-category guarantees that $\tau_0$ can be extended to a $3$-simplex $\tau$ of $\mathcal{C}$. We can then regard the face $d_1(\tau)$ as a homotopy from $g$ to $h$.
\end{proof}
Note that there is a potential asymmetry in the definition of homotopy: if $f,g:X\to Y$ are two morphisms in an $\infty$-category $\mathcal{C}$, then the datum of a homotopy from $f$ to $g$ in the $\infty$-category $\mathcal{C}$ is not equivalent to the datum of a homotopy from $f$ to $g$ in the opposite $\infty$-category $\mathcal{C}^{\op}$. Nevertheless, we have the following:
\begin{proposition}\label{simplicial set inf-cat homotopy symmetric}
Let $\mathcal{C}$ be an $\infty$-category, and let $f,g:X\to Y$ be morphisms of $\mathcal{C}$ having the same source and target. Then $f$ and $g$ are homotopic if and only if they are homotopic when regarded as morphisms of the opposite $\infty$-category $\mathcal{C}^{\op}$. In other words, the following conditions are equivalent:
\begin{itemize}
\item[(a)] There exists a $2$-simplex $\sigma$ of $\mathcal{C}$ satisfying $d_0(\sigma)=\id_Y$, $d_1(\sigma)=g$, and $d_2(\sigma)=f$, as depicted in the diagram
\[\begin{tikzcd}
&Y\ar[rd,"\id_Y"]&\\
X\ar[ru,"f"]\ar[rr,"g"]&&Y
\end{tikzcd}\] 
\item[(b)] There exists a $2$-simplex $\sigma$ of $\mathcal{C}$ satisfying $d_0(\sigma)=f$, $d_1(\sigma)=g$, and $d_2(\sigma)=\id_X$, as depicted in the diagram
\[\begin{tikzcd}
&X\ar[rd,"f"]&\\
X\ar[ru,"\id_X"]\ar[rr,"g"]&&Y
\end{tikzcd}\] 
\end{itemize}
\end{proposition}
\begin{proof}
We will show that (a) implies (b); the proof of the reverse implication is similar. Assume that $f$ is homotopic to $g$. Since the relation of homotopy is symmetric (\cref{simplicial set inf-cat homotopy is equivalence}), it follows that $g$ is also homotopic to $f$. Let $\sigma$ be a homotopy from $g$ to $f$. Then we can regard the tuple of $2$-simplices $(\sigma,s_1(g),\ast,s_0(g))$ as a map of simplicial sets $\rho_0:\Lambda^3_2\to\mathcal{C}$, depicted informally in the diagram
\[\begin{tikzcd}[row sep=15mm,column sep=15mm]
&X\ar[r,"g"]&Y\ar[rd,"\id_Y"]&\\
X\ar[ru,dashed,"\id_X"]\ar[rru,"g"{anchor=south}]\ar[rrr,dashed,"g"]&&&Y\ar[llu,crossing over,dashed,leftarrow,swap,"f"{anchor=south}]
\end{tikzcd}\]
Using our assumption that $\mathcal{C}$ is an $\infty$-category, we can extend $\rho_0$ to a $3$-simplex $\rho$ of $\mathcal{C}$. Then the face $\tau=d_2(\rho)$ has the properties required by (b).
\end{proof}
\begin{corollary}\label{simplicial set homotopy iff Delta^1 x Delta^1}
Let $\mathcal{C}$ be an $\infty$-category, and let $f,g:X\to Y$ be morphisms of $\mathcal{C}$ having the same source and target. Then $f$ and $g$ are homotopic if and only if there exists a map of simplicial sets $H:\Delta^1\times\Delta^1\to\mathcal{C}$ satisfying
\[H|_{\{0\}\times\Delta^1}=f,\quad H|_{\{1\}\times\Delta^1}=g,\quad H|_{\Delta^1\times\{0\}}=\id_X,\quad H|_{\Delta^1\times\{1\}}=\id_Y\]
as indicated in the diagram
\begin{equation}\label{simplicial set homotopy iff Delta^1 x Delta^1-1}
\begin{tikzcd}
X\ar[dd,swap,"f"]\ar[rrdd,"h"{anchor=south}]\ar[rr,"\id_X"]&{}&X\ar[dd,"g"]\ar[ld,phantom,"\tau"description]\\
&{}&\\
Y\ar[rr,"\id_X"]\ar[ru,phantom,"\sigma"description]&&Y
\end{tikzcd}
\end{equation}
\end{corollary}
\begin{proof}
The "only if" direction is clear: if $\sigma$ is a homotopy from $f$ to $g$, then we can extend $\sigma$ to a map $H:\Delta^1\times\Delta^1\to\mathcal{C}$ by taking $\tau$ to be the degenerate simplex $s_0(g)$. Conversely, suppose that there exists a map $H:\Delta^1\times\Delta^1\to\mathcal{C}$, as indicated in the diagram (\ref{simplicial set homotopy iff Delta^1 x Delta^1-1}). Then the $2$-simplex $\sigma$ is a homotopy from $f$ to $h$, and the $2$-simplex $\tau$ guarantees that $g$ is homotopic to $h$ (by virtue of \cref{simplicial set inf-cat homotopy symmetric}). Since homotopy is an equivalence relation (\cref{simplicial set inf-cat homotopy is equivalence}), it follows that $f$ is homotopic to $g$. 
\end{proof}
\subsection{Compositions of morphisms in \texorpdfstring{$\infty$}{inf}-categories}
Let $\mathcal{C}$ be an $\infty$-category. Suppose that we are given objects $X,Y,Z\in\mathcal{C}$ and morphisms $f:X\to Y$, $g:Y\to Z$, and $h:X\to Z$. We say that $h$ is a composition of $f$ and $g$ if there exists a $2$-simplex $\sigma$ of $C$ satisfying $d_0(\sigma)=g$, $d_1(\sigma)=h$, and $d_2(\sigma)=f$, as depicted in the diagram
\[\begin{tikzcd}
&Y\ar[rd,"g"]&\\
X\ar[ru,"f"]\ar[rr,"h"]&&Z
\end{tikzcd}\]
In this case, we also say that the $2$-simplex $\sigma$ \textbf{witnesses} $h$ as a composition of $f$ and $g$. From condition (Kan'), we see that the morphism $h$ is \textit{not} uniquely determined by $f$ and $g$. However, we shall see that it is determined up to homotopy:
\begin{proposition}\label{simplicial set inf-cat composition unique up to homotopy}
Let $\mathcal{C}$ be an $\infty$-category containing morphisms $f:X\to Y$ and $g:Y\to Z$.
\begin{itemize}
\item[(a)] There exists a morphism $h:X\to Z$ which is a composition of $f$ and $g$.
\item[(b)] Let $h:X\to Z$ be a composition of $f$ and $g$, and let $h':X\to Z$ be another morphism in $\mathcal{C}$ having the same source and target. Then $h'$ is a composition of $f$ and $g$ if and only if $h'$ is homotopic to $h$.
\end{itemize}
\end{proposition}
\begin{proof}
The tuple $(g,\ast,f)$ determines a map of simplicial sets $\sigma_0:\Lambda^2_1\to\mathcal{C}$. Since $\mathcal{C}$ is an $\infty$-category, we can extend $\sigma_0$ to a $2$-simplex $\sigma$ of $\mathcal{C}$. Then $\sigma$ witnesses the morphism $h=d_1(\sigma)$ as a composition of $f$ and $g$, so this proves (a). To prove (b), let us first suppose that $h':X\to Z$ is some other morphism in $\mathcal{C}$ which is a composition of $f$ and $g$. Choose a $2$-simplex $\sigma'$ which witnesses $h'$ as a composition of $f$ and $g$. Then the tuple $(s_1(g),\ast,\sigma',\sigma)$ determines a morphism of simplicial sets $\tau_0:\Lambda^3_1\to\mathcal{C}$, which we depict informally as a diagram
\[\begin{tikzcd}[row sep=15mm,column sep=15mm]
&Y\ar[r,"g"]&Z\ar[rd,dashed,"\id_Z"]&\\
X\ar[ru,"f"]\ar[rru,dashed,"h"{anchor=south}]\ar[rrr,dashed,"h'"]&&&Z\ar[llu,crossing over,leftarrow,swap,"g"{anchor=south}]
\end{tikzcd}\]
Using our assumption that $\mathcal{C}$ is an $\infty$-category, we can extend $\tau_0$ to a $3$-simplex $\tau$ of $\mathcal{C}$. Then the face $d_1(\tau)$ is a homotopy from $h$ to $h'$.\par
We now prove the converse. Let $\sigma$ be a $2$-simplex of $\mathcal{C}$ which witnesses $h$ as a composition of $f$ and $g$, and let $h':X\to Z$ be a morphism of $\mathcal{C}$ which is homotopic to $h$. Let $\sigma'$ be a $2$-simplex of $\mathcal{C}$ which is a homotopy from $h$ to $h'$. Then the tuple $(s_1(g),\sigma',\ast,\sigma)$ determines a map of simplicial sets $\rho_0:\Lambda^3_2\to\mathcal{C}$, which we depict informally as a diagram
\[\begin{tikzcd}[row sep=15mm,column sep=15mm]
&Y\ar[r,"g"]&Z\ar[rd,"\id_Z"]&\\
X\ar[ru,dashed,"f"]\ar[rru,"h"{anchor=south}]\ar[rrr,dashed,"h'"]&&&Z\ar[llu,crossing over,leftarrow,dashed,swap,"g"{anchor=south}]
\end{tikzcd}\]
Our assumption that $\mathcal{C}$ is an $\infty$-category guarantees that we can extend $\rho_0$ to a $3$-simplex $\rho$ of $\mathcal{C}$. Then the face $d_2(\rho)$ witnesses $h'$ as a composition of $f$ and $g$.
\end{proof}
Let $\mathcal{C}$ be an $\infty$-category and let $f:X\to Y$ and $g:Y\to Z$ be a pair of morphisms in $\mathcal{C}$. We will write $h=g\circ f$ to indicate that $h$ is a composition of $f$ and $g$. In this case, it should be implicitly understood that we have chosen a $2$-simplex that witnesses $h$ as a composition of $f$ and $g$. We sometimes abuse terminology by referring to $h$ as the composition of $f$ and $g$. However, the reader should beware that only the homotopy class of $h$ is well-defined (\cref{simplicial set inf-cat composition unique up to homotopy}).
\begin{example}
Let $\mathcal{C}$ be an ordinary category containing a pair of morphisms $f:X\to Y$ and $g:Y\to Z$. Then there is a unique morphism $h:X\to Z$ in the $\infty$-category $N_\bullet(\mathcal{C})$ which is a composition of $f$ and $g$, given by the usual composition $g\circ f$ in the category $\mathcal{C}$.
\end{example}
\begin{example}
Let $X$ be a topological space and suppose we are given continuous paths $f,g:[0,1]\to X$ which are composable in the sense that $f(1)=g(0)$, and let $g\ast f:[0,1]\to X$ denote the path obtained by concatenating $f$ and $g$, given concretely by the formula
\[(g\ast f)(t)=\begin{cases}
f(2t)&\text{if $t\in[0,1/2]$},\\
g(2t-1)&\text{if $t\in[1/2,1]$}.
\end{cases}\]
Then $g\ast f$ is a composition of $f$ and $g$ in the $\infty$-category $\Sing_\bullet(X)$. More precisely, the continuous map $\sigma:|\Delta^2|\to X$ given by
\[\sigma(t_0,t_1,t_2)=\begin{cases}
f(t_1+2t_2)&\text{if $t_0\geq t_2$},\\
g(t_2-t_0)&\text{if $t_0\leq t_2$}.
\end{cases}\]
can be regarded as a $2$-simplex of $\Sing_\bullet(X)$ which witnesses $g\ast f$ as a composition of $f$ and $g$.\par
However, note that the concatenation $g\ast f$ is not the only path which is a composition of $f$ and $g$ in the $\infty$-category $\Sing_\bullet(X)$. Any path in $X$ which is path homotopic to $g\ast f$ has the same property, by virtue of \cref{simplicial set inf-cat composition unique up to homotopy} and \cref{simplicial set Sing(X) homotopy iff path homotopy}. For example, we can replace $g\ast f$ by a reparametrization, such as the path
\[h(s)=\begin{cases}
f(3s)&\text{if $s\in[0,1/3]$},\\
g(\frac{3}{2}s-\frac{1}{2})&\text{if $s\i[1/3,1]$}.
\end{cases}\]
When viewing $\Sing_\bullet(X)$ as an $\infty$-category, all of these paths have an equal claim to be regarded as "the" composition of $f$ and $g$.
\end{example}
\begin{proposition}\label{simplicial set inf-cat composotion respect homotopy}
Let $\mathcal{C}$ be an $\infty$-category. Suppose we are given a pair of homotopic morphisms $f,f':X\to Y$ in $\mathcal{C}$ and a pair of homotopic morphisms $g,g':Y\to Z$ in $\mathcal{C}$. Let $h$ be a composition of $f$ and $g$, and $h'$ be a composition of $f'$ and $g'$. Then $h$ is homotopic to $h'$.
\end{proposition}
\begin{proof}
Let $h''$ be a composition of $f$ and $g'$. Since homotopy is an equivalence relation (\cref{simplicial set inf-cat homotopy is equivalence}), it will suffice to show that both $h$ and $h'$ are homotopic to $h''$. We will show that $h$ is homotopic to $h''$, since the proof that $h'$ is homotopic to $h''$ is similar. Let $\sigma_3$ be a $2$-simplex of $\mathcal{C}$ which witnesses $h$ as a composition of $f$ and $g$, $\sigma_2$ be a $2$-simplex of $\mathcal{C}$ which witnesses $h''$ as a composition of f and $g'$, and let $\sigma_0$ be a $2$-simplex of $\mathcal{C}$ which is a homotopy from $g$ to $g'$. Then the tuple $(\sigma_0,\ast,\sigma_2,\sigma_3)$ determines a map of simplicial sets $\tau_0:\Lambda_1^3\to\mathcal{C}$, which we depict informally as a diagram
\[\begin{tikzcd}[row sep=15mm,column sep=15mm]
&Y\ar[r,"g"]&Z\ar[rd,dashed,"\id_Z"]&\\
X\ar[ru,"f"]\ar[rru,dashed,"h"{anchor=south}]\ar[rrr,dashed,"h''"]&&&Z\ar[llu,crossing over,leftarrow,swap,"g'"{anchor=south}]
\end{tikzcd}\]
Using our assumption that $\mathcal{C}$ is an $\infty$-category, we can extend $\tau_0$ to a $3$-simplex $\tau$ of $\mathcal{C}$. Then the face $d_1(\tau)$ is a homotopy from $h$ to $h''$.
\end{proof}
\subsection{Homotopy category of an \texorpdfstring{$\infty$}{inf}-category}
To any topological space $X$, recall that one can associate a category $\Pi(X)$, called the fundamental groupoid of $X$. This category can be described informally as follows: the objects of $\Pi(X)$ are points of $X$, and given a pair of points $x,y\in X$, $\Hom_{\Pi(X)}(x,y)$ is the set of homotopy classes of continuous paths $p:[0,1]\to X$ satisfying $p(0)=x$ and $p(1)=y$, with composition given by concatenation of paths. All of the concepts needed to define the fundamental groupoid $\Pi(X)$ (such as points, paths, homotopies, and concatenation) can be formulated in terms of singular $n$-simplices of $X$ (for $n\geq 2$). Consequently, one can view the fundamental groupoid $\Pi(X)$ as an invariant of the simplicial set $\Sing_\bullet(X)$, rather than the topological space $X$. In this paragraph, we describe an extension of this invariant, where the simplicial set $\Sing_\bullet(X)$ is replaced by an arbitrary $\infty$-category $\mathcal{C}$. In this case, the fundamental groupoid $\Pi(X)$ is replaced by a category $\ho\mathcal{C}$ which we call the \textbf{homotopy category} of $\mathcal{C}$ (beware that the homotopy category $h\mathcal{C}$ is generally not a groupoid: in fact, we will later see that it is a groupoid if and only if $\mathcal{C}$ is a Kan complex).\par
Let $\mathcal{C}$ be an $\infty$-category. For every pair of objects $X,Y\in\mathcal{C}$, we let $\Hom_{\ho\mathcal{C}}(X,Y)$ denote the set of homotopy classes of morphisms from $X$ to $Y$ in $\mathcal{C}$. For every morphism $f:X\to Y$, we let $[f]$ denote its equivalence class in $\Hom_{\ho\mathcal{C}}(X,Y)$. It follows from Propositions~\ref{simplicial set inf-cat composition unique up to homotopy} and \ref{simplicial set inf-cat composotion respect homotopy} that, for every triple of objects $X,Y,Z\in\mathcal{C}$, there is a unique composition law
\begin{align}\label{simplicial set inf-cat composition in homotopy cat}
\circ:\Hom_{\ho\mathcal{C}}(Y,Z)\times\Hom_{\ho\mathcal{C}}(X,Y)\to\Hom_{\ho\mathcal{C}}(X,Z)
\end{align}
satisfying the identity $[g]\circ[f]=[h]$ whenever $h:X\to Z$ is a composition of $f$ and $g$ in the $\infty$-category $\mathcal{C}$.
\begin{proposition}\label{simplicial set inf-cat homotopy cat def}
Let $\mathcal{C}$ be an $\infty$-category.
\begin{itemize}
\item[(a)] The composition law (\ref{simplicial set inf-cat composition in homotopy cat}) is associative. That is, for every triple of composable morphisms $f:W\to X$, $g:X\to Y$, and $h:Y\to Z$ in $\mathcal{C}$, we have an equality $([h]\circ[g])\circ[f]=[h]\circ([g]\circ[f])$ in $\Hom_{\ho\mathcal{C}}(W,Z)$.
\item[(b)] For every object $X\in\mathcal{C}$, the homotopy class $[\id_X]\in Hom_{\ho\mathcal{C}}(X,X)$ is a two-sided identity with respect to the composition law (\ref{simplicial set inf-cat composition in homotopy cat}). That is, for every morphism $f:W\to X$ in $\mathcal{C}$ and every morphism $g:X\to Y$ in $\mathcal{C}$, we have $[\id_X]\circ[f]=[f]$ and $[g]\circ[\id_X]=[g]$.
\end{itemize}
In other word, the composition law (\ref{simplicial set inf-cat composition in homotopy cat}) makes $\ho\mathcal{C}$ a category, called the \textbf{homotopy category} of $\mathcal{C}$.
\end{proposition}
\begin{proof}
We first prove (a). Let $u:W\to Y$ be a composition of $f$ and $g$, let $v:X\to Z$ be a composition of $g$ and $h$, and let $w:W\to Z$ be a composition of $f$ and $v$. Then $([h]\circ[g])\circ[f]=[w]$ and $[h]\circ([g]\circ[f])=[h]\circ[u]$. It therefore suffices to show that $w$ is a composition of $u$ and $h$. Choose a $2$-simplex $\sigma_0$ of $\mathcal{C}$ which witnesses $v$ as a composition of $g$ and $h$, a $2$-simplex $\sigma_0$ of $\mathcal{C}$ which witnesses $w$ as a composition of $f$ and $v$, and a $2$-simplex $\sigma_3$ of $\mathcal{C}$ which witnesses $u$ as a composition of $f$ and $g$. Then the sequence $(\sigma_0,\ast,\sigma_2,\sigma_3)$ determines a map of simplicial sets $\tau_0:\Lambda_1^3\to\mathcal{C}$, which we depict informally as a diagram
\[\begin{tikzcd}[row sep=15mm,column sep=15mm]
&X\ar[r,"g"]&Y\ar[rd,dashed,"h"]&\\
W\ar[ru,"f"]\ar[rru,dashed,"u"{anchor=south}]\ar[rrr,dashed,"w"]&&&Z\ar[llu,crossing over,leftarrow,swap,"v"{anchor=south}]
\end{tikzcd}\]
Using our assumption that $\mathcal{C}$ is an $\infty$-category, we can extend $\tau_0$ to a $3$-simplex $\tau$ of $\mathcal{C}$. Then the $2$-simplex $d_1(\tau)$ witnesses $w$ as a composition of $u$ and $h$.
\end{proof}
\begin{example}
Let $\mathcal{C}$ be an ordinary category. Then the homotopy category of the $\infty$-category $N_\bullet(\mathcal{C})$ can be identified with $\mathcal{C}$. In particular, for each $n\geq 0$, the homotopy category $\ho\Delta^n$ can be identified with $[n]$.
\end{example}
\begin{example}
Let $X$ be a topological space, and regard the singular simplicial set $\Sing_\bullet(X)$ as an $\infty$-category. Then the homotopy category $\ho\Sing_\bullet(X)$ can be identified with the fundamental groupoid $\Pi(X)$.
\end{example}
Let $\mathcal{C}$ be an $\infty$-category and let $\sigma:\Delta^n\to\mathcal{C}$ be an $n$-simplex of $\mathcal{C}$. For $0\leq i\leq n$, let $C_i$ denote the object of $\mathcal{C}$ given by the image of the $i$-th vertex of $\Delta^n$. For $0\leq i\leq j\leq n$, let $f_{ij}:C_i\to C_j$ denote the image under $\sigma$ of the edge of $\Delta^n$ joining the $i$-th vertex to the $j$-th vertex, and let $[f_{ij}]\in\Hom_{\ho\mathcal{C}}(C_i,C_j)$ denote the homotopy class of $f_{ij}$. Then we can regard $(\{C_i\},\{[f_{ij}]\})$ as a functor from the linearly ordered set $[n]$ to the homotopy category $\ho\mathcal{C}$. Let $u(\sigma)$ denote the corresponding $n$-simplex of $N_\bullet(\ho\mathcal{C})$. Then the map $\sigma\mapsto u(\sigma)$ determines a map of simplicial sets
\[u:\mathcal{C}\to N_\bullet(\ho\mathcal{C}).\]
\begin{proposition}\label{simplicial set inf-cat homotopy cat is adjoint}
Let $\mathcal{C}$ be an $\infty$-category and let $u:\mathcal{C}\to N_\bullet(\ho\mathcal{C})$ be the map described above. Then $u$ exhibits $\ho\mathcal{C}$ as a homotopy category of the simplicial set $\mathcal{C}$. In other words, for every category $\mathcal{D}$, the composite map
\[\Hom_{\mathbf{Cat}}(\ho\mathcal{C},\mathcal{D})\to\Hom_{\mathbf{Set}_\Delta}(N_\bullet(\ho\mathcal{C}),N_\bullet(\mathcal{D}))\stackrel{\circ u}{\to}\Hom_{\mathbf{Set}_\Delta}(\mathcal{C},N_\bullet(\mathcal{D}))\]
is bijective.
\end{proposition}
\begin{proof}
Let $F:\mathcal{C}\to N_\bullet(\mathcal{D})$ be a map of simplicial sets. Then $F$ induces a functor of homotopy categories $G:\ho\mathcal{C}\to \ho N_\bullet(D)\cong\mathcal{D}$. By construction, the map of simplicial sets
\[\begin{tikzcd}
\mathcal{C}\ar[r,"u"]&N_\bullet(\ho\mathcal{C})\ar[r,"N_\bullet(G)"]&N_\bullet(\mathcal{D})
\end{tikzcd}\]
agrees with $F$ on the vertices and edges of $\mathcal{C}$, and therefore coincides with $F$ (since a simplex of $N_\bullet(\mathcal{D})$ is determined by its $1$-dimensional facets). On the other hand, since any functor from $\ho\mathcal{C}$ to $\mathcal{D}$ is comcpletely determined by its action on vertices and facets, we see $G$ is the unique functor satisfying this property. 
\end{proof}
\subsection{Isomorphisms in \texorpdfstring{$\infty$}{inf}-categories}
Let $\mathcal{C}$ be an $\infty$-category and let $f:X\to Y$ be a morphism of $\mathcal{C}$. We say that $f$ is an \textbf{isomorphism} if the homotopy class $[f]$ is an isomorphism in the homotopy category $\ho\mathcal{C}$. We say that two objects $X,Y\in\mathcal{C}$ are isomorphic if there exists an isomorphism from $X$ to $Y$ (that is, if $X$ and $Y$ are isomorphic as objects of the homotopy category $\ho\mathcal{C}$). It is clear that if $\mathcal{C}$ be an ordinary category, then a morphism $f:X\to Y$ of $\mathcal{C}$ is an isomorphism if and only if it is an isomorphism when regarded as a morphism of the $\infty$-category $N_\bullet(\mathcal{C})$, so this notion generalize the usual isomorphisms in categories.\par
Now suppose that we are given a pair of morphisms $f:X\to Y$ and $g:Y\to X$ in an $\infty$-category $\mathcal{C}$. We say that $g$ is a \textbf{left homotopy inverse} of $f$ if the identity morphism $\id_X$ is a composition of $f$ and $g$: that is, if we have an equality $[\id_X]=[g]\circ[f]$ in the homotopy category $\ho\mathcal{C}$. Similarly, $g$ is a \textbf{right homotopy inverse} of $f$ if the identity morphism $\id_Y$ is a composition of $g$ and $f$: that is, if we have an equality $[\id_Y]=[f]\circ[g]$ in the homotopy category $\ho\mathcal{C}$. We say that $g$ is a \textbf{homotopy inverse} of $f$ if it is both a left and a right homotopy inverse of $f$. We note that the condition that $g$ is left homotopy inverse to $f$ is equivalent to that $f$ is right homotopy inverse to $g$, since both of these conditions are equivalent to the existence of a $2$-simplex $\sigma$ of $\mathcal{C}$ satisfying $d_0(\sigma)=g$, $d_1(\sigma)=\id_X$, and $d_2(\sigma)=f$, as depicted in the diagram
\[\begin{tikzcd}
&Y\ar[rd,"g"]&\\
X\ar[ru,"f"]\ar[rr,"\id_X"]&&X
\end{tikzcd}\]

On the other hand, suppose that a morphism $f:X\to Y$ admits a left homotopy inverse $g$ and a right homotopy inverse $h$. Then $g$ and $h$ are homotopic: we have
\[[g]=[g]\circ[\id_Y]=[g]\circ([f]\circ[h])=([g]\circ[f])\circ[h]=[\id_Y]\circ[h]=[h].\]
In particular, for a morphism $f:X\to Y$ in an $\infty$-category $\mathcal{C}$, we see that the following are equivalent:
\begin{itemize}
\item[(\rmnum{1})] The morphism $f$ is an isomorphism.
\item[(\rmnum{2})] The morphism $f$ admits a homotopy inverse $g$.
\item[(\rmnum{3})] The morphism $f$ admits both left and right homotopy inverses.
\end{itemize}
In this case, the morphism $g$ is uniquely determined up to homotopy; moreover, any left or right homotopy inverse of $f$ is homotopic to $g$. We will sometimes abuse notation by writing $f^{-1}$ to denote a homotopy inverse to $f$.
\begin{remark}
Let $f:X\to Y$ be a morphism in an $\infty$-category $\mathcal{C}$, and suppose that $g,h:Y\to X$ are left homotopy inverses to $f$. If $f$ does not admit a right homotopy inverse, then $g$ and $h$ need not be homotopic. This follows from the fact that a left-invertible morphism is not necessarily invertible.
\end{remark}
\begin{proposition}\label{simplicial set Kan morphism is isomorphism}
Let $\mathcal{C}$ be a Kan complex. Then every morphisms of $\mathcal{C}$ is an isomorphism. In other words, its homotopy category $\ho\mathcal{C}$ is a groupoid.
\end{proposition}
\begin{proof}
Let $f:X\to Y$ be a morphism in $\mathcal{C}$. Then the tuple $(\ast,\id_X,f)$ determines a map of simplicial sets $\sigma_0:\Lambda^2_0\to\mathcal{C}$, which we depict as
\[\begin{tikzcd}
&Y\ar[rd]&\\
X\ar[ru,"f"]\ar[rr,"\id_X"]&&X
\end{tikzcd}\]
If $\mathcal{C}$ is a Kan complex, then we can extend $\sigma_0$ to a $2$-simplex $\sigma$ of $\mathcal{C}$. Then $\sigma$ exhibits the morphism $g=d_0(\sigma)$ as a left homotopy inverse to $f$. A similar argument shows that $f$ admits a right homotopy inverse, so that $f$ is an isomorphism.
\end{proof}
Let $S_\bullet$ be a Kan complex, then it follows from \cref{simplicial set Kan morphism is isomorphism} that the homotopy category $\ho S_\bullet$ is a groupoid. We denote this groupoid by $\Pi(S_\bullet)$ and call it as the \textbf{fundamental groupoid} of $S_\bullet$.
\begin{remark}
Let $S_\bullet$ be a Kan complex. By definition, the objects of the fundamental groupoid $\Pi(S_\bullet)$ are the vertices of $S_\bullet$, and a pair of vertices $x,y\in S_0$ are isomorphic in $\Pi(S_\bullet)$ if and only if there exists an edge $e:x\to y$ in $S_\bullet$. Applying \cref{simplicial set Kan same component iff connected by edge}, we deduce that $x,y\in S_0$ are isomorphic if and only if they belong to the same connected component of $S_\bullet$. In other words, we have a canonical bijection
\[\pi_0(S_\bullet)\cong\Ob(\Pi(S_\bullet))/\text{isomorphism}.\]
\end{remark}
\begin{example}
Let $X$ be a topological space. Then the singular simplicial set $\Sing_\bullet(X)$ is a Kan complex (\cref{simplicial set singular is Kan}), and its fundamental groupoid $\Pi(\Sing_\bullet(X))$ can be identified with the usual fundamental groupoid $\Pi(X)$ of the topological space $X$ (where objects are the points of $X$ and morphisms are given by homotopy classes of paths in $X$).
\end{example}
\section{Functor of \texorpdfstring{$\infty$}{inf}-categories}
Let $\mathcal{C}$ and $\mathcal{D}$ be categories and $N_\bullet(\mathcal{C})$ and $N_\bullet(\mathcal{D})$ denote the corresponding $\infty$-categories. According to \cref{simplicial set nerve of cat fully faithful}, the nerve functor $N_\bullet$ induces a bijection
\[\Hom_{\mathbf{Cat}}(\mathcal{C},\mathcal{D})\stackrel{\sim}{\to}\Hom_{\mathbf{Set}_\Delta}(N_\bullet(\mathcal{D}),N_\bullet(\mathcal{D})).\]
Consequently, the notion of functor admits an obvious generalization to the setting of $\infty$-categories: if $\mathcal{C}$ and $\mathcal{D}$ are $\infty$-categories, a functor $\mathcal{C}$ to $\mathcal{D}$ is defined to be a map of simplicial sets $F:\mathcal{C}\to\mathcal{D}$. This section is devoted to the study of functors between $\infty$-categories.
\subsection{Example of functors}
Let $F:\mathcal{C}\to\mathcal{D}$ be a functor of $\infty$-categories. Then $F$ carries objects of $\mathcal{D}$ to that of $\mathcal{D}$, and edges of $\mathcal{C}$ to that of $\mathcal{D}$. Also, $F$ is compatible with composition "up to coherent homotopy." For example, suppose that we are given objects $X,Y,Z\in\mathcal{C}$ and morphisms $f:X\to Y$, $g:Y\to Z$, and $h:X\to Z$. If $h$ is a composition of $f$ and $g$, then $F(h)$ is a composition of $F(f)$ and $F(g)$. Moreover, we can say more: if $\sigma$ is a $2$-simplex of $\mathcal{C}$ which witnesses $h$ as a composition of $f$ and $g$, then $F(\sigma)$ is a $2$-simplex of $\mathcal{D}$ which witnesses $F(h)$ as a composition of $F(f)$ and $F(g)$. In particular, we see that $f,g:X\to Y$ are homotopic morphisms of $\mathcal{C}$, then $F(f),F(g):F(X)\to F(Y)$ are homotopic morphisms of $\mathcal{D}$. More precisely, the functor $F$ carries homotopies from $f$ to $g$ (viewed as $2$-simplices of $\mathcal{C}$) to homotopies from $F(f)$ to $F(g)$ (viewed as $2$-simplices of $\mathcal{D}$).
\begin{example}
Let $\mathcal{C}$ be an $\infty$-category and let $\mathcal{D}$ be an ordinary category. Using \cref{simplicial set inf-cat homotopy cat is adjoint}, we obtain a bijection from functors of $\infty$-categories from $\mathcal{C}$ to $N_\bullet(\mathcal{D})$ to that of ordinary categories from $\ho\mathcal{C}$ to $\mathcal{D}$
\end{example}
\begin{example}
Let $X$ be a topological space and let $\mathcal{C}$ be an ordinary category. To specify a functor of $\infty$-categories $F:\Sing_\bullet(X)\to N_\bullet(\mathcal{C})$, one must give a rule which assigns to each continuous map $\sigma:|\Delta^n|\to X$ (viewed as an $n$-simplex of $\Sing_\bullet(X)$) a diagram $F(\sigma)$ of morphisms in $\mathcal{D}$. In particular:
\begin{itemize}
\item[(a)] To each point $x\in X$, the functor $F$ assigns an object $F(x)\in\mathcal{C}$.
\item[(b)] To each continuous path $f:[0,1]\to X$ starting at the point $x=f(0)$ and ending at the point $y=f(1)$, the functor $F$ assigns a morphism $F(f):F(x)\to F(y)$ in the category $\mathcal{C}$. The morphism $F(f)$ is automatically an isomorphism by \cref{simplicial set nerve of cat Kan iff groupoid}.
\item[(c)] For each continuous map $\sigma:|\Delta^2|\to X$ with boundary behavior as depicted in the diagram
\[\begin{tikzcd}
&y\ar[rd,"g"]&\\
x\ar[ru,"f"]\ar[rr,"h"]&&z
\end{tikzcd}\]
we have an identity $F(h)=F(g)\circ F(f)$ in $\Hom_{\mathcal{C}}(F(x),F(z))$.
\end{itemize}
The data of a collection of objects $\{F(x)\}_{x\in X}$ and isomorphisms $\{F(f)\}_{f:[0,1]\to X}$ satisfying (c) is called a \textbf{$\bm{\mathcal{C}}$-valued local system} on $X$. Theerfore the preceding discussion determines a bijection between functors from $\Sing_\bullet(X)$ to $N_\bullet(\mathcal{C})$ and $\mathcal{C}$-valued local system on $X$. By virtue of \cref{simplicial set Sing(X) homotopy cat is Pi(X)}, we can also identify local systems with functors from the fundamental groupoid $\Pi(X)$ into $\mathcal{C}$.\par
More generally, for an arbitrary $\infty$-category $\mathcal{C}$, we can define a $\mathcal{C}$-valued local system on $X$ to be a functor of $\infty$-categories $\Sing_\bullet(X)\to\mathcal{C}$. But note that this notion generally cannot be reformulated in terms of the fundamental groupoid $\Pi(X)$.
\end{example}
\begin{example}
Let $\mathcal{C}$ be an $\infty$-category and let $X$ be a topological space. Then we have a canonical bijection between functors of $\infty$-categories from $\mathcal{C}$ to $\Sing_\bullet(X)$ to continuous functions from $|\mathcal{C}|\to X$, where $|\mathcal{C}|$ denotes the geometric realization of the simplicial set $\mathcal{C}$. Beware that neither side has an obvious interpretation in terms of functors between ordinary categories (even in the special case where $\mathcal{C}$ is the nerve of a category).
\end{example}
\subsection{Commutative diagrams in \texorpdfstring{$\infty$}{inf}-categories}
In ordinary category theory, one can think of a functor $F:\mathcal{C}\to\mathcal{D}$ as a kind of commutative diagram in $\mathcal{D}$, having vertices indexed by the objects of $\mathcal{C}$ and arrows indexed by the morphisms of $\mathcal{C}$. This perspective is quite useful: if the category $\mathcal{C}$ is sufficiently small, one can communicate the datum of a functor by drawing a graphical representation of the corresponding diagram. In this paragraph, we discuss the generalization of this notion in an $\infty$-category.\par
Let $\mathcal{C}$ be an $\infty$-category. A \textbf{diagram} in $\mathcal{C}$ is defined to be a map of simplicial sets $f:K_\bullet\to\mathcal{C}$ (or a \textbf{diagram in $\mathcal{C}$ indexed by $K_\bullet$}, or a $K_\bullet$-indexed diagram). If $K_\bullet$ is the nerve $N_\bullet(I)$ of a partially ordered set $I$, we also say that this is a diagram in $\mathcal{C}$ indexed by $I$, or an $I$-indexed diagram in $\mathcal{C}$.
\begin{example}\label{simplicial set diagram indexed by graph}
Let $\mathcal{C}$ be an $\infty$-category and let $K_\bullet$ be a simplicial set of dimension $\leq 1$, corresponding to a directed graph $G$. In this case, a diagram $K_\bullet\to\mathcal{C}$ can be identified with a pair $(\{C_v\}_{v\in V(G)},\{f_e\}_{e\in E(G)})$, where each $C_v$ is an object of the $\infty$-category $\mathcal{C}$ and each $f_e:C_{s(e)}\to C_{t(e)}$ is a morphism of $C$ (here $s(e)$ and $t(e)$ denote the source and target of the edge $e$). It is often convenient to specify diagrams $K_\bullet\to\mathcal{C}$ by drawing a graphical representation of $G$, where each node is labelled by an object of $\mathcal{C}$ and each arrow is labelled by a morphism in $\mathcal{C}$ (having the indicated source and target).
\end{example}
\begin{example}[\textbf{Noncommutative Squares in Categories}]\label{simplicial set non-commutative square in cat}
Let $K_\bullet$ denote the boundary of the product $\Delta^1\times\Delta^1$: that is, the simplicial subset of $\Delta^1\times\Delta^1$ given by the union of the simplicial subsets $\partial\Delta^1\times\Delta^1$ and $\Delta^1\times\partial\Delta^1$. Then $K_\bullet$ is a $1$-dimensional simplicial set, corresponding to a directed graph which we can depict as
\[\begin{tikzcd}
\bullet\ar[r]\ar[d]&\bullet\ar[d]\\
\bullet\ar[r]&\bullet
\end{tikzcd}\]
We can then display a $K_\bullet$-indexed diagram in an $\infty$-category $\mathcal{C}$ pictorially
\[\begin{tikzcd}
C_{00}\ar[d,swap,"g"]\ar[r,"f"]&C_{01}\ar[d,"g'"]\\
C_{10}\ar[r,"f'"]&C_{11}
\end{tikzcd}\]
where each $C_{ij}$ is an object of $\mathcal{C}$. Note that this square does not necessarily commute.
\end{example}
In classical category theory, it is useful to extend the notational of \cref{simplicial set diagram indexed by graph} to more general situations by introducing the notion of a \textit{commutative diagram}. Let $K_\bullet$ be a simplicial set of dimension $\leq 1$, which we will identify with a directed graph $G$. Assume that $G$ is a \textbf{simple graph}; that is, $G$ satisfies the following additional conditions:
\begin{itemize}
\item[(a)] For every pair of vertices $v,w\in V(G)$, there is at most one edge of $G$ with source $v$ and target $w$. We will denote this edge (if it exists) by $(v,w)\in E(G)$.
\item[(b)] The graph $G$ has no directed cycles. That is, if there is a sequence $v_0,v_1,\dots,v_n\in V(G)$ with the property that the edges $(v_{i-1},v_i)$ exist for $1\leq i\leq n$, then either $n=0$ or $v_0\neq v_n$.
\end{itemize}
Let $\mathcal{C}$ be an ordinary category and suppose we are given a diagram $\sigma:K_\bullet\to N_\bullet(\mathcal{C})$, which we identify with a pair $(\{C_v\}_{v\in V(G)},\{f_e\}_{e\in E(G)})$. We say that the diagram $\sigma$ \textbf{commutes} (or that $\sigma$ is a \textbf{commutative diagram}) if the following additional condition is satisfied:
\begin{itemize}
\item[(c)] Let $v$ and $w$ be vertices of $G$ which are joined by directed paths $(v=v_0,v_1,\dots,v_m=w)$ and $(v=w_0,w_1,\dots,w_n=w)$. Then we have an identity
\[f_{v_m,v_{m-1}}\circ f_{v_{m-1},v_{m-2}}\circ\cdots\circ f_{v_1,v_0}=f_{w_n,w_{n-1}}\circ f_{w_{n-1},w_{n-2}}\circ\cdots\circ f_{w_1,w_0}\]
in the set $\Hom_{\mathcal{C}}(C_v,C_w)$.
\end{itemize}
\begin{proposition}\label{simplicial set diagram commutes iff extend to functor}
Let $K_\bullet$ be a simplicial set of dimension $\leq 1$, corresponding to a directed simple graph $G$. Let $\mathcal{C}$ be an ordinary category, and $f:K_\bullet\to N_\bullet(\mathcal{C})$ be a diagram. Then:
\begin{itemize}
\item[(\rmnum{1})] There is a partial ordering $\preceq$ on the vertex set $V(G)$, where we have $v\preceq w$ if and only if there exists a sequence of vertices $(v=v_0,v_1,\dots,v_n=w)$ with the property that the edges $(v_{i-1},v_i)\in E(G)$ exist for $1\leq i\leq n$.
\item[(\rmnum{2})] There is a unique monomorphism of simplicial sets $K_\bullet\hookrightarrow N_\bullet(V(G))$ which carries each vertex to itself.
\item[(\rmnum{3})] The diagram $\sigma$ extends to a map $\bar{\sigma}:N_\bullet(V(G))\to N_\bullet(\mathcal{C})$ (that is, to a functor $V(G)\to\mathcal{C})$ if and only if it is commutative (here $N_\bullet(V(G))$ is the nerve of the partially ordered set $V(G)$). Moreover, if the extension $\bar{\sigma}$ exists, then it is unique.
\end{itemize}
\end{proposition}
\begin{proof}
It follows immediately from the definitions that the relation $\preceq$ defined in (\rmnum{1}) is reflexive and transitive. Antisymmetry follows from our assumption that the graph $G$ has no directed loops. By construction, we have $v\preceq w$ whenever $v$ and $w$ are connected by an edge $(v,w)\in E(G)$. From the description of the simplicial set $K_\bullet$ given in \cref{simplicial set graph functor description}, we immediately see that there is a unique map of simplicial sets $i:K_\bullet\hookrightarrow N_\bullet(V(G))$ which is the identity on vertices. It follows from assumption (a) that the map $i$ is a monomorphism. Let us henceforth identify $K_\bullet$ with a simplicial subset of $N_\bullet(V(G))$ given by the image of $i$, and identify $\sigma$ with a pair $(\{C_v\}_{v\in V(G)},\{f_e\}_{e\in E(G)})$. Suppose that the diagram $\sigma$ extends to a functor $\bar{\sigma}:N_\bullet(V(G))\to\mathcal{C}$. If $v$ and $w$ are a pair of vertices of $G$ with $v\preceq w$, then we can choose a directed path $(v=v_0,v_1,\dots,v_n=w)$ from $v$ to $w$. The compatibility of $\bar{\sigma}$ with composition then guarantees that $\bar{\sigma}$ must carry the edge $(v,w)$ of $N_\bullet(V(G))$ to the iterated composition $f_{v_n,v_{n-1}}\circ f_{v_{n-1},v_{n-2}}\circ\cdots\circ f_{v_1,v_0}\in\Hom_{\mathcal{C}}(C_v,C_w)$. Since the morphism $\bar{\sigma}(v,w)$ is independent of the choice of directed path, it follows that the diagram $\sigma$ is commutative. Conversely, if $\bar{\sigma}$ is commutative, then we can define $\bar{\sigma}$ on morphisms by the formula $\bar{\sigma}(v,w)=f_{v_n,v_{n-1}}\circ f_{v_{n-1},v_{n-2}}\circ\cdots\circ f_{v_1,v_0}$ to obtain the desired extension of $\sigma$.
\end{proof}
\begin{remark}
In the situation of \cref{simplicial set diagram commutes iff extend to functor}, an arbitrary map of simplicial sets $\sigma:K_\bullet\to N_\bullet(\mathcal{C})$ can be identified with a functor $F:P[G]\to\mathcal{C}$, where $P[G]$ denotes the path category of the graph $G$ (\cref{simplicial set graph homotopy cat is path cat}). The commutativity of the diagram $\sigma$ is equivalent to the requirement that $F$ factors through the quotient functor $P[G]\twoheadrightarrow V(G)$: that is, the value of the functor $F$ on a path depends only the endpoints of that path.
\end{remark}
\begin{example}[\textbf{Commutative Squares in Categories}]\label{simplicial set commutative square in cat}
Let $K_\bullet=\partial(\Delta^1\times\Delta^1)$ be as in \cref{simplicial set non-commutative square in cat}. For any ordinary category $\mathcal{C}$, we can display a diagram $\sigma:K_\bullet\to N_\bullet(\mathcal{C})$ pictorially as
\[\begin{tikzcd}
C_{00}\ar[r,"f"]\ar[d,swap,"g"]&C_{01}\ar[d,"g'"]\\
C_{10}\ar[r,"f'"]&C_{11}
\end{tikzcd}\]
The diagram $\sigma$ is commutative if and only if we have $g'\circ f=f'\circ g$. In this case, \cref{simplicial set diagram commutes iff extend to functor} ensures that $\sigma$ extends uniquely to a diagram $\bar{\sigma}:\Delta^1\times\Delta^1\to N_\bullet(\mathcal{C})$, or equivalently to a functor of ordinary categories $[1]\times[1]\to\mathcal{C}$.
\end{example}
\begin{example}[\textbf{Squares in $\infty$-Categories}]\label{simplicial set inf-cat square diagram}
In the setting of $\infty$-categories, assertion (\rmnum{3}) of \cref{simplicial set diagram commutes iff extend to functor} is false in general. For example, let $I$ denote the partially ordered set $[1]\times[1]$. The simplicial set $N_\bullet(I)\cong\Delta^1\times\Delta^1$ has four vertices (given by the elements of $I$), five nondegenerate edges, and two nondegenerate $2$-simplices. Unwinding the definitions, we see that an $I$-indexed diagram in an $\infty$-category $\mathcal{C}$ is equivalent to the following data:
\begin{itemize}
\item A collection of objects $C_{00},C_{01},C_{10},C_{11}$ of $\mathcal{C}$.
\item A collection of morphisms $f:C_{00}\to C_{01}$, $g:C_{00}\to C_{10}$, $f':C_{10}\to C_{11}$, $g':C_{01}\to C_{11}$, and $h:C_{00}\to C_{11}$.
\item A $2$-simplex $\sigma$ of $\mathcal{C}$ which witnesses $h$ as a composition of $f$ with $g'$, and a 2-simplex $\tau$ of $\mathcal{C}$ which witnesses $h$ as a composition of $g$ with $f'$.
\end{itemize}
This data can be depicted graphically as follows:
\[\begin{tikzcd}
C_{00}\ar[dd,swap,"g"]\ar[rrdd,"h"{anchor=south}]\ar[rr,"f"]&{}&C_{01}\ar[dd,"g'"]\ar[ld,phantom,"\sigma"description]\\
&{}&\\
C_{10}\ar[rr,"f'"]\ar[ru,phantom,"\tau"description]&&C_{11}
\end{tikzcd}\]
Such a diagram is usually not determined by its restriction to the simplicial subset $K_\bullet\sub N_\bullet(I)$ of \cref{simplicial set commutative square in cat}, since homotopies bwtween morphisms are not unique in general.
\end{example}
\begin{example}
Let $\mathcal{C}$ be an $\infty$-category and $K_\bullet\sub\Delta^1\times\Delta^1$ be the simplicial subset in \cref{simplicial set commutative square in cat}. Suppose that we are given a diagram $\sigma:K_\bullet\to\mathcal{C}$, which we depict graphically as
\[\begin{tikzcd}
C_{00}\ar[r,"f"]\ar[d,swap,"g"]&C_{01}\ar[d,"g'"]\\
C_{10}\ar[r,"f'"]&C_{11}
\end{tikzcd}\]
Composing with the unit map $\mathcal{C}\to N_\bullet(\ho\mathcal{C})$, we obtain a diagram $\sigma'$ in the homotopy category $\ho\mathcal{C}$, which we can depict as
\[\begin{tikzcd}
C_{00}\ar[r,"{[f]}"]\ar[d,swap,"{[g]}"]&C_{01}\ar[d,"{[g']}"]\\
C_{10}\ar[r,"{[f']}"]&C_{11}
\end{tikzcd}\]
By definition, the diagram $\sigma'$ is commutative if and only if $[f]\circ[g']=[f']\circ[g]$; that is, if and only if $\sigma$ can be extended to a map $\bar{\sigma}:\Delta^1\times\Delta^1\to\mathcal{C}$ (\cref{simplicial set homotopy iff Delta^1 x Delta^1}). Again, this extension of $\sigma$ is generally not unique.
\end{example}
\cref{simplicial set inf-cat square diagram} illustrates that the notion of "commutative diagram" becomes considerably more subtle in the setting of $\infty$-categories. To specify an $I$-indexed diagram $F:N_\bullet(I)\to\mathcal{C}$ of an $\infty$-category $\mathcal{C}$, one generally needs to specify the values of $F$ on all the simplices of the simplicial set $N_\bullet(I)$. In general, it is not feasible to graphically encode all of this data in a comprehensible way. On the other hand, the formalism of commutative diagrams is too useful to completely abandon. We will therefore sacrifice some degree of mathematical precision in favor of clarity of exposition.\par
Let $\mathcal{C}$ be an $\infty$-category and let $G$ be a directed simple graph, so that the vertex set $V(G)$ inherits a partial ordering (\cref{simplicial set diagram commutes iff extend to functor}). By a commutative diagram $\sigma$ in $\mathcal{C}$, we mean graphically by a collection of objects $\{C_v\}_{v\in V(G)}$ of $\mathcal{C}$, connected by arrows which are labelled by morphisms $\{f_e\}_{e\in E(G)}$. In this case, it should be understood that $\sigma$ is a diagram $N_\bullet(V(G))\to\mathcal{C}$, which carries each vertex $v$ of $N_\bullet(V(G))$ to the object $C_v\in\mathcal{C}$ and each edge $e=(v,w)$ of $G$ to the morphism $f_e$ in $\mathcal{C}$. Note that in this case, the map $\sigma$ need not be completely determined by the pair $(\{C_v\}_{v\in V(G)},\{f_e\}_{e\in E(G)})$ (this pair can instead be identified with the restriction $\sigma|_{K_\bullet}$, where $K_\bullet$ is the $1$-dimensional simplicial subset of $N_\bullet(V(G))$ corresponding to $G$).
\begin{remark}
Suppose that $\mathcal{C}=N_\bullet(\mathcal{C}_0)$, where $\mathcal{C}_0$ is an ordinary category. Then giving a commutative diagram in the $\infty$-category $\mathcal{C}$ is equivalent to giving a commutative diagram in the ordinary category $\mathcal{C}_0$. In this case, commutativity is a \textit{property} that the underlying diagram (indexed by a $\infty$-dimensional simplicial set) does or does not possess. For a general $\infty$-category $\mathcal{C}$, commutativity of a diagram in $\mathcal{C}$ is not a property but a structure; to promote a diagram to a commutative diagram, one must specify additional data to witness the requisite commutativity.
\end{remark}
\begin{example}
Let $\mathcal{C}$ be an $\infty$-category. Then a commutative diagram $\sigma$ is by definition a $2$-simplex of $\mathcal{C}$ satisfying $d_0(\sigma)=g$, $d_1(\sigma)=h$, and $d_2(\sigma)=f$, as depicted as follows:
\[\begin{tikzcd}
&Y\ar[rd,"g"]&\\
X\ar[ru,"f"]\ar[rr,"h"]&&Z
\end{tikzcd}\]
In other words, we mean that $h$ is a $2$-simplex which witnesses $h$ as a composition of $f$ and $g$.
\end{example}
\begin{example}
Let $\mathcal{C}$ be an $\infty$-category. Then by a commutative diagram $\sigma$:
\[\begin{tikzcd}
C_{00}\ar[r,"f"]\ar[d,swap,"g"]&C_{01}\ar[d,"g'"]\\
C_{10}\ar[r,"f'"]&C_{11}
\end{tikzcd}\]
we implicitly assume that $\sigma$ is a map from the entire simplicial set $\Delta^1\times\Delta^1$ to $\mathcal{C}$. In other words, we assume that we have specified another morphism $h:C_{00}\to C_{11}$, which is not indicated in the picture, together with a $2$-simplex $\sigma$ witnessing $h$ as the composition of $f$ and $g'$ and a $2$-simplex $\tau$ witnessing $h$ as the composition of $g$ and $f'$.
\end{example}
\begin{remark}
In ordinary category theory, it is sometimes useful to consider commutativity of diagrams in situations which do not fit the paradigm of our definition. For example, the commutativity of a diagram 
\[\begin{tikzcd}
X\ar[r,"f"]&Y\ar[r,shift left=1mm,"u"]\ar[r,shift right=1mm,swap,"v"]&Z
\end{tikzcd}\]
is often understood as the requirement that $u\circ f=v\circ f$. Beware that this usage is potentially ambiguous (from the shape of the diagram alone, it is not clear that commutativity should
enforce the identity $u\circ f=v\circ f$, but not the identity $u=v$), so we will take special care when applying similar terminology in the $\infty$-categorical setting.
\end{remark}
\subsection{The \texorpdfstring{$\infty$}{inf}-category of functors}
Let $\mathcal{C}$ and $\mathcal{D}$ be categories. Then we can form a new category $\Fun(\mathcal{C},\mathcal{D})$, whose objects are functors from $\mathcal{C}$ to $\mathcal{D}$ and whose morphisms are natural transformations. We now describe an analogous construction in the setting of $\infty$-categories.\par
Let $S_\bullet$ and $T_\bullet$ be simplicial sets. Then the construction
\[([n]\in\Delta^{\op})\mapsto\Hom_{\mathbf{Set}_\Delta}(\Delta^n\times S_\bullet,T_\bullet)\]
determines a functor from the category $\Delta^{\op}$ to the category of sets. We regard this functor as a simplicial set, which we denote by $\Fun(S_\bullet,T_\bullet)$.\par
Note that, given an $n$-simplex $f$ of $\Fun(S_\bullet,T_\bullet)$ and an $n$-simplex $\sigma$ of $S_\bullet$, we can construct an $n$-simplex $\ev(f,\sigma)$ of $T_\bullet$, given by the composition
\[\begin{tikzcd}
\Delta^n\ar[r]&\Delta^n\times\Delta^n\ar[r,"\id\times\sigma"]&\Delta^n\times S_\bullet\ar[r,"f"]&T_\bullet
\end{tikzcd}\]
where $\Delta^n\to\Delta^n\times\Delta^n$ is the diagonal map. This determines a map of simplicial sets $\ev:\Fun(S_\bullet,T_\bullet)\times S_\bullet\to T_\bullet$, which is called the \textbf{evaluation map}.
\begin{proposition}\label{simplicial set Fun evaluation map adjoint prop}
Let $S_\bullet$, $T_\bullet$ and $U_\bullet$ be simplicial sets. Then the composite map
\[\theta:\Hom_{\mathbf{Set}_\Delta}(U_\bullet,\Fun(S_\bullet,T_\bullet))\to\Hom_{\mathbf{Set}_\Delta}(U_\bullet\times S_\bullet,\Fun(S_\bullet,T_\bullet)\times S_\bullet)\stackrel{\ev\circ}{\to}\Hom_{\mathbf{Set}_\Delta}(U_\bullet\times S_\bullet,T_\bullet)\]
is bijective.
\end{proposition}
\begin{proof}
Let $f:U_\bullet\times S_\bullet\to T_\bullet$ be a map of simplicial sets. For each $n$-simplex $\sigma$ of $U_\bullet$, the composite map
\[\begin{tikzcd}
\Delta^n\times S_\bullet\ar[r,"\sigma\times\id"]&U_\bullet\times S_\bullet\ar[r,"f"]&T_\bullet
\end{tikzcd}\]
can be regarded as an $n$-simplex of $\Fun(S_\bullet,T_\bullet)$, which we will denote by $g(\sigma)$. The construction $\sigma\mapsto g(\sigma)$ determines a map of simplicial sets $g:U_\bullet\to\Fun(S_\bullet,T_\bullet)$. By the definition of $\theta$, for any $n$-simplex $(\sigma,\tau)$ of $U_\bullet\times S_\bullet$ (where $\sigma:\Delta^n\to U_\bullet$ and $\tau:\Delta^n\to S_\bullet$), the map $\theta(g)$ sends $(\sigma,\tau)$ to $\ev(g(\sigma),\tau)$, which is given by the composition 
\[\begin{tikzcd}
\Delta^n\to\Delta^n\times\Delta^n\ar[r,"\id\times\tau"]&\Delta^n\times S_\bullet\ar[r,"\sigma\times\id"]&U_\bullet\times S_\bullet\ar[r,"f"]&T_\bullet.
\end{tikzcd}\]
which is equal to $f(\sigma,\tau)$, so $\theta(g)=f$. On the other hand, if $g':U_\bullet\to\Fun(S_\bullet,T_\bullet)$ is another map satisfying $\theta(g')=f$, then for any $n$-simplex $(\sigma,\tau)$ of $U_\bullet\times S_\bullet$, the $n$-simplex $f(\sigma,\tau)$ of $T_\bullet$ is given by the composition
\[\begin{tikzcd}
\Delta^n\ar[r]&\Delta^n\times\Delta^n\ar[r,"1\times\tau"]&\Delta^n\times S_\bullet\ar[r,"g'(\sigma)"]&T_\bullet
\end{tikzcd}\]
It then follows that $g(\sigma)=g'(\sigma)$ for any $n$-simplex $\sigma:\Delta^n\to U_\bullet$, so the map $g$ is uniquely determined by $f$.
\end{proof}
One may wonder why we use the notation $\Fun(S_\bullet,T_\bullet)$ to denote a simplicial set in $\mathbf{Set}_\Delta$, rather than a category of functors. This usage is in fact justified by the following proposition:
\begin{proposition}\label{simplicial set functor of nerve cat isomorphism}
Let $\mathcal{C}$ and $\mathcal{D}$ be categories and let $\ev:\Fun(\mathcal{C},\mathcal{D})\to\mathcal{D}$ denote the evaluation functor, given on objects by the formula $\ev(F,C)=F(C)$. Then the composite map
\[N_\bullet(\Fun(\mathcal{C},\mathcal{D}))\times N_\bullet(\mathcal{C})\cong N_\bullet(\Fun(\mathcal{C},\mathcal{D})\times\mathcal{C})\stackrel{N_\bullet(\ev)}{\to}N_\bullet(\mathcal{D})\]
corresponds, under the bijection of \cref{simplicial set Fun evaluation map adjoint prop}, to an isomorphism of simplicial sets
\[\rho:N_\bullet(\Fun(\mathcal{C},\mathcal{D}))\to\Fun(N_\bullet(\mathcal{C}),N_\bullet(\mathcal{D})).\]
\end{proposition}
\begin{proof}
For each positive integer $n\geq 0$, the map $\rho$ sends an $n$-simplex $\sigma:\Delta^n\to N_\bullet(\Fun(\mathcal{C},\mathcal{D}))$ to the composition
\[\begin{tikzcd}
\Delta^n\times N_\bullet(\mathcal{C})\ar[r,"\sigma\times\id"]&N_\bullet(\Fun(\mathcal{C},\mathcal{D}))\times N_\bullet(\mathcal{C})\ar[r,"\sim"]&N_\bullet(\Fun(\mathcal{C},\mathcal{D})\times\mathcal{C})\ar[r,"N_\bullet(\ev)"]&N_\bullet(\mathcal{D}).
\end{tikzcd}\]
By \cref{simplicial set nerve of cat fully faithful}, we see that
\begin{gather*}
\Hom_{\mathbf{Set}_\Delta}(\Delta^n,N_\bullet(\Fun(\mathcal{C},\mathcal{D})))\cong\Hom_{\mathbf{Cat}}([n],\Fun(\mathcal{C},\mathcal{D}))\cong\Hom_{\mathbf{Cat}}([n]\times\mathcal{C},\mathcal{D}),\\
\Hom_{\mathbf{Set}_\Delta}(\Delta^n,\Fun(N_\bullet(\mathcal{C}),N_\bullet(\mathcal{D})))\cong\Hom_{\mathbf{Set}_\Delta}(\Delta^n\times N_\bullet(\mathcal{C}),N_\bullet(\mathcal{D})),
\end{gather*}
so $\rho$ corresponds to the followng map
\begin{align*}
&\Hom_{\mathbf{Cat}}([n]\times\mathcal{C},\mathcal{D})\stackrel{\gamma}{\to}\Hom_{\mathbf{Set}_\Delta}(N_\bullet([n]\times\mathcal{C}),N_\bullet(\mathcal{D}))\cong\Hom_{\mathbf{Set}_\Delta}(\Delta^n\times N_\bullet(\mathcal{C}),N_\bullet(\mathcal{D}))
\end{align*}
We know that $\gamma$ is an isomorphism (\cref{simplicial set nerve of cat fully faithful}), so the claim follows.
\end{proof}
\begin{corollary}\label{simplicial set homotopy cat Fun of nerve isomorphism}
Let $\mathcal{C}$ and $\mathcal{D}$ be categories. Then there is a canonical isomorphism of categories
\[\Fun(\mathcal{C},\mathcal{D})\cong h\big(\Fun(N_\bullet(\mathcal{C}),N_\bullet(\mathcal{D}))\big).\]
\end{corollary}
\begin{corollary}\label{simplicial set Fun to nerve isomorphism}
Let $S_\bullet$ be a simplicial set having homotopy category $\ho S_\bullet$. Then, for any category $\mathcal{D}$, the composite map
\[N_\bullet(\Fun(\ho S_\bullet,\mathcal{D}))\times S_\bullet\cong N_\bullet(\Fun(\ho S_\bullet,\mathcal{D})\times \ho S_\bullet)\to N_\bullet(\mathcal{D})\]
induces an isomorphism of simplicial sets $\rho_{S_\bullet}:N_\bullet(\Fun(\ho S_\bullet,\mathcal{D}))\cong\Fun(S_\bullet,N_\bullet(\mathcal{D}))$.
\end{corollary}
\begin{proof}
The construction $S_\bullet\mapsto\rho_{S_\bullet}$ carries colimits (in the category $\mathbf{Set}_\Delta$ of simplicial sets) to limits (in the category $\Fun([1],\mathbf{Set}_\Delta$) of morphisms between simplicial sets). Since the category $\mathbf{Set}_\Delta$ is generated under colimits by objects of the form $\Delta^n$, it suffices to prove \cref{simplicial set Fun to nerve isomorphism} in the special case where $S_\bullet=\Delta^n$. In this case, the desired result follows from \cref{simplicial set functor of nerve cat isomorphism}, since $S_\bullet$ is isomorphic to the nerve of the category $\mathcal{C}=[n]$.
\end{proof}
\begin{corollary}\label{simplicial set homotopy cat functor finite product}
The formation of homotopy categories determines a functor $\mathbf{Set}_\Delta\to\mathbf{Cat}$ which commutes with finite products.
\end{corollary}
\begin{proof}
Since the construction $S_\bullet\to \ho S_\bullet$ preserves final objects, it suffices to show that for any pair of simplicial sets $S_\bullet$ and $T_\bullet$, the canonical map
\[u:\ho S_\bullet\times T_\bullet\to \ho S_\bullet\times \ho T_\bullet\]
is an isomorphism of categories. In other words, we want to show that for any category $\mathcal{C}$, composition with $u$ induces a bijection
\[\Hom_{\mathbf{Cat}}(\ho S_\bullet\times \ho T_\bullet,\mathcal{C})\to\Hom_{\mathbf{Cat}}(\ho S_\bullet\times T_\bullet,\mathcal{C}).\]
Unwinding the definitions and using the adjunction of $h$ and $N_\bullet$:
\begin{gather*}
\Hom_{\mathbf{Cat}}(\ho S_\bullet\times \ho T_\bullet,\mathcal{C})\cong\Hom_{\mathbf{Cat}}(\ho S_\bullet,\Fun(\ho T_\bullet,\mathcal{C}))\cong \Hom_{\mathbf{Set}_\Delta}(S_\bullet,N_\bullet(\Fun(\ho T_\bullet,\mathcal{C}))),\\
\Hom_{\mathbf{Cat}}(\ho S_\bullet\times T_\bullet,\mathcal{C})\cong\Hom_{\mathbf{Set}_\Delta}(S_\bullet\times T_\bullet,N_\bullet(\mathcal{C}))\cong\Hom_{\mathbf{Set}_\Delta}(S_\bullet,\Fun(T_\bullet,N_\bullet(\mathcal{C})))
\end{gather*}
we see that this map is in fact given by
\begin{align*}
\Hom_{\mathbf{Set}_\Delta}(S_\bullet,N_\bullet(\Fun(\ho T_\bullet,\mathcal{C})))\stackrel{\rho_{T_\bullet}\circ}{\to}\Hom_{\mathbf{Set}_\Delta}(S_\bullet,\Fun(T_\bullet,N_\bullet(\mathcal{C}))),
\end{align*}
where $\rho_{T_\bullet}$ is the isomorphism appearing in the statement of \cref{simplicial set Fun to nerve isomorphism}.
\end{proof}
We will be primarily interested in the special case where the target simplicial set $T_\bullet$ is an $\infty$-category. In this case, we have the following result:
\begin{theorem}\label{simplicial set Fun of inf-cat is inf-cat}
Let $S_\bullet$ be a simplicial set and $\mathcal{D}$ be an $\infty$-category. Then the simplicial set $\Fun(S_\bullet,\mathcal{D})$ is an $\infty$-category.
\end{theorem}
If $\mathcal{C}$ and $\mathcal{D}$ are $\infty$-categories, by \cref{simplicial set Fun of inf-cat is inf-cat} simplicial set $\Fun(\mathcal{C},\mathcal{D})$ is an $\infty$-category, which is called the $\infty$-category of functors from $\mathcal{C}$ to $\mathcal{D}$. By definition, the objects of the $\infty$-category $\Fun(\mathcal{C},\mathcal{D})$ can be identified with functors from $\mathcal{C}$ to $\mathcal{D}$ (that is, with maps of simplicial sets from $\mathcal{C}$ to $\mathcal{D}$). We say the morphisms of $\Fun(\mathcal{C},\mathcal{D})$ are natural tranformations. That is, if $F,G:\mathcal{C}\to\mathcal{D}$ are objects of $\Fun(\mathcal{C},\mathcal{D})$, we define a natural transformation from $F$ to $G$ to be a map of simplicial sets $u:\Delta^1\times\mathcal{C}\to\mathcal{D}$ satisfying $u|_{\{0\}\times\mathcal{C}}=F$ and $u|_{\{1\}\times\mathcal{C}}=G$.\par
Let us abuse notation by identifying each ordinary category $\mathcal{E}$ with the $\infty$-category $N_\bullet(\mathcal{E})$. In this case, \cref{simplicial set Fun to nerve isomorphism} implies that when $\mathcal{C}$ is an $\infty$-category and $\mathcal{D}$ is an ordinary category, then we have a canonical isomorphism $\Fun(C,D)\cong\Fun(\ho\mathcal{C},\mathcal{D})$. In particular, the functor $\infty$-category $\Fun(\mathcal{C},\mathcal{D})$ is an ordinary category.
\subsection{Digression: lifting properties}
We now review some categorical terminology which will be useful in the proof of \cref{simplicial set Fun of inf-cat is inf-cat}. Let $\mathcal{C}$ be a category. A \textbf{lifting problem} in $\mathcal{C}$ is a commutative diagram
\[\begin{tikzcd}
A\ar[r,"u"]\ar[d,swap,"f"]&X\ar[d,"p"]\\
B\ar[r,"v"]&Y
\end{tikzcd}\]
in $\mathcal{C}$. A solution to the lifting problem $\sigma$ is a morphism $h:B\to X$ in $\mathcal{C}$ satisfying $p\circ h=v$ and $h\circ f=u$, as indicated in the diagram
\[\begin{tikzcd}
A\ar[r,"u"]\ar[d,swap,"f"]&X\ar[d,"p"]\\
B\ar[r,"v"]\ar[ru,dashed,"h"]&Y
\end{tikzcd}\]

Suppose we are given a morphism $f:A\to B$ and $p:X\to Y$ in $\mathcal{C}$. We say that $f$ has the left lifting property with respect to $p$, or that $p$ has the right lifting property with respect to $f$, if, for every pair of morphisms $u:A\to X$ and $v:B\to Y$ satisfying $p\circ u=v\circ f$, the associated lifting problem
\[\begin{tikzcd}
A\ar[r,"u"]\ar[d,swap,"f"]&X\ar[d,"p"]\\
B\ar[r,"v"]\ar[ru,dashed,"h"]&Y
\end{tikzcd}\]
admits a solution (that is, there exists a map $h:B\to X$ satisfying $p\circ h=v$ and $h\circ f=u$).\par
If $S$ is a collection of morphisms in $\mathcal{C}$, we say that a morphism $f:A\to B$ has the \textbf{left lifting property with respect to $\bm{S}$} if it has the left lifting property with respect to every morphism in $S$. Similarly, we say that a morphism $p:X\to Y$ has the \textbf{right lifting property with respect to $\bm{S}$} if it has the right lifting property with respect to every morphism in $S$.
\begin{remark}\label{category lifting problem iff surjectivity}
We note that the lifting problem
\[\begin{tikzcd}
A\ar[r]\ar[d,swap]&X\ar[d]\\
B\ar[r]&Y
\end{tikzcd}\]
in $\mathcal{C}$ can also be translated into the surjectivity of the canonically induced map
\[\Hom_{\mathcal{C}}(B,X)\to\Hom_{\mathcal{C}}(B,X)\times_{\Hom_{\mathcal{C}}(A,Y)}\Hom_{\mathcal{C}}(A,X).\]
\end{remark}
Let $S$ be a collection of morphisms in a category $\mathcal{C}$. We now summarize some closure properties enjoyed by the collection of morphisms which have the left lifting property with respect to $S$. For this. recall that a collection $T$ of morphisms of $\mathcal{C}$ is \textbf{closed under pushouts} if, for every pushout diagram
\[\begin{tikzcd}
A\ar[r]\ar[d,swap,"f"]&A'\ar[d,"f'"]\\
B\ar[r]&B'
\end{tikzcd}\]
in the category $\mathcal{C}$ where the morphism $f$ belongs to $T$, the morphism $f'$ also belongs to $T$.
\begin{proposition}\label{category left lifting closed under pushout}
Let $\mathcal{C}$ be a category which admits pushouts, $S$ be a collection of morphisms of $\mathcal{C}$, and $T$ be the collection of all morphisms of $\mathcal{C}$ having the left lifting property with respect to $S$. Then $T$ is closed under pushouts.
\end{proposition}
\begin{proof}
Suppose we are given a pushout diagram $\sigma$:
\[\begin{tikzcd}
A\ar[r]\ar[d,swap,"f"]&A'\ar[d,"f'"]\\
B\ar[r]&B'
\end{tikzcd}\]
where $f$ belongs to $T$. We want to show that $f'$ also belongs to $T$. For this, we must show that every lifting problem
\[\begin{tikzcd}
A'\ar[r,"u"]\ar[d,swap,"f'"]&X\ar[d,"p"]\\
B'\ar[r,"v"]\ar[ru,dashed]&Y
\end{tikzcd}\]
admits a solution, provided that the morphism $p$ belongs to $S$. Using our assumption that $\sigma$ is a pushout square, we are reduced to solving the associated lifting problem
\[\begin{tikzcd}
A\ar[r,"u\circ g"]\ar[d,swap,"f'"]&X\ar[d,"p"]\\
B\ar[r,"v\circ h"]\ar[ru,dashed]&Y
\end{tikzcd}\]
which is possible by virtue of our assumption that $f$ has the left lifting property with respect to $p$.
\end{proof}
Let $\mathcal{C}$ be a category containing a pair of objects $C$ and $C'$. We say that $C$ is a \textbf{retract} of $C'$ if there exist maps $i:C\to C'$ and $r:C'\to C$ such that $r\circ i=\id_C$. More generally, we say that a morphism $f:C\to D$ of $\mathcal{C}$ is a \textbf{retract} of another morphism $f':C'\to D'$ if it is a retract of $f'$ when viewed as an object of the functor category $\Fun([1],\mathcal{C})$. In other words, we say that $f$ is a retract of $f'$ if there exists a commutative diagram
\[\begin{tikzcd}
C\ar[r,"i"]\ar[d,"f"]&C'\ar[r,"r"]\ar[d,"f'"]&C\ar[d,"f"]\\
D\ar[r,"\bar{i}"]&D'\ar[r,"\bar{r}"]&D
\end{tikzcd}\]
in the category $\mathcal{C}$, where $r\circ i=\id_C$ and $\bar{r}\circ\bar{i}=\id_D$. We say that a collection of morphisms $T$ of $\mathcal{C}$ is closed under retracts if, for every pair of morphisms $f,f'$ in $\mathcal{C}$, if $f$ is a retract of $f'$ and $f'$ belongs to $T$, then $f$ also belongs to $T$.
\begin{proposition}\label{category left lifting closed under retraction}
Let $\mathcal{C}$ be a category, let $S$ be a collection of morphisms of $\mathcal{C}$, and let $T$ be the collection of all morphisms of $\mathcal{C}$ having the left lifting property with respect to $S$. Then $T$ is closed under retracts.
\end{proposition}
\begin{proof}
Let $f'$ be a morphism of $\mathcal{C}$ which belongs to $T$ and let $f$ be a retract of $f'$, so that there exists a commutative diagram
\[\begin{tikzcd}
C\ar[r,"i"]\ar[d,"f"]&C'\ar[r,"r"]\ar[d,"f'"]&C\ar[d,"f"]\\
D\ar[r,"\bar{i}"]&D'\ar[r,"\bar{r}"]&D
\end{tikzcd}\]
with $r\circ i=\id_C$, $\bar{r}\circ\bar{i}=\id_D$. We want to prove that $f$ belongs to $T$. Consider a lifting problem $\sigma$:
\[\begin{tikzcd}
C\ar[r,"u"]\ar[d,swap,"f"]&X\ar[d,"p"]\\
D\ar[r,"v"]\ar[ru,dashed,"h"]&Y
\end{tikzcd}\]
where $p$ belongs to $S$. Our assumption $f'\in T$ ensures that the associated lifting problem
\[\begin{tikzcd}
C'\ar[r,"u\circ r"]\ar[d,swap,"f"]&X\ar[d,"p"]\\
D'\ar[r,"v\circ\bar{r}"]\ar[ru,dashed,"h"]&Y
\end{tikzcd}\]
admits a solution: that is, we can choose a morphism $h':D'\to X$ satisfying $p\circ h'=v\circ\bar{r}$ and $h'\circ g'=u\circ r$. Then the morphism $h=h'\circ\bar{i}$ is then a solution to the lifting problem $\sigma$.
\end{proof}
For every ordinal number $\alpha$, we let $\mathrm{Ord}_{\leq\alpha}=\{\beta:\beta\leq\alpha\}$ denote the collection of all ordinal numbers which are less than or equal to $\alpha$, regarded as a linearly
ordered set. Let $\mathcal{C}$ be a category and let $T$ be a collection of morphisms of $\mathcal{C}$. We say that a morphism $f$ of $\mathcal{C}$ is a \textbf{transfinite composition} of morphisms of $T$ if there exists an ordinal number $\alpha$ and a functor $F:\mathrm{Ord}_{\leq\alpha}\to\mathcal{C}$, given by a collection of objects $\{C_\beta\}_{\beta\leq\alpha}$ and morphisms $\{f_{\gamma,\beta}:C_\beta\to C_\gamma\}_{\beta\leq\gamma}$ with the following properties:
\begin{itemize}
\item For every nonzero limit ordinal $\lambda\leq\alpha$, the functor $F$ exhibits $C_\lambda$ as a colimit of the diagram $(\{C_\beta\}_{\beta<\lambda},\{f_{\gamma,\beta}\}_{\beta\leq\gamma<\lambda})$.
\item For every ordinal $\beta<\alpha$, the morphism $f_{\beta+1,\beta}$ belong to $T$.
\item The morphism $f$ is equal to $f_{\alpha,0}:C_0\to C_\alpha$.
\end{itemize}
We say that $T$ is closed under transfinite composition if, for every morphism $f$ which is a transfinite composition of morphisms of $T$, we have $f\in T$.
\begin{example}
Let $\mathcal{C}$ be a category and let $T$ be a collection of morphisms of $\mathcal{C}$. Then every identity morphism of $\mathcal{C}$ is a transfinite composition of morphisms of $T$ (take $\alpha=0$). In particular, if $T$ is closed under transfinite composition, then it contains every identity morphism of $\mathcal{C}$. Similarly, by taking $\alpha=1$, we see that every morphism of $T$ is a transfinite composition of morphisms of $T$.
\end{example}
\begin{example}
Let $\mathcal{C}$ be a category and let $T$ be a collection of morphisms of $\mathcal{C}$ which contains a pair of composable morphisms $f:C_0\to C_1$ and $g:C_1\to C_2$. Then the composition $g\circ f$ is a transfinite composition of morphisms of $\mathcal{C}$. In particular, if $T$ is closed under transfinite composition, then it is closed under
composition.
\end{example}
\begin{proposition}\label{category left lifting closed under transfinite composition}
Let $\mathcal{C}$ be a category, let $S$ be a collection of morphisms in $\mathcal{C}$, and let $T$ be the collection of all morphisms of $\mathcal{C}$ which have the left lifting property with respect to $S$. Then $T$ is closed under transfinite composition.
\end{proposition}
\begin{proof}
Let $\alpha$ be an ordinal and suppose we are given a functor $\mathrm{Ord}_{\leq\alpha}\to\mathcal{C}$, given by a pair $(\{C_\beta\}_{\beta\leq\alpha},\{f_{\gamma,\beta}:C_\beta\to C_\gamma\}_{\beta\leq\gamma})$ with composition $f=f_{\alpha,0}$. Assume that each of the morphisms $f_{\beta+1,\beta}$ belongs to $T$. We want to show that the morphism $f$ also belongs to $T$. For this, we must show that every lifting problem $\sigma$:
\[\begin{tikzcd}
C_0\ar[r,"u"]\ar[d,swap,"f_{\alpha,0}"]&X\ar[d,"p"]\\
C_\alpha\ar[r,"v"]\ar[ru,dashed]&Y
\end{tikzcd}\]
admits a solution, provided that $p$ belongs to $S$. We construct a collection of morphisms $\{u_\beta:C_\beta\to X\}_{\beta\leq\alpha}$, satisfying the requirements $p\circ u_\beta=v\circ f_{\alpha,\beta}$ and $u_\beta=u_\beta\circ f_{\gamma\beta}$ for $\beta\leq\gamma$, using transfinite recursion. Fix an ordinal $\gamma\leq\alpha$, and assume that the morphisms $\{u_\beta\}_{\beta<\gamma}$ have been constructed. We consider three cases:
\begin{itemize}
\item If $\gamma=0$, we set $u_\gamma=u$.
\item If $\gamma$ is a nonzero limit ordinal, then our hypothesis that $C_\gamma$ is the colimit of the diagram $\{C_\beta\}_{\beta<\gamma}$ guarantees that there is a unique morphism $u_\gamma:C_\gamma\to X$ satisfying $u_\beta=u_\gamma\circ f_{\gamma\circ\beta}$. Moreover, our assumption that the equality $p\circ u_\beta=v\circ f_{\alpha\beta}$ holds for $\beta<\gamma$ guarantees that it also holds for $\beta=\gamma$.
\item Suppose that $\gamma=\beta+1$ is a successor ordinal. In this case, we take $u$ to be any solution to the lifting problem
\[\begin{tikzcd}
C_\beta\ar[r,"u_\beta"]\ar[d,swap,"f_{\beta+1,\beta}"]&X\ar[d,"p"]\\
C_{\beta+1}\ar[ru,dashed]\ar[r,"u\circ f_{\alpha,\beta+1}"]&Y
\end{tikzcd}\]
which exists by virtue of our assumption that $f_{\beta+1,\beta}$ belongs to $T$.
\end{itemize}
We now complete the proof by observing that $u_\alpha$ is a solution to the lifting problem $\sigma$.
\end{proof}
Motivated by the preceding discussion, we introduce the following terminology. Let $\mathcal{C}$ be a category which admits small colimits and let $T$ be a collection of morphisms of $\mathcal{C}$. We say that $T$ is \textbf{weakly saturated} if it is closed under pushouts, retracts, and transfinite composition. By \cref{category left lifting closed under pushout}, \cref{category left lifting closed under retraction} and \cref{category left lifting closed under transfinite composition}, it is clear that we have the following:
\begin{proposition}\label{category left lifting closed weakly saturated}
Let $\mathcal{C}$ be a category which admits small colimits, let $S$ be a collection of morphisms of $\mathcal{C}$, and let $T$ be the collection of all morphisms of $\mathcal{C}$ which have the left lifting property with respect to $S$. Then $T$ is weakly saturated.
\end{proposition}
Let $\mathcal{C}$ be a category and let $T_0$ be a collection of morphisms of $\mathcal{C}$. Then there exists a smallest collection of morphisms $T$ of $\mathcal{C}$ such that $T_0\sub T$ and $T$ is weakly saturated (for example, we can take $T$ to be the intersection of all the weakly saturated collections of morphisms containing $T_0$). The collection $T$ is called the \textbf{weakly saturated collection of morphisms generated by $\bm{T_0}$}, or the \textbf{weak saturation} of $T_0$. It follows from \cref{category left lifting closed weakly saturated} that if every morphism of $T_0$ has the left lifting property with respect to some collection of morphisms $S$, then every morphism of $T$ also has the left lifting property with respect to $S$.
\subsection{Trivial Kan fibrations}
We now return to the realm of simplicial sets. Let $p:X_\bullet\to Y_\bullet$ be a map of simplicial sets. We say that $p$ is a \textbf{trivial Kan fibration} if, for each $n\geq 0$, every lifting problem
\[\begin{tikzcd}
\partial\Delta^n\ar[r]\ar[d,hook]&X_\bullet\ar[d,"p"]\\
\Delta^n\ar[r]&Y_\bullet
\end{tikzcd}\]
All results of the previous paragraph applies to trivial Kan extensions (applied to the opposite category of $\mathbf{Set}_\Delta$). For example, if we are given a pullback diagram of simplicial sets
\[\begin{tikzcd}
X'_\bullet\ar[r]\ar[d,swap,"p'"]&X_\bullet\ar[d,"p"]\\
Y'_\bullet\ar[r]&Y_\bullet
\end{tikzcd}\]
and $p$ is a trivial Kan fibration, then so is $p'$. In fact, it is not hard to see that the collection of trivial Kan fibrations is closed under filtered colimits (when regarded as a full subcategory of the arrow category $\Fun([1],\mathbf{Set}_\Delta))$.
\begin{proposition}\label{simplicial set trivial Kan fibration iff mono}
Let $p:X_\bullet\to Y_\bullet$ be a map of simplicial sets. The following conditions are equivalent:
\begin{itemize}
\item[(\rmnum{1})] $p$ is a trivial Kan fibration.
\item[(\rmnum{2})] $p$ has the right lifting property with respect to every monomorphism of simplicial sets $i:A_\bullet\hookrightarrow B_\bullet$. In other words, every lifting problem
\[\begin{tikzcd}
A_\bullet\ar[r]\ar[d,swap,"i"]&X_\bullet\ar[d,"p"]\\
B_\bullet\ar[r]&Y_\bullet
\end{tikzcd}\] 
admits a solution, provided that $i$ is a monomorphism.
\end{itemize}
\end{proposition}
For the proof of \cref{simplicial set trivial Kan fibration iff mono}, we first prove the following assertion concerning the collection monomorphisms of simplicial sets.
\begin{proposition}\label{simplicial set monomorphism weakly saturated}
Let $T$ be the collection of all monomorphisms in the category $\mathbf{Set}_\Delta$ of simplicial sets. Then:
\begin{itemize}
\item[(a)] The collection $T$ is weakly saturated.
\item[(b)] As a weakly saturated collection of morphisms, $T$ is generated by the collection of inclusion maps $\{\partial\Delta^n\hookrightarrow\Delta^n\}_{n\geq 0}$.
\end{itemize}
\end{proposition}
\begin{proof}
The proof that the collection $T$ is closed under pushouts and retracts are clear. For example, if we are given a pushout diagram of simplicial sets 
\[\begin{tikzcd}
A_\bullet\ar[r]\ar[d,swap,"f"]&A'_\bullet\ar[d,"f'"]\\
B_\bullet\ar[r]&B_\bullet
\end{tikzcd}\]
where $f$ is a monomorphism, then $f'$ is also a monomorphism. This is clear, since we have a pushout diagram
\[\begin{tikzcd}
A_n\ar[r]\ar[d,swap,"f"]&A'_n\ar[d,"f'"]\\
B_n\ar[r]&B_n
\end{tikzcd}\]
in the category of sets for each $n\geq 0$ (where the left vertical map is injective, so the right vertical map is injective as well). To see $T$ is closed under retracts, let $f'$ be a monomorphism and let $f$ be a retract of $f'$, so that there exists a commutative diagram
\[\begin{tikzcd}
A_\bullet\ar[r,"i"]\ar[d,"f"]&A'_\bullet\ar[r,"r"]\ar[d,"f'"]&A_\bullet\ar[d,"f"]\\
B_\bullet\ar[r,"\bar{i}"]&B'_\bullet\ar[r,"\bar{r}"]&B_\bullet
\end{tikzcd}\]
with $r\circ i=\id_{A_\bullet}$, $\bar{r}\circ\bar{i}=\id_{B_\bullet}$. Again, by considering the maps on each $n$-simplicies and a simple diagram chase, we conclude that each $f'_n$ is monic, so $f'$ is also a monomorphism.\par
Now suppose we are given an ordinal $\alpha$ and a functor $S:\mathrm{Ord}_{\leq\alpha}\to\mathbf{Set}_\Delta$, given by a collection of simplicial sets $\{S(\beta)_\bullet\}_{\beta\leq\alpha}$ and transition maps $f_{\gamma,\beta}:S(\beta)_\bullet\to S(\gamma)_\bullet$. Assume that the maps $f_{\beta+1,\beta}$ are monomorphisms for $\beta<\alpha$ and that, for every nonzero limit ordinal $\lambda\leq\alpha$, the induced map $\rlim_{\beta<\lambda}S(\beta)_\bullet\to S(\lambda)_\bullet$ is an isomorphism. We must show that the $f_{\gamma,0}:S(0)_\bullet\to S(\gamma)_\bullet$ is a monomorphism of simplicial sets. In fact, we claim that for each $\gamma\leq\alpha$, the map $f_{\gamma,0}:S(0)_\bullet(0)\to S(\gamma)_\bullet$ is a monomorphism. The proof proceeds by transfinite induction on $\gamma$. In the case $\gamma=0$, the map $f_{\gamma,0}=\id_{S(0)_\bullet}$ is an isomorphism. If $\gamma$ is a nonzero limit ordinal, then the desired result follows from our inductive hypothesis, since the collection of monomorphisms in $\mathbf{Set}_\Delta$ is closed under filtered colimits. If $\gamma=\beta+1$ is a successor ordinal, then we can identify $f_{\gamma,0}$ with the composition
\[\begin{tikzcd}
S(0)_\bullet\ar[r,"f_{\beta,0}"]&S(\beta)_\bullet\ar[r,"f_{\gamma,\beta}"]&S(\gamma)_\bullet
\end{tikzcd}\]
where $f_{\gamma,\beta}$ is a monomorphism by assumption and $f_{\beta,0}$ is a monomorphism by virtue of our inductive hypothesis. This completes the proof of (a).\par
We now prove (b). Let $T'$ be a collection of morphisms in $\mathbf{Set}_\Delta$ which is weakly saturated and contains each of the inclusions $\partial\Delta^n\hookrightarrow\Delta^n$; we show that every monomorphism $i:A_\bullet\hookrightarrow B_\bullet$ belongs to $T'$. For each $k\geq -1$, let $B(k)_\bullet\sub B_\bullet$ denote the simplicial subset given by the union of the skeleton $\sk_k(B_\bullet)$ with the image of $i$. Then the inclusion $i$ can be written as a transfinite composition
\[A_\bullet\cong B(-1)\hookrightarrow B(0)_\bullet\hookrightarrow B(1)_\bullet\hookrightarrow\cdots\]
Since $T'$ is closed under transfinite composition, it will suffice to show that each of the inclusion maps $B(k-1)_\bullet\hookrightarrow B(k)_\bullet$ belongs to $T'$. Applying Proposition~\ref{simplicial set skeleton pushout square} to both $A_\bullet$ and $B_\bullet$, we obtain a pushout diagram
\[\begin{tikzcd}
\coprod_{\sigma\in Q}\partial\Delta^k\ar[r]\ar[d]&\coprod_{\sigma\in Q}\Delta^k\ar[d]\\
B(k-1)_\bullet\ar[r]&B(k)_\bullet
\end{tikzcd}\]
where $Q$ denotes the collection of all nondegenerate $k$-simplices of $B_\bullet$ which do not belong to the image of $i$. Since $T'$ is closed under pushouts, we are reduced to showing that the inclusion map
\[j:\coprod_{\sigma\in Q}\partial\Delta^k\hookrightarrow\coprod_{\sigma\in Q}\Delta^k\]
belongs to $T'$. For this, we choose a well-ordering on the set $Q$. Then $j$ can be written as a transfinite composition of morphisms
\[j_\sigma:\Big(\coprod_{\tau\geq\sigma}\partial\Delta^k\Big)\coprod\Big(\coprod_{\tau<\sigma}\Delta^k\Big)\hookrightarrow\Big(\coprod_{\tau>\sigma}\partial\Delta^k\Big)\coprod\Big(\coprod_{\tau\leq\sigma}\Delta^k\Big)\]
each of which is a pushout of the inclusion $\partial\Delta^k\hookrightarrow\Delta^k$.
\end{proof}
\begin{proof}[\textbf{Proof of \cref{simplicial set trivial Kan fibration iff mono}}]
Let $p:X_\bullet\to Y_\bullet$ be a trivial Kan fibration of simplicial sets and let $T$ be the collection of all morphisms in $\mathbf{Set}_\Delta$ which have the left lifting property with respect to $p$. Then $T$ contains each of the inclusions $\partial\Delta^n\hookrightarrow\Delta^n$ (by virtue of our assumption that $p$ is a trivial Kan fibration) and is weakly saturated (\cref{category left lifting closed weakly saturated}). It follows from \cref{simplicial set monomorphism weakly saturated} that every monomorphism of simplicial sets $i:A_\bullet\hookrightarrow B_\bullet$ belongs to $T$ (and therefore has the left lifting property with respect to $p$).
\end{proof}
\begin{corollary}\label{simplicial set trivial Kan fibration section prop}
Let $p:X_\bullet\to Y_\bullet$ be a trivial Kan fibration of simplicial sets. Then:
\begin{itemize}
\item[(a)] The map $p$ admits a section: that is, there is a map of simplicial sets $s:Y_\bullet\to X_\bullet$ such that the composition $p\circ s=\id_{Y_\bullet}$.
\item[(b)] Let $s$ be any section of $p$. Then the composition $s\circ p:X_\bullet\to X_\bullet$ is fiberwise homotopic to the identity. That is, there exists a map of simplicial sets $h:\Delta^1\times X_\bullet\to X_\bullet$, compatible with the projection to $Y_\bullet$, such that $h|_{\{0\}\times X_\bullet}=s\circ p$ and $h|_{\{1\}\times X_\bullet}=\id_{X_\bullet}$.
\end{itemize}
\end{corollary}
\begin{proof}
To prove (a), we observe that a section of $p$ can be described as a solution to the lifting problem
\[\begin{tikzcd}
\emp\ar[r]\ar[d]&X_\bullet\ar[d,"p"]\\
Y_\bullet\ar[ru,dashed,"s"]\ar[r,"\id"]&Y_\bullet
\end{tikzcd}\]
which exists by virtue of \cref{simplicial set trivial Kan fibration iff mono}. Given any section $s$, a fiberwise homotopy from $s\circ p$ to the identity can be identified with a solution to the lifting problem
\[\begin{tikzcd}
\partial\Delta^1\times X_\bullet\ar[d]\ar[r,"{(s\circ p,\id)}"]&X_\bullet\ar[d,"p"]\\
\Delta^1\times X_\bullet\ar[ru,dashed,"h"]\ar[r]&Y_\bullet
\end{tikzcd}\]
where the bottom horizontal arrow is given by the map $\Delta^1\to\Delta^0$ and $p:X_\bullet\to Y_\bullet$, This solution again exists by virtue of \cref{simplicial set trivial Kan fibration iff mono}.
\end{proof}
\begin{corollary}\label{simplicial set trivial Kan fibration Fun of mono}
Let $p:X_\bullet\to Y_\bullet$ be a trivial Kan fibration of simplicial sets and let $i:A_\bullet\to B_\bullet$ be a monomorphism of simplicial sets. Then the canonical map
\[\theta:\Fun(B_\bullet,X_\bullet)\to\Fun(B_\bullet,Y_\bullet)\times_{\Fun(A_\bullet,Y_\bullet)}\Fun(A_\bullet,X_\bullet)\]
is also a trivial Kan fibration.
\end{corollary}
\begin{proof}
Fix an integer $n\geq 0$; we show that every lifting problem
\[\begin{tikzcd}[column sep=6mm]
\partial\Delta^n\ar[r]\ar[d]&\Fun(B_\bullet,X_\bullet)\ar[d,"\theta"]\\
\Delta^n\ar[r]\ar[ru,dashed]&\Fun(B_\bullet,Y_\bullet)\times_{\Fun(A_\bullet,Y_\bullet)}\Fun(A_\bullet,X_\bullet)
\end{tikzcd}\]
admits a solution. Unwinding the definitions, we see that this is equivalent to the surjectivity of the canonical map (where $\Hom$ is taken in the category $\mathbf{Set}_\Delta$)
\begin{align}\label{simplicial set trivial Kan fibration Fun of mono-1}
\Hom(\Delta^n,\Fun(B_\bullet,X_\bullet))\to\Hom(\partial\Delta^n,\Fun(B_\bullet,X_\bullet))\times_{\Hom(\partial\Delta^n,F_\bullet)}\Hom(\Delta^n,F_\bullet)
\end{align}
where $F_\bullet=\Fun(B_\bullet,Y_\bullet)\times_{\Fun(A_\bullet,Y_\bullet)}\Fun(A_\bullet,X_\bullet)$. By \cref{simplicial set Fun evaluation map adjoint prop}, the right side of (\ref{simplicial set trivial Kan fibration Fun of mono-1}) is equal to the limit of the following diagram
\[\begin{tikzcd}[row sep=20pt, column sep=1pt]
&&&\Hom(\Delta^n\times A_\bullet,X_\bullet)\ar[dd]\ar[ld]\\
\Hom(\Delta^n\times B_\bullet,Y_\bullet)\ar[rr]\ar[dd]&&\Hom(\Delta^n\times A_\bullet,Y_\bullet)&\\
&\Hom(\partial\Delta^n\times B_\bullet,X_\bullet)\ar[ld]&&\Hom(\partial\Delta^n\times A_\bullet,X_\bullet)\ar[ld]\\
\Hom(\partial\Delta^n\times B_\bullet,Y_\bullet)\ar[rr]&&\Hom(\partial\Delta^n\times A_\bullet,Y_\bullet)&
\arrow[from=3-2,to=3-4]
\arrow[from=2-3,to=4-3,crossing over]
\end{tikzcd}\]
However, by first taking products of sets of the form $\Hom(-,X_\bullet)$ and $\Hom(-,Y_\bullet$) respectively, we see that this limit can also be written as
\[\Hom(\Delta^n\times B_\bullet,Y_\bullet)\times_{\Hom((\partial\Delta^n\times B_\bullet)\amalg_{\partial\Delta^n\times A_\bullet}(\Delta^n\times A_\bullet),Y_\bullet)}\Hom((\partial\Delta^n\times B_\bullet)\amalg_{\partial\Delta^n\times A_\bullet}(\Delta^n\times A_\bullet),X_\bullet),\]
so by \cref{category lifting problem iff surjectivity} the surjectivity of (\ref{simplicial set trivial Kan fibration Fun of mono-1}) is then equivalent to solving an associated lifting problem
\[\begin{tikzcd}[column sep=6mm]
(\partial\Delta^n\times B_\bullet)\coprod_{\partial\Delta^n\times A_\bullet}(\Delta^n\times A_\bullet)\ar[r]\ar[d,"i"]&X_\bullet\ar[d,"p"]\\
\Delta^n\times B_\bullet\ar[r]\ar[ru,dashed]&Y_\bullet
\end{tikzcd}\]
This is possible by virtue of \cref{simplicial set trivial Kan fibration iff mono}, since $p$ is a trivial Kan fibration and $i$ is a monomorphism.
\end{proof}
\begin{corollary}\label{simplicial set Fun(B -) of trivial Kan fibration is fibration}
Let $p:X_\bullet\to Y_\bullet$ be a trivial Kan fibration of simplicial sets. Then, for every simplicial set $B_\bullet$, the induced map $\Fun(B_\bullet,X_\bullet)\to\Fun(B_\bullet,Y_\bullet)$ is a trivial Kan fibration.
\end{corollary}
\begin{proof}
Apply \ref{simplicial set trivial Kan fibration Fun of mono} in the special case $A_\bullet=\emp$.
\end{proof}
Let $X_\bullet$ be a simplicial set. We say that $X_\bullet$ is a \textbf{contractible Kan complex} if the projection map $X_\bullet\to\Delta^0$ is a trivial Kan fibration. In other words, this means every map $\sigma_0:\partial\Delta^n\to X_\bullet$ can be extended to an $n$-simplex of $X$.
\begin{example}
Let $X$ be a topological space. Then the singular simplicial set  $\Sing_\bullet(X)$ is a contractible Kan complex if and only if the space $X$ is weakly contractible: that is, if and only if every continuous map $\sigma_0:S^{n-1}\to X$ is nullhomotopic (here $S^{n-1}\cong|\partial\Delta^n|$ denotes the sphere of dimension $n-1$, so that $\sigma_0$ is nullhomotopic if and only if it extends to a continuous map defined on the disk $D^n\cong|\Delta^n|$). In particular, if the topological space $X$ is contractible, then the simplicial set $\Sing_\bullet(X)$ is a contractible Kan complex.
\end{example}
\begin{remark}\label{simplicial set trivial Kan fibarion fiberwise contractible}
Let $p:X_\bullet\to Y_\bullet$ be a trivial Kan fibration. Then, for every vertex $y$ of $Y_\bullet$, the fiber $X_\bullet\times_{Y_\bullet}\{y\}$ of $X_\bullet$ at $y$ is a contractible Kan complex. This follows, for example, from the pullback diagram
\[\begin{tikzcd}
X_\bullet\times_{Y_\bullet}\{y\}\ar[d]\ar[r]&X_\bullet\ar[d,"p"]\\
\{y\}\ar[r]&Y_\bullet
\end{tikzcd}\]
\end{remark}
Applying \cref{simplicial set trivial Kan fibration iff mono} in the case $Y_\bullet=\Delta^0$, we obtain the following:
\begin{corollary}\label{simplicial set contractible Kan iff extension of mono}
Let $X_\bullet$ be a simplicial set. Then $X_\bullet$ is a contractible Kan complex if and only if for every monomorphism of simplicial sets $i:A_\bullet\to B_\bullet$ and every map of simplicial sets $f_0:A_\bullet\to X_\bullet$, there exists a map $f:B_\bullet\to X_\bullet$ such that $f_0=f\circ i$.
\end{corollary}
\begin{corollary}
Let $X_\bullet$ be a contractible Kan complex. Then $X_\bullet$ is a Kan complex, and in particular is an $\infty$-category.
\end{corollary}
\subsection{Uniqueness of composition}
Let $\mathcal{C}$ be an $\infty$-category. Given a composable pair of morphisms $f:X\to Y$ and $g:Y\to Z$ in $\mathcal{C}$, one can form a composition $g\circ f$ by choosing a $2$-simplex $\sigma$ with $d_0(\sigma)=g$ and $d_2(\sigma)=f$, as indicated in the diagram
\[\begin{tikzcd}
&Y\ar[rd,"g"]&\\
X\ar[ru,"f"]\ar[rr,dashed,"g\circ f"]&&Z
\end{tikzcd}\]
In general, neither the $2$-simplex $\sigma$ nor the resulting morphism $g\circ f=d_1(\sigma)$ is uniquely determined. However, we have seen that the composition $g\circ f$ is unique up to homotopy (\cref{simplicial set inf-cat composition unique up to homotopy}). We now prove a stronger result, which asserts that the $2$-simplex $\sigma$ (hence also the composite morphism $g\circ f=d_1(\sigma)$) is unique up to a contractible space of choices.
\begin{theorem}[\textbf{Joyal}]\label{simplicial set inf-cat iff Fun(Delta^2 -) is trivial Kan fibration}
Let $S_\bullet$ be a simplicial set. Then $S_\bullet$ is an $\infty$-category if and only if the inclusion of simplicial sets $\Lambda^2_1\hookrightarrow\Delta^2$ induces a trivial Kan fibration
\[\Fun(\Delta^2,S_\bullet)\to\Fun(\Lambda^2_1,S_\bullet).\]
\end{theorem}
For the proof of \cref{simplicial set inf-cat iff Fun(Delta^2 -) is trivial Kan fibration}, we need to introduce the following terminology. Let $f:A_\bullet\to B_\bullet$ be a morphism of simplicial sets. We say that $f$ is \textbf{inner anodyne} if it belongs to the weakly saturated class of morphisms generated by the collection of all inner horn inclusions $\Lambda^n_i\hookrightarrow\Delta^n$ (so that $0<i<n$). Note that if this case $f$ is necessarily a monomorphism, since the collection of monomorphisms is weakly saturated by \cref{simplicial set monomorphism weakly saturated}. On the other hand, it is not hard to see that the underlying map $f_0:A_0\to B_0$ on verticies is bijective.
\begin{proposition}\label{simplicial set inf-cat iff lifting for inner anodyne}
Let $S_\bullet$ be a simplicial set. The $S_\bullet$ is an $\infty$-category if and only if for every inner anodyne map of simplicial sets $i:A_\bullet\hookrightarrow B_\bullet$ and every map $f_0:A_\bullet\to S_\bullet$ there exists a map $f:B_\bullet\to S_\bullet$ such that $f_0=f\circ i$.
\end{proposition}
\begin{proof}
Since each inner horn inclusion $\Lambda^n_i\hookrightarrow\Delta^n$ is inner anodyne, one direction is clear. Conversely, if $S_\bullet$ is an $\infty$-category. then every inner horn inclusion $\Lambda_i^n\hookrightarrow\Delta^n$ has the left lifting property with respect to the projection map $p:S_\bullet\to\Delta^0$. It then follows from \cref{category left lifting closed weakly saturated} that every inner anodyne map has the left lifting property with respect to $p$.
\end{proof}
\begin{corollary}\label{simplicial set nerve of cat iff unique lifting for inner anodyne}
Let $S_\bullet$ be a simplicial set. Then the simplicial set $S_\bullet$ is isomorphic to the nerve of a category if and only if for every inner anodyne map of simplicial sets $i:A_\bullet\hookrightarrow B_\bullet$ and every map $f_0:A_\bullet\to S_\bullet$, there exists a unique map $f:B_\bullet\to S_\bullet$ such that $f_0=f\circ i$.
\end{corollary}
\begin{proof}
This follows from \cref{simplicial set isomorphic to nerve iff condition Ner} and the proof of \cref{simplicial set inf-cat iff lifting for inner anodyne}.
\end{proof}
We now establish the following lemma about inner anodyne maps that will be used in the proof of \cref{simplicial set inf-cat iff Fun(Delta^2 -) is trivial Kan fibration}.
\begin{lemma}[\textbf{Joyal}]\label{simplicial set inner anodyne generator}
\mbox{}
\begin{itemize}
\item[(a)] For every monomorphism of simplicial sets $i:A_\bullet\hookrightarrow B_\bullet$, the induced map
\[(B_\bullet\times\Lambda^2_1)\coprod_{A_\bullet\times\Lambda^2_1}(A_\bullet\times\Delta^2)\hookrightarrow B_\bullet\times\Delta^2\]
is inner anodyne.
\item[(b)] The collection of inner anodyne morphisms is generated (as a weakly saturated class) by the inclusion maps
\[(\Delta^n\times\Lambda^2_1)\coprod_{\partial\Delta^n\times\Lambda^2_1}(\partial\Delta^n\times\Delta^2)\hookrightarrow\Delta^n\times\Delta^2.\] 
\end{itemize}
\end{lemma}
\begin{proof}

\end{proof}
\begin{proof}[\textbf{Proof of \cref{simplicial set inf-cat iff Fun(Delta^2 -) is trivial Kan fibration}}]
Let $S_\bullet$ be a simplicial set and let $p:\Fun(\Delta^2,S_\bullet)\to\Fun(\Lambda^2_1,S_\bullet)$ denote the restriction map. Then $p$ is a trivial Kan fibration if and only if every lifting problem
\[\begin{tikzcd}
\partial\Delta^n\ar[d]\ar[d]\ar[r]&\Fun(\Delta^2,S_\bullet)\ar[d,"p"]\\
\Delta^n\ar[r]\ar[ru,dashed]&\Fun(\Lambda^2_1,S_\bullet)
\end{tikzcd}\]
admits a solution. Unwinding the definitions, we see that this is equivalent to surjectivity of the canonical map from $\Hom_{\mathbf{Set}_\Delta}(\Delta^n,\Fun(\Delta^2,S_\bullet))$ to the product set
\[\Hom_{\mathbf{Set}_\Delta}(\Delta^n,\Fun(\Lambda^2_1,S_\bullet))\times_{\Hom_{\mathbf{Set}_\Delta}(\partial\Delta^n,\Fun(\Lambda^2_1,S_\bullet))}\Hom_{\mathbf{Set}_\Delta}(\partial\Delta^n,\Fun(\Delta^2,S_\bullet)).\]
By \cref{simplicial set Fun evaluation map adjoint prop}, this is in turn equivalent to the requirement that every lifting problem of the form
\[\begin{tikzcd}[column sep=6mm]
(\Delta^n\times\Lambda^2_1)\coprod_{\partial\Delta^n\times\Lambda^2_1}(\partial\Delta^n\times\Delta^2)\ar[d]\ar[r]&S_\bullet\ar[d]\\
\Delta^n\times\Delta^2\ar[r]\ar[ru,dashed]&\Delta^0
\end{tikzcd}\]
admits a solution. Let $T$ be the collection of all morphisms of simplicial sets which have the left lifting property with respect to the projection $S_\bullet\to\Delta^0$. Then $p$ is a trivial Kan fibration if and only if $T$ contains each of the inclusion map
\[(\Delta^n\times\Lambda^2_1)\coprod_{\partial\Delta^n\times\Lambda^2_1}(\partial\Delta^n\times\Delta^2)\hookrightarrow\Delta^n\times\Delta^2.\]
Since $T$ is weakly saturated by \cref{category left lifting closed weakly saturated}, this is equivalent to the requirement that $T$ contains all inner anodyne morphisms (\cref{simplicial set inner anodyne generator}), which is in turn equivalent to the requirement that $S_\bullet$ is an $\infty$-category (\cref{simplicial set inf-cat iff lifting for inner anodyne}).
\end{proof}
\begin{corollary}\label{simplicial set inf-cat Fun(Delta^2 -) fiberwise contractible}
Let $f:X\to Y$ and $g:Y\to Z$ be a composable pair of morphisms in an $\infty$-category $\mathcal{C}$, so that the tuple $(g,\ast,f)$ determines a map of simplicial sets $\sigma_0:\Lambda^2_1\to\mathcal{C}$. Then the fiber product $\Fun(\Delta^2,\mathcal{C})\times_{\Fun(\Lambda^2_1,\mathcal{C})}\{\sigma_0\}$ is a contractible Kan complex.
\end{corollary}
\begin{proof}
This follows by combining \cref{simplicial set inf-cat iff Fun(Delta^2 -) is trivial Kan fibration} and \cref{simplicial set trivial Kan fibarion fiberwise contractible}.
\end{proof}
\begin{remark}
In the situation of \cref{simplicial set inf-cat Fun(Delta^2 -) fiberwise contractible}, one can think of the simplicial set
\[Z_\bullet=\Fun(\Delta^2,\mathcal{C})\times_{\Fun(\Lambda^2_1,\mathcal{C})}\{\sigma_0\}\]
as a "parameter space" for all choices of $2$-simplex $\sigma$ satisfying $d_0(\sigma)=g$ and $d_2(\sigma)=f$ (note that such $2$-simplices can be identified with the vertices of $Z_\bullet$). Consequently, we can summarize \cref{simplicial set inf-cat Fun(Delta^2 -) fiberwise contractible} informally by saying that this parameter space is contractible.
\end{remark}
By the characterization of \cref{simplicial set inf-cat iff Fun(Delta^2 -) is trivial Kan fibration}, we can now give the proof of \cref{simplicial set Fun of inf-cat is inf-cat}, which asserts that the simplicial set $\Fun(S_\bullet,\mathcal{D})$ is an $\infty$-category if $\mathcal{D}$ is.
\begin{proof}[\textbf{Proof of \cref{simplicial set Fun of inf-cat is inf-cat}}]
Let $S_\bullet$ be a simplicial set and let $\mathcal{D}$ be an $\infty$-category. By \cref{simplicial set inf-cat iff Fun(Delta^2 -) is trivial Kan fibration}, it will suffice to show that the restriction map
\[r:\Fun(\Delta^2,\Fun(S_\bullet,\mathcal{D}))\to\Fun(\Lambda^2_1,\Fun(S_\bullet,\mathcal{D}))\]
is a trivial Kan fibration. But note that we can identify $r$ with the canonical map
\[\Fun(S_\bullet,\Fun(\Delta^2,\mathcal{D}))\to\Fun(S_\bullet,\Fun(\Lambda^2_1,\mathcal{D}))\]
which is a trivial Kan fibration by virtue of \cref{simplicial set inf-cat iff Fun(Delta^2 -) is trivial Kan fibration} and \cref{simplicial set Fun(B -) of trivial Kan fibration is fibration}.
\end{proof}
\subsection{Universality of path categories}
Let $G$ be a directed graph, let $G_\bullet$ denote the associated $1$-dimensional simplicial set, and let $P[G]$ denote the path category of $G$. There is an evident map of simplicial sets $u:G_\bullet\to N_\bullet(P[G])$. By virtue of \cref{simplicial set graph homotopy cat is path cat}, this map exhibits $P[G]$ as the homotopy category of the simplicial set $G_\bullet$. In other words, the path category $P[G]$ is universal among categories $\mathcal{C}$ which are equipped with a $G_\bullet$-indexed diagram. Our goal in this paragraph is to establish a variant of this statement in the setting of $\infty$-categories:
\begin{theorem}\label{simplicial set inf-cat indexed diagram universal by path cat}
Let $G$ be a directed graph and let $\mathcal{C}$ be an $\infty$-category. Then composition with the map of simplicial sets $u:G_\bullet\to N_\bullet(P[G])$ induces a trivial Kan fibration of simplicial sets $\Fun(N_\bullet(P[G]),\mathcal{C})\to\Fun(G_\bullet,\mathcal{C})$.
\end{theorem}
More informally, \cref{simplicial set inf-cat indexed diagram universal by path cat} asserts that any $G$-indexed diagram in an $\infty$-category $\mathcal{C}$ admits an essentially unique extension to a functor of $\infty$-categories $N_\bullet(P[G])\to\mathcal{C}$.
\begin{example}
Let $G$ be the directed graph depicted in the diagram
\[\begin{tikzcd}
\bullet\ar[r]&\bullet\ar[r]&\bullet
\end{tikzcd}\]
Then the map $u:G_\bullet\to N_\bullet(P[G])$ can be identified with the inclusion of simplicial sets $\Lambda^2_1\hookrightarrow\Delta^2$. In this case, \cref{simplicial set inf-cat indexed diagram universal by path cat} reduces to the statement that the map
\[\Fun(\Delta^2,\mathcal{C})\to\Fun(\Lambda^2_1,\mathcal{C})\]
is a trivial Kan fibration, which is equivalent to the assumption that $\mathcal{C}$ is an $\infty$-category by virtue of \cref{simplicial set inf-cat iff Fun(Delta^2 -) is trivial Kan fibration}.
\end{example}
We can in fact prove a precise version (or a somewhat stronger version) of \cref{simplicial set inf-cat indexed diagram universal by path cat}: the canonical map $u:G_\bullet\hookrightarrow N_\bullet(P[G])$ is inner anodyne. As we shall see, this then implies \cref{simplicial set inf-cat indexed diagram universal by path cat}.
\begin{proposition}\label{simplicial set path cat canonical map inner anodyne}
Let $G$ be a directed graph. Then the map of simplicial sets $u:G_\bullet\hookrightarrow N_\bullet(P[G])$ is inner anodyne.
\end{proposition}
\begin{proof}

\end{proof}
\begin{remark}
Let $G$ be a directed graph and let $\mathcal{C}$ be an ordinary category. Combining \cref{simplicial set path cat canonical map inner anodyne} with \cref{simplicial set nerve of cat iff unique lifting for inner anodyne} we deduce that the canonical map
\[\Hom_{\mathbf{Set}_\Delta}(N_\bullet(P[G]),N_\bullet(\mathcal{C}))\to\Hom_{\mathbf{Set}_\Delta}(G_\bullet,N_\bullet(\mathcal{C}))\]
is bijective. Combining this observation with \cref{simplicial set nerve of cat fully faithful}, we obtain a bijection
\[\Hom_{\mathbf{Cat}}(P[G],\mathcal{C})\to\Hom_{\mathbf{Set}}(G_\bullet,N_\bullet(\mathcal{C})).\]
Allowing $\mathcal{C}$ to vary, we recover the assertion that $u:G_\bullet\to N_\bullet(P[G])$ exhibits $P[G]$ as the homotopy category of $G_\bullet$.
\end{remark}
To derive \cref{simplicial set inf-cat indexed diagram universal by path cat} from \cref{simplicial set path cat canonical map inner anodyne}, we now prove that for an $\infty$-category $\mathcal{C}$, the induced map $\Fun(Y_\bullet,\mathcal{C})\to\Fun(X_\bullet,\mathcal{C})$ is a trivial Kan fibartion whenever $X_\bullet\hookrightarrow Y_\bullet$ is inner anodyne (this is a generalization of \cref{simplicial set inf-cat iff Fun(Delta^2 -) is trivial Kan fibration}). To this end, we need the following lemma:
\begin{lemma}\label{simplicial set inner anodyne product prop}
Let $f:X_\bullet\hookrightarrow Y_\bullet$ and $f':X'_\bullet\hookrightarrow Y'_\bullet$ be monomorphisms of simplicial sets. If $f$ is inner anodyne, then the induced map
\[u_{f,f'}:(Y_\bullet\times X'_\bullet)\coprod_{(X_\bullet\times X'_\bullet)}(X_\bullet\times Y'_\bullet)\hookrightarrow Y_\bullet\times Y'_\bullet\]
is inner anodyne.
\end{lemma}
\begin{proof}
Let us fix the map $f':X'_\bullet\hookrightarrow Y'_\bullet$. Let $T$ be the collection of all morphisms $f:X_\bullet\to Y_\bullet$ for which the map $u_{f,f'}$ is inner anodyne. Then $T$ is weakly saturated. To prove the lemma, we must show that $T$ contains all inner anodyne morphisms of simplicial sets. By virtue of \cref{simplicial set inner anodyne generator}, it will suffice to show that $T$ contains every morphism of the form
\[u_{i,j}:(B_\bullet\times\Lambda^2_1)\coprod_{A_\bullet\times\Lambda^2_1}(A_\bullet\times\Delta^2)\hookrightarrow B_\bullet\times\Delta^2,\]
where $i:A_\bullet\hookrightarrow B_\bullet$ is a monomorphism of simplicial sets and $j:\Lambda^2_1\hookrightarrow\Delta^2$ is the inclusion. Setting
\[A'_\bullet=(B_\bullet\times X'_\bullet)\coprod_{(A_\bullet\times X'_\bullet)}(A_\bullet\times Y'_\bullet),\quad B'_\bullet=B_\bullet\times Y'_\bullet\]
we are reduced to the problem of showing that the map
\[u_{i',j}:(B'_\bullet\times\Lambda^2_1)\coprod_{(A'_\bullet\times\Lambda^2_1)}(A'_\bullet\times\Delta^2)\hookrightarrow B'_\bullet\times\Delta^2\]
is inner anodyne, which follows from \cref{simplicial set inner anodyne generator}.
\end{proof}
\begin{proposition}\label{simplicial set Fun(- C) on inner anodyne trivial Kan fibration}
Let $\mathcal{C}$ be an $\infty$-category and let $f:X_\bullet\hookrightarrow Y_\bullet$ be an inner anodyne morphism of simplicial sets. Then the induced map $p:\Fun(Y_\bullet,\mathcal{C})\to\Fun(X_\bullet,\mathcal{C})$ is a trivial Kan fibration.
\end{proposition}
\begin{proof}
To show that $p$ is a trivial Kan fibration, it suffices to show that it has the right lifting property with respect to every monomorphism of simplicial sets $f':X'_\bullet\hookrightarrow Y'_\bullet$. By \cref{category lifting problem iff surjectivity}, this is equivalent to the assertion that every map of simplicial sets
\[g_0:(Y_\bullet\times X'_\bullet)\coprod_{(X_\bullet\times X'_\bullet)}(X_\bullet\times Y'_\bullet)\to\mathcal{C}\]
can be extended to a map $g:Y_\bullet\times Y'_\bullet\to\mathcal{C}$. This follows from \cref{simplicial set inf-cat iff lifting for inner anodyne}, since $\mathcal{C}$ is an $\infty$-category and the map
\[u_{f,f'}:(Y_\bullet\times X'_\bullet)\coprod_{(X_\bullet\times X'_\bullet)}(X_\bullet\times Y'_\bullet)\hookrightarrow Y_\bullet\times Y'_\bullet\]
is inner anodyne (\cref{simplicial set inner anodyne product prop}).
\end{proof}
\begin{proof}[\textbf{Proof of \cref{simplicial set inf-cat indexed diagram universal by path cat}}]
Let $G$ be a graph and let $\mathcal{C}$ be an $\infty$-category; we want to show that the canonical map
\[\Fun(N_\bullet(P[G]),\mathcal{C})\to\Fun(G_\bullet,\mathcal{C})\]
is a trivial Kan fibration. This follows from \cref{simplicial set Fun(- C) on inner anodyne trivial Kan fibration}, since the inclusion $G_\bullet\hookrightarrow N_\bullet(P[G])$ is inner anodyne (\cref{simplicial set path cat canonical map inner anodyne}).
\end{proof}
\chapter{Examples of \boldmath\texorpdfstring{$\infty$}{inf}-categories}
In \cref{simplicial set inf-cat language chapter}, we introduced the notion of an $\infty$-category: that is, a simplicial set which satisfies the weak Kan extension condition. The theory of $\infty$-categories can be understood as a synthesis of classical category theory and algebraic topology. This perspective is supported by the two main examples of $\infty$-categories that we have encountered so far:
\begin{itemize}
\item Every ordinary category $\mathcal{C}$ can be regarded as an $\infty$-category, by identifying $\mathcal{C}$ with the simplicial set $N_\bullet(\mathcal{C})$.
\item Every Kan complex is an $\infty$-category. In particular, for every topological space $X$, the singular simplicial set $\Sing_\bullet(X)$ is an $\infty$-category.
\end{itemize}
Beware that, individually, both of these examples are rather special. An $\infty$-category $\mathcal{C}$ can be regarded as a mathematical structure which encodes information not only about objects and morphisms (given by the vertices and edges of $\mathcal{C}$, respectively), but also about homotopies between morphisms. When $\mathcal{C}$ is (the nerve of) an ordinary category, the notion of homotopy is trivial: two morphisms in $\mathcal{C}$ (having the same source and target) are homotopic if and only if they are identical. On the other hand, if $\mathcal{C}$ is a Kan complex, then every morphism in $\mathcal{C}$ is invertible up to homotopy (\cref{simplicial set nerve of cat Kan iff groupoid}); from a category-theoretic perspective, this is a very restrictive condition.\par
Our goal in this chapter is to supply a larger class of examples of $\infty$-categories, which are more representative of the subject as a whole. To this end, we introduce three variants of the nerve construction $\mathcal{C}\mapsto N_\bullet(\mathcal{C})$ which can be used to produce $\infty$-categories out of other (possibly more familiar) mathematical structures.
\section{Monoidal categories}
Recall that a monoid is a set $M$ endowed with a map $m:M\times M\to M$ which satisfies the following conditions:
\begin{itemize}
\item The multiplication $m$ is associative. That is, we have $x(yz)=(xy)z$ for each triple of elements $x,y,z\in M$.
\item There exists an element $e\in M$ such that $ex=x=xe$ for each $x\in M$ (in this case, the element $e$ is uniquely determined, and is called the \textbf{unit element} of $M$).
\end{itemize}
\begin{example}
Let $\mathcal{C}$ be a category and let $X$ be an object of $\mathcal{C}$. An endomorphism of $X$ is a morphism from $X$ to itself in the category $\mathcal{C}$. We let $\End_{\mathcal{C}}(X)=\Hom_{\mathcal{C}}(X,X)$ denote the set of all endomorphisms of $X$. The composition law on $\mathcal{C}$ determines a map
\[\End_{\mathcal{C}}(X)\times\End_{\mathcal{C}}(X),\quad (f,g)\mapsto f\circ g\]
which exhibits $\End_{\mathcal{C}}(X)$ as a monoid; the unit element of $\End_{\mathcal{C}}(X)$ is the identity morphism $\id_X:X\to X$. We refer to $\End_{\mathcal{C}}(X)$ as the endomorphism monoid of $X$.
\end{example}
\begin{example}\label{monoidal cat Vect_k eg}
Let $k$ be a field. The construction $(U,V)\mapsto U\otimes_kV$ then determines a functor
\[\otimes_k:\mathbf{Vect}_k\times\mathbf{Vect}_k\to\mathbf{Vect}_k\]
which is called the \textbf{tensor product functor}. It is associative in the following sense: for every triple of vector spaces $U,V,W\in\mathbf{Vect}_k$, there exists a canonical isomorphism
\[U\otimes_k(V\otimes_kW)\stackrel{\sim}{\to}(U\otimes_kV)\otimes_kW,\quad u\otimes(v\otimes w)\mapsto(u\otimes v)\otimes w.\]
\end{example}

Our goal in this section is to review the theory of monoidal categories, which axiomatizes the essential features of Example~\ref{monoidal cat Vect_k eg}. To simplify the discussion, we begin by developing the nonunital version of this theory; that is, nonunital monoidal categories. After this, we define the concept of units in a nonunital monoidal category, and show that units are unique up to isomorphisms. We then define a monoidal category to be a nonunital monoidal category $\mathcal{C}$ together with a choice of unit. A basic prototype is the category $\mathbf{Vect}_k$ of vector spaces over a field $k$ (equipped with the tensor product and associativity constraints given in Example~\ref{monoidal cat Vect_k eg}, and the unit given by the object $k\in\mathbf{Vect}_k$).\par
We conclude this section with a brief review of enriched category theory. If $\mathcal{A}$ is a monoidal category, then an $\mathcal{A}$-enriched category $\mathcal{C}$ consists of a collection $\Ob(\mathcal{C})$ of objects of $\mathcal{C}$, a collection of mapping objects $\sHom_{\mathcal{C}}(X,Y)\in\mathcal{A}$ for each pair of objects $X,Y\in\Ob(\mathcal{C})$, and a composition law
\[\sHom_{\mathcal{C}}(Y,Z)\otimes\sHom_{\mathcal{C}}(X,Y)\to\sHom_{\mathcal{C}}(X,Z)\]
which is required to be unital and associative. Enriched category theory will play an important role throughout this chapter: we will be particularly interested in the special case where $\mathcal{A}=\mathbf{Cat}$ is the category of small categories (in which case we recover the notion of strict $2$-category), where $\mathcal{A}=\mathbf{Set}_\Delta$ is the category of simplicial sets (in which case we recover the notion of simplicial category), and where $\mathcal{A}=\mathrm{Ch}(\mathbf{Ab})$ is the category of chain complexes of abelian groups (in which case we recover the notion of differential graded category).
\subsection{Monoidal categories}
Let $\mathcal{C}$ be a category. We begin with the nonunitial version of monoidal categories: a \textbf{nonunital strict monoidal structure} on $\mathcal{C}$ is defined to be a functor
\[\otimes:\mathcal{C}\times\mathcal{C}\to\mathcal{C},\quad (X,Y)\mapsto X\otimes Y\]
which is \textit{strictly associative} in the following sense:
\begin{itemize}
\item For every triple of objects $X,Y,Z\in\mathcal{C}$, we have an equality $X\otimes(Y\otimes Z)=(X\otimes Y)\otimes Z$ (as objects of $\mathcal{C}$).
\item For every triple of morphisms $f:X\to X'$, $g:Y\to Y'$, $h:Z\to Z'$, we have an equality
\[f\otimes(g\otimes h)=(f\otimes g)\otimes h\]
of morphisms in $\mathcal{C}$ from the object $X\otimes(Y\otimes Z)=(X\otimes Y)\otimes Z$ to the object $X'\otimes(Y'\otimes Z')=(X'\otimes Y')\otimes Z'$.
\end{itemize}
A \textbf{nonunital strict monoidal category} is a pair $(\mathcal{C},\otimes)$, where $\mathcal{C}$ is a category and $\otimes:\mathcal{C}\times\mathcal{C}\to\mathcal{C}$ is a nonunital strict monoidal structure on $\mathcal{C}$.
\begin{example}
Let $M$ be a set, which we regard as a category having only identity morphisms. Then nonunital strict monoidal structures on $M$ can be identified with nonunital monoid structures on $M$. In particular, any nonunital monoid can be regarded as a nonunital strict monoidal category (having only identity morphisms).
\end{example}
\begin{example}[\textbf{Endomorphism Categories}]\label{monoidal cat on End of cat}
Let $\mathcal{C}$ be a category, and let $\End(\mathcal{C})=\Fun(\mathcal{C},\mathcal{C})$ denote the category of functors from $\mathcal{C}$ to itself. Then the composition functor
\[\circ:\Fun(\mathcal{C},\mathcal{C})\times\Fun(\mathcal{C},\mathcal{C})\to\Fun(\mathcal{C},\mathcal{C}),\quad (F,G)\mapsto F\circ G\]
is a nonunital strict monoidal structure on $\End(\mathcal{C})$.
\end{example}
For many purposes, the strict associativity is too restrictive. Note that if $k$ is a field, then the tensor product functor $\otimes_k:\mathbf{Vect}_k\times\mathbf{Vect}_k\to\mathbf{Vect}_k$ of \cref{monoidal cat Vect_k eg} does not quite fit this framework. Given vector spaces $X$, $Y$, and $Z$ over $k$, there is no reason to expect the iterated tensor products $X\otimes_k(Y\otimes_kZ)$ and $(X\otimes_kY)\otimes_kZ$ to be identical. To construct the functor $\otimes_k$ explicitly, we need to make certain choices: namely, a choice of universal bilinear map $\beta:U\times V\to U\otimes_kV$ for every pair of vector spaces $U,V\in\mathbf{Vect}_k$. Without an explicit convention for how these choices are to be made, we cannot answer the question of whether the vector spaces $X\otimes_k(Y\otimes_kZ)$ and $(X\otimes_kY)\otimes_kZ$ are equal. However, this is arguably the wrong question to consider: in the setting of vector spaces, the appropriate notion of "sameness" is not equality, but isomorphism. The iterated tensor products $X\otimes_k(Y\otimes_kZ)$ and $(X\otimes_kY)\otimes_kZ$ are isomorphic, because they can be characterized by the same universal property: both are universal among vector spaces $W$ equipped with a $k$-trilinear map $y:X\times Y\times Z\to W$. Even better, there is a canonical isomorphism
\[\alpha_{X,Y,Z}:X\otimes_k(Y\otimes_kZ)\to(X\otimes_kY)\otimes_kZ\]
which depends functorially on $X$, $Y$, and $Z$. Motivated by this example, we introduce the following generalization.\par
Let $\mathcal{C}$ be a category. A nonunital monoidal structure on $\mathcal{C}$ consists of the following data:
\begin{itemize}
\item A functor $\otimes:\mathcal{C}\times\mathcal{C}\to\mathcal{C}$, which is called the tensor product functor.
\item A collection of isomorphisms $\alpha_{X,Y,Z}:X\otimes(Y\otimes Z)\cong(X\otimes Y)\otimes Z$ called the \textbf{associativity constraints} of $\mathcal{C}$. We demand that the associativity constraints $\alpha_{X,Y,Z}$ depend functorially on $X$, $Y$, $Z$ in the following sense: for every triple of morphisms $f:X\to X'$, $g:Y\to Y'$ and $h:Z\to Z'$, the diagram
\[\begin{tikzcd}
X\otimes(Y\otimes Z)\ar[r,"\alpha_{X,Y,Z}","\sim"']\ar[d,"f\otimes(g\otimes h)"]&(X\otimes Y)\otimes Z\ar[d,"(f\otimes g)\otimes h"]\\
X'\otimes(Y'\otimes Z')\ar[r,"\alpha_{X',Y',Z'}","\sim"']&(X'\otimes Y')\otimes Z'
\end{tikzcd}\]
is commutative. In other words, we require that $\alpha=\{\alpha_{X,Y,Z}\}$ can be regarded as a natural isomorphism from the functor $\bullet\otimes(\bullet\otimes\bullet)$ to the functor $(\bullet\otimes\bullet)\otimes\bullet$.
\end{itemize}
The associativity constraints of $\mathcal{C}$ are required to satisfy the following pentagon identity: for every quadruple of objects $W,X,Y,Z\in\mathcal{C}$, the diagram of isomorphisms
\[
\begin{tikzpicture}[scale=1.2]
\node (P0) at (90:2.3cm) {$((X\otimes Y)\otimes Z)\otimes W$};
\node (P1) at (90+72:2cm) {$(X\otimes(Y\otimes Z))\otimes W$} ;
\node (P2) at (90+2*72:2cm) {\makebox[5ex][r]{$X\otimes((Y\otimes Z)\otimes W)$}};
\node (P3) at (90+3*72:2cm) {\makebox[5ex][l]{$X\otimes (Y\otimes(Z\otimes W))$}};
\node (P4) at (90+4*72:2cm) {$(X\otimes Y)\otimes (Z\otimes W)$};
\path[commutative diagrams/.cd, every arrow, every label]
(P0) edge node[swap] {$\alpha_{X,Y,Z}\otimes\id_W$} (P1)
(P1) edge node[swap] {$\alpha_{X,Y\otimes Z,W}$} (P2)
(P2) edge node[swap] {$\id_X\otimes\alpha_{Y,Z,W}$} (P3)
(P4) edge node {$\alpha_{X,Y,Z\otimes W}$} (P3)
(P0) edge node {$\alpha_{X\otimes Y,Z,W}$} (P4);
\end{tikzpicture}
\]
commutes. A \textbf{nonunital monoidal category} is a triple $(\mathcal{C},\otimes,\alpha)$, where $\mathcal{C}$ is a category and $(\otimes,\alpha)$ is a nonunital monoidal structure on $\mathcal{C}$.
\begin{example}
Let $\mathcal{C}$ be a category equipped with a nonunital strict monoidal structure $\otimes:\mathcal{C}\otimes\mathcal{C}\to\mathcal{C}$. Then $\otimes$ determines a nonunital monoidal structure on $\mathcal{C}$ by taking the associativity constraints $\alpha_{X,Y,Z}$ to be identity morphisms. Conversely, if $\mathcal{C}$ is equipped with a nonunital monoidal structure $(\otimes,\alpha)$ where each of the associativity constraints $\alpha_{X,Y,Z}$ is an identity morphism, then $\otimes:\mathcal{C}\times\mathcal{C}\to\mathcal{C}$ is a nonunital strict monoidal structure on $\mathcal{C}$.
\end{example}
\begin{example}\label{monoidal cat nonunital Fun}
Let $\mathcal{C}$ and $\mathcal{D}$ be categories. Then every nonunital monoidal structure $(\otimes,\alpha)$ on $\mathcal{D}$ determines a nonunital monoidal structure on the functor category $\Fun(\mathcal{C},\mathcal{D})$, whose underlying tensor product is given by the composition
\[\Fun(\mathcal{C},\mathcal{D})\times \Fun(\mathcal{C},\mathcal{D})\cong \Fun(\mathcal{C},\mathcal{D}\times\mathcal{D})\stackrel{\otimes\circ}{\to}\Fun(\mathcal{C},\mathcal{D})\]
and whose associativity constraint assigns to each triple of functors $F,G,H:\mathcal{C}\to\mathcal{D}$ the natural isomorphism $F\otimes(G\otimes H)\stackrel{\sim}{\to}(F\otimes G)\otimes H$.
\end{example}
\begin{remark}
Let $\mathcal{C}$ be a category equipped with a nonunital monoidal structure $(\otimes,\alpha)$. We will often abuse terminology by identifying the nonunital monoidal structure $(\otimes,\alpha)$ with the underlying tensor product functor $\otimes:\mathcal{C}\times\mathcal{C}\to\mathcal{C}$. If we say that a functor $\otimes:\mathcal{C}\times\mathcal{C}\to\mathcal{C}$ is a nonunital monoidal structure on $\mathcal{C}$, we implicitly assume that $\mathcal{C}$ has been equipped with associativity constraints $\alpha_{X,Y,Z}:X\otimes(Y\otimes Z)\cong(X\otimes Y)\otimes Z$ satisfying the pentagon identity. Beware that, in the non-strict case, the associativity constraints are an essential part of the data: it is possible to have inequivalent nonunital monoidal categories $(\mathcal{C},\otimes,\alpha)$ and $(\mathcal{C}',\otimes',\alpha')$ with $\mathcal{C}=\mathcal{C}'$ and $\otimes=\otimes'$.
\end{remark}
\begin{remark}\label{monoidal cat nonunital subcat}
Let $\mathcal{C}$ be a category equipped with a nonunital monoidal structure $(\otimes,\alpha)$, and let $\mathcal{C}_0\sub\mathcal{C}$ be a full subcategory. Suppose that, for every pair of objects $X,Y\in\mathcal{C}_0$, the tensor product $X\otimes Y$ also belongs to $\mathcal{C}_0$. Then $\mathcal{C}_0$ inherits a nonunital monoidal structure, with tensor product functor given by the composition
\[\mathcal{C}_0\times\mathcal{C}_0\sub\mathcal{C}\times\mathcal{C}\stackrel{\otimes}{\to}\mathcal{C}\]
(which factors through $\mathcal{C}_0$ by hypothesis), and associativity constraints given by those of $\mathcal{C}$.
\end{remark}
By definition, a nonunital monoid $M$ is a monoid if and only if there exists an element $e\in M$ satisfying $ex=x=xe$ for each $x\in M$. If this condition is satisfied, then the element $e$ is uniquely determined. The categorical analogue of this statement is a bit more subtle. To begin with, we let $\mathcal{C}$ be a category. A \textbf{strict monoidal structure} on $\mathcal{C}$ is a nonunital strict monoidal structure $\otimes:\mathcal{C}\times\mathcal{C}\to\mathcal{C}$ for which there exists an object $\mathbf{1}\in\mathcal{C}$ satisfying the following condition: for every object $X\in\mathcal{C}$, we have $X\otimes\mathbf{1}=X=\mathbf{1}\otimes X$ (as objects of $\mathcal{C}$). Moreover, for every morphism $f:X\to Y$ in $\mathcal{C}$, we have $f\otimes\id_{\mathbf{1}}=f=\id_{\mathbf{1}}\otimes f$ (as morphisms from $X$ to $Y$). A \textbf{strict monoidal category} is a pair $(\mathcal{C},\otimes)$, where $\mathcal{C}$ is a category and $\otimes:\mathcal{C}\times\mathcal{C}\to\mathcal{C}$ is a strict monoidal structure on $\mathcal{C}$.\par
For the sake of applications, it will be convenient to consider a more general notion of unit object, which makes sense in the non-strict setting as well. We will use an efficient formulation due to Saavedra. To motivate the definition, we begin with a simple observation about units in a more elementary setting.
\begin{proposition}\label{monoid left unit iff ell_e}
Let $M$ be a nonunital monoid, let $e$ be an element of $M$, and let $\ell_e:M\to M$ denote the function given by the formula $\ell_e(x)=ex$. The following conditions are equivalent:
\begin{itemize}
\item[(\rmnum{1})] The element $e$ is a left unit of $M$: that is, $\ell_e$ is the identity function from $M$ to itself.
\item[(\rmnum{2})] The element $e$ is idempotent (that is, it satisfies $ee=e$) and the function $\ell_e:M\to M$ is a bijection.
\item[(\rmnum{3})] The element $e$ is idempotent and the function $\ell_e:M\to M$ is a monomorphism.
\end{itemize}
\end{proposition}
\begin{proof}
The implications (\rmnum{1})$\Rightarrow$(\rmnum{2})$\Rightarrow$(\rmnum{3}) are immediate. To complete the proof, assume that $e$ satisfies condition (\rmnum{3}) and let $x$ be an element of $M$. Using the assumption that $e$ is idempotent (and the associativity of the multiplication on $M$), we obtain an identity $\ell_e(x)=ex=(ee)x=e(ex)=\ell_e(ex)$. Since $\ell_e$ is a monomorphism, it follows that $x=ex$.
\end{proof}
\begin{corollary}\label{monoid unit iff idempotent and left right cancel}
Let $M$ be a nonunital monoid. Then an element $e\in M$ is a unit if and only if the following conditions are satisfied:
\begin{itemize}
\item[(a)] The element $e$ is idempotent: that is, we have $ee=e$.
\item[(b)] The element $e$ is left cancellative: that is, the function $x\mapsto ex$ is a monomorphism from $M$ to itself.
\item[(c)] The element e is right cancellative: that is, the function $x\mapsto xe$ is a monomorphism from $M$ to itself. 
\end{itemize}
\end{corollary}
We now adapt the characterization of \cref{monoid unit iff idempotent and left right cancel} to the setting of nonunital monoidal categories. Let $\mathcal{C}$ be a nonunital monoidal category. A unit of $\mathcal{C}$ is a pair $(\mathbf{1},\upsilon)$, where $\mathbf{1}$ is an object of $\mathcal{C}$ and $\upsilon:\mathbf{1}\otimes\mathbf{1}\stackrel{\sim}{\to}$ is an isomorphism, which satisfies the following additional condition: the functors $\mathbf{1}\otimes\bullet$ and $\bullet\otimes\mathbf{1}$ are fully faithful.
\begin{example}\label{monoidal cat strict unit is unit}
Let $\mathcal{C}$ be a strict monoidal category, and let $\mathbf{1}\in\mathcal{C}$ be the strict unit. Then $(\mathbf{1},\id_{\mathbf{1}})$ is a unit of $\mathcal{C}$.
\end{example}
\begin{example}
Let $M$ be a nonunital monoid, regarded as a (strict) nonunital monoidal category having only identity morphisms. Then the converse of \cref{monoidal cat strict unit is unit} holds: a pair $(\mathbf{1},\upsilon)$ is a unit structure on $M$ if and only if $\mathbf{1}$ is a unit element of $M$ and $\upsilon=\id_{\mathbf{1}}$. This is a restatement of \cref{monoid unit iff idempotent and left right cancel}.
\end{example}
Let $\mathcal{C}$ be a category. A \textbf{monoidal structure} on $\mathcal{C}$ is a nonunital monoidal structure $(\otimes,\alpha)$ on $\mathcal{C}$ together with a choice of unit $(\mathbf{1},\upsilon)$. A \textbf{monoidal category} is a category $\mathcal{C}$ together with a monoidal structure $(\otimes,\alpha,\mathbf{1},\upsilon)$ on $\mathcal{C}$. In this case, the object $\mathbf{1}$ is called the unit object of $\mathcal{C}$ and the isomorphism $\upsilon:\mathbf{1}\otimes\mathbf{1}\stackrel{\sim}{\to}\mathbf{1}$ as the unit constraint of $\mathcal{C}$.
\begin{remark}
It is also possible to adopt the following variant: a monoidal category is a nonunital monoidal category $\mathcal{C}$ which admits a unit. This is essentially equivalent to our definition, since a unit $(\mathbf{1},\upsilon)$ of $\mathcal{C}$ is uniquely determined up to unique isomorphism. However, for our purposes it will be more convenient to adopt the convention that a monoidal structure on a category $\mathcal{C}$ includes a choice of unit object $\mathbf{1}\in\mathcal{C}$ and unit constraint $\upsilon:\mathbf{1}\otimes\mathbf{1}\stackrel{\sim}{\to}\mathbf{1}$.
\end{remark}
\begin{example}
Let $\mathcal{C}$ be a category. Then every strict monoidal structure $\otimes:\mathcal{C}\otimes\mathcal{C}\to\mathcal{C}$ can be promoted to a monoidal structure $(\otimes,\alpha,\mathbf{1},\upsilon)$ on $\mathcal{C}$, by taking $\mathbf{1}$ to be the strict unit of $\mathcal{C}$ and the associativity and unit constraints to be identity morphisms of $\mathcal{C}$. Conversely, if $\mathcal{C}$ is equipped with a monoidal structure $(\otimes,\alpha,\mathbf{1},\upsilon)$ for which the associativity and unit constraints are identity morphisms, then $\otimes:\mathcal{C}\times\mathcal{C}\to\mathcal{C}$ is a strict monoidal structure on $\mathcal{C}$ and $\mathbf{1}$ is the strict unit.
\end{example}
\begin{example}
Let $\mathcal{C}$ be a monoidal category and let $\mathcal{C}_0\sub\mathcal{C}$ be a full subcategory. Assume that $\mathcal{C}_0$ contains the unit object $\mathbf{1}$ and is closed under the formation of tensor products in $\mathcal{C}$. Then $\mathcal{C}_0$ inherits the structure of a monoidal category: the underlying nonunital monoidal structure on $\mathcal{C}_0$ is given in \cref{monoidal cat nonunital subcat}, and the unit $(\mathbf{1},\upsilon)$ of $\mathcal{C}_0$ coincides with the unit of $\mathcal{C}$.
\end{example}
\begin{example}
Let $\mathcal{C}$ and $\mathcal{D}$ be categories. Then every monoidal structure on $\mathcal{D}$ determines a monoidal structure on the functor category $\Fun(\mathcal{C},\mathcal{D})$, whose underlying nonunital monoidal structure is given in \cref{monoidal cat nonunital Fun} and whose unit object is the constant functor $\mathcal{C}\to\{\mathbf{1}\}\hookrightarrow\mathcal{D}$ (and whose unit constraint $\upsilon:\mathbf{1}\otimes\mathbf{1}\stackrel{\sim}{\to}\mathbf{1}$ is the constant natural transformation induced by the unit constraint of $\mathcal{D}$).
\end{example}
Let $\mathcal{C}$ be a monoidal category. For each object $X\in\mathcal{C}$, we have canonical isomorphisms
\[
\begin{tikzcd}
\mathbf{1}\otimes(\mathbf{1}\otimes X)\ar[r,"\alpha_{\mathbf{1},\mathbf{1},X}"]&(\mathbf{1}\otimes \mathbf{1})\otimes X\ar[r,"\upsilon\otimes\id_X"]&\mathbf{1}\otimes X
\end{tikzcd}
\]
Since the functor $Y\mapsto\mathbf{1}\otimes Y$ is fully faithful, it follows that there is a unique isomorphism $\lambda_X:\mathbf{1}\otimes X\stackrel{\sim}{\to}X$ for which the diagram
\[\begin{tikzcd}
\mathbf{1}\otimes(\mathbf{1}\otimes X)\ar[rr,"\alpha_{\mathbf{1},\mathbf{1},X}"]\ar[rd,swap,"\id_{\mathbf{1}}\otimes\lambda_X"]&&(\mathbf{1}\otimes\mathbf{1})\otimes X\ar[ld,"\upsilon\otimes\id_X"]\\
&\mathbf{1}\otimes X&
\end{tikzcd}\]
commutes. We say that $\lambda_X$ is the \textbf{left unit constraint}. Similarly, there is a unique isomorphism $\rho_X:X\otimes\mathbf{1}\stackrel{\sim}{\to}X$ for which the diagram
\[\begin{tikzcd}
X\otimes(\mathbf{1}\otimes\mathbf{1})\ar[rr,"\alpha_{X,\mathbf{1},\mathbf{1}}"]\ar[rd,swap,"\id_{X}\otimes\upsilon"]&&(X\otimes\mathbf{1})\otimes\mathbf{1}\ar[ld,"\rho_X\otimes\id_{\mathbf{1}}"]\\
&X\otimes\mathbf{1}&
\end{tikzcd}\]
commutes; we say that $\rho_X$ is the \textbf{right unit constraint}. It is clear that the left and right unit constraints depend functorially on $X$. In other words, for every morphism $f:X\to Y$, the diagram
\[\begin{tikzcd}
\mathbf{1}\otimes X\ar[r,"\lambda_X"]\ar[d,"\id_{\mathbf{1}}"]&X&X\otimes\mathbf{1}\ar[l,swap,"\rho_X"]\ar[d,"f"]\ar[d,"f\otimes\id_{\mathbf{1}}"]\\
\mathbf{1}\otimes Y\ar[r,"\lambda_Y"]&Y&Y\otimes\mathbf{1}\ar[l,swap,"\rho_Y"]
\end{tikzcd}\]
is commutative.
\begin{proposition}[\textbf{The Triangle Identity}]\label{monoidal cat unit constraint triangle identity}
Let $\mathcal{C}$ be a monoidal category with unit object $\mathbf{1}$. Let $X$ and $Y$ be objects of $\mathcal{C}$, and let $\rho_X$ and $\lambda_Y$ be the right and left unit constraints. Then the follwing diagrams of isomorphisms are commutative:
\[\begin{tikzcd}[row sep=6mm,column sep=6mm]
X\otimes(\mathbf{1}\otimes Y)\ar[rd,swap,"\id_X\otimes\lambda_Y"]\ar[rr,"\alpha_{X,\mathbf{1},Y}"]&&(X\otimes\mathbf{1})\otimes Y\ar[ld,"\rho_X\otimes\id_{Y}"]\\
&X\otimes Y&
\end{tikzcd}\quad\begin{tikzcd}[row sep=6mm,column sep=6mm]
X\otimes(Y\otimes\mathbf{1})\ar[rd,swap,"\id_X\otimes\rho_Y"]\ar[rr,"\alpha_{X,Y,\mathbf{1}}"]&&(X\otimes Y)\otimes\mathbf{1}\ar[ld,"\rho_X\otimes\id_{Y}"]\\
&X\otimes Y&
\end{tikzcd}\]
\vspace*{-2mm}
\[\begin{tikzcd}[row sep=6mm,column sep=6mm]
\mathbf{1}\otimes(X\otimes Y)\ar[rd,swap,"\lambda_{X\otimes Y}"]\ar[rr,"\alpha_{\mathbf{1},X,Y}"]&&(\mathbf{1}\otimes X)\otimes Y\ar[ld,"\lambda_X\otimes\id_{Y}"]\\
&X\otimes Y&
\end{tikzcd}\]
\end{proposition}
\begin{proof}
We only give the proof of the first diagram, the others can be done similarly. To this ends, we have a diagram of isomorphisms
\[\begin{tikzcd}[column sep=2mm]
&X\otimes((\mathbf{1}\otimes\mathbf{1})\otimes Y)\ar[d,"\upsilon"]\ar[rr,"\alpha"]&&(X\otimes(\mathbf{1}\otimes\mathbf{1}))\otimes Y\ar[d,"\upsilon"]\ar[rdd,"\alpha"]&\\
&X\otimes(\mathbf{1}\otimes Y)\ar[d,"\alpha"]\ar[rr,"\alpha"]&&(X\otimes\mathbf{1})\otimes Y\ar[lld,"\id"]&\\
X\otimes(\mathbf{1}\otimes(\mathbf{1}\otimes Y))\ar[rrd,"\alpha"]\ar[ru,"\lambda_Y"{anchor=south}]\ar[ruu,"\alpha"]&(X\otimes\mathbf{1})\otimes Y\ar[r,"\rho_X"]&X\otimes Y&X\otimes(\mathbf{1}\otimes Y)\ar[u,"\alpha"]\ar[l,swap,"\lambda_Y"]&((X\otimes\mathbf{1})\otimes\mathbf{1})Y\ar[lu,"\rho_X"{anchor=south}]\\
&&(X\otimes\mathbf{1})\otimes(\mathbf{1}\otimes Y)\ar[lu,"\lambda_Y"{anchor=south}]\ar[ru,"\rho_X"{anchor=south}]\ar[rru,"\alpha"]&&
\end{tikzcd}\]
Here the outer cycle commutes by the pentagon identity, the upper rectangle and outer quadrilaterals by the functoriality of the associativity constraint, the side triangles by the definition of the left and right unit constraints, and the lower quadrilateral by the functoriality of the tensor product $\otimes$. It follows that the middle square is also commutative, which is equivalent to our statement.
\end{proof}
\begin{corollary}\label{monoidal cat unit constraint of 1 char}
Let $\mathcal{C}$ be a monoidal category with unit object $\mathbf{1}$. Then the left and right unit constraints $\lambda_{\mathbf{1}},\rho_{\mathbf{1}}:\mathbf{1}\otimes\mathbf{1}\stackrel{\sim}{\to}\mathbf{1}$ are equal to the unit constraint $\upsilon:\mathbf{1}\otimes\mathbf{1}\stackrel{\sim}{\to}\mathbf{1}$.
\end{corollary}
\begin{proof}
Let $X$ be any object of $\mathcal{C}$. Then the left unit contraint $\lambda_X$ is characterized by the commutativity of the diagram
\[\begin{tikzcd}[row sep=6mm,column sep=6mm]
\mathbf{1}\otimes(\mathbf{1}\otimes X)\ar[rr,"\alpha_{\mathbf{1},\mathbf{1},X}"]\ar[rd,swap,"\id_{\mathbf{1}}\otimes\lambda_X"]&&(\mathbf{1}\otimes\mathbf{1})\otimes X\ar[ld,"\upsilon\otimes\id_X"]\\
&\mathbf{1}\otimes X&
\end{tikzcd}\]
Using \cref{monoidal cat unit constraint triangle identity}, we deduce that $\upsilon\otimes\id_X=\rho_{\mathbf{1}}\otimes\id_X$ as morphisms from $(\mathbf{1}\otimes\mathbf{1})\otimes X$. In other words, the morphisms $\upsilon$ and $\rho_{\mathbf{1}}$ have the same image under the functor $\bullet\otimes X$.  
\end{proof}
If $M$ is a nonunital monoid, then a unit element $e\in M$ is unique if it exists. For nonunital monoidal categories, the analogous statement is more subtle. If a nonunital monoidal category $\mathcal{C}$ admits a unit $(\mathbf{1},\upsilon)$, then it has many others: we can replace $\mathbf{1}$ by any object $\mathbf{1}'$ which is isomorphic to it, and $\upsilon$ by any choice of isomorphism $\upsilon':\mathbf{1}'\otimes\mathbf{1}'\stackrel{\sim}{\to}\mathbf{1}'$. Nevertheless, we have the following strong uniqueness result:
\begin{proposition}[\textbf{Uniqueness of Units}]\label{monoidal cat uniqueness of unit}
Let $\mathcal{C}$ be a nonunital monoidal category equipped with units $(\mathbf{1},\upsilon)$ and $(\mathbf{1}',\upsilon')$. Then there is a unique isomorphism $u:\mathbf{1}\stackrel{\sim}{\to}\mathbf{1}'$ for which the diagram
\[\begin{tikzcd}
\mathbf{1}\otimes \mathbf{1}\ar[r,"\upsilon"]\ar[d,swap,"u\otimes u"]&\mathbf{1}\ar[d,"u"]\\
\mathbf{1}'\otimes \mathbf{1}'\ar[r,"\upsilon'"]&\mathbf{1}
\end{tikzcd}\]
commutes.
\end{proposition}
\begin{proof}
Let $\mathcal{C}$ be a nonunital monoidal category equipped with units $(\mathbf{1},\upsilon)$ and $(\mathbf{1}',\upsilon')$. We can then regard $\mathcal{C}$ as a monoidal category with unit object $\mathbf{1}$ and unit constraint $\upsilon$. For each object $X\in\mathcal{C}$, let $\lambda_X:\mathbf{1}\otimes X\stackrel{\sim}{\to}X$ be the left unit constraint. We want to show that there is a unique isomorphism $u:\mathbf{1}\stackrel{\sim}{\to}\mathbf{1}'$ for which the outer rectangle in the diagram of isomorphisms
\[\begin{tikzcd}
\mathbf{1}\otimes\mathbf{1}\ar[r,"\lambda_{\mathbf{1}}"]\ar[d,"\id_{\mathbf{1}}\otimes u"]&\mathbf{1}\ar[d,"u"]\\
\mathbf{1}\otimes\mathbf{1}'\ar[r,"\lambda_{\mathbf{1}'}"]\ar[d,"u\otimes\id_{\mathbf{1}'}"]&\mathbf{1}'\ar[d,"\id_{\mathbf{1}'}"]\\
\mathbf{1}'\otimes\mathbf{1}'\ar[r,"\upsilon'"]&\mathbf{1}'
\end{tikzcd}\]
is commutative. Since the upper square commutes by functoriality, this is equivalent to the commutativity of the lower square. The existence and uniqueness of $u$ now follows from the assumption that the functor $\bullet\otimes\mathbf{1}'$ is fully faithful.
\end{proof}
We now conclude this paragraph by providing some examples of monoidal categories.
\begin{example}\label{monoidal cat Vect_k construction}
Let $k$ be a field and let $\mathbf{Vect}_k$ denote the category of vector spaces over $k$ (where morphisms are $k$-linear maps). For every pair of vector spaces $V,W\in\mathbf{Vect}_k$, let us choose a vector space $V\otimes_kW$ and a bilinear map
\[V\times W\to V\otimes_kW,\quad (v,w)\mapsto v\otimes w\]
which exhibits $V\otimes_kW$ as a tensor product of $V$ and $W$. The construction $(V,W)\mapsto V\otimes_kW$ determines a functor
\[\otimes_k:\mathbf{Vect}_k\times\mathbf{Vect}_k\to\mathbf{Vect}_k\]
whose value on a pair of $k$-linear maps $\varphi:V\to V'$, $\psi:W\to W'$ is characterized by the identity
\[(\varphi\otimes_k\psi)(v\otimes w)=\varphi(v)\otimes\psi(w).\]
For every triple of vector spaces $U,V,W\in\mathbf{Vect}_k$, there is a canonical isomorphism
\[\alpha_{U,V,W}:U\otimes_k(V\otimes_kW)\stackrel{\sim}{\to}(U\otimes_kV)\otimes_kW,\]
characterized by the identity $\alpha_{U,V,W}(u\otimes(v\otimes w))=(u\otimes v)\otimes w$ for $u\in U$, $v\in V$, and $w\in W$. The pair $(\otimes_k,\alpha)$ is then a nonunital monoidal structure on the category $\mathbf{Vect}_k$. We can upgrade this to a monoidal structure by taking the unit object $\mathbf{1}$ to be the field $k$ (regarded as a vector space over itself), and the unit constraint $\upsilon:\mathbf{1}\otimes_k\mathbf{1}\stackrel{\sim}{\to}\mathbf{1}$ to be the linear map corresponding to the multiplication on $k$ (so that $\upsilon(a\otimes b)=ab$).
\end{example}
\begin{example}\label{monoidal cat Cartesian product construction}
Let $\mathcal{C}$ be a category. Assume that every pair of objects $X,Y\in\mathcal{C}$ admits a product in $\mathcal{C}$. This product is not unique: it is only unique up to (canonical) isomorphism. However, let us choose an object $X\times Y$ together with a pair of morphisms
\[\begin{tikzcd}
X&X\times Y\ar[l,swap,"\pi_1"]\ar[r,"\pi_2"]&Y
\end{tikzcd}\]
which exhibit $X\times Y$ as a product of $X$ and $Y$ in the category $\mathcal{C}$. Then the construction $(X,Y)\mapsto X\times Y$ determines a functor $\mathcal{C}\times\mathcal{C}\to\mathcal{C}$, given on morphisms by the construction
\[((f:X\to X'),(g:Y\to Y'))\mapsto((f\times g):(X\times Y)\to(X'\times Y')),\]
where $f\times g$ is the unique morphism for which the diagram
\[\begin{tikzcd}
X\ar[d,"f"]&X\times Y\ar[d,"f\times g"]\ar[l,swap,"\pi_1"]\ar[r,"\pi_2"]&Y\ar[d,"g"]\\
X'&X'\times Y'\ar[l,swap,"\pi_1"]\ar[r,"\pi_2"]&Y'
\end{tikzcd}\]
is commutative. For every triple of objects $X,Y,Z\in\mathcal{C}$, there is a canonical isomorphism $\alpha_{X,Y,Z}:X\times(Y\times Z)\stackrel{\sim}{\to}(X\times Y)\times Z$, which is characterized by the commutativity of the diagram
\[\begin{tikzcd}[row sep=5mm,column sep=5mm]
&X\times(Y\times Z)\ar[ldd]\ar[rrd]\ar[rr,"\alpha_{X,Y,Z}"]&&(X\times Y)\times Z\ar[rdd]\ar[lld]\\
&X\times Y\ar[rd]\ar[ld]&&Y\times Z\ar[rd]\ar[ld]&\\
X&&Y&&Z
\end{tikzcd}\]
The category $\mathcal{C}$ admits a nonunital monoidal structure, with tensor product given by the functor $(X,Y)\mapsto X\times Y$, and associativity constraints given by $\alpha$. If we assume also that the category $\mathcal{C}$ has a final object $\mathbf{1}$ (so that $\mathcal{C}$ admits all finite products), then we can upgrade the nonunital monoidal structure above to a monoidal structure, where the unit object of $\mathcal{C}$ is $\mathbf{1}$ and the unit constraint $\upsilon$ is the unique morphism from $\mathbf{1}\times\mathbf{1}$ to $\mathbf{1}$ in $\mathcal{C}$. We refer to this monoidal structure as the \textbf{cartesian monoidal structure} on $\mathcal{C}$.
\end{example}
\begin{example}[\textbf{Group Cocycles}]
Let $G$ be a group with identity element $1\in G$, and let $\Gamma$ be an abelian group on which $G$ acts by automorphisms; we denote the action of an element $g\in G$ by $\gamma\mapsto g(\gamma)$. A \textbf{$\bm{3}$-cocycle on $\bm{G}$ with values in $\bm{\Gamma}$} is a map of sets
\[\alpha:G\times G\times G\to\Gamma,\quad (x,y,z)\mapsto\alpha_{x,y,z}\]
which satisfies the equations
\[w(\alpha_{x,y,z})-\alpha_{wx,y,z}+\alpha_{w,xy,z}-\alpha_{w,x,yz}+\alpha_{w,x,yz}=0\]
for every quadruple of elements $w,x,y,z\in G$.\par
Let $\mathcal{C}$ denote the category whose objects are the elements of $G$, and whose morphisms are given by
\[\Hom_{\mathcal{C}}(g,h)=\begin{cases}
\Gamma&\text{if $g=h$},\\
\emp&\text{otherwise}
\end{cases}\]
Using the action of $G$ on $\Gamma$, we can construct a functor $\otimes:\mathcal{C}\times\mathcal{C}\to\mathcal{C}$ given on objects by $(g,h)\mapsto gh$ and on morphisms by
\[((\gamma:g\to g),(\delta:h\to h))\mapsto(\gamma+g(\delta):gh\to gh).\]
Unwinding the definitions, one sees that upgrading the functor $\otimes$ to a nonunital monoidal structure on the category $(\otimes,\alpha)$ on $\mathcal{C}$ is equivalent to choosing a $3$-cocycle $\alpha:G\times G\times G\to\Gamma$. More precisely, any map $\alpha:G\times G\times G\to\Gamma$ can be regarded as a natural transformation of functors
\[\bullet\otimes(\bullet\otimes\bullet)\to(\bullet\otimes\bullet)\otimes\bullet\]
and pentagon identity translates to the cocycle condition above.\par
For any choice of cocycle $\alpha:G\times G\times G\to\Gamma$, we can upgrade the associated nonunital monoidal structure $(\otimes,\alpha)$ to a monoidal structure on the category $\mathcal{C}$, by taking the unit object of $\mathcal{C}$ to be the identity element $1\in G$ and the unit constraint $\upsilon:1\otimes 1\cong 1$ to be the element $0\in\Gamma$.
\end{example}
\begin{example}\label{monoidal cat induced on opposite}
Let $\mathcal{C}$ be a category equipped with a nonunital monoidal structure $(\otimes,\alpha)$. Then the opposite category $\mathcal{C}^{\op}$ inherits a nonunital monoidal structure, which can be described concretely as follows:
\begin{itemize}
\item The tensor product on $\mathcal{C}^{\op}$ is obtained from the tensor product functor $\otimes:\mathcal{C}\times\mathcal{C}\to\mathcal{C}$ by passing to opposite categories.
\item Let $X$, $Y$, and $Z$ be objects of $\mathcal{C}$, and let us write $X^{\op}$, $Y^{\op}$, and $Z^{\op}$ for the corresponding objects of $\mathcal{C}^{\op}$. Then the associativity constraint $\alpha_{X^{\op},Y^{\op},Z^{\op}}$ for $\mathcal{C}^{\op}$ is the inverse of the associativity constraint $\alpha_{X,Y,Z}$ for $\mathcal{C}$.
\end{itemize}
If the nonunital monoidal category $\mathcal{C}$ is equipped with a unit structure $(\mathbf{1},\upsilon)$, then we can regard $(\mathbf{1}^{\op},\upsilon^{-1})$ as a unit structure for the nonunital monoidal category $\mathcal{C}^{\op}$. In particular, every monoidal structure on a category $\mathcal{C}$ determines a monoidal structure on the opposite category $\mathcal{C}^{\op}$.
\end{example}
\begin{example}\label{monoidal cat induced on reverse}
Let $\mathcal{C}$ be a category equipped with a nonunital monoidal structure $(\otimes,\alpha)$. Then we can equip $\mathcal{C}$ with another nonunital monoidal structure $(\otimes^{\rev},\alpha^{\rev})$ defined as follows:
\begin{itemize}
\item The tensor product functor $\otimes^{\rev}:\mathcal{C}\times\mathcal{C}\to\mathcal{C}$ is given on objects by the formula $X\otimes^{\rev}Y=Y\otimes X$ (and similarly on morphisms).
\item The associativity constraint on $\otimes^{\rev}$ is given by the formula $\alpha^{\rev}_{X,Y,Z}=\alpha_{Z,Y,X}^{-1}$.
\end{itemize}
We will refer to the nonunital monoidal structure $(\otimes^{\rev},\alpha^{\rev})$ as the reverse of the nonunital monoidal structure $(\otimes,\alpha)$. In this case, we will write $\mathcal{C}^{\rev}$ to denote the nonunital monoidal category whose underlying category is $\mathcal{C}$, equipped with the nonunital monoidal structure $(\otimes^{\rev},\alpha^{\rev})$.\par
If the nonunital monoidal category $\mathcal{C}$ is equipped with a unit structure $(\mathbf{1},\upsilon)$, then we can also regard $(\mathbf{1},\upsilon)$ as a unit structure for the nonunital monoidal category $\mathcal{C}^{\rev}$. In other words, if $\mathcal{C}$ is a monoidal category, then we can regard $\mathcal{C}^{\rev}$ as a monoidal category (having the same underlying category and unit object, but "reversed" tensor product).
\end{example}
\subsection{Nonunital monoidal functors}
We now study functors between (nonunital) monoidal categories. Let $\mathcal{C}$ and $\mathcal{D}$ be nonunital monoidal categories. A nonunital strict monoidal functor from $\mathcal{C}$ to $\mathcal{D}$ is a functor $F:\mathcal{C}\to\mathcal{D}$ with the following properties:
\begin{itemize}
\item The diagram of functors
\[\begin{tikzcd}
\mathcal{C}\times\mathcal{C}\ar[r,"\otimes"]\ar[d,swap,"F\times F"]&\mathcal{C}\ar[d,"F"]\\
\mathcal{D}\times\mathcal{D}\ar[r,"\otimes"]&\mathcal{D}
\end{tikzcd}\]
is strictly commutative. In particular, for every pair of objects $X,Y\in\mathcal{C}$, we have an equality $F(X)\otimes F(Y)=F(X\otimes Y)$ of objects of $\mathcal{D}$.
\item For every triple of objects $X,Y,Z\in\mathcal{C}$, the functor $F$ carries the associativity constraint $\alpha_{X,Y,Z}$ (for the monoidal structure on $\mathcal{C}$) to the associativity constraint $\alpha_{F(X),F(Y),F(Z)}$ (for the monoidal structure on $\mathcal{D}$).
\end{itemize}
\begin{example}
Let $\mathcal{C}$ be a nonunital monoidal category. Then the identity functor $\id_{\mathcal{C}}$ is a nonunital strict monoidal functor from $\mathcal{C}$ to itself.
\end{example}
For many applications, the above definition is too restrictive. In practice, the definition of a (nonunital) monoidal structure $\otimes:\mathcal{C}\times\mathcal{C}\to\mathcal{C}$ on a category $\mathcal{C}$ often involves constructions which are only well-defined up to isomorphism. In such cases, it is unreasonable to require that a functor $F:\mathcal{C}\to\mathcal{D}$ has the property that $F(X)\otimes F(Y)$ and $F(X\otimes Y)$ are the same object of $\mathcal{D}$. Instead, we should ask for any isomorphism $\mu_{X,Y}:F(X)\otimes F(Y)\stackrel{\sim}{\to}F(X\otimes Y)$. To get a well-behaved theory, we should further demand that the isomorphisms $\mu_{X,Y}$ depend functorially on $X$ and $Y$, and are suitably compatible with the associativity constraints on $\mathcal{C}$ and $\mathcal{D}$. We begin by considering a slightly more general situation, where the morphisms $\mu_{X,Y}$ are not required to be invertible.\par
Let $\mathcal{C}$ and $\mathcal{D}$ be nonunital monoidal categories, and let $F:\mathcal{C}\to\mathcal{D}$ be a functor from $\mathcal{C}$ to $\mathcal{D}$. A \textbf{nonunital lax monoidal structure} on $F$ is defined to be a collection of morphisms $\mu=\{\mu_{X,Y}:F(X)\otimes F(Y)\to F(X\otimes Y)\}_{X,Y\in\mathcal{C}}$ which satisfy the following conditions:
\begin{itemize}
\item[(a)] The morphisms $\mu_{X,Y}$ depend functorially on $X$ and $Y$: that is, for every pair of morphisms $f:X\to X'$, $g:Y\to Y'$ in $\mathcal{C}$, the diagram
\[\begin{tikzcd}
F(X)\otimes F(Y)\ar[r,"\mu_{X,Y}"]\ar[d,swap,"F(f)\otimes F(g)"]&F(X\otimes Y)\ar[d,"F(f\otimes g)"]\\
F(X')\otimes F(Y')\ar[r,"\mu_{X',Y'}"]&F(X'\otimes Y')
\end{tikzcd}\] 
commutes (in the category $\mathcal{D}$). In other words, we can regard $\mu$ as a natural transformation of functors as indicated in the diagram
\[\begin{tikzcd}
\mathcal{C}\times\mathcal{C}\ar[r,"\otimes"]\ar[d,swap,"F\times F"]&\mathcal{C}\ar[d,"F"]\\
\mathcal{D}\times\mathcal{D}\ar[ru,Rightarrow,shorten=3mm,"\mu"]\ar[r,"\otimes"]&\mathcal{D}
\end{tikzcd}\]
\item[(b)] The morphisms $\mu_{X,Y}$ are compatible with the associativity constraints on $\mathcal{C}$ and $\mathcal{D}$ in the following sense: for every triple of objects $X,Y,Z\in\mathcal{C}$, the diagram
\[\begin{tikzcd}[row sep=15mm]
F(X)\otimes(F(Y)\otimes F(Z))\ar[r,"\alpha_{F(X),F(Y),F(Z)}"]\ar[d,swap,"\id_{F(X)}\otimes\mu_{Y,Z}"]&(F(X)\otimes F(Y))\otimes F(Z)\ar[d,"\mu_{X,Y}\otimes\id_{F(Z)}"]\\
F(X)\otimes F(Y\otimes Z)\ar[d,swap,"\mu_{X,Y\otimes Z}"]&F(X\otimes Y)\otimes F(Z)\ar[d,"\mu_{X\otimes Y,Z}"]\\
F(X\otimes(Y\otimes Z))\ar[r,"F(\alpha_{X,Y,Z})"]&F((X\otimes Y)\otimes Z)
\end{tikzcd}\]
commutes (in the category $\mathcal{D}$).
\end{itemize}
A \textbf{nonunital lax monoidal functor} from $\mathcal{C}$ to $\mathcal{D}$ is then a pair $(F,\mu)$, where $F:\mathcal{C}\to\mathcal{D}$ is a functor and $\mu=\{\mu_{X,Y}\}_{X,Y\in\mathcal{C}}$ is a nonunital lax monoidal structure on $F$. In this case, the morphisms $\{\mu_{X,Y}\}$ are called the \textbf{tensor constraints} of $F$. If the tensor constraints $\mu_{X,Y}:F(X)\otimes F(Y)\to F(X\otimes Y)$ is an isomorphism for each $X,Y\in\mathcal{C}$, then we say that $F$ is a \textbf{nonunital monoidal functor}, or that $\mu$ is a nonunital monoidal structure on $F$.
\begin{example}
Let $k$ be a field and let $\mathbf{Vect}_k$ denote the category of vector spaces over $k$, endowed with the monoidal structure $\otimes_k$. The construction of this monoidal structure involved certain choices: for every pair of vector spaces $U,V\in\mathbf{Vect}_k$, we selected a universal $k$-bilinear map $b_{U,V}:U\times V\to U\otimes_kV$. The collection of functions $b=\{b_{U,V}\}$ is then a nonunital lax monoidal structure on the forgetful functor $\mathbf{Vect}_k\to\mathbf{Set}$ (where we equip $\mathbf{Set}$ with the monoidal structure given by cartesian products).
\end{example}
\begin{example}
Let $\mathcal{C}$ and $\mathcal{D}$ be nonunital monoidal categories, and let $F:\mathcal{C}\to\mathcal{D}$ be a nonunital strict monoidal functor. Then $F$ admits a nonunital monoidal structure $\{\mu_{X,Y}\}$, where we take each $\mu_{X,Y}$ to be the identity morphism from $F(X)\otimes F(Y)=F(X\otimes Y)$ to itself. Conversely, if $(F,\mu)$ is a nonunital monoidal functor from $\mathcal{C}$ to $\mathcal{D}$ with the property that the tensor constraints $\mu_{X,Y}$ is an identity morphism in $\mathcal{D}$, then $F$ is a nonunital strict monoidal functor.
\end{example}
\begin{example}
Let $M$ and $M'$ be nonunital monoids, regarded as nonunital monoidal categories having only identity morphisms. Then nonunital lax monoidal functors from $M$ to $M'$ can be identified with nonunital monoid homomorphisms from $M$ to $M'$. Moreover, every nonunital lax monoidal functor from $M$ to $M'$ is automatically strict.
\end{example}
\begin{example}[\textbf{Left Regular Representation}]
Let $\mathcal{C}$ be a nonunital monoidal category and let $\End(\mathcal{C})=\Fun(\mathcal{C},\mathcal{C})$ be the category of functors from $\mathcal{C}$ to itself, endowed with the strict monoidal structure of \cref{monoidal cat on End of cat}. For each object $X\in\mathcal{C}$, let $\ell_X:\mathcal{C}\to\mathcal{C}$ denote the functor given on objects by the formula $\ell_X(Y)=X\otimes Y$. The construction $X\mapsto\ell_X$ then determines a functor $\ell:\mathcal{C}\to\Fun(\mathcal{C},\mathcal{C})$. For every pair of objects $X,Y\in\mathcal{C}$, there is a natural isomorphism $\mu_{X,Y}:\ell_X\circ\ell_Y\stackrel{\sim}{\to}\ell_{X\otimes Y}$, whose value on an object $Z\in\mathcal{C}$ is given by the associativity constraint
\[(\ell_X\circ\ell_Y)=X\otimes(Y\otimes Z)\stackrel{\alpha_{X,Y,Z}}{\to}(X\otimes Y)\otimes Z=\ell_{X\otimes Y}(Z).\]
Then $\mu=\{\mu_{X,Y}\}$ is a nonunital monoidal structure on the functor $X\mapsto\ell_X$: property (a) follows from the naturality of the associativity constraint on $\mathcal{C}$, and property (b) is a reformulation of the pentagon identity.
\end{example}
\begin{remark}
Let $\mathcal{C}$ and $\mathcal{D}$ be nonunital monoidal categories. A nonunital strict monoidal functor from $\mathcal{C}$ to $\mathcal{D}$ is a functor $F:\mathcal{C}\to\mathcal{D}$ possessing certain properties. However, a nonunital (lax) monoidal functor from $\mathcal{D}$ to $\mathcal{D}$ is a functor $F:\mathcal{C}\to\mathcal{D}$ together with additional structure, given by the tensor constraints $\mu_{X,Y}:F(X)\otimes F(Y)\to F(X\otimes Y)$. We will often abuse terminology by identifying a nonunital (lax) monoidal functor $(F,\mu)$ with the underlying functor $F$; in this case, we implicitly assume that the tensor constraints $\mu_{X,Y}$ have been specified.
\end{remark}
Let $\mathcal{D}$ and $\mathcal{D}$ be nonunital monoidal categories. Let $F,F':\mathcal{C}\to\mathcal{D}$ be functors equipped with nonunital lax monoidal structures $\mu$ and $\mu'$, respectively. We say that a natural transformation of functors $\gamma:F\to F'$ is \textbf{nonunital monoidal} if, for every pair of objects $X,Y\in\mathcal{C}$, the diagram
\[\begin{tikzcd}
F(X)\otimes F(Y)\ar[d,swap,"\gamma(X)\otimes\gamma(Y)"]\ar[r,"\mu_{X,Y}"]&F(X\otimes Y)\ar[d,"\gamma(X\otimes Y)"]\\
F'(X)\otimes F'(Y)\ar[r,"\mu'_{X,Y}"]&F'(X\otimes Y)
\end{tikzcd}\]
is commutative. We denote by $\Fun^{\mathrm{lax}}_{\mathrm{nu}}(\mathcal{C},\mathcal{C})$ the category of nonunital lax monoidal functors $(F,\mu):\mathcal{C}\to\mathcal{D}$, whose morphisms are nonunital monoidal natural transformations. We denote by $\Fun_{\mathrm{nu}}^{\otimes}(\mathcal{C},\mathcal{D})$ denote the full subcategory of $\Fun^{\mathrm{lax}}_{\mathrm{nu}}(\mathcal{C},\mathcal{C})$ spanned by the nonunital monoidal functors $(F,\mu)$ from $\mathcal{C}$ to $\mathcal{D}$.
\begin{example}[\textbf{Nonunital Algebras}]\label{monoidal cat nununital algebra structure}
Let $\mathcal{C}$ be a nonunital monoidal category and let $A$ be an object of $\mathcal{C}$. A \textbf{nonunital algebra structure} on $A$ is a map $m:A\otimes A\to A$ for which the diagram
\[\begin{tikzcd}[row sep=6mm,column sep=4mm]
&A\otimes(A\otimes A)\ar[ld,swap,"\id\otimes m"]\ar[rr,"\alpha_{A,A,A}"]&&(A\otimes A)\otimes A\ar[rd,"m\otimes\id"]&\\
A\otimes A\ar[rrd,swap,"m"]&&&&A\otimes A\ar[lld,"m"]\\
&&A&&
\end{tikzcd}\]
is commutative. A \textbf{nonunital algebra object} of $\mathcal{C}$ is a pair $(A,m)$, where $A$ is an object of $\mathcal{C}$ and $m$ is a nonunital algebra structure on $A$. If $(A,m)$ and $(A',m')$ are nonunital algebra objects of $\mathcal{C}$, then we say that a morphism $f:A\to A'$ is a nonunital algebra homomorphism if the diagram
\[\begin{tikzcd}
A\otimes A\ar[r,"m"]\ar[d,swap,"f\otimes f"]&A\ar[d,"f"]\\
A'\otimes A'\ar[r,"m'"]&A'
\end{tikzcd}\]
is commutative. We denote by $\Alg_{\mathrm{nu}}(\mathcal{C})$ the category whose objects are nonunital algebra objects of $\mathcal{C}$ and whose morphisms are nonunital algebra homomorphisms.\par
Let $\{e\}$ denote the trivial monoid, regarded as a (strict) monoidal category having only identity morphisms. Then we can identify objects $A\in\mathcal{C}$ with functors $F:\{e\}\to\mathcal{C}$ (by means of the formula $A=F(e)$). Unwinding the definitions, we see that nonunital lax monoidal structures on the functor $F$ can be identified with nonunital algebra structures on the object $A=F(e)$. Under this identification, nonunital monoidal natural transformations correspond to homomorphisms of nonunital algebras. We therefore have an isomorphism of categories $\Fun^{\mathrm{lax}}_{\mathrm{nu}}(\{e\},\mathcal{C})\cong\Alg_{\mathrm{nu}}(\mathcal{C})$.
\end{example}
\begin{example}
Let $\mathbf{Set}$ denote the category of sets, endowed with the monoidal structure given by cartesian product of sets. For each set $S$, we can identify nonunital algebra structures on $S$ with nonunital monoid structures on $S$. This observation supplies an isomorphism of categories $\Alg_{\mathrm{nu}}(\mathbf{Set})\cong\mathbf{Mon}_{\mathrm{nu}}$, where $\mathbf{Mon}_{\mathrm{nu}}$ is the category whose objects are nonunital monoids and whose morphisms are nonunital monoid homomorphisms.
\end{example}
\begin{example}\label{monoidal cat nonunital lax functor on reverse char}
Let $\mathcal{C}$ and $\mathcal{D}$ be nonunital monoidal categories, and let $\mathcal{C}^{\rev}$ and $\mathcal{D}^{\rev}$ denote the same categories with the reversed nonunital monoidal structure (\cref{monoidal cat induced on reverse}). Then every functor $F:\mathcal{C}\to\mathcal{D}$ can be also regarded as a functor from $\mathcal{C}^{\rev}$ to $\mathcal{C}^{\rev}$, which we will denote by $F^{\rev}$. There is a canonical bijection between nonunital lax monoidal structures on $F$ to that on $F^{\rev}$, which carries a nonunital monoidal structure $\mu$ on $F$ to a nonunital monoidal structure $\mu^{-\rev}$ on $F^{\rev}$, given by the formula $\mu^{\rev}_{X,Y}=\mu_{Y,X}$. Using these bijections, we obtain a canonical isomorphism of categories $\Fun^{\mathrm{lax}}_{\mathrm{nu}}(\mathcal{C},\mathcal{D})\cong\Fun_{\mathrm{nu}}^{\mathrm{lax}}(\mathcal{C}^{\rev},\mathcal{D}^{\rev})$, which restricts to an isomorphism $\Fun^{\otimes}_{\mathrm{nu}}(\mathcal{C},\mathcal{D})\cong\Fun_{\mathrm{nu}}^{\otimes}(\mathcal{C}^{\rev},\mathcal{D}^{\rev})$.
\end{example}
\begin{example}\label{monoidal cat nonunital monoidal functor on opposite char}
Let $\mathcal{C}$ and $\mathcal{D}$ be nonunital monoidal categories, and regard the opposite categories $\mathcal{C}^{\op}$ and $\mathcal{D}^{\op}$ as equipped with the nonunital monoidal structures of \cref{monoidal cat induced on opposite}. Then every functor $F:\mathcal{C}\to\mathcal{D}$ determines a functor $F^{\op}:\mathcal{C}^{\op}\to\mathcal{D}^{\op}$. There is a canonical bijection between nonunital monoidal structures on $F$ to that on $F^{\op}$, which carries a nonunital monoidal structure $\mu$ on $F$ to a nonunital monoidal structure $\mu^{-1}$ on $F^{\op}$. Using these bijections, we obtain a canonical isomorphism of categories $\Fun_{\mathrm{nu}}^{\otimes}(\mathcal{C},\mathcal{D})^{\op}\cong\Fun_{\mathrm{nu}}^{\otimes}(\mathcal{C}^{\op},\mathcal{D}^{\op})$.
\end{example}
\begin{remark}
However, note that the analogue of \cref{monoidal cat nonunital monoidal functor on opposite char} for nonunital lax monoidal functors is false. The notion of nonunital lax monoidal functor is not self-opposite: in general, there is no simple relationship between the categories $\Fun_{\mathrm{nu}}^{\mathrm{lax}}(\mathcal{C},\mathcal{D})^{\op}$ and $\Fun_{\mathrm{nu}}^{\mathrm{lax}}(\mathcal{C}^{\op},\mathcal{D}^{\op})$.\par Motivated by this remark, we introduce the following variant: Let $\mathcal{C}$ and $\mathcal{D}$ be nonunital monoidal categories, and let $F:\mathcal{C}\to\mathcal{D}$ be a functor. A \textbf{nonunital colax monoidal structure} on $F$ is a nonunital lax monoidal structure on
the opposite functor $F^{\op}:\mathcal{C}^{\op}\to\mathcal{D}^{\op}$. In other words, a colax monoidal structure on $F$ is a collection of morphisms $\mu=\{\mu_{X,Y}:F(X\otimes Y)\to F(X)\otimes F(Y)\}$ which satisfy the following pair of conditions:
\begin{itemize}
\item[(a)] The morphisms $\mu_{X,Y}$ depend functorially on $X$ and $Y$: that is, for every pair of $f:X\to X'$, $g:Y\to Y'$ in $\mathcal{C}$, the diagram
\[\begin{tikzcd}
F(X\otimes Y)\ar[r,"\mu_{X,Y}"]\ar[d,swap,"F(f\otimes g)"]&F(X)\otimes F(Y)\ar[d,"F(f)\otimes F(g)"]\\
F(X'\otimes Y')\ar[r,"\mu_{X',Y'}"]&F(X')\otimes F(Y')
\end{tikzcd}\] 
commutes (in the category $\mathcal{D}$).
\item[(b)] For every triple of objects $X,Y,Z\in\mathcal{C}$, the diagram
\[\begin{tikzcd}
F(X\otimes(Y\otimes Z))\ar[r,"F(\alpha_{X,Y,Z})"]\ar[d,swap,"\mu_{X,Y\otimes Z}"]&F((X\otimes Y)\otimes Z)\ar[d,"\mu_{X\otimes Y,Z}"]\\
F(X)\otimes F(Y\otimes Z)\ar[d,swap,"\id\otimes\mu_{Y,Z}"]&F(X\otimes Y)\otimes F(Z)\ar[d,"\mu_{X,Y}\otimes\id"]\\
F(X)\otimes (F(Y)\otimes F(Z))\ar[r,"\alpha_{F(X),F(Y),F(Z)}"]&(F(X)\otimes F(Y))\otimes F(Z)
\end{tikzcd}\] 
\end{itemize}
\end{remark}
\begin{example}[\textbf{Composition of Nonunital Monoidal Functors}]\label{monoidal cat nonunital lax functor composition}
Let $\mathcal{C}$, $\mathcal{D}$ and $\mathcal{E}$ be nonunital monoidal categories, and suppose we are given a pair of functors $F:\mathcal{C}\to\mathcal{D}$ and $G:\mathcal{D}\to\mathcal{E}$. If $\mu=\{\mu_{X,Y}\}_{X,Y\in\mathcal{C}}$ is a nonunital lax monoidal structure on the functor $F$ and $\nu=\{\nu_{U,V}\}_{U,V\in\mathcal{D}}$ is a nonunital lax monoidal structure on $G$, then the composite functor $G\circ F$ inherits a nonunital lax monoidal structure, which associates to each pair of objects $X,Y\in\mathcal{C}$ the composite map
\[\begin{tikzcd}
(G\circ F)(X)\otimes(G\circ F)(Y)\ar[r,"\nu_{F(X),F(Y)}"]&G(F(X)\otimes F(Y))\ar[r,"G(\mu_{X,Y})"]&(G\circ F)(X\otimes Y)
\end{tikzcd}\]
This construction determines a composition law
\[\circ:\Fun^{\mathrm{lax}}_{\mathrm{nu}}(\mathcal{D},\mathcal{E})\times\Fun^{\mathrm{lax}}_{\mathrm{nu}}(\mathcal{C},\mathcal{D})\to\Fun^{\mathrm{lax}}_{\mathrm{nu}}(\mathcal{C},\mathcal{E}).\]
Now suppose that $\mu$ and $\nu$ are nonunital monoidal structures on $F$ and $G$, respectively: that is, assume that all of the tensor constraints $\mu_{X,Y}$ and $\nu_{U,V}$ are isomorphisms. Then the above constriction supplies a nonunital monoidal structure on the composite functor $G\circ F$. We therefore obtain a composition law
\[\circ:\Fun^{\otimes}_{\mathrm{nu}}(\mathcal{D},\mathcal{E})\times\Fun^{\otimes}_{\mathrm{nu}}(\mathcal{C},\mathcal{D})\to\Fun^{\otimes}_{\mathrm{nu}}(\mathcal{C},\mathcal{E}).\]
\end{example}
We close this section by describing an alternative perspective on nonunital lax monoidal functors. First, we need to review a bit of terminology: Let $\mathcal{C}$, $\mathcal{D}$, and $\mathcal{E}$ be categories, and suppose we are given a pair of functors $F:\mathcal{C}\to\mathcal{E}$ and $G:\mathcal{D}\to\mathcal{E}$. We denote by $\mathcal{C}\tilde{\times}_{\mathcal{E}}\mathcal{D}$ the iterated pullback
\[\mathcal{C}\times_{\Fun(\{0\},\mathcal{E})}\Fun([1],\mathcal{E})\times_{\Fun([1],\mathcal{E})}\mathcal{D}\]
which is called the \textbf{oriented fiber product} of $\mathcal{C}$ with $\mathcal{D}$ over $\mathcal{E}$. More concretely:
\begin{itemize}
\item An object of the oriented fiber product $\mathcal{C}\tilde{\times}_{\mathcal{E}}\mathcal{D}$ is a triple $(C,D,\eta)$ where $C$ is an object of the category $\mathcal{C}$, $D$ is an object of the category $\mathcal{D}$, and $\eta:F(C)\to G(D)$ is a morphism in the category $\mathcal{E}$.
\item If $(C,D,\eta)$ and $(C',D',\eta')$ are objects of the oriented fiber product $\mathcal{C}\tilde{\times}_{\mathcal{E}}\mathcal{D}$, then a morphism from $(C,D,\eta)$ to $(C',D',\eta')$ is a pair $(u,v)$, where $u:C\to C'$ is a morphism in the category $\mathcal{C}$, $v:D\to D'$ is a morphism in the category $\mathcal{D}$, and the diagram
\[\begin{tikzcd}
F(C)\ar[r,"\eta"]\ar[d,swap,"F(u)"]&G(D)\ar[d,"G(v)"]\\
F(C')\ar[r,"\eta'"]&G(D')
\end{tikzcd}\]
commutes in the category $\mathcal{E}$.
\end{itemize}
\begin{proposition}\label{monoidal cat nonunital lax functor and oriented product}
Let $\mathcal{C}$ and $\mathcal{D}$ be nonunital monoidal categories, let $G:\mathcal{D}\to\mathcal{C}$ be a functor, and $\mathcal{C}\tilde{\times}_\mathcal{C}\mathcal{D}$ denote the corresponding oriented fiber product. Then:
\begin{itemize}
\item[(a)] Let $\mu=\{\mu_{D,D'}\}_{D,D'\in\mathcal{D}}$ be a nonunital lax monoidal structure on the functor $G$. Then there is a unique nonunital monoidal structure $\otimes_\mu$ on the oriented fiber product $\mathcal{C}\tilde{\times}_\mathcal{C}\mathcal{D}$ with the following properties:
\begin{itemize}
\item[(\rmnum{1})] The forgetful functor
\[U:\mathcal{C}\tilde{\times}_\mathcal{C}\mathcal{D}\to\mathcal{C}\times\mathcal{D},\quad (C,D,\eta)\mapsto(C,D)\]
is a strict nonunital monoidal functor.
\item[(\rmnum{2})] On objects, the tensor product $\otimes_\mu$ is given by the formula
\[(C,D,\eta)\otimes_\mu(C',D',\eta')=(C\otimes C',D\otimes D',t(\eta,\eta'))\]
where $t(\eta,\eta')$ is the composition
\[\begin{tikzcd}
C\otimes C'\ar[r,"\eta\otimes\eta'"]&G(D)\otimes G(D')\ar[r,"\mu_{D,D'}"]&G(D\otimes D')
\end{tikzcd}\]
\end{itemize}
\item[(b)] The construction $\mu\mapsto\otimes_\mu$ induces a bijection between nonunital lax monoidal structures on $G$ and nonunital monoidal structures on $\mathcal{C}\tilde{\times}_{\mathcal{C}}\mathcal{D}$ satisfying (\rmnum{1}).
\end{itemize}
\end{proposition}
\subsection{Lax monoidal functors}
We now introduce a unital version of lax monoidal functors. To motivate the discussion, we begin with a special case. Let $\mathcal{C}$ be a monoidal category with unit object $\mathbf{1}$, and let $A$ be a nonunital algebra object of $\mathcal{C}$ (\cref{monoidal cat nununital algebra structure}) with multiplication $m:A\otimes A\to A$. We say that a morphism $\eps:\mathbf{1}\to A$ is a \textbf{left unit} for $A$ if the composite map
\[\begin{tikzcd}
A\ar[r,"\lambda_A^{-1}"]&\mathbf{1}\otimes A\ar[r,"\eps\otimes\id_A"]&A\otimes A\ar[r,"m"]&A
\end{tikzcd}\]
is the identity map from $A$ to itself; here $\lambda_A:\mathbf{1}\otimes A\stackrel{\sim}{\to}A$ denotes the left unit constraint. Similarly, we say that $\eps$ is a \textbf{right unit} of $A$ if the composite map
\[\begin{tikzcd}
A\ar[r,"\rho_A^{-1}"]& A\otimes\mathbf{1}\ar[r,"\id_A\otimes\eps"]&A\otimes A\ar[r,"m"]&A
\end{tikzcd}\]
is equal to the identity, and $\eps$ is a \textbf{unit} of $A$ if it is both a left and a right unit of $A$.\par
By virtue of \cref{monoidal cat nununital algebra structure}, we can view the theory of nonunital algebras as a special case of the theory of nonunital lax monoidal functors $F:\mathcal{C}\to\mathcal{D}$, where we take C to be the trivial monoid $\{e\}$ (regarded as a category having only identity morphisms). We can then generalize the above definitions to nonunital lax monoidal functors.\par
Let $\mathcal{C}$ and $\mathcal{D}$ be monoidal categories with unit objects $\mathbf{1}_\mathcal{C}$ and $\mathbf{1}_\mathcal{D}$, respectively. Let $F:\mathcal{C}\to\mathcal{D}$ be a nonunital lax monoidal functor with tensor constraints $\mu=\{\mu_{X,Y}\}_{X,Y\in\mathcal{C}}$. Let $\eps:\mathbf{1}_{\mathcal{D}}\to F(\mathbf{1}_{\mathcal{C}})$ be a morphism in $\mathcal{D}$. We say that $\eps$ is a left unit for $F$ if, for every object $X\in\mathcal{C}$, the left unit constraint $\lambda_{F(X)}:\mathbf{1}_{\mathcal{D}}\otimes F(X)\stackrel{\sim}{\to}F(X)$ in the category $\mathcal{D}$ is equal to the composition
\[\begin{tikzcd}
\mathbf{1}_{\mathcal{D}}\otimes F(X)\ar[r,"\eps\otimes\id_{F(X)}"]&F(\mathbf{1}_{\mathcal{C}})\otimes F(X)\ar[r,"\mu_{\mathbf{1}_{\mathcal{C}},X}"]&F(\mathbf{1}_{\mathcal{C}}\otimes X)\ar[r,"F(\lambda_X)"]&F(X)
\end{tikzcd}\]
where $\lambda_{X}:\mathbf{1}_{\mathcal{C}}\otimes X\stackrel{\sim}{\to}X$ is the left unit constraint in the monoidal category $\mathcal{C}$. Similarly, we say that $\eps$ is a right unit for $F$ if, for every object $X\in\mathcal{C}$, the right unit constraint $\rho_{F(X)}:F(X)\otimes\mathbf{1}_{\mathcal{D}}\stackrel{\sim}{\to}F(X)$ is equal to the composition
\[\begin{tikzcd}
F(X)\otimes\mathbf{1}_{\mathcal{D}}\ar[r,"\id_{F(X)}\otimes\eps"]&F(X)\otimes F(\mathbf{1}_{\mathcal{C}})\ar[r,"\mu_{X,\mathbf{1}_{\mathcal{C}}}"]&F(X\otimes\mathbf{1}_{\mathcal{C}})\ar[r,"F(\rho_X)"]&F(X)
\end{tikzcd}\]
We say that $\eps$ is a \textbf{unit} for $F$ if it is both a left and a right unit for $F$.
\begin{example}
Let $\mathcal{C}$ be a monoidal category and let $A$ be a nonunital algebra object of $\mathcal{C}$, which we identify with a nonunital lax monoidal functor $F:\{e\}\to\mathcal{C}$ as in \cref{monoidal cat nununital algebra structure}. Then a map $\eps:\mathbf{1}\to A=F(e)$ is a unit (left unit, right unit) for $A$ if and only if it is a unit (left unit, right unit) for $F$.
\end{example}
We now show that if a nonunital lax monoidal functor $F$ admits a unit $\eps$, then $\eps$ is uniquely determined. This is a consequence of the following proposition:
\begin{proposition}

\end{proposition}
\subsection{Enriched category theory}
Let $\mathcal{C}$ be a category. For every pair of objects $X,Y\in\mathcal{C}$, we let $\Hom_{\mathcal{C}}(X,Y)$ denote the set of morphisms from $X$ to $Y$ in $\mathcal{C}$. In many cases of interest, the sets $\Hom_{\mathcal{C}}(X,Y)$ can be endowed with additional structure, which are respected by the composition law on $\mathcal{C}$. To give a systematic discussion of this phenomenon, it is convenient to use the formalism of \textit{enriched category theory}.\par
Let $\mathcal{A}$ be a monoidal category with unit object $\bm{1}$. An \textbf{$\mathcal{A}$-enriched category $\mathcal{C}$}, or \textbf{a category enriched over $\mathcal{A}$}, is a class consists of the following datum:
\begin{itemize}
\item A collection $\Ob(\mathcal{C})$, whose elements are called the objects of $\mathcal{C}$. We will often abuse notation by writing $X\in\mathcal{C}$ to indicate that $X$ is an element of $\Ob(\mathcal{C})$.
\item For every pair of objects $X,Y\in\Ob(\mathcal{C})$, an object $\sHom_{\mathcal{C}}(X,Y)$ of the monoidal category $\mathcal{A}$.
\item For every triple of objects $X,Y,Z\in\Ob(\mathcal{C})$, a morphism
\[c_{Z,Y,X}:\sHom_{\mathcal{C}}(Y,Z)\otimes\sHom_{\mathcal{C}}(X,Y)\to\sHom_{\mathcal{C}}(X,Z)\]
in the category $\mathcal{A}$, which is called the \textbf{composition law}.
\item For every object $X\in\Ob(\mathcal{C})$, a morphism $e_X:\bm{1}\to\sHom_{\mathcal{C}}(X,X)$ in the category $\mathcal{A}$, which is called the \textbf{identity} of $X$.
\end{itemize}
These data are required to satisfy the following conditions:
\begin{itemize}
\item[(A)] For every quadruple of objects $W,X,Y,Z\in\Ob(\mathcal{C})$, the diagram
\[\begin{tikzcd}[row sep=6mm,column sep=6mm]
&\sHom_{\mathcal{C}}(Y,Z)\otimes\sHom_{\mathcal{C}}(W,Y)\ar[dd,"c_{Z,Y,W}"]\\
\sHom_{\mathcal{C}}(Y,Z)\otimes(\sHom_\mathcal{C}(X,Y)\otimes\sHom_{\mathcal{C}}(W,X))\ar[ru,"\id\otimes c_{Y,X,W}"]\ar[dd,"\alpha"]&\\
&\sHom_{\mathcal{C}}(W,Z)\\
(\sHom_{\mathcal{C}}(Y,Z)\otimes\sHom_\mathcal{C}(X,Y))\otimes\sHom_{\mathcal{C}}(W,X)\ar[rd,swap,"c_{Z,Y,X}\otimes\id"]&\\
&\sHom_{\mathcal{C}}(X,Z)\otimes\sHom_{\mathcal{C}}(W,X)\ar[uu,swap,"c_{Z,X,W}"]
\end{tikzcd}\]
commutes. Here $\alpha$ denotes the associativity constraint on the monoidal category $\mathcal{A}$.
\item[(U)] For every pair of objects $X,Y\in\Ob(\mathcal{C})$, the diagrams
\begin{equation*}
\begin{aligned}
\begin{tikzcd}[row sep=6mm,column sep=6mm]
\bm{1}\otimes\sHom_{\mathcal{C}}(X,Y)\ar[rd,swap,"\lambda"]\ar[rr,"e_Y\otimes\id"]&&\sHom_{\mathcal{C}}(Y,Y)\otimes\sHom_{\mathcal{C}}(X,Y)\ar[ld,"c_{Y,Y,X}"]\\
&\sHom_{\mathcal{C}}(X,Y)
\end{tikzcd}\\
\begin{tikzcd}[row sep=6mm,column sep=6mm]
\sHom_{\mathcal{C}}(X,Y)\otimes\bm{1}\ar[rd,swap,"\rho"]\ar[rr,"\id\otimes e_X"]&&\sHom_{\mathcal{C}}(X,Y)\otimes\sHom_{\mathcal{C}}(X,X)\ar[ld,"c_{Y,X,X}"]\\
&\sHom_{\mathcal{C}}(X,Y)
\end{tikzcd}
\end{aligned}
\end{equation*}
commute, where $\lambda$ and $\rho$ denote the left and right unit constraints on $\mathcal{A}$.
\end{itemize}
\begin{example}[\textbf{Categories Enriched Over Sets}]\label{monoidal cat enriched over Set}
Let $\mathcal{A}=\mathbf{Set}$ be the category of sets, endowed with the monoidal structure given by the cartesian product. Then an $\mathcal{A}$-enriched category can be identified with a category in the usual sense.
\end{example}
\begin{example}\label{monoidal cat enriched as algebra object}
Let $\mathcal{A}$ be a monoidal category. If $\mathcal{C}$ is a category enriched over $\mathcal{A}$ and $X$ is an object of $\mathcal{C}$, then the composition law
\[c_{X,X,X}:\sHom_{\mathcal{C}}(X,X)\otimes \sHom_{\mathcal{C}}(X,X)\to \sHom_{\mathcal{C}}(X,X)\]
exhibits $\sHom_{\mathcal{C}}(X,X)$ as an algebra object of $\mathcal{A}$, in the sense of \cref{*}. Moreover, this construction induces a bijection
\[\{\text{$\mathcal{A}$-enriched categories $\mathcal{C}$ with $\Ob(\mathcal{C})=\{X\}$}\}\stackrel{\sim}{\to}\{\text{algebra objects of $\mathcal{A}$}\}.\]
Consequently, the theory of enriched categories can be regarded as a generalization of the theory of associative algebras.
\end{example}
\begin{remark}[\textbf{Functoriality}]\label{monoidal cat enrichment functoriality}
Let $\mathcal{A}$ and $\mathcal{A}'$ be monoidal categories, and let $F:\mathcal{A}\to\mathcal{A}'$ be a lax monoidal functor (with tensor constraints $\mu_{A,B}:F(A)\otimes F(B)\to F(A\otimes B)$ and unit $\eps:\bm{1}_{\mathcal{A}'}\to F(\bm{1}_{\mathcal{A}}))$. Then every $\mathcal{A}$-enriched category $\mathcal{C}$ determines an $\mathcal{A}'$-enriched category $\mathcal{C}'$, which can be described concretely as follows:
\begin{itemize}
\item The objects of $\mathcal{C}'$ are objects of $\mathcal{C}$.
\item For every pair of objects $X,Y\in\Ob(\mathcal{C}')$, we set $\sHom_{\mathcal{C}'}(X,Y)=F(\sHom_{\mathcal{C}}(X,Y))$.
\item For every triple of objects $X,Y,Z\in\Ob(\mathcal{C}')$, the composition law $c'_{Z,Y,X}$ for $\mathcal{C}'$ is given by the composition
\[F(c_{Z,Y,X})\circ\mu:F(\sHom_{\mathcal{C}}(Y,Z))\otimes F(\sHom_{\mathcal{C}'}(X,Y))\to F(\sHom_{\mathcal{C}}(X,Z)).\]
\item For every object $X\in\Ob(\mathcal{C}')$, the identity morphism $e_X'$ for $X$ in $\mathcal{C}$ is given by the composition
\[\begin{tikzcd}
\bm{1}_{\mathcal{A}'}\ar[r,"\eps"]&F(\bm{1}_{\mathcal{A}})\ar[r,"F(e_X)"]&F(\sHom_{\mathcal{C}}(X,X))=\sHom_{\mathcal{C}'}(X,X)
\end{tikzcd}\]
\end{itemize}
\end{remark}
\begin{example}[\textbf{The Underlying Category of an Enriched Category}]\label{monoidal cat enriched underlying cat}
Let $\mathcal{A}$ be a monoidal category and let $F:\mathcal{A}\to\mathbf{Set}$ be the functor given by $F(\mathcal{A})=\Hom_{\mathcal{A}}(1,A)$, endowed with the lax monoidal structure of \cref{*}. If $\mathcal{C}$ is a category enriched over $\mathcal{A}$, then we can apply the construction of \cref{monoidal cat enrichment functoriality} to obtain a $\mathbf{Set}$-enriched category, which we can identify with an ordinary category (\cref{monoidal cat enriched over Set}). This category is called the \textbf{underlying category} of the $\mathcal{A}$-enriched category $\mathcal{C}$, and we will generally abuse notation by denoting it also by $\mathcal{C}$. Concretely, this underlying category has the same objects as the enriched category $\mathcal{C}$, with morphism sets given by the formula $\Hom_{\mathcal{C}}(X,Y)=\Hom_{\mathcal{A}}(\bm{1},\sHom_{\mathcal{C}}(X,Y))$.
\end{example}
\begin{remark}
Let $\mathcal{A}$ be a monoidal category and let $\mathcal{C}$ be an ordinary category. We define an \textbf{$\mathcal{A}$-enrichment} of $\mathcal{C}$ to be an $\mathcal{A}$-enriched category $\widetilde{\mathcal{C}}$ together with an identification of $\mathcal{C}$ with the underlying category of $\widetilde{\mathcal{C}}$, in the sense of \cref{monoidal cat enriched underlying cat}.
\end{remark}
\begin{example}[\textbf{Enrichment in Vector Spaces}]
Let $k$ be a field and let $\mathbf{Vect}_k$ denote the category of vector spaces over $k$, endowed with the monoidal structure given by tensor product over $k$ (\cref{monoidal cat Vect_k construction}). Then choosing an $\mathbf{Vect}_k$-enrichment of $\mathcal{C}$ is equivalent to endowing each of the sets $\Hom_{\mathcal{C}}(X,Y)$ with the structure of a $k$-vector space, for which the composition maps
\[\Hom_{\mathcal{C}}(Y,Z)\times\Hom_{\mathcal{C}}(X,Y)\to\Hom_{\mathcal{C}}(X,Z)\]
are $k$-bilinear.
\end{example}
\begin{example}[\textbf{Topologically Enriched Categories}]\label{monoidal cat enriched over Top}
Let $\mathbf{Top}$ denote the category of topological spaces, endowed with the monoidal structure given by the cartesian product (\cref{monoidal cat Cartesian product construction}). A $\mathbf{Top}$-enriched category is also called \textbf{a topologically enriched category}. Note that the functor $F$ of \cref{monoidal cat enriched underlying cat} is (canonically isomorphic to) the forgetful functor $Top\to\mathbf{Set}$. Consequently, if $\mathcal{C}$ is a topologically enriched category, then the underlying ordinary category $\mathcal{C}_0$ can be described concretely as follows:
\begin{itemize}
\item The objects of the ordinary category $\mathcal{C}_0$ are the objects of the Top-enriched category $\mathcal{C}$.
\item Given a pair of objects $X,Y\in\mathcal{C}_0$, a morphism $f$ from $X$ to $Y$ (in the ordinary category $\mathcal{C}_0$) is a point of the topological space $\sHom_{\mathcal{C}}(X,Y)$.
\item Given a pair of morphisms $f:X\to Y$ and $g:Y\to Z$ in $\mathcal{C}_0$, the composition $g\circ f$ is given by the image of $(g,f)$ under the continuous map
\[c_{Z,Y,X}:\sHom_{\mathcal{C}}(Y,Z)\otimes\sHom_{\mathcal{C}}(X,Y)\to\sHom_{\mathcal{C}}(X,Z).\]
It follows that, for any ordinary category $\mathcal{C}_0$, promoting $\mathcal{C}_0$ to a topologically enriched category $\mathcal{C}$ is equivalent to endowing each of the morphism sets $\Hom_{\mathcal{C}_0}(X,Y)$ with a topology for which the composition maps $\circ:\Hom_{\mathcal{C}_0}(Y,Z)\times\Hom_{\mathcal{C}_0}(X,Y)\to\Hom_{\mathcal{C}_0}(X,Z)$ are continuous
\end{itemize}
\end{example}
Let $\mathcal{A}$ be a monoidal category, and let $\mathcal{C}$ and $\mathcal{D}$ be $\mathcal{A}$-enriched categories. An \textbf{$\mathcal{A}$-enriched functor} $F:\mathcal{C}\to\mathcal{D}$ consists of the following data:
\begin{itemize}
\item For every object $X\in\Ob(\mathcal{C})$, and object $F(X)\in\Ob(\mathcal{D})$.
\item For every pair of objects $X,Y\in\Ob(\mathcal{C})$, a morphism
\[F_{X,Y}:\sHom_{\mathcal{C}}(X,Y)\to\sHom_{\mathcal{D}}(F(X),F(Y))\]
in the category $\mathcal{A}$.
\end{itemize}
These data are required to satisfy the following conditions:
\begin{itemize}
\item For every object $X\in\Ob(\mathcal{C})$, the morphism $e_{F(X)}:\mathbf{1}\to\sHom_{\mathcal{D}}(F(X),F(X))$ factors as a composition
\[\begin{tikzcd}
\mathbf{1}\ar[r,"e_X"]&\sHom_{\mathcal{C}}(X,X)\ar[r,"F_{X,X}"]&\sHom_{\mathcal{D}}(F(X),F(X))
\end{tikzcd}\]
\item For every triple of objects $X,Y,Z\in\Ob(\mathcal{C})$, the diagram
\[\begin{tikzcd}
\sHom_{\mathcal{C}}(Y,Z)\otimes\sHom_{\mathcal{C}}(X,Y)\ar[d,swap,"F_{Y,Z}\otimes F_{X,Y}"]&\sHom_{\mathcal{C}}(X,Z)\ar[d,"F_{X,Z}"]&\\
\sHom_{\mathcal{D}}(F(Y),F(Z))\otimes\sHom_{\mathcal{D}}(F(X),F(Y))\ar[r]&\sHom_{\mathcal{D}}(F(X),F(Z))
\end{tikzcd}\]
commutes (in the category $\mathcal{A}$); here the horizontal maps are given by the composition laws on $\mathcal{C}$ and $\mathcal{D}$.
\end{itemize}
\begin{remark}[\textbf{The Category of Enriched Categories}]
Let $\mathcal{A}$ be a monoidal category. We say that an $\mathcal{A}$-enriched category $\mathcal{C}$ is small if the collection of objects $\Ob(\mathcal{C})$ is small. The collection of small $\mathcal{A}$-enriched categories can itself be organized into a category $\mathbf{Cat}(\mathcal{A})$, whose morphisms are given by $\mathcal{A}$-enriched functors.
\end{remark}
\begin{example}
Let $\mathcal{C}$ and $\mathcal{D}$ be small categories, which we regard as $\mathbf{Set}$-enriched categories by means of \cref{monoidal cat enriched over Set}. Then $\mathbf{Set}$-enriched functors from $\mathcal{C}$ to $\mathcal{D}$ can be identified with functors from $\mathcal{C}$ to $\mathcal{D}$ in the usual sense. This identification determines an isomorphism of categories $\mathbf{Cat}(\mathbf{Set})\cong\mathbf{Cat}$.
\end{example}
\begin{remark}
Let $F:\mathcal{A}\to\mathcal{A}'$ be a lax monoidal functor between monoidal categories. Then the construction of \cref{monoidal cat enrichment functoriality} determines a functor $\mathbf{Cat}(\mathcal{A})\to\mathbf{Cat}(\mathcal{A}')$. In the special case where $\mathcal{A}\to\mathbf{Set}$ and $F$ is the functor $A\mapsto\Hom_{\mathcal{A}}(\mathbf{1},A)$ corepresented by the unit object $\mathbf{1}\in\mathcal{A}$, we obtain a forgetful functor $\mathbf{Cat}(\mathcal{A})\to\mathbf{Cat}(\mathbf{Set})\cong\mathbf{Cat}$, which assigns to each (small) $\mathcal{A}$-enriched category $\mathcal{C}$ its underlying ordinary category.
\end{remark}
\begin{example}\label{monoidal cat enriched functor of algebra object char}
Let $\mathcal{A}$ be a monoidal category, let $A$ be an algebra object of $\mathcal{A}$, which we can identify with an $\mathcal{A}$-enriched category $\mathcal{C}_A$ having a single object $X$ (\cref{monoidal cat enriched as algebra object}). For any $\mathcal{A}$-enriched category $\mathcal{D}$ containing an object $Y$, we have a canonical bijection
\[
\begin{tikzcd}[row sep=4mm]
\{\text{$\mathcal{A}$-enriched functors $F:\mathcal{C}_A\to\mathcal{D}$ with $F(X)=Y$}\}\ar[d,"\sim"]\\
\{\text{algebra homomorphisms $A\to\sHom_{\mathcal{D}}(Y,Y)$}\}.
\end{tikzcd}
\]
In particular, if $\mathcal{D}=\mathcal{C}_B$ for some other algebra object $B\in\Alg(\mathcal{D})$, we obtain a bijection
\[\Hom_{\mathbf{Cat}(\mathcal{A})}(\mathcal{C}_A,\mathcal{C}_B)\cong\Hom_{\Alg(\mathcal{A})}(A,B).\]
In other words, the construction $A\mapsto\mathcal{C}_A$ induces a fully faithful embedding $\Alg(\mathcal{A})\to\mathbf{Cat}(\mathcal{A})$, whose essential image is spanned by those $\mathcal{A}$-enriched categories having a single object.
\end{example}
\section{The theory of \texorpdfstring{$2$}{2}-categories}
The collection of (small) categories can itself be organized into a (large) category $\mathbf{Cat}$, whose objects are small categories and whose morphisms are functors. However, the structure of $\mathbf{Cat}$ as an abstract category fails to capture many of the essential features of category theory:
\begin{itemize}
\item Given a pair of functors $F,G:\mathcal{C}\to\mathcal{D}$ with the same source and target, we are usually not interested in the question of whether or not $F$ and $G$ are equal. Instead, we should regard $F$ and $G$ as interchangeable if there exists a natural isomorphism $\alpha:F\Rightarrow G$. This sort of information is not encoded in the structure of the category $\mathbf{Cat}$.
\item Given a pair of categories $\mathcal{C}$ and $\mathcal{D}$, we are usually not interested in the question of whether or not $\mathcal{C}$ and $\mathcal{D}$ are isomorphic. Instead, we should regard $\mathcal{C}$ and $\mathcal{D}$ as interchangeable if there exists an equivalence of categories from $F:\mathcal{C}\to\mathcal{D}$. In this case, the functor $F$ need not be invertible when regarded as a morphism in $\mathbf{Cat}$.
\end{itemize}
To remedy the situation, it is useful to contemplate a more elaborate mathematical structure. A \textbf{strict $2$-category} $\mathcal{C}$ consists of the following data:
\begin{itemize}
\item A collection $\Ob(\mathcal{C})$, whose elements we refer to as objects of $\mathcal{C}$. We will often abuse notation by writing $X\in\mathcal{C}$ to indicate that $X$ is an element of $\Ob(\mathcal{C})$.
\item For every pair of objects $X,Y\in\mathcal{C}$, a category $\sHom_{\mathcal{C}}(X,Y)$. We refer to objects $f$ of the category $\sHom_{\mathcal{C}}(X,Y)$ as $1$-morphisms from $X$ to $Y$ and write $f:X\to Y$ to indicate that $f$ is a $1$-morphism from $X$ to $Y$. Given a pair of $1$-morphisms $f,g\in\sHom_{\mathcal{C}}(X,Y)$, we refer to morphisms from $f$ to $g$ in the category $\Hom_{\mathcal{C}}(X,Y)$ as $2$-morphisms from $f$ to $g$.
\item For every triple of objects $X,Y,Z\in\mathcal{C}$, a \textbf{composition functor}
\[\circ:\sHom_{\mathcal{C}}(Y,Z)\times\sHom_{\mathcal{C}}(X,Y)\to\sHom_{\mathcal{C}}(X,Z).\]
\item For every object $X\in\mathcal{C}$, an identity $1$-morphism $\id_X\in\sHom_{\mathcal{C}}(X,X)$.
\end{itemize}
These data are required to satisfy the following conditions:
\begin{itemize}
\item For each object $X\in\mathcal{C}$, the identity $1$-morphism $\id_X$ is a unit for both right and left composition. That is, for every object $Y\in\mathcal{C}$, the functors
\begin{gather*}
\sHom_{\mathcal{C}}(X,Y)\to\sHom_{\mathcal{C}}(X,Y),\quad f\mapsto f\circ\id_X\\
\sHom_{\mathcal{C}}(Y,X)\to\sHom_{\mathcal{C}}(Y,X),\quad g\mapsto \id_X\circ g
\end{gather*}
are both equal to the identity.
\item The composition law of $\mathcal{C}$ is strictly associative. That is, for every quadruple of objects $W,X,Y,Z\in\mathcal{C}$, the diagram of categories
\[\begin{tikzcd}
\sHom_{\mathcal{C}}(Y,Z)\times\sHom_{\mathcal{C}}(X,Y)\times\sHom_{\mathcal{C}}(W,X)\ar[r,"\id\times\circ"]\ar[d,swap,"\circ\times\id"]&\sHom_{\mathcal{C}}(Y,Z)\times\sHom_{\mathcal{C}}(W,Y)\ar[d,"\circ"]\\
\sHom_{\mathcal{C}}(X,Z)\times\sHom_{\mathcal{C}}(W,X)\ar[r,"\circ"]&\sHom_{\mathcal{C}}(W,Z)
\end{tikzcd}\]
commutes (in the ordinary category $\mathbf{Cat}$).
\end{itemize}
The reader might at this point object that the definition of strict $2$-category violates a fundamental principle of category theory: the axioms of strict $2$-categories require that certain functors are equal. In practice, one often encounters mathematical structures $\mathcal{C}$ which do not quite fit in this framework, because the associative law for composition of $1$-morphisms in $\mathcal{C}$ holds only up to isomorphism. To address this point, B\'enabou introduced a more general type of structure which he called a bicategory, which we will refer to here as a $2$-category.\par
Our goal in this section is then to give a brief introduction to the theory of $2$-categories. We begin by reviewing the definition of a $2$-category and establishing some notational and terminological conventions. Every strict $2$-category can be regarded as a $2$-category, but many of the $2$-categories which arise "in nature" fail to be strict. To articulate the relationship between $2$-categories and strict $2$-categories more precisely, it is convenient to view each as the objects of a suitable (ordinary) category: we do this by introducing the notion of a functor between $2$-categories. Roughly speaking, a functor $F:\mathcal{C}\to\mathcal{D}$ is an operation which carries objects, $1$-morphisms, and $2$-morphisms of $\mathcal{C}$ to objects, $1$-morphisms, and $2$-morphisms of $\mathcal{D}$, which is compatible with the composition laws on $\mathcal{C}$ and $\mathcal{D}$. Here again there are several possible definitions, depending on whether one demands that the compatibility holds strictly (in which case we say that $F$ is a strict functor), up to isomorphism (in which case we say that $F$ is a functor), or up to possible non-invertible $2$-morphism (in which case we say that $F$ is a lax functor). We use this notion to introduce an (ordinary) category $2\mathbf{Cat}$, whose objects are $2$-categories and whose morphisms are functors between $2$-categories (and consider several other variations on this theme).
\subsection{\texorpdfstring{$2$}{2}-categories}
Let $\mathcal{C}$ be a strict $2$-category. Then the composition of $1$-morphisms in $\mathcal{C}$ is strictly associative: that is, given a triple of composable $1$-morphisms
\[f:W\to X,\quad g:X\to Y,\quad h:Y\to Z\]
of C, we have an equality $h\circ(g\circ f)=(h\circ g)\circ f$. Our goal in this paragraph is to introduce the more general notion of (non-strict) $2$-category, where we weaken the associativity requirement: rather than demand that the $1$-morphisms $h\circ(g\circ f)$ and $(h\circ g)\circ f$ are identical, we instead ask for a specified isomorphism $\alpha_{h,g,f}:h\circ(g\circ f)\Rightarrow(h\circ g)\circ f$ in the category $\sHom_{\mathcal{C}}(W,Z)$. In order to obtain a sensible theory, we must require that these isomorphisms satisfy an analogue of the pentagon identity: We define a $2$-category $\mathcal{C}$ to be a collection consisting of the following data:
\begin{itemize}
\item A collection $\Ob(\mathcal{C})$, whose elements we refer to as objects of $\mathcal{C}$. We will often abuse notation by writing $X\in\mathcal{C}$ to indicate that $X$ is an element of $\Ob(\mathcal{C})$.
\item For every pair of objects $X,Y\in\Ob(\mathcal{C})$, a category $\sHom_{\mathcal{C}}(X,Y)$. We refer to objects $f$ of the category $\sHom_{\mathcal{C}}(X,Y)$ as $1$-morphisms from $X$ to $Y$ and write $f:X\to Y$ to indicate that $f$ is a $1$-morphism from $X$ to $Y$. Given a pair of $1$-morphisms $f,g\in\sHom_{\mathcal{C}}(X,Y)$, we refer to morphisms from $f$ to $g$ in the category $\sHom_{\mathcal{C}}(X,Y)$ as $2$-morphisms from $f$ to $g$. We will sometimes write $\gamma:f\Rightarrow g$ to indicate that is $\gamma$ a $2$-morphism from $f$ to $g$.
\item For every triple of objects $X,Y,Z\in\Ob(\mathcal{C})$, a composition functor
\[\circ:\sHom_{\mathcal{C}}(Y,Z)\times\sHom_{\mathcal{C}}(X,Y)\to\sHom_{\mathcal{C}}(X,Z).\]
\item For every object $X\in\Ob(\mathcal{C})$, a $1$-morphism $\id_X\in\sHom_{\mathcal{C}}(X,X)$, which we call the identity $1$-morphism from $X$ to itself.
\item For every object $X\in\Ob(\mathcal{C})$, an isomorphism $\upsilon:\id_X\circ\id_X\Rightarrow\id_X$ in the category $\sHom_{\mathcal{C}}(X,X)$. We refer to the $2$-morphisms $\{\upsilon_X\}_{X\in\Ob(\mathcal{C})}$ as the \textbf{unit constraints} of $\mathcal{C}$.
\item For every quadruple of objects $W,X,Y,Z\in\mathcal{C}$, a natural isomorphism $\alpha$ from the functor
\[\sHom_{\mathcal{C}}(Y,Z)\times\sHom_{\mathcal{C}}(X,Y)\times\sHom_{\mathcal{C}}(W,X)\to\sHom_{\mathcal{C}}(W,Z),\quad (h,g,f)\mapsto h\circ(g\circ f)\]
to the functor
\[\sHom_{\mathcal{C}}(Y,Z)\times\sHom_{\mathcal{C}}(X,Y)\times\sHom_{\mathcal{C}}(W,X)\to\sHom_{\mathcal{C}}(W,Z),\quad (h,g,f)\mapsto (h\circ g)\circ f.\]
We denote the value of $\alpha$ on a triple $(h,g,f)$ by $\alpha_{h,g,f}:h\circ(g\circ f)\Rightarrow(h\circ g)\circ f$, and these isomorphisms are called the \textbf{associativity constraints} of $\mathcal{C}$.\par
\end{itemize}
These data are required to satisfy the following conditions:
\begin{itemize}
\item[(C)] For every pair of objects $X,Y\in\Ob(\mathcal{C})$, the functors
\begin{gather*}
\sHom_{\mathcal{C}}(X,Y)\to\sHom_{\mathcal{C}}(X,Y),\quad f\mapsto f\circ\id_X\\
\sHom_{\mathcal{C}}(X,Y)\to\sHom_{\mathcal{C}}(X,Y),\quad f\mapsto\id_Y\circ f
\end{gather*}
are fully faithful.
\item[(P)] For every quadruple of composable $1$-morphisms
\[\begin{tikzcd}
V\ar[r,"e"]&W\ar[r,"f"]&X\ar[r,"g"]&Y\ar[r,"h"]&Z
\end{tikzcd}\]
in $\mathcal{C}$, the diagram of isomorphisms
\[\begin{tikzcd}[row sep=4mm, column sep=1pt]
&h\circ((g\circ f)\circ e)\ar[rr,Rightarrow,"\alpha_{h,g\circ f,e}"]&&(h\circ(g\circ f))\circ e\ar[rd,Rightarrow,"\alpha_{h,g,f\circ\id_e}"]&\\
h\circ(g\circ(f\circ e))\ar[ru,Rightarrow,"\id_h\circ\alpha_{g,f,e}"]\ar[rrd,Rightarrow,"\alpha_{h,g,f\circ e}"]&&&&((h\circ g)\circ f)\circ e\\
&&(h\circ g)\circ(f\circ g)\ar[rru,Rightarrow,"\alpha_{h\circ g,f,e}"]&&
\end{tikzcd}\]
commutes in the category $\sHom_{\mathcal{C}}(V,Z)$.
\end{itemize}
\begin{example}[\textbf{Ordinary Categories}]\label{2-cat ordinary cat example}
Every ordinary category can be regarded as a strict $2$-category. More precisely, to each category $\mathcal{C}$ we can associate a strict $2$-category $\mathcal{C}'$ as follows:
\begin{itemize}
\item The objects of $\mathcal{C}'$ are objects of $\mathcal{C}$.
\item For every pair of objects $X,Y\in\mathcal{C}$, objects of the category $\sHom_{\mathcal{C}'}(X,Y)$ are elements of the set $\Hom_{\mathcal{C}}(X,Y)$, and every morphism in $\sHom_{\mathcal{C}}(X,Y)$ is an identity morphism.
\item For every triple of objects $X,Y,Z\in\mathcal{C}$, the composition functor
\[\circ:\sHom_{\mathcal{C}'}(Y,Z)\times\sHom_{\mathcal{C}'}(X,Y)\to\sHom_{\mathcal{C}'}(X,Z)\]
is given on objects by the composition map of $\mathcal{C}$.
\item For every object $X\in\mathcal{C}$, the identity object $\id_X\in\sHom_{\mathcal{C}'}(X,X)$ coincides with the identity morphism $\id_X\in\Hom_{\mathcal{C}}(X,X)$.
\end{itemize}
In this situation, we will generally abuse terminology by identifying the strict $2$-category $\mathcal{C}'$ with the ordinary category $\mathcal{C}$.
\end{example}
\begin{example}[\textbf{Strict \bm{$2$}-Categories}]\label{2-cat strict as general 2-cat}
Let $\mathcal{C}$ be any strict $2$-category. Then $\mathcal{C}$ can be viewed as a $2$-category by taking the unit and associativity constraints $\upsilon_X$ and $\alpha_{h,g,f}$ to be identity $2$-morphisms in $\mathcal{C}$.
\end{example}
\begin{remark}\label{2-cat underlying cat if strict}
Let $\mathcal{C}$ be a $2$-category. If $\mathcal{C}$ is strict, then we can extract from $\mathcal{C}$ an underlying ordinary category having the same objects and $1$-morphisms. However, this operation has no counterpart for a general $2$-category $\mathcal{C}$: in general, composition of $1$-morphisms in $\mathcal{C}$ is associative only up to isomorphism.
\end{remark}
\begin{remark}
Let $\mathcal{C}$ be a $2$-category. Then $\mathcal{C}$ can be obtained from an ordinary category if and only if every $2$-morphism in $\mathcal{C}$ is an identity $2$-morphism (note that a $2$-category with this property is automatically strict, by virtue of \cref{2-cat strict as general 2-cat}).
\end{remark}
\begin{remark}[\textbf{Endomorphism Categories}]\label{2-cat endomorphism cat monoidal}
Let $\mathcal{C}$ be a $2$-category and let $X$ be an object of $\mathcal{C}$. We will write $\sEnd_{\mathcal{C}}(X)$ for the category $\sHom_{\mathcal{C}}(X,X)$. Then the composition law
\[\circ:\sHom_{\mathcal{C}}(X,X)\times\sHom_{\mathcal{C}}(X,X)\to\sHom_{\mathcal{C}}(X,X)\]
determines a monoidal structure on the category $\sEnd_{\mathcal{C}}(X)$. Note that, if $\mathcal{C}$ is an ordinary category (regarded as a strict $2$-category by \cref{2-cat ordinary cat example}), then the endomorphism category $\sEnd_{\mathcal{C}}(X)$ can be identified with the endomorphism monoid $\End_{\mathcal{C}}(X)$, regarded as a (strict) monoidal category.
\end{remark}
\begin{example}[\textbf{Delooping}]\label{2-cat delooping of monoidal cat}
Let $\mathcal{M}$ be a category equipped with a (strict) monoidal structure $\otimes:\mathcal{M}\times\mathcal{M}\to\mathcal{M}$. We define a strict $2$-category $B\mathcal{M}$ as follows:
\begin{itemize}
\item The set of objects $\Ob(B\mathcal{M})$ is the singleton set $\{X\}$.
\item The category $\sHom_{B\mathcal{M}}(X,X)$ is equal to $\mathcal{M}$.
\item The composition functor $\circ:\sHom_{B\mathcal{M}}(X,X)\times\sHom_{B\mathcal{M}}(X,X)\to\sHom_{B\mathcal{M}}(X,X)$ is equal to the tensor product $\otimes:\mathcal{M}\times\mathcal{M}\to\mathcal{M}$.
\item The identity morphism $\id_X$ is the (strict) unit object of $\mathcal{M}$.
\end{itemize}
The $2$-category $B\mathcal{M}$ is called the \textbf{delooping} of $\mathcal{M}$. Note that the constructions
\[\mathcal{M}\mapsto B\mathcal{M},\quad \mathcal{C}\mapsto\sEnd_{\mathcal{C}}(X)\]
induce mutually inverse bijections from (strict) monoidal categories to (strict) $2$-categories with a single object.
\end{example}
Let $\mathcal{C}$ be a $2$-category. Then there are two different notions of composition for the $2$-morphisms of $\mathcal{C}$:
\begin{itemize}
\item[(V)] Let $X$ and $Y$ be objects of $\mathcal{C}$. Suppose we are given $1$-morphisms $f,g,h:X\to Y$ and a pair of $2$-morphisms
\[\gamma:f\Rightarrow g,\quad \delta:g\Rightarrow h.\]
We can then apply the composition law in the ordinary category $\sHom_{\mathcal{C}}(X,Y)$ to obtain a $2$-morphism $f\Rightarrow h$, which is called the \textbf{vertical composition} of $\gamma$ and $\delta$.
\item[(H)] Let $X,Y,Z$ be objects of $\mathcal{C}$. Suppose we are given 2-morphisms $\gamma:f\Rightarrow g$ in the category $\sHom_{\mathcal{C}}(X,Y)$ and $\gamma':f'\Rightarrow g'$ in the category HomC(Y,Z). Then the image of $(\gamma',\gamma)$ under the composition law
\[\circ:\sHom_{\mathcal{C}}(Y,Z)\times\sHom_{\mathcal{C}}(X,Y)\to\sHom_{\mathcal{C}}(X,Z)\]
is a $2$-morphism from $f'\circ f$ to $g'\circ g$, which is called the \textbf{horizontal composition} of $\gamma$ and $\gamma'$.
\end{itemize}
The terminology is motivated by the following graphical representations of the data described in (V) and (H):
\[\begin{tikzcd}[row sep=15mm,column sep=15mm]
X\ar[r,bend left=60pt,"f",""{name=U, below}]\ar[r,"g"description,""{name=V1,above},""{name=V2,below}]\ar[r,bend right=60pt,swap,"h",""{name=W,above}]&Y
\arrow[Rightarrow,from=U, to=V1,"\gamma"]
\arrow[Rightarrow,from=V2, to=W,"\delta"{anchor=west}]
\end{tikzcd}\quad\quad\quad
\begin{tikzcd}[row sep=15mm,column sep=15mm]
X\ar[r,bend left=30pt,"f",""{name=U1, below}]\ar[r,bend right=30pt,swap,"f'",""{name=U2, above}]&Y\ar[r,bend left=30pt,"g",""{name=V1, below}]\ar[r,bend right=30pt,swap,"g'",""{name=V2, above}]&Z
\arrow[Rightarrow,from=U1, to=U2,"\gamma"]
\arrow[Rightarrow,from=V1, to=V2,"\delta"]
\end{tikzcd}\]
To avoid confusion, we will generally denote the vertical composition of $2$-morphisms $\gamma$ and $\delta$ by $\delta\gamma$ and the horizontal composition of $2$-morphisms $\gamma$ and $\gamma'$ by $\gamma'\circ\gamma$.
\begin{remark}
Let $\mathcal{C}$ be a $2$-category. For each object $X\in\Ob(\mathcal{C})$, the identity $1$-morphism $\id_X$ and the unit constraint $\upsilon_X$ are determined (up to unique isomorphism) by the composition law and associativity constraints. More precisely, given any other choice of identity morphism $\id_X'$ and unit constraint $\upsilon'_X:\id_X'\circ\id_X'\Rightarrow\id_X'$, there exists a unique invertible $2$-morphism $\gamma:\id_X\Rightarrow\id_X'$ for which the diagram
\[\begin{tikzcd}
\id_X\circ\id_X\ar[r,Rightarrow,"\upsilon_X"]\ar[d,swap,Rightarrow,"\gamma\circ\gamma"]&\id_X\ar[d,Rightarrow,"\gamma"]\\
\id'_X\circ\id'_X\ar[r,Rightarrow,"\upsilon_X'"]&\id_X'
\end{tikzcd}\]
commutes. This follows from \cref{monoidal cat uniqueness of unit} applied to the monoidal category $\sEnd_{\mathcal{C}}(X)$ of \cref{2-cat endomorphism cat monoidal}.
\end{remark}
Let $\mathcal{C}$ be a $2$-category. For every $1$-morphism $f:X\to Y$ in $\mathcal{C}$, we have canonical isomorphisms
\[\begin{tikzcd}
\id_Y\circ(\id_Y\circ f)\ar[r,Rightarrow,"\alpha_{\id_Y,\id_Y,f}"]&(\id_Y\circ\id_Y)\circ f\ar[r,Rightarrow,"\upsilon_Y\circ\id_f"]&\id_Y\circ f
\end{tikzcd}\]
Since composition on the left with $\id_Y$ is fully faithful, it follows that there is a unique isomorphism $\lambda_f:\id_Y\circ f\Rightarrow f$ for which the diagram
\[\begin{tikzcd}
\id_Y\circ(\id_Y\circ f)\ar[rd,Rightarrow,"\id_{\id_Y}\circ\lambda_f"]\ar[rr,Rightarrow,"\alpha_{\id_Y,\id_Y,f}"]&&(\id_Y\circ\id_Y)\circ f\ar[ld,"\upsilon_Y\circ\id_f"]\\
&\id_Y\circ f
\end{tikzcd}\]
commutes. The isomorphism $\lambda_f$ is called the left unit constraint. Similarly, there is a unique isomorphism $\rho_f:f\circ\id_X\Rightarrow f$ for which the diagram
\[\begin{tikzcd}
f\circ(\id_X\circ\id_X)\ar[rd,Rightarrow,"\id_{f}\circ\upsilon_X"]\ar[rr,Rightarrow,"\alpha_{f,\id_X,\id_X}"]&&(f\circ\id_X)\circ\id_X\ar[ld,"\rho_f\circ\id_{\id_X}"]\\
&f\circ\id_X
\end{tikzcd}\]
commutes; The isomorphism $\rho_f$ is called the \textbf{right unit constraint}.
\begin{remark}
Let $\mathcal{C}$ be a $2$-category and let $X$ be an object of $\mathcal{C}$. For every $1$-morphism $f:X\to X$ in $\mathcal{C}$, the left and right unit constraints
\[\lambda_f:\id_X\circ f\Rightarrow f,\quad \rho_f:f\circ\id_X\Rightarrow f\]
coincide with the left and right unit constraints of the monoidal category $\sEnd_{\mathcal{C}}(X)$.
\end{remark}
