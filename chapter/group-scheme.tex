\chapter{Group schemes}
\section{Algebraic structures}
\subsection{Algebraic structures on the category of presheaves}
Given a kind of algebraic structure in the category of sets, we propose to extend it to the category $\mathcal{C}$. Let us first consider an example: the case of groups.
\paragraph{Group objects in \texorpdfstring{$\widehat{\mathcal{C}}$}{C}}
Let $G\in\widehat{\mathcal{C}}$, a \textbf{group structure on $\bm{G}$} is defined to be the assignment of a group structure on the set $G(S)$ for any $S\in\Ob(\mathcal{C})$, so that for any morphism $f:S'\to S$ in $\mathcal{C}$, the map $G(f):G(S)\to G(S')$ is a homomorphism of groups. If $G$ and $H$ are groups in $\widehat{\mathcal{C}}$, a \textbf{group homomorphism} from $G$ to $H$ is defined to be a morphism $\theta\in\Hom(G,H)$ such that for any object $S\in\Ob(\mathcal{C})$, the map $\theta(S):G(S)\to H(S)$ is a homomorphism of groups. We denote by $\Hom_{\Grp}(G,H)$ the set of group homomorphisms from $G$ to $H$, and by $\Grp_{\widehat{\mathcal{C}}}$ the category of groups in $\widehat{\mathcal{C}}$.

\begin{example}
Let $E\in\widehat{\mathcal{C}}$, then the object $\sAut(E)$ is endowed with a group structure. The final object $e$ also possesses a unique group structure and is a final object in $\Grp_{\widehat{\mathcal{C}}}$.
\end{example}

Let $G$ be a group in $\widehat{\mathcal{C}}$. For any $S\in\Ob(\mathcal{C})$, let $e_G(S)$ be the unit element in $G(S)$. The family $e_G(S)$ then defines an element $e_G\in\Gamma(G)=\Hom(e,G)$, which is a morphism of groups $e\to G$ and called the \textbf{unit section} of $G$. We also note that giving a group structure over $G$ amounts to given a composition law over $G$, which is a morphism
\[\mu_G:G\times G\to G\]
such that for any $S\in\Ob(\mathcal{C})$, $\mu_G(S)$ is a group structure on $G(S)$. With the same manner, $f:G\to H$ is a group homomorphism is and only if the following diagram is commutative:
\[\begin{tikzcd}
G\times G\ar[r,"\mu_G"]\ar[d,swap,"{(f,f)}"]&G\ar[d,"f"]\\
H\times H\ar[r,"\mu_H"]&H
\end{tikzcd}\]

A sub-object $H$ of $G$ such that for any $S\in\Ob(\mathcal{C})$, $H(S)$ is a subgroup of $G(S)$ possessing evidently a group structure induced by that of $G$: that is, such that the monomorphism $H\to G$ is a morphism of groups. The group $H$ endowed with this structure is called a \textbf{subgroup} of $G$.\par
If $G$ and $H$ are two groups in $\widehat{\mathcal{C}}$, the product $G\times H$ is endowed with a group structure such that for any $S\in\Ob(\mathcal{C})$, $G(S)\times H(S)$ is endowed with the product group structure. The group $G\times H$ endowed with this structure is called the product group of $G$ and $H$ (and this is also the product in the category $\Grp_{\widehat{\mathcal{C}}}$).\par
If $G$ is a group in $\widehat{\mathcal{C}}$ then for any $S\in\Ob(\mathcal{C})$, $G_S$ is also a group in $\widehat{\mathcal{C}_{/S}}$. If $G$ and $H$ are groups in $\widehat{\mathcal{C}}$, then we can define an object $\sHom_{\Grp}(G,H)$ of $\widehat{\mathcal{C}}$ by
\[\sHom_{\Grp}(G,H)(S)=\Hom_{\Grp}(G_S,H_S).\]
One should note that $\sHom_{\Grp}(G,H)$ is in general not a group, nor a fortiori the object $\sHom$ in the category $\Grp_{\widehat{\mathcal{C}}}$. We define similarly objects $\sIso_{\Grp}(G,H)$, $\sEnd_{\Grp}(G)$ and $\sAut_{\Grp}(G)$.

\begin{definition}
Let $G\in\Ob(\mathcal{C})$. A \textbf{group structure over $\bm{G}$} is defined to be a group structure over $h_G\in\widehat{\mathcal{C}}$. If $G$ and $H$ are groups in $\mathcal{C}$, a group homomorphism from $G$ to $H$ is defined to be an element $f\in\Hom(G,H)\cong\Hom(h_G,h_H)$ which is a group homomorphism from $h_G$ to $h_H$. We denote by $\Grp_{\mathcal{C}}$ the category of groups in $\mathcal{C}$. Note that there is a Cartesian square in $\Cat$:
\[\begin{tikzcd}
\Grp_{\mathcal{C}}\ar[r]\ar[d]&\Grp_{\widehat{\mathcal{C}}}\ar[d]\\
\mathcal{C}\ar[r,"h"]&\widehat{\mathcal{C}}
\end{tikzcd}\]
\end{definition}

The preceding definitions and constructions carries over to groups in $\mathcal{C}$, provided that the corresponding functors (products, $\sHom$ objects, etc.) are representable in $\mathcal{C}$. They also applies to categories of the form $\mathcal{C}_{/S}$, and in this case, we denote by $\sHom_{S\dash\Grp}$ for $\sHom_{\Grp}$, etc.\par
More generally, if $\mathcal{T}$ is a kind of structure over $n$ base sets defined by finite projective limits (for example, by the commutativity of some diagrams constructed from Cartesian products: monoid, group, action by group, module over a ring, Lie algebra over a ring, etc.), we can define the notion of $\mathcal{T}$ structure over $n$ objects $F_1,\dots,F_n$ over $\widehat{\mathcal{C}}$: such a structure is the assignment of a $\mathcal{T}$ structure over the sets $F_1(S),\dots,F_n(S)$ for each $S\in\Ob(\mathcal{C})$, so that for any morphism $S'\to S$ in $\mathcal{C}$, the family of maps $(F_i(S)\to F_i(S'))$ is a poly-homomorphism for the $\mathcal{T}$ structure. We define in a similar way the morphisms of the $\mathcal{T}$ structure, whence a category of $\mathcal{T}$ objects in $\widehat{\mathcal{C}}$. The fully faithful functor $h$ permits us to define the category of $\mathcal{T}$ objects in $\mathcal{C}$ as a fiber product in $\Cat$.\par
Suppose now that in $\mathcal{C}$ the pullbacks exist, and let $\mathcal{T}$ be an algebraic structure defined by the data of certain morphisms between Cartesian products satisfying some axioms consisting of the commutativity of certain diagrams constructed by the previous arrows. A $\mathcal{T}$ structure on a family of objects of $\mathcal{C}$ will therefore be defined by certain morphisms between Cartesian products satisfying certain commutation conditions. It follows that if $\mathcal{C}$ and $\mathcal{C}'$ are two categories with products and $\lambda:\mathcal{C}\to\mathcal{C}'$ is a functor commuting with products, then for any family of objects $(F_i)$ of $\mathcal{C}$ equipped with a $\mathcal{T}$ structure, the family $(f(F_i))$ of objects of $\mathcal{C}'$ will thereby be endowed with a $\mathcal{T}$ structure. For example, any group in $\mathcal{C}$ will be transformed into a group in $\mathcal{C}'$, any pair of a ring in $\mathcal{C}$ and a module over this ring will be transformed into an analogous pair in $\mathcal{C}'$, etc.\par
In particular, let $\mathcal{C}$ be a category, then the constant functor $E\mapsto E_S$ commutes with finite projective limits, and hence transforms groups into $S$-groups (i.e. groups in $\mathcal{C}_{/S}$), rings to $S$-rings, etc.

\begin{remark}
It is worth noting that the previous construction, applied to the category $\widehat{\mathcal{C}}$, restores the notions that have already been defined there. In others words, it amounts to the same thing to give oneself a $\mathcal{T}$ structure over an object of $\widehat{\mathcal{C}}$ when we consider this object as a functor on $\mathcal{C}$, or to give ourselves a $\mathcal{T}$ structure on the representable functor over $\mathcal{C}$ defined by this object. For example, let $G\in\widehat{\mathcal{C}}$; if the functor $F\mapsto\Hom_{\widehat{\mathcal{C}}}(F,G)$ is endowed with a group structure, then so is its restriction to $\mathcal{C}$. Conversely, if $G$ is a group in $\widehat{\mathcal{C}}$, then the multiplication morphism $\mu_G:G\times G\to G$ induces for each $F\in\widehat{\mathcal{C}}$ a group structure over $\Hom_{\widehat{\mathcal{C}}}(F,G)$, which is functorial on $F$.
\end{remark}

\paragraph{Group action in \texorpdfstring{$\widehat{\mathcal{C}}$}{PSh}}\label{category group action in PSh paragraph}
Let $E\in\widehat{\mathcal{C}}$ and $G\in\Grp_{\widehat{\mathcal{C}}}$. A \textbf{$\bm{G}$-object structure} over $E$ is defined to be an assignment over $E(S)$, for each $S\in\Ob(\mathcal{C})$, a $G(S)$-set structure on $G(S)$, so that for any morphism $S'\to S$ in $\mathcal{C}$, the map $E(S)\to E(S')$ is compatible with the group homomorphism $G(S)\to G(S')$. As usual, this is equivalent to giving a morphism
\[\mu:G\times E\to E\]
which for each $S$ endows $E(S)$ with a $G(S)$-set structure. On the other hand, since $\Hom(G\times E,E)\cong\Hom(G,\sEnd(E))$, the morphism $\mu$ defines also a morphism $G\to\sEnd(E)$ and it is immediate to see that this is a group homomorphism which sends $G$ into $\sAut(E)$. Therefore, giving a $G$-object structure over $E$ is equivalent to giving a group homomorphism
\[\rho:G\to\sAut(E).\]
In particular, any element $g\in G(S)$ defines an automorphism $\rho(g)$ of the functor $E_S$, that is, an automorphism of $E\times h_S$ which commutes with the projection $E\times h_S\to h_S$, and in particular an automorphism of $E(S')$ for any morphism $S'\to S$.

\begin{definition}
Let $G$ be a group in $\widehat{\mathcal{C}}$ and $E$ be a $G$-object. We denote by $E^G$ the sub-object of $E$ defined by
\[E^G(S)=\{x\in E(S):\text{$x_{S'}$ is invariant under $G(S')$ for any morphism $S'\to S$}\}.\]
Here $x_{S'}$ is the image of $x$ under $E(S)\to E(S')$. It is clear that $E^G$ (called the \textbf{invariant sub-object} of $E$) is the largest sub-object of $E$ on which $G$ acts trivially. If $F$ is a sub-object of $E$, we denote by $N_G(F)$ and $Z_G(F)$ the subgroups of $G$ defined by
\begin{align*}
N_G(F)(S)&=\{g\in G(S):\rho(g)F_S=F_S\}\\
&=\{g\in G(S):\text{$\rho(S)F(S')=F(S')$ for any morphism $S'\to S$}\},\\
Z_G(F)(S)&=\{g\in G(S):\rho(g)|_{F_S}=\id\}\\
&=\{g\in G(S):\text{$\rho(g)|_{F(S')}=\id$ for any morphism $S'\to S$}\}.
\end{align*}
\end{definition}
In particular, let $x\in\Gamma(E)$, i.e. a collection of elements $x_S\in E(S)$, $S\in\Ob(\mathcal{C})$, such that for any morphism $f:S'\to S$, we have $E(f)(x_s)=x_{S'}$ (if $\mathcal{C}$ admits a final object $S_0$, then we have $\Gamma(E)=E(S_0)$). Then $x$ can be considered as a sub-functor of $E$, also denoted by $x$, and we have $N_G(x)=Z_G(x)$. This common functor is also denoted by $\Stab_G(x)$ and called the \textbf{stabilizer} of $x$. For any $S\in\Ob(\mathcal{C})$, we then have
\[\Stab_G(x)(S)=\{g\in G(S):\rho(g)x_S=x_S\}.\]
Suppose that fiber products exist in $\mathcal{C}$. If $G=h_G$ (resp. $E=h_E$), where $G$ is a group in $\mathcal{C}$ (resp. $E\in\Ob(\mathcal{C})$), and if $\mathcal{C}$ possesses a final object $S_0$, so that $x$ is a morphism $S_0\to E$, then the stabilizer $\Stab_G(x)$ is represented by the fiber product $G\times_ES_0$, where $G\to E$ is the composition of $\id_G\times x:G=G\times S_0\to G\times E$ and $\mu:G\times E\to E$.

\begin{remark}
The formation of $E^G$, $N_G(F)$ and $Z_G(F)$ commute with base changes, so for any $S\in\Ob(\mathcal{C})$, weh ave
\[(E^G)_S=(E_S)^{G_S},\quad N_G(F)_S\cong N_{G_S}(F_S),\quad Z_G(F)_S\cong Z_{G_S}(F_S).\]
\end{remark}

If $G$ is a group in $\mathcal{C}$ and $E$ is an object of $\widehat{\mathcal{C}}$ (resp. an object of $\mathcal{C}$), a $G$-object structure over $E$ is defined to be an $h_G$-object structure over $E$ (resp. $h_E$). With this definition, the above notations carries to $\mathcal{C}$, if the corresponding functors are representable. For example, if $N_{h_G}(h_F)$ is representable, then it is represented by a unique sub-object of $G$, which is then a subgroup of $G$ and denoted by $N_G(F)$.\par
We say that the group $G$ in $\widehat{\mathcal{C}}$ acts on a group $H$ in $\widehat{\mathcal{C}}$ if $H$ is endowed with a $G$-object structure such that, for any $g\in G(S)$, the automorphism of $H(S)$ defined by $g$ is a group automorphism. This is the same to say that for any $g\in G(S)$, the automorphism $\rho(g)$ of $H_S$ is an automorphism of groups in $\widehat{\mathcal{C}_{/S}}$, or that the morphism $G\to\sAut(H)$ sends $G$ into $\sAut_{\Grp}(H)$.\par
In the above situation, there exists over $H\times G$ a unique group structure such that, for any $S\in\Ob(\mathcal{C})$, $(H\times G)(S)$ is the semi-direct product of the groups $H(S)$ and $G(S)$ relative to the given action of $G(S)$ on $H(S)$. This group is denoted by $H\rtimes G$ and called the \textbf{semi-direct product} of $H$ by $G$. By definition, we then have
\[(H\rtimes G)(S)=H(S)\rtimes G(S).\]
Let $G$ be a group in $\widehat{\mathcal{C}}$. For any morphism $S'\to S$ of $\mathcal{C}$ and any $g\in G(S)$, let $\inn(g)$ be the automorphism of $G(S')$ defined by $\inn(g)h=ghg^{-1}$. This definition extends to a morphism of groups in $\widehat{\mathcal{C}}$:
\[\inn:G\to\sAut_{\Grp}(G)\sub\sAut(G).\]
The above definitions then apply to $H$ and we have subgroups $N_G(E)$ and $Z_G(E)$ for any sub-object $E$ of $G$.

\begin{definition}
We define the \textbf{center} of $G$ and denote by $Z(G)$ the subgroup $Z_G(G)$ of $G$. We say that $G$ is \textbf{abelian} if $Z_G(G)=G$ or, equivalently, if $G(S)$ is abelian for any $S\in\Ob(\mathcal{C})$. A subgroup $H$ of $G$ is called \textbf{invariant} in $G$ if $N_G(H)=G$, or equivalently, if $H(S)$ is invariant in $G(S)$ for any $S$. Moreover, we say that $H$ is \textbf{cental} in $G$ if $Z_G(H)=G$, or equivalently, if $H(S)$ is cental in $G(S)$ for any $S$.
\end{definition}

\begin{definition}
Let $f:G\to G'$ be a group homomorphism. The kernel of $f$ is the subgroup of $G$ defined by
\[(\ker f)(S)=\{x\in G(S):f(S)x=1\}=\ker f(S)\]
for any $S\in\Ob(\mathcal{C})$. This is an invariant subgroup of $G$. Note that if $G$ and $G'$ belongs to $\mathcal{C}$, $\mathcal{C}$ possesses a final object $S_0$ and fiber products exist in $\mathcal{C}$, then $\ker(f)$ is represented by $S_0\times_{G'}G$.
\end{definition}

\begin{definition}
Let $E\in\widehat{\mathcal{C}}$ and $G$ be a group acting on $E$. We say that the action of $G$ on $E$ is faithful if the kernel of the morphism $G\to\sAut(E)$ is trivial, that is, if for any $S\in\Ob(\mathcal{C})$ and $g\in G(S)$, the condition $g_{S'}\cdot x=x$ for any morphism $S'\to S$ and $x\in E(S')$ implies $g=1$.
\end{definition}

Many definitions and propositions of elementary group theory are easily transported to the setting of groups in $\widehat{\mathcal{C}}$. Let us simply point out the following which will be useful to us:
\begin{proposition}\label{category presheaf group homomorphism section iff semi-direct}
Let $f:W\to G$ be a group homomorphism and put $H(S)=\ker f(S)$ for $S\in\Ob(\mathcal{C})$. Let $u:G\to W$ be a group homomorphism which is a section of $f$. Then $W$ is identified with a semi-direct product of $H$ by $G$ for the action of $G$ over $H$ defined by $(g,h)\mapsto\inn(u(g))h$ for $g\in G(S)$, $h\in H(S)$ and $S\in\Ob(\mathcal{C})$.
\end{proposition}

All the definitions and propositions are transported as usual to $\mathcal{C}$. We define in particular the semi-product of two groups $H$ and $G$ in $\mathcal{C}$, with $G$ acting on $H$, when the Cartesian product $H\times G$ exists in $\mathcal{C}$. We have the following analogue of \cref{category presheaf group homomorphism section iff semi-direct}:
\begin{proposition}\label{category group homomorphism section iff semi-direct}
Let $f:W\to G$ and $i:H\to W$ be group homomorphisms in $\mathcal{C}$ such that for any $S\in\Ob(\mathcal{C})$, $(H(S),i(S))$ is a kernel of $f(S):W(S)\to G(S)$. Let $u:G\to W$ be a homomorphism of groups in $\mathcal{C}$ which is a section of $f$. Then $W$ is identified with the semi-direct product of $H$ by $G$ for the action of $G$ over $H$ such that if $S\in\Ob(\mathcal{C})$, $g\in G(S)$ and $h\in H(S)$, we have $\inn(u(g))i(h)=i(ghg^{-1})$.
\end{proposition}

To end this paragraph, we breifly introduce the concept of modules over a ring in $\widehat{\mathcal{C}}$. So let $A$ and $M$ be objects of $\widehat{\mathcal{C}}$, we say that $F$ is a \textbf{module over the ring $\bm{A}$}, of simply an $A$-module, if for each $S\in\Ob(\mathcal{C})$ the et $A(S)$ is endowed with a ring structure and $M(S)$ with a module structure over this ring, so that for any morphism $S'\to S$, the map $A(S)\to A(S')$ is a ring homomorphism and $M(S)\to M(S')$ is a bi-homomorphism. If the ring $A$ is fixed, we define as usual morphisms of $A$-modules $M$, $M'$, whence the abelian group $\Hom_A(M,M')$, and the category of $A$-modules, which we denote by $\Mod(A)$.

\begin{proposition}\label{category presheaf Mod(A) is AB5 category}
The category $\Mod(A)$ is endowed with an abelian category structure defined "argument by argument". Moreover, $\Mod(A)$ is an (AB5) category, that is, arbitrary direct sums exist in $\Mod(A)$ and if $M$ is an $A$-module, $N$ is a submodule, and $(M_i)_{i\in I}$ is a filtrant family of increasing submodules of $M$, then
\[\bigcup_{i\in I}(M_i\cap N)=\Big(\bigcup_{i\in I}M_i\Big)\cap N.\]
\end{proposition}
\begin{proof}
In fact, let $f:M\to M'$ be a morphism of $A$-modules. We define the $A$-modules $\ker f$ (resp. $\im f$ and $\coker f$) so that for any $S\in\Ob(\mathcal{C})$, $(\ker f)(S)=\ker f(S)$ (resp. $\cdots$). Then $\ker f$ (resp. $\coker f$) is a kernel (resp. cokernel) of $f$, and we have an isomorphism of $A$-modules $M/\ker f\cong\im f$. This proves that $\Mod(A)$ is an abelian category.\par
Arbitrary direct sums exist in $\Mod(A)$ and are defined "argument by argument". Finally, if $M$ is an $A$-module, $N$ is a submodule, and $(M_i)_{i\in I}$ is a filtrant family of increasing submodules of $M$, then the inclusion
\[\bigcup_{i\in I}(M_i\cap N)\sub \Big(\bigcup_{i\in I}M_i\Big)\cap N\]
is an equality: in fact, if $S\in\Ob(\mathcal{C})$ and $x\in N(S)\cap\bigcup_iM_i(S)$, then there exists $i\in I$ such that $x\in N(S)\cap M_i(S)$.
\end{proof}

\begin{proposition}\label{category presheaf Mod(A) generator if small}
If the category $\mathcal{C}$ is $\mathscr{U}$-small, then $A$ is a generator for the category $\Mod(A)$. Concequently, $\Mod(A)$ is a Grothendieck category, hence possesses enough injectives.
\end{proposition}
\begin{proof}
Let $M$ be an $A$-module. For any $S\in\Ob(\mathcal{C})$, let $M_0(S)$ be a system of generators of the $A(S)$-module $M(S)$. Since, by hypothesis, $\mathcal{C}$ is small, we can consider the set $I=\coprod_{S\in\Ob(\mathcal{C})}M_0(S)$. We then have an epimorphism $A^{\oplus I}\to M$. This proves that $A$ is a generator for $\Mod(A)$ (cf. \cite{tohoku} 1.9.1). As $\Mod(A)$ satisfies (AB5), it then follows from (cf. \cite{tohoku} 1.10.2) that $\Mod(A)$ has enough injectives.
\end{proof}

\begin{remark}
If we consider $\Z$ as a constant functor on $\mathcal{C}$, then the category of $\Z$-modules is isomorphic to the category of abelian groups.
\end{remark}

\begin{definition}
If $M$ is an $A$-module, then for any $S\in\Ob(\mathcal{C})$, $M_S$ is an $A_S$-module, so we can define an abelian group $\sHom_A(M,N)$ by
\[\sHom_A(M,N)(S)=\Hom_{A_S}(M_S,N_S).\]
We define similarly objects $\sIso_A(M,N)$, $\sEnd_A(M)$ and $\sAut_A(M)$, which are groups in $\widehat{\mathcal{C}}$ endowed with the structure of composition.
\end{definition}

\begin{definition}
Let $A$ be a ring in $\widehat{\mathcal{C}}$, $M$ be an $A$-module and $G$ be a group in $\widehat{\mathcal{C}}$. We denote by $A[G]$ the group algebra in $\widehat{\mathcal{C}}$ of $G$ over $A$, so that for any $S\in\Ob(\mathcal{C})$, we have 
\[(A[G])(S)=A(S)[G(S)].\]
An \textbf{$\bm{A[G]}$-module structure} on $M$ is defined to be a $G$-object structure such that for any $S\in\Ob(\mathcal{C})$ and $g\in G(S)$, the automorphim of $F(S)$ defined by $g$ is an automorphism of $A(S)$-module. Equivalently, this means the group homomorphism
\[\rho:G\to\sAut(M)\]
sends $G$ to the subgroup $\sAut_A(M)$ of $\sAut(M)$. Therefore, given an $A[G]$-module structure on $M$, we have a group homomorphism
\[\rho:G\to\sAut_A(M).\]
We define similarly the abelian group $\Hom_{A[G]}(M,N)$ for $A[G]$-modules $M,N$, whence an additive category $\Mod(A[G])$.
\end{definition}
The constructions above are immediately specialized in the case where $G$ (or $A$, or both) are representable by objects of $\mathcal{C}$ which are thereby endowed with corresponding algebraic structures.

\subsection{Algebraic structures on the category of schemes}
We now apply the constructions of the previous paragraph to the category of schemes $\Sch$, and more generally to categories $\Sch_{/S}$. We will simplify the notations in the following way: a group in $\Sch$ will also be called a \textbf{group scheme}, and a group scheme in $\Sch_{/S}$ will be called a \textbf{group scheme over $\bm{S}$}, or an \textbf{$\bm{S}$-group}, or $A$-group when $S$ is the spectrum of a ring $A$.
\paragraph{Constant schemes}
The category of schemes admits direct sums and fiber products, while direct sums commute with base changes. We can then define the constant objects: for any set $E$, we have a scheme $E_\Z$ and for any scheme $S$, an $S$-scheme $E_S=(E_\Z)_S$. Recall that for any $S$-scheme $T$, $\Hom_S(T,E_S)$ is the set of locally constant maps from the space $T$ to $E$.\par
The functor $E\mapsto E_S$ commutes with finite projective limits. In particular, if $G$ is a group, then $G_S$ is a group scheme over $S$; if $A$ is a ring, then $A_S$ is a ring scheme over $S$, etc.
\paragraph{Affine \texorpdfstring{$S$}{S}-groups}
Let $T$ be an affine $S$-scheme, or an $S$-scheme that is affine over $S$. Then the $\mathscr{O}_S$-algebra $f_*(\mathscr{O}_T)$ (also denoted by $\mathscr{A}(T)$), where $f:T\to S$ is the structural morphism, is then quasi-coherent. Conversely, any quasi-coherent $\mathscr{O}_S$-algebra $\mathscr{A}$ corresponds to an affine $S$-scheme $\Spec(\mathscr{A})$, and the constructions $T\mapsto\mathscr{A}(T)$, $\mathscr{A}\mapsto\Spec(\mathscr{A})$ are quasi-inverses of each other. It follows that giving an algebraic structure on an affine $S$-scheme $T$ is equivalent to giving the corresponding structure on $\mathscr{A}(T)$ in the opposite category to that of quasi-coherent $\mathscr{O}_S$-algebras. In particular, if $G$ is an affine $S$-group over $S$, $\mathscr{A}(G)$ is endowed with an augmented $\mathscr{O}_S$-bialgebra structure, that is, we have the following homomorphisms of $\mathscr{O}_S$-algebras
\[\Delta:\mathscr{A}(G)\to\mathscr{A}(G)\otimes_{\mathscr{O}_S}\mathscr{A}(G),\quad \eps:\mathscr{A}(G)\to\mathscr{O}_S,\quad \tau:\mathscr{A}(G)\to\mathscr{A}(G)\]
corresponding to the morphisms of $S$-schemes 
\[\pi:G\times G\to G,\quad e_G:S\to G,\quad i:G\to G.\]
The maps $\Delta$, $\eps$ and $\tau$ satisfy the following conditions (which express that $G$ is an $S$-monoid):
\begin{enumerate}[leftmargin=40pt]
    \item[(HA1)] $\Delta$ is coassociative: the following diagram is commutative
    \[\begin{tikzcd}
    \mathscr{A}(G)\ar[r,"\Delta"]\ar[d,swap,"\Delta"]&\mathscr{A}(G)\otimes_{\mathscr{O}_S}\mathscr{A}(G)\ar[d,"\id\otimes\Delta"]\\
    \mathscr{A}(G)\otimes_{\mathscr{O}_S}\mathscr{A}(G)\ar[r,"\Delta\otimes\id"]&\mathscr{A}(G)\otimes_{\mathscr{O}_S}\mathscr{A}(G)\otimes_{\mathscr{O}_S}\mathscr{A}(G)
    \end{tikzcd}\]
    \item[(HA2)] $\Delta$ is compatible with $\eps$: the following compositions are identities:
    \[\begin{tikzcd}
    \mathscr{A}(G)\ar[r,"\Delta"]&\mathscr{A}(G)\otimes_{\mathscr{O}_S}\mathscr{A}(G)\ar[r,"\id\otimes\eps"]&\mathscr{A}(G)\otimes_{\mathscr{O}_S}\mathscr{O}_S\ar[r,"\sim"]&\mathscr{A}(G)
    \end{tikzcd}\]
    \vspace*{-4mm}
    \[\begin{tikzcd}
    \mathscr{A}(G)\ar[r,"\Delta"]&\mathscr{A}(G)\otimes_{\mathscr{O}_S}\mathscr{A}(G)\ar[r,"\eps\otimes\id"]&\mathscr{O}_S\otimes_{\mathscr{O}_S}\mathscr{A}(G)\ar[r,"\sim"]&\mathscr{A}(G)
    \end{tikzcd}\]
\end{enumerate}
Also, in this case $(\mathscr{A}(G),\Delta,\eps,\tau)$ is a Hopf algebra. Let us take advantage of the circumstance to notice once again that it follows from the definition of an $S$-group structure that in order to give such a structure on a $S$-scheme $G$ affine over $S$, it is not necessary to verify anything on $\mathscr{A}(G)$, but simply endow each $G(S')$ for $S'$ above $S$ with a group structure functorial in $S'$. This remark applies mutatis mutandis to morphisms.

\begin{example}\label{scheme group affine of order 2 char}
Let $S=\Spec(R)$ be an affine scheme and $G=\Spec(A)$ be a group scheme over $S$, with $A$ free of rank $2$. Then $A$ is a Hopf algebra of rank $2$ over $R$. Let $I$ be the aumentation ideal of $A$ (i.e. the kernel of $\eps:A\to R$), and assume that it is free of rank $1$ over $R$, so that we have $A=R\oplus I$. If $x$ is a generator of $I$, there then exists $a,c\in R$ such that
\begin{equation}\label{scheme group affine of order 2 char-1}
x^2=ax,\quad \Delta(x)=1\otimes x+x\otimes 1-c x\otimes x
\end{equation}
(the second equation follows from the fact that $I$ is a Hopf ideal, so $\Delta(I)\sub I\otimes A+A\otimes I$). Since $\Delta$ is a ring homomorphism, we have the following equality:
\begin{align*}
\Delta(ax)&=\Delta(x^2)=\Delta(x)\Delta(x)=(x\otimes 1+1\otimes x)^2+c^2(x\otimes x)^2-2c\cdot x\otimes x(x\otimes 1+1\otimes x)\\
&=a(1\otimes x+x\otimes 1)+(a^2c^2-4ac+2)x\otimes x.
\end{align*}
Comparing the both sides, we then conclude that $ac=2$. Conversely, it is easily seen that any factorization $ac=2\in R$ defines a group scheme of order $2$ over $R$ by (\ref{scheme group affine of order 2 char-1}).
\end{example}

\paragraph{The groups \texorpdfstring{$\G_a$}{G} and \texorpdfstring{$\G_m$}{G}}\label{scheme group G_a and G_m paragraph}
We consider the \textbf{additive group functor} $\G_a:\Sch^{\op}\to\Set$ defined by the formula
\[\G_a(S)=\Gamma(S,\mathscr{O}_S),\]
endowed with the group structure defined by the additive group structure of the ring $\Gamma(S,\mathscr{O}_S)$. This is represented by the affine scheme, which we denote also by $\G_a$, and which is then a group scheme
\[\G_a=\Spec(\Z[T]).\]
In fact, we have bijections
\[\Hom(S,\G_a)=\Hom_{\Alg}(\Z[T],\Gamma(S,\mathscr{O}_S))\cong\Gamma(S,\mathscr{O}_S).\]
For any scheme $S$, we then have an affine $S$-group over $S$, which we denote by $\G_{a,S}$, and it corresponds to the $\mathscr{O}_S$-bigebra $\mathscr{O}_S[T]$ with the comultiplication and counit given by
\[\Delta(T)=T\otimes 1+1\otimes T,\quad \eps(T)=0.\]

Let $\G_m:\Sch^{\op}\to\Set$ be the \textbf{multiplication group functor} defined by
\[\G_m(S)=\Gamma(S,\mathscr{O}_S)^{\times},\]
where $\Gamma(S,\mathscr{O}_S)^{\times}$ denotes the multiplication group of invertible elements in the ring $\Gamma(S,\mathscr{O}_S)$, endowed with the canonical group structure. This is represented by an affine group, which is still denoted by $\G_m$:
\[\G_m=\Spec(\Z[T,T^{-1}])=\Spec(\Z[\Z])\]
where $\Z[\Z]$ is the group algebra of the additive group $\Z$ over the ring $\Z$. In fact,
\[\Hom(S,\Spec(\Z[T,T^{-1}]))=\Hom_{\Alg}(\Z[T,T^{-1}],\Gamma(S,\mathscr{O}_S))\cong\Gamma(S,\mathscr{O}_S)^\times.\]
For any scheme $S$, we then have an affine $S$-group $\G_{m,S}$ over $S$, which corresponds to the $\mathscr{O}_S$-bigebra $\mathscr{O}_S[\Z]$, with the comultiplication and counit given by
\[\Delta(x)=x\otimes x,\quad \eps(x)=1\for x\in\Z.\]

We also note that the set $\Gamma(S,\mathscr{O}_S)$ is a ring for each scheme $S$, so we can endow the functor $\G_a$ with a natural ring structure, which we denote by $\mathbb{O}$. The ring $\mathbb{O}$ is represented by the scheme $\Spec(\Z[T])$, which is also denoted by $\mathbb{O}$, which is then a ring scheme in $\widehat{\Sch}$. For any scheme $S$, $\mathbb{O}_S=S\times_{\Spec(\Z)}\Spec(\Z[T])=\Spec(\mathscr{O}_S[T])$ is then an affine ring scheme over $S$. Note that this ring is also denoted by $S[T]$.\par
For any object $F$ in $\widehat{\Sch}$, the set $\mathbb{O}(F):=\Hom(F,\mathbb{O})$ is then endowed with a ring structure and is functorial on $F$. In particular, if $X$ is a scheme and we are given morphisms $x:X\to F$ and $f:F\to\mathbb{O}$ (that is, $x\in F(X)$ and $f\in\mathbb{O}(F)$), then $f(x):=f\circ x$ is an element in $\mathbb{O}(X)=\Gamma(X,\mathscr{O}_X)$.

\begin{definition}
Let $\pi:M\to X$ be a morphism in $\widehat{\Sch}$, and $\mathbb{O}_X=\mathbb{O}\times X$. We say that $M$ is an \textbf{$\mathbb{O}_X$-module} if for each $X$-scheme $X'$, we are given an $\mathbb{O}(X')$-module structure on $\Hom_X(X',M)$, which is functorial on $X'$. Equivalently, this amounts to giving oneself an $X$-abelian group structure $\mu:M\times_XM\to M$ on $M$ and an "external law"
\[\mathbb{O}\times M=\mathbb{O}_X\times_XM\to M,\quad (f,m)\mapsto f\cdot m\]
which is an $X$-morphism and for any $x\in X(S)$, endows $M(x)=\{m\in M(S):\pi(m)=x\}$ an $\mathbb{O}(S)$-module structure. In this case, for any $Y\in\widehat{\Sch}_{/X}$ (not necessarily representable), the set $\Hom_X(Y,M)=\Gamma(M_Y/Y)$ is an $\mathbb{O}(Y)$-module, which is functorial on $Y$.
\end{definition}

\begin{example}
Let $k$ be a field of characteristic zero and $A$ be a $k$-algebra. Then the set
\[\Hom_{\Grp}(\G_a,\G_m)(\Spec(A))\]
consists of nilpotent elements of $A$; more precisely, all group homomorphisms from $\G_{a,\Spec(A)}$ to $\G_{m,\Spec(A)}$ are of the form $x\mapsto e^{ax}$ with $a\in A$ nilpotent. To see that, note that the underlying schemes of $\G_{a,\Spec(A)}$ and $\G_{m,\Spec(A)}$ are $\Spec(A[X])$ and $\Spec(A[Y,Y^{-1}])$, so any group homomorphism is of the form $Y\mapsto\sum_if_iX^i$ for some $f_i\in A$. The condition that this is a group homomorphism is that
\[\sum_if_i(X_1+X_2)^i=\Big(\sum_if_iX_1^i\Big)\Big(\sum_jf_jX_2^j\Big).\]
Expanding this, we conclude that $f_{i+j}/(i+j)!=f_i/i!f_j/j!$, so every such homomorphism is of the form $f_i=a^i/i!$, and $a$ must be nilpotent since the sum is finite.\par
Now, we conclude that the functor $\sHom_{\Grp}(\G_a,\G_m)$ is not representable. For any positive integer $n$, let $A_n=k[t]/(t^n)$. Then the morphism $x\mapsto e^{tx}$ is in $\Hom_{\Grp}(\G_a,\G_m)(\Spec(A_n))$ for each $n$. However, if $A$ is the inverse limit $k\llbracket t\rrbracket$, then there is no corresponding morphism in $\Hom_{\Grp}(\G_a,\G_m)(\Spec(A))$, so $\sHom_{\Grp}(\G_a,\G_m)$ is not representable.
\end{example}

\paragraph{Diagonalizable groups}\label{scheme diagonalizable group paragraph}
The construction of $\G_m$ can be generalized in the following manner. Let $M$ be an abelian group and $M_\Z$ be the constant group scheme associated with $M$. We then consider the functor $D(M):\Sch^{\op}\to\Set$ defined by
\[D(M)(S)=\Hom_{\Grp}(M_\Z(S),\G_m(S))\cong \Hom_{S\dash\Grp}(M_S,\G_{m,S})\cong \sHom_{\Grp}(M_\Z,\G_m)(S).\]
This is an abelian group in $\widehat{\Sch}$ and is represented by the group scheme $\Spec(\Z[M])$, which is still denoted by $D(M)$. In fact, for any scheme $S$, we have
\[\Hom(S,\Spec(\Z[M]))=\Hom_{\Alg}(\Z[M],\Gamma(S,\mathscr{O}_S))\cong\Hom_{\Grp}(M,\Gamma(S,\mathscr{O}_S)^{\times}).\]

For any scheme $S$, we then obtain an affine group scheme over $S$:
\[D_S(M)=D(M)_S=\sHom_{\Grp}(M_\Z,\G_m)_S=\sHom_{\Grp}(M_S,\G_{m,S}).\]
This is associated with the $\mathscr{O}_S$-bigebra $\mathscr{O}_S[M]$, whose comultiplication and counit are defined by
\[\Delta(x)=x\otimes x,\quad \eps(x)=1\for x\in M.\]

If $f:M\to N$ is a homomorphism of abelian groups, we then have obtain a morphism of $S$-groups
\[D_S(f):D_S(N)\to D_S(M),\]
whence a functor $D_S:M\mapsto D_S(M)$ from the category of abelian groups to the category of affine groups over $S$, which can also be described as the composition of the functor $M\mapsto M_S$ with the functor $M_S\mapsto\sHom_{\Grp}(M_S,\G_{m,S})$. This functor clearly commutes with base changes. An $S$-group isomorphic to a group of the form $D_S(M)$ is called \textbf{diagonalizable}. We note that the elements of $M$ can be interpreted as some characters of $D_S(M)$, that is, certain elements of $\Hom_{\Grp}(D_S(M),\G_{m,S})$ (in fact, this latter group is isomorphic to $M_S$, as we shall see).
\begin{example}
It is clear that we have $D(\Z)=\G_m$ and $D(\Z^n)=(\G_m)^n$. We now consider the group scheme
\[\bm{\mu}_n=D(\Z/n\Z)\]
which is called the \textbf{group of $n$-th roots of unity}. In fact, we have
\[\bm{\mu}_n(S)=\Hom_{\Grp}(\Z/n\Z,\Gamma(S,\mathscr{O}_S)^\times)=\{f\in\Gamma(S,\mathscr{O}_S):f^n=1\}.\]
The $S$-group $\bm{\mu}_{n,S}$ corresponds to the $\mathscr{O}_S$-algebra $\mathscr{O}_S[T]/(T^n-1)$. Suppose in particular that $S$ is the spectrum of a field $k$ of characteristic $p$. Then by putting $T-1=s$, we have
\[k[T]/(T^p-1)=k[s]/(s^p),\]
which shows that the underlying space of $\bm{\mu}_{p,S}$ is reduced to a single point, and the local ring of this point is the Artinian $k$-algebra $k[s]/(s^p)$. By the same ideas, we see that the $S$-schemes $\G_{a,S}$, $\G_{m,S}$, $\mathbb{O}_S$ are smooth on $S$, that $D_S(M)$ is flat on $S$ and that it is formally smooth (resp. smooth) on $S$ if and only if the residual characteristic of $S$ does not divide the torsion of $M$ (resp. and if moreover $M$ is finite type).
\end{example}
\begin{example}
The above proceedure applies to "classical groups" (linear groups $\GL_n$, symplectic groups $\Sp_n$, orthogonal groups $\O_n$, etc.). We define for example $\GL_n$ as representing the functor such that
\[\GL_n(S)=\GL(n,\Gamma(S,\mathscr{O}_S))=\Aut_{\mathscr{O}_S}(\mathscr{O}_S^n).\]
We can construct it for example as the open set of $\Spec(\Z[T_{ij}])$ ($1\leq i,j\leq n$) defined by the function $\det(T_{ij})$, which is $\Spec(\Z[T_{ij},\det(T_{ij})^{-1}])$. Similarly, the special linear group $\SL_n$ is defined to be the kernel of the function $\det:\GL_n\to\G_m$, and it is the closed subscheme $\Spec(\Z[T_{ij}]/(\det(T_{ij}-1)))$ of $\GL_n$. The group structure of $\SL_n$ is thus the induced one by $\GL_n$.
\end{example}

\paragraph{Module functors in the category of schemes}\label{scheme module Gamma functor paragraph}
We now associate with any $\mathscr{O}_S$-module over the schema $S$, an $\mathbb{O}_S$-module (where $\mathbb{O}_S$ denotes the ring functor introduced in \ref{scheme group G_a and G_m paragraph}). This can be done in two different ways, as we shall now define.
\begin{definition}
Let $S$ be a scheme. For any $\mathscr{O}_S$-module $\mathscr{F}$, we denote by $\mathbf{W}(\mathscr{F})$ and $\mathbf{V}(\mathscr{F})$ the contravariant functors over $\Sch_{/S}$ defined by
\[\mathbf{W}(\mathscr{F})(S')=\Gamma(S',\mathscr{F}\otimes_{\mathscr{O}_{S}}\mathscr{O}_{S'}),\quad \mathbf{V}(\mathscr{F})(S')=\Hom_{\mathscr{O}_{S'}}(\mathscr{F}\otimes_{\mathscr{O}_{S}}\mathscr{O}_{S'},\mathscr{O}_{S'}).\]
Then $\mathbf{W}(\mathscr{F})$ and $\mathbf{V}(\mathscr{F})$ are endowed with natural structures of $\mathbb{O}_S$-modules, so that we obtain functors $\mathbf{W}$ and $\mathbf{V}$ from the category of $\mathscr{O}_S$-modules to that of $\mathbb{O}_S$-modules, $\mathbf{W}$  being convariant and $\mathbf{V}$ being contracovariant.
\end{definition}

We often restrict ourselves to the category of quasi-coherent $\mathscr{O}_S$-modules, so that $\mathbf{W}$  and $\mathbf{V}$ are considered as functors from $\Qcoh(\mathscr{O}_S)$ to the category of $\mathbb{O}_S$-modules:
\[\mathbf{W}:\Qcoh(\mathscr{O}_S)\to\Mod(\mathbb{O}_S),\quad \mathbf{V}:\Qcoh(\mathscr{O}_S)^{\op}\to\Mod(\mathbb{O}_S).\]
The reader should however note that most of the propositions in this paragraph do not rely on the quasi-coherence hypothesis.
\begin{proposition}\label{scheme Gamma module functor prop}
Let $S$ be a scheme.
\begin{enumerate}
    \item[(a)] The functors $\mathbf{W}$ and $\mathbf{V}$ commute with base changes: if $S'\to S$ is a morphism and $\mathscr{F}$ is a quasi-coherent $\mathscr{O}_S$-module, then $\mathbf{W}(\mathscr{F}\otimes\mathscr{O}_{S'})\cong\mathbf{W}(\mathscr{F})_{S'}$ and $\mathbf{V}(\mathscr{F}\otimes\mathscr{O}_{S'})\cong \mathbf{V}(\mathscr{F})_{S'}$.
    \item[(b)] The functors $\mathbf{W}$ and $\mathbf{V}$ are fully faithful: the canonical maps
    \begin{gather*}
    \Hom_{\mathscr{O}_S}(\mathscr{F},\mathscr{F}')\to\Hom_{\mathbb{O}_S}(\mathbf{W}(\mathscr{F}),\mathbf{W}(\mathscr{F}')),\quad \Hom_{\mathscr{O}_S}(\mathscr{F},\mathscr{F}')\to\Hom_{\mathbb{O}_S}(\mathbf{V}(\mathscr{F}'),\mathbf{V}(\mathscr{F}))
    \end{gather*}
    are bijective.
    \item[(c)] The functors $\mathbf{W}$ and $\mathbf{V}$ are additive: we have $\mathbf{W}(\mathscr{F}\oplus\mathscr{F}')\cong \mathbf{W}(\mathscr{F})\times_S\mathbf{W}(\mathscr{F}')$ and $\mathbf{V}(\mathscr{F}\oplus\mathscr{F}')\cong \mathbf{V}(\mathscr{F})\times_S\mathbf{V}(\mathscr{F}')$.
\end{enumerate}
\end{proposition}
\begin{proof}
Assertions (a), (c) are clear from the definitions. As for (b), we note that by taking $S'$ to be the open subsets of $S$, we can construct a homomorphism $u:\mathscr{F}\to\mathscr{F}'$ from an $\mathbb{O}_S$-homomorphism $f:\mathbf{W}(\mathscr{F})\to\mathbf{W}(\mathscr{F}')$, and it is immediate to verify that this gives an inverse of the canonical map $\Hom_{\mathscr{O}_S}(\mathscr{F},\mathscr{F}')\to\Hom_{\mathbb{O}_S}(\mathbf{W}(\mathscr{F}),\mathbf{W}(\mathscr{F}'))$. A similar argument proves the case for $\mathbf{V}$.
\end{proof}

We recall that if $F,F'$ are $\mathbb{O}_S$-modules, then $\sHom_{\mathbb{O}_S}(F,F')$ denotes that $S$-functor which associates any morphism $S'\to S$ with $\Hom_{\mathbb{O}_{S'}}(F_{S'},F'_{S'})$.

\begin{proposition}\label{scheme Gamma module functor of sHom morphism}
We have the following canonical morphisms in $\Mod(\mathbb{O}_S)$:
\[\begin{tikzcd}[column sep=3mm]
\sHom_{\mathbb{O}_S}(\mathbf{W}(\mathscr{F}),\mathbf{W}(\mathscr{F}'))\ar[rr,"\sim"]&&\sHom_{\mathbb{O}_S}(\mathbf{V}(\mathscr{F}'),\mathbf{V}(\mathscr{F}))\\
&\mathbf{W}(\sHom_{\mathscr{O}_S}(\mathscr{F},\mathscr{F}'))\ar[ru]\ar[lu]&
\end{tikzcd}\]
\end{proposition}
\begin{proof}
For each $S$-scheme $S'$, we have a canonical homomorphism
\begin{gather*}
\mathbf{W}(\sHom_{\mathscr{O}_S}(\mathscr{F},\mathscr{F}'))(S')=\Gamma(S',\sHom_{\mathscr{O}_S}(\mathscr{F},\mathscr{F}')\otimes\mathscr{O}_{S'})\to \Hom_{\mathscr{O}_{S'}}(\mathscr{F}\otimes\mathscr{O}_{S'},\mathscr{F}'\otimes\mathscr{O}_{S'}).
\end{gather*}
The proposition then follows from \cref{scheme Gamma module functor prop}~(a) and (b).
\end{proof}

\begin{remark}\label{scheme Gamma functor representable by Spec of sym}
Let $\mathscr{F}$ be a quasi-coherent $\mathscr{O}_S$-module. Recall that the $S$-functor $\mathbf{V}(\mathscr{F})$ is represented by an affine $S$-scheme which is denoted by $\V(\mathscr{F})$ and called the vector bundle defined by $\mathscr{F}$:
\[\V(\mathscr{F})=\Spec(\bm{S}(\mathscr{F})),\]
where $\bm{S}(\mathscr{F})$ denotes the symmetric algebra over $\mathscr{F}$. On the other hand, the article (\cite{Nitsure_rep_of_Hom}) shows that if $S$ is Noetherian and $\mathscr{F}$ is a coherent $\mathscr{O}_S$-module, then $\mathbf{W}(\mathscr{F})$ is representable if and only if $\mathscr{F}$ is locally free, and in this case we have an isomorphism $\mathbf{W}(\mathscr{F})\cong\mathbf{V}(\mathscr{F})$.
\end{remark}

\begin{proposition}\label{scheme Gamma module functor Hom with Spec prop}
Let $\mathscr{F}$ and $\mathscr{F}'$ be quasi-coherent $\mathscr{O}_S$-modules and $\mathscr{A}$ be a quasi-coherent $\mathscr{O}_S$-algebra. Then we have a functorial isomorphism
\begin{equation}\label{scheme Gamma module functor Hom with Spec prop-1}
\Hom_S(\Spec(\mathscr{A}),\sHom_{\mathbb{O}_S}(\mathbf{W}(\mathscr{F}'),\mathbf{W}(\mathscr{F})))\stackrel{\sim}{\to} \Hom_{\mathscr{O}_S}(\mathscr{F}',\mathscr{F}\otimes_{\mathscr{O}_S}\mathscr{A}).
\end{equation}
\end{proposition}
\begin{proof}
If we put $X=\Spec(\mathscr{A})$, then by \cref{scheme Gamma module functor prop}, the left side of (\ref{scheme Gamma module functor Hom with Spec prop-1}) is given by
\begin{align*}
\sHom_{\mathbb{O}_S}(\mathbf{W}(\mathscr{F}'),\mathbf{W}(\mathscr{F}))(X)&\cong\Hom_{\mathbb{O}_X}(\mathbf{W}(\mathscr{F}'\otimes\mathscr{O}_X),\mathbf{W}(\mathscr{F}\mathscr{O}_X))\cong\Hom_{\mathscr{O}_X}(\mathscr{F}'\otimes\mathscr{O}_X,\mathscr{F}\otimes\mathscr{O}_X)\\
&\cong \Hom_{\mathscr{O}_S}(\mathscr{F}',\varphi_*(\varphi^*(\mathscr{F})))
\end{align*}
where $\varphi:X\to S$ is the structural morphism. On the other hand, by \cref{scheme S-affine qcoh general product char} we have $\varphi_*(\varphi^*(\mathscr{F}))\cong\mathscr{F}\otimes\mathscr{A}$, so the assertion follows.
\end{proof}

\begin{corollary}\label{scheme Gamma module functor of tensor with algbera char}
We have a canonical isomorphism $\mathbf{W}(\mathscr{F}\otimes\mathscr{A})\cong\sHom_S(\Spec(\mathscr{A}),\mathbf{W}(\mathscr{F}))$.
\end{corollary}
\begin{proof}
Let $f:S'\to S$ be an $S$-scheme and $X'=X\times_SS'$, we then have a Cartesian diagram
\[\begin{tikzcd}
X'\ar[r,"\varphi'"]\ar[d,swap,"f'"]&S'\ar[d,"f"]\\
X\ar[r,"\varphi"]&S
\end{tikzcd}\]
By \cref{scheme S-affine stable under base change} and \cref{scheme S-affine algebra under base change prop}, $X'$ is affine over $S'$ and $\varphi'_*(\mathscr{O}_{X'})=f^*(\mathscr{A})$, so
\[\sHom_S(\Spec(\mathscr{A}),\mathbf{W}(\mathscr{F}))(S')=\Hom_{S'}(\Spec(f^*(\mathscr{A})),\mathbf{W}(f^*(\mathscr{F})))\]
and by \cref{scheme Gamma module functor Hom with Spec prop} applied to $f^*(\mathscr{F})$, $\mathscr{F}'=\mathscr{O}_{S'}$ and $f^*(\mathscr{A})$, this is equal to
\begin{equation*}
\Gamma(S',f^*(\mathscr{F})\otimes f^*(\mathscr{A}))=\Gamma(S',f^*(\mathscr{F}\otimes\mathscr{A}))=\mathbf{W}(\mathscr{F}\otimes\mathscr{A})(S').\qedhere
\end{equation*}
\end{proof}

\begin{proposition}\label{scheme Gamma module functor of sHom locally free prop}
If $\mathscr{F}$ and $\mathscr{F}'$ are locally free of finite type, then the morphisms in \cref{scheme Gamma module functor of sHom morphism} are isomorphisms.
\end{proposition}
\begin{proof}
In fact, for any morphism $S'\to S$, we then have
\[\mathbf{W}(\sHom_{\mathscr{O}_S}(\mathscr{F},\mathscr{F}'))(S')=\Gamma(S',\sHom_{\mathscr{O}_S}(\mathscr{F},\mathscr{F}')\otimes\mathscr{O}_{S'})=\Hom_{\mathscr{O}_S}(\mathscr{F},\mathscr{F}').\]
But this is also isomorphic to $\sHom_{\mathbb{O}_S}(\mathbf{W}(\mathscr{F}),\mathbf{W}(\mathscr{F}'))(S')$ and to $\sHom_{\mathbb{O}_S}(\mathbf{W}(\mathscr{F}),\mathbf{W}(\mathscr{F}'))(S')$, in view of \cref{scheme Gamma module functor prop}~(b).
\end{proof}

\begin{corollary}\label{scheme Gamma module functor isomorphic if locally free}
Let $\mathscr{F}$ be a locally free $\mathscr{O}_S$-module of finite type and put $\check{\mathscr{F}}=\sHom_{\mathscr{O}_S}(\mathscr{F},\mathscr{O}_S)$. Then we have canonical isomorphisms
\begin{align*}
\mathbf{W}(\check{\mathscr{F}})\cong\sHom_{\mathbb{O}_S}(\mathbf{W}(\mathscr{F}),\mathbb{O}_S)\cong\mathbf{V}(\mathscr{F}),\quad \mathbf{V}(\check{\mathscr{F}})\cong\sHom_{\mathbb{O}_S}(\mathbf{V}(\mathscr{F}),\mathbb{O}_S)\cong\mathbf{W}(\mathscr{F}),
\end{align*}
\end{corollary}
\begin{proof}
This follows from \cref{scheme Gamma module functor of sHom locally free prop} by taking $\mathscr{F}'=\mathscr{O}_S$ and note that $\mathbf{W}(\mathscr{O}_S)=\mathbb{O}_S$.
\end{proof}

\begin{proposition}\label{scheme Gamma module functor monomorphism iff split}
If $u:\mathscr{F}\to\mathscr{F}'$ is a morphism of locally free $\mathscr{O}_S$-modules of finite rank, then for $\mathbf{W}(u):\mathbf{W}(\mathscr{F})\to\mathbf{W}(\mathscr{F}')$ to be a monomorphism, it is necessary and sufficient that $f$ identifies $\mathscr{F}$ locally as a direct factor of $\mathscr{F}'$.
\end{proposition}
\begin{proof}
One direction follows essentially from \cref{sheaf of module homomorphism ft to local free prop}. Conversely, if $\mathscr{F}$ is a direct factor of $\mathscr{F}'$, then for any $f:S'\to S$, $f^*(\mathscr{F})$ is a submodule of $f^*(\mathscr{F}')$, so $\mathbf{W}(\mathscr{F})(S')=\Gamma(S',f^*(\mathscr{F}))$ is a submodule of $\mathbf{W}(\mathscr{F}')(S')=\Gamma(S',f^*(\mathscr{F}'))$.
\end{proof}

\paragraph{The category of \texorpdfstring{$\mathscr{O}_S[G]$}{O}-modules}
Let $G$ be an $S$-group and $\mathscr{F}$ be an $\mathscr{O}_S$-module. Then an \textbf{$\mathscr{O}_S[G]$-module structure} on $\mathscr{F}$ is defined to be an $\mathbb{O}_S[h_G]$-module structure on $\mathbf{W}(\mathscr{F})$. A morphism of $\mathscr{O}_S[G]$-modules is by definition a morphism of the associated $\mathbb{O}_S[h_G]$-modules. We thus obtain a category $\Mod(\mathscr{O}_S[G])$ of $\mathscr{O}_S[G]$-modules and the full subcategory $\Qcoh(\mathscr{O}_S[G])$ formed by quasi-coherent $\mathscr{O}_S$-modules. By definition, giving an $\mathscr{O}_S[G]$-module structure on $\mathscr{F}$ is equivalent to giving a morphism of groups
\[\rho:h_G\to\sAut_{\mathbb{O}_S}(\mathbf{W}(\mathscr{F})).\]

\begin{remark}
Since by \cref{scheme Gamma module functor prop} we have an anti-isomorphism
\[\sAut_{\mathbb{O}_S}(\mathbf{W}(\mathscr{F}))\cong\sAut_{\mathbb{O}_S}(\mathbf{V}(\mathscr{F})),\]
we see that an $\mathbb{O}_S[h_G]$-module structure on $\mathbf{W}(\mathscr{F})$ is equivalent to an $\mathbb{O}_S[h_G]$-module structure on $\mathbf{V}(\mathscr{F})$, and these two structures are connected by the operation $\rho(g)\mapsto \rho^*(g^{-1})$, where $\rho^*$ denotes the image of $\rho:h_G\to\sAut_{\mathbb{O}_S}(\mathbf{W}(\mathscr{F}))$ under the above isomorphism.
\end{remark}

\begin{remark}
The categories we have just constructed can also be defined by the following Cartesian squares:
\[\begin{tikzcd}
\Qcoh(\mathscr{O}_S[G])\ar[r,hook]\ar[d]&\Mod(\mathscr{O}_S[G])\ar[r]\ar[d]&\Mod(\mathbb{O}_S[h_G])\ar[d,"\text{forget}"]\\
\Qcoh(\mathscr{O}_S)\ar[r,hook]&\Mod(\mathscr{O}_S)\ar[r]&\Mod(\mathbb{O}_S)
\end{tikzcd}\]
The categories $\Mod(\mathscr{O}_S)$ and $\Mod(\mathbb{O}_S)$ are abelian, but one should be careful that in general the functor $\mathbf{W}$ is not exact, neither left nor right.
\end{remark}

\begin{remark}\label{scheme module over group invariant subshaef def}
Let $\mathscr{F}$ be an $\mathscr{O}_S[G]$-module. The \textbf{subsheaf of invariants} $\mathscr{F}^G$ is defined as follows: for any open subset $U$ of $S$,
\[\mathscr{F}^G(U)=\mathbf{W}(\mathscr{F})^G(U)=\{x\in\mathscr{F}(U):\text{$g\cdot x_{S'}=x_{S'}$ for any morphism $f:S'\to U$ and $g\in G(S')$}\}\]
where $x_{S'}$ denotes the image of $x$ in $\Gamma(S',f^*(\mathscr{F}))=\Gamma(U,f_*(f^*(\mathscr{F})))$.\par
Be careful that the natural morphism $\mathbf{W}(\mathscr{F}^G)\to\mathbf{W}(\mathscr{F})^G$ is not an isomorphism in general. For example, if $S=\Spec(\Z)$ and $G$ is the constant group $\Z/2\Z=\{1,\tau\}$ acting on $\mathscr{F}=\mathscr{O}_S$ via $\tau\cdot 1=-1$, then we have $\mathscr{F}^G=0$ since the ring $\Gamma(U,\mathscr{F})$ has characteristic zero for any standard open $U$ of $S$. However, it is clear that $\mathbf{W}(\mathscr{F})^G(\Spec(R))=R$ for any $\F_2$-algebra $R$.
\end{remark}

From now on, we restrict ourselves to the case where the group scheme $G$ is affine over $S$. Then, in view of \cref{scheme Gamma module functor Hom with Spec prop}, giving a morphism of $S$-functors
\[\rho:h_G\to\sAut_{\mathbb{O}_S}(\mathbf{W}(\mathscr{F}))\]
is equivalent to giving a morphism of $\mathscr{O}_S$-modules
\[\mu:\mathscr{F}\to\mathscr{F}\otimes_{\mathscr{O}_S}\mathscr{A}(G).\]
The condition that $\rho$ is a group homomorphism is then translated into the folllowing conditions on $\mu$:
\begin{enumerate}[leftmargin=40pt]
    \item[(CM1)] the following diagram is commutative:
    \[\begin{tikzcd}
    \mathscr{F}\ar[r,"\mu"]\ar[d,swap,"\mu"]&\mathscr{F}\otimes_{\mathscr{O}_S}\mathscr{A}(G)\ar[d,"\id\otimes\Delta"]\\
    \mathscr{F}\otimes_{\mathscr{O}_S}\mathscr{A}(G)\ar[r,"\mu\otimes\id"]&\mathscr{F}\otimes_{\mathscr{O}_S}\mathscr{A}(G)\otimes_{\mathscr{O}_S}\mathscr{A}(G)
    \end{tikzcd}\]
    \item[(CM2)] the following composition is the identity:
    \[\begin{tikzcd}
    \mathscr{F}\ar[r,"\mu"]&\mathscr{F}\otimes_{\mathscr{O}_S}\mathscr{A}(G)\ar[r,"\id\otimes\eps"]&\mathscr{F}\otimes\mathscr{O}_S\ar[r,"\sim"]&\mathscr{F}
    \end{tikzcd}\]
\end{enumerate}
These two axioms then endow a \textit{comodule structure} on $\mathscr{F}$ over the bigebra $\mathscr{A}(G)$.\par
Put $\mathscr{A}=\mathscr{A}(G)$. If $\mathscr{F}$ and $\mathscr{F}'$ are $\mathscr{A}$-comodules, a morphism $f:\mathscr{F}\to\mathscr{F}'$ of comodules is then defined to be a morphism of $\mathscr{O}_S$-modules such that the following diagram is commutative:
\[\begin{tikzcd}
\mathscr{F}\ar[r,"f"]\ar[d,swap,"\mu_{\mathscr{F}}"]&\mathscr{F}'\ar[d,"\mu_{\mathscr{F}'}"]\\
\mathscr{F}\otimes\mathscr{A}\ar[r,"f\otimes\id"]&\mathscr{F}'\otimes\mathscr{A}
\end{tikzcd}\]
We thus obtain a category $\CoMod(\mathscr{A})$ of comodules over $\mathscr{A}$, and we denote by $\CoQcoh(\mathscr{A})$ the full subcategory formed by quasi-coherent $\mathscr{O}_S$-modules. From the above remarks, it is also clear that we have the following:
\begin{proposition}\label{scheme module over affine group cat equivalence}
Let $G$ be an affine $S$-group. Then we have equivalences of categories:
\[\Mod(\mathscr{O}_S[G])\cong\CoMod(\mathscr{A}(G)),\quad \Qcoh(\mathscr{O}_S[G])\cong\CoQcoh(\mathscr{A}(G)).\]
If moreover $S=\Spec(A)$ is affine and we put $A[G]=\Gamma(S,\mathscr{A}(G))$, then we have an equivalence of categories
\[\CoQcoh(\mathscr{A}(G))\cong\CoMod(A[G]).\]
\end{proposition}

\begin{proposition}\label{scheme module over flat affine group cat is abelian}
If $G$ is affine and flat over $S$, then the category $\Mod(\mathscr{O}_S[G])$ (resp. $\Qcoh(\mathscr{O}_S[G])$), being equivalent to the category of $\mathscr{A}(G)$-comodules (resp. quasi-coherent over $\mathscr{O}_S$), is abelian.
\end{proposition}
\begin{proof}
Suppose that $\mathscr{A}=\mathscr{A}(G)$ is a flat $\mathscr{O}_S$-module. Let $\mathscr{E}$ be an $\mathscr{A}$-comodule and $\mathscr{F}$ be a sub-$\mathscr{O}_S$-module of $\mathscr{E}$. As $\mathscr{A}$ is flat over $\mathscr{O}_S$, we can identify $\mathscr{F}\otimes\mathscr{A}$ (resp. $\mathscr{F}\otimes\mathscr{A}\otimes\mathscr{A}$) as a sub-$\mathscr{O}_S$-module of $\mathscr{E}$ (resp. $\mathscr{E}\otimes\mathscr{A}\otimes\mathscr{A}$). Assume that $\mu_{\mathscr{E}}$ sends $\mathscr{F}$ into $\mathscr{F}\otimes\mathscr{A}$, then the restriction $\mu_\mathscr{F}:\mathscr{F}\to\mathscr{F}\otimes\mathscr{A}$ induces a comodule structure on $\mathscr{F}$, and we say that $\mathscr{F}$ is a sub-comodule of $\mathscr{E}$. By passing to quotient, $\mu_\mathscr{E}$ then defies a morphism of $\mathscr{O}_S$-modules $\mathscr{E}/\mathscr{F}\to\mathscr{E}/\mathscr{F}\otimes\mathscr{A}$, which endows $\mathscr{E}/\mathscr{F}$ with an $\mathscr{A}$-comodule structure.\par
Now if $f:\mathscr{E}\to\mathscr{E}'$ is a morphism of $\mathscr{A}$-comodules, then $\ker f$ (resp. $\im f$) is a sub-$\mathscr{A}$-comodule of $\mathscr{E}$ (resp. $\mathscr{E}'$), and $f$ induces an isomorphism $\mathscr{E}/\ker f\stackrel{\sim}{\to} \im f$ of $\mathscr{A}$-comodules. Moreover, if $\mathscr{E}$ and $\mathscr{E}'$ are quasi-coherent $\mathscr{O}_S$-modules, then so are $\ker f$ and $\im f$. Therefore, we conclude that $\CoMod(\mathscr{A})$ and $\CoQcoh(\mathscr{A})$ are abelian categories.
\end{proof}

We now suppose further that $G$ is a diagonalizable group, which means $\mathscr{A}(G)$ is the algebra of an abelian group $M$ over the ring $\mathscr{O}_S$. If $\mathscr{F}$ is an $\mathscr{O}_S$-module, we then have
\[\mathscr{F}\otimes\mathscr{A}(G)=\bigoplus_{m\in M}\mathscr{F}\otimes m\mathscr{O}_S,\]
so giving a morphism $\mu:\mathscr{F}\to\mathscr{F}\otimes\mathscr{A}(G)$ is equivalent to giving a family of endomorphisms $(\mu_m)_{m\in M}$ of $\mathscr{F}$ such that for any section $x$ of $\mathscr{F}$ over an open subset $S$, $(\mu_m(x))$ is a section of the direct sum $\bigoplus_{m\in M}\mathscr{F}$ (this means that over any sufficiently small open subset, there are only a finite number of restrictions of the $\mu_m(x)$ which are non-zero). For a morphism $\mu$ defined by
\[\mu(x)=\sum_{m\in M}\mu_m(x)\otimes m\]
to satisfy (CM1) and (CM2), it is necessary and sufficient that we have 
\[\mu_m\circ\mu_n=\delta_{mn}\mu_m,\quad \sum_{m\in M}\mu_m=\id_\mathscr{F}\]
which signify that the $\mu_m$ are orthogonal projections adding up to the identity. We have therefore proved the following result:
\begin{proposition}\label{scheme module over diagonalizable group cat equivalent to graded module}
If $G=D_S(M)$ is a diagonalizable group over $S$, then the category of $\mathscr{O}_S[G]$-modules (resp. quasi-coherent $\mathscr{O}_S[G]$-modules) is equivalent to the category of graded $\mathscr{O}_S$-modules (resp. quasi-coherent $\mathscr{O}_S[G]$-modules) of type $M$.
\end{proposition}

\begin{remark}
More precisely, if $\mathscr{F}$ is an $\mathscr{O}_S[G]$-module and $\mathscr{F}=\bigoplus_{m\in M}\mathscr{F}_m$ is the corresponding graduation on $\mathscr{F}$, then for any $S'$ over $S$ and $g\in G(S')$, the action of $g$ on a section $x$ of $\mathscr{F}\otimes_{\mathscr{O}_S}\mathscr{O}_{S'}$ over an open subset $S$ is given by
\[g\cdot x=\sum_m\phi(m)x_m\]
where $\phi:M\to\Gamma(S',\mathscr{O}_{S'})^\times$ is the morphism corresponding to $g$.
\end{remark}

\begin{corollary}\label{scheme affine acted by diagonalizable group equivalent to graded alg}
The functor $\mathscr{A}\mapsto\Spec(\mathscr{A})$ induces an equivalence from the category of graded quasi-coherent $\mathscr{O}_S$-algebras of type $M$ to the opposite category of that of affine $S$-schemes acted by the group $G=D_S(M)$.
\end{corollary}
\begin{proof}
If $X$ is an affine scheme over $S$ acted by the affine $S$-group $D_S(M)$, then $\mathscr{A}(S)$ is a quasi-coherent $\mathscr{O}_S$-algebra which is acted by $G$, whence a graded $\mathscr{O}_S$-algebra of type $M$. The converse of this is immediate.
\end{proof}

\begin{proposition}\label{scheme module over diagonalizable group sequence split iff}
Let $G$ be a diagonalizable group over $S$. If
\[\begin{tikzcd}
0\ar[r]&\mathscr{F}_1\ar[r]&\mathscr{F}_2\ar[r]&\mathscr{F}_3\ar[r]&0
\end{tikzcd}\]
is an exact sequence of quasi-coherent $\mathscr{O}_S[G]$-modules which split as a sequence of $\mathscr{O}_S$-modules, then it splits as a sequence of $\mathscr{O}_S[G]$-modules..
\end{proposition}
\begin{proof}
If $G=D_S(M)$, then each $\mathscr{F}_i$ is graded by the $(\mathscr{F}_i)_m$ and for each $m\in M$ the sequence
\[\begin{tikzcd}
0\ar[r]&(\mathscr{F}_1)_m\ar[r]&(\mathscr{F}_2)_m\ar[r]&(\mathscr{F}_3)_m\ar[r]&0
\end{tikzcd}\]
of $\mathscr{O}_S$-modules is splitting. The proposition then follows from \cref{scheme module over diagonalizable group cat equivalent to graded module}, since the corresponding result for graded modules is true.
\end{proof}

\subsection{Cohomology of groups}\label{category cohomology of group subsection}
\paragraph{The standard complex}\label{category cohomology of group standard complex paragraph}
Let $\mathcal{C}$ be a category, $G$ be a group in $\widehat{\mathcal{C}}$, $A$ be a ring and $M$ be a $A[G]$-module. For $n\geq 0$, we put
\[C^n(G,M)=\Hom(G^n,M),\quad \mathcal{C}^n(G,M)=\sHom(G^n,M),\]
where $G^0$ is the final object $e$ of $\widehat{\mathcal{C}}$. Then $\mathcal{C}^n(G,M)$ (resp. $C^n(G,M)$) is endowed evidently with a structure of $\mathbb{O}$-module (resp. $\Gamma(\mathbb{O})$-module), and we have
\[C^n(G,M)\cong\Gamma(\mathcal{C}^n(G,M)),\quad \mathcal{C}^n(G,M)(S)=C^n(G_S,M_S).\]
Giving an element of $C^n(G,M)$ is then equivalent to giving for each $S\in\Ob(\mathcal{C})$ an $n$-cochain of $G(S)$ in $M(S)$, which is functorial on $S$. The boundary operator
\[d:C^n(G(S),M(S))\to C^{n+1}(G(S),M(S)),\]
which is defined by the formula
\begin{align*}
(df)(g_1,\dots,g_{n+1})&=g_1\cdot f(g_2,\dots,g_{n+1})+\sum_{i=1}^{n}(-1)^if(g_1,\dots,g_ig_{i+1},\dots,g_{n+1})\\
&+(-1)^{n+1}f(g_1,\dots,g_n)
\end{align*}
is then functorial on $S$ and hence defines a homomorphism
\[d:C^n(G,M)\to C^{n+1}(G,M)\]
such that $d\circ d=0$. We then obtain a complex of abelian groups, which we denote by $C^\bullet(G,M)$. We define similarly a complex of $A$-modules $\mathcal{C}^n(G,M)$, and we have
\[C^\bullet(G,M)=\Gamma(\mathcal{C}^n(G,M)).\]
We denote by $H^n(G,M)$ (resp. $\mathcal{H}^n(G,M)$) the cohomology group of the complex $C^\bullet(G,M)$ (resp. $\mathcal{C}^\bullet(G,M)$). In particular, we have
\[\mathcal{H}^0(G,M)=M^G,\quad H^0(G,M)=\Gamma(M^G).\]

\begin{remark}
The set-theoretic definition of $d$ is given to verify that $d\circ d=0$. We can also define $d$ in terms of the multiplication $m:G\times G\to G$ and the action $\mu:G\times M\to M$ as follows: for any $f\in C^n(G,M)$,
\[df=\mu\circ(\id_G\times f)+\sum_{i=1}^{n}(-1)^if\circ(\id_{G^{i-1}}\times m\times\id_{G^{n-i}})+(-1)^{n+1}f\circ\pr_{[1,n]},\]
where $\pr_{[1,n]}$ is the projection of $G^{n+1}=G^{n}\times G$ to $G^n$. Similarly, for any $S\in\Ob(\mathcal{C})$ and $f\in\Ob(\mathcal{C})^n(G,M)(S)=C^n(G_S,M_S)$, we have
\[df=\mu_S\circ(\id_G\times f)+\sum_{i=1}^{n}(-1)^if\circ(\id_{G_S^{i-1}}\times m_S\times\id_{G_S^{n-i}})+(-1)^{n+1}f\circ\pr_{[1,n]},\]
where $m_S$ and $\mu_S$ are defined by base change.
\end{remark}

We recall that $\Mod(A[G])$ is endowed with an abelian category structure, defined "argument by argument" (\cref{category presheaf Mod(A) is AB5 category}); therefore a sequence of $A[G]$-modules
\[\begin{tikzcd}
0\ar[r]&M'\ar[r]&M\ar[r]&M''\ar[r]&0
\end{tikzcd}\]
is exact if and only the sequence of abelian groups
\[\begin{tikzcd}
0\ar[r]&M'(S)\ar[r]&M(S)\ar[r]&M''(S)\ar[r]&0
\end{tikzcd}\]
is exact for any $S\in\Ob(\mathcal{C})$. If $\mathcal{C}$ is $\mathscr{U}$-small, then by \cref{category presheaf Mod(A) generator if small}, $\Mod(A[G])$ possesses enough injectives, so that the derived functors of the left exact functors $\mathcal{H}^0$ and $H^0$ can be defined. We now show that the functors $\mathcal{H}^n$ and $H^n$ are isomorphic to the derived functors of $\mathcal{H}^0$ and $H^0$, respectively.

\begin{definition}
Let $H$ be a subgroup functor of $G$. For any $A[H]$-module $P$, we denote by $\CoInd^G_H(P)$ the object $\sHom_H(G,P)$ of $\widehat{\mathcal{C}}$, endowed with the structure of an $A[G]$-module defined as follows: for any $S\in\Ob(\mathcal{C})$, the set $\sHom_H(G,P)(S)=\Hom_{H_S}(G_S,P_S)$ consists of $H$-equivalent morphisms $\phi:G_S\to P_S$, where $H$ acts by the inversion of right regular representation, i.e. such that we have
\[\phi(gh)=h^{-1}\cdot\phi(g),\quad h\in H(S'),g\in G(S'),S'\to S;\]
we act $g\in G(S)$ and $a\in A[S]$ on $\phi\in\Hom_{H_S}(G_S,P_S)$ by the formule
\[(g\cdot\phi)(g')=\phi(g^{-1}g'),\quad (a\cdot\phi)(g')=a\phi(g'),\]
for any $g'\in G(S')$ and $S'\to S$\footnote{Note that $(g\cdot\phi)(g'h)=\phi(g^{-1}g'h)=h^{-1}\cdot\phi(g^{-1}g')=h^{-1}\cdot((g\cdot\phi)(g'))$, so $g\cdot\phi\in\Hom_{H_S}(G_S,P_S)$.}. Moreover, for any $\phi\in\Hom_{H_S}(G_S,P_S)$, we set
\[\eps(\phi)=\phi(1)\in P(S)\]
where $1$ denotes the unit element of $G(S)$. Then it is clear that the construction of $\CoInd^G_H(P)$ is functorial on $P$, and we have thus defined a functor $\CoInd^G_H:\Mod(A[H])\to\Mod(A[G])$ and a natural transform $\Res^G_H\circ\CoInd^G_H\to\id$, where $\Res^G_H$ denotes the forgetful functor (the restriction functor). If $H$ is the trivial subgroup of $G$, then we simply write $\CoInd(P)$; note that in this case $\CoInd(P)=\sHom(G,P)$.
\end{definition}

\begin{proposition}\label{category presheaf group module forgetful CoInd adjoint}
Let $H$ be a subgroup functor of $G$.
\begin{enumerate}
    \item[(a)] For $P\in\Mod(A[H])$ the map $\eps_P:\CoInd^G_H(P)\to P$ is a homomorphism of $A[H]$-modules.
    \item[(b)] The functor $\CoInd^G_H$ is right adjoint to the forgetful functor $\Res^G_H:\Mod(A[G])\to\Mod(A[H])$. More precisely, $\eps:\CoInd^G_H\to\id$ induces a bijection 
    \[\Hom_{A[G]}(M,\CoInd^G_H(P))\stackrel{\sim}{\to}\Hom_{A[H]}(\Res^G_H(M),P)\]
    for any $M\in\Mod(A[G])$ and $P\in\Mod(A[H])$.
\end{enumerate}
Therefore, if $I$ is an injective object of $\Mod(A)$, then $\CoInd(I)$ is an injective object of $\Mod(A[G])$.
\end{proposition}
\begin{proof}
The first assertion follows from the simple observation that for $g\in G(S)$, we have $g\cdot\eps(\phi)=\phi(g)=\eps(g\cdot\phi)$. To prove (b), for any $A[H]$-morphism $f:M\to P$, we associate an element $\phi_f\in\Hom_{A[G]}(M,\CoInd^G_H(P))$ defined as follows: for $S\in\Ob(\mathcal{C})$ and $m\in M(S)$, $\phi_f(m)$ is the element of $\Hom_{H_S}(G_S,P_S)$ such that for any $g\in G(S')$, $S'\to S$,
\[\phi_f(m)(g)=f(g^{-1}m)\in P(S').\]
Note that this is an $A[H]$-equivalent map for $\CoInd^G_H(P)$ because we have
\[h^{-1}\cdot\phi_f(m)(g)=h^{-1}\cdot f(g^{-1}m)=f(h^{-1}g^{-1}m)=\phi_f(m)(gh).\]
Moreover, for any $g\in G(S)$, we note that
\[(\phi_f(gm))(g')=f(g'^{-1}gm)=\phi_f(m)(g^{-1}g')=(g\cdot\phi_f(m))(g'),\]
so $\phi_f$ belongs to $\Hom_{A[G]}(M,\CoInd^G_H(P))$. Conversely, if $\phi\in\Hom_{A[G]}(M,\CoInd(P))$ and we write, for $m\in M(S)$, $f(m)=\phi(m)(1)$, then $f$ is an $A[H]$-morphism from $M$ to $P$\footnote{In fact, for $h\in G(S)$, we have $f(hm)=\phi(hm)(1)=(h\cdot\phi(m))(1)=\phi(m)(h^{-1})=h\cdot\phi(m)(1)$, so $f$ is $A[G]$-linear.} and
\[\phi_f(m)(g)=f(g^{-1}m)=\phi(g^{-1}m)(1)=(g\cdot\phi(m))(1)=\phi(m)(g),\]
so $\phi_f=\phi$. It is clear that $\phi_f(m)(1)=f(m)$, whence the assertion in (b). The last assertion then follows since the forgetful functor $\Res^G_H$ is exact.
\end{proof}

\begin{definition}\label{category presheaf group module forgetful CoInd unit def}
Let $M$ be an $A[G]$-module; the identity map on $M$ (considered as an $A$-module) corresponds by adjunction to a morphism of $A[G]$-modules
\[\eta_M:M\to \CoInd(M)\]
such that for $S\in\Ob(\mathcal{C})$ and $m\in M(S)$, $\eta_M(m)$ is the morpism $G_S\to M_S$ such that for any $S'\to S$ and $g\in G(S')$,
\[\eta_M(m)(g)=g\cdot m_{S'}\in M(S').\]
Note that $\eta_M$ is a monomorphism: in fact, $\eps_M:\CoInd(M)\to M$ is a morphism of $A$-modules such that $\eps_M\circ\eta_M=\id_M$. Therefore, $M$ is a direct factor of the $A$-module $\CoInd(M)$.
\end{definition}

\begin{proposition}\label{category presheaf group module cohomology of coinduction zero}
For any $P\in\Mod(A)$, we have
\[H^n(G,\sHom(G,P))=0,\quad \mathcal{H}^n(G,\sHom(G,P))=0\for n>0.\]
Therefore, the functors $H^n(G,-)$ and $\mathcal{H}^n(G,-)$ are effacable for $n>0$.
\end{proposition}
\begin{proof}
It suffices to prove that $\mathcal{C}^\bullet(G,\sHom(G,P))$ and $C^\bullet(G,\sHom(G,P))$ are null-homotopic at positive degrees. To this end, we only need to consider the second one, since the corresponding result can be derived via base changes. Now, we define for $n\geq 0$ a morphism
\[\sigma:C^{n+1}(G,\sHom(G,P))\to C^n(G,\sHom(G,P)).\]
Let $f\in C^{n+1}(G,\sHom(G,P))$; for any $S\in\Ob(\mathcal{C})$ and $g_1,\dots,g_n\in G(S)$, $\sigma(f)(g_1,\dots,g_n)$ is the element of $\Hom_S(G_S,P_S)$ such that for any $S'\to S$ and $x\in G(S')$, 
\[\sigma(f)(g_1,\dots,g_n)(x)=f(x,g_1,\dots,g_n)(1)\in P(S'),\]
where $1$ denotes the unit element of $G(S')$. Then $\sigma$ is a null homotopy at positive degrees. In fact, for any $g_1,\dots,g_{n+1}\in G(S)$ and $x\in G(S')$, we have, on the one hand,
\begin{align*}
d\sigma(f)(g_1,\dots,g_{n+1})(x)&=f(xg_1,g_2,\dots,g_{n+1})(1)+\sum_{i=1}^{n}(-1)^if(x,g_1,\dots,g_ig_{i+1},\dots,g_{n+1})(1)\\
&+(-1)^{n+1}f(x,g_1,\dots,g_n)(1),
\end{align*}
and on the other hand,
\begin{align*}
\sigma(df)(g_1,\dots,g_{n+1})(x)&=(xf(g_1,\dots,g_{n+1}))(1)-f(xg_1,g_2,\dots,g_{n+1})(1)\\
&+\sum_{i=1}^{n}(-1)^{i+1}f(x,g_1,\dots,g_ig_{i+1},g_{n+1})+(-1)^{n+2}f(x,g_1,\dots,g_n)(1),
\end{align*}
whence
\[(d\sigma(f)+\sigma(df))(g_1,\dots,g_{n+1})(x)=(xf(g_1,\dots,g_{n+1}))(1)=f(g_1,\dots,g_{n+1})(x),\]
i.e. $d\sigma+\sigma d$ is the identity map on $C^{n+1}(G,\sHom(G,P))$, for any $n\geq 0$.
\end{proof}

\begin{proposition}\label{category presheaf group module cohomology is derived}
Suppose that $\mathcal{C}$ is $\mathscr{U}$-small, finite products exist in $\mathcal{C}$, and that $G$ is representable. Then the functors $H^n(G,-)$ (resp. $\mathcal{H}^n(G,-)$) are the derived functors of $H^0(G,-)$ (resp. $\mathcal{H}^n(G,-)$) over the category of $A[G]$-modules.
\end{proposition}
\begin{proof}
In view of (\cite{tohoku} 2.2.1 and 2.3), it suffices to show that the $H^n(G)$ (resp. $\mathcal{H}^n(G,-)$) form a cohomological functors, since they are effacable for $n>0$ in view of \cref{category presheaf group module cohomology of coinduction zero}. Let 
\[\begin{tikzcd}
0\ar[r]&M'\ar[r]&M\ar[r]&M''\ar[r]&0
\end{tikzcd}\]
be an exact sequence of $A[G]$-modules, and let $S\in\Ob(\mathcal{C})$. By hypothesis, $G$ is represented by an object $G\in\Ob(\mathcal{C})$, and finite products exist in $\mathcal{C}$. In particular, $\mathcal{C}$ possesses a final object $e$. For each $n\geq 0$, the product $G^n\times h_S$ is then represented by $G^n\times S$ (where $G^0=e$), and the sequence
\[\begin{tikzcd}
0\ar[r]&M'(G^n\times S)\ar[r]&M(G^n\times S)\ar[r]&M''(G^n\times S)\ar[r]&0
\end{tikzcd}\]
is exact. Therefore, the sequence of $A$-modules
\[\begin{tikzcd}
0\ar[r]&\mathcal{C}^n(h_G,M')\ar[r]&\mathcal{C}^n(h_G,M)\ar[r]&\mathcal{C}^n(h_G,M'')\ar[r]&0
\end{tikzcd}\]
is exact, which means $\mathcal{C}^\bullet(G,-)$, considered as a functor from $\Mod(A[G])$ to the category of complexes of $\Mod(A)$, is exact. It then follows from the induced long exact sequence that $\mathcal{H}^n(G,-)$ form a cohomological functor. As the functor $\mathbf{W}$ is exact, the same holds for the functors $H^n(G,-)$.
\end{proof}

\paragraph{Cohomology of \texorpdfstring{$\mathscr{O}_S[G]$}{O}-modules}\label{scheme group cohomology of qcoh G-module}
Let $S$ be a scheme, $G$ be an $S$-group and $\mathscr{F}$ be a quasi-coherent $\mathscr{O}_S[G]$-module. We define the cohomology groups of $G$ with values in $\mathscr{F}$ by
\[H^n(G,\mathscr{F})=H^n(h_G,\mathbf{W}(\mathscr{F})).\]
Suppose that $G$ is affine over $S$, then by \cref{scheme Gamma module functor of tensor with algbera char}, this cohomology can be calculated in the following way: $H^n(G,\mathscr{F})$ is the $n$-th cohomology group of the complex $C^\bullet(G,\mathscr{F})$ whose $n$-th term is 
\[C^n(G,\mathscr{F})=\Gamma(S,\mathscr{F}\otimes\underbrace{\mathscr{A}(G)\otimes\cdots\otimes\mathscr{A}(G)}_{\text{$n$-fold}}).\]
If $f$ (resp. $a_i$) is a section of $\mathscr{F}$ (resp. $\mathscr{A}(G)$) over an open subset of $S$, we then have
\begin{align*}
d(f\otimes a_1\otimes\cdots\otimes a_n)&=\mu_\mathscr{F}(f)\otimes a_1\otimes\cdots\otimes a_n+\sum_{i=1}^{n}(-1)^if\otimes a_1\cdots\otimes \Delta a_i\otimes\cdots\otimes a_n\\
&+(-1)^{n+1}f\otimes a_1\otimes\cdots\otimes a_n\otimes 1
\end{align*}
where $\Delta:\mathscr{A}(G)\to\mathscr{A}(G)\otimes\mathscr{A}(G)$ and $\mu_\mathscr{F}:\mathscr{F}\to\mathscr{F}\otimes\mathscr{A}(G)$ are induced from the cogebrea structure of $\mathscr{A}(G)$ and the comodule structure on $\mathscr{F}$. Note in passing that the cohomology of $G$ with values in $\mathscr{F}$ therefore depends only on the comodule structure of $\mathscr{F}$ and the monoid structure of $G$. In particular, we obtain a functor
\[H^0(G,\mathscr{F})=\Gamma(S,\mathscr{F}^G)\]
where $\mathscr{F}^G$ is the invariant sheaf of $\mathscr{F}$ defined in \cref{scheme module over group invariant subshaef def}.

\begin{theorem}\label{scheme group module over affine flat cohomology is derived}
Let $S$ be an affine scheme and $G$ be an affine and flat group over $S$. Then the functors $H^n(G,-)$ are the derived functors of $H^0(G,-)$ over the category of quasi-coherent $\mathscr{O}_S[G]$-modules. 
\end{theorem}

If $G$ is affine and flat over $S$, then by \cref{scheme module over flat affine group cat is abelian}, the category $\Qcoh(\mathscr{O}_S[G])$ is equivalent to the category $\CoQcoh(\mathscr{A}(G))$ of quasi-coherent $\mathscr{A}(G)$-comodules over $\mathscr{O}_S$ and is abelian. On the other hand, $\mathscr{A}(G)$ being a flat $\mathscr{O}_S$-module, the functor $\mathscr{F}\mapsto\mathscr{F}\otimes_{\mathscr{O}_S}\mathscr{A}(G)^{\otimes n}$ is exact; as $S$ is also affine, we conclude that $C^\bullet(G,-)$ is an exact functor over $\Qcoh(\mathscr{O}_S[G])$.\par
We denote by $\Delta$ (resp. $\eta$) the coultiplication (resp. counit) of $\mathscr{A}(G)$. For any quasi-coherent $\mathscr{O}_S$-module $\mathscr{P}$, we denote by $\Ind(\mathscr{P})=\mathscr{P}\otimes_{\mathscr{O}_S}\mathscr{A}(G)$ endowed with the $\mathscr{A}(G)$-comodule structure defined by
\[\id_\mathscr{P}\otimes\Delta:\mathscr{P}\otimes_{\mathscr{O}_S}\mathscr{A}(G)\to\mathscr{P}\otimes_{\mathscr{O}_S}\mathscr{A}(G)\otimes_{\mathscr{O}_S}\mathscr{A}(G);\]
this defines a functor $\Ind:\Qcoh(\mathscr{O}_S)\to\Qcoh(\mathscr{O}_S[G])$. It follows from \cref{scheme Gamma module functor of tensor with algbera char} that we have an isomorphism of $\mathbb{O}_S[G]$-modules
\begin{equation}\label{scheme module over group Ind and CoInd relation}
\mathbf{W}(\Ind(\mathscr{P}))\cong \CoInd(\mathbf{W}(\mathscr{P}))=\sHom(G,\mathbf{W}(\mathscr{P})).
\end{equation}
Via this identification, the morphism $\eps:\CoInd(\mathbf{W}(\mathscr{P}))\to\mathbf{W}(\mathscr{P})$ then corresponds to the morphism $\id_\mathscr{P}\otimes \eta:\Ind(\mathscr{P})\to\mathscr{P}$ of $\mathscr{O}_S$-modules, where we use \cref{scheme Gamma module functor prop}. From \cref{category presheaf group module forgetful CoInd adjoint}, we then conclude the following corolalry:
\begin{corollary}\label{scheme module over group forgetful Ind adjoint}
Let $S$ be a scheme and $G$ be an affine group over $S$. Then the functor $\Ind$ is right adjoint to the forgetful functor $\iota:\Qcoh(\mathscr{O}_S[G])\to\Qcoh(\mathscr{O}_S)$. More precisely, the map $\id_\mathscr{P}\otimes\eta:\Ind(\mathscr{P})\to\mathscr{P}$ induces for any object $\mathscr{M}$ of $\Qcoh(\mathscr{O}_S[G])$ a bijection
\[\Hom_{\mathscr{O}_S[G]}(\mathscr{M},\Ind(\mathscr{P}))\stackrel{\sim}{\to}\Hom_{\mathscr{O}_S}(\mathscr{M},\mathscr{P}).\]
Therefore, if $\mathscr{I}$ is an injective object in $\Qcoh(\mathscr{O}_S)$, then $\Ind(\mathscr{I})$ is an injective object in $\Qcoh(\mathscr{O}_S)$.
\end{corollary}

Let $\mathscr{F}$ be an $\mathscr{O}_S[G]$-module and $\mu_\mathscr{F}:\mathscr{F}\to\Ind(\mathscr{F})$ be the map defining the $\mathscr{A}(G)$-comodule structure. It follows from the axioms (CM1) and (CM2) that $\mu_\mathscr{F}$ is a morphism of $\mathscr{O}_S[G]$-modules, and that $(\id_\mathscr{F}\otimes\eta)\circ\mu_\mathscr{F}=\id_\mathscr{F}$, so that $\mathscr{F}$ is a direct factor of $\Ind(\mathscr{F})$ considered as $\mathscr{O}_S$-modules. In particular, $\mu_\mathscr{F}$ is a monomorphism. As we have, by (\ref{scheme module over group Ind and CoInd relation}) and \cref{category presheaf group module cohomology of coinduction zero},
\[H^n(G,\mathbf{W}(\Ind(\mathscr{F})))\cong H^n(G,\sHom_S(G,\mathbf{W}(\mathscr{F})))=0\for n>0\]
we conclude that $H^n(G,-)$ is effacable for $n>0$.\par
Finally, as $S$ is affine, $\Qcoh(\mathscr{O}_S)$ possesses enough injectives. Let $\mathscr{F}\rightarrowtail\mathscr{I}$ be a monomorphism of $\mathscr{O}_S$-modules where $\mathscr{I}$ is injective object of $\Qcoh(\mathscr{O}_S)$; then, $\mathscr{A}(G)$ being flat over $\mathscr{O}_S$, $\Ind(\mathscr{F})$ is a sub-$\mathscr{O}_S[G]$-module of $\Ind(\mathscr{I})$, so we conclude that
\begin{corollary}\label{scheme group module over affine Qcoh enough injective}
Under the hypothesis of \cref{scheme group module over affine flat cohomology is derived}, the abelian category $\Qcoh(\mathscr{O}_S[G])$ possesses enough injectives.
\end{corollary}

In view of (\cite{tohoku} 2.2.1 and 2.3), the proof of \cref{scheme group module over affine flat cohomology is derived} is completed.

\begin{remark}
We can also prove \cref{scheme module over group forgetful Ind adjoint} by the following calculation. To any morphism of $\mathscr{O}_S[G]$-modules $\phi:\mathscr{M}\to\mathscr{P}\otimes_{\mathscr{O}_S}\mathscr{A}(G)$, we associate the $\mathscr{O}_S$-morphism $(\id_\mathscr{P}\otimes\eta)\circ\phi:\mathscr{M}\to\mathscr{P}$. Conversely, to any $\mathscr{O}_S$-morphism $f:\mathscr{M}\to\mathscr{P}$ we associate the $\mathscr{O}_S[G]$-morphism $(f\otimes\id_{\mathscr{A}(G)})\circ\mu_\mathscr{M}:\mathscr{M}\to\Ind(\mathscr{P})$. On the one hand, from axiom (CM2) we see that
\[(\id_\mathscr{P}\otimes\eta)\circ(f\circ\id_{\mathscr{A}(G)})\circ\mu_\mathscr{M}=(f\circ\id_{\mathscr{O}_S})\circ(\id_\mathscr{P}\otimes\eta)\circ\mu_\mathscr{M}=f.\]
On the other hand, for any $\phi$ the following diagram is commutative:
\[\begin{tikzcd}
\mathscr{M}\ar[r,"\phi"]\ar[d,swap,"\mu_\mathscr{M}"]&\mathscr{P}\otimes_{\mathscr{O}_S}\mathscr{A}(G)\ar[d,"\id_\mathscr{P}\otimes\Delta"]\\
\mathscr{M}\otimes_{\mathscr{O}_S}\mathscr{A}(G)\ar[r,"\phi\otimes\id_{\mathscr{A}(G)}"]&\mathscr{P}\otimes_{\mathscr{O}_S}\mathscr{A}(G)\otimes_{\mathscr{O}_S}\mathscr{A}(G)
\end{tikzcd}\]
so it follows that
\begin{align*}
\big(((\id_\mathscr{P}\otimes\eta)\circ\phi)\otimes\id_{\mathscr{A}(G)}\big)\circ\mu_\mathscr{M}&=(\id_\mathscr{P}\otimes\eta\otimes\id_{\mathscr{A}(G)})\circ(\phi\otimes\id_{\mathscr{A}(G)})\circ\mu_\mathscr{M}\\
&=(\id_\mathscr{P}\otimes\eta\otimes\id_{\mathscr{A}(G)})\circ(\id_\mathscr{P}\otimes\Delta)\circ\phi=\phi.
\end{align*}
This proves the first claim of \cref{scheme module over group forgetful Ind adjoint}, and the second one then follows.
\end{remark}

Let $\mathscr{F}$ be an $\mathscr{O}_S[G]$-module. We have seen that the axiom (CM2) shows that considered as $\mathscr{O}_S$-modules, $\mathscr{F}$ is a direct factor of $\CoInd(\mathscr{F})$. This implies the following proposition:

\begin{proposition}\label{scheme module over flat group cohomology zero if}
Let $S$ be an affine scheme and $G$ be an affine and flat group scheme over $S$. Suppose that for any exact sequence
\[\begin{tikzcd}
0\ar[r]&\mathscr{F}_1\ar[r]&\mathscr{F}_2\ar[r]&\mathscr{F}_3\ar[r]&0
\end{tikzcd}\]
of quasi-coherent $\mathscr{O}_S[G]$-modules, which splits as a sequence of $\mathscr{O}_S$-modules, also split as $\mathscr{O}_S[G]$-modules. Then the functors $H^n(G,-)$ are zero for $n>0$.
\end{proposition}
\begin{proof}
In fact, by the hypothesis, the sequence of $\mathscr{O}_S[G]$-modules
\[\begin{tikzcd}
0\ar[r]&\mathscr{F}\ar[r]&\CoInd(\mathscr{F})\ar[r]&\CoInd(\mathscr{F})/\mathscr{F}\ar[r]&0
\end{tikzcd}\]
is splitting, so $\mathscr{F}$ is a direct factor of $\CoInd(\mathscr{F})$ as an $\mathscr{O}_S[G]$-module. Since $\CoInd(\mathscr{F})$ has trivial higher cohomology, so does $\mathscr{F}$.
\end{proof}

\begin{theorem}\label{scheme module over diagonalizable group cohomology zero}
Let $S$ be an affine scheme and $G$ be a diagonalizable $S$-group. Then for any quasi-coherent $\mathscr{O}_S[G]$-module $\mathscr{F}$, we have $H^n(G,\mathscr{F})=0$ for $n>0$.
\end{theorem}
\begin{proof}
This follows from \cref{scheme module over flat group cohomology zero if} and \cref{scheme module over diagonalizable group sequence split iff}.
\end{proof}

\subsection{\texorpdfstring{$G$}{G}-equivariant objects and modules}
Let $\mathcal{C}$ be a category with a final object $e$ and such that fiber products exist in $\mathcal{C}$. Let $G$ be a group in $\widehat{\mathcal{C}}$, $\pi:M\to X$ be a morphism in $\widehat{\mathcal{C}}$, and $\lambda=\lambda_X:G\times X\to X$ be an action of $G$ on $X$. In this paragraph, we denote by $Y\times_fM$ the fiber product of $\pi:M\to X$ and an $X$-functor $f:Y\to X$.\par
For any $U\in\Ob(\mathcal{C})$ and $x\in X(U)$, the \textbf{fiber} of $M$ at $x$ is defined by $M_x=U\times_xM$, i.e. for any $\phi:U'\to U$, we have
\[M_x(U')=\{m\in M(U'):\pi(m)=x_{U'}=\phi^*(x)\}.\]
Finally, if $g\in G(U)$, we denote by $g(x)$ the element $\lambda(g,x)$ in $X(U)$.

\begin{definition}
We say that $M$ is a \textbf{$\bm{G}$-equivariant object over $\bm{X}$}, or a \textbf{$\bm{G}$-equivariant $\bm{X}$-object}, if we are given an action $\Lambda:G\times M\to M$ of $G$ on $M$ compatible with $\lambda$, i.e. such that the following diagram is commutative:
\[\begin{tikzcd}
G\times M\ar[r,"\Lambda"]\ar[d,"\id_G\times\pi"]&M\ar[d,"\pi"]\\
G\times X\ar[r,"\lambda"]&X
\end{tikzcd}\]
This is equivalent to saying that we are given, for any morphism $(g,x):U\to G\times X$, morphisms
\[\Lambda_x^U(g):M_x(U)\to M_{g(x)}(U),\quad m\mapsto g\cdot m\]
satisfying $1\cdot m=m$ and $g\cdot(h\cdot m)=(gh)\cdot m$ and functorial on the $(G\times X)$-object $U$. Alternatively, this means we are given morphisms of $U$-objects
\[\Lambda_x(g):M_x\to M_{g(x)}\]
such that $\Lambda_x(1)=\id$ and $\Lambda_{h(x)}(g)\circ\Lambda_x(h)=\Lambda_x(gh)$.\par
Now let $A$ be a ring in $\widehat{\mathcal{C}}$ and $A_X=A\times X$. Under the condition described above, we say that $M$ is a \textbf{$\bm{G}$-equivariant $\bm{A_X}$-module} if it is an $A_X$-module and the action $\Lambda$ is compatible with the $A_X$-module structure on $M$, that is, if for any morphism $(g,x):U\to G\times X$, the map $\Lambda_x(g):M_x\to M_{g(x)}$ is a morphism of $A_U$-modules.
\end{definition}

\begin{remark}
In the above definition for $G$-equivariant objects, the conditions $\Lambda_x(1)=\id$ and $\Lambda_{h(x)}(g)\circ\Lambda_x(h)=\Lambda_x(gh)$ implies that $\Lambda_x(g)$ is an isomorphism, with inverse $\Lambda_{g(x)}(g^{-1})$. Conversely, if we suppose that each $\Lambda_x(g)$ is an isomorphism, the condition $\Lambda_{h(x)}(g)\circ\Lambda_x(h)=\Lambda_{x}(gh)$, applied to $h=1$, then implies that $\Lambda_x(1)=\id$.
\end{remark}

\begin{remark}\label{category of presheaf G-equivariant object iff isomorphism on product}
If $M$ is an $A_X$-module, then in view of the universal property of fiber products, giving a morphism $\Lambda:G\times M\to M$ which is compatible with $\lambda$ is equivalent to giving a homomorphism of $A_{G\times X}$-modules
\[\theta:G\times M=(G\times X)\times_{\pr_X}M \to (G\times X)\times_\lambda M,\quad (g,x,m)\mapsto(g,g(x),g\cdot m),\]
and the morphisms $\Lambda_x(g):M_x\to M_{g(x)},m\mapsto g\cdot m$ are isomorphisms of $A_U$-modules if and only if $\theta$ is an isomorphism. As we have supposed that each $\Lambda_x(h)$ is an isomorphism, the equality $\Lambda_x(1)=\id$ follows from the equality $\Lambda_{h(x)}(g)\circ\Lambda_x(h)=\Lambda_{x}(gh)$. Therefore, $\Lambda$ is an action of $G$ over $M$ if and only the following diagram of $(G\times G\times X)$-isomorphisms is commutative (where we denote by $m$ the multiplication of $G$ and $f^*(\theta)$ is the isomorphism induced from $\theta$ under a base change $f:G\times G\times X\to G\times X$)
\[\begin{tikzcd}
(G\times G\times X)\times_{\pr_X\circ\pr_{23}}M\ar[r,"\pr^*_{23}(\theta)","\sim"']\ar[d,equal]&(G\times G\times X)\times_{\lambda\circ\pr_{23}}M\ar[d,equal]\\
(G\times G\times X)\times_{\pr_X\circ(m\times\id_X)}M\ar[d,swap,"(m\times\id_X)^*(\theta)","\sim"']&(G\times G\times X)\times_{\pr_X\circ(\id_G\times\lambda)}M\ar[d,"(\id_G\times\lambda)^*(\theta)","\sim"']\\
(G\times G\times X)\times_{\lambda\times(m\times\id_X)}M\ar[r,equal]&(G\times G\times X)\times_{\lambda\circ(\id_G\times\lambda)}M
\end{tikzcd}\]
\end{remark}

\begin{remark}
The above definitions extend to the case where $G$ is only a monoid. In this case, giving an action $\Lambda:G\times M\to M$ that is compatible with $\lambda$ and such that each $\Lambda_x(g):M_x\to M_{g(x)}$ is a morphism of $A_U$-modules is equivalent to giving a morphism 
\[\theta:G\times M=(G\times X)\times_{\pr_X}M \to (G\times X)\times_\lambda M,\quad (g,x,m)\mapsto(g,g(x),g\cdot m),\]
such as the diagram in \cref{category of presheaf G-equivariant object iff isomorphism on product} (without the signs $\sim$ under the arrows) is commutative, and such that $\pr_M\circ\theta\circ(\eps_G\times\id_M)=\id_M$, where $\eps_G$ denotes the unit section of $G$ and $\pr_M$ the projection on $M$ (this is added since in this case the equality $\Lambda_x(1)=\id$ can not be derived).
\end{remark}

Let $Y$ be another object of $\widehat{\mathcal{C}}$ which is endowed with an action $\lambda_Y:G\times Y\to Y$ by $G$ and $N$ be a $G$-equivariant $A_X$-module. A morphism $f:Y\to X$ in $\widehat{\mathcal{C}}$ (resp. a homomorphism of $A_X$-modules $\phi:M\to X$) is called $G$-equivariant if it commutes with the action of $G$, i.e. if we have $f(g\cdot y)=g\cdot f(y)$ (resp. $\phi(g\cdot m)=g\cdot\phi(m)$), which is equivalent to $f\circ\lambda_Y=\lambda_X\circ\id_G\times f$ (resp. $\phi\circ\Lambda_M=\Lambda_N\circ(\id_G\times\phi)$). We then obtain the following lemma:

\begin{lemma}\label{category of presheaf G-equivariant pullback}
Let $f:Y\to X$ be a $G$-equivariant morphism and $M$ be a $G$-equivariant $A$-module. Then the inverse image $f^*(M)=Y\times_fM$ is a $G$-equivariant $A_Y$-module.
\end{lemma}
\begin{proof}

\end{proof}

\section{Tangent spaces and Lie algebras}
In this section, we construct the tangent spaces and Lie algebras in scheme theory. It will be useful not to restrict oneself to the diagrams themselves, but to also be intersted to certain functors on the category of schemes which are not necessarily representable. The exposition we give here easily generalize beyond the theory of schemes. For example, it is valid for the theory of complex analytic spaces, with suitable modifications.
\subsection{The tangent bundle and tangent space}
\paragraph{The functor \texorpdfstring{$\sHom_{Z/S}(X,Y)$}{Hom}}\label{scheme tangent bundle functor sHom_Z/S(X,Y) paragraph}
Let $\mathcal{C}$ be a category and $S$ be an object of $\mathcal{C}$. We consider objects $X,Y,Z$ in $\widehat{\mathcal{C}}$ with $X,Y$ lying over $Z$ and $Z$ lying over $S$:
\[\begin{tikzcd}[row sep=4mm, column sep=4mm]
X\ar[rd,swap,"p_X"]&&Y\ar[ld,"p_Y"]\\
&Z\ar[d]&\\
&S
\end{tikzcd}\]

\begin{definition}
We define an object $\sHom_{Z/S}(X,Y)$ in $\widehat{\mathcal{C}_{/S}}$ by the formula
\[\sHom_{Z/S}(X,Y)(S')=\Hom_{Z_{S'}}(X_{S'},Y_{S'})=\Hom_Z(X\times_SS',Y),\]
where $S'$ is an object of $\mathcal{C}_{/S}$. We see that $\sHom_{Z/S}(X,Y)$ is none other than the sub-object of $\sHom_S(X,Y)$ formed by morphisms compatible with $p_X$ and $p_Y$, that is, it is the kernel of the morphisms
\[\begin{tikzcd}
\sHom_S(X,Y)\ar[r,shift left=2pt]\ar[r,shift right=2pt]&\sHom_S(X,Z)
\end{tikzcd}\]
where the first map is defined by composing with $p_Y$ and the seond one is the constant map of $p_X$.
\end{definition}

On the other hand, we see as in (\ref{category presheaf Hom functor adjoint prop-1}) that, for any object $T$ of $\widehat{\mathcal{C}}$ over $S$, we have a natural bijection
\[\Hom_S(T,\sHom_{Z/S}(X,Y))\cong \Hom_Z(X\times_ST,Y).\]
Moreover, by (\ref{category presheaf Hom functor adjoint prop-1}), if $E,F$ are objects of $\widehat{\mathcal{C}}$ lying over $Z$, then
\[\Hom_Z(E,\sHom_Z(F,Y))\cong\Hom_Z(E\times_ZF,Y)\cong\Hom_Z(F,\sHom_Z(E,Y)).\]
Apply this to $E=X$ and $F=Z\times_ST$, we then obtain the following bijections for any object $T$ of $\widehat{\mathcal{C}_{/S}}$:
\begin{equation}\label{category of presheaf functor Hom_Z/S(X,Y) isomorphism-1}
\Hom_S(T,\sHom_{Z/S}(X,Y))\cong\Hom_Z(X\times_ST,Y)\cong\begin{cases}
\Hom_Z(Z\times_ST,\sHom_Z(X,Y)),\\
\Hom_Z(X,\sHom_Z(Z\times_ST,Y)).
\end{cases}
\end{equation}
Since these bijections are functorial over $T$, we then obtain isomorphisms of $S$-functors
\begin{equation}\label{category of presheaf functor Hom_Z/S(X,Y) isomorphism-2}
\begin{tikzcd}[row sep=5mm, column sep=2mm]
\sHom_S(T,\sHom_{Z/S}(X,Y))\ar[rd,"\sim"]\ar[rr,"\sim"]&&\sHom_{Z/S}(X,\sHom_Z(Z\times_ST,Y))\\
&\sHom_{Z/S}(X\times_ST,Y)\ar[ru,"\sim"]&
\end{tikzcd}
\end{equation}

\paragraph{The restriction functor $\Res_{Z/S}Y$} We now consider the special case where $X=Z$: we put
\[\Res_{Z/S}Y=\sHom_{Z/S}(Z,Y).\]
By definition, we then have
\[\Res_{Z/S}(Y)(S')=\Hom_Z(Z\times_SS',Y)=\Gamma(Y_{S'}/Z_{S'}).\]
The functor $\Res_{Z/S}:\widehat{\mathcal{C}_{/Z}}\to\widehat{\mathcal{C}_{/S}}$ is a right adjoint of the base change functor from $S$ to $Z$. In fact, for any $S$-functor $U$, by (\ref{category of presheaf functor Hom_Z/S(X,Y) isomorphism-1}) we have
\[\Hom_S(U,\Res_{Z/S}Y)=\Hom_S(U,\sHom_{Z/S}(Z,Y))\cong\Hom_Z(U\times_SZ,Y).\]
(If $\mathcal{C}=\Sch$ and $Z$ is an $S$-scheme, the functor $\Res_{Z/S}$ is called the \textbf{Weil restriction}.) We also ntoe that since for any $S'\in\Ob(\mathcal{C}_{/S})$ we have 
\begin{align*}
\sHom_{Z/S}(X,Y)(S')&=\Hom_{Z}(X_{S'},Y)\cong\Hom_X(X_{S'},Y\times_ZX)=\sHom_{X/S}(X,Y\times_ZX),
\end{align*}
so we obtain an isomorphism
\[\sHom_{Z/S}(X,Y)\cong\sHom_{X/S}(X,Y\times_ZX)=\Res_{X/S}(Y\times_ZX),\]
which for $Z=S$ gives an isomorphism 
\[\sHom_S(X,Y)\cong \Res_{X/S}Y_X.\]

\begin{remark}\label{category of presheaf functor Hom product commutes}
The functore $Y\mapsto\sHom_{Z/S}(X,Y)$ commutes with products in the sense that we have a functorial isomorphism
\begin{equation}\label{category of presheaf functor Hom product commutes-1}
\sHom_{Z/S}(X,Y\times_ZY')\cong\sHom_{Z/S}(X,Y)\times_S\sHom_{Z/S}(X,Y')
\end{equation}
It follows that if $Y$ is a $Z$-group (resp. $Z$-ring, etc.), then $\sHom_{Z/S}(X,Y)$ is an $S$-group (resp. $S$-ring, etc.).
\end{remark}

\begin{remark}\label{category of presheaf functor Hom module structure}
Let $\pi:M\to Y$ be a $Y$-functor in $\mathbb{O}_Y$-modules. Put $H=\sHom_{Z/S}(X,Y)$, then the functor $\sHom_{Z/S}(X,M)$ is endowed with a natural $\mathbb{O}_H$-module structur. In fact, denote by $m:M\times_YM\to M$ and $\lambda:\mathbb{O}_Y\times_YM\to M$ the defining morphisms of abelian group structure and module structure of $M$. Let $H'$ be an $S$-scheme over $H$, that is, we are given a $Z$-morphism $f:X\times_SH'\to Y$, which makes $X\times_SH'$ a $Y$-object. Then $\Hom_H(H',\sHom_{Z/S}(X,M))$ is the set of $Z$-morphisms $\phi:X\times_SH'\to M$ such that $\pi\circ\phi=f$, that is, the $Y$-morphisms $X\times_SH'\to M$.\par
Let $\phi,\psi$ be two such morphisms, we define $\phi+\psi$ as the composition of $Y$-morphisms
\[\begin{tikzcd}
X\times_SH'\ar[r,"\phi\times\psi"]&M\times_YM\ar[r,"m"]&M
\end{tikzcd}\]
and this endows $\sHom_{Z/S}(X,M)$ an abelian group structure over $H=\sHom_{Z/S}(X,Y)$.\par
Similarly, if $a$ is an element of $\mathbb{O}(X\times_SH')$, i.e. an $S$-morphism $a:X\times_SH'\to\mathbb{O}_S$, we define $a\phi$ as the composition $\lambda\circ(a\times\phi)$, where $a\times\phi$ denotes the $Y$-morphism from $X\times_SH$ to $\mathbb{O}_Y\times_YM\cong\mathbb{O}_S\times_SM$ with components $a$ and $\phi$. We verify that this endows $\Hom_H(H',\sHom_{Z/S}(X,M))$ with an $\mathbb{O}(X\times_SH')$-module structure, which is functorial on $H'$.
\end{remark}

\paragraph{The scheme \texorpdfstring{$I_S(\mathscr{M})$}{I}}\label{scheme tangent bundle I_S paragraph}
\begin{definition}
Let $S$ be a scheme and $\mathscr{M}$ be a quasi-coherent $\mathscr{O}_S$-module. We denote by $\mathscr{D}_{\mathscr{O}_S}(\mathscr{M})$ the quasi-coherent algebra $\mathscr{O}_S\oplus\mathscr{M}$ (where $\mathscr{M}$ is considered as a square zero ideal). We denote by $I_S(\mathscr{M})$ the $S$-scheme $\Spec(\mathscr{D}_{\mathscr{O}_S}(\mathscr{M}))$. In particular, we have $\mathscr{D}_{\mathscr{O}_S}=\mathscr{D}_{\mathscr{O}_S}(\mathscr{O}_S)$, $I_S=I_S(\mathscr{O}_S)$, which are called the \textbf{algebra of dual numbers over $\bm{S}$} and the \textbf{dual number scheme over $\bm{S}$}.
\end{definition}

We then obtain a contravariant functor $\mathscr{M}\mapsto I_S(\mathscr{M})$ from the category of quasi-coherent $\mathscr{O}_S$-modules to the category of $S$-schemes. In particular, the morphisms $0\to\mathscr{M}$ and $\mathscr{M}\to 0$ define respectively the structural morphism $\rho:I_S(\mathscr{M})\to I_S(0)=S$ and a section $\eps_\mathscr{M}:S\to I_S(\mathscr{M})$, which is called the \textbf{zero section} of $I_S(\mathscr{M})$.\par

As $\mathscr{M}\mapsto I_S(\mathscr{M})$ is a contravariant functor, for any endomorphism $a\in\End_{\mathscr{O}_S}(\mathscr{M})$, we have an $S$-endomorphism $a^*$ of $I_S(\mathscr{M})$, and
\[1^*=\id,\quad (ab)^*=b^*\circ a^*,\quad 0^*=\eps_\mathscr{M}\circ\rho,\quad a^*\circ\eps_\mathscr{M}=\eps_\mathscr{M}.\]
Therefore, the $S$-scheme $I_S(\mathscr{M})$ is endowed with a right action of the (multiplicative) monoid $\End_{\mathscr{O}_S}(\mathscr{M})$, which commutes with $S$-morphisms $I_S(\mathscr{M})\to I_S(\mathscr{M}')$ induced by homomorphisms $\mathscr{M}\to\mathscr{M}'$. In particular, the operations $a^*$ preserves the zero section of $I_S(\mathscr{M})$.\par
For any endomorphism $a\in\End_{\mathscr{O}_S}(\mathscr{M})$, $f:S'\to S$ and $m\in I_S(\mathscr{M})(S')$, we write $m\cdot a=a^*(m)$. Then we have
\[m\cdot 1=m,\quad (m\cdot a)\cdot b=m\cdot(ab),\quad m\cdot 0=\eps_\mathscr{M}(\rho(m)).\]
Moreover, if $m=\eps_\mathscr{M}(f)$, then $m\cdot a=m$.

\begin{remark}
The formation of $I_S(\mathscr{M})$ commutes with base changes: we have a canonical isomorphism
\[I_S(\mathscr{M})_{S'}\cong I_{S'}(\mathscr{M}\otimes_{\mathscr{O}_S}\mathscr{O}_{S'}).\]
For simplicity, we shall write $I_{S'}(\mathscr{M})$ for $I_S(\mathscr{M})_{S'}$. More generally, if $X$ is an $S$-functor (not necessarily representable), then we define $I_X(\mathscr{M}):=I_S(\mathscr{M})\times_SX$.
\end{remark}

\begin{remark}\label{scheme tangent bundle I_S action of O(S)}
By consider the homotheties on $\mathscr{M}$, we see that the multiplicative monoid $\mathbb{O}(S')$ acts on the $S'$-scheme $I_{S'}(\mathscr{M})$, which is functorial on $\mathscr{M}$, i.e. the $S$-scheme $I_S(\mathscr{M})$ is endowed with a structure of an $\mathbb{O}_S$-object, which is functorial on $\mathscr{M}$. We then have a morphism of $S$-schemes
\[\lambda:I_S(\mathscr{M})\times_S\mathbb{O}_S\to I_S(\mathscr{M}),\]
which satisfies the evident conditions. For any $S$-functor $X$, we then obtain by base change a morphism of $X$-functors
\[\lambda_X:I_X(\mathscr{M})\times_S\mathbb{O}_S\to I_X(\mathscr{M})\]
which defines an action of monoid $\mathbb{O}(X)$ on the $S$-functor $I_X(\mathscr{M})$: any element $a$ of $\mathbb{O}(X)=\Hom_S(X,\mathbb{O}_S)$ defines an $X$-endomorphism $a^*$ of $I_X(\mathscr{M})$. More precisely, if $x\in X(S')$ and $m\in I_S(\mathscr{M})(S')=I_{S'}(\mathscr{M})(S')$, then $a(x)=a\circ x$ belongs to $\mathbb{O}(S')$ and we have
\[(m,x)\cdot a=(m\cdot a(x),x).\]
This operation is functorial on $\mathscr{M}$ and preserves the zero section $\eps_\mathscr{M}:X\to I_X(\mathscr{M})$, i.e. $a^*\circ\eps_\mathscr{M}=\eps_\mathscr{M}$ for any $a\in\mathbb{O}(X)$.\par
Even further, this operation is functorial on $X$ in the following sense: if $\pi:Y\to X$ is a morphism of $S$-functors and $u:\mathbb{O}(X)\to\mathbb{O}(Y)$ is the corresponding ring homomorphism (i.e. $u(a)=a\circ\pi$ for $a\in\mathbb{O}(X)$), then the following diagram is commutative
\[\begin{tikzcd}
I_Y(\mathscr{M})\ar[r,"u(a)^*"]\ar[d,swap,"\pi"]&I_Y(\mathscr{M})\ar[d,"\pi"]\\
I_X(\mathscr{M})\ar[r,"a^*"]&I_X(\mathscr{M})
\end{tikzcd}\]
\end{remark}

We now consider two quasi-coherent $\mathscr{O}_S$-modules $\mathscr{M}$ and $\mathscr{N}$. The commutative diagram of the direct sum then defines a commutative diagram of $S$-schemes
\begin{equation}\label{scheme dual number of direct sum Cartesian diagram-1}
\begin{tikzcd}[row sep=4mm,column sep=4mm]
&I_S(\mathscr{M}\oplus\mathscr{N})&\\
I_S(\mathscr{M})\ar[ru]&&I_S(\mathscr{N})\ar[lu]\\
&S\ar[ru,swap,"\eps_\mathscr{N}"]\ar[lu,"\eps_\mathscr{M}"]\ar[uu,swap,"\eps_{\mathscr{M}\oplus\mathscr{N}}"]&
\end{tikzcd}
\end{equation}

\begin{proposition}\label{scheme dual number of direct sum Cartesian diagram}
For any $S$-scheme $X$, the diagram of functors over $S$ obtained by applying the functor $\sHom_S(-,X)$ to (\ref{scheme dual number of direct sum Cartesian diagram-1}) is Cartesian:
\[\begin{tikzcd}
\sHom_S(I_S(\mathscr{M}\oplus\mathscr{N}),X)\ar[r]\ar[d]&\sHom_S(I_S(\mathscr{N}),X)\ar[d]\\
\sHom_S(I_S(\mathscr{M}),X)\ar[r]&\sHom_S(S,X)=X
\end{tikzcd}\]
\end{proposition}
\begin{proof}
It suffices to verify that for any $S'\to S$, the diagram obtained by applying the functors on $S'$ is Cartesian. As the formation of $I_S(-)$ commutes with base change, it then suffices to prove this for $S'=S$, hence to verify that the following diagram is Cartesian:
\[\begin{tikzcd}[row sep=12mm,column sep=8mm]
X(I_S(\mathscr{M}\oplus\mathscr{N}))\ar[r]\ar[d]\ar[rd,"X(\eps_{\mathscr{M}\oplus\mathscr{N}})",pos=0.4]&X(I_S(\mathscr{N}))\ar[d,"X(\eps_\mathscr{N})"]\\
X(I_S(\mathscr{M}))\ar[r,"X(\eps_\mathscr{M})"]&X(S)
\end{tikzcd}\]
Now if we consider $x\in X(S)$, it follows from (\cite{SGA1} \Rmnum{3}, 5.1) that the fiber $X(\eps_\mathscr{M})^{-1}(x)$ of $x$ is isomorphic to $\Hom_{\mathscr{O}_S}(x^*(\Omega_{X/S}^1),\mathscr{M})$. Since this latter functor clearly commutes with finite direct sums of $\mathscr{O}_S$-modules, our assertion follows.
\end{proof}

\begin{corollary}\label{scheme dual number isomorphic to product}
Let $X$ be an $S$-scheme and $\mathscr{M}$ be a free $\mathscr{O}_X$-module of finite type. Then $\sHom_S(I_S(\mathscr{M}),X)$ is isomorphic to a finite product of copies of $\sHom_S(I_S,X)$.
\end{corollary}

\begin{remark}\label{scheme dual number Hom represented by vector bundle of Omega}
It follows from the proof of \cref{scheme dual number of direct sum Cartesian diagram} that $\sHom_S(I_S,X)$ is isomorphic to the $X$-functor $\mathbf{V}(\Omega^1_{X/S})$, and hence represented by the vector bundle $\V(\Omega_{X/S}^1)$. 
\end{remark}

\paragraph{The tangent bundle and condition (E)}
\begin{definition}
Let $S$ be a scheme and $\mathscr{M}$ be a free $\mathscr{O}_S$-module of finite rank. Let $X$ be a functor over $S$. The \textbf{tangent bundle of $X$ over $S$ relative to the $\mathscr{O}_S$-module $\mathscr{M}$} is defined to be the $S$-functor
\[T_{X/S}(\mathscr{M})=\sHom_S(I_S(\mathscr{M}),X).\]
In particular, the \textbf{tangent bundle of $X$ over $S$} is the functor
\[T_{X/S}=T_{X/S}(\mathscr{O}_S)=\sHom_S(I_S,X).\]
\end{definition}

The construction $\mathscr{M}\mapsto T_{X/S}(\mathscr{M})$ is then a covariant functor from the category of free $\mathscr{O}_S$-modules of finite type to the category of $S$-functors. In particular, the morphisms $\mathscr{M}\to 0$ and $0\to\mathscr{M}$ define respectively an $S$-morphism $\pi_\mathscr{M}:T_{X/S}(\mathscr{M})\to T_{X/S}(0)\cong X$ and a section $\tau:X\to T_{X/S}(\mathscr{M})$, called the \textbf{zero section}. Moreover, it follows from the preceding remarks that $\mathbb{O}(S)$ is a monoid acting on the $X$-functor $T_{X/S}(\mathscr{M})$, which is functorial on $\mathscr{M}$.

\begin{remark}\label{scheme tangent bundle projection zero section char}
We note that the projection $\pi_\mathscr{M}:T_{X/S}(\mathscr{M})\to X$ is induced by the zero section $\eps_\mathscr{M}:S\to I_S(\mathscr{M})$, while the zero section $\tau:X\to T_{X/S}(\mathscr{M})$ is induced by the structural morphism $\rho:I_S(\mathscr{M})\to S$. For any point $t\in T_{X/S}(\mathscr{M})(S')$ (resp. $x\in X(S')$), which corresponds to an $S$-morphism $f:I_{S'}(\mathscr{M})\to X$ (resp. $g:S'\to X$), we have 
\[\pi(t)=f\circ(\id_{S'}\times\eps_\mathscr{M}),\quad  \text{(resp. $\tau(x)=g\circ(\id_{S'}\times\rho)$)}.\]
It follows from the above definition that $\mathscr{M}\mapsto T_{X/S}(\mathscr{M})$ is a covariant functor from the category of free $\mathscr{O}_X$-modules of finite rank to that of functors over $X$. In particular, $\mathbb{O}(S)$ is a monoid operating on the $X$-functor $T_{X/S}(\mathscr{M})$, which \textit{respects the functoriality of $\mathscr{M}$}.
\end{remark}

\begin{remark}\label{scheme tangent bundle Sigma action construction}
In particular, the above arguments motivates the following construction. For any $S$-morphism $X'\to X$, we put
\[\Sigma(X',\mathscr{M})=\Hom_X(X',T_{X/S}(\mathscr{M})).\]
We have an action of the multiplicative monoid $\End_{\mathscr{O}_S}(\mathscr{M})$ over $\Sigma(X',\mathscr{M})$, denoted by $(\lambda,x)\mapsto\lambda\ast x$, such that
\begin{equation}\label{scheme tangent bundle Sigma action construction-1}
\lambda\ast(\mu\ast x)=(\lambda\mu)\ast x,\quad 1\ast x=x,\quad 0\ast x=\tau_0\ast\phi
\end{equation}
where $\tau_0$ is the zero section $X\to T_{X/S}(\mathscr{M})$. We have similarly an action of $\End_{\mathscr{O}_S}(\mathscr{M}\oplus\mathscr{M})$ over $\Sigma(X',\mathscr{M}\oplus\mathscr{M})$.\par
Moreover, let $m:\mathscr{M}\oplus\mathscr{M}\to\mathscr{M}$ (resp. $\delta:\mathscr{M}\to\mathscr{M}\oplus\mathscr{M}$) the addition (resp. diagonal map) of $\mathscr{M}$, and put $m_{X'}:\Sigma(X',\mathscr{M}\oplus\mathscr{M})\to\Sigma(X',\mathscr{M})$ and $\delta_{X'}:\Sigma(X',\mathscr{M})\to\Sigma(X',\mathscr{M}\oplus\mathscr{M})$ be the induced morphisms. For $\lambda,\mu\in\mathbb{O}(S)$, let $h_\lambda$ (resp. $h_{\lambda,\mu}$) be the multiplication by $\lambda$ on $\mathscr{M}$ (resp. by $(\lambda,\mu)$ on $\mathscr{M}\oplus\mathscr{M}$). Since $m\circ h_{\lambda,\lambda}=h_\lambda\circ m$ and $m\circ h_{\lambda,\mu}=h_{\lambda+\mu}$, we have, for $z\in\Sigma(X',\mathscr{M}\oplus\mathscr{M})$ and $x\in\Sigma(X',\mathscr{M})$:
\begin{equation}\label{scheme tangent bundle Sigma action construction-2}
\lambda\ast m(z)=m((\lambda,\lambda)\ast z),\quad m((\lambda,\mu)\ast\delta(x))=(\lambda+\mu)\ast x.
\end{equation}
\end{remark}

\begin{definition}
Let $x\in X(S)=\Hom_S(S,X)=\Gamma(X/S)$. We then define the tangent space of $X$ over $S$ at the point $x$ relative to $\mathscr{M}$ to be the $S$-functor obtained from $T_{X/S}(\mathscr{M})$ by base change via the morphism $x:S\to X$:
\[\begin{tikzcd}
T_{X/S,x}(\mathscr{M})\ar[r]\ar[d]&T_{X/S}(\mathscr{M})\ar[d,"\pi"]\\
S\ar[r,"x"]&X
\end{tikzcd}\]
In particular, $T_{X/S,x}(\mathscr{O}_X)$ is denoted by $T_{X/S,x}$, which is called the \textbf{tangent space of $\bm{X}$ over $\bm{S}$ at the point $\bm{x}$}.
\end{definition}

\begin{remark}\label{scheme tangent bundle fiber char by morphism}
It follows from \cref{scheme tangent bundle projection zero section char} that, for any $t:S'\to S$, $T_{X/S,x}(\mathscr{M})(S')$ is the set of $S$-morphisms $f:I_{S'}(\mathscr{M})\to X$ such that $f\circ(\id_{S'}\times\eps_\mathscr{M})=x\circ t$, where $\eps_\mathscr{M}:S\to I_{S}(\mathscr{M})$ is the zero section.
\end{remark}

\begin{proposition}\label{scheme tangent bundle representable if}
If $X$ is representable, then $T_{X/S}(\mathscr{M})$ and $T_{X/S,x}(\mathscr{M})$ are representable. In particular, $T_{X/S}$ and $T_{X/S,x}$ are represented by the vector bundles $\V(\Omega_{X/S}^1)$ and $\V(x^*(\Omega_{X/S}^1))$.
\end{proposition}
\begin{proof}
It suffices to prove for $T_{X/S}(\mathscr{M})$, since the analogous result follows from base change. By \cref{scheme dual number isomorphic to product}, it suffices to consider $T_{X/S}$, which follows from \cref{scheme dual number Hom represented by vector bundle of Omega}.
\end{proof}

\begin{remark}
By \cref{scheme tangent bundle representable if}, we can give a simple description of the vector bundle representing $T_{X/S,x}$: if $x:S\to X$ is an $S$-morphism, then the image of $x$ is locally closed in $S$ by \cref{scheme morphism graph is immersion}, hence defined by a quasi-coherent ideal $\mathscr{I}$ of an open subscheme of $X$. The quotient $\mathscr{I}/\mathscr{I}^2$ can then be considered as a quasi-coherent module over $S$, whose vector bundle $\V(\mathscr{I}/\mathscr{I}^2)$ is the desired representing scheme.\par
For example, let $X$ be an algebraic scheme over a field $X$ and $x$ be a rational point of $X$ over $k$. Let $\m_x$ be the maximal ideal of the local ring $\mathscr{O}_{X,x}$, then we have $T_{X/k,x}=\V(\m_x/\m_x^2)$.
\end{remark}

We now retun to the general situation. We first note that $T_{X/S,x}$ is a covariant functor from the category of free $\mathscr{O}_S$-modules of finite rank to that of functors over $S$. In particular, $\mathbb{O}_S$ is a set of perators of the functor $T_{X/S,x}(\mathscr{M})$, which respects the functoriality on $\mathscr{M}$.

\begin{proposition}\label{scheme tangent bundle commutes with base change}
The formulation of $T_{X/S}(\mathscr{M})$ and $T_{X/S,x}(\mathscr{M})$ commutes with base changes: for any $S$-scheme $S'$, we have functorial isomorphisms
\begin{align*}
T_{X_{S'}/S'}(\mathscr{M}\otimes\mathscr{O}_S)\stackrel{\sim}{\to} T_{X/S}(\mathscr{M})_{S'},\\
T_{X_{S'}/S',x'}(\mathscr{M}\otimes\mathscr{O}_S)\stackrel{\sim}{\to} T_{X/S,x}(\mathscr{M})_{S'}
\end{align*}
where $x'=x_{S'}$.
\end{proposition}
\begin{proof}
This follows from the fact that $\sHom$ commutes with base changes.
\end{proof}

\begin{corollary}\label{scheme tangent bundle commutes with base change module isomorphism}
The $X$-functor $T_{X/S}(\mathscr{M})$ (resp. the $S$-functor $T_{X/S,x}(\mathscr{M})$) is naturally endowed with an $\mathbb{O}_X$-object (resp. $\mathbb{O}_S$-object) structure, which is functorial on $\mathscr{M}$, and the isomorphism of \cref{scheme tangent bundle commutes with base change} are isomorphism of $\mathbb{O}_{X_{S'}}$-objects (resp. $\mathbb{O}_{S'}$-objects).
\end{corollary}
\begin{proof}
We first prove the case for $T_{X/S,x}(\mathscr{M})$. For any $S'$ over $S$, $\mathbb{O}(S')$ acts on $\mathscr{M}\otimes\mathscr{O}_{S'}$, and hence on $T_{X_{S'}/S',x'}(\mathscr{M}\otimes\mathscr{O}_{S'})=T_{X/S,x}(\mathscr{M})_{S'}$. It is easy to verify that this operation is functorail on $S'$, so $T_{X/S,x}(\mathscr{M})$ is endowed with an $\mathbb{O}_S$-object structure.\par
For $T_{X/S}(\mathscr{M})$ this is more complicated. For each $X'$ over $X$, put $T_{X/S}(\mathscr{M})_{X'}=T_{X/S}(\mathscr{M})\times_XX'$; we need to endow $T_{X/S}(\mathscr{M})_{X'}(X')=\Hom_X(X',T_{X/S}(\mathscr{M}))$ with a structure of $\mathbb{O}(X')$-set which is functorial in $X'$. For this we construct the following diagram, where $X_{X'}=X\times_SX'$ and $f'$ is the section of $X_{X'}$ over $X'$ defined by $f:X'\to X$:
\[\begin{tikzcd}[row sep=4mm,column sep=2mm]
&T_{X_{X'}/X'}(\mathscr{M})\ar[ld]\ar[dd]&\\
T_{X/S}(\mathscr{M})\ar[dd]&&T_{X/S}(\mathscr{M})_{X'}\ar[lu]\ar[dd]\ar[ll,crossing over]\\
&X_{X'}\ar[ld]\ar[ld]&\\
X\ar[rd]&&X'\ar[ll,swap,"f"]\ar[ld]\ar[lu,swap,"f'"]\\
&S&
\end{tikzcd}\]
This diagram, together with \cref{scheme tangent bundle fiber char by morphism}, shows that $T_{X/S}(\mathscr{M})_{X'}(X')$ is identified with
\begin{equation}\label{scheme tangent bundle commutes with base change module isomorphism-1}
T_{X_{X'}/X',f'}(\mathscr{M})(X')=\{\text{$X'$-morphisms $\psi:I_{X'}(\mathscr{M})\to X_{X'}$ such that $\psi\circ\eps_{\mathscr{M}}=f'$}\},
\end{equation}
over which any $a\in\mathbb{O}(X')$ operates via the action over $I_{X'}(\mathscr{M})$, i.e. with the notations of \ref{scheme tangent bundle I_S paragraph}, we have $a\psi=\psi\circ a^*$, so for any $X''\to X'$ and $x\in I_{X'}(\mathscr{M})(X'')$, $(a\psi)(x)=\psi(x\cdot a)$. We then verify that this construction is functorial on $X'$.
\end{proof}

\begin{remark}\label{scheme trangent bundle operation of O_X define as morphism}
The operation of $\mathbb{O}_X$ over $T_{X/S}(\mathscr{M})$ can be simply defined as follows. For any $f:X'\to X$, by (\ref{scheme tangent bundle commutes with base change module isomorphism-1}) we have\footnote{If $X'$ is representable, this equality can also be deduced from \cref{scheme tangent bundle projection zero section char} and the equivalence $\widehat{\Sch}_{/X}\stackrel{\sim}{\to}\widehat{\Sch_{/X}}$. In fact, the equivalence $\alpha:\widehat{\Sch}_{/X}\to\widehat{\Sch_{/X}}$ commutes with Yoneda embedding, so we have
\[\Hom_X(X',T_{X/S}(\mathscr{M}))\cong\Hom_{X}(X',\alpha(T_{X/S}(\mathscr{M})))=\alpha(T_{X/S}(\mathscr{M}))(X')=\{\phi\in\Hom_S(I_{X'}(\mathscr{M}),X):\pi_{\mathscr{M}}(\phi)=f\}.\]
and \cref{scheme tangent bundle projection zero section char} shows that $\pi_\mathscr{M}(\phi)=\phi\circ\eps_\mathscr{M}$.}
\begin{align*}
\Hom_X(X',T_{X/S}(\mathscr{M}))=T_{X/S}(\mathscr{M})_{X'}(X')=\{\phi\in\Hom_S(I_{X'}(\mathscr{M}),X)\mid\phi\circ\eps_\mathscr{M}=f\},
\end{align*}
and we have seen in \cref{scheme tangent bundle I_S action of O(S)} that $I_{X'}(\mathscr{M})$, considered as an $S$-functor, is endowed with an operation by the monoid $\mathbb{O}(X')$ which conserve the zero section $\eps_\mathscr{M}:X'\to I_{X'}(\mathscr{M})$. Therefore, if we denote by $a^*$ the endomorphism of $I_{X'}(\mathscr{M})$ defined by $a\in\mathbb{O}(X')$, then we have $a^*\phi=\phi\circ a$, which means for any $S'\to S$ and $(m,x')\in\Hom_S(S',I_S(\mathscr{M})\times_SX')$,
\[(a\phi)(m,x')=\phi(m\cdot a(x'),x')\]
(note that $a^*\circ\eps_\mathscr{M}=\eps_\mathscr{M}$, whence $(a\phi)\circ\eps_\mathscr{M}=f$). Similarly, the operation of $\mathbb{O}_S$ over $T_{X/S,x}(\mathscr{M})$ can be described as follows. For any $t:S'\to S$, $T_{X/S,x}(\mathscr{M})(S')$ is the set of $S$-morphisms $\phi:I_{S'}(\mathscr{M})\to X$ such that $\phi\circ\eps_\mathscr{M}=u\circ t$; for such a $\phi$ and $a\in\mathbb{O}(S')$, we have $a\phi=\phi\circ a^*$.
\end{remark}

Let $S$ be a scheme and $X$ be an $S$-functor. We say that \textbf{$\bm{X}$ satisfies conditon (E) relative to $\bm{S}$} if, for any $S'\to S$ and any free $\mathscr{O}_{S'}$-module $\mathscr{M}$ and $\mathscr{N}$ of finite rank, the diagram of sets
\[\begin{tikzcd}[row sep=4mm,column sep=4mm]
&X(I_{S'}(\mathscr{M}\oplus\mathscr{N}))\ar[rd]\ar[ld]&\\
X(I_{S'}(\mathscr{M}))\ar[rd]&&X(I_{S'}(\mathscr{N}))\ar[ld]\\
&X(S')
\end{tikzcd}\]
obtained by applying $X$ to the diagram (\ref{scheme dual number of direct sum Cartesian diagram-1}), is Cartesian. Equivalently, this means the functor $\mathscr{M}\mapsto T_{X/S}(\mathscr{M})$ transforms direct sums of free $\mathscr{O}_S$-modules of finite rank to products of $X$-functors. If this is the case, the same holds for the functor $\mathscr{M}\mapsto T_{X/S,x}(\mathscr{M})=S\times_XT_{X/S}(\mathscr{M})$, for any $x\in\Gamma(X/S)$. By \cref{scheme dual number of direct sum Cartesian diagram}, we see that any representable functor satisfies condition (E).\par
We often say that "$X/S$ satisfies condition (E)" to abbreviate that $X$ satisfies condition (E) relative to $S$. In this case, the functor $\mathscr{M}\mapsto T_{X/S}(\mathscr{M})$ commutes with products, hence transforms groups to groups. In particular, $T_{X/S}(\mathscr{M})$ is an abelian $X$-group, and for the same reason $T_{X/S,x}(\mathscr{M})$ is an abelian $S$-group.

\begin{proposition}\label{scheme tangent bundle condition (E) module structure}
If $X/S$ satisfies condition (E), the abelian group structure over $T_{X/S}(\mathscr{M})$ (resp. $T_{X/S,x}(\mathscr{M})$) and the operation of $\mathbb{O}_X$ (resp. $\mathbb{O}_S$) endow $T_{X/S}(\mathscr{M})$ (resp. $T_{X/S,x}(\mathscr{M})$) with the structure of an $\mathbb{O}_X$-module (resp. $\mathbb{O}_S$-module).
\end{proposition}
\begin{proof}
The operation of $\mathbb{O}_X$ (resp. $\mathbb{O}_S$) is functorial on $\mathscr{M}$, so it respects the abelian group structure induced by the functoriality of $\mathscr{M}$. In fact, retain the notations of \cref{scheme tangent bundle Sigma action construction}. The structure of (abelian) $X$-group of $T_{X/S}(\mathscr{M})$ is deduced by the composition
\[T_{X/S}(\mathscr{M})\times_X T_{X/S}(\mathscr{M})\cong T_{X/S}(\mathscr{M}\oplus\mathscr{M})\stackrel{m}{\to} T_{X/S}(\mathscr{M}),\]
and on the other hand the morphism
\[T_{X/S}(\mathscr{M})\stackrel{\delta}{\to} T_{X/S}(\mathscr{M}\oplus\mathscr{M})\cong T_{X/S}(\mathscr{M})\times_XT_{X/S}(\mathscr{M})\]
is the diagonal morphism. We then deduce from the equality (\ref{scheme tangent bundle Sigma action construction-2}) and \cref{scheme tangent bundle Sigma action construction} that
\[\lambda(x+y)=\lambda x+\lambda y,\quad (\lambda+\mu)x=\lambda x+\mu x,\]
for any $f:X'\to X$, $x,y\in\Hom_X(X',T_{X/S}(\mathscr{M}))$ and $\lambda,\mu\in\mathbb{O}(X')$.
\end{proof}

\begin{remark}
If $X$ is representable, then it satisfies (E) and $T_{X/S}$ and $T_{X/S,x}$ are represented by vector bundles. The previous laws are the same as those which are deduced from the vector bundle structures.
\end{remark}

\begin{proposition}\label{scheme tangent bundle condition (E) base change}
If $X/S$ satisfies condition (E), then $X_{S'}/S'$ satisfies condition (E) and the isomorphisms of \cref{scheme tangent bundle condition (E) module structure} respects the $\mathbb{O}_{X_{S'}}$-module (resp. $\mathbb{O}_{S'}$-module) structure.
\end{proposition}
\begin{proof}
The formulation of $I_S(\mathscr{M})$ commutes with base change, so the first assertion is immediate. The second one follows from the proof of \cref{scheme tangent bundle condition (E) module structure}.
\end{proof}

\begin{proposition}\label{scheme tangent bundle functorial on X}
The functors $T_{X/S}(\mathscr{M})$ and $T_{X/S,x}(\mathscr{M})$ are functorial on $X$, which means if $f:X\to X'$ is an $S$-morphism, we have commutative diagrams
\[\begin{tikzcd}
T_{X/S}(\mathscr{M})\ar[r,"T(f)"]\ar[d]&T_{X'/S}(\mathscr{M})\ar[d]\\
X\ar[r]&X'
\end{tikzcd}\quad\quad
\begin{tikzcd}
T_{X/S,x}(\mathscr{M})\ar[rd]\ar[rr,"T_x(f)"]&&T_{X'/S,f\circ x}(\mathscr{M})\ar[ld]\\
&S&
\end{tikzcd}\]
Moreover, if $f$ is a monomorphism, so are $T(f)$ and $T_x(f)$.
\end{proposition}
\begin{proof}
The existence of $T(f)$ and $T_x(f)$, as well as the last assertion, follow immediately from definition. The commutativity of the diagrams then follows from the functoriality of these morphisms with respect to $\mathscr{M}$ and of the fact that $X=T_{X/S}(0)$.
\end{proof}

\begin{remark}\label{scheme tangent space of representable isomorphism if etale}
In the situation of \cref{scheme tangent bundle functorial on X}, suppose that $X$ and $X'$ are representable and $r$ is the rank of the free $\mathscr{O}_S$-module $\mathscr{M}$. Then by \cref{scheme dual number isomorphic to product}, $T_{X/S}(\mathscr{M})$ is isomorphic to the product over $X$ of $r$ copies of $\V(\Omega_{X/S}^1)$, and similarly for $T_{X'/S}(\mathscr{M})$. Therefore, the square in \cref{scheme tangent bundle functorial on X} are Cartesian if $f$ is an open immersion, of more generally if $f^*(\Omega_{X'/S}^1)=\Omega_{X/S}^1$ (for example if $f$ is \'etale). In this case, we have an isomorphism of $S$-functors 
\[T_{X/S,x}(\mathscr{M})\stackrel{\sim}{\to} T_{X'/S,f\circ x}(\mathscr{M}).\]
More generally, the Cartesian square of \cref{scheme tangent bundle functorial on X} defines a morphism of $X$-functors
\[\begin{tikzcd}
T_{X/S}(\mathscr{M})\ar[rr]\ar[rd]&&T_{X'/S}(\mathscr{M})\times_{X'}X\ar[ld]\\
&X&
\end{tikzcd}\]
\end{remark}

\begin{proposition}\label{scheme tangent bundle condition (E) functorial on X}
Let $f:X\to X'$ be an $S$-morphism. If $X$ and $X'$ satisfy condition (E) relative to $S$, then
\[T(f):T_{X/S}(\mathscr{M})\to T_{X'/S}(\mathscr{M})_X\quad\quad (\text{resp.}\quad T_x(f):T_{X/S,x}(\mathscr{M})\to T_{X'/S,f\circ x}(\mathscr{M}))\]
is a morphism of $\mathbb{O}_X$-modules (resp. $\mathbb{O}_S$-modules).
\end{proposition}
\begin{proof}
This follows from \cref{scheme tangent bundle functorial on X} by the functoriality on $\mathscr{M}$.
\end{proof}

\begin{proposition}\label{scheme tangent bundle fiber product commutes}
Let $X$ and $Y$ be functors over $S$. We have isomorphisms functorial on $\mathscr{M}$:
\begin{align}
T_{X/S}(\mathscr{M})\times_ST_{Y/S}(\mathscr{M})&\stackrel{\sim}{\to} T_{(X\times_SY)/S}(\mathscr{M}),\label{scheme tangent bundle fiber product commutes-1}\\
T_{X/S,x}(\mathscr{M})\times_ST_{Y/S,y}(\mathscr{M})&\stackrel{\sim}{\to} T_{(X\times_SY)/S,(x,y)}(\mathscr{M}),\label{scheme tangent bundle fiber product commutes-2}
\end{align}
\end{proposition}
\begin{proof}
The first isomorphism follows from (\ref{category of presheaf functor Hom product commutes-1}), and the second one is deduced by base change via $(x,y):S\to X\times_SY$.
\end{proof}

\begin{corollary}\label{scheme tangent bundle induce algebraic structure}
If $X/S$ is endowed with an algebraic structure defined by finite Cartesian products, then $T_{X/S}(\mathscr{M})$ is endowed with the same structure and the projection $T_{X/S}(\mathscr{M})\to X$ is a morphism of that structure.
\end{corollary}

\begin{proposition}\label{scheme tangent bundle condition (E) fiber product commutes}
If $X/S$ and $Y/S$ satisfy condition (E), then $(X\times_SY)/S$ satisfies condition (E) and (\ref{scheme tangent bundle fiber product commutes-1}) (resp. (\ref{scheme tangent bundle fiber product commutes-2})) is an isomorphism of $\mathbb{O}_{X\times_SY}$-modules (resp. $\mathbb{O}_S$-modules).
\end{proposition}
\begin{proof}
Suppose that $X/S$ and $Y/S$ satisfy condition (E). Then by (\ref{scheme tangent bundle fiber product commutes-1}), so does $(X\times_SY)/S$. Let $(x,y):Z\to X\times_SY$ be an $S$-morphism. To see that (\ref{scheme tangent bundle fiber product commutes-1}) is a morphism of $\mathbb{O}_{X\times_SY}$-modules, in view of \cref{scheme trangent bundle operation of O_X define as morphism}, it suffices to show that the map
\begin{align*}
\{\phi\in\Hom_S(I_Z(\mathscr{M}),X):\phi\circ\eps_\mathscr{M}=x\}&\times\{\psi\in\Hom_S(I_Z(\mathscr{M}),Y):\psi\circ\eps_\mathscr{M}=y\}\\
&\to\{\theta\in\Hom_S(I_Z(\mathscr{M}),X\times_SY):\theta\circ\eps_\mathscr{M}=(X,y)\}
\end{align*}
which to $(\phi,\psi)$ associated $\phi\times\psi$, is a morphism of $\mathbb{O}(Z)$-modules. But this is immediate, since for $a\in\mathbb{O}(Z)$ we have $a\cdot(\phi,\psi)=(\phi\circ a^*,\psi\circ a^*)$, and 
\[(\phi\circ a^*)\times(\psi\circ a^*)=(\phi\times\psi)\circ a^*=a\cdot(\phi\times\psi).\]
Similarly, by using \cref{scheme tangent bundle fiber char by morphism}, we can show that (\ref{scheme tangent bundle fiber product commutes-2}) is a morphism of $\O_{S}$-modules.
\end{proof}

If $X$ is an $S$-group and $e:S\to X$ is the unit section, we define
\[\mathfrak{Lie}(X/S,\mathscr{M})=T_{X/S,e}(\mathscr{M}),\]
that is, $\mathfrak{Lie}(X/S,\mathscr{M})$ is defined by the Cartesian square
\[\begin{tikzcd}
\mathfrak{Lie}(X/S,\mathscr{M})\ar[d]\ar[r,"i"]&T_{X/S}(\mathscr{M})\ar[d,"\pi"]\\
S\ar[r,"e"]&X
\end{tikzcd}\]
By \cref{scheme tangent bundle induce algebraic structure}, the projection $\pi:T_{X/S}(\mathscr{M})\to X$ is a morphism of $S$-groups, and it then follows that $\mathfrak{Lie}(X/S,\mathscr{M})$ is endowed with an $S$-group structure, and is isomorphic via $i$ to the kernel of $\pi$.\par
If, moreover, $X/S$ satisfies condition (E), we shall see in \cref{scheme H-object condition (E) Lie structure induced coincide} that the $S$-group structure of $\mathfrak{Lie}(X/S,\mathscr{M})$, induced by that of $X$, coincides with the abelian group structure induced by functoriality of $\mathscr{M}$. To this end we introduce the following terminology: an \textbf{H-set} is a set $X$ endowed with a composition law with a two-sided unit, denoted by $e_X$ or simply $e$. If $f:X\to Y$ is a morphism of H-sets, its kernel $\ker f$ is defined to be $f^{-1}(e_Y)$, which is a sub-H-set of $X$.\par
An H-object in a category $\mathcal{C}$ is defined by the usual manner: this is an object $X$ of $\mathcal{C}$, endowed with a morphism $X\times X\to X$ such that there exists a section of $X$ (over the final object) possessing the property of being a two-sided unit. Any $\mathcal{C}$-monoid, and in particular any $\mathcal{C}$-group is therefore an H-object. In particular, an H-object of the category of functors over a scheme $S$ is called an \textbf{$\bm{S}$-H-functor}. If $X$ is an $S$-H-functor (for example, an $S$-group), and $e:S\to X$ is the unit section of $X$, we define
\[\mathfrak{Lie}(X/S,\mathscr{M})=T_{X/S,e}(\mathscr{M}),\quad \mathfrak{Lie}(X/S)=\mathfrak{Lie}(X/S,\mathscr{O}_S).\]
By \cref{scheme tangent bundle induce algebraic structure}, we see that $T_{X/S}(\mathscr{M})$ and $\mathfrak{Lie}(X/S,\mathscr{M})$ are also $S$-H-functors, and we have morphisms of $S$-H-functors
\begin{equation}\label{scheme H-object Lie exact sequence}
\begin{tikzcd}
\mathfrak{Lie}(X/S,\mathscr{M})\ar[r,"i"]&T_{X/S}(\mathscr{M})\ar[r,shift left=2pt,"\pi"]&X\ar[l,shift left=2pt,"\tau"]
\end{tikzcd}
\end{equation}
where $i$ is an isomorphism from $\mathfrak{Lie}(X/S,\mathscr{M})$ to $\ker\pi$ and $\tau$ is a section of $\pi$.

\begin{proposition}\label{scheme H-object condition (E) Lie structure induced coincide}
Let $X$ be an $S$-H-object satisfying condition (E) relative to $S$. Then the $S$-H-object structure of $\mathfrak{Lie}(X/S,\mathscr{M})$ induced by that of $X$ coincides with the $S$-group structure induced by functoriality on $\mathscr{M}$.
\end{proposition}

Since $X$ satisfies condition (E), we see that $\mathfrak{Lie}(X/S,\mathscr{M})$ is an H-object in the category of $\mathbb{O}_S$-modules. The proposition then follows from the following lemma:

\begin{lemma}\label{category H-object of H-object is commutative}
Let $\mathcal{C}$ be a category. Let $G$ be an H-object in the category of $\mathcal{C}$-H-objects (i.e. $G$ is a $\mathcal{C}$-H-object endowed with a morphism of $\mathcal{C}$-H-objects $h:G\times G\to G$). Then $h$ coincides with the composition law of $G$ and is commutative.
\end{lemma}
\begin{proof}
By taking the values of the functors on a variable argument, we are reduced to the case where $\mathcal{C}$ is the category of sets. We then have a set $G$ and two maps $f,h:G\times G\to G$ such that
\begin{equation}\label{category H-object of H-object is commutative-1}
h(f(x,y),f(z,t))=f(h(x,z),h(y,t)),
\end{equation}
and we have two elements $e,u$ of $G$ such that $f(e,x)=f(X,e)=x$ and $h(u,x)=h(x,u)=x$. This is the famous Eckmann-Hilton argument\footnote{This argument is used to prove, for example, that higher homotopy groups are abelian.}, which we now provide a proof. We first note that by (\ref{category H-object of H-object is commutative-1}),
\begin{equation}\label{category H-object of H-object is commutative-2}
h(f(u,y),f(x,u))=f(x,y)=h(f(x,u),f(u,y)).
\end{equation}
In particular, for $y=e$ (resp. $x=e$), we obtain, respectively,
\begin{align*}
x=f(x,e)=h(f(u,e),f(x,u))=h(u,f(x,u))=f(x,u),\\
y=f(e,y)=h(f(e,u),f(u,y))=h(u,f(u,y))=f(u,y),
\end{align*}
whence the equality $h(y,x)=f(x,y)=h(x,y)$ in view of (\ref{category H-object of H-object is commutative-2}). This proves the lemma, whence \cref{scheme H-object condition (E) Lie structure induced coincide}.
\end{proof}

\begin{remark}\label{scheme S-H-functor condition (E) morphism i prop}
The assertion of \cref{scheme H-object condition (E) Lie structure induced coincide} can also be interpreted as follows: if we endow $\mathfrak{Lie}(X/S,\mathscr{M})$ with the abelian group structure induced by functoriality on $\mathscr{M}$, then the morphism $i:\mathfrak{Lie}(X/S,\mathscr{M})\to T_{X/S}(\mathscr{M})$ is a morphism of $S$-H-objects.
\end{remark}

\begin{corollary}\label{scheme S-H-functor condition (E) invertible if project to unit}
If $X$ is an $S$-H-functor satisfying condition (E) relative to $S$, any element of $X(I_S(\mathscr{M}))$, which projects to the unit element of $X(S)$, is invertible.
\end{corollary}
\begin{proof}
This follows from the sequence (\ref{scheme H-object Lie exact sequence}) and \cref{scheme H-object condition (E) Lie structure induced coincide}, since $\mathfrak{Lie}(X/S,\mathscr{M})$ is a group hence any element has an inverse.
\end{proof}

\begin{corollary}\label{scheme S-monoid condition (E) invertible iff image in X(S)}
If $X$ is an $S$-monoid satisfying condition (E) relative to $S$, an element of $X(I_S(\mathscr{M}))$ is invertible if and only if its image in $X(S)$ is invertible.
\end{corollary}
\begin{proof}
One direction is immediate, so assume that $x\in X(I_S(\mathscr{M}))$ is an element whose projection $s$ to $X(S)$ is invertible in $X(S)$. Let $s^{-1}$ be the inverse of $s$ in $X(S)$, then $y=x\tau(s^{-1})=x\tau(s)^{-1}$ is projective to the unit element of $X(S)$, and hence is invertible in $X(I_S(\mathscr{M}))$. If $y^{-1}$ is this inverse, we then have
\begin{align*}
x\cdot\tau(s)^{-1}y^{-1}=(x\tau(s)^{-1})\cdot(x\tau(s)^{-1})^{-1}=e,
\end{align*}
so $x$ is right invertible. Similarly, by considering $y'=\tau(s^{-1})x=\tau(s)^{-1}x$, we see that $x$ is also left invertible, so it is invertible in $X(I_S(\mathscr{M}))$.
\end{proof}

\begin{corollary}
If $X$ is an $S$-group satisfying condition (E) relative to $S$, the two $S$-group laws on $\mathfrak{Lie}(X/S,\mathscr{M})$ coincide.
\end{corollary}

\begin{corollary}\label{scheme S-group condition (E) power by n}
Let $G$ be an $S$-group satisfying condition (E) relative to $S$. For $n\in\Z$, let $n_G:G\to G$ be the morphism of $S$-functors defined by $g\mapsto g^n$. Then the induced morphism $\mathfrak{Lie}(n_G):\mathfrak{Lie}(G/S)\to\mathfrak{Lie}(G/S)$ is the multiplication by $n$, i.e. the map which to any $x\in\mathfrak{Lie}(G/S)(S')$ associates $nx$.
\end{corollary}
\begin{proof}
We first note that $n_G$ is in general not a morphism of groups, but it perverses the unit section $e:S\to G$, hence the induced morphism $\mathfrak{Lie}(n_G)=T_e(n_G)$ sends $\mathfrak{Lie}(G/S)$ into itself. If we denote by $i:\mathfrak{Lie}(G/S)\to T_{G/S}$ the inclusion, then $\mathfrak{Lie}(n_G)$ is defined by the equality $i(\mathfrak{Lie}(n_G)(x))=i(x)^n$, for any $S'\to S$ and $x\in\mathfrak{Lie}(G/S)(S')$. Now by \cref{scheme S-H-functor condition (E) morphism i prop} we have $i(x)^n=i(nx)$, whence $\mathfrak{Lie}(n_G)(x)=nx$.
\end{proof}

Before deducing other consequences from \cref{scheme H-object condition (E) Lie structure induced coincide}, let us prove another result of functoriality:

\begin{proposition}\label{scheme tangent bundle functor and sHom commutes}
In the situation of \ref{scheme tangent bundle functor sHom_Z/S(X,Y) paragraph}, we have a functorial isomorphism on $\mathscr{M}$:
\[T_{\sHom_{Z/S}(X,Y)/S}(\mathscr{M})\stackrel{\sim}{\to} \sHom_{Z/S}(X,T_{Y/Z}(\mathscr{M})).\]
\end{proposition}
\begin{proof}
In fact, by definition we have
\[T_{\sHom_{Z/S}(X,Y)/S}(\mathscr{M})=\sHom_S(I_S(\mathscr{M}),\sHom_{Z/S}(X,Y))\cong\sHom_{Z/S}(X,\sHom_Z(Z\times_SI_S(\mathscr{M}),Y)),\]
where we have used the isomorphism (\ref{category of presheaf functor Hom_Z/S(X,Y) isomorphism-1}) with $T=I_S(\mathscr{M})$. In view of the isomorphism $Z\times_SI_S(\mathscr{M})\cong I_Z(\mathscr{M})$, we then obtain
\begin{equation*}
T_{\sHom_{Z/S}(X,Y)/S}(\mathscr{M})\cong\sHom_{Z/S}(X,\sHom_Z(I_Z(\mathscr{M}),Y))=\sHom_{Z/S}(X,T_{Y/Z}(\mathscr{M})).\qedhere
\end{equation*}
\end{proof}

\begin{corollary}\label{scheme tangent bundle functor and sHom module structure}
If $Y/Z$ satisfies condition (E), then $\sHom_{Z/S}(X,Y)/S$ satisfies condition (E) and the isomorphism of \cref{scheme tangent bundle functor and sHom commutes} respects the $\mathbb{O}$-module structure over $\sHom_{Z/S}(X,Y)$.
\end{corollary}
\begin{proof}
Let $\mathscr{M}$, $\mathscr{N}$ be two free $\mathscr{O}_S$-modules of finite rank. If $Y/Z$ satisfies condition (E), then
\[T_{Y/Z}(\mathscr{M}\oplus\mathscr{N})\cong T_{Y/Z}(\mathscr{M})\times_Y T_{Y/Z}(\mathscr{N}).\]
The right side is a sub-functor of $T_{Y/Z}(\mathscr{M})\times_ST_{Y/Z}(\mathscr{N})$ and via the isomorphism (\ref{category of presheaf functor Hom product commutes-1}), we obtain an isomorphism
\begin{align*}
\sHom_{Z/S}(X,T_{Y/Z}(\mathscr{M}\oplus\mathscr{N}))\cong\sHom_{Z/S}(X,T_{Y/Z}(\mathscr{M}))\times_{\sHom_{Z/S}(X,Y)}\sHom_{Z/S}(X,T_{Y/Z}(\mathscr{N})).
\end{align*}
Combined with \cref{scheme tangent bundle functor and sHom commutes}, this implies
\[T_{\sHom_{Z/S}(X,Y)/S}(\mathscr{M}\oplus\mathscr{N})\cong T_{\sHom_{Z/S}(X,Y)/S}(\mathscr{M})\times_{\sHom_{Z/S}(X,Y)}T_{\sHom_{Z/S}(X,Y)/S}(\mathscr{N}),\]
so $\sHom_{Z/S}(X,Y)$ satisfies condition (E).\par
For the second assertion, let $H=\sHom_{Z/S}(X,Y)$ and consider an $S$-morphism $\Delta:H'\to\sHom_{Z/S}(X,Y)$, that is, an $Z$-morphism $\delta:H'\times_SX\to Y$, which makes $H'\times_SX$ a $Y$-object. We then have a commutative diagram
\[\begin{tikzcd}
\Hom_H(H',\sHom_{Z/S}(X,T_{Y/Z}(\mathscr{M})))\ar[r,hook]\ar[d,equal]&\Hom_S(H',\sHom_{Z/S}(X,T_{Y/Z}(\mathscr{M})))\ar[d,equal]\\
\Hom_Y(H'\times_SX,T_{Y/Z}(\mathscr{M}))\ar[r,hook]\ar[d,equal]&\Hom_Z(H'\times_SX,T_{Y/Z}(\mathscr{M}))\ar[d,equal]\\
\{\psi\in\Hom_Z(I_{H'\times_SX}(\mathscr{M}),Y):\psi\circ\eps_\mathscr{M}=\delta\}\ar[r,hook]&\Hom_Z(I_{H'\times_SX}(\mathscr{M}),Y).
\end{tikzcd}\]
By \cref{category of presheaf functor Hom module structure}, the action of $\alpha\in\mathbb{O}(H'\times_SX)$ over $\Psi\in\Hom_Y(H'\times_SX,T_{Y/Z}(\mathscr{M}))$ is given as follows: for any $U\to S$ and $(h,x)\in\Hom_S(U,H'\times_SX)$ ($U$ is then an $Y$-object via $\delta\circ(h,x)$), we have
\[(\alpha\Psi)(h,x)=\alpha(h,x)\Psi(h,x),\]
where $\alpha(h,x)\in\mathbb{O}(U)$ acts on $\Psi(h,x)\in T_{Y/Z}(\mathscr{M})(U)$ via the $\mathbb{O}_Y$-module structure of $T_{Y/Z}(\mathscr{M})$. By \cref{scheme trangent bundle operation of O_X define as morphism}, the latter is given, via the identification
\[\Hom_Y(H'\times_SX,T_{Y/Z}(\mathscr{M}))=\{\psi\in\Hom_Z(I_{H'\times_SX}(\mathscr{M}),Y):\psi\circ\eps_\mathscr{M}=\delta\},\]
by the following: for any $(m,h,x)\in\Hom_S(U,I_S(\mathscr{M})\times_SH'\times_SX)$,
\begin{equation}\label{scheme tangent bundle functor and sHom module structure-1}
(\alpha\psi)(m,h,x)=\psi(m\cdot\alpha(h,x),h,x).
\end{equation}
On the other hand, consider the tangent space $T_{H/S}(\mathscr{M})=\sHom_S(I_S(\mathscr{M}),H)$; we have a commutative diagram
\[\begin{tikzcd}
\Hom_H(H',T_{H/S}(\mathscr{M}))\ar[r,hook]\ar[d,equal]&\Hom_S(H',T_{H/S}(\mathscr{M}))\ar[d,equal]\\
\{\Phi\in\Hom_S(I_{H'}(\mathscr{M}),H):\Phi\circ\eps_\mathscr{M}=\Delta\}\ar[r,hook]\ar[d,equal,"(*)"]&\Hom_S(I_{H'}(\mathscr{M}),H)\ar[d,equal]\\
\{\phi\in\Hom_Z(I_{H'\times_SX}(\mathscr{M}),Y):\phi\circ\eps_\mathscr{M}=\delta\}\ar[r,hook]&\Hom_Z(I_{H'\times_SX}(\mathscr{M}),Y)
\end{tikzcd}\]
where the bijection $(*)$ is given as follows: for any $U\to S$ and $(m,h,x)\in\Hom(U,I_S(\mathscr{M})\times_SH'\times_SX)$ (so that $U$ is over $Z$ via $U\stackrel{x}{\to}X\to Z$), we have $\Phi(m,h)\in\Hom_Z(X\times_SU,Y)$ and
\begin{equation}\label{scheme tangent bundle functor and sHom module structure-2}
\phi(m,h,x)=\Phi(m,h)\circ(x\times\id_U)\in\Hom_Z(U,Y).
\end{equation}
By \cref{scheme trangent bundle operation of O_X define as morphism} (where we replace $X$ by $\sHom_{Z/S}(X,Y)$ and $X'$ by $H'$), the action of $a\in\mathbb{O}(H')$ over $\Phi\in\Hom_S(I_{H'}(\mathscr{M}),H)$ is given by
\[(a\Phi)(m,h)=\Phi(m\cdot a(h),h)\]
where $U\to S$ and $(m,h)\in\Hom_S(U,I_S(\mathscr{M})\times_SH')$. Therefore, if $\phi$ (resp. $a\phi$) is the element of $\Hom_Z(I_{H'\times_SX}(\mathscr{M}),Y)$ associated with $\Phi$ (resp $a\Phi$), we have, by (\ref{scheme tangent bundle functor and sHom module structure-2}),
\begin{equation}
(a\Phi)(m,h,x)=\Phi(m\cdot a(h),h)\circ(x\times\id_U)=\phi(m\cdot a(h),h,x).
\end{equation}
Together with (\ref{scheme tangent bundle functor and sHom module structure-1}), this shows that the isomorphism $T_{H/S}(\mathscr{M})\stackrel{\sim}{\to} \sHom_{Z/S}(X,T_{Y/Z}(\mathscr{M}))$ of \cref{scheme tangent bundle functor and sHom commutes} is an isomorphism of $\mathbb{O}(H)$-modules. Moreover, for any $H'\to H$, the $\mathbb{O}(H')$-module structure of $\Hom_H(H',T_{H/S}(\mathscr{M}))$ extends, in a functorial way on $H'$, to an $\mathbb{O}(H'\times_SX)$-module structure.
\end{proof}

In particular, for $Z=S$, we obtain the following corollary:
\begin{corollary}\label{scheme tangent bundle functor and sHom global commute}
We have a functorial isomorphism on $\mathscr{M}$:
\[T_{\sHom_S(X,Y)/S}(\mathscr{M})\stackrel{\sim}{\to} \sHom_S(X,T_{Y/S}(\mathscr{M})).\]
Moreover, if $Y/S$ satisfies condition (E), then $\sHom_S(X,Y)/S$ satisfies condition (E) and the preceding isomorphism respects the $\mathbb{O}$-module structure over $\sHom_S(X,Y)$.
\end{corollary}

Let $u:X\to Y$ be an $S$-morphism, which can be identified with a constant morphism $\bm{u}:S\to\sHom_S(X,Y)$ such that $\bm{u}(f)=u_{S'}$ for any $f:S'\to S$. The fiber product of $\bm{u}$ and $\sHom_S(X,T_{Y/S}(\mathscr{M}))\to\sHom_S(X,Y)$ is then identified with $\sHom_{Y/S}(X,T_{Y/S}(\mathscr{M}))$, where $X$ is over $Y$ via $u$. Therefore, we deduce from the definition of $T_{\sHom_S(X,Y)/S,\bm{u}}(\mathscr{M})$ and \cref{scheme tangent bundle functor and sHom global commute} the following:

\begin{corollary}\label{scheme tangent bundle fiber and sHom commute}
Let $u:X\to Y$ be an $S$-morphism. We have a functorial isomorphism on $\mathscr{M}$ (where $X$ is over $Y$ via $u$):
\[T_{\sHom_S(X,Y)/S,\bm{u}}(\mathscr{M})\stackrel{\sim}{\to} \sHom_{Y/S}(X,T_{Y/S}(\mathscr{M})).\]
This is an isomorphism of $\mathbb{O}_S$-modules if $Y/S$ satisfies condition (E).
\end{corollary}

In particular, for $Y=X$, $\sEnd_S(X)$ is an $S$-functor in monoids, hence a fortiori an $S$-H-functor. Since $\mathfrak{Lie}(\sEnd_S(X)/S,\mathscr{M})$ is by definition $T_{\sEnd_S(X)/S,e}(\mathscr{M})$, where $e$ is the unit section, we obtain (recall that $\sHom_{X/S}(X,T_{X/S}(\mathscr{M}))\cong\Res_{X/S}T_{X/S}(\mathscr{M})$):

\begin{corollary}\label{scheme tangent bundle Lie and Weil restriction isomorphism}
We have a functorial isomorphism on $\mathscr{M}$:
\[\mathfrak{Lie}(\sEnd_S(X)/S,\mathscr{M})\stackrel{\sim}{\to}\Res_{X/S}T_{X/S}(\mathscr{M}).\]
This is an isomorphism of $\mathbb{O}_S$-modules if $X/S$ satisfies condition (E).
\end{corollary}

\begin{remark}\label{scheme tangent bundle Weil restriction module structure}
Suppose that $X/S$ satisfies condition (E). Then the functor
\[\Res_{X/S}T_{X/S}(\mathscr{M})=\sHom_{X/S}(X,T_{X/S}(\mathscr{M}))\]
is endowed with a $\Res_{X/S}\mathbb{O}_X$-module structure, i.e. for any $S'\to S$,
\[\sHom_{X/S}(X,T_{X/S}(\mathscr{M}))(S')=\{\psi\in\Hom_X(I_{S'}(\mathscr{M})\times_SX,X):\psi\circ(\eps_\mathscr{M}\times\id_X)=\pr_X\}\]
is endowed with a $\mathbb{O}(X\times_SS')$-module structure, which is functorial on $S'$. This follows either from \cref{scheme tangent bundle condition (E) module structure} and the properties of the functor $\Res_{X/S}$, or from the proof of \cref{scheme tangent bundle functor and sHom module structure}.
\end{remark}

We now give a geometric interpretation of the tangent bundle. Let $U$ be an $S$-functor; by (\ref{category presheaf Hom functor adjoint prop-3}), we have isomorphism functorial on $\mathscr{M}$:
\begin{align*}
T_{X/S}(\mathscr{M})(U)&=\Hom_S(U,\sHom_S(I_S(\mathscr{M}),X))\cong\Hom_S(I_S(\mathscr{M}),\sHom_S(U,X))\\
&=\Hom_{I_S(\mathscr{M})}(U_{I_S(\mathscr{M})},X_{I_S(\mathscr{M})}).
\end{align*}
In particular, the morphism $\mathscr{M}\to 0$ induces a commutative diagram
\[\begin{tikzcd}
\Hom_S(U,T_{X/S}(\mathscr{M}))\ar[r,"\sim"]\ar[d,"\circ\pi_\mathscr{M}"]&\Hom_{I_S(\mathscr{M})}(U_{I_S(\mathscr{M})},X_{I_S(\mathscr{M})})\ar[d]\\
\Hom_S(U,X)\ar[r,equal]&\Hom_S(U,X)
\end{tikzcd}\]
where the second vertical arrow is given by base change $\eps_\mathscr{M}:S\to I_S(\mathscr{M})$. We therefore obtain the following proposition:

\begin{proposition}\label{scheme tangent bundle Hom to char}
Let $h_0:U\to X$ be an $S$-morphism. Then $\Hom_X(U,T_{X/S}(\mathscr{M}))$ is identified with the set of $I_S(\mathscr{M})$-morphisms $h:U_{I_S(\mathscr{M})}\to X_{I_S(\mathscr{M})}$ that extend $h_0$ (we view $U$ (resp. $X$) as a sub-object of $U\times_SI_S(\mathscr{M})$ (resp. $X\times_SI_S(\mathscr{M})$) via $\id_U\times_S\eps_\mathscr{M}$ (resp. $\id_X\times_S\eps_\mathscr{M}$)).
\end{proposition}

In particular, for $U=X$ and $h_0=\id_X$, we obtain:

\begin{corollary}\label{scheme tangent bundle section over X char}
The set $\Gamma(T_{X/S}(\mathscr{M})/X)$ is identified with the set of $I_S(\mathscr{M})$-endomorphisms $\phi$ of $X_{I_S(\mathscr{M})}$ which induce identity on $X$, i.e. such that the following diagram is commutative:
\[\begin{tikzcd}
I_X(\mathscr{M})\ar[rr,"\phi"]&&I_X(\mathscr{M})\\
&X\ar[lu,"\eps_\mathscr{M}"]\ar[ru,swap,"\eps_\mathscr{M}"]&
\end{tikzcd}\]
\end{corollary}

On the other hand, by \cref{scheme tangent bundle fiber and sHom commute}, $\Gamma(T_{X/S}(\mathscr{M})/X)\cong\mathfrak{Lie}(\sEnd_S(X)/S,\mathscr{M})(S)$. If $X/S$ satisfies condition (E), then $\sEnd_S(X)/S$ satisfies condition (E) and $\mathfrak{Lie}(\sEnd_S(X)/S,\mathscr{M})$ is then an $\mathbb{O}_S$-module (and in fact a $\Res_{X/S}\mathbb{O}_X$-module). Applying \cref{scheme H-object condition (E) Lie structure induced coincide}, we then deduce that

\begin{proposition}\label{scheme tangent bundle condition (E) section over X abelian group char}
If $X/S$ satisfies condition (E), the abelian group $\Gamma(T_{X/S}(\mathscr{M})/X)$ is identified with the set of $I_S(\mathscr{M})$-endomorphisms of $X_{I_S}(\mathscr{M})$ which induce identity on $X$. In particular, any $I_S(\mathscr{M})$-endomorphism of $X_{I_S}(\mathscr{M})$ which induces the identity on $X$ is an automorphism.
\end{proposition}

\begin{corollary}\label{scheme tangent bundle condition (E) morphism extension is iso}
Let $u:X\to Y$ be an $S$-isomorphism with $Y/S$ satisfying condition (E). Any $I_S(\mathscr{M})$-morphism of $X_{I_S(\mathscr{M})}$ to $Y_{I_S(\mathscr{M})}$ which extends $u$ is an isomorphism.
\end{corollary}
\begin{proof}
By \cref{scheme tangent bundle Hom to char} the considered set is identified with $\Hom_Y(X,T_{Y/S}(\mathscr{M}))$, which is isomorphic to $\Gamma(T_{Y/S}(\mathscr{M})/Y)$ by our hypothesis.
\end{proof}

\begin{corollary}\label{scheme tangent bundle condition (E) fiber of Iso to Hom}
If $Y/S$ satisfies condition (E), the monomorphism $\sIso_S(X,Y)\to\sHom_S(X,Y)$ induces, for any $u\in\Iso_S(X,Y)$, an isomorphism
\[T_{\sIso_S(X,Y)/S,u}(\mathscr{M})\stackrel{\sim}{\to} T_{\sHom_S(X,Y)/S,u}(\mathscr{M}).\]
\end{corollary}
\begin{proof}
It suffices to see that $T_{\sIso_S(X,Y)/S,u}(\mathscr{M})\stackrel{\sim}{\to} T_{\sHom_S(X,Y)/S,u}(\mathscr{M})$ is a bijection, for any $S'\to S$. By base change (cf. \cref{scheme tangent bundle fiber product commutes}), it suffices to consider $S'=S$. In this case, we note that $T_{\sHom_S(X,Y)/S,u}(\mathscr{M})(S)$ (resp. $T_{\sIso_S(X,Y)/S,u}(\mathscr{M})(S)$) is the set of $I_S(\mathscr{M})$-morphisms (resp. automorphims) $X_{I_S(\mathscr{M})}\to Y_{I_S(\mathscr{M})}$ which extends $u$, and we can apply \cref{scheme tangent bundle condition (E) morphism extension is iso}.
\end{proof}

\begin{corollary}\label{scheme tangent bundle condition (E) fiber of Aut to End}
If $X/S$ satisfies (E), the monomorphism $\sAut_S(X)\to\sEnd_S(X)$ induces, for any $u\in\sAut_S(X)$, an isomorphism $T_{\sAut_S(X)/S,u}(\mathscr{M})\stackrel{\sim}{\to} T_{\sEnd_S(X)/S,u}(\mathscr{M})$. In particular, we have
\[\mathfrak{Lie}(\sAut_S(X)/S,\mathscr{M})\stackrel{\sim}{\to} \mathfrak{Lie}(\sEnd_S(X)/S,\mathscr{M})\stackrel{\sim}{\to}\Res_{X/S}T_{X/S}(\mathscr{M})\]
so that $\mathfrak{Lie}(\sAut_S(X)/S,\mathscr{M})$ is endowed with a $\Res_{X/S}\mathbb{O}_X$-module structure.
\end{corollary}

\begin{example}\label{scheme functor condition (E) non-example}
There exist functors possessing infinitesimal endomorphisms which are not automorphisms, and hence a fortiori do not satisfy condition (E). For any pointed set $(E,x_0)$, let $M(E)$ be the free commutative monoid generated by $E$ and $M_P(E,x_0)$ be the commutative monoid obtained by quotient $M(E)$ by the equivalence relation generated by $m\sim x_0+m$. Then $(E,x_0)\to M_P(E,x_0)$ is the left adjoint of the forgetful functor from the category of commutative monoid to that of pointed sets. We say that $M_P(E,x_0)$ is the \textbf{free commutative monoid over the pointed set $(E,x_0)$}.\par
Let $X$ be the functor which associates any scheme $S$ to the free commutative monoid over the set $\mathbb{O}(S)$, pointed by the zero element. A morphism $f:S\to I_{\Z}=\Spec(\Z[t])$ corresponds to a square zero element $u_f$ of $\mathbb{O}(S)$, hence defines an endomorphism of $X(S)$ by $x\mapsto x+u_f$ (taken in $M_P(\mathbb{O}(S),0)$). We thus obtain an endomorphism $\phi$ of $X_{I_{\Z}}=X\times_{\Z}I_{\Z}$, defined as follows. For any $f\in I_{\Z}(S)$ and $x\in X(S)$,
\[\phi(x,f)=(x+u_f,f).\]
If $f_0:S\to I_{\Z}$ is the composition of the structural morphism $S\to\Spec(\Z)$ and the zero section of $I_\Z$, the corresponding element $u_{f_0}=0$, and hence $\phi(x,f_0)=(x,f_0)$ (since $x+0=x$ in $M_P(\mathbb{O}(S),0)$). Since the map $X(S)\to X_{I_{\Z}}(S)$ is given by $x\mapsto (x,f_0)$, this shows that $\phi$ induces the identity on $X$, hence is an infinitesimal endomorphism of $X$ which is evidently not an automorphism.
\end{example}

Suppose that $X$ is representable. In this case, we have seen in \cref{scheme tangent bundle representable if} that the $X$-functor $T_{X/S}$ is represented by $\V(\Omega_{X/S}^1)$, whence the bijections
\begin{equation}\label{scheme group representable tangent bundle section and derivation}
\Gamma(T_{X/S}/X)\cong\Hom_X(\Omega_{X/S}^1,\mathscr{O}_S)\cong\Der_{\mathscr{O}_S}(\mathscr{O}_X).
\end{equation}
This can also be deduced as follows. According to \cref{scheme tangent bundle condition (E) section over X abelian group char}, $\Gamma(T_{X/S}/X)$ is identified with the set of \textbf{infinitesimal endomorphisms} of $X$ (i.e. $I_S$-endomorphisms of $X_{I_S}$ inducing the identity on $X$). Now $X$ and $X_{I_S}$ have the same underlying topological space, with structural sheaves being $\mathscr{O}_X$ and $\mathscr{D}_{\mathscr{O}_X}=\mathscr{O}_X\oplus\mathscr{M}$, where $\mathscr{M}=\mathscr{O}_X$ is considered as a square zero ideal. Let $\pi:\mathscr{D}_{\mathscr{O}_X}\to\mathscr{O}_X$ be the morphism of $\mathscr{O}_X$-algebras which is zero on $\mathscr{M}$, we then deduce that giving an infinitesimal endomorphism of $X$ is equivalent to giving a morphism of $\mathscr{O}_S$-algebras $\phi:\mathscr{O}_X\to\mathscr{D}_{\mathscr{O}_X}$ such that $\pi\circ\phi=\id_{\mathscr{O}_X}$, which then amouts to giving an $\mathscr{O}_S$-derivation of the sheaf of rings $\mathscr{O}_X$.\par
Moreover, we see that if $D,D'\in\Der_{\mathscr{O}_S}(\mathscr{O}_X)$ and if we denote by $\phi_D$ the infinitesimal endomorphism correponding to $D$, then
\[\phi_{D+D'}=\phi_{D}\circ\phi_{D'}.\]
This shows that the identification
\[\{\text{infinitesimal endomorphisms of $X$}\}\cong\Der_{\mathscr{O}_S}(\mathscr{O}_X)\]
is an isomorphism of abelian groups. In view of \cref{scheme tangent bundle condition (E) section over X abelian group char} (and \cref{scheme tangent bundle Weil restriction module structure}), we have then isomorphism of abelian groups (as well as $\mathbb{O}(X)$-modules)
\[\Gamma(T_{X/S}/X)\stackrel{\sim}{\to} \Der_{\mathscr{O}_S}(\mathscr{O}_X)\]
which ressume the classical interpretation of tangent vectors in view of derivations of the structural sheaf. Recall also that $\Gamma(T_{X/S}/X)$ is equal to $H^0(X,\g_{X/S})$, where $\g_{X/S}$ is the dual of $\Omega_{X/S}^1$.

\subsection{Tangent space of a group}
Let $G$ be a functor in groups over $S$. By \cref{scheme tangent bundle induce algebraic structure}, $T_{G/S}(\mathscr{M})$ and $\mathfrak{Lie}(G/S,\mathscr{M})$ are endowed with group structures over $S$ and we have group morphisms
\begin{equation}\label{scheme group tangent space and Lie split exact sequence}
\begin{tikzcd}
\mathfrak{Lie}(G/S,\mathscr{M})\ar[r,"i"]&T_{G/S}(\mathscr{M})\ar[r,shift left=2pt,"\pi"]&G\ar[l,shift left=2pt,"\tau"]
\end{tikzcd}
\end{equation}
By definition $i$ is an isomorphism from $\mathfrak{Lie}(G/S)(\mathscr{M})$ onto the kernel of $\pi$, and $\tau$ is a section of $\pi$. It then follows from \cref{category group homomorphism section iff semi-direct} that we can identify $T_{G/S}(\mathscr{M})$ with a semi-direct product of $G$ by $\mathfrak{Lie}(G/S,\mathscr{M})$.

\begin{definition}
The corresponding operation of $G$ on $\mathfrak{Lie}(G/S,\mathscr{M})$ is denoted by
\[\Ad:G\to\sAut_{\Grp}(\mathfrak{Lie}(G/S,\mathscr{M}))\]
and called the adjoint representation (relative to $\mathscr{M}$) of $G$. For any $S'\to S$, we then have by definition, for $x\in G(S')$ and $X\in\mathfrak{Lie}(G/S,\mathscr{M})(S')$, that
\[\Ad(x)(X)=i^{-1}(\tau(x)i(x)\tau(x)^{-1}).\]
\end{definition}
\begin{definition}
If $G$ and $H$ are two functors in groups over $S$ and if $f:G\to H$ is a group morphism, then we have an induced morphism of exact sequences which is compatible with sections:
\[\begin{tikzcd}
1\ar[r]&\mathfrak{Lie}(G/S,\mathscr{M})\ar[r]\ar[d,"\mathfrak{Lie}(f)"]&T_{G/S}(\mathscr{M})\ar[r]\ar[d,"T(f)"]&G\ar[r]\ar[d,"f"]&1\\
1\ar[r]&\mathfrak{Lie}(H/S,\mathscr{M})\ar[r]&T_{H/S}(\mathscr{M})\ar[r]&H\ar[r]&1
\end{tikzcd}\]
The morphism $\mathfrak{Lie}(f)=T_e(f)$ is the derived morphism of $f$. If $G/S$ and $H/S$ satisfy condition (E), then $\mathfrak{Lie}(f)$ respects the $\mathbb{O}_S$-module structure induced by functoriality on $\mathscr{M}$ (cf. \cref{scheme tangent bundle condition (E) functorial on X}).
\end{definition}

\begin{proposition}\label{scheme group Lie Ad is derived of Inn}
Let $g\in G(S)$, then $\Ad(g):\mathfrak{Lie}(G/S,\mathscr{M})\to\mathfrak{Lie}(G/S,\mathscr{M})$ is the derived morphism of $\inn(g):G\to G$.
\end{proposition}
\begin{proof}
In fact, $\Ad(g)X=i^{-1}(\inn(g)i(X))$, which is none other than $T(\inn(g))X$ by the definition of the derived morphism.
\end{proof}

If $G/S$ satisfies condition (E), then by \cref{scheme H-object condition (E) Lie structure induced coincide}, the group structure of $\mathfrak{Lie}(G/S,\mathscr{M})$ defined from $G$ coincides with that induced by the $\mathbb{O}_S$-module structure of $\mathscr{M}$. We then deduce from the preceding proposition and the functoriality of the operation of $\mathbb{O}_S$ (\cref{scheme tangent bundle condition (E) functorial on X}) that:

\begin{corollary}\label{scheme group condition (E) Ad is linear representation}
Suppose that $G/S$ satisfies condition (E). Then the morphism $\Ad$ sends $G$ into the subgroup $\sAut_{\mathbb{O}_S}(\mathfrak{Lie}(G/S,\mathscr{M}))$ of $\sAut_{\Grp}(\mathfrak{Lie}(G/S),\mathscr{M})$, that is, for any $g\in G(S')$, $\Ad(g)$ respects the $\mathbb{O}(S')$-module structure of $\mathfrak{Lie}(G_{S'}/S',\mathscr{M})$. In other words, $\Ad$ is a linear representation of $G$ on the $\mathbb{O}_S$-module $\mathfrak{Lie}(G/S,\mathscr{M})$.
\end{corollary}

\begin{remark}
Suppose that $G/S$ satisfies condition (E). Then the derived morphism of the group law $m:G\times_SG\to G$ is none other than the addition law of $\mathfrak{Lie}(G/S,\mathscr{M})$ ($m$ is not a morphism of groups, but $m(e,e)=e$, so the derived morphism $\mathfrak{Lie}(m)$ sends $T_{(G\times_SG)/S,(e,e)}(\mathscr{M})=\mathfrak{Lie}(G/S,\mathscr{M})\times_S\mathfrak{Lie}(G/S,\mathscr{M})$ into $\mathfrak{Lie}(G/S,\mathscr{M})$). For any $n\in\Z$, we show similarly that if $n_G:G\to G$ is the morphism of $S$-functors defined by $g\mapsto g^n$, then the derived morphism $\mathfrak{Lie}(n_G)$ is the multiplication by $n$ on $\mathfrak{Lie}(G/S)$, cf. \cref{scheme S-group condition (E) power by n}.
\end{remark}

Now consider the $S$-functor $\sHom_{G/S}(G,T_{G/S}(\mathscr{M}))$; for any $S'\to S$, we have $T_{G/S}(\mathscr{M})_{S'}\cong T_{G_{S'}/S'}(\mathscr{M})$ and hence
\[\sHom_{G/S}(G,T_{G/S}(\mathscr{M}))(S')\cong\Hom_{G_{S'}}(G_{S'},T_{G_{S'}/S'}(\mathscr{M}))=\Gamma(T_{G_{S'}/S'}(\mathscr{M})/G_{S'}).\]
Note that we have an isomorphism, functorial on $S'$,
\begin{equation}\label{scheme group tangent bundle Hom of Lie and section isomorphism}
\Hom_{S'}(G_{S'},\mathfrak{Lie}(G_{S'}/S',\mathscr{M}))\stackrel{\sim}{\to}\Gamma(T_{G_{S'}/S'}(\mathscr{M})/G_{S'})
\end{equation}
which to any $f:G_{S'}\to\mathfrak{Lie}(G_{S'}/S',\mathscr{M})$ associates the section $s_f:G_{S'}\to T_{G_{S'}/S'}(\mathscr{M})$ such that, for any $S''\to S'$ and $g\in G(S'')$,
\[s_f(g)=i(f(g))\tau(g).\]
Let $h$ be an automorphism of the functor $G_{S'}$ over $S'$ (not necessarily respects the group structure). To any section $s$ of $T_{G_{S'}/S'}(\mathscr{M})$, we can associate $h(s)$ defined by transport the structure: this for example the only section of $T_{G_{S'}/S'}(\mathscr{M})$ fitting into the commutative diagram
\[\begin{tikzcd}
G_{S'}\ar[r,"s"]\ar[d,swap,"h"]&T_{G_{S'}/S'}(\mathscr{M})\ar[d,"T(h)"]\\
G_{S'}\ar[r,"h(s)"]&T_{G_{S'}/S'}(\mathscr{M})
\end{tikzcd}\]
In particular, we can take $h$ to be the right translation $t_x$ by an element $x$ of $G(S')$, that is, $h(g)=t_x(g)=g\cdot x$, for any $g\in G(S'')$, $S''\to S'$. We have immediately
\[t_x(s_f)=s_{t_x(f)},\]
where $t_x(f):G_{S'}\to\mathfrak{Lie}(G_{S'}/S',\mathscr{M})$ is defined by
\[t_x(f)(g)=f(g\cdot x^{-1})\]
for any $g\in G(S'')$, $S''\to S'$. It follows that if we operate $G$ on  $\sHom_{G/S}(G,T_{G/S}(\mathscr{M}))$ and $\sHom_S(G,\mathfrak{Lie}(G/S,\mathscr{M}))$ by right translation in the following way: for any $S'\to S$, $x\in G(S')$, $\sigma\in\Gamma(T_{G_{S'}/S'}(\mathscr{M}/G_{S'}))$ and $f\in\Hom_{S'}(G_{S'},\mathfrak{Lie}(G_{S'}/S',\mathscr{M}))$,
\[(\sigma\cdot x)(g)=\sigma(g\cdot x^{-1})\cdot \tau(x),\quad (f\cdot x)(g)=f(g\cdot x^{-1}),\]
for any $g\in G(S'')$, $S''\to S'$, then the isomorphism (\ref{scheme group tangent bundle Hom of Lie and section isomorphism}) respects the action of $G$.\par
In particular, by this isomorphism, the elements of $\sHom_{G/S}(G,T_{G/S}(\mathscr{M}))^G(S')$ (called \textbf{right invariant sections} of $T_{G_{S'}/S'}(\mathscr{M})$) corresponds to constant morphisms of $G_{S'}$ into $\mathfrak{Lie}(G_{S'}/S',\mathscr{M})$ (i.e. which factors through the projection $G_{S'}\to S'$), or to elements of $\mathfrak{Lie}(G_{S'}/S',\mathscr{M})(S')=\mathfrak{Lie}(G/S,\mathscr{M})(S')$. We then have the following proposition:

\begin{proposition}\label{scheme group Lie and right invariant section}
The map $\mathfrak{Lie}(G/S,\mathscr{M})(S)\to\Gamma(T_{G/S}(\mathscr{M})/G)$ which associates an element $X\in\mathfrak{Lie}(G/S,\mathscr{M})(S)$ the section $x\mapsto X(\pi(x))$ is a bijection from $\mathfrak{Lie}(G/S,\mathscr{M})(S)$ onto the set of right invariant sections of $\Gamma(T_{G/S}(\mathscr{M})/G)$.
\end{proposition}

Similarly, we can act $G$ on $\sEnd_{I_S(\mathscr{M})/S}(G_{I_S(\mathscr{M})})$ as follows: for any $S'\to S$, $x\in G(S')$ and $u\in\sEnd_{I_S(\mathscr{M})/S}(G_{I_S(\mathscr{M})})(S')=\End_{I_{S'}}(G_{I_{S'}(\mathscr{M})})$,
\[(u\cdot x)(g)=u(g\cdot x^{-1})\cdot x,\]
for any $g\in G(S'')$, $S''\to I_{S'}(\mathscr{M})$. Then the morphism of \cref{scheme tangent bundle section over X char}
\[\sHom_{G/S}(G,T_{G/S}(\mathscr{M}))\to\sEnd_{I_S(\mathscr{M})/S}(G_{I_S(\mathscr{M})})\]
respects the operation of $G$ and induces for any $S'\to S$ a bijection from $\Gamma(T_{G_{S'}/S'}(\mathscr{M})/G_{S'})$ and the set of $I_{S'}(\mathscr{M})$-endomorphisms $u$ of $G_{I_{S'}(\mathscr{M})}$ inducing the identity on $G$ and are invariant under right translations, i.e. satisfies $u_{S''}\cdot x=u_{S''}$ for any $S''\to S'$ and $x\in G(S'')$. By \cref{scheme tangent bundle condition (E) section over X abelian group char}, we then conclude the following theorem:

\begin{proposition}\label{scheme group Lie and right invariant I_S-endomorphism}
There exists a bijection (functorail on $G$) from the set $\mathfrak{Lie}(G/S,\mathscr{M})(S)$ to the set of $I_S(\mathscr{M})$-endomorphisms of $G_{I_S(\mathscr{M})}$ inducing the identity on $G$ and commutes with right translations of $G$, and this is a group isomorphism if $G/S$ satisfies condition (E).
\end{proposition}

By considering the case $\mathscr{M}=\mathscr{O}_S$, we thus obtain the classical definitions of the Lie algebra of a group.\par

Before going further, let us establish some new corollaries of \cref{scheme tangent bundle functor and sHom commutes}. Let $X,Y$ be over $Z$ and $Z$ be over $S$, as in \ref{scheme tangent bundle functor sHom_Z/S(X,Y) paragraph}. As we have seen in \cref{scheme tangent bundle functor and sHom commutes}, the isomorphisms (\ref{category of presheaf functor Hom_Z/S(X,Y) isomorphism-2}):
\begin{equation}\label{scheme group tangent bundle of sHom isomorphism-1}
\begin{tikzcd}[column sep=1mm]
\sHom_S(I_S(\mathscr{M}),\sHom_{Z/S}(X,Y))\ar[rd,"\sim"]\ar[rr,"\sim"]&&\sHom_{Z/S}(X,\sHom_Z(I_Z(\mathscr{M}),Y))\\
&\sHom_{Z/S}(X\times_SI_S(\mathscr{M}),Y)\ar[ru,"\sim"]
\end{tikzcd}
\end{equation}
induces the isomorphism $\theta$ below
\begin{equation}\label{scheme group tangent bundle of sHom isomorphism-2}
\begin{tikzcd}[column sep=1mm]
T_{\sHom_{Z/S}(X,Y)}(\mathscr{M})\ar[rd,"\sim"]\ar[rr,swap,"\sim","\theta"']&&\sHom_{Z/S}(X,T_{Y/Z}(\mathscr{M}))\\
&\sHom_{Z/S}(X\times_SI_S(\mathscr{M}),Y)\ar[ru,"\sim"]
\end{tikzcd}
\end{equation}
By \cref{category of presheaf functor Hom product commutes}, if $Y$ is a $Z$-group, so is $\sHom_Z(V,Y)$ for any $V\to Z$ (in particular for $V=I_Z(\mathscr{M})$); explicitly, if $Z''\to Z'\to Z$ and $\phi,\psi\in\Hom_Z(V_{Z'},Y)$, then $\phi\cdot\psi$ is defined by
\[(\phi\cdot\psi)(v)=\phi(v)\psi(v)\]
for any $v\in V_{Z'}(Z'')$.

\begin{definition}
Suppose that $X$ and $Y$ are $Z$-groups. Let $\sHom_{(Z/S)\dash\Grp}(X,Y)$ be the sub-functor of $\sHom_{Z/S}(X,Y)$ defined as follows: for any $S'\to S$,
\begin{equation}\label{scheme group tangent bundle of sHom isomorphism-3}
\sHom_{(Z/S)\dash\Grp}(X,Y)(S')=\Hom_{Z_{S'}\dash\Grp}(X_{S'},Y_{S'}).
\end{equation}
This definition applies equally if we replace $Y$ by the $Z$-group $T_{Y/Z}(\mathscr{M})$.
\end{definition}

We then easily see that $T_{\sHom_{(Z/S)\dash\Grp}(X,Y)/S}(\mathscr{M})(S')$ corresponds, under the isomorphisms of (\ref{scheme group tangent bundle of sHom isomorphism-2}), to $Z_{S'}$-morphisms $\phi:X_{S'}\times_{S'}I_{S'}(\mathscr{M})\to Y_{S'}$ which is multiplicative on $X$, that is, which satisfies $\phi(x_1x_2,m)=\phi(x_1,m)\phi(x_2,m)$, and these correspond to $Z_{S'}$-group morphisms $X_{S'}\to T_{Y/Z}(\mathscr{M})_{S'}$. We then obtain the following:

\begin{proposition}\label{scheme group tangent bundle of sHom_Z/S isomorphism}
Let $X,Y$ be $Z$-groups and $Z$ be over $S$. We have an isomorphism of $S$-functors, functorial on $\mathscr{M}$:
\[T_{\sHom_{(Z/S)\dash\Grp}(X,Y)}(\mathscr{M})\stackrel{\sim}{\to}\sHom_{(Z/S)\dash\Grp}(X,T_{Y/Z}(\mathscr{M})).\]
\end{proposition}

In particular, for $Z=S$, we obtain the following corollary. Before stating it, we note that if $Y$ is an abelian $S$-group, then so is $T_{Y/S}(\mathscr{M})$, and hence $H=\Hom_{S\dash\Grp}(X,Y)$ and $\Hom_{S\dash\Grp}(X,T_{Y/S}(M))$, and finally is $T_{H/S}(\mathscr{M})$.

\begin{corollary}\label{scheme group tangent bundle of sHom isomorphism}
Let $X,Y$ be $S$-groups. We have an isomorphism of $S$-functors, functorial on $\mathscr{M}$:
\[T_{\sHom_{S\dash\Grp}(X,Y)/S}(\mathscr{M})\stackrel{\sim}{\to} \sHom_{S\dash\Grp}(X,T_{Y/S}(\mathscr{M})).\]
If $Y$ is commutative, then this is an isomorphism of abelian $S$-groups.
\end{corollary}

If $Y$ is an $\mathbb{O}_S$-module, the functor $T_{Y/S}(\mathscr{M})$ (resp. $\mathfrak{Lie}(Y/S,\mathscr{M})$) is endowed with an $\mathbb{O}_S$-module structure deduced by that of $Y$, which we denote by $T_{Y/S}'(\mathscr{M})$ (resp. $\mathfrak{Lie}'(Y/S,\mathscr{M})$). Therefore, if $X,Y$ are $\mathbb{O}_S$-modules, then $T_{Y/S}'(\mathscr{M})=\sHom_S(I_S(\mathscr{M}),Y)$ and $H=\sHom_{\mathbb{O}_S}(X,Y)$, and hence $\sHom_{\mathbb{O}_S}(X,T'_{Y/S}(\mathscr{M}))$ and $T'_{H/S}(\mathscr{M})$, are endowed with $\mathbb{O}_S$-module structures, and we have:

\begin{corollary}\label{scheme group tangent bundle of sHom O_S-module isomorphism}
If $X,Y$ are $\mathbb{O}_S$-modules, we have an isomorphism of $\mathbb{O}_S$-modules, functorial on $\mathscr{M}$:
\[T'_{\sHom_{\mathbb{O}_S}(X,Y)/S}(\mathscr{M})\stackrel{\sim}{\to} \sHom_{\mathbb{O}_S}(X,T_{Y/S}(\mathscr{M})).\]
\end{corollary}

\begin{definition}
Let $X,M$ be $S$-groups and $X$ acts on $M$ by groups automorphisms. We define the sub-functor $\mathcal{Z}_S^1(X,M)$ of $\sHom_S(X,M)$ as follows: for any $S'\to S$, $\mathcal{Z}_S^1(X,M)(S')$ is the set
\[\{\phi\in\Hom_{S'}(X_{S'},M_{S'}):\text{$\phi(x_1x_2)=\phi(x_1)(x_1\cdot\phi(x_2))$ for any $x_1,x_2\in X(S'')$, $S''\to S'$}\}.\]
The functor $\mathcal{Z}_S^1(X,M)$ is called the \textbf{functor of cross homomorphisms} from $X$ to $M$.
\end{definition}

\begin{remark}
If $M$ is an $\mathbb{O}_S[X]$-module, then $\mathcal{Z}_S^1(X,M)$ coincides with the kernel of the differential
\[d:\sHom_S(X,M)\to\sHom_S(X^2,M)\]
defined in \ref{category cohomology of group standard complex paragraph}. In particular, $\mathcal{Z}_S^1(X,M)$ is an $\mathbb{O}_S$-module in this case.
\end{remark}

Let $u:X\to Y$ be a morphism of $S$-groups. We have seen in \cref{scheme tangent bundle fiber and sHom commute} that we have an isomorphism of $S$-functors, functorial on $\mathscr{M}$:
\begin{equation}\label{scheme group tangent space of morphism isomorphism-1}
T_{\sHom_{S}(X,Y)/S,u}(\mathscr{M})\stackrel{\sim}{\to} \sHom_{Y/S}(X,T_{Y/S}(\mathscr{M})).
\end{equation}
On the other hand, as $Y$ is an $S$-group, we have $T_{Y/S}(\mathscr{M})=\mathfrak{Lie}(Y/S,\mathscr{M})\rtimes Y$, whence an isomorphism
\begin{equation}\label{scheme group tangent space of morphism isomorphism-2}
\begin{aligned}
\sHom_{Y/S}(X,T_{Y/S}(\mathscr{M}))&\stackrel{\sim}{\to}\sHom_{Y/S}(X,\mathfrak{Lie}(Y/S,\mathscr{M})\rtimes Y)\\
&\stackrel{\sim}{\to}\sHom_{Y/S}(X,\mathfrak{Lie}(Y,S,\mathscr{M})_Y)\\
&\stackrel{\sim}{\to}\sHom_S(X,\mathfrak{Lie}(Y,S,\mathscr{M})).
\end{aligned}
\end{equation}

For any $S'\to S$, denote by $u':X'\to Y'$ the morphism induced by $u$ from base change. Consider the $S$-functor defined as follows:
\begin{align*}
\sHom_{(Y/S)\dash\Grp}(X,\mathfrak{Lie}(Y/S,\mathscr{M})\rtimes Y)(S')&=\Hom_{Y'\dash\Grp}(X',(\mathfrak{Lie}(Y/S,\mathscr{M})\rtimes Y)_{S'})\\
&=\Hom_{Y'\dash\Grp}(X',\mathfrak{Lie}(Y'/S',\mathscr{M})\rtimes Y').
\end{align*}
The isomorphism (\ref{scheme group tangent space of morphism isomorphism-1}) then induces an isomorphism
\begin{equation}\label{scheme group tangent space of morphism isomorphism-3}
T_{\sHom_{S\dash\Grp}(X,Y)/S,u}(\mathscr{M})\stackrel{\sim}{\to} \sHom_{(Y/S)\dash\Grp}(X,\mathfrak{Lie}(Y/S,\mathscr{M})\rtimes Y).
\end{equation}

The isomorphism (\ref{scheme group tangent space of morphism isomorphism-2}) can be made explicit as follows: If $\Phi\in\sHom_{Y/S}(X,\mathfrak{Lie}(Y/S,\mathscr{M})\rtimes Y)$, then for any $S''\to S'\to S$ and $x\in X(S'')$, we can write
\[\Phi(S')(x)=\phi(S')(x)\cdot u'(x)\quad\text{where}\quad \phi(S')(x)\in\mathfrak{Lie}(Y'/S',\mathscr{M})(S''),\]
which determines an element $\phi$ of $\sHom_S(X,\mathfrak{Lie}(Y/S,\mathscr{M}))$. On the other hand, the composition of the morphisms
\[\begin{tikzcd}
X\ar[r,"u"]&Y\ar[r,"\Ad"]&\Aut_{S\dash\Grp}(\mathfrak{Lie}(Y/S,\mathscr{M}))
\end{tikzcd}\]
defines an operation of $X$ on $L=\mathfrak{Lie}(Y/S,\mathscr{M})$ by group automorphisms, and we note that $\Phi(S')$ is a group morphism if and only if for any $x_1,x_2\in X(S'')$, we have
\begin{align*}
\phi(S')(x_1x_2)=\phi(S')(x_1)(u(x_1)\phi(S')(x_2)u(x_1)^{-1})=\phi(S')(x_1)(x_1\cdot\phi(S')(x_2)),
\end{align*}
that is, if and only if $\phi\in\mathcal{Z}_S^1(X,\mathfrak{Lie}(Y/S,\mathscr{M}))$. We therefore obtain the following result:

\begin{proposition}\label{scheme group tangent space of morphism isomorphism}
Let $u:X\to Y$ be a morphism of $S$-groups. We have an isomorphism of $S$-functors, functorial on $\mathscr{M}$:
\[T_{\sHom_{S\dash\Grp}(X,Y)/S,u}(\mathscr{M})\stackrel{\sim}{\to}\mathcal{Z}_S^1(X,\mathfrak{Lie}(Y/S,\mathscr{M})).\]
\end{proposition}

Suppose moreover that $Y/S$ satisfies condition (E). Then it folows from \cref{scheme group tangent bundle of sHom isomorphism}, by the same proof of \cref{scheme tangent bundle functor and sHom module structure}, that $\sHom_{S\dash\Grp}(X,Y)/S$ satisfies condition (E). We then have (this also follows from \cref{scheme group tangent space of morphism isomorphism})
\[T_{\sHom_{S\dash\Grp}(X,Y)/S,u}(\mathscr{M}\oplus\mathscr{N})\cong T_{\sHom_{S\dash\Grp}(X,Y)/S,u}(\mathscr{M})\times_ST_{\sHom_{S\dash\Grp}(X,Y)/S,u}(\mathscr{N}).\]
Therefore, $T_{\sHom_{S\dash\Grp}(X,Y)/S,u}(\mathscr{M})$ is endowed, as $\mathcal{Z}_S^1(X,\mathfrak{Lie}(Y/S,\mathscr{M}))$, with an $\mathbb{O}_S$-module structure induced by functoriality on $\mathscr{M}$. We then deduce that the isomorphism \cref{scheme group tangent space of morphism isomorphism} is an isomorphism of $\mathbb{O}_S$-modules in this case:

\begin{proposition}\label{scheme group condition (E) tangent space of morphism module isomorphism}
Let $u:X\to Y$ be a morphism of $S$-groups and suppose that $Y/S$ satisfies condition (E). We have an isomorphism of $\mathbb{O}_S$-modules, functorial on $\mathscr{M}$:
\[T_{\sHom_{S\dash\Grp}(X,Y)/S,u}(\mathscr{M})\stackrel{\sim}{\to}\mathcal{Z}_S^1(X,\mathfrak{Lie}(Y/S,\mathscr{M})).\]
\end{proposition}

Moreover, if $Y/S$ satisfies condition (E), we deduce from \cref{scheme tangent bundle condition (E) morphism extension is iso}, as the proof of \cref{scheme tangent bundle condition (E) fiber of Iso to Hom}, that for any $u\in\Iso_{S\dash\Grp}(X,Y)$ we have an isomorphism functorial on $\mathscr{M}$
\[T_{\sIso_{S\dash\Grp}(X,Y)/S,u}(\mathscr{M})\stackrel{\sim}{\to } T_{\sHom_{S\dash\Grp}(X,Y)/S,u}(\mathscr{M}).\]
We then deduce the following corollaries:

\begin{corollary}\label{scheme group condition (E) tangent space of sIso isomorphism}
Let $u:X\to Y$ be a morphism of $S$-groups. If $Y/S$ satisfies condition (E), we have an isomorphism of $\mathbb{O}_S$-modules, functorial on $\mathscr{M}$:
\[T_{\sIso_{S\dash\Grp}(X,Y)/S,u}(\mathscr{M})\stackrel{\sim}{\to}\mathcal{Z}_S^1(X,\mathfrak{Lie}(Y/S,\mathscr{M})).\]
\end{corollary}
\begin{corollary}\label{scheme group condition (E) tangent space of sAut isomorphism}
Let $X$ be an $S$-group. If $X/S$ satisfies condition (E), we have an isomorphism of $\mathbb{O}_S$-modules, functorial on $\mathscr{M}$:
\[\mathfrak{Lie}(\sAut_{S\dash\Grp}(X)/S,\mathscr{M})\stackrel{\sim}{\to} \mathcal{Z}_S^1(X,\mathfrak{Lie}(X/S,\mathscr{M})).\]
\end{corollary}

If $Y$ is abelian, then the adjoint representation of $Y$ on $L=\mathfrak{Lie}(Y/S,\mathscr{M})$ is trivial, so we have $\mathcal{Z}_S^1(X,L)=\sHom_{S\dash\Grp}(X,L)$. We thus have:
\begin{corollary}\label{scheme group abelian tangent space of sHom isomorphism}
Let $Y$ be an abelian $S$-group. We have an isomorphism of $S$-functors, functorial on $\mathscr{M}$:
\[T_{\sHom_{S\dash\Grp}(X,Y)/S,u}(\mathscr{M})\stackrel{\sim}{\to}\sHom_{S\dash\Grp}(X,\mathfrak{Lie}(Y/S,\mathscr{M})).\]
If $Y/S$ satisfies condition (E), this is an isomorphism of $\mathbb{O}_S$-modules.
\end{corollary}

Consider now the case where $X,Y$ are $\mathbb{O}_S$-modules. Recall that we denote by $T_{Y/S}'(\mathscr{M})$ (resp. $\mathfrak{Lie}'(Y/S,\mathscr{M})$) the functor $T_{Y/S}(\mathscr{M})$ (resp. $\mathfrak{Lie}(Y/S,\mathscr{M})$) endowed with the $\mathbb{O}_S$-module structure induced by that of $Y$. If $Y/S$ satisfies condition (E), we always endow $\mathfrak{Lie}(Y/S,\mathscr{M})$ the $\mathbb{O}_S$-module structure defined by functoriality on $\mathscr{M}$. In this case, the abelian group structures of $\mathfrak{Lie}(Y/S,\mathscr{M})$ and $\mathfrak{Lie}'(Y/S,\mathscr{M})$ coincide (cf. \cref{scheme H-object condition (E) Lie structure induced coincide}), but this is in general not true for the module structures. For any $S'\to S$ and $a\in\mathbb{O}(S')$, we denote by $a\cdot'm$ (resp. $a\cdot m$) the action of $a$ on $m\in\mathfrak{Lie}'(Y/S,\mathscr{M})(S')$ (resp. $m\in\mathfrak{Lie}(Y/S,\mathscr{M})(S')$), and similarly for the actions of $a$ on $T'_{Y/S}(\mathscr{M})$ and $T_{Y/S}(\mathscr{M})$.\par
We have $T'_{Y/S}(\mathscr{M})\cong\mathfrak{Lie}'(Y/S,\mathscr{M})\oplus Y$ as $\mathbb{O}_S$-modules; therefore, we obtain, as in \cref{scheme group abelian tangent space of sHom isomorphism}, that:

\begin{proposition}\label{scheme O_S-module tangent space of sHom isomorphism}
Let $u:X\to Y$ be a morphism of $\mathbb{O}_S$-modules. We have an isomorphism of $S$-functors, functorial on $\mathscr{M}$:
\begin{equation}\label{scheme O_S-module tangent space of sHom isomorphism-1}
T_{\sHom_{\mathbb{O}_S}(X,Y)/S,u}(\mathscr{M})\stackrel{\sim}{\to} \sHom_{\mathbb{O}_S}(X,\mathfrak{Lie}'(Y/S,\mathscr{M})).
\end{equation}
If $Y/S$ satisfies condition (E), then $\sHom_{\mathbb{O}_S}(X,Y)/S$ satisfies condition (E) and (\ref{scheme O_S-module tangent space of sHom isomorphism-1}) is an isomorphism of $\mathbb{O}_S$-modules if we endow both sides the $\mathbb{O}_S$-module structure induced by functoriality on $\mathscr{M}$.
\end{proposition}

\begin{remark}\label{scheme O_S-module tangent bundle of sIso and sHom isomorphism}
Let $u:X\to Y$ be a morphism of $\mathbb{O}_S$-modules. Denote by $\tau_u$ the map which associates to any morphism $\phi:X\to\mathfrak{Lie}'(Y/S,\mathscr{M})$ of $\mathbb{O}_S$-modules the morphism
\[\phi\oplus u:X\to T'_{Y/S}(\mathscr{M})=\mathfrak{Lie}'(Y/S,\mathscr{M})\oplus Y.\]
Then the isomorphism of \cref{scheme O_S-module tangent space of sHom isomorphism} fits into the following diagram, functorial on $\mathscr{M}$:
\begin{equation}\label{scheme O_S-module tangent bundle of sIso and sHom isomorphism-1}
\begin{tikzcd}
T_{\sHom_{\mathbb{O}_S}(X,Y)/S,u}\ar[r,"\sim"]\ar[d,hook]&\sHom_{\mathbb{O}_S}(X,\mathfrak{Lie}'(Y/S,\mathscr{M}))\ar[d,hook,"\tau_u"]\\
T_{\sHom_{\mathbb{O}_S}(X,Y)/S}(\mathscr{M})\ar[r,"\sim"]&\sHom_{\mathbb{O}_S}(X,T'_{Y/S}(\mathscr{M}))
\end{tikzcd}
\end{equation}
Moreover, if $Y/S$ satisfies condition (E), we deduce from \cref{scheme tangent bundle condition (E) morphism extension is iso}, as the proof of \cref{scheme tangent bundle condition (E) fiber of Iso to Hom}, that for any $u\in\Iso_{\mathbb{O}_S}(X,Y)$, we have
\begin{equation}\label{scheme O_S-module tangent bundle of sIso and sHom isomorphism-1}
T_{\sIso_{\mathbb{O}_S}(X,Y)/S}(\mathscr{M})\cong T_{\sHom_{\mathbb{O}_S}(X,Y)/S}(\mathscr{M}).
\end{equation}
\end{remark}

\begin{corollary}\label{scheme O_S-module condition (E) Lie of sAut isomorphism}
Let $X$ be an $\mathbb{O}_S$-module satisfying condition (E) relative to $S$. We have an isomorphism, functorial on $\mathscr{M}$:
\[\mathfrak{Lie}(\sAut_{\mathbb{O}_S}(X)/S,\mathscr{M})\stackrel{\sim}{\to} \sHom_{\mathbb{O_S}}(X,\mathfrak{Lie}'(X/S,\mathscr{M}))\]
which respects the $\mathbb{O}_S$-module structure induced by functoriality on $\mathscr{M}$. In particular, $\sAut_{\mathbb{O}_S}(X)/S$ satisfies condition (E).
\end{corollary}
\begin{proof}
The first assertion follows from (\ref{scheme O_S-module tangent space of sHom isomorphism-1}) and (\ref{scheme O_S-module tangent bundle of sIso and sHom isomorphism-1}). For the second one, as $X/S$ satisfies condition (E), we have an isomorphism of $\mathbb{O}_S$-modules $\mathfrak{Lie}'(X/S,\mathscr{M}\oplus\mathscr{N})\cong \mathfrak{Lie}'(X/S,\mathscr{M})\times_S\mathfrak{Lie}'(X/S,\mathscr{N})$, and hence
\[\mathfrak{Lie}(\sAut_{\mathbb{O}_S}(X)/S,\mathscr{M}\oplus\mathscr{N})\cong\mathfrak{Lie}(\sAut_{\mathbb{O}_S}(X)/S,\mathscr{M})\times_S\mathfrak{Lie}(\sAut_{\mathbb{O}_S}(X)/S,\mathscr{N}).\]
In view of the sequence (\ref{scheme group tangent space and Lie split exact sequence}), this proves that $\sAut_{\mathbb{O}_S}(X)/S$ satisfies condition (E).
\end{proof}

Before going further towards this direction, let us take a closer look at the relations between $Y$, $\mathfrak{Lie}(Y/S)$ and $\mathfrak{Lie}'(Y/S)$. We first notice that (cf. \cref{scheme tangent bundle fiber char by morphism})
\begin{equation}\label{scheme Lie and Lie' of O_S isomorphism to Gamma-1}
\mathfrak{Lie}(\mathbb{O}_S/S,\mathscr{M})=\mathfrak{Lie}'(\mathbb{O}_S/S,\mathscr{M})=\bm{W}(\mathscr{M})
\end{equation}
and that we have a canonical isomorphism
\begin{equation}\label{scheme Lie and Lie' of O_S isomorphism to Gamma-2}
d:\mathbb{O}_S\stackrel{\sim}{\to} \mathfrak{Lie}(\mathbb{O}_S/S).
\end{equation}

Now let $F$ be an $\mathbb{O}_S$-module. For any $S_2\to S_1\to S$, we have a bihomomorphism
\begin{equation}\label{scheme Lie and Lie' relation morphism-1}
F(S_1)\to F(S_2),\quad \mathbb{O}(S_1)\to\mathbb{O}(S_2),
\end{equation}
whence a morphism of $\mathbb{O}(S_2)$-modules
\[F(S_1)\otimes_{\mathbb{O}(S_1)}\mathbb{O}(S_2)\to F(S_2).\]
In particular, for $S_1=S'$ and $S_2=I_{S'}(\mathscr{M})$, we deduce a morphism of $\mathbb{O}(S')$-modules, functorial on $\mathscr{M}$
\[F(S')\otimes_{\mathbb{O}(S')}T_{\mathbb{O}_S/S}(\mathscr{M})(S')\to T'_{F/S}(\mathscr{M})(S').\]
With $S'$ varies, we obtain morphisms of $\mathbb{O}_S$-modules, functorial on $\mathscr{M}$:
\begin{equation}\label{scheme Lie and Lie' relation morphism-2}
F\otimes_{\mathbb{O}_S}T_{\mathbb{O}_S/S}(\mathscr{M})\to T'_{F/S}(\mathscr{M}).
\end{equation}
These morphisms are functorial on $\mathscr{M}$, hence compatible with the projections of tangent bundles onto their bases; they then define morphisms of $\mathbb{O}_S$-modules
\begin{equation}\label{scheme Lie and Lie' relation morphism-3}
F\otimes_{\mathbb{O}_S}\mathfrak{Lie}(\mathbb{O}_S/S,\mathscr{M})\to\mathfrak{Lie}'(F/S,\mathscr{M})
\end{equation}
such that the following diagram is commutative:
\[\begin{tikzcd}
0\ar[r]&F\otimes_{\mathbb{O}_S}\mathfrak{Lie}(\mathbb{O}_S/S,\mathscr{M})\ar[d]\ar[r]&F\otimes_{\mathbb{O}_S}T_{\mathbb{O}_S/S}(\mathscr{M})\ar[d]\ar[r]&F\ar[r]\ar[d,equal]&0\\
0\ar[r]&\mathfrak{Lie}'(F/S,\mathscr{M})\ar[r]&T'_{F/S}(\mathscr{M})\ar[r]&F\ar[r]&0
\end{tikzcd}\]
We can consider the morphisms (\ref{scheme Lie and Lie' relation morphism-3}) as morphisms of abelian $S$-groups:
\begin{equation}\label{scheme Lie and Lie' relation morphism-4}
F\otimes_{\mathbb{O}_S}\mathfrak{Lie}(\mathbb{O}_S/S,\mathscr{M})\to\mathfrak{Lie}(F/S,\mathscr{M}).
\end{equation}
By tensoring $F$ with the isomorphism $d:\mathbb{O}_S\stackrel{\sim}{\to}\mathfrak{Lie}(\mathbb{O}_S/S)$, we then deduce (for $\mathscr{M}=\mathscr{O}_S$) a morphism of abelian $S$-groups
\begin{equation}\label{scheme Lie and Lie' relation morphism-5}
d:F\stackrel{\sim}{\to} F\otimes_{\mathbb{O}_S}\mathfrak{Lie}(\mathbb{O}_S/S)\to\mathfrak{Lie}(F/S).
\end{equation}

\begin{remark}\label{scheme O_S-module morphism F to Lie not module}
If $F/S$ satisfies condition (E), the morphisms (\ref{scheme Lie and Lie' relation morphism-4}) and (\ref{scheme Lie and Lie' relation morphism-5}) are not necessarily morphisms of $\mathbb{O}_S$-modules, if we endow both sides the module structure induced by functoriality on $\mathscr{M}$. For example, let $k$ be a field with characteristic $p>0$, $S=\Spec(k)$, and $F$ be the $\mathbb{O}_S$-module which to any $S$-scheme $T$ associates $F(T)=\Gamma(T,\mathscr{O}_T)$, endowed with the $\mathbb{O}(T)$-module structure obtained by acting a scalar via its $p$-th power, that is, $r\cdot f=r^pf$ for $r\in\mathbb{O}(T)$ and $f\in F(T)$. As an $S$-group, $F$ is isomorphic to $\G_{a,S}$, so $F$ satisfies condition (E) and $\mathfrak{Lie}(F/S)$ is identified with $\mathfrak{Lie}(\G_{a,S}/S)\cong\mathbb{O}_S$. Then, the morphism $d:F\to\mathfrak{Lie}(F/S)$ is, for any $T\to S$, the identity map $F(T)\to\mathbb{O}(T)$; it respects the abelian group structure, but not the $\mathbb{O}_S$-module structure.
\end{remark}

\begin{remark}
We can explicit the morphisms (\ref{scheme Lie and Lie' relation morphism-2}) and (\ref{scheme Lie and Lie' relation morphism-3}) as follows. The morphism $\Theta:F\otimes_{\mathbb{O}_S}T_{\mathbb{O}_S/S}(\mathscr{M})\to T'_{F/S}(\mathscr{M})=\sHom_S(I_S(\mathscr{M}),F)$ is defined so that for any $S'\to S$, $\alpha\in\mathbb{O}(I_{S'}(\mathscr{M}))$, and $f:S'\to F$,
\[\Theta(f\otimes\alpha)=\alpha(\tau_0\circ f)=\alpha\cdot(f\circ\rho)\]
where $\tau_0:F\to T'_{F/S}(\mathscr{M})$ is the zero section and $\rho:I_{S'}(\mathscr{M})\to S'$ is the structural morphism.
\end{remark}

\begin{definition}
We say that $F$ is a \textbf{good $\mathbb{O}_S$-module} if the morphisms $F\otimes_{\mathbb{O}_S}T_{\mathbb{O}_S/S}(\mathscr{M})\to T_{F/S}(\mathscr{M})$ (or equivalently, the morphisms $F\otimes_{\mathbb{O}_S}\mathfrak{Lie}(\mathbb{O}_S/S,\mathscr{M})\to \mathfrak{Lie}(F/S,\mathscr{M})$) are isomorphisms of abelian $S$-groups (so that $F/S$ satisfies condition (E)) and if moreover they respect the $\mathbb{O}_S$-module structures induced by functoriality on $\mathscr{M}$.
\end{definition}

\begin{proposition}\label{scheme O_S-module good Lie and Lie' coincide}
Let $F$ be an $\mathbb{O}_S$-module. Consider the following conditions:
\begin{enumerate}
    \item[(\rmnum{1})] $F$ is a good $\mathbb{O}_S$-module.
    \item[(\rmnum{2})] $F/S$ satisfies condition (E) and $d:F\to\mathfrak{Lie}(F/S)$ is an isomorphism of $\mathbb{O}_S$-modules.
    \item[(\rmnum{3})] $\mathfrak{Lie}(F/S,\mathscr{M})=\mathfrak{Lie}'(F/S,\mathscr{M})$.
\end{enumerate}
Then we have (\rmnum{1})$\Leftrightarrow$(\rmnum{2})$\Rightarrow$(\rmnum{3}).
\end{proposition}
\begin{proof}
The implication (\rmnum{1})$\Rightarrow$(\rmnum{2}) follows from definition. To see that (\rmnum{2})$\Rightarrow$(\rmnum{2}), it suffices to show that the morphisms of abelian $S$-groups
\[F\otimes_{\mathbb{O}_S}\mathfrak{Lie}(\mathbb{O}_S/S,\mathscr{M})\stackrel{\sim}{\to} \mathfrak{Lie}(F/S,\mathscr{M})\]
are isomorphisms of $\mathbb{O}_S$-modules. As $F/S$ satisfies condition (E), the two members transform finite direct sums of copies of $\mathscr{O}_S$ into finite products of abelian $S$-groups. We are then reduced to the case where $\mathscr{M}=\mathscr{O}_S$, which follows by the hypothesis.\par
Finally, (\rmnum{1})$\Rightarrow$(\rmnum{3}) follows from the definition and the fact that the isomorphisms
\[F\otimes_{\mathbb{O}_S}\mathfrak{Lie}(\mathbb{O}_S/S,\mathscr{M})\stackrel{\sim}{\to}\mathfrak{Lie}'(F/S,\mathscr{M})\]
of (\ref{scheme Lie and Lie' relation morphism-3}) is an isomorphism of $\mathbb{O}_S$-modules.
\end{proof}

\begin{example}\label{scheme O_S-module Gamma is good}
For any quasi-coherent $\mathscr{O}_S$-module $\mathscr{E}$, the $\mathbb{O}_S$-module $\mathbf{W}(\mathscr{E})$ and $\mathbf{V}(\mathscr{E})$ are good. In fact, for any $f:S'\to S$, the morphisms
\begin{align*}
\mathbf{W}(\mathscr{E})(S')\otimes_{\mathbb{O}(S')}\mathbb{O}(I_{S'}(\mathscr{M}))&\to T_{\mathbf{W}(\mathscr{E})/S}(\mathscr{M})(S')\\
\mathbf{V}(\mathscr{E})(S')\otimes_{\mathbb{O}(S')}\mathbb{O}(I_{S'}(\mathscr{M}))&\to T_{\mathbf{V}(\mathscr{E})/S}(\mathscr{M})(S')
\end{align*}
correspond, respectively, to morphisms
\begin{align*}
\Gamma(S',f^*(\mathscr{E}))\otimes_{\mathbb{O}(S')}\Gamma(S',\mathscr{D}_{\mathscr{O}_{S'}}(\mathscr{M}))&\to\Gamma(S',f^*(\mathscr{E})\otimes_{\mathscr{O}_{S'}}\mathscr{D}_{\mathscr{O}_{S'}}(\mathscr{M})),\\
\Hom_{\mathscr{O}_{S'}}(f^*(\mathscr{E}),\mathscr{O}_{S'})\otimes_{\mathbb{O}(S')}\Gamma(S',\mathscr{D}_{\mathscr{O}_{S'}}(\mathscr{M}))&\to \Hom_{\mathscr{O}_{S'}}(f^*(\mathscr{E}),\mathscr{D}_{\mathscr{O}_{S'}}(\mathscr{M}));
\end{align*}
which are both isomorphisms since $\mathscr{D}_{\mathscr{O}_{S'}}(\mathscr{M})$ is isomorphic, as $\mathscr{O}_{S'}$-module, to a finite direct sum of copies of $\mathscr{O}_{S'}$ (recall that $\mathscr{M}$ is assumed to be free). 
\end{example}

\begin{proposition}\label{scheme O_S-module good Lie of sAut isomorphism}
Let $F$ be a good $\mathbb{O}_S$-module. Then $\sAut_{\mathbb{O}_S}(F)/S$ satisfies condition (E) and we have a isomorphism (functorial on $\mathscr{M}$)
\[\mathfrak{Lie}(\sAut_{\mathbb{O}_S}(F)/S,\mathscr{M})\stackrel{\sim}{\to} \sHom_{\mathbb{O}_S}(F,\mathfrak{Lie}(F/S,\mathscr{M}))\]
which respects the $\mathbb{O}_S$ induced by the functoriality on $\mathscr{M}$. In particular, we have an isomorphism of $\mathbb{O}_S$-modules
\[\mathfrak{Lie}(\sAut_{\mathbb{O}_S}(F)/S)\stackrel{\sim}{\to} \sEnd_{\mathbb{O}_S}(F).\]
Morover, $\sEnd_{\mathbb{O}_S}(F)$ is a good $\mathbb{O}_S$-module.
\end{proposition}
\begin{proof}
In fact, by \cref{scheme O_S-module good Lie and Lie' coincide}, $F/S$ satisfies condition (E) and
\begin{equation}\label{scheme O_S-module good Lie of sAut isomorphism-1}
\mathfrak{Lie}(F/S,\mathscr{M})=\mathfrak{Lie}'(F/S,\mathscr{M})\cong F\otimes_{\mathbb{O}_S}\mathfrak{Lie}(\mathbb{O}_S/S,\mathscr{M}).
\end{equation}
The first assertion then follows from \cref{scheme O_S-module condition (E) Lie of sAut isomorphism}. Put $E=\sEnd_{\mathbb{O}_S}(F)$; by (\ref{scheme O_S-module good Lie of sAut isomorphism-1}) and (\cite{SGA3-1} remarque 4.3.5), we have the following commutative diagram of abelian groups
\[\begin{tikzcd}
\sEnd_{\mathbb{O}_S}(F)\otimes_{\mathbb{O}_S}\mathfrak{Lie}(\mathbb{O}_S/S,\mathscr{M})\ar[d,equal]\ar[r,"d_E"]&\mathfrak{Lie}(\sEnd_{\mathbb{O}_S}(F)/S,\mathscr{M})\\
\sHom_{\mathbb{O}_S}(F,F\otimes_{\mathbb{O}_S}\mathfrak{Lie}(\mathbb{O}_S/S,\mathscr{M}))\ar[r,"d_F"]&\sHom_{\mathbb{O}_S}(F,\mathfrak{Lie}(\sEnd_{\mathbb{O}_S}(F)/S,\mathscr{M}))\ar[u,"\sim","(*)"']
\end{tikzcd}\]
where $d_F$ and ($*$) are isomorphisms of $\mathbb{O}_S$-modules; therefore, so is $d_E$, and this proves the proposition.
\end{proof}

\begin{remark}\label{scheme tangent space of Aut and infinitesimal endomorphism}
We can provide an explicit description of the isomorphism in \cref{scheme O_S-module good Lie of sAut isomorphism}. For this, put $\mathscr{O}_{I_S}=\mathscr{O}_S\oplus t\mathscr{O}_S$ (with $t^2=0$) and let $F$ be a good $\mathbb{O}_S$-module. Then, for any $S'\to S$, the morphism\footnote{The equality on the right follows from the exact sequence (\ref{scheme group tangent space and Lie split exact sequence}).} (which is the identity on $F(S')$)
\[F(S')\oplus tF(S')=F(S')\otimes_{\mathbb{O}(S')}\mathbb{O}(I_{S'})\to F(I_{S'})=F(S')\oplus\mathfrak{Lie}(F/S)(S')\]
induces an isomorphism of $\mathbb{O}(S')$-modules $tF(S')\cong\mathfrak{Lie}(F/S)(S')$. Since this is functorial over $S'$, we then obtain an isomorphism $\mathfrak{Lie}(F/S)\cong tF$. For any $S'\to S$, we have, by \cref{scheme O_S-module good Lie of sAut isomorphism}, a commutative diagram
\[\begin{tikzcd}
\End_{\mathbb{O}_{S'}}(F_{S'})\ar[r,"\sim"]&\Hom_{\mathbb{O}_{S'}}(F_{S'},tF_{S'})\ar[r,"\sim"]\ar[d,hook]&\mathfrak{Lie}(\sAut_{\mathbb{O}_S}(F)/S)(S')\ar[d,hook]\\
&\Aut_{\mathbb{O}(I_{S'})}(F_{I_{S'}})\ar[r,equal]&T_{\sAut_{\mathbb{O}_S}(F)/S}(S')
\end{tikzcd}\]
and we deduce from the commutative diagram (\ref{scheme O_S-module tangent bundle of sIso and sHom isomorphism}) (take $u=\id$ in the diagram) that any $X\in\End_{\mathbb{O}_{S'}}(F_{S'})$ corresponds to the element $\id+tX$ of $\Aut_{\mathbb{O}_{I_{S'}}}(F_{I_{S'}})$.
\end{remark}

We say that the $S$-group $G$ is \textbf{good} if $G/S$ satisfies condition (E) and $\mathfrak{Lie}(G/S)$ is a good $\mathbb{O}_S$-module. Note that if $F$ is a good $\mathbb{O}_S$-module, it is also a good $S$-groups: in fact, $F/S$ satisfies condition (E) and $\mathfrak{Lie}(F/S)\cong F$ (cf. \cref{scheme O_S-module good Lie and Lie' coincide}~(\rmnum{2})) is a good $\mathbb{O}_S$-module.

\begin{example}\label{scheme group representable is good}
If $G$ is representable, then it is good. In fact, $G/S$ satisfies condition (E) and $\mathfrak{Lie}(G/S)$ is of the form $\V(\mathscr{E})$ by \cref{scheme tangent bundle representable if}, hence good by \cref{scheme O_S-module Gamma is good}.
\end{example}

\begin{lemma}\label{scheme group condition (E) Lie of Lie module morphism}
Let $G$ be an $S$-group such that $G/S$ satisfies condition (E), and $F=\mathfrak{Lie}(G/S)$. Then $F/S$ satisfies condition (E) and the abelian group morphism $d:F\to\mathfrak{Lie}(F/S)$ respects the $\mathbb{O}_S$-module structure. Therefore, $G$ is good if and only if $d:F\to\mathfrak{Lie}(F/S)$ is bijective.
\end{lemma}
\begin{proof}

\end{proof}

\begin{theorem}\label{scheme O_S-module good Aut is good}
If $F$ is a good $\mathbb{O}_S$-module, the $S$-group $\sAut_{\mathbb{O}_S}(F)$ is good.
\end{theorem}
\begin{proof}
By \cref{scheme O_S-module good Lie of sAut isomorphism}, $\sAut_{\mathbb{O}_S}(F)/S$ satisfies condition (E) and $\mathfrak{Lie}(\sAut_{\mathbb{O}_S}(F)/S)\cong\sEnd_{\mathbb{O}_S}(F)$ is a good $\mathbb{O}_S$-module.
\end{proof}

\begin{example}\label{scheme O_S-module p-twisted G_m not good}
Let $F$ be the $\mathbb{O}_S$-module defined in \cref{scheme O_S-module morphism F to Lie not module}. Then, the canonical morphism $d:F\to\mathfrak{Lie}(F/S)$ is, for any $T\to S$, the identity morphism $F(T)\to\mathbb{O}(T)$. It respects the abelian group structure, but not the module structure, so $F$ is not good.
\end{example}

Let $G$ be an $S$-group and $F$ be a good $\mathbb{O}_S$-module. Suppose that we are given a linear representation of $G$ on $F$, that is, an $S$-group morphism
\[\rho:G\to\sAut_{\mathbb{O}_S}(F).\]
If $G/S$ satisfies condition (E), we deduce from \cref{scheme O_S-module good Lie of sAut isomorphism} and \cref{scheme tangent bundle condition (E) functorial on X} a morphism of $\mathbb{O}_S$-modules
\[d\rho:\mathfrak{Lie}(G/S)\to\mathfrak{Lie}(\sAut_{\mathbb{O}_S}(F)/S)\cong\sEnd_{\mathbb{O}_S}(F).\]
Moreover, put $\mathscr{O}_{I_S}=\mathscr{O}_S\oplus t\mathscr{O}_S$ (with $t^2=0$), we then deduce from \cref{scheme tangent space of Aut and infinitesimal endomorphism} that, for $S'\to S$ and $X\in\mathfrak{Lie}(G/S)(S')\sub T_{G/S}(S')=G(I_{S'})$, we have the following equality in $\Aut_{\mathbb{O}_{I_{S'}}}(F_{I_{S'}})$:
\begin{equation}\label{scheme group derived morphism of linear representation equality}
\rho(X)=\id+t\,d\rho(X),
\end{equation}
that is, for any $S''\to I_{S'}$ and $f\in F(S')$, we have $\rho(X)(f)=f+td\rho(X)(f)$ in $F(S'')$.\par
As a special case, if $G$ is a good $S$-group, then $\mathfrak{Lie}(G/S)$ is a good $\mathbb{O}_S$-module, and we have a morphism of $S$-groups
\[\Ad:G\to\sAut_{\mathbb{O}_S}(\mathfrak{Lie}(G/S)),\]
from which we then deduce a morphism of $\mathbb{O}_S$-modules
\[\ad:\mathfrak{Lie}(G/S)\to\sEnd_{\mathbb{O}_S}(\mathfrak{Lie}(G/S)),\]
or equivalently, an $\mathbb{O}_S$-bilinear morphism
\[\mathfrak{Lie}(G/S)\times_S\mathfrak{Lie}(G/S)\to\mathfrak{Lie}(G/S),\quad (x,y)\mapsto [x,y]:=\ad(x)\cdot y\]
where $x,y$ denote elements of $\mathfrak{Lie}(G/S)(S')=\mathfrak{Lie}(G_{S'}/S')(S')$. If $G$ is commutative, then $\Ad$ is trivial, and we have $[x,y]=0$.

\begin{remark}\label{scheme group Lie bracket definition by diagram}
We can give an equivalent definition of the bracket: it suffices to do this for $x,y\in\mathfrak{Lie}(G/S)(S)$. We then note that there is a canonical isomorphism $I_S\times_SI_S\cong I_{I_S}$; to avoid confusions, we denote by $I$ and $I'$ the two copies of $I_S$ and put $\mathscr{O}_I=\mathscr{O}_S[t]$, $\mathscr{O}_{I'}=\mathscr{O}_S[t']$, where $t^2=t'^2=0$. We then have a commutative diagram
\[\begin{tikzcd}
I\times I'\ar[d]\ar[r]&I'\ar[d]\\
I\ar[r]&S
\end{tikzcd}\]
(the two arrows from $I\times I'$ identifying it as the dual number scheme over $I$ or over $I'$), which gives a commutative diagram of abelian groups ($L=\mathfrak{Lie}(G/S)$) by (\ref{scheme group tangent space and Lie split exact sequence}):
\begin{equation}
\begin{tikzcd}[row sep=6mm,column sep=6mm]
&&1\ar[d]&1\ar[d]&\\
&&L(I)\ar[d]\ar[r]&L(S)\ar[d]\ar[r]&1\\
1\ar[r]&L(I')\ar[r]\ar[d]&G(I\times I')\ar[r]\ar[d]&G(I')\ar[r]\ar[d]&1\\
1\ar[r]&L(S)\ar[r]\ar[d]&G(I)\ar[r]\ar[d]&G(S)\ar[r]\ar[d]&1\\
&1&1&1
\end{tikzcd}
\end{equation}
The final piece of this diagram is none other than $\mathfrak{Lie}(L/S)(S)$. If $G$ is good, this is isomorphic to $L(S)$ and we then have the following diagram, where the rows and columns are exact sequences of groups and in view of the identification $L(I)=L(S)\oplus tL(S)$ (resp. $L(I')=L(S)\oplus t'L(S)$), the injection $L(S)\hookrightarrow L(I)$ (resp. $L(S)\hookrightarrow L(I')$) is given by $u\mapsto tu$ (resp. $u\mapsto t'u$):
\begin{equation}
\begin{tikzcd}[row sep=6mm,column sep=6mm]
L(S)\ar[r,"t"]\ar[d,"t'"]&L(I)\ar[r]\ar[d]&L(S)\ar[d]\\
L(I')\ar[r]\ar[d]&G(I\times I')\ar[r]\ar[d]&G(I')\ar[d]\\
L(S)\ar[r]&G(I)\ar[r]&G(S)
\end{tikzcd}
\end{equation}

Now in this diagram, if we take two elements $x$ and $y$ in $L(S)$ and choose arbitrarily element $\tilde{x}\in L(I)$ (resp. $\tilde{y}\in L(I')$) which maps to $x$ (resp. to $y$), then the commutator $\tilde{x}\tilde{y}\tilde{x}^{-1}\tilde{y}^{-1}$ in $G(I\times I')$ does not depend on the choice of $\tilde{x}$ and $\tilde{y}$, and it is the image of an element $z\in L(S)$. In fact, if we identify $x$ with its image under the canonical section $L(S)\to L(I)$ (and similarly for $y$), then $\tilde{x}=xu$ and $\tilde{y}=yv$, with $u,v\in L(S)=L(I)\cap L(I')$, and since $L(I)$, $L(I')$ are abelian, we have
\[\tilde{x}\tilde{y}\tilde{x}^{-1}\tilde{y}^{-1}=xuyvu^{-1}x^{-1}v^{-1}y^{-1}=xuyu^{-1}vx^{-1}v^{-1}y^{-1}=xyx^{-1}y^{-1}.\]
Moreover, this element is send to the unit element of $G(I)$ and of $G(I')$, hence comes from an element $z\in L(S)$. Finally, consider $y$ (resp. $x$) as element of $L(I')$ (resp. $L(S)\sub G(I')$), by (\ref{scheme group derived morphism of linear representation equality}) we have
\[xyx^{-1}=\Ad(x)(y)=(\id+t'\ad(x))(y)=y+t'[x,y],\]
so the element $xyx^{-1}y^{-1}$ of $L(I')$ is the iamge of $z=[x,y]\in L(S)$.\par
From the above construction, we see that the bracket has the following properties:
\begin{enumerate}
    \item[(\rmnum{1})] The bracket is functorial on $G$: more precisely, $G\mapsto\mathfrak{Lie}(G/S)$ is a functor from the category of good $S$-groups to the category of good $\mathbb{O}_S$-modules endowed with an $\mathbb{O}_S$-bilinear composition law.
    \item[(\rmnum{2})] We have $[x,y]+[y,x]=0$. In fact, the diagram is symmetric, and by exchanging $x$ and $y$ we are considering the element $\tilde{y}\tilde{x}\tilde{y}^{-1}\tilde{x}^{-1}$, which is the inverse of $\tilde{x}\tilde{y}\tilde{x}^{-1}\tilde{y}^{-1}$.
\end{enumerate}
\end{remark}

\begin{proposition}\label{scheme O_S-module good Lie bracket expression}
Let $F$ be a good $\mathbb{O}_S$-module. Via the identification $\mathfrak{Lie}(\sAut_{\mathbb{O}_S}(F)/S)=\sEnd_{\mathbb{O}_S}(F)$, we have
\[\Ad(g)\cdot Y=g\circ Y\circ g^{-1},\quad [X,Y]=X\circ Y-Y\circ X,\]
for any $S'\to S$, $g\in\Aut_{\mathbb{O}_S}(F_{S'})$ and $X,Y\in\mathfrak{Lie}(\sAut_{\mathbb{O}_S}(F)/S)(S')=\End_{\mathbb{O}_S}(F_{S'})$.
\end{proposition}
\begin{proof}
By base change, we can assume that $S'=S$, which makes it possible to simplify the notations. Put $I=I_S$ and $\mathscr{O}_I=\mathscr{O}_S[t]$ (with $t^2=0$). Recall that the inclusion $i:\End_{\mathbb{O}_S}(F)\hookrightarrow\sAut_{\mathbb{O}_I}(F_I)$ sends $Y$ to $\id+tY$, so by the definition of $\Ad(g)$, we have
\[\id+t\Ad(g)(Y)=g\circ(\id+tY)\circ g^{-1}=\id+t(g\circ Y\circ g^{-1}),\]
whence $\Ad(g)(Y)=g\circ Y\circ g^{-1}$.\par
Let $I'$ be a second copy of $I_S$, and put $\mathscr{O}_{I'}=\mathscr{O}_S[t']$ (with $t'^2=0$). Apply the result of \cref{scheme group Lie bracket definition by diagram} to $G=\sAut_{\mathbb{O}_S}(F)$ and $L=\mathfrak{Lie}(G/S)=\sAut_{\mathbb{O}_S}(F)$, where we identify $X$ with its image under the canonical section $L(S)\hookrightarrow L(I)$; its image in $G(I\times I')$ is then $\id+t'X$, hence the inverse is $\id-t'X$. Similarly, $Y$ is send to $\id+tY$, so the inverse is $\id-tY$. Then the commutator
\[(\id+t'X)\circ(\id+tY)\circ(\id-t'X)\circ(\id-tY)=\id+tt'(X\circ Y-Y\circ X)\]
is the image of $Z=X\circ Y-Y\circ X$ in $G(I\times I')$ (in fact, $Z$ is send to $tZ\in L(I)$, hence to $\id+tt'Z\in G(I\times I')$). By \cref{scheme group Lie bracket definition by diagram}, we conclude that $[X,Y]=X\circ Y-Y\circ X$.
\end{proof}

\begin{corollary}\label{scheme O_S-module good Jacobi identity}
Let $G$ be a good $S$-group and $x,y,z\in\mathfrak{Lie}(G/S)(S')$. We have
\[[x,[y,z]]+[y,[z,x]]+[z,[x,y]]=0.\]
\end{corollary}
\begin{proof}
In fact, as $G$ is good, $\mathfrak{Lie}(G/S)$ is a good $\mathbb{O}_S$-module and hence, by \cref{scheme O_S-module good Aut is good}, $\sAut_{\mathbb{O}_S}(\mathfrak{Lie}(G/S))$ is a good $S$-group. Then, the morphism of $S$-groups
\[\Ad:G\to\sAut_{\mathbb{O}_S}(\mathfrak{Lie}(G/S))\]
gives by functoriality $\ad([x,y])=[\ad(x),\ad(y)]$. Combined with \cref{scheme O_S-module good Lie bracket expression}, this shows that
\[\ad([x,y])=[\ad(x),\ad(y)]=\ad(x)\circ\ad(y)-\ad(y)\circ\ad(x),\]
which implies the Jacobi identity after applied to an element $z$.
\end{proof}

\begin{corollary}\label{scheme O_S-module good representation induced}
Let $G$ be a good $S$-group linearly acted on a good $\mathbb{O}_S$-module $F$ (i.e. $F$ is an $\mathbb{O}_S[G]$-module, $G$ and $F$ being good). Then the linear map $d\rho:\mathfrak{Lie}(G/S)\to\sEnd_{\mathbb{O}_S}(F)$ is a representation, that is, we have
\[d\rho([x,y])=d\rho(x)\circ d\rho(y)-d\rho(y)\circ d\rho(x).\]
\end{corollary}
\begin{proof}
This follows from the functoriality of bracket and \cref{scheme O_S-module good Lie bracket expression}.
\end{proof}

To any good $S$-group (for example representable), we have associated a good $\mathbb{O}_S$-module $\mathfrak{Lie}(G/S)$ endowed functorially a bilinear map verifying
\[[x,y]+[y,x]=0,\quad [x,[y,z]]+[y,[z,x]]+[z,[x,y]]=0.\]
We therefore say that $\mathfrak{Lie}(G/S)$, endowed with this structure, is the \textbf{quasi-Lie algebra} of $G$ over $S$. For any linear representation of $G$ over a good $\mathbb{O}_S$-module $F$, we can associate a representation of the quasi-Lie algebra $\mathfrak{Lie}(G/S)$. In particular, the adjoint representation of $G$ is associated to the adjoint representation of the quasi-Lie algebra.\par

A group functor $G$ over $S$ is called \textbf{very good} if it is good and $\mathfrak{Lie}(G/S)$ is a Lie algebra over $\mathbb{O}_S$ (that is, if we have the identity $[x,x]=0$). The following $S$-groups are very good: $\sAut_{\mathbb{O}_S}(F)$ for any good $\mathbb{O}_S$-module $F$ (cf. \cref{scheme O_S-module good Lie bracket expression} and \cref{scheme O_S-module good Jacobi identity}), any representable group (see below), any good $S$-group admitting a monomorphism into a very good $S$-group (cf. \cref{scheme tangent bundle functorial on X}), for example any good subgroup of a very good representable group, or any good $S$-group admitting a faithful representation over a good $\mathbb{O}_S$-module, for example any good $S$-group such that $\Ad$ is faithful.\par

Now suppose that $G$ is a group scheme. By \cref{scheme group Lie and right invariant I_S-endomorphism}, $\mathfrak{Lie}(G/S)(S)$ is identified with right invariant infinitesimal automorphisms of $G$, hence by (\ref{scheme group representable tangent bundle section and derivation}) with derivations of $\mathscr{O}_G$ over $\mathscr{O}_S$ invariant under right translations. Moreover, this identification respects the module structure and is an \textit{anti-isomorphism} of Lie algebras: put $\mathscr{O}_I=\mathscr{O}_S[t]$ and $\mathscr{O}_{I'}=\mathscr{O}_S[t']$ and let $x\in L(I)$ and $y\in L(I')$\footnote{As before, we write $L=\mathfrak{Lie}(G/S)$.}. The left translation $\lambda_x$ (resp. $\lambda_y$) is an $S$-automorphism of $G_{I\times I'}$ which induces the identity on $G_{I'}$ (resp. $G_I$) and which corresponds to an $\mathscr{O}_S$-automorphism
\[u=\id+td_x,\quad\quad (\text{resp.}\quad v=\id+t'd_y)\]
of $\mathscr{O}_{G_{I\times I'}}=\mathscr{O}_G[t,t']/(t^2,t'^2)$, where $d_x,d_y$ are $\mathscr{O}_S$-derivations of $\mathscr{O}_G$ invariant under right translations. As the correspondence of $S$-automorphisms of $G_{I\times I'}$ and $\mathscr{O}_S$-automorphisms of $\mathscr{O}_{G_{I\times I'}}$ is contravariant, $\lambda_x\lambda_y\lambda_x^{-1}\lambda_y^{-1}$ corresponds to $v^{-1}u^{-1}vu=\id+tt'(d_yd_x-d_xd_y)$. We then deduce from \cref{scheme group Lie bracket definition by diagram} that the map $x\mapsto-d_x$ is an isomorphism of Lie algebras. The preceding argument is valid for $\mathfrak{Lie}(G/S)(S')=\mathfrak{Lie}(G_{S'}/S')(S')$ for any $S'\to S$, so we recover the following classical definition:

\begin{proposition}\label{scheme group scheme Lie algebra isomorphic to derivation}
Via the isomorphism $x\mapsto -d_x$, $\mathfrak{Lie}(G/S)$ is identified with the functor which associates any $S'\to S$ to the Lie algebra of derivations of $G_{S'}$ over $S'$ invariant under right translations.
\end{proposition}

As we have seen in \cref{scheme group representable is good} that any representable group is good, we conclude the following corollary:

\begin{corollary}\label{scheme group representable is very good}
Any representable grop is very good.
\end{corollary}

Let $e:S\to G$ be the unit section of $G$. Put $\omega_{G/S}^1=e^*(\Omega_{G/S}^1)$ and recall that (cf. \cref{scheme tangent bundle representable if}) $\mathfrak{Lie}(G/S)$ is represented by the vector bundle $\mathrm{Lie}(G/S)=\V(\omega_{G/S}^1)$. We then have assocaited functorially to any $S$-group scheme $G$ a vector bundle $\V(\omega_{G/S}^1)$ over $S$, which represents the functor $\mathfrak{Lie}(G/S)$, hence is endowed with the structure of a Lie algebra $S$-scheme over $\mathbb{O}_S$. Moreover, this construction commutes with base change and finite products.

\begin{remark}\label{scheme group omega_G/S differential module prop}
Let $\pi:G\to S$ be the structural morphism. The $\mathscr{O}_G$-module $\Omega_{G/S}^1$ is evidently $(G\times_SG)$-equivariant and hence, by (\cite{SGA3-1} \Rmnum{1}, 6.8.1), we have $\Omega_{G/S}^1\cong\pi^*(\omega_{G/S}^1)$. It follows for example that $\Omega_{G/S}^1$ is locally free (resp. locally free of finite rank) if $\omega_{G/S}^1$ is, which is in particular the case if $S$ is the spectrum of a field (resp. if $S$ is the spectrum of a field and $G$ is locally of finite type over $S$). Moreover, by (\cite{SGA3-1} \Rmnum{1}, 6.8.2), $\omega_{G/S}^1$ is endowed with a canonical $\mathbb{O}_S[G]$-module structure, which induces over $\V(\omega_{G/S}^1)=\mathrm{Lie}(G/S)$ the adjoint representation.\par
On the other hand, $e$ is an immersion, and is a closed immersion if $G$ is separated over $S$ (cf. \cref{scheme morphism cartesian square with diagonal morphism}). Hence $\omega_{G/S}^1$ is identified with $\mathscr{I}/\mathscr{I}^2$, where $\mathscr{I}$ is the quasi-coherent ideal defining $e(S)$ in an open subset $U$ of $G$ in which $e(G)$ is closed (if $G$ is separated over $S$, we can put $U=G$, and if $G=\Spec(\mathscr{A}(G))$ is affine over $S$, $\mathscr{I}$ is none other than the augmented ideal of $\mathscr{A}(G)$, i.e. the kernel of $e^{\sharp}:\mathscr{A}(G)\to\mathscr{O}_S$).
\end{remark}

\begin{remark}\label{scheme group omega_G/S invariant sheaf of differential}
We deduce from the isomorphism $\Omega_{G/S}^1\cong\pi^*(\omega_{G/S}^1)$ that the $\mathscr{O}_S$-module $\omega_{G/S}^1$ is identified with the sheaf $\pi_*^G(\Omega_{G/S}^1)$ of right invariant differentials of $G$ over $S$, that is, the sheaf whose sections over an open subset $U$ of $S$ are the sections of $\Omega_{G/S}^1$ over $\pi^{-1}(U)$ which are invariant under right translations (cf. (\cite{SGA3-1} \Rmnum{1}, 6.8.3)).
\end{remark}

We denote by $\sLie(G/S)$ the sheaf of sections of the vector bundle $\mathrm{Lie}(G/S)\to S$, which is the $\mathscr{O}_S$-module $(\omega_{G/S}^1)^{\vee}=\sHom_{\mathscr{O}_S}(\omega_{G/S}^1,\mathscr{O}_S)$ dual to $\omega_{G/S}^1$ (cf. \cref{scheme qcoh associated vector bundle def}). It is endowed with a Lie algebra structure over $\mathscr{O}_S$. As this construction does not commute with base change (in general), the Lie algebra structure on $\sLie(G/S)$ does not allow us to reconstruct the $S$-scheme structure on the $\mathbb{O}_S$-Lie algebra $\mathrm{Lie}(G/S)$. However, we have:

\begin{proposition}\label{scheme group omega_G/S locally free construct Lie}
Suppose that $\omega_{G/S}^1$ is locally free of finite type. Then $\sLie(G/S)^{\vee}\cong(\omega_{G/S})^{\vee\vee}\cong\omega_{G/S}^1$ and hence
\[\mathrm{Lie}(G/S)=\V(\omega_{G/S}^1)=\V(\sLie(G/S)^{\vee})=\mathbf{W}(\sLie(G/S)).\]
\end{proposition}
\begin{proof}
In fact, $\omega_{G/S}^1$ is reflexive if it is locally free of finite type, and the assertion follows from \cref{scheme Gamma module functor isomorphic if locally free}.
\end{proof}

Finally, let $G\to H$ be a monomorphism of group functors. Then $\mathfrak{Lie}(G/S)\to\mathfrak{Lie}(H/S)$ is also a monomorphism (cf. \cref{scheme tangent bundle functorial on X}). As any monomorphism of vector bundles is a closed immersion\footnote{Let $f:\mathscr{M}\to\mathscr{N}$ be a morphism of $\mathscr{O}_S$-modules and $\mathscr{P}=\coker f$. If $\V(\mathscr{N})\to\V(\mathscr{M})$ is a monomorphism, the surjective morphism $\bm{S}(\mathscr{N})\to\bm{S}(\mathscr{P})$ factors through $\mathscr{O}_S$, hence $\mathscr{P}=0$.}, we obtain:

\begin{corollary}
Let $G\to H$ be a monomorphism of $S$-groups.
\begin{enumerate}
    \item[(\rmnum{1})] $\mathrm{Lie}(G/S)\to\mathrm{Lie}(H/S)$ is a closed immersion and hence $\omega_{H/S}^1\to\omega_{G/S}^1$ is an epimorphism.
    \item[(\rmnum{2})] If $\omega_{G/S}^1$ is locally free of finite type, then the corresponding morphism $\sLie(G/S)\to\sLie(H/S)$ is an isomorphism from $\sLie(G/S)$ to a submodule of $\sLie(H/S)$ which is locally a direct factor. 
\end{enumerate}
\end{corollary}

\begin{example}\label{scheme O_S-module alpha_p not good}
Let $S=\Spec(k)$ with $k$ a field of characteristic $p>0$. Let $\bm{\alpha}_{p,S}$ be the $S$-functor which to any $S$-scheme $T$ associates
\[\bm{\alpha}_{p,S}(T)=\{x\in\mathscr{O}(T):x^p=0\}.\]
Then $\bm{\alpha}_{p,S}$ is represented by $\Spec(\mathscr{O}_S[X]/(X^p))$, and hence is a very good $S$-group. It is also endowed with an $\mathbb{O}_S$-module structure, which is not very good, because the canonical morphism
\[\bm{\alpha}_{p,S}\to\mathfrak{Lie}(\bm{\alpha}_{p,S}/S)=\G_{a,S}\]
is not bijective\footnote{This can be deduced from the exact sequence (\ref{scheme group tangent space and Lie split exact sequence}), or we can also note that $\omega_{G/k}^1=k[X]$.}.
\end{example}

\begin{example}\label{scheme group condition (E) but not good}
Let $\mathrm{Nil}$ be the $\Z$-functor defined as follows: for any scheme $S$, $\mathrm{Nil}(S)$ is the nilideal of $\mathscr{O}_S$:
\[\Nil(S)=\{x\in\mathscr{O}(S):\text{there exists $n\in\N$ such that $x^n=0$}\}.\]
Let $\Nil^2$, $\mathbb{O}_{\red}$ and $F$ be the $\Z$-functors in groups which associate to any scheme $S$, respectively, the ideal $\Nil(S)^2$ and \[\mathbb{O}_{\red}(S)=\mathscr{O}(S)/\Nil(S),\quad F(S)=\mathscr{O}(S)/\Nil(S)^2.\]
It is easily seen that $\mathfrak{Lie}(\mathbb{O}_{\red}/\Z)=0$, hence the $\mathbb{O}_{\Z}$-module $\mathbb{O}_{\red}$ is not good (although it is a good $\Z$-group). If $M,N$ are free $\Z$-modules of finite rank, we have
\[\Nil^2(I_S(M\oplus N))=\Nil^2(S)\oplus\Nil^2(S)\otimes_{\Z}M \oplus\Nil(S)\otimes_{\Z}N\]
and hence
\[F(I_S(M\oplus N))=F(S)\oplus\mathbb{O}_{\red}(S)\otimes_{\Z}M\oplus\mathbb{O}_{\red}(S)\otimes_{\Z}N.\]
We then deduce, on the one hand, that the $\Z$-functor $F$ satisfies condition (E) and, on the other hand, that $\mathfrak{Lie}(F/\Z)=\mathbb{O}_{\red}$ (cf. (\ref{scheme group tangent space and Lie split exact sequence})); as the latter is not a good $\mathbb{O}_\Z$-module, this shows that $F$ is a $\Z$-group which satisfies condition (E) but is not good.
\end{example}

\subsection{Calculation of some Lie algebras}
\paragraph{Lie algebras of diagonalizable groups}
Let $G=D_S(M)$ be a diagonalizable group over $S$ (cf. \ref{scheme diagonalizable group paragraph}). The formation of $\mathfrak{Lie}(G/S)$ commutes with base change, so it suffices to consider this construction for $G=D(M)$. We then have
\[G(I_S)=\Hom_{\Grp}(M,\Gamma(I_S,\mathscr{O}_{I_S})^\times)=\Hom_{\Grp}(M,\Gamma(S,\mathscr{D}_{\mathscr{O}_S})^\times).\]
Now the section $S\to I_S$ induces a split exact sequence
\[\begin{tikzcd}
1\ar[r]&\Gamma(S,\mathscr{O}_S)\ar[r]&\Gamma(S,\mathscr{D}_{\mathscr{O}_S})^\times\ar[r]&\Gamma(S,\mathscr{O}_S)^\times\ar[r]&0
\end{tikzcd}\]
which implies that $\mathfrak{Lie}(G)(S)$ is identified with $\Hom_{\Grp}(M,\mathbb{O}_S)$, endowed with the evident $\mathbb{O}(S)$-module structure. We then obtain by base change the following:

\begin{proposition}\label{scheme diagonalizable group Lie isomorphism}
We have isomorphisms
\[\sHom_{S\dash\Grp}(M_S,\mathbb{O}_S)\stackrel{\sim}{\to}\mathfrak{Lie}(D_S(M)/S),\quad \sHom_{\Grp}(\widetilde{M}_S,\mathscr{O}_S)\stackrel{\sim}{\to} \sLie(D_S(M)/S),\]
where, in the second isomorhism, $\widetilde{M}_S$ is the sheaf of constant group over $S$ defined by $M$, and $\sHom_{\Grp}$ is the sheaf of homomorphisms of groups.
\end{proposition}

\begin{corollary}\label{scheme diagonalizable group of free group Lie isomorphism}
If $M$ is free of finite rank, then
\[\mathbf{W}(\sLie(D_S(M)/S))\stackrel{\sim}{\to} \mathfrak{Lie}(D_S(M)/S),\quad M^{\vee}\otimes_{\Z}\mathscr{O}_S\stackrel{\sim}{\to} \sLie(D_S(M)/S).\]
In particular, $\mathbb{O}_S\cong\mathfrak{Lie}(\G_{m,S}/S)$ and $\mathscr{O}_S\cong\sLie(\G_{m,S}/S)$.
\end{corollary}
\begin{proof}
The second isomorphism follows from \cref{scheme diagonalizable group Lie isomorphism} the isomorphism
\[M^\vee\otimes_{\Z}\mathscr{O}_S=\Hom_{\Z}(\widetilde{M}_S,\mathscr{O}_S)=\Hom_{\Grp}(\widetilde{M}_S,\mathscr{O}_S),\]
which it implies that $\mathbf{W}(\sLie(D_S(M)/S))=\sHom_{S\dash\Grp}(M_S,\mathbb{O}_S)$, whence the first isomorphism.
\end{proof}

\paragraph{Inverse system of group schemes}
Let $(G_i,\varphi_{ij})$ be an inverse system of group schemes over $S$ indexed by a finite set $I$. For each $i$, the sequence
\[\begin{tikzcd}
1\ar[r]&\mathfrak{Lie}(G_i)\ar[r]&T_{G_i/S}\ar[r]&G_i\ar[r]&1
\end{tikzcd}\]
is exact, so by passing to inverse limit, we obtain an exact sequence
\[\begin{tikzcd}
0\ar[r]&\llim(\mathfrak{Lie}(G_i))\ar[r]&\llim T_{G_i/S}\ar[r]&\llim G_i\ar[r]&0
\end{tikzcd}\]
so we conclude that
\[\llim(\mathfrak{Lie}(G_i))\stackrel{\sim}{\to}\mathfrak{Lie}(\llim G_i).\]
For example, an exact sequence of groups $e\to G'\to G\to G''$ gives an exact sequence of Lie algebras
\[\begin{tikzcd}
0\ar[r]&\mathfrak{Lie}(G')\ar[r]&\mathfrak{Lie}(G)\ar[r]&\mathfrak{Lie}(G'')
\end{tikzcd}\]
and the functor $\mathfrak{Lie}$ commutes with fiber products:
\[\mathfrak{Lie}(H_1\times_GH_2)\cong\mathfrak{Lie}(H_1)\times_{\mathfrak{Lie}(G)}\mathfrak{Lie}(H_2).\]
In particular, if $H_1$ and $H_2$ are subgroups of $G$, then $\mathfrak{Lie}(H_1)$ and $\mathfrak{Lie}(H_2)$ are sub-$\mathscr{O}_S$-modules of $\mathfrak{Lie}(G)$, and we have
\[\mathfrak{Lie}(H_1\cap H_2)=\mathfrak{Lie}(H_1)\cap\mathfrak{Lie}(H_2).\]

\begin{example}
Consider for example the subgroups $\SL_2$ and $\G_m$ in $\GL_2$ over a field $k$ of characteristic $2$. Then we have $\SL_2\cap\G_m=\bm{\mu}_2$, and
\[\mathfrak{Lie}(\SL_2)\cap\mathfrak{Lie}(\G_m)=\left\{\begin{pmatrix}
a&0\\
0&a
\end{pmatrix}:a\in k\right\}=\mathfrak{Lie}(\mu_2).\]
\end{example}

\begin{proposition}\label{scheme alg group Lie algebra equal then euqal}
Let $H\sub G$ be algebraic groups over a field $k$ such that $\mathfrak{Lie}(H)=\mathfrak{Lie}(G)$. If $H$ is smooth and $G$ is connected, then $H=G$.
\end{proposition}
\begin{proof}
Let $\g=\mathfrak{Lie}(G)$ and $\h=\mathfrak{Lie}(H)$. We recall that $\dim(\g)=\dim(G)$, with equality if and only if $G$ is smooth. From the inequalities
\[\dim(H)=\dim(\h)=\dim(\g)\geq\dim(G)\geq\dim(H)\]
we conclude that $\dim(\g)=\dim(G)$, so $G$ is smooth and $\dim(G)=\dim(H)$, hence $G=H$.
\end{proof}

\begin{corollary}\label{scheme alg subgroup Lie algebra equal then equal}
Let $H_1$ and $H_2$ be smooth connected subgroups of $G$ such that $\mathfrak{Lie}(H_1)=\mathfrak{Lie}(H_2)$. If $H_1\cap H_2$ is smooth, then $H_1=H_2$.
\end{corollary}
\begin{proof}
In fact, from $\mathfrak{Lie}(H_1\cap H_2)=\mathfrak{Lie}(H_1)\cap\mathfrak{Lie}(H_2)=\mathfrak{Lie}(H_1)$ we conclude that $H_1\cap H_2=H_1$, and similarly it equals to $H_2$.
\end{proof}

\paragraph{Normalizers and centralizers}
We recall that a sequence $0\to F'\to F\to F''\to 0$ of $\mathbb{O}_S$-modules is called \textbf{exact} if for any $S'\to S$ the sequence
\[0\to F'(S')\to F(S')\to F''(S')\to 0\]
is exact. Similarly, a sequence $1\to G'\to G\to G''\to 1$ of $S$-groups is exact if for any $S'\to S$ the sequence of groups
\[1\to G'(S')\to G(S')\to G''(S')\to 1\]
is exact.

\begin{lemma}\label{scheme group condition (E) good and exact sequence}
Let $1\to G'\to G\to G''\to 1$ be an exact sequence of $S$-groups.
\begin{enumerate}
    \item[(\rmnum{1})] The sequences
    \[1\to T_{G'/S}(\mathscr{M})\to T_{G/S}(\mathscr{M})\to T_{G''/S}(\mathscr{M})\to 1\]
    \[1\to\mathfrak{Lie}(G'/S,\mathscr{M})\to \mathfrak{Lie}(G/S,\mathscr{M})\to \mathfrak{Lie}(G''/S,\mathscr{M})\to 1\]
    are exact.
    \item[(\rmnum{2})] Let $1\to H'\to H\to H''\to 1$ be a second exact sequence of groups; it is exact if and only if the following sequence is exact:
    \[\begin{tikzcd}
    1\ar[r]&G'\times_SH'\ar[r]&G\times_SH\ar[r]&G''\times_SH''\ar[r]&1
    \end{tikzcd}\]
    \item[(\rmnum{3})] If two of the $S$-groups $G',G,G''$ satisfy condition (E), then the third one satisfies condition (E).
    \item[(\rmnum{4})] If $0\to F'\to F\to F'\to 0$ is an exact sequence of $\mathbb{O}_S$-modules and two of the modules $F',F,F''$ are good, the third one is good.
    \item[(\rmnum{5})] If two of the $S$-groups are good, the third one is good.
\end{enumerate}
\end{lemma}

\begin{lemma}\label{scheme O_S-module good and invariant under group}
Let $G$ be an $S$-group, $E,F$ be $G$-objects, $M$ be an $\mathbb{O}_S[G]$-module.
\begin{enumerate}
    \item[(a)] The canonical homomorphism $E^G\times_SF^G\to (E\times_SF)^G$ is an isomorphism.
    \item[(b)] If $M$ is a good $\mathbb{O}_S$-module, so is $M^G$.
\end{enumerate}
\end{lemma}

If $E$ is an $S$-group and $F$ is a sub-$S$-group of $E$, we denote by $E/F$ the $S$-functor which to any $S'\to S$ associates the set $E(S')/F(S')$ of classes $\bar{x}=xF(S')$, $x\in E(S')$. If $E$ is an abelian group over $S$, then $E/F$ is endowed with an abelian group structure.\par
Now let $G$ be an $S$-group and $H$ be a sub-$S$-group of $G$; put $E=\mathfrak{Lie}(G/S,\mathscr{M})$ and $F=\mathfrak{Lie}(H/S,\mathscr{M})$. The adjoint action of $H$ on $E$ stablizes $F$, hence induces an action of $H$ over the $S$-functor $E/F$. For any $S'\to S$, we then have
\[(E/F)^H(S')=\{\bar{x}\in E(S')/F(S'):\text{$x^{-1}_{S''}\Ad(h)x_{S''}\in F(S'')$ for $S''\to S'$, $h\in H(S'')$}\}\]
where $x_{S''}$ denotes the image of $x$ in $E(S'')$.

\begin{theorem}\label{scheme group normalizer and centralizer Lie prop}
Let $G$ be an $S$-group, $H$ be a sub-$S$-group of $G$, $N=N_G(H)$ and $Z=Z_G(H)$.
\begin{enumerate}
    \item[(\rmnum{1})] If the group law of $\mathfrak{Lie}(G/S,\mathscr{M})$ is abelian, then
    \[\mathfrak{Lie}(N/S,\mathscr{M})/\mathfrak{Lie}(H/S,\mathscr{M})=\big(\mathfrak{Lie}(G/S,\mathscr{M})/\mathfrak{Lie}(H/S,\mathscr{M})\big)^H.\]
    \item[(\rmnum{2})] If the group law of $\mathfrak{Lie}(G/S,\mathscr{M})$ is abelian, then $\mathfrak{Lie}(Z/S,\mathscr{M})=\mathfrak{Lie}(G/S,\mathscr{M})^H$.
    \item[(\rmnum{3})] If $G$ satisfies condition (E) (resp. if $G$ and $H$ satisfy condition (E)), then $Z$ satisfies condition (E) (resp. $N$ satisfies condition (E)).
    \item[(\rmnum{4})] If $G$ is good (resp. very good), then $Z$ is good (resp. very good).
    \item[(\rmnum{5})] If $G$ and $H$ are good, then $N$ is good. If moreover $G$ is very good, then $N$ is very good.
\end{enumerate}
\end{theorem}

\begin{corollary}\label{scheme group Lie of centralizer char}
We have $\mathfrak{Lie}(Z(G)/S)=\mathfrak{Lie}(G/S)^G$ if the group law of $\mathfrak{Lie}(G/S)$ is abelian.
\end{corollary}

\begin{corollary}\label{scheme group normal subgroup Lie invariant under Ad}
If the group law of $\mathfrak{Lie}(G/S)$ is abelian and $H$ is a normal subgroup of $G$, then
\[\mathfrak{Lie}(G/S,\mathscr{M})/\mathfrak{Lie}(H/S,\mathscr{M})=\big(\mathfrak{Lie}(G/S,\mathscr{M})/\mathfrak{Lie}(H/S,\mathscr{M})\big)^H.\]
\end{corollary}

Let $G$ be a good $S$-group acting linearly on a good $\mathbb{O}_S$-module $F$ via
\[\rho:G\to\sAut_{\mathbb{O}_S}(F).\]
We have defined a corresponding linear representation
\[d\rho:\mathfrak{Lie}(G/S)\to\sEnd_{\mathbb{O}_S}(F).\]
The subgroups $N_G(E)$ and $Z_G(E)$ are defined for any subset $E$ of $F$. Similarly, for any $S'\to S$, we define
\begin{align*}
N_{\mathfrak{Lie}(G/S)}(E)(S')&=\{X\in\mathfrak{Lie}(G/S):d\rho(X)E_{S'}\sub E_{S'}\},\\
Z_{\mathfrak{Lie}(G/S)}(E)(S')&=\{X\in\mathfrak{Lie}(G/S):d\rho(X)E_{S'}=0\}.
\end{align*}
called the \textbf{normalizer} and \textbf{centralizer}, respectively, of $E$ in $F$.

\begin{theorem}\label{scheme group normalizer and centralizer of rep Lie prop}
Let $G$ be a good $S$-group acting linearly on a good $\mathbb{O}_S$-module $F$, and $E$ be a sub-$\mathbb{O}_S$-module of $F$.
\begin{enumerate}
    \item[(a)] We have $\mathfrak{Lie}(Z_G(E)/S)=Z_{\mathfrak{Lie}(G/S)}(E)$ and $Z_G(E)$ is a good $S$-group; it is very good if $G$ is.
    \item[(b)] Suppose that $E$ is a good $\mathbb{O}_S$-module. Then $\mathfrak{Lie}(N_G(E)/S)=N_{\mathfrak{Lie}(G/S)}(E)$ and $N_G(E)$ is a good $S$-group; it is very good if $G$ is.
\end{enumerate}
\end{theorem}

\begin{example}
Let $G$ be a good $S$-group. Then \cref{scheme group normalizer and centralizer of rep Lie prop} can be applied to the adjoint representation of $G$. Let $E$ be a good submodule of $\mathfrak{Lie}(G/S)$, for which we can associate the normalizer and centralizer. By \cref{scheme group normalizer and centralizer of rep Lie prop}, their Lie algebras are respectively the normalizer and centralizer of $E$ in $\mathfrak{Lie}(G/S)$, given by the usual definition:
\begin{align*}
N_{\mathfrak{Lie}(G/S)}(E)(S')&=\{X\in\mathfrak{Lie}(G/S):d\rho(X)E_{S'}\sub E_{S'}\},\\
Z_{\mathfrak{Lie}(G/S)}(E)(S')&=\{X\in\mathfrak{Lie}(G/S):d\rho(X)E_{S'}=0\}.
\end{align*}
\end{example}

\begin{example}
Let $H$ be a sub-$S$-group of $G$, then $\mathfrak{Lie}(H/S)$ is a sub-$\mathbb{O}_S$-module of $\mathfrak{Lie}(G/S)$. Suppose that $\mathfrak{Lie}(H/S)$ is a good $\mathbb{O}_S$-module; we evidently have
\[N_G(H)\sub N_G(\mathfrak{Lie}(H/S)),\quad Z_G(H)\sub Z_G(\mathfrak{Lie}(H/S))\]
whence, by \cref{scheme group normalizer and centralizer of rep Lie prop}, we obtain
\[\mathfrak{Lie}(N_G(H)/S)\sub N_{\mathfrak{Lie}(G/S)}(\mathfrak{Lie}(H/S)),\quad \mathfrak{Lie}(Z_G(H)/S)\sub Z_{\mathfrak{Lie}(G/S)}(\mathfrak{Lie}(H/S)),\]
but none of these four inclusions is a priori an identity. In particular, if $H$ is a normal subgroup of $G$, then $\mathfrak{Lie}(H/S)$ is an ideal of $\mathfrak{Lie}(G/S)$.
\end{example}

\begin{example}
Let $k$ be an algebraically closed field of characteristic $p>0$ and $G$ be the algebraic group over $k$ such that for any $k$-scheme $S'$, $G(S')$ consists of the matrices
\[A(a,b)=\begin{pmatrix}
a&0&0\\
0&a^p&b\\
0&0&1
\end{pmatrix},\quad a\in\Gamma(S',\mathscr{O}_{S'})^\times,b\in\Gamma(S',\mathscr{O}_{S'}).\]
Defined regular functions on $G$ by $X:A(a,b)\mapsto a$ and $Y:A(a,b)\mapsto b$, then $\mathscr{O}(G)=k[X,Y,X^{-1}]$, which is an integral domain, and so $G$ is connected and smooth. We note that
\[\begin{pmatrix}
a_1&0&0\\
0&a_1^p&b_1\\
0&0&1
\end{pmatrix}\begin{pmatrix}
a_2&0&0\\
0&a_2^p&b_2\\
0&0&1
\end{pmatrix}\begin{pmatrix}
a_1&0&0\\
0&a_1^p&b_1\\
0&0&1
\end{pmatrix}^{-1}=\begin{pmatrix}
a_2&0&0\\
0&a_2^p&b_1-a_2^pb_1+a_1^pb_2\\
0&0&1
\end{pmatrix}\]
so $G$ is not commutative and its center consists of the elements $A(a,b)$ with $a^p=1$ and $b=0$. It then follows that
\[\mathscr{O}(Z(G))=\mathscr{O}(G)/(X^p-1,Y)\cong k[X]/(X^p-1),\]
which is not reduced (in fact $Z(G)=\bm{\mu}_p$). On the other hand, the kernel of the morphism $G(k[t])\to G(k)$ is given by
\[\left\{\begin{pmatrix}
1+at&0&0\\
0&1&bt\\
0&0&1
\end{pmatrix}:a,b\in k\right\},\]
so the Lie algebra of $G$ is equal to
\[\mathfrak{Lie}(G/k)=\left\{\begin{pmatrix}
a&0&0\\
0&0&b\\
0&0&0
\end{pmatrix}:a,b\in k\right\},\]
which is obviously commutative. Moreover,
\[\begin{pmatrix}
a_1&0&0\\
0&a_1^p&b_1\\
0&0&1
\end{pmatrix}\begin{pmatrix}
1+a_2t&0&0\\
0&1&b_2t\\
0&0&1
\end{pmatrix}\begin{pmatrix}
a_1&0&0\\
0&a_1^p&b_1\\
0&0&0
\end{pmatrix}^{-1}=\begin{pmatrix}
1+a_2t&0&0\\
0&1&a_1^pb_2t\\
0&0&1
\end{pmatrix},\]
so the kernel of the morphism $\Ad:G\to\GL(\g)$ consists of the elements $A(a,b)$ with $a^p=1$, hence
\[\ker\Ad=\Spec(\mathscr{O}(G)/(X^p-1))=\Spec(k[X,Y]/(X^p-1)),\]
which is not reduced, and so $\Ad$ is not smooth. We also note that
\[\dim(\z(\g))=2>\dim(\ker\Ad)=1>\dim(Z(G))=0.\]
\end{example}

\begin{example}\label{scheme group Lie and normalizer nonequal example}
Let $S$ be a scheme, $F$ be the good $\mathbb{O}_S$-module $\mathbb{O}_S^2$ endowed with the natural action of the good $S$-group $G=\GL_{2,S}$, and $E$ be the sub-$\mathbb{O}_S$-module of $F$ formed by couples $(x_1,x_2)$ such that $x_2$ is nilpotent. Put $N=N_G(E)$, then $\mathfrak{Lie}(N/S)=\mathfrak{Lie}(G/S)$ while, for any $S'\to S$, we have
\[N_{\mathfrak{Lie}(G/S)}(E)(S')=\Big\{\begin{pmatrix}
a&b\\
x&c
\end{pmatrix}:\text{$a,b,c,x\in\mathscr{O}(S')$, $x$ nilpotent}\Big\}\]
hence $\mathfrak{Lie}(N_G(E)/S)\neq N_{\mathfrak{Lie}(G/S)}(E)$.
\end{example}

\section{Equivalence relations and passing to quotient}\label{category equivalence relation and quotient section}
\subsection{Universally effective equivalence relations}
\paragraph{Equivalence relations}
\begin{definition}
Let $\mathcal{C}$ be a category. A \textbf{$\mathcal{C}$-equivalence relation} over $X\in\Ob(\mathcal{C})$ is defined to be a representable sunfunctor $R$ of $X\times X$, such that for any $S\in\Ob(\mathcal{C})$, $R(S)$ is the graph of an equivalence relation over $X(S)$.
\end{definition}
This definition is applicable in particular to the category $\widehat{\mathcal{C}}$. If we consider $X$ as an object of $\widehat{\mathcal{C}}$, then a $\widehat{\mathcal{C}}$-equivalance relation over $X$ is none other than a subfunctor $R$ of $X\times X$ (not necessarily representable in $\mathcal{C}$) such that $R(S)$ is the graph of an equivalence relation on $X(S)$ for any $S\in\Ob(\mathcal{C})$. In fact, this conditon is evidently necessary. Conversely, if for any $S\in\Ob(\mathcal{C})$, $R(S)$ is the graph of an equivalence relation, then this equivalence relation extends to $R(F)$ for any $F\in\Ob(\widehat{\mathcal{C}})$ by declearing two morphisms $\phi,\psi:F\to R$ to be equivalent if, for any $S\in\Ob(\mathcal{C})$ and $x\in F(S)$, $\phi(x)$ is equivalent to $\psi(x)$ in $X(S)$.\par
If $R$ is a $\mathcal{C}$-equivalence relation on $X$, we denote by $p_i:R\to X$ the morphism induced by the projection $\pr_i:X\times X\to X$. We then have a diagram
\[p_1,p_2:R\rightrightarrows X.\]
A morphism $u:X\to Z$ is called \textbf{compatible with $R$} if $up_1=up_2$. The cokernel in $\mathcal{C}$ of the couple $(p_1,p_2)$ is also called the \textbf{quotient object} of $X$ by $R$, and denoted by $X/R$. We then have an exact diagram
\[\begin{tikzcd}
R\ar[r,shift left=2pt,"p_1"]\ar[r,shift right=2pt,swap,"p_2"]&X\ar[r,"p"]&X/R
\end{tikzcd}\]
and $X/R$ represents the covariant functor
\[\Hom_{\mathcal{C}}(X/R,Z)=\{\text{morphisms $X\to Z$ compatible with $R$}\}.\]
Since the quotient objects have been chosen in $\mathcal{C}$, the quotient $X/R$ is unique (when it exists).\par
These definitions immediately generalize to $\widehat{\mathcal{C}}$-equivalence relations on $X$, but note that the Yoneda embedding functor $\mathcal{C}\to\widehat{\mathcal{C}}$ does not commutes with the formation of quotients, so the quotient $X/R$ of $X$ by $R$ in $\mathcal{C}$ if not a priori a quotient of $X$ by $R$ in $\widehat{\mathcal{C}}$. Therefore, we will be careful not to identity $\mathcal{C}$ indiscriminetly with its image in $\widehat{\mathcal{C}}$ when dealing with questions involving passages to the quotient. In the following, by "equivalence relation", we simply mean $\widehat{\mathcal{C}}$-equivalence relations.\par

If $X$ is an object of $\mathcal{C}$ over $S$, an \textbf{equivalence relation on $\bm{X}$ over $\bm{S}$} is defined to be an equivalence relation $R$ over $X$ such that the structural morphism $X\to S$ is compatible with $R$. In this case, the canonical morphism $R\to X\times X$ then factors through the monomorphism
\[X\times_SX\to X\times X\]
and defines an equivalence relation over the object $X\to S$ of $\mathcal{C}_{/S}$. If the quotient $X/R$ exists, it is endowed with a canonical morphism to $S$ and the corresponding object of $\mathcal{C}_{/S}$ is a quotient of $X\in\Ob(\mathcal{C}_{/S})$ by the preceding equivalence relation. Conversely, if $S$ is a squarable object of $\mathcal{C}$ and $Y\to S$ is a quotient of $X$ by this equivalence relation (in $\mathcal{C}_{/S}$), then $Y$ is a quotient by $R$ in $\mathcal{C}$.

\begin{definition}
If $X$ (resp. $X'$) is an object of $\mathcal{C}$ endowed with an equivalence relation $R$ (resp. $R'$), a morphism $u:X\to X'$ is called compatible with $R$ and $R'$ if the following equivalence relations are satisfied:
\begin{enumerate}
    \item[(\rmnum{1})] For any $S\in\Ob(\mathcal{C})$, two points of $X(S)$ congruent modulo $R(S)$ are transformed by $u$ to two points of $X'(S)$ congruent modulo $R'(S)$
    \item[(\rmnum{2})] There exists a morphism $R\to R'$ (necessarily unique) fitting into the diagram
    \[\begin{tikzcd}
    R\ar[r]\ar[d]&R'\ar[d]\\
    X\times X\ar[r,"u\times u"]&X'\times X'
    \end{tikzcd}\]
\end{enumerate}
By the universal property of $X/R$, there then exists (if the quotients $X/R$ and $X'/R'$ exists) a unique morphism $v$ fitting into the commutative diagram
\[\begin{tikzcd}
X\ar[r,"p"]\ar[d,"u"]&X/R\ar[d,"v"]\\
X'\ar[r,"p'"]&X'/R'
\end{tikzcd}\]
\end{definition}

\begin{definition}
A sub-object $Y$ of $X$ is called \textbf{stable} under the equivalence relation $R$ if the following equivalent conditions are satisfied:
\begin{enumerate}
    \item[(\rmnum{1})] For any $S\in\Ob(\mathcal{C})$, the subset $Y(S)$ of $X(S)$ is stable under $R(S)$.
    \item[(\rmnum{2})] The inverse images of $Y$ under $p_1$ and $p_2$ are identical. 
\end{enumerate}
\end{definition}
A particular important case is the following: the quotient $X/R$ exists and $Y$ is the inverse image of a sub-object of $X/R$ in $X$.

\begin{definition}
Let $R$ be an equivalence relation over $X$ and $X'\to X$ ve a morphism. The equivalence relation $R'$ over $X$ obtained by the Cartesian diagram
\[\begin{tikzcd}
R'\ar[r]\ar[d]&R\ar[d]\\
X'\times X'\ar[r]&X\times X
\end{tikzcd}\]
is called the inverse image of $R$ in $X'$. In particular, if $X'$ is a sub-object of $X$, the corresponding equivalence relation is called the induced relation on $X'$, and denoted by $R_{X'}$.
\end{definition}

The morphism $X'\to X$ is compatible with $R'$ and $R$; we then have, if the quotients exist, a morphism $X'/R'\to X/R$. If $X'$ is a sub-object of $X$, we shall see that in certain case we can prove that $X'/R'\to X/R$ is a monomorphism, hence identifies $X'/R'$ with a sub-object of $X/R$. If this is the case, the inverse image of this sub-object in $X$ will be a sub-object of $X$ containing $X'$ and stable under $R$, called the \textbf{saturation} of $X'$ for the equivalence relation $R$.

\begin{proposition}\label{category equivalence relation subobject Cartesian diagram}
If the sub-object $Y$ of $X$ is stable under $R$, we have two Cartesian squares for $i=1,2$:
\[\begin{tikzcd}
R_Y\ar[r]\ar[d,swap,"p_i"]&R\ar[d,"p_i"]\\
Y\ar[r]&X
\end{tikzcd}\]
\end{proposition}
\begin{proof}
This follows from the definition of $R_Y$ and the stability of $Y$ under $R$.
\end{proof}

\paragraph{Equivalence relation defiend by a free group action}
\begin{definition}
Let $X$ be an object of $\mathcal{C}$ and $H$ be a $\mathcal{C}$-group acting on $X$. We say that $H$ acts freely on $X$ if the following conditions are satisfied:
\begin{enumerate}
    \item[(\rmnum{1})] For any $S\in\Ob(\mathcal{C})$, the group $H(S)$ acts freely on $X(S)$.
    \item[(\rmnum{2})] The morphism of functors $H\times X\to X\times X$ defined by $(h,x)\mapsto(hx,x)$ is a monomorphism. 
\end{enumerate}
\end{definition}
If $H$ acts freely on $X$, the image of $H\times X$ by the morphism in (\rmnum{2}) is an equivalence relation on $X$, called the \textbf{equivalence relation defined by the action of $\bm{H}$ over $\bm{X}$}. The quotient of $X$ by this equivalence relation, if exists, is denoted by $H\backslash X$. It represents the following covariant functor: if $Z$ is an object of $\mathcal{C}$, we have
\[\Hom(H\backslash X,Z)=\{\text{morphisms $X\to Z$ invariant under $H$}\}\]
where a morphism $f:X\to Z$ is invariant under $H$ if for any $S\in\Ob(\mathcal{C})$, the corresponding morphism $X(S)\to Z(S)$ is invariant under the group $H(S)$.

\begin{lemma}
Let $H$ be a group acting freely on $X$ and $Y$ be a sub-object of $X$. The following conditions are equivalent:
\begin{enumerate}
    \item[(\rmnum{1})] $Y$ is stable under the equivalence relation defined by $H$.
    \item[(\rmnum{2})] For any $S\in\Ob(\mathcal{C})$, the subset $Y(S)$ of $X(S)$ is stable under $H(S)$.
    \item[(\rmnum{3})] There exists a morphism $f$ (necessarily unique) fitting into the commutative diagram
    \[\begin{tikzcd}
    H\times Y\ar[r,"f"]\ar[d]&Y\ar[d]\\
    H\times X\ar[r]&X
    \end{tikzcd}\]
\end{enumerate}
Under these conditions, $f$ defines a morphism of $\widehat{\mathcal{C}}$-groups $H\to\sAut(Y)$ and the equivalence relation over $Y$ defined by $H$ is the induced one from $X$.
\end{lemma}
\begin{proof}
The proof is immediate, by the definition of stable objects and the equivalence relation induced by $H$. The operation of $H$ on $Y$ is called the induced action.
\end{proof}

Now consider the following situation: $H$ and $G$ are two $\mathcal{C}$-groups and we are given a group morphism $u:H\to G$. Then $H$ acts on $G$ by translations (we put $h\cdot g=u(h)g$) and it acts freely on $G$ if and only if $u$ is a monomorphism. In this case, the quotient of $G$ by this action of $H$ is denoted (if exists) by $H\backslash G$. Similarly, we can define a right action of $H$ on $G$, and a quotient $G/H$. These quotients are functorial relative to the two groups. More precisely, we have the following lemma for right actions of $H$:

\begin{lemma}\label{category equivalence relation by free action compatible morphism iff}
Let $u:H\to G$ and $u':H'\to G'$ be two monomorphisms of $\mathcal{C}$-groups. Suppose that we are given a morphism of $\mathcal{C}$-groups $f:G\to G'$, then the following conditions are equivalent:
\begin{enumerate}
    \item[(\rmnum{1})] $f$ is compatible with the equivalence relations defined by $H$ and $H'$.
    \item[(\rmnum{2})] For any $S\in\Ob(\mathcal{C})$, we have $f(u(H(S)))\sub u'(H(S))$.
    \item[(\rmnum{3})] There exists a morphism $g:H\to H'$ (necessarily unique and multiplicative) such that the following diagram is commutative:
    \[\begin{tikzcd}
    H\ar[r,"g"]\ar[d,swap,"u"]&H'\ar[d,"u'"]\\
    G\ar[r,"f"]&G'
    \end{tikzcd}\] 
\end{enumerate}
Under these conditions, if the quotients $G/H$ and $G'/H'$ exist, there is a unique morphism $\bar{f}$ fitting into the commutative diagram
\[\begin{tikzcd}
G\ar[r,"f"]\ar[d,swap,"p"]&G'\ar[d,"p'"]\\
G/H\ar[r,"\bar{f}"]&G'/H'
\end{tikzcd}\]
\end{lemma}
\begin{proof}
The first assertion can be verified element-wisely, and the second one then follows from (\rmnum{1}).
\end{proof}

We can then translate the notions introduced above for general equivalence relations to the present situation. Let us simply point out the following lemma, whose proof is immediate by reduction to the set case:
\begin{lemma}\label{category equivalence relation by free action stable subobject iff}
Let $u:H\to G$ be a monomorphism of $\mathcal{C}$-groups and $G'$ be a sub-group of $G$. For a sub-object $G'$ of $G$ to be stable under the equivalence relation defined by $H$, it is necessary and sufficient that $u$ factors through the canonical monomorphism $G'\to G$. In this case, the induced action of $H$ on $G'$ is none other than that deduced by the monomorphism $H\to G'$ factorizing $u$.
\end{lemma}

\paragraph{Universally effective equivalence relations}\label{category universally effective relation paragraph}
\begin{definition}
Let $f:X\to Y$ be a morphism. The image of the canonical monomorphism
\[X\times_YX\to X\times X\]
then defines a $\widehat{\mathcal{C}}$-equivalence relation on $X$, called the \textbf{equivalence relation defined by $\bm{f}$ over $\bm{X}$} and denoted by $R(f)$.
\end{definition}
\begin{definition}
Let $R$ be an equivalence relation over $X$. We say that $R$ is effective if
\begin{enumerate}
    \item[(\rmnum{1})] $R$ is representable (i.e. is a $\mathcal{C}$-equivalence relation);
    \item[(\rmnum{2})] the quotient $Y/R$ exists in $\mathcal{C}$;
    \item[(\rmnum{3})] the diagram
    \[\begin{tikzcd}
    R\ar[r,shift left=2pt,"p_1"]\ar[r,shift right=2pt,swap,"p_2"]&X\ar[r,"p"]&Y
    \end{tikzcd}\]
    makes $R$ the fiber product of $X$ over $Y$, that is, $R$ is the equivalence relation defined by $p$.
\end{enumerate}
\end{definition}
If $R$ is an effective equivalence relation over $X$, then $p$ is an effective epimorphism. If $f:X\to Y$ is an effective epimorphism, then $R(f)$ is an effective equivalence relation over $X$ whose quotient is $Y$. There then exists a correspondence bewteen effective equivalence relations over $X$ and effective quotients of $X$.

\begin{definition}
An equivalence relation $R$ over $X$ is called \textbf{universally effective} if the quotient $Y=X/R$ exists and if, for any $Y'\to Y$, the fiber product $X'=X\times_YY'$ and $R'=R\times_YY'$ exist and $R'$ is a fiber product of $X'$ over $Y'$. Equivalently, this amouts to saying that $R$ is effective and $p:X\to X/R$ is a universally effective epimorphism.
\end{definition}

\begin{remark}\label{category universal effective equivalence section is invariant}
Suppose that $\mathcal{C}$ is the category of $S$-schemes and denote by $\G_{a,S}$ the additive group over $S$. Let $R\sub X\times_SX$ be a universally effective equivalence relation and $p:X\to Y$ be the quotient. Then, for any open subset $U$ of $Y$, $\mathscr{O}(U)=\Hom_S(U,\G_{a,S})$ is the set of elements $\phi\in\mathscr{O}(p^{-1}(U))=\Hom_S(p^{-1}(U),\G_{a,S})$ such that $\phi\circ p_1=\phi\circ p_2$. In particular, if $R$ is given by a free right action over $X$ of a group $H$, then $\mathscr{O}(U)$ is the set of $\phi\in\mathscr{O}(p^{-1}(U))$ such that $\phi(xh)=\phi(x)$ for any $S'\to S$ and $x\in X(S')$, $h\in H(S')$. 
\end{remark}

\begin{proposition}\label{category universal effective equivalence factor monomorphism iff}
Let $R$ be a universally effective equivalence relation over $X$, $f:X\to Z$ be a morphism compatible with $R$, with a factorization $g:X/R\to Z$. The following conditions are equivalent:
\begin{enumerate}
    \item[(\rmnum{1})] $g$ is a monomorphism;
    \item[(\rmnum{2})] $R$ is the equivalence relation defined by $f$. 
\end{enumerate}
\end{proposition}
\begin{proof}
In fact, (\rmnum{1}) clearly implies (\rmnum{2}), and the converse follows from \cref{category descent morphism monomorphism if fiber product isomorphic}. 
\end{proof}

\begin{definition}
Let $H$ be a $\mathcal{C}$-group acting freely on $X$. We say that $H$ acts \textbf{effectively} on $X$, or the action of $H$ on $X$ is \textbf{effective} (resp. \textbf{universally effective}), if the equivalence relation defined by $H$ is effective (resp. universally effective).
\end{definition}

In practive, it is often difficult to characterize universally effective epimorphisms. We often have, however, a certain number of morphisms of this type, for example, faithfully flat and quasi-compact morphisms of schemes. This leads to the following definition: Let $\mathcal{M}$ be a family of morphisms of $\mathcal{C}$ satisfying the following properties:
\begin{enumerate}[leftmargin=40pt]
    \item[(M1)] $\mathcal{M}$ is \textit{stable under base change}, i.e. for any morphism $u:T\to S$ in $\mathcal{M}$ is squarable and for any $S'\to S$, $u':T\times_SS'\to S'$ belongs to $\mathcal{M}$. 
    \item[(M2)] The composition of two morphisms in $\mathcal{M}$ belongs to $\mathcal{M}$.
    \item[(M3)] Any isomorphism belongs to $\mathcal{M}$.
    \item[(M4)] Any morphism in $\mathcal{M}$ is an effective epimorphism.  
\end{enumerate}
Note that (M1) and (M2) imply:
\begin{enumerate}[leftmargin=40pt]
    \item[(M1')] The Cartesian product of two morphisms in $\mathcal{M}$ is in $\mathcal{M}$: Let $u:X\to Y$ and $u':X'\to Y'$ be two $S$-morphisms belonging to $\mathcal{M}$. If $Y\times_SY'$ exsits, then $X\times_SX'$ exists and $u\times_Su'$ belongs to $\mathcal{M}$.
\end{enumerate}
This follows from the diagram
\[\begin{tikzcd}
&Y'&X'\ar[l,swap,"u'"]\\
X\ar[d,"u"]&X\times_SY'\ar[l]\ar[d]\ar[u]&X\times_SX'\ar[ld,"u\times_Su'"]\ar[l]\ar[u]\\
Y&Y\times_SY'\ar[l] 
\end{tikzcd}\]
Similarly, (M1) and (M4) imply:
\begin{enumerate}[leftmargin=40pt]
    \item[(M4')] Any morphism of $\mathcal{M}$ is a universally effective epimorphism.
\end{enumerate}

The family $\mathcal{M}_0$ of universally effective morphisms verifies the conditions (M1)--(M4). In fact, (M1), (M3) and (M4) follows by definition, (M2) follows from (\cite{SGA3-1} \Rmnum{4}, 1.8). In the following, we suppose that $\mathcal{M}$ is a family of morphisms in $\mathcal{C}$ verifying the above conditions. In particular, our result is applicable to the family $\mathcal{M}_0$.

\begin{definition}
We say that an equivalence relation $R$ over $X$ is \textbf{of type $\mathcal{M}$} if it is representable and if $p_1\in\mathcal{M}$\footnote{This by (M2) and (M3) implies $p_2\in\mathcal{M}$, since $p_1$ and $p_2$ are exchanged by an isomorphism of $X\times X$.}. We say that $R$ is \textbf{$\mathcal{M}$-effective} if it is effective and if the canonical morphism $X\to X/R$ belongs to $\mathcal{M}$. Finally, we say the quotient $Y$ of $X$ is \textbf{$\mathcal{M}$-effective} if the canonical morphism $X\to Y$ belongs to $\mathcal{M}$.
\end{definition}

\begin{proposition}\label{category equivalence relation M-effective prop}
Let $\mathcal{M}$ be a family of morphisms in $\mathcal{C}$ as above.
\begin{enumerate}
    \item[(a)] An $\mathcal{M}$-effective equivalence relation is of type $\mathcal{M}$ and universally effective. 
    \item[(b)] An $\mathcal{M}$-effective quotient is universally effective.
    \item[(c)] The map $R\mapsto X/R$ and $p\mapsto R(p)$ is a bijective correspondence from the set of effective equivalence relations over $X$ to the set of $\mathcal{M}$-effective quotients of $X$.
    \item[(d)] $\mathcal{M}_0$-effectivity is equivalent to universally effectivity.
\end{enumerate}
\end{proposition}
\begin{proof}
Let $R$ be $\mathcal{M}$-effective, so that we have a Cartesian square
\[\begin{tikzcd}
R\ar[r,"p_2"]\ar[d,swap,"p_1"]&X\ar[d,"p"]\\
X\ar[r,"p"]&X/R
\end{tikzcd}\]
and $p\in\mathcal{M}$. By (M1), $p_1$ and $p_2$ belong to $\mathcal{M}$, so $R$ is of type $\mathcal{M}$.\par
Put $Y=X/R$ and let $Y'\to Y$ be a morphism. By (M1), the fiber products $X'=X\times_YY'$ and $R'=R\times_YY'$ exist and the morphisms $X'\to Y'$ and $p'_i:R'\to X'$ belong to $\mathcal{M}$. Finally, as $R=X\times_YX$, we obtain, by associativity of fiber products:
\[R'=X\times_YX\times_YY'=X'\times_{Y'}X'\]
so $R'$ is $\mathcal{M}$-effective and in particular $R$ is universally effective. This proves (a) and also (d). The assertions of (b) and (c) then follows from this and the definition.
\end{proof}

\begin{example}
Let $H$ be an $S$-group whose structural morphism belongs to $\mathcal{M}$. If $H$ acts freely on the $S$-object $X$, then it defines an equivalence relation of type $\mathcal{M}$. In fact, by (M1) the fiber product $H\times_SX$ exists and $p_2:H\times_SX\to X$ belongs to $\mathcal{M}$. We say that the operation of $H$ is \textbf{$\mathcal{M}$-effective} if the equivalence relation over $X$ defined by $H$ is $\mathcal{M}$-effective.
\end{example}

\begin{proposition}[\textbf{$\mathcal{M}$-effectivity and Base Change}]\label{category equivalence relation M-effective and base change}
Let $R$ be an $\mathcal{M}$-effective equivalence relation on $X$ over $S$ and put $Y=X/R$. Let $S'\to S$ be a base change morphism such that $Y'=Y\times_SS'$ exists. Then $X'=X\times_SS'$ exists, $R'=R\times_SS'$ exists and is an $\mathcal{M}$-effective equivalence relation on $X'$ over $S'$ and $X'/R'\cong(X/R)'$.
\end{proposition}
\begin{proof}
In fact, the canonical morphisms $X\to Y$ and $R\to Y$ belong to $\mathcal{M}$, hence by (M1'), $X'$ and $R'$ are representable. By associativity of fiber products, $R'$ is the equivalence relation defined by the canonical morphism $X'\to Y'$ which belongs to $\mathcal{M}$, whence the conclusion.
\end{proof}

\begin{proposition}[\textbf{$\mathcal{M}$-effectivity and Cartesian Product}]\label{category equivalence relation M-effective and product}
Let $R$ (resp. $R'$) be an $\mathcal{M}$-effective equivalence relation on $X$ (resp. $X'$) over $S$. If $(X/R)\times_S(X'/R')$ exists, then $X\times_SX'$ exists, $R\times_SR'$ is an $\mathcal{M}$-effective equivalence relation on $X\times_SX'$ over $S$ and
\[(X\times_SX')/(R\times_SR')\cong (X/R)\times_S(X'/R').\]
\end{proposition}
\begin{proof}
Put $Y=X/R$ and $Y'=X'/R'$. By (M1'), the fiber product $X\times_SX'$ exists and the canonical morphism $q:X\times_SX'\to Y\times_SY'$ belongs to $\mathcal{M}$. Now the formula
\[(X\times X')\times_{Y\times Y'}(X\times X')\cong (X\times_YX)\times(X'\times_{Y'}X')\]
(where the product without subscript is taken over $S$) shows that $R\times_SR'$ is the equivalence relation defined by $q$ on $X\times_SX'$, whence the proposition.
\end{proof}

Suppose that $\mathcal{C}$ possesses a final object $e$ and let $f:G\to G'$ be a morphism of $\mathcal{C}$-groups such that $f\in\mathcal{M}$. Then by (M1), the kernel $\ker f$ is representable by $e\times_{G'}G$, and the morphism $\ker f\to e$ belongs to $\mathcal{M}$. On the other hand, the equivalence relation defined by $f$ is the same as that defined by the action of $\ker f$ (right, say) over $G$, that is, the image of the morphism $G\times \ker f\to G\times G$, defined by $(g,h)\mapsto(g,gh)$.

\begin{corollary}\label{category M-morphism kernel M-effective}
Suppose that $\mathcal{C}$ possesses a final object $e$ and let $f:G\to G'$ be a morphism of $\mathcal{C}$-groups such that $f\in\mathcal{M}$. Then the action of $\ker f$ on $G$ is $\mathcal{M}$-effective and $G'$ is the the quotient $G/\ker f$.
\end{corollary}
\begin{proof}
Since $f$ is a universally effective epimorphism by (M4'), $G'$ is identified with the quotient of $G$ by the equivalence relation defined by $f$, that is, by the action of $\ker f$. Since $\ker\to e$ belongs to $\mathcal{M}$, this equivalence relation is therefore representable by (M1), and we conclude the corollary.
\end{proof}

\paragraph{Construction of quotients by descent}
\begin{definition}
We say that a descent data over $X'$ relative to $S'\to S$ is \textbf{effective} if $X'$ endowed with the descent data is isomorphic to the inverse image over $S'$ of an object $X$ over $S$.
\end{definition}

If $S'\to S$ is a descent morphism, then the $X$ in the above definition is unique up to unique isomorphism. The morphism $S'\to S$ is an effective descent morphism if it is a descent morphism and any descent data relative to $S'\to S$ is effective.\par
Now consider an equivalence relation $R$ over an object $X$ over $S$. Let $X'$ (resp. $X''$, resp. $X'''$) be the inverse image of $X$ over $S'$, $S''=S'\times_SS'$ and $S'''=S'\times_SS'\times_SS'$ and let $R'$, $R''$, $R'''$ be the induced equivalence relations of $R$ by inverse image. Suppose that the equivalence relation $R'$ on $X'$ is $\mathcal{M}$-effective, and consider the quotient $Y'=X'/R'$ which is an object over $S'$. Its inverse images under the two projections from $S''$ are isomorphic to $X''/R''$ by \cref{category equivalence relation M-effective and base change}, so the $S'$-object $Y'$ is endowed with a canonical glueing data. Using the same uniqueness for $X'''/R'''$, we see that this is a descent data (note that we have implicitly assumed have all these fiber products exist, for example if $S'\to S$ is squarable).

\begin{proposition}\label{category equivalence relation M-effective and descent}
Let $R$ be an equivalence relation on an object $X$ over $S$, and $S'\to S$ be a universally effective epimorphism. Suppose that any $S$-morphism whose inverse image over $S'$ belongs to $\mathcal{M}$ is itself in $\mathcal{M}$. Then the following conditions are equivalent:
\begin{enumerate}
    \item[(\rmnum{1})] $R$ is $\mathcal{M}$-effective on $X$;
    \item[(\rmnum{2})] $R'$ is $\mathcal{M}$-effective and the canonical descent date over $X'/R'$ is effective.
\end{enumerate}
Moreover, if this is the case, the descent object of $X'/R'$ is canonically isomorphic to $X/R$.
\end{proposition}
\begin{proof}
The fact that (\rmnum{1}) implies (\rmnum{2}) follows directly from the definition of $\mathcal{M}$-effectivity and \cref{category equivalence relation M-effective prop}~(a). If the converse is true, then the last assertion follows from the fact that a universally effective epimorphism is a descent morphism, so the descent object is unique (up to isomorphism).\par
We now prove that (\rmnum{2})$\Rightarrow$(\rmnum{1}). Let $Y'=X'/R'$ and $Y$ be the descent object. As the canonical morphism $p':X'\to X'/R'=Y'$ is compatible with the descent data (its inverse image over $S''$ coincides with the canonical morphism $X''\to X''/R''$ by \cref{category equivalence relation M-effective and base change}), it comes from an $S$-morphism $p:X\to Y$. As $p'$ belongs to $\mathcal{M}$, it follows from the hypothesis made on the morphism $S'\to S$ that $p$ also belongs to $\mathcal{M}$. As $p'$ is compatible with the equivalence relation $R'$, $p$ is compatible with $R$, since a universally effective epimorphism is a descent morphism. We then have a morphism
\[R\to X\times_YX.\]
To see that $R$ is $\mathcal{M}$-effective and that $Y$ is isomorphic to $X/R$, it suffices to prove that this morphism is an isomorphism. Now since $R'$ is effective, this becomes an isomorphism after base change to $S'$, and it is therefore an isomorphism for the same reason.
\end{proof}

We note that the hypothesis of \cref{category equivalence relation M-effective and descent} is verified if we take $\mathcal{M}=\mathcal{M}_0$ to be the family of universally effective epimorphisms and if $\mathcal{C}$ possesses fiber products (cf. \cite{SGA3-1}, \Rmnum{4}, Corollaire 1.10). We then deduce the following corollary:

\begin{corollary}\label{category equivalence relation universally effective and descent}
Suppose that $\mathcal{C}$ possesses fiber products (over $S$). Let $R$ be an equivalence relation on $X$ over $S$ and $S'\to S$ be a universally effective epimorphism. Then the following conditions are equivalent:
\begin{enumerate}
    \item[(\rmnum{1})] $R$ is universally effective on $X$;
    \item[(\rmnum{2})] $R'$ is universally effective and the canonical descent date over $X'/R'$ is effective.
\end{enumerate}
Moreover, if this is the case, the descent object of $X'/R'$ is canonically isomorphic to $X/R$.
\end{corollary}

\subsection{Equivalence relations in the category of sheaves}

\paragraph{Equivalence relations in \texorpdfstring{$\widetilde{\mathcal{C}}$}{C}}

Let $\mathcal{C}$ be a site and $\widetilde{\mathcal{C}}$ be the category of sheaves over $\mathcal{C}$. Let $i:\mathcal{C}\to\widehat{\mathcal{C}}$ be the inclusion functor.

\begin{proposition}\label{site sheaf equivalence relation is universally effective}
Any equivalence relation in $\widetilde{\mathcal{C}}$ is universally effective: let $R$ be a $\widetilde{\mathcal{C}}$-equivalence relation on the sheaf $X$, then the sheaf associated with the separated presheaf
\[i(X)/i(R):S\mapsto X(S)/R(S)\]
is a universally effective quotient sheaf of $X$ by $R$.
\end{proposition}
\begin{proof}
Let $X/R=(i(X)/i(R))^\#$ be the quotient sheaf of $X$ by $R$, which exists by (\cite{SGA3-1}, \Rmnum{4}, 4.4.1(\rmnum{2})). It is necessary to show that $X\to X/R$ is a universally effective epimorphism, and that the morphism $f:R\to X\times_{X/R}X$ is an isomorphism. The first assertion follows from the proof of (\cite{SGA3-1}, \Rmnum{4}, 4.4.3). As for $f$, it comes from the sheafification of the morphism $i(R)\to i(X)\times_{i(X/R)}i(X)$, or, as $i(X)/i(R)$ is separated (\cite{SGA3-1}, \Rmnum{4}, 4.4.5(\rmnum{2})) so that $i(X)/i(R)\to i(X/R)$ is a monomorphism, from the canonical morphism $i(R)\to i(X)\times_{i(X)/i(R)}i(X)$.\par
We are therefore reduced to the same assertion for the category of presheaves. But $i(X)/i(R)$ is the presheaf $S\mapsto X(S)/R(S)$ and we are then reduced to the same assertion for the category of sets, which is immediate.
\end{proof}

\begin{proposition}\label{site sheaf quotient of subsheaf is image}
Under the conditions of \cref{site sheaf equivalence relation is universally effective}, let $Y$ be a subsheaf of $X$. Denote by $R_Y$ the equivalence relation induced on $Y$ by $R$, then the canonical morphism $Y/R_Y\to X/R$ is a monomorphism: it identifies $Y/R_Y$ with the subsheaf of $X/R$, which is the image sheaf of the composition morphism $Y\to X\to X/R$.
\end{proposition}
\begin{proof}
The morphism of presheaves
\[i(Y)/i(R_Y)=i(Y)/i(R)_{i(Y)}\to i(X)/i(R)\]
is a monomorphism. As the functor $\#$ is left exact, it preserves monomorphisms and hence $Y/R_Y\to X/R$ is a monomorphism. The last assertion then follows from the commutative diagram
\[\begin{tikzcd}
Y\ar[r]\ar[d]&X\ar[d]\\
Y/R_Y\ar[r]&X/R
\end{tikzcd}\]
and the fact that $Y\to Y/R_Y$ is covering.
\end{proof}
In view of \ref{site sheaf quotient of subsheaf is image}, we can identify $Y/R_Y$ with a subsheaf of $X/R$.

\begin{proposition}\label{site sheaf stable subsheaf and subquotient correspond}
Let $R$ be a $\widetilde{\mathcal{C}}$-equivalence relation on a sheaf $X$. For any subsheaf $Y$ of $X$ stable under $R$, denote by $Y'=Y/R_Y$ the quotient considered as a subsheaf of $X'=X/R$. Then $Y=Y'\times_{X'}X$ and the maps $Y\mapsto Y/R_Y$ and $Y'\mapsto Y'\times_{X'}X$ give a bijective correspoondence between the set of subsheaves $Y$ of $X$ stable under $R$ and the set of subsheaves $Y'$ of $X'$.
\end{proposition}
\begin{proof}
If $Y'$ is a subsheaf of $X'$, then $Y'\times_{X'}X$ is a subsheaf of $X$ stable under $R$, and we have $(Y'\times_{X'}X)/R=Y'$. If $Y'$ is obtained by passing to quotient of a subsheaf $Y$ of $X$, then $Y$ is a subobject of $Y'\times_{X'}X$. It then suffices to show that if we have two subobjects $Y_1$ and $Y_2$ of $X$, stable under $R$ and $Y_1\sub Y_2$, and if the quotients $Y_1/R_{Y_1}$ and $Y_2/R_{Y_2}$ are identical, then $Y_1=Y_2$. For this, we are evidently reduced to the same assertion in the case $Y_2=X$. Denote then by $P$ (resp. $Q$) the presheaf $i(X)/i(R)$ (resp. $i(Y)/i(R_Y)$), the diagram
\[\begin{tikzcd}
Y\ar[d,hook]\ar[r]&Q\ar[d,hook]\\
X\ar[r]&P
\end{tikzcd}\]
is Cartesian. As we have a commutative diagram
\[\begin{tikzcd}
Q\ar[r,hook]\ar[d,hook]&Q^\#\ar[d,equal]\\
P\ar[r,hook]&P^\#
\end{tikzcd}\]
and $Q\mapsto Q^\#$ is covering, the monomorphism $Q\hookrightarrow P$ is covering, so $Q$ is a refinement of $P$. By base change, $Y$ is then a refinement of $X$. As $X$ and $Y$ are both sheaves, we conclude that $Y=X$.
\end{proof}
In particular, if $Y$ is a subsheaf of $X$ and $Y'=Y/R_Y$, then the preceding correspondence ddefines a subsheaf $\widebar{Y}$ of $X$, stable under $R$, containing $Y$ and minimal with these properties; this subsheaf is called the saturation of $Y$ for the equivalence relation $R$.

\paragraph{Description of the quotient of a sheaf by an equivalence relation}\label{site sheaf quotient by equivalence relation paragraph}
Now assume that the topology of $\mathcal{C}$ is subcanonical. In this case, we know that any covering sieve is universally effective epimorphic, and the canonical functor $i:\mathcal{C}\to\widehat{\mathcal{C}}$ factors through $\widetilde{\mathcal{C}}$.

\begin{proposition}\label{site sheaf quotient by equivalence relation char}
Let $R$ be a $\widetilde{\mathcal{C}}$-equivalence relation on a sheaf $X$. Let $F\in\Ob(\widehat{\mathcal{C}})$ be the presheaf defined as follows: for any $S\in\Ob(\mathcal{C})$\footnote{$R\times S$ is the equivalence relation on $X\times S$ defined by $R\times S\sub X\times X\times S\times S$ (induced by the diagonal) and $R_Z$ is the equivalence relation induced over $Z$.},
\begin{equation*}
F(S)=\{\text{sub-$S$-sheaves $Z$ of $X\times S$ stable under $R\times S$ whose quotient by $R_Z$ is $S$}\}.
\end{equation*}
Then for any sheaf $Y$, $\Hom(Y,F)$ is identified with the set
\[\{\text{sub-$Y$-sheaves of $X\times Y$ stable under $R\times Y$ whose quotient is $Y$}\}.\]
In particular, the subsheaf $R$ of $X\times X$ corresponds to a morphism $p:X\to F$ and the diagram
\[\begin{tikzcd}
R\ar[r,shift left=2pt,"p_1"]\ar[r,shift right=2pt,swap,"p_2"]&X\ar[r,"p"]&F
\end{tikzcd}\]
is exact, hence identifies $F$ with the quotient sheaf $X/R$.
\end{proposition}
\begin{proof}
Let $Q=X/R$. For any sheaf $Y$ and any morphism $f:Y\to Q$ corresponding to a section $s:Y\to Q\times Y$, consider the diagram
\begin{equation}\label{site sheaf quotient by equivalence relation char-1}
\begin{tikzcd}
&Z\ar[r]\ar[d,hook]&Y\ar[d,hook,"s"]\\
R\times Y\ar[r,shift left=2pt]\ar[r,shift right=2pt]&X\times Y\ar[r]&Q\times Y
\end{tikzcd}
\end{equation}
where the square is Cartesian. It is immediate from \cref{site sheaf stable subsheaf and subquotient correspond} that $Z$ is a sub-$Y$-sheaf of $X\times Y$ stable under $R\times Y$ whose quotient is $Y$; conversely, any $Z$ with these properties provides a unique section of $Q\times Y$ over $Y$. Taking $Y$ to be representable or arbitrary, we obtain an isomorphism $Q\cong F$ and the desired form of $\Hom(Y,F)$. Finally, consider the canonical morphism $X\to Q$, we immdediately see that it corresponds to the sub-$X$-sheaf $R$ of $X\times X$, which proves our assertion.
\end{proof}

\begin{corollary}\label{site sheaf quotient factor through if}
Let $G$ be a subfunctor of $F$ such that $\Hom(X,G)\sub\Hom(X,F)$ contains $R$. Then the canonical morphism $p:X\to F$ factors through $G$. As $p$ is covering, it follows that $G$ is a refinement of $F$. In particular, any subsheaf $G$ of $F$ verifying the preceding condition is equal to $F$.
\end{corollary}
\begin{proof}
By the identification of \cref{site sheaf quotient by equivalence relation char}, the hypothesis implies that $p:X\to F$ belongs to the image of $\Hom(X,G)$, whence it factors through $G$.
\end{proof}

We now consider the case where $X$ and $R$ are representable. Let's first introduce some terminology. In addition to the conditions (M1)--(M4) introduced in \ref{category universally effective relation paragraph}, we will use other conditions on a family $\mathcal{M}$ of morphisms of $\mathcal{C}$ (for completeness, we recall conditions (M1)--(M3)):
\begin{enumerate}[leftmargin=40pt]
    \item[(M1)] $\mathcal{M}$ is stable under base change.
    \item[(M2)] The composition of two elements of $\mathcal{M}$ belongs to $\mathcal{M}$.
    \item[(M3)] Any isomorphism belongs to $\mathcal{M}$.
    \item[($\text{M4}_\mathcal{T}$)] Any element of $\mathcal{M}$ is covering.
    \item[($\text{M5}_\mathcal{T}$)] Let $f:X\to Y$ be a morphism in $\mathcal{C}$. If there exists a covering sieve $R\hookrightarrow Y$ such that for any $Y'\to R$, $X\times_YY'\to Y'$ belongs to $\mathcal{M}$, then $f$ belongs to $\mathcal{M}$.
\end{enumerate}
Recall that (M1) and (M2) implies
\begin{enumerate}[leftmargin=40pt]
    \item[(M1')] The Cartesian product of two morphisms in $\mathcal{M}$ belongs to $\mathcal{M}$.
\end{enumerate}
and (M1) and ($\text{M4}_\mathcal{T}$) implies (by \cite{SGA3-1}, \Rmnum{4}, 4.3.9):
\begin{enumerate}[leftmargin=40pt]
    \item[(M4')] Any morphism in $\mathcal{M}$ is a universally effective epimorphism.
\end{enumerate}

The preceding conditions are verified for the family of covering morphisms, denoted by $\mathcal{M}_\mathcal{T}$, if $\mathcal{C}$ possesses fiber products. The results we are going to establish for a family $\mathcal{M}$ satisfying these conditions will apply in particular to the family $\mathcal{M}_\mathcal{T}$. In particular, we can take for $\mathcal{T}$ the canonical topology and for $\mathcal{M}$ the family of universally effective epimorphisms.

\begin{proposition}\label{site sheaf M-effective relation iff quotient representable}
Let $\mathcal{M}$ be a family of morphisms verifying conditions (M1)--($\text{M5}_\mathcal{T}$). Let $R$ be a $\widetilde{\mathcal{C}}$-equivalence relation on $X\in\Ob(\mathcal{C})$ of type $\mathcal{M}$, $\widetilde{X}$ be the sheaf associated with $X$, $\widetilde{R}$ the $\widetilde{\mathcal{C}}$-equivalence relation on $\widetilde{X}$ defined by $R$, and $\widetilde{X}/\widetilde{R}$ the quotient sheaf. For $R$ to be $\mathcal{M}$-effective, it is necessary and sufficient that $\widetilde{X}/\widetilde{R}$ is representable, and in this case it is represented by the quotient $X/R$.
\end{proposition}
\begin{proof}
Suppose that $R$ is $\mathcal{M}$-effective and let $Y=X/R$. The canonical morphism $p:X\to Y$ belongs to $\mathcal{M}$, hence is covering by ($\text{M4}_\mathcal{T}$). The corresponding morphism
\[\tilde{p}:\widetilde{X}\to\widetilde{Y}\]
is then a universally effective epimorphism in $\widetilde{\mathcal{C}}$, hence identifies $\widetilde{Y}$ with the quotient of $\widetilde{X}$ by the equivalence relation $R'$ defined by $\tilde{p}$. As the functor $\mathcal{C}\to\widetilde{\mathcal{C}}$ commutes with fiber products, $R'$ is none other than $\widetilde{R}$, because $R$ is the equivalence relation defined by $R$ (since it is effective). We then conclude that $\widetilde{X}/\widetilde{R}$ is represented by $Y$.\par
Conversely, suppose that $\widetilde{X}/\widetilde{R}$ is represented by an object $Y$ of $\mathcal{C}$. Let $p:X\to Y$ be the morphism induced by the canonical morphism $\widetilde{X}\to\widetilde{X}/\widetilde{R}$, which is a covering morphism by (\cite{SGA3-1}, \Rmnum{4}, 4.4.3). It is clear as before that $R$ is the equivalence relation defined by $p$, so it remains to show that $p\in\mathcal{M}$. But the Cartesian square
\[\begin{tikzcd}
R\ar[r,"\sim"]&X\times_YX\ar[d,"p_2"]\ar[r,"p_1"]&X\ar[d,"p"]\\
&X\ar[r,"p"]&Y
\end{tikzcd}\]
shows that the base change of $p$ by the covering morphism $p$ belongs to $\mathcal{M}$ (since $p_2\in\mathcal{M}$ by our hypothesis). We then conclude from (M1) and ($\text{M5}_\mathcal{T}$) that $p\in\mathcal{M}$.
\end{proof}

\begin{corollary}\label{site sheaf M-morphism kernel M-effective}
Let $\mathcal{M}$ be a family of morphisms verifying conditions (M1)--($\text{M5}_\mathcal{T}$) and $f:G\to G'$ be a morphism of $\mathcal{C}$-groups belonging to $\mathcal{M}$. Suppose that $\ker f$ is representable (for example, if $\mathcal{C}$ has a final object $e$), then the equivalence relation on $G$ defined by $H=\ker f$ is $\mathcal{M}$-effective and $G'$ represents the quotient sheaf $\widetilde{G}/\widetilde{H}$ for the topology $\mathcal{T}$.
\end{corollary}
\begin{proof}
This follows from \cref{category M-morphism kernel M-effective} and \cref{site sheaf M-effective relation iff quotient representable}.
\end{proof}

We are now in a position to state the main theroem of this paragraph. Before this, let recall the following result:
\begin{proposition}\label{site sheaf representable iff descent data effective}
Let $\{S_i\to S\}$ be a covering family and $Z$ be a sheaf over $S$. Suppose that for each $i$, the $S_i$-functor $Z\times_SS_i$ is represented by an object $T_i$. Then the family $T_i$ is endowed with a canonical descent data relative to $\{S_i\to S\}$. For $Z$ to be representable, it is necessary and sufficient that this descent data is effective, and in this case the descent object represents $Z$.
\end{proposition}
\begin{proof}
By (\cite{SGA3-1}, \Rmnum{4}, 4.4.3), $\{S_i\to S\}$ is universally effective epimorphic in $\widetilde{\mathcal{C}}$, hence is a descent family in $\widetilde{\mathcal{C}}$. If $Z$ is represented by an object $T$, the $T\times_SS_i$ (considered as sheaves) is isomorphic to $Z\times_SS_i$, hence the descent data over $T_i$ is effective and the descent object (necessarily unique) is isomorphic to $Z$. Conversely, suppose that the canonical descent data over $T_i$ is effective and let $T$ be a descent object. As the family $\{S_i\to S\}$ is a descent family, there exsits an $S$-morphism $T\to Z$ whose base change to $S_i$ is the canonical morphism $T_i\to Z\times_SS_i$. This morphism is therefore locally an isomorphism, and it follows from (\cite{SGA3-1}, \Rmnum{4}, 4.4.8) that it is an isomorphism.
\end{proof}

\begin{theorem}\label{site sheaf quotient by M-effective char}
Let $\mathcal{M}$ be a family of morphisms verifying conditions (M1)--($\text{M5}_\mathcal{T}$) and $R$ be a $\mathcal{C}$-equivalence relation of type $\mathcal{M}$ on an object $X$ of $\mathcal{C}$. Consider the functor $F\in\Ob(\widehat{\mathcal{C}})$ defined as follows:
\[F(S)=\{\text{sub-$S$-sheaf $Z$ of $X\times S$ stable under $R\times S$ whose quotient by $R_Z$ is $S$}\}.\]
Let $F_0$ be the sub-functor of $F$ such that $F_0(S)$ is formed by representable $Z\in F(S)$, that is,
\[F_0(S)=\left\{\parbox{4in}{%
sub-$\mathcal{C}_{/S}$-objects $Z$ of $X\times S$ stable under $R\times S$ such that $R_Z$ is $\mathcal{M}$-effective and the quotient of $Z$ by $R_Z$ is $S$%
}\right\}.\]
\begin{enumerate}
    \item[(a)] The morphism $p:X\to F$ defined by the sub-object $R$ of $X\times X$ identifies $F$ with the quotient sheaf of $X$ by $R$.
    \item[(b)] The following conditions are equivalent:
    \begin{enumerate}
        \item[(\rmnum{1})] $F$ is representable.
        \item[(\rmnum{2})] $F_0$ is representable.
        \item[(\rmnum{3})] $R$ is $\mathcal{M}$-effective.
    \end{enumerate}
    and under these conditions, we have $F=F_0=X/R$.
    \item[(c)] Let $\mathcal{N}$ be a family of morphisms which is stable under base change and such that for any covering family $\{S_i\to S\}$ and any family $\{T_i\to S_i\}$ of morphisms in $\mathcal{N}$, any descent data on the $T_i$ relative to $\{S_i\to S\}$ is effective. Suppose that $X$ is squarable and the morphism $R\to X\times X$ belongs to $\mathcal{N}$, then $F_0=F$.
\end{enumerate}
\end{theorem}
\begin{proof}
The proof of (\rmnum{1}) follows from \cref{site sheaf quotient by equivalence relation char}. As for (\rmnum{2}), we have seen the equivalence of (\rmnum{1}) and (\rmnum{3}) as well as the equality $F=X/R$ (cf. \cref{site sheaf M-effective relation iff quotient representable}). It remains to prove that (\rmnum{2}) or (\rmnum{3}) implies $F_0=F$, but for this we first note that $F_0$ is indeed a sub-functor of $F$. In fact, for any $S\in\Ob(\mathcal{C})$ and $Z\in F_0(S)$, the morphism $Z\to S$ belongs to $\mathcal{M}$ and hence is squarable, so $Z\times_SS'$ belongs to $F_0(S')$ for any $S'\to S$. As $R\in F(X)$ belongs to $F_0(X)$, \cref{site sheaf quotient factor through if} shows that (\rmnum{2}) implies $F_0=F$.\par
Now suppose that (\rmnum{3}) is satisfied and let $Q$ be an object of $\mathcal{C}$ representing $X/R$. Then the morphism $X\to Q$ belongs to $\mathcal{M}$ and, for any $S\in\Ob(\mathcal{C})$ and any $Z\in F(S)$, the diagram (\ref{site sheaf quotient by equivalence relation char-1}) of \cref{site sheaf quotient by equivalence relation char} shows that $Z=S\times_{(Q\times S)}X\times S$ is representable, and $Z\to S$ belongs to $\mathcal{M}$, hence $Z\in F_0(S)$.\par
Finally, to prove (c), let $f:S\to F$ be a morphism corresponding to $Z\in F(S)$. We must show that $f$ factors through $F_0$, which means $Z$ is representable. For this, we first note that if $f$ factors through $X$, then it is the image of an element $x_0\in X(S)$, and the corresponding sheaf $Z$ is defined by the Cartesian squares (since the morphism $p:X\to F$ corresponds to the subsheaf $R$ of $X\times X$)
\[\begin{tikzcd}
Z\ar[r]\ar[d]&R_S\ar[r]\ar[d]&R\ar[d]\\
X_S\ar[r,"{\id_{X_S}\times\tau_{x_0}}"]&X_S\times_SX_S\ar[r]&X\times X
\end{tikzcd}\]
where $\tau_{x_0}$ is the morphism $X_S\to X_S$ defined by $(x,s)\mapsto(x_0(s),s)$\footnote{Unwinding the definitions, we see that for any $S'\to S$, $Z(S)$ consists of elements $(x,s)\in X(S')\times S'$ such that $(x_0(s),x)\in R(S')$, so its quotient by $R_Z$ is $S$.}. Moreover, as $R\to X\times X$ belongs to $\mathcal{N}$, so is $Z\to X_S$.\par
For the general case, as $X\to F$ is a covering morphism, there exists a covering family $\{S_i\to S\}$ and for each $i$ a morphism $S_i\to X$ fitting into the diagram
\[\begin{tikzcd}
X\ar[r]&F\\
S_i\ar[r]\ar[u]&S\ar[u,swap,"f"]
\end{tikzcd}\]
By the preceding arguments, the morphism $f_i:S_i\to F$ defined by this diagram belongs to $F_0(S_i)$ and corresponds to the subsheaf $Z\times_SS_i$ of $X_{S_i}$, and the morphism $Z\times_SS_i\to X_{S_i}$ belongs to $\mathcal{N}$. As the family $\{X_{S_i}\to X_S\}$ is covering, the descent data on $Z\times_SS_i$ provides a descent object over $S$ by our hypothesis, which must represents $Z$ in view of \cref{site sheaf representable iff descent data effective}.
\end{proof}

\begin{corollary}\label{site sheaf quotient by M-effective Hom set char}
Let $R$ be an $\mathcal{M}$-effective equivalence relation on $X$. For any sheaf $F$, the map
\[\Hom(X/R,F)\to\Hom(X,F)\]
identifies $\Hom(X/R,F)$ with the subset formed by morphisms $X\to F$ compatible with $R$.
\end{corollary}
\begin{proof}
By \cref{site sheaf quotient by M-effective char}, $X/R$ represents the quotient sheaf $\widetilde{X}/\widetilde{R}$, and the defining property of $\widetilde{X}/\widetilde{R}$ gives the assertion.
\end{proof}

\begin{remark}\label{site sheaf quotient by M-effective descent morphism in N remark}
In the hypothesis of \cref{site sheaf quotient by M-effective char}~(\rmnum{3}), if we further suppose that the descent object $T$ is such that the morphism $T\to S$ belongs to $\mathcal{N}$, then the inclusion morphism $Z\to X_S$ also belongs to $\mathcal{N}$, as it is obtained by the descent data on the morphisms $Z\times_SS_i\to X_{S_i}$, which are in $\mathcal{N}$.
\end{remark}

\begin{remark}\label{site sheaf quotient by M-effective M1 to M4 remark}
We have proved the implications (\rmnum{3})$\Rightarrow$(\rmnum{2})$\Rightarrow$(\rmnum{1}) and (\rmnum{3})$\Rightarrow$$[F_0=F=X/R]$ in \cref{site sheaf quotient by M-effective char} without resorting the "sufficient" part of \cref{site sheaf quotient by M-effective char}, which is the only place we use condition ($\text{M5}_\mathcal{T}$). Therefore, they remain valid if $\mathcal{M}$ only satisfies conditions (M1)--($\text{M4}_\mathcal{T}$). An example of such a family of that of squarable covering morphisms.
\end{remark}


\begin{corollary}\label{site sheaf quotient by M-effective quotient for subcanonical topology}
Under the conditions of \cref{site sheaf quotient by M-effective char}~(\rmnum{2}), $X/R$ is also the quotient sheaf of $X$ by $R$ for any intermediate topology between $\mathcal{T}$ and the canonical topology.
\end{corollary}
\begin{proof}
If $\mathcal{T}'$ is an intermediate topology between $\mathcal{T}$ and the canonical topology, then $\mathcal{M}$ satisfies (M1)--($\text{M4}_{\mathcal{T}'}$), so $F_0$ is identified with the quotient sheaf of $X$ by $R$ for $\mathcal{T}'$ (\cref{site sheaf quotient by M-effective M1 to M4 remark}), which is $X/R$.
\end{proof}

\begin{corollary}\label{site sheaf quotient by universal effective represents quotient sheaf}
Let $R$ be a universally effective equivalence relation on $X$. Then the object $X/R$ of $\mathcal{C}$ represents the quotient sheaf of $X$ by $R$ for the canonical topology. Moreover, $(X/R)(S)$ is the set of sub-$\mathcal{C}_{/S}$-objects $Z$ of $X_S$ stable under $R\times S$ such that $R_Z$ is universally effective and the quotient of $Z$ by $R_Z$ is $S$.
\end{corollary}

\begin{corollary}
Let $\mathcal{M}$ be the family of squarable covering morphisms. If $R$ is an $\mathcal{M}$-effective equivalence relation on $X$, then $X/R$ of $\mathcal{C}$ represents the quotient sheaf of $X$ by $R$ and it also represents the functor $F_0$ of \cref{site sheaf quotient by M-effective char}.
\end{corollary}

While in questions involving exclusively projective limits (fiber products, algebraic structures, etc.) we can identify $\mathcal{C}$ indiscriminately with a full subcategory of $\widetilde{\mathcal{C}}$ or of $\widehat{\mathcal{C}}$, it is not the same in those which combine projective and inductive limits. In questions involving both projective limits and inductive limits (in particular passages to the quotient), we should consider the given category as embedded in the category of sheaves. Thus if $R$ is a $\mathcal{C}$-equivalence relation on the object $X$ of $\mathcal{C}$, $X/R$ will denote the quotient sheaf of $X$ by $R$ (designated previously by $(i(X)/i(R))^\#$), so in the case where this sheaf is representable, the object representing it. The previous results show that in the most important cases, a quotient in $\mathcal{C}$ will also be a quotient in the category of sheaves.\par

We now give an example of the usage of effectivity criteria. As before, let $\mathcal{T}$ be a subcanonical topology on $\mathcal{C}$ and choose a family $\mathcal{M}$ of morphisms satisfying conditions (M1)--($\text{M5}_\mathcal{T}$). We consider a family $\mathcal{N}$ of morphisms in $\mathcal{C}$ with the following properties:
\begin{enumerate}[leftmargin=40pt]
    \item[(N1)] $\mathcal{N}$ is stable under base change.
    \item[($\text{N}_\mathcal{T}$)] The morphisms of $\mathcal{N}$ have descent property for the given topology. That is, for any $S\in\Ob(\mathcal{C})$, any covering family $\{S_i\to S\}$ and any family $\{T_i\to S_i\}$ of morphisms in $\mathcal{N}$, any dascent data on $T_i$ relative to $\{S_i\to S\}$ is effective, and if $T$ is the descent object, the morphism $T\to S$ belongs to $\mathcal{N}$.
\end{enumerate}
As any element of $\mathcal{M}$ is covering, ($\text{N}_\mathcal{T}$) implies the following property:
\begin{enumerate}[leftmargin=40pt]
    \item[($\text{N}_\mathcal{M}$)] If $Y'\to X'$ belongs to $\mathcal{N}$ and $X'\to X$ belongs to $\mathcal{M}$, any descent data over $Y'$ relative to $X'\to X$ is effective. If $Y$ is the descent object, then $Y\to X$ belongs to $\mathcal{N}$.
\end{enumerate}
A particular important example is the following: $\mathcal{C}$ is the category of schemes, $\mathcal{T}$ is the fpqc topology, $\mathcal{M}$ is the family of faithfully flat and quasi-compact morphisms, $\mathcal{N}$ is the family of closed immersions, or that of quasi-compact immersions.\par
By \cref{site sheaf quotient by M-effective char}, we then have the following result (cf. \cref{site sheaf quotient by M-effective descent morphism in N remark}):
\begin{proposition}\label{site sheaf quotient by type MN char}
Let $X$ be a squarable object in $\mathcal{C}$ and $R$ be an equivalence relation on $X$ of type $\mathcal{M}$ such that $R\to X\times X$ belongs to $\mathcal{N}$. Then the quotient sheaf $X/R$ is defined by
\[(X/R)(S)=\left\{\parbox{4in}{%
sub-$\mathcal{C}_{/S}$-objects $Z$ of $X\times S$ stable under $R\times S$ such that $Z\to X_S$ belongs to $\mathcal{N}$, that $R_Z$ is $\mathcal{M}$-effective, and that the quotient of $Z$ by $R_Z$ is $S$%
}\right\}.\]
\end{proposition}

Moreover, we have the following correspondence of stable subobjects of $X$ and $\mathcal{M}$-effective equivalence relations.
\begin{proposition}\label{site sheaf quotient by M-effective stable N-subobject correspond}
Let $X\in\Ob(\mathcal{C})$ and $R$ be an $\mathcal{M}$-effective equivalence relation on $X$.
\begin{enumerate}
    \item[(a)] For any sub-object $Y$ of $X$, stable under $R$ and such that $Y\to X$ belongs to $\mathcal{N}$, the equivalence relation induced on $Y$ by $R$ is $\mathcal{M}$-effective and the quotient $Y/R_Y=Y'$ is a sub-object of $X'=X/R$ such that $Y'\to X'$.
    \item[(b)] The map $Y\mapsto Y'$ is a bijection from the set of sub-objects $Y$ of $X$ stable under $R$ such that $Y\to X$ belongs to $\mathcal{N}$ to the set of sub-objects $Y'$ of $X'$ such that $Y'\to X'$ belongs to $\mathcal{N}$. The inverse map is given by $Y'\to Y'\times_{X'}X$.
\end{enumerate}
\end{proposition}
\begin{proof}
As $R$ is $\mathcal{M}$-effective, the morphism $X\to X'$ belongs to $\mathcal{M}$. Let $Y'$ be a sub-object of $X'$ such that the canonical morphism $Y'\to X'$ belongs to $\mathcal{N}$. Then, the sub-object $Y=Y'\times_{X'}X$ of $X$ is stable under $R$, and the morphism $Y\to X$ (resp. $Y\to Y'$) belongs to $\mathcal{N}$ (resp. $\mathcal{M}$) since $\mathcal{N}$ and $\mathcal{M}$ are stable under base change. By \cref{site sheaf stable subsheaf and subquotient correspond}, the quotient sheaf $R/R_Y$ is represented by $Y'$ and hence, by \cref{site sheaf M-effective relation iff quotient representable}, $R_Z$ is $\mathcal{M}$-effective.\par
Conversely, any sub-object $Y$ of $X$, stable under $R$ and such that the morphism $Y\to X$ belongs to $\mathcal{N}$, is obtained in this way. In fact, if $Y$ is stable under $R$, its two inverse images in $R=X\times_{X'}X$ are identical and $Y$ is endowed with a descent data relative to $X\to X'$; our assertion then follows from ($\text{N}_\mathcal{M}$).
\end{proof}

\begin{corollary}
Let $X\in\Ob(\mathcal{C})$ and $R$ be an $\mathcal{M}$-effective equivalence relation on $X$. Suppose that $R\to X\times X$ belongs to $\mathcal{N}$, then for any $Y$ as in \cref{site sheaf quotient by M-effective stable N-subobject correspond}, $R_Y\to Y\times Y$ belongs to $\mathcal{N}$ and hence, by \cref{site sheaf quotient by type MN char}, we have
\[(Y/R_Y)(S)=\left\{\parbox{4in}{%
sub-$\mathcal{C}_{/S}$-objects $Z$ of $Y\times S$ stable under $R_Y\times S$ such that $Z\to Y_S$ belongs to $\mathcal{N}$, that $R_Z$ is $\mathcal{M}$-effective, and that the quotient of $Z$ by $R_Z$ is $S$%
}\right\}.\]
\end{corollary}

\subsection{Passage to quotient and algebraic structures}
\paragraph{Principal homogeneous bundles} 
We recall that an object $X$ in $\widehat{\mathcal{C}}$ with a (right) group action by a group functor $H$ is called \textbf{formally principal homogeneous} under $H$ if the canonical morphism
\[X\times H\to X\times X,\quad (x,h)\mapsto (x,xh)\]
is an isomorphism. Equivalently, this means for any $S\in\Ob(\mathcal{C})$, $X(S)$ is formally principal homogeneous under $H(S)$, which is therefore empty or principal homogeneous under $H(S)$. In particular, if we act $H$ on itself by (right) translations, then $H$ is formally principal homogeneous under itself. The $H$-object $X$ is called trivial if it is isomorphic to $H$ acted by right translations.
\begin{proposition}\label{category formally principal homogeneous global section char}
If $X$ be formally principal homogeneous under $H$, we have an isomorphism
\[\Gamma(X)\stackrel{\sim}{\to}\Iso_H(H,X)\]
of principal homogeneous sets under $\Gamma(H)$.
\end{proposition}
\begin{proof}
To any section $x$ of $X$, we can associate the morphism $H\to X$ defined set-wise by $h\mapsto xh$. The assertion is then immediate.
\end{proof}

\begin{corollary}
We have an isomorphism of $H$-objects
\[X\stackrel{\sim}{\to} \sIso_H(H,X).\]
Moreover, for $X$ to be trivial, it is necessary and sufficient that $X$ is formally principal homogeneous and possesses a global section.
\end{corollary}

\begin{definition}
Let $\mathcal{C}$ be a site. An $S$-object $X$ with an action by $H$ is called a \textbf{principal homogeneous bundle under $H$} if it is \textbf{locally trivial}, that is, if the following equivalent conditions are satisfied:
\begin{enumerate}
    \item[(\rmnum{1})] The set of morphisms $T\to S$ such that (the functor) $X\times_ST$ is trivial under $H\times_ST$ is a refinement of $S$.
    \item[(\rmnum{2})] There exists a covering family $\{S_i\to S\}$ such that for each $i$, the $S_i$-functor $X\times_SS_i$ is trivial under $H\times_SS_i$.
\end{enumerate}
\end{definition}

\begin{proposition}\label{site formally principal homogeneous under M-group iff}
Let $\mathcal{C}$ be a site and $\mathcal{M}$ be a family of morphisms in $\mathcal{C}$ satisfying conditions (M1)--($\text{M}5_\mathcal{T}$) of \ref{site sheaf quotient by equivalence relation paragraph}. Let $H$ be an $S$-group such that the structural morphism $H\to S$ belongs to $\mathcal{M}$ and $P$ be an $S$-object acted by $H$. The following conditions are equivalent:
\begin{enumerate}
    \item[(\rmnum{1})] $P$ is a principal homogeneous bundle under $H$.
    \item[(\rmnum{2})] $P$ is formally principal homogeneous under $H$ and the structural morphism $P\to S$ belongs to $\mathcal{M}$.
    \item[(\rmnum{3})] There exists a morphism $S'\to S$ in $\mathcal{M}$ such that the base change of $P$ to $S'$ is trivial, that is, $P\times_SS'$ is trivial under $H\times_SS'$.
    \item[(\rmnum{4})] $H$ acts freely and $\mathcal{M}$-effectively on $P$ and the quotient $P/H$ is isomorphic to $S$.
\end{enumerate}
\end{proposition}
\begin{proof}
We first note that (\rmnum{2}) and (\rmnum{4}) are equivalent, in view of the fact that, in either case, $P\to S$ belongs to $\mathcal{M}$, hence is squarable, which ensures the representability of $H\times_SP$ and $P\times_SP$. It is clear that (\rmnum{2}) implies (\rmnum{3}), because we can take $S'=P$, and the hypothesis that $P$ is formally principal homogeneous implies that $P\times_SP$ is trivial under $H\times_SP$, since it has a section (the diagonal section $P\to P\times_SP$). On the other hand, (\rmnum{3}) implies (\rmnum{1}), since $\{S'\to S\}$ is a covering family by condition ($\text{M4}_\mathcal{T}$). It then remains to show that (\rmnum{1})$\Rightarrow$(\rmnum{2}). In this case, the morphism of sheaves $P\times_SH\to P\times_SP$ is locally an isomorphism, hence an isomorphism (\cite{SGA3-1} \Rmnum{4}, 4.5.8); $P$ is then formally principal homogeneous. The structural morphism $P\to S$ is locally isomorphic to the structural morphism $H\to S$, which belongs to $\mathcal{M}$. It is then an element of $\mathcal{M}$ by (M1) and ($\text{M5}_\mathcal{T}$).
\end{proof}

We note that if $H$ acts freely on an $S$-object $X$ and $p:X\to Y=X/H$ is the quotient morphism, then we have an induced morphism
\[(H\times_SY)\times_YX=H\times_SX\to X.\]
Therefore, $H\times_SY$ has an induced action on $X$ over $Y$, and the quotient $X/H\times_SY$ is $Y$. The equivalence of (\rmnum{1}) and (\rmnum{4}) in \cref{site formally principal homogeneous under M-group iff} can therefore be generalized to the following proposition:
\begin{proposition}\label{site M-effective principal homogeneous bundle iff}
Under the same hypothesis of \cref{site formally principal homogeneous under M-group iff}, assume that the topology $\mathcal{T}$ is subcanonical. Let $H$ be an $S$-group and $X$ be an $S$-object over which $H$ acts (on right). Suppose that the structural morphism $H\to S$ belongs to $\mathcal{M}$, then the following conditions are equivalent:
\begin{enumerate}
    \item[(\rmnum{1})] $H$ acts freely and $\mathcal{M}$-effectively on $X$. 
    \item[(\rmnum{2})] There exists an $S$-morphism $p:X\to Y$ compatible with the equivalence relation on $X$ defined by $H$ and such that the induced action of $H\times_SY$ on $X$ over $Y$ makes $X$ a principal homogeneous bundle under $H_Y$ over $Y$.
\end{enumerate}
Under these conditions, $p$ identifies $Y$ with the quotient $X/H$.
\end{proposition}

\begin{corollary}\label{site covering morphism of quotient by ker is torsor}
Let $\mathcal{C}$ be a category possessing a final object, arbitrary fiber products, and endowed with a subcanonical topology $\mathcal{T}$. Let $f:G\to H$ be a morphism of $\mathcal{C}$-groups and $K=\ker f$, and suppose that $f$ belongs to a family $\mathcal{M}$ satisfying conditions (M1)--($\text{M}5_\mathcal{T}$). Then $H$ represents the quotient sheaf $G/K$, and $f$ is a $K_H$-torsor\footnote{We also say that $G$ is a $K$-torsor over $H$.}.
\end{corollary}
\begin{proof}
In fact, as $f$ is covering, it is a universally effective epimorphism, so $H$ is the quotient of $G$ by the equivalence relation $R(f)=G\times_HG$, which is also the equivalence relation defined by $K$. On the other hand, the morphism $G\times K\to G\times_HG$, $(g,k)\mapsto(g,gk)$ is an isomorphism of $K_G=G\times_HK_H$-objects. Since the morphism $f:G\to H$ is covering, $f$ is a $K_H$-torsor by \cref{site formally principal homogeneous under M-group iff}~(\rmnum{2}).
\end{proof}

We can now specify \cref{site sheaf quotient by M-effective char} in the case of passage to quotient by a group action:
\begin{proposition}\label{site sheaf quotient by M-effective free group action char}
Under the hypothesis of \cref{site formally principal homogeneous under M-group iff}, assume that the topology $\mathcal{T}$ is subcanonical and denote by $F_0$ the functor over $S$ defined as follows: for any $S'\to S$, $F_0(S')$ is the set of representable sub-$S'$-functors $Z$ of $X\times_SS'$, stable under $H\times_SS'$ and is a principal homogeneous bundle under the induced $S'$-group action.
\begin{enumerate}
    \item[(a)] The following conditions are equivalent:
    \begin{enumerate}
        \item[(\rmnum{1})] The action of $H$ on $X$ is $\mathcal{M}$-effective and free.
        \item[(\rmnum{2})] $F_0$ is representable.
    \end{enumerate}
    Under these conditions, we have $F_0=X/H$.
    \item[(b)] Let $\mathcal{N}$ be a family of morphisms which is stable under base change and such that for any covering family $\{S_i\to S\}$ and any family $\{T_i\to S_i\}$ of morphisms in $\mathcal{N}$, any descent data on the $T_i$ relative to $\{S_i\to S\}$ is effective. Suppose that $X$ is squarable and the morphism $X\times_SH\to X\times_SX$ belongs to $\mathcal{N}$, then the morphism $p:X\to F_0$ corresponding to the sub-object $X\times_SH$ of $X\times_SX$ identifies $F_0$ with the quotient sheaf $X/H$.
\end{enumerate}
\end{proposition}

\paragraph{Group structure and passage to quotient}
We are now intersted in the algebraic structure induced on the quotient $G/H$ of a group by a subgroup. We first consider category of sheaves over $\mathcal{C}$ for an arbitrary topology. By taking the canonical topology and apply \cref{site sheaf quotient by M-effective M1 to M4 remark}, we then obtain results for the passage to universally effective quotients in $\mathcal{C}$.

\begin{proposition}\label{site quotient sheaf by subgroup G-action structure}
Let $u:H\to G$ be a monomorphisms of sheaves of groups. Then there exists a unique $G$-object structure on the quotient sheaf $G/H$ such that the canonical morphism
\[p:G\to G/H\]
is a morphism of $G$-objects. This structure is functorial relative to $(G,H)$: if we have a commutative diagram
\[\begin{tikzcd}
H\ar[r]\ar[d]&G\ar[d]\\
H'\ar[r]&G'
\end{tikzcd}\]
Then the induced morphism $G/H\to G'/H'$ is compatible with $G\to G'$.
\end{proposition}
\begin{proof}
The sheaf $G/H$ is the sheaf associated with the presheaf
\[i(G)/i(H):S\mapsto G(S)/H(S);\]
as $\#$ is left exact, it transforms objects acted by groups into objects acted by groups. Since the presheaf $i(G)/i(H)$ is endowed with an action by $i(G)$, $G/H=(i(G)/i(H))^\#$ is endowed with an action by $(i(G))^\#=G$. This structure clearly has the required properties.
\end{proof}

\begin{corollary}\label{site quotient by universally effective subgroup G-action structure}
Let $u:H\to G$ be a monomorphism of $\mathcal{C}$-groups. Suppose that the action of $H$ on $G$ is universally effective, then there exists a unique $G$-object structure on the quotient object $G/H$ in $\mathcal{C}$ such that $p:G\to G/H$ is a morphism of $G$-objects. This structure is functorial relative to $(G,H)$.
\end{corollary}

\begin{proposition}\label{site quotient sheaf by normal subgroup group structure}
Let $u:H\to G$ be a monomorphism of sheaves of groups which identifies $H$ with a normal subsheaf of $G$. Then there exists a unique group structure on the quotient sheaf $G/H$ such that the caonical morphism $p:G\to G/H$ is a group morphism. This structure is functorial relative to the couple $(G,H)$ ($H$ being normal).
\end{proposition}
\begin{proof}
The proof is the same as \cref{site quotient sheaf by subgroup G-action structure}.
\end{proof}

\begin{corollary}\label{site quotient by universally effective normal subgroup group structure}
Let $u:H\to G$ be a monomorphism of $\mathcal{C}$-groups identifying $H$ with a normal subgroup of $G$. Suppose that the action of $H$ on $G$ is universally effective, then there exists a group structure on the quotient object $G/H$ in $\mathcal{C}$ such that $p:G\to G/H$ is a morphism of groups. This structure is functorial relative to $(G,H)$ ($H$ being normal and acts universally effectively).
\end{corollary}

We can characterize the group structure of $G/H$ in the following way:
\begin{proposition}\label{site quotient by normal subgroup universal prop}
Under the conditions of \cref{site quotient sheaf by normal subgroup group structure}, let $K$ be a $\mathcal{C}$-group and $f:G\to K$ be a morphism. The following conditions are equivalent:
\begin{enumerate}
    \item[(\rmnum{1})] $f$ is a morphism of groups compatible with the equivalence relation defined by $H$.
    \item[(\rmnum{2})] $f$ is a morphism of groups inducing the trivial morphism $H\to K$.
    \item[(\rmnum{3})] $f$ factors into a morphism of groups $G/H\to K$.
\end{enumerate}
\end{proposition}
\begin{proof}
The equivalence of (\rmnum{1}) and (\rmnum{2}) is proved set-wisely. We evidently have (\rmnum{3})$\Rightarrow$(\rmnum{2}). The equivalence of (\rmnum{3}) and (\rmnum{2}) then follows from the formula
\[\Hom(G/H,K)\cong\Hom(i(G)/i(H),K)\]
and the definition of the group structure of $G/H$.
\end{proof}

\begin{remark}\label{site quotient by normal subgroup factorization mono iff kernel}
In the preceding situation, if the kernel of $f$ is exactly $H$, then the morphism $G/H\to K$ which factors $f$ is a monomorphism. This follows from \cref{category universal effective equivalence factor monomorphism iff}.
\end{remark}

In the case of sheaves of groups, we can precise \cref{site sheaf stable subsheaf and subquotient correspond} as following:

\begin{proposition}\label{site sheaf quotient by normal subgroup correspond}
Let $G$ be a sheaf of groups, $H$ be a normal subsheaf of groups. For any subsheaf of groups $K$ of $G$ containing $H$, let $K'$ be the quotient group $K/H$ considered as a a subgroup of $G'=G/H$. Then we have $K=K'\times_{G'}G$, and the maps $K\mapsto K/H$, $K'\mapsto K'\times_{G'}G$ define a bijection between the set of subsheaves of groups of $G$ containing $H$ and the set of subsheaves of groups of $G'$. In this correspondence, normal subgroups of $G$ corresponds to that of $G'$.
\end{proposition}
\begin{proof}
The first assertion follows equally from \cref{site sheaf stable subsheaf and subquotient correspond} and \cref{category equivalence relation by free action stable subobject iff}. It remains to show that $K$ is normal in $G$ if and only if $K'$ is normal in $G'$. If $K$ is normal in $G$, then the presheaf $i(K)/i(H)$ is normal in $i(G)/i(H)$, and the same is true for the associated sheaves. Conversely, if $K'$ is normal in $G'$, then the fiber product $K'\times_{G'}G$ is normal in $G$, which is equal to $K$.
\end{proof}
Now if $L$ is a subsheaf of groups of $G$, then there exists a smallest normal subsheaf of groups $\widebar{L}$ of $G$ containing $L$, called the saturation of $L$. In fact, we have $\widebar{L}=L\cdot H$.

\begin{proposition}\label{site sheaf quotient by normal subgroup second isomorphism}
Under the preceding conditions, $L\cdot H$ is a subsheaf of groups of $G$ containing $H$ and the image of $L$ in $G/H$ is identified with
\[(L\cdot H)/H\cong L/(H\cap L).\]
\end{proposition}
\begin{proof}
Denote by $L'$ the image sheaf of $L$ in $G/H$. This is a subsheaf of groups of $G/H$ corresponding to $L\cdot H$ by \cref{site sheaf quotient by normal subgroup correspond}. As the morphism $L\to L'$ is covering, hence a universally effective epimorphism of sheaves, it follows from \cref{site sheaf equivalence relation is universally effective} that $L'$ is identified with the quotient of $L$ by the kernel of $L\to L'$, which is evidently $H\cap L$.
\end{proof}

Finally, we consider the following case: we have a sheaf of groups $G$, a subsheaf of groups $K$ of $G$ and a subsheaf of groups $H$ of $K$, which is normal in $K$. Let us first define a (right) action of the sheaf in groups $H\backslash K$ ($=K/H$) on $G/H$. The group $K$ operates by right translations on $G$. As $H$ is normal in $K$, this operation is compatible with the equivalence relation defined by the action of $H$ and thus defines an operation of $K$ on $G/H$, that is, a morphism of the opposite group $K^\op$ to $\sAut(G/H)$. Since the latter is a sheaf (cf. \cite{SGA3-1}, \Rmnum{4}, 4.5.13) and that this morphism is trivial on $H$, it factors through $K/H$ and defines the desired operation. Since the right and left operations of $G$ on itself commute, the operations of $G$ and $K/H$ on $G/H$ commute.

\begin{proposition}\label{site sheaf quotient by normal subgroup third isomorphism}
Under the preceding conditions, $K/H$ acts freely on $G/H$ (on the right) and we have a canonical isomorphism of sheaves operated by $G$:
\[(G/H)/(K/H)\cong G/K.\]
If $K$ is normal in $G$, then $K/H$ is normal in $G/H$ and this isomorphism is a group isomorphism.
\end{proposition}
\begin{proof}
We have an isomorphism of presheaves
\[i(G)/i(K)\stackrel{\sim}{\to} (i(G)/i(H))/(i(K)/i(H))\]
which respects the action of $i(G)$. The result then follows by applying $\#$ on both sides.
\end{proof}

\begin{corollary}\label{site sheaf quotient by normal subgroup third isomorphism M-effective iff}
Let $G$ be a $\mathcal{C}$-group, $K$ be a sub-$\mathcal{C}$-group of $G$, $H$ be a normal sub-$\mathcal{C}$-group of $K$. Let $\mathcal{M}$ be a family of morphisms in $\mathcal{C}$ verifying the conditions (M1)--($\text{M5}_\mathcal{T}$). Suppose that the right action of $H$ on $G$ (resp. $K$) is $\mathcal{M}$-effective, then $K/H$ acts freely on $G/H$, and this action commutes with that of $G$. The following conditions are equivalent:
\begin{enumerate}
    \item[(\rmnum{1})] The action of $K$ on $G$ is $\mathcal{M}$-effective.
    \item[(\rmnum{2})] The action of $K/H$ on $G/H$ is $\mathcal{M}$-effective.
\end{enumerate}
Under these conditions, we have an isomorphism of $G$-objects in $\mathcal{C}$:
\[(G/H)/(K/H)\cong G/K.\]
\end{corollary}
\begin{proof}
Since $H$ acts $\mathcal{M}$-effectively on $G$ and $K$, by \cref{site sheaf quotient by normal subgroup third isomorphism} we have a diagram
\[\begin{tikzcd}
H\ar[r,hook]&K\ar[r,hook]\ar[d]&G\ar[d]\\
&K/H\ar[r,hook]&G/H\ar[d]\\
&&G/K
\end{tikzcd}\]
where the square is Cartesian. Since $\mathcal{M}$ is stable under composition, if $K/H$ acts on $G/H$ $\mathcal{M}$-effectively, then we conclude that $G\to G/K\in\mathcal{M}$, so $K$ acts on $G$ $\mathcal{M}$-effectively. Conversely, if $G\to G/K$ belongs to $\mathcal{M}$, then consider the Cartesian diagram
\[\begin{tikzcd}
K\times G\ar[r]\ar[d]&G\ar[d]\\
(K/H)\times(G/H)\ar[r]&(G/H)
\end{tikzcd}\]
By hypothesis, the morphism $K\times G\to G\times G$ belongs to $\mathcal{M}$ (\cref{category equivalence relation M-effective prop}), and $G\to G/H$ belongs to $\mathcal{M}$. We then conclude from (M1) and ($\text{M5}_\mathcal{T}$) that $(K/H)\times(G/H)\to G/H$ belongs to $\mathcal{M}$, so the equivalence relation on $G/H$ defined by $K/H$ is of type $\mathcal{M}$. Since the quotient of $G/H$ by this equivalence relation is represented by $G/K$, we conclude from \cref{site sheaf quotient by M-effective char} that the action of $K/H$ on $G/H$ is $\mathcal{M}$-effective. 
\end{proof}

Now let $\mathcal{N}$ be a family of morphisms in $\mathcal{C}$ verifying conditions (N1) and ($\text{N}_\mathcal{M}$) of \cref{site sheaf quotient by equivalence relation paragraph}. By \cref{site sheaf quotient by normal subgroup correspond} and \cref{site sheaf quotient by M-effective stable N-subobject correspond}, we obtain:
\begin{proposition}\label{site sheaf quotient by normal N-subgroup correspond}
Let $G$ be a $\mathcal{C}$-group and $H$ be a normal sub-$\mathcal{C}$-group of $G$ whose action on $G$ is $\mathcal{M}$-effective.
\begin{enumerate}
    \item[(a)] For any sub-$\mathcal{C}$-group $K$ of $G$ containing $H$ and such that the morphism $K\to G$ belongs to $\mathcal{N}$, $H$ acts $\mathcal{M}$-effectively on $K$ and the quotient $K/H\to K'$ is a sub-$\mathcal{C}$-group of $G'=G/H$ such that the morphism $K'\to G'$ belongs to $\mathcal{N}$.
    \item[(b)] The map $K\mapsto K'=K/H$ is a bijection from the set of sub-$\mathcal{C}$-groups $K$ of $G$ containing $H$ and such that the morphism $K\to G$ belongs to $\mathcal{N}$, $H$ acts $\mathcal{M}$-effectively on $K$ to the set of sub-$\mathcal{C}$-groups $K'$ of $G'$ such that the morphism $K'\to G'$ belongs to $\mathcal{N}$. Under this correspondence, the normal subgroups of $G$ correspond to that of $G'$.
\end{enumerate}
\end{proposition}
\begin{corollary}\label{site sheaf quotient by normal N-subgroup unit section in N}
If $H\to G$ belongs to $\mathcal{N}$, then $\mathcal{C}$ possesses a final object $e$ and the unit section $e\to G/H$ belongs to $\mathcal{N}$.
\end{corollary}
\begin{proof}
This follows form \cref{site sheaf quotient by normal N-subgroup correspond} by taking $K=H$.
\end{proof}

\subsection{Applications to the category of schemes}\label{site topology on Sch subsection}
Let $\Sch$ be the category of schemes, to which we can assocate the Zariski topology, that is, the topology generated by the family of morphisms $\{S_i\to S\}$, where each $S_i\to S$ is an open immersion and the union of images of $S_i$ is equal to $S$. A sheaf over the Zariski topology is also called of local nature: this is a contravariant functor $F:\Sch^{\op}\to\Set$ such that for any scheme $S$ and any covering $\{S_i\to S\}$, we have an exact diagram
\[\begin{tikzcd}
F(S)\ar[r]&\prod_iF(S_i)\ar[r,shift left=2pt]\ar[r,shift right=2pt]&\prod_{i,j}F(S_i\cap S_j)
\end{tikzcd}\]
In particular, a functor of local natura transforms direct sums to products. As any representable functor is a sheaf, this topology is coarser than the canonical topology.\par
To introduce (and handle) more topologies on $\Sch$, we need a general criterion to identify the covering families of the topology generated by certain family of morphisms. This is contained in the following proposition.
\begin{proposition}\label{site topology generated by refining family prop}
Let $\mathcal{C}$ be a category and $\mathcal{C}'$ be a full subcategory. Let $P$ be a set of families of morphisms of $\mathcal{C}$ with the same codomains, which is stable under composition and base change, and $P'$ be a set of families of morphisms of $\mathcal{C}'$ containing the families of identity morphisms. We endow $\mathcal{C}$ with the topology generated by $P$ and $P'$ and suppose that the following conditions are satisfied:
\begin{enumerate}
    \item[(a)] If $\{S_i\to S\}\in P'$ (hence $S_i,S\in\Ob(\mathcal{C}')$) and $T\to S$ is a morphism in $\mathcal{C}'$, then the fiber products $S_i\times_ST$ (in $\mathcal{C}$) exist and the family $\{S_i\times_ST\to T\}$ belongs to $P'$.
    \item[(b)] For any $S\in\Ob(\mathcal{C})$, there exists $\{S_i\to S\}\in P$ with $S_i\in\Ob(\mathcal{C}')$ for each $i$.
    \item[(c)] In the following situation
    \[\begin{tikzcd}
    S_{ijk}\ar[r,"(P')"]&S_{ij}\ar[r,"(P)"]&S_i\ar[r,"(P')"]&S
    \end{tikzcd}\]
    where $S,S_i,S_{ij},S_{ijk}\in\Ob(\mathcal{C}')$, $\{S_i\to S\}\in P'$, $\{S_{ij}\to S_i\}\in P$ for each $i$, $\{S_{ijk}\to S_{ij}\}\in P'$ for any $i,j$, there exists a family $\{T_n\to S\}\in P'$ and for each $n$ a multi-index $ijk$ and a commutative diagram
    \[\begin{tikzcd}[row sep=4mm, column sep=4mm]
    T_n\ar[rd]\ar[rr]&&S_{ijk}\\
    &S\ar[ru]&
    \end{tikzcd}\]
\end{enumerate}
Then for a sieve $R$ of $S\in\Ob(\mathcal{C})$ to be covering, it is necessary and sufficient that there exists a composite family
\begin{equation}\label{site topology generated by refining family prop-1}
\begin{tikzcd}[row sep=4mm, column sep=4mm]
S_{ij}\ar[d,swap,"(P')"]\ar[r,dashed]&R\ar[d]\\
S_i\ar[r,"(P)"]&S
\end{tikzcd}
\end{equation}
where $S_i,S_{ij}\in\Ob(\mathcal{C}')$, $\{S_i\to S\}\in P$, $\{S_{ij}\to S_i\}\in P'$ for each $i$, and the morphisms $S_{ij}\to S$ factors through $R$.
\end{proposition}
\begin{proof}
Since the families in $P$ and $P'$ are covering, any family which is the composite of such families is again covering, so a sieve of the form indicated in the proposition is covering for $\mathcal{C}$, since it contains a covering sieve. Conversely, it suffices to prove that sieves of the form (\ref{site topology generated by refining family prop-1}) form a topology, i,e, it suffices to verify the axioms (T1)--(T3).\par
To verify (T3), let $S\in\Ob(\mathcal{C})$. There exists by (b) a family $\{S_i\to S\}\in P$ with $S_i\in\Ob(\mathcal{C}')$. The families $\{\id_{S_i}:S_i\to S_i\}$ belong to $P'$ by hypothesis, so the sieve $S$ of $S$ is of the following form:
\[\begin{tikzcd}[row sep=4mm, column sep=4mm]
S_i\ar[r]\ar[d,swap,"(P')"]&S\ar[d,"\id_S"]\\
S_i\ar[r,"(P)"]&S
\end{tikzcd}\]

Now let $R$ be a sieve of $S$ with desired form (\ref{site topology generated by refining family prop-1}) and $R'$ be a sieve such that for any $T\to R$ in $\mathcal{C}$, the sieve $R'\times_TS$ is of the desired from. Then as the morphism $S_{ij}\to S$ factors through $R$, the sieve $R'_{ij}=R'\times_SS_{ij}$ of $S_{ij}$ is of the desicred form by hypothesis:
\[\begin{tikzcd}[row sep=4mm, column sep=4mm]
R'_{ij}\ar[dd]\ar[r,hook]&S_{ij}\ar[d,swap,"(P')"]\ar[rd]&\\
&S_i\ar[d,swap,"(P)"]&R\ar[ld,hook']\\
R'\ar[r,hook]&S&
\end{tikzcd}\]
so for each $ij$, we have a diagram of the form
\[\begin{tikzcd}[row sep=4mm, column sep=4mm]
S_{ijkl}\ar[rd]\ar[d,swap,"(P')"]&\\
S_{ijk}\ar[d,swap,"(P)"]&R'_{ij}\ar[ld,hook']\\
S_{ij}
\end{tikzcd}\]
We have thus proved that there exists a composite family
\[\begin{tikzcd}[row sep=4mm, column sep=4mm]
S_{ijkl}\ar[r,"(P')"]&S_{ijk}\ar[r,"(P)"]&S_{ij}\ar[r,"(P')"]&S_i\ar[r,"(P)"]&S
\end{tikzcd}\]
belonging to $P\circ P'\circ P\circ P'$, which factors through $R'$ and with all objects (except $S$) belong to $\mathcal{C}'$. Applying condition (c) to the family $\{S_{ijkl}\to S_i\}$, we then obtain for each $i$ a family $\{T_{in}\to S_i\}\in P'$, such that $T_{in}\to S$ factors through one of the $S_{ijkl}$, hence through $R'$:
\[\begin{tikzcd}[row sep=4mm, column sep=4mm]
T_{in}\ar[r]\ar[d,swap,"(P')"]&S_{ijkl}\ar[dd]\\
S_i\ar[d,swap,"(P)"]&\\
S&R'\ar[l,hook']
\end{tikzcd}\]
The sieve $R'$ of $S$ is therefore of the desired form (\ref{site topology generated by refining family prop-1}), which verifies axiom (T2).\par

Fianlly, as for axiom (T1), let $R$ be a sieve of $S$ of the desired form and $T\to S$ be a morphism in $\mathcal{C}$. Let $T_i=S_i\times_ST$; the family $\{T_i\to T\}$ then belongs to $P$, and applying condition (b), we obtain for each $i$ a family $\{U_{ik}\to T_i\}\in P$, with $U_{ik}\in\Ob(\mathcal{C}')$. By the hypotesis on $P$, we have $\{U_{ik}\to T\}\in P$, so by condition (a), $U_{ik}\times_{S_i}S_{ij}=U_{ikj}$ is an object of $\mathcal{C}'$ and for each $ik$, $\{U_{ikj}\to U_{ik}\}\in P'$.
\[\begin{tikzcd}
U_{ikj}\ar[d,swap,"(P')"]\ar[rr]&&S_{ij}\ar[rdd,bend left=20pt]\ar[d,"(P')"]&\\
U_{ik}\ar[r,"(P)"]\ar[rd,swap,"(P)"]&T_i\ar[r]\ar[d,"(P)"]&S_i\ar[d,"(P)"]&\\
&T\ar[r]&S&R\ar[l,hook']
\end{tikzcd}\]
We therefore conclude that the family $\{U_{ikj}\to T\}$ factors through the sieve $T\times_SR$ of $T$, which is hence of the desired form. This proves axiom (T1) and completes the proof.
\end{proof}

\begin{corollary}\label{site topology generated by refining family covering sieve iff}
If $S\in\Ob(\mathcal{C}')$ and $R$ is a sieve of $S$, then $R$ is covering if and only if there exists a family $\{T_i\to S\}\in P'$ which factors through $R$.
\end{corollary}
\begin{proof}
In fact, any such sieve is covering. Conversely, it suffices to apply (c) to the family $\{S_i\to S\}$ and the identity morphisms of $S_i$ to deduce that any covering sieve is of the indicated form. 
\end{proof}

\begin{corollary}\label{site topology generated by refining family sheaf iff}
For a presheaf $F\in\PSh(\mathcal{C})$ to be separated (resp. a sheaf), it is necessary and sufficient that the morphisms
\[\begin{tikzcd}
F(S)\ar[r]&\prod_iF(S_i)
\end{tikzcd}\]
is injective (resp. that the diagram
\[\begin{tikzcd}
F(S)\ar[r]&\prod_iF(S_i)\ar[r,shift left=2pt]\ar[r,shift right=2pt]&\prod_{i,j}F(S_i\times_SS_j)
\end{tikzcd}\]
is exact) for $\{S_i\to S\}\in P$ and $\{S_i\to S\}\in P'$, respectively.
\end{corollary}
\begin{proof}
In fact, these conditions are necessary, because the families above are covering. Conversely, if $R$ is the sieve of $S$ of a family of morphisms $\{S_{ij}\stackrel{(P')}{\to} S_i\stackrel{(P)}{\to} S\}$, a diagram chasing shows that the above conditions imply that $\Hom(S,F)\to\Hom(R,F)$ is injective (resp. bijective). But any covering sieve $R'$ of $S$ contains a sieve generated by such a family and we have a commutative diagram
\[\begin{tikzcd}[row sep=6mm,column sep=4mm]
\Hom(S,F)\ar[rr,"f"]\ar[rd,swap,"g"]&&\Hom(R,F)\ar[ld,"h"]\\
&\Hom(R',F)&
\end{tikzcd}\]
If $g$ is injective, then so is $f$, so in this case $F$ is separated. In this case, since the morphism $R'\to R$ is covering, we see that $h$ is also injective (cf. \cref{site morphism of presheaf covering def}). Therefore, if $g$ is bijective, so is $f$, hence $F$ is a sheaf.
\end{proof}

\begin{remark}
The condition (c) of \cref{site topology generated by refining family prop} is satisfies if $P'$ is stable under composition and if any family $\{S_i\to S\}$ of morphisms in $P$ with $S_i,S\in\Ob(\mathcal{C}')$ has a subfamily belonging to $P'$.
\end{remark}

We now let $\mathcal{C}=\Sch$ be the category of schemes, and $\mathcal{C}'$ be the full subcategory formed by affine schemes. We shall consider the following sets $P'$:
\begin{enumerate}[leftmargin=35pt]
    \item[$P_1'$:] finite and surjective families, formed by flat morphisms;
    \item[$P_2'$:] finite and surjective families, formed by flat morphisms of finite presentation;
    \item[$P_3'$:] finite and surjective families, formed by \'etale morphisms;
    \item[$P_4'$:] finite and surjective families, formed by finite \'etale morphisms;
\end{enumerate}
For each of these sets $P_i'$ (except $P_4'$), the conditions of \cref{site topology generated by refining family prop} are satisfied (as for (c), note that an affine scheme is quasi-compact, so any family of morphisms of $\mathcal{C}'$, belonging to $P$, contains a finite subfamily which is equally in $P$, hence in $P_i'$ for $i=1,2,3$). The corresponding topology $\mathcal{T}_i$ generated by $P$ and $P'_i$ is denoted and called by the following manner:
\begin{enumerate}[leftmargin=35pt]
    \item[$\mathcal{T}_1$] is the faifufully flat and quasi-compact topology (\fpqc);
    \item[$\mathcal{T}_2$] is the faifufully flat and finite presented topology (\fppf);
    \item[$\mathcal{T}_3$] is the \'etale topology (\et);
    \item[$\mathcal{T}_4$] is the finite \'etale topology (\etf).
\end{enumerate}
As $P_1'\sups P_2'\sups P_3'\sups P_4'$, we have
\[\fpqc\geq\fppf\geq\et\geq \etf\geq \Zar.\]

\begin{proposition}\label{scheme topology on Sch by refining family prop}
Let $\mathcal{T}_i$ ($i=1,2,3,4$) be the topologies on $\Sch$ defined above.
\begin{enumerate}
    \item[(a)] For a sieve $R$ of $S$ to be covering for $\mathcal{T}_i$ ($1\leq i\leq 3$), it is necessary and sufficient that there exists a covering $(S_\alpha)$ of $S$ by affine opens and for each $\alpha$ a family $\{S_{\alpha\beta}\to S_\alpha\}\in P_i'$, with $S_{\alpha\beta}$ affine, such that the the family $\{S_{\alpha\beta}\to S\}$ factors through $R$.
    \item[(b)] For a presheaf $F$ over $\Sch$ to be a sheaf for the fpqc topology (resp. fppf, \'etale, finite \'etale), it is necessary and sufficient that
    \begin{enumerate}
        \item[(\rmnum{1})] $F$ is a sheaf over the Zariski topology, i.e. a functor of local nature.
        \item[(\rmnum{2})] For any faithfully flat morphism (resp. faithfully flat morphism of finite presentation, resp. surjective \'etale, resp. finite surjective \'etale) $T\to S$ of affine schemes, we have an exact diagram
        \[\begin{tikzcd}
        F(S)\ar[r]&F(T)\ar[r,shift left=2pt]\ar[r,shift right=2pt]&F(T\times_ST)
        \end{tikzcd}\]
    \end{enumerate}   
    \item[(c)] The topologies $\mathcal{T}_i$ ($1\leq i\leq 4$) are subcanonical.
    \item[(d)] Any surjective family formed by open and flat morphisms (resp. flat and locally of finite presentation, resp. \'etale, resp. finite and \'etale) is covering for the fpqc topology (resp. fppf, resp. \'etale, resp. finite \'etale).
    \item[(e)] Any finite and surjective family, formed by flat and quasi-compact morphisms, is covering for the fpqc topology. 
\end{enumerate}
\end{proposition}
\begin{proof}
Assertion (a) follows from \cref{site topology generated by refining family prop}, and (b) follows from \cref{site topology generated by refining family sheaf iff}, since a sheaf for the Zariski topology transforms direct sums into products. Any representable functor is a sheaf for Zariski topology, and satisfies condition (\rmnum{2}) by (\cite{SGA1}, \Rmnum{8}, 5.3), so $\mathcal{T}_1$ is subcanonical, which proves (c).\par
Let $\{S_i\to S\}$ be a family of morphisms as in (d). By considering a covering of $S$ by affine opens, we are reduced to the case where $S$ is affine. We first deal with the case where each $S_i\to S$ is flat and open (resp. \'etale). Let $S_{ij}$ be a covering og $S_i$ be affine opens. As the morphisms considered are open, the images $T_{ij}$ of $S_{ij}$ in $S$ form an open covering of $S$. As $S$ is affine, hence quasi-compact, there exists a finite subcover of $T_{ij}$, with $i,j$ belongs to a finite set $F$. Then $S'=\coprod_FS_{ij}$ is affine, and the morphism $S'\to S$ belongs to $P_1'$ (resp. $P_3'$), hence is covering. As this factors through the given family $\{S_i\to S\}$, the latter is also covering.\par
In the case of the finite \'etale topology, each $S_i$ is finite over $S$, hence is affine; in the preceding argument, we can then take for $\{S_{ij}\}$ the covering $\{S_i\}$ of $S_i$, and we obtain a morphism $S'\to S$ belonging to $P_4'$.\par
Now consider the case where $f_i:S_i\to S$ are flat and locally of finite presentation. For any $s\in S$, there exists (by the proof of (\cite{EGA4-4}, 17.16.2)) an affine subscheme $X(s)$ of one of the $S_i$ such that $s\in f_i(X(s))$ and that the morphism $g_i:X(s)\to S$, restriction of $f_i$, is flat and quasi-finite. Then $g_i(X(s))$ is an open neighborhood $U(s)$ of $s$ ((\cite{EGA4-2}, 2.4.6)), and as $S$ is affine, it is covered by a finite number of such opens $U(s_j)$, $j=1,\dots,n$. Therefore, $X'=\coprod_jX(s_j)$ is affine, and the morphism $X'\to S$ is surjective, flat, of finite presentation and quasi-finite, hence belongs to $P_2'$, which completes the proof of (d).\par
Finally, let $\{S_i\to S\}$ be a finite and surjective family of flat and quasi-compact morphisms. Let $T_j$ be a covering of $S$ be affine opens. Then $S_{ij}=T_j\times_SS_i$ is quasi-compact and hence has a finite affine covering $T_{ijk}$. Each morphism $T_{ijk}\to T_j$ is flat, and the family $\{T_{ijk}\to T_j\}$ is finite and surjective, hence covering for $\mathcal{T}_1$. The family $\{T_{ijk}\to S\}$ is hence also, by composition, covering, and it factors through the given family:
\[\begin{tikzcd}
T_{ijk}\ar[rd]\ar[r]&S_{ij}\ar[d]\ar[r]&S_i\ar[d]\\
&T_j\ar[r]&S
\end{tikzcd}\]
so the given family $\{S_i\to S\}$ is also covering for $\mathcal{T}_1$.
\end{proof}

\begin{proposition}\label{scheme topology T_i family M_i}
Let $\mathcal{M}_i$ be the family of the following morphisms:
\begin{enumerate}[leftmargin=35pt]
    \item[$\mathcal{M}_1$:] faithfully flat and quasi-compact morphisms.
    \item[$\mathcal{M}_2$:]  faithfully flat and locally of finite presentaion morphisms.
    \item[$\mathcal{M}_3$:]  surjective \'etale morphisms.
    \item[$\mathcal{M}_4$:]  finite surjective \'etale morphisms. 
\end{enumerate}
Then the family $\mathcal{M}_i$ verifies conditions (M1)--($\text{M}_{\mathcal{T}_i}$) of \ref{site sheaf quotient by equivalence relation paragraph}.
\end{proposition}
\begin{proof}
For (M1)--(M3), these are classical results. By \cref{scheme topology on Sch by refining family prop}~(d) and (e), we see that $\mathcal{M}_i$ satisfies ($\text{M4}_{\mathcal{T}_i}$), so it remains to verify ($\text{M5}_{\mathcal{T}_i}$). For this, it suffices to show that each $\mathcal{M}_i$ satisfies condition ($\text{M5}_{\mathcal{T}_1}$), since it implies the others. This follows from (\cite{SGA1}, \Rmnum{8}, n4 and n5).
\end{proof}

\begin{corollary}\label{scheme equivalence relation M_i-effective iff quotient represent}
If $X$ is a scheme and $R$ is an equivalence relation of type $\mathcal{M}_i$, then $R$ is $\mathcal{M}_i$-effective if and only if the quotient sheaf $X$ by $R$ for $\mathcal{T}_i$ is representable and in this case it is represented by $X/R$.
\end{corollary}
\begin{proof}
In fact, this is a concequence of \cref{site sheaf M-effective relation iff quotient representable}.
\end{proof}

We also consider families $\mathcal{N}$ of morphisms verifying conditions (N1) and ($\text{N}_{\mathcal{T}_i}$). But note, as above, that condition ($\text{N}_{\mathcal{T}_1}$) implies the others.

\begin{proposition}\label{scheme morphism descent by fpqc eg}
The following families satisfy conditions (N1) and ($\text{N}_{\mathcal{T}_1}$) of \ref{site sheaf quotient by equivalence relation paragraph}, that is, have descent property for the fpqc topology:
\begin{enumerate}[leftmargin=35pt]
    \item[$\mathcal{N}$:] open immersions.
    \item[$\mathcal{N}'$:] closed immersions.
    \item[$\mathcal{N}''$:] quasi-compact immersions.
\end{enumerate}
\end{proposition}
\begin{proof}
In view of \cref{scheme topology on Sch by refining family prop}~(b), it suffices to consider the descent data relative to the Zariski topology and to a faithfully flat and quasi-compact morphism. The first assertion is clear; for the second one, the case for $\mathcal{N}$ and $\mathcal{N}'$ follows from (\cite{SGA1}, \Rmnum{8}, 4.4) and (\cite{SGA1}, \Rmnum{8}, 1.9). The case for $\mathcal{N}''$ can be deduce as in (\cite{SGA1}, \Rmnum{8}, 5.5), by using the previous two results.
\end{proof}

We can therefore apply the results of the previous subsections to the families of morphisms given above. Let us give one as an example (\cref{site sheaf quotient by M-effective quotient for subcanonical topology} and \cref{site sheaf quotient by type MN char}):

\begin{corollary}\label{scheme quotient by fpqc equivalence relation prop}
Let $X$ be a scheme and $R$ be an equivalence relation on $X$. Suppose that $R\to X$ is faithfully flat and quasi-compact and $R\to X\times X$ is a closed immersion (resp. open immersion, resp. quasi-compact immersion). Then the quotient sheaf $X/R$ is the same for the fpqc topology and the canonical topology, and for any scheme $S$, we have
\[(X/R)(S)=\left\{\parbox{4in}{%
closed (resp. open, resp. quasi-compact) subschemes $Z$ of $X\times S$ stable under $R\times S$ such that $Z\to X_S$ belongs to $\mathcal{N}$, that $R_Z$ is faithfully flat and quasi-compact, and that diagram $R_Z\rightrightarrows Z\to S$ is exact%
}\right\}.\]
\end{corollary}

\paragraph{Homogeneous spaces}
Let $G$ be an $S$-group scheme, $X$ an $S$-scheme acted (on right) by $G$, and
\[\Phi:G\times_SX\to X\times_SX\]
be the $S$-morphism defined setwise by $(g,x)\mapsto(gx,x)$. Recall that $X$ is called formally principal homogeneous under $G$ if the following equivalent conditions are satisfied\footnote{The equivalence of (\rmnum{1}) and (\rmnum{2}) is clear, and (\rmnum{2})$\Leftrightarrow$(\rmnum{3}) since $\mathcal{C}$ is a full subcategory of $\widehat{\mathcal{C}}$.}:
\begin{enumerate}
    \item[(\rmnum{1})] For any $T\to S$, the set $X(T)$ is either empty of pincipal homogeneous under $G(T)$,
    \item[(\rmnum{2})] $\Phi$ is an isomorphism of $S$-functors,
    \item[(\rmnum{3})] $\Phi$ is an isomorphism of $S$-schemes.  
\end{enumerate}
The definition of \textbf{formally homogeneous spaces} (not necessarily principal) is obtained by demanding that $\Phi$ is an epimorphism in the category of sheaves for an appropriate topology $\mathcal{T}$. On the other hand, the condition that $\Phi$ is an epimorphism of $S$-functors is equivalent to that, for any $T\to S$, the set $X(T)$ is empty or homogeneous (not necessarily principal) under $G(T)$. But this condition is too restrictive, as shown in the following example.

\begin{example}
Let $S=\Spec(\R)$, $G=\G_{m,\R}$ and $X=\G_{m,\R}$ over which $G$ acts by $t\cdot x=t^2x$. Then the morphism $\Phi:G\times_SX\to X\times_SX$ is \'etale, finite and surjective, hence an epimorphism in the category of sheaves for the finite \'etale topology (a fortiori, an epimorphism of $S$-schemes). However, the points $1$ and $-1$ of $X(\R)$ are not conjugate under $G(\R)$, so that the morphism $G(\R)\times X(\R)\to X(\R)\times X(\R)$ is not surjective\footnote{Obviously, this difficulty comes from the fact that if $\mathcal{C}'$ is a full subcategory of $\widehat{\mathcal{C}}$ containing $\mathcal{C}$, for example, the category of sheaves on $\mathcal{C}$ for a subcanonical topology $\mathcal{T}$, and if $f:X\to Y$ is a morphism in $\mathcal{C}$, then the implications
\[\text{$f$ is an epimorphism in $\widehat{\mathcal{C}}$}\Rightarrow\text{$f$ is an epimorphism in $\mathcal{C}'$}\Rightarrow\text{$f$ is an epimorphism in $\mathcal{C}$}\]
are in general strict.
}.
\end{example}

\begin{definition}
Let $G$ be an $S$-group, $X$ be an $S$-group scheme acted by $G$ and $\mathcal{T}$ be a subcanonical topology over $\Sch_{/S}$. We say that $X$ is a \textbf{formally homogeneous space under $G$} (relative to the topology $\mathcal{T}$) if the following equivalent conditions are satisfied:
\begin{enumerate}
    \item[(\rmnum{1})] the morphism $\Phi:G\times_SX\to X\times_SX$ is an epimorphism in the category of sheaves for the topology $\mathcal{T}$.
    \item[(\rmnum{2})] for any $T\to S$ and $x,y\in X(T)$, there exists a covering morphism $T'\to T$ for the topology $\mathcal{T}$ and $g\in G(T')$ such that $y_{T'}=g\cdot x_{T'}$.
\end{enumerate}
\end{definition}

\begin{remark}
Condition (\rmnum{1}) implies that $\Phi$ is a universally effective epimorphism in $\Sch_{/S}$ (cf. \cite{SGA3-1}, \Rmnum{4}, 4.4.3). This implies, in particular, that $\Phi$ is a surjective morphism of schemes.
\end{remark}

\begin{proposition}\label{scheme formally homogeneous iff}
Let $G$ be an $S$-group, $X$ be an $S$-scheme acted by $G$, and $\mathcal{T}$ be a subcanonical topology over $\Sch_{/S}$. The following conditions are equivalent:
\begin{enumerate}
    \item[(\rmnum{1})] $X$ verifies the following conditions:
    \begin{enumerate}
        \item[(a)] the morphism $\Phi:G\times_SX\to X\times_SX$ is covering, i.e. $X$ is a formally homogeneous $G$-space.
        \item[(b)] the morphism $X\to S$ is covering, i.e. locally for the topology $\mathcal{T}$, it possesses a section\footnote{Note that the morphism $X\to S$ is an epimorphism in $\widehat{\Sch_{/S}}$ if and only if for any $T\to S$, the morphism $X(T)\to S(T)=\{T\to S\}$ is surjective, that is, the morphism $T\to S$ factors through $X$. But since $T\to S\in S(T)$ is the image of the identity morphism $\id_S:S\to S$, this is true if and only if $\id_S$ factors through $X$, which means $X$ admits a section.}.
    \end{enumerate}
    \item[(\rmnum{2})] $X$ is locally isomorphic (as a $G$-object) to the quotient sheaf (for $\mathcal{T}$) of $G$ by a subgroup scheme $H$, that is, there exists a covering family $\{S_i\to S\}$ such that each $X\times_SS_i$ represents the quotient sheaf of $G\times_SS_i$ by a subgroup scheme $H_i$.
\end{enumerate}
Under these conditions, we say that $X$ is a \textbf{homogeneous $G$-space} (relative to the topology $\mathcal{T}$).
\end{proposition}
\begin{proof}
Suppose that (\rmnum{2}) is satisfied. Put $G_i=G\times_SS_i$ and $X_i=X\times_SS_i$, then $X_i$ possesses a section over $S_i$, namely the composition of the unit section $\eps_i:S_i\to G_i$ and the projection $\pi_i:G_i\to X_i=G_i/H_i$. We then conclude that $X\to S$ is covering.\par
On the other hand, $\pi_i$ is covering, so $\pi_i\times\pi_i$ is also covering, and we have a commutative diagram
\[\begin{tikzcd}
G_i\times_{S_i}G_i\ar[r,"\Psi_i"]\ar[d,"\id\times\pi_i"]&G_i\times_{S_i}X_i\ar[d,"\pi_i\times\pi_i"]\\
G_i\times_{S_i}X_i\ar[r,"\Phi_i","\sim"']&G_i\times_{S_i}X_i
\end{tikzcd}\]
where $\Phi_i$ is induced by $\Phi$ by base change $S_i\to S$ and $\Psi_i$ is the isomorphism defined by $(g,g')\mapsto(gg',g)$. Then $(\pi_i\times\pi_i)\circ\Psi_i$ is covering, hence $\Phi_i$ is a covering. This shows that $\Phi$ is locally covering, hence is covering, whence (\rmnum{2})$\Rightarrow$(\rmnum{1}).\par
Conversely, suppose that (\rmnum{1}) is satisfied, and moreover the structural morphism $X\to S$ possesses a section $\sigma\in X(S)$. By \cref{scheme morphism graph is immersion}, $\sigma$ is an immersion. Define $H=G\times_XS$ by the diagram below, where the two squares are Cartesian:
\[\begin{tikzcd}
H\ar[r]&G\ar[r,"\id_G\boxtimes\sigma"]\ar[d,"\pi"]&G\times_SX\ar[d,"\Phi"]\\
S\ar[r,"\sigma"]&X\ar[r,"\id_X\boxtimes\sigma"]&X\times_SX
\end{tikzcd}\]
where $\pi$, $\id_G\boxtimes\sigma$ and $\id_X\boxtimes\sigma$ denote the morphisms defined setwisely, for $T\to S$ and $g\in G(T)$, $x\in X(T)$, by
\[\pi(g)=g\cdot\sigma_T,\quad (\id_G\boxtimes\sigma)(g)=(g,\sigma_T),\quad (\id_X\boxtimes\sigma)(x)=(x,\sigma_T).\]
Then $\pi$ is covering, and $H$ is a subgroup scheme of $G$, represented by the stabilizer $\Stab_G(\sigma)$ of $\sigma$, that is, for any $T\to S$, we have
\[H(T)=\{g\in G(T):g\cdot\sigma_T=\sigma_T\}.\]
Denote by $G/H$ the presheaf $T\mapsto G(T)/H(T)$, and $(G/H)^\#$ the associated sheaf for the topology $\mathcal{T}$. From the above arguments, we obtain a commutative diagram of morphisms of presheaves acted by $G$:
\[\begin{tikzcd}
G\ar[r,"\pi"]\ar[d]&X\\
G/H\ar[ru,swap,"\bar{\pi}"]
\end{tikzcd}\]
where $\bar{\pi}$ is a monomorphism (cf. \cref{category universal effective equivalence factor monomorphism iff}). As $\pi$ is covering, $\bar{\pi}$ is also a covering, so $\bar{\pi}$ induces an isomorphism $(G/H)^\#\cong X$. Therefore, we have shown that: if $X$ is a homogeneous $G$-space such that $X\to S$ admits a section $\sigma$, then $X$ represents the quotient sheaf $G/H$, where $H=G\times_XS$ is the stabilizer of $\sigma$.\par
In the general case, by hypothesis there exists a covering family $\{S_i\to S\}$ such that each morphism $X_i=X\times_SS_i\to S_i$ possesses a section $\sigma_i$. Put $G_i=G\times_SS_i$, then the morphism $\Phi_i=G_i\times_{S_i}X_i\to X_i\times_{S_i}X_i$ deduced from $\Phi$ by base change $S_i\to S$ is covering, hence, by the preceding arguments, $X_i\cong G_i/H_i$ where $H_i=\Stab_{G_i}(\sigma_i)$. This proves the implication (\rmnum{1})$\Rightarrow$(\rmnum{2}).
\end{proof}

\section{Construction of quotient schemes}
\subsection{\texorpdfstring{$\mathcal{C}$}{C}-groupoids}
Let $\mathcal{C}$ be a category which has finite products and coproducts. Recall that a diagram
\[\begin{tikzcd}
X_1\ar[r,shift left=2pt,"d_1"]\ar[r,shift right=2pt,swap,"d_0"]&X_0\ar[r,"p"]&Y
\end{tikzcd}\]
in $\mathcal{C}$ is called \textbf{exact} if $pd_0=pd_1$ and if, for any $T\in\mathcal{C}$, $T(p)$ is a bijection from $T(Y)$ to the subset of $T(X_0)$ formed by morphisms $f:X_0\to T$ such that $fd_0=fd_1$. We also say that $(Y,p)$ is the cokernel of $(d_0,d_1)$, and write
\[(Y,p)=\coker(d_0,d_1).\]
Let $\mathcal{C}$ be the category $\Rsp$ of ringed spaces. In this case, there always exists a cokernel $(Y,p)$, of which we can give the following description: the underlying topological space $Y$ is obtained from $X_0$ by identifying the points $d_0(x)$ and $d_1(x)$, endowed with the quotient topology. The canonical morphisms $\pi:X_0\to Y$ and $d_0$, $d_1$ then induce a double arrow of sheaves of rings over $Y$:
\[\begin{tikzcd}
\pi_*(\mathscr{O}_{0})\ar[r,shift left=2pt,"\delta_1"]\ar[r,shift right=2pt,swap,"\delta_0"]&\pi_*((d_0)_*(\mathscr{O}_{1}))=\pi_*((d_1)_*(\mathscr{O}_{1}))
\end{tikzcd}\]
where $\mathscr{O}_i$ is the structural sheaf of $X_i$. We choose $\mathscr{O}_Y$ to be the sheaf of rings over $Y$ whose sections $s$ are such that $\delta_0(s)=\delta_1(s)$. The morphism $p:X_0\to Y$ is defined in the evident way.\par
Let $d_0,d_1:X_1\rightrightarrows X_0$ be a diagram in $\Rsp$ and $(Y,p)$ be the cokernel. We say that an open subset $U$ of $X_0$ is saturated if $d_0^{-1}(U)=d_1^{-1}(U)$, which is equivalent to that $U=p^{-1}(p(U))$. In this case, as $Y$ is endowed with the quotient topology, $p(U)$ is an open subset of $Y$.

\begin{lemma}\label{ringed space groupoid restriction to saturated prop}
Let $U$ be a saturated open subset of $X$ and $V=p(U)$. If we denote by $U_1=d_0^{-1}(U)=d_1^{-1}(U)$ the open subset of $X_1$, and $\tilde{d}_0$, $\tilde{d}_1$ and $\tilde{p}$ the restriction of $d_0$, $d_1$ to $U_1$ and $p$ to $U$, then $(V,\tilde{p})$ is the cokernel of the following diagram in $\Rsp$:
\[\begin{tikzcd}
U_1\ar[r,shift left=2pt,"\tilde{d}_1"]\ar[r,shift right=2pt,swap,"\tilde{d}_0"]&U\ar[r,"\tilde{p}"]&V
\end{tikzcd}\]
\end{lemma}
\begin{proof}
Since $U$ is saturated, the morphisms $d_0,d_1$ and $p$ restricts to give the desired diagram. The claim that $(V,\tilde{p})$ is the cokernel is an immediate verification.
\end{proof}

\begin{remark}
The result of \cref{ringed space groupoid restriction to saturated prop} is not true in the category of schemes. For example, let $S=\Spec(\C)$, $X_0=\A_S^2=\Spec(\C[x_1,x_2])$, $d_1:\G_{m,S}\times_S\A_S^2\to\A_S^2$ be the action of $\G_{m,S}$ on $\A_S^2$ by multiplication, and $d_0:\G_{m,S}\times_S\A_S^2\to\A_S^2$ the projection to the second factor. Let $U=\A_S^2-\{\m\}$, where $\m$ is the point $(0,0)$. Then the projective space $\P_S^1$ is the cokernel of $(\tilde{d}_0,\tilde{d}_1)$ in $\Rsp$ and $\Sch_{/S}$, and the cokernel $Y$ of $(d_0,d_1)$ in $\Rsp$ is the union of $\P^1_S$ and the point $y_0=\{p(\m)\}$, with the unique open subset containing $y_0$ being $Y$ and we have $\Gamma(Y,\mathscr{O}_Y)=\C$ (note that $Y$ is not a scheme). If $f:\A_S^2\to T$ is a morphism of $S$-schemes such that $fd_0=fd_1$ and $\bar{f}:Y\to T$ is the induced morphism of ringed spaces, then for any affine open subset $V=\Spec(A)$ of $T$ containing the point $t_0=f(\m)$, we have $f^{-1}(V)=Y$, so the ring homomorphism $A\to\C[x_1,x_2]$ factors through $\C$. This shows that $S=\Spec(\C)$ is the cokernel of $(d_0,d_1)$ in the category $\Sch_{/S}$.
\end{remark}

\begin{lemma}\label{ringed space groupoid cokernel in scheme if}
Let $d_0,d_1:X_1\rightrightarrows X_0$ be a diagram in $\Sch$ and $(Y,p)$ be the cokernel in $\Rsp$.
\begin{enumerate}
    \item[(a)] If $Y$ is a scheme and $p$ is a morphism of schemes, then $(Y,p)$ is a cokernel of $(d_0,d_1)$ in $\Sch$.
    \item[(b)] Suppose that any point of $X_0$ has a saturated open neighborhood $U$ such that, if $(\tilde{d}_0,\tilde{d}_1)$ is the induced diagram to $d_0^{-1}(U)=d_1^{-1}(U)$ and $(Q,q)$ is the cokernel of $(\tilde{d}_0,\tilde{d}_1)$ in $\Rsp$, then $Q$ is a scheme and $q$ is a morphism of schemes. Then $(Y,p)$ is a cokernel of $(d_0,d_1)$ in $\Sch$.
\end{enumerate}
\end{lemma}
\begin{proof}
In the situation of (a), since $(Y,p)$ is a cokernel of $(d_0,d_1)$ in $\Rsp$, every morphism $f:X_0\to T$ of schemes such that $fd_0=fd_1$ factors into a morphism $\bar{f}:Y\to T$ of ringed spaces. Now we know that $f=\bar{f}p$, and $f$, $p$ are both morphisms of locally ringed spaces with $p$ being surjective; it follows that $\bar{f}$ is also a morphism of locally ringed spaces, hence $(Y,p)$ is a cokernel on $(d_0,d_1)$ in $\Sch$. Now (b) follows from (a) and \cref{ringed space groupoid restriction to saturated prop} by glueing.
\end{proof}

In this section, we consider the existence of $\coker(d_0,d_1)$ if the two morphisms arise from a groupoid. More precisely, denote by $X_2=X_1\times_{d_1,d_0}X_1$ the fiber product and $d_0',d_2'$ the two projections of $X_2$ to $X_1$, so that we have a Cartesian square:
\begin{equation}\label{category groupoid square-1}
\begin{tikzcd}
X_2\ar[r,"d_0'"]\ar[d,swap,"d_2'"]&X_1\ar[d,"d_1"]\\
X_1\ar[r,"d_0"]&X_0
\end{tikzcd}
\end{equation}
Moreover, suppose that we are given a third morphism $d'_1:X_2\to X_1$. We say that $(d_0,d_1:X_1\rightrightarrows X_0,d_1')$ is a \textbf{$\mathcal{C}$-groupoid} if for any object $T$ of $\mathcal{C}$, $X_1(T)$ is the set of morphisms of a groupoid $X_*(T)$ whose set of objects is $X_0(T)$, with source map $d_1(T)$, target map $d_0(T)$, and whose composition map is $d'_1(T)$ (we identify $(X_1\times_{d_1,d_0}X_1)(T)$ with $X_1(T)\times_{d_1(T),d_0(T)}X_1(T)$)\footnote{Therefore, in this case, $X_2(T)$ is the set of pairs of composable morphisms $(f_2,f_1)$, that is, such that $d_0(f_1)=d_1(f_2)$, and $d_0'$, $d_1'$ and $d_2'$ send $(f_2,f_1)$ to $f_2$, $f_2\circ f_1$, $f_1$, respectively.}.\par

If $\varphi$ is a morphism of the groupid $X_*(T)$, the map $f\mapsto\varphi f$ is a bijection from the set of morphisms $f$ whose target coincides with the source of $\varphi$ to the set of morphisms with the same target as $\varphi$. We then conclude that there is a Cartesian square
\begin{equation}\label{category groupoid square-2}
\begin{tikzcd}
X_2\ar[r,"d_1'"]\ar[d,swap,"d_0'"]&X_1\ar[d,"d_0"]\\
X_1\ar[r,"d_0"]&X_0
\end{tikzcd}
\end{equation}
Moreover, this square is also cocartesian: if $\varphi:X_1\to Y$ is a morphism in $\mathcal{C}$ such that $\varphi d_1'=\varphi d_0'$, then for any $T\in\Ob(\mathcal{C})$, the value of the morphism $\varphi(T):X_1(T)\to Y(T)$ on $f\in X_1(T)$ only depends on the target of $f$ (if $g$ is another morphism with the same target as $f$, then $f^{-1}g$ is in the image of $d_1'$ and we have $\varphi(f)=\varphi d_1'(g,g^{-1}f)=\varphi d_0'(g,g^{-1}f)=\varphi(g)$), so it factors through $d_0(T)$.\par

Similarly, the map $g\mapsto g\circ\varphi$ is a bijection from the set of morphisms $g$ whose source coincides with the target of $\varphi$ to the set of morphisms with the same source as $\varphi$. We then conclude that there is a Cartesian square
\begin{equation}\label{category groupoid square-3}
\begin{tikzcd}
X_2\ar[r,"d_1'"]\ar[d,swap,"d_2'"]&X_1\ar[d,"d_1"]\\
X_1\ar[r,"d_1"]&X_0
\end{tikzcd}
\end{equation}
which is also cocartesian.\par
On the other hand, let $s:X_0\to X_1$ be the unique morphis in $\mathcal{C}$ such that, for any $T\in\Ob(\mathcal{C})$, $s(T):X_0(T)\to X_1(T)$ associates to any object of $X_*(T)$ the identity morphism of this object. The morphism $s$ satisfies the following equalities:
\begin{align}
d_1s=\id_{X_0},\label{category groupoid morphism s equality-1}\\
d_0s=\id_{X_0}.\label{category groupoid morphism s equality-2}
\end{align}
Finally, the associativity of the composition maps is expressed by the commutativity of the following diagram
\begin{equation}\label{category groupoid square-4}
\begin{tikzcd}
X_1\times_{d_1,d_0}X_1\times_{d_1,d_0}X_1\ar[r,"d_1'\times\id_{X_1}"]\ar[d,swap,"\id_{X_1}\times d'_1"]&X_1\times_{d_1,d_0}X_1\ar[d,"d_1"]\\
X_1\times_{d_1,d_0}X_1\ar[r,"d_1'"]&X_0
\end{tikzcd}
\end{equation}

Conversely, the conditions (\ref{category groupoid square-2}), (\ref{category groupoid square-3}) and (\ref{category groupoid square-4}) and the existence of a morphism $s$ satisfying (\ref{category groupoid morphism s equality-1}) imply that $(d_0,d_1:X_1\rightrightarrows X_0,d'_1)$ is a groupoid. In the rest of this section, we mainly use the squares (\ref{category groupoid square-1}), (\ref{category groupoid square-2}) and (\ref{category groupoid square-3}), which are summarized into the following diagram:
\begin{equation}\label{category groupoid commutative diagram}
\begin{tikzcd}
X_2\ar[r,shift left=2pt,"d_0'"]\ar[r,shift right=2pt,swap,"d_1'"]\ar[d,shift left=2pt,"d_1'"]\ar[d,shift right=2pt,swap,"d_2'"]&X_1\ar[r,"d_0"]\ar[d,"d_1"]&X_0\\
X_1\ar[d,swap,"d_1"]\ar[r,"d_0"]&X_0\\
X_0
\end{tikzcd}
\end{equation}
where the square is Cartesian and the first row and first column are exact.\par
We only use associativity in an indirect way, for example to ensure the existence of a morphism $s$ satisfying (\ref{category groupoid morphism s equality-1}) and (\ref{category groupoid morphism s equality-2}), or to ensure the existence of a morphism $\sigma:X_1\to X_1$ such that
\begin{equation}\label{category groupoid morphism sigma equality}
d_0\sigma=d_1,\quad d_1\sigma=d_0.
\end{equation}
(We choose $\sigma$ so that $\sigma(T):X_1(T)\to X_1(T)$ sends a morphism of $X_*(T)$ to its inverse.)\par
By abusing of languages, a $\mathcal{C}$-groupoid is also defined to be a diagram
\[\begin{tikzcd}[column sep=12mm]
X_2\ar[r,shift left=8pt,"d_0'"]\ar[r,shift right=8pt,swap,"d_2'"]\ar[r,"d_1'"description]&X_1\ar[r,shift left=2pt,"d_0"]\ar[r,shift right=2pt,swap,"d_1"]&X_0
\end{tikzcd}\]
such that (\ref{category groupoid square-1}), (\ref{category groupoid square-2}) and (\ref{category groupoid square-3}) are Cartesian, that (\ref{category groupoid square-4}) is commutative and that there exists $s$ satisfying (\ref{category groupoid morphism s equality-1}) and (\ref{category groupoid morphism s equality-2})\footnote{For a gropoid $X_*$, we often say that $X_0$ is the base of the groupoid and $X_1$ is the equvialence prerelation.}.

\begin{example}\label{category groupoid by group action}
Let $X$ be an object in $\mathcal{C}$ and $G$ be a $\mathcal{C}$-group acting on $X$ (on the left). We denote by $d_0:G\times X\to X$ the morphism defining the action of $G$ over $X$, by $d_1:G\times X\to X$ the projection onto the second factor, by $\mu:G\times G\to G$ the multiplication of $G$, and finally by $\pr_{2,3}$ the projection of $G\times G\times X=G\times(G\times X)$ onto the second and third factors. Then
\[\begin{tikzcd}[column sep=15mm]
G\times G\times X\ar[r,shift left=8pt,"\pr_{2,3}"]\ar[r,shift right=8pt,swap,"\id_G\times d_0"]\ar[r,"\mu\times\id_X"description]&G\times X\ar[r,shift left=2pt,"d_0"]\ar[r,shift right=2pt,swap,"d_1"]&X_0
\end{tikzcd}\]
is a groupoid in $\mathcal{C}$. For any $T\in\Ob(\mathcal{C})$, the groupoid $X_*(T)$ has object set $X(T)$ and morphisms $(g,x)$, where $g\in G(T)$ and $x\in X(T)$. Moreover, $X_*(T)$ is a setoid if and only if for any $x\in G(T)$, the automorphism group $\Aut(x)$ is trivial, that is, if and only if $G(T)$ acts freely on $X(T)$.
\end{example}

\begin{example}\label{category groupoid from equivalence relation}
Let $d_0,d_1:X_1\to X_0$ be an \textbf{equivalence couple}, that is, if $d_0\boxtimes d_1:X_1\to X_0\times X_0$ is the morphism with components $d_0$, $d_1$, then for any $T\in\Ob(\mathcal{C})$, $(d_0\boxtimes d_1)(T)$ is a bijection from $X_1(T)$ to the graph of an equivalence relation on $X_0(T)$. The set $X_1(T)$ is then identified with the set of couples $(x,y)$ formed by elements of $X_1(T)$ such that $x\sim y$; similarly, the set $X_2(T)=(X_1\times_{d_1,d_0}X_1)(T)$ is identified with the set of triples $(x,y,z)$ of elements of $X_0(T)$ such that $x\sim y$ and $y\sim z$. There is then a unique morphism $d_1':X_2\to X_1$ fitting into the squares (\ref{category groupoid square-2}) and (\ref{category groupoid square-3}): $d_1'(T)$ sends $(x,y,z)\in X_2(T)$ to $(x,z)\in X_1(T)$. For this choice of $d_1'$, $(d_0,d_1:X_1\rightrightarrows X_0,d_1')$ is a $\mathcal{C}$-groupoid.\par
Conversely, consider a $\mathcal{C}$-groupoid $X_*$ such that $d_0\boxtimes d_1:X_1\to X_0\times X_0$ is a monomorphism (in other words, for any $T\in\Ob(\mathcal{C})$ and $x,y\in X_0(T)$, there exists a unique morphism from $x$ to $y$). Then $(d_0,d_1)$ is an equivalence couple and $X_*$ can be reconstructed from $(d_0,d_1)$ as explained above\footnote{In particular, if $G$ is a $\mathcal{C}$-group acting on the left on an object $X$ of $\mathcal{C}$ and $X_*$ is the associated groupoid, then $(d_0,d_1)$ is an equivalence couple if and only if $G$ acts freely on $X$.}. 
\end{example}

\begin{example}
Let $p:X\to Y$ be a morphism in $\mathcal{C}$ and $\pr_1,\pr_2$ be two projections from $X\times_{p,p}X$ to $X$. Then $(\pr_1,\pr_2):X\times_{p,p}X\rightrightarrows X$ is an equivalence couple. We say that $p$ is an \textbf{effective epimorphism} if the diagram
\[\begin{tikzcd}
X\times_{p,p}X\ar[r,shift left=2pt,"\pr_1"]\ar[r,shift right=2pt,swap,"\pr_2"]&X\ar[r,"p"]&Y
\end{tikzcd}\]
is exact, that is, if $(Y,p)=\coker(\pr_1,\pr_2)$.\par
For example, let $S$ be a Noetherian scheme and $\mathcal{C}$ be the category of schemes finite over $S$. We show that an epimorphism in $\mathcal{C}$ is not necassarily effective: we choose $S=\Spec(k[T^3,T^5])$, where $k$ is a field, $Y=S$ and $X=\Spec(k[T])$. If $i$ denotes the inclusion of $B=k[T^3,T^5]$ to $A=k[T]$ and $p=\Spec(i)$, then $X\times_{p,p}X$ is identified with $\Spec(A\otimes A)$ and $\coker(\pr_1,\pr_2)$ is equal to $\Spec(B')$, where $B'$ is the subring of $A$ formed by elements $a$ such that $a\otimes_B1=1\otimes_Ba$. Now
\[T^7\otimes_B1=(T^2T^5)\otimes_B1=T^2\otimes_BT^5=T^2\otimes_B(T^3T^2)=T^5\otimes_BT^2=1\otimes_BT^7\]
hence $T^7\in B'$, $T^7\notin B$ and $\Spec(B)\neq\Spec(B')$, whence a couterexample\footnote{The same arguments apply to $B=k[T^n,T^{n+r}]$ and the element $T^{n+2r}\otimes_B1$, provided that $2r\nmid n$.}.
\end{example}

Consider a $\mathcal{C}$-groupoid
\[\begin{tikzcd}[column sep=12mm]
X_2\ar[r,shift left=8pt,"d_0'"]\ar[r,shift right=8pt,swap,"d_2'"]\ar[r,"d_1'"description]&X_1\ar[r,shift left=2pt,"d_0"]\ar[r,shift right=2pt,swap,"d_1"]&X_0
\end{tikzcd}\]
and let $f_0:Y_0\to X_0$ be a morphism in $\mathcal{C}$. Then by base change to $Y_0$, we obtain a $\mathcal{C}$-groupoid
\[\begin{tikzcd}[column sep=12mm]
Y_2\ar[r,shift left=8pt,"e_0'"]\ar[r,shift right=8pt,swap,"e_2'"]\ar[r,"e_1'"description]&Y_1\ar[r,shift left=2pt,"e_0"]\ar[r,shift right=2pt,swap,"e_1"]&Y_0
\end{tikzcd}\]
which is said to be \textbf{induced} by $X_*$ and $f_0$. We also say that $Y_*$ is the \textbf{inverse image} of $X_*$ by the base change morphism $f_0:Y_0\to X_0$. More precisely, we choose for $Y_1$ the fiber product of the diagram
\[\begin{tikzcd}
Y_1\ar[d,dashed]\ar[r,dashed,"f_1"]&X_1\ar[d,"d_0\boxtimes d_1"]\\
Y_0\times Y_0\ar[r,"f_0\times f_0"]&X_0\times X_0
\end{tikzcd}\]
for $e_0$ and $e_1$ the composition of the canonical morphism $Y_1\to Y_0\times Y_0$ and the first and second projections of $Y_0\times Y_0$. The morphism $Y_1\to Y_0\times Y_0$ is then $e_0\boxtimes e_1$, and we have $f_0\circ e_i=d_i\circ f_1$ for $i=0,1$, where we denote by $f_1$ the projection of $Y_1$ to $X_1$. We put $Y_2=Y_1\times_{e_0,e_1}Y_1$. We can say that the couple $(e_0,e_1)$ is defined such that, for any $T\in\Ob(\mathcal{C})$, and any couple $(y,x)$ of elements of $Y_0(T)$, there is a one-to-one correspondence $\psi\mapsto{_y\psi_x}$ between the set of morphisms $\psi$ of $X_*(T)$ with source $f_0(x)$, target $f_0(y)$ and the arrows ${_y\psi_x}$ of $Y_*(T)$ with source $x$ and target $y$. We therefore determine $e'_1:Y_2\to Y_1$ by defining for all $T\in\Ob(\mathcal{C})$ the composition of the morphism of $Y_*(T)$ using the formula
\[{_z\psi_y}\circ{_y\varphi_x}={_z(\psi\circ\varphi)_x}.\]
It is then clear that this makes $Y_*(T)$ a groupoid.\par
From the $\mathcal{C}$-groupoid $X_*$ and the base change $f_0:Y_0\to X_0$, we can reconstruct the couple $(e_0,e_1):Y_1\rightrightarrows Y_0$ in another way: consider $Y_0\times_{X_0}X_1$ and $\pr_1,\pr_2$ such that the square
\begin{equation}\label{category groupoid base change Y_1 square-1}
\begin{tikzcd}
Y_0\times_{X_0}X_1\ar[d,swap,"\pr_1"]\ar[r,"\pr_2"]&X_1\ar[d,"d_0"]\\
Y_0\ar[r,"f_0"]&X_0
\end{tikzcd}
\end{equation}
is Cartesian. We then verify that we have a Cartesian square
\begin{equation}\label{category groupoid base change Y_1 square-2}
\begin{tikzcd}
Y_1\ar[r,"e_0\boxtimes f_1"]\ar[d,swap,"e_1"]&Y_0\times_{X_0}X_1\ar[d,"d_1\circ\pr_2"]\\
Y_0\ar[r,"f_0"]&X_0
\end{tikzcd}
\end{equation}
where $f_1$ denotes the canonical projection from $Y_1=(Y_0\times Y_0)\times_{X_0\times X_0}X_1$ to $X_1$.

\begin{example}\label{category groupoid base change by d_0 and d_1}
Let's take $Y_0=X_1$, $f_0=d_0$. For any object $T$ of $\mathcal{C}$, $Y_1(T)$ is then identified with the set of diagrams of the form
\[\begin{tikzcd}
b\ar[r,"\varphi"]&d\\
a\ar[u,"f"]&c\ar[u,"g"]
\end{tikzcd}\]
of $X_*(T)$. The source of this diagram is the morphism $f$, the target is the morphism $g$, and the composition of two diagrams is clear (by taking the compositin of the horizontal morphisms).\par
Similarly, by choosing $Y_0'=X_1$ and $f_0'=d_1$, the set $Y_1'(T)$ is identified for any $T\in\Ob(\mathcal{C})$ with the set of diagrams of the form
\[\begin{tikzcd}
b&d\\
a\ar[r,"\psi"]\ar[u,"f"]&c\ar[u,"g"]
\end{tikzcd}\]
of the groupoid $X_*(T)$. The source of this diagram is the morphism $f$, the target is the morphism $g$, and the composition of two diagrams is evident.\par
Now since $X_*(T)$ is a groupoid for any $T\in\Ob(\mathcal{C})$, the identity map on $Y_0(T)$ and the map
\[\begin{tikzcd}
b\ar[r,"\varphi"]&d\\
a\ar[u,"f"]&c\ar[u,"g"]
\end{tikzcd}\mapsto\begin{tikzcd}
b&d\\
a\ar[r,"g^{-1}\varphi f"]\ar[u,"f"]&c\ar[u,"g"]
\end{tikzcd}\]
from $Y_1(T)$ to $Y'_1(T)$ define an isomorphism of groupoids $Y_*(T)$ and $Y'_*(T)$. Moreover, this isomorphism depends functorial on $T$, hence is an isomorphism of the $\mathcal{C}$-groupoids $Y_*$ and $Y_*'$.
\end{example}

\begin{proposition}\label{category groupoid cokernel under universal effective base change prop}
Let $X_*$ be a $\mathcal{C}$-groupoid and suppose that $f_0:Y_0\to X_0$ is a universally effective epimorphism. Then $\coker(d_0,d_1)$ exists if and only if $\coker(e_0,e_1)$ exists. Moreover, in this case $f_0$ induces an isomorphism
\[\coker(d_0,d_1)\stackrel{\sim}{\to}\coker(e_0,e_1).\]
\end{proposition}
\begin{proof}
We denote by $C(d_0,d_1)$ the covariant functor from $\mathcal{C}$ to the category of sets which associates to any $T\in\Ob(\mathcal{C})$ the kernel of the couple $T(d_0),T(d_1):T(X_0)\rightrightarrows T(X_1)$ (in $\Set$), and similarly for $C(e_0,e_1)$. For any $T\in\mathcal{C}$, we then have a commutative diagram
\[\begin{tikzcd}
C(d_0,d_1)(T)\ar[r]\ar[d,"T(f)"]&T(X_0)\ar[r,shift left,"T(d_1)"]\ar[r,shift right,swap,"T(d_0)"]\ar[d,"T(f_0)"]&T(X_1)\ar[d,"T(f_1)"]\\
C(e_0,e_1)(T)\ar[r]&T(Y_0)\ar[r,shift left,"T(e_1)"]\ar[r,shift right,swap,"T(e_0)"]&T(Y_1)
\end{tikzcd}\]
where $T(f)$ is the injection induced by the injection $T(f_0)$. If we can show that $T(f)$ is a surjection for any $T$, then we obtain a functorial isomorphsim $f:C(d_0,d_1)\stackrel{\sim}{\to}C(e_0,e_1)$, which implies the proposition. For this, consider the diagram
\[\begin{tikzcd}
&Y_1\ar[r,"f_1"]\ar[d,shift left=2pt,"e_0"]\ar[d,shift right=2pt,swap,"e_1"]&X_1\ar[d,shift left=2pt,"d_0"]\ar[d,shift right=2pt,swap,"d_1"]\\
Y_0\times_{X_0}Y_0\ar[ru,"\Delta"]\ar[r,shift left=2pt,"\pr_2"]\ar[r,shift right=2pt,swap,"\pr_1"]&Y_0\ar[d,"g"]\ar[r,"f_0"]&X_0\ar[ld,dashed,"h"]\\
&T&
\end{tikzcd}\]
where $\Delta$ is the section of $Y_1\to Y_0\times Y_0$ defined by the morphism $s\circ f_0\circ\pr_1:Y_0\times Y_0\to X_1$, the morphism $s:X_0\to X_1$ satisfying the equalities (\ref{category groupoid morphism s equality-1}) and (\ref{category groupoid morphism s equality-2}). If a morphism $g:Y_0\to T$ is such that $ge_0=ge_1$, we then have $ge_0\Delta=ge_1\Delta$, hence $g\pr_1=g\pr_2$. As $f_0$ is an effective epimorphism, $g$ is then the composition of $f_0$ with a morphism $h:X_0\to T$, that is, we have $g=T(f_0)(h)$. It remains to show that $h$ belongs to $C(d_0,d_1)(T)$, which means $hd_0=hd_1$; now we have
\[hd_0f_1=hf_0e_0=ge_0=ge_1=hf_0e_1=hd_1f_1\]
whence the desired equality since $f_1$ is an epimorphism (because $f_0$ is a universally epimorphism).
\end{proof}

Consider now a scheme $S$ and choose $\mathcal{C}=\Sch_{/S}$. Then a $\mathcal{C}$-groupoid
\[\begin{tikzcd}[column sep=12mm]
X_2\ar[r,shift left=8pt,"d_0'"]\ar[r,shift right=8pt,swap,"d_2'"]\ar[r,"d_1'"description]&X_1\ar[r,shift left=2pt,"d_0"]\ar[r,shift right=2pt,swap,"d_1"]&X_0
\end{tikzcd}\]
permits us to define an equivalence relation on the underlying set $|X_0|$ : if $x,y\in|X_0|$, we write $x\sim y$ if there exists $z\in|X_1|$ such that $x=d_1(z)$ and $y=d_0(z)$. The reflecxivity and symmetricity of this equation is evident\footnote{The reflecxivity follows from the existence of $s:X_0\to X_1$ which is a section of $d_0$ and $d_1$; the symmetricity follows from the existence of an involution $\sigma$ of $X_1$ which exchanges $d_0$ and $d_1$.}. As for the transtivity, if $x\sim y$ and $y\sim z$, then there exists $u,v\in |X_1|$ such that $x=d_1(u)$, $y=d_0(u)$, $y=d_1(v)$, $z=d_0(v)$. It then follows that $(v,u)$ belongs to the set $|X_1|\times_{d_1,d_0}|X_1|$. As the canonical map
\[|X_1\times_{d_1,d_0}X_1|\to |X_1|\times_{d_1,d_0}|X_1|\]
on underlying sets is surjective, $(v,u)$ is the image of some $w\in|X_2|$. We then have $x=d_1d_1'(w)$ and $z=d_0d_1'(w)$, then $x\sim z$.\par
Now let $f_0:Y_0\to X_0$ be a base change morphism of schemes over $S$. If $x,y$ are points of $|Y_0|$, we see that $x\sim y$ if and only if $f_0(x)\sim f_0(y)$. In fact, if $x\sim y$, then there exists $z\in|Y_1|$ such that $x=e_1(z)$ and $y=e_0(z)$. As $f_0\circ e_i=d_i\circ f_1$ for $i=0,1$, we then have $f_0(x)=d_1f_1(z)$ and $f_0(y)=d_0f_1(z)$, whence $f_0(x)\sim f_0(y)$.\par
Conversely, if $f_0(x)\sim f_0(y)$ and $z\in|X_1|$ is such that $f_0(y)=d_1(z)$ and $f_0(x)=d_0(z)$, then by the square (\ref{category groupoid base change Y_1 square-1}), there exists a point $t\in|Y_0\times_{X_0}X_1|$ such that $\pr_1(t)=x$ and $\pr_2(t)=z$. Similarly, as $f_0(y)=d_1\pr_2(t)$, there exists $s\in|Y_1|$ such that $y=e_1(s)$ and $(e_0\boxtimes f_1)(s)=t$ (cf. the square (\ref{category groupoid base change Y_1 square-2})). We then have $e_0(s)=\pr_1(e_0\boxtimes f_1)(s)=\pr_1(t)=x$, whence $x\sim y$.

\subsection{Quotient for a finite locally free groupoid}
Let $S$ be a scheme and consider a $\Sch_{/S}$-grupoid 
\[\begin{tikzcd}[column sep=12mm]
X_2\ar[r,shift left=8pt,"d_0'"]\ar[r,shift right=8pt,swap,"d_2'"]\ar[r,"d_1'"description]&X_1\ar[r,shift left=2pt,"d_0"]\ar[r,shift right=2pt,swap,"d_1"]&X_0
\end{tikzcd}\]
In this subsection, we prove the existence of a quotient of $X_*$ under the hypothesis that the strucutral morphism is finite and locally free. More precisely, we shall prove the following theorem:
\begin{theorem}\label{scheme groupoid quotient by locally free finite prop}
Suppose that $X_*$ satisfies the following conditions\footnote{As $d_0$ and $d_1$ are exchanged by the involution $\sigma$, these conditions are symemtric on $d_0$ and $d_1$. Moreover, for any $x\in X_0$ we have $d_0d_1^{-1}(x)=d_1d_0^{-1}(x)$.}:
\begin{enumerate}
    \item[(\rmnum{1})] $d_1$ is locally free and finite.
    \item[(\rmnum{2})] For any $x\in X_0$, the set $d_0d_1^{-1}(x)$ is contained in an affine open subset of $X_0$.
\end{enumerate}
Then we have the following:
\begin{enumerate}
    \item[(a)] There exists a cokernel $(Y,p)$ of $(d_0,d_1)$ in $\Sch_{/S}$. Moreover, such a pair $(Y,p)$ is a cokernel of $(d_0,d_1)$ in the category of ringed spaces.
    \item[(b)] The morphism $p$ is open and integral, and $Y$ is affine if $X_0$ is affine.
    \item[(c)] The morphism $X_1\to X_0\times_YX_0$ with components $d_0$ and $d_1$ is surjective.
    \item[(d)] If $(d_0,d_1)$ is an equivalence couple, then the morphism $X_1\to X_0\times_YX_0$ in (c) is an isomorphism and $p:X_0\to Y$ is locally free and finite. Further, $(Y,p)$ is a cokernel of $(d_0,d_1)$ in the category of sheaves for the fppf topology and, for any base change $Y'\to Y$, $Y'$ is the cokernel of the groupoid $X_*\times_YY'$ induced from $X_*$ by base change. 
\end{enumerate}
In particular, for any base change $S'\to S$, $Y'=Y\times_SS'$ is the cokernel of the $S'$-groupoid $X_*'=X_*\times_SS'$. Hence, in this case, the formation of quotient commutes with base change.
\end{theorem}

It follows from \cref{scheme groupoid quotient by locally free finite prop}~(a) that the underlying topological space of $Y$ is the quotient of that of $X_0$ by the equivalence relation defined by the groupoid $X_*$. The rest of this subsection is devoted to the proof of \cref{scheme groupoid quotient by locally free finite prop}.

\paragraph{Quotient by a finite and locally free groupoid (affine case)}\label{scheme groupoid quotient by locally free finite affine case paragraph}
We first prove the theorem under the asssumption that $X_0$ is affine and $d_1$ is locally free of constant rank $n$ (then we shall see how to reduce the general case to this particular one). In this case, $X_0$, $X_1$ and $X_2$ are all affine, so we can suppose that $X_i=\Spec(A_i)$, $d_j=\Spec(\delta_j)$, $d_k'=\Spec(\delta_k')$, where $\delta_j$, $\delta_k'$ are homomorphism of rings. From (\ref{category groupoid commutative diagram}), we then obtain a solid diagram
\begin{equation}\label{scheme groupoid quotient by locally free finite prop-1}
\begin{tikzcd}
A_2&A_1\ar[l,shift left=2pt,"\delta_0'"]\ar[l,shift right=2pt,swap,"\delta_1'"]&A_0\ar[l,swap,"\delta_0"]\\
A_1\ar[u,"\delta_2'"]&A_0\ar[l,shift left=2pt,"\delta_0"]\ar[l,shift right=2pt,swap,"\delta_1"]\ar[u,"\delta_1"]&B\ar[u,dashed,"\delta"]\ar[l,swap,"\delta"]
\end{tikzcd}
\end{equation}
where the two squares are cocartesian and the first row is exact. We denote by $B$ the subring of $A_0$ formed by $a\in A_0$ such that $\delta_0(a)=\delta_1(a)$, and let $\delta:B\to A_0$ be the canonical inclusion. If $a_0\in A_0$, let
\[P_{\delta_1}(T,\delta_0(a))=T^n-\sigma_1T^{n-1}+\cdots+(-1)^n\sigma_n\]
be the characteristic polynomial of $\delta_0(a)$ if we consider $A_1$ as an $A_0$-algebra via the homomorphism $\delta_1$ (cf. \cite{Bourbaki_CA1-7} \Rmnum{2} \S 5, exercice 9). As the two squares of (\ref{scheme groupoid quotient by locally free finite prop-1}) are cocartesian, we have
\begin{align*}
\delta_0(P_{\delta_1}(T,\delta_0(a)))=P_{\delta_2'}(T,\delta_0'\delta_0(a)),\quad \delta_1(P_{\delta_1}(T,\delta_0(a)))=P_{\delta_2'}(T,\delta_1'\delta_0(a)).
\end{align*}
As $\delta_0'\delta_0=\delta_1'\delta_0$, we then conclude that
\[\delta_0(P_{\delta_1}(T,\delta_0(a)))=\delta_1(P_{\delta_1}(T,\delta_0(a)))\]
that is, $\delta_0(\sigma_i)=\delta_1(\sigma_i)$ for each $i$. By Hamilton-Cayley theorem, we then have
\begin{align*}
0=\delta_1(P_{\delta_1}(T,\delta_0(a)))(\delta_0(a))&=\delta_0(a)^n-\delta_1(\sigma_1)\delta_0(a)^{n-1}+\cdots+(-1)^{n}\delta_1(\sigma_n)\\
&=\delta_0(a)^n-\delta_0(\sigma_1)\delta_0(a)^{n-1}+\cdots+(-1)^{n}\delta_0(\sigma_n),
\end{align*}
whence
\[a^n-\sigma_1a^{n-1}+\cdots+(-1)^{n}\sigma_n=0\]
because $\delta_0$ has a section $\tau:A_1\to A_0$ such that $\tau\delta_0=\id_{A_0}$, so it is injective. We then conclude that \textit{$A_0$ is integral over $B$}.\par
Now consider two prime ideals $\p,\q$ of $A_0$; we show that the equality $\p\cap B=\q\cap B$ implies the existence of a prime ideal $\r$ of $A_1$ such that $\p=d_0(\r)$ and $\q=d_1(\r)$. In fact, if the assertion was not true, $\p$ would be distinct from $\delta_0^{-1}(\n)$ any prime ideal $\n$ of $A_1$ such that $\delta_1^{-1}(\n)=\q$. For such an ideal $\n$ we would have $\delta_0^{-1}(\n)\cap B=\delta_1^{-1}(\n)\cap B=\q\cap B=\p\cap B$, so by \cref{integral ring extension prime lying over same contraction}, $\p$ is not contained in $\delta_0^{-1}(\n)$. But there are finitely many prime ideals $\n$ of $A_1$ such that $\delta_1^{-1}(\n)=\q$ (\cref{integral finite ring lying over prime finite}), hence, by prime aviodence, there exists $a\in\p$ which is not in any of the $\delta_0^{-1}(\n)$. Therefore, $\delta_0(a)$ is not conained in these ideals $\n$, and hence, by the lemma bolow, the norm $N_{\delta_1}(\delta_0(a))$ does not belong to $B\cap\q$ (the norm is calculated by considering $A_1$ as an $A_0$-algebra via the homomorphism $\delta_1$, and we have $N_{\delta_1}(\delta_0(a))=\sigma_n$ with the notations above). As $(-1)^{n-1}\sigma_n=a^n+\sum_{i=1}^{n-1}(-1)^i\sigma_ia^{n-i}$, this norm belongs to $B\cap\p=B\cap\q$, which is a contradiction.

\begin{lemma}\label{ring homomorphism lying over prime iff norm belong to p}
Let $A\to A'$ be a ring homomorphism such that $A'$ is a projective $A$-module of rank $n$. Let $\p$ be a prime ideal of $A$ and $\q_1,\dots,\q_r$ be the prime ideals of $A'$ lying over $\p$. Let $a\in A'$, then $a$ belongs to $\q_1\cup\cdots\cup\q_r$ if and only if its norm $N(a)$ belongs to $\p$.
\end{lemma}
\begin{proof}
By replacing $A$ and $A'$ by the localizations $A_\p$ and $A'_\p$, we may assume that $(A,\p)$ is local and $A'$ is semi-local, with $\Spec(A')=\{\q_1,\dots,\q_r\}$. In this case, $A'$ is a free $A$-module of rank $n$, and $N(a)$ is the determinant of the endomorphism $h_a:A'\to A'$ with ratio $a$. We then conclude that $N(a)\notin\p$ if and only if $N(a)$ is invertible, if and only if $h_a$ is invertible, and this is equivalent to that $a\notin\q_1\cup\cdots\cup\q_r$. 
\end{proof}

We now prove assertion (a) of \cref{scheme groupoid quotient by locally free finite prop} in this case. Let $Y=\Spec(B)$ and $p=\Spec(\delta)$, where $\delta:B\to A_0$ is the inclusion. By the preceding arguments, $p:X_0\to Y$ is integral, hence surjective, and the underlying space of $\Spec(B)$ is obtained from that of $X_0$ by identifying points $x,y$ such that thre exists $z\in X_1$ such that $d_1(z)=y$, $d_0(z)=x$. Moreover, as $i$ is integral, $p$ is closed (\cref{integral ring closed map on spec}) so that $Y$ is endowed with the quotient topology of that of $X_0$. In particular, $p$ is open: if $U$ is an open subset of $X_0$, as $d_1$ is surjective and locally free and finite (hence faithfully flat and finitely presented), hence open, the saturation $U'=d_1d_0^{-1}(U')$ of $U'$ under the equivalence relation defined by $X_*$ is open, so $p(U')=p(U)$ is open, since $Y$ is endowed with the quotient topology.\par
Finally, it follows from the choice of $B$ and the fact that $p$, $d_0$, $d_1$ are affine that the canonical sequence of sheaf of rings
\[\begin{tikzcd}
\mathscr{O}_Y\ar[r]&p_*(\mathscr{O}_{X_0})\ar[r,shift left=2pt,"p_*(\delta_1)"]\ar[r,shift right=2pt,swap,"p_*(\delta_0)"]&p_*((d_0)_*(\mathscr{O}_{X_1}))=p_*((d_1)_*(\mathscr{O}_{X_1}))
\end{tikzcd}\]
is exact. It remains to prove that $(Y,p)$ is also the cokernel of $(d_0,d_1)$ in the category of schemes (or more generally in the category of locally ringed spaces). Let $f:X_0\to Z$ be a morphism of schemes such that $fd_0=fd_1$. By the above arguments, there exists a unique morphism of ringed spaces $\tilde{f}:Y\to Z$ such that $f=\tilde{f}p$. Since the composition $f$ and $p$ are both local morphisms, we conclude that $\tilde{f}$ is a local morphism, hence a morphism of schemes.\par
Now the assertion (b) of \cref{scheme groupoid quotient by locally free finite prop} follows immediately. On the other hand, since $p:|X_0|\to |Y|$ is a quotient map, the following map
\[d_0\boxtimes d_1:|X_1|\to |X_0|\times_{|Y|}|X_0|\]
is surjective. This map factors into
\[|X_1|\stackrel{d_0\boxtimes d_1}{\longrightarrow} |X_0\times_YX_0|\stackrel{q}{\longrightarrow}|X_0|\times_{|Y|}|X_0|\]
where $q$ is the canonical map. We therefore conclude that the image of $d_0\boxtimes d_1$ then contains points $v\in X_0\times_YX_0$ such that $\{v\}=q^{-1}(q(v))$, which is staisfied if $v$ is a rational point over $Y$ (that is, if the residue field $\kappa(v)$ is identified with $\kappa(w)$, where $w$ is the image of $v$ in $Y$). If $v\in X_0\times_YX_0$ is not rational over $Y$, let $w$ be the image of $v$ in $Y$. By (\cite{EGA3} $0_{\Rmnum{3}}$, 10.3.1) there exists a local ring $C$ and a flat local homomorphism $\rho:\mathscr{O}_w\to C$ such that $C/\m_wC$ is isomorphic to $\kappa(v)$ as $\kappa(w)$-algebras. If we put $Y'=\Spec(C)$ and $\pi:Y'\to Y$ is the morphism induced by $\rho$, it is clear that the canonical projection of $(X_0\times_YX_0)\times_YY'$ onto $X_0\times_YX_0$ sends $v$ to a point $v'$ of $(X_0\times_YX_0)\times_YY'$ which is rational over $Y'$. As
\[(X_0\times_YX_0)\times_YY'\cong(X_0\times_YY')\times_{Y'}(X_0\times_YY'),\]
and as the hypothesis of \cref{scheme groupoid quotient by locally free finite prop} and the preceding results, in particular that of (b), is valid under base change $\pi:Y'\to Y$, we conclude that $v'$ is the image of an element $u'\in X_1\times_YY'$ under the morphism deduced from $d_0\boxtimes d_1$ by base change. If $u$ is the image of $u'$ in $X_1$, we then have $v=(d_0\boxtimes d_1)(u)$.\par
Finally, we prove assertion (d) of \cref{scheme groupoid quotient by locally free finite prop}. By hypothesis, $X_0=\Spec(A_0)$, $X_1=\Spec(A_1)$, and for $i=0,1$, the morphism $\delta_i:A_0\to A_1$ is finite; hence the morphism $A_0\otimes_BA_0\to A_1$ is finite. Since $(d_0,d_1)$ is assumed to be an equivalence couple, we may further assume that $d_0\boxtimes d_1:X_1\to X_0\times_YX_0$ is a monomorphism; then, by (\cite{EGA4-4}, 18.12.6), $d_0\boxtimes d_1$ is a closed immersion, so $A_0\otimes_BA_0\to A_1$ is surjective. We will show that this is an isomorphism (and also that $p:X_0\to Y$ is finite and locally free). For this, it suffices to show that for any prime ideal $\p$ of $B$, the homomorphism $(A_0)_\p\otimes_{B_\p}(A_0)_\p\to (A_1)_\p$ with components $(\delta_0)_\p$ and $(\delta_1)_\p$ is bijective. In other words, we may assume that $B$ is local. It then follows from \cref{integral finite ring lying over prime finite} that $(A_0)_\p$ is semi-local. By applying (\cite{EGA3} $0_{\Rmnum{3}}$, 10.3.1) to perform a faithfully flat base change, we may also assume that the residue field of $B$ is infinite, so that we can use the following lemma:

\begin{lemma}\label{semilocal ring submodule generating contain basis}
Let $B$ be a local ring with infinite residue field, $A$ be a semi-local ring and $i:B\to A$ be a homomorphism wich sends the maximal ideal $\m$ of $B$ into the radical $\r$ of $A$. Let $M$ be a free $A$-module of rank $n$ and $N$ be a sub-$B$-module of $M$ which generates $M$ as an $A$-module. Then $N$ contains a basis of $M$ over $A$.
\end{lemma}
\begin{proof}
We recall that, by Nakayama's lemma, a sequence $m_1,\dots,m_n$ of elements of $M$ is an $A$-basis of $M$ if and only if the canonical images of $m_1,\dots,m_n$ in $M/\r M$ form a basis of $M/\r M$ over $A/\r$. We can then replace $M$ by $M/\r M$, $N$ by $N/(N\cap\r M)$, $A$ by $A/\r$ and $B$ by $B/\m$. In this case, we then have $A=K_1\times\cdots\times K_r$, and $M$ can be identified with $K_1^n\times\cdots\times K_r^n$. Now we choose elements $(x_{ij})_{1\leq i\leq r,1\leq j\leq n}$ of $N$ such that for each $1\leq i\leq r$, the $i$-th component of $(x_{i,1},\dots,x_{i,n})$ in $K_i^n$ is linearly independent over $K_i$. We can consider the polynomial
\[f(a_{1,1},\dots,a_{n,r})=\prod_{i=1}^{r}\det_{K_i}(\sum_{j=1}^{n}a_{j,1}x_{j,1},\dots,\sum_{j=1}^{n}a_{j,r}x_{j,r})\]
where $\det_{K_i}(z_1,\dots,z_n)$ denotes the determinant of the $i$-th components of $z_1,\dots,z_n$ over $K_i$. As $k$ is an infinite field and each polynomial $\det_{K_i}(\sum_{j=1}^{n}a_{j,1}x_{j,1},\dots,\sum_{j=1}^{n}a_{j,r}x_{j,r})$ with coefficient in $A$ is nonzero (take $a_{i,1}=\cdots=a_{i,n}=1$ and others to be zero), we conclude that there exists a family $(y_\ell)_{1\leq\ell\leq n}$ of $k$-linear combinations of the $x_{ij}$ such that for any $1\leq i\leq n$, the $i$-th component of $(y_\ell)_{1\leq\ell\leq n}$ is linearly independent over $K_i$. Then $(y_\ell)_{1\leq\ell\leq n}$ is easily seen to be a basis of $M$ over $A$.
\end{proof}

We shall apply \cref{semilocal ring submodule generating contain basis} for the ring homomorphism $B\to A_0$, $M=A_1$ (as an $A_0$-module via the homomorphism $\delta_1$), and $N=\delta_0(A_0)$. In fact, as $d_0\boxtimes d_1:X_1\to X_0\times_YX_0$ is a closed immersion, the homomorphism $A_0\otimes_BA_0\to A_1$ with components $\delta_0$ and $\delta_1$ is surjective; this signifies that $\delta_0(A_0)$ generates the $A_0$-module $A_1$.\par
Let $a_1,\dots,a_n$ be elements of $A_0$ such that $\delta_0(a_1),\dots,\delta_0(a_n)$ form a basis of $A_1$ over $A_0$. If we can show that $a_1,\dots,a_n$ is a basis of $A_0$ over $B$, then the homomorphism $A_0\otimes_BA_0\to A_1$ sends the basis $(1\otimes a_i)_{1\leq i\leq n}$ to the basis $(\delta_0(a_i))_{1\leq i\leq n}$, hence is bijective. Therefore, if $\eps:\Z^n\to A_0$ is the homomorphism of abelian groups sending the natural basis of $\Z^n$ to $a_1,\dots,a_n$, it suffices to prove that the map $B\otimes_{\Z}\Z^n\to A_0$ with components $i$ and $\eps$ is bijective. Now the diagram (\ref{scheme groupoid quotient by locally free finite prop-1}) gives the following commutative diagram:
\[\begin{tikzcd}
A_2&A_1\ar[l,shift left=2pt,"\delta_0'"]\ar[l,shift right=2pt,swap,"\delta_1'"]&A_0\ar[l,swap,"\delta_0"]\\
A_1\otimes_{\Z}\Z^n\ar[u,"u_2"]&A_0\otimes_{\Z}\Z^n\ar[l,shift left=2pt,"\delta_0\otimes 1"]\ar[l,shift right=2pt,swap,"\delta_1\otimes 1"]\ar[u,"u_1","\sim"']&B\otimes_{\Z}\Z^n\ar[l,swap,"\delta\otimes 1"]\ar[u,"u_0"]
\end{tikzcd}\]
where $u_0$, $u_1$ and $u_2$ have components $\delta$ and $\eps$, $\delta_1$ and $\delta_0\eps$, $\delta_2'$ and $\delta_0'\delta_0\eps$, respectively. We see that $u_1$ is an isomorphism. As the two squares in (\ref{scheme groupoid quotient by locally free finite prop-1}) are cocartesian, $u_2$ is then an isomorphism. Since the two horizontal rows of the above diagram are exact, we conclude that $u_0$ is bijective. This shows that $A_0$ is a locally free $B$-module of rank $n$, whence $\delta_0\otimes\delta_1:A_0\otimes_BA_0\to A_1$ is an isomorphism. This proves \cref{scheme groupoid quotient by locally free finite prop} in the particular case where $X_0$ is affine and $d_1$ is locally free of rank $n$.

\paragraph{Quotient by a finite and locally free groupoid (general case)}\label{scheme groupoid quotient by locally free finite general case paragraph}
Let $U^n$ be the largest open subset of $X_0$ over which $d_1$ is finite and locally free of rank $n$. We know that $X_0$ is the direct sum of these $U^n$. On the other hand, it follows from the Cartesian squares
\[\begin{tikzcd}
X_2\ar[r,"d_0'"]\ar[d,swap,"d_2'"]&X_1\ar[d,"d_1"]\\
X_1\ar[r,"d_0"]&X_0
\end{tikzcd}\quad\quad \begin{tikzcd}
X_2\ar[r,"d_1'"]\ar[d,swap,"d_2'"]&X_1\ar[d,"d_1"]\\
X_1\ar[r,"d_1"]&X_0
\end{tikzcd}\]
that the inverse images of $U^n$ under $d_0$ and $d_1$ both coincide with the lagest open subset of $X_1$ over which $d_2'$ is locally free of rank $n$: in fact, as $d_0$ (resp. $d_1$) is surjective, finite and flat, hence faithfully flat and affine, $d_2'$ is of rank $n$ at a point $x$ of $X_1$ if and only if $d_1$ is of rank $n$ on a neighborhood of $d_0(x)$ (resp. $d_1(x)$). We then have $d_0^{-1}(U^n)=d_1^{-1}(U^n)$ so that the groupoid $X_*$ is the direct sum of the groupoids $X_*^n$ induced from $X_*$ on the open and closed subsets $U^n$. It then suffices to prove \cref{scheme groupoid quotient by locally free finite prop} for each $X_*^n$, so we can assume that $d_1$ is locally free of finite rank $n$.\par 
We now prove the theorem in the general case. By the above arguments, we can assume that $d_1$ is locally free of rank $n$. Let $(Y,p)$ be a cokernel of $(d_0,d_1)$ in the category of ringed spaces. Then as before, for (a) it suffices to prove that $Y$ is a scheme and $p:X_0\to Y$ is a morphism of schemes. By \cref{ringed space groupoid cokernel in scheme if}, this question is local over $Y$: let $y\in Y$ and $x_0\in X_0$ such that $p(x)=y$; if $x$ has a saturated affine open neighborhood $U$, then $p(U)$ is an affine open of $Y$ by \cref{scheme groupoid quotient by locally free finite prop}~(b) in the affine case and $p|_U$ is a morphism of schemes. It then suffices to prove that any $x\in X_0$ has a saturated affine open neighborhood $U$. Here is how we proceed (the proof is taken from \cite{SGA1}, \Rmnum{8}, cor. 7.6):
\[d_1(d_0^{-1}(x))\sub U=\widetilde{V}_f\sub V_f\sub \widetilde{V}\sub V\sub X_0.\]

By condition (\rmnum{2}) of \cref{scheme groupoid quotient by locally free finite prop}, there exists an open affine subset $V$ of $X_0$ containing $d_1(d_0^{-1}(X))$; if $F=X_0-V$, $d_1(d_0^{-1}(F))$ is closed because $d_1$ is integral and $\widetilde{V}=X_0-d_1(d_0^{-1}(F))$ is the largest saturated open subset containing $V$. As $\widetilde{V}$ is an neighborhood of the finite set $d_1(d_0^{-1}(x))$ ($d_0$ is also finite, hence has finite fiber), there exists a section $f$ of the structural sheaf of $V$ which vanishes over $V-\widetilde{V}$ and such that $d_1(d_0^{-1}(x))$ is contained in the open subset $V_f\sub V$ formed by points where $f$ is non-vanishing\footnote{This can be proved by applying the prime avoidence lemma, since $V$ is assumed to be affine.}. We then see that the largest saturated open subset $\widetilde{V}_f$ of $V_f$ is affine: in fact, let $Z(f)=\widetilde{V}-V_f$. Then $d_0^{-1}(Z(f))$ is the set of points of $d_0^{-1}(\widetilde{V})=d_1^{-1}(\widetilde{V})$ where the image $d_0^*(f)$ of $f$ under the map induced by $d_0$ vanishes. On the other hand, as $d_1$ induces a locally free morphism of rank $n$ from $d_0^{-1}(\widetilde{V})=d_1^{-1}(\widetilde{V})$ to $\widetilde{V}$, by \cref{ring homomorphism lying over prime iff norm belong to p}, $d_1(d_0^{-1}(Z(f)))$ is the set of points where the norm $N(d_0^*(f))$ relative to the morphism $d_1$ vanishes. It follows that $\widetilde{V}_f=\widetilde{V}-d_1(d_0(Z(f)))$ is the set of points of $V_f$ where $N(d_0^*(f))$ does not vanish, so $\widetilde{V}_f$ is affine.\par
We therefore conclude assertion (a), and (b), (c), together with the first part of (d), are then clear from the affine case. It remains to see the other concequences in (d). By hypothesis, the groupoid $X_*$ is given by an equivalence relation $R\to X_0\times_SX_0$, and we have proved that $R$ is effective and that $p:X_0\to Y=X_0/R$ is surjective, finite and locally free, hence, in particular, faithfully flat and finitely presented. Therefore, if $\mathcal{M}$ is the family of faithfully flat morphisms locally of finite presentaion, then $R$ is $\mathcal{M}$-effective. By \cref{scheme equivalence relation M_i-effective iff quotient represent}, we conclude that $(Y,p)$ represents the quotient sheaf of $X_0$ by $R$ for the fppf topology, and the assertion concerning base change follows from \cref{category equivalence relation M-effective prop}.

\begin{remark}
With the hypothesis and notations of \cref{scheme groupoid quotient by locally free finite prop}, suppose that $S$ is locally Noetherian and $\pi_0:X_0\to S$ is quasi-projective. Let $\mathscr{A}$ be an ample $\mathscr{O}_{X_0}$-module relative to $\pi_0$. By \cref{scheme morphism finite locally free iff direct image}, $p_*(\mathscr{A})$ is an invertible $p_*(\mathscr{O}_{X_0})$-module, so there exists a covering $(V_i)_{i\in I}$ of $Y$ by affine opens, such that $\mathscr{A}$ is trivial over each saturated affine open subset $U_i=p^{-1}(V_i)$. For each $i\in I$, let $A_{i,0}=\mathscr{O}_{X_0}(U_i)$, $A_{i,1}=\mathscr{O}_{X_1}(d_0^{-1}(U_i))=\mathscr{O}_{X_1}(d_1^{-1}(U_i))$, $\delta_{i,0}$ (resp. $\delta_{i,1}$) be the morphism $A_{i,0}\to A_{i,1}$ induced by $d_0$ (resp. $d_1$), and $B_i=\mathscr{O}_Y(V_i)=\{b\in A_{i,0}:\delta_{i,0}(b)=\delta_{i,1}(b)\}$. Following \autoref{scheme norm of invertible sheaf subsection}, consider the invertible $\mathscr{O}_{X_0}$-module $N_{d_1}(d_0^*(\mathscr{A}))$, the norm of $d_0^*(\mathscr{A})$ relative to the finite and locally free morphism $d_1:X_1\to X_0$. If $\mathscr{A}$ is given, relative to the open covering $(U_i)_{i\in I}$, by the transition functions $c_{ij}\in\mathscr{O}_{X_0}(U_i\cap U_j)^{\times}$, then $N_{d_1}(d_0^*(\mathscr{A}))$ is given by the transition functions $N_{\delta_1}(\delta_0(c_{ij}))\in\mathscr{O}_{X_0}(U_i\cap U_j)^\times$. As, by the proof of \ref{scheme groupoid quotient by locally free finite affine case paragraph}, these elements belong to $\mathscr{O}_Y(U_i\cap U_j)^\times$, they define an invertible $\mathscr{O}_Y$-module $\mathscr{L}$, such that $p^*(\mathscr{L})=N_{d_1}(d_0^*(\mathscr{A}))$. We also note that, for any $n\in\N$, we have $p^*(\mathscr{L}^n)=N_{d_1}(d_0^*(\mathscr{A}^n))$.\par
We now prove that $\mathscr{L}$ is ample for the morphism $\pi:Y\to S$ (the proof that $\pi:Y\to S$ is of finite type follows from that of \cref{scheme groupoid quasi-section lemma}~(b)). For this, by replacing $S$ with an affine open, we may assume that $S$ is affine. Let $y\in Y$, $x\in X_0$ such that $p(x)=y$, $V$ be an open subset of $Y$ containing $y$, and $U=p^{-1}(V)$. As $\mathscr{A}$ is $\pi_0$-ample, there exists an integer $n\geq 1$ and a section $s\in\Gamma(X_0,\mathscr{A}^n)$ such that the open subset $(X_0)_s$ satisfies $x\in (X_0)_s\sub U$ (cf. \cref{scheme ample sheaf iff}). With the preceding notations, $s$ is given by the sections $a_i\in A_{i,0}=\mathscr{O}_{X_0}(U_i)$ such that $a_i=c_{ij}a_j$ over $U_i\cap U_j$, and $(X_0)_s$ is the union of $U_i'=\{p\in\Spec(A_{i,0}):a_i\notin\p\}$. For each $i\in I$, put $N(a_i)=N_{\delta_1}(\delta_0(a_i))\in B_i$. By \cref{scheme groupoid quotient by locally free finite prop}~(a) and \cref{ring homomorphism lying over prime iff norm belong to p}, we have
\[p(U'_i)=pd_1d_0^{-1}(U_i')=pd_1(\{\q\in\Spec(A_{i,1}):\delta_{i,0}(a_i)\notin\q\})\]
and $d_1(\{\q\in\Spec(A_{i,1}):\delta_{i,0}(a_i)\notin\q\})=\{\p\in\Spec(A_{i,0}):N_{\delta_1}(\delta_{i,0}(a_i))\notin\p\}$, whence
\[p(U_i')=\{\p\in\Spec(B_i):N(a_i)\notin\p\}.\]
It then follows that $p((X_0)_s)=Y_{N(s)}$, so if we denote by $N(s)$ the section of $\mathscr{L}^n$ over $Y$ defined by the sections $N(a_i)\in\mathscr{O}_Y(V_i)$, we then have
\begin{equation}
y\in p((X_0)_s)=Y_{N(s)}\sub p(U)=V.
\end{equation} 
This proves that $\mathscr{L}$ is ample for $\pi:Y\to S$ (\cref{scheme ample sheaf iff}), so $\pi:Y\to S$ is quasi-projective.
\end{remark}

\subsection{Quasi-sections for a groupoid}\label{scheme groupoid quotient if quasi-section subsection}
We now prove a techniqucal lemma which will be used in the proof of the forecoming two theorems. Let $S$ be a scheme and
\[\begin{tikzcd}[column sep=12mm]
X_2\ar[r,shift left=8pt,"d_0'"]\ar[r,shift right=8pt,swap,"d_2'"]\ar[r,"d_1'"description]&X_1\ar[r,shift left=2pt,"d_0"]\ar[r,shift right=2pt,swap,"d_1"]&X_0
\end{tikzcd}\]
be a $\Sch_{/S}$-groupoid. A \textbf{quasi-section} of $X_*$ is defined to be a subscheme $U$ of $X_0$ such that
\begin{enumerate}
    \item[(a)] The restriction of $d_1$ to $d_0^{-1}(U)$ is a finite, locally free and surjective morphism from $d_0^{-1}(U)$ to $X_0$.
    \item[(b)] Any subset $E$ of $U$ formed by equivalent points for the equivalence relation defined by $X_*$ is contained in an affine open subset of $U$\footnote{If $x,y\in E$, then there exists $z\in X_1$ such that $d_1(z)=x$, $d_0(z)=y$, that is, $z\in (d_1|_{d_0^{-1}(U)})^{-1}(x)$, which is a finite set by (a). Hence $E$ is contained in the finite subset $d_0d_1^{-1}(x)\cap U$.}.
\end{enumerate}
If $U$ is a quasi-section of $X_*$, the $\Sch_{/S}$-groupoid
\[\begin{tikzcd}[column sep=12mm]
U_2\ar[r,shift left=8pt,"u_0'"]\ar[r,shift right=8pt,swap,"u_2'"]\ar[r,"u_1'"description]&U_1\ar[r,shift left=2pt,"u_0"]\ar[r,shift right=2pt,swap,"u_1"]&U
\end{tikzcd}\]
induced from $X_*$ and the inclusion $U\to X_0$ verifies the hypotheses of \cref{scheme groupoid quotient by locally free finite prop}. In fact, put $V=d_0^{-1}(U)$ and let $u,v$ be morphisms induced by $d_0$ and $d_1$:
\[\begin{tikzcd}
X_0&V\ar[r,"u"]\ar[l,swap,"v"]&U
\end{tikzcd}\]
By (\ref{category groupoid base change Y_1 square-2}), we then have a Cartesian square
\[\begin{tikzcd}
U_1\ar[r]\ar[d,swap,"u_1"]&V\ar[d,"v"]\\
U\ar[r,hook]&X_0
\end{tikzcd}\]
hence $u_1$ is surjective and finite locally free by (a). With (b), condition (a) then assures that the groupoid $U_*$ satisfies the hypotheses of \cref{scheme groupoid quotient by locally free finite prop}. In particular, $\coker(u_0,u_1)$ exists in $\Sch_{/S}$. Moreover, as $d_0$ possesses a section (the morphism $s:X_0\to X_1$), $u$ is a universally effective epimorphism by (\cite{SGA3-1}, \Rmnum{4}, 1.12); this ensures, by \cref{category groupoid cokernel under universal effective base change prop}, that $\coker(u_0,u_1)$ coincides with the cokernel $\coker(v_0,v_1)$ of the groupoid $V_*$:
\[\begin{tikzcd}[column sep=12mm]
V_2\ar[r,shift left=8pt,"v_0'"]\ar[r,shift right=8pt,swap,"v_2'"]\ar[r,"v_1'"description]&V_1\ar[r,shift left=2pt,"v_0"]\ar[r,shift right=2pt,swap,"v_1"]&V
\end{tikzcd}\]
induced by $U_*$ and the base change $u:V\to U$, which is also the inverse image of $X_*$ under the base change
\[V\hookrightarrow X_1\stackrel{d_0}{\to} X_0.\]
By \cref{category groupoid base change by d_0 and d_1}, $V_*$ is isomorphic to the groupoid $V_*'$, the inverse image of $X_*$ under the base change
\[V\hookrightarrow X_1\stackrel{d_1}{\to} X_0\]
and hence $V_*'$ admits a cokernel in $\Sch_{/S}$. Now, being flat, surjective and finite, $v:V\to X_0$ is faithfully flat and quasi-compact, hence a universally effective epimorphism by \cref{scheme topology T_i family M_i}. Therefore by \cref{category groupoid base change by d_0 and d_1}, the groupoid $X_*$ also admits a cokernel $\coker(d_0,d_1)$ in $\Sch_{/S}$. We have therefore proved the first assertion of (a) in the following lemma:

\begin{lemma}\label{scheme groupoid quasi-section lemma}
Suppose that the $\Sch_{/S}$-groupoid $X_*$ possesses a quasi-section. Then:
\begin{enumerate}
    \item[(a)] There exists a cokernel $(Y,p)$ of $(d_0,d_1)$ in $\Sch_{/S}$. Moreover, such a couple $(Y,p)$ is also a cokernel of $(d_0,d_1)$ in the category of ringed spaces.
    \item[(a')] $p$ is surjective, and is open (resp. universally closed) if $d_0$ is.
    \item[(b)] Suppose that $S$ is locally Noetherian and $X_0$ is locally of finite type (resp. of finite type) over $S$. Then $p$ and $Y\to S$ are locally of finite presentation (resp. of finite presentation).
    \item[(c)] The morphism $X_1\to X_0\times_YX_0$ with components $d_0$ and $d_1$ is surjective.
    \item[(d)] If $(d_0,d_1)$ is an equivalence couple, $X_1\to X_0\times_YX_0$ is an isomorphism. Moreover, if $d_0:X_1\to X_0$ is flat, then $p$ is faithfully flat.   
\end{enumerate}
\end{lemma}
\begin{proof}
Before proving the second assertion of (a), let us first consider (a'), (b) and (c). We have seen that $(Y,p)$ is identified with $\coker(v_0,v_1)$ and $\coker(u_0,u_1)$. Let $q$ and $r$ be the canonical epimorphisms of $U$ and $Y$ onto $Y$:
\begin{equation}\label{scheme groupoid quasi-section lemma-1}
\begin{tikzcd}
X_0\ar[rd,swap,"p"]&V\ar[l,swap,"v"]\ar[d,"r"]\ar[r,"u"]&U\ar[ld,"q"]\\
&Y&
\end{tikzcd}
\end{equation}
By hypothesis, $v$ is surjective and finite locally free, hence open. On the other hand, if $d_0:X_1\to X_0$ is open (resp. universally closed), then $u$, which is induced from $d_0$ by restriction, is also open (resp. universally closed). As, by \cref{scheme groupoid quotient by locally free finite prop}, $q$ is surjective, integral and open, we conclude that $r$ is surjective, and open (resp. universally closed) if $d_0$ is. The same assertion then holds for $p$, since $v$ is surjective. This proves (a').\par
Now suppose that $S$ is locally Noetherian and $X_0$ is locally of finite type over $S$, so that $X_0$ is locally Noetherian. Let $S'=\Spec(R)$ be an open affine subset of $S$, $Y'=\Spec(B)$ an open affine subset of $Y$ which projections into $S'$, and $U'=\Spec(A)$ be the inverse image of $Y'$ in $U$. As $R$ is Noetherian, it suffices to show that $B$ is a finite type $R$-algebra, and this follows from the fact that $R$ is Noetherian and $A$ is integral over $B$ (cf. (\ref{Artin-Tate lemma})). Finally, as $X_0\to S$ is locally of finite type, so is $p$ (\cref{scheme morphism local ft permanence prop}), hence $p$ is locally of finite presentation since $Y$ is locally Neotherian.\par
It remains to see that second assertion of (b). Suppose that $X_0$ is of finite type over $S$. Then, as $p$ is surjective, $Y$ is also quasi-compact over $S$, hence of finite type over $S$. As $S$ is locally Noetherian, $X_0\to S$ and $Y\to S$ are then finitely presented, and hence $p:X_0\to Y$ is also finitele presented (\cref{scheme morphism fp permanence prop}).\par
Finally, as the groupoid $V_*$ with base $V$ is isomorphic to the inverse image of $U_*$ by the base change $u$ and to the inverse image of $X_*$ by the base change $v$, we have Cartesian squares
\begin{equation}\label{scheme groupoid quasi-section lemma-2}
\begin{tikzcd}
X_1\ar[d,"d_0\boxtimes d_1"]&V_1\ar[l]\ar[r]\ar[d,"v_0\boxtimes v_1"]&U_1\ar[d,"u_0\boxtimes u_1"]\\
X_0\times_YX_0&V\times_YV\ar[l,swap,"v\times v"]\ar[r,"u\times u"]&U\times_YU
\end{tikzcd}
\end{equation}
As $u_0\boxtimes u_1$ is surjective, so is $v_0\boxtimes v_1$. Since $v\times v$ is surjective, the composition $V_1\to X_0\times_YX_0$ is also surjective, whence so is $d_0\boxtimes d_1$.\par
We now turn to the proof of the second assertion of (a). To see that $(Y,p)$ is a cokernel of $(d_0,d_1)$ in $\Rsp$, we first show that $Y$ is obtained from $X_0$ by identifying points $x,y$ such that there exists $z\in X_1$ with $d_0(z)=x$ and $d_1(z)=y$. In fact, $p$ is surjective and we have $pd_0=pd_1$; conversely, if $p(x)=p(y)$, there exists a point $z'$ of $X_0\times_YX_0$ whose first projection is $x$ and second projection is $y$. If $z$ is a point of $X_1$ such that $(d_0\boxtimes d_1)(z)=z'$, then $d_0(z)=x$ and $d_1(z)=y$, whence our assertion. Also, if $W$ is a saturated open subset of $X_0$, $W\cap U$ is saturated open in $U$, so by \cref{scheme groupoid quotient by locally free finite prop}, $q(W\cap U)$ is open in $Y$. As $q(W\cap U)=p(W)$, we see that $Y$ is endowed with the quotient topology of that of $X_0$.\par
It remains to show that the canonical sequence 
\[\begin{tikzcd}
\mathscr{O}_Y\ar[r]&p_*(\mathscr{O}_{X_0})\ar[r,shift left=2pt]\ar[r,shift right=2pt,swap]&p_*((d_0)_*(\mathscr{O}_{X_1}))=p_*((d_1)_*(\mathscr{O}_{X_1}))
\end{tikzcd}\]
is exact. Let $Y'$ be an open subset of $Y$ and put $U'=q^{-1}(Y')$, $X_0'=p^{-1}(Y')$, etc. Then, $U'$ is a saturated open subset of $U$ for the equivalence relation defined by the groupoid $U_*$, and it follows from \cref{ringed space groupoid restriction to saturated prop} and \cref{ringed space groupoid cokernel in scheme if} that $Y'$ is its cokernel, in $\Sch_{/S}$ and in $\Rsp$. Similarly, $X_0'$ is saturated open in $X_0$ for the equivalence relation defined by $X_*$ and we have the following Cartesian squares
\[\begin{tikzcd}
X_0'\ar[d]&V'=d_0^{-1}(U')\ar[l,swap,"\tilde{d}_1"]\ar[r,"\tilde{d}_0"]\ar[d]&U'\ar[d]\\
X_0&V=d_0^{-1}(U)\ar[l,swap,"d_1"]\ar[r,"d_0"]&U'
\end{tikzcd}\]
Hence $\tilde{d}_1$ is surjective and finite locally free. On the other hand, let $x\in U'$. As $U$ is a quasi-section, the set $E=d_0d_1^{-1}(x)\cap U$ is finite and contained in an affine open $W$ of $U$. The intersection $E'=E\cap U'$ is then a finite subset contained in a quasi-affine open subset $W\cap U'$. Therefore, there exists an affine open $W'$ of $W\cap V$ containing $E'$, so $U'$ is a quasi-section of the groupoid $X_*'$ induced by $X_*$ over $X_0'$. The first assertion of (a), applied to $X_*'$ and $U'$, shows that $Y'$ is the cokernel of $X_*'$ in $\Sch_{/S}$. In particular, for any $S$-scheme $T$, we have an exact sequence
\[\begin{tikzcd}
T(Y')\ar[r,"T(p|_{X_0'})"]&T(X_0')\ar[r,shift left=2pt,"T(d_1|_{X_1'})"]\ar[r,shift right=2pt,swap,"T(d_0|_{X_1'})"]&T(X_1')
\end{tikzcd}\] 
Now if $T=\G_{a,S}$, this sequence is identified with
\[\begin{tikzcd}
\Gamma(Y',\mathscr{O}_Y)\ar[r]&\Gamma(p^{-1}(Y'),\mathscr{O}_{X_0})\ar[r,shift left=2pt,"\delta_1"]\ar[r,shift right=2pt,swap,"\delta_0"]&\Gamma(d_0^{-1}p^{-1}(Y'),\mathscr{O}_{X_1})=\Gamma(d_1^{-1}p^{-1}(Y'),\mathscr{O}_{X_1})
\end{tikzcd}\]
which is then exact for any open subset $Y'$ of $Y$. This conclude the proof of (a).\par
Finally, if $(d_0,d_1)$ is an equivalence couple, so is $(u_0,u_1)$ and the morphism $u_0\boxtimes u_1:U_1\to U\times_YU$ is an isomorphism (\cref{scheme groupoid quotient by locally free finite prop}). As $v\times v$ is faithfully flat and quasi-compact, we conclude from (\ref{scheme groupoid quasi-section lemma-2}) that $d_0\boxtimes d_1$ is an isomorphism (\cite{SGA1}, \Rmnum{8}, 5.4). Moreover, if $d_0$ is flat, $u$ is also flat. Now $q$ is flat by \cref{scheme groupoid quotient by locally free finite prop}, so $r$ is also flat from the diagram (\ref{scheme groupoid quasi-section lemma-1}). As $v$ is faithfully flat, $p$ is then flat, and hence faithfully flat since it is surjective.
\end{proof}

\subsection{Quotient for a flat proper groupoid}

\subsection{Quotient by a group scheme}
We now consider the action of a group scheme $G$ over $S$ on an $S$-scheme $X$, and use the preceding results to construct the quotient $G\backslash X$. We first recall the following result:

\begin{theorem}\label{scheme morphism local fp factorization by quotient iff}
Let $S$ be a scheme and $f:X\to Y$ be an $S$-morphism. Suppose that one of the following conditions is satisfied:
\begin{enumerate}
    \item[($\alpha$)] The morphism $f$ is locally of finite presentation.
    \item[($\beta$)] The scheme $S$ is locally Noetherian and $X$ is locally of finite type over $S$. 
\end{enumerate} Then the following conditions are equivalent:
\begin{enumerate}
    \item[(\rmnum{1})] There exists an $S$-scheme $X'$ and a factorization of $f$:
    \[f:X\stackrel{f'}{\to} X'\stackrel{i}{\to}Y\]
    where $f'$ is a faithfully flat $S$-morphism locally of finite presentation and $i$ is a monomorphism.
    \item[(\rmnum{2})] The first projection $\pr_1:X\times_YX\to X$ is a flat morphism.
\end{enumerate}
Moreover, under these conditions, $(X',f')$ is a quotient of $X$ by the equivalence relation induced by $f$ (for the fppf topology), so that the factorization $f=i\circ f'$ is unique up to isomorphisms.
\end{theorem}
\begin{proof}
The implication (\rmnum{1})$\Rightarrow$(\rmnum{2}) is trivial. In fact, the first projection $\pr_1':X\times_{X'}X\to X$ factors through $X\times_YX$:
\[\pr_1':X\times_{X'}X\stackrel{u}{\to} X\times_YX\stackrel{\pr_1}{\to} X.\]
The morphism $u$ is an isomorphism, since $i$ is a monomorphism, and $\pr_1'$ is flat since $f'$ is flat, hence $\pr_1$ is flat.\par
To prove that (\rmnum{2})$\Rightarrow$(\rmnum{1}), we first note that the assertions of \cref{scheme morphism local fp factorization by quotient iff} are local on $Y$ (hence are local on $S$); they are also local on $X$, as it easily follows from the fact that a flat morphism locally of finite presentation is open (\cite{EGA4-3}, 11.3.1).\par
The case where $Y$ is locally Noetherian and $X$ is of finite type over $Y$ is treated in (\cite{Murre1964}, cor.2 du th.2). We now see how to reduce ourselves to this case. Under the hypothesis ($\alpha$), we can therefore assume that $X,Y$ are affine and $f$ is finitely presented. By replacing $S$ with $Y$, we may also assume that $X$ and $Y$ are finitely presented over $S$. We then reduce to the case where $S$ is Noetherian thanks to (\cite{EGA4-3}, 11.2.6).\par
Under the hypothesis ($\beta$), we can suppose that $X,Y,S$ are affine, $S$ is Noetherian and $X$ is of finite type over $S$. Consider $Y$ as a filtered projective limit of affine schemes $Y_i$ which are of finite type over $S$. The schemes $X\times_{Y_i}X$ then form a descreasing filtered system of closed subschemes of $X\times_SX$, whose projective limit is $X\times_YX$. As $X\times_SX$ is Noetherian, we have $X\times_{Y_i}X=X\times_YX$ for $i$ large enough, so that the composition
\[f_i:X\stackrel{f}{\to} Y\to Y_i\]
satisfies the hypothesis of (\rmnum{2}) if so does for $f$. As the equivalence relation defined by $f$ over $X$ coincides with that by $f_i$, it is clear that it suffices to prove (\rmnum{2})$\Rightarrow$(\rmnum{1}) for $f_i$, which means we can reduce to the case where $Y$ is of finite type over $S$. 
\end{proof}

Now let $S$ be a scheme, $G$ be a group scheme over $S$ which is locally of finite presentation over $S$, and $X$ be an $S$-scheme acted by $G$. If $X\to S$ possesses a section $\xi$, we recall that the stabilizer $\Stab_G(\xi)$ is represented by a group subscheme of $G$ (in fact, by the group subscheme $G\times_XS$, cf. \ref{category group action in PSh paragraph}).

\begin{theorem}\label{scheme group action by local fp quotient by stabilizer exist}
Let $S$ be a scheme and $G$ be a $S$-group scheme locally of finite presentation over $S$, which acts on an $S$-scheme $X$. Suppose that $X\to S$ possesses a section $\xi$ such that the stabilizer $H$ of $\xi$ in $G$ is flat over $S$. If one of the following conditions is satisfied:
\begin{enumerate}
    \item[(a)] $X$ is locally of finite type over $S$;
    \item[(b)] $S$ is locally Noetherian,
\end{enumerate}
then the fppf quotient sheaf $G/H$ is representable by an $S$-scheme which is locally of finite presentation over $S$, and the $S$-morphism induced by $\xi$:
\[f:G=G\times_SS\to G\times_SX\to X,\quad g\mapsto g\cdot\xi\]
factors into
\[\begin{tikzcd}
G\ar[rd,swap,"p"]\ar[rr,"f"]&&X\\
&G/H\ar[ru,hook,swap,"j"]&
\end{tikzcd}\]
where $p$ is the canonical projection, which is a faithfully flat morphism locally of finite presentation, and $j$ is a monomorphism.
\end{theorem}
\begin{proof}
The morphism $f$ makes $G$ an $X$-scheme, and the definition of the stabilizer of $\xi$, the morphism
\[G\times_SH\to G\times_XG,\quad (g,h)\mapsto (g,gh)\]
is an isomorphism. As $H$ is flat over $S$, $G\times_SH$ is flat over $G$, hence the first projection $\pr_1:G\times_SG\to G$ is flat. Therefore, if $X$ is locally of finite type over $S$, then $f$ is locally of finite presentation (\cref{scheme morphism local fp permanence prop}), and otherwise $S$ is supposed to be Noetherian. It then suffices to apply \cref{scheme morphism local fp factorization by quotient iff} to the morphism $f$. Also, it follows from (\cite{EGA4-4}, 17.7.5) that $G/H$ is locally of finite presentation over $S$.
\end{proof}

\begin{corollary}\label{scheme group morphism factorization by quotient if flat}
Let $S$ be a scheme and $u:G\to H$ be a morphism of $S$-group schemes. Suppose that $G$ is locally of finite presentation over $S$ and that, either $H$ is locally of finite type over $S$, or $S$ is locally Noetherian. Then, if $K=\ker u$ is flat over $S$, the quotient group $G/K$ is representable by an $S$-group scheme which is of finite presentation over $S$, and $u$ factors into
\[\begin{tikzcd}
G\ar[rd,swap,"p"]\ar[rr,"u"]&&H\\
&G/K\ar[ru,hook,swap,"j"]&
\end{tikzcd}\]
where $p$ is the canonical projection which is faithfully flat and locally of finite presentation, and $j$ is a monomorphism.
\end{corollary}
\begin{proof}
We can apply \cref{scheme group action by local fp quotient by stabilizer exist} to $X=H$ and the unit section of $H$.
\end{proof}

\section{Generalities on algebraic groups}
In this section, we denote by $A$ a local Artinian ring with residue field $k$. A group scheme $G$ over $\Spec(A)$ is called simply an $A$-group. This $A$-group is called \textbf{locally of finite type} if the underlying scheme is locally of finite type over $A$, and it is called \textbf{algebraic} if the underlying scheme is of finite type over $A$.

\subsection{Some preliminary remarks}\label{scheme algebraic group preliminary remark subsection}
Consider a group scheme $G$ over an arbitrary scheme $S$. The structural morphism $\mu:G\times_SG\to G$ is called the multiplication morphism of $G$ and the inversion morphism $c:G\to G$ is defined by $c(T)(x)=x^{-1}$ ($T$ being a scheme over $S$ and $x\in G(T)$). If $U$ and $V$ are subsets of $G$, we denote by $U\cdot V$ the image under the multiplication morphism of the subset of $G\times_SG$ formed by points whose first projection belongs to $U$ and the second projection belongs to $V$. The notations $U^{-1}$ or $c(U)$ are defined similarly.\par
Let $\pr_1:G\times_SG\to G$ be the first projection and $\sigma:G\times_SG\to G\times_SG$ be the morphism with components $\pr_1$ and $\mu$. For any $S$-scheme $T$, $\sigma(T)$ is the map $(x,y)\mapsto(x,xy)$; so $\sigma$ is an automorphism. The composition of this automorphism and the projection $\pr_2$ is the multiplication $\mu$. Therefore, if $G$ is flat (resp. smooth, etc.) over $S$, then $\pr_2$, and hence $\mu$, are flat (resp. smooth, etc.).\par

We now suppose that $S$ is the spectrum of a local Artinian ring $A$ with residue field $k$. We denote by $(\Sch_{/k})_{\red}$ the category of reduced schemes over $k$. For any scheme $X$ over $A$, the reduced scheme $X_{\red}$ is then an object of $(\Sch_{/k})_{\red}$, and the functor $X\mapsto X_{\red}$ is right adjoint to the inclusion of $(\Sch_{/k})_{\red}$ to $\Sch_{/A}$. This ensures that, for any $A$-group $G$, $G_{\red}$ is a group in the category $(\Sch_{/k})_{\red}$, that is, for any reduced $k$-scheme $T$, $G_{\red}(T)$ is endowed with a group structure, functorial on $T$. We note that $G_\red$ is not necessarily a $k$-group, because the multiplication is only a morphism $(G_\red\times_k G_\red)_\red\to G_\red$.\par 
Nevertheless, if $k$ is a \textit{perfect} field, the inclusion $(\Sch_{/k})_\red$ into $\Sch_{/k}$ commutas with products (in this case a product of reduced $k$-schemes is still reduced) so that the groups in $(\Sch_{/k})_\red$ are identified with $k$-groups whose underlying scheme is reduced. In this case, if $G$ is a $k$-group, $G_\red$ is a subgroup scheme of $G$; but this subgroup is in genral not normal in $G$.\par
For example, if $k$ is a field with characteristic $3$, the constant group $(\Z/2\Z)_k$ acts nontrivially on the diagonalisable $D_k(\Z/3\Z)$. If $G$ denotes the semi-direct prodcut of $D_k(\Z/3\Z)$ by $(\Z/2\Z)_k$ defined by this action, then $G_\red$ is identified with $(\Z/2\Z)_k$ and is not normal in $G$.\par
Let $k$ be an arbitrary field, $k^{\perf}=k^{p^{-\infty}}$ bethe perfect closure of $k$, and $H$ be a group in the category $(\Sch_{/k})_\red$. Then $(H\otimes_kk^{\perf})_\red$ is a group scheme over $k^\perf$. As $H\otimes_kk^\perf$ and $(H\otimes_kk^{\perf})$ have the same underlying topological space, we see that the groups in $(\Sch_{/k})_\red$ have in common with the $k$-groups certain topological properties invariant by extension of the base field: for example, any group of $(\Sch_{/k})_\red$ is separated.\par

An $A$-group $G$ is always \textit{separated}, becasue the unit section $e:\Spec(A)\to G$ is a closed immersion. In fact, let $x$ be the unique point of $\Spec(A)$ and $\eta$ be the structural morphism $G\to\Spec(A)$. As $\eta\circ e=\id_{\Spec(A)}$, for any affine open $U=\Spec(B)$ of $G$ containing $e(x)$, the morphism $B\to A$ possesses a section, hence is surjective. It follows that $e$ is a closed immersion (this argument is valid for any zero dimension local ring $A$, not necessarily Artinian). Now, the diagonal $G\times_AG$ is identified with the inverse image of the unit section:
\[\begin{tikzcd}
G\ar[r]\ar[d,swap,"\Delta_G"]&\Spec(A)\ar[d,"e"]\\
G\times_SG\ar[r,"\varphi"]&G
\end{tikzcd}\]
where $\varphi:G\times_SG$ is defined by $(x,y)\mapsto xy^{-1}$.\par

Let $G$ be an $A$-scheme. We say that a point $g$ of $G$ is \textbf{strictly rational} over $A$ if there exists an $A$-morphism $s:\Spec(A)\to G$ which sends the unique point of $\Spec(A)$ to $g$, i.e. if the morphism $A\to\mathscr{O}_{G,g}$ admits a retraction. We note that in this case $\kappa(g)=k$, and hence $hg$ is a clsed point of $G$.\par
Suppose that $G$ is an $A$-group, then such a morphism $s:\Spec(A)\to G$ defines an automorphism $r_s$ of the scheme $G$ over $A$, which is called the \textbf{right translation} by $s$: for any morphism $\pi:S\to\Spec(A)$, $r_s(\pi)$ is the automorphism of $G(S)$ defined by $x\mapsto x\cdot G(\pi)(s)$, for any $x\in G(S)$. Similarly, we denote by $\ell_s$ the left translation defined by $s$, which is the automorphism of $G$ defined by $\ell_s(\pi)(x)=G(\pi)(s)\cdot x$, for any $x\in G(S)$.\par
As $G\otimes_Ak$ and $G$ have the same underlying topological space $|G|$, that $G\otimes_Ak$ is a $k$-group and that $s\otimes_Ak$ depends only on $g$ and not on $s$, we see that the automorphism of $|G|$ induced by $r_s$ and $\ell_s$ (or by $r_{s\otimes k}$ and $\ell_{s\otimes k}$) only depends on the point $g$ and not on $s$. If $P$ is a subset of $|G|$, we then denote by $r_g(P)$ or $P\cdot g$ (resp. $\ell_g(P)$ or $g\cdot P$) the subset $r_s(P)$ (Resp. $\ell_s(P)$).

\begin{remark}\label{scheme A-group strict rational stable under base change remark}
If $g$ is a strictly rational point of $G$ and if $A\to A'$ is a morphism of local Artinian rings, then $G'=G\otimes_AA'$ possesses a unique point $g'$ over $g$, and $g'$ is strictly rational over $A'$. Moreover, if we denote by $P'$ the inverse image of $P$ in $G'$, then $P'\cdot g'$ is the inverse image of $P\cdot g$ (cf. \cref{scheme inverse image under base change prop}).
\end{remark}

\begin{proposition}\label{scheme A-group product of open dense is G}
Let $G$ be an $A$-group and $U,V$ be open dense subsets in $G$. Then $U\cdot V$ (i.e. the image of $U\times_AV$ under the multiplication morphism) is equal to $G$.
\end{proposition}
\begin{proof}
In fact, as $G$ and $G\otimes k$ have the same underlying space, we may suppose, by replacing $A$ by $k$ and $G$ by $G\otimes k$, that $A=k$. Let $g\in G$ and put $K=\kappa(g)$, then the left translation $\ell_g$ is an automorphism of $G_K$. As the projection $G_K\to G$ is open (cf. \cite{EGA4-2}, 2.4.10), $U_K$ and $V_K$ are two open dense subsets of $G_K$, and so is the image of $V_K$ by $\ell_g$. There then exists $v\in V_K$ such that $u=\ell_g(v)$ belongs to $U_K$. Let $L$ be an extension of $K$ containing $\kappa(v)$, and hence $\kappa(u)$, and $g_L$, $v_L$ be the $L$-points of $G_L$ defined by $g$ and $v$. Then $g_L\cdot v_L=u'$ is a point of $G_L$ lying over $u$, and hence $g_L=u'\cdot v_L^{-1}$ is lying over $U\cdot V$, whence $g\in U\cdot V$. This proves the proposition.
\end{proof}

\begin{corollary}\label{scheme A-group irreducible is quasi-compact}
If $G$ is an irreducible $A$-group, then $G$ is quasi-compact.
\end{corollary}
\begin{proof}
If $U$ is a nonempty affine open subset of $G$, then $U$ is dense in $G$, so by \cref{scheme A-group product of open dense is G} the morphism $\mu:U\times_AU\to G$ is surjective, hence $G$ is quasi-compact (since $U\times_AU$ is).
\end{proof}

\begin{corollary}\label{scheme A-group subgroup is closed}
Let $G$ be an $A$-group and $H$ be a sub-$A$-group of $G$. Then $H$ is closed in $G$.
\end{corollary}
\begin{proof}
Let $k'$ be the perfert closure of $k$. As the underlying spaces of $G$ and $H$ are unchanged by base changing to $k'$, we may suppose that $A=k$ is a perfect field. We can also suppose that $G$ and $H$ are reduced, hence geometrically reduced.\par
Let $\widebar{H}$ be the closure of $H$, then $\mu^{-1}(\widebar{H})$ is a closed subset of $G\times G$ containing $H\times H$. As the morphism $H\to\Spec(k)$ (resp. $\widebar{H}\to\Spec(k)$) is universally open, and as $H$ is dense in $\widebar{H}$, we see that $H\times H$ is dense in $H\times\widebar{H}$ and $H\times\widebar{H}$ is dense in $\widebar{H}\times\widebar{H}$, hence $H\times H$ is dense in $\widebar{H}\times\widebar{H}$. We conclude that $\mu(\widebar{H}\times\widebar{H})\sub\widebar{H}$, and as $\widebar{H}\times\widebar{H}$ is reduced, $\mu$ induces a morphism $\mu':\widebar{H}\times\widebar{H}\to\widebar{H}$.\par
Let $g\in\widebar{H}$ and put $K=\kappa(g)$. As the projection $\widebar{H}_K\to\widebar{K}$ is open, $H_K$ and $\ell_g(H_K)$ are open dense subsets of $\widebar{H}_K$, so there exists $u,v\in H_K$ such that $\ell_g(v)=u$. We then conclude, as the proof of \cref{scheme A-group product of open dense is G}, that $g$ belongs to $H\cdot H=H$, so $\widebar{H}=H$.
\end{proof}

\subsection{Local properties for algebraic groups}
Without further specifications, we now suppose that $G$ is an $A$-group locally of finite type.

\begin{proposition}\label{scheme alg group flat local ring CM}
Let $x$ be a point of a $A$-group $G$ locally of finite type and flat over $A$. Then the local ring $\mathscr{O}_{G,x}$ is Cohen-Macaulay and there exists a regular sequence $a_1,\dots,a_n$ of $\mathscr{O}_{G,x}$ such that $\mathscr{O}_{G,x}/(a_1,\dots,a_n)$ is a finitely generated and flat $A$-module (hence free).
\end{proposition}
\begin{proof}
We first suppose that $A=k$ is a field. It then suffices to prove that $\mathscr{O}_{G,x}$ is Cohen-Macaulay and we can limit ourselves to the case where $x$ is a closed point (cf. \cite{EGA4-1}, $0_{\Rmnum{4}}$, 16.5.13). By \cref{scheme local algebraic over Artinian CM closed point} below, $G$ contains a closed point $y$ such that $\mathscr{O}_{G,y}$ is Cohen-Macaulay. By (\cite{EGA4-2}, 6.7.1), for any finite type extension $K$ of $k$ and any point $\bar{y}$ of $\widebar{G}=G\otimes_kK$ over $y$, $\mathscr{O}_{\widebar{G},\bar{y}}$ is then Cohen-Macaulay. If the extension $K$ is chosen large enough, i.e. if $K$ contains a normal extension of $k$ containing the residue fields $\kappa(x)$ and $\kappa(y)$, then $\bar{y}$ is strictly rational over $K$ and thus so is any point $\bar{x}$ of $\widebar{G}$ lying over $x$\footnote{In fact, the hypothesis on $K$ implies that, for any extension $L$ of $K$, any $k$-morphism $\kappa(x)\to L$ (resp. $\kappa(y)\to L$) factors through $K$. Therefore, any point of $G\otimes_kK$ lying over $x$ or $y$ has residue field $K$, and hence is (strictly) rational over $K$.}. As the automorphism $r_{\bar{x}}\circ r_{\bar{y}}^{-1}$ sends $\bar{y}$ to $\bar{x}$, we conclude that $\mathscr{O}_{\widebar{G},\bar{x}}$, and hence $\mathscr{O}_{G,x}$, is Cohen-Macaulay (\cite{EGA4-2}, 6.7.1).
\end{proof}

\begin{lemma}\label{scheme local algebraic over Artinian CM closed point}
Any nonempty scheme $X$, locally of finite type over an Artinian ring $A$, contains a closed point $x$ whose local ring is Cohen-Macaulay. 
\end{lemma}
\begin{proof}
We can evidently assume that $X$ is affine with ring $B$ and prove by induction on $\dim(X)$ (the assertion is trivial if $X$ is discrete, since then any local ring is Artinian, hence Cohen-Macaulay, cf. \cref{CM module if length finite}). As $B$ is of finite type over $A$, if $\dim(B)>0$, $B$ contains a non-invertible element $a$ which is not a zero-divisor\footnote{In fact, $B$ is a Noetherian Jacobson ring. If any non-invertible element of $B$ is a zero divisor, then by \cref{associated prime and homothety injective}, any prime ideal is an associated prime of $B$. In particular, $B$ has only finitely many maximal ideals $\m_1,\dots,\m_r$ (cf. \cref{associated prime ideal finite if Noe finite}). As $B$ is Jacobson, the intersection of $\m_i$ is the nilradical of $B$, and it follows from \cref{prime ideal contain intersection} that each $\m_i$ is a minimal prime ideal of $B$, so $\dim(B)=0$.}. The closed subscheme $X'=\Spec(B/(a))$ of $X$ then has dimension strictly smaller than $\dim(X)$, and hence by induction hypothesis has a closed point $x$ such that $\mathscr{O}_{X',x}$ is Cohen-Macaulay. As $\mathscr{O}_{X',x}=\mathscr{O}_{X,x}/(a)$ and $a$ is non-invertible and not a zero-divisor in $\mathscr{O}_{X,x}$, we conclude that $\mathscr{O}_{X,x}$ is Cohen-Macaulay (\cref{CM over local Noe iff quotient by regular is CM}).
\end{proof}

\begin{proposition}\label{scheme alg group flat closed point locally rational}
Let $A$ be a local Artinian ring, $G$ be an $A$-group locally of finite type and flat over $A$, and $x$ be a closed point of $G$. Then there exists a local $A$-algebra $A'$, finite and free over $A$, such that any point $x'$ of $G\otimes_AA'$ lying over $x$ is strictly rational over $A'$.
\end{proposition}
\begin{proof}
Let $k_1$ be a normal extension of finite type of $k$ containing the residue field $\kappa(x)$ of $x$. By (\cite{EGA3} $0_{\Rmnum{3}}$, 10.3.1), there exists a local $A$-algebra $A_1$ which is finite and free over $A_1$ with residue field $k_1$. In this case, the points $g_1,\dots,g_n$ of $G\otimes_AA_1$ lying over $x\in G$ has residue field $k_1$, so they are rational over $A_1$. Let $B_1,\dots,B_n$ be the local rings of $g_1,\dots,g_n$. By \cref{scheme alg group flat local ring CM}, $B_1,\dots,B_n$ possess quotients $B_1',\dots,B_n'$ which are Artinian and finite and free over $A_1$. Put $A'=B_1'\otimes_{A_1}\cdots\otimes_{A_1}B_n'$, then $A'$ is local, finite and free over $A_1$, and for each $i=1,\dots,n$, we have a surjective homomorphism
\[B_i\otimes_{A_1}A'\twoheadrightarrow B_i'\otimes_{A_1}A'\twoheadrightarrow A',\]
where the second one is induced by the multiplication map $B_i'\otimes_{A_1}B_i'\twoheadrightarrow B_i'$. Therefore, $A'$ satisfies our requirements.
\end{proof}

Now let $e:\Spec(A)\to G$ be the unit element of $G$, which is the image of the unit section $\Spec(A)\to G$. By definition, $e$ is strictly rational over $A$.

\begin{proposition}\label{scheme alg group flat smooth iff at unit}
Let $G$ be a group locally of finite type and flat over a local Artinian ring $A$ and $k^{\perf}$ (resp. $\bar{k}$) be the perfect closure (resp. algebraic closure) of the residue field $k$ of $A$.
\begin{enumerate}
    \item[(a)] For any closed point $x$ of $\widebar{G}=G\otimes_k\bar{k}$, the local ring $\mathscr{O}_{\widebar{G},e}$ and $\mathscr{O}_{\widebar{G},x}$ are isomorphic. In particular, the tangent spaces $T_e\widebar{G}$ and $T_x\widebar{G}$ are isomorphic.
    \item[(b)] The following assertions are equivalent:
    \begin{enumerate}
        \item[(\rmnum{1})] $G\otimes_kk^\perf$ is reduced.
        \item[(\rmnum{2})] $\mathscr{O}_{G,e}\otimes_Ak^{\perf}$ is reduced.
        \item[(\rmnum{3})] $G$ is smooth over $A$.
        \item[(\rmnum{4})] $G$ is smooth over $A$ at $e$.   
    \end{enumerate} 
\end{enumerate}
\end{proposition}
\begin{proof}
Let $x$ be a closed point of $\widebar{G}$, so that there exists a unique $\bar{k}$-morphism $s:\Spec(\bar{k})\to\widebar{G}$ with image $x$. The right translation $r_s$ then induces an isomorphism from $\mathscr{O}_{\widebar{G},e}=\mathscr{O}_{G,e}\otimes_k\bar{k}$ to $\mathscr{O}_{\widebar{G},x}$, whence the assertion of (a).\par
To prove assertion (b), in view of \cref{scheme morphism local fp smooth at point iff}, we may assume that $A=k$ is a field. The implications (\rmnum{1})$\Rightarrow$(\rmnum{2}), (\rmnum{3})$\Rightarrow$(\rmnum{4}), (\rmnum{3})$\Rightarrow$(\rmnum{1}) and (\rmnum{4})$\Rightarrow$(\rmnum{2}) are clear, so it suffices to show that (\rmnum{2})$\Rightarrow$(\rmnum{3}). In this case, $\mathscr{O}_{G,e}\otimes_k\bar{k}$ is then reduced, so by (a), $\mathscr{O}_{\widebar{G},x}$ is reduced for any closed point $x$ of $\widebar{G}$, so that $\widebar{G}$ is reduced. As $\widebar{G}$ is locally of finite type over $k$, there then exists a closed point $y$ such that $\mathscr{O}_{\widebar{G},y}$ is regular, and by (a), this implies that $\widebar{G}$ is regular at any closed point, hence smooth over $\widebar{k}$. It then follows from (\cite{EGA4-4}, 17.7.1) that $G$ is smooth over $k$.
\end{proof}

We can now give the examples below, indicated by M. Raynaud, of group schemes $G$ over a non-perfect field $k$, such that $G_{\red}$ is not a $k$-group.

\begin{example}\label{scheme alg group G_red not group example}
Let $k$ be a non-perfect field with characteristic $p>0$, $t\in k-k^p$, $\bar{k}$ be an algebraic closure of $k$, and $\alpha\in\bar{k}$ such that $\alpha^p=t$.
\begin{enumerate}
    \item[(a)] Consider the additive group $\G_{a,k}=\Spec(k[X])$ and let $G$ be the subscheme, finite over $k$, defined by the additive polynomial $X^{p^2}-tX^p$. Then we have
    \[G_{\red}=\Spec(k[X]/(X(X^{p(p-1)}-t)))\]
    which is \'etale at the origin. If this was a group scheme over $k$, it would be smooth over $k$ by \cref{scheme alg group flat smooth iff at unit}; but $G_\red$ is not geometrically reduced, so this is not true.
    \item[(b)] Consider $\G_{a,k}^4=\Spec(k[X,Y,U,V])$ and let $G$ be the subgroup scheme defined by the ideal $\mathfrak{I}$ generated by the additive polynomials $P=X^p-tY^p$, $Q=U^p-tV^p$. Then $G$ is of dimension $2$ and is irreducible, because $(G_{\bar{k}})_\red\cong\Spec(\bar{k}[Y,V])$ is irreducible.\par
    Let $A=k[X,Y,U,V]$ and $\m$ be the augmentation ideal (that is, the ideal generated by $X,Y,U,V$). Denote by $x,y,u,v$ the images of $dX,dY,dU,dV$ in $\Omega_{A/k}^1\otimes_A(A/\m)$, considered as linear forms over the tangent space $k^4=T_0\G_{a,k}^4$. We now prove that the subspace $E=T_0G_\red$ is equal to $k^4$. Otherwise, there should exist a linear form $f=ax+by+cu+dv$, with $a,b,c,d\in k$ not all zero, which vanishes on $E$. Since the formation of $\Omega_{A/k}^1$ (and hence that of the tangent space) commutes with base change, we can identify $f$ with its image in $(\bar{k}^4)^\times$. As $(G_{\bar{k}})_\red\sub(G_\red)_{\bar{k}}$, $f$ then vanishes over the subspace $T_0(G_{\bar{k}})_\red$ of $\bar{k}^4$, which is defined by the equations $g_1=x-\alpha y$ and $g_2=u-\alpha v$, and hence $f=\lambda g_1+\mu g_2$, with $\lambda,\mu\in\bar{k}$. Now $\lambda g_1+\mu g_2$ does not belong to $k^4$ unless $\lambda=\mu=0$, and this contradiction implies $E=k^4$, whence $T_0(G_\red)_{\bar{k}}=\bar{k}^4$.\par
    On the other hand, $R=XV-YU$ belongs to $\sqrt{\mathfrak{I}}$ because $R^p=(X^p-tY^p)V^p-Y^p(U^p-tV^p)$, so the tangent space $F$ at point $(\alpha,1,\alpha,1)$ of $(G_\red)_{\bar{k}}$ is contained in the hyperplane $H$ of $\bar{k}^4$ defined by the equation $\alpha dV+dX-dU-\alpha dY=0$, hence is of dimension $\leq 3$\footnote{In fact, $\sqrt{\mathfrak{I}}$ is generated by $P,Q,R$, so the tangent space of $(G_\red)_{\bar{k}}$ at a point $(x_0,y_0,u_0,v_0)$ is given by the hyperplane defined by the equation $v_0dX-u_0dY+x_0dV-y_0dU=0$.}. By \cref{scheme alg group flat smooth iff at unit}~(a), $G_\red$ is not a $k$-group scheme.
\end{enumerate}
\end{example}

In fact, any $k$-group locally of finite type over a field $k$ of characteristic zero is smooth. To see this, we need the following useful criterion for the smoothness of a general group scheme (we recall that a scheme $S$ is called of \textbf{characteristic zero} if for any $s\in S$, the residue field $\kappa(s)$ has characteristic zero):

\begin{proposition}\label{scheme group over char 0 smooth at unit section iff omega}
Let $S$ be a scheme of characteristic zero and $G$ be an $S$-group scheme locally of finite presentation over $S$ at the unit section $e(S)$\footnote{By this, we mean that $G$ is locally of finite presentation over $S$ at any point of $e(S)$.}. For $G$ to be smooth at the unit section $e(S)$, it is necessary and sufficient that the $\mathscr{O}_S$-module $\omega_{G/S}=e^*(\Omega_{G/S}^1)$ (called the \textbf{conormal module} of the unit section of $G$) is locally free.
\end{proposition}

\begin{proof}
We first recall that, if $\pi:G\to S$ is the structural morphism, we have $\Omega_{G/S}^1=\pi^*(\omega_{G/S})$ (cf. \cref{scheme group omega_G/S differential module prop}), so the $\mathscr{O}_S$-module $\omega_{G/S}$ is locally free if and only if the $\mathscr{O}_G$-module $\Omega_{G/S}^1$ is locally free, which is in turn the case if $G$ is smooth over $S$.\par
Conversely, if $\omega_{G/S}$ is locally free, then so is $\Omega_{G/S}^1$, and as $S$ has characteristic zero, it follows from (\cite{EGA4-4}, 16.12.2) that $G$ is differentially smooth over $S$. It then follows from (\cite{EGA4-4}, 17.12.5) that $G$ is smooth at the unit section.
\end{proof}

\begin{corollary}[\textbf{Cartier}]\label{scheme alg group smooth over char 0}
Let $k$ be a field of characteristic zero. Then any $k$-group locally of finite type is smooth over $k$.
\end{corollary}
\begin{proof}
In fact, in this case $\omega_{G/S}$ is always locally free, so by \cref{scheme group over char 0 smooth at unit section iff omega}, $G$ is smooth at the identity, hence smooth by \cref{scheme alg group flat smooth iff at unit}.
\end{proof}

\subsection{Connected components}
Consider first an $A$-group $G$ and let $G'$ be the connected component of the identity $e$ of $G$. This connected component is clearly closed, so that we can identify it with the reduced closed subscheme of $G$ which has $G'$ as underlying space.

\begin{proposition}\label{scheme A-group connected component is geometric}
For any field extension $K$ of $k$, the underlying space of $G'\otimes_AK$ is the connected component of the identity of the $K$-group $G\otimes_AK$ (i.e. $G'$ is geometrically connected). 
\end{proposition}
\begin{proof}
Let $(G\otimes_AK)'$ be the connected component of the identity in $G\otimes_AK$. As the image of $(G\otimes_AK)'$ in $G$ is connected and contains the identity element, it is contained in $G'$, so $(G\otimes_AK)'$ is contained in the inverse image $G'\otimes_AK$ of $G'$ in $G\otimes_AK$. The proposition then follows from the connectedness of $G'\otimes_AK$, which results from (\cite{EGA4-2}, 4.5.8 and 4.5.14).
\end{proof}

It is clear that $G'$ is a reduced $k$-scheme, and (\cite{EGA4-2} 4.5.8 and 4.5.14) shows that $G'\times_kG'$ is connected, so that $(G'\times_kG')_\red$ is the reduced subscheme of $G\times_AG$ whose underlying space is the connected component of the identity. In particular, the multiplication morphism $\mu:G\times_AG\to G$ induces a morphism $\mu':(G'\times_kG')_\red\to G'$, which makes $G'$ a group in $(\Sch_{/k})_\red$.

We recall that for a scheme $P$ we denote by $|P|$ the underlyig topological space of $P$. Then, we define a sub-$A$-functor $G^0$ of $G$ so that for any $A$-scheme $S$,
\[G^0(S)=\{u\in G(S):u(|S|)\sub |G'|\}.\]
Let $c:G\to G$ be the inversion morphism; as $c(|G'|)=|G'|$, we have $c\circ u\in G^0(S)$ for any $u\in G^0(S)$. On the other hand, if $u,v\in G^0(S)$, then $u\boxtimes v$ sends $|S|$ into the subspace $|G\times_AG|$ formed by points whose two projections belong to $|G'|$. This subspace is identified with the underlying space of $G'\times_AG'$, which is connected by (\cite{EGA4-2} 4.5.8). Therefore, $\mu\circ(u\boxtimes v)$ sends $|S|$ into $|G'|$, and we conclude that $G^0$ is a sub-$A$-functor in groups of $G$.\par
If the connected component of $e$ is an open subset of $|G|$, then the sub-functor $G^0$ is representable by this open subscheme of $G$, which is then a subgroup scheme of $G$, also denoted by $G$. In this case, we have $G'=(G^0)_\red$, and the underlying spaces $|G'|$ and $|G^0|$ coincide.\par
We now assume that $G$ is locally of finite type over $A$, so that $G$ is locally Noetherian, hence locally connected (\cref{topo space local Noe is local connected}). In this case, any connected component of $G$ is open. We then denote by $G^0$ the induced open subscheme of $G$ over $|G'|$. By the arguments above, $G^0$ is a subgroup of $G$, called the \textbf{identity component} of $G$. For any $A$-scheme $S$, we then have
\[G^0(S)=\{u\in G(S):u(|S|)\sub |G^0|=|G'|\}.\]
Let $G^\alpha$ be a connected component of $G$ and $\nu^\alpha:G^\alpha\times_AG^0\to G$ be the morphism induced by the equality
\[\nu^\alpha(S)(g,\gamma)=g\gamma g^{-1},\]
for any $S\in\Sch_{/A}$, $g\in G^\alpha(S)$, $\gamma\in G^0(S)$. If $e$ is the identity of $G$, the restriction of $\nu^\alpha$ to $G^\alpha\times_A\{e\}$ is trivial; as $G^\alpha\times_AG^0$ is connected by (\cite{EGA4-2} 4.5.8), we see that $\nu^\alpha$ factors through $G^0$. Hence, for any $A$-scheme $S$, $G^0(S)$ is a normal subgroup of $G(S)$. We then obtain assertion (a) of the following proposition:

\begin{proposition}\label{scheme alg group identity component prop}
Let $G$ be an $A$-group locally of finite type.
\begin{enumerate}
    \item[(a)] $G^0$ is an irreducible normal (in fact characteristic) subgroup of $G$ and $G^0\otimes_Ak$ is geometrically irreducible over $k$.
    \item[(b)] $G^0$ is quasi-compact, hence of finite type over $A$.
\end{enumerate}
\end{proposition}
\begin{proof}
As $G^0$ and $G^0\otimes_Ak$ have the same underlying topological space, it suffices to prove that $G^0\otimes_Ak$ is geometrically irreducible. Let $\bar{k}$ be an algebraic closure of $k$. Then $(G\otimes_A\bar{k})_\red$ is a $\bar{k}$-group locally of finite type and reduced, hence smooth over $\bar{k}$ (\cref{scheme alg group flat smooth iff at unit}). A fortiori, the local rings $(G\otimes_A\bar{k})_\red$ are integral, hence, since $G\otimes_A\bar{k}$ is locally Noetherian, the connected components of $G\otimes_A\bar{k}$ are irreducible (cf. \cref{topo space locally Noe irre component open iff} and \cref{scheme irreducible component open and nilradical}). In particular, the connected component $G^0\times_A\bar{k}$ is irreducible. Finally, as $G^0$ is locally of finite type over $A$, it suffices to prove that $G^0$ is quasi-compact, which follows from \cref{scheme A-group irreducible is quasi-compact} since $G^0$ is irreducible.
\end{proof}

\begin{corollary}\label{scheme alg group connected component prop}
Any connected component of $G$ is irreducible, of finite type over $A$, and of the same dimension as $G^0$.
\end{corollary}
\begin{proof}
We can suppose that $A=k$. Let $C$ be a connected component of $G$, $x$ be a closed point of $C$, $\kappa(x)$ be the residue field of $x$ and $k'$ be a finite normal extension of $k$ containing $\kappa(x)$ over $k$ ($\kappa(x)$ is a finite extension, cf. \cref{scheme algebraic is Jacobson and closed point char}). The canonical projection $\pi:C\otimes_kk'\to C$ is open and closed (cf. \cref{scheme morphism integral is universally closed} and \cite{EGA4-2} 2.4.10), therefore, if $C'$ is the connected component of $C\otimes_kk'$, the projection $C'\to C$ is  surjective, hence $C'$ contains a point $y\in\pi^{-1}(x)$, and such a point is rational over $k'$ (cf. the proof of \cref{scheme alg group flat closed point locally rational}). We then conclude that $C'$ is the disjoint union of $G^0\times_kk'$ under the translation $r_y$ for $y\in\pi^{-1}(x)$. Now $G^0\times_kk'$ is of finite type over $k'$ by \cref{scheme alg group identity component prop} and $\pi^{-1}(x)$ is finite (with cardinality $\leq[k':k]$), so $C\otimes_kk'$ is of finite type over $k'$, and hence $C$ is of finite type over $k$ (cf. \cite{EGA4-4}, 17.7.4). On the other hand, as $G^0\times_kk'$ is irreducible by \cref{scheme alg group identity component prop}, so is $C'$, and hence is $C$, since the projection $C'\to C$ is surjective.\par
We therefore conclude from above that $C\otimes_kk'$ is the disjoint union of a finite number of translations of $G^0\otimes_kk'$. As the dimension is invariant under base change of fields, it follows that $C$ has the same dimension as $G^0$ (moreover, it follows from (\cite{EGA4-2}, 5.2.1) that we have $\dim_g(G)=\dim(G^0)$ for any point $g\in G$).
\end{proof}

\begin{example}\label{scheme A-group connected component not geometrically irreducible example}
One should note that a connected component (not containing the identity) may not be geometrically connected in general. For example, if $k=\R$, the group $\bm{\mu}_{3,\R}$, represented by $\R[X]/(X^3-1)$, has two connected components:
\[\{e\}=\Spec(\R),\quad C=\Spec(\R[X]/(X^2+X+1)),\]
and $C\otimes_{\R}\C$ has two components. 
\end{example}

Before proceeding further, we establish the following simply lemma, which allows us to convert many problems about $A$-group schemes to those over the residue field $k$.

\begin{lemma}\label{scheme over local Artinian ft closed immersion if closed fiber}
Let $(A,\m)$ be a local Artinian ring and $k=A/\m$ be the residue field.
\begin{enumerate}
    \item[(a)] Let $X$ be an $A$-scheme such that $X\otimes_Ak$ is locally of finite type (resp. of finite type), then so is $X$.
    \item[(b)] Let $u:X\to Y$ be a morphism of $A$-schemes. If $u\otimes_Ak$ is an immersion (resp. a closed immersion), then so is $u$.
\end{enumerate}
\end{lemma}
\begin{proof}
Suppose that $X\otimes_Ak$ is locally of finite type over $k$. Let $U=\Spec(B)$ be an affine open of $X$. By the hypothesis of (a), there exists elements $x_1,\dots,x_n$ of $B$ whose image generate $B/\m B$ as a $k$-algebra, and it follows from Nakayama's lemma that $x_i$ generate $B$ as an $A$-algebra. This shows that $X$ is locally of finite type over $A$. If $X\otimes_Ak$ is also quasi-compact, then so is $X$ (they have the same underlying topological space), and hence $X$ is of finite type over $A$. This proves (a).\par
Now let $u:X\to Y$ be a morphism of $A$-schemes and suppose that $u\otimes_Ak$ is an immersion (resp. a closed immersion). Then $u$ is a homeomorphism from $X$ to a locally closed (resp. closed) subset of $Y$ and, for any $x\in X$, the ring homomorphism $\phi_x:\mathscr{O}_{Y,\phi(x)}\to\mathscr{O}_{X,x}$ is such that $\phi_x\otimes_Ak$ is surjective. By Nakayama's lemma, it follows that $\phi_x$ is surjective, so $u$ is an immersion (resp. a closed immersion) by \cref{scheme morphism immersion iff stalk}~(b).
\end{proof}

\begin{proposition}\label{scheme A-group homomorphism of local ft image prop}
Let $A$ be a local Artinian ring with residue field $k$ and let $u:G\to H$ be a quasi-compact morphism between $A$-group schemes locally of finite type.
\begin{enumerate}
    \item[(a)] The set $u(G)$ is closed in $H$, whose connected components are irreducible and of the same dimension.
    \item[(b)] We have $\dim(G)=\dim(u(G))+\dim(\ker u)$.
    \item[(c)] If $u$ is a monomorphism, it is a closed immersion.
\end{enumerate}
\end{proposition}
\begin{proof}
By \cref{scheme over local Artinian ft closed immersion if closed fiber}, it suffices to prove the proposition in the case where $A=k$. Moreover, as the considered properties are stable under fpqc descent, and as the dimension is stable under base change of fields, we may assume that $k$ is algebraically closed.\par
Denote by $C$ the reduced subscheme of $H$ whose underlying space if $\widebar{u(G)}$. As $u(G)$ is stable under the inversion morphism of $H$, so is $C$. On the other hand, $u:G\to C$ is quasi-compact and dominant, hence by (\cite{EGA4-2}, 2.3.7), so is $u\times_k\id_G$ and $\id_H\times_ku$, hence is their composition $u\times_ku:G\times_kG\to C\times_kC$. Therefore, the multiplication of $H$ sends $C\times_kC$ into $C$, and $C$ is a subgroup of $H$.\par
Hence, by replacing $H$ with $C$, we may assume that $u$ is dominant. Since $k$ is algebraically closed and $G$ is of finite type over $k$, we see that $u(G(k))$ is dense in $H$, hence meets any connected component of $H$, and acts transitively on the set of connected components. It then suffices to show that $u(G)$ contains $H^0$. By replacing $G$ with $u^{-1}(H^0)$, we can then suppose that $H=H^0$; in this case, by \cref{scheme alg group identity component prop}, $H$ is irreducible and of finite type over $k$, hence Noetherian. On the other hand, $u$ is locally of finite type by \cref{scheme morphism local ft permanence prop} and quasi-compact, hence of finite type. The constructibility theorem of Chevalley (\cite{EGA4-1}, 1.8.5), $u(G)$ is a constructible subset (and dense) of $H=\widebar{u(G)}$, hence contains an open dense subset $U$ of $H$ (\cite{EGA3} $0_{\Rmnum{3}}$, 9.2.2). Then by \cref{scheme A-group product of open dense is G}, we have $H=U\cdot U\sub u(G)$, whence $u(G)=H$. In view of \cref{scheme alg group connected component prop}, this proves the assertion of (a).\par
To prove (b), recall first that the functor $\ker u$ is representable by $u^{-1}(e)$, where $e$ is the identity element of $H$. As $u$ is of finite type, $\ker u$ is of finite type over $k$. On the other hand, by replacing $H$ with the reduced closed subscheme $u(G)$, we may assume that $u$ is surjective. Denote by $u^0$ the restriction of $u$ to $G^0$. As $G$ and $\ker u$ are equidimensional and $(\ker u)^0\sub\ker u^0$, we are then reduced to the case where $G$, and hence $H$, are irreducible.\par
By (\cite{EGA4-3}, 9.2.6.2 et 10.6.1(\rmnum{2})), the set of $y\in H$ such that $\dim(u^{-1}(y))=\dim(G)-\dim(H)$ contains a nonempty open subset $V$. Since $u$ is surjective, $U=u^{-1}(V)$ is then a nonempty open subset of $G$, hence contains a closed point $x$ of $G$, since $G$ is a Jacobson scheme (\cite{EGA4-3}, 10.4.8). Then the right translation $r_x$ is an isomorphism from $\ker u$ to $u^{-1}(u(x))$, whence
\[\dim(\ker u)=\dim(u^{-1}(u(x)))=\dim(G)-\dim(H).\]

Now suppose that $u$ is a momomorphism. If $C$ is a connected component of $G$, there exists a closed point $x\in G$ such that $C=r_x(G^0)$,  and if we denote by $u_C$ (resp. $u^0$) the restriction of $u$ to $C$ (resp. to $G^0$), we have $u_C=r_{u(x)}\circ u^0\circ r_x^{-1}$, so it suffices to show that $u^0$ is a closed immersion. We can then suppose that $G=G^0$, so that $G$ is irreducible and of finite type over $k$.\par
Let $\xi$ be the generic point of $G$, then $\mathscr{O}_{G,\xi}$ if a local Artinian ring, with maximal ideal $\m_\xi$. Let $h=u(\xi)$, $\m_h$ be the maximal ideal of $\mathscr{O}_{H,h}$, and $A=\mathscr{O}_{G,\xi}/\m_h\mathscr{O}_{G,\xi}$. As $u$ is a monomorphism, so is the morphism $u_h:\Spec(A)\to\Spec(\kappa(h))$ induced by base change, hence the multiplication map $A\otimes_{\kappa(h)}A\to A$ is an isomorphism, and we conclude $A=\kappa(h)$. By Nakayama's lemma (since $\m_h\mathscr{O}_{G,\xi}$ is contained in $\m_\xi$, hence nilpotent), it follows that the morphism $\mathscr{O}_{H,h}\to\mathscr{O}_{G,\xi}$ is surjective.\par
Let $V$ be an affine open subset of $H$ containing $h$, $U$ be a nonempty affine open subset of $G$ contained in $u^{-1}(V)$, $\phi:\mathscr{O}_H(V)\to\mathscr{O}_G(U)$ be the induced morphism of $k$-algebras, $\p$ be the prime ideal of $\mathscr{O}_G(U)$ corresponding to $\xi$, and $\q=\phi^{-1}(\p)$. As $G$ is of finite type over $k$, $\mathscr{O}_G(U)$ is generated by finitely many elements $a_1,\dots,a_n$ as a $k$-algebra. By the preceding arguments, there then exist elements $b_1,\dots,b_n$ and $s$ in $\mathscr{O}_H(V)$ such that $s\notin\q$ and we have the equalities $a_i/1=\phi(b_i)/\phi(s)$ in $\mathscr{O}_G(U)_\p$. By definition, this means there are elements $t_1,\dots,t_n$ of $\mathscr{O}_G(U)-\p$ such that $t_i(a_i\phi(s)-\phi(b_i))=0$. Then, putting $t=t_1\cdots t_n\phi(s)\in\mathscr{O}_G(U)-\p$, the equalities $a_i/1=\phi(b_i)/\phi(s)$ are valid in $\mathscr{O}_G(U)_t$ and, as $t\in\mathscr{O}_G(U)=k[a_1,\dots,a_n]$, there exists $b\in\mathscr{O}_H(V)$ such that $t/1=\phi(b)/\phi(s)^r$ for an integer $r\in\N$. Therefore $\phi$ induces a surjection from $\mathscr{O}_H(V)_{sb}$ to $\mathscr{O}_G(U)_t$, and hence $u$ is a local immersion at the generic point $\xi$.\par
The open subset $W$ of $G$ formed by points over which $u$ is a local immersion is therefore nonempty. As $G$ is a Jacobson scheme, $W$ contains a closed point $y$ and, to show that $W=G$, it suffices to show that any closed point of $G$ belongs to $W$. But any closed point $x$ is the image of $y$ under the translation $r_x\circ r_y^{-1}$, and hence belongs to $W$. This proves that $u$ is a local immersion.\par
As $G$ is irreducible, it then follows that $u$ is an immersion. In fact, for any $x\in G$, let $U_x$ and $V_x$ be open subsets of $G$ and $H$, respectively, such that $x\in U_x$ and that $u$ induces a closed immersion from $U_x$ into $V_x$. As $U_x$ is dense in $G$, $u(U_x)$ is in $u(G)\cap V_x$, and as $u(U_x)$ is closed in $V_x$, we then have $u(U_x)=V_x$. As $u$ is also injective, we also have $U_x=u^{-1}(V_x)$, so $u$ induces a closed immersion from $G$ into the open subscheme of $H$ formed by the union of the $V_x$; whence $u:G\to H$ is an immersion. Since we have proved that $u(G)$ is closed in $H$, it is a closed immersion. 
\end{proof}

\subsection{Orbits under an algebraic group}
\begin{lemma}\label{scheme A-group action orbit map flat iff at point}
Let $A$ be a local Artinian ring, $k$ be its residue field, $G$ be a flat $A$-group, $X$ be an $A$-scheme endowed with a right action $\mu:G\times_AX\to X$ of $G$ and a section $s_0:\Spec(A)\to X$. Let $\phi$ be the morphism $\mu\circ(\id_G\times s_0)$ from $G=G\times_AA$ to $X$. If $\phi$ is flat at a point $g\in G$, then it is flat.
\end{lemma}
\begin{proof}
As $G$ is flat over $A$, by the fiber criterion of flatness (\cite{EGA4-3}, 11.3.10.2), it suffices to show that $\phi\otimes_Ak$ is flat, hence we can assume that $A=k$. In this case, $s_0$ can be considered as a $k$-point $x_0\in X(k)$, and $\phi$ is the orbit morphism $h\mapsto h\cdot x_0$.\par
Let $h\in G$, we show that $\phi$ is flat at $h$. Let $K$ be an extension of $k$ containing $\kappa(g)$ and $\kappa(h)$; we have a Cartesian square
\[\begin{tikzcd}
G_K\ar[r,"\phi_K"]\ar[d]&X_K\ar[d]\\
G\ar[r,"\phi"]&X
\end{tikzcd}\]
where the vertical morphisms are faithfully flat. By (\cite{SGA3-1}, \Rmnum{5}, Lemme 7.4), $\phi_K$ is flat at any point $g'\in G_K$ lying over $g$, and to show that $\phi$ is flat at $h$, it suffices to show that $\phi_K$ is flat at any point $h'$ lying over $h$. We are then reduced to the case where $g$ and $h$ are rational. Let $u=hg^{-1}$ and $\ell_u$ (resp. $\mu_u$) be the left translation of $G$ (resp. of $X$) defined by $u$. As $\phi\circ\ell_u=\mu_u\circ\phi$, we obtain a commutative square
\[\begin{tikzcd}
\mathscr{O}_{X,h\cdot x_0}\ar[r,"\sim"]\ar[d]&\mathscr{O}_{X,g\cdot x_0}\ar[d]\\
\mathscr{O}_{G,h}\ar[r,"\sim"]&\mathscr{O}_{G,g}
\end{tikzcd}\]
in which the horizontal morphisms are isomorphisms. As the morphism $\mathscr{O}_{X,g\cdot x_0}\to\mathscr{O}_{G,g}$ is flat by hypothesis, we conclude that $\mathscr{O}_{X,h\cdot x_0}\to\mathscr{O}_{G,h}$ is also flat.
\end{proof}

\begin{example}\label{scheme alg group morphism nonclosed eg}
Let $k$ be a field of characteristic zero, $G$ be the constant group $\Z_k$ over $k$ and $H$ be the additive group $\G_{a,k}$. Let $u:G\to H$ be a morphism of $k$-groups. If $u\neq 0$, then $u(G)$ is not closed in $H$. In fact, in this case $u(G)$ is an infinite subset of closed points of the underlying scheme $\A_k^1$ of $H$, which is not closed by \cref{scheme algerbraic curve topology prop}.
\end{example}

We recall that if $G$ is a Lie group, then a homogeneous space $X=G/H$ has a natural manifold structure and its dimension is given by $\dim(G)-\dim(H)$. In the case of algebraic groups, we still have the following analogous result:

\begin{proposition}\label{scheme A-group-object transitive prop}
Let $A$ be a local Artinian ring, $k$ be its residue field, $G$ be an $A$-group locally of finite type, $X$ be a nonempty $A$-scheme locally of finite type endowed with a left action by $G$. Suppose that the morphism $\phi:G\times_SX\to X\times_SX$ defined by $(g,x)\mapsto(gx,x)$ is surjective, then:
\begin{enumerate}
    \item[(a)] The connected components of $X$ are of finite type, irredcuible and have the same dimension.
    \item[(b)] More precisely, let $\bar{k}$ be an algebraic closure of $k$ and $x$ be a closed point of $X\otimes_A\bar{k}$. Then the stabilizer $F=\Stab_{G\otimes_A\bar{k}}(x)$ is a closed subgroup of $G\otimes_A\bar{k}$, and the dimension of the irreducible components of $X$ is $\dim(G)-\dim(F)$.
\end{enumerate}
\end{proposition}
\begin{proof}
In view of \cref{scheme over local Artinian ft closed immersion if closed fiber}, we may suppose that $A=k$. We first consider the case where $k$ is algebraically closed. Then $G_\red$ is a $k$-group locally of finite type and hence, by replacing $G$ with $G_\red$ and $X$ with $X_\red$, we may assume that $G$ and $X$ are reduced.\par
As $G\times_kX$ is locally of finite type over $k$, $\phi$ is locally of finite type by \cref{scheme morphism local ft permanence prop}, hence locally of finite presentation since $X\times_kX$ is locally Noetherian. Let $x$ be a rational point of $X$, then the orbit map $\phi_x:G\to X$, induced from $\phi$ by base change along $\id\times x:X\to X\times_AX$, is surjective and locally of finite presentation. If $\eta$ is a maximal point of $X$, then $\mathscr{O}_{X,\eta}$ is a field (since $X$ is reduced), so $\phi_x$ is flat at any point of $G$ lying over $\eta$. By \cref{scheme A-group action orbit map flat iff at point}, we then conclude that $\phi_x$ is flat. Now $\phi_x$ is faithfully flat and locally of finite presentation, hence open (cf. \cite{EGA4-2}, 2.4.6). As $G^0$ is open in $G$, irreducible and quasi-compact (\cref{scheme alg group identity component prop}), each orbit $G^0\cdot x=\phi_x(G^0)$, for $x$ a rational point of $X$, is an open subset of $X$, irreducible and quasi-compact, hence of finite type over $k$ (since $X$ is locally of finite type over $k$).\par
As any nonempty open subset of $X$ contains a rational point (equivalently, a closed point, since $k$ is assumed to be algebraically closed), it then follows that $X$ is covered by the orbits of $G^0$. Moreover, any two orbits are either disjoint or equal. In fact, if $\phi_x(G^0)\cap\phi_y(G^0)$ is nonempty, it contains a rational point $z$ and there then exists $g,h\in G^0$ such that $g\cdot x=z=h\cdot y$, whence $x=g^{-1}\cdot z$ and $y=h^{-1}\cdot z$ and hence $\phi_x(G^0)=\phi_z(G^0)=\phi_y(G^0)$. Therefore, the orbits $\phi_x(G^0)$ are the irreducible components of $X$, and also the connected components of $X$.\par
Finally, let $x,y$ be two rational points of $X$. As $\phi_x$ is surjective, there exists a point $g\in G$ such that $y=g\cdot x$ and, as $G^0$ is a normal subgroup of $G$, the orbit $G^0\cdot y$ is then the image of $G^0$ under the left translation $\ell_g$ of $X$, so that $G^0\cdot y$ and $G^0\cdot x$ have the same dimension.\par
If $x$ is a closed point of $X$, the stabilizer of $x$ is represented by the closed subscheme $F$ of $G$ defined by the Cartesian square:
\[\begin{tikzcd}
F\ar[r]\ar[d]&G\ar[d,"\phi_x"]\\
\Spec(k)\ar[r,"x"]&X
\end{tikzcd}\]
Then $F$ is a $k$-group locally of finite type, $F\cap G^0$ is a $k$-group of finite type containing $F^0$, and by \cref{scheme alg group connected component prop}, $F$ and $F\cap G^0$ are equidimensional, with the same dimension as $F^0$. Let $C=\phi_x(G^0)$ be the irreducible component of $X$ containing $x$. By the same arguments as \cref{scheme A-group homomorphism of local ft image prop}, we can show that $\dim(C)=\dim(G^0)-\dim(F^0)=\dim(G)-\dim(F)$.\par
In the general case (i.e. $k$ is an arbitrary field), let $\bar{k}$ be an algebraic closure of $k$. Let $C$ be a connected component of $X$ and $C'$ be a connected component of $C\otimes_k\bar{k}$, then $C'$ is a connected component of $X'=X\otimes_k\bar{k}$. The morphism $\pi:X'\to X$ is open by (\cite{EGA4-2}, 2.4.10), and as it is integral, it is also closed, so $\pi(C')=C$. As $C'$ is irreducible and quasi-compact, so is $C$, and hence $C$ is of fnite type over $k$ (because $X$ is locally of finite type over $k$).\par
Finally, since the dimension is invariant under base change of fields (\cite{EGA4-2}, 4.1.4), $\dim(C)=\dim(C')$, and as any irreducible components of $X'$ are of the same dimension, the same is true for $X$.
\end{proof}

\begin{proposition}[\textbf{Closed Orbit Lemma}]\label{scheme alg group closed orbit lemma}
Let $A$ be a local Artinian ring, $k$ be its residue field, $G$ be an $A$-group locally of finite type and $X$ be a nonempty $A$-scheme of finite type endowed with a left action by $G$. Then each orbit of a rational point of $X$ is open in its closure, and the boundary of each orbit is a union of orbits of lower dimension. In particular, the orbits of minimal dimension are closed.
\end{proposition}
\begin{proof}
As in \cref{scheme A-group-object transitive prop}, we may assume that $A=k$ is algebraically closed and $G$, $X$ are reduced. In this case, if $x\in X(k)$ is a rational point of $X$ and $M=\phi_x(G)$ is the orbit of $x$ under $G$, then $G$ is clearly stable under $G$. We note that any two rational points of $M$ are conjugate: in fact, if $y\in M(k)$, then by definition there exists $g\in G$ such that $\phi_x(g)=y$. Since $y$ is rational and $\phi_x$ is a $k$-morphism, it follows that $g$ is rational over $k$, whence $\ell_g(x)=y$.\par
Now as $X$ is of finite type over $k$, it is Noetherian, so by Chevalley's constructibility theorem  (\cite{EGA4-1}, 1.8.5), $M$ is a constructible subset (and dense) of $\widebar{M}$, hence contains an open dense subset $U$ of $\widebar{M}$ (\cite{EGA3} $0_{\Rmnum{3}}$, 9.2.2). It is clear that $U$ contains the closed point $x$ of $M$, so by homogeneity, every closed point of $\widebar{M}$ is contained in an open subset of $\widebar{M}$ contained in $M$. We therefore conclude from (\cite{EGA4-3}, 10.1.2) that $M$ is open in $\widebar{M}$, so $\widebar{M}-M$ is closed and of lower dimension, as well as being $G$-stable; it is therefore a union of orbits of $G$ of lower dimension.
\end{proof}

\begin{remark}
We note that if $G$ is reduced, then the morphism $\phi_x:G\to X$ is faithfully flat, so $\mathscr{O}_M\to(\phi_x)_*(\mathscr{O}_G)$ is an injective homomorphism, and remains so after base change of fields. Therefore, if $G$ is smooth, then $M$ is geometrically reduced, and since its smooth locus is nonempty (hence open dense in $M$, cf. \cite{EGA4-4} 17.15.12), we see that $M$ is smooth over $k$ by homogeneity. 
\end{remark}

\begin{example}
Let $k$ be an algebraically closed field and consider the action
\[\SL_2\times\A_k^2\to \A_k^2,\quad \begin{pmatrix}
a&b\\
c&d
\end{pmatrix}\begin{pmatrix}
x\\
y
\end{pmatrix}=\begin{pmatrix}
ax+by\\
cx+dy
\end{pmatrix}.\]
There are two orbits of $\A_k^2$, namely $\{(0,0)\}$ and its complement. We see that the smaller one is closed, but the larger one is not locsed, and not even affine.
\end{example}

\subsection{Morphisms of algebraic groups}
Let $A$ be a local Artinian ring, $G$ and $H$ be $A$-groups and $u:G\to H$ be a morphism of $A$-groups. Then $u$ induces a morphism of groups $u(A):G(A)\to H(A)$. As $H(A)$ acts on $H$ by right translations, $u(A)$ defines an action of $G(A)$ on $H$, which is compatible with the morphism $u$ and the action of $G(A)$ on $G$ defined by left translations. As $G(A)$ acts transitively on strictly rational points, we see that these points "share the same properties with respect to $u$". For example, we have the following result:

\begin{proposition}\label{scheme alg group morphism prop iff point}
Let $A$ be a local Artinian ring with residue field $k$, $G$ be an $A$-group locally of finite type and flat, $u:G\to H$ be a morphism of $A$-groups. The following conditions are equivalent:
\begin{enumerate}
    \item[(\rmnum{1})] $u$ is flat (resp. quasi-finite, resp. unramified, resp, smooth, resp. \'etale) at a point of $G$.
    \item[(\rmnum{2})] $u$ is flat (resp. quasi-finite, resp. unramified, resp, smooth, resp. \'etale)
\end{enumerate}
\end{proposition}
For the proof of \cref{scheme alg group morphism prop iff point}, we need the following lemma:

\begin{lemma}\label{ring homomorphism ft and fp after nilpotent quotient}
Let $A\to B\to C$ be ring homomorphisms and $\n$ be a nilpotent ideal of $A$. Suppose that $C/\n C$ is a $(B/\n B)$-algebra of finite type.
\begin{enumerate}
    \item[(a)] $C$ is a $B$-algebra of finite type.
    \item[(b)] If $C$ is flat over $A$ and $C/\n C$ is a $(B/\n B)$-algebra of finite presentation, then $C$ is a $B$-algebra of finite presentation.
\end{enumerate}
\end{lemma}
\begin{proof}
Let $x_1,\dots,x_n$ be elements of $C$ whose images in $C/\n C$ generate it as a $(B/\n B)$-algebra. By nilpotent Nakayama's lemma, the $x_i$ generate $C$ as a $B$-algebra, and this proves (a). Let $\phi:B[X_1,\dots,X_n]\to C$ be the surjective homomorphism thus obtained, and $\mathfrak{I}=\ker\phi$. Suppose that $C$ is flat over $A$ and $C/\n C$ is of finite presentation over $B/\n B$. Then $\mathfrak{I}/\n\mathfrak{I}$, which is identified with the kernel of $\bar{\phi}=\phi\otimes_A(A/\n)$, is finitely generated by \cref{algebra fp iff polynomial map finite kernel}. Let $P_1,\dots,P_s$ be the polynomials whose images generate $\mathfrak{I}/\n\mathfrak{I}$, then by nilpotent Nakayama's lemma, they generate $\mathfrak{I}$, which proves (b).
\end{proof}

\begin{proof}[Proof of \cref{scheme alg group morphism prop iff point}]
It suffices to prove that (\rmnum{1})$\Rightarrow$(\rmnum{2}), so let $x$ be an arbitrary point of $G$. As $G$ is flat over $A$, by the fiber criterion of flatness (\cite{EGA4-3}, 11.3.10.2), $u$ will be flat at $x$ if $u\otimes_Ak$ is. Similarly, by the preceding lemma, we see that $u$ will be locally of finite type (resp. locally of finite presentation) if $u\otimes_Ak$. As the other properties are verified over fibers, we may then assume that $A=k$.\par
Now let $x$ be a point of $G$ where condition (a) is satisfied. As the considered properties are perserved under fpqc descent (cf. \cite{EGA4-2}, 2.5.1, 2.7.1 et \cite{EGA4-4}, 17.7.1), we may assume that $k$ is an algebraic closure of $\kappa(x)$, so that $k$ is algebraically closed and $x\in G(k)$.\par
As $G$ is a Jacobson scheme (cf. \cite{EGA4-3}, 10.4.7) as as the set $W$ of points of $G$ where $u$ is flat, quasi-finite, unramified, smooth or \'etale is open, it suffices to show that any point $y$ of $G(k)$ belongs to $W$. Now, for such a point $y$, the translation $r_y\circ r_x^{-1}$ sends $x$ to $y$, hence $u$ possesses the same property at $y$, i.e. $y\in W$. 
\end{proof}

\begin{corollary}\label{scheme alg group prop iff at point}
Let $A$ be a local Artinian ring with residue field $k$, $G$ be a flat $A$-group. The following assertions are equivalent:
\begin{enumerate}
    \item[(a)] $G$ is locally quasi-finite (resp. unramified, resp, smooth, resp. \'etale) over $A$ at a point.
    \item[(b)] $G$ is locally quasi-finite (resp. unramified, resp, smooth, resp. \'etale) over $A$
\end{enumerate}
\end{corollary}
\begin{proof}
In fact, if $G$ satisties one of the conditions of (a) at a point $x$, there exists an open neighborhood $U$ of $x$ which is of finite type over $A$. Therefore, it suffices to apply \cref{scheme alg group morphism prop iff point} to the case where $H$ is the trivial $A$-group, in view of \cref{scheme group local ft over local Artin if open subset} below.
\end{proof}

\begin{lemma}\label{scheme group local ft over local Artin if open subset}
Let $A$ be a local Artinian ring and $G$ be an $A$-group. If there exists a nonempty open subset $G$ of finite type over $A$, then $G$ is locally of finite type over $G$.
\end{lemma}
\begin{proof}
By \cref{ring homomorphism ft and fp after nilpotent quotient}, we can suppose that $A=k$ is a field. Moreover, by fpqc descent, we can assume that $k$ is algebraically closed (cf. \cite{EGA4-2}, 2.7.1). Let $V$ be an open subset of $G$ formed by points where $G$ is of finite type over $k$; by hypothesis, $V\neq\emp$. As $G$ is a Jacobson scheme, $V$ contains a closed point $x$ and, to show that $V=G$, it suffices to show that any closed point $y$ of $G$ belongs to $V$. Now for such a point $y$, the translation $r_y\circ r_x^{-1}$ sends $x$ to $y$, so $y\in V$.
\end{proof}

\begin{corollary}\label{scheme alg group morphism open iff}
Let $A$ be a local Artinian ring and $u:G\to H$ be a morphism between $A$-groups locally of finite type. The following conditions are equivalent:
\begin{enumerate}
    \item[(\rmnum{1})] $u$ is universally open,
    \item[(\rmnum{2})] $u$ is open;
    \item[(\rmnum{3})] $u$ is open at a point of $G$;
    \item[(\rmnum{4})] the morphism $u^0:H^0\to H^0$ induced by $u$ is dominant;
    \item[(\rmnum{4}')] $u^0$ is surjective;
    \item[(\rmnum{5})] there exists a connected component $C$ of $G$ such that, if $D$ denotes the connected component of $H$ containing $u(C)$, the morphism $u':C\to D$ induced by $u$ is dominant.
\end{enumerate}
\end{corollary}
\begin{proof}
The implications (\rmnum{1})$\Rightarrow$(\rmnum{2})$\Rightarrow$(\rmnum{3}) and (\rmnum{4}')$\Rightarrow$(\rmnum{4})$\Rightarrow$(\rmnum{5}) are clear. As $G^0$ of finite type over $A$ (\cref{scheme alg group identity component prop}), hence Noetherian, the induced morphism $u^0$ is quasi-compact so $u^0(G^0)$ is closed in $H^0$ by \cref{scheme A-group homomorphism of local ft image prop}, whence (\rmnum{4})$\Rightarrow$(\rmnum{4}'). On the other hand, since $G^0$ (resp. $C$) is open in $G$ (\cref{scheme alg group identity component prop}) and $H^0$ (resp. $D$) is irreducible (\cref{scheme alg group connected component prop}), we see that (\rmnum{2}) implies (\rmnum{4}) (resp. that (\rmnum{3}) implies (\rmnum{5})). It then remains to show that (\rmnum{5}) implies (\rmnum{1}).\par
Let $C$ and $D$ be as in (\rmnum{5}) and endowed them with the induced scheme structure; denote by $u':C\to D$ the induced morphism. Let $k$ be a residue field of $A$. As the base change $\Spec(k)\to\Spec(A)$ is a universally homeomorphism (Cf. \cref{EGA4-4}, 18.12.11), we can suppose that $A=k$. By hypothesis, $u'$ is dominant and, since $C$ is of finite type over $k$ (\cref{scheme alg group connected component prop}), hence Noetherian, $u'$ is quasi-compact. By (\cite{EGA4-2}, 2.3.7), $u'\otimes_k\bar{k}$ is then quasi-compact and dominant, where $\bar{k}$ is an algebraic closure of $k$. Then, since $C\otimes_k\bar{k}$ is a union of connected components of $G\otimes_k\bar{k}$, the morphism $u\otimes_k\bar{k}:G\otimes_k\bar{k}\to H\otimes_k\bar{k}$ satisfies condition (\rmnum{5}). We are then reduced to the case where $A=k$ is algebraically closed (cf. \cref{EGA4-2}, 2.6.4).\par
In this case, we can further replace $u$ by $u_\red$, and therefore assume that $H$ is reduced. Let $\xi$ (resp. $\eta$) be the generic point of $C$ (resp. $D$). Since $u'$ is quasi-compact and dominant, $u'(\xi)=\eta$ by \cref{scheme morphism qs dominant char}. On the other hand, as $H$ is reduced, the local ring $\mathscr{O}_{H,\eta}$ is a field, and hence $u'$ is flat at $\xi$. Hence by \cref{scheme alg group morphism prop iff point}, $u$ is flat; moreover, since $u$ is locally of finite type and $H$ is locally Noetherian, $u$ is locally of finite presentation, so by (\cite{EGA4-2}, 2.4.6) it is universally open.
\end{proof}

\begin{proposition}\label{scheme alg group morphism proper iff}
Let $A$ be a local Artinian ring and $u:G\to H$ be a quasi-compact morphism between $A$-groups locally of finite type. The following assertions are equivalent:
\begin{enumerate}
    \item[(\rmnum{1})] $u$ is proper;
    \item[(\rmnum{2})] there exists $h\in H$ such that the fiber $u^{-1}(h)$ is nonempty and proper over $\kappa(h)$;
    \item[(\rmnum{2})] $\ker u$ is proper over $A$.
\end{enumerate}
\end{proposition}
\begin{proof}
It is clear that (\rmnum{1}) implies (\rmnum{3}), and (\rmnum{3}) implies (\rmnum{2}). On the other hand, it follows from the hypothesis that $u$ is of finite type and, since $G$ is separated, $u$ is also separated. It then remains to show that condition (\rmnum{2}) implies that $u$ is universally closed, for which we can assume that $A=k$. Let $k'$ be an algebraic closure of $\kappa(h)$, $u':G'\to H'$ be the morphism induced by base change, $h'$ be a point of $H'$ over $h$; then the fiber $u'^{-1}(h')=u^{-1}(h)\times_{\kappa(h)}k'$ is nonempty and proper, and it suffices to show that $u'$ is proper (\cite{EGA4-2}, 2.6.4). We can therefore assume that $k$ is algebraically closed and $h\in H(k)$.\par
We have seen in \cref{scheme A-group homomorphism of local ft image prop} that $u(G)$ is the underlying space of a closed and reduced subscheme of $H$; any closed immersion being proper, we can then suppose that $u$ is surjective, and that $H$ is reduced. In this case, $G(k)$ acts simply and transitively on closed points of $H$; for any closed point $y\in H$, $u^{-1}(y)$ is therefore proper over $\kappa(y)$. By (\cite{EGA4-3}, 9.6.1), the set of points $y\in H$ such that $u^{-1}(y)$ is not proper over $\kappa(y)$ is locally constructible; since it contains no closed point, it is therefore empty (cf. \cite{EGA4-3}, 10.3.1 et 10.4.7).\par
Now consider the generic point $\eta$ of $H^0$; by the preceding arguments, the fiber $u^{-1}(\eta)=G\times_H\Spec(\kappa(\eta))$ is proper over $\kappa(\eta)$. On the other hand, since $H$ is reduced, $\kappa(\eta)=\mathscr{O}_{H,\eta}$. As $\mathscr{O}_{H,\eta}$ is the inductive limit of $\mathscr{O}_H(V)$, for $V$ runs through nonempty open subsets of $H^0$, it follows from (\cite{EGA4-3}, 8.1.2(a) et 8.10.5(\rmnum{12})) that there exists a nonempty open subset $V$ of such that the restriction $u|_{u^{-1}(V)}:u^{-1}(V)\to V$ is proper. It is then clear that the $g\cdot V$, for $g\in G(k)$, form an open covering of $H$; we then deduce that $u$ is proper (cf. \cref{scheme morphism proper over closed subscheme}).
\end{proof}

\begin{corollary}\label{scheme alg group morphism local qf iff}
Let $A$ be a local Artinian ring and $u:G\to H$ be a morphism between $A$-groups locally of finite type. The following conditions are equivalent:
\begin{enumerate}
    \item[(\rmnum{1})] $u$ is locally quasi-finite;
    \item[(\rmnum{2})] $u$ is quasi-finite at a point;
    \item[(\rmnum{3})] $\ker u$ is discrete;
    \item[(\rmnum{4})] the restriction of $u$ to a connected component of $G$ is finite.    
\end{enumerate}
Finally, if $u$ is quasi-compact, these conditions are equivalent to:
\begin{enumerate}
    \item[(\rmnum{5})] $u$ is finite.
\end{enumerate}
\end{corollary}
\begin{proof}
It is clear that (\rmnum{4}) implies (\rmnum{3}), that (\rmnum{3}) implies (\rmnum{2}), and that in the case where $u$ is quasi-compact, (\rmnum{5}) is equivalent to (\rmnum{4}). We have seen in \cref{scheme alg group morphism prop iff point} that conditions (\rmnum{1}) and (\rmnum{2}) are equivalent.\par
We now show that (\rmnum{1})$\Rightarrow$(\rmnum{4}). Let $C$ be a connected component of $G$; since $C$ is of finite type over $A$ (\cref{connected component}) and that $G$, $H$ are separated, the restriction of $u'$ of $u$ to $C$ is separated and of finite type. As the fibers of $u'$ are discrete, it follows that $u'$ is quasi-finite (\cref{scheme morphism ft quasi-finite iff fiber discrete}). As any quasi-finite and proper morphism is finite (\cref{scheme morphism over local Noe finite iff affine proper}), it then suffices to show that $u'$ is universally closed, for which we may suppose that $A=k$ is a field. Then by fpqc descent (cf. \cite{EGA4-2}, 2.6.4), it suffices to show that $u'\otimes_k\bar{k}$ is universally closed, where $\bar{k}$ is an algebraic closure of $k$. Further, as $C$ is of fnite type over $k$, $C\otimes_k\bar{k}$ is a finite sum of connected components $C_1',\dots,C_n'$ of $G\otimes_k\bar{k}$, so it suffices to consider the claim for each $C'_i$; this means we can reduce to the case where $k$ is algebraically closed.\par
Now let $g$ be a closed point of $G$; if $u^0:G^0\to H$ is the restriction of $u$ to $G^0$, we have $u'=r_{u(g)}\circ u^0\circ r_{g^{-1}}$, so we only need to show that $u^0$ is universally closed. By hypothesis, $u$ is locally quasi-finite, so the fiber $\ker u$ is discrete (and nonempty); we note that $u^0$ is of finite type since $G^0$ is of finite type over $k$ (hence Noetherian), so the fiber $\ker u^0$ is finite, hence proper. It then follows from \cref{scheme alg group morphism proper iff} that $u^0$ is proper, and a fortiori universally closed.
\end{proof}

\begin{corollary}\label{scheme alg group qc morphism monomorphism iff}
Let $A$ be a local Artinian ring and $u:G\to H$ be a quasi-compact morphism between $A$-groups locally of finite type. The following conditions are equivalent:
\begin{enumerate}
    \item[(\rmnum{1})] $u$ is a closed immersion;
    \item[(\rmnum{2})] $u$ is a monomorphism;
    \item[(\rmnum{3})] $\ker u$ is trivial, ie,e, isomorphic to the trivial $k$-group.
\end{enumerate}
\end{corollary}
\begin{proof}
It is clear that (\rmnum{1}) implies (\rmnum{2}), and (\rmnum{2}) and (\rmnum{3}) are equivalent. Finally, if $\ker u$ is trivial, then it is proper over $k$ and nonempty, so $u$ is a proper monomorphism by \cref{scheme alg group morphism proper iff}, and is of finite presentation because $H$ is locally Noetherian. We then conclude from (\cite{EGA4-3}, 8.11.5) that $u$ is a closed immersion.
\end{proof}

\begin{example}\label{scheme k-group monomorphism nonclosed eg}
Let $k$ be a field of characteristic zero, $G$ be the constant $k$-group $\Z_k$ and $H$ be the $k$-group $\G_{a,k}$. Let $u:G\to H$ be a morphism of $k$-groups. If $u\neq 0$, then $\ker u=0$ and $u$ is a monomorphism. But $u$ is not a closed immersion (cf. \cref{scheme alg group morphism nonclosed eg}).
\end{example}

\subsection{Construction of the quotient \texorpdfstring{$F\backslash G$}{FG}}
Let $A$ be a local Artinian ring and $u:F\to G$ be a homomorphism of $A$-groups. If $\mu_F:F\times_AF\to F$ and $\mu_G:G\times_AG\to G$ denote the multiplication morphisms and $\lambda$ is the composition morphism
\[F\times_AG\stackrel{u\times\id_G}{\to} G\times_AG\stackrel{\mu_G}{\to} G,\]
we define the \textbf{left quotient} $F\backslash G$ of $G$ by $F$ to be the cokernel of the following $\Sch_{/A}$-groupoid $G_*$:
\[\begin{tikzcd}[column sep=15mm]
F\times_A F\times_AG\ar[r,shift left=8pt,"\pr_{2,3}"]\ar[r,shift right=8pt,swap,"\id_F\times\lambda"]\ar[r,"\mu\times\id_G"description]&F\times_AG\ar[r,shift left=2pt,"\lambda"]\ar[r,shift right=2pt,swap,"\pr_2"]&G
\end{tikzcd}\]
(where $\pr_2$ and $\pr_{2,3}$ are the projections of $F\times_AG$ and $F\times_A(F\times_AG)$.) We say that $G_*$ is the groupoid of base $G$ defined by $u$. As the unique $A$-morphism $F\to\Spec(A)$ is universally open (\cite{EGA4-2}, 2.4.9), $\pr_2$ is an open morphism, hence so is $\lambda$ which is the composition of $\pr_2$ and the automorphism $\sigma$ of $F\times_AG$ defined by $\sigma(S)(x,y)=(x,u(S)(x)\cdot y)$, where $S$ is an $A$-scheme and $x\in F(S)$, $y\in G(S)$. Similarly, we see that $\pr_2$ and $\lambda$ are flat if $F$ is flat over $A$.\par
On the other hand, as 
We also note that any $A$-morphism $s:\Spec(A)\to G$ defines an automorphism of the groupoid $G_*$ which induces over $G$, $F\times_AG$ and $F\times_AF\times_AG$ the automorphisms $r_s$, $\id_F\times r_s$ and $\id_F\times\id_F\times r_s$, respectively. We then denote this automorphism of $G_*$ by $r_s$ and call it the right translation of $G_*$ defined by $s$.

\begin{theorem}\label{scheme alg group group quotient exists}
Let $F$ and $G$ be groups flat and locally of finite type over a local Artinian ring $A$ and $u:F\to G$ be a quasi-compact homomorphism of $A$-groups with kernel finite over $A$. Then
\begin{enumerate}
    \item[(a)] The left quotient $F\backslash G$ of $G$ by $F$ exists in $\Sch_{/A}$ and $F\backslash G$ is a quotient of $G_*$ in the category of ringed spaces.
    \item[(b)] The canonical morphism $p:G\to F\backslash G$ is surjective and open, and $F\backslash G$ is locally of finite type over $A$.
    \item[(b')] More precisely, $X=F\backslash G$ is endowed with a right action of $G$ such that $p(e)\cdot g=p(g)$ for any $g\in G$. Therefore, the connected components of $X$ are of finite type over $A$, irreducible and of the same dimension $\dim(G)-\dim(F)$.
    \item[(c)] The canonical morphism $(\lambda,\pr_2):F\times_AG\to G\times_{(F\backslash G)}G$ is surjective.
    \item[(d)] If $u$ is a monomorphism, then:
    \begin{enumerate}
        \item[(\rmnum{1})] $F\times_AG\to G\times_{(F\backslash G)}G$ is an isomorphism and $G\to F\backslash G$ is fiathfully flat and locally of finite presentation.
        \item[(\rmnum{1}')] $F\backslash G$ represents the fppf sheaf quotient $\widetilde{F\backslash G}$ and $G\to F\backslash G$ is an $F$-torsor locally trivial for the fppf topology.
        \item[(\rmnum{2})] $F\backslash G$ is flat over $A$, and is smooth over $A$ is $G$ is.
        \item[(\rmnum{3})] $u:F\to G$ is a closed immersion and $F\backslash G$ is separated.
        \item[(\rmnum{4})] If $F$ is a normal subgroup of $G$, there exists a unique $A$-group structure on $F\backslash G$ such that $p:G\to F\backslash G$ is a morphism of $A$-groups.    
    \end{enumerate}  
\end{enumerate}
\end{theorem}

In the proof of this theorem, we denote by $A'$ a local $A$-algebra, finite and free over $A$. If $\mathcal{R}$ is a relation concerning $A'$, we shall say that "$\mathcal{R}(A')$ is valid for $A'$ large enough" if there exists a local algebra $A_1$, finite and free over $A$, such that the relation $\mathcal{R}(A')$ is verified for any local algebra $A'$, finite and free over $A_1$. 

\paragraph{Passage to the quotient \texorpdfstring{$F\backslash G$}{FG} (case where \texorpdfstring{$F$}{F} and \texorpdfstring{$G$}{G} are of finite type over $A$)}\label{scheme alg group group quotient ft case paragraph}
In this paragraph, we prove the theorem for the case where $F$ and $G$ are of finite type over $A$. For this, we first make the following additional hypothesis:
\begin{equation}\label{scheme alg group group quotient exists-1}
\parbox{5in}{%
    Every point of $G$ has a saturated open neighborhood $W$ such that the groupoid induced by $G_*$ on $W$ has a quasi-section.%
}
\end{equation}
(cf. \autoref{scheme groupoid quotient if quasi-section subsection}). Then, by \cref{scheme groupoid quasi-section lemma}, we have the assertions (a), (b), (c) and (d)(\rmnum{1}), and $F\backslash G$ is of finite type over $k$. Moreover, under the assumption of (d), since $G\to F\backslash G$ is faithfully flat and locally of finite presentation, assertion (d)(\rmnum{2}) follows from (\cite{EGA4-4}, 17.7.4). On the other hand, (d)(\rmnum{1}') follows from (d)(\rmnum{1}), by \cref{category equivalence relation M-effective prop}, \cref{site sheaf quotient by universal effective represents quotient sheaf} and \cref{site formally principal homogeneous under M-group iff}.\par

We now prove the following assertion:
\begin{equation}\label{scheme alg group group quotient exists-2}
\text{Any finite subset of $F\backslash G$ is contained in an affine open subset.}
\end{equation}
If $U$ is a quasi-section of the groupoid induced by $G_*$ over a saturated open subset $W$ of $G$, then $U\otimes_AA'$ is a quasi-section of the $\Sch_{/A'}$-groupoid induced by $G_*\otimes_AA'$ over $W\otimes_AA'$. Further, if $U_*$ is the $\Sch_{/A}$-groupoid induced by $G_*$ over $U$, then $U_*\otimes_AA'$ is identified with the $\Sch_{/A'}$-groupoid induced by $G_*\otimes_AA'$ over $U\otimes_AA'$. It then follows from the proof of \cref{scheme groupoid quasi-section lemma} that the construction of the quotient $X=F\backslash G$ commutes with any base change $A\to A'$, where $A'$ is a local $A$-algebra finite and free over $A$\footnote{For this, it amounts to note that the formation of the direct images under $p$, $\lambda$ and $\pr_2$ commute with flat base changes $A\to A'$: As $F$ and $G$ are of finite type and separated over an Artinian ring $A$, any morphism $f$ considered in question is quasi-compact and separated, and the equality $f_*(\mathscr{O}_X)\otimes_AA'=f'_*(\mathscr{O}_{X'})$ (with the evident notations) follows from \cref{scheme morphism qcsp base change formula}.}.\par

Now let $x_1,\dots,x_n$ be elements of $X=F\backslash G$, which we can assume to be closed\footnote{In fact, let $y_1,\dots,y_n$ be arbitrary points of $X$; as $X$ is of finite type over $A$, each $y_i$ is in the closure of a closed point $x_i$ of $X$, and any open subset containing $x_i$ also contains $y_i$.}, and $g_1,\dots,g_n$ be closed points of $G$ which project to $x_1,\dots,x_n$. Let $V$ be an affine everywhere dense open subset of $X$\footnote{Such an open subset exists, since $X$ is of finite type over $k$: $X$ has finitely many irreducible components $C_1,\dots,C_r$, and it suffices to choose for each $i$ a nonempty open affine subset contained in $C_i-\bigcup_{j\neq i}C_i$ (however, by (b') we know that these $C_i$ are disjoint).}, and let $U$ be the inverse image of $V$ in $G$. By \cref{scheme alg group flat closed point locally rational}, there exists a local $A$-algebra $A'$, finite and free over $A$, such that the points $g_1',\dots,g_r'$ of $G'=G\otimes_AA'$ lying over the $g_1,\dots,g_n$ are strictly rational over $A'$. As the morphisms $G'\to G$ and $G\to X$ are open, $U'=U\otimes_AA'$ is dense in $G'$, hence the open subset $\bigcap_{i=1}^{r}(U')^{-1}\cdot g'_i$ is nonempty, hence contains a closed point $x$. Therefore, by \cref{scheme alg group flat closed point locally rational} (and \cref{scheme A-group strict rational stable under base change remark}), we can suppose, by enlarging $A'$ if necessary, that $x$ is strictly rational over $A'$. Then, as $x\in(U')^{-1}\cdot g'_i$, we have $g_i'\in U'\cdot x$.\par

Denote by $V'$ the inverse image of $V$ in $X'=X\otimes_AA'$; this is an affine open subset of $X'$, and is also the image of $U'$ under the projection $G'\to X'$. As the right translation $r_x$ is an automorphism of the groupoid $G_*\otimes_AA'$, it induces an automorphism, still denoted by $r_x$, or the quotient $X'$. Therefore, the image $V'\cdot x=r_x(V')$ of $U'\cdot x$ in $X'$ is an affine open subset of $X'$ containing the images $x_1',\dots,x_r'$ of $g_1',\dots,g_r'$. Now consider the equivalence relation over $X'=X\otimes_AA'$ defined by the projection $X\otimes_AA'\to X$:
\[\begin{tikzcd}
X\otimes_AA'\otimes_AA'\ar[r,shift left=2pt,"d_1"]\ar[r,shift right=2pt,swap,"d_0"]&X\otimes_AA'\ar[r]&X
\end{tikzcd}\]
where $d_0$ and $d_1$ are induced by the two canonical injections of $A'$ into $A'\otimes_AA'$. As $A'$ is a finite and free $A$-algebra, say of rank $n$, we see that $d_0$ and $d_1$ are finite and locally free of rank $n$. Therefore, we can apply the reasoning of \ref{scheme groupoid quotient by locally free finite general case paragraph} (the proof taken from \cite{SGA1}, \Rmnum{8}, 7.6) to obtain a saturated affine open subset $W'\sub V'\cdot x$ containing $x_1',\dots,x_r'$. The image of $W'$ in $X$ then contains $x_1,\dots,x_n$ and is an affine open subset of $X$, by \cref{scheme groupoid quotient by locally free finite prop}\footnote{We note that $X\otimes_AA'\to X$ is a universally effective epimorphism since it is faithfully flat and locally of finite presentation, cf. \cref{scheme topology T_i family M_i}.}.\par

We return to the case where $F$ and $G$ are of finite type over $A$ (i.e. we do note assume (\ref{scheme alg group group quotient exists-1})). For any local algebra $A'$, finite and free over $A$, we now denote by $U(A')$ the set of points of $G\otimes_AA'$ admitting a satusrated open subset $W$ such that the groupoid induced by $G_*\otimes_AA'$ over $W$ possesses a quasi-section. It is clear that $U(A')$ is saturated for the operation of $G(\Spec(A'))$ over $G\otimes_AA'$. We claim that, if $A'$ is taken large enough, then $U(A')=G\otimes_AA'$.\par
Since $\ker u$ is assumed to be finite over $A$ (and hence discrete), we see that the morphism $(\lambda,\pr_2):F\times_AG\to G\times_AG$, which is the composition of $u\times\id_G:F\times_AG\to G\times_AG$ with the automorphism $\sigma$ of $G\times_AG$ defined by $(x,y)\mapsto(xy,y)$, is quasi-finite (cf. \cref{scheme alg group morphism local qf iff}). It then follows from (\cite{SGA3-1} \Rmnum{5}, theoreme 8.1) that $U(A)$ is not empty, hence contains a closed point $y$. We now prove our claim by induction on $\dim(G-U(A))$. Let $g_1,\dots,g_n$ be closed points belonging to distinct irreducible components of $G-U(A)$. By \cref{scheme alg group flat closed point locally rational}, there exists a local $A$-algebra $A'$, finite and free over $A$, such that the points $g_1',\dots,g_r'$ (resp. $x=x_1,\cdots, x_r$) of $G'=G\otimes_AA'$ lying over $g_1,\dots,g_n$ (resp. $y$) are strictly rational over $A'$. Then $U(A')$ contains $(U(A)\otimes_AA')\cdot x^{-1}g_i'$ for each $i$, so it contains $g_1',\dots,g_r'$ and we have
\[\dim(G'-U(A'))<\dim(G-U(A)).\]
The induction hypothesis implies the existence of a local algebra $A''$, finite and free over $A'$, such that $U(A'')=G'\otimes_{A'}A''=G\otimes_AA''$.\par

With the assertion in (\ref{scheme alg group group quotient exists-2}), we are now able to prove the existence of the existence $F\backslash G$ for $F,G$ of finite type over $A$. By the arguments above, we can take a local algebra $A'$, finite and free over $A$, such that $U(A')=G\otimes_AA'$. We put $A''=A'\otimes_AA'$ and, for any $A$-scheme $X$, we put $X'=X\otimes_AA'$ and $X''=X\otimes_AA''$. By what we have already proven, the quotients $F'\backslash G'$ and $F''\backslash G''$ exist and we have the following commutative diagram, where the first two rows and columns are exact:
\begin{equation}\label{scheme alg group group quotient exists-3}
\begin{tikzcd}[row sep=12mm, column sep=12mm]
F''\otimes_{A''}G''\ar[r,shift left=2pt,"\pr_2''"]\ar[r,shift right=2pt,swap,"\lambda''"]\ar[d,shift left=2pt,"w_2"]\ar[d,shift right=2pt,swap,"w_1"]&G''\ar[r,"p''"]\ar[d,shift left=2pt,"v_2"]\ar[d,shift right=2pt,swap,"v_1"]&F''\backslash G''\ar[d,shift left=2pt,"u_2"]\ar[d,shift right=2pt,swap,"u_1"]\\
F'\times_{A'}G'\ar[r,shift left=2pt,"\pr_2'"]\ar[r,shift right=2pt,swap,"\lambda'"]\ar[d,"h"]&G'\ar[r,"p'"]\ar[d,"g"]&F'\backslash G'\\
F\otimes_AG\ar[r,shift left=2pt,"\pr_2"]\ar[r,shift right=2pt,swap,"\lambda"]&G
\end{tikzcd}
\end{equation}
In this diagram, $\pr_2'$ and $\lambda'$ (resp. $\pr_2''$ and $\lambda''$) are obtained from $\pr_2$ and $\lambda$ by base change; the morphisms $g$ and $h$ are induced by the canonical injection $A\to A'$. We denote by $p'$ and $p''$ the canonical morphisms, and the morphisms $v_1,v_2$ and $w_1,w_2$ are induced by the two canonical injections from $A'$ to $A''$. Finally, as the construction of the quotient $F'\backslash G'$ commutes with base changes $f_1,f_2:\Spec(A'')\rightrightarrows\Spec(A')$, we have, with $\pi':F'\backslash G'\to\Spec(A')$ being the structural morphism, the canonical isomorphisms for $i=1,2$:
\[\tau_i:F''\backslash G''\stackrel{\sim}{\to} (F'\backslash G')\times_{\pi',f_i}\Spec(A''),\]
and the morphism $u_i$ is the composition of $\tau_i$ with the projection $(F'\backslash G')\times_{\pi',f_i}\Spec(A'')\to F'\backslash G''$.\par
If we have a diagram of the form (\ref{scheme alg group group quotient exists-3}), the first two rows and columns being exact, we then verify that $\coker(\pr_2,\lambda)$ exists if and only if $\coker(u_1,u_2)$ exists, and these two cokernels are identified. Now if follows from the compatibility of the formation of $F\backslash G$ with base extensions (as we have mentioned) that the composition morphism
\[(F'\backslash G')\times_{\pi',f_1}\Spec(A'')\stackrel{\tau_1^{-1}}{\to} F''\backslash G''\stackrel{\tau_2}{\to} (F'\backslash G')\times_{\pi',f_2}\Spec(A'')\]
is a descent data over $F'\backslash G'$ relative to $f:\Spec(A')\to\Spec(A)$. By (\ref{scheme alg group group quotient exists-2}) and (\cite{SGA1} \Rmnum{8}, 7.6), this descent data is effective, which means $\coker(u_1,u_2)$ exists.\par
To complete the proof of assertions (a), (b), (c) and (d)(\rmnum{1}) of \cref{scheme alg group group quotient exists} in the case where $F$ and $G$ are of finite type over $A$, it remains to study the quotient $F\backslash G$. According to \cref{scheme groupoid quasi-section lemma}, the assertions (b), (c) and (d)(\rmnum{1}) become true after a suitable base change $f:\Spec(A')\to\Spec(A)$, so they were true before the base change (cf. \cite{EGA4-2}, 2.6.1, 2.6.2 et 2.7.1). To see the second assertion of (a), i.e. that $F\backslash G$ is the cokernel of $(\pr_2,\lambda)$ in the category of ringed spaces, we can proceed as in  \ref{scheme groupoid quotient if quasi-section subsection}.\par
Finally, we prove assertion (d)(\rmnum{3}) of \cref{scheme alg group group quotient exists}, the proof of (b') and (d)(\rmnum{4}) will be postponed to \ref{scheme alg group group quotient and sheaf quotient paragraph}. Denote by $X=F\backslash G$ and let $d:F\times_AG\to G\times_AG$ be the morphism with components $\lambda$ and $\pr_2$. As $u$ is a closed immersion by \cref{scheme A-group homomorphism of local ft image prop}, and as $d=\sigma\circ(u\times\id_G)$, where $\sigma$ is the automorphism of $G\times_AG$ defined by $\sigma(x,y)=(xy,y)$, we see that $d$ is also a closed immersion. On the other hand, by (d)(\rmnum{1}), we have a Cartesian square
\[\begin{tikzcd}
F\times_AG\ar[r,"d"]\ar[d]&G\times_AG\ar[d,"p\times p"]\\
X\ar[r,"\Delta_X"]&X\times_AX
\end{tikzcd}\]
and $p$ hence also $p\times p$ is faithfully flat and locally of finite presentation. By fppf descent, as $d$ is a closed immersion, so is $\Delta_X$ (cf. \cref{scheme morphism descent by fpqc eg}), i.e. $X$ is separted.

\begin{remark}\label{scheme alg group monomorphism from flat group quotient exists}
In \cref{scheme alg group group quotient exists}, the hypothesis that $G$ is flat can be removed if $u:F\to G$ is a monomorphism.
\end{remark}

\begin{corollary}\label{scheme A-group quotient by subgroup section is invariant}
Let $A$ be a local Artinian ring, $G$ be an $A$-group locally of finite type, $H$ be a closed subgroup of $G$, flat over $A$. Denote by $p$ the morphism $g\to G/H$ and $\lambda$ (resp. $\pr_1$) the morphism $G\times H\to G$ defined by $\lambda(g,h)=gh$ (resp. the projection $G\times H\to G$). Then for any open subset $U$ of $G/H$, we have
\[\mathscr{O}(U)=\{\phi\in\mathscr{O}(p^{-1}(U)):\phi\circ\lambda=\phi\circ\pr_1\}\]
i.e. $\mathscr{O}(U)$ is the set of $\phi\in\mathscr{O}(p^{-1}(U))$ such that $\phi(gh)=g$ for any $A$-scheme $S$ and $g\in G(S)$, $h\in H(S)$.
\end{corollary}
\begin{proof}
In fact, as $p:G\to G/H$ is faithfully flat and locally of finite presentation, hence a universally effective epimorphism, this follows from \cref{category universal effective equivalence section is invariant}.
\end{proof}

\begin{remark}
If $A=\Spec(k)$ is the spectrum of an algebraically closed field, then we have $(F\backslash G)(k)=F(k)\backslash G(k)$ for the quotient $F\backslash G$ constructed in \cref{scheme alg group group quotient exists}. In fact, if $B$ is a faithfully flat $k$-algebra of finite presentation, then there exists a maximal ideal $\m$ such that $B/\m$ is a finite extension of $k$ by \cref{field ft algebra Zariski lemma}, which is therefore equal to $k$ since $k$ is algebraically closed. We then have a composition $k\to B\to k$, which is equal to the identity on $k$, and this implies that any section $F(B)\backslash G(B)$ is the inverse image of a section in $F(k)\backslash G(k)$, so in particular the morphism $X(k)/G(k)\to (X/G)(k)$ is an isomorphism.\par
In the general case, the equality $(F\backslash G)(k)=F(k)\backslash G(k)$ still holds if we interpret the left side as the fppf quotient sheaf defined by the action of $G$ on $F$. However, there may be orbits of $G(k)$ on $F(k)$ that are not closed. If so, then $(F\backslash G)(k)=F(k)\backslash G(k)$ cannot have a topology where all closed points are closed and for which the natural map $F(k)\to(F\backslash G)(k)$ is continuous, so $F\backslash G$ can not be a scheme.
\end{remark}

\paragraph{Passage to the quotient \texorpdfstring{$F\backslash G$}{FG} (general case)}\label{scheme alg group group quotient general case paragraph}

We now return to the general case of \cref{scheme alg group group quotient exists}, so $F$ and $G$ are not necessarily of finite type over $A$.\par
Let $G^\alpha$ be a connected component of $G$. We show that the saturation $\mathcal{S}(G^\alpha)$ of $G^\alpha$ for the equivalence relation defined by the groupoid $G_*$ is closed in $G$. By definition, this saturation is the image of $F\times_AG^\alpha$ under $\lambda$, hence is open in $G$ (recall that $\lambda$ is open). If $k$ is the residue field of $A$ and $\bar{k}$ is an algebraic closure of $k$, then it suffies to show that the image of $(F\times_AG^\alpha)\otimes_A\bar{k}$ under $\lambda\otimes_A\bar{k}$ is closed in $G\otimes_A\bar{k}$ (cf. \cref{scheme morphism descent by fpqc eg}). As $G^\alpha\otimes_A\bar{k}$ is a union of connected components of $G\otimes_A\bar{k}$, we are then reduced to the case where $A=k$ is an algebraically closed field. In this case, $\mathcal{S}(G^\alpha)$ is the union of the images of $G^\alpha$ under the left translations $\ell_{u(x)}$, where $x$ runs through closed points of $F$\footnote{If $x,y\in F$ are contained in the same connected component $C$ of $F$, then $\ell_{u(x)}(G^\alpha)$ and $\ell_{u(y)}(G^\alpha)$ is contained in the image of $F\times G^\alpha$, which is connected (since we have assumed that $k$ is algebraically closed). Therefore, we have $\ell_{u(x)}(G^\alpha)=\ell_{u(y)}(G^\alpha)$, and our assertion follows from the fact that any connected component contains a closed point.}. The assertion therefore follows from the fact that these images are connected components of $G$.\par

In particular, consider the identity component $G^0$ of $G$. Then $\mathcal{S}(G^0)$ contains evidently the image of $F$ under $u$, which is none other than the equivalent class of the identity of $G$. On the other hand, if $F^\beta$ is a connected component of $F$, then $F^\beta\times_AG^0$ is connected (\cref{scheme A-group connected component is geometric}) so that its image under $\lambda$ is contained in the connected component of $u(F^\beta)$ in $G$. In other words, $\mathcal{S}(G^0)$ is the union of the connected components that meet the image of $F$.\par
We also remark that the open subscheme of $G$ with $\mathcal{S}(G^0)$ as underlying topological space is a subgroup of $G$ (still denoted by $\mathcal{S}(G^0)$): the inversion morphism of $G$ preserves the image of $F$ and permutes the connected components of $G$ which meet this image. It then suffices to show that $\mu_G:G\times_AG\to G$ sends $S(G^0)\times_A\mathcal{S}(G^0)$ into $\mathcal{S}(G^0)$ and for this we can suppose that $A$ is an algebraically closed field (in fact, $\mathcal{S}(G^0)\otimes_A\bar{k}$ is identified with the saturation of $(G\otimes_A\bar{k})^0$ for the equivalence relation defined by $u\otimes_A\bar{k}$). In this case, if $G^\gamma$ and $G^\delta$ are connected components of $\mathcal{S}(G^0)$, then $G^\gamma\times_AG^\delta$ is connected and its image under $\mu_G$ meets the image of $F$; therefore, $u(G^\gamma\times_AG^\delta$) is contained in a connected component of $G$ meeting $u(F)$.\par

We therefore obtain that the groupoid $G_*$ induced by $u$ on $G$ is a direct sum of groupoids $\mathcal{S}(G_*^\alpha)$ induced by $G_*$ over the distinct clopen subsets of $G$ of the form $\mathcal{S}(G^\alpha)$. The cokernel of $G_*$ is then the direct sum of the cokernels of these groupoids $\mathcal{S}(G^\alpha_*)$, which we now study separately.\par
First consider the groupoid $\mathcal{S}(G^0_*)$ induced by $G_*$ over $\mathcal{S}(G^0)$. It is clear that $\mathcal{S}(G^0_*)$ is the groupoid with base $\mathcal{S}(G^0)$ defined by the homomorphism from $F$ to $\mathcal{S}(G^0)$ induced by $u$. The cokernel whose existence we want to prove is therefore identified with $F\backslash\mathcal{S}(G^0)$. On the other hand, consider the groupoid
\[\begin{tikzcd}[column sep=12mm]
G^0_2\ar[r,shift left=8pt,"\ell_0'"]\ar[r,shift right=8pt,swap,"\ell_2'"]\ar[r,"\ell_1'"description]&G^0_1\ar[r,shift left=2pt,"\ell_0"]\ar[r,shift right=2pt,swap,"\ell_1"]&G_0^0=G^0
\end{tikzcd}\]
induced by $\mathcal{S}(G^0)$ over $G^0$. If we recall the construction of the inverse image groupoid, the object denoted $Y_0\times_{X_0}X_1$ is none other than $F\times_AG^0$, so that $G^0_1$ is the inverse image of $G^0$ under the morphism $F\times_AG^0\to\mathcal{S}(G^0)$ induced by $\lambda$. 
We claim that this inverse image is $F_0\times_AG^0$, where we note by $F_0$ the inverse image of $G^0$ under $u$. In fact, since $G^0$ is clopen in $G$, its inverse image is also clopen in $F$. If $F^\beta$ is a connected component of $F$ contained in $F^0$, $F^\beta\times_AG^0$ is connected (\cref{scheme A-group connected component is geometric}) and $\lambda(F^\beta\times_AG^0)$ is contained in $G^0$. Conversely, if $F^\beta$ is a connected component of $F$ not contained in $F_0$, then the image $\lambda(F^\beta\times_AG^0)$ is still connected and contains $u(F^\beta)$. Since $u(F^\beta)$ is not contained in $G^0$, which is a connected component in $G$, we conclude that $\lambda(F^\beta\times_AG^0)$ does not meet $G^0$.\par
It then follows that the groupoid $G^0_*$ induced by $G_*$ over $G^0$ is the one defined by the homomorphism $F_0\to G^0$ induced by $u$. As $G^0$ (and hence $F_0$, since $u$ is \textit{quasi-compact}) is of finite type over $A$, by \ref{scheme alg group group quotient ft case paragraph}, $G_*^0$ possesses a cokernel which is none other than $F_0\backslash G^0$. We claim that $F_0\backslash G^0$ is identified with $F\backslash\mathcal{S}(G^0)$. In fact, the proof is similar to that of \cref{scheme groupoid quasi-section lemma}~(a). Consider the diagram
\[\begin{tikzcd}
\mathcal{S}(G^0)&F\times_AG^0\ar[l,swap,"v"]\ar[r,"\pr_2"]&G^0
\end{tikzcd}\]
where $v$ is the morphism induced by $\lambda$. As $\pr_2$ possesses a section, $\pr_2$ is a universally effective epimorphism (cf. \cite{SGA3-1} \Rmnum{4}, 1.12) so that $F_0\backslash G^0$ coincides with the cokernel $\coker(v_0,v_1)$, where
\[\begin{tikzcd}[column sep=12mm]
V_2\ar[r,shift left=8pt,"v_0'"]\ar[r,shift right=8pt,swap,"v_2'"]\ar[r,"v_1'"description]&V_1\ar[r,shift left=2pt,"v_0"]\ar[r,shift right=2pt,swap,"v_1"]&V=F\times_AG^0
\end{tikzcd}\]
is the inverse image of the groupoid $G_*^0$ under $\pr_2$ (cf. \cref{category groupoid cokernel under universal effective base change prop}), which is also the inverse image of $\mathcal{S}(G^0)$ under the composition morphism
\[F\times_AG^0\hookrightarrow F\times_A\mathcal{S}(G^0)\stackrel{\pr_2}{\to}\mathcal{S}(G^0).\]
Similarly, as $v$ is faithfully flat and quasi-compact (hence a universally effective epimorphism), $F\backslash\mathcal{S}(G^0)$ coincides with the cokernel of the inverse image of $\mathcal{S}(G_*^0)$ under $v$. Now this inverse image is isomorphic to $V_*$ by \cref{category groupoid base change by d_0 and d_1}, so the canonical inclusion $G_*^0$ in $\mathcal{S}(G_*^0)$ induces an isomorphism $F_0\backslash G^0\cong F\backslash\mathcal{S}(G^0)$.\par
Finally, we notice that the construction of $F\backslash\mathcal{S}(G^0)$ commutes with finite and locally free base changes, because this is true for $F_0\backslash G^0$.\par

It remains to construct the cokernel of the groupoid $\mathcal{S}(G_*^\alpha)$ for an arbitrary connected component of $G$. By \cref{scheme alg group flat closed point locally rational}, if we choose a large enough local $A$-algebra $A'$, finite and free over $A$, then $G^\alpha\otimes_AA'$ is the union of finitely many connected components $C^1,\dots,G^n$ of $G\otimes_AA'$, each of which has a strict rational point. For each $i$, there then exists a right translation $r_i$ of $G\otimes_AA'$ which sends $G^0\otimes_AA'$ to $C^i$. This translation induces an isomorphism of the groupoid $\mathcal{S}(G^0_*)\otimes_AA'$ to $\mathcal{S}(C^i_*)$, so that the latter has a cokernel. As $\mathcal{S}(G^\alpha_*)\otimes_AA'$ is the direct sum of a certain number of the $\mathcal{S}(C^i_*)$, it therefore possesses a cokernel. This cokernel is a direct sum of a certain number of copies of $(F_0\otimes_AA')\backslash(G^0\otimes_AA')$, so any finite subset of it is contained in an affine open subset; moreover, the construction of this cokernel commutes with finite and free base changes. We then conclude as in \ref{scheme alg group group quotient ft case paragraph} that this cokernel is of the form $Y\otimes_AA'$, where $Y$ is a cokernel of $\mathcal{S}(G^\alpha_*)$.\par
We have therefore constructed $F\backslash G$ and shown that it is a direct sum of schemes of finite type over $A$. The other assertions of \cref{scheme alg group group quotient exists} reduce directly to those concerning the groupoids $\mathcal{S}(G^\alpha_*)$. As in \autoref{scheme groupoid quotient if quasi-section subsection}, the second assertion of (a) follows from the first one and (b), (c), so it suffices to prove (b), (c) and (d)(\rmnum{1}). As $A'$ is a local, finite and free $A$-algebra, the morphism $A\to A'$ is faithfully flat and of finite presentation, so according to (\cite{SGA1} \Rmnum{8} 3.1, 4.6, 5.4), it suffices to check the corresponding assertions for the groupoid $\mathcal{S}(G^\alpha_*)\otimes_AA'$. But this is isomorphic to the direct sum of a finite number of copies of $\mathcal{S}(G_*^0)\otimes_AA'$, so we are reduced to the groupoid $\mathcal{S}(G^0_*)$. For this, we can still modify the proof established in \autoref{scheme groupoid quotient if quasi-section subsection}, as we have already done in this paragraph. We have therefore completed the proof of \autoref{scheme groupoid quotient if quasi-section subsection}. We have therefore completed the proof of \cref{scheme alg group group quotient exists}\par

With the notations of (\cite{SGA3-1}, \Rmnum{5} \S 9.a), we now consider a condition under which the formation of the cokernel of $X_*$ in $\Sch_{/S}$ commutes with a given base extension $\pi:S'\to S$. As the cokernel of $X_*$ and $X_*'$ are identified with the cokernel of the groupoid $U_*$ and $U'_*$ induced by $X_*$ and $X_*'$ over the quasi-sections $U$ and $U'$, respectively, we are then reduced to the case where $\Sch_{/S}$-groupoids verifying the hypothesis of \cref{scheme groupoid quotient by locally free finite prop}. If we denote by $Y$ the cokernel of $U_*$, $Y'=Y\times_SS'$ and $Y_1$ the cokernel of $U_*'$, we have seen in (\cite{SGA3-1}, \Rmnum{5} \S 9.a) that the canonical morphism $Y_1\to Y'$ is a homeomorphism (and even a universal homeomorphism); we can therefore identify $Y_1$ and $Y'$ as topological spaces. If $p:U\to Y$ is the canonical morphism and if $p':U'\to Y'$ is that induced by base change, we want the sequence of $\mathscr{O}_{Y'}$-modules
\begin{equation}\label{scheme groupoid quotient commutes with base change if locally quasi-section-1}
\begin{tikzcd}
\mathscr{O}_{Y'}\ar[r]&p'_*(\mathscr{O}_{U'})\ar[r,shift left=2pt]\ar[r,shift right=2pt]&p'_*(u_1')_*(\mathscr{O}_{U_1'})=p'_*(u_0')_*(\mathscr{O}_{U_1'})
\end{tikzcd}
\end{equation}
to be exact. As we are reduced to the hypothesis of \cref{scheme groupoid quotient by locally free finite prop}, $u_0$ and $u_1$ are finite and locally free; and by \cref{scheme groupoid quotient by locally free finite prop}, $p$ is integral. Then $p$ and $p\circ u_i$ are affine, hence separated and quasi-compact.\par
Therefore, if $S'$ is flat over $S$, it follows from \cref{scheme morphism qcsp base change formula} that (\ref{scheme groupoid quotient commutes with base change if locally quasi-section-1}) is identified with the inverse image of the sequence
\begin{equation}\label{scheme groupoid quotient commutes with base change if locally quasi-section-2}
\begin{tikzcd}
\mathscr{O}_{Y}\ar[r]&p_*(\mathscr{O}_{U})\ar[r,shift left=2pt]\ar[r,shift right=2pt]&p_*(u_1)_*(\mathscr{O}_{U_1})=p_*(u_0)_*(\mathscr{O}_{U_1})
\end{tikzcd}
\end{equation}
which is exact. An analogous reasoning is applicable if the groupoid $X_*$ possesses "locally" a quasi-section (cf. the proof of (\cite{SGA3-1}, \Rmnum{5} 7.1)), and we therefore obtain the following:

\begin{proposition}\label{scheme groupoid quotient commutes with base change if locally quasi-section}
The construction of the cokernel of $X_*$ commutes with any flat base change if $X_*$ possesses a quasi-section.
\end{proposition}

Now consider the case of the groupoid $G_*$ of \cref{scheme alg group group quotient exists} where we suppose that $F$ and $G$ are of finite type over $A$. By the proof in \ref{scheme alg group group quotient ft case paragraph}, there exists a local algebra $A'$, finite and free over $A$, such that the groupoid $G_*\otimes_AA'$ possesses quasi-sections. For any extension $T\to\Spec(A)$, the seqence
\[(F''\backslash G'')\times_{\Spec(A)}T\rightrightarrows (F'\backslash G')\times_{\Spec(A)}T\to(F\backslash G)\times_{\Spec(A)}T\]
induced from the diagram (\ref{scheme alg group group quotient exists-3}) is exact. If we suppose that $T$ is flat over $\Spec(A)$, then $(F''\backslash G'')\times_{\Spec(A)}T$ and $(F'\backslash G')\times_{\Spec(A)}T$ are identified, respectively, with the cokernel of the groupoids $(G_*\otimes_AA')\times_{\Spec(A)}T$ and $(G_*\otimes_AA')\times_{\Spec(A)}T$. The diagram induced from (\ref{scheme alg group group quotient exists-3}) by base change $T\to\Spec(A)$ then show that $(F\backslash G)\times_{\Spec(A)}T$ is identified with the cokernel of $G_*\times_{\Spec(A)}T$. An analogous reasoning is applicable in the general case (i.e. if $G$ and $F$ are locally of finite type over $A$), hence we obtain:

\begin{proposition}\label{scheme alg group group quotient and base change}
Under the hypothesis of \cref{scheme alg group group quotient exists}, for any flat $A$-scheme $T$, $(F\backslash G)\times_{\Spec(A)}T$ is identified with the quotient of $G\times_{\Spec(A)}T$ by $F\times_{\Spec(A)}T$.
\end{proposition}

\paragraph{Connections with \autoref{category equivalence relation and quotient section} and consequences}\label{scheme alg group group quotient and sheaf quotient paragraph}
We recall the notations and hypothesis of \cref{scheme alg group group quotient exists}. We then have the following commutative diagram
\[\begin{tikzcd}
F\times_AG\times_AG\ar[r,"\id_F\times\mu_G"]\ar[d,shift left=2pt,"\lambda\times\id_G"]\ar[d,shift right=2pt,swap,"\pr_2\times\id_G"]&F\times_AG\ar[d,shift left=2pt,"\lambda"]\ar[d,shift right=2pt,swap,"\pr_2"]\\
G\times_AG\ar[r,"\mu_G"]\ar[d,swap,"p\times\id_G"]&G\ar[d,"p"]\\
(F\backslash G)\times_AG\ar[r,dashed,"\rho"]&F\backslash G
\end{tikzcd}\]
which satisfies the equalities $\pr_2\circ(\id_F\times\mu_G)=\mu_G\circ(\pr_2\times\id_G)$ and $\lambda\circ(\id_F\times\mu_G)=\mu_G\circ(\lambda\times\id_G)$. Moreover, as $G$ is supposed to be flat over $A$, the left column is exact by \cref{scheme groupoid quotient commutes with base change if locally quasi-section}, so that $\mu_G$ induces a morphism of $A$-schemes:
\[\rho:(F\backslash G)\times_AG\to F\backslash G\]
This morphism $\rho$ induces a right action of $G$ on $F\backslash G$ as we immediately verify; moreover, the canonical morphism $G\to F\backslash G$ commutes to the right actions of $G$ on $G$ and $F\backslash G$. This proves the first assertion of (b') of \cref{scheme alg group group quotient exists}. By \cref{scheme A-group-object transitive prop}, we conclude that the connected components of $X=F\backslash G$ are of finite type, irreducible, and all of the same dimension. To evaluate this dimension, we can assume that $A=k$ is an algebraically closed field. By \ref{category group action in PSh paragraph}, the stabilizer of the $k$-point $p(e)$ is represented by the fiber $H=p^{-1}(p(e))$, and since $F\backslash G$ is the quotient of $G$ by $F$ in the category of ringed spaces, this fiber has underlying space $u(F)$, and since $\ker u$ is finite, we therefore have $\dim(H)=\dim(u(F))=\dim(F)$. By \cref{scheme A-group-object transitive prop}, we have therefore obtained that $\dim(X)=\dim(G)-\dim(F)$. This proves (b') of \cref{scheme alg group group quotient exists}.\par
If the homomorphism of $A$-groups $u:F\to G$ is a monomorphism, the above arguments can also be deduced from the results of \autoref{category equivalence relation and quotient section}. In fact, the canonical morphism $p:G\to F\backslash G$ is faithfully flat and open by \cref{scheme alg group group quotient exists}~(b) and (d)(\rmnum{1}); it is then covering for the fpqc topology (\cref{scheme topology on Sch by refining family prop}) and we can apply \cref{site quotient by universally effective subgroup G-action structure} and \cref{site quotient by universally effective normal subgroup group structure}. In particular, if we suppose in \cref{scheme alg group group quotient exists} that $F$ is a normal subgroup of $G$, then there exists a unique $A$-group structure over $F\backslash G$ such that $p:G\to F\backslash G$ is a homomorphism of $A$-groups. This proves assertion (d)(\rmnum{4}) of \cref{scheme alg group group quotient exists}.

We now review the assertions of \autoref{category equivalence relation and quotient section} in the present situation. The statements of \cref{site sheaf quotient by normal subgroup correspond} and \cref{site sheaf quotient by normal N-subgroup correspond} are translated as follows: Let $F$ and $G$ be two groups locally of finite type and flat over $A$, $F$ being a closed normal subgroup of $G$. The maps $H\mapsto\to F\backslash H$ and $H'\mapsto H'\times_{(F\backslash G)}G$ define a bijection correspondence between flat $A$-subgroups of $G$ containing $F$ and flat $A$-subgroups of $F\backslash G$. Under this bijection, closed (resp. normal) subgroups of $G$ containing $F$ correspond to closed (resp. normal) subgroups of $F\backslash G$. Moreover, since $p:G\to F\backslash G$ is faithfully flat, an $A$-subgroup $H$ of $G$ is flat over $A$ if and only if $F\backslash H$ is.\par

The result of \cref{site sheaf quotient by normal subgroup third isomorphism} implies the following: Let $F$, $H$ and $G$ be groups locally of finite type and flat over $A$; suppose that $F\sub H\sub G$, with $F$ closed in $G$ and normal in $H$. Then $F\backslash H$ acts freely on the left on $F\backslash G$, the quotient scheme $(F\backslash H)\backslash(F\backslash G)$ exists and we have a canonical isomorphism of scheme acted by $G$:
\[(F\backslash H)\backslash(F\backslash G)=H\backslash G.\]

Finally, \cref{site sheaf quotient by normal subgroup second isomorphism} implies the following: Let $F,H$ and $G$ are groups locally of finite type and flat over $A$; suppose that $F$ is contained in $G$ and closed, normal in $G$, that $H$ is contained in $G$ and that $F\cap H$ is flat over $A$. Let $F\rtimes_AH$ be the semi-product $A$-group of $F$ by $H$, $u:F\cap H\to F\rtimes_AH$ be the monomorphism defined by $x\mapsto (x^{-1},x)$, and $F\cdot H$ be the quotient $(F\cap H)\backslash(F\rtimes_AH)$. Then there is a canonical isomorphism
\[F\backslash(F\cdot H)=(F\cap H)\backslash H.\]

\paragraph{Factorization for a quasi-compact group homomorphism}
Let $u:G\to H$ be a quasi-compact morphism between $A$-groups locally of finite type such that the kernel $N$ of $u$ is flat over $A$. In this case, by \cref{scheme alg group monomorphism from flat group quotient exists}, the quotient $A$-group $C=N\backslash G$ exists and the morphism $p:G\to C$ is faithfully flat and locally of finite presentation. On the other hand, by \cref{site quotient by normal subgroup universal prop}, $u$ induces a monomorphism $i:C\to H$, which is quasi-compact (because $u$ is quasi-compact and $p$ is surjective, cf. \cref{scheme morphism qc cancelled by surjective}), hence it is a closed immersion by \cref{scheme A-group homomorphism of local ft image prop}. We then obtain the following proposition:

\begin{proposition}\label{scheme alg group morphism factorization if ker flat}
Let $u:G\to H$ be a quasi-compact morphism between $A$-groups locally of finite type such that $N=\ker u$ is flat over $A$. Then we have a factorization
\[\begin{tikzcd}
G\ar[rr,"u"]\ar[rd,swap,"p"]&&H\\
&N\backslash G\ar[ru,swap,"i"]&
\end{tikzcd}\]
where $p$ is faithfuly flat, locally of finite presentation, and $i$ is a closed immersion.
\end{proposition}

Suppose moreover that $G$ is flat over $A$. Then by \cref{scheme alg group monomorphism from flat group quotient exists}, $C=N\backslash G$ is flat over $A$ and hence the quotient $X=C\backslash H$ exists in $\Sch_{/A}$ and represents the fppf sheaf quotient $\widetilde{C\backslash H}$, and $q:H\to X$ is a $C$-torsor. Therefore, denoting by $e:\Spec(A)\to G$ the unit section of $G$, $i$ induces an isomorphism of fppf sheaves between $\widetilde{C}$ and the fiber product of $q$ and $q\circ e:\Spec(A)\to X$, which is represented by a closed subscheme $H$. We then conclude that $i$ is an isomorphism from $C$ to a closed subgroup $K$ of $H$ (equal to the stabilizer of the $A$-point $q\circ e$ of $X$). (This provides another proof of the fact that a quasi-compact monomorphism between $A$-groups locally of finite type is a closed immersion.)\par
If $C$ is a normal subgroup of $H$, then the $A$-group $\widebar{H}=C\backslash H$ is a cokernel in the category of $A$-groups of the morphism $u:G\to H$, and $K$ is the kernel of the morphism $H\to\widebar{H}$. If $G$ and $H$ are abelian $A$-groups, then $K$ is the image of $u$ in the category of abelian $A$-groups and $C=N\backslash G$ is the coimage of $u$. In view of the isomorphism $C\cong K$, we obtain:

\begin{theorem}\label{scheme abelian k-algebraic group cat is abelian}
Let $k$ be a field. The category of abelian $k$-algebraic groups is abelian.
\end{theorem}
\begin{proof}
In fact, since $k$ is a field, $\ker u$ is always flat over $k$.
\end{proof}

We note that the full subcategory of affine abelian algebraic $k$-groups is thick. In fact, consider an exact sequence of abelian algebraic $k$-groups:
\[\begin{tikzcd}
1\ar[r]&N\ar[r]&G\ar[r]&G/N\ar[r]&1
\end{tikzcd}\]
If $G$ is affine, it is clear that so is $N$, and $G/N$ is also affine by Chevalley theorem (cf. \cite{SGA3-1}, $\Rmnum{6}_B$, 11.17). Conversely, if $N$ and $G/N$ are affine, then $G$ is also affine by (cf. \cite{SGA3-1}, $\Rmnum{6}_B$, 9.2(\rmnum{8})). We therefore conclude that:

\begin{corollary}\label{scheme affine abelian k-algebraic group cat is abelian}
Let $k$ be a field. The category of affine abelian $k$-algebraic groups is abelian.
\end{corollary}

\begin{remark}
The category of affine abelian groups (not necessarily of finite type) is also abelian; this is deduced in (\cite{SGA3-1}, $\Rmnum{6}_B$, 11.17 et 11.18.2) (cf. \cite{Demazure} \S\Rmnum{3}.3, 7.4), see also (\cite{SGA3-1}, $\Rmnum{7}_B$) for a proof using formal groups.
\end{remark}

\paragraph{Complements on the identity component}
Let $G$ be a group locally of finite type and flat over a local Artinian ring $A$. We have seen in \cref{scheme alg group identity component prop} that the identity component $G^0$ is an open and normal subgroup of $G$, hence also flat over $A$. By \cref{scheme alg group group quotient exists}, $G^0\backslash G$ is a flat $A$-group. Moreover, as each component $G^\alpha$ of $G$ is saturated for the equivalence relation defined by $G^0$, $G^0\backslash G$ is the direct sum of these $G^0\backslash G^\alpha$\footnote{The saturation $\mathcal{S}(G^\alpha)$ under this equivalence relation is the image of $G^\alpha\times G^0$ under multiplication, which is equal to $G^\alpha$ since it is connected ($G^0$ is geometrically connected) and contains $G^0$.}. In particular, the identity component of $G^0\backslash G$ is none other than $G^0\backslash G^0\cong\Spec(A)$, and hence $G^0\backslash G\to\Spec(A)$ is a local isomorphism at the identity. Therefore, $G^0\backslash G$ is \'etale over $\Spec(A)$ by \cref{scheme alg group morphism prop iff point}. If $A=k$ is an algebraically closed field, then $G^0\backslash G$ is a constant $k$-group, operating simply transitively on the set of components connected of $G$ (in particular, if $G$ is of finite type, then $G^0\backslash G$ is finite). The group $G^0\backslash G$ is also denoted by $\pi_0(G)$, together with the canonical morphism $\pi:G\to\pi_0(G)$. This notation is justified by the following characterization:

\begin{proposition}\label{scheme alg group pi_0 universal prop}
Let $G$ be a group locally of finite type and flat over a local Artinian ring $A$. Then $(\pi_0(G),\pi)$ is the satisfies the following universal property: for any \'etale $k$-group $H$ and any homomorphism $u:G\to H$, there exists a unique homomorphism $\tilde{u}:G$ such that $u=\tilde{u}\circ\pi$. Moreover, the fibers of $\pi$ are the connected components of $G$, and its kernel is $G^0$.
\end{proposition}
\begin{proof}
The universal property for $\pi_0(G)$ follows from \cref{site quotient by normal subgroup universal prop}, since any \'etale $k$-group $H$ is discrete. By \cref{scheme alg group group quotient exists}, the morphism $\pi:G\to\pi_0(G)$ is identified with the quotient map on the underlying space defined by the equivalence relation induced by $G^0$, so the fibers of $\pi$ are connected components of $G$. Its kernel is the fiber at the unit section, which is $G^0$.
\end{proof}

\begin{proposition}\label{scheme alg group extension of connected prop}
Let $1\to H\to G\to Q\to 1$ be an exact sequence of algebraic groups over a field $k$. 
\begin{enumerate}
    \item[(a)] If $H$ and $Q$ are connected, then $G$ is connected.
    \item[(b)] If $G$ is connected, then $Q$ is connected.
\end{enumerate}
\end{proposition}
\begin{proof}
As the morphism $G\to Q$ is surjective, assertion (b) follows immediately. To prove (a), we note that if $H$ is connected, then it maps to $1$ in $\pi_0(G)$, so the morphism $G\to\pi_0(G)$ factors through $Q$. By \cref{scheme alg group pi_0 universal prop}, this morphism then factors through $\pi_0(Q)$:
\[G\to Q\to\pi_0(Q)\to\pi_0(G).\]
Since $\pi_0(Q)$ is trivial, it follows that $\pi_0(G)$ is trivial, so $G$ is connected.
\end{proof}

\paragraph{The quotient \texorpdfstring{$G_\red\backslash G$}{G}}
Now let $k$ be a perfect field and $G$ be a $k$-group locally of finite type. We have remarked in \autoref{scheme algebraic group preliminary remark subsection} that $G_\red$ is then a subgroup of $G$. Moreover, the equivalence class of the identity of $G$ for the left action of $G_\red$ on $G$ is the whole space $G$. Therefore, by \cref{scheme alg group group quotient exists}, we obtain:

\begin{proposition}\label{scheme alg group over perfect G/G_red char}
Let $k$ be a perfect field and $G$ be a $k$-group locally of finite type. Then the $k$-scheme $G_\red\backslash G$ is the spectrum of a local Artinian $k$-algebra, with residue field $k$.
\end{proposition}
\begin{proof}
In fact, $G_\red\backslash G$ reduces to a point, with residue field $k$, and is of finite type over $k$. It is therefore the spectrum of a local Artinian $k$-algebra (\cref{scheme algebraic Artinian iff}).
\end{proof}

\begin{proposition}\label{scheme alg group over perfect flat morphism iff G/G_red}
Let $u:F\to G$ be a morphism between groups locally of finite type over a perfect field $k$. Then the following assertions are equivalent:
\begin{enumerate}
    \item[(\rmnum{1})] $u$ is flat.
    \item[(\rmnum{2})] $u^0:F^0\to G^0$ is dominant and the morphism $\tilde{u}:F_\red\backslash F\to G_\red\backslash G$ induced by $u$ is flat. 
\end{enumerate}
\end{proposition}
\begin{proof}
Consider the following commutative diagram
\[\begin{tikzcd}
F\ar[r,"p"]\ar[d,swap,"u"]&F_\red\backslash F\ar[d,"\tilde{u}"]\\
G\ar[r,"q"]&G_\red\backslash G
\end{tikzcd}\]
where $p$ and $q$ are canonical projections. By \cref{scheme alg group group quotient exists}~(d), $p$ and $q$ are faithfully flat; therefore, if $u$ is flat, then $q\circ u=\tilde{u}\circ p$ is flat, hence so is $\tilde{u}$. On the other hand, since $F^0$ and $G^0$ are open and irreducible in $F$ and $G$, respectively, the induced morphism $u^0:F^0\to G^0$ is flat, hence dominant by (\cite{EGA4-2}, 2.3.5).\par
Conversely, suppose that $\tilde{u}$ is flat and $u^0$ is dominant. As $F^0$ and $G^0$ are irreducible, $\tilde{u}^0$ sends the generic point $\xi$ of $F^0$ to the generic point $\eta$ of $G^0$\footnote{Let $y=\tilde{u}^0(\xi)$ be the image of $\xi$ and consider the generalization $\eta\rightsquigarrow y$. By (\cite{EGA4-2}, 2.3.5), there exists a generalization $x\rightsquigarrow \xi$ such that $\tilde{u}^0(x)=\eta$. Since $\xi$ is generic in $F^0$, we must have $x=\xi$, so $\eta=\tilde{u}^0(\xi)$.}. Let $R$ be the local Artinian $k$-algebra such that $G_\red\backslash G=\Spec(R)$, and $\m$ be its maximal ideal. We have a local homomorphism of local rings $R\to\mathscr{O}_{G,\eta}\to\mathscr{O}_{F,\xi}$. Note that we have a Cartesian square:
\[\begin{tikzcd}
G_\red\ar[d]\ar[r]&G\ar[d,"q"]\\
\Spec(R/\m)\ar[r]&\Spec(R)
\end{tikzcd}\] 
and hence $\mathscr{O}_{G,\eta}/\m\mathscr{O}_{G,\eta}\cong\mathscr{O}_{G_\red,\eta}=\kappa(\eta)$, so that $\mathscr{O}_{F,\xi}/\m\mathscr{O}_{F,\xi}$ is flat over $\mathscr{O}_{G,\eta}/\m\mathscr{O}_{G,\eta}$. On the other hand, as $q$ and $\tilde{u}\circ p$ are flat, $G$ and $F$ are flat over $R$. Therefore, by the local criterion of flatness (cf. \cite{EGA4-3}, 11.3.10.2), $\mathscr{O}_{F,\xi}$ is flat over $\mathscr{O}_{G,\eta}$, that is, $u$ is flat at the point $\xi$, so it is flat by \cref{scheme A-group action orbit map flat iff at point}.
\end{proof}

\paragraph{Complements on $k$-groups not necessarily of finite type}
We now generalise the results of this subsection to groups not necessarily of finite type over a fixed field $k$. Before this, let point out the following result.

\begin{lemma}\label{scheme k-group decompose as maximal point product}
Let $G$ be a $k$-group. For any $x\in G$, there exists a point $u\in G\times G$ such that $\mu(u)=x$ and the two projections $p_1(u)$ and $p_2(u)$ are maximal points of $G$.
\end{lemma}
\begin{proof}
Let $K=\kappa(x)$. As the projection $G_K\to G$ is flat, it sends maximal points to maximal points (cf. \cite{EGA4-2}, 2.3.5), we are reduced to the case where $x$ is rational. Then the translation $\ell_x$ (resp. $r_x$) gives a morphism $G\to G\times G$, $g\mapsto(\ell_x(g^{-1}),g)$ (resp. $g\mapsto(g,r_x(g^{-1}))$) which induces an isomorphism from $G$ to $\mu^{-1}(x)$. Then, if $u$ is a maximal point of $\mu^{-1}(x)$, $p_1(u)$ and $p_2(u)$ are maximal points of $G$ (since $p_1$ and $p_2$ are flat).
\end{proof}

\begin{corollary}\label{scheme k-group morphism qc dominant surjective}
Let $f:G\to H$ be a quasi-compact and dominant morphism of $k$-groups.
\begin{enumerate}
    \item[(a)] $f$ is surjective.
    \item[(b)] If $H$ is reduced, $f$ is faithfully flat.
\end{enumerate}
\end{corollary}
\begin{proof}
We denote by $\mu_G$ (resp. $\mu_G$) the multiplication of $H$ (resp. $G$). Let $h\in H$, then by \cref{scheme k-group decompose as maximal point product}, there exists $u\in H\times H$ such that $\mu_H(u)=h$ and that $\alpha=p_1(u)$ and $\beta=p_2(u)$ are maximal points of $H$. As $f$ is quasi-compact and dominant, $f^{-1}(\alpha)$ and $f^{-1}(\beta)$ are nonempty (\cref{scheme morphism qs dominant char}), and hence there exists $v\in G\times G$ such that $(f\times f)(v)=u$ (\cref{scheme fiber product inverse image char by composition extension}). Then $g=\mu_G(v)$ satisfies $f(g)=h$, so $f$ is surjective.\par
Suppose morover that $H$ is reduced. Then $\mathscr{O}_{H,\alpha}$ is a field, and we have $f^{-1}(\alpha)\neq\emp$, so $f$ is flat at a point $\xi$ of $f^{-1}(\alpha)$, hence flal by \cref{scheme A-group action orbit map flat iff at point}.
\end{proof}

Recall that a morphism $f:X\to Y$ is called \textbf{scheme-theoretic dominant} if it satisfies the following condition: for any open subset $U$ of $Y$, if $Z$ is a closed subscheme of $U$ such that the morphism $f^{-1}(U)\to U$ factors through $Z$, then $Z=U$. If $f$ is quasi-compact and quasi-separated, this is equivalent to saying that the scheme-theoretic image of $X$ under $f$ is $Y$ (cf. \cref{scheme theoretic image exist if}).

\begin{proposition}\label{scheme k-group morphism qc factorization by image}
Let $f:H\to G$ be a quasi-compact morphism of $k$-groups. Then the scheme-theoretic image of $f$\footnote{This scheme-theoretic image exists by \cref{scheme theoretic image exist if}, since $f$ is separated (recall that $G$ and $H$ are both separated).} is a closed subgroup $H'$ of $G$, and $f$ factors into
\[\begin{tikzcd}
H\ar[rr,"f"]\ar[rd,swap,"f'"]&&G\\
&H'\ar[ru,swap,"i"]
\end{tikzcd}\]
where $f'$ is scheme-theoretic dominant, quasi-compact and surjective.
\end{proposition}
\begin{proof}
Denote by $c_G$ and $\mu_G$ (resp. $c_H$ and $\mu_G$) the inversion and multiplication of $G$ (resp. $H$). Then $c_G\circ f=f\circ c_H$ factors through the closed subscheme $i(H')$, whence $H'\sub c_G(H')$ and hence $H'=c_G(H')$ (since $c_H^2=\id_H$). Similarly, as $f\circ\mu_H=\mu_G\circ(f\times f)$ factors through $H'$, $f\times f$ factors through the closed subscheme $\mu_G^{-1}(H')$ of $G\times G$. On the other hand, as the formation of scheme-theoretic image commutes with flat base change (\cref{scheme morphism qcsp base change formula} and \cref{scheme theoretic image exist if}), the scheme-theoretic image of $f\times\id_H$ (resp. $\id_{H'}\times f$) is $H'\times H$ (resp. $H'\times H'$). Therefore, by the transitivity of scheme-theoretic image (\cref{scheme theoretic image transitivity}), the scheme-theoretic image of $f\times f$ is $H'\times H'$, which is then contained in $\mu_G^{-1}(H')$, i.e. the restriction of $\mu_G$ to $H'\times H'$ factors through $H'$. This shows that $H'$ is a closed subgroup of $G$. We denote by $i:H'\hookrightarrow G$ the inclusion. Then $f=i\circ f'$, where $f':H\to H'$ is scheme-theoretic dominant and quasi-compact (since $f$ is quasi-compact and $i$ is separated). By \cref{scheme k-group morphism qc dominant surjective}, $f'$ is surjective.
\end{proof}

We can now state the following theorem due to D. Perrin.
\begin{theorem}[\textbf{D. Perrin}]\label{scheme k-group qc is projective limit and quotient}
Let $G$ be a quasi-compact $k$-group, then
\begin{enumerate}
    \item[(a)] $G$ is the projective limit of a filtered system $(G_i)$ of $k$-groups of finite type (where the transition morphisms $u_{ij}:G_j\to G_i$ are affine for $i$ large enough) and the morphisms $G\to G_i$ are faithfully flat (and affine for $i$ large enough).
    \item[(b)] Let $H$ be a closed sub-$k$-group of $G$. Then the fpqc quotient sheaf $\widetilde{G/H}$ is a $k$-scheme in the following two cases:
    \begin{enumerate}
        \item[(\rmnum{1})] The immersion $H\to G$ is of finite presentation; in this case, $G/H$ is of finite type over $k$.
        \item[(\rmnum{2})] $H$ is normal in $G$.
    \end{enumerate}
\end{enumerate}
\end{theorem}

\begin{corollary}\label{scheme k-group morphism qc scheme dominant is faithfully flat}
Let $f:G\to H$ be a quasi-compact morphism of $k$-groups. If $f$ is scheme-theoretic dominant, it is faithfully flat (this is the case if $H$ is affine and the morphism $f^\#:\mathscr{O}(H)\to\mathscr{O}(G)$ is surjective). 
\end{corollary}
\begin{proof}

\end{proof}

\begin{corollary}\label{scheme k-group morphism qc quotient representable}
Let $u:G\to H$ be a quasi-compact morphism of $k$-groups and $N=\ker u$.
\begin{enumerate}
    \item[(a)] The fpqc quotient sheaf $\widetilde{G/N}$ is represented by a $k$-group $G/N$, and $u$ factors into
    \[\begin{tikzcd}
    G\ar[rr,"u"]\ar[rd,swap,"p"]&&G\\
    &G/N\ar[ru,swap,"i"]
    \end{tikzcd}\]
    where $p$ is faithfully flat and $i$ is a closed immersion.
    \item[(b)] If $u$ is a monomorphism, it is a closed immersion; if $u$ is scheme-theoretic dominant, then it is faithfully flat.
\end{enumerate}
\end{corollary}

\begin{corollary}\label{scheme abelian k-group qc cat is abelian}
The category of abelian quasi-compact $k$-groups is abelian.
\end{corollary}

\begin{corollary}\label{scheme alg group over char 0 geometric reduced}
If $\char(k)=0$, any $k$-group $G$ is geometrically reduced.
\end{corollary}
\begin{proof}

\end{proof}

\begin{corollary}
Let $G$ be a quasi-compact $k$-group and suppose that $k$ is algebraically closed. 
\begin{enumerate}
    \item[(a)] Let $f:G\to H$ be a faithfully flat morphism of $k$-groups, then the induced map $G(k)\to H(k)$ is surjective.
    \item[(b)] The set of rational points is dense in $G$. 
\end{enumerate}
\end{corollary}

\section{Generalities on group schemes}
\subsection{Open properties for group morphisms}
In this subsection, $S$ denotes an arbitrary scheme; an $S$-group scheme is simply called an $S$-group. Given an $S$-group $G$, we denote by $e:S\to G$ the unit section, $c:G\to G$ the inversion and $\mu$ the multiplication morphism $G\times_SG\to G$. For any $S$-scheme $X$, we denote by $\pi$ or $\pi_X$ the structural morphism $X\to S$.\par
Given a property $\mathcal{P}$ for a morphism of $S$-schemes $u:X\to Y$, we say that $\mathcal{P}$ is stable under base change if, for any $u$ verifying $\mathcal{P}$, so is the morphism $u_{(Y')}$ for any $S$-morphism $Y'\to Y$. We say that $\mathcal{P}$ is \textbf{local for the topology $\mathcal{T}$} if $\mathcal{P}$ verifies the following conditions:
\begin{enumerate}
    \item[(a)] $\mathcal{P}$ is stable under base change,
    \item[(b)] if $\{Y_i\to Y\}$ is a covering family of $S$-morphisms for the topology $\mathcal{T}$ and each $u_{(Y_i)}$ verifies $\mathcal{P}$, then $u$ verifies $\mathcal{P}$.
\end{enumerate}

\begin{proposition}\label{scheme group restriction to locus of fpqc local prop}
Let $\mathcal{P}$ be a property for a morphism of $S$-schemes which is local for the fpqc topology. Let $u:G\to H$ be a morphism of $S$-groups, and suppose that $G$ is flat and universally open over $S$. Let $W$ be the largest open subset of $H$ over which $u$ verifies the property $\mathcal{P}$ and put $V=u^{-1}(W)$. Then $U=\pi_G(V)$ is an open subset of $S$ and $V$ is an open subgroup of $G|_U$ (we use $G|_U$ to denote the $U$-group $\pi^{-1}_G(U)$ induced by $G$).
\end{proposition}
\begin{proof}
The existence of a largets open subset $W$ of $H$ over which $u$ verifies $\mathcal{P}$ follows from that fact that $\mathcal{P}$ is local for the Zariski topology. Since $\pi_G$ universally open, $\pi_G(V)$ is an open subset of $S$, and it suffices to show that $V$ is a subgroup of $G|_U$. We can then assume that $U=S$.\par
Let $G'=G\times_SV$, $H'=H\times_SV$, $V'=V\times_SV$, $W'=W\times_SV$ and $u'=u_{(V)}$; let $W_1'$ be the largest open subset of $H'$ over which $u'$ satisfies $\mathcal{P}$. Since $V$ is flat and universally open over $S$, so is $H'$ over $H$, and \cref{scheme locus of fpqc local prop stable under base change} below shows that $W_1'=W'$. Consider then the automorphism $a$ (resp. $b$) of the $V$-scheme $G'$ (resp. $H'$) defined by
\[a(g,v)=(g\cdot v^{-1},v),\quad (\text{resp. }b(h,v)=(h\cdot u(v^{-1}),v)),\]
where $g\in G(T)$, $v\in G(T)$ and $h\in H(T)$, for any $T\to S$. It is clear that $u'\circ a=b\circ u'$, so $W'$ is stable under $b$ and hence $V'$ is stable under $a$. But this then implies that $V$ is a subgroup of $G$.
\end{proof}

\begin{lemma}\label{scheme locus of fpqc local prop stable under base change}
Let $\mathcal{P}$ be a property for a morphism of $S$-schemes which is local for the fpqc topology. Consider the following Cartesian diagram of $S$-schemes:
\[\begin{tikzcd}
X'\ar[d]\ar[r,"f'"]&Y'\ar[d,"g"]\\
X\ar[r,"f"]&Y
\end{tikzcd}\]
where $g$ is flat and open. Let $W$ (resp. $W'_1$) be the largest open subset of $Y$ (resp. $Y'$) over which $f$ (resp. $f'=f_{(Y')}$) verifies $\mathcal{P}(u)$. Then $W_1'=W\times_YY'$.
\end{lemma}
\begin{proof}
Put $W'=W\times_YY'$; since $\mathcal{P}$ is table under base change, we have, we have $W'\sub W'_1$. As $g$ is open, $W_1=g(W'_1)$ is open in $Y$. Put $V_1=f^{-1}(W_1)$ and $V_1'=V_1\times_{W_1}W_1'$; it is clear that $V_1'=f'^{-1}(W_1')$. Since $g$ is flat and open, the morphism $W_1'\to W_1=g(W_1')$ induced by $g$ is faithfully flat and open, hence covering for the fpqc topology (cf. \cref{scheme topology on Sch by refining family prop}). Since the morphism $V_1'\to W_1'$ induced by $f'$ verifies $\mathcal{P}$, so does the morphism $V_1\to W_1$ induced by $f$, and hence $W_1\sub W$. We then conclude that $W_1'\sub g^{-1}(W_1)\sub g^{-1}(W)=W'$, so $W'=W_1'$.
\end{proof}

\begin{example}
The following properties for a morphism are local for the fpqc topology: flat, (universally) open, (locally) of finite type, of finite presentation, quasi-finite (cf. \cite{EGA4-2}, 2.5.1, 2.6.1 et 2.7.1), unramified, smooth, \'etale (EGA \cite{EGA4-4}, 17.7.3).
\end{example}

\begin{remark}
We note that the proof of \cref{scheme group restriction to locus of fpqc local prop} in fact only uses basis changes by flat morphisms, so the proposition applies to any property satisfying condition (b) for the fpqc topology and stable under base changes by flat morphisms (eg. quasi-compact and dominant).\par
Of course, we can state an analogous proposition concerning the properties local for a topology $\mathcal{T}$ finer than the Zariski topology, with the required condition on $G$ being then that $\pi_G$ is universally open and covering for the $\mathcal{T}$ topology. In particular, if $G$ is flat and locally of finite presentation over $S$, we have an analogous statement for properties stable under base changes by flat morphisms locally of finite presentation and satisfying condition (b) with respect to the fpqc topology (for example, regular, reduced, Cohen-Macaulay, etc., cf. \cite{EGA4-2}, 6.8).
\end{remark}

\begin{proposition}\label{scheme group morphism restriction to local fp smooth uo locus}
Let $G$ and $H$ be $S$-groups and $u:G\to H$ be a morphism of $S$-groups.
\begin{enumerate}
    \item[(a)] Suppose that $G$ or $H$ is flat over $S$ and $G$ or $H$ is locally of finite presentation over $S$, and let $V$ be the largest open subset of $G$ such that the restriction of $u$ to $V$ is flat and locally of finite presentation (resp. smooth, resp. \'etale). Then $U=\pi_V(V)$ is an open subset of $S$ and $V$ is an open subgroup of $G|_U$.
    \item[(b)] Suppose that $G$ or $H$ is universally open over $S$, and let $V$ be the largest open subset of $G$ such that the restriction of $u$ to $V$ is universally open\footnote{We note that $V$ is the largest open subset contained in the set $E$ of points of $G$ over which $u$ is universally open, but $E$ may not be open, as the following example shows: let $k$ be a field, $H=S=\Spec(A)$, where $A=k[x]$, and $G$ be the $S$-group $\Spec(A[y]/(xy))$; then $E$ equals to the unit section of $G$, which is not open.}. Then $U=\pi_V(V)$ is an open subset of $S$ and $V$ is an open subgroup of $G|_U$.
\end{enumerate}
\end{proposition}
\begin{proof}
To prove (a), we fist note that the restriction $\pi_V$ of $\pi_G$ to $V$ is flat and locally of finite presentation: if $\pi_G$ is flat (resp. locally of finite presentation), then so is its restriction $\pi_V$; on the other hand, if $\pi_H$ is flat (resp. locally of finite presentation), then so is $\pi_V$ since we have $\pi_V=\pi_H\circ u$. Hence, in any case the morphism $\pi_V$ is flat and locally of finite presentation, whence universally open (\cite{EGA4-2}, 2.4.6). Let $U=\pi_V(V)$, which is then open in $S$. It suffices to show that $V$ is an open subgroup of $G|_U$, and we may assume that $U=S$.\par
Put $G'=G\times_SV$, $H'=H\times_SV$, $V'=V\times_SV$ and $u'=u\times_S\id_V$. Then since $V$ is flat and locally of finite presentation over $S$, so is $H'$ over $H$. By (\cite{EGA4-4}, 17.4), $V'$ is then the largest open subset of $G'$ such that the restriction of $u'$ to $V'$ is flat and locally of finite presentation (resp. smooth, resp. \'etale). Consider the automorphisms $a$ and $b$ defind in the proof of \cref{scheme group restriction to locus of fpqc local prop}; then $V'$ is stable under $a$, hence $V$ is a subgroup of $G$.\par
Now consider (b). The restriction $\pi_V$ of $\pi_G$ to $V$ is a universally open morphism, either because it is the restriction of $\pi_G$ or is the composition of $u$ and $\pi_V$. Again, let $U=\pi_V(V)$, it suffices to show that $V$ is an open subgroup of $G|_U$, for which we can assume that $U=S$.\par
Define $G',H',V',u'$ as before; then $\pi_V:V\to S$ is surjective and universally open, hence so is $G'\to G$. Therefore, by (\cite{EGA4-3}, 13.3.4(\rmnum{1}) et (\rmnum{2})), $V'$ is the largest open subset of $G'$ over which $u'$ is universally open. The same arguments then apply to show that $V'$ is stable under $a$, so $V$ is a subgroup of $G$.
\end{proof}

\begin{corollary}\label{scheme group local fp smooth uo locus is open subgroup}
Let $G$ be an $S$-group and $V$ be the largest open subset of $G$ which is flat and locally of finite presentation (resp. smooth, \'etale, universally open) over $S$. Then $U=\pi_V(V)$ is open in $S$ and $V$ is an open subgroup of $G|_U$.
\end{corollary}
\begin{proof}
It suffices to apply \cref{scheme group morphism restriction to local fp smooth uo locus} to the trivial $S$-group $H$ and where $u$ is the morphism $G\to H$, bacause then $\pi_H$ is an isomorphism and $\pi_G=\pi_H\circ u$.
\end{proof}

\begin{corollary}\label{scheme group local fp smooth uo at unit section iff open subgroup}
Let $G$ be an $S$-group, if there exists a neighborhood $V$ of the unit section with one (or more) of the following properties: $V$ is flat and locally of finite presentation (resp. smooth, \'etale, universally open) over $S$, then there exists an open subgroup of $G$ with the same properties.
\end{corollary}
\begin{proof}
It suffices to apply \cref{scheme group local fp smooth uo locus is open subgroup}, since with the notations of \cref{scheme group morphism restriction to local fp smooth uo locus} we have $e(S)\sub V$, so $U=S$.
\end{proof}

\begin{proposition}\label{scheme group locus on base and group prop}
Let $u:G\to H$ be a morphism of $S$-groups.
\begin{enumerate}
    \item[(a)] Suppose that $G$ (resp. $H$) is flat and of finite presentation (resp. flat and of finite type) over $S$ at the unit section\footnote{The hypothesis here is that $G$ is flat and of finite presentation over $S$ at the unit section of $G$ \textit{or} $H$ is flat and of finite presentation over $S$ at the unit section of $G$; and the conclusion is that the sets $U_{\mathrm{flat}}\supset U_{\mathrm{smooth}}\supset U_{\mathrm{\'et}}$ are then open in $S$. The same interpretation is valid in (b).}. Then the sets
    \[U_{\mathrm{flat}}\supset U_{\mathrm{smooth}}\supset U_{\mathrm{\'et}}\]
    formed by points $s\in S$ such that $u_s$ is flat (resp. smooth, \'etale), are open in $S$.\par
    If $G$ (resp. $H$) is flat and locally of finite presentation (resp. flat and of finite type) over $S$, then the set $V_{\mathrm{flat}}$ (resp. $V_{\mathrm{smooth}}$, $V_{\mathrm{\'et}}$) of points of $G$ where $u$ is flat (resp. smooth, \'etale) is the inverse image of $U_{\mathrm{flat}}$ (resp. $U_{\mathrm{smooth}}$, $U_{\mathrm{\'et}}$) under $\pi_G$.
    \item[(b)] Suppose that for any $s\in S$, the fiber $G_s$ is locally of finite type over $\kappa(s)$, and that $u$ is locally of finite type (resp. locally of finite presentation) at the unit section of $G$. Then the sets
    \[U_{\mathrm{lqf}}\supset U_{\mathrm{u.ram}}\]
    formed by $s\in S$ such that $u_s$ is locally quasi-finite (resp. unramified), are open in $S$.\par
    If $u$ is also locally of finite type (resp. locally of finite presentation), then the set $V_{\mathrm{lqf}}$ (resp. $V_{\mathrm{u.ram}}$) of points of $G$ where $u$ is quasi-finite (resp. unramified) is the inverse image of $U_{\mathrm{lqf}}$ (resp. $U_{\mathrm{u.ram}}$) under $\pi_G$.
\end{enumerate}
\end{proposition}
\begin{proof}

\end{proof}

\begin{corollary}\label{scheme group radiciel morphism locus of open immersion}
Let $u:G\to H$ be a morphism of $S$-groups which is radiciel and suppose that $G$ (resp. $H$) is flat and locally of finite presentation (resp. flat and locally of finite type) over $S$. Then the set $U$ of $s\in S$ such that $u_s$ is an open immersion is open in $S$, and the restriction of $u$ to $U$ is an open immersion.
\end{corollary}
\begin{proof}
By \cref{scheme group locus on base and group prop}~(a), the set $U'$ of points $s\in S$ such that $u_s$ is \'etale is open in $S$. Since $u$ is radiciel, so is $u_s$ for any $s\in S$, and by (\cite{EGA4-4}, 17.9.1) we have $U=U'$, which shows that $U$ is open. Finally, \cref{scheme group locus on base and group prop}~(a), the restriction of $u$ to $U$ is \'etale; since $u$ is radiciel, this restriction is an open immersion by (\cite{EGA4-4}, 17.9.1).
\end{proof}

\begin{proposition}\label{scheme group unramified at unit section iff}
Let $G$ be an $S$-group. The following conditions are equivalent:
\begin{enumerate}
    \item[(\rmnum{1})] $G$ is unramified at the unit section.
    \item[(\rmnum{2})] The unit section is an open immersion.
    \item[(\rmnum{3})] $G$ is of finite presentation over $S$ at the unit section, and $G_s$ is unramified over $\kappa(s)$ for any $s\in S$.
\end{enumerate}
If $G$ is locally of finite presentation over $S$, then the above conditions are equivalent to:
\begin{enumerate}
    \item[(\rmnum{4})] $G$ is unramified over $S$.
\end{enumerate}
\end{proposition}
\begin{proof}
The implication (\rmnum{1})$\Rightarrow$(\rmnum{2}) follows from \cref{scheme morphism local fp unramified at point iff}, since an injective local isomorphism is an open immersion. Conversely, if $e:S\to G$ is an open immersion, then the restriction of $\pi_G$ to $e(S)$ is an isomorphism, so $G$ is unramified at the points of $e(S)$. We also note that by \cref{scheme group local ft over local Artin if open subset}, either (\rmnum{1}) or (\rmnum{3}) implies that for any $s\in S$, $G_s$ is locally of finite presentation over $\kappa(s)$. Then by \cref{scheme morphism local fp unramified at point iff}, condition (\rmnum{1}) is equivalent to that for any $s\in S$, $G_s$ is unramified over $\kappa(s)$ at $e_s$, the identity of $G_s$, or by \cref{scheme alg group prop iff at point} that $G_s$ is unramified over $\kappa(s)$ for any $s\in S$, hence (\rmnum{1})$\Leftrightarrow$(\rmnum{3}). Finally, if $G$ is locally of finite presentation over $S$, then by \cref{scheme morphism local fp unramified at point iff}, condition (\rmnum{4}) is equivalent to that $G_s$ is unramified over $\kappa(s)$ for any $s\in S$, whence to (\rmnum{3}).
\end{proof}

\begin{corollary}\label{scheme group morphism local qf or unram iff}
Let $u:G\to H$ be a morphism of $S$-groups. Suppose that for any $s\in S$, $G_s$ is locally of finite type over $\kappa(s)$.
\begin{enumerate}
    \item[(a)] If $u$ is locally of finite type, the following conditions are equivalent:
    \begin{enumerate}
        \item[(\rmnum{1})] $u$ is locally quasi-finite;
        \item[(\rmnum{2})] for any $s\in S$, $u_s:G_s\to H_s$ is locally quasi-finite;
        \item[(\rmnum{3})] $\ker u$ is locally quasi-finite over $S$;
        \item[(\rmnum{4})] the fibers of $\ker u$ are discrete.
    \end{enumerate}
    \item[(b)] If $u$ is locally of finite presentation, the following conditions are equivalent:
    \begin{enumerate}
        \item[(\rmnum{5})] $u$ is unramified;
        \item[(\rmnum{6})] for any $s\in S$, $u_s:G_s\to H_s$ is unramified;
        \item[(\rmnum{7})] $\ker u$ is unramified;
        \item[(\rmnum{8})] the unit section $S\to\ker u$ is an open immersion.
    \end{enumerate}
\end{enumerate}
\end{corollary}
\begin{proof}
The equivalences (\rmnum{1})$\Leftrightarrow$(\rmnum{2}) and (\rmnum{5})$\Leftrightarrow$(\rmnum{6}) follows from \cref{scheme group locus on base and group prop}~(b); also, since $\ker u$ is the inverse image of $e_H(S)$ under $u$, we see that (\rmnum{1})$\Rightarrow$(\rmnum{3}) and (\rmnum{5})$\Rightarrow$(\rmnum{7}).\par
For any point $s\in S$, denote by $e_s$ the identity element of $H_s$. Then (\rmnum{3}) (resp. (\rmnum{7})) implies that, for any $s\in S$,
\[(\ker u)_s=\ker u_s=u_s^{-1}(e_s)\]
is locally quasi-finite (resp. unramified) over $\kappa(s)=\kappa(e_s)$, hence that $u_s$ is quasi-finite (resp. unramified) at the identity of $G_s$. By \cref{scheme alg group morphism prop iff point}, $u_s$ is locally quasi-finite (resp. unramified), so (\rmnum{3})$\Rightarrow$(\rmnum{2}) and (\rmnum{7})$\Rightarrow$(\rmnum{6}). Finally, (\rmnum{2})$\Leftrightarrow$(\rmnum{4}) by \cref{scheme alg group morphism local qf iff} and (\rmnum{7})$\Leftrightarrow$(\rmnum{8}) by \cref{scheme group unramified at unit section iff}.
\end{proof} 

\begin{lemma}\label{scheme morphism with section isomorphism if}
Let $\pi:X\to S$ be a morphism admitting an $S$-section $e:S\to X$.
\begin{enumerate}
    \item[(a)] If $\pi$ is injective, it is integral.
    \item[(b)] If $\pi$ is locally of finite type and if for any $s\in S$, $\pi_s$ is an isomorphism, then $\pi$ is an isomorphism.
\end{enumerate}
\end{lemma}
\begin{proof}
In the situation of (a), $\pi$ is then a bijective, hence a homeomorphism (and in particular affine). Since $e$ is a surjective immersion, $e(S)$ is defined by a nilideal $\mathscr{I}$ of $\mathscr{O}_X$. Since $e$ is a section of $\pi$, we have $\mathscr{O}_X=\mathscr{O}_S\oplus\mathscr{I}$ as $\mathscr{O}_S$-modules. The ideal $\mathscr{I}$ is clearly integral over $\mathscr{O}_S$ (since it is nilpotent), so $\mathscr{O}_X$ is integral over $\mathscr{O}_S$, and $\pi$ is integral.\par
Now assume the condition in (b), in particular, $\pi$ is bijective. As $\pi\circ e=\id_S$, $e$ is then locally of finite presentation (\cref{scheme morphism local fp permanence prop}), so $\mathscr{I}$ is of fintie type over $\mathscr{O}_X$. For any $s\in S$, we have $\mathscr{O}_{X_s}=\kappa(s)\oplus(\mathscr{I}\otimes_{\mathscr{O}_S}\kappa(s))$, and since $e_s$ is an isomorphism, $\mathscr{I}\otimes_{\mathscr{O}_S}\kappa(s)=0$, hence a fortiori $\mathscr{I}\otimes_{\mathscr{O}_X}\kappa(x)=0$. It then follows from Nakayama's lemma that $\mathscr{I}=0$, so $\pi$ is an isomorphism.
\end{proof}

\begin{proposition}\label{scheme group local ft trivial iff fiber trivial}
Let $G$ be an $S$-group locally of finite type. Suppose that for any $s\in S$, $G_s$ is the trivial $\kappa(s)$-group. Then $G$ is the trivial $S$-group.
\end{proposition}
\begin{proof}
By hypothesis, for any $s\in S$, the morphism $\pi_s$ is an isomorphism. Since $\pi$ has a section $e$, we conclude from \cref{scheme morphism with section isomorphism if} that $\pi$ is an isomorphism from $G$ to $S$, so $G$ is the trivial $S$-group.
\end{proof}

\subsection{Identity component of a group scheme}

Given a subset $A$ (resp. $B$) of an $S$-scheme $X$ (resp. $Y$), by abusing the notation, we denote by $A\times_SB$ the subset of $X\times_SY$ formed by points whose first projection belongs to $A$ and second one belongs to $B$. If $A$ is a subset of an $S$-group $G$, we say that $A$ is \textbf{stable under the group law of $G$} if we have $A^{-1}\sub A$ and $\mu(A\times_SA)\sub A$.\par

Let $G$ be an $S$-functor in groups. We say that $G$ is \textbf{pointwise reprentable} if for any $s\in S$, the functor $G_s=G\otimes_S\kappa(s)$ is representable. In this case, we denote by $G^0_s$ the identity component of the $\kappa(s)$-group $G_s$. We define an $S$-subgroup functor of $G$, called the \textbf{identity component} of $G$ and denoted by $G^0$, as follows: for any $T\to S$,
\[G^0(T)=\{u\in G(T):\text{for any $s\in S$, $u_s(|T_s|)\sub|G_s^0|$}\}.\]
We therefore construct a functor $G\mapsto G^0$.

\begin{remark}\label{scheme group Lie alg of connected component}
If $G$ is an $S$-functor in groups that is pointwise reprentable, then $\mathfrak{Lie}(G^0/S)=\mathfrak{Lie}(G/S)$. In fact, for any $S$-scheme $T$, if $I_T$ is the dual number scheme over $T$ and $\eps:T\to I_T$ is the zero section, then by \cref{scheme tangent bundle fiber char by morphism} we have
\[\mathfrak{Lie}(G/S)(T)=\{u\in G(I_T):u\circ\eps=e\}\]
where $e:T\to G$ denotes the composition of $T\to S$ and the unit section $S\to G$; similarly,
\begin{align*}
\mathfrak{Lie}(G^0/S)(T)&=\{u\in G^0(I_T):u\circ\eps=e\}\\
&=\{u\in G(I_T):\text{$u\circ\eps=e$ and $u_s(|(I_T)_s|)\sub|G_s^0|$ for any $s\in S$}\}.
\end{align*}
But for any $s\in S$, $T_s$ and $(I_T)_s=I_{T_s}$ have the same underlying space, so for $u\in\mathfrak{Lie}(G/S)(T)$ we have $u_s(I_{T_s})=e_s$, where $e_s$ denotes the unit section of $G_s$, whence $u\in \mathfrak{Lie}(G^0/S)(T)$. Since this argument is valid for any $T$, we conclude that $\mathfrak{Lie}(G^0/S)=\mathfrak{Lie}(G/S)$.
\end{remark}

\begin{remark}
If $G$ and $H$ are pointwise reprentable $S$-functors in groups, then:
\begin{enumerate}
    \item[(a)] if $G\sub H$, then $G^0\sub H^0$.
    \item[(b)] if $G\sub H$ amd $H^0\sub G$, then $G^0=H^0$.
    \item[(c)] if for any $s\in S$, $G_s$ is locally of finite type over $\kappa(s)$, then $(G^0)_s$ is represented by the identity component of $G_s$, so $G^0$ is pointwise representable and we have $(G^0)^0=G^0$.
\end{enumerate}
\end{remark}

\begin{proposition}\label{scheme group functor identity component and base change}
Let $G$ be a pointwise representable $S$-functor in groups and $S'$ be an $S$-scheme. Then $(G\times_SS')^0=G^0\times_SS'$, so the formation of identity components commutes with base change.
\end{proposition}
\begin{proof}
It suffices to see that, for any $s'\in S'$ with projection $s\in S$, we have
\[((G\times_SS')\otimes_{S'}\kappa(s'))^0=(G_s\otimes_{\kappa(s)}\kappa(s'))^0=G_s^0\otimes_{\kappa(s)}\kappa(s');\]
this follows from \cref{scheme A-group connected component is geometric}. Note (for later use) that we did not use the group structure of $G_s$, but only the fact that $G_s^0$ has a rational point, namely $e_s$, so it is geometrically connected (\cite{EGA4-2}, 4.5.14).
\end{proof}

\begin{example}\label{scheme group identity component represented by |G^0|}
Let $G$ be an $S$-group and denote by $|G^0|$ the subset of $G$ which is the union of $|G^0_s|$ for $s\in S$. Then $|G^0|$ is a subset of $G$ stable under the group law of $G$, and for any $S'\to S$, we have
\[G^0(S')=\{u\in G(S'):u(|S'|)\sub |G^0|\}.\]
If $|G^0|$ is an open subset of $G$, then it is clear that $G^0$ is represented by the open subscheme induced over $|G^0|$.
\end{example}

\begin{proposition}\label{scheme group over qcqs fiber ft identity component qc}
Let $S$ be a quasi-compact and quasi-separated scheme, $G$ be an $S$-group whose fibers are locally of finite type. Then there exists a quasi-compact open subset $U$ of $G$ containing $|G^0|$.
\end{proposition}
\begin{proof}
The unit section being an immersion, $e(S)$ is a quasi-compact subspace of $G$, hence there exists a quasi-compact open subset $V$ of $G$ containing $e(S)$. Since $S$ is quasi-separated and $V$ is quasi-compact, $V$ is then quasi-compact over $S$ (\cref{scheme morphism from into qs prop}~(\rmnum{3})), hence $V\times_SV$ is quasi-compact over $S$. Put $V_s=V\cap G_s$ and $V^0_s=V\cap G^0_s$, then $V_s^0$ is an open subset of $G^0_s$, dense in $G_s^0$ since $G^0_s$ is irreducible (\cref{scheme alg group identity component prop}), hence $V_s^0\cdot V_s^0=G^0_s$ (\cref{scheme A-group product of open dense is G}). This shows that $V_s\cdot V_s\supset|G^0|$, so $V\cdot V\supset|G^0|$. Finally, since $V\cdot V$ is quasi-compact, there exists a quasi-compact open subset $U$ of $G$ containing $V\cdot V$, and a fortiori $|G^0|$.
\end{proof}

\begin{corollary}\label{scheme group fiber ft connected is qc}
Let $G$ be an $S$-group whose fibers are locally of finite type and connected. Then $G$ is quasi-compact over $S$.
\end{corollary}
\begin{proof}
Since the assertion is local over $S$, we may assume that $S$ is affine. By \cref{scheme group over qcqs fiber ft identity component qc}, there then exists a quasi-compact open subset $U$ of $G$ containing $|G^0|=G$, so $G$ is quasi-compact, hence quasi-compact over the affine scheme $S$ (\cref{scheme morphism qc permanence prop}~(\rmnum{4})).
\end{proof}

\begin{proposition}\label{scheme group local fp |G^0| constructible prop}
Let $G$ be an $S$-group locally of finite presentation.
\begin{enumerate}
    \item[(a)] $|G^0|$ is ind-constructible in $G$.
    \item[(b)] If $G$ is quasi-separated over $S$ and $S$ is quasi-compact and quasi-separated, then $|G^0|$ is constructible.
    \item[(c)] If $G$ is quasi-separated over $S$, then $|G^0|$ is locally constructible. 
\end{enumerate}
\end{proposition}
\begin{proof}
We first consider the assertion (a). Since $\pi:G\to S$ is locally of finite presentation, for any $s\in S$, there exists an open subset $U$ of $G$ containing $e(s)$ and an open subset $V$ of $S$ containing $s$ such that $\pi(U)\sub V$ and that the morphism $\pi':U\to V$ induced by $\pi$ is of finite presentation. Then $T=e^{-1}(U)$ is an open subset of $S$ and if we denote by $W=\pi'^{-1}(T)$ and $\pi''=\pi'|_W$, then $\pi'':W\to T$ is of finite presentation, and admits a section $e'':T\to W$ induced by $e$.\par
For any $t\in T$, as $G_t^0$ is irreducible by \cref{scheme alg group identity component prop}, $W\cap G_t^0$ is dense in $G_t^0$, hence irreducible, hence connected: it is then the connected component of $\pi''^{-1}(t)$ containing $e''(t)$. It then follows from (\cite{EGA4-3}, 9.7.12) that the union $W^0$ of the $W\cap G_t^0$, for $t\in T$, is locally constructible in $W$. On the other hand, it follows from \cref{scheme A-group product of open dense is G} that $|G^0|\cap\pi^{-1}(T)=W^0\cdot W^0$, i.e. $|G^0|\cap\pi^{-1}(T)$ is the image of $W\times_TW$ under the morphism $\mu'':W\times_TW\to\pi^{-1}(T)$ induced by $\mu$. As $W\times_TW$ (resp. $\pi^{-1}(T)$) is of finite presentation (resp. locally of finite presentation) over $T$, $\mu''$ is locally of finite presentation and quasi-separated by \cref{scheme morphism local fp permanence prop} and \cref{scheme morphism local ft permanence prop}; if $\pi^{-1}(T)$ is also quasi-separated over $T$, then $\mu''$ is quasi-compact by \cref{scheme morphism qc permanence prop}, hence of finite presentation. As $W^0\times_TW^0$ is locally constructible in $W\times_TW$ (since $W^0$ is in $W$), it follows from Chevalley's constructible theorem (\cite{EGA4-1}, 1.8.4 et 1.9.5(\rmnum{8})) that $|G^0|\cap\pi^{-1}(T)$ is ind-constructible in $\pi^{-1}(T)$ and is locally constructible in $\pi^{-1}(T)$ if $G$ (and hence $\pi^{-1}(T)$) is quasi-separated over $T$. This proves the assertions of (a) and (c).\par
Now suppose that $G$ is quasi-separated over $S$ and $S$ is quasi-compact and quasi-separated. Then by \cref{scheme group over qcqs fiber ft identity component qc} there exists a quasi-compact open subset $U$ of $G$ containing $|G^0|$. Since $G$ is quasi-separated over $S$, $G$ is quasi-separated, hence the open subset $U$ is retrocompact (\cref{scheme qs iff intersection of qc open}), and it suffices to show that $|G^0|$ is constructible in $U$ (\cite{EGA3-1}, $0_{\Rmnum{3}}$, 9.1.8). Moreover, $U$ being quasi-compact, hence quasi-compact over $S$ (\cref{scheme morphism from into qs prop}~(\rmnum{3})), and quasi-separated over $S$, the restriction of $\pi$ to $U$ is of finite presentation, hence by (\cite{EGA4-3}, 9.7.12), $|G^0|$ is locally constructible in $U$, hence constructible in $U$, since $U$ is quasi-compact and quasi-separated (\cite{EGA4-1}, 1.8.1). This proves (b), and it follows that for any quasi-compact and quasi-separated open subset $T$ of $S$ (for example, any affine open of $S$), $|G^0|\cap\pi^{-1}(T)$ is constructible.
\end{proof}

\begin{corollary}\label{scheme group over qcqs affine filtered limit isomorphism}
Let $S_0$ be a quasi-compact and quasi-separated scheme, $I$ be a directed set, $(\mathscr{A}_i)_{i\in I}$ be an inductive system of quasi-coherent $\mathscr{O}_{S_0}$-algebras, $\mathscr{A}=\rlim\mathscr{A}_i$, $S_i=\Spec(\mathscr{A}_i)$ for $i\in I$, and $S=\Spec(\mathscr{A})$. Let $G$ be an $S_0$-group locally of finite presentation, then the canonical map $\rlim G^0(S_i)\to G^0(S)$ is bijective.
\end{corollary}
\begin{proof}
Since $G$ is locally of finite presentation over $S$, the canonical map $\rlim G(S_i)\to G(S)$ is bijective by (\cite{EGA4-2}, 8.14.2(c)). It then ensures that the canonical map $\rlim G^0(S_i)\to G^0(S)$ is injective. To see that it is surjective, let $g\in G^0(S)\sub G(S)$. There exists $i\in I$ such that $g$ factors through some $g_i\in G(S_i)$; by hypothesis, $g^{-1}(|G^0|)=S$, but by \cref{scheme group local fp |G^0| constructible prop}, $|G^0|$ is ind-constructible in $G$, so $g_i^{-1}(|G^0|)$ is in $S_i$. It follows from (\cite{EGA4-2}, 8.3.4) that there exists an index $j\geq i$ such that $g_j^{-1}(|G^0|)=S_j$, where $g_j$ is the map induced from $g_i$ by base change $S_j\to S_i$. This shows that $g_j\in G^0(S_j)$, and since $g_j$ maps to $g$, this proves our assertion.
\end{proof}

\begin{proposition}\label{scheme group local fp identity component representable is qc open}
Let $G$ be an $S$-group locally of finite presentation. Suppose that $G^0$ is representable, then the canonical morphism $i:G^0\to G$ is an open immersion. Moreover, $G^0$ is quasi-compact over $S$.
\end{proposition}
\begin{proof}
Since $G^0$ is a subfunctor of $G$, the morphism $i$ is a monomorphism, hence is radiciel. By the definition of $G^0$, we verify immediately that $i$ is formally \'etale (and note that $|G^0|$ is the image of $i$). Finally, it follows from the characterization (\cite{EGA4-3}, 8.14.2(c)) on $S$-schemes locally of finite presentation and from \cref{scheme group over qcqs affine filtered limit isomorphism} that, since $G$ is locally of finite presentation over $S$, so is $G^0$ (\cref{scheme morphism local fp permanence prop}). Hence $i$ is locally of finite presentation; it is then a radiciel and \'etale morphism, hence an open immersion (\cite{EGA4-4}, 17.9.1). Finally, the last assertion follows from \cref{scheme group fiber ft connected is qc}.
\end{proof}

\begin{theorem}\label{scheme group smooth at unit section iff}
Let $G$ be an $S$-group. The following conditions are equivalent:
\begin{enumerate}
    \item[(\rmnum{1})] $G$ is smooth over $S$ at the unit section.
    \item[(\rmnum{2})] $G$ is flat and locally of finite presentation over $S$ at the unit section, and for any $s\in S$, $G_s$ is smooth over $\kappa(s)$.
    \item[(\rmnum{3})] There exists an open subgroup $G'$ of $G$ which is smooth over $S$.
    \item[(\rmnum{4})] $G^0$ is representable by an open subscheme of $G$ which is smooth over $S$,
\end{enumerate}
\end{theorem}
\begin{proof}
By \cref{scheme alg group morphism prop iff point} and \cref{scheme group local fp smooth uo at unit section iff open subgroup}, it is clear that (\rmnum{4})$\Rightarrow$(\rmnum{3})$\Rightarrow$(\rmnum{1}) and (\rmnum{1}) implies (\rmnum{2}) and (\rmnum{3}). We first show that (\rmnum{3})$\Rightarrow$(\rmnum{4}). In the situation of (\rmnum{3}), \cref{scheme group fiber ft open subgruop identity component equal} below shows that $G'$ contains $|G^0$, and that $|G^0|=|G'^0|$. It then suffices to prove that $|G^0|$ is open in $G$, because we have as in \cref{scheme group identity component represented by |G^0|} that $G^0$ is representable by the smooth open subscheme induced over $|G'^0|=|G^0|$. We can therefore suppose that $G'=G$.\par
To show that $|G^0|$ is open, it suffices to prove that any $s\in S$ possesses a neighborhood $T$ in $S$ such that $|G^0|\cap\pi^{-1}(T)$ is open in $\pi^{-1}(T)$. Let $s\in S$; since $G=G'$, $\pi$ is then locally of finite presentation, so we can construct, as in the proof of \cref{scheme group local fp |G^0| constructible prop}, an open subset $T$ of $S$ containing $s$ and an open subset $W$ of $G$ containing $e(S)$ such that the morphism $\pi'':W\to T$ induced by $\pi$ is of finite presentation and admits a section $e'':T\to W$, which is induced by $e$. For any $t\in T$, $W\cap G_t^0$ is then the connected component of $\pi''^{-1}(t)$ containing $e''(t)$. Since $\pi$ is smooth, so is $\pi''$, which is then smooth of finite presentation. Then by (\cite{EGA4-3}, 15.6.5), the union $W^0$ of the $W\cap G_t^0$ for $t\in T$ is open in $W$.\par
On the other hand, by \cref{scheme A-group product of open dense is G}, we have $W^0\cdot W^0=|G^0|\cap\pi^{-1}(T)$, and it is necessary to show that it is open in $\pi^{-1}(T)$. We are then reduced to the case where $T=S$, and it rests to demonstrate that $W^0\cdot W^0$ is open in $G$. Since $\pi$ is universally open, so is $\mu$, and since $W^0$ is open in $G$, so is $W^0\cdot W^0=\mu(W^0\times_SW^0)$.
\end{proof}

\begin{lemma}\label{scheme group fiber ft open subgruop identity component equal}
Let $G$ be an $S$-group whose fibers are locally of finite type. Then any open subgroup $H$ of $G$ contains $|G^0|$ and we have $|G^0|=|H^0|$.
\end{lemma}
\begin{proof}
Let $s\in S$ and put $G_s'=H_s\cap G^0_s$. Then $G_s'$ is an open subset of $G_s^0$, which is dense in $G^0_s$ since $G_s^0$ is irreducible (\cref{scheme alg group identity component prop}), hence $G_s'\cdot G_s'=G^0_s$ (\cref{scheme A-group product of open dense is G}). This shows that $G^0_s=G'_s\cdot G_s'\sub H_s\cdot H_s=H_s$, so we have $G_s^0=H_s^0$ for any $s\in S$, whence $|G^0|\sub H$ and $|H^0|=|G^0|$.
\end{proof}

\begin{proposition}\label{scheme group local fp morphism flat iff identity component surjective}
Let $u:G\to H$ be a morphism between $S$-groups locally of finite presentation. If $u$ is flat, the map $u^0:|G^0|\to|H^0|$ induced from $u$ is surjective. The converse is true if $G$ is flat over $S$ and $H$ has reduced fibers.
\end{proposition}
\begin{proof}
If $u$ is flat, then for any $s\in S$, $u_s$ is flat and locally of finite presentation, hence open by (\cite{EGA4-2}, 2.4.6), so the morphism $u^0_s:G^0_s\to H^0_s$ is surjective by \cref{scheme alg group morphism open iff}. The morphism $u^0:|G^0|\to|H^0|$ is then surjective.\par
Conversely, suppose that the map $u^0:|G^0|\to|H^0|$ is surjective, $G$ is flat over $S$ and $H$ has reduced fibers. Then for any $s\in S$, the morphism $u^0_s:G^0_s\to H^0_s$ is surjective, hence flat at any point lying over the generic point $\eta$ of $H^0_s$ (since $\mathscr{O}_{H^0_s,\eta}$ is a field), and $u$ is flat by \cref{scheme alg group morphism prop iff point}. The morphism $u$ is therefore flat by fiber criterion of flatness (\cite{EGA4-3}, 11.3.11).
\end{proof}

\subsection{Dimension of fibers}
\begin{proposition}\label{scheme group local ft fiber dimension functor semicontinuous}
Let $G$ be an $S$-scheme locally of finite type, endowed with an $S$-section $e$ and such that for any $s\in S$, we have $\dim(G_s)=\dim_{e(s)}(G_s)$ (this is the case if $G$ is an $S$-group, cf. \cref{scheme alg group connected component prop}).
\begin{enumerate}
    \item[(a)] The function $s\mapsto\dim(G_s)$ is upper semi-continuous on $S$.
    \item[(b)] If $G$ is locally of finite presentation over $S$, then this function is locally constructible.
\end{enumerate}
\end{proposition}
\begin{proof}
Let $\pi:G\to S$ be the structural morphism. By Chevalley's semicontinuous theorem (\cite{EGA4-3}, 13.1.3), the function $x\mapsto\dim_x(\pi^{-1}(\pi(x)))$ is upper semi-continuous on $G$. Now for any $s\in S$, we have
\[\dim(G_s)=\dim(\pi^{-1}(s))=\dim_{e(s)}(\pi^{-1}(\pi(e(s))));\]
and since the function $s\mapsto e(s)$ is continuous on $S$, the composition functor $s\mapsto\dim(G_s)$ is upper semi-continuous on $G$.\par
Suppose that $G$ is locally of finite presentation over $S$. To show that the functor $s\mapsto\dim(G_s)$ is locally constructible, we see from the preceding arguments that it suffices to show that the function $x\mapsto\dim_x(\pi^{-1}(\pi(x)))$ is locally constructible on $G$, which follows from (\cite{EGA4-3}, 9.9.1).
\end{proof}

\begin{proposition}\label{scheme group uo at identity component of s iff}
Let $\pi:G\to S$ be an $S$-scheme locally of finite presentation, endowed with an $S$-section $e$ and satisfies the following conditions:
\begin{enumerate}
    \item[(a)] For any $s\in S$ and any $x\in G_s$, we have $\dim(G_s)=\dim_x(G_s)$ (or equivalently, the irreducible components of $G_s$ have the same dimension).
    \item[(b)] For any $s\in S$, if $G_s^0$ is the connected component of $G_s$ containing $e(s)$, then $G^0_s$ is gemetrically irreducible.
\end{enumerate}
Let $s\in S$, the following conditions are equivalent:
\begin{enumerate}
    \item[(\rmnum{1})] $G$ is universally open over $S$ at points of $G^0_s$.
    \item[(\rmnum{2})] $G$ is universally open over $S$ at any point of a neighborhood of $e(s)$ in $G_s^0$. 
    \item[(\rmnum{3})] The function $t\mapsto\dim(G_t)$ is constant in a neighborhood of $s$ in $S$. 
    \item[(\rmnum{4})] $|G^0|$ is universally open over $S$ at points of $G^0_s$, that is, for any $S'\to S$, $s'\in S'$ lying over $s$ and $V$ an open neighborhood of $g$ in $G'=G_{(S')}$, $\pi(V\cap |G'^0|)$ is an open neighborhood of $s'$ in $S'$. 
\end{enumerate}
\end{proposition}
\begin{proof}
It is clear that (\rmnum{1})$\Rightarrow$(\rmnum{2}). By (\cite{EGA4-3}, 14.3.3.1(\rmnum{2})), the set of points of $G^0_s$ where $\pi_G$ is universally open is closed in $G^0_s$. Hence, as $G^0_s$ is irreducible, we have (\rmnum{2})$\Rightarrow$(\rmnum{1}).\par
We now prove that (\rmnum{1})$\Rightarrow$(\rmnum{3}). Let $T$ be the set of $t\in S$ such that $\dim(G_t)=\dim(G_s)$. By \cref{scheme group local ft fiber dimension functor semicontinuous}, $T$ is locally constructible, hence by (\cite{EGA4-1}, 1.10.1), to show that $T$ is a neighborhood of $s$, it suffices to show that any generalization $s'$ of $s$ belongs to $T$. Let $\xi$ be the generic point of $G^0_s$ and $U$ be an open subset of $G$ containing $\xi$. As $\pi_G$ is universally open at $\xi$, by (\cite{EGA4-3}, 14.3.13), for any generalization $s'$ of $s$, we have $\dim(U\cap G_{s'})\geq\dim_x(U\cap G_s)$. In view of the hypothesis (a), this then implies $\dim(G_{s'})\geq\dim(G_s)$. Now the function $s\mapsto\dim(G_s)$ is upper semi-continuous by \cref{scheme group local ft fiber dimension functor semicontinuous}, so we also have $\dim(G_{s'})\leq\dim(G_s)$, whence $s'\in T$. This show the implication (\rmnum{1})$\Rightarrow$(\rmnum{3}).\par
It is clear that (\rmnum{4})$\Rightarrow$(\rmnum{1}); we prove that (\rmnum{3})$\Rightarrow$(\rmnum{4}). As the dimension of fibers is stable under base change of fields and as the formation of $|G^0|$ commutes with base change (cf. the proof of \cref{scheme group functor identity component and base change}), we can suppose that $S'=S$ and $s'=s$. Moreover, as any open subset $V$ of $G$ meeting $G^0_s$ contains the generic point $\eta$ of $G^0_s$, we can suppose that $g=\eta$.\par
We can also suppose that $S$ is affine. Let $U$ be an affine open neighborhood of $e(s)$, which is then of finite presentation over $S$. Replacing $S$ by $e^{-1}(U)$ and then $U$ by $U\cap\pi^{-1}(S)$, we are reduced to the case where $\pi:U\to S$ is of finite presentation and admits a section $e:S\to U$. Then by (\cite{EGA4-3}, 9.7.12), $|G^0|\cap U$ is constructible in $U$, and by replacing $V$ by an affine open subset in $V\cap U$, we obtain that $|G^0|\cap V$ is constructible in $V$. As $\pi:V\to S$ is of finite presentation, $\pi(|G^0|\cap V)$ is then locally constructible in $S$, by Chevalley's constructibility theorem (cf. \cite{EGA4-1}, 1.8.4).\par
Hence, by (cf. \cite{EGA4-1}, 1.10.1), to show that $\pi(|G^0|\cap V)$ is an open neighborhood of $s$, it suffices to show that for any generalization $t$ of $s$, there exists a generalization $\xi$ of $\eta$ belonging to $|G^0|$ (and hence to $|G^0|\cap V$). Now the generic point $\xi$ of $G_t^0$ is a generalization of $\eta$. In fact, $e(s)$ belongs to the closure $X$ of $\{\xi\}$ in $G$, hence, by Chevalley's semi-continuous theorem (cf. \cite{EGA4-3}, 13.1.3), we have $\dim_{e(s)}(X_s)\geq\dim_\xi(X_t)$; on the other hand, the hypothesis of (\rmnum{3}) implies that $\dim_\xi(G_t^0)=\dim_{e(s)}(G_s)$. It then follows that an irreducible components of $X_s$ containing $e(s)$ is equal to $G_s^0$, whence $\eta\in X$. This proves (\rmnum{2})$\Rightarrow$(\rmnum{3}).
\end{proof}

\begin{corollary}\label{scheme group local fp fiber dimension is locally constant}
Let $G$ be an $S$-group locally of finite presentation. Then the function $s\mapsto\dim(G_s)$ is locally constant over $S$.
\end{corollary}
\begin{proof}
This follows immediately from \cref{scheme group uo at identity component of s iff}, because any flat morphism locally of finite presentation is universally open (\cite{EGA4-2}, 2.4.6).
\end{proof}

\begin{corollary}\label{scheme group flat local ft smooth at unit section iff}
Let $G$ be a flat $S$-group locally of finite presentatio over $S$ at the unit section. Consider the following conditions:
\begin{enumerate}
    \item[(\rmnum{1})] $G$ is smooth over $S$ at the unit section.
    \item[(\rmnum{2})] For any $s\in S$, $G_s$ is smooth over $\kappa(s)$, and the function $s\mapsto\dim(G_s)$ is locally constant on $S$.
    \item[(\rmnum{3})] For any $s\in S$, $G_s$ is smooth over $\kappa(s)$, and there exists an neighborhood $V$ of the unit section such that $\pi_V:V\to S$ is universally open.
    \item[(\rmnum{4})] For any $s\in S$, $G_s^0$ is smooth over $\kappa(s)$, and $G^0$ is representable by an open subgroup of $G$, which is universally open over $S$.
\end{enumerate} 
Then we have the implications (\rmnum{1})$\Rightarrow$(\rmnum{2})$\Leftrightarrow$(\rmnum{3})$\Leftrightarrow$(\rmnum{4}). If we further suppose that $S$ is reduced, then the above conditions are equivalent, and they imply that $G^0$ is smooth over $S$.
\end{corollary}
\begin{proof}
We first show that (\rmnum{1})$\Rightarrow$(\rmnum{2}). For any $x\in G$, we have $\dim_x(\pi^{-1}(\pi(x)))=\dim(\pi^{-1}(\pi(x)))$ by \cref{scheme alg group connected component prop}, so by (\cite{EGA4-4}, 17.10.2), the function
\[x\mapsto\dim_x(\pi^{-1}(\pi(x)))=\dim(\pi^{-1}(\pi(x)))\]
is continuous in a neighborhood of the unit section. Hence the function $s\mapsto\dim(G_s)$ is continuous on $S$, hence locally constant on $S$. On the other hand, for any $s\in S$, $G_s$ is smooth over $\kappa(s)$ by \cref{scheme alg group flat smooth iff at unit}.\par
Now assume the conditions in (\rmnum{2}), we prove (\rmnum{4}). It suffices to show that $|G^0|$ is open in $G$, because then, by \cref{scheme group identity component represented by |G^0|}, $G^0$ is representable by the open subscheme induced over $|G^0|$, and the properties of $G^0$ given in (\rmnum{4}) follow from \cref{scheme group uo at identity component of s iff}. Given a point $s\in S$, we construct $W$, $T$, $\pi''$, $e''$ and $W^0$ as in \cref{scheme group local fp |G^0| constructible prop}. Then by (\cite{EGA4-3}, 15.6.7), $W^0$ is open in $W$. On the other hand, under the hypothesis of (\rmnum{2}), it follows from \cref{scheme group uo at identity component of s iff} that $\pi$ is universally open at any point of $W^0$, hence $\mu$ is universally open at any point of $W^0\times_SW^0$, and this shows that $W^0\cdot W^0$ is open in $G$. We can then conclude (\rmnum{4}) as in the proof of \cref{scheme group smooth at unit section iff}.\par
It is clear that (\rmnum{4})$\Rightarrow$(\rmnum{3}), and (\rmnum{3})$\Rightarrow$(\rmnum{2}) follows from {scheme group uo at identity component of s iff} applied to $V$. Finally, suppose that (\rmnum{2})--(\rmnum{4}) are verified. To show that $G$ is smooth at the unit section, in view of \cref{scheme group smooth at unit section iff}, we may assume that $G=G^0$. Then $G$ is of finite presentation over $S$ by \cref{scheme group local fp connected fiber uo is separated and qc}, thus $\pi_G$ is of finite presentation, with geometrically integral fibers, whose dimension are locally constant over $S$. By (\cite{EGA4-3}, 15.6.7), the morphism $G\times_SS_{\red}\to S{\red}$ induced from $\pi_G$ is flat, hence $\pi_G$ is flat if $S$ is reduced. In this case, $G=G^0$ is smooth over $S$ by \cref{scheme group smooth at unit section iff}.
\end{proof}

\begin{example}\label{scheme group fiber smooth nonsmooth example}
If $k$ is a field and $S=\Spec(k[\delta])$ with $\delta^2=0$, then the trivial $k$-group $G=\Spec(k)$ is an $S$-group verifying (\rmnum{2})--(\rmnum{4}) of \cref{scheme group flat local ft smooth at unit section iff}. But it is not flat, hence not smooth, over $S$. 
\end{example}

\subsection{Separation of groups and homogeneous spaces}
\begin{proposition}
For an $S$-group $G$ to be separated, it is necessary and sufficient that the unit section of $G$ is a closed immersion.
\end{proposition}
\begin{proof}
This condition is necessary by \cref{scheme morphism to separated graph is closed}; it is sufficient in view of the following diagram:
\begin{equation*}
\begin{tikzcd}[row sep=12mm, column sep=12mm]
G\ar[d,"\Delta_{G/S}"]\ar[r,"\pi"]&S\ar[d,"e"]\\
G\times_SG\ar[r,"\mu\circ(\id_G\times c)"]&G
\end{tikzcd}\qedhere
\end{equation*}
\end{proof}

\begin{proposition}\label{scheme group over discrete base separated}
If $S$ is discrete, any $S$-group is separated.
\end{proposition}
\begin{proof}
In fact, $S$ is then equal to $\coprod_{s\in S}\Spec(\mathscr{O}_{S,s})$, and by \cref{scheme morphism separated local on target}, it suffices to show that for any $s\in S$, $G\otimes_S\Spec(\mathscr{O}_{S,s})$ is separated, which is true since $\mathscr{O}_{S,s}$ is a local ring of zero dimension, 
\end{proof}

\begin{theorem}\label{scheme homogeneous space sp if oepn dense fiber sp}
Let $S$ be a scheme, $G$ be an $S$-group locally of finite presentation and universally open over $S$ at a neighborhood of the unit section, $X$ be an $S$-scheme acted by $G$ such that the morphism
\[\Phi:G\times_SX\to X\times_SX,\quad (g,x)\mapsto (gx,x)\]
is surjective. Suppose that for any $s\in S$:
\begin{enumerate}
    \item[(\rmnum{1})] there exists an open subscheme $U$ of $X$, separated over $S$, such that $U_s$ is dense in $X_s$,
    \item[(\rmnum{2})] the fiber $X_s$ is locally of finite type over $\kappa(s)$.
\end{enumerate}
Then $X$ is seprated over $S$.
\end{theorem}
Before proving this theorem, let's see some of its concequences. First, we consider the following weaker form of \cref{scheme homogeneous space sp if oepn dense fiber sp}:

\begin{corollary}\label{scheme homogeneous space sp if connected ft fiber}
Let $S$ be a scheme, $G$ be an $S$-group locally of finite presentation and universally open over $S$ at a neighborhood of the unit section, $X$ be an $S$-scheme acted by $G$ such that the morphism
\[\Phi:G\times_SX\to X\times_SX,\quad (g,x)\mapsto (gx,x)\]
is surjective. Suppose that $X$ has connected and locally of finite type fibers, then:
\begin{enumerate}
    \item[(a)] $X$ is separated over $S$.
    \item[(b)] If there exists an open subset $V$ of $X$, quasi-compact over $S$ and meets each fiber $X_s$, then $X$ is quasi-compact over $S$. 
\end{enumerate}
\end{corollary}
\begin{proof}
In fact, let $s\in S$ such that $X_s\neq\emp$. As the morphism $\Phi_s:G_s\times_{\kappa(s)}X_s\to X_s\times_{\kappa(s)}X_s$ induced by $\Phi$ is surjective and as $X_s$ is connected, by \cref{scheme A-group-object transitive prop}, $X_s$ is irreducible. Hence, if $U$ is an affine open subset of $X$ such that $U_s$ is nonempty, then $U_s$ is dense in $X_s$, and \cref{scheme homogeneous space sp if oepn dense fiber sp} implies that $X$ is separated over $S$.\par
To prove (b), we may assume that $S$ is affine. Then $V$ is quasi-compact and, by \cref{scheme group over qcqs fiber ft identity component qc}, there exists a quasi-compact open subset $U$ of $G$ containing $|G^0|$. Let $s\in S$ be such that $X_s\neq\emp$, then $X_s$ is irreducible by (\cite{SGA3-1}, $\Rmnum{6}_A$, 2.6.6), and hence quasi-compact by (\cite{SGA3-1}, $\Rmnum{6}_A$, 2.6.4(\rmnum{1}')). It then follows that $X_s$ is of finite type over $\kappa(s)$, hence Noetherian. As $U_s$ contains $G_s^0$, the morphism $U_s\times_{\kappa(s)}V_s\to X_s,(g,x)\mapsto gx$ is then surjective by (\cite{SGA3-1}, $\Rmnum{6}_A$, 2.6.4(\rmnum{2})), as $U_s$ is retrocompact in $X_s$. Therefore the morphism $U\times_SV\to X$ is surjective and hence $X$ is quasi-compact (since $U$ and $V$ are quasi-compact); as $S$ is affine, hence separated, this implies that $X$ is quasi-compact over $S$ (cf. \cref{scheme morphism from into qs prop}).
\end{proof}

\begin{corollary}\label{scheme group local fp connected fiber uo is separated and qc}
Let $S$ be a scheme, $G$ be an $S$-group locally of finite presentation, with connected fibers, and universally open over $S$. Then $G$ is separated and of finite presentation over $S$.
\end{corollary}
\begin{proof}
By \cref{scheme group fiber ft connected is qc} and \cref{scheme homogeneous space sp if connected ft fiber}, $G$ is quasi-compact and separated over $S$, whence of finite presentation over $S$.
\end{proof}

\subsection{Representability of sub-functors in groups}\label{sheme functor representability of Res subsection}
Let $X$ be an $S$-functor, $G$ be an $S$-functor in groups, and $u,v$ be two $S$-morphisms from $X$ to $G$. The \textbf{transporter} of $u$ to $v$, denoted by $\Trans(u,v)$, is the sub-$S$-functor of $G$ defined by
\begin{align*}
\Trans(u,v)(S')&=\{g\in G(S'):\inn(g)\circ u_{S'}=v_{S'}\}\\
&=\{g\in G(S'):\text{$g_{S''}u_{S''}(x)g_{S''}^{-1}=v_{S''}(x)$ for any $x\in X(S'')$, $S''\to S'$}\}.
\end{align*}
In particular, $\Trans(u,u)$ is the sub-$S$-functor in groups of $G$, called the \textbf{centralizer} of $u$ and denoted by $\Centr(u)$.\par
If $X$ and $Y$ are two sub-$S$-functors of $G$, the \textbf{transporter} of $X$ to $Y$, denoted by $\Trans_G(X,Y)$, is the sub-$S$-functor of $G$ defined by
\begin{align*}
\Trans_G(X,Y)(S')&=\{g\in G(S'):\inn(g)(X_{S'})\sub Y_{S'}\}\\
&=\{g\in G(S'):\text{$g_{S''}X(S'')g_{S''}^{-1}\sub Y(S'')$ for any $S''\to S'$}\}.
\end{align*}
We also define the strict transporter of $X$ to $Y$ by
\begin{align*}
\STrans_G(X,Y)(S')&=\{g\in G(S'):\inn(g)(X_{S'})=Y_{S'}\}\\
&=\{g\in G(S'):\text{$g_{S''}X(S'')g_{S''}^{-1}=Y(S'')$ for any $S''\to S'$}\}.
\end{align*}
Note that we have
\[\STrans_G(X,Y)=\Trans_G(X,Y)\cap c(\Trans_G(Y,X)),\]
where $c$ is the inversion morphism of $G$.\par
Now let $H$ be a sub-$S$-functor of $G$ and $i:H\to G$ be the canonical $S$-morphism; the cantralizer and normalizer of $H$ in $G$ is then the sub-$S$-functors
\[Z_G(H)=\Centr(i)=\Trans(i,i),\quad N_G(H)=\STrans_G(H,H).\]
Finally, the center of $G$ is the $S$-functor $Z(G)=Z_G(G)=\Centr(\id_G)$. From our definition, it is clear that these functors are stable under base change.

\begin{definition}\label{scheme morphism essentially free definition}
Let $f:X\to S$ be a morphism of schemes. We say that $f$ is \textbf{essentially free}, or that $X$ is \textbf{essentially free over $S$}, if there exists an affine open covering $(S_i)$ of $S$, for each $i$ an $S_i$-scheme $S_i'$ which is affine and faithfully flat over $S_i$, and an affine open covering $(X_{ij}')$ of $X_i'=X\times_SS_i'$ such that for any $i,j$, the ring of $X_{ij}'$ is a free module over that of $S_i'$.
\end{definition}

\begin{proposition}\label{scheme morphism essentialy free base change prop}
Let $S$ be a schems.
\begin{enumerate}
    \item[(a)] If $X$ is essentially free over $S$, it is flat over $S$. The converse is true if $S$ is Artinian.
    \item[(b)] If $S$ is the spectrum of a field, any $S$-scheme is essentially free over $S$.
    \item[(b)] If $X$ is essentially free over $S$, then $X'=X\times_SS'$ is essentially free over $S'$ for any $S'\to S$. The converse is true if $S'\to S$ is faithfully flat and quasi-compact.
\end{enumerate}
\end{proposition}
\begin{proof}
The proof is immediate, by noting that a flat module over a local Artinian ring is free.
\end{proof}

\begin{example}\label{scheme group diagonalizable is essentially free}
Let $G$ be a diagonalizable group over $S$ (or more generally, which become diagonalizable after a suitable faithfully flat and quasi-compact base extension to any affine open subset of $S$, i.e. $G$ is "of multiplicative type"). Then $H$ is essentially free over $S$. In fact, if $G$ is diagonalizable, it is affine over $S$ and defined by a group algebra, which is free over $S$.
\end{example}

The introduction of the notion of essentially freeness is justified by the following theorem:
\begin{theorem}\label{scheme Weil restriction of closed of essentialy free representable prop}
Let $S$ be a sheme, $Z$ be an essentially free $S$-scheme, and $Y$ be a closed subscheme of $Z$. Consider the following functor
\[F=\Res_{Z/S}Y:\Sch_{/S}^{\op}\to\Set,\quad F(S')=\Gamma(Y_{S'}/Z_{S'})=\begin{cases}
\emp&\text{if $Z_{S'}\neq Y_{S'}$},\\
\{\id_{Z_{S'}}\}&\text{if $Z_{S'}=Y_{S'}$}.
\end{cases}\]
Then $F$ is representable by a closed subscheme $T$ of $S$. If $Y\to Z$ is of finite presentation, so is $T\to S$.
\end{theorem}
\begin{proof}
We first note that $F$ is a sheaf for the fpqc topology: as $F(S')=\emp$ or $\{\ast\}$ for any $S'$, it reduces to verify that if $(S_i)$ is an open covering of $S$ (resp. $S'\to S$ is a faithfully flat and quasi-compact morphism), and if each $Y_{S_i}\to Z_{S_i}$ (resp. if $Y_{S'}\to Z_{S'}$) is an isomorphism, then so is $Y\to Z$. But this is clear (resp. follows from \cite{SGA1} \Rmnum{8}, 5.4 ou \cite{EGA4-2}, 2.7.1). Moreover, by (\cite{SGA1} \Rmnum{8}, 1.9), faithfully flat and quasi-compact morphisms are effective descent for the fibre category of closed immersions. This allows us to limit ourselves to the case where $S_i'=S$ (with the notations of \cref{scheme morphism essentially free definition}).\par
Let $(Z_j)$ be an affine open covering of $Z$ such that $\mathscr{O}(Z_j)$ is a free module over $A=\mathscr{O}(S)$, and let $Y_j=Y\cap Z_j$ and $F_j=\Res_{Z_j/S}Y_j:\Sch_{/S}^{\op}\to\Set$. Each $F_j$ is a sub-functor of the final functor, and we have evidently $F=\bigcap_jF_j$, which allows us to reduce to proving that each $F_j$ is representable by a closed subscheme $T_j$ of $S$ (bacause then $F$ is representable by the intersection of the $T_j$). We can hence suppose that $Z$ is equally affine, $Z=\Spec(B)$, where $B$ is a free $A$-module. Let $(e_\lambda)_{\lambda\in\Lambda}$ be a basis of $B$ over $A$, and $\varphi_\lambda:B\to A$ be the "coordinate forms" relative to this basis, that is, $\varphi_\lambda(\sum_\mu a_\mu e_\mu)=e_\lambda$ for any $\lambda\in\Lambda$. If $\mathfrak{J}$ is an ideal of $B$ defining the subscheme $Y$ of $Z$, we let $\mathfrak{I}$ be the ideal in $A$ generated by the coordinates $\varphi_\lambda(x)$, with $x\in\mathfrak{J}$. Then the subscheme $T=V(\mathfrak{I})=\Spec(A/\mathfrak{I})$ satisfies the desired condition: in fact, for any $A$-algebra $C$, since $B$ is flat over $A$, we see that the morphism $B\otimes_AC\to(B/\mathfrak{J})\otimes_AC$ is an isomorphism if and only if the image of $x\otimes 1$ in $B\otimes_AC$ is zero for $x\in\mathfrak{J}$, which is equivalent to that the kernel of $A\to C$ contains the ideal $\mathfrak{I}$.
\end{proof}

\begin{example}\label{scheme subfunctor Weil restriction example}
Let $S$ be a scheme and $X,Y,Z$ be schemes over $S$.
\begin{enumerate}
    \item[(a)] Consider a morphism $q:X\to\sHom_S(Y,Z)$, i.e. $X$ acts on $Y$ with value in $Z$, and a subscheme $Z'$ of $Z$, whence a monomorphism
    \[\sHom_S(Y,Z')\to\sHom_S(Y,Z).\]
    Let $X'$ be the inverse image of $\sHom_S(Y,Z')$ in $X$ under the morphism $q$, which is the sub-functor of $X$ such that $X'(T)$ is the set of $x\in X(T)$ such that $q(x):Y_T\to Z_T$ factors through $Z'_T$. The functor $X'$ can also be described as follows: we put $P=X\times_SY$, and let $P'$ the inverse image of $Z'$ under $r:P\to Z$, the corresponding morphism of $q$; then there is an isomorphism of $X$-functors
    \[X'\cong\Res_{P/X}P'.\]
    We then conclude that if $Y$ is essentially free over $S$ and $Z'$ is closed in $Z$, the subfunctor $X'$ of $X$ is representable by a closed subscheme of $X$. If $Z'\to Z$ is of finite presentation, then so is $X'\to X$.
    \item[(b)] Let $q_1,q_2$ be two actions of $X$ over $Y$ with value in $Z$, i.e. two morphisms
    \[q_1,q_2:X\rightrightarrows\sHom_S(Y,Z),\]
    and put $X'=\ker(q_1,q_2)$, which is the sub-functor of $X$ such that $X'(T)$ is the set of $x\in X(T)$ such that the two morphisms $q_1(x),q_2(x):Y_T\rightrightarrows Z_T$ coincide. Now giving the morphisms $q_1,q_2$ is equivalent to a morphism
    \[q:X\to\sHom_S(Y,Z\times_SZ),\]
    and hence, a morphism $r:X\times_SY\to Z\times_SZ$. If $U=Z\times_SZ$, and $U'$ is the diagonal subscheme of $Z\times_SZ$, then $X'$ is none other than the inverse image of the sub-functor $\sHom_S(Y,U')$ of $\sHom_S(Y,U)$ under $q$, so it is isomorphism to $\Res_{P/X}P'$, where $P=X\times_SY$ and $P'$ is the inverse image of the diagonal under $r$, i.e. the kernel of $r_1,r_2:X\times_SY\rightrightarrows Z$. We therefore conclude that if $Y$ is essentially free over $S$ and $Z$ is separted over $S$, then the sub-functor $X'$ of $X$ is representable by a closed subscheme of $X$. If $Z\to S$ is locally of finite type, then $X'\to X$ is of finite presentation.
    \item[(c)] Consider a morphism $q:X\to\sHom_S(Y,Y)$, i.e. $X$ acts over $Y$. Let $X'$ be the kernel of this morphism, which is defined such that $X'(T)$ is the set of $x\in X(T)$ such that $q(X):Y_T\to Y_T$ is the identity. This functor is a special case of (b), as we can always introduce the morphism
    \[q':X\to\sHom_S(Y,Y)\]
    so that $X$ acts trivially on $Y$. Therefore, if $Y$ is essentially free and separated over $S$, the subfunctor $X'$ of $X$ is representable by a closed subscheme of $X$. If $Y\to S$ is locally of finite type, then $X'\to X$ is of finite presentation.
    \item[(d)] Under the conditions of (c), consider the sub-functor $Y'$ of invariants of $Y$, that is, $Y'(T)$ is the set of $y\in Y(T)$ such that the corresponding morphism $\bar{q}(y):X_T\to Y_T$ is the constant $T$-morphism with value $y$ (where $\bar{q}:Y\to\sHom_S(X,Y)$ is the morphism induced by $q$). If $q'$ is as in (c), with the corresponding morphisms
    \[\bar{q},\bar{q}':Y\rightrightarrows\sHom_S(X,Y)\]
    then $Y'=\ker(\bar{q},\bar{q}')$. Therefore, if $X$ is essentially free over $S$ and $Y$ is separted over $S$, then the sub-functor $Y'$ of invariants of $Y$ is representable by a closed subscheme of $Y$. If $Y\to S$ is locally of finite type, then $Y'\to Y$ is of finite presentation.
\end{enumerate}
\end{example}

The constructions of the type explained in the previous examples frequently occurs in group theory. For example, if $G$ is a group $S$-scheme acting on the $S$-schema $X$:
\[q:G\to\sAut_S(X),\]
the kernel of $q$ (the subgroup of $G$ acting trivially on $X$) is a closed subscheme of $G$ provided that $X$ is essentially free and separated over $S$, and the sub-object $X^G$ of invariants of $X$ is a closed subscheme of $X$ provided that $G$ is essentially free over $S$ and $X$ is separted over $S$.\par
Let $u,v:X\to G$ be morphisms of schemes and consider the sub-functor $\Trans(u,v)$ of $G$ whose value at an $S$-scheme $T$ is the set of $g\in G(T)$ such that $\inn(g)(u_T)=v_T$. If we consider the morphism $G\to\sHom_S(X,G)$ defined by the composition of the inner automorphism of $G$ and $u$ and by $v$, then we are in the situation of \cref{scheme subfunctor Weil restriction example}~(b). Therefore, if $X$ is essentially free over $S$ and $G$ is separated over $S$, then $\Trans(u,v)$ is a closed subscheme of $G$.\par
Let $Y,Z$ be subschemes of $X$ and consider the sub-functor $\Trans_G(Y,Z)$ of $G$, whose value at an $S$-scheme $T$ is the set $g\in G(T)$ such that the corresponding automorphism of $X_T$ satisfies $g(Y_T)\sub Z_T$, i.e. such that the induced morphism $Y_T\to X_T$ factors through $Z_T$. Therefore, by \cref{scheme subfunctor Weil restriction example}~(a), if $Y$ is essentially free over $S$ and $Z$ is closed in $X$, then $\Trans_G(Y,Z)$ is a closed subscheme of $G$.\par
We can also consider the strict transporer of $Y$ to $Z$, whose value at an $S$-scheme $T$ is the set $g\in G(T)$ such that $g(Y_T)=Z_T$. Since we have $\STrans_G(Y,Z)=\Trans_G(Y,Z)\cap c(\Trans_G(Z,Y))$, we obtain the same result concerning $\STrans_G(Y,Z)$. That is, it is a closed subscheme of $X$ if $Y$ is essentially free over $S$ and $Z$ is closed in $X$.\par
A particular important case is $X=G$, with $G$ act on its self by inner automorphisms. If $H$ is a subscheme of $G$, the transporter of $H$ to $H$ is then the normalizer $N_G(H)$ of $H$. Hence, if $H$ is a closed subgroup of $G$ and essentially free over $S$, then $N_G(H)$ is representable by a closed subgroup of $G$.\par
Finally, let $X$ be a subscheme of $G$; then the centralizer $Z_G(X)$ in $G$ is the sun-functor in groups of $G$ defined by the proceedure of \cref{scheme subfunctor Weil restriction example}~(d), where we act $Z$ on $G$ by inner automorphisms. Therefore, if $Z$ is essentially free over $S$ and $G$ is separated over $S$, $Z_G(X)$ is a closed subgroup of $G$. In particular, if $G$ is essentially free and separated over $S$, then its center $Z(G)$ is a closed subgroup of $G$.\par
If $S$ is the spectrum of a field, then any scheme over $S$ is essentially free over $S$; also, any $k$-group is separted over $k$, so we obtain the following corollary:

\begin{corollary}\label{scheme k-group transporter representable by closed}
Let $G$ be a group scheme over a field $k$ and $Y,Y'$ be two subscheme of $G$. Then
\begin{enumerate}
    \item[(a)] The centralizer of $Y$ in $G$ is a closed subgroup of $G$.
    \item[(a')] More generally, for any $u,v:X\to G$ morphisms of schemes, the transporter $\Trans(u,v)$ is representable by a closed subscheme of $G$.
    \item[(b)] If $Y$ is closed, the transporter $\Trans_G(Y',Y)$ is a closed subscheme of $G$. If $Y'$ is also closed, we have the same conclusion for $\STrans_G(Y',Y)$.
    \item[(c)] For any subgroup $H$ of $G$, $N_G(H)$ is a closed subgroup of $G$.
\end{enumerate}
\end{corollary}

\begin{corollary}\label{scheme k-group invariant subscheme exist}
Let $G$ be a group scheme over a field $k$ and $X$ be a separated scheme over $S$. Then the sub-functor $X^G$ of invariants of $X$ under $G$ is represented by a closed subscheme of $X$. The morphism $X^G\to X$ is of finite presentation over $S$ if $X$ is of finite type over $S$.
\end{corollary}
\begin{proof}
This follows from \cref{scheme subfunctor Weil restriction example}~(d) since any scheme is essentially free over $k$.
\end{proof}

\begin{corollary}\label{scheme alg group center is closed and quotient affine}
Let $k$ be a field and $G$ be a connected algebraic group over $k$. Then $Z(G)$ is representable by a closed subscheme of $G$, and $G/Z(G)$ is an affine algebraic $k$-group. 
\end{corollary}
\begin{proof}
The first assertion is contained in \cref{scheme k-group transporter representable by closed}, but we will see that it also follows from the proof of the second assertion. In fact, $G$ acts by the adjoint representation on the finite-dimensional $k$-vector space $P_n=\mathscr{O}_{G,e}/\m_e^{n+1}$ (where $\m_e$ is the maximal ideal of $\mathscr{O}_{G,e}$); denote by $H_n$ the kernel of $\rho_n:G\to\GL(P_n)$. By \cref{scheme A-group homomorphism of local ft image prop}, $\rho_n$ induces a closed immersion $G/H_n\hookrightarrow\GL(P_n)$, hence each $G/H_n$ is affine (a linear algebraic group). As $G$ is Noetherian, the intersection $H$ of $H_n$ is equal to one of $H_n$, so $G/H$ is affine.\par
On the other hand, let $Z$ be the center of $G$, which is clearly contained in $H$. Let $\widehat{\mathscr{O}}_{G,e}$ be the completion of $\mathscr{O}_{G,e}$ for the $\m_e$-adic topology and $\widehat{S}$ be the spectrum of $\widehat{\mathscr{O}}_{G,e}$ (resp. $S=\Spec(\mathscr{O}_{G,e})$). As $\widehat{S}\to S$ is faithfully flat and quasi-compact and the two morphisms
\[H\times_kS\to S,\quad (g,x)\mapsto gxg^{-1}\quad (\text{resp.}\quad (g,x)\mapsto x)\]
coincide after base change $\widehat{S}\to S$, they coincide, i.e. $H$ acts trivially on $\mathscr{O}_{G,e}$. Now by \cref{scheme subfunctor Weil restriction example}~(d), the sub-object $G^H$ of invariants of $G$ under $H$ (which is none other than $Z_G(H)$) is a closed subscheme of $G$, hence is defined by a qausi-coherent ideal $\mathscr{I}$ of $\mathscr{O}_G$. As $G^H$ dominates the open subscheme $S=\Spec(\mathscr{O}_{G,e})$ and $\mathscr{O}_{G^H,e}=\mathscr{O}_{G,e}/\mathscr{I}_e$, it follows that $\mathscr{I}_e=0$. Since $\mathscr{I}$ is of finite type ($G$ being Noetherian), there then exists an open neighborhood $U$ of $e$ such that $\mathscr{I}|_U=0$ (cf. \cref{sheaf of module ft local prop}). Then the subgroup $G^H$ contains $U$ and hence also $U\cdot U$, which is equal to $G$ since $G$ is irreducible (\cref{scheme A-group product of open dense is G} and \cref{scheme alg group identity component prop}), therefore $Z_G(H)=G$, whence $H\sub Z$ and $Z=H$.
\end{proof}

\subsection{Properties preversed by quotients}
\begin{definition}
Given a monomorphism $u:H\to G$ of $S$-groups, we denote by $G/H$ (resp. $H\backslash G$) the sheaf (for the fpqc topology) of $G$ by the equivalence relation defined by the monomorphism
\[G\times_SH\stackrel{\delta\circ(\id_G\times u)}{\longrightarrow} G\times_SG\quad (\text{resp.}\quad H\times_SG\stackrel{\gamma\circ(\id_G\times u)}{\longrightarrow} G\times_SG)\]
where $\delta$ (resp. $\gamma$) is the automorphism of $G\times_SG$ defined by $(g,h)\mapsto(g,gh)$ (resp. $(h,g)\mapsto(hg,g)$) for $g,h\in G(T)$.
\end{definition}

\begin{proposition}\label{scheme group fpqc quotient of monomorphism representable prop}
Let $u:H\to G$ be a monomorphism of $S$-groups. Suppose that $G/H$ is represented by an $S$-scheme $G'$, then:
\begin{enumerate}
    \item[(\rmnum{1})] The canonical morphism $p:G\to G'$ is covering for the fpqc topology.
    \item[(\rmnum{2})] If we put $e'=p\circ e$ (this is called the unit section of $G'$), the following diagrams are Cartesian:
    \[\begin{tikzcd}[column sep=12mm, row sep=12mm]
    G\times_SH\ar[r,"\delta\circ(\id_G\times u)"]\ar[d,swap,"\pr_1"]&G\ar[d,"p"]\\
    G\ar[r,"p"]&G'
    \end{tikzcd}\quad\quad\quad\begin{tikzcd}[column sep=12mm, row sep=12mm]
    H\ar[r,"u"]\ar[d,swap,"\pi_H"]&G\ar[d,"p"]\\
    S\ar[r,"e'"]&G'
    \end{tikzcd}\]
    In particular, $u$ is an immersion.
    \item[(\rmnum{3})] There exists a unique $S$-scheme structure on $G'$ acted by $G$ such that $p$ is a morphism of $S$-schemes acted by $G$.
    \item[(\rmnum{4})] If we suppose that $H$ is normal in $G$, then there exists a unique $S$-group structure on $G'$ such that $p$ is a morphism of $S$-groups.
    \item[(\rmnum{5})] Let $S_0$ be an $S$-scheme and put $G_0=G\times_SS_0$, $H_0=H\times_SS_0$. Then $G_0/H_0$ is representable by $G'_0=G'\times_SS_0$.
    \item[(\rmnum{6})] Let $\mathcal{P}$ be a property for $S$-morphisms. Suppose that $\mathcal{P}$ is stable under base change, then if $p:G\to G'$ verifies $\mathcal{P}$, so does the structural morphism $\pi_H:H\to S$.
    \item[(\rmnum{7})] Let $\mathcal{P}$ be a property for $S$-morphisms. Suppose that $\mathcal{P}$ is local for the fpqc topology. Then, for the morphism $p:G\to G'$ to verify $\mathcal{P}$, it is necessary and sufficient that so deos $\pi_H:H\to S$.
    \item[(\rmnum{8})] Let $\mathcal{P}$ be a property for $S$-morphisms. Suppose that $\mathcal{P}$ is local for the fpqc topology and stable under composition. Then, if the structural morphisms $H\to S$ and $G'\to S$ verify $\mathcal{P}$, so does the structural morphism $G\to S$.
    \item[(\rmnum{9})] If $G$ is reduced, then $G'$ is reduced.
    \item[(\rmnum{10})] For $G'$ to be separated over $S$, it is necessary and sufficient that $u$ is a closed immersion.
    \item[(\rmnum{11})] For $H$ to be flat over $S$, it is necessary and sufficient that $p$ is a flat morphism.
    \item[(\rmnum{12})] For $H$ to be faithfully flat and locally of finite presentation over $S$, it is necessary and sufficient that $p$ is faithfully flat and localy of finite presentation. In this case, for $G'$ to be locally of finite presentation (resp. locally of finite type, of finite type, unramified, smooth, \'etale, locally quasi-finite, quasi-finite) over $S$, it is sufficient that so is $G$ over $S$.
    \item[(\rmnum{13})] Suppose that $H$ is flat and of finite presentation over $S$. Then $p$ is faithfully flat and of finite presentation. Moreover, for $G$ to be of finite presentation over $S$, it is necessary and sufficient that so is $G'$.
\end{enumerate}
\end{proposition}

\begin{remark}
Under the general hypothesis of \cref{scheme group fpqc quotient of monomorphism representable prop}, if we suppose that $H$ is flat and locally of finite presentation over $S$, then $p$ is covering for the fppf topology by \cref{scheme group fpqc quotient of monomorphism representable prop}~(\rmnum{7}), so assertions (\rmnum{7}) and (\rmnum{8}) of \cref{scheme group fpqc quotient of monomorphism representable prop} can be extended to properties local for the fppf topology.
\end{remark}

\begin{remark}
The question that whether a quotient $G/H$ is representable is often tricky. In general, to be able to affirm that the quotient $G/H$ is representable, it is not sufficient to suppose that $G$ and $H$ are of finite presentation on $S$ and $H$ flat on $S$. In fact, suppose further that $G$ is smooth with connected fibres. In this case, if $G/H$ is a scheme, then it is separated by \cref{scheme homogeneous space sp if connected ft fiber}, and hence $H\hookrightarrow G$ is a closed immersion according to \cref{scheme group fpqc quotient of monomorphism representable prop}~(\rmnum{9}). Therefore, if $H$ is not closed in $G$, then $G/H$ is not representable.\par
To obtain a counter-example, we can choose $S$ to be the spectrum of a DVR (so $S$ consists of a generic point and a special point), and put $G=\G_{m,S}$. Consider on the other hand an integer $n>1$ which is invertible over $S$; then $\bm{\mu}_n=\ker(G\stackrel{n}{\to}G)$ is a closed subgroup of $G$ which is \'etale over $S$ (cf. \cite{SGA3-1} $\Rmnum{7}_A$, 8.4). Let $H$ be the open subgroup of $\bm{\mu}_n$ obtaine by deleting from $\bm{\mu}_n$ the subset of the special fiber of $\bm{\mu}_n$ complementray to the origin. Then $H$ is not closed in $G$, hence $G/H$ is not representable.
\end{remark}

\subsection{Affine group schemes}
\paragraph{Complements on \texorpdfstring{$G_{\aff}$}{G} and anti-affine groups}
We note the following lemma, which extends (\cite{SGA3-1} $\Rmnum{6}_B$ 11.18.1) to the case where $G$ is not assumed to be of finite type.
\begin{lemma}\label{scheme k-group qc monomorphism is closed immersion}
Let $k$ be a field, $u:G\to H$ be a monomorphism of $k$-groups, with $H$ affine. Suppose that $u$ is quasi-compact, then $u$ is a closed immersion.
\end{lemma}

\begin{theorem}
Let $G$ be an algebraic group over a field $k$. We denote by $\rho:G\to\G_{\aff}$ the canonical morphism and $N$ be its kernel.
\begin{enumerate}
    \item[(\rmnum{1})] The canonical morphism $G/N\to G_{\aff}$ is an isomorphism, and hence $G_{\aff}$ is an affine algebraic group and $\rho$ is faithfully flat.
    \item[(\rmnum{2})] We have a canonical isomorphism $(G/N)_{\aff}=G_{\aff}$.
    \item[(\rmnum{3})] $N$ is a characteristic subgroup of $G$.
    \item[(\rmnum{4})] $\mathscr{O}(N)=k$.
    \item[(\rmnum{5})] $N$ is smooth, conncected and commutative.    
\end{enumerate}
\end{theorem}
\begin{proof}
Since the morphism $\rho^\sharp:\mathscr{O}(G_\aff)\to\mathscr{O}(G)$ is an isomorphism, assertion (\rmnum{1}) is a particular case of (\cite{SGA3-1} $\Rmnum{6}_B$ 11.18.1), and (\rmnum{2}) follows from the universal properties of $G_{\aff}$ and $(G/N)_{\aff}$. To prove (\rmnum{3}), we note that for any $k$-automorphism $\phi:G\to G$, the $k$-algebra $\mathscr{O}(G_\aff)=\mathscr{O}(G)$ is invairant under $\phi^\sharp$, so we obtain a commutative diagram
\[\begin{tikzcd}
G\ar[r,"\phi"]\ar[d]&G\ar[d]\\
G_\aff\ar[r,"\phi"]&G
\end{tikzcd}\]
Therefore $N$, as the kernel of the canonical morphism $G\to G_\aff$, is invariant under $\phi$, i.e. $N$ is characteristic.\par
Now if $\phi:G\to G$ is a $k$-automorphism of $G$, then we obtain a $k$-scheme $G'=G\stackrel{\phi}{\to} G\to\Spec(k)$. Applying the above arguments to $G'$, we then see that $(G')_\aff=$ $N_S$, being the kernel of the canonical morphism $G_S\to(G_S)_{\aff}$, is invariant under any automorphism of $G_S$, i.e. $N$ is a characteristic subgroup of $G$.\par
Now put $N'=\ker(N\to N_\aff)$; by (b), this is a normal subgroup of $G$. As $N$ is algebraic (being a closed subgroup of $G$), by (a) we have $N/N'\cong N_{\aff}$. Moreover, by \cref{scheme alg group group quotient exists} and \cref{site sheaf quotient by normal subgroup third isomorphism}, we have an isomorphism of $k$-groups
\[(G/N')/N_{\aff}\cong(G/N')/(N/N')\cong G/N\cong G_{\aff}.\]
As $N_\aff$ is affine, the projection $G/N'\to G/N$ is also affine, so $G/N'$ is also affine ($G/N=\G_{\aff}$ being affine). Hence, by the universal property of $G_{\aff}$, the projection $p':G\to G/N'$ factors through $G/N=G_{\aff}$, whence $N\sub N'$ and $N=N'$. We then conclude that $N_{\aff}$ is the trivial group, so $\mathscr{O}(N)=k$.\par
Finally, assertion (\rmnum{5}) follows from the following general lemma:
\begin{lemma}\label{scheme alg group anti-affine is smooth connected commutative}
Let $k$ be a field and $N$ be an algebraic group over $k$ such that $\mathscr{O}(N)=k$. Then $N$ is smooth connected and commutative.
\end{lemma}
In fact, we can suppose that $k$ is algebraically closed. Then $H=N^0_\red$ is a subgroup of $N$, and the quotient $X=N/H$ is finite (hence affine) over $k$, by \cref{scheme alg group over perfect G/G_red char}. On the other hand, as $p:N\to X$ is faithfully flat, we have $\mathscr{O}(X)\sub\mathscr{O}(N)\sub k$, so it follows that $\mathscr{O}(X)=k$ and $X$ is trivial, i.e. $N=N_\red^0$, hence $N$ is smooth and connected. Finally, if $Z$ is the center of $N$, then by \cref{scheme alg group center is closed and quotient affine}, $N/Z$ is affine and we obtain similarly that $\mathscr{O}(N/Z)=k$, whence $N=Z$.
\end{proof}

We say that an algebraic group $N$ over a field $k$ is \textbf{anti-affine} if $\mathscr{O}(N)=k$. By \cref{scheme alg group anti-affine is smooth connected commutative}, such a group is therefore smooth connected and commutative, and for any algebraic group $G$, we have an exact sequence
\[\begin{tikzcd}
1\ar[r]&G_{\mathrm{ant}}\ar[r]&G\ar[r]&G^\aff\ar[r]&1
\end{tikzcd}\]
If $k$ is algebraically closed, a basis class of anti-affine algebraic groups is that of \textbf{abelian varieties}, that is, complete connected $k$-algebraic groups.

\section{Diagonalizable groups}
\subsection{Duality for group schemes}\label{scheme group duality}
Let $\mathcal{C}$ be a category, which we identity as a full subcategory of $\widehat{\mathcal{C}}=\PSh(\mathcal{C})$. Let $I$ be an abelina group functor over $\mathcal{C}$, i.e. an object of $\widehat{\mathcal{C}}$ endowed with an abelian group structure. For any $F\in\Ob(\widehat{\mathcal{C}})$, the object $\sHom(F,I)$ is then endowed with an abelian group structure, induced by that of $I$. For any group $G$ in $\widehat{\mathcal{C}}$, let $D(G)=\sHom_{\Grp}(G,I)$ be the sub-object of $\sHom(G,I)$ defined, for any $S\in\Ob(\mathcal{C})$, by:
\begin{equation}\label{category group D(G) functor def}
D(G)(S)=\Hom_{S\dash\Grp}(G_S,I_S),
\end{equation}
where $G_S=G\times S$ and $I_S=I\times S$ are considered as $S$-groups, i.e. as groups in $\widehat{\mathcal{C}}_{/S}$. Then $D(G)$ is a sub-$\widehat{\mathcal{C}}$-group of $\sHom(G,I)$. In this way, we obtain a contravariant functor $D$ from the category of $\widehat{\mathcal{C}}$-groups to the category of abelian $\widehat{\mathcal{C}}$-groups.\par
The right hand side of (\ref{category group D(G) functor def}) can also be interpreted as the subset of $\Hom(G\times S,I)$ formed by morphisms $G\times S\to I$ which is "multiplicative relative to the first argument $G$". Moreover, the formula (\ref{category group D(G) functor def}) also valid when $S$ is any object of $\widehat{\mathcal{C}}$, not necessarily in $\mathcal{C}$.\par
If now we let $S$ be a group in $\widehat{\mathcal{C}}$, which we denote by $G'$, then in the first mumber of (\ref{category group D(G) functor def}), we can distinguish the subset $\Hom_{\Grp}(G',D(G))$ formed by morphisms which respect the group structures of $G'$ and $D(G)$. It then corresponds to the subset of $\Hom(G\times G',I)$ formed by morphisms which are multiplicative relative to the first and second arguments, which will then be called \textbf{bilinear morphisms} from $G\times G'$ to $I$, or pairings of $G$ and $G'$ with values in $I$. We thus obtain
\begin{equation}\label{category pairing of D(G) isomorphism to bilinear form-1}
\Hom_{\Grp}(G',D(G))\stackrel{\sim}{\to}\mathrm{Bil}(G\times G',I),
\end{equation}
which is functorial on the couple $(G,G')$. As the second member of (\ref{category pairing of D(G) isomorphism to bilinear form-1}) on $G$ and $G'$, we then deduce a functorial bijection
\begin{equation}\label{category pairing of D(G) isomorphism to bilinear form-2}
\Hom_{\Grp}(G',D(G))\stackrel{\sim}{\to}\Hom_{\Grp}(G,D(G')).
\end{equation}
In other words, a group morphism $G'\to D(G)$ is equivalent to a morphism $G'\to D(G)$, which are both equivalent to giving a bilinear form $G\times G'\to I$. Applying this to the special case where $G'=D(G)$ and to the identity morphism of $D(G)$, we thus obtain a canonical homomorphism
\begin{equation}\label{category D(D(G)) canonical morphism}
G\to D(D(G)).
\end{equation}
We say that $G$ is \textbf{reflexive} (relative to $I$) if this homomorphism is an isomorphism. We note that this implies in particular that $G$ is abelian. We thus obtain:

\begin{proposition}
The functor $D$ induces an anti-equivalence on the category of reflexive $\widehat{\mathcal{C}}$-groups.
\end{proposition}

In particular, if $G,H$ are two reflexive groups, $D$ induces an isomorphism
\[\Hom_{\Grp}(G,H)\stackrel{\sim}{\to} \Hom_{\Grp}(D(H),D(G))\]
(in fact it is sufficient that $H$ is reflexive, as can be seen from the formula (\ref{category pairing of D(G) isomorphism to bilinear form-2})).\par
As usual, we say that a $\mathcal{C}$-group is reflexive if it is reflexive as an $\widehat{\mathcal{C}}$-group. We thus obtain similarly an anti-equivalence on the category of reflexive $\mathcal{C}$-groups.

\begin{remark}
We note that the formation of $D(G)$ commutes with base changes, which therefore transforms reflexive groups into reflective groups.
\end{remark}

We are interesting in particular in the case where $\mathcal{C}=\Sch_{/S}$, the category of schemes over $S$, and $I=\G_{m,S}$, the multiplicative grou over $S$. For any (ordinary) group $M$, we consider the constant $S$-group $M_S$. We also see that for any group scheme $G$ over $S$, we have a canonical isomorphism (functorial on $M$ and $G$ and compatible with base changes):
\[\Hom_{S\dash\Grp}(M_S,G)=\Hom_{\Grp}(M,G(S))\]
Apply this to $G=I=\G_{m,S}$ and to a scheme $S'$ over $S$, we then obtain a functorial isomorphism
\begin{equation}\label{scheme D(M_S) isomorphism to group homo to G_m}
D(M_S)(S')\stackrel{\sim}{\to}\Hom_{\Grp}(M,\G_{m,S}(S')).
\end{equation}
We thus recover the functor $D_S(M)$ considered in \ref{scheme diagonalizable group paragraph}, which is representable for $M$ abelian, and in this case,
\[D_S(M)=D(M_S)=\Spec(\mathscr{O}_S[M]),\]
where $\mathscr{O}_S[M]$ is the group algebra of $M$ with coefficients in $\mathscr{O}_S$ (note that by definition $D(M)$ is unchanged if we replace $M$ by its abelianization, so we can always assume that $M$ is abelian).

\begin{definition}
A group scheme $G$ over $S$ is called \textbf{diagonalizable} if it is isomorphic to a scheme of the form $D_S(M)=D(M_S)=\sHom_{S\dash\Grp}(M_S,\G_{m,S})$ for an abelian group $M$. We say that $G$ is \textbf{locally diagonalizable} if any point of $S$ admits an open neighborhood $U$ such that $G|_U$ is doagonalizable.
\end{definition}

\begin{theorem}\label{scheme group D(M_S) is reflexive}
Let $\Gamma$ be a constant abelian group scheme over $S$, i.e. isomorphic to a group scheme of the form $M_S$, where $M$ is an ordinary abelian group. Then $\Gamma$ is reflexive, i.e. the canonical homomorphism
\[\Gamma\to D(D(\Gamma))\]
is an isomorphism. The diagonalizable group $D(M_S)$ is hence also reflexive.
\end{theorem}

In view of the definitions, this theorem follows from the following result (applied to any scheme $S'$ over $S$):

\begin{corollary}\label{scheme group D(M_S) to G_m morphism is locally constant}
Let $G=D(M_S)$, then any homomorphism of $S$-groups
\[\chi:G\to\G_{m,S}\]
is defined by a uniquely determined section of $M_S$ over $S$, i.e. by a uniquely determined locally constant map from $S$ to $M$.
\end{corollary}
\begin{proof}
We first recall that $\Gamma(M_S/S)$ is identified with the set of locally constant maps from $S$ to $M$ (cf. \cite{SGA3-1} \Rmnum{1}, 1.8). As by definition we have
\[\G_{m,S}=\GL(1)_S=\sAut_{\mathscr{O}_S}(\mathscr{O}_S),\]
we see that giving a group homomorphism $\chi:G\to\G_{m,S}$ is equivalent to giving a $\mathscr{O}_S[G]$-module structure over $\mathscr{O}_S$, which is compatible with the natural $\mathscr{O}_S$-module structure of $\mathscr{O}_S$. By \cref{scheme module over diagonalizable group cat equivalent to graded module}, this amounts to giving a graduation of type $M$ over $\mathscr{O}_S$, i.e. a decomposition of $\mathscr{O}_S$ into modules $\mathscr{L}_m$ ($m\in M$). But in view of \cref{sheaf of module local free inverse image injective iff}, a direct factor of a locally free module of finite type is locally free of finite type, hence each $\mathscr{L}_m$ is, in a neighborhood of each point of $S$, either zero of free of rank $1$, and in this case is identified with $\mathscr{O}_S$ in this neighborhood. Let $S_m$ be the open subset of $S$ formed by points where $\mathscr{L}_m$ is isomorphic to $\mathscr{O}_S$. Since $\mathscr{O}_S$ is the direct sum of $\mathscr{L}_m$, we see that the union of $S_m$ is equal to $S$, and that $S_m$ are pairwise disjoint. Therefore, giving a group homomorphism $G\to\G_{m,S}$ is equivalent to giving a decomposition of $S$ as union of disjoint open subsets $S_m$ ($m\in M$), i.e. to giving a locally constant map from $S$ to $M$. This proves \cref{scheme group D(M_S) to G_m morphism is locally constant} and hence \cref{scheme group D(M_S) is reflexive}.
\end{proof}

\begin{corollary}\label{scheme group locally diagonalizable reflexive}
Any locally diagonalizable group is reflexive. If $M,N$ are two ordinary abelian groups, then the natural homomorphism
\[\Hom_{S\dash\Grp}(M_S,N_S)\to\Hom_{S\dash\Grp}(D(N_S),D(M_S))\]
is bijective.
\end{corollary}
\begin{proof}
The statement for locally diagonalizable groups follows from the uniqueness part of \cref{scheme group D(M_S) to G_m morphism is locally constant}, and the last assertion has alreay been remarked.
\end{proof}

Since the preceding isomorphism is compatible with base changes, we then deduce an isomorphism of $S$-groups
\begin{equation}\label{scheme group diagonalizable sHom isomorphism by D}
\sHom_{S\dash\Grp}(M_S,N_S)\stackrel{\sim}{\to} \sHom_{S\dash\Grp}(D(N_S),D(M_S)).
\end{equation}
For any $S$-scheme $T$, by adjunction, we have
\begin{equation}\label{scheme sHom of constant group char by global section}
\sHom_{S\dash\Grp}(M_S,N_S)(T)=\Hom_{\Grp}(M,\Hom_{T}(T,N_T))=\Hom_{\Grp}(M,\Gamma(N_T/T))
\end{equation}
and $\Gamma(N_T/T)$ is the abelian group of locally constant maps $T\to N$ by (\cite{SGA3-1} \Rmnum{1}, 1.8). On the other hand, let $\Hom_{\Grp}(M,N)_S$ be the constant $S$-group associated with the ordinary abelian group $\Hom_{\Grp}(M,N)$. We then have an evident homomorphism of abelian $S$-functors in groups
\begin{equation}\label{scheme group diagonalizable of Hom group morphism}
\theta:\Hom_{\Grp}(M,N)_S\to \sHom_{S\dash\Grp}(M_S,N_S).
\end{equation}
which is always a monomorphism. Moreover, this is an isomorphism if $M$ is finitely generated:

\begin{proposition}\label{scheme group diagonalizable of Hom isomorphism if ft}
Let $M,N$ be ordinary abelian groups and assume that $M$ is finitely generated. Then (\ref{scheme group diagonalizable of Hom group morphism}) is an isomorphism.
\end{proposition}
\begin{proof}
Let $F(M)$ and $G(M)$ be the left (resp. right) member of (\ref{scheme group diagonalizable of Hom group morphism}). If $M=M_1\oplus M_2$, then we have a canonical isomorphism $F(M)=F(M_1)\oplus F(M_1)$, and similarly for $G$. Therefore, it suffices to prove that $\theta$ is an isomorphism if $M=\Z/r\Z$ for an integer $r\geq 0$. In this case, $F(M)=(N[r])(S)$, where $N[r]$ is the kernel of $r\cdot\id_N$, and for any $T\to S$, the homomorphism
\[F(M)(T)=\Gamma(N[r]_T/T)\to G(M)(T)=\Gamma(N_T/T)[r]\]
is easily seen to be bijective, whence the assertion.
\end{proof}

\begin{corollary}\label{scheme group sHom of D(M) representable if ft}
Let $M,N$ be ordinary abelian groups and assume that $M$ is finitely generated. Then we have an isomorphism
\[\Hom_{\Grp}(M,N)_S\stackrel{\sim}{\to} \sHom_{S\dash\Grp}(D(N_S),D(M_S));\]
therefore $\sHom_{S\dash\Grp}(D(N_S),D(M_S))$ is representable.
\end{corollary}
\begin{proof}
This follows from \cref{scheme group diagonalizable of Hom isomorphism if ft} and \cref{scheme group locally diagonalizable reflexive}.
\end{proof}

\begin{corollary}\label{scheme group sHom of D(M) over connected base char}
Under the conditions of \cref{scheme group diagonalizable of Hom isomorphism if ft}, if $S$ is connected, we have
\[\Hom_{S\dash\Grp}(D_S(N),D_S(M))\stackrel{\sim}{\to}\Hom_{\Grp}(M,S),\quad \Iso_{S\dash\Grp}(D_S(N),D_S(M))\stackrel{\sim}{\to}\Iso_{\Grp}(M,S)\]
\end{corollary}
\begin{proof}
If $S$ is connected, then the set $\Hom_{\Grp}(M,N)_S(S)$ is identified with $\Hom_{\Grp}(M,N)$, whence the first assertion by \cref{scheme group sHom of D(M) representable if ft}. The case for $\Iso_{\Grp}(D_S(N),D_S(M))$ can be deduced from this, using the functoriality of $D_S$.
\end{proof}

To extend the preceding results to locally diagonalizable groups, we need to use the technique of descents. Let $S$ be a scheme and $X,Y,T$ be $S$-schemes. If $(T_i)$ is an open covering of $T$, and if we put $T_{ij}=T_i\cap T_j=T_i\times_TT_j$, then as a morphism of $T$-schemes $X_T\to Y_T$ cen be glued locally over $T$, we see that the following sequence is exact:
\begin{equation}\label{scheme sHom functor is sheaf}
\Hom_T(X_T,Y_T)\to\prod_i\Hom_{T_i}(X_{T_i},Y_{T_i})\rightrightarrows\prod_{ij}\Hom_{T_{ij}}(X_{T_{ij}},Y_{T_{ij}})
\end{equation}
i.e. the functor $\sHom_S(X,Y)$ is a sheaf over $\Sch_{/S}$ for the Zariski topology. More generally, by (\cite{SGA3-1} \Rmnum{4}, 4.5.13), it is a sheaf for any subcanonical topology over $\Sch_{/S}$.\par
Let $G,H$ be $S$-groups, we then deduce that the $S$-functor $\sHom_{S\dash\Grp}(X,Y)$ is a sheaf for the fpqc topology.

\begin{lemma}\label{scheme Zariski functor representable by descent lemma}
Let $F$ be a Zariski sheaf over $\Sch_{/S}$.
\begin{enumerate}
    \item[(a)] Suppose that there exists an open covering $(S_i)$ of $S$ such that the fiber product $F_i=F\times_SS_i$ is representable by an $S_i$-scheme $X_i$. Then $F$ is representable by an $S$-scheme $X$.
    \item[(b)] Suppose that $F$ is a fpqc sheaf and there exists a faithfully flat and quasi-compact morphism $S'\to S$ such that $F'=F\times_SS'$ is representable by an $S'$-scheme $X'$. Then $X'$ is endowed with a canonical descent data relative to $S'\to S$. If this descent data is effective (for example if $X'$ is affine over $S$), then $F$ is representable by an $S$-scheme $X$.
\end{enumerate}
\end{lemma}
\begin{proof}
We first consider the situation of (a). It follows from the hypothesis that $X_i\times_SS_j$ and $X_j\times_SS_i$ represent the two restriction of $F$ to $S_{ij}=S_i\times_SS_j$, hence, by Yoneda lemma, there exists a unique isomorphism of $S_{ij}$-schemes
\[\phi_{ji}:X_i\times_SS_j\stackrel{\sim}{\to} X_j\times_SS_i;\]
we then have isomorphisms of schemes over $S_{ijk}=S_i\times_SS_j\times_SS_k$:
\[\begin{tikzcd}[row sep=12mm, column sep=12mm]
X_i\times_SS_j\times_SS_k\ar[d,equal]\ar[r,"\phi_{ji}\times\id_{S_k}"]&X_j\times_SS_i\times_SS_k\ar[r,equal]&X_j\times_SS_k\times_SS_i\ar[d,"\phi_{kj}\times\id_{S_i}"]\\
X_i\times_SS_k\times_SS_j\ar[r,"\phi_{ki}\times\id_{S_j}"]&X_k\times_SS_i\times_SS_j\ar[r,equal]&X_k\times_SS_j\times_SS_i
\end{tikzcd}\]
and as these objects all represent the restriction of $F$ to $S_{ijk}$, this diagram is commutative, i,e, the $\phi_{ji}$ is a descent data over $X_i$. Thus we can glue $X_i$ to an $S$-scheme $X$ such that $X\times_SS_i=X_i$ for each $i$. For any $Y$ over $S$, the $Y_i=Y\times_SS_i$ form an open covering of $Y$; put $Y_{ij}=Y_i\times_SY_j=Y\times_SS_{ij}$. As $F$ (resp. $h_X$) is a Zariski sheaf by hypothesis, we have the following commutative diagram with exact rows:
\[\begin{tikzcd}
F(Y)\ar[r]\ar[d]&\prod_iF(Y_i)\ar[r,shift left=2pt]\ar[r,shift right=2pt]\ar[d,equal]&\prod_{ij}F(Y_{ij})\ar[d,equal]\\
h_X(Y)\ar[r]&\prod_ih_X(Y_i)\ar[r,shift left=2pt]\ar[r,shift right=2pt]&\prod_{ij}h_X(Y_{ij})
\end{tikzcd}\]
It follows that $X$ represents $F$, which proves (a).\par
As for (b), it follows from the hypotheses in (b) that $F_1''=F\times_{S'}S_1''$ (where $S_1''=S''=S'\times_SS'$, considered as an $S'$-scheme via the first projection) is represented by $X_1''=X'\times_{S'}S_1''$; similarly, $F_2''=F'\times_{S'}S_2''$ is represented by $X_2''=X'\times_{S'}S_2''$ (again, $S_2''=S''$ and is considered as an $S'$-scheme via the second projection). Now $F_1''=F\times_SS''=F_2''$, so there exists a unique $S''$-isomorphism $\phi:X_1''\to X_2''$.; then, if we denote by $q_i$ (resp. $\pr_{ji}$) the projections of $S'''=S'\times_SS'\times_SS'$ onto the $i$-th factor (resp. to the $i$-th and $j$-th factor), $X_i'''=X'\times_{S'}S_i'''$ (where $S_i'''=S'''$ is considered as an $S'$-scheme using $q_i$), and $\pr_{ji}^*(\phi):X_i'''\stackrel{\sim}{\to}X_j'''$ the isomorphism of $S'''$-schemes induced from $\phi$ by base change, we obtain a diagram of isomorphisms of $S'''$-schemes
\[\begin{tikzcd}
X_1'''\ar[rd,swap,"\pr_{31}^*(\phi)"]\ar[r,"\pr_{21}^*(\phi)"]&X_2'''\ar[d,"\pr_{32}^*(\phi)"]\\
&X_3'''
\end{tikzcd}\]
and as these objects all represent the restriction of $F$ to $S'''$, this diagram is commutative, so $\phi$ is a descent data over $X'$ relative to $S'\to S$. Suppose further that this descent data is effective, i.e. there exists an $S$-scheme $X$ such that $X'\cong X\times_SS'$ (by \cite{SGA1} \Rmnum{8} 2.1, this is the case if $X'$ is affine over $S'$). For any $Y\to S'$, put $Y'=Y\times_SS'$ and $Y''=Y'\times_YY'=Y\times_SS''$, then $Y'\to Y$ is, as $S'\to S$, faithfully flat and quasi-compact, hence a universally effective epimorphism (\cref{scheme topology T_i family M_i}). Therefore, the equivalence relation
\[Y'\times_YY'\rightrightarrows Y'\]
is $\mathcal{M}$-effective ($\mathcal{M}$ is the family of faithfully flat and quasi-compact morphisms) with quotient $Y$. As $F$ (resp. $h_X$) is a fpqc sheaf by hypothesis, we again obtain a commutative diagram with exact rows:
\[\begin{tikzcd}
F(Y)\ar[r]\ar[d]&F(Y')\ar[r,shift left=2pt]\ar[r,shift right=2pt]\ar[d,equal]&F(Y'\times_YY')\ar[d,equal]\\
h_X(Y)\ar[r]&h_X(Y')\ar[r,shift left=2pt]\ar[r,shift right=2pt]&h_X(Y'\times_YY')
\end{tikzcd}\]
It then follows that $X$ represents $F$, whence assertion (b).
\end{proof}

\begin{corollary}\label{scheme fpqc functor representable by descent}
Let $F$ be a fpqc sheaf over $\Sch_{/S}$. Suppose that there exists an open covering $(S_i)$ of $S$ and for each $i$ a faithfully flat and quasi-compact morphism $S'_i\to S_i$ such that $F_i'=F\times_SS_i'$ is representable by an $S_i'$-scheme $X_i'$ which is affine over $S'_i$. Then $F$ is representable by an $S$-scheme which is affine over $S$ (such that $X\times_SS_i'=X_i'$ for each $i$). If moreover each $X_i'\to S_i'$ is a closed immersion (resp. a finite \'etale morphism), so is $X\to S$.
\end{corollary}
\begin{proof}
The first assertion follows from \cref{scheme Zariski functor representable by descent lemma}. For the second one, it suffices to verify that each morphism $X\times_SS_i\to S_i$ is a closed immersion (resp. finite \'etale), which follows from (\cite{EGA4-2}, 2.7.1) (resp. \cite{EGA4-4}, 17.7.3).
\end{proof}

\begin{corollary}\label{scheme group local diagonalizabl sHom representable}
If $G$ and $H$ are locally diagonalizable groups over $S$ with $H$ of finite type over $S$, then $\sHom_{S\dash\Grp}(G,H)$ is representable.
\end{corollary}
\begin{proof}
By \cref{scheme group sHom of D(M) representable if ft}, there exists an open covering $(S_i)$ of $S$ such that $\sHom_{S_i\dash\Grp}(G|_{S_i},H_{S_i})$ is representable for each $i$. Since $\sHom_{S\dash\Grp}$ is a Zariski sheaf, the corollary then follows form \cref{scheme Zariski functor representable by descent lemma}.
\end{proof}

\subsection{Scheme-theoretic properties}
\begin{proposition}\label{scheme group diagonalizable schematic prop}
Let $S$ be a nonempty scheme, $M$ be an ordinary abelian group, $G=D(M_S)$ the diagonalizable $S$-group defined by $M$.
\begin{enumerate}
    \item[(a)] $G$ is faithfully flat and affine over $S$ (a fortiori quasi-compact over $S$).
    \item[(b)] $G$ is of fintie type over $S$ if and only if $M$ is finitely generated, and in this case $G$ is of fintie presentation over $S$.
    \item[(c)] $G$ is finite over $S$ if and only if $M$ is finite, if and only if $G$ is of finite type over $S$ and annihilated by an integer $n>0$. In this case, we have $\deg(G/S)=\Card(M)$.
    \item[(c')] $G$ is integral over $S$ if and only if $M$ is a torsion group.
    \item[(d)] $G$ is the trivial $S$-group if and only if $M$ is trivial.
    \item[(e)] $G$ is smooth over $S$ if and only if $M$ is finitely generated and the order of the torsion subgroup of $M$ is coprime to the residue characteristic of $S$.
\end{enumerate}
\end{proposition}
\begin{proof}
Since $D_S(M)$ is obtained by base change $D_\Z(M)$ to $S$, the first assertion (a) is clear. As for (b), it suffices to reduce to affine case and note that for any ring $A$, the group algebra $A[M]$ is of finite type over $A$ if and only if $M$ is finitely generated, and in this case $\Z\to\Z[M]$ is finitely presented, hence $A\to A[M]$ is finitely presented for any ring $A$. This proves (b), and (c) follows from a similar argument, noting that for a finitely generated abelian group $M$ is finite if and only if it is torsion. Finally, (c') follows from the fact that for any ring $A$, the $A$-algebra $A[M]$ is integral over $A$ if and only if $M$ is torsion.\par
We now prove (e). If $G$ is smooth over $S$, it is locally of finite presentation over $S$, so $M$ is finitely generated and $G$ is of finite presentation over $S$. Conversely, if $G$ is of finite presentation over $S$ (and flat over $S$ by (a)), then it is smooth over $S$ if and only if the geometric fibers are smooth, so we are reduced to the case where $S$ is the spectrum of an algebraically closed field $k$. Write $M=T\oplus L$, where $T$ is the torsion subgroup of $M$ and $L$ is a free abelian group, then we have $D(M)=D(T)\times D(L)$, where $D(L)\cong\G_m^r$ is smooth over $k$. Therefore, $G=D(M)$ is smooth over $k$ if and only if $D(T)$ is. Now $T$ is isomorphic to a direct sum of groups of the form $\Z/n_i\Z$, with $n$ being the product of $n_i$, so $D(T)$ is a product of $D(\Z/n_i\Z)=\bm{\mu}_{n_i}$. Since the $k$-group is represented by $\Spec(k[T]/(T^{n_i}-1))$, we then conclude that $D(T)$ is smooth if and only if each $n_i$ is coprime to $p=\char(k)$, i.e. if and only if $n$ is coprime to $p$.
\end{proof}

\begin{theorem}\label{scheme group diagonalizable exact sequence prop}
Let $S$ be a scheme and
\[\begin{tikzcd}
0\ar[r]&M'\ar[r,"u"]&M\ar[r,"v"]&M''\ar[r]&0
\end{tikzcd}\]
be an exact sequence of ordinary abelian groups. Consider the following sequence of $S$-schemes:
\[\begin{tikzcd}
0\ar[r]&D_S(M'')\ar[r,"v^t"]&D_S(M)\ar[r,"u^t"]&D_S(M'')\ar[r]&0
\end{tikzcd}\]
\begin{enumerate}
    \item[(a)] $v^t$ induces an isomorphism from $D_S(M'')$ to the kernel of $u^t$, and $u^t$ is faithfully flat and quasi-compact.
    \item[(b)] $D_S(M')$ represents the fpqc quotient $D_S(M)/D_S(M'')$.
\end{enumerate}
\end{theorem}
\begin{proof}
Let $\mathcal{M}$ be the family of faithfully flat and quasi-compact morphisms. First, (\rmnum{2}) follows from (\rmnum{1}). In fact, the equivalence relation of $D_S(M)$ defined by $u^t$ is the same as that defined by the subgroup $\ker(u^t)=D_S(M'')$; as $u^t\in\mathscr{M}$,  this equivalence relation is $\mathcal{M}$-effective, and hence $D_S(M')$ represents the quotient sheaf for the fpqc topology (cf. \cref{site sheaf M-morphism kernel M-effective}).\par
The first assertion of (\rmnum{1}) is a trivial concequence of the definition of $D_S(-)$. More generally, for any exact sequence
\[\begin{tikzcd}
M'\ar[r]&M\ar[r]&M''\ar[r]&0
\end{tikzcd}\]
the transposed sequence is exact:
\[\begin{tikzcd}
0\ar[r]&D_S(M'')\ar[r]&D_S(M)\ar[r]&D_S(M')
\end{tikzcd}\]
On the other hand, as $D_S(M)$ and $D_S(M')$ are affine over $S$, $u^t$ is necessarily an affine morphism, a fortiori quasi-compact. The second assertion of (\rmnum{1}) now follows from \cref{scheme group diagonalizable transpose mono epi iff}~(a) below. 
\end{proof}

\begin{corollary}\label{scheme group diagonalizable transpose mono epi iff}
Let $S$ be a nonempty scheme and $u:M'\to M$ be a homomorphism of ordinary abelian groups, $u^t:G\to G'$ be the transpose morphism.
\begin{enumerate}
    \item[(a)] For $u$ to be a monomorphism, it is necessary and sufficient that $u^t$ is faithfully flat.
    \item[(b)] For $u$ to be an epimorphism, it is necessary and sufficient that $u^t$ is a monomorphism (and in this case $u^t$ is a closed immersion).
\end{enumerate}
\end{corollary}
\begin{proof}
To prove (a), we note that if $u$ is a monomorphism, then $\mathscr{O}_S[M]$ is a module over $\mathscr{O}_S[M']$ admitting a nonempty basis (namely, the system of sections defined by any system of representatives of $M$ modulo $M'$), a fortiori it is faithfully flat. Conversely, if this is the case, then $u^t:\mathscr{O}_S[M']\to\mathscr{O}_S[M]$ is injective, which implies (for $S\neq\emp$) that $u:M'\to M$ is injective.\par
To prove (b), we note that if $u$ is an epimorphism, then $\mathscr{O}_S[M']\to\mathscr{O}_S[M]$ is surjective, hence $u^t$ is a closed immersion and a fortiori a monomorphism. Conversely, if $u^t$ is a monomorphism, then $\ker u^t$ is trivial. If we put $M''=\coker u$, then we see that $\ker u^t=D_S(M'')$ by the left-exactness of $D_S$, so by \cref{scheme group diagonalizable schematic prop} we have $M''=0$ hence $u$ is an epimorphism.
\end{proof}

\begin{corollary}
Let $M'\stackrel{u}{\to} M\stackrel{v}{\to}M''$ be an exact sequence of ordinary abelian groups, and consider the transpose sequence
\[G''\stackrel{v^t}{\longrightarrow}G\stackrel{u^t}{\longrightarrow}G'.\]
Then $v^t$ induces a faithfully flat and quasi-compact morphism from $G''$ to $\ker u^t$, and the latter is a diagonalizable group isomorphic to $D_S(v(M))=D_S(\coker u)$.
\end{corollary}
\begin{proof}
It suffices to consider the extended sequence $M'\stackrel{u}{\to} M\stackrel{v}{\to}v(M)=\coker u$ and apply \cref{scheme group diagonalizable exact sequence prop}. 
\end{proof}

\begin{corollary}\label{schem group diagonalizable morphism quotient by ker im prop}
Let $S$ be a scheme, $u:G\to H$ be a homomorphism of locally diagonalizable $S$-groups, with $H$ of finite type over $S$. Put $G'=\ker u$, then:
\begin{enumerate}
    \item[(a)] $G'$ is locally diagonalizable, and is of finite type over $S$ if $G$ is.
    \item[(b)] The quotient $G/G'$ exists, or more precisely the equivalence relation defined by $G'$ over $G$ is $\mathcal{M}$-effective ($\mathcal{M}$ is the family of faithfully flat and quasi-compact morphisms). Moreover $G/G'$ is locally diagonalizable, of finite type over $S$.
    \item[(c)] The homomorphism $u:G\to H$ factors uniquely into
    \[\begin{tikzcd}
    G\ar[r,"v"]&G/G'\ar[r,"w"]&H
    \end{tikzcd}\]
    where $v$ is the canonical homomorphism (hence faithfully flat and quasi-compact) and $w$ is a closed immersion.
    \item[(d)] The quotient $H'=H/\im w=\coker w=\coker u$ exists. More precisely, the equivalence relation defined by $G/G'$ over $H$ is $\mathcal{M}$-effective, and $H'$ is of finite type over $S$.
\end{enumerate}
\end{corollary}
\begin{proof}
The first assertion is a concequence of (b), by the definition of the quotient $G'/G$. To show that the fpqc quotient sheaf $\widetilde{G}/\widetilde{G}'$ is representable, since it is local over $S$, we can suppose that $G$ and $H$ are diagonalizable, of the form $D_S(M)$ and $D_S(N)$, and $S$ is connected. As $H$ is of finite type over $S$, $N$ is finitely generated by \cref{scheme group diagonalizable schematic prop}, so by \cref{scheme group sHom of D(M) representable if ft}, $u$ is defined by a homomorphism $u':N\to M$. Then, in view of \cref{scheme group diagonalizable exact sequence prop} and \cref{scheme group diagonalizable transpose mono epi iff}, $G'$ is isomorphic to $D_S(\coker u')$, and $\widetilde{G}/\widetilde{G}'$ is represented by $D_S(\im u')$. Further, consider the exact sequence
\[\begin{tikzcd}
0\ar[r]&\ker u'\ar[r]&N\ar[r,"w^t"]&\im u'\ar[r]&0
\end{tikzcd}\]
we obtain that $w$ is a closed immersion and the quotient $H'=H/\im w$ is represented by $D_S(\ker u')$; this is of finite type over $S$ since $N$, and hence $\ker u'$, is of finite type.
\end{proof}

\begin{corollary}\label{scheme group diagonalizable n-torsion subgroup prop}
Let $G$ be a diagonalizable group scheme over $S$, and $n\neq 0$ be an integer. Then the subgroup ${_nG}$ of $G$, the kernel of the homomorphism $n\cdot\id_G:G\to G$, is integral over $S$, and finite over $S$ if $G$ is of finite type over $S$.
\end{corollary}
\begin{proof}
If $G=D_S(M)$, then ${_nG}=D_S(M/nM)$ in view of \cref{scheme group diagonalizable exact sequence prop}, and we conclude by \cref{scheme group diagonalizable schematic prop}.
\end{proof}

\subsection{Torsors under a diagonalizable group}
Let $S$ be a scheme and $G=D_S(M)$ be a diagonalizable group over $S$. We consider \textbf{$G$-torsors} (or \textbf{principal homogeneous $G$-bundles}) for the fpqc topology. Recall that this is a scheme $P$ over $S$, acted by $G$ (on the right), such that any point of $S$ admits an open neighborhood $U$ and a faithfully flat and quasi-compact morphism $S'\to U$ such that $P'=P\times_SS'$ is isomorphic to $G'=G\times_SS'$. As $G$ is affine over $S$, it follows from (\cite{SGA1}, \Rmnum{8} 5.6) that such a $P$ is necessarily affine over $S$. We also note that since $G$ is itself faithfully flat and quasi-compact over $S$, a scheme $P$ is principal homogeneous under $G$ if and only if it is formally principal homogeneous, and it is moreover faithfully flat and quasi-compact over $S$ (cf. \cref{site formally principal homogeneous under M-group iff}~(\rmnum{3})).\par
Recall on the other hand that (cf. \cref{scheme module over diagonalizable group cat equivalent to graded module}) giveing an $S$-scheme $P$, affine over $S$, acted by the group $G=D_S(M)$ is equivalent to giving a quasi-coherent graded algebra $\mathscr{A}$ of type $M$ over $S$, i.e. a quasi-coherent algebra $\mathscr{A}$ over $S$, endowed with a decomposition into direct sums (as $\mathscr{O}_S$-modules):
\[\mathscr{A}=\bigoplus_{m\in M}\mathscr{A}_m\]
where $\mathscr{A}_m\cdot\mathscr{A}_n\sub\mathscr{A}_{m+n}$ for any $m,n\in M$.

\begin{proposition}\label{scheme group diagonalizable torsor char by algebra}
For a scheme $P$ affine over $S$ acted by $G=D_S(M)$ (defined by a quasi-coherent graded algebra $\mathscr{A}$ of type $M$) to be a $G$-torsor, it is necessary and sufficient that $\mathscr{A}$ satisfies the following conditions:
\begin{enumerate}
    \item[(\rmnum{1})] For any $m\in M$, $\mathscr{A}_m$ is an invertible module over $S$.
    \item[(\rmnum{2})] For any $m,n\in M$, the homomorphism
    \[\mathscr{A}_m\otimes_{\mathscr{O}_S}\mathscr{A}_n\to\mathscr{A}_{m+n}\] 
    induced by multiplication of $\mathscr{A}$, is an isomorphism.
\end{enumerate}
\end{proposition}
\begin{proof}
The necessity of these conditions are immediate by fpqc descent (\cite{SGA1}, \Rmnum{8}, remarque 1.12), since they are verified for the trivial $G$-torsor, i.e. where $\mathscr{A}=\mathscr{O}_S[M]$. On the otherh and, we first note that (\rmnum{1}) implies that $P$ is faithfully flat over $S$, and is also quasi-compact over $S$ since it is affine over $S$. Therefore, it remains to verify that $P$ is formally homogeneous under $G$, i.e. that the homomorphism
\[P\times_SG\to P\times_SP\]
is an isomorphism. Now this morphism corresponds to the homomorphism
\[\mathscr{A}\otimes\mathscr{A}\to\mathscr{A}\otimes\mathscr{O}_S(M)\]
whose $(m,n)$-component (where $m,n\in M$) is given by (cf. the arguments before \cref{scheme module over diagonalizable group cat equivalent to graded module})
\[x_m\otimes y_n\mapsto x_my_n\otimes e_n.\]
Therefore, we see that condition (\rmnum{2}) is equivalent to that $P$ is formally principal homogeneous, and this proves the proposition.
\end{proof}

\begin{corollary}\label{scheme group diagonalizable torsor O_S to A_0 isomorphism}
The conditions of \cref{scheme group diagonalizable torsor char by algebra} imply that the homomorphism $\mathscr{O}_S\to\mathscr{A}_0$ is an isomorphism.
\end{corollary}
\begin{proof}
In fact, we have $\mathscr{A}_0\otimes_{\mathscr{O}_S}\mathscr{A}_0\cong\mathscr{A}_0$ and $\mathscr{A}_0$ is invertible.
\end{proof}

If for example $M=\Z$, then the conditions of \cref{scheme group diagonalizable torsor char by algebra} are equivalent to that $\mathscr{A}_1=\mathscr{L}$ is invertible and 
\[\mathscr{A}=\bigoplus_{n\in\Z}\mathscr{L}^{\otimes n}\]
(isomorphism of graded algebras). We therefore obtain the following special case:

\begin{corollary}\label{scheme G_m-torsor and invertible sheaf}
There is an equivalence between the category of $\G_{m,S}$-torsors $P$ over $S$ to the category of invertible modules $\mathscr{L}$ over $S$, which associates an invertible module $\mathscr{L}$ over $S$ to the spectrum of the graded algebra $\bigoplus_{n\in\Z}\mathscr{L}^{\otimes n}$.
\end{corollary}

\begin{corollary}\label{scheme G_m-torsor isomorphism class and Picard group}
The group of isomorphism classes of $\G_{m,S}$-torsors over $S$ is isomorphic to the Picard group $\Pic(S)$, i.e. to $H^1(S,\mathscr{O}_S^\times)$.
\end{corollary}

Since $\G_{m,S}$ can also be considered as the scheme $\sAut_{\mathscr{O}S}(\mathscr{O}_S)$ of automorphisms of the module $\mathscr{O}_S$, we see that \cref{scheme G_m-torsor isomorphism class and Picard group} is equivalent to the following statement, which is a variant of Hilbert's theorem 90:

\begin{corollary}\label{scheme G_m-torsor Zariski local trivial}
Any $\G_{m,S}$-torsor over $G$ is locally trivial (in the sense of the Zariski topology).
\end{corollary}
\begin{proof}
For any $S$-scheme $Y$, we note that
\[\Hom_{\mathscr{O}_S\text{-}\mathbf{alg}}(\bigoplus_{n\in\Z}\mathscr{L}^{\otimes n},\mathscr{A}(Y))=\Hom_{\mathscr{O}_Y\dash\mathbf{alg}}(\bigoplus_{n\in\Z}\mathscr{L}^{\otimes n}\otimes_{\mathscr{O}_S}\mathscr{O}_Y,\mathscr{O}_Y)\]
is isomorphic to $\Iso_{\mathscr{O}_Y}(\mathscr{L}\otimes\mathscr{O}_S\mathscr{O}_Y,\mathscr{O}_Y)$, so the equivalence in \cref{scheme G_m-torsor and invertible sheaf} associates an invertible sheaf $\mathscr{L}$ with the $\G_{m,S}$-torsor $\sIso_{\mathscr{O}_S}(\mathscr{L},\mathscr{O}_S)$. The fact that this is an equivalence of categories means that every $\G_{m,S}$-torsor is of the form $\sIso_{\mathscr{O}_S}(\mathscr{L},\mathscr{O}_S)$, which is clearly Zariski trivial since $\mathscr{L}$ is invertible (this can also be proved using the fact that invertible sheaves have fpqc descent).
\end{proof}

\begin{remark}
We note that the preceding result is not valid for the group $\bm{\mu}_n$, or for a twisted form of $\G_m$. For example, let $S^1$ be the kernel of the norm morphism $N:\Res_{\C/\R}\G_{m,\C}\to\G_{m,\R}$ (induced by the norm from $\C$ to $\R$), which is a $\C/\R$-form of $\G_{m,\R}$. The equation $N(z)=-1$ in $\Res_{\C/\R}\G_{m,\C}$ defines a $S^1$-torsor $X$ over $\Spec(\R)$, which is locally trivial for the \'etale topology, but nontrivial since $X(\R)=\emp$. We now prove that $H_{\et}^1(\R,S^1)=\Z/2\Z$, so that there is no other $S^1$-torsors over $\Spec(\R)$. In fact, we have an exact sequence of smooth algebraic $\R$-groups
\[\begin{tikzcd}
1\ar[r]&S^1\ar[r]&\Res_{\C/\R}\G_{m,\C}\ar[r]&\G_{m,\R}\ar[r]&1
\end{tikzcd}\]
so we obtain a long exact sequence on \'etale cohomology:
\[\begin{tikzcd}
0\ar[r]&S^1(\R)\ar[r]&\C^\times\ar[r,"N"]&\R^\times\ar[r]&H^1_{\et}(\R,S^1)\ar[r]&H^1_{\et}(\R,\Res_{\C/\R}\G_{m,\C})
\end{tikzcd}\]
But we have $H_{\et}^1(\R,\Res_{\C/\R}\G_{m,\C})\cong H^1_{\et}(\C,\G_{m,\C})$ (cf. \cite{SGA3-3} \Rmnum{24}, 8.4), and this is trivial by \cref{scheme G_m-torsor isomorphism class and Picard group}. We then obtain an isomorphism $H^1_{\et}(\R,S^1)\cong\R^\times/N(\C^\times)\cong\Z/2\Z$.
\end{remark}

\begin{proposition}\label{scheme group diagonalizable torsor char by A_0}
Under the hypothesis in \cref{scheme group diagonalizable torsor char by algebra}, the conditions (\rmnum{1}) and (\rmnum{2}) are equivalent the following conditions:
\begin{enumerate}
    \item[(\rmnum{1}')] $\mathscr{O}_S\to\mathscr{A}_0$ is an isomorphism.
    \item[(\rmnum{2}')] For any $m\in M$, we have $\mathscr{A}_m\cdot\mathscr{A}_{-m}=\mathscr{A}_0$.
\end{enumerate}
\end{proposition}
\begin{proof}
These conditions are necessary by \cref{scheme group diagonalizable torsor char by algebra}. To prove the converse, we first consider the case $M=\Z$. That is, we show that if $A=\bigoplus_{n\in \Z}A_n$ is a graded ring such that $A_1\cdot A_{-1}=A_0$, then each $A_n$ is a projective $A_0$-module and
\[A_m\otimes_{A_0}A_n\to A_{m+n},\quad m,n\in\Z\]
is an isomorphism. For this, we note that by hypothesis, there exists $f_i\in A_1$, $g_i\in A_{-1}$ such that
\begin{equation}\label{scheme group diagonalizable torsor char by A_0-1}
\sum_if_ig_i=1.
\end{equation}
As the conclusion to be established is local over $\Spec(A_0)$ and as, by \cref{scheme group diagonalizable torsor char by A_0-1}, $\Spec(A_0)$ is covered by the affine open subsets $D(f_ig_i)$, we may reduce ourselves to the case where there exists an element $f\in A_1$ which is invertible in $A$. Then for each $n\in\Z$, we obtain an isomorphism $h\mapsto f^nh$ from $A_0$ to $A_n$, whence0 an isomorphism $A_0[t,t^{-1}]\to A$ of $A_0$-algebras, from which our assertion is clear.\par
Now in the general case, we see that under the conditions of \cref{scheme group diagonalizable torsor char by A_0}, each $\mathscr{A}_m$ ($m\in M$) is invertible. To prove the second condition of \cref{scheme group diagonalizable torsor char by algebra}, we can suppose that $\mathscr{A}_m$ and $\mathscr{A}_n$ have basis $f_m$ and $f_n$, with inverses $f_m^{-1}\in\Gamma(S,\mathscr{A}_{-m})$, $f_n^{-1}\in\Gamma(S,\mathscr{A}_{-n})$. Then the homothety with ratio $f_m^{-1}f_n^{-1}\in\Gamma(S,\mathscr{A}_{-m-n})$ defines an isomorphism $\mathscr{A}_{m+n}\to\mathscr{A}_0\cong\mathscr{O}_S$, which sends the image of $f_m\otimes f_n$ in $\mathscr{A}_{m+n}$ to the unit section $1$ of $\mathscr{O}_S$. In the diagram
\[\begin{tikzcd}
\mathscr{A}_m\otimes\mathscr{A}_n\ar[r,"u"]&\mathscr{A}_{m+n}\ar[r,"v"]&\mathscr{A}_0\cong\mathscr{O}_S
\end{tikzcd}\]
we see that $w$ and $v$ are epimorphisms of invertible sheaves, hence are isomorphisms, whence $u$ is an isomorphism.
\end{proof}

\subsection{Quotient by diagonalizable groups}
In this subsection, we denote by $\mathcal{M}$ the family of faithfully flat and quasi-compact morphisms, and the torsors are considered for the fpqc topology. Our main result is the following:

\begin{theorem}\label{scheme group diagonalizable free action quotient}
Let $S$ be a scheme, $M$ be an ordinary abelian group, $G=D_S(M)$ be the diagonalizable group over $S$ defined by $M$, $P$ be an $S$-scheme affine over $S$ acted freely by $G$ on the right. Then the equivalence relation defined by $G$ over $P$ is $\mathcal{M}$-effective, i.e. the quotient sheaf $X=P/G$ exists and $P$ is a $G_X=D_X(M)$-torsor over $X$. Further, $P/G$ is affine over $S$; more precisely, if $P$ is defined by a graded algebra $\mathscr{A}$ of type $M$, then $P/G$ is isomorphic to $\Spec(\mathscr{A}_0)$, where $\mathscr{A}_0=\mathscr{A}^G$ is the zero-th component of $\mathscr{A}$.
\end{theorem}

Put $X=\Spec(\mathscr{A}_0)$, then we have a structural morphism $P\to X$ induced by $\mathscr{A}_0\to\mathscr{A}$, which is invariant under the action of $G$. In this way, $P$ is an $X$-scheme which is affine over $X$ and acted by $G_X=D_X(M)$, and the hypothesis that $G$ acts freely on $P/S$ implies that $G_X$ acts freely on $P/X$. It then remains to show that $P$ is a $G_X$-torsor, using the fact that $\mathscr{B}_0=\mathscr{O}_X$, where $\mathscr{B}$ is the graded $\mathscr{O}_X$-algebra of type $M$ defining $P/X$. We can then suppose that $X=S$ and $S$ is affine, hence $P$ is affine, given by a graded ring $A$ of type $M$ whose homogeneous components is denoted by $A_m$, and $S=\Spec(A_0)$. In view of \cref{scheme group diagonalizable torsor char by A_0}, we only need to prove that
\begin{equation}\label{scheme group diagonalizable free action quotient-1}
A_m\cdot A_{-m}=A_0\for m\in M.
\end{equation}
As in the proof of \cref{scheme group diagonalizable torsor char by algebra}, we see that (\ref{scheme group diagonalizable free action quotient-1}) is equivalent to that the morphism $P\times_SG\to P\times_SP$ is a closed immersion (not only a monomorphism), i.e. that the ring homomorphism
\[\theta:A\otimes_{A_0}A\to A[M],\quad b_n\otimes a_m\mapsto b_na_m\otimes e_m\]
is surjective\footnote{In fact, $\theta$ is surjective if and only if each piece $A_0\otimes e_m$ is in the image of $A\otimes_{A_0}A$, which means $A_m\cdot A_{-m}\to A_0$ is surjective.}. This is the case if we assume further that the equivalence relation defined by $G$ over $P$ is closed, but we will show that this follows from the assumption that $G$ acts freely over $P$ (this is in fact implicitly contained in the conclusion of \cref{scheme group diagonalizable free action quotient}, since $G\times_SP=G_X\times_XP$ is then isomorphic to $P\times_XP$, which is closed in $P\times_SP$ since $X$ is affine (hence separated) over $S$).\par
Let $R=P\times_SG$. The hypothesis that $G$ acts freely, i.e. that $R\to P\times_SP$ is a monomorphism, is equivalent to that the diagonal
\[R\to R'=R\times_{(P\times_SP)}R\]
is an isomorphism. We have $R=\Spec(A[M])$ and $R'=\Spec(A[M\times M]/\mathfrak{K})$, where $\mathfrak{K}$ is the ideal generated by the elements of the form
\[x_m(e_{m,0}-e_{0,m}),\quad m\in M,x_m\in A_m\]
Let $\phi:A[M\times M]\to A[M]$ be the surjective ring homomorphism defined by
\[xe_{m,n}\mapsto xe_{m+n},\quad m,n\in M,x\in A\]
(where $e_m$, $e_{m,n}=e_m\otimes e_n$ are elements of the canonical basis of $A[M]$ and $A[M\times M]$). Then the diagonal morphism $R\to R'$ corresponds to the ring homomorphism
\[\bar{\phi}:A[M\times M]/\mathfrak{K}\to A[M]\]
obtained by passing to quotient. Now the kernel of $\bar{\phi}$ is the ideal $\mathfrak{K}'$ generated by the elements $d_m=e_{m,0}-e_{0,m}$. We have $\mathfrak{K}\sub\mathfrak{K}'$, and the hypothesis that $G$ acts free on $P$, i.e. that $\bar{\phi}$ is an isomorphism, is equivalent to the equality $\mathfrak{K}=\mathfrak{K}'$, which can be expressed by the relations
\begin{equation}\label{scheme group diagonalizable free action quotient-2}
d_m\in\mathfrak{K}=\sum_pA[M\times M]A_pd_p,\for m\in M.
\end{equation}
Using the natural tri-graduation over $A[M\times M]$, and the fact that the first degree of $d_m$ is zero for any $m\in M$, this signifies that we can write $d_m$ into the form
\[d_m=fe_{r,s}(e_{p,0}-e_{0,p}),\quad f\in A_{-p}\cdot A_p,\]
and using the fact that the total degree of $d_m$ is $m$, we can reduce ourselves to the terms such that $r+s+p=m$. We then conclude that we have, for any $m\in M$, an expression
\begin{equation}\label{scheme group diagonalizable free action quotient-3}
d_m=e_{m,0}-e_{0,m}=\sum_{r,s}\lambda_{r,s}(e_{m-s,s}-e_{r,m-r})
\end{equation}
where $\lambda_{r,s}\in\mathfrak{J}_p=A_p\cdot A_{-p}\sub A_0$, and $p=m-(r+s)$.\par
To deduce (\ref{scheme group diagonalizable free action quotient-1}) from this, it suffices to establish the same relation module any maximal ideal of $A_0$. As the hypotheses are invariant under such a reduction, we may therefore assume that $A_0$ is a field.

\begin{lemma}\label{scheme group diagonalizable free action quotient lemma-1}
Under the preceding conditions (with $A_0$ be a field), if $M\neq 0$, there exists $p\in M\setminus\{0\}$ such that $\mathfrak{J}_p=A_0$.
\end{lemma}
\begin{proof}
If this is not the case, then each $\mathfrak{J}_p$ is zero in $A_0$ except $\mathfrak{J}_0$, so $\lambda_{r,s}=0$ except $r+s=m$. By comparing the coefficients of $e_{m,0}$ in (\ref{scheme group diagonalizable free action quotient-3}), we then obtain that $\lambda_{m,0}-\lambda_{m,0}=1$, which is absurd.
\end{proof}

\begin{lemma}\label{scheme group diagonalizable free action quotient lemma-2}
Under the preceding conditions (with $A_0$ be a field), for any proper subgroup $N$ of $M$, there exists $p\in M-N$ such that $\mathfrak{J}_p=A_0$.
\end{lemma}
\begin{proof}
Let $M'=M/N$ and consider the graded ring $A'$ of fype $M'$, whose underlying ring is $A$, and whose graduation is given by
\[A'_{m'}=\bigoplus_{m\in\pi^{-1}(m')}A_m\]
where $\pi:M\to M'=M/N$ is the canonical homomorphism. Geometrically, this construction means we consider the action of $P$ by the subgroup $G'=D_S(M')$ induced by $G$, which is therefore free. That is, the couple $(M',A')$ satisfies the hypothesis of \cref{scheme group diagonalizable free action quotient lemma-1}. We thus conclude from \cref{scheme group diagonalizable free action quotient lemma-1} that there exists $f_i\in A_{m_i}$ and $g_i\in A_{-m_i}$, with $m_i\in M$ belonging to a common coset of $M/N$, such that $\sum_if_ig_i=1$. Now as $A_0$ is a field, we can choose $m_j\in M-N$ and write
\[1=\Big(1-\sum_{i\neq j}f_ig_i\Big)^{-1}\cdot f_jg_j\]
and this shows that $\mathfrak{J}_{m_j}=A_0$, whence the lemma.
\end{proof}

Now note that we have $\mathfrak{J}_p\cdot\mathfrak{J}_q\sub\mathfrak{J}_{p+q}$ and $\mathfrak{J}_p=\mathfrak{J}_{-p}$, so if $N$ denotes the subset of $M$ of $m\in M$ such that $\mathfrak{J}_p=A_0$, then $N$ is a subgroup of $M$. Using \cref{scheme group diagonalizable free action quotient lemma-2}, we see that $N=M$, and this completes the proof of \cref{scheme group diagonalizable free action quotient}. 

\begin{corollary}\label{scheme group diagonalizable free action is closed}
Under the conditions of \cref{scheme group diagonalizable free action quotient}, the graph morphism
\[P\times_SG\to P\times_SP\]
is a closed immersion. In particular, if $\sigma$ is a section of $P$ over $S$, then the morphism $g\mapsto \sigma\cdot g$ from $G$ to $P$ defined by $\sigma$ is a closed immersion.
\end{corollary}

\begin{corollary}\label{scheme group diagonaizable quotient by monomorphism prop}
Let $G,H$ be two $S$-groups, with $G$ diagonalizable and $H$ affine over $S$, and let $u:G\to H$ be a monomorphism of $S$-groups. Then $u$ is a closed immersion, $H\backslash G=X$ exists and $H$ is a $G_X$-torsor over $X$, with $X$ affine over $S$.
\end{corollary}
\begin{proof}
By \cref{scheme group diagonalizable free action is closed}, the induced morphism $H\times_SG\to G\times_SG$ by $u$ is a closed immersion; since $G$ is faithfully flat and quasi-compact over $S$, it follows that $u$ is a closed immersion (cf. \cite{EGA1}, \Rmnum{8} 4.8 et \cite{EGA4-3}, 8.11). The other assertions are contained in \cref{scheme group diagonalizable free action quotient}.
\end{proof}

\begin{corollary}\label{scheme group diagonalizable free action finite condition}
Under the conditions of \cref{scheme group diagonalizable free action quotient}, if $G$ is of finite type over $S$ and $P$ is of finite type (resp. of finite presentation) over $S$, so is $X=P/G$.
\end{corollary}
\begin{proof}
By hypothesis, $G_X$ is then of finite type over $X$, hence of finite presentation over $X$ by \cref{scheme group diagonalizable schematic prop}. As $P$ is a $G_X$-torsor over $X$, we conclude that it is of finite presentation over $X$ (cf. \cite{EGA4-2}, 2.7.1). As it is also faithfully flat over $X$, we conclude the corollary from (\cite{EGA4-4}, 17.7.5).
\end{proof}

\section{Group schemes of multiplicative type}
\begin{definition}
Let $S$ be a scheme, $G$ be a group scheme over $S$. We say that $G$ is a group \textbf{of multiplicative type} if it is locally diagonalizable for the fpqc topology, i.e. if for any $s\in S$, there exists an open neighborhood $U$ of $s$ and a faithfully flat and quasi-compact morphism $S'\to U$ such that $G'=G\times_SS'$ is a diagonalizable $S'$-group.\par
We say that $G$ is of \textbf{quasi-isotrivial} multiplicative type if it is locally diagonalizable for the \'etale topology, i.e. if in the above definition we can take $S'\to U$ to be surjective \'etale, or equivalently (by taking the direct sum of $S'$) if there is an \'etale and surjective morphism $S'\to S$ such that $G'=G\times_SS'$ is a locally diagonalizable $S'$-group. If we can take $S'\to S$ to be finite and surjective \'etale, then we say that $G$ is of \textbf{isotrivial} multiplicative type.\par
Finally, we say that $G$ is of \textbf{locally trivial} multiplicative type (resp. \textbf{locally isotrivial}) if any $s\in S$ admits an open neighborhood $U$ such that $G\times_SU$ is a diagonalizable $U$-group (resp. a group of isotrivial multiplicative type, i.e. there exists a finite \'etale surjective morphism $S'\to U$ such that $G\times_SS'$ is a diagonalizable $S'$-group).
\end{definition}
Note that the previous notions all follow from that of diagonalizable group by a localization process, in the sense that we consider different topologies associated with $\Sch$. It is generally agreed that when the word "locally" is not made explicit, it is for the Zariski topology. In the terminology introduced here, "of locally trivial multiplicative type" is equivalent to "locally diagonalizable", and similarly "trivial" is equaivalent to "diagonalizable". For the five concepts introduced above, we have the following implication diagram, which results from the corresponding relations between the topologies in play:
\[\begin{tikzcd}[row sep=5mm,column sep=2mm]
&\text{general multiplicative type}&\\
&\text{quasi-isotrivial}\ar[u,Rightarrow]&\\
&\text{locally isotrivial}\ar[u,Rightarrow]&\\
\text{isotrivial}\ar[ru,Rightarrow]&&\text{locally trivial}\ar[lu,Rightarrow]\\
&\text{trivial}\ar[lu,Rightarrow]\ar[ru,Rightarrow]&
\end{tikzcd}\]

From a practical point of view, let us point out that all the groups of multiplicative type that we encounter will be quasi-isotrivial: we will see that if $G$ is of finite type over $S$, then $G$ is automatically quasi-isotrivial. But we will give examples where the group is not locally isotrivial. We will also see there that $G$ can be locally trivial, without being isotrivial nor a fortiori trivial (which easily implies that the inclusions of the diagram above are strict).\par
On the other hand, we will also see that when $S$ is locally Noetherian and normal (or more generally geometrically unibranch), any group of multiplicative finite type over $S$ is necessarily isotrivial, and moreover trivial as soon as it is locally trivial. This explains that most of the groups of multiplicative type which we meet in practice will be without doubt isotrivial, especially since we will see later that the maximal tori of semi-simple group schemes are automatically isotrivial.

\begin{definition}
Let $S$ be a scheme and $G$ be an $S$-group. We say that $G$ is a \textbf{torus} if it is locally (for the fpqc topology) isomorphic to a group of the form $\G_m^r$ (where $r\geq 0$ is an integer).
\end{definition}
With the above notions, this means we can choose morphisms $S'\to U\hookrightarrow S$ such that $G'$ is isomorphic to a group of the form $(\G_{m,S'})^r$. We note that the integer $r$ depends on $s\in S$, which is equal to the dimension of the fiber $G_s=G\otimes_S\kappa(s)$. This is clearly a locally constant function, as one can see easily. This observation can be generalized:

\begin{definition}
Let $G$ be a diagonalizable group scheme over a field $k$, hence $G$ is isomorphic to $D_k(M)$ for some ordinary abelian group $M$, defined up to isomorphisms and $M\cong\Hom_{k\dash\Grp}(G,\G_{m,k})$. The isomorphic class of $M$ is called the \textbf{type} of the diagonalizable group $G$, which is evidently invariant under base field change.
\end{definition}

Now if $G$ is a group scheme of multiplicative type over a scheme $S$, then for any $s\in S$, there exists an extension $k$ of $\kappa(s)$ such that $G\times_S\Spec(k)$ is a diagonalizable $k$-group\footnote{In fact, by hypothesis there exists an open neighborhood $U$ of $s$ and a faithfully flat and quasi-compact morphism $S'\to U$ such that $G_{S'}$ is diagonalizable. Then for any $s'\in S'$ lying over $s$, $\kappa(s')$ is an extension of $\kappa(s)$ and $G_{s'}=G\times_S\Spec(\kappa(s'))$ is diagonalizable.}, whose type is then independent of the chosen extension, and called the \textbf{type of $G$ at $s$}, or the \textbf{type of $G_s$}. In particular, if $G$ itself is diagonalizable, then the type of $G$ at $s$ is equal to that of $G$.\par
In the general case, if $G$ is a group of multiplicative type over $S$ and $M$ is an ordinary abelian group, we say that \textbf{$G$ is of type $M$} if it is of type $M$ at any point $s\in S$, i.e. if it is locally isomorphic to $D_S(M)$ for the fpqc topology.

\begin{remark}\label{scheme group multiplicative partiton base remark}
We see immediately that for a group $G$ of multiplicative type over $S$, the function $s\mapsto\text{type of $G_s$}$ is locally constant over $S$: in fact, with the notations above, if $G'$ is of type $M$, then $G$ is of type $M$ over the neighborhood $U$ of $s$. We thus obtain a canonical partition of $S$ into subschemes $S_i$ such that for each $i$, $G_{S_i}$ is of type $M_i$, where $M_i$ are non-isomorphic ordinary abelian groups.\par
In particular, if $S$ is connected, then the type of fibers of $G$ is constant, i.e. there exists an ordinary abelian group $M$ such that $G$ is of type $M$. Finally, if $G$ is a torus, then the type of $G$ at $s$ is characterized by the integer $\dim(G_s)$ (in fact, $G_s$ is of type $\Z^r$, where $r=\dim(G_s)$).
\end{remark}

\begin{remark}
It is clear that the above definitions are stable under base change. Thus, if $G$ is a group scheme over $S$ and $S'\to S$ is a base change morphism, then if $G$ is of multiplicative type (resp. isotrival, etc.), so is $G'$. If the morphism $S'\to S$ is also faithfully flat and quasi-compact, then $G'$ is of multiplicative type (resp. a torus) if and only if $G$ is. Similarly, if $S'\to S$ is \'etale (resp. finite \'etale), then $G'$ is quasi-isotrivial (resp. isotrivial) if and only if $G$ is. Finally, for a general base change morphism $S'\to S$, if $s'\in S'$ and $s=f(s')$, then the type of $G'$ at $s'$ is equal to that of $G$ at $s$, since we have $G'_{s'}=G_s\otimes_{\kappa(s)}\kappa(s')$.
\end{remark}

\subsection{Extension of cartain properties to groups of multiplicative type}
We denote by $\mathcal{M}$ the family of faithfully flat and quasi-compact morphisms. As we shall see, certain properties for diagonalizable groups extends immedaitely to groups of multiplicative type.

\begin{proposition}\label{scheme group multiplicative schematic prop}
Let $G$ be a group of multiplicative type over a scheme $S$. We have the following:
\begin{enumerate}
    \item[(a)] $G$ is faithfully flat over $S$ and affine over $S$ (a fortiori quasi-compact over $S$).
    \item[(b)] $G$ is of finite type over $S$ if and only if for any $s\in S$, the type of $G$ at $s$ is a finitely generated abelian group. In this case, $G$ is of finite presentation over $S$.
    \item[(c)] $G$ is finite over $S$ if and only if for any $s\in S$, the type of $G$ at $s$ is given by a finite abelian group. If $S$ is quasi-compact, this is equivalent to that $G$ is of fintie type over $S$ and annihilated by an integer $n>0$.
    \item[(c')] $G$ is integral over $S$ if and only if for any $s\in S$, the type of $G$ at $s$ is given by a torsion abelian group.
    \item[(d)] $G$ is the trivial $S$-group if and only if for any $s\in S$, the type of $G$ at $s$ is the trivial groups.
    \item[(e)] $G$ is smooth over $S$ if and only if for any $s\in S$, the type of $G$ at $s$ is given by an abelian group whose torsion subgroup has order coprime to the characteristic of $\kappa(s)$.
\end{enumerate}
\end{proposition}
\begin{proof}
These results are concequences of \cref{scheme group diagonalizable schematic prop}, since they have fpqc descent (cf. \cite{SGA1} \Rmnum{8} ou \cite{EGA4-2}, \S 2).
\end{proof}

\begin{proposition}\label{scheme group multiplicative n-torsion finite}
Let $G$ be a group of multiplicative and fintie type over $S$. Then for any integer $n\neq 0$, the kernel ${_nG}$ of $n\cdot\id_G$ is a subgroup of multiplicative type and finite over $S$.
\end{proposition}
\begin{proof}
This follows from \cref{scheme group diagonalizable n-torsion subgroup prop}.
\end{proof}

\begin{proposition}\label{scheme group multiplicative free action quotient prop}
Let $G$ be a group of multiplicative and fintie type over $S$, acting freely (on the right) over an $S$-scheme $X$ which is affine over $S$. Then:
\begin{enumerate}
    \item[(a)] The equivalence relation defined by $G$ on $X$ is $\mathcal{M}$-effective, and $Y=X/G$ is affine over $S$.
    \item[(b)] If $X$ is of finite presentation (resp. of finite type) over $S$, so is $Y$. 
\end{enumerate}
\end{proposition}
\begin{proof}
The first assertion follows from \cref{scheme group diagonalizable free action quotient}, dealing with the case where $G$ is diagonalizable, and from \cref{category equivalence relation M-effective and descent}, since faithfully flat and quasi-compact morphisms are effective descent for the fibre category of schemes with affine morphisms, i.e. for any $Y'$ affine over $S$, endowed with a descent dat relative to $S'\to S$, this descent data is effective, i.e. $Y'$ comes from a scheme $Y$ which is affine over $S$ (cf. \cite{SGA1} \Rmnum{8} 2.1).\par
For the second assertion, we are equally reduced to the diagonalizable case, which follows from \cref{scheme group diagonalizable free action finite condition}, bacause the finite conditions can be descent by faithfully flat and quasi-compact morphisms (cf. \cite{SGA1} \Rmnum{8} 3.3 et 3.6).
\end{proof}

As in \cref{scheme group diagonalizable free action is closed} and \cref{scheme group diagonaizable quotient by monomorphism prop}, we obtain from \cref{scheme group multiplicative free action quotient prop} the following corollaries (note that closed immersions are fpqc local, cf. \cite{EGA1}, \Rmnum{8} 4.8 et \cite{EGA4-3}, 8.11):

\begin{corollary}\label{scheme group multiplicative free action graph closed}
Under the conditions of \cref{scheme group multiplicative free action quotient prop}, the graph morphism
\[X\times_SG\to X\times_SX\]
is a closed immersion. In particular, for any section $\sigma$ of $X$ over $S$, the corresponding morphism $G\to X,g\mapsto \sigma\cdot g$ is a closed immersion.
\end{corollary}

\begin{corollary}\label{scheme group mono multiplicative to affine prop}
Let $u:G\to H$ be a monomorphism of $S$-groups, with $G$ of multiplicative type and $H$ affine over $S$. Then $u$ is a closed immersion, $H/G=Y$ exists and is affine over $S$. Finally, $H$ is a $G_Y$-torsor over $Y$.
\end{corollary}
\begin{proof}
If we act $G$ on $H$ via $u$, then the morphism corresponding to the unit section of $H$ over $S$ is exactly $u$, so by \cref{scheme group multiplicative free action graph closed} it is a closed immersion. The other claims follow from \cref{scheme group multiplicative free action quotient prop}.
\end{proof}

\begin{remark}\label{scheme group mono multiplicative to fp sp prop}
Let $u:G\to H$ be a monomorphism of $S$-groups, with $G$ of multiplicative finite type over $S$ and $H$ of finite presentation and separated over $S$. Then by (\cite{SGA3-2} \Rmnum{8} 7.12 and 7.13 (a)), we can prove that $u$ is a closed immersion.
\end{remark}

\begin{proposition}\label{scheme group multiplicative morphism factorization}
Let $u:G\to H$ be a homomorphism of $S$-groups of multiplicative type, with $H$ of finite type over $S$. Then:
\begin{enumerate}
    \item[(a)] $G'=\ker u$ is an $S$-group of multiplicative type, and if of finite type if $G$ is.
    \item[(b)] The equivalence relation defined by $G'$ over $G$ is $\mathcal{M}$-effective, hence $u$ factors into
    \[\begin{tikzcd}
    G\ar[r]&G/G'\ar[r]&H
    \end{tikzcd}\]
    where $G/G'\to H$ is a closed immersion of $S$-groups and $G\to G/G'$ is faithfully flat and quasi-compact. Moreover, $G/G'$ is of multiplicative finite type over $S$.
    \item[(c)] The equivalence relation on $H$ defined by $I=G/G'$ is $\mathcal{M}$-effective, $H'=H/I$ exists, and $H'$ is of fintie type over $S$.
\end{enumerate}
\end{proposition}
\begin{proof}
As in \cref{scheme group multiplicative free action quotient prop}, we may assume that $G$ and $H$ are diagonalizable, and then the proposition reduces to \cref{schem group diagonalizable morphism quotient by ker im prop}.
\end{proof}

\begin{corollary}\label{scheme group multiplicative morphism ft category prop}
Let $S$ be a scheme.
\begin{enumerate}
    \item[(a)] The category of $S$-groups of multiplicative finite type over $S$ is abelian.
    \item[(b)] Let $u:G\to H$ be a homomorphism of the category in (a); for $u$ to be a monomorphism (resp. an epimorphism, resp. an isomorphism) in this category, it is necessary and sufficient that it is a monomorphism of schemes (resp. faithfully flat, resp. an isomorphism of schemes).
\end{enumerate}
\end{corollary}
\begin{proof}
The first assertion follows from \cref{scheme group multiplicative morphism factorization}, noting that the product of two $S$-groups of multiplicative type is again of multiplicative type, and of finite type over $S$ if these two groups are. The rest are immediate, for example, $u:G\to H$ is a monomorphism if and only if $\ker u$ is trivial, if and only if it is a closed immersion in view of \cref{scheme group multiplicative morphism factorization}. We also note that since $G$ and $H$ are affine over $S$, any $S$-morphism between them is quasi-compact by \cref{scheme morphism qc permanence prop}~(\rmnum{5}).
\end{proof}

\begin{corollary}\label{scheme group multiplicative ft mono epi locus prop}
Let $u:G\to H$ be a homomorphism of $S$-groups of multiplicative finite type over $S$. Let $U$ be the set of points $s\in S$ such that $u_s:G_s\to H_s$ is a monomorphism (resp. faithfully flat, resp. an isomorphism). Then $U$ is clopen and the induced morphism $u_U:G|_U\to H|_U$ is a monomorphism (resp. faithfully flat, resp. an isomorphism).
\end{corollary}
\begin{proof}
Let $P$ (resp. $Q$) be the kernel (resp. cokernel) of $u$. By \cref{scheme group multiplicative morphism factorization}, $Q$ exists and $P$, $Q$ are of multiplicative type, and their formation are stable under base change $S'\to S$; in particular, stable under taking fibers. On the other hand, $u$ is a monomorphism (resp. faithfully flat, resp. an isomorphism) if and only if $P=0$ (resp. $Q=0$, resp. $P=Q=0$). We are therefore reduced to verify that: if $R$ is an $S$-group of multiplicative type, then the set $U$ of points $s\in S$ such that $R_s=0$ is clopen, and $R|_U=0$. This is contained in \cref{scheme group multiplicative partiton base remark} and \cref{scheme group multiplicative schematic prop}~(d).
\end{proof}

\begin{corollary}\label{scheme group multiplicative subgroup inclusion locus prop}
Let $G$ be an $S$-group of multiplicative finite type over $S$, $H$, $H'$ be subgroup of multiplicative type.
\begin{enumerate}
    \item[(a)] $H''=H\cap H'$ is a subgroup of multiplicative type of $G$ and of finite type over $S$.
    \item[(b)] Let $U$ be the set of $s\in S$ such that $H_s\sub H'_s$ (resp. $H_s=H'_s$), then $U$ is clopen and $H|_U\sub H'|_U$ (resp. $H|_U=H'|_U$). 
\end{enumerate}
\end{corollary}
\begin{proof}
Of course, the notation $H\cap H'$ means the intersection in the sense of functors, i.e. $H\times_GH'$, which is a sub-$S$-group of $G$. By applying \cref{scheme group multiplicative morphism factorization} to the inclusion $H\to G$, we see that $H$ is of finite type; similarly $H'$ is of finite type. It then follows that $H\cap H'$ is of multiplicative finite type over $S$ (we note that the canonical functor from the category considered in \cref{scheme group multiplicative morphism ft category prop} to $\Sch_{/S}$ commutes with finite products).\par
The formation of $H\cap H'$ commutes with base changes, and in particular with taking fibers. On the other hand, $H\sub H'$ (resp. $H=H'$) if and only if $H'=H$ (resp. $H''=H$ and $H''=H'$). Consider the canonical homomorphism $H''\to H$ and $H''\to H'$. Then $U$ is the set $s\in S$ such that the induced homomorphism $H''_s\to H_s$ is an isomorphism (resp. $H''_s\to H_s$ and $H''_s\to H'_s$ are isomorphisms), so the conclusion follows from \cref{scheme group multiplicative ft mono epi locus prop}.
\end{proof}

\begin{proposition}\label{scheme group multiplicative Zariski trivial prop}
Let $S$ be a scheme, $G$ be an $S$-group of multiplicative finite type over $S$, $H$ be a subgroup of multiplicative type of $G$, and $K=G/H$ (which is a group of multiplicative type).
\begin{enumerate}
    \item[(a)] Suppose that $G$ is trivial (i.e. diagonalizable). Then there exists a partition $(S_i)$ of $S$ by clopen subsets such that for each $i$, $H|_{S_i}$ and $K|_{S_i}$ are diagonalizable. In particular, if $S$ is connected, then $H$ and $K$ are diagonalizable.
    \item[(b)] The same assertion is true as in (a) by replacing "diagonalizable" with "isotrivial", provided that $S$ is connected or quasi-compact.
    \item[(c)] Suppose that $G$ is locally trivial (resp. locally isotrivial, resp. quasi-isotrivial), then so is $K$ and $H$. 
\end{enumerate}
\end{proposition}
\begin{proof}
By hypothesis of (a), we have $G=D_S(M)$, where $M$ is a finitely generated abelian group. For any quotient group $M_i$ of $M$, let $H_i=D_S(M_i)$ be the diagonalizable group corresponding to $G$. Let $S_i$ be the set of $s\in S$ such that $H_s=(H_i)_s$; in view of \cref{scheme group multiplicative subgroup inclusion locus prop}, $S_i$ is clopen in $S$, and we have $H|_{S_i}=(H_i)|_{S_i}$, hhence $H|_{S_i}$ is diagonalizable, and also $K|_{S_i}$ by \cref{scheme group diagonalizable exact sequence prop}. Evidently, the $S_i$ are pairwise disjoint, and they cover $S$: this follows from the fact that for any $s\in S$, $H_s$ is diagonalizable and a subgroup of $G_s$, which is also diagonalizable (cf. \cref{scheme group diagonalizable transpose mono epi iff}). Restricting to the nonempty $S_i$, we then obtain (a).\par
For (b), by hypothesis there exists $S'\to S$ being \'etale and surjective, such that $G_{S'}$ is diagonalizable. Then any point of $S'$ has a neighborhood $U'$, clopen in $S'$, such that $H|_{U'}$ and $K|_{U'}$ are diagonalizable. The image $U$ of $U'$ in $S$ is then clopen, and $S'\to S$ induces a finite and \'etale surjective morphism $U'\to U$, so that any point $s\in S$ has a clopen neighborhood $U$ such that $H|_U$ and $K|_U$ are isotrivial. The assertion of (b) then follows, since we can take the direct sum of these morphisms $U'\to U\hookrightarrow S$ to obtain a finite \'etale surjective morphism. Finally, (c) follows from (a) and (b).
\end{proof}

\subsection{Infinitesimal properties: lifting and conjugation}\label{scheme group multiplicative infinitesimal prop subsection}
\begin{theorem}\label{scheme group multiplicative over affine cohomology zero}
Let $S$ be an affine scheme, $G$ be a group of multiplicative type, $\mathscr{F}$ be a quasi-coherent sheaf over $S$ over which $G$ acts. Then we have 
\[H^i(G,\mathscr{F})=0\for i>0,\]
where $H^i$ denotes the Hochschild cohomology of \autoref{category cohomology of group subsection}.
\end{theorem}

In fact, by \cref{scheme group cohomology of qcoh G-module}, if $S=\Spec(A)$, $G=\Spec(B)$ and $M$ is the $A$-module defining $\mathscr{F}$, then the Hochschild cohomology is calculated as the cohomology of the complex $C^\bullet(G,\mathscr{F})$, where
\[C^n(G,\mathscr{F})=\Gamma(S,\mathscr{F}\otimes\underbrace{\mathscr{A}(G)\otimes\cdots\otimes\mathscr{A}(G)}_{\text{$n$-fold}})=M\otimes\underbrace{B\otimes\cdots\otimes B}_{\text{$n$-fold}}.\]
If $f$ (resp. $a_i$) is a section of $\mathscr{F}$ (resp. $\mathscr{A}(G)$) over $S$, we then have
\begin{align*}
d(f\otimes a_1\otimes\cdots\otimes a_n)&=\mu_M(f)\otimes a_1\otimes\cdots\otimes a_n+\sum_{i=1}^{n}(-1)^if\otimes a_1\cdots\otimes \Delta a_i\otimes\cdots\otimes a_n\\
&+(-1)^{n+1}f\otimes a_1\otimes\cdots\otimes a_n\otimes 1
\end{align*}
where $\Delta:B\to B\otimes B$ and $\mu_M:M\to M\otimes B$ are induced from the cogebrea structure of $B$ and the comodule structure on $\mathscr{F}$. We also note that the formation of $C^\bullet(G,\mathscr{F})$ commutes with base change. Therefore, for any base change morphism $A\to A'$ with $A'$ flat over $A$, we have
\[H^i(G',\mathscr{F}')=H^i(G,\mathscr{F})\otimes_AA',\]
and therefore, if we suppose that $A'$ is faithfully flat over $A$, to show that $H^i(G,\mathscr{F})$ vanishes, it suffices to consider the cohomology $H^i(G',\mathscr{F}')$. By the definition of multiplicative type, we are therefore reduced to the case where $G$ is diagonalizable, which follows from \cref{scheme module over diagonalizable group cohomology zero}.\par
Using the results of (\cite{SGA3-1} \Rmnum{3}), we then obtain from \cref{scheme group multiplicative over affine cohomology zero} the following concequences:

\begin{theorem}\label{scheme group multiplicative morphism conjugation if I^2 reduction}
Let $S$ be an affine scheme and $S_0$ be an affine subscheme of $S$ defined by an ideal $\mathscr{I}$ such that $\mathscr{I}^2=0$. Let $u,v:H\to G$ be group homomorphisms over $S$, where $H$ is of multiplicative type, and $u_0,v_0:H_0\to G_0$ be the morphisms induced by base change $S_0\to S$.
\begin{enumerate}
    \item[(a)] If $u_0=v_0$, then there exists $g\in G(S)$ such that $v=\inn(g)\circ u$ and $g_0=e$.
    \item[(b)] In particular, if $H$ and $H'$ are subgroups of $G$ of multiplicative type such that $H_0=H_0'$, then there exists $g\in G(S)$ such that $H'=\inn(g)(H)$ and $g_0=e$.
\end{enumerate}
\end{theorem}
\begin{proof}
The first assertion follows from (\cite{SGA3-1} \Rmnum{3}, 2.1(\rmnum{2}) et 3.1), and the second one follows by applying (a) to the morphisms $H\times_SH'\to H\hookrightarrow G$ and $H\times_SH'\to H'\hookrightarrow G$ (note that $H\times_SH'$ is of multiplicative type).
\end{proof}

\begin{corollary}\label{scheme group multiplicative morphism conjugation if nilpotent reduction}
Under the conditions of \cref{scheme group multiplicative morphism conjugation if I^2 reduction}, suppose that $G$ is smooth over $S$ and $\mathscr{I}$ is nilpotent.
\begin{enumerate}
    \item[(a)] If there exists $g_0\in G_0(S_0)$ such that $v_0=\inn(g_0)\circ u_0$, then $g_0$ comes from an element $g\in G(S)$ such that $v=\inn(g)\circ u$.
    \item[(b)] In particular, if $H$ and $H'$ are subgroups of $G$ of multiplicative type such that $H_0'=\inn(g_0)(H_0)$ with $g_0\in G_0(S_0)$, then $g_0$ comes from an element $g\in G(S)$ such that $H'=\inn(g)(H)$.
\end{enumerate}
\end{corollary}
\begin{proof}
To prove (a), by induction on the integer $n>0$ such that $\mathscr{I}^n=0$, we are reduced to the case where $\mathscr{I}$ is square zero. Moreover, as $G$ is smooth over $S$, we can lift $g_0$ into an element $g'$ of $G(S)$. By replacing $v$ with $v'=\inn(g')\circ u$, we are then reduced to the situation of \cref{scheme group multiplicative morphism conjugation if I^2 reduction}. Finally, (b) follows from (a) by a similar argument as in \cref{scheme group multiplicative morphism conjugation if I^2 reduction}.
\end{proof}

\begin{corollary}\label{scheme group multiplicative morphism equal if nilpotent reduction}
Under the conditions of \cref{scheme group multiplicative morphism conjugation if I^2 reduction}, with $\mathscr{I}$ nilpotent, suppose that $v$ is central. Then $u_0=v_0$ implies $u=v$. In particular, if $u_0$ is the trivial homomorphism, then $u$ is trivial.
\end{corollary}
\begin{proof}
The reduction to the case where $\mathscr{I}$ is square zero is still immediate, and then it suffices to apply \cref{scheme group multiplicative morphism conjugation if I^2 reduction}.
\end{proof}

\begin{theorem}\label{scheme group multiplicative morphism nilpotent lifting exist}
Let $S$ be an affine scheme and $S_0$ be an affine subscheme of $S$ defined by a nilpotent ideal $\mathscr{I}$. Let $H$ be an $S$-group of multiplicative type and $G$ be a smooth $S$-group.
\begin{enumerate}
    \item[(a)] Let $u_0:H_0\to G_0$ be a morphism of $S_0$-groups induced by base change $S_0\to S$. Then there exists a morphism $u:H\to G$ lifting $u_0$ (and by \cref{scheme group multiplicative morphism conjugation if I^2 reduction}, two such liftings $u,u'$ are conjugate by an element of $G(S)$ which reduces to the identity in $G(S_0)$).
    \item[(b)] In particular, if $H_0$ is a subgroup of $G_0$ of multiplicative type, there exists a subgroup $H$ of $G$ such that $H\times_SS_0=H_0$, which is of multiplicative type and flat over $S$. Moreover, any two such subgroups $H$, $H'$ are conjugate by an element of $g\in G(S)$ which reduces to the identity in $G(S_0)$.
\end{enumerate}
\end{theorem}
\begin{proof}
The first assertion follows from (\cite{SGA3-1} \Rmnum{3} 2.1, 2.3 et 3.1) (vanishing of $H^2$ implies the existence of a lifting of $u_0$, vanishing of $H^1$ implies the uniqueness).\par
Now let $H_0$ be a subgroup of $G_0$. We then have an closed immersion $u_0:H_0\to G_0$, which by (a) comes from a morphism $u:H\to G$. As $H$ and $H_0$ (resp. $G$ and $G_0$) have the same underlying topological space and as for any $h\in H$, the morphism $\mathscr{O}_{G,u(h)}\to\mathscr{O}_{H,h}$ is surjective (by the nilpotent version of Nakayama's lemma), $u$ is equally a closed immersion. It then remains to note that for any lifting $H$ of $H_0$ of a flat subgroup of $G$, $H$ is necessarily of multiplicative type by \cref{scheme group flat multiplicative iff nilpotent reduction}\footnote{We note that the proof of \cref{scheme group flat multiplicative iff nilpotent reduction} only depends on \cref{scheme group multiplicative morphism nilpotent lifting exist}~(a), so there is no circular reasoning.}.
\end{proof}

\begin{proposition}\label{scheme group multiplicative action trivial if nilpotent reduction}
Let $S$ be an affine scheme and $S_0$ be an affine subscheme of $S$ defined by a nilpotent ideal $\mathscr{I}$. Let $G$ be an $S$-group of multiplicative type, $X$ be an $S$-scheme locally of fintie type acted by $G$ so that $G_0$ acts trivially on $X_0$. Then $G$ acts trivially on $X$.
\end{proposition}
\begin{proof}
The proof is that of (\cite{SGA3-1} \Rmnum{3} 2.4(b)), so we are reduced to the case where $G$ is diagonalizable, which is the case considered in (\cite{SGA3-1} \Rmnum{3} 2.4(b)).
\end{proof}

\begin{corollary}\label{scheme group multiplicative morphism central if nilpotent reduction}
Let $S$ be an affine scheme and $S_0$ be an affine subscheme of $S$ defined by a nilpotent ideal $\mathscr{I}$. Let $u:G\to H$ be a homomorphism of $S$-groups, with $G$ of multiplicative type and $H$ locally of finite presentation over $S$. Suppose that the homomorphism $u_0:G_0\to H_0$ induced by base change $S_0\to S$ is central, then $u$ is central.
\end{corollary}
\begin{proof}
It suffices to apply \cref{scheme group multiplicative action trivial if nilpotent reduction} to the action of $G$ on $H$ defined by $(g,h)\mapsto\inn(u(g))\cdot h$.
\end{proof}

\subsection{The density theorem of torsion subgroups}
The dentity theorem (\cref{scheme group multiplicative G[n] schematically dense}), together with the algebricity theorem (\cref{scheme group over complete Noe formal homomorphism is algebraic}), will be the essential tools in the present and next two sections, to pass from the infinitesimal properties of the groups of multiplicative type, which we have just developed, to finite properties.

\begin{definition}
Let $X$ be a scheme. We say that a family $(Z^i)_{i\in I}$ of subschemes of $X$ is \textbf{schematically dense} if for any open subset $U$ of $X$ and any closed subscheme $Z$ of $U$ which dominates the $Z^i\cap U$, we have $Z=U$. We say that a subscheme $Z$ of $X$ is \textbf{schematically dense} in $X$ if this is true for the family reduced to $Z$.
\end{definition}

We see immediately (\cite{EGA4-3} 11.10.1) that the above definition is equivalent to that for any open subset $U$ of $X$, any section $f$ of $\mathscr{O}_U$ which vanishes over the $Z^i\cap U$, is zero. This also signifies that the intersection of the kernels of the canonical homomorphisms
\[u_i:\mathscr{O}_X\to (v_i)_*(\mathscr{O}_{Z^i})\]
is zero, where $v_i:Z^i\to X$ is the canonical immersion. If $X$ is over a scheme $S$, this amounts to saying that for any open subset $U$ of $X$ and any couple $(u,v)$ of morphisms of $U$ to a separated $S$-scheme $Y$ which coincides over the $Z^i\cap U$, we have $u=v$\footnote{In fact, the relation $u=v$ is equivalent to the relation $Z=U$, where $Z$ is the inverse image of the diagonal of $Y\times_SY$ under the $S$-morphism $X\to Y\times_SY$ defined by $(u,v)$; this diagonal is a closed subscheme of $Y\times_SY$, hence $Z$ is a closed subscheme of $U$, which dominates $Z^i\cap U$ by the hypothesis on $(u,v)$, hence if the family $(Z^i)$ is schematically dense, we have $Z=U$, whence $u=v$. The converse implication is deduced by putting $Y=\Spec(\mathscr{O}_S[T])$.}. With the terminology introduced in \autoref{scheme theoretic image subsection}, we see that the subscheme $Z$ of $X$ is schematically dense in $X$ if and only if $X$ coincides with the scheme-theoretic closure of $Z$ in $X$.

\begin{example}
Let $X=\Spec(k[x,y]/(x^2,xy))$ (i.e. a line with an embedded point at the origin) and $U=D(y)$. Then $U$ is of course dense in $X$ as a topological space, but is not schematically dense, since $U\hookrightarrow X$ factors through $U\hookrightarrow\Spec(k[y])\hookrightarrow\Spec(k[x,y]/(x^2,xy))$.
\end{example}

\begin{lemma}\label{scheme flat schematically dense if nilpotent reduction}
Let $X$ be a flat $S$-scheme, $(Z^i)_{i\in I}$ be a family of subschemes of $X$, which are flat over $S$. Let $S_0$ be the subscheme of $S$ defined by a nilpotent ideal $\mathscr{I}$, and suppose that the modules $\mathscr{I}^n/\mathscr{I}^{n+1}$ are locally free over $S_0$. Let $X_0$ and $(Z_0^i)_{i\in I}$ be the induced schemes by base change $S_0\to S$. If the family $(Z^i_0)_{i\in I}$ is schematically dense in $X_0$, then $(Z^i)_{i\in I}$ is schematically dense in $X$.
\end{lemma}
\begin{proof}
Suppose that $\mathscr{I}^{n+1}=0$ (where $n\geq 0$), we prove by induction on $n$. The assertion is trivial for $n=0$, so assume that $n>0$ and denote by $S_m$, $X_m$, $Z^i_m$ the schemes obtained by reduction modulo $\mathscr{I}^{m+1}$. The induction hypothesis implies that $(Z^i_{n-1})_{i\in I}$ is schematically dense in $X_{n-1}$. By replacing $X$ with an open subset, we are then reduced to proving that any section $f$ of $\mathscr{O}_X$ which vanishes over the $Z^i$ is itself zero.\par
Now the section $f_{n-1}$ of $\mathscr{O}_{X_{n-1}}=\mathscr{O}_X/\mathscr{I}^n\mathscr{O}_X$ defined by $f$ vanishes over the $Z^i_{n-1}$, so is itself zero, hence $f$ is a section of $\mathscr{I}^n\mathscr{O}_X$. As $X$ is flat over $S$, we have
\[\mathscr{I}^n\mathscr{O}_X\stackrel{\sim}{\to}\mathscr{E}\otimes_{\mathscr{O}_{S_0}}\mathscr{O}_{X_0}\]
where $\mathscr{E}=\mathscr{I}^n=\mathscr{I}^n/\mathscr{I}^{n+1}$. Similarly, as each $Z^i$ is flat over $S$, the restriction $f_i$ of $f$ to $Z^i$ can be regareded as a section of
\[\mathscr{I}^n\mathscr{O}_{Z^i}\stackrel{\sim}{\to}\mathscr{E}\otimes_{\mathscr{O}_{S_0}}\mathscr{O}_{Z_0^i}.\]
Now $\mathscr{E}$ is locally free by hypothesis, hence so is $\mathscr{F}=\mathscr{E}\otimes_{\mathscr{O}_{S_0}}\mathscr{O}_{X_0}$, and $f$ is therefore a section of a locally free module $\mathscr{F}$ over $X_0$, such that for any $i$ its restriction to $Z_0^i$ is zero. As $(Z_0^i)_{i\in I}$ is schematically dense in $X_0$, it follows from (\cite{EGA4-3} 11.10.1) that $f$ is itself zero.
\end{proof}

\begin{lemma}\label{scheme Noe flat schematically dense if fiber is}
Let $X$ be a locally Noetherian and flat $S$-scheme, $(Z^i)_{i\in I}$ be a family of subschemes of $X$ which are flat over $S$. Suppose that for any $s\in S$, the family $(Z^i_s)_{i\in I}$ of fibers at $s$ is schematically dense in $X_s$, then the family $(Z^i)_{i\in I}$ is schematically dense in $X$. Further, in this case, if $S$ is locally Noetherian, then $(Z^i)_{i\in I}$ is universally schematically dense in $X$.
\end{lemma}

\begin{lemma}\label{scheme schematically dense faithfully flat descent}
Let $X$ be an $S$-scheme, $(Z^i)_{i\in I}$ be a family of subschemes of $X$, and $S'\to S$ be a faithfully flat morphism. If the family $(Z'^i)_{i\in I}$ is schematically dense in $X'$, then $(Z^i)_{i\in I}$ is schematically dense in $X$.
\end{lemma}
\begin{proof}
This follows from (\cite{EGA4-3} 11.10.6 (\rmnum{1}) et 11.9.10 (\rmnum{1})). 
\end{proof}

\begin{corollary}\label{scheme flat open schematically dense if fiber is}
Let $X$ be a flat $S$-scheme and $U$ be an open subset of $X$. Suppose that for any $s\in S$, $U_s$ is schematically dense in $X_s$, and that $X$ is locally Noetherian or locally of finite presentation over $S$. Then $U$ is schematically dense in $X$.
\end{corollary}

\begin{theorem}\label{scheme group multiplicative G[n] schematically dense}
Let $S$ be a scheme and $G$ be a group of multiplicative finite type over $S$. For any integer $n>0$, let ${_nG}$ be the kernel of the morphism $n\cdot\id_G$. Then the family $({_nG})_{n>0}$ of subschemes of $G$ is schematically dense in $G$.
\end{theorem}
\begin{proof}
We first consider the case where $S$ is locally Noetherian. Then by \cref{scheme Noe flat schematically dense if fiber is}, we are reduced to the case where $S$ is the spectrum of a field $k$. By \cref{scheme schematically dense faithfully flat descent}, we can also suppose that $k$ is algebraically closed and $G$ is diagonalizable, i.e. of the form $D_k(M)$, where $M$ is a finitely generated abelian group. Then $M$ is of the form $\Gamma\times\Z^r$, with $\Gamma$ finite, hence $G$ is isomorphic to $G_1\times T$, where $G_1=D(\Gamma)$ and $T=\G_m^r$. Therefore for $n$ multiplicatively large (i.e. $n$ is a multiple of the order of $\Gamma$), we have ${_nG}=G_1\times {_nT}$ (since ${_nG_1}=G_1$).\par
Applying again \cref{scheme Noe flat schematically dense if fiber is} to the projection $G\to G_1$, we are reduced to the case where $G=\G_m^r$. As $G$ is then reduced, the familt $({_nG})_{n>0}$ is schematically dense in $G$ if and only if the union of ${_nG}$ is dense in $G$ for the usual topology. As ${_nG}=({_n\G_m})^r$, we are also reduced to the case $G=\G_m$, hence $G$ is an irreducible algebraic curve. Now the theorem follows from the fact that the union of the ${_nG}$ (equals to the set of roots of unity in $k$) is infinite (cf. \cref{scheme algerbraic curve topology prop}).
In the general case, for any point $s\in S$, there exists a neighborhood $U$ of $s$ and a faithfully flat and quasi-compact morphism $S'\to U$ such that $G'=G_{S'}$ is diagonalizable, i.e. of the form $D_{S'}(M)$ with $M$ a finitely generated abelian group. By \cref{scheme schematically dense faithfully flat descent}, we can then reduce to the case where $G$ is diagonalizable, hence consider the absolute group $H=D_{\Z}(M)$ over $\Spec(\Z)$. By the preceding arguments, for any $s\in\Spec(\Z)$, the family $({_nH_s})_{n>0}$ is schematically dense in $H$; it then suffices to apply \cref{scheme Noe flat schematically dense if fiber is}.
\end{proof}

\begin{corollary}\label{scheme group multiplicative subgroup equal if dominates G[n]}
Let $G$ be an $S$-group of multiplicative finite type and $H$ be a subgroup of $G$ which dominates ${_nG}$ for each $n>0$. Then $G=H$.
\end{corollary}
\begin{proof}
In view of \cref{scheme group multiplicative G[n] schematically dense}, we are reduced to show that the subscheme $H$ is closed, or equivalently $|H|=|G|$. This reduces us to the case where $S$ is the spectrum of a field, but then any subgroup of $G$ is closed (\cref{scheme A-group homomorphism of local ft image prop}).
\end{proof}

\begin{corollary}\label{scheme group multiplicative morphism equal on G[n] equal}
Let $S$ be a scheme and $G$ be an $S$-group.
\begin{enumerate}
    \item[(a)] Let $u,v:H\to G$ be homomorphisms of $S$-groups, with $H$ of multiplicative finite type, and suppose that for any integer $n>0$, the restriction of $u,v$ to ${_nH}$ are identical. Then $u=v$.
    \item[(b)] Let $H_1$, $H_2$ be subgroups of multiplicative finite type of $G$, and suppose that for any integer $n>0$, we have ${_nH_1}={_nH_2}$. Then $H_1=H_2$.  
\end{enumerate}
\end{corollary}
\begin{proof}
The first assertion follows from the second, by considering the graph subgroups $H_1$ and $H_2$ of $H\times G$ induced by $u$ and $v$. To prove (b), let $H=H_1\cap H_2$, which is a subgroup of $H_i$ ($i=1,2$). We must show that this is identical to $H_i$, but the hypothesis means that it dominates ${_nH_i}$ for each $n>0$, we are therefore reduced to the situation of \cref{scheme group multiplicative subgroup equal if dominates G[n]}.
\end{proof}

\begin{remark}
Under the conditions of \cref{scheme group multiplicative G[n] schematically dense}, let $m>0$ be an integer which have the following properties: for any $s\in S$, $m$ is not a power of the characteristic of $\kappa(s)$, and if $G_s$ is of type $M$, the prime divisors of the order of the torsion subgroup of $M$ divides $m$ (this second condition is verified if $G$ is a torus). Then the proof of \cref{scheme group multiplicative G[n] schematically dense} implies that in the statement of \cref{scheme group multiplicative G[n] schematically dense} and \cref{scheme group multiplicative subgroup equal if dominates G[n]} and \cref{scheme group multiplicative morphism equal on G[n] equal}, we can only consider the subgroups of the form $G[m^r]$, with $r>0$.
\end{remark}

\subsection{Central homomorphisms of groups of multiplicative type}
\begin{lemma}\label{scheme local Noe subscheme equal if closed point reduction}
Let $(A,\m)$ be a local Noetherian ring, $S=\Spec(A)$, and $H$ be a finite scheme over $S$, hence $H=\Spec(B)$ with $B$ a finite $A$-algebra. Let $K$ be a subscheme of $H$ such that $K_n=H_n$ by reduction modulo $\m^{n+1}$ for any $n$. Then $K=H$.
\end{lemma}
\begin{proof}
Let $s$ be the closed point of $S$. We first note that $K$ is a closed subscheme of $H$. In fact, it is a priori a closed subscheme of an open subset $U$ of $H$. But $K$, hence $U$, contains the fiber $K_s=H_s$; as the morphism $H\to S$ is finite, hence closed, this ensures that the complement of $U$ is empty, i.e. $U=H$\footnote{We note that the image of $H-K$ in $S$ (if not empty) is a closed subset of $S$, whence is defined by an ideal $\a$ of $A$. This ideal must be contained in $\m$, so $\m$ belongs to the image of $H-K$.}. Therefore $K$ is defined by an ideal $\mathfrak{I}$ of $B$. The hypothesis implies that $\mathfrak{I}$ is comtained in $\m^nB$ for any $n$; as $B$ is a finite $A$-module, it is separated for the $\m$-adic topology, whence $\mathfrak{I}=0$.
\end{proof}

\begin{theorem}\label{scheme group multiplicative morphism lifting from fiber}
Let $u,v:H\to G$ be homomorphisms of $S$-groups, with $H$ of multiplicative finite type over $S$ and $S$ locally Noetherian or $G$ of finite presentation over $S$. Let $s\in S$ be such that $u_s=v_s$, and suppose that $v_s$ is central. Then there exists a neighborhood $U$ of $s$ such that $u|_U=v|_U$.
\end{theorem}
\begin{proof}
We first assume that $S$ is locally Noetherian. Let $K=\ker(u,v)$ be the inverse image of the diagonal of $G\times_SG$ under the morphism $(u,v)$; this is a subgroup of $H$, and we want to choose $U$ such that $K|_U=H|_U$. Note that as $S$ is locally Noetherian and $H$ is of finite type over $S$, $H$ is also locally Noetherian, so the immersion $K\hookrightarrow H$ is of finite type (\cref{scheme ft over local Noe base morphism is ft}). Then $K$ is of finite type over $S$, hence of finite presentation over $S$ ($S$ is locally Noetherian). Therefore, by (\cite{EGA4-3} 8.8.2.4), to show that there exists an open neighborhood $U$ of $s$ such that $K|_U=H|_U$, it suffices to prove that $K_{S'}=H_{S'}$, where $S'=\Spec(A)$, $A=\Spec(\mathscr{O}_{S,s})$. We can then reduce to the case where $S$ is local with closed point $s$.\par
In view of \cref{scheme group multiplicative morphism equal if nilpotent reduction} (here we use the hypothesis that $v_s$ is central), we have $u_n=v_n$ for any $n$, where as usual the index $n$ indicates the reduction modulo $\m^{n+1}$ ($\m$ being the maximal ideal of $A$). For any integer $m>0$, denote by ${_mu}$, ${_mv}$ the homomorphism ${_mH}\to G$ induced by $u,v$, we then have $({_mu})_n=({_mv})_n$. As this is valid for any $n$ and ${_nH}$ is finite over $S$ in view of \cref{scheme group multiplicative n-torsion finite}, it follows from \cref{scheme local Noe subscheme equal if closed point reduction} that ${_mu}={_mv}$. Since this is valid for any $m$, we conclude that $u=v$ from \cref{scheme group multiplicative G[n] schematically dense}.\par
We now consider the case where $G$ is of finite presentation. As $H$ is also of finite presentation over $S$, by (\cite{EGA4-3} 8.8.2), we can suppose that $S$ is local with closed point $s$, and show that $u=v$ in this case. If $f:S'\to S$ is a faithfully flat and quasi-compact morphism and we denote by $f_H$ the morphism $H'\to H$ and by $u',v':H'\to G'$ the morphisms induced by $u,v$, then the equality $u'=v'$ implies $u\circ f_H=v\circ f_H$, whence $u=v$ since $f_H$ is an epimorphism. Therefore, by taking a base change by faithfully flat and quasi-compact morphisms, we can suppose that $H$ is diagonalizable, hence of the form $D_S(M)$, with $M$ a finitely generated abelian group.\par
Now introduce as in the proof of \cref{scheme flat open schematically dense if fiber is} the directed family of finite type sub-$\Z$-algebras $A_i$ of $A=\mathscr{O}_{S,s}$, and $S_i=\Spec(A_i)$. Note that $H=D_S(M)$ provides, for any $i$, a diagonalizable group $H_i=D_{S_i}(M)$. As $G$ is of finite presentation over $S$, by (\cite{EGA4-3} 8.8.2) (also see \cite{SGA3-1} $\Rmnum{6}_B$, 10.2 et 10.3), there exists an index $i$, a group scheme $G_i$ of finite presentation over $S_i$, and morphisms of $S_i$-groups $u_i,v_i:H_i\to G_i$ which become $u,v$ under base change. Let $s_i$ be the image of $s$ in $S_i$ and $\rho_i:H_i\times_{S_i}G_i\to G_i$ be the morphism of $S_i$-schemes defined by $\rho_i(h,g)=u_i(h)gu_i(h)^{-1}$. Then, as $u_s$ is central, $\rho_s=\rho_i\times_{\kappa(s_i)}\kappa(s)$ is equal to the second projection, and hence so is $\rho_i$ (since $\kappa(s_i)\to\kappa(s)$ is faithfully flat and quasi-compact), i.e. $(u_i)_{s_i}$ is central. Similarly, as $u_s=v_s$, we have $(u_i)_{s_i}=(v_i)_{s_i}$. We can then apply the locally Noetherian case of the theorem to the situation over $S_i$, which proves the conclusion.  
\end{proof}

\begin{corollary}\label{scheme group multiplicative morphism trivial on fiber lifting}
Under the hypothesis of \cref{scheme group multiplicative morphism lifting from fiber}, let $s\in S$ and suppose that $u_s$ is trivial. Then there exists an open neighborhood $U$ of $s$ such that $u|_U$ is trivial.
\end{corollary}
\begin{proof}
It suffices to apply \cref{scheme group multiplicative morphism lifting from fiber} to $u$ and the trivial homomorphism, which is central.
\end{proof}

\begin{corollary}\label{scheme group multiplicative morphism trivial locus prop}
Under the hypothesis of \cref{scheme group multiplicative morphism lifting from fiber}, assume that $G$ is separated over $S$. Let $U$ be the set of $s\in S$ such that $u_s$ is trivial, then $U$ is clopen in $S$ and $u|_U:H_U\to G_U$ is trivial homomorphism.
\end{corollary}
\begin{proof}
It remains to show that $U$ is also closed. For this, let $K=\ker u$; as $G$ is separated over $S$, this is a closed subscheme of $H$, and $U$ is the set of $s\in S$ such that $K_s=H_s$. We then easily see that $U$ is closed, for example by applying \cref{scheme Weil restriction of closed of essentialy free representable prop} ($H$ is essentially free on $S$ according to \cref{scheme group diagonalizable is essentially free}). In fact, we note that if $K_{S'}=H_{S'}$ for a morphism $S'\to S$, then for any $s'\in S'$ with image $s\in S$, the fiber $K_{s'}=H_{s'}$, and hence $K_s=H_s$ (cf. \cite{SGA1} \Rmnum{8} 5.4). This implies $s\in U$, so the morphism $S'\to S$ factors through $U\to S$. The converse of this is immediate, and we conclue that $U$ represents the functor $\Res_{H/S}K$.
\end{proof}

\begin{corollary}\label{scheme group multiplicative morphism effective epi descent}
Let $S$ be a scheme, $H$ and $G$ be $S$-groups, with $H$ of multiplicative finite type and $G$ separated and of finite presentation over $S$. Let $\pi:S'\to S$ be a universally effective epimorphism with geometrically connected fibers. Let $u':H'\to G'$ be a central homomorphism of $S'$-groups, then there exists a unique homomorphism $u:H\to G$ of $S$-groups such that $u\times_SS'=u'$. If $S'$ admits a section $g$ over $S$, then $u$ is the morphism induced from $u'$ by base change $g:S\to S'$.
\end{corollary}
\begin{proof}
As $\pi$ is a universally effective epimorphism, so is $H'\to H$, whence the uniqueness of $u$. If $\pi$ admits a section $g$, then $u'=\pi^*(u)$ implies $u=g^*\pi^*(u)=g^*(u')$. For the existence of $u$, as $\pi$ is effective, we are reduced to show that the two homomorphisms $u_1'',u_2'':H''\to G''$ of $S''$-groups induced from $u'$ by base change $\pr_1,\pr_2:S''=S'\times_SS'\to S'$, are identical. Now these coincide over the diagonal of $S''$, since the inverse image of $u_1'',u_2''$ under the diagonal morphism $S'\to S''$ are identical (equal to $u'$). As $u_1''$ and $u_2''$ are central, we can apply \cref{scheme group multiplicative morphism trivial on fiber lifting} to the morphism $u_1''(u_2'')^{-1}$. There then exists a clopen subset $U$ of $S''$, containing the diagonal of $S''$, such that $u_1''$ and $u_2''$ coincide over $U$. Since the fibers of $S'$ over $S$ are geometrically connected, so are those of $S''$ over $S$, which are a fortiori connected, whence $U$ contains these fibers, and hence equals to $S''$. This completes the proof of the corollary.
\end{proof}

\begin{corollary}\label{scheme group multiplicative normal subgroup is central}
Let $S$ be a scheme, $K$ be an $S$-group locally of finite type with connected fibers, $H$ be a normal subgroup of multiplicative finite type of $K$. Then $H$ is a central subgroup of $K$.
\end{corollary}
\begin{proof}
We first note that $\pi:K\to S$ has connected fibers, since for a group scheme locally of finite type over a field, connected implies geometrically connected (\cref{scheme alg group identity component prop}). We can then apply \cref{scheme group multiplicative morphism effective epi descent} to $G=H$, $S'=K$, and to the homomorphism of $K$-groups $u':H_K\to H_K$ induced setwisely by $(h,k)\mapsto(khk^{-1},k)$, which is central since $H$ is commutative. The inverse image of $u'$ under the unit section $e:S\to K$ is the identity homomorphism of $H$, so by \cref{scheme group multiplicative morphism effective epi descent} the same is true for $H_K\to H_K$, which implies that $H$ is central in $K$.
\end{proof}

\begin{remark}\label{scheme subgroup multiplicative coincidence locus prop}
Let $G$ be a group scheme over $S$, $H_1,H_2$ be subgroups of $K$ of multiplicative finite type over $S$, and assume that $S$ is locally Noetherian or $G$ is of finite presentation over $S$. By applying the above results to the projections of $H_1\times H_2$, we obtain similar results on the coincidence locus of $H_1$ and $H_2$. For example, if $s\in S$, and $(H_1)_s$ is central, then there exists an open neighborhood $U$ of $s$ such that $H_1|_U=H_2|_U$. Moreover, if $G$ is separated over $S$, then the set $U$ of $s\in S$ such that $(H_1)_s=(H_2)_s$ is clopen in $S$, and $(H_1)|_U=(H_2)|_U$.
\end{remark}

\begin{theorem}\label{scheme group multiplicative morphism central at fiber prop}
Let $S$ be a scheme, $u:H\to G$ be a homomorphism of $S$-groups, with $H$ of multiplicative finite type and $G$ of finite presentation over $S$.
\begin{enumerate}
    \item[(a)] Suppose that $G$ has connected fibers, and let $s\in S$ be such that $u_s:H_s\to G_s$ is a central homomorphism. Then there exists an open neighborhood $U$ of $s$ such that the homomorphism $u|_U:H|_U\to G|_U$ is central.
    \item[(b)] Suppose that for any $s\in S$, $u_s:H_s\to G_s$ is central, then $u$ is central. 
\end{enumerate}
\end{theorem}
\begin{proof}
To prove (a), by reasoning as in \cref{scheme group multiplicative morphism lifting from fiber}, we can assume that $S$ is local with closed point $s$, that $H=D_S(M)$ is diagonalizable, and that there eixsts a finite type sub-$\Z$-algebra $A_i$ of $A=\mathscr{O}_{S,s}$ and a morphism $u_i:D_{S_i}(M)\to G_i$ such that $u_{s_i}$ is central ($s_i$ is the image of $s$ in $S_i$) and induces $u$ by base change. We must show that $u$ is central in this case.\par
Let $K$ be the subscheme $\ker(v,w)$, where $v,w:H\times_SG\rightrightarrows G$ are defined by
\[v(h,g)=g,\quad w(h,g)=\inn(u(h))\cdot g=u(h)gu(h)^{-1}.\]
Then $K$ is a subgroup of the $G$-group $H_G=H\times_SG$, and we need to show that it equals to $H_G$. In view of \cref{scheme group multiplicative morphism equal on G[n] equal}~(b), it suffices to show that ${_mH_G}={_mH}\times_SG$ for any integer $m>0$, and we can then assume that $H={_mH}$, so $H$ is finite over $S$.\par
Let $e$ be the unit element of the fiber $G_s$, then $S'=\Spec(\mathscr{O}_{G,e})$ is a Noetherian local scheme ($G$ is locally Noetherian since $S$ is Noetherian); put $S_n'=S'\times_SS_n$, where $S_n=\Spec(A/\m^{n+1})$. Then $K_{S'}=K\times_GS'$ is a subscheme of $H_{S'}=H\times_SS'$ and, by \cref{scheme group multiplicative morphism central if nilpotent reduction}, we have $K_{S_n'}=H_{S_n'}$ for any $n$. As $H_{S'}$ is finite over $S'$, we conclude from \cref{scheme local Noe subscheme equal if closed point reduction} that $K_{S'}=H_{S'}$.\par
On the other hand, as $H\times_SG$ is Noetherian (being of finite presentation over $S$), the immersion $K\hookrightarrow H\times_SG$ is of finite type by \cref{scheme ft over local Noe base morphism is ft}, so $K$ is of finite type over $G$, and hence of finite presentation over $G$. Then the equality $K_{S'}=H_G\times_GS'$ implies, by (\cite{EGA4-3} 8.8.2.4), that there exists an open neighborhood $W$ of $e$ in $G$ such that $K\times_GW=H_G\times_GW=H\times_SW$. Now $K$ and $V=H\times_SW$ can both be considered as subschemes of the scheme $G_H$ over $K$, and $K\times_GW=K\times_{G_H}V$; therefore, we conclude that $K$ dominates the open neighborhood $V$ of the unit section of $G_H$ over $H$.\par
For any $t\in H$, the fiber $G_t$ (being an algebraic $\kappa(t)$-group) is connected, hence irreducible (\cref{scheme alg group identity component prop}). By \cref{scheme theoretic closure exist if qc}, the open subset $V_t$ is then schematically dense in $G_t$, so $V$ is schematically dense in $G_H$ by \cref{scheme Noe flat schematically dense if fiber is}. Moreover, as $K$ is a sub-$H$-group of $G_H$, it induces on each fiber $G_h$ a subgroup $K_h$, and as the latter dominates an open neighborhood of the identity element of $G_h$ (which is dense in $G_h$), it follows that $K_h=G_h$ for each $h$, whence $|K|=|G_H|$. Now $K$ is then a closed subscheme of $G_H$ which dominates the schematically dense subscheme $V$, so $K=G_H$ and this proves (a), and (b) is a direct concequence of (a), since the formation of $K$ commutes with restrictions.
\end{proof}

\subsection{Canonical factorization of morphisms from groups of multiplicative type}

\begin{lemma}\label{scheme group fp qf over qc is torsion}
Let $S$ be a quasi-compact scheme, $G$ be an abelian $S$-group of finite presentation and quasi-finite over $S$. Then there exists an integer $n>0$ such that $n\cdot\id_G=0$, i.e. $G={_nG}$.
\end{lemma}
\begin{proof}

\end{proof}

\begin{corollary}\label{scheme multiplicative closed subgroup fiber qf prop}
Let $S$ be an $S$-scheme, $H$ be an $S$-group of multiplicative finite type, and $K$ be a closed subgroup of $H$ which is of finite presentation over $S$. Let $s\in S$ be such that $K_s$ is quasi-finite over $\kappa(s)$ at the identity. Then there exists an open neighborhood $U$ of $s$ such that $K|_U$ is contained in ${_nH}|_U$ for an integer $n>0$, and a fortiori such that $K|_U$ is finite over $U$.
\end{corollary}
\begin{proof}
By \cref{scheme group locus on base and group prop}, we see that there exists an neighborhood $U$ of $s$ such that $K|_U$ is locally quasi-finite over $U$. By shrinking $U$ if necessary, we may assume that $U$ is affine, hence quasi-compact, and $K|_U$ is quasi-finite and of finite presentation over $U$. As $H$ is commutative (hence so is $K$), \cref{scheme group fp qf over qc is torsion} then implies that $n\cdot\id_{K|_U}=0$, so $K|_U$ is contained in ${_nH}|_U$.
\end{proof}

From \cref{scheme multiplicative closed subgroup fiber qf prop}, we deduce the finiteness of certain closed subgroups of a multiplicative group. By applying Nakayama's lemma, we immediately deduce the following important result:

\begin{proposition}\label{scheme multiplicative closed subgroup fiber trivial prop}
Let $S$ be an $S$-scheme, $H$ be an $S$-group of multiplicative finite type, and $K$ be a closed subgroup of $H$ which is of finite presentation over $S$. Assume that $K_s$ is the trivial group, then there exists an open neighborhood $U$ of $s$ such that $K|_U$ is the trivial group.
\end{proposition}
\begin{proof}
By \cref{scheme multiplicative closed subgroup fiber qf prop}, there exists an open neighborhood $V$ of $s$ such that $K|_V$ is finite over $V$. By taking an open affine neighborhood in $V$, we may assume that $V=\Spec(A)$ is affine, and then $K|_V=\Spec(B)$ for a finite $A$-algebra $B$. Let $\p$ be the prime ideal of $A$ corresponding to $s$, then by hypothesis, we have $A_\p/\p A_\p\cong B_\p/\p B_\p$, so $A_\p\cong B_\p$ in view of Nakayama's lemma. As $K$ is of finite presentation over $S$, so is $K|_V$ over $V$, and we conclude from (\cite{EGA4-3} 8.8.2.4) that there exists an open neighborhood $U$ of $s$ in $V$ such that $K|_U\cong U$, i.e. $K|_U$ is the trivial group.
\end{proof}

\begin{remark}
We already know that (cf. \cref{scheme group local ft trivial iff fiber trivial}) a group $G$ locally of finite type over $S$ is trivial if and only if its fibers $G_s$ are trivial. \cref{scheme multiplicative closed subgroup fiber trivial prop} then shows that this property is "open" for certain closed subgroups of a multiplicative group.
\end{remark}

\begin{corollary}\label{scheme group morphism from multiplicative mono locus open}
Let $u:H\to G$ be a homomorphism of $S$-groups, with $H$ of multiplicative finite type and $G$ separated over $S$. Suppose that $S$ is locally Noetherian or $G$ is locally of finite type over $S$. Let $s\in S$ be such that $u_s:H_s\to G_s$ is a monomorphism, then there exists an open neighborhood $U$ of $s$ such that $u|_U:H|_U\to G|_U$ is a monomorphism.
\end{corollary}
\begin{proof}
Let $K=\ker u$, the hypothesis implies that $K_s$ is the trivial group. Now $G$ being separated over $S$, $K$ is a closed subgroup of $H$, and in the case where $S$ is not locally Noetherian but $G$ is of finite presentation over $S$, as $H\to S$ is of finite presentation, by \cref{scheme morphism local fp permanence prop}~(\rmnum{5}) we see that $H\to G$ is locally of finite presentation, hence so is $K\to S$, and $K$ is of finite presentation over $S$ since it is separated over $S$ and quasi-compact over $S$ (being closed in $H$, which is quasi-compact over $S$). We can then apply \cref{scheme multiplicative closed subgroup fiber trivial prop} to conclude the corollary.
\end{proof}

\begin{remark}
Under the hypothese of \cref{scheme group morphism from multiplicative mono locus open}, we note that if $G$ is affine over $S$ (resp. of finite presentation over $S$), then $H|_U\to G|_U$ is in fact a closed immersion (\cref{scheme group mono multiplicative to affine prop} and \cref{scheme group mono multiplicative to fp sp prop}).
\end{remark}

\begin{corollary}\label{scheme group morphism from multiplicative mono if on fiber}
Let $u:H\to G$ be a homomorphism of $S$-groups, with $H$ of multiplicative finite type and $G$ separated over $S$. Suppose that $S$ is locally Noetherian or $G$ is locally of finite type over $S$. If for any $s\in S$, the homomorphism $u_s:H_s\to G_s$ is a monomorphism, then $u$ is a monomorphism.
\end{corollary}
\begin{proof}
The reasoning is the same as \cref{scheme group morphism from multiplicative mono locus open}, the hypothesis that $G$ is separated over $S$ ensures that $K$ is closed in $H$, since the hypothesis that $u_s$ are monomorphisms imply that $K$ is reduced setwisely to the unit section of $G$.
\end{proof}

\begin{theorem}\label{scheme group morphism from multiplicative factorization}
Let $u:H\to G$ be a homomorphism of $S$-groups, with $H$ of multiplicative finite type and $G$ separated over $S$. Suppose that $S$ is locally Noetherian or $G$ is of finite presentation over $S$. Then $K=\ker u$ is a subgroup of multiplicative finite type over $H$, and $u$ factors into
\[\begin{tikzcd}
H\ar[r,"\bar{u}"]&H/K\ar[r,"u'"]&G
\end{tikzcd}\]
where $H/K$ is of multiplicative finite type, $\bar{u}$ is the canonical homomorphism (which is faithfully flat and affine), and $u'$ is a monomorphism.
\end{theorem}
\begin{proof}
It suffices to prove that $K$ is of multiplicative type, since the rest of the proposition then follows from \cref{scheme group multiplicative morphism factorization}~(c). We first suppose that $G$ is of finite presentation over $S$. This hypothesis is stable under base change, so we may assume that $H$ is diagonalizable, i.e. $H=D_S(M)$ with $M$ a finitely generated abelian group, and prove that $K$ is of multiplicative type. Let $s\in S$, then $K_s$ is a closed subgroup of $H_s=D_{\kappa(s)}(M)$, hence is of the form $D_{\kappa(s)}(N)$ (\cref{scheme group multiplicative Zariski trivial prop}), where $N$ is a quotient group of $M$. Put $K'=D_S(N)$, then $K'$ is a subgroup of multiplicative type of $H$; let $v':K'\to G$ be the morphism induced by $u$. Then $v'_s$ is the trivial homomorphism by our construction, so by \cref{scheme group multiplicative morphism trivial on fiber lifting} there exists an open neighborhood $U$ of $s$ such that $v'|_U:K'|_U\to G|_U$ is trivial. By replacing $S$ with $U$, we may hence assume that $v'$ is the trivial homomorphism, so $u$ factors into
\[\begin{tikzcd}
H\ar[r,"\bar{u}"]&H/K'\ar[r,"u'"]&G
\end{tikzcd}\]
Now since $u'_s$ is induced from $u_s$ by factorizing through $H_s\to H_s/(\ker u_s)$, we see that it is a monomorphism (\cref{site quotient by normal subgroup factorization mono iff kernel}), hence in view of \cref{scheme group morphism from multiplicative mono locus open}, there exists an open neighborhood $V$ of $s$ such that $u'|_V:(H/K')|_V\to G|_V$ is a monomorphism. By replacing $S$ with $V$, we can therefore assume that $u'$ is a monomorphism, whence $\ker u=\ker\bar{u}=K'$, and this proves that $\ker u$ is of multiplicative type by our construction.\par
The same proof is valid if we assume that $S$ is locally Noetherian, at least in the case where $H$ is diagonalizable. If we do not make this assumption on $H$, we must show that there exists a faithfully flat and quasi-compact morphism $S'\to S$, with $S'$ locally Noetherian, which trivializes $H$. This is indeed what we will see in the next section (\autoref{scheme group multiplicative quasi-isotrivial theorem subsection}).
\end{proof}

\subsection{Algebricity of formal homomorphisms}
\begin{theorem}\label{scheme group over complete Noe formal homomorphism is algebraic}
Let $A$ be a Noetherian ring and $\mathfrak{I}$ be an ideal of $A$ such that $A$ is complete for the $\mathfrak{I}$-adic topology. Let $S=\Spec(A)$, $S_n=\Spec(A/\mathfrak{I}^{n+1})$, and $H$, $G$ be $S$-groups with $H$ of isotrivial multiplicative type and $G$ affine over $S$. Then the canonical map
\begin{equation}
\theta:\Hom_{S\dash\Grp}(H,G)\to\llim_n\Hom_{S_n\dash\Grp}(H_n,G_n)
\end{equation}
is bijective ($H_n$ and $G_n$ are the $S_n$-groups induced by base change $S_n\to S$).
\end{theorem}


\begin{corollary}\label{scheme group over complete Noe morphism from multiplicative lifting exist}
Under the hypotheses of \cref{scheme group over complete Noe formal homomorphism is algebraic}, suppose that $G$ is smooth over $S$, and let $u_0:H_0\to G_0$ be a homomorphism of $S_0$-groups. Then there exists a homomorphism of $S$-groups $u:H\to G$ lifting $u_0$, and any two such liftings $u,u'$ are conjugate by an element $g\in G(S)$ which reduces to the identity in $G_0(S_0)$.
\end{corollary}
\begin{proof}
The conjugation statement follows from \cref{scheme group multiplicative morphism nilpotent lifting exist}. To construct $u$, we construct inductively for each $n$ a morphism $u_n:H_n\to G_n$ such that $u_n$ is deduced from $u_{n+1}$ by reduction, which is possible by \cref{scheme group multiplicative morphism nilpotent lifting exist}. By virtue of \cref{scheme group over complete Noe formal homomorphism is algebraic}, the system $(u_n)$ comes from a morphism $u:H\to G$. Given two liftings $u$ and $u'$, to construct $g$ such that $u'=\inn(g)u$ and $g_0=1$, we can construct $g_n$ step by step, such that $g_0=1$, $g_n$ is deduced from $g_{n+1}$ by reduction, and $u'_n=\Int(g_n)u_n$; this is possible thanks to \cref{scheme group multiplicative morphism nilpotent lifting exist}. As $A$ is separate and complete, the $g_n$ come from an element $g\in G(S)$ and to prove that $u'=\inn(g)u$, it suffices to use the injectivity of \cref{scheme group over complete Noe formal homomorphism is algebraic}. 
\end{proof}

\begin{remark}
We recall that a similar result is proved in (\cite{EGA3-1} \Rmnum{3} 5.1), with the assumption that $H$ is proper over $S$ and $G$ is separated and locally of finite type over $S$. The fact that \cref{scheme group over complete Noe formal homomorphism is algebraic} is true without any properness assumption is quite unexpected, and can be interpreted as a "rigidity" of the structure of a group of multiplicative type. The analogous statement with $G=H=\G_a$ (additive group) is false in general, as we see by taking $A$ of characteristic $p>0$ and defining the $u_n$ to be the reduction mod $\mathfrak{I}_{n+1}$ of an additive formal series
\[u(T)=\sum_{n\geq 0}a_nT^{p^n}\]
where $a_n$ are elements of $A$ which tends zero for the $\mathfrak{I}$-adic topology, but not for the discrete topology (i.e. $(a_n)$ is not eventually zero). The statement of \cref{scheme group over complete Noe formal homomorphism is algebraic} is also false if we drop the hypothesis that $G$ is affine, even for $H=\G_m$. To see an example, consider a discrete valuation ring $A$ and an elliptic curve over the fraction field $K$ of $A$, which reduces (in the reduction theory of N\'eron-Kodaira, say) to the group $\G_m$ over the residual field $k$. We will then have a smooth commutative group scheme $G$ over $S$, whose special fiber is $\G_{m,k}$ (which by \cref{scheme group multiplicative morphism nilpotent lifting exist} allows us to define a projective system of isomorphisms $u_n:H_n\stackrel{\sim}{\to}G_n$, where $H=\G_{m,A}$), but whose generic fiber is an abelian variety, so that there is no nontrivial homomorphism of $S$-groups $H\to G$.
\end{remark}

\subsection{Groups of multiplicative type over a field}
Let $k$ be a field and $G$ be an affine $k$-group. In this subsection, we establish some properties of diagonalizable and multiplicative groups over $k$. In particular, we shall see that any subgroup or quotient group of a diagonalizable (resp. multiplicative) group is diagonalizable (resp. multiplicative), and the property of being diagonalizable (resp. multiplicative) for a $k$-group $G$ has intimate relation with its linear representations.\par
We denote by $\mathscr{O}(G)$ the corresponding Hopf algebra of $G$ and by $X(G)=\Hom_{k\dash\Grp}(G,\G_{m})$ the group of characters of $G$. By the lemma of independence of characters, we see that $X(G)$ is a free subset of $\mathscr{O}(G)$ over $k$, and it corresponds to the group of group-like elements of $\mathscr{O}(G)$, i.e. to elements $e\in\mathscr{O}(G)$ such that $\Delta(e)=e\otimes e$ and $\eps(e)=1$.

\begin{lemma}\label{scheme k-group character linearly independent}
Let $A$ be a Hopf algebra over a field $k$. Then the group-like elements in $A$ are linearly independent over $k$. In particular, if $G$ is an affine $k$-group, then distinct characters of $G$ are linearly independent.
\end{lemma}
\begin{proof}
If not, it will be possible to express one group-like element $e$ as a linear combination of group-like elements
\[e=\sum_ia_ie_i,\quad a_i\in k.\]
We may even suppose that the $e_i$ occurring in the sum are linearly independent. Now we have
\begin{align*}
\Delta(e)&=e\otimes e=\sum_{ij}a_ia_je_i\otimes e_j,\\
\Delta(e)&=\sum_ia_i\Delta(e_i)=\sum_ia_ie_i\otimes e_i.
\end{align*}
Since $e_i\otimes e_j$ are linearly independent, this implies that
\[\begin{cases*}
a_ia_i=a_i&\text{for all $i$},\\
a_ia_j=0&\text{if $i\neq j$}.
\end{cases*}\]
We also know that $1=\eps(e)=\sum_ia_i\eps(e_i)=\sum a_i$, so combining these statements, we see that the $a_i$ form a complete set of orthogonal idempotents in the field $k$, and so one of them equals $1$ and the remainder are zero, which contradicts our assumption that $e$ is not equal to any of the $e_i$.
\end{proof}

\begin{proposition}\label{scheme k-group diagonalizable iff group-like element}
An affine $k$-group $G$ is diagonalizable if and only if $\mathscr{O}(G)$ is generated by group-like elements.
\end{proposition}
\begin{proof}
If $G=D_k(M)$ for an abelian group $M$, then $\mathscr{O}(G)$ is isomorphic to $k[M]$, and we know that $k[M]$ is generated by its group-like elements. Conversely, if $\mathscr{O}(G)$ is generated by its subgroup $M$ of group-like elements, then by \cref{scheme k-group character linearly independent}, $M$ is a basis over $k$ for $\mathscr{O}(G)$, so the induced homomorphism $k[M]\to\mathscr{O}(G)$ is an isomorphism of $k$-vector spaces. This isomorphism is compatible with the comultiplication maps because it is on the basis elements $m\in M$.
\end{proof}

\begin{corollary}\label{scheme k-group multiplicative quotient subgroup is multiplicative}
Let $G$ be a diagonalizable group over a field $k$.
\begin{enumerate}
    \item[(\rmnum{1})] Any subgroup or quotient group of $G$ is diagonalizable.
    \item[(\rmnum{2})] Any inductive limit or projective limit of multiplicative $k$-groups is diagonalizable.
\end{enumerate}
\end{corollary}
\begin{proof}
We note that (\rmnum{2}) follows from (\rmnum{1}) and the formula $D_k(\bigoplus_iM_i)\cong\prod_iD_k(M_i)$. To prove (\rmnum{1}), by taking an fpqc base change, we may assume that $G$ is diagonalizable. Using (\cite{SGA3-1} $\Rmnum{6}_B$ 11.13) and \cref{scheme group diagonalizable exact sequence prop}, we can reduce to the case where $G$ is of finite type over $k$. If $H$ is a subgroup of $G$, then $H\hookrightarrow G$ is a closed immersion, so $H$ is affine and the ring homomorphism $\mathscr{O}(G)\to\mathscr{O}(H)$ is surjective, and it sends group-like elements to group-like elements (being a homomorphism of Hopf algebras). As the group-like elements of $\mathscr{O}(G)$ span it, the same is true of $\mathscr{O}(H)$, so $H$ is diagonalizable. It then follows from \cref{scheme group diagonalizable exact sequence prop} that $G/H$ is identified with $D_k(N)$. 
\end{proof}

\begin{corollary}\label{scheme k-group G_m nonzero quotient is G_m}
Let $k$ be a field. Then any nonzero quotient of $\G_{m,k}$ is isomorphic to $\G_{m,k}$.
\end{corollary}
\begin{proof}
By \cref{scheme k-group multiplicative quotient subgroup is multiplicative}, the quotient group $H$ is diagonalizable, hence $H=D_k(M)$ with $M$ a nontrivial sub-$\Z$-module of $\Z$. But $\Z$ is free, so $M$ is also free, i.e. $H\cong\G_{m,k}$.
\end{proof}

\begin{theorem}\label{scheme k-group diagonalizable iff representation}
Let $G$ be a $k$-group, then the following conditions are equivalent:
\begin{enumerate}
    \item[(\rmnum{1})] $G$ is diagonalizable.
    \item[(\rmnum{2})] The regular representation of $G$ is diagonalizable.
    \item[(\rmnum{3})] Any linear representation of $G$ is diagonalizable.
    \item[(\rmnum{4})] Any finite dimensional linear representation is diagonalizable. 
\end{enumerate}
\end{theorem}
\begin{proof}
In view of the description of representations of a diagonalizable group $D_k(M)$ before \cref{scheme module over diagonalizable group cat equivalent to graded module}, we se that (\rmnum{1})$\Rightarrow$(\rmnum{3}), and (\rmnum{3})$\Rightarrow$(\rmnum{4})$\Rightarrow$(\rmnum{2}) is trivial. Now assume that the regular representation of $G$ is diagonalizable, i.e. $\mathscr{O}(G)$ is generated by its eigenvectors. If $f\in\mathscr{O}(G)$ is an eigenvector with corresponding character $\chi$, then for any $k$-algebra $R$, we have
\[f(hg)=f(h)\chi(g),\quad g,h\in G(R)=\Hom_k(\mathscr{O}(G),R)\]
Taking $h=e$, we then see that $f(g)=f(e)\chi(g)$, so the eivenvectors of $\mathscr{O}(G)$ are multiplies of characters. In view of \cref{scheme k-group diagonalizable iff group-like element}, this implies that $G$ is diagonalizable.
\end{proof}

\begin{theorem}\label{scheme k-group abelian multiplicative iff}
Let $G$ be an abelian $k$-group. Then the following conditions are equivalent:
\begin{enumerate}
    \item[(\rmnum{1})] $G$ is multiplicative.
    \item[(\rmnum{2})] There exists an extension $k'$ of $k$ such that $G\otimes_kk'$ is diagonalizable.
    \item[(\rmnum{3})] For any linear representation $G\to\GL(V)$, we have $H^i(G,V)=0$ for $i>0$.
    \item[(\rmnum{4})] Any homomorphism from $G$ to $\G_a$ is trivial.
    \item[(\rmnum{5})] Any linear representation of $G$ is semi-simple.
    \item[(\rmnum{6})] The regular representation of $G$ is semi-simple.    
\end{enumerate}
\end{theorem}
\begin{proof}
(\cite{DG} \Rmnum{4} \S 1, $\text{n}^\circ2$, 2.2).
\end{proof}
\begin{corollary}\label{scheme k-group multiplicative premanence prop}
Let $k$ be a field.
\begin{enumerate}
    \item[(\rmnum{1})] Any subgroup and quotient of a multiplicative $k$-group is multiplicative.
    \item[(\rmnum{2})] Any projective limit and inductive limit of multiplicative $k$-groups is multiplicative.
    \item[(\rmnum{3})] For an affine $k$-group to be multiplicative, it is necessary and sufficient that its quotient groups are.
    \item[(\rmnum{4})] If $k'$ is an extension of $k$, a $k$-group $G$ is multiplicative if and only if $G\otimes_kk'$ is.
\end{enumerate}
\end{corollary}
\begin{proof}
The first two assertions follow easily from \cref{scheme k-group multiplicative quotient subgroup is multiplicative}, and the others follow from \cref{scheme k-group abelian multiplicative iff}.
\end{proof}

\begin{proposition}\label{scheme k-group multiplicative extension prop}
Let $H$, $K$ be groups of multiplicative finite type over a field $k$, and $G$ be a $k$-group such that we have an exact sequence
\[\begin{tikzcd}
1\ar[r]&H\ar[r]&G\ar[r]&K\ar[r]&1
\end{tikzcd}\]
\begin{enumerate}
    \item[(a)] If $G$ is abelian or $K$ is connected, then $G$ is of multiplicative type.
    \item[(b)] If $K$ and $H$ are diagonalizable and $K$ is a torus, then $G$ is diagonalizable. 
\end{enumerate}
\end{proposition}
\begin{proof}
For the proof of (a), we refer to (\cite{SGA3-2} \Rmnum{17} 7.1.1), of which (a) is a particular case; the case of a field is treated in part (a) of the proof of (\cite{SGA3-2} \Rmnum{17} 7.1.1). Now suppose that $K$ and $H$ are diagonalizable:
\[K\cong D_k(M),\quad H\cong D_k(N)\]
and that $G$ is of multiplicative type (this follows from (a), since $K$ is a torus, hence connected). Then by (\cite{SGA3-2} \Rmnum{10} 1.4), $G$ is isotrivial over $k$, i.e. there exists a finite separable extension $k'/k$ such that $G'=G\times_kk'$ is diagonalizable, hence $G'=D_{k'}(E)$ for an abelian group $E$, and we have an exact sequence
\begin{equation}\label{scheme k-group multiplicative extension prop-1}
\begin{tikzcd}
0\ar[r]&D_{k'}(N)\ar[r]&D_{k'}(E)\ar[r]&D_{k'}(M)\ar[r]&0
\end{tikzcd}
\end{equation}
Hence, by \cref{scheme group diagonalizable exact sequence prop} and \cref{scheme group diagonalizable transpose mono epi iff}, $M$ is a subgroup of $E$ and we have an exact sequence
\begin{equation}\label{scheme k-group multiplicative extension prop-2}
\begin{tikzcd}
0\ar[r]&M\ar[r]&E\ar[r]&N\ar[r]&0
\end{tikzcd}
\end{equation}

Now for a given extension $E_0$ of $N$ by $M$, consider the diagonalizable $k$-group $G_0=D_k(E_0)$; the $k$-group functor of automorphisms of the extension
\begin{equation}\label{scheme k-group multiplicative extension prop-3}
\begin{tikzcd}
1\ar[r]&H\ar[r]&G_0\ar[r]&K\ar[r]&1
\end{tikzcd}
\end{equation}
i.e. the subfunctor in groups of $\sAut_{k\dash\Grp}(G_0)$ whose points over a $k$-scheme $T$ are the $\phi\in\Aut_{T\dash\Grp}(G_T)$ inducing the identity over $H_T$ and $K_T$, is identified with $L=\sHom_{k\dash\Grp}(K,H)$, which is, by \cref{scheme group sHom of D(M) representable if ft}, the constant $k$-group associated with $\Hom_{\Grp}(N,M)$. We hence see that the classification of extensions $G$ of $K$ by $H$, which over a separable closure $k^{\sep}$ become isomorphic to the extension (\ref{scheme k-group multiplicative extension prop-3}), is the same as that of $k$-torsors for the \'etale topology under the constant group $L_k$, which is classified by the Galois cohomology group ($L$ being a trivial $\Gamma$-module)
\[H^1_{\et}(k,L_k)=H^1(\Gamma,L)=\Hom_{\mathbf{Cont}.\Grp}(\Gamma,L)\]
If $K$ is a torus, then $M$ is torsion-free, hence so is $L$, whence $\Hom_{\mathbf{Cont}.\Grp}(\Gamma,L)=0$; it then follows that any extension of $K$ by $H$ is diagonalizable.
\end{proof}

\begin{example}\label{scheme alg group diagonalizable extension not diagonalizable example}
An extension of diagonalizable groups may not be diagonalzable. For example, let $G$ be the group of functor sending $S'$ to the group of \textbf{monomial matrices} over $\mathscr{O}_{S'}$ (that is, a matrix with exactly one nonzero element on each row and each column). Let $D_n\cong D_(\Z^n)$ be the diagonal subgroup of $\GL_n$, then it is not hard to see that we have $G=D_n\rtimes\mathfrak{S}_n$, where $\mathfrak{S}_n$ is the constant group associated with the permutation group of $n$ elements. We thus obtain an exact sequence
\[\begin{tikzcd}
e\ar[r]&D_n\ar[r]&G\ar[r]&\mathfrak{S}_n\ar[r]&e
\end{tikzcd}\]
Now consider the case $n=2$; the constant group $\mathfrak{S}_2$ is isomorphic to $D(\Z/2\Z)$, hence diagonalizable, and $G$ is not diagonalizable since it is not abelian.
\end{example}

\section{Characterization and classification of groups of multiplicative type}
\subsection{Classification of isotrivial groups of multiplicative type}
We recall that a group of multiplicative type $H$ over a scheme $S$ is called \textbf{isotrivial} if there exists a finite surjective \'etale morphism $S'\to S$ such that $H'=H\times_SS'$ is diagonalizable. We note that any such scheme $S'$ over $S$ is dominated by a finite \'etale connected $S$-scheme $S_1'$, which is Galois, i.e. a homogeneous principal bundle over $S$ under the group $\Gamma_S$, where $\Gamma$ is an ordinary finite group (cf. \cite{EGA4-4} 18.2.9). We may thus suppose that $S'$ is of this form, and we propose the determination of groups of multiplicative type over $S$ which \textit{splits} over $S'$, i.e. such that $H'=H\times_SS'$ is diagonalizable. By descent theory (\cite{SGA1} \Rmnum{8} 2.1 et 5.4), the category of such $H$ is equivalent to the category of diagonalizable groups $H'$ over $S'$, endowed with an action of $\Gamma$ compatible with the action of $\Gamma$ on $S'$. Now as $S'$ is connected, the contravariant functor
\[M\mapsto D_{S'}(M)\]
is an anti-equivalence from the category of ordinary abelian groups to the category of diagonalizable groups over $S'$, whose quasi-inverse is given by $H\mapsto\Hom_{S'\dash\Grp}(H,\G_{m,S'})$\footnote{Note that this is the set of characters of $H$ \textit{defined over $S'$}.} (cf. \cref{scheme group sHom of D(M) over connected base char}). We have therefore proved the following proposition:

\begin{proposition}\label{scheme group multiplicative split over Galois cat equivalence}
Let $S$ be a connected scheme and $S'$ be a Galois covering of $S$ with group $\Gamma$. Then the category of groups of multiplicative type over $S$ which split over $S'$ is anti-equivalent to the category of $\Gamma$-modules, i.e. the ordinary abelian groups endowed with an action $\Gamma\to\Aut_{\Grp}(M)$.
\end{proposition}

Now we want to generalize \cref{scheme group multiplicative split over Galois cat equivalence} to the category of any isotrivial multiplicative group over $S$, without fixing a trivialization. For this, we recall that the category of finite surjective \'etale morphisms $S'\to S$ is a Galois category, provided we choose a base point $\bar{s}$, i.e. a geometric point $\bar{s}:\Spec(\Omega)\to S$. 

\begin{corollary}\label{scheme group multiplicative cat equivalence}
Let $S$ be a connected scheme and $\bar{s}:\Spec(\Omega)\to S$ be a geometric point of $S$ (where $\Omega$ is a separably closed field). Consider the fundamental group of $S$ with base $\bar{s}$:
\[\pi=\pi_1(S,\bar{s}).\]
Then the category of isotrivial multiplicative groups $H$ over $S$ is anti-equivalent to the category of Galois $\pi$-modules\footnote{By definition, a Galois $\pi$-module $M$ is an abelian group endowed with a continuous action by $\pi$. Equivalently, this means the stabilizer of any point of $M$ is open in $\pi$.} whose action comes from a finite quotient of $\pi$ (i.e. such that the kernel $\pi\to\Aut_{\Grp}(M)$ is open).
\end{corollary}
\begin{proof}
For any isotrivial multiplicative group $H$ over $S$, we can associate the group
\[M=\Hom_{\Omega\dash\Grp}(H_{\bar{s}},\G_{m,\Omega}),\]
where $H_{\bar{s}}$ is the geometric fiber of $H$ at $\bar{s}$; this group is endowed in a natural way an action of $\pi_1(S,\bar{s})$, which factors through the Galois group of a trivilizing Galois cover of $S$. Conversely, if $M$ is a Galois $\pi$-module whose action comes from a finite quotient $\Gamma$, then this corresponds to a Galois covering $S'$ over $S$, and we can form the diagonalizable group $H'=D_{S'}(M)$, which is naturally endowed with an action of $\Gamma$. By \cref{scheme group multiplicative split over Galois cat equivalence}, we then obtain a group $H$ of multiplicative type over $S$ which split over $S'$. By taking a section of $\Spec(\Omega)$ to $S'_{\bar{s}}$, which is isomorphic to $\coprod_{g\in\Gamma}\Spec(\Omega)$, we then conclude that $\sHom_{\Omega\dash\Grp}(H_{\bar{s}},\G_{m,\Omega})(S_{\bar{s}})\cong M$.\par
Finally, if $H$ is an isotrivial multiplicative group over $S$ and $S'\to S$ is a Galois covering which trivializes $H$ with Galois group $\Gamma$. If $\Gamma_1=G/\ker(\pi\to\Aut(M))$ with $M=\Hom_{\Omega\dash\Grp}(H_{\bar{s}},\G_{m,\Omega})$, then the action of $\pi$ factors through $\Gamma$, so $\Gamma_1$ is a quotient of $\Gamma$ and hence the Galois covering $S'_1$ corresponding to $\Gamma_1$ dominates $S'$. Now the diagonalizable group $D_{S_1'}(M)$, endowed with the descent data given by the action of $\Gamma_1$, has descent object $D_{S'}(M)$ over $S'$ (since the action of $\Gamma$ is induced from that of $\Gamma_1$). The corresponding descent objects of these descent datum are therefore isomorphic, which proves that the above two constructions are inverses of each other.
\end{proof}

We will see below (cf. \cref{*}) that if $S$ is normal, or more generally geometrically unibranch, then any group of multiplicative finite type over $S$ is necessarily isotrivial, so classification principle \cref{scheme group multiplicative cat equivalence} is applicable to groups of multiplicative finite type on $S$, which correspond to Galois $\pi$-modules which are finitely generated over $\Z$. In particular, we have the following result:

\begin{proposition}\label{scheme alg group multiplicative cat equivalence}
Let $k$ be a field, $H$ be a group of multiplicative finite type over $k$. Then $H$ is isotrivial, i.e. there exists a finite separable extension $k'$ over $k$ which trivializes $H$. Concequently, if $\pi$ is the absolute Galois group of $k$, then the category of groups of multiplicative finite type over $k$ is anti-equivalent to the category of finitely generated\footnote{A Galois $\pi$-module $M$ is finitely generated if and only if it is finitely generated over $\Z$. In fact, if $m_1,\dots,m_n$ is a generating family of $M$ over $\Z[\pi]$, then the stabilizer of each $m_i$ has finite index in $\pi$, so its orbit under $\pi$ is finite. Therefore, the $m_i$, together with their conjugates, form a finite subset of $M$ which generates $M$ over $\Z$.} Galois $\pi$-modules\footnote{We note that if $M$ is a finitely generated Galois $\pi$-module, then the kernel of $\pi\to\Aut_{\Grp}(M)$ is necessarily open.}.
\end{proposition}
\begin{proof}
Since $H$ is of finite type over $k$, this follows from the "principle of finite extension" (cf. \cite{EGA4-3} 9.1.4): By hypothesis, there exists a diagonalizable group $G$ of finite type over $k$, a faithfully flat and quasi-compact morphism $S'\to S=\Spec(k)$, and an isomorphism of $S'$-groups $u:H_{S'}\cong G_{S'}$. By replacing $S'$ with the residue field of a point of $S'$, we may asume that $S'$ is the spectrum of a field extension $K$ of $k$. The latter is the inductive limit of subalgebras $A_i$ of finite type, it follows from (\cite{EGA4-3} 8.8.2.4) that $u$ comes from an $A_i$-isomorphism $u_i:H_{A_i}\cong G_{A_i}$, for $i$ large enough. In view of Hilbers Nullstellensatz, there exists a quotient field $k'$ of $A_i$ which is a finite extension of $k$, this extension then trivilizes $H$. Now $k'$ is a purely inseparable extension of a separable extension $k_s'$ of $k$. In view of \cref{scheme group multiplicative morphism effective epi descent} (the extension $k'\to k_s'$ is purely inseparable, so the corresponding morphism has geometrically connected fibers), the isomorphism $u':H_{k'}\cong G_{k'}$ comes from an isomorphism $H_{k'_s}\cong G_{k'_s}$, which completes the proof.
\end{proof}

\begin{example}
The correspondence of \cref{scheme group multiplicative cat equivalence} gives in particular a characterization of isotrivial tori over $S$ of relative dimension $n$: put $\pi=\pi_1(S,\bar{s})$, then the isomorphism classes of such tori correspond to representations $\pi\to\GL(n,\Z)$, whose kernel is an open subgroup of $\pi$.
\end{example}

\begin{example}\label{scheme alg group multiplicative character construction}
Let $k$ be a field and $\bar{k}$ be an algebraic closure of $k$. Then any finite separable extension of $k$ can be embedded into $\bar{k}$, so in the correspondence of \cref{scheme alg group multiplicative cat equivalence}, if $M$ is a finitely generated $\pi$-module, the corresponding multiplicative group $H$ over $k$ can be deduced by descenting $D_{\bar{k}}(M)$. In particular, for any extension $K$ of $k$, we then have
\[H(\Spec(K))=D_{\bar{k}}(M)^{\pi}(K)=D_{\bar{k}}(M)^{\pi_K}(K)=\Hom_{\Grp}(M,\bar{k}^\times)^{\pi_K},\]
where $\pi_K$ is the Galois group $\Gal(\bar{k}/K)$. In other words, the value of $H$ over $K$ is the set of group homomorphisms $M\to\bar{k}^{\times}$ commuting with the actions of $\pi_K$.\par
As a particular example, consider $k=\R$ and $\bar{k}=\C$, so $\pi=\Z/2\Z$, and let $M=\Z$. Then the automorphism group of $M$ is isomorphic to $\Z/2\Z$, so there are two possible actions of $\pi$ on $M$ (denote by $H$ the corresponding multiplicative group):
\begin{enumerate}
    \item[(a)] The trivial action, and in this case $H(\R)=\R^\times$, and $H\cong\G_{m,\R}$.
    \item[(b)] The action given by the identity on $\Z/2\Z$. Then $H(\R)=\Hom(\Z,\C^\times)^{\pi}$, which consists of elements of $\C^\times$ fixed by the conjugate action of $\Z/2\Z$:
    \[z\mapsto\bar{z}^{-1}.\]
    Thus $H(\R)=\{z\in\C^{\times}:|z|^2=1\}=S^1$, which is compact. 
\end{enumerate}
\end{example}

\begin{example}\label{scheme group multiplicative non-isotrivial example}
Even if $S$ is an algebraic curve, there may exist over $S$ tori (of relative dimension 2) which are not locally isotrivial (and a fortiori not isotrivial); there can also exist non-isotrivial locally trivial tori. (Note however that such phenomena can occur only if $S$ is not normal, cf. \cite{SGA3-2} \Rmnum{10}, 5.16). For example, let $S$ be an irreducible algebraic curve (over an algebraically closed field) having a double point $a$, $S'$ be its normalization, and $b$, $c$ be the points of $S'$ lying over $a$. We then constructs a connected principal homogeneous bundle $P$ over $S$, of structural group $\Z$, by linking infinite copies of $S'$ according to the diagram
\[\begin{tikzcd}
b\ar[rd]& &b\ar[rd]& &b\ar[rd]&\\
 &c\ar[ru]& &c\ar[ru]& &c
\end{tikzcd}\]
(This is a principal bundle for the \'etale topology.) Now we have a homomorphism
\[\varphi:\Z\to\GL(2,\Z),\quad n\mapsto\begin{pmatrix}
1&n\\
0&1
\end{pmatrix}\]
which allows us to construct a torus $T$ over $S$, of relative dimension $2$, from the trivial torus $\G_m^2$ over $P$ and the descent data induced from $\varphi$. (We note that the projection $P\to S$ is covering for the \'etale topology and a fortiori for the canonical topology of $\Sch$, and that the considered descent data is necessarily effective, since $\G_{m,P}^2$ is affine over $P$). It is not difficult to prove that $T$ is not isotrivial in the neighborhood of $a$ (see \cite{SGA3-2} \Rmnum{10}, 7.3) (it is trivial in $S-\{a\}$).
\end{example}

\subsection{Variations of the infinitesimal structure}
In this subsection, we let $S$ be a (fixed) scheme and $S_0$ be a subscheme of $S$ with the same underlying space (i.e. such that the immersion $S_0\to S$ is a homeomorphism, or equivalently, defined by a nilideal $\mathscr{I}$). We recall that by (\cite{EGA4-4} 18.1.2), the functor
\[X\mapsto X_0=X\times_SS_0\]
is an equivalence between the category of \'etale schemes over $S$ and the analogous category over $S_0$. We now consider the similar question for multiplicative groups over $S$ and $S_0$:

\begin{proposition}\label{scheme group multiplicative nilideal reduction}
The functor $H\mapsto H_0=H\times_SS_0$ from the category of groups of multiplicative type over $S$ to that over $S_0$ is fully faithful, and it induces an equivalence between the category of quasi-isotrivial multiplicative groups over $S$ and the analogous category over $S_0$.
\end{proposition}
\begin{proof}
We first prove the fully faithfulness, i.e. if $H,G$ over $S$ are of multiplicative type, then
\[\Hom_{S\dash\Grp}(H,G)\to\Hom_{S_0\dash\Grp}(H_0,G_0)\]
is bijective. This question is local over $S$, so we can assume that $S$ is affine, and there then exists a faithfully flat and quasi-compact morphism $S'\to S$ which splits $H$ and $G$. Let $S''=S'\times_SS'$, and denote by $H',G'$ (resp. $H'',G''$) the groups deduced from $H,G$ by base change $S'\to S$ (resp. $S''\to S$); similarly, we define $S_0'$ and $S_0''$, the latter being isomorphic to $S_0'\times_{S_0}S_0'$. We then have a commutative diagram with exact rows:
\[\begin{tikzcd}
\Hom_{S\dash\Grp}(H,G)\ar[d,"u"]\ar[r]&\Hom_{S'\dash\Grp}(H',G')\ar[r,shift left=2pt]\ar[r,shift right=2pt]\ar[d,"u'"]&\Hom_{S''\dash\Grp}(H'',G'')\ar[d,"u''"]\\
\Hom_{S_0\dash\Grp}(H_0,G_0)\ar[r]&\Hom_{S'_0\dash\Grp}(H'_0,G'_0)\ar[r,shift left=2pt]\ar[r,shift right=2pt]&\Hom_{S''_0\dash\Grp}(H''_0,G''_0)
\end{tikzcd}\]
so to prove that $u$ is bijective, it suffices to show that so are $u'$ and $u''$, which then reduces us to the case where $H,G$ are diagonalizable, i.e. $H=D_S(M)$ and $G=D_S(N)$, with $M$ and $N$ being ordinary abelian groups. We have similarly $H_0=D_{S_0}(M)$, $G_0=D_{S_0}(N)$, so there is a commutative diagram
\[\begin{tikzcd}
\Hom_{S\dash\Grp}(N_S,M_S)\ar[r,"\sim"]\ar[d]&\Hom_{S\dash\Grp}(D_S(M),D_S(N))\ar[d]\\
\Hom_{S_0\dash\Grp}(N_{S_0},M_{S_0})\ar[r,"\sim"]&\Hom_{S_0\dash\Grp}(D_{S_0}(M),D_{S_0}(N))
\end{tikzcd}\]
where the horizontal arrows are isomorphisms in view of \cref{scheme group locally diagonalizable reflexive}. We are then reduced to prove that the homomorphism
\begin{equation}\label{scheme group multiplicative nilideal reduction-1}
\Hom_{S\dash\Grp}(N_S,M_S)\to\Hom_{S_0\dash\Grp}(N_{S_0},M_{S_0})
\end{equation}
is bijective, i.e. the functor $M_S\mapsto M_{S_0}$ is faithfully flat. Now (\ref{scheme group multiplicative nilideal reduction-1}) is also identified with the natural map
\[\Hom_{\Grp}(N,\Gamma(M_S))\to\Hom_{\Grp}(N,\Gamma(M_{S_0}))\]
induced from $\Gamma(M_S)\to\Gamma(M_{S_0})$ (cf. (\cref{scheme sHom of constant group char by global section})), which is evidently bijective (because $\Gamma(M_S)$ is the set of locally constant maps from $M$ to $S$, which only depends on the underlying topological space of $S$), whence the first assertion.\par
To prove the second assertion of \cref{scheme group multiplicative nilideal reduction}, we must show that any quasi-isotrivial multiplicative group $H_0$ over $S_0$ comes from a quasi-isotrivial multiplicative group $H$ over $S$. For this, let $S_0'\to S_0$ be a \'etale surjective morphism trivilizing $H_0$. As we have remarked, there then exists an \'etale morphism $S'\to S$ and an $S_0$-isomorphism $S'\times_SS_0\cong S'_0$. As $H_0'$ is diagonalizable, we immediately see that it comes from a diagonalizable group $H'$ over $S'$ by base change $S_0'\to S'$ (if $H_0'=D_{S_0'}(M)$, we can put $H'=D_{S'}(M)$). Define as usual $S''=S'\times_SS'$, $S'''=S'\times_SS'\times_SS'$, and $S''_0$, $S_0'''$. Using the fully faithfulness for the case $(S''',S_0''')$ and $(S''',S_0''')$, we see that the natural descent data over $H_0'$ relative to $S_0'\to S_0$ comes from a well-defined descent data over $H'$ relative to $S'\to S$. This descent data is effective since $H'$ is affine over $S'$ (\cite{SGA1} \Rmnum{8} 2.1), so there exists an $S$-group $H$ such that $H\times_SS'=H'=D_{S'}(M)$, and hence is quasi-isotrivial multiplicative. We then verify easily, using the fully faithfulness for $(S',S_0')$ and $(S,S_0)$, that the given isomorphism between $H_0'$ and $H'\times_{S'}S_0'$ comes from an isomorphism between $H_0$ and $H\times_SS_0$.
\end{proof}

\begin{corollary}\label{scheme group multiplicative nilideal reduction triviality}
Let $H$ be an $S$-group of multiplicative type, and $H_0=H\times_SS_0$. For $H$ to be quasi-isotrivial (resp. locally isotrivial, resp. isotrivial, resp. locally trivial, resp. trivial), it is necessary and sufficient that $H_0$ be so.
\end{corollary}
\begin{proof}
The necessary direction is trivial, and the sufficient part is done for "quasi-isotrivial", because thanks to full fidelity, it suffices to know that any quasi-isotrivial group over $S_0$ can be lifted into a quasi-isotrivial group over $S$. The same argument works for "trivial". For the "isotrivial" case, we use the reasoning establishing the second assertion of \cref{scheme group multiplicative nilideal reduction}, but taking $S_0'\to S_0$ to be finite \'etale surjective. The "locally isotrivial" and "locally trivial" cases result immediately from the "isotrivial" and "trivial" cases.
\end{proof}

\begin{corollary}\label{scheme group flat multiplicative iff nilpotent reduction}
Suppose that the ideal $\mathscr{I}$ defining $S_0$ is locally nilpotent. Let $H$ be a flat $S$-group and $H_0=H\times_SS_0$. For $H$ to be of quasi-isotrivial multiplicative, it is necessary and sufficient that $H_0$ be so.
\end{corollary}
\begin{proof}
Suppose that $H_0$ is quasi-isotrivial multiplicative, and we show that $H$ is so. As the question is local for the \'etale topology and the category of \'etale schemes over $S$ is equivalent to that over $S_0$ under the function $S'\mapsto S'\times_SS_0$, we are then reduced to the case where $H_0$ is diagonalizable, hence isomorphic to $D_{S_0}(M)$. Let $G=D_S(M)$, we then have an isomorphism $u_0:H_0\stackrel{\sim}{\to} G_0$; we claim that this comes from a unique homomorphism $u:H\to G$, which must be an isomorphism\footnote{Since $u_0$ is an isomorphism it suffices to see that for any $h\in H$, the morphism $\phi:\mathscr{O}_{G,u(h)}\to\mathscr{O}_{H,h}$ is an isomorphism. This is true after reduction modulo $\mathfrak{I}=\mathscr{I}_s$ ($s$ being the image of $h$ in $S$), so its cokernel $C$ satisfies $C=\mathfrak{I}C$, whence $C=0$ as $\mathfrak{I}$ is nilpotent. As $\mathscr{O}_{H,h}$ is flat over $\mathscr{O}_{S,s}$, the kernel of $\phi$ also verifies $K=\mathfrak{I}K$, whence $K=0$.}. To see this, note that we have (cf. \cref{category pairing of D(G) isomorphism to bilinear form-2})
\[\Hom_{S\dash\Grp}(H,G)\cong\Hom_{S\dash\Grp}(M_S,\sHom_{S\dash\Grp}(H,\G_{m,S}))\]
and the second member is identified with $\Hom_{\Grp}(M,\Hom_{S\dash\Grp}(H,\G_{m,S}))$, so the homomorphism
\begin{equation}\label{scheme group flat multiplicative iff nilpotent reduction-1}
\Hom_{S\dash\Grp}(H,G)\to\Hom_{S_0\dash\Grp}(H_0,G_0)
\end{equation}
is isomorphic to the homomorphism
\begin{equation}\label{scheme group flat multiplicative iff nilpotent reduction-2}
\Hom_{\Grp}(M,\Hom_{S\dash\Grp}(H,\G_{m,S}))\to\Hom_{\Grp}(M,\Hom_{S_0\dash\Grp}(H_0,\G_{m,S_0}))
\end{equation}
induced from the restriction homomorphism
\begin{equation}\label{scheme group flat multiplicative iff nilpotent reduction-3}
\Hom_{S\dash\Grp}(H,\G_{m,S})\to\Hom_{S_0\dash\Grp}(H_0,\G_{m,S_0}).
\end{equation}
Therefore, to prove that (\ref{scheme group flat multiplicative iff nilpotent reduction-1}) is bijective, it suffices to show that (\ref{scheme group flat multiplicative iff nilpotent reduction-3}) is bijective. Since this question is local over $S$, we can assume that $S$ is affine, Now $H_0$ is of multiplicative type and $\G_{m,S}$ is smooth over $S$ and abelian, so the assertion follows from \cref{scheme group multiplicative morphism nilpotent lifting exist}~(a), which completes the proof\footnote{We recall that the proof of \cref{scheme group multiplicative morphism nilpotent lifting exist}~(a) only depends on the vanishing of $H^1$ and $H^2$; in view of (\cite{SGA3-1} \Rmnum{3} 2.1), we only need $H_0$ to be of multiplicative type.}.
\end{proof}

\begin{corollary}\label{scheme group multiplicative over local Artin closed reduction}
Let $A$ be a local Artinian ring with residue field $k$, $S=\Spec(A)$, and $S_0=\Spec(k)$.
\begin{enumerate}
    \item[(a)] Let $H$ be a flat $S$-group locally of finite type, such that $H_0=H\times_SS_0$ is of multiplicative type. Then $H$ is of isotrivial multiplicative finite type over $S$. In particular, any $S$-group $H$ of multiplicative finite type is isotrivial.
    \item[(b)] The functor $H\mapsto H_0$ is an equivalence between the category of groups of multiplicative finite type over $A$ to the analogous category over $k$.
\end{enumerate}
\end{corollary}
\begin{proof}
Let $H$ be as in (a), then $H_0$ is of multiplicative type and locally of finite type over $S_0$, hence of finite type, and isotrivial by \cref{scheme alg group multiplicative cat equivalence}. By \cref{scheme group multiplicative nilideal reduction triviality} and \cref{scheme group flat multiplicative iff nilpotent reduction}, $H$ is of multiplicative type (hence of finite type) and isotrivial, and assertion (b) follows from \cref{scheme group multiplicative nilideal reduction}.
\end{proof}

\begin{corollary}\label{scheme group multiplicative fpqc radiciel base change}
Let $S'\to S$ be a faithfully flat, quasi-compact and radiciel morphism.
\begin{enumerate}
    \item[(a)] The functor $H\mapsto H'=H\times_SS'$, from the category of multiplicative type over $S$ to the analogous category over $S'$, is fully faithful, and it induces an equivalence between the subcategory formed by quasi-isotrivial multiplicative groups.
    \item[(b)] If $H$ is of multiplicative type, for it to be quasi-isotrivial (resp. locally isotrivial, resp. isotrivial, resp. locally trivial, resp. trivial), it is necessary and sufficient that $H_0$ be so.
\end{enumerate}
\end{corollary}
\begin{proof}
Let $S''=S'\times_SS'$ and $S'''=S'\times_SS'\times_SS'$, then the hypothesis on $S'\to S$ implies that the diagonal immersions $S'\to S''$ and $S'\to S'''$ are surjective, hence base change along these immersions is justified by \cref{scheme group multiplicative nilideal reduction} and \cref{scheme group multiplicative nilideal reduction triviality}. Given that $S'\to S$ is an effective descent morphism for the fiber category of groups of multiplicative type over schemes (because it is faithfully flat and quasi-compact), our assertion follows formally from \cref{scheme group multiplicative nilideal reduction} and \cref{scheme group multiplicative nilideal reduction triviality}.
\end{proof}

\subsection{Variations of finite structure and the quasi-isotrivial theorem}\label{scheme group multiplicative quasi-isotrivial theorem subsection}
\paragraph{Variations of finite structure: case of complete base} In this paragraph, we consider a (fixed) Noetherian ring $A$ endowed with an ideal $\mathfrak{I}$ such that $A$ is separated and complete for the $\mathfrak{I}$-adic topology. Let $S=\Spec(A)$, $S_0=\Spec(A/\mathfrak{I})$, we want to extend the results of the previous subsection to multiplicative groups over $S$ and $S_0$.

\begin{lemma}\label{scheme group over Noe complete homomorphism reduction}
Let $G,H$ be $S$-groups with $G$ of isotrivial multiplicative type, $H$ affine over $S$ and flat over $S$ at the points of $H_0$, and $H_0$ of quasi-isotrivial multiplicative type. Then the following natural map is bijective:
\[\Hom_{S\dash\Grp}(G,H)\to\Hom_{S_0\dash\Grp}(G_0,H_0).\]
\end{lemma}
\begin{proof}
For any integer $n\geq 0$, we put $S_n=\Spec(A/\mathfrak{I}^{n+1})$, and let $G_n$, $H_n$ be the induced groups by base change $S_n\to S$. As $G/S$ is of isotrivial multiplicative type and $H$ is affine over $S$, by \cref{scheme group over complete Noe formal homomorphism is algebraic}, the natural homomorphism
\[\Hom_{S\dash\Grp}(G,H)\to\llim_n\Hom_{S_n\dash\Grp}(G_n,H_n)\]
is bijective. On the other hand, in view of \cref{scheme group flat multiplicative iff nilpotent reduction}, the $H_n$ are of quasi-isotrivial multiplicative type, and in view of \cref{scheme group multiplicative nilideal reduction}, the transition homomorphisms in the projective system $(\Hom_{S_n\dash\Grp}(G_n,H_n))_n$ are isomorphism, whence our assertion.
\end{proof}

\begin{theorem}\label{scheme group isotrivial multiplicative over Noe complete reduction}
Consider a Noetherian ring $A$ endowed with an ideal $\mathfrak{I}$ such that $A$ is separated and complete for the $\mathfrak{I}$-adic topology. Let $S=\Spec(A)$ and $S_0=\Spec(A/\mathfrak{I})$.
\begin{enumerate}
    \item[(a)] The functor $H\mapsto H_0=H\times_SS_0$ is an equivalence between the category of isotrivial multiplicative groups over $S$ to the analogous category over $S_0$.
    \item[(b)] For an $S$-group $H$ of multiplicative finite type, $H$ is isotrivial if and only if $H_0$ is.
\end{enumerate}
\end{theorem}
\begin{proof}
For the first assertion, one can either repeat the proof of \cref{scheme group multiplicative nilideal reduction}, or use \cref{scheme group multiplicative cat equivalence}, noting that in both cases the functor
\[S'\mapsto S'_0=S'\times_SS_0\]
from the category of finite \'etale schemes over $S$ to the category of finite \'etale schemes over $S_0$, is an equivalence (\cite{SGA1} \Rmnum{1} 8.4).\par
We now prove the second assertion, i.e. that if $H_0$ is isotrivial, so is $H$. In view of what we have proved, there exists an isotrivial multiplicative group $G$ over $S$ and an $S_0$-isomorphism
\[u_0:G_0\stackrel{\sim}{\to} H_0.\]
As $H$ is of finite type, so are $H_0$ and $G_0$, so by \cref{scheme group multiplicative schematic prop}~(b), the type of $G$ at each point of $S$ is a finitely generated abelian group, and hence $G$ is of finite type over $S$. On the other hand, in view of \cref{scheme group over Noe complete homomorphism reduction}, $u_0$ comes from a homomorphism of $S$-groups
\[u:G\to H.\]
Now $G,H$ are of multiplicative finite type over $S$, and $u_0$ is an isomorphism, so by \cref{scheme group multiplicative ft mono epi locus prop}, $u$ is an isomorphism (any neighborhood of $S_0$ in $S$ is equal to $S$).
\end{proof}

\begin{corollary}\label{scheme group multiplicative ft over complete Noe local isotrivial}
Let $A$ be a complete Noetherian local ring with residue field $k$.
\begin{enumerate}
    \item[(a)] Any group of multiplicative finite type over $A$ is isotrivial.
    \item[(b)] The functor $H\mapsto H_0=H\otimes_Ak$ is an equivalence between the category of groups of multiplicative finite type over $A$ and over $k$.
\end{enumerate}
\end{corollary}
\begin{proof}
The first assertion follows from \cref{scheme group isotrivial multiplicative over Noe complete reduction}~(b) and \cref{scheme alg group multiplicative cat equivalence}, and the second one follows from \cref{scheme group isotrivial multiplicative over Noe complete reduction}~(a), in view of the fact that $H$ is of finite type if and only if $H_0$ is.
\end{proof}

\begin{remark}
We note that in view of \cref{scheme alg group multiplicative cat equivalence}, \cref{scheme group multiplicative ft over complete Noe local isotrivial} gives a complete classification of groups of multiplicative fintie type ove $A$ in terms of modules under the absolute Galois group of $k$.
\end{remark}

\begin{remark}
Under the hypothesis of \cref{scheme group isotrivial multiplicative over Noe complete reduction} (i.e. $A$ is Noetherian, separated and complete for the $\mathfrak{I}$-adic topology), it follows from \autoref{*} that the functor $H\mapsto H_0$, form the category of groups of multiplicative finite type over $S$ to that over $S_0$, is fully faithful (without the hypothesis of isotrivality).\par
However, it is not in general essentially surjective, in fact there can be an $S_0$-group $H_0$, of multiplicative finite type, locally trivial if we want (but not isotrivial), which does not come from a group of multiplicative type $H$ over $S$ by reduction. To see this, let us take as in \cref{scheme group multiplicative non-isotrivial example} a non-isotrivial multiplicative group over a non-normal curve. We can obviously assume $S$ to be affine, and suppose that it is embedded in the affine space of dimension $2$, thus defined by an ideal $\mathfrak{I}$ in $k[X,Y]$. We will take for $A$ the completion of this ring for the $\mathfrak{I}$-adic topology, so that $A$ is a regular ring of dimension $2$, a fortiori normal. We will see in (\cite{SGA3-2} \Rmnum{10} 5.16) that any group of multiplicative finite type over $S=\Spec(A)$ is isotrivial; hence $H_0$, which is non-isotrivial and of finite type over $S_0$, does not come from a group of multiplicative type $H$ over $S$ (because $H$ would necessarily be of finite type, hence isotrivial).
\end{remark}

\begin{lemma}\label{scheme alg group open multiplicative subgroup torsion dense}
Let $H$ be an abelian algebraic group over a field $k$ which admits an open subgroup $G$ of multiplicative type. Then the family of subschemes $({_nH})_{n>0}$ of $H$ is schematically dense, and in particular if ${_nH}={_nG}$ for any $n>0$, then $H=G$.
\end{lemma}
\begin{proof}
Let $\bar{k}$ be an algebraic closure of $k$; it suffices to prove that the family $({_nH_{\bar{k}}})_{n>0}$ is schematically dense in $H_{\bar{k}}$, because then $({_nH})_{n>0}$ is schematically dense in $H$ (\cref{scheme schematically dense faithfully flat descent}). Hence, we may assume that $k$ is algebraically closed, hence $G=D_k(M)$ for a finitely generated abelian group $M$. Let $M_0=M/M_{\mathrm{tor}}$, where $M_{\mathrm{tor}}$ denotes the torsion subgroup of $M$, then $T=D_k(M_0)$ is the largest torus in $G$, and $H/T$ is finite\footnote{We note that $(H/T)/(G/T)$ is isomorphic to $H/G$, which is discrete and quasi-compact, hence finite (over $k$). As $G/T$ is by construction also finite over $k$, we conclude that $H/T$ is finite over $k$.}, so $H(k)/T(k)$ is annihilated by an integer $\nu>0$. We can find finitely many elements $g_i\in H(k)$ such that $H=\coprod_ig_iG$, and we have $g_i^\nu\in T(k)$. As $k$ is algebraically closed, $\nu\cdot\id_T$ is surjective in $T(k)\cong (k^\times)^d$ for some $d>0$, hence by replacing the $g_i$ by $g_it_i^{-1}$, where $t_i\in T(k)$ is such that $t_i^\nu=g_i^\nu$, we can assume that $g_i^\nu=1$. If $n$ is a multiple of $\nu$, we then have
\[{_nH}\supseteq g_i\cdot {_nG}\]
and as the family ${_nG}$ is schematically dense in $G$ in view of \cref{scheme group multiplicative G[n] schematically dense}, we conclude that ${_nH}$ is schematically dense.
\end{proof}

\begin{theorem}\label{scheme group over Noe complete reduction multiplicative clopen lifting}
Let $A$ be a Noetherian ring endowed with an ideal $\mathfrak{I}$ such that $A$ is separated and complete for the $\mathfrak{I}$-adic topology. Let $S=\Spec(A)$, $S_0=\Spec(A/\mathfrak{I})$, and $H$ be an affine $S$-group of finite type and affine over $S$ at the points of $H_0$, such that $H_0=H\times_SS_0$ is of isotrivial multiplicative type. Then there exists an clopen subgroup $G$ of $H$ of isotrivial multiplicative finite type such that $G_0=H_0$.
\end{theorem}
\begin{proof}
In view of \cref{scheme group isotrivial multiplicative over Noe complete reduction}, there exists a group of isotrivial multiplicative type over $G$ and an isomorphism
\[u_0:G_0\stackrel{\sim}{\to} H_0.\]
By \cref{scheme group over Noe complete homomorphism reduction}, $u_0$ comes from a unique homomorphism of $S$-groups
\[u:G\to H.\]
Using \cref{scheme group morphism from multiplicative mono locus open}, we see that $u$ is a monomorphism (because if $U$ is the set of $s\in S$ such that $u_s:G_s\to H_s$ is a monomorphism, then $U$ is an open neighborhood of $S_0$ in $S$, and $G|_U\to H|_U$ is a monomorphism). In view of \cref{scheme group mono multiplicative to affine prop}, $u$ is then a closed immersion, so $G$ is of finite type, hence of finite presentation over $S$. By \cref{scheme group morphism restriction to local fp smooth uo locus}, there exists an open neighborhood $U$ of $S_0$, hence equals to $S$, such that $G|_U\to H|_U$ is \'etale. The morphism $u$ is then \'etale, whence an open immersion (as it is a \'etale monomorphism, cf. \cite{EGA4-4} 17.9.1), and this proves the assertion.
\end{proof}

\begin{corollary}\label{scheme group multiplicative over Noe complete isotrivial and reduction}
Under the hypotheses of \cref{scheme group over Noe complete reduction multiplicative clopen lifting}, let $H$ be an $S$-group of finite type, affine and flat over $S$. For $H$ to be of isotrivial multiplicative type, it is necessary and sufficient that $H_0$ be so, and that $H$ satisfies the following equivalent conditions:
\begin{enumerate}
    \item[(\rmnum{1})] The fibers of $H$ are of multiplicative type, and their types are constant over each connected component of $S$.
    \item[(\rmnum{2})] $H$ is abelian and the subgroups ${_nH}$ ($n>0$) are finite over $S$.
    \item[(\rmnum{3})] The fibers of $H$ are connected.  
\end{enumerate}
\end{corollary}
\begin{proof}
Of course, if $H$ is of multiplicative type (and isotrivial), then conditions (\rmnum{1}) and (\rmnum{2}) are verified, by \cref{scheme group multiplicative schematic prop}~(c), so we only need to prove the sufficient part. We shall utilize the group $G$ indicated in \cref{scheme group over Noe complete reduction multiplicative clopen lifting}. If condition (\rmnum{3}) is satisfied, then we have evidently $H=G$. In the case (\rmnum{2}), we note that the open immersion $u:G\to H$ induces an open immersion
\[{_nu}:{_nG}\to {_nH}\]
which induces an isomorphism $({_nG})_0\stackrel{\sim}{\to}({_nH})_0$; as ${_nH}$ is finite over $S$, this immediately implies that ${_nu}$ is an isomorphism (this follows by Nakayama's lemma, cf. \cref{admissible ring nilideal contained in Jacobson radical}). In view of \cref{scheme alg group open multiplicative subgroup torsion dense}, this ensures that the induced morphisms $u_s:G_s\to H_s$ on the fibers are isomorphisms, so $u$ is surjective, whence an isomorphism.\par
Finally, in case (\rmnum{1}), we can assume that $S$ is connected, and then for any $s\in S$, $u_s:G_s\to H_s$ is a monomorphism of algebraic groups of multiplicative type of the same type over $\kappa(s)$\footnote{Since $G_s$ and $H_s$ are of the same type for any $s\in S_0$, hence for any $s\in S$ (as $A$ is complete and separated, the connected components of $S$ and $S_0$ in bijection under $C\mapsto C\cap S_0$, cf. \cite{EGA4-4} 18.5.4 et 18.15.13).}. We claim that this monomorphism is necessarily an isomorphism. In fact, we can suppose, by replacing the base field, that the two groups over $\kappa(s)$ are diagonalizable, and then this follows from \cref{scheme group diagonalizable transpose mono epi iff} and the fact that a surjective homomorphism of isomorphic finitely generated $\Z$-modules is necessarily bijective.
\end{proof}

\begin{corollary}\label{scheme group over Noe complete isotrivial torus iff reduction}
Under the hypotheses of \cref{scheme group over Noe complete reduction multiplicative clopen lifting}, let $H$ be an $S$-group of finite type, affine and flat over $S$, with connected fibers. Then for $H$ to be an isotrivial torus, it is necessary and sufficient that $H_0$ be so.
\end{corollary}
\begin{proof}
If $H_0$ is an isotrivial torus, then its type (determined by the relative dimension) is contant over each connected component of $S$. Therefore, it follows from \cref{scheme group multiplicative over Noe complete isotrivial and reduction} that $H$ is of isotrivial multiplicative over $S$; since $H_0$ is a torus, we conclude that so is $H$.
\end{proof}

\paragraph{Case of arbitrary base. Theorem of quasi-isotrivial}
We recall that a local ring is called Henselian if any finite algebra $B$ over $A$ is a product of finite local algebras over $A$. If $A$ is a Henselian local ring, $k$ its residue field, $S=\Spec(A)$, $S_0=\Spec(k)$, and $\xi$ is a geometric point of $S_0$, then the functor
\[X\mapsto X_0=X\times_SS_0\]
is an equivalence from the category of \'etale coverings over $S$ to the analogous category over $S_0$ (cf. \cite{EGA4-4} \S 18.5). In particular, we have $\pi_1(S_0,\xi)=\pi_1(S,\xi)$.\par
Suppose that $A$ is Noetherian and denote by $A'$ its completion, $S'=\Spec(A')$. Then $A'$ is a complete local Noetherian ring, hence Henselian (\cite{EGA4-4} 18.5.14), and the functor
\[X\mapsto X'=X\times_SS'\] 
from the category of \'etale coverings over $S$ to the analogous category over $S'$, fits into the following commutative diagram:
\[\begin{tikzcd}
\FEt_S\ar[rr]\ar[rd,swap,"\sim"]&&\FEt_{S'}\ar[ld,"\sim"]\\
&\FEt_{S_0}&
\end{tikzcd}\]
hence is also an equivalence of categories, and $\pi_1(S_0,\xi)=\pi_1(S',\xi)$. We also note that as $S$ is connected ($A$ being local), it follows from \cref{scheme group multiplicative cat equivalence} that (since $\pi_1(S,\xi)=\pi_1(S_0,\xi)$) the category of groups of isotrivial multiplicative type over $S$ is equivalent to the analogous category over $S_0$ (and also over $S'$ if $A$ is Noetherian).

\begin{lemma}\label{scheme group multiplicative over Hensel reduction}
Let $A$ be a local Henselian ring with residue field $k$, $S=\Spec(A)$, $S_0=\Spec(k)$.
\begin{enumerate}
    \item[(a)] The functor $H\mapsto H_0=H\times_SS_0$ is an equivalence from the category of groups of multiplicative type and finite over $S$ to the analogous category over $S_0$.
    \item[(b)] If $A$ is Noetherian, denote by $A'$ its completion and $S'=\Spec(A')$, the functor $H\mapsto H'=H\times_SS'$ is an equivalence from the category of groups of multiplicative type and finite over $S$ to the analogous category over $S'$.
\end{enumerate}
\end{lemma}
\begin{proof}
The second assertion is a concequence of (a), and we see that the considered functor in (a) is essentially surjective, because any group of multiplicative type $H_0$ finite over $S_0=\Spec(k)$ (hence of finite type over $S_0$) is isotrivial by \cref{scheme alg group multiplicative cat equivalence}, hence comes from a group of isotrivial multiplicative type over $S$ (by the remarks above).\par
To prove the fully faithfulness, i.e. that for any groups $G,H$ of multiplicative type and finite over $S$, the following map is bijective:
\[\Hom_{S\dash\Grp}(G,H)\to\Hom_{S_0\dash\Grp}(G_0,H_0)\]
or equivalently, denoting $F=\sHom_{S\dash\Grp}(G,H)$, that the natural map
\[\Hom_S(S,F)\to\Hom_{S_0}(S_0,F_0)\]
induced by base change $S_0\to S$, is bijective. For this, in view of the equivalence remarked above, we can utilize \cref{scheme group multiplicative finite sHom representable by finite etale}.
\end{proof}

\begin{lemma}\label{scheme group multiplicative finite sHom representable by finite etale}
Let $G,H$ be groups of multiplicative type and finite over $S$. Then $F=\sHom_{S\dash\Grp}(G,H)$ is representable by a finite \'etale over $S$.
\end{lemma}
\begin{proof}
Let $f:S'\to S$ be a faithfully flat and quasi-compact morphism such that $G'$ and $H'$ are diagonalizable. It then suffices to show that $F_{S'}$ is representable by a scheme $X'$ which is \'etale and finite (hence affine) over $S'$, because the induced descent data over $X'$ relative to $f$ (cf. \cref{scheme Zariski functor representable by descent lemma}) is then effective (\cite{SGA1} \Rmnum{8} 2.1), whence the existence of a scheme $X$ over $S$ such that $X\times_SS'=X'$, which represents $F$, and is \'etale and finite over $S$ (cf. \cite{SGA1} \Rmnum{8} 5.7 et \cite{EGA4-4} 17.7.3(\rmnum{2})).\par
We can hence suppose that $G=D_S(M)$ and $H=D_S(N)$, where $M,N$ are finite abelian groups (cf. \cref{scheme group diagonalizable schematic prop}~(c)). Then $K=\Hom_{\Grp}(N,M)$ is a finite abelian group, and by \cref{scheme group sHom of D(M) representable if ft}, we have an isomorphism
\[\sHom_{S\dash\Grp}(G,H)\cong K_S,\]
which completes the proof of \cref{scheme group multiplicative finite sHom representable by finite etale} (note that $K_S$ is a finite direct sum of copies of $S$, whence finite \'etale).
\end{proof}

\begin{lemma}\label{scheme ft over local Hensel local scheme decomposition}
Let $S$ be a local Henselian scheme, $s$ be its closed point, $X$ be a scheme locally of fintie type over $S$, $x$ be an isolated point in its fiber $X_s$.
\begin{enumerate}
    \item[(a)] $\mathscr{O}_{X,x}$ is finite over $\mathscr{O}_{S,s}$.
    \item[(b)] If $X$ is separated over $S$, then $X'=\Spec(\mathscr{O}_{X,x})$ is a clopen subscheme of $X$, i.e. we have a decomposition $X=X'\coprod X''$.
\end{enumerate}
\end{lemma}
\begin{proof}
By the local form of Zariski's main theorem (\cite{*} \Rmnum{4} Th.1), $x$ admits an affine open neighborhood $U=\Spec(B)$ of finite type and quasi-finite over $A=\mathscr{O}_{S,s}$, and there is an open immersion $U\hookrightarrow Y=\Spec(C)$, where $C$ is a finite $A$-algebra. As $A$ is Henselian, $Y$ is the direct sum of local schemes $Y_1,\dots,Y_n$ (with $Y_i=\Spec(C_i)$, $C_i$ being a finite local $A$-algebra), each is finite over $S$, and the points of $Y$ lying over $s$ are the closed points $y_1,\dots,y_n$. Therefore $x=y_i$ for certain index $i$, and $\mathscr{O}_{X,x}=\mathscr{O}_{U,x}=C_i$ is finite over $A$. Moreover, $X'=\Spec(C_i)$ is an open subscheme of $U$, hence of $X$.\par
Suppose that $X$ is separated over $S$. Then as the morphism $X'\to S$ is finite ($C_i$ being finite over $A$), so is the immersion $X'\to X$ (\cref{scheme morphism integral finite permanence prop}), and hence $X'$ is also closed in $X$.
\end{proof}

\begin{proposition}\label{scheme group multiplicative over local Hensel homomorphism reduction}
Let $A$ be a Noetherian local Henselian ring, $A'$ be its completion, $S=\Spec(A)$, $S'=\Spec(A')$, and $s$ be the closed point of $S$. Let $G$ be a group of multiplicative finite type over $S$ and $H$ be a $S$-group locally of finite type and separated over $S$, such that $H_s$ is of multiplicative type and $H$ is flat over $S$ at the points of $H_s$. Then the following natural map
\[\Hom_{S\dash\Grp}(G,H)\stackrel{\sim}{\to}\Hom_{S'\dash\Grp}(G',H')\]
is bijective, where $G',H'$ are induced by base change $S'\to S$.
\end{proposition}

\begin{theorem}\label{scheme group multiplicative fp fiber etale subgroup lifting}
Let $S$ be a scheme, $H$ be an affine $S$-group of finite presentation over $S$, and $s\in S$. Suppose that
\begin{enumerate}
    \item[($\alpha$)] $H$ is flat over $S$ at the points of $H_s$.
    \item[($\beta$)] $H_s$ is of multiplicative type. 
\end{enumerate}
Then there exsits an \'etale morphism $S'\to S$, a point $s'$ of $S'$ lying over $s$ such that $\kappa(s')=\kappa(s)$, and a clopen subgroup $G'$ of $H'=H\times_SS'$ of isotrivial multiplicative finite type, such that $G'_{s'}=H'_{s'}$.
\end{theorem}

\begin{corollary}\label{scheme group multiplicative ft is quasi-isotrivial}
Let $S$ be a scheme and $H$ be an $S$-group of multiplicative finite type. Then $H$ is quasi-isotrivial, i.e. is trivialized by a surjective \'etale morphism $S'\to S$.
\end{corollary}
\begin{proof}
For any point $s\in S$, by \cref{scheme group multiplicative fp fiber etale subgroup lifting}, there exists an \'etale morphism $S'\to S$, a point $s'\in S'$ lying over $s$ such that $\kappa(s)=\kappa(s')$, and a clopen subgroup $G'$ of $H'$ of isotrivial multiplicative finite type such that $G'_{s'}=H'_{s'}$. As $G'$ and $H'$ are of multiplicative finite type, by \cref{scheme group multiplicative ft mono epi locus prop} there exists an open neighborhood $U'$ of $s'$ such that $G'|_{U'}=H'|_{U'}$. As \'etale morphisms are open, by replacing $U'$ by a finite \'etale surjective morphism, we may assume that there exists an open neighborhood $U$ of $s$ such that $U'\to U$ is \'etale surjective, and $H\times_UU'$ is diagonalizable; we therefore conclude that $H$ is quasi-isotrivial.
\end{proof}

\begin{corollary}\label{scheme group multiplicative ft over local Hensel isotrivial prop}
Let $A$ be a local Henselian ring, $k$ be its residue field, and $\pi$ be the Galois group of an algebraic closure of $k$.
\begin{enumerate}
    \item[(a)] Any group of multiplicative finite type over $S=\Spec(A)$ is isotrivial.
    \item[(b)] The category of groups of multiplicative finite type over $S$ is equivalent to the analogous category over $S_0=\Spec(k)$, and hence is anti-equivalent to the category of finitely generated Galois $\pi$-modules.  
\end{enumerate}
\end{corollary}
\begin{proof}
With the notations of \cref{scheme group multiplicative fp fiber etale subgroup lifting}, we note that since $S'\to S$ is \'etale, hence locally quasi-finite, by \cref{scheme ft over local Hensel local scheme decomposition}~(a), $\mathscr{O}_{S',s'}$ is finite over $\mathscr{O}_{S,s}$. As we have $\kappa(s')=\kappa(s)$ and $\m_{s'}=\m_s$ ($S'\to S$ is unramified, cf. \cref{scheme morphism local fp unramified at point iff}), we conclude from Nakayama's lemma that $\mathscr{O}_{S',s'}\cong\mathscr{O}_{S,s}$. As the image of $\Spec(\mathscr{O}_{S',s'})$ in $S'$ is contained in $U'$ and $S\cong\Spec(\mathscr{O}_{S,s})$, we conclude that there is a section $\sigma:S\to S'$ of $S'$ over $S$ such that the base change of $G'$ along $\sigma$ is isomorphic to $H$, whence the assertion in (a). The second assertion now follows from \cref{scheme alg group multiplicative cat equivalence}.
\end{proof}

\begin{corollary}\label{scheme group multiplicative fp fiber extend to nbhd}
Under the conditions of \cref{scheme group multiplicative fp fiber etale subgroup lifting}, suppose that one of the following conditions is satisfied:
\begin{enumerate}
    \item[(\rmnum{1})] For any generalization $t$ of $s$, $H_t$ is of multiplicative type and of the same type as $H_s$.
    \item[(\rmnum{2})] $H$ is abelian and ${_nH}$ ($n>0$) are finite over $S$ in a neighborhood of $s$.
    \item[(\rmnum{3})] For any generalization $t$ of $s$, the fiber $H_t$ is connected.
\end{enumerate}
Then there exists an open neighborhood $U$ of $s$ such that $H|_U$ is of multiplicative type (and hence the above conditions are equivalent).
\end{corollary}
\begin{proof}
Since \'etale morphisms are open, we are reduced to prove that there exists (with the notations of \cref{scheme group multiplicative fp fiber etale subgroup lifting}) an open neighborhood $U'$ of $s'$ such that $G'|_{U'}=H'|_{U'}$. Let $S''=\Spec(\mathscr{O}_{S',s'})$; as $G'$ and $H'$ are of finite presentation over $S'$, it suffices to prove, according to (\cite{EGA4-3} 8.8.2), that $G''=H''$. We can then assume that $S=S''$, and hence reduce to the case where $S$ is local and $s$ is its closed point. In this case, any point of $S$ is a generalization of $s$, and any neighborhood of $s$ is equal to $S$ (cf. \cref{scheme local canonical morphism prop}), so it suffices to apply \cref{scheme group ft over local fiber multiplicative subgroup equal iff} below.
\end{proof}

\begin{lemma}\label{scheme group ft over local fiber multiplicative subgroup equal iff}
Let $S$ be a local scheme, $s$ be its closed point, $H$ be an $S$-group of finite type, and $G$ be an clopen subgroup of $H$ of multiplicative type such that $G_s=H_s$. Suppose that one of the following conditions is satisfied:
\begin{enumerate}
    \item[(\rmnum{1})] The fibers of $H$ are of multiplicative type and of the same type as $H_s$.
    \item[(\rmnum{2})] $H$ is abelian and the $H_n$ ($n>0$) are finite over $S$.
    \item[(\rmnum{3})] The fibers of $H$ are connected.
\end{enumerate}
Then $G=H$ (and hence the above conditions are equivalent).
\end{lemma}
\begin{proof}
The proof is the same as \cref{scheme group multiplicative over Noe complete isotrivial and reduction}.
\end{proof}

\begin{corollary}\label{scheme group flat fp multiplicative fiber is multiplicative iff}
Let $S$ be a scheme, $H$ be an affine $S$-group, flat and of finite presentation over $S$, with multiplicative fibers. For $H$ to be multiplicative, it is necessary and sufficient that the following equivalent conditions are satisfied:
\begin{enumerate}
    \item[(\rmnum{1})] The type of $H_s$ is a locally constant function on $S$.
    \item[(\rmnum{2})] $H$ is abelian, and the $H_n$ ($n>0$) are finite over $S$.
    \item[(\rmnum{3})] The fibers of $H$ are connected.
\end{enumerate}
\end{corollary}

\begin{corollary}\label{scheme group flat fp affine fiber torus extend nbhd}
Let $S$ be a scheme, $H$ be a flat group scheme of finite presentation over $S$. Suppose that $H$ is affine over $S$ and has connected fibers. If $s\in S$ is such that $H_s$ is a torus, there exists an open neighborhood $U$ of $s$ such that $H|_U$ is a torus. In particular, if the fibers of $H$ are tori, then $H$ is a torus.
\end{corollary}

\subsection{Twisted constant groups}\label{scheme enlarged fundamental group subsection}
Let $S$ be a scheme and $R$ be a group scheme over $S$. We say that $R$ is a \textbf{twisted constant group} over $S$ if it is locally, in the sense of the fpqc topology, isomorphic to a constant group, i.e. of the form $M_S$, where $M$ is an abelian group.\par
We say that a twisted constant group $R$ over $S$ is quasi-isotrivial (resp. isotrivial, resp. locally isotrivial, resp. locally trivial, resp. trivial) if in the above definition we can replace the fpqc topology by the \'etale topology (resp. global finite \'etale topology, resp. finite \'etale topology, resp. Zariski topology, resp. chaotic topology). For example, to say that $R$ is quasi-isotrivial (resp. isotrivial) signifies that there exists a surjective \'etale (resp. and finite) morphism $S'\to S$ such that $R'=R\times_SS'$ is a constant group over $S$.\par
We define similarly the \textbf{type} of a twisted constant group $R$ over $S$ at a point $s\in S$; this is an isomorphism class of ordinary groups, which is locally constant over $S$, hence constant if $S$ is connected. We say that $R$ is of type $M$ if the fibers of $R$ are of type $M$. We note that $R$ is quasi-compact over $S$ only if it is \textit{finite} over $S$, i.e. if its fiber at any $s\in S$ is a finite group.\par
The most interesting case for us is the one where $R$ is abelian. We then say that $R$ is "\textit{finitely generated}" if its type at any point $s\in S$ is given by a finitely generated $\Z$-module (this should not be confused with the schematic notion "$R$ of finite type over $S$").

\begin{remark}
We can also consider $S$-schemes $X$ which are locally isomorphic (for the fpqc topology) to constant schemes. We then say that $X$ is a twisted constant bundle over $S$, and the terminologies introduced above can be extended to these schemes. Of course, we shall pay attention that when $X$ is endowed with a structure of $S$-group, the meaning of the expressions "twisted constant", "isotrivial" etc. changes accordingly if we take into account the group structure over $S$.
\end{remark}

\begin{proposition}\label{scheme twisted constant abelian duality prop}
Let $R$ be a twisted constant abelian group over $S$.
\begin{enumerate}
    \item[(a)] The functor $H=D_S(R)=\sHom_{S\dash\Grp}(G,\G_{m,S})$ (cf. \autoref{scheme group duality}) is representable by a group of multiplicative type over $S$.
    \item[(b)] For any $s\in S$, the type of $R$ at $s$ is equal to that of $H$ at $s$.
    \item[(c)] For $R$ to be quasi-isotrivial (resp. global finite \'etale topology, resp. finite \'etale topology, resp. Zariski topology, resp. chaotic topology)
\end{enumerate}
\end{proposition}
\begin{proof}
As the covering familes for the fpqc topology are effective descent for the fibre category of affine groups schemes over $S$ (cf. \cite{SGA1} \Rmnum{8} 2.1), we see that $H$ is representable (and is of multiplicative type over $S$), since this is true if $R$ is constant (cf. \cref{scheme Zariski functor representable by descent lemma}). The fact that $H$ is of multiplicative type is clear by definition, and its type is easily seen to be the same as $R$ at $s\in S$. Finally, since $H_{S'}=D_{S'}(R_{S'})$, the last assertion is reduced to the trivial case, i.e. to verify that $R$ is trivial if and only if $H$ is; this follows from the biduality theorem (\cref{scheme group D(M_S) is reflexive}).
\end{proof}

To specify the correspondence between twisted constant groups and groups of multiplicative type, it is necessary to start from a group of multiplicative type $H$, and study $R=D_S(H)$. If the latter is representable, it is obviously a twisted consant group, and we have $H\cong D_S(R)$. In other words, the functor $R\mapsto D_S(R)$ is an anti-equivalence between the category of twisted constant groups over $S$ and that of groups of multiplicative type $H$ over $S$ such that $D_S(H)$ is representable. In general, it is hard to see that whether $D_S(H)$ is representable for a group of multiplicative type $H$, but we shall see that this is the case if $H$ is quasi-isotrivial, and in particular if $H$ is of finite type (cf. \cref{scheme group multiplicative ft is quasi-isotrivial}).

\begin{lemma}\label{scheme sp fppf effective descent}
Let $S'\to S$ be a faithfully flat and locally presented morphism, and $X'$ be a separated $S'$-scheme, locally of finite presentation and locally quasi-finite over $S$. Then any descent data over $X'$ relative to $S'\to S$ is effective.
\end{lemma}
\begin{proof}
If $X'\to S'$ is quasi-compact, hence of finite presentation and quasi-finite, then it is a quasi-affine morphism (cf. \cite{EGA4-3} 8.11.2), and the effectivity follows from (\cite{SGA1} \Rmnum{8} 7.9). In the general case, we can reduce to the case where $S$ and $S'$ are affine. Let $(U_i')$ be an affine open covering of $X'$ and $V_i'$ be the saturation of $U_i'$ under the equivalence relation of $X'$ defined by the descent data, i.e. $V_i'=q_2(q_1^{-1}(U_i'))$, where $q_1,q_2$ are the projections of $X_1''=X'\times_{S',\pr_2}S''$ over $X'$ ($q_1=\pr_1$, and $q_2$ is deduced from the first projection of $X_2''=X'\times_{S',\pr_2}S''$, thanks to the descent data $X_1''\cong X_2''$). As $S'\to S$ is faithfully flat, locally of finite presentation, and quasi-compact (recall that $S'$ and $S$ are affine), so is $p_1:S''=S'\times_SS'\to S'$, hence also $q_1$ and $q_2$, which are therefore quasi-compact open morphisms (\cite{EGA4-2} 2.4.6). Therefore, $V_i'$ is an open subset of $X'$. From what we have already seen, the descent data induced on the $V_i'$ are effective, from which it follows that the same is true for $X'$ (\cite{SGA1} \Rmnum{8} 7.2).
\end{proof}

\begin{corollary}\label{scheme twisted constant group fppf effective descent}
A faithfully flat and locally finite presented morphism $S'\to S$ is effective descent for the fibre category of twisted constant groups.
\end{corollary}
\begin{proof}
In fact, this amounts to the effectiveness of a descent datum under the conditions of \cref{scheme sp fppf effective descent}, if $X'$ is a constant $S'$-scheme.
\end{proof}

\begin{theorem}\label{scheme group multiplicative sHom representable if quasi-isotrivial}
Let $S$ be a scheme, $G$, $H$ be $S$-groups of quasi-isotrivial multiplicative type, with $G$ of finite type over $S$. Then $\sHom_{S\dash\Grp}(H,G)$ is representable by a quasi-isotrivial twisted constant group over $S$. For any $s\in S$, it the type of $G_s$ (resp. $H_s$) is $M$ (resp. $N$), then that of $\sHom_{S\dash\Grp}(H,G)_s$ is $\Hom_{\Grp}(M,N)$.
\end{theorem}
\begin{proof}
We proceed as in \cref{scheme group multiplicative finite sHom representable by finite etale}, utilizing the fact that the assertion has been established if $G$ and $H$ are trivial. The effectivity of the descent data is justified by \cref{scheme twisted constant group fppf effective descent} (in the case of an \'etale surjective morphism $S'\to S$\footnote{From this, we see that the assertions of \cref{scheme group multiplicative sHom representable if quasi-isotrivial} are already valid if $G$ and $H$ are trivial for the fppf topology.}).
\end{proof}

\begin{corollary}\label{scheme group multiplicative D_S representable if quasi-isotrivial}
Let $S$ be a scheme and $H$ be an $S$-group of multiplicative type.
\begin{enumerate}
    \item[(a)] The $S$-group functor $D_S(H)$ is representable by a quasi-isotrivial twisted constant group over $S$.
    \item[(b)] The functors $H\mapsto D_S(H)$ and $R\mapsto D_S(R)$ give an anti-equivalence between the category of quasi-isotrivial twisted constant groups over $S$ and that of groups of quasi-isotrivial multiplicative type over $S$.
\end{enumerate}
\end{corollary}
\begin{proof}
This follows directly from \cref{scheme group multiplicative sHom representable if quasi-isotrivial} by taking $G=\G_{m,S}$, which is of finite type over $S$.
\end{proof}

\begin{corollary}\label{scheme group multiplicative ft sHom representable}
Let $S$ be a scheme and $G,H$ be $S$-groups of multiplicative finite type. Then $\sHom_{S\dash\Grp}(H,G)$ is representable by a finitely generated quasi-isotrivial twisted constant group over $S$.
\end{corollary}
\begin{proof}
This follows from \cref{scheme group multiplicative sHom representable if quasi-isotrivial}, as any group of multiplicative finite type over $S$ is quasi-isotrivial (\cref{scheme group multiplicative ft is quasi-isotrivial}).
\end{proof}

We also note that in \cref{scheme twisted constant abelian duality prop}, $R$ is finitely generated if and only if $H=D_S(R)$ is of finite type over $S$ (cf. \cref{scheme group multiplicative schematic prop}~(b)). In view of \cref{scheme group multiplicative ft is quasi-isotrivial}, $H$ is then quasi-isotrivial, so $R$ is quasi-isotrivial. We thus obtain the following corollaries:

\begin{corollary}\label{scheme group multiplicative ft cat equivalent to ft twisted constant}
The functors in \cref{scheme group multiplicative D_S representable if quasi-isotrivial} induce an anti-equivalence between the category of $S$-groups of multiplicative finite type and that of finitely generated twisted constant groups over $S$. Moreover, any such group $R$ is quasi-isotrivial.
\end{corollary}

\begin{corollary}\label{scheme group multiplicative ft sIso representable}
Let $H,G$ be $S$-groups of multiplicative finite type. Then $\sIso_{S\dash\Grp}(H,G)$ is representable by a clopen subscheme of $\sHom_{S\dash\Grp}(H,G)$, and it is a twisted constant $S$-scheme. In particular, $\sAut_{S\dash\Grp}(H,G)$ is representable by a twisted constant $S$-group (not abelian in general).
\end{corollary}
\begin{proof}
This can be proves as in \cref{scheme group multiplicative sHom representable if quasi-isotrivial}, using the fact that the assertions are valid if $G$ and $H$ are trivial.
\end{proof}

\begin{proposition}\label{scheme twisted constant abelian isotrivial and connected component prop}
Let $S$ be a scheme, $R$ be a twisted constant abelian group over $S$, $H=D_S(R)$ the group of multiplicative type it defines. Consider the following conditions:
\begin{enumerate}
    \item[(\rmnum{1})] $H$ is isotrivial (i.e. $R$ is isotrivial).
    \item[(\rmnum{2})] $R$ is the union of clopen subschemes $R_i$, which are quasi-compact over $S$ (and hence nesessarily finite over $S$).
    \item[(\rmnum{3})] The connected components of $R$ are finite over $S$.
\end{enumerate}
Then we have (\rmnum{1})$\Rightarrow$(\rmnum{2})$\Rightarrow$(\rmnum{3}), (\rmnum{2})$\Leftrightarrow$(\rmnum{3}) if $S$ is locally Noetherian, and (\rmnum{1})$\Leftrightarrow$(\rmnum{2}) if $R$ is finitely generated (i.e. if $H$ is of finite type over $S$) and $S$ is quasi-compact or its connected components are open.
\end{proposition}
\begin{proof}
The assertion in the parenthesis of (\rmnum{2}) follows from \cref{scheme twisted constant group qc closed is finite} below. By decomposing $S$ into a sum of subschemes $S_i$ over which $H$ is of constant type, we are reduced to the case where $H$, hence $R$, is of constant type $M$. It is clear that (\rmnum{2})$\Rightarrow$(\rmnum{3}), since the connected components of $R$ are closed; they are clopen in $R$ if $R$ is locally Noetherian (cf. \cref{topo space local Noe is local connected}, since $R$ is of finite type over $S$, hence locally Noetherian), whence (\rmnum{3})$\Rightarrow$(\rmnum{2}) in this case.\par
To prove that (\rmnum{1})$\Rightarrow$(\rmnum{2}), let $S'\to S$ be a finite surjective \'etale morphism which trivializes $H$, hence $R$, so that $R'\cong M_{S'}=\coprod_{m\in M}R_m'$, where $R_m'$ are disjoint open subsets of $R'$, $S'$-isomorphic to $S'$. Let $g:R'\to R$ be the projection morphism, which is finite surjective \'etale, hence an open and closed morphism. Then the subsets $R_m=g(R_m')$ are clopen in $R$, and evidently quasi-compact over $S$ since the $R'_m$ are ($g$ being surjective, \cref{scheme morphism qc cancelled by surjective}).\par
Finally, suppose that $H$ is of finite type over $S$, and we show that (\rmnum{2})$\Rightarrow$(\rmnum{1}). The case where the connected components of $S$ are open is immediately reduced to the case where $S$ is connected, so we can assume that $S$ is quasi-compact or connected. Since $M$ is finitely generated, we can write $M=\Z^r\times N$, where $r>0$ is an integer and $N$ a finite abelian group. Let $G=D_S(M)$ and consider the schemes (cf. \cref{scheme group multiplicative ft sIso representable})
\[P=\sIso_{S\dash\Grp}(H,G)\sub\sHom_{S\dash\Grp}(H,G)=Q.\]
We have isomorphisms
\[Q\cong\sHom_{S\dash\Grp}(M_S,R)\cong\sHom_{S\dash\Grp}(\Z_S^r,R)\times\sHom_{S\dash\Grp}(N_S,R)\cong R^r\times E,\]
where $E=\sHom_{S\dash\Grp}(N_S,R)$ is finite over $S$ (because it is a twisted constant group of type $\End_{\Grp}(N)$). The hypothesis on $R$ then implies that $Q$ is the union of clopen subschemes $Q_i$ which are finite over $S$, so $P$ is the union of clopen subschemes $P_i=P\cap Q_i$, which are finite over $S$. As they are \'etale over $S$, their image in $S$ are clopen subsets $S_i$ of $S$, and cover $S$. If $S$ is connected or quasi-compact, there then exists finitely many indices $i$ such that the $S_i$ cover $S$; let $S'$ be the union of the corresponding $P_i$. Then $S'\to S$ is finite surjective \'etale, and putting $P'=P\times_SS'=\sIso_{S'\dash\Grp}(H',G')$, we see that $P'$ has a section over $S'$, i.e. there exists an isomorphism of $S'$-groups
\[H'=H\times_SS'\stackrel{\sim}{\to} G'=G\times_SS'=D_{S'}(M),\]
which proves that $S'$ trivializes $H$. 
\end{proof}

\begin{lemma}\label{scheme twisted constant group qc closed is finite}
Let $S$ be a scheme and $R$ be a twisted constant scheme over $S$. Then any closed subscheme $Z$ of $R$ which is quasi-compact over $S$ is finite over $S$.
\end{lemma}
\begin{proof}
In fact, by fpqc descent, we may assume that $R$ is constant, hence of the form $I_S$, where $I$ is a set. Then $R$ is a directed union of $J_S$, where $J$ runs through finite subsets of $I$. We can also suppose that $S$ is affine, so $Z$ is quasi-compact, hence contained in one of the $J_S$. As $J_S$ is finite over $S$, so is $Z$.
\end{proof}

Let $S$ be a locally Noetherian scheme, and $\widetilde{S}$ be the normalization of $S_\red$. Recall that $S$ is called \textit{geometrically unibranch} (cf. \cref{EGA4-1} $0_{\Rmnum{4}}$, \S 23.2 et \cite{EGA4-2} \S 6.15) if the morphism $\widetilde{S}\to S$ is radiciel (and hence a universal homeomorphism). In particular, the connected components of $S$ are irreducible (cf. \cref{scheme irreducible component open and nilradical}~(c)).\par
Suppose then that $S$ is connected, hence irreducible, let $\eta$ be its generic point, and $f:P\to S$ be a flat and locally quasi-finite morphism. Let $P_i$ be the irreducible components of $P$ and $\xi_i$ be the generic point of $P_i$. As $P$ is flat over $S$, each $\xi_i$ is lying over $\eta$ (cf. \cite{EGA4-2} 2.3.4), and hence $(P_i)_\eta=P_i\cap P_\eta$ is the closure of $\xi_i$ in $P_\eta$. Since the fiber $P_\eta$ is discrete by hypothesis, we then have $(P_i)_\eta=\{\xi_i\}$. This remark applies in particular if $f$ is \'etale; in this case, $P$ is also locally Noetherian and geometrically unibranch (cf. \cite{EGA4-4} 17.5.7), hence its connected components are irreducible, and open.

\begin{corollary}\label{scheme twisted constant group over Noe geo.uni connected component finite}
Let $S$ be a locally Noetherian and geometrically unibranch scheme, $P$ be a quasi-isotrivial twisted constant scheme over $S$. Then the connected components of $P$ are finite over $S$.
\end{corollary}
\begin{proof}
We can evidently suppose that $S$ is connected, hence irreducible; let $\eta$ be its generic point. By the remark above, each connected component $P_i$ of $P$ is open and closed, and meets the fiber at a single point. Therefore \cref{scheme twisted constant over Noe connected clopen finite if fiber} below applies and shows that each $P_i$ is finite over $S$.
\end{proof}

\begin{lemma}\label{scheme twisted constant over Noe connected clopen finite if fiber}
Let $S$ be a locally Noetherian and connected scheme, $P$ be a quasi-isotrivial twisted constant $S$-scheme, and $Z$ be a clopen subset of $P$ such that there exists $s\in S$ such that $Z_s$ is finite. Then $Z$ is finite over $S$.
\end{lemma}
\begin{proof}
Let us first consider the non-connected case of $S$. Let $U$ be the set of $s\in S$ such that $Z_s$ is finite, we shall prove that $U$ is clopen and that $Z|_U$ is finite over $U$. This assertion is essentially equivalent to \cref{scheme twisted constant over Noe connected clopen finite if fiber}, but has the advantage that it is local for the \'etale topology, so we can reduce to the case where $P$ is constant, i.e. of the form $I_S$, where $I$ is a set\footnote{The locally Noetherian hypothesis is preserved under \'etale base change; this is where we use the quasi-isotriviality of $P$ over $S$.}.\par
We may then assume that $S$ is connected, since the connected components of $S$ are open ($S$ being locally Noetherian). But then we must have $Z=J_S$ for a subset $J$ of $I$, and hence $U=\emp$ or $U=S$, according to 
\end{proof}

\begin{corollary}\label{scheme group multiplicative ft over Noe geo.uni is isotrivial}
Let $S$ be a locally Noetherian and geometrically unibranch scheme. Then any $S$-group $H$ of multiplicative finite type is isotrivial.
\end{corollary}
\begin{proof}
We can suppose that $S$ is connected, hence $H$ is of constant type $M$. We can then apply \cref{scheme twisted constant group over Noe geo.uni connected component finite} to $P=R=D_S(H)$, and then utilize \cref{scheme twisted constant abelian isotrivial and connected component prop}.
\end{proof}


\subsection{Principal Galois bundles and enlarged fundamental group}
Let $S$ be a scheme, we consider the question of determine principal homogeneous bundles $P$ over $S$ (for the fpqc topology) with structure group of the form $G_S$, the constant group over $S$ defined by an ordinary group $G$ (not necessarily finite), which are also called \textbf{principal Galois bundles over $S$ with group $G$}. We note that as $G_S$ is \'etale and the structure morphism $G_S\to G$ is surjective, any such $P$ is \'etale over $S$, and the structural morphism $P\to S$ is surjective (cf. \cite{SGA1} \Rmnum{8} 3.1 et \cite{EGA4-4} 17.7.3), hence $P$ is covering for the \'etale topology, and $P$ is also locally trivial for the \'etale topology (\cref{site formally principal homogeneous under M-group iff}).\par
Since we often consider locally Noetherian schemes, we may suppose that $S$ is a sum of connected schemes, i.e. its connected components are open, and hence assume that $S$ is connected. Let $\xi:\Spec(\Omega)\to S$ be a geometric point of $S$, where $\Omega$ is a separably closed field. Then for any principal Galois bundle $P$ over $S$ with group $G$, $P_\xi$ is a principal Galois bundle over $\Omega$ with group $G$, whence is trivial. We therefore specify the initial problem by proposing to determine the category of the principal Galois bundles over $S$ \textbf{pointed over $\xi$}, i.e. endowed with an $S$-morphism $\xi\to P$, or equivalently a trivialization of $P_\xi$. For fixed $G$, the set of isomorphic classes of such bundles, up to isomorphisms fixing the base point $\xi$, is denoted by $\pi^1(S,\xi;G)$. Then the set $\pi^1(S;G)$ of isomorphism classes of principal Galois bundles over $S$ with group $G$ (without a specified base point) is isomorphic to the set of orbits of $G$ in $\pi^1(S,\xi;G)$:
\[\pi^1(S;G)=\pi^1(S,\xi;G)/G,\]
where $G$ acts naturally on $P$ by definition. In fact, since any principal Galois bundle over $S$ is trivialized over $\xi$, the map $\pi^1(S,\xi;G)\to\pi^1(S;G)$ is surjective. If two $\xi$-pointed principal Galois bundle $\sigma:\xi\to P$ and $\sigma':\xi\to P'$ over $S$ are identified in $\pi^1(S;G)$, then there exists a $G$-equivariant isomorphism $\phi:P\to P'$, which then induces a map
\[\phi^*:P(\xi)\to P'(\xi).\]
If we denote by $g\in G$ the (unique) element such that $\sigma'\cdot g=\phi^*(\sigma)=\sigma\circ\phi$, then $\phi$ induces an isomorphism $P\cong P'\cdot g$ of $\xi$-pointed principal Galois bundles, whence our assertion.\par
For any morphism $S'\to S$ which is universally effective descent for the fibre category of twisted constant schemes over a variable base (for example fppf morphisms, cf. \cref{scheme twisted constant group fppf effective descent}), we propose the determination of the subsets of the preceding sets, denoted by $\pi^1(S'/S,\xi;G)$ and $\pi^1(S'/S;G)$, formed by principal Galois bundles over $S$ which are trivialized by $S'$. We determine in fact the fibre category of principal Galois bundles $P$ over $S$ which are trivialized by $S'$. Of course we have
\[\pi^1(S,\xi;G)=\rlim_{S'}\pi^1(S'/S,\xi;G)\]
where $S'$ runs through a cofinal system of the set of covering morphisms $S'\to S$ for the \'etale topology (for example, if $S$ is quasi-compact, we can take the set of $S'$ over $S$ which are quasi-compact and have \'etale surjective structural morphism). Similarly, the fibre category of principal Galois bundles over $S$ is the inductive limit of its subcategories defined by the $S'$ (formed by bundles trivialized over $S'$).\par
Thanks to the assumption made on $S'\to S$, the fiber category of principal Galois bundles over $S$ trivialized by $S'$ is equivalent to the fibre category of principal Galois bundles trivial over $S'$ (hence of the form $G_{S'}$, where $G$ acts by right translation), endowed with a descent data relative to $S'\to S$. The datum of a base point over a principal Galois bundle $P$ over $S$ trivialized by $S'$ then translates, in terms of the trivial bundle $P'$ over $S'$ and its descent data, to the data of a trivialization of $P'\times_{S'}S'_\xi$ compatible with the induced descent data under base change, relative to $S'_\xi\to\xi$ (we put $S'_\xi=S'\times_S\xi$), i.e. a section $\sigma$ of $P'_\xi$ over $S'_\xi$ compatible with the descent data. For a general morphism $S'\to S$ (not necessarily universally effective descent for the fibre category of twisted constant groups), we can then define $\pi^1(S'/S;G)$ and $\pi^1(S'/S,\xi;G)$ to be the set of isomorphism classes of trivial bundle $P'$ over $S'$ endowed with a descent data (and a trivialization of $P'_\xi$ compatible with the induced descent data).\par
We can now state the most important result of this subsection, which gives the description of the functor $G\mapsto\pi^1(S,\xi;G)$ in terms of the simplicial set $S_\bullet$ induced by $S'\to S$, and identify the fiber category of principal Galois bundles over $S$:

\begin{proposition}\label{scheme relative fundamental group representable prop}
Suppose that the connected components of $S'$ and $S''$ are open.
\begin{enumerate}
    \item[(a)] The functor $G\mapsto\pi^1(S'/S,\xi;G)$ from the category of groups to the category of sets, is representable by a group, denoted by $\pi_1(S'/S,\xi)$ and called the fundamental group of $S$ at $\xi$ relative to $S'\to S$. We then have a functorial isomorphism
    \[\pi^1(S'/S,\xi;G)\stackrel{\sim}{\to}\Hom_{\Grp}(\pi_1(S'/S,\xi),G).\]
    \item[(b)] The group $\pi_1(S'/S,\xi;G)$ admits a set of generators bijective to $\pi_0(S'')$, and is described in terms of these generators by relations bijective to elements of $\pi_0(S''')$. In particular, $\pi_1(S'/S,\xi)$ is finitely generated (resp. of finite presentation) if $\pi_0(S'')$ (resp. as well as $\pi_0(S''')$) is finite.
    \item[(c)] The fibre category of principal Galois bundles over $S$ trivialized by $S'$, pointed over $\xi$, is equivalent to the category of ordinary groups $G$, endowed with a homomorphism $\pi_1(S'/S,\xi)\to G$.
\end{enumerate}
\end{proposition}

If $S$ is connected and Noetherian, then any \'etale scheme $S'$ over $S$ is locally Noetherian, hence its connected components are open. We then conclude from the arguments above that the functor $G\mapsto\pi^1(S,\xi;G)$, from the category of groups to that of sets, is strictly pro-representable, i.e. there exists a projective system 
\[\Pi=\Pi_1(S;\xi)=(\pi_i)_{i\in I}\]
of ordinary groups, with a filtered index set $I$, that is strict (i.e. the transition morphisms $\pi_j\to\pi_i$ are surjective), and an isomorphism of functors on $G$:
\[\pi^1(S,\xi;G)\stackrel{\sim}{\to}\rlim_i\Hom_{\Grp}(\pi_i,G).\]
The right member is also denoted by $\Hom_{\pro\dash\Grp}(\Pi,G)$.\par
In the case where the projective limit $\pi=\llim\pi_i$ is "large enough", i.e. if the canonical homomorphisms $\pi\to\pi_i$ are surjective for $i$ sufficiently large, it is necessary to endow $\pi$ with the projective limit topology of the discrete topologies of the $\pi_i$, and the isomorphism can also be written:
\[\pi^1(S,\xi;G)\stackrel{\sim}{\to} \Hom_{\mathrm{Cont}\Grp}(\pi,G),\]
where the right member denotes the set of continous homomorphism of topological groups, where $G$ is endowed with the discrete topology.\par
The hypothesis that we have just formulated on the projective system $\Pi$ is verified, as it is well known, when the $\pi_i$ are finite groups (cf. \cite{BEns}, \Rmnum{3} \S 7.4, Th.1). This last condition also means that any principal Galois bundle over $S$ is isotrivial, i.e. is trivialized by a finite surjective \'etale morphism, which is the case when $S$ is geometrically unibranched (for example normal) as it follows immediately from \cref{scheme group multiplicative ft over Noe geo.uni is isotrivial}. In the case where the $\pi_i$ are finite, the group $\pi$ also coincides with the \'etale fundamental group $\pi_1(S,\xi)$ of $S$ at $\xi$.\par
Also, in the prefered case ($\pi\to\pi_i$ being surjectives), we could also call $\pi$ the extended group fundamental of $S$ at $\xi$. In other cases, $\pi$ itself does not present much interest, and the role of the usual fundamental group is played by the projective system itself, which we will call the \textbf{enlarged fundamental pro-group of $S$ at $\xi$}.\par

Let us quickly indicate the calculation of $\pi_1(S'/S,\xi)$. Let $S_i$ be the $(i+1)$-th fiber power of $S'$ over $S$ (i.e. $S_0=S'$, $S_1=S''$, etc.). We have obvious simplicial operations over the $S_i$, which make $(S_i)_{i\in\N}$ a simplicial object of $\Sch_{/S}$. Transforming this simplicial object by the functor of connected component
\[\pi_0:\Sch_{/S}\to\Set,\]
we then obtain a simplicial set $K_\bullet=(K_i)_{i\in\N}$, with $K_i=\pi_0(S_i)$. Similarly, the $(S_i)_\xi$ (the $(i+1)$-th fiber product of $S'_\xi$ over $\xi$) form a simplicial object of $\Sch_{/\xi}$, hence of $\Sch_{/S}$. It is endowed with a natural homomorphism of simplicial objects to $(S_i)_{i\in\N}$ (induced by the morphism $\xi\to S$), so we have a simplicial set $k_\bullet$ (with $k_i=\pi_0((S_i)_\xi)$) and a canonical homomorphism $k_\bullet\to K_\bullet$. We can form a new simplicial set by taking the cone of this morphism (see \cite{SGA3-2} \Rmnum{10} 9.5.1):
\[\widetilde{K}_\bullet=\Cone(k_\bullet\to K_\bullet).\]
In this way, we obtain a pointed simplicial set $\widetilde{K}_\bullet$ (i.e. a simplicial set endowed with a homomorphism $\tilde{\xi}:e_\bullet\to\widetilde{K}_\bullet$, where $e_\bullet$ is the final simplicial set). We can then construct the well-known combinatorial invariants $\pi_0(\widetilde{K}_0,\tilde{\xi})$ and $\pi_1(\widetilde{K}_\bullet,\tilde{\xi})$, the construction of which only involves the components of degree $\leq 1$ (resp. of degree $\leq 2$) of $\widetilde{K}_\bullet$ (these invariants are defined without any restriction on $S$ or $S'$). We then verifies without difficulty that, if the connected components of $S_0=S'$ and $S_1=S''$ are open (in fact, it is sufficient that $\widetilde{K}_\bullet$ is connected), then $\pi_1(\widetilde{K}_\bullet,\tilde{\xi})$ represents the functor i.e. we have:
\[\pi_1(S'/S,\xi)\cong\pi_1(\widetilde{K}_\bullet,\tilde{\xi}).\]
We also remark that if the morphism $S'\to S$ is universally submersive (cf. \cite{SGA1} \Rmnum{4} 2.1), and the connected components of $S'$ are open, then the simplicial set $K_\bullet$, and hence $\widetilde{K}_\bullet$, is connected.

\begin{example}\label{scheme enlarged fundamental group example}

\end{example}

Let $S$ be a scheme, which we may assume to be locally Noetherian, so that certain schemes over $S$ (namely schemes of finite type over $S$) are locally Noetherian, so their connected components are open. Using the enlarged fundamental group, we can now give a classification of twisted consant groups over $S$. 

\begin{proposition}\label{scheme twisted constant group fppf trivial is quasi-isotrivial}
Any twisted constant scheme $X$ over $S$ which is locally trivial for the fppf topology is quasi-isotrivial.
\end{proposition}
\begin{proof}
We can suppose that $S$ is connected, hence $X$ is of constant type $I$, where $I$ is a set. Let $S'\to S$ be a faithfully flat morphism locally of finite presentation that trivializes $X$. Then $X'=X\times_SS'$ is isomorphic to $I_{S'}$, so $I_{S'}$ is endowed with a descent data relative to $S'\to S$, i.e. we have an isomorphism $I_{S''}\stackrel{\sim}{\to}I_{S''}$ satisfying the cocycle condition. Now $S''=S'\times_SS'$ is locally Noetherian, so its connected components are open, and the automorphisms of $I_{S''}$ corresponding to sections of $G_{S''}$, where $G=\Aut(I)$ is the permutation group of $I$. In this way, we obtain a descent data over $G_{S'}$ (considered as the trivial Galois bundle) relative to $S'\to S$. In view of \cref{scheme sp fppf effective descent}, this descent data is effective, so it corresponds to a principal Galois bundle $P$ over $S$, with group $G$. By construction, it represents the functor $\sIso_S(I_S,X)$ in the category of schemes over $S$ which are locally Noetherian, and we easily see that $P\times_SX\cong I_S$, so the \'etale surjective base change $P\to S$ trivializes $X$, hence $X$ is quasi-isotrivial.
\end{proof}

The proof given above shows at the same time that the classification of twisted constant scheme $X$ over $S$, quasi-isotrivial and of type $I$, is equivalent to that of principal Galois bundles over $S$, with group $G=\Aut(I)$. In fact, in this way, we obtain an equivalence of categories.\par
Using the enlarged fundamental group $\Pi$, we can make use the above correspondence to obtain an action of $\Pi$. For this, suppose that $S$ is connected, and endowed with a geometric point $\xi$. Then the enlarged fundamental group $\Pi=\Pi_1(S,\xi)$ is defined. On the other hand, for any quasi-isotrivial twisted constant scheme $X$ over $S$, let $I=X(\xi)$ be its setwise fiber at $\xi$, so $X$ is of type $I$, and is associated as we have just seen with a principal Galois bundle $P=\Iso_S(I_S,X)$ over $S$, with group $G=\Aut(I)$. According to the definition of $\Pi$, we therefore obtain a canonical homomorphism $\Pi$ to $G$, i.e. of one of the $\pi_i$ to $G$. As $G$ is the permutation group of $I=X(\xi)$, this means "$G$ acts continuously on $I=X(\xi)$", being understood that the $\pi_i$ (large $i$) act on $I$, in a compatible way with the transition morphisms of $\Pi$.\par
If $X\to Y$ is an $S$-morphism between quasi-isotrivial twisted constant schemes over $S$, then we obtain an induced map $X(\xi)\to Y(\xi)$, which is compatible with the operation of $\Pi$. We thus obtain an equivalence of categories:

\begin{proposition}\label{scheme twisted constant quasi-isotrivial over Noe connected equivalence to Pi-module}
Let $S$ be a connected locally Noetherian scheme, $\xi$ be a geometric point of $S$, $\Pi=\Pi_1(S,\xi)$ be the pro-fundamental group of $S$ at $\xi$. Then the functor
\[X\mapsto X(\xi)\]
is an equivalence between the category of quasi-isotrivial twisted constant schemes over $S$ and the category of $\Pi$-sets.
\end{proposition}

The functor $X\mapsto X(\xi)$ is compatible with finite sums and finite projective limits. Therefore, it transforms quasi-isotrivial twisted constant groups (or rings, etc.) over $S$ to ordinary groups (or rings, etc.) endowed with a continuous action of $\Pi$. In particular:

\begin{corollary}\label{scheme twisted constant group over Noe connected equivalence to Pi-module}
The category of twisted constant abelian groups over $S$ is equivalent to the category of $\Pi$-modules.
\end{corollary}

Using \cref{scheme group multiplicative ft is quasi-isotrivial} and \cref{scheme group multiplicative D_S representable if quasi-isotrivial}, we therefore conclude the following results:

\begin{theorem}\label{scheme group multiplicative quasi-isotrivial over Noe connected equivalence to Pi-module}
Let $S$ be a connected locally Noetherian scheme, $\xi$ be a geometric point of $S$, $\Pi=\Pi_1(S,\xi)$ be the pro-fundamental group of $S$ at $\xi$. Then the functor
\[G\mapsto\Hom_{\kappa(\xi)\dash\Grp}(G_\xi,\G_{m,\xi})\]
induces an anti-equivalence from the category of groups of quasi-isotrivial multiplicative type over $S$ to the category of $\Pi$-modules.
\end{theorem}

\begin{corollary}\label{scheme group multiplicative ft over Noe connected equivalence to ft Pi-module}
The preceding functor induces an anti-equivalence between the category of groups of multiplicative finite type over $S$ and the category of $\pi$-modules which are finitely generated over $\Z$.
\end{corollary}

\begin{example}
Let $S$ be a complete rational curve over an algebraically closed field, with exactly a point with $n+1$ distinct branches. By \cref{scheme enlarged fundamental group example}, the enlarged fundamental group $\Pi(S,\xi)$ is a free group with $n$ generators. Therefore, by \cref{scheme group multiplicative ft over Noe connected equivalence to ft Pi-module}, the classification of tori of relative dimension $m$ over $S$ is equivalent to that of systems of $n$ endomorphisms $A_1,\dots,A_n$ of the $\Z$-module $\Z^m$, up to isomorphisms of $\Z^m$. 
\end{example}

\begin{remark}
If we make no assumption on $S$, it remains true that for an ordinary finitely generated abelian group $M$, the category of groups of multiplicative type of type $M$ over $S$ is anti-equivalent to the category of principal Galois bundles over $S$, of group $G=\Aut_{\Grp}(M)$. This follows easily from \cref{scheme group multiplicative ft cat equivalent to ft twisted constant} and \cref{scheme group multiplicative ft sIso representable}.
\end{remark}

\section{Criterion of representability and applications to subgroups of multiplicative type}
As we have already seen in \autoref{scheme group multiplicative quasi-isotrivial theorem subsection} and \autoref{scheme enlarged fundamental group subsection}, the representability of certain functors, in particular of functors of the form $\sHom_S(X,Y)$, plays an important role in many questions concerning group schemes. Among the results particularly useful in this direction, let us point out (in addition to the
questions of the representability of quotient) the question of the representability of functors of the form $\Res_{X/S}Y$ ($Y$ a sub-object of $X$) studied in \autoref{sheme functor representability of Res subsection}. In this section, we shall give various variants of this results, which will provide us the representativeness of various centralizers, standardizers, and transporters.

\subsection{Formally smooth functors}


We now interpret the results stated in \autoref{scheme group multiplicative infinitesimal prop subsection}, concerning the infinitesimal extensions of a homomorphism from a group of multiplicative type, into the language introduced above.

\begin{proposition}\label{scheme multiplicative to smooth functor formally smooth}
Let $S$ be a scheme and $G$ be a smooth group over $S$. Consider a homomorphism $u:H_1\to H_2$ of groups of multiplicative type over $S$, whence a morphism of functors over $S$:
\[M_{H_1}\to M_{H_2},\quad M_{H_i}=\sHom_{S\dash\Grp}(H_i,G),\]
defined by $w\mapsto w\circ u$. Then each $M_{H_i}$ ($i=1,2$) is formally smooth, and the homomorphism $M_{H_1}\to M_{H_2}$ is formally smooth.
\end{proposition}
\begin{proof}
The first assertion follows from \cref{scheme group multiplicative morphism nilpotent lifting exist}~(a). In view of the definition, the second one signifies the following: if $S$ is affine, $S_0$ is a subscheme of $S$ defined by a nilpotent ideal, $v:H_1\to G$ is a homomorphism of $S$-groups, and $w_0:(H_2)_{S_0}\to G_{S_0}$ is a homomorphism of $S_0$-groups such that $w_0\circ u_{S_0}=v_{S_0}$, then there exists a homomorphism of $S$-groups
\[w:H_2\to G\]
which extends $w_0$ and such that $w\circ u=v$. For this, we can first extend $w_0$ into a homomorphism of $S$-groups $w':H_2\to G$, which is possible by \cref{scheme group multiplicative morphism nilpotent lifting exist}~(a). Then we consider $v'=w'\circ u:H_1\to G$, which is such that $v'_{S'}=v_{S_0}$ by the hypothesis on $w_0$. In view of \cref{scheme group multiplicative morphism nilpotent lifting exist}~(a), there then exists an element $g\in G(S)$, whose image in $G(S_0)$ is the identity, and such that $v=\inn(g)v'$, whence $v=\inn(g)w'\circ u$. It then suffices to take $w=\inn(g)w'$.
\end{proof}

\begin{corollary}\label{scheme multiplicative to smooth transporter formally smooth}
With the notations of \cref{scheme multiplicative to smooth functor formally smooth}, let $v_1,v_2:H\to G$ be two morphisms of $S$-groups, and $\Trans(v_1,v_2)$ be the subfunctor of $G$ formed by $g$ such that $\inn(g)v_1=v_2$. Then this functor is formally smooth over $S$. In particular, if $v_1=v_2=v$, then the functor $\Centr(v)$, the subgroup of $G$ formed by $h$ such that $\inn(h)v=v$, is formally smooth over $S$.
\end{corollary}
\begin{proof}
By base change $S'\to S$, we can let $S$ be affine and $S_0$ be a subscheme of $S$ defined by a nilpotent ideal. For any $g_0\in G(S_0)$ such that $\inn(g_0)(v_1)_{S_0}=(v_2)_{S_0}$, we must extend $g_0$ to a section $g\in G(S)$ such that $\inn(g)v_1=v_2$. For this, we can first extend $g_0$ to a section $g'$ of $G$ over $S$, which is possible since $G$ is smooth over $S$. Put $v_2'=\inn(g')v_1$, we then note that $v_2$ and $v_2'$ have the same restriction to $S_0$, so by \cref{scheme group multiplicative morphism nilpotent lifting exist}~(a) there exists $g''\in G(S)$, which induces the identity over $S_0$, such that $v_2=\inn(g'')v_2'$. Then $v_2=\inn(g'')\inn(g')v_1=\inn(g''g')v_1$, so it suffices to choose $g=g''g'$.
\end{proof}

\begin{corollary}\label{scheme multiplicative to smooth action morphism formally smooth}
With the notations of \cref{scheme multiplicative to smooth functor formally smooth}, consider $M=M_H$ as a functor acted by $G$ (by $(v,g)\mapsto\inn(g)v$). Then the corresponding morphism
\[\Phi:G\times_SM\to M\times_SM,\quad (g,v)\mapsto(\inn(g)v,v)\]
is formally smooth.
\end{corollary}
\begin{proof}
By base change $S'\to S$, we may consider the absolute case for $S$. A section $M\times_SM$ over $S$ is a couple $(v_1,v_2)$, and its inverse image under $\Phi$ is given by the transporter $\Trans(v_1,v_2)$. Therefore \cref{scheme multiplicative to smooth transporter formally smooth} implies our assertion.
\end{proof}

\begin{proposition}\label{scheme group smooth functor of subgroup multiplicative formally smooth}
Let $S$ be a scheme, $G$ be a smooth $S$-group over $S$, and consider the functor $M:\Sch_{/S}^{\op}\to\Set$ such that
\[M(S')=\{\text{the set of subgroups of multiplicative type of $G_{S'}$}\}.\]
Then $M$ is formally smooth over $S$.
\end{proposition}
\begin{proof}
This is a reformulation of \cref{scheme group multiplicative morphism nilpotent lifting exist}~(b).
\end{proof}

\begin{corollary}\label{scheme group smooth functor of subgroup n-torsion multiplicative formally smooth}
With the notations of \cref{scheme group smooth functor of subgroup multiplicative formally smooth}, let $n>0$ be an integer and consider the morphism of functors
\[\varphi_n:M\to M,\quad \varphi_n(H)={_nH}=\ker(n\cdot\id_H).\]
Then $\varphi_n$ is a formally smooth morphism. If for any integer $p>0$, $M_p$ denotes the subfunctor of $M$ such that $M_p(S')$ is the set of subgroups of multiplicative type $H$ of $G_{S'}$ such that ${_pH}=H$. Then the morphism $M_{np}\to M_n$ induced by $\varphi_n$ is formally smooth.
\end{corollary}
\begin{proof}
The second assertion is trivially contained in the first one. The proof of the first one is analogous to that of \cref{scheme multiplicative to smooth functor formally smooth}, by involking \cref{scheme group multiplicative morphism nilpotent lifting exist}~(b).
\end{proof}

\begin{corollary}\label{scheme group smooth subgroup multiplicative transporter formally smooth}
With the notations of \cref{scheme group smooth functor of subgroup multiplicative formally smooth}, let $H_1,H_2$ be two subgroups of multiplicative type of $G$, and $\Trans_G(H_1,H_2)$ be the subfunctor of $G$ formed by $g$ such that $\inn(g)(H_1)=H_2$. Then this functor is formally smooth over $S$. In particular, if $H_1=H_2=H$, then the subfunctor $N_G(H)$ is formally smooth.
\end{corollary}
\begin{proof}
The proof is analogous to \cref{scheme multiplicative to smooth transporter formally smooth}, by using \cref{scheme group multiplicative morphism nilpotent lifting exist}~(b).
\end{proof}

\begin{corollary}\label{scheme group smooth functor of subgroup multiplicative action morphism formally smooth}
With the notations of \cref{scheme group smooth functor of subgroup multiplicative formally smooth}, consider $M$ as a functor acted by $G$ (by $(g,H)\mapsto\inn(g)(H)$). Then the corresponding morphism
\[\Phi:G\times_SM\to M\times_SM,\quad (g,H)\mapsto(\inn(g)(H),H)\]
is formally smooth.
\end{corollary}
\begin{proof}
Again this follows from \cref{scheme group smooth subgroup multiplicative transporter formally smooth}, since the inverse images under $\Phi$ are given by transporters.
\end{proof}

\begin{proposition}\label{scheme group smooth to multiplicative subgroup transporter formally smooth}
Let $S$ be a scheme, $G$ be a smooth $S$-group over $S$, $H$ be an $S$-group of multiplicative type, and $u:H\to G$ be a homomorphism of $S$-groups. Let $K$ be a subgroup of $G$ of multiplicative type, and consider the functor $\Trans_G(u,K)$ formed by $g\in G(S')$ such that $\inn(g)u_{S'}:H_{S'}\to G_{S'}$ factors through $K_{S'}$. Then this functor is formally smooth over $S$.
\end{proposition}
\begin{proof}
By base change $S'\to S$, we can let $S$ be affine and $S_0$ be a subscheme of $S$ defined by a nilpotent ideal. Let $g_0\in G(S_0)$ be such that $\inn(g_0)u_{S_0}$ factors through $K_{S_0}$, then by \cref{scheme group multiplicative nilideal reduction}, the induced morphism $v_0:H_{S_0}\to K_{S_0}$ can be lifted to a morphism $v:H\to K$. If $j:K\to G$ is the inclusion morphism and $g$ is an extension of $g_0$ to a section of $G$ (exists since $G$ is smooth over $S$), then $jv$ and $\inn(g)u$ have the same restriction on $S_0$, so by \cref{scheme group multiplicative morphism nilpotent lifting exist}~(a), there exists $g'\in G(S)$, induceing the identity on $S_0$, such that $\inn(g')\inn(g)u=jv$. It then suffices to take $g'g\in G(S)$.
\end{proof}

\subsection{Auxiliary results on representability}
\begin{proposition}\label{scheme functor representable by open immersion iff}
Let $F\to S$ a functor over a scheme $S$. The following conditions are equivalent:
\begin{enumerate}
    \item[(\rmnum{1})] $F$ is representable, and $F\to S$ is an open immersion.
    \item[(\rmnum{2})] $F$ is a fpqc sheaf, commutes with inductive limits of rings, $F\to S$ is a monomorphism, and the followng conditions are verified: for any local scheme $S'$ over $S$ with residue field $k$, any $S$-morphism $\Spec(k)=S_0'\to F$ can be extended to an $S$-morphism $S'\to S$.
    \item[(\rmnum{3})] (If $S$ is locally Noetherian) $F$ is a fpqc sheaf, commutes with inductive limits of rings and adic projective limits of rings, $F\to S$ is a monomorphism, and is infinitesimally \'etale.  
\end{enumerate}
\end{proposition}

Before proving \cref{scheme functor representable by open immersion iff}, let us explain some terminologies. We say that a contravariant functor $F$ over $S$ \textbf{commutes with inductive limits of rings} if for any filtered projective system $(S_i')_{i\in I}$ of rings over an open affine subset of $S$, with ring $A_i'$, the natural homomorphism
\begin{equation}\label{scheme presheaf commutes with inductive limit of rings-1}
\rlim \Hom_{S}(S_i',F)\to \Hom_S(S',F),\quad \text{where $S'=\Spec(A')$, $A'=\rlim_iA_i'$}
\end{equation}
is bijective. We note that the scheme $S'$ is none other than the projective limit of $(S_i')$ in the category of schemes (and even in that of ringed spaces), so the considered condition is naturally considered as a "right-exactness" (commutation with certain inductive limits in $\Sch_{/S}$), like the condition of being a sheaf for some topology. Note that the considered condition is essentially relative, i.e. involves the morphism $F\to S$ and not only the functor $F:\Sch^{\op}\to\Set$ itself. Thus, if $F$ is representable, the considered condition means that $F$ is locally of finite presentation over $S$ (cf. \cite{EGA4-3} 8.14.2).\par
On the other hand, we say that a functor $F$ over $S$ \textbf{commutes with adic projective limits of rings}, if for any $S'$ over $S$ which is the spectrum of a complete Noetherian local ring $A'$ with maximal ideal $\m$, putting $S_n'=\Spec(A'/\m^{n+1})$, the natural map
\begin{equation}\label{scheme presheaf commutes with inductive limit of rings-2}
\Hom_S(S',F)\to\llim_n\Hom_S(S_n',F)
\end{equation}
is bijective. We note that this condition, which is naturally a "right-exactness" condition, is satisfied whenever $F$ is representable. However, we easily seen that this condition only concerns the functor $F$ as an object of $\widehat{\Sch}$, i.e. is independent of the morphism $F\to S$.

\begin{remark}\label{scheme functor commutes with inductive limit of rings representable iff on local fp}
Let $F$ be a functor over $S$ which is a Zariski sheaf. Let $(S_i)$ be a covering of $S$ by open subsets, then we esily verify that (by a glueing method) $F$ is representable if and only if each $F_i=F\times_SS_i$ is representable, which permits us for example to reduce to the case where $S$ is affine. Suppose that $F$ also commutes with inductive limits of rings, then for $F$ to be representable, it is necessary and sufficient that its restriction to the category of schemes locally of finite presentation over $S$ be representable. In fact, in this case, if $X$ is a scheme locally of finite presentation over $S$ and $X\to F$ is a morphism such that, for any $S'$ locally of finite presentation over $S$, the induced morphism
\[\Hom_S(S',X)\to\Hom_S(S',F)\]
is bijective, then as $X$ and $F$ are Zariski sheaves which commute with the inductive limits of rings\footnote{Note that any algebra $B$ over a ring $A$ is the inductive limit of finitely presented sub-$A$-algebras.}, $X\to F$ must be an isomorphism.
\end{remark}

We now prove \cref{scheme functor representable by open immersion iff}. The implications (\rmnum{1})$\Rightarrow$(\rmnum{2}) and (\rmnum{1})$\Rightarrow$(\rmnum{3}) are evident, so it suffices to prove the converse. In the situation of (\rmnum{2}), let $U$ be the set of $s\in S$ such that the canonical monomorphism $\Spec(\kappa(s))\to S$ factors through $F$. In view of the last condition of (\rmnum{2}), we see that $U$ is also the set of $s\in S$ such that the canonical monomorphism $\Spec(\mathscr{O}_{S,s})\to U$ factors through $F$. Now $\mathscr{O}_{S,s}$ is the inductive limit of the rings of affine neighborhoods of $s$ in $S$, so by our hypothesis on $F$, there exists an open neighborhood $U_s$ such that the canonical immersion $U_s\to S$ factors through $F$. This implies $U_s\sub U$, so the set $U$ is open in $S$. As $F\to S$ is a monomorphism and $F$ is a Zariski sheaf, the $S$-morphism $U_s\to F$ can be glued over $U_s\cap U_{s'}$ ($s,s'\in U$), hence provides an $S$-morphism $U\to F$ (which is monomorphism, since $F\to S$ is a monomorphism). It remains to show that this is an isomorphism, or equivalently that any $S$-morphism $S'\to F$ factors uniquely through $U$ (where $S'$ is an $S$-scheme). As $F\to S$ and $U\to S$ are monomorphism, this is also equivalent to that the structural morphism $S'\to S$ factors through $U$, for which we can reduce to the case where $S'$ is the spectrum of a field (as $U$ is open in $S$, the morphism $S'\to S$ factors through $U$ if and only if its image in $S$ is contained in $U$, and this can be verified using the morphism $\Spec(\kappa(s'))$, $s'\in S'$). Let $s\in S$ be a point lying over the unique point $s'$ of $S'$, we claim that the $S$-morphism $S'\to S$ factors through $S_0=\Spec(\kappa(s))\to F$ (this implies $s\in U$ and we are done). For this, since $S'\to S_0$ is covering for the fpqc topology and $F$ is a fpqc sheaf, it suffices to note that the two compositions
\[S''=S'\times_{S_0}S'\rightrightarrows S'\to F\]
are equal, which follows from the fact that $F\to S$ is a monomorphism.\par
Now assume that $S$ is locally Noetherian, and we prove that (\rmnum{3})$\Rightarrow$(\rmnum{2}). For this, it suffices to show that the last condition of (\rmnum{3}), and it suffices to assume that $S'=\Spec(\mathscr{O}_{S,s})$, with $s\in S$ (this is all we have used to prove (\rmnum{1}), which in turn implies (\rmnum{2})). Let $A=\mathscr{O}_{S,s}$, $A_n=A/\m^{n+1}$, $S_n=\Spec(A_n)$, then it follows from the hypothesis that $F\to S$ is infinitesimally smooth, so the given morphism $S_0\to F$ extends to morphisms $S_n\to F$. As $F\to S$ is a monomorphism, we thus obtain an element of $\llim_nF(S_n)$, and as $F$ commutes with adic projective limits of rings, the morphisms $S_n\to F$ provides an $S$-morphism $\Spec(\widehat{A})=\widehat{S}'\to F$. Again, as $F\to S$ is a monomorphism, $F$ is a fpqc sheaf, and $\widehat{S}'\to S'$ is covering for this topology ($S$ being Noetherian), the morphism $\widehat{S}'\to F$ factors through $S'\to F$, which completes the proof.

\begin{proposition}\label{scheme functor over Noe representable if surjective subopen functor}
Let $S$ be a locally Noetherian scheme, $F\to S$ be a functor over $S$, $(X_i,u_i)_{i\in I}$ be a family of $S$-morphisms $u_i:X_i\to F$, where each $X_i$ is a scheme locally of finite type over $S$. Suppose that the following conditions are satisfied:
\begin{enumerate}
    \item[(a)] $F$ is a fpqc sheaf, commutes with inductive limits of rings and adic projective limits of rings.
    \item[(b)] The $u_i:X_i\to F$ are monomorphisms, and are infinitesimally \'etale.
    \item[(c)] The family $(u_i)$ is "jointly surjective", i.e. any point of $F$ with values in a field $k$ comes from a point of $X_i$ with values in $k$.
\end{enumerate}
Then $F$ is representable by a scheme locally of finite type over $S$, the $u_i$ are open immersions, and the family $X_i$ covers $F$.
\end{proposition}
\begin{proof}
For each couple of indices $(i,j)$, we write $X_{ij}=X_i\times_FX_j$, and consider the projections
\[v_{ij}:X_{ij}\to X_i,\quad w_{ij}:X_{ij}\to X_j.\]
We claim that these are represented by open immersions. To see this, we apply the criterion \cref{scheme functor representable by open immersion iff}~(\rmnum{3}): $X_{ij}$ satisfies the exactness conditions, because $F$, $X_i$, $X_j$ all satisfy them, and the conditions are stable under finite projective limits (in particular fiber products). As $X_i\to F$ is a monomorphism, so is $v_{ij}:X_{ij}\to X_i$, which is induced by base change $X_j\to F$, and similarly $v_{ij}$ is a monomorphism; finally, the infinitesimal \'etale condition is preserved by base change. This proves that we are under the conditions of \cref{scheme functor representable by open immersion iff}~(\rmnum{3}).\par
We now use the $X_i$, $X_{ij}$, $v_{ij}$ and $w_{ij}$ to construct an $S$-scheme $X$ such that $X_i$ is identified with an open subset of $X$, and $X_{ij}$ is identified with $X_i\cap X_j$, with $v_{ij}$, $w_{ij}$ be canonical immersions. We note that $X$ is also the quotient of $X'=\coprod_iX_i$ by the equivalence relation $R=\coprod_{ij}X_{ij}$ (the two projections $v,w:R\rightrightarrows Y$ being defined by the $v_{ij}$ and $w_{ij}$). More precisely, $F$ being a fpqc sheaf, the $u_i:X_i\to F$ provides a morphism $u':X'\to F$, and $R$ is none other than the equivalence relation defined by $u'$. Finally, it follows from definition that the quotient $X=X'/R$ is also a quotient in the category of fpqc sheaves. Therefore, $u'$ factors into a unique morphism
\[u:X\to F\]
which is a monomorphism. It remains to show that this is an isomorhism. As $F$ is a Zariski sheaf, we can suppose that $S$ is affine, and as $F$ commutes with inductive limit of rings, it suffices to verify that for any $T$ affine of finite presentation over $S$, any morphism $T\to F$ factors through $X$ (cf. \cref{scheme functor commutes with inductive limit of rings representable iff on local fp}). For this, consider $G=X\times_FT\to T$; as $T$ is Noetherian, we see as above that it is an open immersion. Since $X\to F$ is jointly surjectie and this conditions is stable under base change, the morphism $G\to T$ is also jointly surjective, hence an isomorphism because it is an open immersion.
\end{proof}

\begin{proposition}\label{scheme functor over Noe representable if morphism to projective limit}
Let $S$ be a locally Noetherian scheme, $I$ be a directed set, $(T_i)_{i\in I}$ a projective system of $S$-schemes locally of finite type, $T=\llim_iT_i$ the projective limit functor, $F$ a functor over $S$, and $u:F\to T$ an $S$-morphism. Suppose that the following conditions are satisfied:
\begin{enumerate}
    \item[(a)] $F$ is a fpqc sheaf, commutes with inductive limits of rings and adic projective limits of rings.
    \item[(b)] The morphism $u:F\to T$ is a monomorphism.
    \item[(b')] The morphism $u:F\to T$ is infinitesimally \'etale.
    \item[(c)] For any point $\xi$ of $F$ with values in a field $k$, denote by $\xi_i\in T_i(\Spec(k))$ its image and by $t_i$ the corresponding point of $T_i$, then there exists an index $i\in I$ such that for any $j\geq i$ the transition morphism $p_{ij}:T_j\to T_i$ is \'etale at $t_j$.
    \item[(d)] For any scheme $X$ locally of finite type over $S$, and any $S$-morphism $X\to F$, the set of $x\in X$ over which this morphism is infinitesimally \'etale is open. 
\end{enumerate}
Then $F$ is representable by a scheme locally of finite type over $S$.
\end{proposition}

In practice, we will check conditions (c) and (d) of \cref{scheme functor over Noe representable if morphism to projective limit} via the following way:

\begin{corollary}\label{scheme functor over Noe representable if morphism to projective limit condition verify}
With the conditions (a), (b), (b') of \cref{scheme functor over Noe representable if morphism to projective limit}, the conditions (c) and (d) of \cref{scheme functor over Noe representable if morphism to projective limit} are implied by the following:
\begin{enumerate}
    \item[(c')] The $T_i$ are smooth over $S$, and the transition morphisms $p_{ij}:T_j\to T_i$ are smooth.
    \item[(d')] For any point $\xi$ of $F$ with values in a field $k$, let $t_i(\xi)$ be the point of $T_i$ defined by $\xi$, $d_i(\xi)$ the relative dimension of $T_i$ over $S$ at $t_i(\xi)$, and $d(\xi)=\sup_id_i(\xi)$. Then $d(\xi)<+\infty$ and for any scheme $X$ locally of finite type over $S$, any $S$-monomorphism $v:X\to F$, the function $x\mapsto d(\xi_x)$ on $X$ is locally constant (where for $x\in X$, we denote by $\xi_x$ the point of $F$ with values in $\kappa(x)$ induced by $v$). 
\end{enumerate}
\end{corollary}

\begin{corollary}\label{scheme morphism projective limit qc open eventual restrict}
Under the hypothesis of \cref{scheme functor over Noe representable if morphism to projective limit}, for any quasi-compact open subset $U$ of $F$ which is separated over $S$, there exists an index $i\in I$ such that $j\geq i$ the morphism $u_j|_U:U\to T_j$ is an open immersion. In particular, if the $T_i$ are quasi-affine over $S$, then any open subset $U$ of $F$ which is quasi-compact over $S$, i.e. of finite type over $S$, is quasi-affine over $S$.
\end{corollary}

\begin{proposition}\label{scheme group affine functor of qf multiplicative subgroup representable if}
Let $S$ be a scheme and $G$ be an affine group scheme over $S$.
\begin{enumerate}
    \item[(a)] Let $F:\Sch_{/S}^{\op}\to\Set$ such that, for any $S'$ over $S$,
    \[F(T)=\{\text{set of subgroups of multiplicative type of $G_{S'}$ which is finite over $S'$}\}.\]
    Suppose that $S$ is locally Noetherian or $G$ is of finite presentation over $S$. Then $F$ is representable and affine over $S$. If $G$ is of finite presentation over $S$, then $F$ is locally of finite presentation over $S$.
    \item[(b)] Let $H$ be a group of multiplicative type over $S$ and finite over $S$. Then $\sHom_{S\dash\Grp}(H,G)$ is representable and affine over $S$. If $G$ is of finite type (resp. of finite presentation) over $S$, so is $\sHom_{S\dash\Grp}(H,G)$.
\end{enumerate}
\end{proposition}

\begin{remark}
Except for the assertion that $\Hom_{S\dash\Grp}$ is affine, and in the case where $G$ is of finite presentation over $S$ (which will suffice), \cref{scheme group affine functor of qf multiplicative subgroup representable if} is an immediate consequence from Hilbert's Schema Theory (\cite{TGA4}); it even sufficies that $G$ be quasi-projective over $S$. In case (a), we can also represent the larger functor
\[F'(S')=\{\text{set of subgroups of $G_{S'}$, flat proper and of finite presentation over $S'$}\},\]
(the canonical monomorphism $F\to F'$ is an open immersion, as it follows from \cref{scheme functor representable by open immersion iff} and \cref{scheme group multiplicative over Noe complete isotrivial and reduction}), so that the representability of $F'$ implies that of $F$. In case (b), we can confine ourselves to assuming that $H$ is projective and of finite presentation over $S$. In both cases, we obtain a functor locally of finite presentation over $S$. In the section, \cref{scheme group affine functor of qf multiplicative subgroup representable if} is only a technique lemma to prove a key result at the next subsection, so we will sketch an easy direct proof, without using Hilbert schemes.
\end{remark}

\subsection{The functor of subgroups of multiplicative type}\label{scheme group affine smooth functor of multiplicative type subsection}
The main result of this subsection is the following theorem, which gives the representability of the functor of subgroups of multiplicative type for an affine smooth group scheme.

\begin{theorem}\label{scheme group affine smooth functor of subgroup multiplicative representable}
Let $S$ be a scheme, $G$ be an affine smooth $S$-group, and consider the functor $F:\Sch_{/S}^{\op}\to\Set$ defined by 
\[F(S')=\{\text{set of subgroups of multiplicative type of $G_{S'}$}\}.\]
Then $F$ is representable, and is smooth and separated over $S$.
\end{theorem}

\begin{corollary}\label{scheme group affine smooth sHom from multiplicative ft representable}
Let $G,H$ be $S$-groups, with $G$ smooth and affine over $S$, $H$ of multiplicative finite type over $S$. Then $\sHom_{S\dash\Grp}(H,G)$ is representable, and is smooth and separated over $S$.
\end{corollary}

\begin{remark}
We have deduced \cref{scheme group affine smooth sHom from multiplicative ft representable} from \cref{scheme group affine smooth functor of subgroup multiplicative representable}, which is immediate if $H$ is also smooth over $S$. For the deduction of the general case, the representability result of \cref{scheme group affine smooth functor of subgroup multiplicative representable} would have to be established without supposing $G$ smooth over $S$, but only affine of finite presentation over $S$ (of course, then $F$ will no longer be smooth over $S$ in general!). There is a little doubt that whether \cref{scheme group affine smooth functor of subgroup multiplicative representable} remains true under these more general assumptions, but the proof seems to have to be more delicate. Note, however, that when $G$ is a closed subgroup of a smooth affine group $G'$ over $S$, then the functor $F$ representing the subgroups of multiplicative type of $G$ is represented by a closed subscheme of the scheme representing the analogous functor $F'$ for $G'$, as can easily be seen by applying \cref{scheme Weil restriction of closed of essentialy free representable prop}. This also raises the following question: for a group scheme $G$ over an affine scheme $S$, which is affine and of presentation finite over $S$, is it isomorphic to a group subscheme of a suitable $\GL_{n,S}$? This is true when $S$ is the spectrum of a field (cf. \cite{SGA3-1} $\Rmnum{6}_B$ 11.11), but unfortunately false in general, even for tori. Finally, note that we can also directly prove \cref{scheme group affine smooth sHom from multiplicative ft representable} by exactly the same method as \cref{scheme group affine smooth functor of subgroup multiplicative representable}.
\end{remark}

\begin{corollary}\label{scheme multiplicative to affine smooth transporter representable}
Under the conditions of \cref{scheme group affine smooth functor of subgroup multiplicative representable}, let $u_1,u_2:H\rightrightarrows G$ be two homomorphisms of $S$-groups. Then the subfunctor $\Trans(u_1,u_2)$ of $G$ is representable by a closed subscheme of $G$, which is smooth over $S$.
\end{corollary}
\begin{proof}
By \cref{scheme multiplicative to smooth transporter formally smooth}, it remains to show that $\Trans(u_1,u_2)\to G$ is a closed immersion, which follows from the fact that it is the kernel of the morphisms $G\to M$, defined by $g\mapsto\inn(g)\circ u_1$ and the constant morphism $g\mapsto u_2$, and that $M$ is separated over $S$.
\end{proof}

\begin{corollary}\label{scheme group multiplicative to affine smooth sHom action smooth}
Under the conditions of \cref{scheme group affine smooth functor of subgroup multiplicative representable}, let $M=\sHom_{S\dash\Grp}(H,G)$, which is a smooth and separated scheme over $S$. Let $G$ acts on $M$ via $(g,u)\mapsto\inn(g)\circ u$, then the canonical morphism
\begin{equation}\label{scheme group multiplicative to affine smooth sHom action smooth-1}
\Phi:G\times_SM\to M\times_SM
\end{equation}
is smooth.
\end{corollary}
\begin{proof}
This follows from \cref{scheme multiplicative to affine smooth transporter representable}. In fact, by \cref{scheme multiplicative to smooth action morphism formally smooth}, the morphism $\Phi$ is formally smooth, it is locally of finite presentation because $G\times_SM$ and $M\times_SM$ are (cf. \cref{scheme morphism local fp permanence prop}).
\end{proof}

\begin{corollary}\label{scheme group multiplicative to affine smooth centralizer quotient representable}
Under the conditions of \cref{scheme group affine smooth functor of subgroup multiplicative representable}, let $u:H\to G$ be a morphism of $S$-groups. Then $\Centr(u)=\Trans(u,u)$ is represented by a closed subgroup of $G$, which is smooth over $S$. Further, $C/\Centr(u)$ is representable by an open subscheme of $M$.
\end{corollary}
\begin{proof}
The morphism $g\mapsto\inn(g)\circ u$ from $G$ to $M$ is smooth of finit etype in view of \cref{scheme group multiplicative to affine smooth sHom action smooth}, and hence an open morphism (\cite{EGA4-2} 2.4.6). If $U$ denotes its image, with the induced scheme structure by $M$, then the induced morphism $G\to U$ is smooth, surjective, of finite type (cf. \cref{scheme morphism ft permanence prop}), hence covering for the fpqc topology. Moreover, it is evident that the morphism $G\to M$ makes $G$ a formally principal homogeneous sheaf under $\Centr(u)_M$ (with action defined by right translations), which implies that the sheaf $G/\Centr(u)$ is representable by $U$ (cf. \cref{site M-effective principal homogeneous bundle iff}).
\end{proof}

\begin{proposition}\label{scheme group multiplicative to affine smooth morphism conjugate iff}
Under the conditions of \cref{scheme group affine smooth functor of subgroup multiplicative representable}, let $u_1,u_2:H\to G$ be two homomorphism of $S$-groups. Then the following conditions are equivalent:
\begin{enumerate}
    \item[(\rmnum{1})] $u_1$ and $u_2$ are locally conjugate for the \'etale topology.
    \item[(\rmnum{1}')] $u_1$ and $u_2$ are locally conjugate for the fpqc topology.
    \item[(\rmnum{2})] For any $s\in S$, denote by $\bar{s}$ the spectrum of an algebraic closure of $\kappa(s)$, then the morphisms $(u_1)_{\bar{s}}$ and $(u_2)_{\bar{s}}$ are conjugate by an element of $G(\kappa(\bar{s}))$.
    \item[(\rmnum{2}')] The structural morphism $\Trans(u_1,u_2)\to S$ is surjective.
    \item[(\rmnum{3})] $\Trans(u_1,u_2)$ is a torsor under the action of the smooth $S$-group $\Centr(u_1)$.   
\end{enumerate}
\end{proposition}
\begin{proof}
The implications (\rmnum{1})$\Rightarrow$(\rmnum{1}') and (\rmnum{2})$\Rightarrow$(\rmnum{2}') are trivial, and (\rmnum{1}')$\Rightarrow$(\rmnum{2}) follows from the principle of finite extension (cf. \cite{EGA4-3} 9.1.4). On the other hand, (\rmnum{2}')$\Rightarrow$(\rmnum{3}) because $\Trans(u_1,u_2)$ is smooth and of finite type over $S$, hence flat and quasi-compact over $S$; so it is faithfully flat and quasi-compact over $S$ if and only if the structural morphism is surjective (it is easy to see that $\Trans(u_1,u_2)$ is a formally principal homogeneous bundle over $S$ under the action of $\Centr(u_1)$). Finally, (\rmnum{3})$\Rightarrow$(\rmnum{1}) follows from the fact that the morphism $\Trans(u_1,u_2)\to S$ is smooth and surjective (cf. \cref{EGA4-4} 17.16.3). 
\end{proof}

\begin{remark}\label{scheme group multiplicative to affine smooth functor of conjugate morphism representable}
For $u_1=u$ fixed, the functor $\Sch_{/S}^{\op}\to\Set$ which associates a scheme $S'$ over $S$ with the set of homomorphisms of $S'$-groups $u_2:H_{S'}\to G_{S'}$ which are locally conjugate to $u_{S'}:H_{S'}\to G_{S'}$ for the \'etale topology (or equivalently, the fpqc topology), is represented by the open subset $M\cong G/\Centr(u)$ of $G$, considered in \cref{scheme group multiplicative to affine smooth centralizer quotient representable}.
\end{remark}

Let us now sketch the variants of the previous results, obtained by applying \cref{scheme group affine smooth functor of subgroup multiplicative representable} instead of \cref{scheme group affine smooth sHom from multiplicative ft representable}. Let $G$ be a smooth and affine group scheme over $S$, and denote by $M$ the smooth and separated $S$-scheme which represents the functor considered in \cref{scheme group affine smooth functor of subgroup multiplicative representable}. We still have an action of $G$ on $M$:
\[G\times_SM\to M,\quad (g,H)\mapsto\inn(g)(H),\]
whence as above a morphism
\begin{equation}\label{scheme group affine smooth functor of subgroup multiplicative action morphism}
\Phi:G\times_SM\to M\times_SM.
\end{equation}
By \cref{scheme group affine smooth functor of subgroup multiplicative representable} and \cref{scheme group smooth functor of subgroup multiplicative action morphism formally smooth}, this morphism is smooth, and we can also prove the following corollary:

\begin{corollary}\label{scheme smooth affine subgroup multiplicative transporter representable}
Let $H_1$, $H_2$ be subgroups of multiplicative type of $G$. Then the subfunctor $\Trans_G(H_1,H_2)$ of $G$ are representable by a closed subscheme of $G$, which is smooth over $S$.
\end{corollary}
\begin{proof}
This follows from \cref{scheme group smooth subgroup multiplicative transporter formally smooth}, and $\Trans_G(H_1,H_2)$ is closed in $G$ for the same reason as in \cref{scheme multiplicative to affine smooth transporter representable}.
\end{proof}

\begin{corollary}\label{scheme smooth affine subgroup multiplicative normalizer representable}
Let $H$ be a subgroup of multiplicative type of $G$. Then the subfunctor $N_G(H)$ of $G$ is representable by a closed subgroup of $G$ which is smooth over $S$. Moreover, the quotient $G/N_G(H)$ is representable by an open subscheme of $M$.
\end{corollary}

\begin{corollary}\label{scheme smooth affine subgroup multiplicative conjugate iff}
Let $H_1,H_2$ be subgroups of multiplicative type of $G$. Then the following conditions are equivalent:
\begin{enumerate}
    \item[(\rmnum{1})] $H_1$ and $H_2$ are locally conjugate for the \'etale topology.
    \item[(\rmnum{1}')] $H_1$ and $H_2$ are locally conjugate for the fpqc topology.
    \item[(\rmnum{2})] For any $s\in S$, denote by $\bar{s}$ the spectrum of an algebraic closure of $\kappa(s)$, then $(H_1)_{\bar{s}}$ and $(H_2)_{\bar{s}}$ are conjugate by an element of $G(\kappa(\bar{s}))$.
    \item[(\rmnum{2}')] The structural morphism $\Trans_G(H_1,H_2)\to S$ is surjective.
    \item[(\rmnum{3})] $\Trans_G(H_1,H_2)$ is a torsor under the action of the smooth $S$-group $N_G(H_1)$.
\end{enumerate}
\end{corollary}

\begin{remark}\label{scheme smooth affine subgroup multiplicative functor of conjugate representable}
Similar to \cref{scheme group multiplicative to affine smooth functor of conjugate morphism representable}, for a subgroup of multiplicative type $H$ of $G$, the functor $\Sch_{/S}^{\op}\to S$ which assocaites a scheme $S'$ over $S$ with the set of subgroups $H'$ of multiplicative type of $G_{S'}$ which are locally conjugate to $H_{S'}$ for the \'etale topology, is representable by the open subset $C/Z_G(H)$ of $G$.
\end{remark}

\begin{remark}
As the morphism (\ref{scheme group multiplicative to affine smooth sHom action smooth-1}) (resp. (\ref{scheme group affine smooth functor of subgroup multiplicative action morphism})) is smooth hence open, its image is open in $M\times_SM$. We let $R$ be this image, endowed with the induced scheme structure of $M\times_SM$. We then easily see that $R$ is an equivalence relation over $M$, whose definition is none other than the condition considered in \cref{scheme group multiplicative to affine smooth morphism conjugate iff} (resp. \cref{scheme smooth affine subgroup multiplicative conjugate iff}). It is therefore interesting that whether the quotient sheaf $M/R$ (which is formally \'etale over $S$, cf. \cref{scheme group multiplicative morphism nilpotent lifting exist}) is representable (it is then representable by a \'etale scheme over $S$); this is true for example if $S$ is the spectrum of a field. We also note that it may happen that $R$ is not closed in $M\times_SM$, which signifies that (if $M/R$ is representable) $M/R$ may not be separated over $S$.
\end{remark}

\begin{theorem}\label{scheme group affine smooth multiplicative fiber etale lifting}
Let $S$ be a scheme, $G$ be a smooth and affine group over $S$, $s\in S$.
\begin{enumerate}
    \item[(a)] For any subgroup of multiplicative type $H_0$ of $G_s$, there exists an \'etale morphism $S'\to S$, a point $s'\in S'$ lying over $s$ such that $\kappa(s')=\kappa(s)$, and a subgroup of multiplicative type $H'$ of $G'=G\times_SS'$ such that $H'_{s'}=H_0\otimes_{\kappa(s)}\kappa(s')$.
    \item[(b)] For any group homomorphsim $u_0:H_s\to H_s$, where $H$ is an $S$-group of multiplicative finite type, there exists an \'etale morphism $S'\to S$, a point $s'\in S$ lying over $s$ such that $\kappa(s')=\kappa(s)$, and a morphism $u':H'\to G'$ such that $u'_{s'}=u_0\otimes_{\kappa(s)}\kappa(s')$.
\end{enumerate}
\end{theorem}
\begin{proof}
This follows from \cref{scheme group affine smooth functor of subgroup multiplicative representable} and \cref{scheme group affine smooth sHom from multiplicative ft representable}. For example, by \cref{scheme group affine smooth functor of subgroup multiplicative representable} we know that the functor $M$ considered in it is representable and smooth over $S$. Now a subgroup of multiplicative type $H_0$ of $G_s$ corresponds to a morphism $\Spec(\kappa(s))\to M$, located at $m\in M$. Then by (\cite{EGA4-4} 17.16.3 (\rmnum{1})), there exists a neighborhood $U$ of $s$ and a subscheme $M'$ of $M$ (contained in the inverse image of $U$ in $M$) such that $m\in M'$ and the induced morphism $M'\to U$ is \'etale. We then obtain a factorization $\Spec(\kappa(s))\to M'$, and it suffices to take $S'=M'$ and $s'=m\in M'$, noting that the morphism $S'\to S'=M'$ corresponds to a subgroup of multiplicative of $G'=G\times_SS'$, and as $\Spec(\kappa(s))\to M$ is an $S$-morphism, we have $\kappa(s')=\kappa(m)=\kappa(s)$.
\end{proof}

\begin{proposition}\label{scheme group affine smooth Weyl group representable}
Let $S$ be a scheme, $G$ be a smooth and affine group over $S$, $H$ be a subgroup of multiplicative finite type of $G$. Then $Z_G(H)$ is a clopen subgroup of $N_G(H)$, and the quotient sheaf
\[W_G(H)=N_G(H)/Z_G(H)\]
is representable by an open subgroup of $\sAut_{S\dash\Grp}(H)$, which is hence quasi-finite, \'etale and separted over $S$.
\end{proposition}
\begin{proof}
Consider the evident homomorphism
\[\theta:N_G(H)\to\sAut_{S\dash\Grp}(H),\]
whose kernel is by definition $Z_G(H)$. As $\sAut_{S\dash\Grp}(H)$ is representable by a \'etale and separated group over $S$ (cf. \cref{scheme group multiplicative ft sIso representable}), its unit section is an open and closed immersion, hence its inverse image under $\theta$ is an open and closed subgroup of $N_G(H)$. We also claim that $\theta$ is a smooth morphism: the formally smoothness follows formally from the definition, and the fact that $N_G(H)$ is smooth over $S$ and $\sAut_{S\dash\Grp}(H)$ is \'etale over $S$. We then conclude as in \cref{scheme group multiplicative to affine smooth centralizer quotient representable} that the image of $\theta$ is an open subset $U$ of $\sAut_{S\dash\Grp}(H)$, and represents the quotient sheaf $N_G(H)/Z_G(H)$ (note that $N_G(H)$ is affine over $S$, as a closed subscheme of $G$). This scheme is hence \'etale and separated over $S$, since $\sAut_{S\dash\Grp}(H)$ is, and it is quasi-finite over $S$ because it is quasi-compact as the image of $N_G(H)$.
\end{proof}

\begin{corollary}\label{scheme group affine smooth Weyl group fiber rank semicontinuous}
Under the hypothesis of \cref{scheme group affine smooth Weyl group representable}, for any $s\in S$, let 
\[w(s)=\rank_{\kappa(s)}(N_{G_s}(H_s)/Z_{G_s}(H_s))\]
(which is also the index of $Z_{G(k)}(H(k))$ in $N_{G(k)}(H(k))$, where $k$ is an algebraic closure of $\kappa(s)$). Then the function $s\mapsto w(s)$ is lower semi-continuous. For it to be constant in a neighborhood of $s$, it is necessary and sufficient that $W_G(H)$ be finite over $S$ in a neighborhood of $s$.
\end{corollary}
\begin{proof}
In fact, for any $S$-scheme $W$ which is \'etale, of finite type and separated over $S$, the function $s\mapsto w(s)=\rank_{\kappa(s)}(W_s)$ is lower semi-continuous, and it is constant in a neighborhood of $s$ if and only if $W$ is finite over $S$ at a neighborhood of $s$. To see this, we note that we can reduce to the case where $S$ is locally Noetherian: we can clearly assume that $S=\Spec(A)$ is affine, and then $S$ is the inductive limit of $S_i=\Spec(A_i)$, where $A_i$ are finitely generated sub-$\Z$-algebras of $A$. By (\cite{EGA4-3} 8.8.2 et 8.10.5), there exists an index $i$, a scheme $W_i$ over $S_i$ which is \'etale, of finite type and separated over $S_i$, such that $W$ is isomorphic to $W_i\times_{S_i}S$. We recall that the rank function of fibers is invariant under base change (cf. \cref{scheme finite over field count point in a.c. extension}), so it suffices to prove the assertion for $W_i$ and $S_i$, which in turn follows from (\cite{EGA4-3} 15.5.1, \cite{EGA4-2} 2.4.6, et \cref{scheme morphism over local Noe finite iff affine proper}).
\end{proof}

\section{Maximal tori, Weyl group, Cartan subgroups and reductive center}
\subsection{Maximal tori, Cartan subgroups and Weyl group}
\paragraph{Maximal tori}
Let $G$ be an algebraic group over an algebraically closed field $k$. We say that an algebraic subgroup $T$ of $G$ is a \textbf{maximal torus} if $T$ is a torus (since $k$ is algebraically closed, this means $T$ is isomorphic to a group of the form $\G_m^r$), and is maximal for this property. Note that, $k$ being perfect, $G_\red$ is a subgroup of $G$ which is smooth over $k$, and any reduced subgroup of $G$ is automatically a subgroup of $G_\red$. Therefore, the maximal tori of $G$ coincide with that of $G_\red$. If $G$ is affine, hence $G_\red$ is affine, a fundamental theorem of Borel tells us that any two maximal tori of $G$ are conjugate under an element of $G(k)=G_\red(k)$ (\cite{Chevalley1958} 6, th.4 (c)), and in particular have the same dimension. This common dimension is called the \textbf{reductive rank} of $G$. Note also that the restriction that $G$ is affine is harmless, as it follows from a theorem known from Chevalley that any smooth connected algebraic group over $k$ is a extension of an abelian variety by an affine group. In this subsection, we will most often limit ourselves to group schemes affine over the base.\par
Let $G$ be a smooth algebraic group over $k$ and $T$ be a maximal torus of $G$; the centralizer $C=Z_G(T)$ of $T$ in $G$ is called the \textbf{Cartan subgroup} of $G$ associated with $T$, which is a subgroup of $G$ thanks to \cref{scheme k-group transporter representable by closed}. Note that as $G$ is smooth over $k$, $C$ is also smooth over $k$ in view of \cref{scheme smooth affine subgroup multiplicative normalizer representable} and \cref{scheme group affine smooth Weyl group representable}, so in this case $C$ is the unique subgroup of $G$ smooth over $k$ such that $C(k)$ is the centralizer of $T(k)$ in $G(k)$. By the conjugation theorem (\cite{Chevalley1958} 6, th.4 (c)), the Cartan subgroups of different maximal tori are conjugate to each other, so they have the same dimension, which is called the \textbf{nilpotent rank} of $G$, and is equal to that of $G_\red$. Let $\rho_r(G)$ and $\rho_n(G)$ be the reductive rank and nilpotent rank of $G$, respectively. Then we have the inequality:
\[\rho_r(G)\leq\rho_n(G),\]
and their difference
\[\rho_u(G)=\rho_n(G)-\rho_r(G)=\dim(C/T)\]
can be calld, if $G$ is affine, the \textbf{unipotent rank} of $G$. If $G$ is smooth, affine and connected, then $C$ is a maximal nilpotent and connected algebraic group, and equals to its connected normalizer (\cite{Chevalley1958} 6, th.6 (a) et (c)), hence isomorphic to a product $C_s\times C_u$, where $C_s=T$ is the maximal torus and $C_u$ is a smooth unipotent subgroup, i.e. a successive extension of groups isomorphic to $\G_a$ (\cite{Chevalley1958} 6, th.1 cor.1 et 7, th.4). In this case, we also have
\[\rho_u(G)=\dim(C_u).\]

\begin{remark}
In addition to the notions of rank that we have just specified for an affine algebraic group, there are two others which are useful, namely the \textit{semisimple rank} $\rho_s(G)$, which is by definition the reductive rank of the quotient $G/\rad(G)$, where $\rad(G)$ is the radical of G, and the \textit{infinitesimal rank} $\rho_i(G)$, which is defined to be the nilpotent rank of the Lie algebra of $G$ (this will be defined and studied later). We only remark that there are inequalities
\[\rho_s\leq\rho_r\leq\rho_n\leq\rho_i.\]
\end{remark}

For a general group scheme over $S$, the notion of a maximal torus is defined using the fibers over residue fields. For this to work, we must take an algebraic closure of each residue field $\kappa(s)$, $s\in S$. The validity of taking such a field extension is justified by the following lemma:

\begin{lemma}\label{scheme alg group maximal torus iff base change}
Let $G$ be an algebraic group over an algebraically closed field $k$, $T$ be an algebraic subgroup of $G$, $k'$ be an algebraically closed extension of $k$, $G'$ and $T'$ be the induced group under base change. For $T$ to be a maximal torus, it is necessary and sufficient that $T'$ is a maximal torus of $G'$.
\end{lemma}
\begin{proof}
Since $k$ and $k'$ are algebraically closed, by our arguments above, we can replace $G$ with $G_\red$ and hence assume that $G$ is smooth over $k$. In this case, we know from (\cite{Chevalley1958}) that $T$ is a maximal torus in $G$ if and only if $Z_G(T)/T$ is unipotent, and this holds for $T$ if and only if it holds for $T'$ (\cite{Chevalley1958}).
\end{proof}

\begin{definition}\label{scheme group maximal torus def}
Let $S$ be a scheme, $G$ be an $S$-group of finite type, $T$ be a subgroup of $G$. We say that $T$ is a maximal torus of $G$ if
\begin{enumerate}
    \item[(a)] $T$ is a torus, i.e. is locally isomorphic to $\G_m^r$ for the fpqc topology;
    \item[(b)] for any $s\in S$, denote by $\bar{s}$ the spectrum of an algebraic closure of $\kappa(s)$, $T_{\bar{s}}$ is a maximal torus in $G_{\bar{s}}$.  
\end{enumerate}
\end{definition}
It follows from \cref{scheme alg group maximal torus iff base change} that if $S$ is the spectrum of a an algebraically closed field, then we recover the usual definition of maximal tori, and that the above definition is stable under base change. We also note that a maximal torus in the sense of \cref{scheme group maximal torus def} is maximal in the set of sub-torus of $G$ (this follows easily from \cref{scheme group multiplicative ft mono epi locus prop}). But the converse of this question is more tricky, that is, whether $G$ admits effectively a maximal torus in the sense of \cref{scheme group maximal torus def}, which is in general not true even if $G$ is semi-simple. However, we will see that this is true if $S$ is artinian, or if $S$ is a local scheme and $G$ is "reductive": in this case, every torus of $G$ is contained in a maximal torus.

\begin{definition}
Let $G$ be an algebraic group over a field $k$. We define the \textbf{reductive rank} (resp. \textbf{nilpotent rank}, resp. \textbf{unipotent rank}, etc.) of $G$ to be that of $G_{\bar{k}}$, where $\bar{k}$ is an algebraic closure of $k$.
\end{definition}

We see in view of \cref{scheme alg group maximal torus iff base change} and the commutation of $Z_G(T)$ with base change that the notion of various ranks of $G$ is stable under base field extensions. On the other hand, if $k$ is algebraically closed, then these coincide with the already defined ones.

\begin{remark}\label{scheme group no maximal torus example}
It is not hard to construct a affine group scheme smooth over the spectrum $S$ of a DVR, whose generic fiber is isomorphic to $\G_m$ and the special fiber is isomorphic to $\G_a$. For example, let $R=k\llbracket\pi\rrbracket$ be a complete DVR and consider the affine scheme $G=\Spec(R[t,(1-\pi t)^{-1}])$. Then $G$ is a group scheme under the multiplication and inversion:
\[\mu(t_1,t_2)=t_1+t_2-\pi t_1t_2,\quad c(t)=-t(1-\pi t)^{-1}.\]
On the level of algebras, these maps correspond to the comultiplication and antipode of $R[t,(1-\pi t)^{-1}]$, given by
\[\Delta(t)=1\otimes t+t\otimes 1-\pi t\otimes t,\quad S(t)=-t(1-\pi t)^{-1}.\]
When $\pi$ is invertible, then the $k$-linear homomorphism
\[\varphi:k[\pi,\pi^{-1},t,(1-\pi t)^{-1}]\to k[\pi,\pi^{-1},s,s^{-1}],\quad t\mapsto\pi^{-1}(1-s)\]
gives an isomorphism from $G$ to the multiplicative group scheme $\G_{m,S}$ (and, in fact, that's how the above formulas for multiplication and inversion are easiest to obtain). When $\pi$ is zero, it is clear that $G$ reduces to the additive group.\par
For such a group $G$, it does not contain any torus except the trivial one $T$ (reduced to the identity), which is evidently not a maximal torus in the sense of \cref{scheme group maximal torus def}. More precisely, in the special fiber $G_0=\G_{a,k}$, $T_0$ is of course a maximal torus, but in the generic fiber $G_1=\G_{m,K}$, $T_1$ is not maximal ($k$ is the residue field, $K$ is the fraction field). We also see that in this example the reductive rank of $G_s$ ($s\in S$) is not a continuous function on $S$.
\end{remark}

Despite the counter-example in \cref{scheme group no maximal torus example}, we have the following result:

\begin{theorem}\label{scheme group affine smooth rho_r and rho_n and maximal tori prop}
Let $G$ be an affine smooth group over $S$. For any $s\in S$, consider $\rho_r(s)=\rho_r(G_s)$ and $\rho_n(s)=\rho_n(G_s)$, the reductive rank and nilpotent rank of $G_s$. With these notations, we have:
\begin{enumerate}
    \item[(a)] The function $\rho_r$ is lower semi-continuous on $S$, and $\rho_n$ is upper semi-continous on $S$. Hence $\rho_u=\rho_n-\rho_r$ is upper semi-continuous on $S$.
    \item[(b)] The following conditions (stable under arbitrary base change) are equivalent:
    \begin{enumerate}
        \item[(\rmnum{1})] The function $\rho_r$ is locally constant on $S$ (in this case, we say that $G$ is of locally constant reductive rank).
        \item[(\rmnum{2})] There exists, locally for the \'etale topology, a maximal torus in $G$.
        \item[(\rmnum{3})] There exists, locally for the fpqc topology, a maximal torus in $G$.
    \end{enumerate}
    \item[(c)] Let $T_1$, $T_2$ be maximal tori of $G$, then $T_1$, $T_2$ are conjugate locally for the \'etale topology, i.e. there exists a surjective \'etale morphism $S'\to S$ such that the subgroups $(T_1)_{S'}$ and $(T_2)_{S'}$ of $G_{S'}$ are conjugate by a section of $G_{S'}$ over $S'$.
    \item[(d)] If $\rho_r$ is locally constant, so is $\rho_n$ (hence also $\rho_u$).
\end{enumerate}
\end{theorem}
\begin{proof}
Note that for any morphism $S'\to S$, if $G'=G\times_SS'$, the functions $\rho'_n,\rho'_r$ and $\rho'_u$ over $S'$ defined in terms of $G'$, are obtained by composing $\rho_n,\rho_r$ and $\rho_u$ with the given morphism $S'\to S$. If $S'\to S$ is faithfully flat and quasi-compact, then $\rho'$ is upper (resp. lower) semi-continuous if and only if $\rho$ is, because the topology of $S$ is obtained by a quotient of $S$ (\cite{SGA1} \Rmnum{8} 4.3). Therefore, the assertions of (a) are local for the fpqc topology. Let $s\in S$, we can show that the set $U$ of $t\in S$ such that $\rho_r(t)\geq\rho_r(s)$ (resp. $\rho_n(t)\leq\rho_n(s)$) is an open neighborhood of $s$. By taking an algebraic closure of $\kappa(s)$, we see that we are reduced to the case where $G_s$ has a maximal torus $T_s$. Moreover, thanks to \cref{scheme group affine smooth multiplicative fiber etale lifting}~(a), by replacing $S$ with an \'etale scheme $S'$ over $S$ endowed with a point $s'$ over $s$, we can suppose that $T_s$ is the fiber of a torus $T$ of $G$. Then for any $t\in S$, we have
\[\rho_r(t)\geq\rho_r(G_t)\geq\dim(T_t)=\dim(T_s)=\rho_r(G_s)=\rho_r(s),\]
which proves that $\rho_r$ is lower semi-continuous. On the other hand, in view of \cref{scheme group affine smooth Weyl group representable}, the functor $C=Z_G(T)$ is representable by a closed subgroup of $G$ which is smooth over $S$. Hence by (\cite{EGA4-4} 17.10.2) there exists an open neighborhood $U$ of $s$ such that $t\in U$ implies $\dim(C_t)=\dim(C_s)=\rho_n(s)$. The upper semi-continuity of $\rho_n$ then follows from the relation
\[\rho_n(t)\leq\dim(C_t)\for t\in S,\]
which is contained in the following observation: if $G$ is an affine smooth algebraic group over a field $k$ and $T$ is a torus in $G$, $C$ its centralizer, then we have $\rho_n(G)\leq\dim(G)$. In fact, we can suppose that $k$ is algebraically closed and choose a maximal torus $T'$ containing $T$. Then the centralizer $C'$ of $T'$ is contained in $C$, hence $\dim(C')\leq\dim(C)$.\par
If $\rho_r$ is locally constant, then for any torus $T$ in $G$ and any $s\in S$, if $T_s$ is a maximal torus in $G_s$, then there exists an open neighborhood $U$ of $s$ such that $T|_U$ is a maximal torus in $G|_U$. Now using the reasoning of (a), we see that (\rmnum{1})$\Rightarrow$(\rmnum{3}). On the other hand, (\rmnum{3})$\Rightarrow$(\rmnum{1}), because if $G$ admits a maximal torus $T$, then it is evident that $\rho_r(s)=\dim(T_s)$ is a locally constant function on $s$, but we note that the question of the continuity of $\rho_r$ is local for the fpqc topology. It remains to show that (\rmnum{1})$\Rightarrow$(\rmnum{2}).\par
For this, we introduce the functor $F$ of \cref{scheme group affine smooth functor of subgroup multiplicative representable}, which is a scheme smooth and separated over $S$, and consider the sub-functor $\mathscr{T}$ of $F$ whose value over $S'\to S$ is the set of maximal tori in $G_{S'}$. We claim that $\mathscr{T}$ is representable by an open subscheme of $F$, hence is smooth and separated over $S$. To see this, we shall use \cref{scheme functor representable by open immersion iff}. We first note that, since tori and their maximality are bith defined fpqc locally and the functor $F$ is a fpqc sheaf, it is easy to see that $\mathscr{T}$ is a fpqc sheaf. Now let $U$ be the set of $x\in F$ such that the canonical monomorphism $\Spec(\kappa(x))\to F$ factors through $\mathscr{T}$. Then any point $x\in U$ (lying over $s\in S$) corresponds to a maximal torus $T_{\kappa(x)}$ of $G_{\kappa(x)}$, which comes from a torus $T$ of $G_{\Spec(\mathscr{O}_{F,x})}$\footnote{Since the canonical monomorphism $\Spec(\kappa(x))\to F$ factors through $\Spec(\mathscr{O}_{F,x})$, $T_{\kappa(x)}$ comes from a subgroup $T$ of $G_{\Spec(\mathscr{O}_{F,x})}$ of multiplicative type. Now since the image of $\Spec(\mathscr{O}_{F,x})$ in $S$ is contained in any neighborhood of $s$ (cf. \cref{scheme local canonical morphism prop}), we see that $T$ is in fact a torus of $G_{\Spec(\mathscr{O}_{F,x})}$.}. Since $T_{\kappa(x)}$ is a maximal torus of $G_{\kappa(x)}$, we see that $T_s$ is a torus in $G_s$, so there exists an open neighborhood $U$ of $s$ such that $T|_U$ is a maximal torus in $G|_U$. But by \cref{scheme local canonical morphism prop}, the inverse image of $U$ in $F$ then contains the image of $\Spec(\mathscr{O}_{F,x})$, so $T$ is in fact a maximal torus of $G_{\Spec(\mathscr{O}_{F,x})}$. In other words, we have proved that the canonical morphism $\Spec(\mathscr{O}_{F,x})\to F$ factors through $\mathscr{T}$. Now since $\mathscr{O}_{F,x}$ is the inductive limit of the rings of affine neighborhoods of $x$ in $F$, we see that the torus $T$ of $G_{\Spec(\mathscr{O}_{F,x})}$ in fact comes from a subgroup $T'$ of multiplicative type of $G|_{U'}$, where $U'$ is an affine open neighborhood of $x$. By possibly shrinking $U'$, we may also assume that $T'$ is a torus in $G|_{U'}$. Again, since $T'$ is a maximal torus at $x$ (hence the image of $\Spec(\mathscr{O}_{F,x})$), we may further assume that $T'$ is a maximal torus of $G|_{U'}$. In this case, for any point $x'\in U'$, the canonical morphism $\Spec(\kappa(x'))\to F$ then factors through $T'$, so we have $U'\sub U$, and hence $U$ is open in $F$. Now following the proof of (\rmnum{2})$\Rightarrow$(\rmnum{1}), we conclude that the open subset $U$ represents $\mathscr{T}$, and hence $\mathscr{T}$ is smooth and separated over $S$. As the structural morphism $\mathscr{T}\to S$ is evidently surjective, it admits locally for the \'etale topology a section over $X$ in view of (\cite{EGA4-4} 17.16.3), and this proves (\rmnum{1})$\Rightarrow$(\rmnum{2}).\par
As for (c), this is an immediate concequence of \cref{scheme smooth affine subgroup multiplicative transporter representable} and (\cite{EGA4-4} 17.16.3), in view of Borel's conjugation theorem\footnote{The transporter of two maximal tori of $G$ is representable by a smooth and separated scheme over $S$, so we can proceed as in the proof of \cref{scheme group affine smooth multiplicative fiber etale lifting}.}. Finally, in view of the remarks in the proof of (a), if $\rho_r$ is locally constant, we can assume that there is a maximal torus $T$ in $G$. If $C$ is its centralizer, then $C$ is representable and is smooth over $S$ by \cref{scheme group affine smooth Weyl group representable}, so the function $s\mapsto\rho_n(s)=\dim(C_s)$ is indeed locally constant.
\end{proof}

\begin{corollary}\label{scheme group affine smooth rho_u zero prop}
Let $G$ be as in \cref{scheme group affine smooth rho_r and rho_n and maximal tori prop} and let $s\in S$ be such that $\rho_u(s)=0$, i.e. $\rho_r(s)=\rho_n(s)$ (that is, the maximal tori in $G_k$ are central, where $k$ is an algebraic closure of $\kappa(s)$). Then there exists an open neighborhood $U$ of $s$ such that $\rho_r$ and $\rho_n$ are constant over $U$, and in particular, for any $t\in U$ the unipotent rank $\rho_u(t)$ of $G_t$ is zero.
\end{corollary}
\begin{proof}
This follows immediately from \cref{scheme group affine smooth rho_r and rho_n and maximal tori prop} and the inequality $\rho_r(t)\leq\rho_n(t)$ for any $t\in S$.
\end{proof}

In the proof of \cref{scheme group affine smooth rho_r and rho_n and maximal tori prop}, we have also proved the following result:

\begin{corollary}\label{scheme group affine smooth maximal tori functor prop}
Let $G$ be as in \cref{scheme group affine smooth rho_r and rho_n and maximal tori prop} and suppose that $G$ is of locally constant reductive rank. Consider the functor
\[\mathscr{T}:\Sch_S^{\op}\to\Set\]
such that for any $S'\to S$, we have
\[\mathscr{T}(S')=\{\text{the set of maximal tori in $G_{S'}$}\}.\]
Then $\mathscr{T}$ is representable by a scheme which is smooth, separated and of finite type over $S$.
\end{corollary}
\begin{proof}
It remains to show that $\mathscr{T}$ is of finite type over $S$, for which we can assume that $G$ admits a maxila torus $T$. By \cref{scheme smooth affine subgroup multiplicative normalizer representable}, $N_G(T)$ and $G/N_G(T)$ are representable by schemes, and $\mathscr{T}$ is isomorphic to $G/N_G(T)$. The morphism $G\to\mathscr{T}$ defined by $g\mapsto\inn(g)(T)$ being surjective, and $G$ quasi-compact over $S$, we see that $\mathscr{T}$ is also quasi-compact over $S$, whence the corollary. 
\end{proof}

The functor $\mathscr{T}$ of \cref{scheme group affine smooth maximal tori functor prop} is called the \textbf{scheme of maximal tori of $G$}. We will see in \ref{scheme group smooth affine subgroup multplicative application paragraph} that it is in fact affine over $S$.

\begin{remark}\label{scheme group affine smooth universal subgroup multiplicative}
As the functor $F$ in \cref{scheme group affine smooth functor of subgroup multiplicative representable} is representable by a scheme $M$ over $S$, by definition, the identity morphism on $M$ corresponds to a subgroup $\widetilde{H}$ of $G_M$, which is called the \textbf{universal subgroup of multiplicative type} of $G_M$. Its name is justified by the following observation: for any subgroup $H$ of multiplicative type of $G$, which corresponds to an $S$-section $\sigma:S\to M$, consider the following commutative diagram
\[\begin{tikzcd}
F(S)\ar[d,swap,"\sim"]\ar[r]&F(M)\ar[d,"\sim"]\\
M(S)\ar[r]&M(M)
\end{tikzcd}\]
As the image of $\sigma\in M(S)$ in $M(M)$ is the identity (it is a section), we conclude that the image of $H\in F(S)$ in $F(M)$ is equal to $\widetilde{H}$, i.e. the inverse image of $H$ under the morphism $G_M\to G$ equals to $\widetilde{H}$. On the other hand, since $\sigma$ is a section of $M$ over $S$, it induces a section $G_M$ over $G$, and we then see that the inverse image of $\widetilde{H}$ under this section is equal to $H$, that is, \textit{any subgroup of $H$ of multiplicative type can be obtained from $\widetilde{H}$}.\par
More generally, for $S'\to S$, we see that a section $\sigma:S'\to M$ (which corresponds to a subgroup $H$ of multiplicative type of $G_{S'}$) induces a commutative diagram with Cartesian squares:
\[\begin{tikzcd}
H\ar[r]\ar[d,hook]&\widetilde{H}\ar[d,hook]&\\
G_{S'}\ar[d]\ar[r]&G_M\ar[d]\ar[r]&G\ar[d]\\
S'\ar[r]&M\ar[r]&S
\end{tikzcd}\]
If the section $\sigma$ factors through $\widetilde{H}$, i.e. $\sigma$ comes from an element of $\widetilde{H}(S')$, then we conclude from the above diagram that we obtain a section of $H$ over $S'$. Therefore, the subscheme $\widetilde{H}\sub G_M$ represents the following functor:
\begin{equation*}
\widetilde{H}(S')=\left\{
    \parbox{4in}{%
    set of couples $(H,\sigma)$, where $H$ is a subgroup of multiplicative type of $G_{S'}$, and $\sigma$ is a section of $H$ over $S'$%
    }
\right\}.
\end{equation*}

Now assume that $G$ is of constant reductive rank $r$, so the functor $\mathscr{T}$ is representable (by an open subscheme of $G$). Let $H$ be a subgroup of multiplicative type of $G$ and $s\in S$, such that $H_s$ is a torus of $G_s$ of relative dimension $r$ (a maximal torus). Then the fiber $H_s$ corresponds to a morphism $\Spec(\kappa(s))\to M$, and we have a commutative diagram
\[\begin{tikzcd}
F(S)\ar[d,"\sim"]\ar[r]&F(M)\ar[d,"\sim"]\ar[r]&F(\Spec(\kappa(s)))\ar[d,"\sim"]\\
M(S)\ar[r]&M(M)\ar[r]&M(\kappa(s))
\end{tikzcd}\]
so the fiber of $\widetilde{H}$ along this morphism is equal to $H_s$. In other words, the underlying subspace of $\mathscr{T}$ consists of points $x\in M$ such that $\widetilde{H}_x$ is a torus of $(G_M)_x$ of relative dimension $r$.
\end{remark}

We now conclude this paragraph by giving some examples where maximal tori exsit.

\begin{proposition}\label{scheme group multiplicative fy unique maximal torus}
Let $G$ be an $S$-group of multiplicative finite type over $S$. Then $G$ admits a unique maximal torus, and any torus of $G$ is contained in this maximal torus. 
\end{proposition}
\begin{proof}
The uniqueness follows clearly from the last assertion, which characterizes the maximal torus as the largest torus of $G$. From the uniqueness, we see that the question of existence is local for the fpqc topology, which allows us to suppose that $G$ diagonalizable, i.e. of the form $D_S(M)$, $M$ a finitely generated abelian group finite type. Let $M_0$ be the quotient of $M$ by its torsion subgroup, then the torus $T=D_S(M_0)$ in $G$ is maximal and the greatest sub-torus. Indeed, a sub-torus $T'$ of $G$ is locally diagonalizable for the fpqc topology, so to prove that $T'\sub T$, we can suppose that $T'$ is diagonalizable, therefore of the form $D_S(N)$, where $N$ is a free quotient of $M$, hence $N$ is a quotient of $M_0$. Since the construction of $T$ as $D_S(M_0)$ is compatible with base change, this shows at the same time that $T$ is a maximal torus of $G$, and completes the proof. 
\end{proof}

\begin{proposition}\label{scheme group fp fpqc local maximal central torus prop}
Let $G$ be an $S$-group of finite presentation over $S$. Suppose that $G$ admits fpqc locally a central maximal torus, then it admits (globally) a unique maximal torus, and this is the largest torus of $G$. 
\end{proposition}
\begin{proof}
The uniqueness follows from the last assertion, so the question is fpqc local and we can assume that $G$ admits a central maximal torus $T$. Then any torus $R$ of $G$ is contained in $T$, in view of \cref{scheme group fp torus commutes with maximal torus inclusion} below.
\end{proof}

\begin{lemma}\label{scheme group fp torus commutes with maximal torus inclusion}
Let $G$ be an $S$-group of finite presentation over $S$ and $T$ be a maximal torus of $G$. If $R$ is a sub-torus of $G$ and $R$ commutes with $T$, then $R\sub T$.
\end{lemma}
\begin{proof}
As $R$ commutes with $T$, the morphism $R\times_ST\to G$ defined by $(r,t)\mapsto rt$ is a homomorphism of groups, hence it admits an image subgroup $T'$ in $G$ (\cref{scheme group morphism from multiplicative factorization}), which is a quotient group of multiplicative type of $R\times_ST$, hence a torus, and contains $T$. As $T$ is a maximal torus, we then have $T'=T$, hence $R\sub T$.
\end{proof}

\begin{corollary}\label{scheme group smooth abelian affine rho_r constant maximal torus}
Let $G$ be an abelian $S$-group, smooth and affine over $S$ with locally constant reductive rank. Then $G$ admits a unique maximal torus, and it contains any sub-torus of $G$.
\end{corollary}
\begin{proof}
This follows from \cref{scheme group fp fpqc local maximal central torus prop} and \cref{scheme group affine smooth rho_r and rho_n and maximal tori prop}~(b).
\end{proof}

\begin{corollary}\label{scheme group affine smooth rho_r constant maximal torus if fiber nilpotent}
Let $G$ be a smooth and affine group over $S$. Suppose that for any $s\in S$, denote by $\bar{s}$ the spectrum of an algebraic closure $k$ of $\kappa(s)$, the geometric fiber $G_{\bar{s}}$ is a connected and nilpotent algebraic group. Suppose further that the reductive rank of $G$ is locally constant, then $G$ admits a unique maximal torus $T$, and $T$ is central and is the largest sub-torus of $G$.
\end{corollary}
\begin{proof}
In view of \cref{scheme group affine smooth rho_r and rho_n and maximal tori prop}~(b), $G$ admits fpqc locally a maximal torus, so by \cref{scheme group fp fpqc local maximal central torus prop} we are reduced to prove that any maximal torus of $G$ is central. In view of \cref{scheme group multiplicative morphism central at fiber prop}~(b), we can assume that $S$ is the spectrum of an algebraically closed field. Then $T(k)$ is in the center of $G(k)$ by (\cite{Chevalley1958} 6 th.2), which implies that $T$ is in the center of $G$, bacause then $Z_G(T)$ is a closed subscheme of $G$ which contains the points of $G(k)$, and hence by Nullstellensatz is equal to $G$.
\end{proof}

\paragraph{The Weyl group}
We first consider an algebraically closed field $k$ and let $G$ be an algebraic group over $k$, smooth and affine over $k$. If $T$ is a maximal torus of $G$, $C$ is its centralizer and $N$ the normalizer, then in view of \cref{scheme smooth affine subgroup multiplicative normalizer representable} and \cref{scheme group affine smooth Weyl group representable}, these are closed smooth subgroups of $G$, and $C$ is an open subgroup of $N$, so that $W=N/C$ is a finite \'etale group over $k$, hence determined by the group $W(k)$ of closed points of $k$ (in fact, $W=W(k)_k$, the constant $k$-group associated with $W(k)$). The finite (ordinary) group $W(k)$ is called the \textbf{geometric Weyl group} (or simply the \textbf{Weyl group}) \textbf{of $G$ relative to $T$}. In view of Borel's conjugation theorem, the Weyl groups relative to different maximal tori are all isomorphic, so we can say that Weyl group of $G$ without specifying the maximal torus. As the formation of $C$, $N$ and $N/C$ commutes with arbitrary base extension, we see that if $k'$ is algebraically closed field containing $k$, the geometric Weyl group of $G_{k'}$ relative to $T_{k'}$ is canonically isomorphic to that of $G$ relative to $T$; therefore, the geometric Weyl group of $G$ coincides with that of $G_{k'}$.\par
This allows us to define the geometric Weyl group of an algebraic group $G$, smooth and affine over an arbitrary field $k$, to be that of $G_{k'}$ for $k'$ an algebraic closure of $k$. If $G$ admits a maximal torus $T$, then we can evidently form $C=Z_G(T)$, $N=N_G(T)$, and $W=N/C$, which is a finite \'etale group over $k$, called the \textbf{geometric Weyl group relative to $T$}. This is then nothing other than the group of points of $W$ with values in an arbitrary algebraically closed extension $k'$ of $k$. However, the knowledge of the geometric Weyl group $W(k')$ is obviously not sufficient in general, to reconstruct the algebraic group $W$: it is also necessary to know the operation of the Galois group $\Gal(k'/k)$ on $W(k')$.\par
If finally $G$ is a group scheme over an arbitrary base $S$, $G$ is smooth and affine over $S$, and if $T$ is a maximal torus of $G$, then \cref{scheme group affine smooth Weyl group representable} allows us to define the group
\[W(T)=N_G(T)/Z_G(T)\]
which is an \'etale $S$-group, separated and quasi-finite over $S$. Its geometric fiber (relative to an algebraic closure of the residue field $\kappa(x)$, $s\in S$) is then the geometric Weyl groups of the fiber $G_s$. As a result, \cref{scheme group affine smooth Weyl group fiber rank semicontinuous} gives us information on the variation of these groups with $s\in S$. We can specify and generalize this information as follows:

\begin{theorem}\label{scheme group affine smooth Weyl group variation prop}
Let $S$ be a scheme, $G$ be an $S$-group, smooth and affine over $S$. For any $s\in S$, let $w(s)$ be the geometric Weyl group of $G_s$, which is an isomorphism class of finite groups. In the set $E$ of isomorphism classes of finite groups, we introduce the following order: $w\leq w'$ if $w$ and $w'$ are represented by finite groups $W$ and $W'$, respectively, such that $W$ is isomorphic to a quotient of $W'$.
\begin{enumerate}
    \item[(a)] The function $s\mapsto w(s)$ from $S$ to $E$ is lower semi-continuous.
    \item[(b)] Suppose that the reductive rank of $G$ is locally constant. Then the following conditions are equivalent:
    \begin{enumerate}
        \item[(\rmnum{1})] The function $s\mapsto w(s)$ is locally constant.
        \item[(\rmnum{2})] The function $s\mapsto\Card(w(s))$ is locally constant.
        \item[(\rmnum{3})] There exists, locally for the \'etale topology, a maximal torus $T$ of $G$ such that $W(T)$ is finite over $S$.
        \item[(\rmnum{4})] For any $S'\to S$ and any maximal torus $T$ of $G_{S'}$, the Weyl group $W(T)$ is finite over $S'$.
    \end{enumerate}
\end{enumerate}
\end{theorem}

Preceding as in \cref{scheme group affine smooth rho_r and rho_n and maximal tori prop}~(a), we are reduced to prove that for any $s\in S$, there exists an open neighborhood $U$ of $s$ such that $t\in U$ implies $w(t)\geq w(s)$, provided that threre exists a torus $R$ in $G$ such that $R_s$ is a maximal torus of $G_s$. Let $W(R)=N_G(R)/Z_G(R)$ be as in \cref{scheme group affine smooth Weyl group representable}, which is a \'etale group scheme, separated and quasi-finite over $S$. For any $t\in S$, let $w'(t)\in E$ be the geometic fiber of $W(R)$ at $t$. As $R_s$ is a maximal torus in $G_s$ and the formation of $N_G$, $Z_G$ and $N_G/Z_G$ are compatible with base change, we see that
\[w(s)=w'(s);\]
We further assert that $w(t)\geq w'(t)$ and $w'(t)\geq w'(s)$ for $t$ in a neighborhood of $s$, and this proves (a). These two inequalities are contained in the following two lemmas.

\begin{lemma}\label{scheme group etale quasi-finite fiber type lower semicontinuous}
Let $S$ be a scheme, $W$ be an $S$-group which is \'etale, separated and quasi-finite over $S$. For any $s\in S$, let $f(s)$ be the class of the geometric fiber of $W$ at $s$, which is an element of the ordered set $E$ of isomorphism classes of finite groups. Then the function $f:S\to E$ is lower semi-continuous. For it to be constant in a neighborhood of $s\in S$, it is necessary and sufficient that so is the function $s\mapsto\Card(f(s))$, and for this it is necessary and sufficient that $W$ is finite over $S$ on an open neighborhood $U$ of $s$.
\end{lemma}
\begin{proof}
This result is a refinement of the fact invoked in the proof of \cref{scheme group affine smooth Weyl group fiber rank semicontinuous}, we confine ourselves to a sketch of the demonstration, see also (\cite{EGA4-3} 15.5.1) and (\cite{EGA4-4}, 18.10.7). We can again assume that $S$ is affine and Noetherian, and hence the function $f$ is constructible (\cite{EGA3-1} $0_{\Rmnum{3}}$, 9.3.1 et 9.3.2). In view of (\cite{EGA3-1} $0_{\Rmnum{3}}$, 9.3.4), we only need to prove that if $t$ is a generalization of $s$, then $f(t)\geq f(s)$. Then thanks to \cref{scheme morphism from valuation specturm char}, we can reduced to the case where $S$ is the spectrum of a discrete valuation ring, which we can assume to be complete with algebraically closed residue field. But then, as $G$ is \'etale and separated over $S$, it contains a clopen subscheme $G'$, finite over $S$, such that $G_s'=G_s$ (\cref{scheme qf sp over complete Noe locla sum}), and we see immediately that $G'$ is a subgroup of $G$ in this case. Moreover, as $G'$ is \'etale and finite over $S=\Spec(R)$, with $R$ complete with algebraically closed residue field, it follows from (\cite{EGA4-4} 18.8.1) that $G$ is a constant group, hence of the form $A_S$, where $A=G'(\kappa(s))=G(\kappa(s))$ has class $f(s)$. If $B$ is the geometric fiber of $G$ at the generic point $t$ of $S$, we then have a canonical monomorphism $A\to B$, and this proves $f(s)\leq f(t)$. The fact that $f$ is constant near $s$ (or equivalently, $\Card(f)$ is constant near $s$) if and only if $W$ is finite over $S$ in a neighborhood of $s$ follows from \cref{scheme morphism over local Noe finite iff affine proper} and (\cite{EGA4-3} 15.5.1). 
\end{proof}

\begin{lemma}\label{scheme alg group affine Weyl group of inclusion torus prop}
Let $G$ be an affine smooth algebraic group over an algebraically closed field $k$, $R\sub T$ be two sub-torus of $G$, $W(R)$ and $W(T)$ be the finite groups relative to $R$ and $T$. Then $W(R)$ is isomorphic to a quotient of $W(T)$.
\end{lemma}
\begin{proof}
We consider the following diagram:
\[\begin{tikzcd}
T\ar[r,hook]&C(T)\ar[r,hook]\ar[d,hook]&C(R)\ar[d,hook]\\
&N(T)\cap N(R)\ar[d,hook]\ar[r,hook]&N(R)\\
&N(T)
\end{tikzcd}\]
Then $(N(T)\cap N(R))/C(T)$ is a subgroup of $W(T)=N(T)/C(T)$, and we have an evident homomorphism
\[(N(T)\cap N(R))/C(T)\to W(R)=N(R)/C(R),\]
so it remains to show that this is surjective, i.e. for any point $g$ of $N(R)$ with values in $k$, there exists a point $c$ of $C(R)$ with values in $k$ such that $cg$ normalizes $T$, that is, such that
\[\inn(c)(\inn(g)T)=T.\]
For this, it suffices to note that $\inn(g)T$ is a torus of $N(R)$, hence of $C(R)$ (which is an open subgroup). Then $T$ and $\inn(g)T$ are two maximal tori of $C(R)$, since they are maximal in $G$, and we conclude by Borel's conjugation theorem.
\end{proof}

We therefore conclude assertion (a) of \cref{scheme group affine smooth Weyl group variation prop}. As for (b), we have already pointed out that (\rmnum{1}) and (\rmnum{2}) are trivially equivalent, and they imply (\rmnum{4}) according to the converse of \cref{scheme group etale quasi-finite fiber type lower semicontinuous}. On the other hand, (\rmnum{4})$\Rightarrow$(\rmnum{3}) thanks to \cref{scheme group affine smooth rho_r and rho_n and maximal tori prop}~(b) and \cref{scheme group etale quasi-finite fiber type lower semicontinuous}. Finally, (\rmnum{3})$\Rightarrow$(\rmnum{2}), because we have seen in \cref{scheme group affine smooth rho_r and rho_n and maximal tori prop}~(a) that condition (\rmnum{2}) is local for fpqc topology, which allows us to assume that $G$ admits a maximal torus $T$ such that $W(T)$ is finite over $S$, and we conclude again using \cref{scheme group etale quasi-finite fiber type lower semicontinuous}. This completes the proof of \cref{scheme group affine smooth Weyl group variation prop}.

\paragraph{Cartan subgroups}
\begin{definition}\label{scheme group smooth ft Cartan subgroup def}
Let $G$ be a group scheme smooth and of finite type over a scheme $S$. A subgroup $C$ of $G$ is called a \textbf{Cartan subgroup} of $G$ if it is smooth over $S$ and such that for any $s\in S$, if $\bar{s}$ denotes the spectrum of an algebraic closure of $\kappa(s)$, then $C_{\bar{s}}$ is a Cartan subgroup of $G_{\bar{s}}$.
\end{definition}
It is immediate that if $C$ is a Cartan subgroup of $G$, then for any $S'$ over $S$, $C_{S'}$ is also a Cartan subgroup of $G_{S'}$. We also note that the fact for a subgroup $C$ of $G$ to be a Cartan subgroup is local for the fpqc topology.

\begin{theorem}\label{scheme smooth affine ft maximal tori and Cartan subgroup correspond}
Let $G$ be smooth group scheme, affine and of finite type over $S$, with locally constant reductive rank. Then the map
\[T\mapsto Z_G(T)\]
induces a bijection from the set of maxmal tori of $G$ to that of Cartan subgroups of $G$\footnote{The proof given here in fact only proves the theorem for closed Cartan subgroups. However, (\cite{SGA3-2} \Rmnum{12} 7.1(a)) proves the theorem in the given form and shows that any Cartan subgroup of $G$ is in fact closed.}. If $C$ corresponds to $T$, then $T$ is the unique maximal torus of $C$.
\end{theorem}
\begin{proof}
Let $T$ be a maximal torus of $G$, the functor $Z_G(T)$ is representable by a closed and smooth subscheme of $G$ (\cref{scheme group affine smooth Weyl group representable}), $C=Z_G(T)$, and it follows by definition that $C$ is a Cartan subgroup of $G$. Moreover, $T$ is evidently a maximal torus in $C$, and being central in $C$, it is the unique maximal torus of $C$ (\cref{scheme group fp fpqc local maximal central torus prop}). Hence the map $T\mapsto Z_G(T)$ is injective.\par
To see it is surjective, let $C$ be a Cartan subgroup of $G$. It suffices to find a maximal torus $T$ of $C$, because then $T$ is a maximal torus of $G$ (for any $s\in S$, $C_s$ and $G_s$ have the same reductive rank), and in view of (\cite{SGA3-1} \Rmnum{9} 5.6(b)), $T$ is in the center of $C$, hence $C\sub C'=Z_G(T)$. But $C$ is a smooth subgroup of the smooth group $C'$ over $S$ and coincides with $C'$ fiber by fiber, whence $C=C'$. Now since $G$ is locally of constant reductive rank, so is $C$, and hence $C$ admits locally a maximal torus (for the fpqc topology), in view of \cref{scheme group affine smooth rho_r and rho_n and maximal tori prop}~(b). As this torus is central by the preceding arguments, it follows from \cref{scheme group fp fpqc local maximal central torus prop} that $C$ admits a maximal torus, whence the assertion.
\end{proof}

\begin{corollary}\label{scheme smooth affine functor of Cartan subgroup representable if constant reductive rank}
Let $G$ be a group scheme smooth and affine over $S$ with locally constant reductive rank. Consider the functor $\mathscr{C}:\Sch_{/S}^{\op}\to\Set$ defined by
\[\mathscr{C}(S')=\{\text{the set of Cartan subgroups of $G_{S'}$}\},\]
then $\mathscr{C}$ is isomorphic to the functor $\mathscr{T}$ of \cref{scheme group affine smooth maximal tori functor prop}, hence is representable by a scheme which is smooth, separated and of finite type over $S$.
\end{corollary}

\begin{corollary}\label{scheme smooth affine ft normalizer of torus and Cartan equal}
Under the conditions of \cref{scheme smooth affine ft maximal tori and Cartan subgroup correspond}, if $C=Z_G(T)$, we have
\[N_G(C)=N_G(T).\]
\end{corollary}
\begin{proof}
In fact, for $S'\to S$ and $g\in G(S')$, if $\inn(g)(C)\sub C$, then $\inn(g)$ fixes the unique maximal torus of $C_{S'}$, which is $T_{S'}$, so we have $g\in N_G(T)(S')$. The converse follows from definition: if $g\in N_G(T)(S')$, then $\inn(g)(T)=T$, and for any $t\in T(S')$, $c\in C(S')$, we have
\[\inn(g)(c)\cdot\inn(g)(t)=\inn(g)(ct)=\inn(g)(tc)=\inn(g)(t)\cdot\inn(g)(c)\]
so $\inn(g)(C)\sub C$, and thus $g\in N_G(C)$.  
\end{proof}

\subsection{The reductive center}
\begin{definition}\label{scheme group fp affine fiber reductive center def}
Let $S$ be a scheme, $G$ be an $S$-group of finite presentation over $S$ with affine fibers, and $Z$ be a subgroup of $G$. We say that $Z$ is a \textbf{reductive center} of $G$ if
\begin{enumerate}
    \item[(\rmnum{1})] $Z$ is central in $G$ and of multiplicative finite type over $S$,
    \item[(\rmnum{2})] for any morphism $S'\to S$ and any central homomorphism $u:H\to G_{S'}$, where $H$ is a group of multiplicative finite type over $S$, $u$ factors through $Z_{S'}$.
\end{enumerate}
\end{definition}
We note that such a subgroup $Z$ is uniquely determined as the largest central subgroup of $G$ of multiplicative fintie type over $S$. It is easy to give examples where $G$ (smooth and affine over $S$) admits a largest central subgroup $Z$ of multiplicative finite type over $S$, but where $Z$ is not a reductive center (cf. \cref{scheme group no maximal torus example}). In fact, it follows from \cref{scheme group morphism from multiplicative factorization} that a subgroup $Z$ of $G$ is a reductive center if and only if it is a largest central subgroup of multiplicative type and preserves this property under any base change.\par
It is evident that if $Z$ is a reductive center of $G$, then for any base change $S'\to S$, $Z_{S'}$ is the reductive center of $G_{S'}$. By fpqc descent (\cite{SGA1} \Rmnum{8}) and the uniqueness of reductive center, we see that the existence of the reductive center of $G$ is local for the fpqc topology.

\begin{proposition}\label{scheme group fp reductive center iff fiberwise}
Let $G$ be an $S$-group of finite presentation with affine fibers, $Z$ be a subgroup of $G$. For $Z$ to be a reductive center of $G$, it is necessary and sufficient that it is of multiplicative type and for any $s\in S$, $Z_s$ is a reductive center of $G_s$.
\end{proposition}
\begin{proof}
The condition is clearly necessary. Conversely, let $Z$ be such that $Z_s$ is a reductive center of $G_s$ for any $s\in S$. It then follows from \cref{scheme group multiplicative morphism central at fiber prop} that $Z$ is central in this case. As the considered properties are invariant under base change, it remains to show that any central homomorphism $u:H\to G$, with $H$ of multiplicative and fintie type, factors through $Z$. Now since $Z$ is central, the canonical immersion $Z\to G$ and $u$ define a group homomorphism
\[w:H\times_SZ\to G.\]
In view of \cref{scheme group morphism from multiplicative factorization}, this admits a image group $K$, which is a subgroup of multiplicative type of $G$, and we must show that $K=Z$. But this can be justified fiberwise using the hypothesis on $Z$, and then apply \cref{scheme subgroup multiplicative coincidence locus prop}.
\end{proof}

We also note that in \cref{scheme group fp reductive center iff fiberwise}, the hypothesis that for any $s\in S$, $Z_s$ is a reductive center of $G_s$ is in fact purely geometric, i.e. it suffices to verify this over the algebraic closure of $\kappa(s)$ (as it is local for the fpqc topology).

\begin{theorem}\label{scheme alg group affine reductive center exist}
Let $G$ be an affine algebraic group over a field $k$. Then $G$ admits a reductive center.
\end{theorem}
\begin{proof}
As the center of $G$ is represented by a closed subgroup of $G$ (cf. \cref{scheme alg group center is closed and quotient affine}), we are easily reduced to the case where $G$ is abelian. Moreover, since the considered property is fpqc local, we may asume that $k$ is algebraically closed. 

\begin{lemma}\label{scheme group multiplicative morphism to G_a trivial}
Let $H$ be an $S$-group of multiplicative type, then any group homomorphism $u:H\to\G_a$ is trivial.
\end{lemma}
In fact, as a faithfully flat and quasi-compact morphism is effective descent for affine morphisms, it suffices to consider the case where $H=D_S(M)$ is diagonalizable. Consider the module $E=\mathscr{O}_S^2$ as an extension of $E'=\mathscr{O}_S$ by $E''=\mathscr{O}_S$. Then $\G_a$ is identified with the scheme of automorphisms of this extension, hence a homomorphism $u:H\to\G_a$ is identified with an $H$-module structure over $E$ respecting this extension, i.e. such that $E'$ is stable under the action of $H$ and that the operation induced by $H$ on $E''$ is trivial. In view of \cref{scheme module over diagonalizable group cat equivalent to graded module}, if $E=\bigoplus_mE_m$ is the corresponding graduation of $E$, this implies that $E_m=E_m\cap E'$ for $m\neq 0$, so $H$ acts trivially on $E$, hence $u$ is trivial. 
\end{proof}

\begin{remark}
If $G$ is a nontrivial abelian variety over $k$, then $G$ does not admits reductive center in the sense of \cref{scheme group fp affine fiber reductive center def}, where we omit the "affine" restriction. In fact, for $n$ coprime to the characteristic of $k$, ${_nG}$ is \'etale over $k$ with order coprime to $\char(k)$, hence is of multiplicative type. But the family $({_nG})$ is schematically dense in $G$, so if there exists a reductive center, then it must be equal to $G$, which is absurd.
\end{remark}

\begin{lemma}\label{scheme multiplicative to smooth affine central factor by maximal torus}
Let $S$ be a scheme, $G$ be an affine smooth group over $S$, with connected fibers. Let $T$ be a maximal torus of $G$ and $u:H\to G$ be a central homomorphism, with $H$ a group of multiplicative finite type over $S$. Then $u$ factors through $T$.
\end{lemma}
\begin{proof}
Let $C$ be the centralizer of $T$, which is a closed subscheme of $G$ and smooth over $S$ (\cref{scheme group affine smooth Weyl group representable}), hence affine over $S$. As $T$ is contained in the center of $C$, it is normal, and we can consider the quotient group $C/T=U$, which is representable (\cref{scheme group mono multiplicative to affine prop}). Now as $u$ is central, it factors through $C$ (its image commutes with $T$), and it remains to prove that the composition homomorphism $H\to C\to U=C/T$ is trivial. In view of \cref{scheme group multiplicative morphism central at fiber prop}, we may reduce to the case where $S$ is the spectrum of a field, which we may assume to be algebraically closed. In this case, by (\cite{Chevalley1958} 6, th.2), $U$ is a connected unipotent algebraic group (smooth over $k$), which signifies that it admits a composition series with quotients isomorphic to $\G_a$. Therefore any homomorphism from a group of multiplicative finite type $H$ to $U$ is trival, whence our assertion. 
\end{proof}

\begin{corollary}\label{scheme smooth affine reductive center in any maximal torus}
Let $G$ be an affine smooth group over $S$ with connected fibers. If $G$ admits a reductive center, then it is contained in any maximal torus of $G$.
\end{corollary}
\begin{proof}
This follows from \cref{scheme multiplicative to smooth affine central factor by maximal torus} and condition (b) in \cref{scheme group fp affine fiber reductive center def}.
\end{proof}

\begin{theorem}\label{scheme smooth affine connected fiber reductive center exist if}
Let $S$ be a scheme, $G$ be an affine smooth group over $S$ with connected fibers.
\begin{enumerate}
    \item[(a)] For any $s\in S$, let $z(s)$ be the type of the reductive center of $G_s$ (exists by \cref{scheme alg group affine reductive center exist}). Let $E$ be the ordered set of isomorphic classes of finitely generated $\Z$-modules, which $M\leq M'$ if and only if $M$ is isomorphic to a quotient of $M'$. Then the map $S\to E,s\mapsto z(s)$ is lower semi-continuous.
    \item[(b)] For $G$ to admit a reductive center $Z$, it is necessary and sufficient that the function $z:S\to E$ in (a) is locally constant. In this case, $G/Z$ is representable (cf. \cref{scheme group multiplicative free action quotient prop}), and $G/Z$ admits the trivial subgroup as its reductive center.
    \item[(c)] Suppose that the reductive rank of $G$ is locally constant. Then $G$ admits a reductive center $Z$, and the maximal tori $T$ of $G/Z$ (resp. Cartan subgroups $C$ of $G$) correspond bijectively to the maximal tori $T'$ of $G'=G/Z$ (resp. Cartan subgroups $C'$ of $G$) via $T'=T/Z$ (resp. $C'=C/Z$).
    \item[(d)] Let $T$ be a maximal torus of $G$, $\g=\mathfrak{Lie}(G)$ be the Lie algebra of $G$, and consider the homomorphism
    \[\theta:T\to\GL(\g)\]
    induced by the adjoint representation of $G$. Then the kernel of $\theta$ is a reductive center of $G$.
\end{enumerate}
\end{theorem}

Let us give a useful translation of (d), in the case where $T$ is diagonalizable, hence of the form $D_S(M)$ where $M$ is a free $\Z$-module of finite rank. Then by \cref{scheme module over diagonalizable group cat equivalent to graded module}, the $T$-module $\g$ admits a graduation of type $M$:
\[\g=\bigoplus_{m\in M}\g^m\]
where $\g^m$ are sub-$T$-modules of $\g$ (which are necessarily locally free, since $G$ is smooth over $S$). Suppose that for any $m\in M$, the rank of $\g^m$ is constant, then the set $R$ of elements $m\in M$ such that $\g^m\neq 0$ (roots of $T$) is finite. We then conclude the following corollary:

\begin{corollary}\label{scheme smooth affine diagonalizable maximal torus reductive center char}
Under the conditions of \cref{scheme smooth affine reductive center in any maximal torus}~(d), the reductive center of $G$ is the intersection of kernels of roots $\alpha\in R$. We then have an isomorphism
\[Z\cong D_S(N)\]
where $N$ is the quotient of $M$ by the subgroup generated by $R$.
\end{corollary}

\begin{corollary}\label{scheme alg group reductive center trivial and nilpotent Lie alg is unipotent}
Let $G$ be an algebraic group over an algebraically closed field $k$. Suppose that $G$ is smooth, connected, affine, with trivial reductive center, and that the Lie algebra $\g$ of $G$ is nilpotent. Then $G$ is unipotent, i.e. admits a composition series with quotients isomorphic to $\G_a$.
\end{corollary}
\begin{proof}
In view of (\cite{Chevalley1958} 6, th.4, cor.3), it suffices to show that any maximal torus $T$ of $G$ is reduced to the trivial group, or equivalently, that the Lie algebra $\t$ of $T$ is reduced to $0$. But this follows from the fact that the $T$-module $\g$ decomposes according to roots $\alpha$ of $T$, that for any $t\in\t$, the operator $\ad_{\g}(t)$ on $\g$ is semi-simple ($T$ is semisimple). As $\g$ is nilpotent and $\t\sub\g$, this then implies that $\ad_{\g}(t)=0$. Now in view of \cref{scheme smooth affine connected fiber reductive center exist if}~(d), as the reductive center of $G$ reduces to the neutral element, the homomorphism $T\to\GL(\g)$ is a monomorphism, and therefore induces an injective map on Lie algebras, which implies that (\cref{scheme O_S-module good Lie of sAut isomorphism}) for $t\in\t$, the relation $\ad_{\g}(t)=0$ implies $t=0$. This proves that $t=0$ and completes
the demonstration.
\end{proof}

\begin{proposition}\label{scheme alg group reductive center is intersection of maximal tori}
Let $G$ be an affine smooth connected algebraic group over an algebraically closed field $k$. Then the reductive center of $G$ is the intersection of maximal tori of $G$.
\end{proposition}
\begin{proof}
Of course, this is the intersection in the schematic sense (or equivalently, in the sense of subfunctors of $G$), i.e. the largest closed subscheme of $G$ dominated by the maximal tori of $G$. As $G$ is Noetherian, this is also the intersection of a suitable finite set of maximal tori of $G$.\par
Let $Z$ be this intersection, which is a closed subgroup of a torus, hence is of multiplicative type (cf. \cref{scheme k-group multiplicative quotient subgroup is multiplicative}). By \cref{scheme multiplicative to smooth affine central factor by maximal torus}~(d), $Z$ contains the reductive center of $G$; to prove the equality, it remains to prove that it is central. Since $G$ is connected, it suffices in view of \cref{scheme group multiplicative normal subgroup is central} to prove that $Z$ is normal. But by construction $Z$ is invariant under the $\Int(g)$, with $g\in G(k)$, hence the normalizer $N=N_G(Z)$ is a closed subgroup of $G$ which contains the rational points of $G$. As $G$ is reduced, we then have $N=G$, which completes the proof.
\end{proof}

\begin{proposition}\label{scheme smooth affine zero unipotent rank reductive center is center}
Let $S$ be a scheme and $G$ be an affine smooth group over $S$ with connected fibers and of zero unipotent rank. Let $\mathscr{T}$ be the scheme of maximal tori of $G$, which is smooth, separated and of finite type over $S$ (cf. \cref{scheme group affine smooth rho_r and rho_n and maximal tori prop} and \cref{scheme group affine smooth maximal tori functor prop}). Let $G$ acts on $\mathscr{T}$ by inner automorphisms, so we have a homomorphism of group functors
\[u:G\to\sAut_S(\mathscr{T}).\]
Then the following subfunctors of $G$ are identical:
\begin{enumerate}
    \item[(\rmnum{1})] The reductive center $Z$ of $G$.
    \item[(\rmnum{2})] The center $Z(G)$ of $G$.
    \item[(\rmnum{3})] The kernel $Z'$ of the homomorphism $u$.
\end{enumerate}
In particular, the center of $G$ is representable by a subgroup of multiplicative type of $G$.
\end{proposition}
\begin{proof}
It is clear that we have $Z\sub Z(G)\sub Z'$, so it remains to show that $Z'\sub Z$, which amounts to saying that (the hypothesis being stable under base change) any section $g\in G(S)$ acting trivially on $\mathscr{T}$ is a section of $Z$. Putting $G'=G/Z$ and utilizing \cref{scheme smooth affine connected fiber reductive center exist if}~(b) and (c), which imply in particular that the scheme $\mathscr{T}'$ of maximal tori of $G'$ is canonically isomorphic to $\mathscr{T}$, we are reduced to the case where $G=G'$, i.e. where the reductive center of $G$ is trivial (note that in view of \cref{scheme smooth affine connected fiber reductive center exist if}~(c), the unipotent rank of $G'$ is equal to that of $G$, so is zero). It is then necessary to prove in this case that $g$ is the unit section of $G$. The usual proceedure of passing to limit reduces us to the case where $S$ is Noetherian, and similarly to the case where $S$ is local Artinian. In this case, the kernel $Z'$ of $u$ is representable (\cref{scheme morphism essentialy free base change prop}~(a) and \cref{scheme subfunctor Weil restriction example}~(c)), so to show that $Z'$ reduces to the identity, it suffices in view of Nakayama's lemma to prove this for the fiber $Z'_0$. This then reduces us to the case where $S$ is the spectrum of a field $k$, which we can evidently assume to be algebraically closed. Now $Z'$ is contained in the stabilizer of any point of $\mathscr{T}(k)$, i.e. in the normalizer $N_G(T)$ of any maximal torus $T$ of $G$. As the unipotent rank of $G$ is zero, by \cref{scheme group affine smooth Weyl group representable}, $T=Z_G(T)$ is an open subgroup of $N_G(T)$, hence the Lie algebra of $N_G(T)$ is identified with that of $T$, and the Lie algebra of $Z'$ is contained in $T$. On the other hand, it follows from \cref{scheme alg group reductive center is intersection of maximal tori} that the intersection of the Lie algebras of maximal tori $T$ of $G$ is none other than the Lie algebra of the reductive center $Z$, so it is zero, since we have supposed $Z=0$. Therefore, the Lie algebra of $Z'$ is zero, i.e. $Z'$ is \'etale over $k$. Furthermore, $Z'$ is evidently normal in $G$, and as $G$ is connected, it follows easily that $Z'$ is contained in the center of $G$, and hence in $T=Z_G(T)$ for any maximal torus $T$. We then conclude that $Z'$ is contained in the intersection of maximal tori of $G$, which is zero by \cref{scheme alg group reductive center is intersection of maximal tori}, and this completes the proof.
\end{proof}

\paragraph{Application to subgroups of multiplicative type}\label{scheme group smooth affine subgroup multplicative application paragraph}
\begin{theorem}\label{scheme smooth affine functor of subgroup multiplicative n-torsion prop}
Let $S$ be a scheme, $G$ be an affine smooth $S$-group, $M$ be the scheme of subgroups of multiplicative type of $G$ (cf. \cref{scheme group affine smooth functor of subgroup multiplicative representable}). For any integer $n>0$, let $T_n$ be the sub-functor of $M$ such that $T_n(S')$ is the set of subgroups of multiplicative type $H$ of $G_{S'}$ such that $n\cdot\id_H=0$ (which is representable and affine over $S$, cf. \cref{scheme group affine functor of qf multiplicative subgroup representable if}). Let $u_n:M\to T_n$ be the morphism defined by $u_n(H)={_nH}$, where ${_nH}=\ker(n\cdot\id_H)$.
\begin{enumerate}
    \item[(a)] Any subscheme $U$ of $M$, of finite type over $S$, is contained in a closed subscheme of finite type over $S$, and any closed subscheme $Z$ of $M$, of finite type over $S$, is affine over $S$.
    \item[(b)] Suppose that $S$ is quasi-compact, and let $Z$ be a closed subscheme of $M$ of finite type over $S$. Then there exists an integer $n>0$ such that for any multiple $m$ of $n$, the morphism
    \[u_m|_Z:Z\to T_m\]
    is a closed immersion. 
\end{enumerate}
\end{theorem}

\begin{corollary}\label{scheme smooth affine functor of subgroup multiplicative clopen subscheme prop}
With the notations of \cref{scheme smooth affine functor of subgroup multiplicative n-torsion prop}, let $U$ be a clopen subset of $M$, of finite type over $S$. Then $U$ is affine over $S$ for the induced scheme structure by $M$, and if $S$ is quasi-compact, there exists $n>0$ such that for any multiple $m$ of $n$, the induced morphism $u_m|_U:U\to T_m$ is an open and closed immersion.
\end{corollary}
\begin{proof}
The first assertion follows from \cref{scheme smooth affine functor of subgroup multiplicative n-torsion prop}~(a), and the second one from \cref{scheme smooth affine functor of subgroup multiplicative n-torsion prop}~(b), as the morphism $u_m:M\to T_m$ is smooth (\cref{scheme group smooth functor of subgroup n-torsion multiplicative formally smooth}) and that a smooth immersion (i.e. \'etale immersion) is an open immersion.
\end{proof}

\begin{corollary}\label{scheme smooth affine functor of maximal tori is affine}
Let $S$ be a scheme, $G$ be an affine smooth $S$-group, of locally constant reductive rank, $\mathscr{T}$ be the scheme of maximal tori of $G$ (cf. \cref{scheme group affine smooth maximal tori functor prop}). Then $\mathscr{T}$ is smooth and affine over $S$. If $T$ is a maximal torus of $G$, $N_G(T)$ its normalizer, then $G/N_G(T)$ is affine over $S$ (cf. \cref{scheme smooth affine subgroup multiplicative normalizer representable}). The same is true for $G/Z_G(T)$ provided that $W(T)=N_G(T)/Z_G(T)$ is finite over $S$ (cf. \cref{scheme group affine smooth Weyl group variation prop}~(b)).
\end{corollary}
\begin{proof}
The second asertion is contained in the first one, because by the conjugation theorem, $G/N_G(T)$ is isomorphic to $\mathscr{T}$. Note that by construction, $\mathscr{T}$ is isomorphic to a clopen subscheme of $M$. To see this, we can suppose that the reductive rank of $G$ is constant and equals to $r$; then $\mathscr{T}$ is the subscheme of $M$ which corresponds to sub-tori of relative dimension $r$, i.e. the largest subscheme of $M$ over which the universal subgroup of multiplicative type $\widetilde{H}\sub G_M$ is of type $\Z^r$ (cf. \cref{scheme group affine smooth universal subgroup multiplicative}), which is clopen in $M$. We can then apply \cref{scheme smooth affine functor of subgroup multiplicative clopen subscheme prop}. Finally, for the last assertion, we note that $G/Z_G(T)$ is finite over $G/N_G(T)\cong(G/Z_G(T))/W(T)$\footnote{Recall that $G/Z_G(T)\to (G/Z_G(T))/W(T)$ is a $W(T)$-torsor, so we can apply (\cite{SGA1} \Rmnum{8} 5.7).}, hence is affine if $G/N_G(T)$ is.
\end{proof}

\begin{corollary}\label{scheme alg group smooth affine functor of maximal tori direct sum of affine}
Let $G$ be an affine smooth algebraic group over a field $k$. Then the scheme $M$ of subgroups of multiplicative type of $G$ is a direct sum of affine schemes over $S$. For any subgroup $H$ of multiplicative type of $G$, if $C=Z_G(T)$ and $N=N_G(T)$, the quotients $G/C$ and $G/N$ are affine. 
\end{corollary}
\begin{proof}
Using \cref{scheme group smooth functor of subgroup multiplicative action morphism formally smooth}, we see that the saturation of any finite closed subset of $M$ under the action of $G$ is open: in fact, we are reduced to the case where $k$ is algebraically closed, hence to the case of the orbit of a rational point $x$ over $k$. But then by \cref{scheme group smooth functor of subgroup multiplicative action morphism formally smooth}, the morphism $g\mapsto g\cdot x$ from $G$ to $M$ is smooth\footnote{This is the base change of the morphism $\Phi$ in \cref{scheme group smooth functor of subgroup multiplicative action morphism formally smooth} by the morphism $M\to M\times_SM,y\mapsto(y,x)$}, so its image is open. Let $U$ be the union of orbits of closed points of $M$ under the equivalence relation defined by the action of $G$. Then $U$ is open and contains any closed point of $M$, hence by Nullstellensatz is equal to $M$. Thus $M$ is a disjoint union of open subsets, which are necessarily closed, hence $M$ is a sum of clopen subschemes $M_i$, where each $M_i$ is an orbit under $G$ of a closed point, hence is quasi-compact and of finite type (as an image of $G$). In view of \cref{scheme smooth affine functor of subgroup multiplicative clopen subscheme prop}, each $M_i$ is then affine. If $H$ is a subgroup of multiplicative type of $G$, it corresponds to a rational point of $M$ over $k$, and $G/C$ is identified with the orbit of $x$ under $G$ (\cref{scheme smooth affine subgroup multiplicative functor of conjugate representable}), hence is affine by the arguments above. As $C$ is an open subgroup of $N$ (\cref{scheme group affine smooth Weyl group representable}), $G/C$ is finite over $G/N$, and hence is affine if $G/C$ is (\cref{scheme morphism integral finite permanence prop}).
\end{proof}

\begin{corollary}\label{scheme alg group smooth affine subgroup multiplicative functor of conjugate clopen}
Under the conditions of \cref{scheme alg group smooth affine functor of maximal tori direct sum of affine} on $G$ and $H$, the subscheme $U$ of $M$ of subgroups of multiplicative type of $G$ which is locally conjugate to $H$ (cf. \cref{scheme smooth affine subgroup multiplicative functor of conjugate representable}) is a clopen subscheme of $G$.
\end{corollary}
\begin{proof}
In the proof of \cref{scheme alg group smooth affine functor of maximal tori direct sum of affine} we have seen that any orbit in $M$ under $H$ is clopen.
\end{proof}

\section{Regular elements of algebraic groups and Lie algebras}
\subsection{An auxilary lemma for schemes with operators}\label{scheme group regular element auxilary lemma subsection}
Let $S$ be a scheme, $G$ be an $S$-group acting (on the left) on an $S$-scheme $V$, $W$ be a closed subscheme of $V$ and $N=\Trans_G(W)$ be its stabilizer in $G$. We endow $\Sch_{/S}$ with the fpqc topology, and identify $G$, $V$, $M$ with the corresponding sheaves over $\Sch_{/S}$. Consider the quotient sheaf $G/N$, it is easy to see that this is isomorphic to the functor which for any $S'$ over $S$ associates the set of subsheaves $W'$ of $V_{S'}$ which is locally conjugate with $W_{S'}$ under $G$. Let $X$ be the \textbf{twisted product of $G$ and $V$ over $N$}, i.e. the subsheaf of $(G/N)\times_SV$ whose value, for any $S'$ over $S$, is the set of couples $(W',v)$, where $W'$ is as above and $v\in W'(S')$. Let $Z$ be the inverse image of $X$ in $G\times_SV$, so that we have a Cartesian diagram
\begin{equation}\label{scheme G-action stailizer quotient of closed subscheme lemma-1}
\begin{tikzcd}
Z\ar[r]\ar[d]&X\ar[d,hook,"i"]\\
G\times_SV\ar[r]&(G/N)\times_SV
\end{tikzcd}
\end{equation}
where $i$ is the canonical inclusion and the bottom arrow is induced by the canonical morphism $G\to G/N$ sending $g$ to $g\cdot W_{S'}$. We then see that for $S'$ over $S$, $Z(S')$ is the set of couples $(g,v)\in G(S')\times V(S')$ such that $v\in g\cdot W_{S'}(S')$. Therefore, $Z$ is isomorphic to the sheaf $G\times_SW$ via the isomorphism
\[G\times_SW\stackrel{\sim}{\to}Z,\quad (g,w)\mapsto (g,g\cdot w).\]
The preceding Cartesian diagram then gives a Cartesian diagram
\begin{equation}\label{scheme G-action stailizer quotient of closed subscheme lemma-2}
\begin{tikzcd}
G\times_SW\ar[r,"q"]\ar[d,swap,"\lambda"]&X\ar[d,"i"]\\
G\times_SV\ar[r]&(G/N)\times_SV
\end{tikzcd}
\end{equation}
where $\lambda(g,w)=(g,g\cdot w)$, hence $q(g,w)=(\bar{g},g\cdot w)$, where $\bar{g}$ denotes the image of $g$ under the canonical map $G(S')\to(G/N)(S')$. Finally, we see from the diagram (\ref{scheme G-action stailizer quotient of closed subscheme lemma-1}) that $Z\to X$ is a principal bundle under the group $N$, acted by $(g,v)\cdot n=(gn,v)$, so that in (\ref{scheme G-action stailizer quotient of closed subscheme lemma-2}), the morphism $q:G\times_SW\to X$ is a principal bundle under the right action of $N$ defined by
\[(g,w)\cdot n=(gn,n^{-1}\cdot w).\]
We summarize the previous morphisms in the following diagram:
\begin{equation}\label{scheme G-action stailizer quotient of closed subscheme lemma-3}
\begin{tikzcd}[row sep=12mm, column sep=12mm]
&G\times_SW\ar[rd,"\varphi"]\ar[d,"q"]&\\
G/N&X\ar[d,hook,"i"]\ar[r,"\psi"]\ar[l,swap,"p"]&V\\
&(G/N)\times_SV\ar[ru,dashed,swap,"\pr_2"]\ar[lu,dashed,swap,"\pr_1"]&
\end{tikzcd}
\end{equation}
where $p=\pr_1\circ i$, $\psi=\pr_2\circ i$ and $\varphi=\psi\circ q$, i.e. $\varphi(g,w)=g\cdot w$. If $v$ is a section of $V$ over $S$, the subsheaf $X_v\sub X$ of the inverse image of $v$ under $\psi$ is then such that $X_v(S')$ the set of subsheaves $W'$ of $V_{S'}$ which is locally conjugate to $W_{S'}$ under $G$ and contains the section $v_{S'}$ of $V_{S'}$. On the other hand, the inverse image of $v$ under $\varphi$ is isomorphic to the subsheaf $M_v$ of $G$ defined such that $M_v(S')$ the set of $g\in G(S')$ such that $v_{S'}\in g\cdot W(S')$, i.e. such that $g^{-1}v_{S'}\in W(S')$. If $v$ is a section of $W$ and not only of $V$, then $M_v$ obviously contains $N$.\par
Now suppose that $N$ is representable and faithfully flat and quasi-compact over $S$, and that $G/N$ is representable (this is true if $S$ is the spectrum of a field $k$ and $G$ is of finite type over $k$, cf. \cref{scheme k-group transporter representable by closed}). Then using the theory of fpqc descent and the fact that $Z\to G\times_SV$ is a closed immersion (recall that $Z$ is isomorphic to $G\times_SW$), we see from the Cartesian diagram (\ref{scheme G-action stailizer quotient of closed subscheme lemma-2}) that $X$ is representable (it is obtained by descent of the closed subscheme $Z$ from $G\times_SV$ by the faithfully flat and quasi-compact morphism $G\times_SV\to (G/N)\times_SV$), so (\ref{scheme G-action stailizer quotient of closed subscheme lemma-3}) is a diagram of morphisms of schemes over $S$.\par
We suppose thereafter that $S$ is the spectrum of a field $k$, and that $G$, $V$, $W$ are of finite type over $k$. Let $\n$ be the Lie algebra of $N$, so we have $\dim(N)\leq\dim_k(\n)$, with the equality if and only if $N$ is smooth over $k$. Let $a\in W(k)$, and consider the subscheme $M_a=\varphi^{-1}(a)$ of $G$ defined as above, which contains $N$. Let $\m_a$ the tangent space of $M_a$ at the identity element $e$ of $N$, so that we have 
\begin{equation}\label{scheme action normalizer and inverse image prop-1}
\n\sub\m_a,\quad \dim(N)\leq\dim_k(\n)\leq\dim_k(\m_a).
\end{equation}

\begin{lemma}\label{scheme action normalizer and inverse image prop}
Consider the following conditions:
\begin{enumerate}
    \item[(\rmnum{1})] $n=\m_a$ and $N$ is smooth over $k$.
    \item[(\rmnum{1}')] $\dim(N)=\dim_k(\m_a)$.
    \item[(\rmnum{2})] The morphism $\psi:X\to V$ is unramified at $(\bar{e},a)$.
    \item[(\rmnum{3})] $M_a$ and $N$ coincide in a neighborhood of $e$.  
\end{enumerate}
Then we have the implications (\rmnum{1})$\Leftrightarrow$(\rmnum{1}')$\Rightarrow$(\rmnum{2})$\Leftrightarrow$(\rmnum{3}). Further, suppose that $\varphi:G\times_SW\to V$ is smooth at $(e,a)$, then $M_a$ is smooth over $k$ at $e$, and $\psi$ is smooth at $(\bar{e},a)$.
\end{lemma}
\begin{proof}
The equivalence of (\rmnum{1}) and (\rmnum{1}') follows from (\ref{scheme action normalizer and inverse image prop-1}), and the fact that $N$ is smooth if and only if $\dim(N)=\dim_k(\n)$. On the other hand, consider the inclusion morphism $N\to M_a$, by (\cite{EGA4-4} 17.11.1 (d)), if $N$ is smooth over $k$ at $e$ and the tangent map at $e$ is surjective, then $N\to M_a$ is smooth at $e$, hence (being an immersion) is an isomorphism at $e$; this proves (\rmnum{1})$\Rightarrow$(\rmnum{3}). To prove the equivalence of (\rmnum{2}) and (\rmnum{3}), consider $X_a=\phi^{-1}(a)$; using the isomorphism $M_a\cong\varphi^{-1}(a)=q^{-1}(X_a)$, we obtain a morphism $p_a:M_a\to X_a$ which makes $M_a$ a principal homogeneous bundle over $X_a$ with group $N$. Consider the following diagram:
\[\begin{tikzcd}
N\ar[r,"j_a"]\ar[d]&M_a\ar[d,"p_a"]\\
S=\Spec(k)\ar[r,"j_a'"]&X_a
\end{tikzcd}\]
where $j_a':\Spec(k)\to X_a$ is defined by the point $(\bar{e},a)$ of $X_a$, and $j_a:N\to M_a$ is the canonical immersion. To say that $\psi$ is unramified at $(\bar{e},a)$ signifies that $j'_a$ is an open immersion (\cref{scheme morphism local fp unramified at point iff}), or equivalently that it induces an isomorphism $S\stackrel{\sim}{\to}\Spec(\mathscr{O}_{X_a,a})$. As $p_a$ is flat, this is equivalent to saying that the morphism induced from the preceding morphism by base change $\Spec(\mathscr{O}_{M_a,e})\to\Spec(\mathscr{O}_{X_a,b})$ is an isomorphism; but this induced morphism is none other than the morphism $\Spec(\mathscr{O}_{N,e})\to\Spec(\mathscr{O}_{M_a,e})$, so we obtain the equivalence of (\rmnum{2}) and (\rmnum{3}). Finally, as $M\cong\varphi^{-1}(a)$, the last assertion follows from the fact that $q$ is flat and $q(e,a)=(\bar{e},a)$.
\end{proof}

\begin{remark}
We note that under the hypothesis of \cref{scheme action normalizer and inverse image prop}~(\rmnum{3}), $N$ is in fact a clopen subscheme of $M_a$ (closed because it is a subgroup of $M_a$, and open because the it is open at the identity).
\end{remark}

\subsection{Regular elements of algebraic groups and density theorem}
\paragraph{The case of an algebraic group over a field}\label{scheme alg group regular element paragraph}
We now apply the construction and notations of the previous subsection to the case where $G$ is a smooth connected algebraic group over $k$, $V=G$ over which $G$ acts by inner automorphisms, and $W=H$ is a smooth connected subgroup of $G$. We denote by $\g,\h$ the Lie algebras of $G$ and $H$, respectively, by $N$ the normalizer of $H$ in $G$, and $\n$ be its Lie algebra. If $a\in G(k)$, we denote by $M_a$ the subgroup of $G$ isomorphic to $\varphi^{-1}(a)$, so that if $a\in H(k)$, we have $N\sub M_a$. In this case, we denote by $\m_a$ the tangent space of $M_a$ at the identity element of $G$. Note that for $a\in H(k)$, we have
\[\h\sub\n\sub\m_a\sub\g.\]
We shall utilize the following lemma:

\begin{lemma}\label{scheme alg group smooth connected subgroup equal normalizer component iff}
For that $H=N^0$, it is necessary and sufficient that we have $(\g/\h)^H=0$ (where $\g/\h$ denotes the subspace of invariants under the adjoint action of $H$). If this condition is satisfied, $N$ is smooth and we have $\dim(X)=\dim(G)$. In any case, $\dim(X)\leq\dim(G)$, and the equality holds if and only if $H$ has finite index in $N$.
\end{lemma}
\begin{proof}
In view of \cref{scheme group normalizer and centralizer Lie prop}~(\rmnum{1}), $\n$ is equal to the inverse image of $(\g/\h)^H$ under the quotient map $\g\to\g/\h$, hence $(\g/\h)^H=0$ if and only if $\n=\h$, which is equivalent to ($H$ being a smooth connected subgroup of $N$) that $H=N^0$ (cf. \cref{scheme group Lie alg of connected component} and \cref{scheme alg group Lie algebra equal then euqal}). This implies evidently that $N$ is smooth. On the other hand, we have
\[\dim(X)=\dim(G)-\dim(N)+\dim(H)=\dim(G)-(\dim(N)-\dim(H)),\]
hence $\dim(X)\leq\dim(G)$, with the equality holds if and only if $\dim(H)=\dim(N)$, i.e. if and only if $H$ has finite index in $N$. 
\end{proof}

\begin{theorem}\label{scheme alg smooth connected subgroup contain Cartan iff}
Let $G$ be a smooth connected algebraic group over an algebraically closed field $k$, $H$ be a smooth connected subgroup of $G$, $N$ be its normalizer, and $X=G\times^NH$ be the twisted product of $G$ and $H$ over $N$. Let $\psi:X\to G$ be the canonical morphism, defined by $(\bar{g},h)\mapsto\inn(g)h=ghg^{-1}$. Then the following conditions are equivalent:
\begin{enumerate}
    \item[(\rmnum{1})] $H$ contains a Cartan subgroup $C$ of $G$.
    \item[(\rmnum{1}')] The reductive rank and nilpotent rank $H$ equal to those of $G$.
    \item[(\rmnum{2})] $H$ contains a maximal torus $T$ of $G$, and $(\g/\h)^T=0$.
    \item[(\rmnum{3})] The set of conjugates of $H$ containing a given maximal torus is nonempty and finite, and $H$ has finite index in $N$.
    \item[(\rmnum{4})] There exists $a\in H(k)$ which is contained in finitely many conjugates of $H$ (or such that $\psi^{-1}(a)$ has an isolated point), and $H$ has finite index in $N$.
    \item[(\rmnum{4}')] The morphism $\psi:X\to G$ is generically quasi-finite (i.e. there exists an open dense subset of $X$ over which $\psi$ is quasi-finite), and $H$ has finite index in $N$.
    \item[(\rmnum{5})] There exists an open dense subset $U$ of $G$ such that for any $x\in U(k)$, the set of conjugates of $H$ containing $x$ is nonempty and finite, i.e. $\psi:X\to G$ is dominant and generically quasi-finite.
    \item[(\rmnum{6})] There exists an open dense subset $U$ of $G$ such that any $x\in U(k)$ is contained in a conjugate of $H$, i.e. $\psi:X\to G$ is dominant.
    \item[(\rmnum{7})] There exists $a\in H(k)$ such that $(\g/\h)^{\Ad(a)}=0$.
\end{enumerate}
Furthermore, these conditions imply that $H=N^0$, i.e. $N$ is smooth and $\dim(H)=\dim(N)$, and that $\psi:X\to G$ is generically \'etale.
\end{theorem}

By \cref{scheme alg group smooth connected subgroup equal normalizer component iff}, we have $\dim(X)\leq\dim(G)$, with the equality holds if and only if $\dim(H)=\dim(N)$, i.e. $H$ has finite index in $N$. Now the inequality $\dim(X)\leq\dim(G)$ implies that $\psi:X\to G$ is dominant if and only if it is domimant and generically quasi-finite, or equivalantly if $\psi$ is generically quasi-finite and $\dim(X)=\dim(G)$. By \cref{scheme alg group smooth connected subgroup equal normalizer component iff} again, we then conclude the equivalence of (\rmnum{4}'), (\rmnum{5}) and (\rmnum{6}). The equivalence of (\rmnum{4}) and (\rmnum{4}') is immdiaite by \cref{scheme morphism ft isolated point nbhd finite}, as $X$ is irreducible (being the image of $G\times_SH$).\par
The equivalence of (\rmnum{1}) and (\rmnum{1}') is immediate from definition. On the other hand, if $H$ contains a Cartan subgroup $C$ of $G$, it contains the maximal torus $T$ of $C$ (cf. \cref{scheme group fp fpqc local maximal central torus prop}), which is a maximal torus of $G$. As $C$ is the centralizer of $T$, its Lie algebra $\c$ is given by
\[\c=\g^T\]
(cf. \cref{scheme group normalizer and centralizer Lie prop}~(\rmnum{2})). Hence as $H\supseteq C$, whence $\h\sups\c$, it ensures that $\h\sups\g^T$, which in view of the root decomposition (\cref{scheme module over diagonalizable group cat equivalent to graded module}) implies that
\begin{equation}\label{scheme alg group smooth connected subgroup contain Cartan iff-1}
(g/\h)^T=0.
\end{equation}
Conversely, if $H$ contains a maximal torus $T$ and the preceding relation is verified, i.e. $\h\sups\c=\g^T$, we claim that $H$ contains the centralizer $C$ of $T$ (whence the equivalence (\rmnum{1}) and (\rmnum{2})). This follows form the following lemma:

\begin{lemma}\label{scheme alg group smooth subgroup contain centralizer iff Lie algebra}
Let $G$ be a smooth algebraic group over a field $k$, $T$ be a subgroup of multiplicative type of $G$, $C$ be its connected centralizer (i.e. the connected component of $Z_G(T)$\footnote{This is equal to $Z_G(T)$ if $G$ is connected and $T$ is a torus, cf. (\cite{SGA3-2} \Rmnum{12} 6.6 (b)).}), $H$ be a smooth subgroup of $G$ containing $T$. For $H$ to contain $C$, it is necessary and sufficient that its Lie algebra $\h$ contains the Lie algebra $\c$ of $C$.
\end{lemma}
\begin{proof}
We have seen in \cref{scheme group affine smooth Weyl group representable} that $Z_G(H)$ is smooth over $k$, hence $C$ is smooth over $k$; similarly $Z_H(T)$ is smooth over $k$. Now the intersection $Z_H(T)=Z_G(T)\cap H$ has Lie algebra $\h\cap\c$, so the hypothesis $\h\sups\c$ implies that the smooth subgroup $Z_H(T)$ of the smooth group $Z_G(T)$ have the same Lie algebra, so it contains the connected component $C$ of $Z_G(T)$, which means $H$ contains $C$. The reverse implication is immediate.
\end{proof}

To prove the equivalence of (\rmnum{1}) and (\rmnum{3}), it suffices to prove that if $H$ contains the maximal torus $T$ of a Cartan subalgebra $C$ of $G$, then the condition $H\sups C$ (which is equivalent to (\ref{scheme alg group smooth connected subgroup contain Cartan iff-1}), as we have already seen above) is equivalent to that $H$ has finite index in $N$ and the set of conjugates of $H$ containing $T$ is finite. If $H$ contains $C$, hence $(\g/\h)^T=0$, then a fortiori
\begin{equation}\label{scheme alg group smooth connected subgroup contain Cartan iff-2}
(\g/\h)^H=0,
\end{equation}
now recall that $\n$ is the inverse image of $(\g/\h)^H$ under the canonical morphism $\g\to\g/\h$ (\cref{scheme group normalizer and centralizer Lie prop}~(\rmnum{1})), so the preceding relation is equivalent to that $H=N^0$ by \cref{scheme alg group smooth connected subgroup equal normalizer component iff}, a fortiori $H$ has finite index in $N$. Now consider the diagraom of subgroups
\[\begin{tikzcd}
T\ar[r]&N(T)\cap H\ar[r]\ar[d]&H\ar[d]\\
&N(T)\cap N(H)\ar[d]\ar[r]&N(H)\\
&N(T)
\end{tikzcd}\]
Using the conjugation theorem of maximal tori in $H$ (\cite{SGA3-2} 6.6 (a)), we see that any conjugate of $H$ containing $T$ is conjugate to $H$ by an element of $N(T)(k)$, hence the set of conjugates of $H$ containing $T$ is in bijection to the set of points of $N(T)/(N(T)\cap N(H))$. Now as $H\sups C$, we have $N(T)\cap H\sups C$, hence the preceding set is a quotient of $(N(T)/C)(k)$ (which is a finite set), hence is finite. Conversely, suppose that $N(T)/(N(T)\cap N(H))$ are $N(H)/H$ are finite. Using the conjugation theorem in $H$, we see that the homomorphism
\[(N(T)\cap N(H))/(N(T)\cap H)\to N(H)/H\]
induced by the preceding diagram is bijective on $k$-values points (in fact, it is an isomorphism), so as the second member is finite, so is the first one. It then follows that $N(T)\cap H$ has finite index in $N(T)$, so it contains $C=N(T)^0$, whence $H\sups C$. By now, we have proved that conditions (\rmnum{1}), (\rmnum{1}'), (\rmnum{2}) and (\rmnum{3}) are equivalent.\par
Now consider the implication (\rmnum{2})$\Rightarrow$(\rmnum{7}). We easily see that the conditions (\rmnum{2}) and (\rmnum{7}) are each invariant under base extension $k\to k'$, with $k'$ algebraically closed, so we can suppose that $k$ is of infinite transcendendal degree over its prime field. Then it is well known (and we immediately verify) that there exists an element $a\in T(k)$ such that the subgroup of $T(k)$ it generates is dense in $T$ for the Zariski topology. We then conclude that $(\g/\h)^T=(\g/\h)^{\Ad(a)}$, and as by hypothesis the first member is zero, we then conclude (\rmnum{7}).\par
Now as $G$ is irreducible and the smooth locus of $\psi$ is open (cf. \cite{EGA4-4} 17.11.4), the proof of (\rmnum{7})$\Rightarrow$(\rmnum{6}) is contained in the following corollary, which refines \cref{scheme alg smooth connected subgroup contain Cartan iff}:

\begin{corollary}\label{scheme alg group smooth twisted product morphism smooth iff}
Let $G$ be a smooth algebraic group over a field $k$, $H$ be a smooth subgroup of $G$, $N$ be its normalizer in $G$, and $\varphi:G\times_SH\to G$ be the morphism defined by $\varphi(g,h)=\inn(g)h=ghg^{-1}$. Let $\psi:X=G\times^NH\to G$ be the morphism induced from $\varphi$ by passing to quotient, and $a\in H(k)$. Then the following conditions are equivalent:
\begin{enumerate}
    \item[(\rmnum{1})] $\varphi$ is smooth at $(e,a)$.
    \item[(\rmnum{2})] $\psi$ is \'etale at $(\bar{e},a)$ and $N$ is smooth over $k$.
    \item[(\rmnum{3})] $(\g/\h)^{\Ad(a)}=0$ (where $\g,\h$ are the Lie algebras of $G$, $H$, respectively).  
\end{enumerate}
These conditions imply $H^0=N^0$.
\end{corollary}
\begin{proof}
We see that the smoothness of $\varphi$ (which is a morphism of smooth $k$-schemes) at a rational point over $k$ is equivalent to the surjectivity of the tangent map at this point. Now an immedaite calculus shows that this tangent map is given by
\[d\varphi_{(e,a)}(\xi,\eta)=(\Ad(a^{-1})-\id)(\xi)+\eta,\]
consider as a map from $\g\times\h\to\g$ (as usual, we identity the tangent spaces of $G$ and $H$ with their Lie algebras, using left invariant vector fields)\footnote{As the tangent map can be computed componentwise (i.e. by computing the partial derivatives), it amouts to consider the morphism $G\to G,g\mapsto gag^{-1}$. This morphism sends $e$ to $a$, and to compute the tangent map on $\g$, we may compose it with the left multiplication $\ell_{a^{-1}}$, but then note that $a^{-1}gag^{-1}=\mu(C_{a^{-1}}(g),g^{-1})$, where $\mu$ is the multiplication morphism of $G$.}. Its surjectivity is then equivalent to the surjectivity of $\id-\Ad(a^{-1})=\id-(\Ad(a))^{-1}$ on $\g/\h$, i.e. to the condition in (\rmnum{3}). Now (\rmnum{3}) implies a fortiori $(\g/\h)^H=0$, i.e. $H^0=N^0$ (cf. \cref{scheme alg group smooth connected subgroup equal normalizer component iff}). We deduce that $N$ is also smooth, and $\dim(H)=\dim(N)$, whenc $\dim(X)=\dim(G)$. Now as $q:G\times_SH\to X$ is flat and $\psi=\varphi\circ q$, this implies that $\psi$ is smooth at $q(e,a)=(\bar{e},a)$, hence \'etale at this point by dimension consideration. We have therefore proved (\rmnum{1})$\Leftrightarrow$(\rmnum{3})$\Rightarrow$(\rmnum{2}), but (\rmnum{2})$\Rightarrow$(\rmnum{1}) because the smoothness of $N$ implies that of $q$.
\end{proof}

Finally, we prove that (\rmnum{6})$\Rightarrow$(\rmnum{1}), which, together with the already established implications, prove the theorem. Suppose first that $G$ is affine, and let $U$ be the nonempty open subset of $G$ such that $x\in U(k)$ implies that $x$ is contained in a conjugate of $H$. Let $C$ be a Cartan subgroup of $G$. Using the implication (\rmnum{1})$\Rightarrow$(\rmnum{5}) for $C$ replacing $H$ (this is the density theorem of Borel), we see that we can find a conjugate of $C$ which meets $U$, hence we can suppose that $U\cap C\neq\emp$, i.e. there exists a nonempty open subset $V$ of $C$ such that for any $x\in V(k)$, $x$ is contained in a conjugate of $H$. Now write $C$ as a product
\[C=T\cdot C_u\]
where $T$ is the maximal torus of $C$ (also a maximal torus of $G$) and $C_u$ is the unipotent part of $C$, $T$ being the center of $C$ (\cite{Chevalley1958} 6 th.2). We can still assume that $k$ has infinite transcendental degree over its prime field, which allows us to find an element $t\in T(k)$ which belongs to the projection of $V$ onto $T$ (which is a non-empty open set of $T$), i.e. $(t\cdot C_u)\cap V\neq\emp$, and such that $t$ "generates" $T$. As any algebraic subgroup of $G$ which contains a product $t\cdot u$ (with $t\in T(k)$, $u\in C_u(k)$) contains both factors (\cite{Chevalley1958} 4 th. 3), it follows, with the previous choice of $t$, and taking $t\cdot u\in V(k)$, that there exists a conjugate of $H$ containing $t$, hence $T$. Therefore, by taking conjugate of $C$ (and $T$), we can assume that $T\sub H$.\par
If $W\sub C_u$ is the inverse image of $V$ under the morphism $u\mapsto t\cdot u$, then we see that for any element $x\in(T\cdot W)(k)$, there exists a conjugate of $H$ which contains $T$ and $x$. As we have already remarked (\cite{SGA3-2} 6.6 (a)), such a conjugate is of the form $\inn(g)\cdot H$, where $g\in N(T)(k)$. Consider then the morphism
\[f:N(T)\times_S H\to G,\quad (g,h)\mapsto\inn(g)h=ghg^{-1},\]
then the image of $f$ contains $T\cdot W$, hence, as $N(T)$ is a finite union of translates $C\cdot g_i$ with $g_i\in N(T)(k)$ (since $C$ has finite index in $N(T)$), there exists an open dense subset $V'$ of $C=T\cdot C_u$ which is contained in the image of $(C\cdot g_i)\times_S H$ under $f$. By replacing $H$ with $\inn(g_i)\cdot H$, we can suppose that $g_i=e$, i.e. $f(C\times H)\sups V'$. Hence for any $u\in V'(k)$, there exists $v\in C(k)$ and $h\in H(k)$ such that $v^{-1}hv=u$, whence $vuv^{-1}\in H(k)$. Putting $C'=C\cap H=Z_H(T)$, we then get $vuv^{-1}\in C'$, whence $u\in\inn(v)\cdot C'$. This proves that the union of conjugates of $C'$ in $C$ (under the elements of $C(k)$) is dense, which implies (in view of the implication (\rmnum{6})$\Rightarrow$(\rmnum{4})) that $C'$ has finite index in its normalizer in $C$. By (\cite{Chevalley1958} 7 lemma du n1), this implies that $C'=C$, hence $C=H$, and this proves (\rmnum{6})$\Rightarrow$(\rmnum{1}) if $G$ is affine.\par
In the general case, we prceed by induction on $n=\dim(G)$, the assertion being trivial if $n=0$. Let $Z$ be the center of $G$, and distinguish two cases: if $\dim(Z\cap H)>0$, then putting $G'=G/(Z\cap H)$, we have $\dim(G')<n$; on the other hand, the hypothesis (\rmnum{6}) on $H$ implies the same condition for the image $H'$ of $H$ in $G'$, so $H'$ contains a Cartan subgroup $C'$ of $G'$, and then $H$ contains its inverse image $C$ of $G'$, which is a Cartan subgroup in view of (\cite{SGA3-2} \Rmnum{12} 6.6 (e)).\par
In the case where $\dim(Z\cap H)=0$, the canonical morphism $H\to G/Z$ is a finite morphism, and as $G/Z$ is affine in view of (\cite{SGA3-2} \Rmnum{12} 6.1), it then follows that $H$ is affine, hence any homomorphism from $H$ into an abelian variety is zero: this follows from the fact that a connected affine smooth algebraic group over an algebraically closed field is a rational variety, or simply that it is the union of its Borel subgroups (\cite{Chevalley1958} 6 th.5 (b)). Using Chevalley's structure theorem, we see that $G$ is an extension of an abelian veriety $A$ by a smooth affine group. Then the image of $H$ in $A$ is zero, $H$ being affine; but it is identical to $A$ because the union of its conjugates in $A$ must be dense, so $A=0$ and $G$ is affine. We are reduced to the affine case. This completes the proof of \cref{scheme alg smooth connected subgroup contain Cartan iff}.

\begin{corollary}\label{scheme alg group smooth twisted product morphism smooth prop}
Suppose that the equivalent conditions of \cref{scheme alg group smooth twisted product morphism smooth iff} are verified.
\begin{enumerate}
    \item[(a)] Let $k(X)$ (resp. $k(G)$) be the rational function field of $X$ (resp. $G$), then $k(X)$ is a separable finite extension of $k(G)$ (let $d$ be its degree).
    \item[(b)] Let $T$ be a maximal torus of $G$ contained in $H$ (which exists by \cref{scheme alg smooth connected subgroup contain Cartan iff}~(\rmnum{2})) and $C$ be the corresponding Cartan subgroup of $G$; then $C\sub H$. On the other hand, $N_G(T)$ is a smooth subgroup of $G$ and $N_G(T)\cap N_G(H)=N_{N_G(H)}(T)$ is a smooth subgroup of finite index $d$ in $N_G(T)$. The number of conjugates of $H$ containing a given maximal torus or a Cartan subgroup of $G$ is equal to $d$.
    \item[(c)] Let $U$ be the largest open subset of $G$ such that $\psi:X\to G$ induces a finite \'etale morphism $\psi^{-1}(U)\to U$. Then $U$ is an open dense subset of $G$ and for $g\in G(k)$, we have $g\in U(k)$ if and only if there exists exactly $d$ conjugates of $H$ containing $g$, or equivalently there exists $d$ distinct cojugates $H_i$ of $H$ containing $g$ such that for each $i$, $(\g/\h_i)^{\Ad(g)}=0$ (where $\h_i=\mathfrak{Lie}(H_i)$).
\end{enumerate}
\end{corollary}
\begin{proof}
Assertion (a) follows from the fact that $\psi$ is generically \'etale; this also implies that the open subset $U$ introduces in (c) is nonempty, i.e. dense, and also the two characterization given for the elements of $U(k)$ ($\psi$ is separated, $X$ is integral and $G$ is integral normal, so this follows from (\cite{SGA1} \Rmnum{1} 10.11) and the fact that $(\g/\h_i)^{\Ad(g)}=0$ signifies that $\psi$ is \'etale at the point $x_i$ of $\psi^{-1}(g)$ corresponding to $H_i$). If $H$ contains a maximal torus $T$ of $G$, then the centralizers of $T$ in $H$ and $G$ have the same dimension, and are smooth and connected (\cite{SGA3-2} \Rmnum{12} 6.6 (b)), hence are equal, which proves that $C\sub H$. Moreover, we know that the normalizer of $T$ in a smooth group containing it is smooth (\cref{scheme smooth affine subgroup multiplicative normalizer representable}), so $N_G(T)$ and $N_{N_G(H)}(T)$ are smooth (we note that $N=N_G(H)$ is smooth by the last assertion of \cref{scheme alg smooth connected subgroup contain Cartan iff}), moreover $N_{N_G(H)}(T)$ contains $C$, which has finite index in $N_G(T)$, so it also has finite index in $N_G(T)$. Using the conjugation theorem for maximal tori of $H$, we see that this index is equal to the number of conjugates of $H$ which contain $T$, or equivalently, those contain $C$. Now as the union of conjugates of $C$ in $G$ is dense (in view of \cref{scheme alg smooth connected subgroup contain Cartan iff}~(\rmnum{6}) applied to $C$), and the open subset $U$ defined in (c) is evidently stable under inner automorphisms, we see that $C\cap U\neq\emp$. Proceed as in the proof of (\rmnum{6})$\Rightarrow$(\rmnum{1}) of \cref{scheme alg smooth connected subgroup contain Cartan iff}, we conclude that (by a base change of field) that there exists $g\in(C\cap U)(k)$ such that any conjugate $H$ containing $g$ also contains $T$, and therefore also contains $C$ (cf. the proof of (\rmnum{1})$\Leftrightarrow$(\rmnum{3}) of \cref{scheme alg smooth connected subgroup contain Cartan iff}). Hence the conjugates of $H$ containing $C$ are those containing $g$, and as $g\in U(k)$, the number of such conjugates is equal to $d$, which proves the assertions of (b). We have in fact established that the set of conjugates of $H$ containing $T$ is a homogeneous set under the group of rational points of
\[W_G(T)=N_G(T)/Z_G(T)\]
which proves in particular that $d$ is smaller or equal to the order of the Weyl group of $G$.
\end{proof}

\begin{corollary}\label{scheme alg group smooth twisted product morphism birational iff}
With the notations of \cref{scheme alg smooth connected subgroup contain Cartan iff}, the following conditions are equivalent:
\begin{enumerate}
    \item[(\rmnum{1})] $\psi:X\to G$ is a birational morphism.
    \item[(\rmnum{2})] There exists a unique conjugate of $H$ containing a given Cartan subgroup of $G$.
    \item[(\rmnum{3})] $H$ contains a Cartan subgroup $C$ of $G$, and $N_G(H)\sups N_G(C)$.
    \item[(\rmnum{4})] There exists a nonempty open subset $V$ of $G$ such that $g\in V(k)$ implies that $g$ is contained in exactly one conjugate of $H$.   
\end{enumerate}
\end{corollary}
\begin{proof}
This follows from \cref{scheme alg group smooth twisted product morphism smooth prop} and \cref{scheme alg smooth connected subgroup contain Cartan iff}, since in this case $\psi$ is birational if and only if $d=1$, which is equivalent by \cref{scheme alg group smooth twisted product morphism smooth prop} to that $N_G(T)\cap N_G(H)=N_G(T)$ (recall that $N_G(T)=N_G(C)$ by \cite{scheme smooth affine ft normalizer of torus and Cartan equal}).
\end{proof}

\begin{corollary}\label{scheme alg group smooth conjugates containing g unique iff}
Suppose that the conditions of \cref{scheme alg group smooth twisted product morphism birational iff} are satisfied, and let $g\in G(k)$. Then the following conditions are equivalent:
\begin{enumerate}
    \item[(\rmnum{1})] $g\in U(k)$, where $U$ is defined as in \cref{scheme alg group smooth twisted product morphism smooth prop}, i.e. $g$ is contained in a unique conjugate of $H$.
    \item[(\rmnum{2})] The set of conjugates of $H$ containing $g$ is finite and nonempty.
    \item[(\rmnum{3})] The scheme $\psi^{-1}(g)$ of conjugates of $H$ containing $g$ contains an isolated point.
    \item[(\rmnum{4})] There exists a conjugate $H'$ of $H$ containing $g$, and we have $(\g/\h')^{\Ad(g)}=0$, where $\h'=\mathfrak{Lie}(H')$.
\end{enumerate}
Finally, $U$ is also the largest open subset of $G$ such that $\psi$ induces an isomorphism $\psi^{-1}(U)\stackrel{\sim}{\to}U$.
\end{corollary}
\begin{proof}
The equivalence of (\rmnum{1}), (\rmnum{2}) and (\rmnum{3}), as well as the last assertion, follow from the Zariski's Main Theorem applied to the birational morphism $\psi:X\to G$, given that $G$ is normal. The equivalence of these conditions to (\rmnum{4}) follows immediately from the last assertion of \cref{scheme alg smooth connected subgroup contain Cartan iff}, which characterizes the set of elements of $X$ over which $\psi$ is \'etale.
\end{proof}

\begin{theorem}\label{scheme alg group smooth twisted product morphism for Cartan group and eigenvalue of 1}
Let $G$ be a connected smooth algebraic group over an algebraically closed field $k$, $C$ be a Cartan subgroup, with maximal torus $T$, $N=N_G(C)=N_G(T)$ (cf. \cite{scheme smooth affine ft normalizer of torus and Cartan equal}), Let $X=G\times^NC$ be the twisted product of $G$ and $C$ over $N$, where $N$ acts over $C$ by inner automorphisms, and $\psi:X\to G$ be the canonical morphism.
\begin{enumerate}
    \item[(a)] The morphism $\psi$ is birational.
    \item[(b)] Let $U$ be the largest open subset of $G$ such that $\psi$ induces an isomorphism $\psi^{-1}(U)\stackrel{\sim}{\to}U$, and
    \[\rho=\rho_n(G)=\dim(C)\]
    be the nilpotent rank of $G$. Then for any $g\in G(k)$, the multiplicity of the eigenvalue $1$ of $\Ad(g)$ over $\g$ is $\geq\rho$, and for it to be equal to $\rho$, it is necessary and sufficient that we have $g\in U(k)$. 
\end{enumerate}
\end{theorem}
\begin{proof}
As the conditions of \cref{scheme alg smooth connected subgroup contain Cartan iff}, we can apply \cref{scheme alg group smooth twisted product morphism birational iff}, which implies (a). In (\cite{Chevalley1958} 7) (in the case where $G$ is affine), the points of $U(k)$ are called regular points of $G(k)$, and we will follow this terminology by calling $U$ the \textbf{open subset of the regular points of $G$}. To prove (b), we introduce for any $g\in G(k)$ the characteristic polynomial
\[P(\Ad(g),t)=t^n+c_1(g)t^{n-1}+\cdots+c_n(g).\]
By replacing $k$ with an arbitrary $k$-algebra $R$, we easily see that the $c_i(g)$ comes from well-defined sections $c_i\in\Gamma(G,\mathscr{O}_G)$. If $g\in G(k)$ is an element contained in a Cartan subgroup (for example a regular element), which we can assume to be in $C$, then by \cref{scheme alg group smooth conjugates containing g unique iff}~(\rmnum{4}), we see that we have $(\g/\c)^{\Ad(g)}=0$ if and only if $\g$ is regular (where $\c$ denotes the Lie algebra of $C$). On the other hand, as $C$ is nilpotent, we see immediately that $\Ad_{\c}(g)$ only has eigenvalue $1$, which proves that the multiplicity of the eigenvalue $1$ for $\Ad_{\g}(g)$ is $\geq\rho$, and exactly equals to $\dim(C)=\rho$ if and only if $g$ is regular. In particular, the polynomial above is divisible by $(t-1)^\rho$. As the relation of divisibility by $(t-1)^\rho$ is expressed by linear relations (with integer coefficients) between the coefficients of the polynomial, and that these relations hold for $g\in U(k)$, $U$ being a dense open set, it follows ($G$ being reduced) that they hold for any $g$, so that we have a relation
\begin{equation}\label{scheme alg group smooth twisted product morphism for Cartan group and eigenvalue of 1-1}
t^n+c_1t^{n-1}+\cdots+c_0=(t-1)^\rho(t^{n-\rho}+b_1t_1^{n-\rho-1}+\cdots+b_{n-\rho})
\end{equation}
in the ring of polynomials over $\Gamma(G,\mathscr{O}_G)$. In particular, for any $g\in G(k)$, the multiplicity of $1$ of $\Ad(g)$ is $\geq\rho$. Moreover, we see that we have the equality if $g$ is regular. For the converse, suppose that $G$ is affine, and write $g$ as a product
\[g=g_sg_u\]
of semi-simple and unipotent part (\cite{Chevalley1958} 4 n4), then
\[\Ad(g)=\Ad(g_s)\Ad(g_u)\]
is the analogous decomposition for $\Ad(g)$ (\cite{Chevalley1958} 4 n4 cor au th.3), and therefore $\Ad(g)$ and $\Ad(g_s)$ have the same eigenvalue (considering multilicities), in particular the eigenvalue $1$ has the same multiplicity in $\Ad(g)$ and $\Ad(g_s)$.\par
On the other hand, in view of (\cite{Chevalley1958} 7 th.2 cor.1), $g$ is regular if and only if $g_s$ is. Hence to prove (b), we can suppose that $g=g_s$ is semi-simple, hence contained in a maximal torus in view of (\cite{Chevalley1958} 7 th.2 cor.1), and a fortiori in a Cartan subgroup of $G$, which is the case we have already treated. This proves (b) in the case where $G$ is affine. In the general case, let $Z=Z(G)_\red$, then in view of (\cite{SGA3-2} \Rmnum{12} 6.6 (e)), the Cartan subgroups of $G$ are the inverse images of that of $G'=G/Z$, hence $g$ is regular in $G$ if and only if its image $g'$ in $G'$ is regular in $G'$. On the other hand, as $Z$ is smooth, the Lie algebra $\g'$ of $G'$ is equal to $\g/\z$, where $\z=\mathfrak{Lie}(Z)$, and $\Ad(g')$ is equal to $\Ad_{\g/\z}(g)$, hence the multiplicity of eigenvalue $1$ in $\Ad(g')$ is equal to $d=\dim(Z)$ plus the multiplicity of eigenvalue $1$ in $\Ad(g')$, so the former is equal to the reductive rank of $G$ if and only if the latter is equal to that of $G'$. We are therefore reduced to the case of $G'$, but $G'$ is affine in view of (\cite{SGA3-2} \Rmnum{12} 6.1), so this completes the proof.
\end{proof}

\begin{corollary}\label{scheme alg group smooth open of regular element given by G_b}
With the notations of the preceding proof, let
\[b=1+b_1+\cdots+b_{n-\rho}\in\Gamma(G,\mathscr{O}_G).\]
Then the open subset of regular elements of $G$ is given by
\[U=G_b\]
(set of points of $G$ where $b$ is invertible), in particular, $U$ is an affine open subset if $G$ is affine.
\end{corollary}
\begin{proof}
By the arguments above, $g$ is a regular element if and only if the polynomial $t^{n-\rho}+b_1t^{n-\rho-1}+\cdots+b_{n-\rho}$ does not divide $t-1$, which means, as $k$ is algebraically closed, that $1$ is not a root of it, whence the claim.
\end{proof}

\begin{remark}\label{scheme alg group regular element contained in unique Cartan subgroup}
We note that condition of \cref{scheme alg group smooth twisted product morphism for Cartan group and eigenvalue of 1}~(b) is independent of $C$ (it only concerns the action of $g\in G(k)$ on $\g$). Therefore, we conclude that any regular point $g\in G(k)$ is in contained in a Cartan subgroup of $G$ (necessarily unique, in view of \cref{scheme alg group smooth conjugates containing g unique iff}~(\rmnum{1}), since the Cartan subgroups of $G$ are conjugate when $k$ is algebraically closed, cf. \cite{Chevalley1958} 6, th.4 (c)). Conversely, if a point $g\in G(k)$ is contained in a unique Cartan subgroup $C$ of $G$, then it is regular in view of \cref{scheme alg group smooth conjugates containing g unique iff}.
\end{remark}

\begin{example}\label{scheme alg group GL_2 regular element example}
Let $G=\GL_2$ be the general linear group of order $2$ defined over a field $k$ with $\char(k)\neq 2$. Then $G$ contains a maximal torus $T$ formed by diagonal matrices, and we have the following computation:
\[\begin{pmatrix}
a&b\\
c&d
\end{pmatrix}\begin{pmatrix}
\lambda_1&0\\
0&\lambda_2
\end{pmatrix}\begin{pmatrix}
a&b\\
c&d
\end{pmatrix}^{-1}=\frac{1}{ad-bc}\begin{pmatrix}
\lambda_1ad-\lambda_2bc&(\lambda_2-\lambda_1)ab\\
(\lambda_1-\lambda_2)dc&\lambda_2ad-\lambda_1bc
\end{pmatrix}.\]
Therefore, we see that the Cartan subgroup $C=Z_G(T)$ is equal to $T$, and its normalizer $N=N_G(C)=N_G(T)$ is equal to monomial matrices of order $2$ (cf. \cref{scheme alg group diagonalizable extension not diagonalizable example}). We see that the identity component of $N$ is equal to $C$, and another component is given by $(\begin{smallmatrix}0&1\\1&0\end{smallmatrix})\cdot C$, which justifies \cref{scheme alg group smooth twisted product morphism smooth iff}. On the other hand, for an element $g=(\begin{smallmatrix}a&b\\c&d\end{smallmatrix})$, the matrix of $\Ad(g)$ under the canonical basis $(E_{ij})_{1\leq i,j\leq 2}$ is computed as follows:
\[\Ad(g)=\frac{1}{ad-bc}\begin{pmatrix}
ad&-ac&bd&-bc\\
-ab&a^2&-b^2&ab\\
cd&-c^2&d^2&-cd\\
-bc&ac&-bd&ad
\end{pmatrix}\]
so its characteristic polynomai $P(\Ad(g),t)$ is given by
\[P(\Ad(g),t)=(t-1)^2\Big(t^2-\frac{(a^2+2bc+d^2)}{ad-bc}t+1\Big).\]
Therefore, the subset $U$ of regular elements of $G$ is defined by the open subset
\[U=\{g\in G:2(ad-bc)-(a^2+2bc+d^2)\neq 0\},\]
which is easily seen to be an open dense subset of $G$.
\end{example}

\begin{corollary}\label{scheme alg group smooth regular element of subgroup prop}
Let $H$ be a connected smooth algebraic subgroup of $G$ containing a Cartan subgroup of $G$.
\begin{enumerate}
    \item[(a)] Let $C$ be an algebraic subgroup of $H$. For $C$ to be a Cartan subgroup of $H$, it is necessary and sufficient that it be a Cartan subgroup of $G$.
    \item[(b)] Let $g\in G(k)$ and $d$ be the degree of $\psi$ (cf. \cref{scheme alg group smooth twisted product morphism smooth prop}). For $g$ to be a regular point of $G$, it is necessary and sufficient that there exists exactly $d$ conjugates $H_i$ of $H$ containing $g$, and that for each $i$, $g$ is a regular element of $H_i$ (or equivalently, is regular in one of them). If this is the case, and if $C$ is the unique Cartan subgroup containing $g$ (cf. \cref{scheme alg group regular element contained in unique Cartan subgroup}), then the conjugates of $H$ containing $g$ are the conjugates of $H$ containing $C$. 
    \item[(c)] Let $g\in H(k)$, for $g$ to be regular in $G$, it is necessary and sufficient that it be rgular in $H$, and we have $(\g/\h)^{\Ad(g)}=0$. 
\end{enumerate}
\end{corollary}
\begin{proof}
Under one of the hypothesis of (a), the unique maximal torus $T$ of $C$ is a maximal torus of $G$ and of $G$ ($H$ having the same reductive rank as $G$, cf. \cref{scheme alg smooth connected subgroup contain Cartan iff}), so as $Z_H(T)\sub Z_G(T)$ are smooth connected groups with the same dimension, they are equal, so it is equivalent to saying that $C$ is equal to one of these subgroups, whence the assertion of (a).\par
Now consider the assertions of (b). Suppose first that $g$ is regular in $G$, let $C$ be the unique Cartan subgroup of $G$ containing $g$, then in view of \cref{scheme alg group smooth twisted product morphism smooth prop}~(b), there exists exactly $d$ conjugates $H_i$ of $H$ containing $C$. As $(\g/\c)^{\Ad(g)}=0$, i.e. $\Ad(g)$ has no eigenvalue $1$ on $\g/\c$, we have a fortiori $(\g/\h_i)^{\Ad(g)}=0$, hence in view of \cref{scheme alg group smooth twisted product morphism smooth prop}~(c), there exists exactly $d$ conjugates of $H$ containing $g$, namely the $H_i$. For such an $H_i$, a Cartan subgrou of $H_i$ containing $g$ is a Cartan subgroup of $G$ containing $g$ in view of (a), hence is equal to $C$; this proves that $g$ is regular in $H_i$ (\cref{scheme alg group regular element contained in unique Cartan subgroup}). Conversely, suppose that there exists at most $d$ conjugates $H_i$ of $H$ containing $g$, and that $g$ is regular in one of them, which can be assumed to be equal to $H$. Let us prove that $g$ is regular in $G$. As $g$ is regular in $H$, it is contained in a unique Cartan subgroup $C$ of $H$, which by (a) is a Cartan subgroup of $G$. Let $C'$ be a Cartan subgroup of $G$ containing $g$, let us prove $C'=C$ (which implies that $g$ is regular in $C$). Indeed, by virtue of \cref{*}~(b), there exist exactly $d$ conjugates of $H$ containing $C'$, and since they all contain $g$, they are necessarily the $H_i$, therefore the $H_i$, and in particular $H$, contain $C'$. Now $C$ and $C'$ are two Cartan subgroups of $H$ (in view of (a)), which contains the same regular element $g$ of $H$, hence they are equal.\par
Finally, let $g\in H(k)$. Denote by $\nu(u)$ the dimension of the kernel of $\id-u$, for an endomorphism $u$ of a finite-dimensional vector space, we have
\[\nu(\Ad_{\g}(g))=\nu(\Ad_\h(g))+\nu(\Ad_{\g/\h}(g))\]
now the two members on the right are respectively $\geq\rho_n(H)=\rho_n(G)$ and $\geq 0$, so we have $\nu(\Ad_{\g}(g))=\rho$ if and only if $\nu(\Ad_{\h}(g))=\rho$ if and only if $\nu(\Ad_{\g/\h}(g))=0$, i.e. $g$ is regular in $G$ if and only if it is regular in $H$ and $\Ad_{\g/\h}(g)$ has no nontrivial invariants.
\end{proof}

\begin{example}
In the statement of \cref{scheme alg smooth connected subgroup contain Cartan iff}, one cannot weaken the condition (\rmnum{3}) by assuming only that $H$ contains a maximal torus and has finite index in its normalizer, even if we require this normalizer to be smooth, i.e. that we have $H=N^0$, and even when $G$ is affine solvable. As an example, consider the group $G$ of $\GL_3$ of the form
\[G(k)=\left\{\begin{pmatrix}
t&a&c\\
0&1&b\\
0&0&1
\end{pmatrix}:t\in k^\times,a,b,c\in k\right\},\]
and the subgroup $H$ of matrices of the preceding form, with $b=c=0$. The maximal torus $T$ of $G$ is given by matrices $g$ with $a=b=c=0$, and its centralizer $C$ and normalizer $N$ are given by matrices of the form
\[C(k)=N(k)=\left\{\begin{pmatrix}
t&0&0\\
0&1&b\\
0&0&1
\end{pmatrix}:t\in k^\times,b\in k\right\}.\]
On the other hand, the normalizer of $H$ is equal to $H$, so $H$ contains a maximal torus $T$ and is self-normalizing, but it does not contain the Cartan subgroup $C$ defined by $T$.
\end{example}

\paragraph{The case of a group over arbitrary base}
Suppose first that we are over a base field $k$, not necessarilly algebraically closed. As the conditions (\rmnum{1}'), (\rmnum{4}'), (\rmnum{5}) and (\rmnum{6}) of \cref{scheme alg smooth connected subgroup contain Cartan iff} are invariant under base field change, we see by passing to the algebraic closure $\bar{k}$ of $k$ that they are equivalent to each other, and equivalent to that $H_{\bar{k}}$ contains a Cartan subgroup of $G_{\bar{k}}$. If these conditions are satisfied, then (with the notations of \cref{scheme alg group smooth twisted product morphism smooth prop}) the rational function field $k(X)$ is a finite separable extension of $k(G)$, of degree $d$, which is independent of base field change. If $U$ is the largest open subset of $G$ such that $\psi$ induces a morphism $\psi^{-1}(U)\stackrel{\sim}{\to}U$ which is finite and \'etale, then the formation of $U$ commutes with base field change (cf. \cref{scheme locus of fpqc local prop stable under base change}). If $\psi$ is birational, then $U$ is also the largest open subset of $G$ such that $\psi$ induces an isomorphism $\psi^{-1}(U)\stackrel{\sim}{\to}U$, and then for $g\in U(k)$, there exists a unique subgroup $H'$ of $G$, conjugate to $H$ over the algebraic closure $\bar{k}$, such that $g\in H'$.\par
A point $g\in G(k)$ is called \textbf{regular} if it is regular as an element $G(\bar{k})=G_{\bar{k}}(\bar{k})$. More generally, the construction of \cref{scheme alg group smooth open of regular element given by G_b} gives us an open subset of $G$, whose formation commutes with any base field change, so is called the \textbf{open subset of regular points of $G$}. This is also characterized by the fact that for any algebraically closed field extension $K$ of $k$ and any point $g\in G(K)$, $g$ is a regular point of $G_K$ if and only if $g\in U(K)$. If $g\in U(K)$, we shall see that $g$ is contained in a unique Cartan subgroup of $G$.\par
Let $G$ be a smooth and separated scheme over $S$, of finite type over $S$ and has connected fibers. Consider the functor $\mathscr{C}:\Sch_{/S}^{\op}\to\Set$ defined by
\[\mathscr{C}(S')=\{\text{set of Cartan subgroups of $G_{S'}$}\}\]
Suppose that this functor is representable by a smooth scheme over $S$ (we will give in (\cite{SGA3-2} \Rmnum{15}) an equivalent condition for this to be so, but we already know that this hypothesis is satisfied if $G$ is affine over $S$ of locally constant reductive rank (\cref{scheme smooth affine functor of Cartan subgroup representable if constant reductive rank}), or more generally if $G$ admits locally for the fpqc topology a maximal torus (\cite{SGA3-2} 7.1 (a)), for example if $S$ is the spectrum of a field). Let $X$ be the "universal Cartan subgroup of the $\mathscr{C}$-group $G_{\mathscr{C}}$". As a scheme over $S$, $X$ then represents the following functor (cf. \cref{scheme group affine smooth universal subgroup multiplicative})
\[X(S')=\{\text{set of couples $(C,g)$, $C$ being a Cartan subgroup of $G_{S'}$, and $g$ be a section of $C$ over $S'$}\}.\]
Consider the canonical projection $\psi:X\to G,(C,g)\mapsto g$. We then have the following theorem:

\begin{theorem}\label{scheme group regular set twisted product morphism birational}
Under the preceding conditions on $G$ and notations, let $U$ be the set of $g\in G$ such that $g$ is a regular element in the fiber $G_s$. Then $U$ is open, and it is also the largest open subset $U$ of $G$ such that $\psi$ induces an isomorphism $\psi^{-1}(U)\stackrel{\sim}{\to} U$.
\end{theorem}
\begin{proof}
We first prove that $U$ is open. By the hypothesis of the representability of $\mathscr{C}$ as a smooth scheme over $S$. as its structural morphism is evidently surjective, we conclude that $G$ admits locally for the \'etale topology a Cartan subgroup (cf. \cref{EGA4-4} 17.16.3), and that the nilpotent rank of the fibers of $G$ is locally constant. The same is true for the dimension of fibers of $G$ (as $G$ is smooth over $S$), and by taking localization of $S$, we can suppose that they are both constant, say $\rho$ and $n$. Consider the \textbf{Kill polynomial}
\[P_G(t)=t^n+c_1t^{n-1}+\cdots+c_n\in A[t],\quad A=\Gamma(G,\mathscr{O}_G)\]
(the polynomial in the proof of \cref{scheme alg group smooth twisted product morphism for Cartan group and eigenvalue of 1}.) The restriction of this polynomial to fibers $G_s$ of $G$, and in particular to the maximal points of $S$, is divisible by $(t-1)^\rho$, which is expressed by the fact that certain linear combinations with integer coefficients of the $c_i$ are zero on the fibers $G_s$. If $S$ is reduced (which we can assume to establish that $U$ is open), it follows that they are themselves zero, so that the Killing polynomial itself is divisible by $(t-1)^\rho$, say
\[P_G(t)=(t-1)^\rho(t^{n-\rho}+b_1t^{n-r-1}+\cdots+b_{n-\rho}).\]
Let $b$ be the sum of these coefficients $b_0=1,b_1,\dots,b_{n-\rho}$, then in view of \cref{scheme alg group smooth open of regular element given by G_b} applied to the fibers of $G$, we see that $U=G_b$, which proves that $U$ is open.\par
To show that $\psi^{-1}(U)\to U$ is an isomorphism, we are reduced by (\cite{SGA1} \Rmnum{1} 5.7) to verify this fiber by fiber, hence to the case where the base is a field, which we may assume to be algebraically closed. Then there exists a Cartan subgroup $C$ of $G$, and if $N$ is its normalizer, then $\mathscr{C}$ is identified, by the conjugation theorem (\cite{SGA3-2} \Rmnum{12} 7.1 (a) et (b)), with $G/N$, and the morphism $\psi:X\to G$ is none other than that defined in \ref{scheme alg group regular element paragraph}. We then conclude by \cref{scheme alg group smooth twisted product morphism for Cartan group and eigenvalue of 1}~(b). The same reasoning shows that $U$ is the largest open subset of $G$ such that $\psi$ induces an isomorphism $\psi^{-1}(U)\to U$.
\end{proof}


\begin{corollary}\label{scheme group regular section contained in unique Cartan subgroup}
Under the conditions of \cref{scheme group regular set twisted product morphism birational}, let $g$ be a regular section of $G$, i.e. such that for any $s\in S$, $g(s)$ is a regular point of $G_s$. Then there exists a unique Cartan subgroup $C$ of $G$ such that $g$ is a section of $C$.
\end{corollary}
\begin{proof}
The hypothesis on $g$ signifies that $g$ is a section of $U$, and the conclusion that there exists a unique section of $X$ which dominates $g$ is none other an equivalent expression that $\psi^{-1}(U)\to U$ is an isomorphism.
\end{proof}

Note that the open subset $\psi^{-1}(U)$ of the Cartan subgroup $X$ of $G_\mathscr{C}$ is none other than the open subset of $X$ formed by points of $X$ which are regular in $G_\mathscr{C}$ (regular in the fibers). We thus obtain a natural "fibration" of the open dense subset $U$ of regular points of $G$ over the scheme $\mathscr{C}$, whose fiber at a point $x$ of $\mathscr{C}$ (corresponding to a Cartan subgroup $C$ of $G$) equals to the set of sections of $C_{\kappa(x)}$ over $\kappa(x)$. We conclude for example the following corollary:

\begin{corollary}\label{scheme alg group generic Cartan subgroup exist}
Let $G$ be a connected smooth algebraic group over the field $k$, $\mathscr{T}$ be the scheme of maximal tori of $G$ (isomorphic to the scheme of Cartan subgroups of $G$). Then the function field $k(G)$ of $G$ is isomorphic to the function field of a connected nilpotent affine smooth algebraic group $C$ over the function field $k(\mathscr{T})$ of $\mathscr{T}$ (called the \textbf{generic Cartan subgroup} of $G$). If $G$ is affine with zero unipotent rank, i.e. if the Cartan subgroups of $G_{\bar{k}}$ are tori, then $k(G)$ is a unirational extension of $k(\mathscr{T})$. 
\end{corollary} 

Of course, the generic Cartan subgroup is just the Cartan subgroup of $G_{k(\mathscr{T})}$ correponding to the generic fiber of $X$ over $\mathscr{T}$. It only remains to prove the last assertion of \cref{scheme alg group generic Cartan subgroup exist}, which is contained in the following well-known result (due to Chevalley):

\begin{lemma}\label{scheme group torus function field is unirational}
Let $k$ be a field, $T$ be a torus over $k$, $k(T)$ be the rational function field of $T$. Then $k(T)$ is a unirational extension of $k$, i.e. is contained in a purely transcendental extension of $k$.
\end{lemma}
\begin{proof}
Let $k'$ be a finite separable extension of $k$ which split $X$ (\cref{scheme alg group multiplicative cat equivalence}), then $T\otimes_kk'$ is a rational variety, i.e. admits an open dense subset isomorphic to an open dense subset of the affine space $\A_k^n$, hence $T'=\Res_{\Spec(k')/\Spec(k)}T_{k'}$ is a rational variety (it admits an open dense subset isomorphic to an open dense subset $\Res_{\Spec(k')/\Spec(k)}\A_{k'}^n$, which is isomorphic to $\A_{k}^{mn}$, where $m=[k':k]$). Consider the norm hommorphism $T'\to T$ (defined for abelian algebraic groups over $k$); then the composition $T\to T'\to T$ is the $m$-th power on $T$, hence dominant, so $T'\to T$ is dominant, which proves that $T$ is unirational.
\end{proof}

Now let us return to the conditions of \cref{scheme group regular set twisted product morphism birational}, and suppose that $G$ admits locally for the fpqc topology a maximal torus (cf. \cite{SGA3-2} \Rmnum{12} 7.1). Let $T$ be the maximal torus of the Cartan subgroup $C$ of $G$, so that the morphism $j:C\to G$ induces a morphism $T\to G$ whose image is formed setwisely of semi-simple elements of fibers of $G$ (\cite{SGA3-2} \Rmnum{12} 8). Further, it follows from \cref{scheme group regular set twisted product morphism birational} that the restriction of $j$ to the open set $T^{\reg}$ of regular points of $T$ induces a closed immersion
\[T^{\reg}\to U=G^{\reg}.\]

\begin{corollary}\label{scheme group smooth sp functor of regular section in torus representable}
Let $G$ be a smooth and separated $S$-group of finite type over $S$ with connected fibers, admitting locally for the fpqc topology a maximal torus. Let $Z:\Sch_{/S}\to\Set$ be the functor defined by
\[Z(S')=\{\text{set of regular sections of $G_{S'}$ over $S'$ which is contained in a maximal torus of $G_{S'}$}\}.\]
Then $Z$ is representable by a closed scheme of the open subset $U=G^{\reg}$ of $G$, and is smooth over $S$. 
\end{corollary}
\begin{proof}
In fact, if $T$ is a maximal torus of $G$, then this functor is represented by the intersection of $U$ with the image $\psi(\psi^{-1}(T))$. As $\psi$ is birational on $U$ and $\psi^{-1}(T)\cap\psi^{-1}(U)$ is closed in $\psi^{-1}(U)$, we see that this intersection is closed in $U$.
\end{proof}

\begin{corollary}\label{scheme group smooth sp conjugate by regular in torus dominant}
Under the conditions of \cref{scheme group smooth sp functor of regular section in torus representable}, let $C$ be a Cartan subgroup of $G$, and consider the morphism
\[\varphi:Z\times_SC\to G,\quad (g,h)\mapsto\inn(g)h=ghg^{-1}.\]
Then $\varphi$ is dominant.
\end{corollary}
\begin{proof}
It suffices to prove this fiber by fiber, so we are reduced to the case where $S$ is the spectrum of an algebraically closed field. Let $T$ be the maximal torus of $C$, $t_0$ be an element of $T(k)$ regular in $G$, $c_0$ be a point of $C(k)$ regular in $G$, consider $\varphi^{-1}(\varphi(t_0,c_0))$, whose rational points over $k$ are the couples $(t,c)$, with $t\in Z(k)$, $c\in C(k)$, and such that $\inn(t)c=\inn(t_0)c_0$, i.e. $c=\inn(t^{-1}t_0)c_0$. Its points therefore correspond to $t\in Z(k)$ such that $\inn(t^{-1}t_0)c_0\in C$, or equivalently, as $c_0$ is regular, such that $tt_0^{-1}\in N$ (normalizer of $C$), i.e. $t\in N$. By considering the the points $t\in Z(k)$ such that $t\in C(k)$, we obtain a clopen subset of this fiber (cf. \cref{scheme alg smooth connected subgroup contain Cartan iff}), so there is a connected component of $\varphi^{-1}(\varphi(t_0,c_0))$ isomorphic to $T$. The generic fiber of $\varphi$ therefore has dimension $\leq\dim(T)$, and
\[\dim(\im\varphi)\geq\dim(Z\times_SC)-\dim(T)=\dim(Z)+\dim(C)-\dim(T),\]
but we have
\[\dim(Z)=\dim(T^\reg)=\dim(G)-\dim(C)+\dim(T),\]
whence finally $\dim(\im\varphi)\geq\dim(G)$, so $\varphi$ is dominant. 
\end{proof}

\begin{remark}
Note that the above reasoning also shows that the connected component of the fiber $\varphi^{-1}(\varphi(t_0,c_0))$ at $(t_0,c_0)$ is isomorphic to $T$; in particular, it is smooth over $k$, and has the same dimension as the generic fiber, which implies that $\varphi$ is in fact smooth at $(t_0,c_0)$ (which we should also be able to verify by calculating the tangent map). It follows that under the conditions of \cref{scheme group smooth sp conjugate by regular in torus dominant}, the induced morphism $Z\times_SC^{\reg}\to G^{\reg}$ (where we set $C^{\reg}=C\cap G^{\reg}$) is a smooth morphism. Similarly, we see that the analogous morphism $Z\times T^{\reg}\to Z$ (where $T$ is a maximal torus of $G$) is smooth. More generally, for any connected smooth normal subgroup $H$ of $C$ containing a regular element $c_0$ of $G(k)$, the image of $Z\times_SH\to G$ is dense in that of $G\times_SH\to G$.
\end{remark}

\subsection{Cartan subalgebras and regular elements of Lie algebras over a field}
Let $\g$ be a Lie algebra over a ring $k$. For any $x\in\g$, recall that we denote by $\ad(x)$ the endomorphism
\[\ad(x)(y)=[x,y]\]
of $\g$, which is a derivation of the Lie algebra $\g$. For any derivation $D$ on $\g$, the nilspace of $D$, i.e. the union of the kernel of the powers of $D$, which is a Lie subalgebra of $\g$, as we see from the Leibniz formula
\[D^n([x,y])=\sum_{p=0}^{n}\binom{n}{p}[D^px,D^{n-p}y].\]
As in \autoref{Lie algebra Cartan subalgebra section}, for $a\in\g$, we denote by $\g^0(a)$ the nilspace of $\ad(a)$, which is the union of kernel of $\a(a)^n$.

\begin{proposition}\label{scheme group Lie algebra nilspace self-normalizing}
For any $a\in\g$, its nilspace $\g^0(a)$ is a Lie subalgebra of $\g$, and is self-normalizing.
\end{proposition}
\begin{proof}
It remains to prove that it is self-normalizing, i.e. that any element of $\g/\g^0(a)$ annihilated by the adjoint representation of $\g^0(a)$ on $\g/\g^0(a)$ is zero, but this is trivial (because any element in this quotient annihilated by $\ad(x)$ is zero).
\end{proof}

In this subsection, we suppose that $k$ is a field (not necessarily of characteristic zero), and $\g$ be finite dimension over $k$. Following \cite{SGA3-2}, we denote by $W(\g)$ the scheme over $k$ defined by $\g$, whose points on a $k$-algebra $A$ are the elements of $\g\otimes_kA$. In other words, we have
\[\Hom_{\Spec(k)}(\Spec(A),W(\g))=\Hom_k(k,\g\otimes_kA)=\Hom_{k\dash\Alg}(\bm{S}_k(\g^\vee),A)\]
so $W(\g)$ is represented by the $k$-algebra $\bm{S}_k(\g^\vee)$, the symetric algebra over $\g^\vee$ ($W(\g)$ is the functor $\mathbf{W}(\g)$ defined in \ref{scheme module Gamma functor paragraph}, cf. \cref{scheme Gamma functor representable by Spec of sym} and \cref{scheme Gamma module functor isomorphic if locally free}). If $x\in\g$, the characteristic polynomial of $\ad(a)$ is also called the \textbf{characteristic polynomial} or \textbf{Killing polynomial} of $a$ in $\g$, say
\[P_\g(x,t)=t^n+c_1(a)t^{n-1}+\cdots+c_n(a)\]
where $n=\rank_k(\g)$ and $c_i(a)\in k$. By considering these polynomials for $a\in\g\otimes_kA$, where $A$ is an arbitrary $k$-algebra, we see that the $c_i(a)$ comes from well-defined sections of the structural sheaf of $W(\g)$, i.e. from elements of $\bm{S}_k(\g^\vee)$ (if $k$ is an infinite field, the $c_i$ are determined by the corresponding polynomial functions $g\to k$, but this is no longer true if $k$ is finite). Let $r$ be the largest integer such that the Killing polynomial
\[P_\g(t)=t^n+c_1t^{n-1}+\cdots+c_n\]
is divisible by $r$, i.e. we have
\[P_\g(t)=t^n+c_1t^{n-1}+\cdots+c_{n-r}t^r,\quad c_{n-r}\neq 0.\]
The integer $r$ is called the \textbf{nilpotent rank} of the Lie algebra $\g$ (this coincides with the rank of $\g$ defined in \autoref{Lie algebra Cartan subalgebra section}, if $k$ has characteristic zero), and denoted by $\rho_n(\g)$. It is clearly invariant under base field extension.

\begin{proposition}\label{scheme group Lie algebra g^0(a) regular iff c_n-r}
Let $r$ be the nilpotent rank of $\g$, and $x\in\g$. Then we have
\[\dim_k(\g^0(a))\geq r,\]
with equality if and only if $c_{n-r}(a)\neq 0$. In this case, $\g^0(a)$ is a nilpotent subalgebra of $\g$.
\end{proposition}
\begin{proof}
The first assertion is trivial from our definition, since $\dim_k(\g^0(a))$ is the multiplicity of eigenvalue $0$ for $\ad(a)$. Now assume that $c_{n-r}(a)\neq 0$, we need to show that $\g^0(a)$ is nilpotent, which signifies that for any $x\in\g^0(a)$, $\ad_{\g^0(a)}(x)$ is a nilpotent endomorphism (cf. \cref{Lie algebra nilpotent iff ad nilpotent}). For this, we may assume that $k$ is algebraically closed, then as $\ad(a)$ is injective on $\g/\g^0(a)$, by Nullstellensatz there exists a nonempty open subset $U$ of $W(\g^0(a))$ such that for any $x\in U(k)$, $\ad_{\g/\g^0(a)}(x)$ is injective\footnote{We note that $W(\g^0(a))(k)=\g^0(a)$, so to define the open subset $U$, it suffices by Nullstellensatz to consider the subset of $x\in\g^0(a)$ such that $\ad_{\g/\g^0(a)}(x)$ is injective, and prove that it is open. But this is immediate: this is the complement of the kernel of the adjoint representation $\ad:\g^0(a)\to\GL(\g/\g^0(a))$.}, hence $\g^0(x)\sub\g^0(a)$. We can also suppose that $U$ is contained in the open subset of points where $c_{n-r}$ do not vanish, i.e. the regular elements of $\g$ (since this open subset is nonempty in view of $c_{n-r}(a)\neq 0$). Then $g^0(x)$ has the same dimension as $\g^0(a)$ (both equal to the nilpotent rank of $\g$), so they are equal, i.e. for any $x\in U(k)$, $\ad_{\g^0(a)}(x)$ is nilpotent. Since the nilpotent condition for $x\in\g^0(a)$ can be detected by the vanishing of the Killing polynomial of $\g^0(a)$ and $U$ is a dense subset of $W(\g^0(a))$ ($W(\g^0(a))$ being irreducible, since $\bm{S}_k(\g^0(a)^\vee)$ is an integral domain), we conclude that it is equal to $W(\g^0(a))$, i.e. $\ad_{\g^0(a)}(x)$ is nilpotent for $x\in\g^0(a)$, so $\g^0(a)$ is nilpotent.
\end{proof}

We say that an element $a\in\g$ is \textbf{regular} if $c_{n-r}(a)\neq 0$, i.e. if $\dim_k(\g^0(a))=r$. If $k$ is infinite, this also signifies that $\dim_k(\g^0(a))$ is as small as possible\footnote{If $k$ is finite, it may happen that $c_{n-r}\neq 0$ in $\bm{S}_k(\g^\vee)$, but $c_{n-r}(a)=0$ for any $a\in\g$. In this case, for any $a\in\g$, we will have $\dim_k(\g^0(a))>r$.} (for $a$ runs through $\g$). In any case, the notion of regularity is invariant under base field change, and the set of points of $W(\g)$ which are regular (i.e. those come from the regular points of $W(\g)$ with values in a suitable extension field of $k$) is open, because it is identified with $W(\g)_{c_{n-r}}$ (the subset of $W(\g)$ where $c_{n-r}$ is invertible). 

\begin{corollary}\label{scheme group Lie algebra g^0(a) maximal nilpotent}
Let $a\in\g$ be a regular element and $\h$ be a subalgebra of $\g$ containing $a$. Then $\h$ is nilpotent if and only if $\h\sub\g^0(a)$. In particular, $\g^0(a)$ is a maximal nilpotent subalgebra of $\g$.
\end{corollary}
\begin{proof}
As $\g^0(a)$ is nilpotent, the relation $\h\sub\g^0(a)$ implies that $\h$ is nilpotent; conversely, if $\h$ is nilpotent, then $\ad_{\h}(a)$ is nilpotent, so $\h$ is contained in the nilspace of its element $a$, i.e. $\h\sub\g^0(a)$.
\end{proof}

\begin{proposition}\label{scheme group Lie subalgebra maximal nilpotent contain regular iff}
Suppose that $k$ is infinite and let $\d$ be a subalgebra of $\g$. Consider the following conditions:
\begin{enumerate}
    \item[(\rmnum{1})] $\d$ is maximal nilpotent and contains a regular element of $\g$.
    \item[(\rmnum{1}')] $\d$ is of the form $\g^0(a)$, where $a$ is a regular element of $\g$.
    \item[(\rmnum{2})] $\d$ is nilpotent and of the form $\g^0(a)$, where $a\in\g$.
    \item[(\rmnum{2}')] $\d$ is nilpotent, and there exists $a\in\d$ such that $\ad_{\g/\d}(a)$ is injective.
    \item[(\rmnum{3})] $\d$ is nilpotent and self-normalizing.
\end{enumerate}
We have the implications (\rmnum{1})$\Leftrightarrow$(\rmnum{1}')$\Rightarrow$(\rmnum{2})$\Leftrightarrow$(\rmnum{2}')$\Leftrightarrow$(\rmnum{3})\footnote{We shall see in \cref{scheme alg group smooth Lie subalgebra Cartan and regular element iff} that if $\g$ is the Lie algebra of a smooth algebraic group, then these conditions are all equivalent.}.
\end{proposition}
\begin{proof}
The equivalence of (\rmnum{1}) and (\rmnum{1}') is trivial by \cref{scheme group Lie algebra g^0(a) maximal nilpotent}, and these conditions imply trivially (\rmnum{2}). The equivalence of (\rmnum{2}) and (\rmnum{2}') is equally trivial, as well as (\rmnum{2}')$\Rightarrow$(\rmnum{3}) (cf. \cref{scheme group Lie algebra nilspace self-normalizing}). It remains to prove (\rmnum{3})$\Rightarrow$(\rmnum{2}'), which follows from \cref{*} below (this is the point where we use the fact that $k$ is infinite).
\end{proof}

\begin{lemma}\label{scheme group Lie algebra nilpotent acting non injective eigenvector}
Let $\d$ be a nilpotent Lie algebra over an infinite field $k$, acting on a finite dimensional vector space $V$. Suppose that for any $x\in\d$, the endomorphism $u(x)$ is not injective, then there exists a nonzero element $v\in V$ annihilated by $\d$.
\end{lemma}
\begin{proof}
We can suppose that $k$ is algebraically closed and $\d$ is of finite dimension. We then see that $V$ is a direct sum of finitely many nonzero subspaces $V_i$ stable under $\d$, such that for any $i$, and any $x\in\d$, $u(x)|_{V_i}$ has a unique eigenvalue $\lambda_i(x)$ (cf. \cite{Borubaki_LieI} \Rmnum{1}, \S 4, Exercice 22). Let $c_i(x)$ be the constant term of the characteristic polynomial of $u(x)|_{V_i}$, so that $\lambda_i(x)=0$ if and only if $c_i(x)=0$. Then $c_i$ is a polynomial function over $\d$, and the hypothesis signifies that the union of zero sets of $c_i$ is equal to $\d$. Hence one of the $c_i$ is zero\footnote{Recall that over an infinite field, a vector space can not be written as a finite union of its proper subspaces.}, which reduces us to the case (by replacing $V$ with $V_i$) where $V$ is such that $u(x)$ ($x\in\d$) are nilpotent. But then Engel's theorem (\cref{Lie subalg of gl(V) Engel theorem}) implies that there exists nonzero $v\in V$ such that $x\cdot v=0$ for $x\in\d$.
\end{proof}

We easily see that ($k$ being finite) the equivalent conditions (\rmnum{1}) and (\rmnum{1}') of \cref{scheme group Lie subalgebra maximal nilpotent contain regular iff} are invariant under base field extension. If they are satisfied, we then say that $\d$ is a \textbf{Cartan subalgebra} of $\g$. In the general case (where $k$ is not necessarily infinite), we say that $\d$ is a Cartan subalgebra of $\g$, if it becomes a Cartan subalgebra of $\g$ after a (hence any) base field extension $k\to k'$, with $k'$ infinite. This then implies that $\d$ is nilpotent and self-normalizing.

\begin{proposition}\label{scheme group Lie algebra regular element in unique Cartan subgroup}
Let $\g$ be a Lie algebra over a field $k$.
\begin{enumerate}
    \item[(a)] If $a\in\g$ is a regular element, it is contained in a unique Cartan subalgebra of $\g$.
    \item[(b)] Let $\d$ be a Cartan subgroup of $\g$ and $a\in\d$, then $a$ is regular in $\g$ if and only if $\ad_{\g/\d}(a)$ is injective.
\end{enumerate}
\end{proposition}
\begin{proof}
In fact, for (a) we note that if $a$ is regular then $\g^0(a)$ is a Cartan subalgebra of $\g$ (because this is true over an infinite field $k'$ over $k$), and it then follows from \cref{scheme group Lie algebra g^0(a) maximal nilpotent} that any Cartan subalgebra of $\g$ containing $a$ is equal to $\g^0(a)$. As for (b), we note that the dimension of $\g^0(a)$ is equal to the sum of $\d^0(a)$ and $(\g/\d)^0(a)$, and as the first is equal to $r$, this sum is equal to $r$ if and only if $\ad_{\g/\d}(a)$ is injective.
\end{proof}

\begin{corollary}\label{scheme group Cartan subalgebra fixed iff regular element}
Let $a$ be a regular element of $\g$, $\d$ be a Cartan subalgebra of $\g$ containing $a$, $A$ be a $k$-algebra, $\g_A$ and $\d_A$ be the Lie algebras over $A$ induced by base change, and $a_A$ be the image of $a$ in $\g_A$. Let $u$ be an automorphism of $\g$, then for that $u(\d_A)=\d_A$, it is necessary and sufficient that $u(a_A)\in\d_A$.
\end{corollary}
\begin{proof}
The condition is clearly necessary, to show that it is also sufficient, note that if $u(a_A)\in\d_A$, then $\d'=u(\d_A)$ is a subalgebra of $\g$ containing $a_A$, and any element $b\in\d'$ is such that $\ad_{\d'}(b)$ is nilpotent (because $\d'$ is isomorphic to $\d_A$, which has this property). Putting $b=a_A$, we then see that the nilspace $\d_A=\g_A^0(a_A)$ contains $\d'$. As $\d'$ is locally a direct factor of the module $\g_A$ ($\d$ being so), and hence of $\d_A$, and it is a projective module of the same rank $r$ as $\g_A$, we conclude that $\d'=\d_A$.
\end{proof}

\begin{example}
Let $\g=\gl_2$ be the Lie algebra of $\GL_2$ over a field $k$. Then for a general element $x=(\begin{smallmatrix}a&b\\c&d\end{smallmatrix})$, the matrix of $\ad(x)$ under the canonical basis $(E_{ij})_{1\leq i,j\leq 2}$ has the form
\[\ad(x)=\begin{pmatrix}
0&-c&b&0\\
-b&a-d&0&b\\
c&0&-a+d&-c\\
0&c&-b&0
\end{pmatrix}\]
so the characteristic polynomial of $\ad(x)$ is given by
\[P_\g(\ad(x),t)=t^2(t^2-(a^2+4bc+d^2-2ad)),\]
so $x$ is regular if and only if $a^2+4bc+d^2-2ad\neq 0$. For such an element, the Catan subalgebra $\g^0(x)$ is generated by the following matrices:
\[v_1=\begin{pmatrix}
a-d&b\\
c&0
\end{pmatrix},\quad v_2=\begin{pmatrix}
1&0\\
0&1
\end{pmatrix}\]
which are eigenvectors of $\ad(x)$ with eigenvalue $0$. In this case, it is easy to see that $\g^0(x)$ is abelian.
\end{example}

\begin{proposition}\label{scheme group Lie subalgebra contain Cartan subalgebra iff}
Let $\h$ be a Lie subalgebra of $\g$.
\begin{enumerate}
    \item[(a)] The following conditions are equivalent if $k$ is infinite:
    \begin{enumerate}
        \item[(\rmnum{1})] $\h$ contains a Cartan subalgebra $\d$ of $\g$.
        \item[(\rmnum{2})] $\h$ contains a regular element $a$ of $\g$, and an element $b$ such that $\ad_{\g/\h}(b)$ is injective.
        \item[(\rmnum{3})] $\h$ has the same nilpotent rank as $\g$, and contains a regular element $a$ of $\g$.  
    \end{enumerate}
    These conditions are invariant under base field change.
    \item[(b)] Suppose that the conditions in (a) are varified over an infinite field $k'$ over $k$. Let $a\in\h$, then $a$ is regular in $\g$ if and only if it is regular in $\h$ are $\ad_{\g/\h}(a)$ is injective, i.e. if and only if $\h^0(a)=\g^0(a)$ and if this is a Cartan subalgebra of $\h$.
    \item[(c)] Under the conditions of (b), let $\d$ be a subalgebra of $\h$. For that this is a Cartan subalgebra of $\h$, it is sufficient that it is a Cartan subalgebra of $\g$\footnote{We shall see in \cref{scheme alg group subalgebra of smooth subgroup Cartan subgroup iff} that if $\g$ and $\h$ are Lie algebras of smooth algebraic groups, then this conditions is also necessary.}. 
\end{enumerate}
\end{proposition}
\begin{proof}
We see immediately that the conditions (\rmnum{2}) and (\rmnum{3}) of (a) are invariant under base field extension of $k$ (assumed to be infinite), and that in statements (b) and (c), we can assume that $k$ is infinite. If $\h$ contains the Cartan subalgebra $\d=\g^0(a)$, then $a$ is a regular element of $\g$ such that $\ad_{\g/\h}(a)$ is injective, so (\rmnum{1})$\Rightarrow$(\rmnum{2}). Conversely, if (\rmnum{2}) holds, then for a "generic" element $a$ of $\h$, $a$ simultaneously satisfies the two conditions considered in (\rmnum{2}), so $\g^0(a)$ is a Cartan subalgebra of $\g$ and is contained in $\h$, whence (\rmnum{1}). Now we have seen that (\rmnum{1}) and (\rmnum{2}) are equivalent. Suppose that they are satisfied, and let $a$ be a variable element of $\h$, then
\begin{equation}\label{scheme group Lie subalgebra contain Cartan subalgebra iff-1}
\dim_k(\g^0(a))=\dim_k(\h^0(a))+\dim_k((\g/\h)^0(a)),
\end{equation}
On the other hand, the two terms on the right are respectively $\geq r'=\rho_n(\h)$ and $\geq 0$, and the equalities is attained for a "generic" element of $\h$. Moreover, we also have $\dim_k(\g^0(a))\geq r=\rho_n(\g)$, the equality being attained if and only if $a$ is regular in $\g$. We then conclude that we have $r=r'$, and that $a$ is regular in $\g$ if and only if the two terms on the right of (\ref{scheme group Lie subalgebra contain Cartan subalgebra iff-1}) are equal to $r'$ and $0$, respectively, i.e. if and only if $a$ is regular in $\h$ and $\ad_{\g/\h}(a)$ is injective; this proves (b), and (c) then follows trivially by consider a regular element of $\g$ in $\d$, so that $\g^0(a)=\d$. Moreover, the preceding result shows that (\rmnum{1})$\Rightarrow$(\rmnum{3}), and finally (\rmnum{3})$\Rightarrow$(\rmnum{1}) because with (\rmnum{3}), a generic element $a$ of $\h$ is regular in $\h$ and is in $\g$, hence $\h^0(a)\sub\g^0(a)$ are respectively Cartan subalgebras of $\h$ and $\g$. As $\g$ and $\h$ have the same nilpotent rank over $k$, the two subalgebras are therefore equal, which proves (\rmnum{1}).
\end{proof}

\paragraph{Case of the Lie algebra of a smooth algebraic group: density theorem}\label{scheme alg group smooth Lie algebra regular element paragraph}
Let $G$ be a smooth algebraic group over a field $k$ and let $\g$ be its Lie algebra. Let $\h$ be a Lie subalgebra of $\g$, and acts $G$ on $W(\g)$ by adjoint representation. We can then consider the closed subscheme $W(\h)$ of $W(\g)$, and introduce as in \autoref{scheme group regular element auxilary lemma subsection} the normalizer
\[N=N_G(\h)=N_G(W(\h))\]
which is a subgroup of $G$ (not necessarily smooth). Let $\n=\mathfrak{Lie}(N)$, and consider the twisted product 
\[X=G\times^NW(\h)\]
which is the quotient of $G\times_kW(\h)$ by the right action $(g,x)\cdot n=(gn,\Ad(n^{-1})x)$ of $N$. We then have the following canonical morphisms:
\[\begin{tikzcd}
G\times_kW(\h)\ar[d,swap,"q"]\ar[rd,"\varphi"]&\\
X\ar[r,"\psi"]&W(\g)
\end{tikzcd}\]
Again, for $a\in\g$, we denote by $M_a$ the subfunctor of $G$ defined such that $M_a(S')$ is the set of $g\in G(S')$ such that $a_{S'}\in g\cdot W(\g)(S')$; this is isomorphic to the inverse image of $a$ under $\varphi$.

\begin{theorem}\label{scheme alg group smooth Lie subalgebra contain Cartan iff}
With the preceding notations, suppose further that $k$ is infinite. Consider the following conditions:
\begin{enumerate}
    \item[(a1)] $\h$ contains a Cartan subalgebra $\d$ of $\g$.
    \item[(a2)] There exists $a\in\h$ such that $\ad_{\g/\h}(a)$ is injective.
    \item[(a3)] $\varphi:G\times_kW(\h)\to W(\g)$ is generically smooth.
    \item[(a4)] $\varphi:G\times_kW(\h)\to W(\g)$ is dominant, and $\h$ has the same nilpotent rank as $\g$.
    \item[(b1)] $\psi:X\to W(\g)$ is generaically smooth and $\h=\n$.
    \item[(b2)] $\psi:X\to W(\g)$ is dominant and $\h=\n$.
    \item[(b3)] $\psi:X\to W(\g)$ is dominant.
    \item[(c1)] $\psi:X\to W(\g)$ is dominant and $N$ is smooth.
    \item[(c2)] $\psi:X\to W(\g)$ is dominant, $\h=\n$, and $\h$ is the Lie algebra of a smooth subgroup $H$ of $G$.
    \item[(c3)] $\psi:X\to W(\g)$ is genraically quasi-finite, and $\h=\n$.
    \item[(d1)] $\psi:X\to W(\g)$ is generically \'etale, and $\h=\n$.
    \item[(d2)] There exists $a\in\h$ such that $N$ and the transporter $M_a$ coincide in a neighborhood of $e$, and $\h=\n$.
    \item[(d3)] There exists a smooth algebraic subgroup $H$ of $G$ with Lie algebra $\h$, and $\h$ contains a Cartan subalgebra $\d$ of $\g$.
\end{enumerate}
Then we have the following implications:
\[\begin{tikzcd}
&\textup{(d1)}\ar[r,Leftrightarrow]&\textup{(d2)}\ar[d,Rightarrow]\ar[r,Leftrightarrow]&\textup{(d3)}\ar[r,Rightarrow]&\textup{(c1)}\ar[r,Leftrightarrow]&\textup{(c2)}\ar[r,Leftrightarrow]\ar[d,Rightarrow]&\textup{(c3)}\\
\textup{(a1)}\ar[r,Leftrightarrow]&\textup{(a2)}\ar[r,Leftrightarrow]&\textup{(a3)}\ar[r,Leftrightarrow]&\textup{(a4)}\ar[r,Rightarrow]&\textup{(b1)}\ar[r,Rightarrow]&\textup{(b2)}\ar[r,Rightarrow]&\textup{(b3)}
\end{tikzcd}\]
Moreover, we have (d1)$\Leftrightarrow$(a1)+(c1)$\Leftrightarrow$(b1)+(c1), and if $k$ has characteristic zero, all these conditions are equivalent.
\end{theorem}
\begin{proof}
We first note the following trivial implications ($W(\g)$ being irreducible):
\[\textup{(b1)}\Rightarrow\textup{(b2)}\Rightarrow\textup{(b3)},\quad \textup{(c2)}\Rightarrow\textup{(b2)},\quad \textup{(d1)}\Leftrightarrow\textup{(c3)+(b1)}.\]
Now we prove the equivalence of conditions (a1)--(a4), and that they imply (b1). The implication (a1)$\Rightarrow$(a2) is trivial; on the other hand, (a3) signifies, if $k$ is algebraically closed, that there exists a rational point of $G\times_kW(\h)$ over $k$ over which the tangent map of $\varphi$ is surjective, and we see easily that we can take this point of the form $(e,a)$, where $a\in\h$ (by transforming it by an action of $G(k)$). We then conclude that if $k$ is infinite (not necessarily algebraically closed), this condition of generic smoothness is still necessary. Now the tangent map of $\varphi$ is easily calculated: by identifying the tangent space of $W(\h)$ at $a$ with $\h$, it is the map\footnote{The map $\varphi:G\times W(\h)\to W(\g)$ is defined by $(g,a)\mapsto\Ad(g)(a)$, so it suffices to compute that tangent map of the component morphism $G\to W(\g),g\mapsto\Ad(g)(a)$. For this, one can use the identity $\Ad_{\exp(t\xi)}(a)=e^{\ad(t\xi)}(a)$.}
\[d\varphi_{(e,a)}:\g\times\h\to\g,\quad (\xi,x)\mapsto[\xi,a]+x\]
This map is surjective if and only if $\ad_{\g/\h}(a)$ is surjective, or equivalently, injective; this proves the equivalence of (a2) and (a3). Moreover, (a2) implies evidently that $\h=\n$, and (a3) implies that $\psi$ is generically smooth, bacause if  $\varphi$ is smooth at a point $u$, then $\psi$ is smooth at $q(u)$ ($q$ being faithfully flat); hence (a2), (a3) imply (b1). To prove that they imply (a1), we note that if $\psi$ is dominant, then since the set of regular points of $\g$ is open dense, $\h$ contains a regular element of $\g$, so a "generic element" $b\in\h$ is regular in $\g$ and satisfies $\ad_{\g/\h}(b)$ is injective, hence $\g^0(b)\sub\h$, and $\h$ then contains the Cartan subalgebra $\d=\g^0(b)$. Hence (a1), (a2) and (a3) are equivalent. Finally, (a1)$\Leftrightarrow$(a4) because we have remarked that if $\h$ contains a Cartan subalgebra, it has the same nilpotent rank as $\g$ (\cref{scheme group Lie subalgebra contain Cartan subalgebra iff}), hence (a1)$\Rightarrow$(a4); conversely, if (a4) is verified, then $\h$ contains a regular element of $\g$, as and it has the same nilpotent rank as $\g$, it contains a Cartan subalgebra in view of \cref{scheme group Lie subalgebra contain Cartan subalgebra iff}.\par
To prove the equivalence of (c1)--(c3), we first note the following fact:
\begin{enumerate}
    \item[($\alpha$)] If $N$ is smooth, then $\dim(X)\leq\dim(G)$, with the equality if and only if $\h=\n$. 
    \item[($\beta$)] If $\h=\n$, then $\dim(X)\geq\dim(G)$, with the equality if and only if $N$ is smooth. 
\end{enumerate}
In fact, these assertions follows from the formula
\[\dim(X)=\dim(G)-\dim(N)+\dim_k(\h)\leq\dim(G)-\dim_k(\n)+\dim_k(\h),\]
and the fact that $\dim(N)=\dim_k(\n)$ if and only if $N$ is smooth. Given this, (c1)$\Rightarrow$(c3) because (c1) implies that $\dim(X)\geq\dim(G)$, so in virtue of ($\alpha$), the equality of these dimensions and $\h=\n$, whence (3); we similarly see that (c3)$\Rightarrow$(c1) by applying ($\beta$). On the other hand, (c1) implies (c2), because it implies $\h=\n$, so $\h$ is the Lie algebra of the smooth algebraic subgroup $N$ of $G$. Conversely, (c2)$\Rightarrow$(c1) because then $H$ normalizes its Lie algebra $\h$ and hence is contained in $N$; but $H$ is smooth and has the same Lie algebra as $N$, so $N$ is smooth (its connected component then coincides with that of $H$).\par
Next we prove the equivalence of (d1)--(d3) and that they imply (a3) (this completes the implications of the diagram in \cref{scheme alg group smooth Lie subalgebra contain Cartan iff}). We have (d1)$\Leftrightarrow$(d2) because as $W(\g)$ is irreducible, (d1) is equivalent to the fact that $\psi$ is generically unramified (by generically flatness, cf. \cite{EGA4-2} 6.9.1), which is also equivalent to (d2) tanks to the equivalence of (\rmnum{2})$\Leftrightarrow$(\rmnum{3}) of \cref{scheme action normalizer and inverse image prop}. As (d1)$\Rightarrow$(c3)$\Leftrightarrow$(c2)$\Leftrightarrow$(c1) by what we have already seen, we see that (d1) implies that $N$ is smooth and $\h$ is a Lie algebra of a smooth subgroup of $G$, i.e. $q:G\times_kW(\h)\to X$ is smooth (it is a torsor under $N$), hence the composition $\varphi=\psi\circ q$ is generically smooth (that is, we have (a3)). As (a3)$\Rightarrow$(a1), it is also true that (d1)$\Rightarrow$(d3). Finally, (d3)$\Rightarrow$(d1) becasue we trivially have (d3)$\Rightarrow$(a1), so as (a1)$\Leftrightarrow$(a3)$\Rightarrow$(b1), we have (d3)$\Rightarrow$(b1), whence (d3)$\Rightarrow$(c2). Since (c2)$\Rightarrow$(c3), we see that (d3)implies (b1) and (c3), whence (d1) since \'etale $=$ smooth $+$ quasi-finite.\par
Finally, if $k$ has characteristic zero, then (b3)$\Rightarrow$(c1) because $N$ is then automatically smooth (\cref{scheme alg group smooth over char 0}), and ((c1)+(c3))$\Rightarrow$(d1) because in characteristic zero, for a morphism of integral schemes, generically \'etale $=$ dominant $+$ generically quasi-finite (again by generically flatness, cf. \cite{EGA4-2} 6.9.1). This proves that in this case, all conditions above are equivalent.
\end{proof}

\begin{corollary}\label{scheme alg group smooth twisted product dominant prop}
Under the equivalence conditions (c1)--(c3) of \cref{scheme alg group smooth Lie subalgebra contain Cartan iff}, there exists a unique smooth connected subgroup $H$ of $G$ with Lie algebra $\h$, and we have
\[N_G(H)=N_G(\h)=N,\quad H=N^0.\]
\end{corollary}
\begin{proof}
In fact, $H=N^0$ satisfies the first conditions, and if $H$ also satisfies, then (as $H'$ normalizes its Lie algebra $\h$) we have $H'\sub N$, hence $H'=N^0$ as $H'$ is smooth and connected, with the same Lie algebra as $N^0$. For the identity $N_G(H)=N_G(\h)$, we can suppose that $k$ is algebraically closed. Then from what we have just seen, the points of the two groups with values in $k$ are the same. On the other hand, the inclusions $H\sub N_G(H)\sub N$ shows that $N_G(H)$ and $H$ have the same Lie algebra, hence they are identical.
\end{proof}

\begin{corollary}\label{scheme alg group smooth subalgebra regular element iff}
Under the equivalence conditions (a1)--(a4) of \cref{scheme alg group smooth Lie subalgebra contain Cartan iff}, let $a\in\h$. Then the following conditions are equivalent, and is satisfied if $a$ is regular in $\g$:
\begin{enumerate}
    \item[(\rmnum{1})] $\varphi$ is smooth at $(e,a)$.
    \item[(\rmnum{2})] $M_a$ is smooth at $e$, and $\dim_e(M_a)=\dim_k(\h)$.
    \item[(\rmnum{3})] $\ad_{\g/\h}(a)$ is injective. 
\end{enumerate}
If we are under the equivalence conditions (d1)--(d3), let $H$ be the subgroup of $G$ considered in \cref{scheme alg group smooth twisted product dominant prop}. Then the preceding conditions are also equivalent to the following:
\begin{enumerate}
    \item[(\rmnum{4})] $\psi$ is \'etale at $(\bar{e},a)$.
    \item[(\rmnum{5})] Denoting by $M_a^0$ the identity component of $M_a$, we have $M_a^0=H$.
\end{enumerate}
\end{corollary}
\begin{proof}
Since $M_a$ is isomorphic to the fiber $\varphi^{-1}(a)$, we evidently have (\rmnum{1})$\Rightarrow$(\rmnum{2}), the point $e$ corresponding to $(e,a)$, and (\rmnum{2})$\Rightarrow$(\rmnum{1}) because (\rmnum{2}) implies that $\varphi$ is "equidimensional" at $(e,a)$ (i.e. the dimension of the fiber passing through this point is equal to that of the generic fiber), which then implies that it is flat at $(e,a)$ ($G\times_kW(\h)$ and $W(\g)$ being regular, cf. \cite{EGA4-2} 6.1.5). The equivalence of (\rmnum{1}) and (\rmnum{3}) is contained in the proof of \cref{scheme alg group smooth Lie subalgebra contain Cartan iff}. Moreover, in view of \cref{scheme group Lie algebra regular element in unique Cartan subgroup}~(b), we see that $a$ is regular in $\g$ implies (\rmnum{3}). Under the conditions (d1)--(d3), as $q:G\times_kW(\h)\to X$ is smooth ($N$ being smooth), it follows that (\rmnum{1}) is equivalent to $\psi$ being smooth at $(\bar{e},a)$, and as $\psi$ is generically \'etale, this is equivalent to (\rmnum{4}). Also, as has been remarked in \cref{scheme action normalizer and inverse image prop}, (\rmnum{4}) implies that $N$ is a clopen subset of $M_a$, whence (\rmnum{5}). Finally, it is trivial that (\rmnum{5})$\Rightarrow$(\rmnum{2}), and this completes the proof.
\end{proof}

\begin{corollary}\label{scheme alg smooth subgroup contain Cartan inclusion iff Lie algebra}
Let $G$ be a smooth algebraic group over a field $k$, $H$ be a smooth subgroup such that its Lie algebra $\h$ contains (over a suitable field extension of $k$) a Cartan subalgebra of the Lie algebra $\g$ of $G$. Let $K$ be a connected subgroup of $G$ (not necessarily smooth), with Lie algebra $\k$, such that $\k$ contains a regular element $a\in\g$ (over a suitable field extension of $k$). Then $H$ contains $K$ if and only if $\h$ contains $\k$.
\end{corollary}
\begin{proof}
In view of (\rmnum{3})$\Rightarrow$(\rmnum{5}) of \cref{scheme alg group smooth subalgebra regular element iff}, we have $H^0=M_a^0$. On the other hand, if $\k\sub\h$, then $\Ad(K)$ sends $a\in\k$ into $\k\sub\h$, whence $K\sub M_a$. As $K$ is connected, we then have $K\sub M_a^0$, so $K\sub H$.
\end{proof}

\begin{corollary}\label{scheme alg group smooth Lie subalgebra Cartan and regular element iff}
Let $\g$ be the Lie algebra of a smooth algebraic group $G$ over a field $k$, then:
\begin{enumerate}
    \item[(a)] Let $\d$ be a subalgebra of $\g$. For $\d$ to be a Cartan subalgebra, it is necessary and sufficient that $\d$ be nilpotent and self-normalizing.
    \item[(b)] Let $a\in\g$, for $a$ to be regular, it is necessary and sufficient that $\g^0(a)$ be nilpotent.
\end{enumerate}
\end{corollary}
\begin{proof}
By taking a base field extension, we can suppose that $k$ is infinite. In view of \cref{scheme group Lie subalgebra maximal nilpotent contain regular iff}, we are reduced for (a) to prove that if $\d$ is nilpotent and contains an element $a$ such that $\ad_{\g/\d}(a)$ is injective, then $\d$ is a Cartan subalgebra. Now in view of (a2)$\Rightarrow$(a1) of \cref{scheme alg group smooth Lie subalgebra contain Cartan iff}, $\d$ contains a Cartan subalgebra $\d'=\g^0(a)$, and in view of \cref{scheme group Lie algebra g^0(a) maximal nilpotent}, we conclude that $\d=\d'$ (as $\d$ is nilpotent). To prove (b), we note that $\g^0(a)$ is a Cartan subalgebra of $\g$ in view of (a) (cf. \cref{scheme group Lie algebra nilspace self-normalizing}), hence $a$ is regular,
\end{proof}

\begin{corollary}\label{scheme alg group subalgebra of smooth subgroup Cartan subgroup iff}
Let $G$ be a smooth algebraic group over a field $k$, $H$ be a smooth algebraic subgroup, $\g,\h$ be their Lie algebras, and suppose that over a suitable field extension of $k$, $\h$ contains a Cartan subalgebra of $\g$. Let $\d$ be a subalgebra of $\h$, then it is a Cartan subalgebra of $\h$ if and only if it is a Cartan subalgebra of $\g$.
\end{corollary}
\begin{proof}
In view of \cref{scheme group Lie subalgebra contain Cartan subalgebra iff}, it remains to show that if $\d$ is a Cartan subalgebra of $\h$, it is a Cartan subalgebra of $\g$, and for this we are reduced to show that $\d$ contains a regular element $a$ of $\g$, so we can suppose that $k$ is algebraically closed. But as there is an open dense subset of $\h$ formed by elements regular in $\g$, this assertion follows from the implication (a1)$\Rightarrow$(b3) of \cref{scheme alg group smooth Lie subalgebra contain Cartan iff} applied to $(\h,\d)$.
\end{proof}

\paragraph{Cartan subalgebras and subgroups of type (C)}
For simplicity, we limit ourselves in the following theorem to the case of algebraic groups over an algebraically closed field (the case of any basic pre-scheme being treated in next section):

\begin{theorem}\label{scheme alg group smooth Cartan subalgebra conjugate prop}
Let $G$ be a smooth algebraic group over an algebraically closed field. Then:
\begin{enumerate}
    \item[(a)] The Cartan subalgebras of $\g$ are conjugate.
    \item[(b)] Let $\d$ be a Cartan subalgebra of $\g$. Then its normalizer $N$ in $G$ is smooth, and $D=N^0$ is the unique smooth connected subgroup of $G$ whose Lie algebra is $\d$. We have
    \[N_G(\d)=N_G(D)=N,\quad D=N_G(D)^0.\]
    \item[(c)] With $\d$ as in (d), put $X=G\times^NW(\d)$, and consider the canonical morphism
    \[\psi:X\to W(\g).\]
    Then $\psi$ is a birational morphism.
    \item[(d)] With the notations in (c), let $U$ be the largest open subset of $W(\g)$ such that $\psi$ induces an isomorphism $\psi^{-1}(U)\stackrel{\sim}{\to} U$, then for $a\in\g$, the following conditions are equivalent:
    \begin{enumerate}
        \item[(\rmnum{1})] $a\in U(k)$.
        \item[(\rmnum{2})] $a$ is contained in a unique Cartan subalgebra of $\g$.
        \item[(\rmnum{3})] The set of Cartan subalgebras of $\g$ containing $a$ is nonempty and finite.
        \item[(\rmnum{4})] The fiber $\psi^{-1}(a)$ has an isolated point.
        \item[(\rmnum{5})] (If $a\in\d$) The morphism $\psi$ is \'etale (or equivalently, quasi-finite) at the point $(\bar{e},a)$.
        \item[(\rmnum{6})] $a$ is a regular element of $\g$.
        \item[(\rmnum{7})] We have $N=M_a$. 
    \end{enumerate}
\end{enumerate}
\end{theorem}
\begin{proof}
By applying \cref{scheme alg group smooth Lie subalgebra contain Cartan iff} to the Cartan subalgebra $\d$ of $\g$, we see that the strongest conditions (d1)--(d3) are verified: This is evident under the form (d3) in view of \cref{scheme group Cartan subalgebra fixed iff regular element}, or in the form (d1)=(c3)+(a1), because the condition (a1) is trivial and the condition (c3) follows from the fact that a regular point of $\g$ is contained in a unique Cartan subalgebra of $\g$ (\cref{scheme group Lie algebra regular element in unique Cartan subgroup}~(a)), and a fortiori in a unique Cartan subalgebra conjugate of $\d$. Then (b) follows from \cref{scheme alg group smooth twisted product dominant prop} and (c) follows from the fact that $\psi$ is generically \'etale and that a generic point (more precisely, a regular point) of $\g$ is contained in a unique Cartan subalgebra of $\g$. Under these conditions, the equivalence of conditions (\rmnum{1})--(\rmnum{5}) over $a$ is an immediate concequence of Zariski's Main Theorem for the separated birational morphism $\psi$ (cf. \cref{scheme morphism ft isolated point nbhd finite}), as $W(\g)$ is normal and $X$ is integral. The equivalence of (\rmnum{5}) and (\rmnum{6}) is a particular case of (\rmnum{3})$\Leftrightarrow$(\rmnum{4}) of \cref{scheme alg group smooth subalgebra regular element iff} (reducing the case where $a\in\d$ by transforming $a$ by a suitable element $g\in G(k)$), taking into account of \cref{scheme group Lie algebra regular element in unique Cartan subgroup}~(b). Moreover, by \cref{scheme alg group smooth subalgebra regular element iff}, (\rmnum{5}) and (\rmnum{6}) are also equivalent to $M_a^0=N^0$, and in view of \cref{scheme group Cartan subalgebra fixed iff regular element}, (\rmnum{6}) implies that $N=M_a$, which proves (d).\par
Of course, (b) and (c), and the equivalence of (\rmnum{1}), (\rmnum{4}), (\rmnum{5}), (\rmnum{6}), (\rmnum{7}) remain valid over an arbitrary field. We now prove (a) using the fact that $k$ is algebraically closed. In view of (a1)$\Rightarrow$(b3) of \cref{scheme alg group smooth Lie subalgebra contain Cartan iff}, $\psi:X\to W(\g)$ is dominant, hence there exists an open dense subset $V$ of $W(\g)$ such that any $a\in V(k)$ is the image of an element of $X(k)$, i.e. is contained in a conjugate of $\d$. Applying this result to another Cartan subalgebra $\d'$ of $\g$, we see that we can choose $V$ such that any element $V(k)$ is conjugate to an element of $\d$ and an element of $\d'$. For a regular element in $V(k)$, this then implies that there is a conjugate $\d''$ of $\d'$ which contains a regular element of $\d$, hence is equal to $\d$ in view of \cref{scheme group Lie algebra regular element in unique Cartan subgroup}~(a). This completes the proof.
\end{proof}

For a smooth algebraic group $G$ over a field $k$, we define a \textbf{subgroup of type (C)} of $G$ to be a smooth connected subgroup of $G$ whose Lie algebra is a Cartan subalgebra of $\g$. We define the \textbf{infinitesimal rank} of $G$ to be the nilpotent rank of its Lie algebra $\g$, which is equal to the dimension of any subgroup of type (C) of $G$. By \cref{scheme alg group smooth Cartan subalgebra conjugate prop}~(b), we then conclude the following corollary:

\begin{corollary}\label{scheme alg group smooth subgroup type (C) and Cartan subalgebra correspondence}
Let $G$ be a smooth algebraic group over a field $k$. The map $D\mapsto\d=\mathfrak{Lie}(D)$ establishes a bijective correspondence between subgroups of type (C) of $G$ and Cartan subalgebras $\d$ of $\g$. If $D$ is a subgroup of type (C) of $G$, then it is self-normalizing and connected, and $D=N_G(D)^0$.
\end{corollary}

Combine \cref{scheme alg group smooth Cartan subalgebra conjugate prop}~(a) and \cref{scheme alg group smooth subgroup type (C) and Cartan subalgebra correspondence}, we also get the following:

\begin{corollary}\label{scheme alg group smooth subgroup type (C) conjugate}
If $k$ is algebraically closed, the subgroups of type (C) of $G$ are conjugate.
\end{corollary}

\begin{corollary}\label{scheme alg group smooth subgroup contain type (C) iff Cartan subalgebra}
Let $G$ be a smooth algebraic group over an algebraically closed field $k$, $H$ be a smooth subgroup of $G$, $\g$, $\h$ be their Lie algebras. For $\h$ to contain a Cartan subalgebra of $\g$, it is necessary and sufficient that $H$ contains a subgroup $D$ of type (C) of $G$.
\end{corollary}
\begin{proof}
This is evidently sufficient, and is also necessary because for that we have $H\sups D$, it is necessary and sufficient that $\h\sups\d$, in view of \cref{scheme alg smooth subgroup contain Cartan inclusion iff Lie algebra}.
\end{proof}

\begin{remark}
Suppose that $k$ is algebraically closed, and let $D$ be a subgroup of type (C) of $G$. Then it is easy to see that $D$ contains a Cartan subgroup $C$ of $G$: in fact, let $V=\g/\d$, then for a generic element $a\in\d$, $\ad(a)$ acts on $V$ injectively (it suffices to take $a$ regular in $\g$), from which we conclude that for a generic element $g\in D(k)$, we have $V^{\Ad(g)}=0$ (cf. \ref{scheme group derived morphism of linear representation equality}), which allows us to apply \cref{scheme alg smooth connected subgroup contain Cartan iff}.
\end{remark}

\begin{remark}
One should be careful not to confuse the notions of Cartan subgroups with subgroups of type (C): the subgroups of type (C) of $G$ are Cartan subgroups if and only if they are nilpotent (Cartan subgroups being indeed maximal nilpotent subgroups). Note that it can happen that $\g$ is nilpotent without $G$ is (for example, $\SL_{2,k}$ for $k$ of characteristic $2$), then the subalgebras of Cartan of $\g$ are identical to $\g$, i.e. $G$ is a subgroup of type (C) of itself if $G$ is connected, but it is not a Cartan subgroup of $G$!
\end{remark}

\subsection{Construction of Cartan subgroups and maximal tori for a smooth algebraic group}
\begin{theorem}\label{scheme alg group admits maximal torus}
Let $G$ be a smooth algebraic group over a field $k$. Then $G$ admits a maximal torus $T$, hence a Cartan subgroup $C=Z_G(T)$.
\end{theorem}
In view of \cref{scheme smooth affine ft maximal tori and Cartan subgroup correspond}, it amounts to the same thing to find a maximal torus $T$ of $G$, or a Cartan subgroup $C$ of $G$. Moreover, as the maximal tori of $G$ are those of $G^0$, we can suppose that $G$ is connected. We distinguish two cases.\par
First assume that the field $k$ is finite. Let $\mathscr{T}$ be the scheme of maximal tori of $G$ (cf. \cref{scheme group affine smooth maximal tori functor prop}), which is a smooth scheme over $k$. Note that $G$ acts on $\mathscr{T}$ via inner automorphisms, and in view of the conjugation theorem (cf. \cite{SGA3-2} \Rmnum{12} 6.6 (a)), two points of $\mathscr{T}_{\bar{k}}$ over $\bar{k}$ are conjugate under $G_{\bar{k}}(\bar{k})$. Since $\mathscr{T}$ is smooth over $k$, hence $\mathscr{T}_{\bar{k}}$ smooth over $\bar{k}$, this implies that $\mathscr{T}_{\bar{k}}$ is isomorphic to $G_{\bar{k}}/\widebar{N}$, where $\widebar{N}$ is the stabilizer of an element of $\mathscr{T}_{\bar{k}}(\bar{k})$, i.e. the normalizer of a maximal torus $\widebar{T}$ of $G_{\bar{k}}$. Therefore, $\mathscr{T}$ is a homogeneous space under the action of $G$. A well-known theorem of Lang (\cite{Lang1956}) shows that any homogeneous space under a connected algebraic group over a finite field admits a rational point. In particular, $\mathscr{T}$ admits a rational point, i.e. $G$ admits a maximal torus $T$.\par
In the case where the field $k$ is infinite, we utilize the following lemma:
\begin{lemma}\label{scheme alg group smooth admits subgroup type (C)}
Let $G$ be a smooth algebraic group over a field $k$. Then $G$ admits a subgroup of type (C).
\end{lemma}
In view of \cref{scheme alg group smooth subgroup type (C) and Cartan subalgebra correspondence}, this is equivalent to saying that $\g$ contains a Cartan subalgebra $\d$. This is trivial if $k$ is infinite, because then $\g$ contains a regular element $a$, and we can put $\d=\g^0(a)$. The finite case is treated exactly in the same way as above, but requires the prior construction of the scheme $\mathscr{D}$ of the Cartan subalgebras of $\g$ and the fact that the latter is smooth over $k$, which will be seen below (\cref{scheme Lie group functor of Cartan subalgebra representable}). However, to establish \cref{scheme alg group admits maximal torus} in the case where $k$ is infinite, it is sufficient to apply \cref{scheme alg group smooth admits subgroup type (C)} for $k$ infinite.\par
We can now give a proceedure for constructing Cartan subgroups of $G$ (also valid when $k$ is finite, assuming \cref{scheme alg group smooth admits subgroup type (C)} in this case). First suppose that $G$ is affine. We proceed by induction on $n=\dim(G)$, the assertion being trivial for $n=0$, so suppose $n>0$ and the assertion is proved for dimensions $n'<n$. Let $Z$ be the reductive center of $G$ (exist by \cref{scheme alg group affine reductive center exist}), and let
\[u:G\to G'=G/Z\]
be the canonical homomorphism. In view of \cref{scheme smooth affine connected fiber reductive center exist if}~(c), the Cartan subgroups $C'$ of $G'$ corresponds to those of $G$, so by replacing $G$ with $G'$, we can suppose that the reductive center of $G$ is trivial (cf. \cref{scheme smooth affine connected fiber reductive center exist if}~(b)). By \cref{scheme alg group smooth admits subgroup type (C)}, $G$ admits a subgroup $D$ of type (C), so we are reduced to find a Cartan subgroup in $D$. If $\dim(D)=\dim(G)$, i.e. $D=G$, then the Lie algebra of $G$ is a Cartan subalgebra of itself, hence is nilpotent, and by \cref{scheme alg group reductive center trivial and nilpotent Lie alg is unipotent}, $G$ is then nilpotent, whence a Cartan subgroup of itself. If $\dim(D)<\dim(G)$, then by the induction hypothesis, there exists a Cartan subgroup in $D$, which is hence a Cartan subgroup of $G$. This proves the theorem in the case where $G$ is affine. In the general case, let $Z$ be the center of $G$; then $G/Z=G'$ is affine by \cref{scheme alg group center is closed and quotient affine} and for any Cartan subgroup $C'$ of $G'$, its inverse image $C$ in $G$ is a Cartan subgroup of $G$ (\cref{scheme smooth affine connected fiber reductive center exist if}~(e)). We are then reduced to find a Cartan subgroup in the affine group $G/Z$, which completes the proof.

\begin{corollary}\label{scheme group over Artinian admits maximal torus}
Let $G$ be a smooth group of finite type over an Artinian scheme $S$. Then $G$ admits a maximal torus $T$, hence a Cartan subgroup $C=Z_G(T)$. Any torus $S$ in $G$ is contained in a maximal torus.
\end{corollary}
\begin{proof}
We can suppose that $S$ is local with residue field $k$, then in view of \cref{scheme alg group admits maximal torus}, $G_0=G\times_S\Spec(k)$ admits a maximal torus $T_0$. By \cref{scheme group multiplicative morphism nilpotent lifting exist} and \cref{scheme group flat multiplicative iff nilpotent reduction}, $T_0$ comes from a torus $T$ of $G$, which is evidently a maximal torus. The last assertion follows from
\end{proof}

\begin{remark}
When $k$ is infinite, \cref{scheme alg group admits maximal torus} is a consequence of the much more precise result that the scheme $\mathscr{T}$ of the maximal tori of $G$ is a rational variety, proved below (\cref{*}). The method is essentially a conjunction of the proof of \cref{scheme alg group admits maximal torus} and the explaination of the structure of the scheme $\d$ of the Cartan subalgebras of $\g$. To achieve the desired result, we must first generalize cartain results of the previous section to a scheme over arbitrary base (this is the goal of the following two subsections), and refine the previous construction in proving \cref{scheme alg group admits maximal torus}, using the fact that any Cartan subgroup of $G$ is contained in a in a unique subgroup of type (C) of $G$.
\end{remark}

\subsection{Lie algebras over an abitrary scheme: regular sections and Cartan subalgebras}
\begin{lemma}\label{scheme Lie algebra over ring fp nilpotent pointwise iff}
Let $A$ be a ring, $\d$ be a Lie algebra over $A$, and for any $s\in\Spec(A)$, let $\d(s)=\d\otimes_A\kappa(s)$ be the Lie algebra over the residue field $\kappa(s)$. Suppose that the $A$-module $\d$ is of finite presentation, then the following conditions are equivalent:
\begin{enumerate}
    \item[(\rmnum{1})] For any $s\in\Spec(A)$, $\d(s)$ is nilpotent.
    \item[(\rmnum{2})] For any $x\in\d$, $\ad(x)$ is nilpotent.
    \item[(\rmnum{3})] There exists an integer $n\geq 0$ such that for any seequence $x_1,\dots,x_N$ of elements of $\d$, we have
    \[\ad(x_1)\ad(x_2)\cdots\ad(x_N)=0.\]
\end{enumerate}
\end{lemma}
\begin{proof}
If $A$ is a field, the equivalence of (\rmnum{1}) and (\rmnum{3}) follows from the definition of nilpotence, and that of (\rmnum{2}) and (\rmnum{3}) is a concequence of Engel's theorem (\cref{Lie algebra nilpotent iff ad nilpotent}). In the general case, we have trivially (\rmnum{3})$\Rightarrow$(\rmnum{2}), and (\rmnum{2})$\Rightarrow$(\rmnum{1}) thanks to the preceding result and the fact that (\rmnum{2}) is stable under base change. It remains to show that (\rmnum{1})$\Rightarrow$(\rmnum{3}). If $A$ is local Artinian with maximal ideal $\m$, let $n>0$ be an integer such that $\m^n=0$, and $N$ be an integer such that the condition of (\rmnum{3}) is verified for $\d(s)=\d\otimes_A(A/\m)$, putting $N'=nN$, we easily see that this integer satisfies (\rmnum{3})\footnote{In fact, we see that in this case any $N$ composition of $\ad(x)$, $x\in\h$ sends $\d$ to $\m\d$, so the assertion is immediate.}. If $A$ is Noetherian, then by \cref{associated prime module over Noe injection to product over Artinian}, there exists a finitely many prime ideals $\p_i\in\Ass(\d)$ and integers $n_i\geq 0$ such that the map
\[\d\to\prod_i\d\otimes_AA_{\p_i}/\p_i^{n_i}A_{\p_i}\]
is injective. Then in view of the preceding result, there exists for each $i$ an integer $N$ satisfying (\rmnum{3}) for the Lie algebra $\d\otimes_AA_i$. Putting $N$ to be the largest of $N_i$, we see that (\rmnum{3}) is satisfied for $\d$ with $N$. Finaly, the general case is reduced to the Noetherian case by a limit process in view of (\cite{EGA4-3} 8.5.2).
\end{proof}

Let $S$ be a scheme, $\d$ be a quasi-coherent Lie algebra over $S$, which is assumed to be a module of finite presentation. We say that $\d$ is \textbf{nilpotent} if for any $s\in S$, the Lie algebra $\d(s)$ over $\kappa(s)$ is nilpotent. We say that $\d$ is \textbf{strictly nilpotent} if it is locally free and for any open subset $U$ of $S$ over which it has constant rank $r$, its Killing polynomial is reduced to $P_\d(t)=t^r$, where if $\g$ is a locally free Lie algebra over $S$, we define its Killing polynomial to be a polynomial $P_\d(t)\in A(t)$, where
\[A=\Gamma(\bm{S}_{\mathscr{O}_S}(\d^\vee))\cong\Gamma(W(\d))\]
is the global section ring of the vector bundle $W(\d)=\Spec(\bm{S}_{\mathscr{O}_S}(\d^\vee))$ defined by $\d$. It is evident that these two notions are stable under base change, and local for the fpqc topology.

\begin{proposition}\label{scheme Lie algebra locally nipotent and strict prop}
If is $\d$ is strictly nilpotent, then it is nilpotent, and the converse is valid if $\d$ is locally free and $S$ is reduced.
\end{proposition}
\begin{proof}
One direction is immediate, conversely, assume that $\d$ is nilpotent and locally free, and $S$ is reduced, we prove that $\d$ is strictly nilpotent. For this, we may assume that $S=\Spec(A)$ is affine, and also Noetherian by passing to limit (\cite{EGA4-3} 8.5.2). Then as $\d$ is nilpotent at each residue field, the assertion follows from \cref{Noe ring minimal prime and Ass localization prop}~(c). 
\end{proof}

Let $S$ be a scheme, $\g$ be a Lie algebra over $S$ which is a locally free of finite type. Let $\d$ be a subalgebra of $\g$, we say that this is a \textbf{Cartan subalgebra} of $\g$ if it satisfies the following conditions:
\begin{enumerate}
    \item[(\rmnum{1})] $\d$ is locally a direct factor of $\g$ (hence also locally free and of finite type).
    \item[(\rmnum{2})] For any $s\in S$, $\d(s)$ is a Cartan subalgebra of $\g(s)$.
\end{enumerate}
A section $a$ of $\g$ over $S$ is called \textbf{quasi-regular} if for any $s\in S$, $a(s)\in\g(s)$ is a regular element of the Lie algebra $\g(s)$ over $\kappa(s)$. We say that $a$ is a \textbf{regular section} if it is quasi-regular and contained in a Cartan subalgebra of $\g$.\par
The above notions are clearly stable under base change, and local for the fpqc topology. For this, we note that the Cartan subalgebra containing a given regular section is uniquely determined. More precisely:

\begin{proposition}\label{scheme Lie algebra unique Cartan subalgebra containing quasi-regular section}
Let $S$ and $\g$ be as above, and $a$ be a quasi-regular section of $\g$. Then there exists at most one Cartan subalgebra of $\g$ containing $a$. For it to exist, i.e. for $a$ to be a regular section, it is necessary and sufficient that $a$ satisfies the following condition: $\d=\g^0(a)$ is locally a direct factor of $\g$, and $\ad(a)$ induces an automorphism of $\g/\d$. In this case, $\d$ is the unique Cartan subalgebra containing $a$.
\end{proposition}
\begin{proof}
Suppose that $a$ is contained in a Cartan subalgebra $\d$ of $\g$. Then $\ad_{\g/\d}(a)$ is bijective on each fiber, hence (as $\g/\d$ is locally a direct factor) is an automorphism of $\g/\d$. But in view of \cref{scheme Lie algebra over ring fp nilpotent pointwise iff}, $\ad_\d(a)$ is locally nilpotent, hence $\d=\g^0(a)$, which proves the uniqueness of $\d$ and the necessity of the regularity criterion. For the sufficiency, we note that the hypothesis made on $a$ implies that the formation of $\g^0(a)$ commutes with base change, and in particular with passing to fibers. This proves in particular that the fibers $\d(s)$ of $\d=\g^0(a)$ are Cartan subalgebras of $\g(s)$. As $\d$ is a subalgebra of $\g$ in view of \cref{scheme group Lie algebra nilspace self-normalizing}, it is then a Cartan subalgebra of $\g$ containing $a$.
\end{proof}

\begin{corollary}\label{scheme Cartan subalgebra fixed iff regular section}
Under the hypothesis of \cref{scheme Lie algebra unique Cartan subalgebra containing quasi-regular section}, let $\d$ be a Cartan subalgebra of $\g$, $a$ be a section of $\d$ which is regular in $\g$, and $u$ be an automorphism of $\g$. For $\d$ to be stable under $u$, it is necessary and sufficient that $u(a)$ is a section of $\d$.
\end{corollary}
\begin{proof}
In fact, by transporting of structure the image $u(a)$ is a regular section of $\g$, contained in two Cartan subalgebras $\d$ and $u(\d)$, hence they are identical.
\end{proof}

\begin{corollary}\label{scheme Cartan subalgebra locally has regular section}
Under the hypothesis of \cref{scheme Lie algebra unique Cartan subalgebra containing quasi-regular section}, let $\d$ be a Cartan subalgebra of $\g$. Then for any $s\in S$ such that $\kappa(s)$ is infinite, there exists an open neighborhood $V$ of $s$ and a regular section $a$ of $\g$ over $V$ such that $\d=(\g|_V)^0(a)$ (i.e. such that $a$ is a section of $\d|_V$).
\end{corollary}
\begin{proof}
The fact that $\kappa(s)$ is infinite assures the existence of a regular element $a_0$ of $\g(s)$ contained in $\d(s)$ (\cref{scheme group Lie subalgebra maximal nilpotent contain regular iff}), and we can extend this to a section of $\d$ to a neighborhood $V$ of $s$, and as $\ad_{\g/\d}(a)$ induces an automorphism of $\g(s)/\d(s)$, by restricting $V$, it induces an automorphism of $\g/\d$ over $V$, which implies that $a$ is a quasi-regular section of $\g|_V$. As $\d|_V=(\g|_V)^0(a)$, we see that $a$ satisfies the condition of \cref{scheme Lie algebra unique Cartan subalgebra containing quasi-regular section}, so it is regular.
\end{proof}

Again let $\g$ be a Lie algebra over $S$ which is a locally free of finite type, then by examing its Killing polynomial, we easily see that the function ($\rho_n$ denotes the nilpotent rank)
\[s\mapsto\rho_n(\g(s))\]
over $S$ is upper semi-continuous (in fact, if a section $c_i$ is nonzero at $s$, then it is nonzero in a neighborhood of $s$). We are mainly interested in the case where this function is in fact continuous, i.e. locally constant. Here are some variations of this property:

\begin{proposition}\label{scheme Lie algebra nilpotent rank locally constant iff}
Let $S$ and $\g$ be as in \cref{scheme Lie algebra unique Cartan subalgebra containing quasi-regular section}, and consider the following conditions:
\begin{enumerate}[leftmargin=40pt]
    \item[(C0)] The nilpotent rank $\g(s)$ ($s\in S$) is a locally constant function on $S$.
    \item[(C1)] There exists locally for the fpqc topology a Cartan subalgebra of $\g$.
    \item[(C1')] There exists locally for the \'etale topology a Cartan subalgebra of $\g$.
    \item[(C2)] Condition (C0) is staisfied and for any $S'\to S$, any quasi-regular section of $\g_{S'}=\g\otimes_SS'$ is regular.
    \item[(C3)] Any $s\in S$ has an open neighborhood $V$ over which the Killing polynoial of $\g$ is of the form
    \[P_{\g|_V}(t)=t^r(t^{n-r}+c_1t^{n-r-1}\cdots+c_{n-r})\]
    where for any $s\in V$, $c_{n-r}(s)\in\bm{S}(\g(s)^\vee)$ is nonzero. 
\end{enumerate}
Then we have the implications (C3)$\Rightarrow$(C2)$\Rightarrow$(C1')$\Rightarrow$(C1)$\Rightarrow$(C0), and they are equivalent if $S$ is reduced.
\end{proposition}

We note that the conditions considered above are manifestly stable under base change, and local for the fpqc topology. The implications (C1')$\Rightarrow$(C1)$\Rightarrow$(C0) are trivial, and (C0)$\Rightarrow$(C3) is immediate if $S$ is reduced, since in this case $c_{n-r}\neq 0$ if and only if it is nonzero at each point. We also note that:

\begin{corollary}\label{scheme Lie algebra section regular locus is open}
Suppose that condition (C0) is satisfied. Let $U$ be the set of elements of $W(\g)$ which are regular in their fibers, then $U$ is open; in particular, for any section $a$ of $\g$ over $S$, the set $V$ of $s\in S$ such that $a(s)\in\g(s)$ is regular is open.
\end{corollary}
\begin{proof}
In fact, the first assertion follows from the second one (applied to $\g_{S'}$ for any base change $S'\to S$). For the second one, as we can suppose that $S$ is reduced, hence (C3) is satisfied, it suffices to consider the Killing polynomial of $a$ on $\g$.
\end{proof}

The implication (C2)$\Rightarrow$(C1') follows easily from the following more general result:

\begin{corollary}\label{scheme Lie algebra condition (C2) Cartan subalgebra regular stalk extension prop}
Suppose that condition (C2) is verified, then:
\begin{enumerate}
    \item[(a)] For any $s\in S$ and any Cartan subalgebra $\d_0$ of $\g(s)$ such that $\d_0$ contains a regular element of $\g(s)$ (automatically satisfied if $\kappa(s)$ is infinite), there exists an open neighborhood $V$ of $s$ and a Cartan subalgebra $\d$ of $\g|_V$ whose fiber at $s$ is equal to $\d_0$. If $S_1$ is a subscheme of $S$ containing $s$ and if we have extended $\d_0$ to a Cartan subalgebra $\d_1$ of $\g\otimes_SS_1$, then we can find an open neighborhood $V$ of $s$ in $S$ and a Cartan subalgebra $\d$ of $\g|_V$ such that $\d\otimes_V(S_1\cap V)$ is equal to $\d_1|_{S_1\cap V}$.
    \item[(b)] For any $s\in S$ such that $\g(s)$ contains a regular element (automatically satisfied if $k$ is infinite), there exists an open neighborhood $V$ of $s$ and a Cartan subalgebra $\d$ of $\g|_V$.
\end{enumerate}
\end{corollary}
\begin{proof}
The assertion (b) follows from (a) by setting $\d_0=\g(s)^0(a_0)$, $a_0$ being a regular element of $\g(s)$. To prove (a), say the second assertion, we consider a regular element $a_0$ of $\g(s)$ contained in $\d_0$, we extend it over a neighborhood of $s$ in $S_1$ to a section of $\d_1$, and we extend the latter into a section of $\g$ in a neighborhood of $s$. In view of \cref{scheme Lie algebra section regular locus is open}, this section is quasi-regular in an open neighborhood $V$ of $s$, hence regular under (C2). Hence $\d=(\g|_V)^0(a)$ satisfies the desired conditions of (a) in view of the uniqueness in \cref{scheme Lie algebra unique Cartan subalgebra containing quasi-regular section}.
\end{proof}

It then remains to prove the implication (C3)$\Rightarrow$(C2). We easily note the following equivalence form of (C3):
\begin{enumerate}[leftmargin=40pt]
    \item[(C3')] The nilpotent rank of $\g(s)$ ($s\in S$) is locally constant, and over any open subset $V$ of $S$ where this value is equal to $r$, the Killing polynomial of $\g|_V$ is divisible by $t^r$.
\end{enumerate}
It is then necessary to prove that this condition implies that any quasi-regular section $a$ of $\g$ is regular. In view of the criterion of \cref{scheme Lie algebra unique Cartan subalgebra containing quasi-regular section}, this is contained in the implication (\rmnum{4})$\Rightarrow$(\rmnum{3}) of the following lemma (applied to the endomorphism $\ad(a)$ of $\g$):

\begin{lemma}\label{ring finite projective module endomorphism nilpotent space direct factor}
Let $A$ be a ring, $M$ be a finitely generated projective module, $u$ be an endomorphism of $M$. Then the following conditions are equivalent:
\begin{enumerate}
    \item[(\rmnum{1})] $M$ is a direct sum of two submodules $M'$, $M''$ stable under $u$, such that $u|_{M'}$ is nilpotent and $u|_{M''}$ is an automorphism of $M''$.
    \item[(\rmnum{2})] There exists an integer $n>0$ such that $\im u^n+\ker u^n=M$.
    \item[(\rmnum{3})] The nilspace $N=\bigcup_{n>0}\ker u^n$ is a direct factor in $M$, and we have $M=N+u(M)$.
\end{enumerate}
and they are implied by the following condition (and are equivalent if $A$ is reduced):
\begin{enumerate}
    \item[(\rmnum{4})] Locally over $\Spec(A)$ (for the Zariski topology), the characteristic polynomial $P_u(t)$ of $u$ can be written into the form
    \[P_u(t)=t^r(t^{n-r}+c_1t^{n-r-1}+\cdots+c_{n-r})\]
    where $c_{n-r}$ is invertible.
\end{enumerate}
\end{lemma}
\begin{proof}
The equivalence of (\rmnum{1}), (\rmnum{2}) and (\rmnum{3}) is immediate. The fact that (\rmnum{1}) implies (\rmnum{4}) if $A$ is reduced comes from the fact that in this case, a nilpotent endomorphism of a projective module of rank $r$ has characteristic polynomial $t^r$, while in the general case, the characteristic polynomial of an automorphism of a projected module of finite type has as constant term the determinant of $u$ up to the sign (locally on $\Spec(A)$), hence an invertible element of $A$. Finally, to prove that (\rmnum{4})$\Rightarrow$(\rmnum{1}), we note that $M$ is a module over the ring of polynomials $A[t]$, with $t$ acting by $u$, and the well-known identity (Hamilton-Calley theorem)
\[P(u)=0\]
shows that $M$ is annihilated by $PA[t]$, hence can be considered as a module over $A[t]/PA[t]$. Now as $P=t^rQ$, where the constant term of $A$ is invertible, we see easily that 
\[PA[t]=t^rA[t]\cap QA[t],\]
so $A[t]/PA[t]$ decomposes into a product of rings $A[t]/t^rA[t]$ and $A[t]/QA[t]$, whence a corresponding decomposition of $M$ into $A[t]$-modules, i.e. into sum of two sub-$A$-modules $M'$ and $M''$ stable under $u$, which is the decomposition considered in (\rmnum{1}).
\end{proof}

\begin{corollary}\label{scheme Lie algebra Cartan subalgebra strict local nilpotent if (C3)}
The following conditions are equivalent to (C3):
\begin{enumerate}
    \item[(\rmnum{1})] For any $S'$ over $S$, any quasi-regular section of $\g_{S'}$ is regular and any Cartan subalgebra of $\g_{S'}$ is strictly nilpotent.
    \item[(\rmnum{2})] For any $S'$ over $S$, any quasi-regular section of $\g_{S'}$ is contains in a strictly nilpotent Cartan subalgebra of $\g_{S'}$.
\end{enumerate}
In particular, if (C3) is verified, the Cartan subalgebras of $\g$ are strictly nilpotent.
\end{corollary}
\begin{proof}
It is clear that (\rmnum{1})$\Rightarrow$(\rmnum{2}). If (C3) is satisfied, then it is satisfied for $\g_{S'}$ for any $S'\to S$, so it suffices to show (\rmnum{1}) for $\g$. To this end, we note that for any $s\in S$, after taking a base change, there exists an open neighborhood $V$ of $s$ such that the Killing polynomial of $\g_V$ is of the form
\[P_{\g|_V}(t)=t^r(t^{n-r}+c_1t^{n-r-1}+\cdots+c_{n-r}).\]
If $\d$ is a Cartan subalgebra of $\g$, then by \cref{scheme Lie algebra over ring fp nilpotent pointwise iff}, we see that for any section $x$ of $\d$ over $V$, the endomorphism $\ad(x)$ is nilpotent over $\d$, so the Killing polynomial of $\d$ is reduced to $t^r$.
\end{proof}

\begin{remark}
If $\g$ is the Lie algebra of a smooth group scheme of finite presentation over $S$, then we will see that conditions (C0), (C1), (C1') and (C2) are equivalent for $\g$ (\cite{SGA3-2} \Rmnum{14} 5.2(a)). On the other hand, even if $S$ is local Artinian, it is not true in general that (C2) implies (C3). For example, we can consider an affine smooth group $G$ over the spectrum $S=\Spec(A)$ of a discrete valuation ring, such that the Lie algebra of the generic fiber is not nilpotent, and that the special fiber is nilpotent. Then for $n$ sufficiently large, the Lie algebra $G_n=G\times_SS_n$ (where if $\m$ is the maximal ideal of $A$, $S_n=\Spec(A/\m^{n+1})$) is not strictly nilpotent, but it is nilpotent.
\end{remark}

Return to the conditions of \cref{scheme Lie algebra unique Cartan subalgebra containing quasi-regular section} and consider the functor $\mathscr{D}:\Sch_{/S}^{\op}\to\Set$ defined by
\[\mathscr{D}(S')=\{\text{set of Cartan subalgebras of $\g_{S'}$}\}.\]
We also introduce the functor $X:\Sch_{/S}^{\op}\to\Set$ defined as follows:
\[X(S)=\{\text{set of couples $(\d,a)$, where $\d$ is a Cartan subalgebra of $\g_{S'}$ and $a$ is a section of $\d$}\},\]
for which we have two projections $p:X\to\mathscr{D},(\d,a)\mapsto\d$ and $\psi:X\to W(\g),(\d,a)\mapsto a$. If (C0) condition is verified, we also consider the open subset of regular points of $W(\g)$ (cf. \cref{scheme Lie algebra section regular locus is open}); then condition (C2) is expressed by the fact that the morphism
\[\psi^{-1}(U)\to U\]
induced by $\psi$ is an isomorphism (a priori, this is a monomorphisn tanks to \cref{scheme Lie algebra unique Cartan subalgebra containing quasi-regular section}). We note that the morphism $p:X\to\mathscr{D}$ is representable by a projection of vector bundle, i.e. for any $S$-morphism $S'\to\d$, corresponding to a Cartan subalgebra $\d$ of $\g_{S'}$, the fiber product $X\times_{\mathscr{D}}S'$ is representable by a vector bundle over $S'$, namely $W(\d)$. Hence if $\mathscr{D}$ is representable, so is $X$, and in fact $X$ is representable by $W(\tilde{\d})$, where $\tilde{\d}$ is the univeral Cartan subalgebra of $\g_{\mathscr{D}}$.

\begin{theorem}\label{scheme Lie group functor of Cartan subalgebra representable}
Let $S$ be a scheme, $\g$ be a Lie algebra over $S$ which is a locally free $\mathscr{O}_S$-module of finite type, and suppose that condition (C0) of \cref{scheme Lie algebra nilpotent rank locally constant iff} is satisfied.
\begin{enumerate}
    \item[(a)] The functor $\mathscr{D}$ of Cartan subalgebras of $\g$ is representable by a quasi-projective scheme over $S$ of finite presentation. Therefore, the functor $X$ defined as above is also representable.
    \item[(b)] If condition (C2) of \cref{scheme Lie algebra nilpotent rank locally constant iff} is satisfied, $\mathscr{D}$ and $X$ are smooth over $S$, and the morphism $\psi^{-1}(U)\to U$ induced by $\psi$ is an isomorphism.
    \item[(c)] Suppose that the condition (C2) is satisfied, and let $s\in S$, $\d_0$ be a Cartan subalgebra of $\g(s)$, corresponding to a point $d\in\mathscr{D}(\kappa(s))$. Suppose that $\d_0$ contains a regular point of $\g(s)$ (automatically satisfied if $\kappa(s)$ is infinite), let $r$ be the nilpotent rank of $\g(s)$ and $n$ be its dimension over $\kappa(s)$. Then there exists an open neighborhood $V$ of $d$ in $\mathscr{D}$ which is $S$-isomorphic to an open subset $V'$ of $S[t_1,\dots,t_{n-r}]$.
\end{enumerate}
\end{theorem}
\begin{proof}
We can suppose that $\g$ is of constant rank $n$, and of constant nilpotent rank $r$. We note that the assertion made over $X$ in (a) and (b) easily follows from the assertions made over $\mathscr{D}$ and the fact that $X$ is a vector bundle over $\mathscr{D}$ defined by a locally free module.\par
The functor $\mathscr{D}$ is a subfunctor of the grassmanian functor $\sGr_{n-r}(\g)$ whose value at $S'$ is the set of locally free quotient modules of rank $n-r$ of $\g_{S'}$, and it is well known that the latter is representable by a projective smooth scheme over $S$ (cf. \cite{EGA1_new}). By considering a morphism $S'\to\sGr_{n-r}(\g)$, where $S'$ is a scheme over $S$, and taking the fiber product $\mathscr{D}\times_{\sGr_{n-r}(\g)}S'$, we are reduced to the following problem: given a locally free quotient module of rank $n-r$ of $\g$, or equivalently, a submodule $\d$ of rank $r$ which is locally a direct factor, represent the following functor:
\[F(S')=\begin{cases*}
\emp&\text{if $\d_{S'}$ is not a Cartan subalgebra of $\g_{S'}$},\\
\{*\}&\text{otherwise}.
\end{cases*}\]
To this end, we start by expressing the condition that $\d_{S'}$ is a Lie subalgebra of $\g_{S'}$: this is justified by the fact that $S'\to S$ factors through certain closed subscheme $S_1$ of $S$, of finite presentation over $S$ (whose local equations over $S$ can be written immediately using a basis of $\g$ adapted to the submodule $\d$). We can therefore suppose that we already have $S=S_1$. We must then express the condition that $\d_{S'}$ contains fpqc locally a quasi-regular section of $\g_{S'}$, and for this we consider $V=W(\d)\cap U$, where $U$ is the open set of regular points of $W(\g)$ (\cref{scheme Lie algebra section regular locus is open}); then the structural morphism $V\to S$ is smooth and quasi-compact, so its image $S_2$ is an open subset of $S$ and the immersion $S_2\to S$ is quasi-compact, i.e. of finite presentation. The considered condition over $S'$ is then expressed by the condition that $S'\to S$ factors through $S_2$, hence we are reduced to the case where $S_2=S$, and using the theory of descent, to the case where $\d$ admits a quasi-regular section $a$ of $\g$. Finally, it is necessary to express the condition that the section $a_{S'}$ of $\g_{S'}$ induced from $a$ satiaties $\ad_{\g_{S'}/\d_{S'}}(a_{S'})$ being injective, which amounts to say that $S'\to S$ factors through an open subscheme of finite presentation over $S$, say $S_D$, where $D$ is the determinant of $\ad_{\g/\d}(a)$. But then we immediately see that $\d|_{S_D}$ is a Cartan subalgebra of $\g|_{S_D}$, so $S_D$ represents the functor $F$, which completes the proof of (a) (in this case, the scheme $\sGr_{n-r}(\g)\times_SS_D$ then represents $\mathscr{D}$). Now (b) is immediate thanks to \cref{scheme Lie algebra condition (C2) Cartan subalgebra regular stalk extension prop}~(a).\par
We now prove (c), so let $a_0$ be a regular point of $\g(s)$, contained in $\d_0$, extends to a section $a$ of $\g$ in a neighborhood $V$ of $s$. We can evidently suppose that $V=S$. Let $M_0$ be a supplementary of $\d_0$ in $\g(s)$, then in a neighborhood $V$ of $S$, there exists a direct factor $M$ of $\g$ such that $M(s)=M_0$, and we can also suppose that $V=S$. Now let $mathscr{V}$ be the subfunctor of $\mathscr{D}$ such that $mathscr{V}(S')$ is the set of Cartan subalgebras $\d'$ of $\g_{S'}$ which satisfies the following conditions:
\begin{enumerate}
    \item[($\alpha$)] $\d'$ is supplementary to $M_{S'}$.
    \item[($\beta$)] The unique section of $(a_{S'}+M_{S'})\cap\d$ is a regular section of $\g_{S'}$.
\end{enumerate}
Condition ($\alpha$) corresponds to an open subset $V_1$ of $\d$ (induced by the open subset of $\sGr_{n-r}(\g)$ defined by the same condition), and the conjuction of ($\alpha$) and ($\beta$) corresponds to an open subset of $V_1$ in view of (\cref{scheme Lie algebra section regular locus is open}) and (C2). Hence $\mathscr{V}$ is represented by an open subscheme $V$ of $\mathscr{D}$, evidently contains $a$. On the other hand, let $\mathscr{V}'$ be the subfunctor of $W(M)$ defined by 
\[\mathscr{V}'(S')=\left\{\parbox{4in}{%
set of sections $u'$ of $M_{S'}$ such that:
\begin{enumerate}
    \item[(\rmnum{1})] $a_{S'}+u'$ is a regular section of $\g_{S'}$, and
    \item[(\rmnum{2})] the unique Cartan subalgebra $\d'$ of $\g_{S'}$ containing $a_{S'}+u'$ is asupplementary of $M_{S'}$
\end{enumerate}
%
}\right\}.\]
Then condition (\rmnum{1}) corresponds to an open subscheme $V_1'$ of $W(M)$, namely over the inverse image of the open subset $U$ of regular points of $W(\g)$ under the translation $m\mapsto a+m$. The conjugaction of (\rmnum{1}) and (\rmnum{2}) then corresponds to an open subset $V'$ of $V_1'$, namely the inverse image of $V$ under the evident morphism $V_1'\to\mathscr{D}$ (associating to $u'$ the unique Cartan subalgebra of $\g_{S'}$ containing $a_{S'}+u'$). The restriction of the latter morphism to $V'$ is then a morphism $V'\to V$, which is evidently an isomorphism. This proves the assertion (c) and completes the proof.
\end{proof}

\begin{corollary}\label{scheme Lie group over field functor of Cartan subalgebra is rational if regular element}
Let $\g$ be a finite dimensional Lie algebra over a field $k$. Then the scheme $\mathscr{D}$ of Cartan subalgebras of $\g$ is quasi-projective, smooth and irreducible. If $\g$ contains a regular element (for eaxample if $k$ is infinite), then $\mathscr{D}$ is a rational variety, i.e. its function field is a purly transcendental extension of $k$.
\end{corollary}
\begin{proof}
The fact that $\mathscr{D}$ is irreducible comes from the fact that we have a surjective morphism $\psi^{-1}(U)\to\mathscr{D}$, and $\psi^{-1}(U)$ is irreducibel, being isomorphic to an open subset $U$ of $W(\g)$. The last assertion follows immediately from \cref{scheme Lie group functor of Cartan subalgebra representable}~(c).
\end{proof}

\subsection{Subgroups of type (C) of group schemes over arbitrary base}
\begin{theorem}\label{scheme group smooth Lie algebra regular element transporter representable}
Let $S$ be a scheme, $G$ be a smooth $S$-group, $\g$ be its Lie algebra (which is locally free of finite type over $S$), $\h$ be a subalgebra of $\g$ which is locally a direct factor of $\g$, and such that for any $s\in S$, the geometric fiber $\h_{\bar{s}}$ contains a Cartan subgroup of $\g_{\bar{s}}$. Let $a$ be a quasi-regular section of $\g$, then the subfunctor $M_a\cong\Trans_G(a,\h)$\footnote{The isomorphism $M_a\stackrel{\sim}{\to}\Trans_G(a,\h)$ is realized via the inversion morphism of $G$, cf. the definition of $M_a$ in \ref{scheme alg group smooth Lie algebra regular element paragraph}.} is representable by a closed subscheme of $G$ smooth over $S$ with surjective structural morphism.
\end{theorem}
\begin{proof}
Consider the canonical morphism
\[\varphi:G\times_SW(\h)\to W(\g),\quad (g,x)\mapsto\Ad(g)\cdot x,\]
then $M_a$ is $S$-isomorphic to $\varphi^{-1}(a)$, the inverse image of $a$ (considered as a section of $W(\g)$ over $S$) under $\varphi$. It then suffices for the smoothness of $M_a$ to show that $\varphi$ is smooth at the points of $G\times_SW(\h)$ lying over $a$; more generally, $\varphi$ is smooth at any point lying over a point of $W(\g)$ which is regular in its fiber $W(\g(s))$ over $S$. To see this, as the source and target of $\varphi$ are smooth, hence flat and locally of finite presentation over $S$, we are reduced to the verification fiber by fiber, hence to the case where $S$ is the spectrum of an algebraically closed field $k$, $\h$ is a subalgebra of $\g$, containing a Cartan subalgebra of $\g$, and $a$ is a regular element of $\g$. We can evidently suppose (as $\varphi$ is a $G$-morphism) that the considered point of $G\times W(\h)$ is of the form $(e,a)$, and that $G$ is connected, hence of finite type over $k$. But then the assertion if none other than \cref{scheme alg group smooth subalgebra regular element iff}. Moreover, the fact that $M_a$ is a closed subscheme of $G$ (of finite presentation over $S$) is trivial, since it is the inverse image of $W(\h)$ under the morphism $g\mapsto\Ad(g)\cdot a$ from $G$ to $W(\g)$. The surjectivity of the structural morphism $M_a\to S$ reduces equally to the case of an algebraically closed base field, but then $\h$ contains a Cartan subalgebra by hypothesis, which is hence conjugate to the Cartan subalgebra $\d=\g^0(a)$ by the conjugation theorem (\cref{scheme alg group smooth Cartan subalgebra conjugate prop}), so $\h$ contains a conjugate of $a$. This completes the proof.
\end{proof}

\begin{corollary}\label{scheme group Lie algbera transporter of subalgebra representable if}
Let $G,\g$ be as in \cref{scheme group smooth Lie algebra regular element transporter representable}, with $G$ of finite type over $S$. Let $\k$ and $\h$ be subalgebras of $\g$, which are locally direct factors of $\g$. Suppose that one of the following conditions is satisfied:
\begin{enumerate}
    \item[(a)] For any $s\in S$, the geometric fiber $\k_{\bar{s}}$ is nilpotent and contains a regular element of $\g_{\bar{s}}$, and the geometric fiber $\h_{\bar{s}}$ contains a Cartan subalgebra of $\g_{\bar{s}}$.
    \item[(b)] $\k$ is a Cartan subalgebra of $\g$. 
\end{enumerate}
Then $\Trans_G(\k,\h)$ is a closed subscheme of $G$ and smooth over $S$. Moreover, in case (a), its structural morphism is surjective.
\end{corollary}
\begin{proof}
The functor $\Trans_G(\k,\h)$ is representable by a closed subscheme of $G$ in view of (\cite{SGA3-1} $\Rmnum{6}_B$ 6.5.2) and \cref{scheme subfunctor Weil restriction example}~(a), and is of finite presentation over $S$. To prove the smoothness in case (a), we first suppose that there exists a section $a$ of $\k$ which is quasi-regular in $\g$. Then it suffices to apply \cref{scheme group smooth Lie algebra regular element transporter representable} and the following lemma:
\begin{lemma}\label{scheme group smooth Cartan subalgebra transporter equal to regular element}
Under the conditions of \cref{scheme group Lie algbera transporter of subalgebra representable if}~(a), if $a$ is a section of $\k$ which is quasi-regular in $\g$, then we have
\[\Trans_G(\k,\h)=\Trans_G(a,\h).\]
\end{lemma}
In fact, in view of the definition and the hypothesis made on $\h$, it suffices to show that if $a$ is mreover a section of $\h$, then we have $\k\sub\h$. Now since $\k$ is nilpotent, it follows from \cref{scheme Lie algebra over ring fp nilpotent pointwise iff} that $\k\sub\g^0(a)$. On the other hand, $\g^0(a)\sub\h$ because $\ad_{\g/\h}(a)$ is injective (this is ture on each fiber in view of \cref{scheme group Lie subalgebra contain Cartan subalgebra iff}~(b)), whence the conclusion.\par
In the general case, by passing to limit, we are reduced to the case where $S$ is affine Noetherian, hence to the case where $S$ is local Artinian (the smoothness being a infinitesimal property), and finally by flat descent to the case where the residue fields are infinite. The fiber $\k_0$ then admits a regular element in $\g_0$, and we can extend this element to a section of $\k$, which then berings us back to the previous case (by conjugating $\h$). Finally, to see the the structural morphism is smooth in case (a), we can assume that $S$ is the spectrum of an algebraically closed field, hence $\k$ contains a regular section of $\g$, and we can then apply \cref{scheme group smooth Lie algebra regular element transporter representable}.\par
In order to prove the smoothness under condition (b), we are reduced by definition to proving that if $S$ is affine, $S_0$ is a subscheme defined by a nilpotent quasi-coherent ideal $\mathscr{I}$, $g_0\in G(S_0)$ transforms $\k_0$ to $\h_0$, then $g_0$ lifts to an element $g\in G(S)$ which transforms $\k$ to $\h$. But the hypothesis made on $g_0$ implies that we are in fact under the condition (a), whence the corollary.
\end{proof}

Applying (\cite{EGA4-4} 17.16.3) to the smooth scheme $\Trans_G(a,\h)$ and $\Trans_G(\k,\h)$, we conclude the following corollary, which shows that Cartan subalgebras are "\'etale locally" conjugate to each other.

\begin{corollary}\label{scheme group smooth Lie algebra quasi-regular etale local in Cartan}
Under the conditions of \cref{scheme group smooth Lie algebra regular element transporter representable} for $G$ and $\h$, suppose that $G$ is of finite type over $S$.
\begin{enumerate}
    \item[(a)] For any quasi-regular section $a$ of $\g$, there exists locally for the \'etale topology a conjugate of $a$ which is a section of $\h$.
    \item[(b)] For any Cartan subalgebra $\d$ of $\g$, $\d$ is locally for the \'etale topology conjugate to a subalgebra of $\h$.
\end{enumerate}
\end{corollary}

\begin{corollary}\label{scheme group smooth Cartan subalgebra etale local conjugate}
Let $G$ be a smooth $S$-group of finite type, $\g$ be its Lie algebra, $\d$ and $\d'$ be Cartan subalgebras of $\g$. Then $\Trans_G(\d,\d')$ is identified with the strict transporter of $\d$ to $\d'$, and is a closed subscheme of $G$ smooth over $S$, with surjective structural morphism. In particular, locally for the \'etale topology, $\d$ and $\d'$ are conjugate.
\end{corollary}
\begin{proof}
The fact that the transporter is identical to the strict one comes from the fact that $\d$ and $\d'$ are locally direct factors of $\g$, and have the same rank at each point. Then the corollary is a special case of \cref{scheme group smooth Lie algebra quasi-regular etale local in Cartan}.
\end{proof}

\begin{corollary}\label{scheme group smooth Cartan subalgebra normalizer closed subgroup}
Let $G$ be a smooth $S$-group of finite type and $\d$ be a Cartan subalgebra of $\d=\mathfrak{Lie}(G)$. Then $N_G(\d)$ is a closed subgroup of $G$ smooth over $S$, whose Lie algebra is equal to $\d$.
\end{corollary}
\begin{proof}
The first assertion follows from \cref{scheme group smooth Cartan subalgebra etale local conjugate}, and the second one amounts to saying that $\d$ is self-normalizing, which can be verified fiber by fiber.
\end{proof}

\begin{corollary}\label{scheme group smooth Lie algebra quasi-regular is regular}
Let $G$ be a smooth $S$-group of finite type and $\g$ be its Lie algebra. Then conditions (C2), (C1'), (C1) of \cref{scheme Lie algebra nilpotent rank locally constant iff} are equivalent. In other words, if $\g$ admits locally for the fpqc topology a Cartan subalgebra, then any quasi-regular section of $\g$ is regular.
\end{corollary}
\begin{proof}
Let $a$ be a quasi-regular section, and we need to prove that it is regular. The question is local for the fpqc topology, so we can suppose that $\g$ admits a Cartan subalgebra $\d$. In view of \cref{scheme group smooth Lie algebra quasi-regular etale local in Cartan}~(a), $a$ is locally for the \'etale topology conjugate to a section of $\d$, which allows us to assume that $a$ is a section of $\d$. But then the assertion is trivial by definition.
\end{proof}

\begin{definition}
Let $G$ be a smooth $S$-group over $S$. A subgroup $D$ of $G$ is called a \textbf{subgroup of type (C)} if it is smooth over $S$, with connected fibers, and such that $\d=\mathfrak{Lie}(D)$ is a Cartan subalgebra of $\g=\mathfrak{Lie}(G)$, i.e. such that for any $s\in S$, $D_s$ is a subgroup of type (C) of $G_s$.
\end{definition}

\begin{theorem}\label{scheme group smooth subgroup of type (C) prop}
Let $G$ be a smooth $S$-group of finite type, $\g$ be its Lie algebra.
\begin{enumerate}
    \item[(a)] The map $D\mapsto\d=\mathfrak{Lie}(D)$ is a bijective correspondence between subgroups of type (C) of $G$ and Cartan subalgebras of $\g$.
    \item[(b)] If $D$ is a subgroup of type (C) and $\d=\mathfrak{Lie}(D)$, we have $N_G(D)=N_G(\d)$, which is a closed subscheme of $G$ smooth over $S$, and
    \[D=N_G(D)^0=D_G(\d)^0.\]
    \item[(c)] Two subgroups $D$ and $D'$ of type (C) are conjugate locally for the \'etale topology.
\end{enumerate}
\end{theorem}
\begin{proof}
Let $D$ be a subgroup of type (C) of $G$, and $\d$ be its Lie algebra. Then $D\sub N_G(\d)$, and in view of our definition and \cref{scheme group smooth Cartan subalgebra normalizer closed subgroup}, this is an inclusion of smooth subgroups over $S$, inducing an isomorphism on Lie algebras. As $D$ has connected fibers, we then have $D=N_G(\d)^0$, so the map considered in (a) is injective. Now let $\d$ be a Cartan subalgebra of $\g$, then in view of \cref{scheme group smooth Cartan subalgebra normalizer closed subgroup}, $N=N_G(\d)$ is a closed subgroup of $G$ smooth over $S$, with Lie algebra $\d$. It then follows from \cref{scheme group smooth at unit section iff} that the identity component $N^0$ is representable by an open subscheme of $G$ smooth over $S$, and it is evidently a subgroup of type $C$ with Lie algebra $\d$. This proves the assertion of (a), and the second assertion in (b) also follows, together with the formula $D=N_G(\d)^0$. Finally, assertion (c) follows from (a) and \cref{scheme group smooth Lie algebra quasi-regular etale local in Cartan}.
\end{proof}

\begin{corollary}\label{scheme group smooth fp functor of subgroup of type (C) representable}
Let $G$ be a smooth $S$-group of finite type, $\g$ be its Lie algebra, and suppose that $\g$ admits locally for the fpqc topology a Cartan subalgebra (or equivalently a subgroup of type (C), in view of \cref{scheme group smooth subgroup of type (C) prop}~(a)). Consider the functor $\mathscr{D}:\Sch_{/S}^{\op}\to\Set$ defined by
\[\mathscr{D}(S')=\{\text{set of subgroups of type (C) of $G_{S'}$}\}.\]
Then $\mathscr{D}$ is representable by a quasi-projective and smooth scheme over $S$, with geometrically connected fibers. If $S$ is the spectrum of a field $k$ and $\g$ admits a regular element (automatically satisfied if $k$ is infinite), then $\mathscr{D}$ is a rational variety over $k$.
\end{corollary}
\begin{proof}
In fact, in view of \cref{scheme group smooth subgroup of type (C) prop}~(a), the functor $\mathscr{D}$ is naturally isomorphic to the functor of Cartan subalgebras of $\g$, which is considered in \cref{scheme Lie group functor of Cartan subalgebra representable}. On the other hand, in view of \cref{scheme group smooth Lie algebra quasi-regular is regular}, condition (C2) is satisfied, so the assertion follows from \cref{scheme Lie group functor of Cartan subalgebra representable} and \cref{scheme Lie group over field functor of Cartan subalgebra is rational if regular element}.
\end{proof}

\begin{corollary}\label{scheme group smooth subgroup of type (C) normalizer quotient representable}
Let $D$ be a subgroup of type (C) of $G$ and $N$ be its normalizer in $G$. Then the quotient sheaf $G/N$ is canonically isomorphic to the functor $\mathscr{D}$ of \cref{scheme group smooth fp functor of subgroup of type (C) representable}, hence representable by a quasi-projective and smooth scheme over $S$, with geometrically connected fibers.
\end{corollary}
\begin{proof}
By \cref{scheme group smooth Cartan subalgebra etale local conjugate}, the canonical morphism from $G/N$ to $\mathscr{D}$ is an isomorphism, so the assertion follows from \cref{scheme group smooth fp functor of subgroup of type (C) representable}.
\end{proof}

\begin{corollary}\label{scheme group smooth fpqc local subgroup of type (C) if}
Let $G$ be a smooth $S$-group of finite type and $H$ be a subgroup of $G$ of finite type. Suppose that for any $s\in S$, the geometric fiber $\h_{\bar{s}}$ contains a Cartan subalgebra of $\g_{\bar{s}}$, and moreover $\g$ admits locally for the fpqc topology a Cartan subalgebra. Then locally for the fpqc topology, $H$ contains a subgroup of type (C) of $G$.
\end{corollary}
\begin{proof}
In view of \cref{scheme group smooth Lie algebra quasi-regular is regular} and \cref{scheme group smooth subgroup of type (C) prop}, $G$ admits locally for the \'etale topology a subgroup of type (C), say $D$. Then the hypothesis on $\h$ implies that the structual morphism of the transporter $\Trans_G(D,H)$ is surjective (in view of the conjugation theorem \cref{scheme alg group smooth Cartan subalgebra conjugate prop}), so we conclude the corollary by (\cite{EGA4-4} 17.16.3).
\end{proof}

\begin{proposition}\label{scheme group smooth subgroup contain iff Lie algbera}
Let $G$ be a smooth $S$-group of finite type over $S$, $H$, $K$ be smooth subgroups of $G$ of finite type of $S$, with $K$ having connected fibers. Suppose that one of the following conditions is satisfied ($\h,\k,\h$ being the Lie algebras of $G$, $H$, $K$):
\begin{enumerate}
    \item[(a)] For any $s\in S$, $\k_{\bar{s}}$ contains a regular element of $\g_{\bar{s}}$, and $\h_{\bar{s}}$ contains a Cartan subalgebra of $\g_{\bar{s}}$.
    \item[(b)] For any $s\in S$, $\k_{\bar{s}}$ contains a Cartan subalgebra of $\g_{\bar{s}}$. 
\end{enumerate}
Then for $H\sups K$, it is necessary and sufficient that $\h\sups\k$.
\end{proposition}
\begin{proof}
Of course, we only need to prove that if $\h\sups\k$, then $H\sups K$. In case (b), the inclusion $\h\sups\k$ implies that we are under condition (a), so it suffices to prove (a). Proceding as in \cref{scheme group Lie algbera transporter of subalgebra representable if} by reducing to the case where $S$ is local Artinian, we are reduced to the case where there exists a section $a$ of $\k$ which is quasi-regular in $\g$. In this case, in view of the proof of \cref{scheme alg smooth subgroup contain Cartan inclusion iff Lie algebra}, we are reduced to the following statement:
\begin{lemma}
Let $G$ be a smooth $S$-group of finite type, $H$ be a smooth subgroup of $G$ of finite type over $S$, $\g,\h$ be their Lie algebras, and $a$ be a section of $\h$ which is quasi-regular in $\g$. Suppose that for any $s\in S$, the geometric fiber $\h_{\bar{s}}$ contains a subalgebra of $\g_{\bar{s}}$. Let $M_a=\Trans_G(a,\h)$, which is a closed subscheme of $G$ smooth over $S$, so that its identity component $M_a^0$ is an open subgroup of $M_a$. Then we have $H^0=M_a^0$.
\end{lemma}
In fact, as $H$ sends $\h$ into $\h$, we have $H\sub M_a$, whence $H^0\sub M_a^0$. Since the latter is the union of identity components of a smooth scheme over $S$, we can verify the equality fiber by fiber, which then reduces us to the case where $S$ is the spectrum of an algebraically closed field. But this is justified by \cref{scheme alg group smooth subalgebra regular element iff}.
\end{proof}

\begin{corollary}\label{scheme group smooth subgroup of type (C) transporter representable}
Let $G$ be a smooth $S$-group of finite type, $D$ be a subgroup of type (C) of $G$, $H$ be a subgroup of $G$ of finite type, $\d$ and $\h$ be their Lie algebras. Then we have
\[\Trans_G(D,H)=\Trans_G(\d,\h),\]
and this functor is representable by a closed subscheme of $G$ smooth over $S$.
\end{corollary}
\begin{proof}
The identity of transporters follows from \cref{scheme group smooth subgroup contain iff Lie algbera}, which allows us to apply \cref{scheme group Lie algbera transporter of subalgebra representable if}.
\end{proof}

\begin{corollary}\label{scheme group smooth transporter of subgroup representable if}
Under the conditions of \cref{scheme group smooth subgroup contain iff Lie algbera}~(a), suppose that for any $s\in S$, the geometric fiber $\k_{\bar{s}}$ is nilpotent (i.e. $\k$ is nilpotent). Then we have
\[\Trans_G(K,H)=\Trans_G(\k,\h)\]
and this functor is representable by a closed subscheme of $G$ smooth over $S$, with surjective structural morphism. Moreover, $H$ contains locally for the \'etale topology a subgroup conjugate to $K$.
\end{corollary}
\begin{proof}
The identity of the two transporters is contained in \cref{scheme group smooth subgroup contain iff Lie algbera}, and we can apply \cref{scheme group Lie algbera transporter of subalgebra representable if}~(a) to conclude that it is representable. The last assertion follows from (\cite{EGA4-4} 17.16.3).
\end{proof}

To conclude this subsection, we examine the case where $G$ is "semi-simple" over $S$. In this case, as we will show below, any subgroup of type (C) of $G$ is necessarily a Cartan subgroup.

\begin{theorem}\label{scheme group adjoint semi-simple subgroup type (C) is Cartan}
Let $G$ be a smooth $S$-group whose geomeric fibers are adjoint semi-simple algebraic groups, i.e. semi-simple with trivial reductive center. Then the subgroups of type (C) of $G$ are identical to maximal tori of $G$, hence also to Cartan subgroups of $G$.
\end{theorem}
\begin{proof}
In view of the definitions, we are reduced to the case where $S$ is the spectrum of an algebraically closed field. Since any subgroup of type (C) contains a Cartan subgroup of $G$ (\cref{scheme group smooth subgroup contained in subgroup of type (C) if}~(d)), it suffices to show that for a maximal torus $T$ of $G$, the Lie algebra $\t$ of $T$ is a Cartan subalgebra of $G$, that is (in view of the inequality $\rho_n(\g)\geq\rho_n(G)=\dim(T)=\dim(\t)=r$), there exists $x\in\t$ with $\dim(\g^0(x))=r$. As $\t$ is abelian and a fortiori nilpotent, this is equivalent to saying that there exists $a\in\t$ such that $\ad_{\g/\t}(a)$ is injective (\cref{scheme group Lie subalgebra contain Cartan subalgebra iff}~(a)). Now consider the characters $\alpha$ of $T$ in the adjoint representation over $T$ (roots relative to $T$). The theory of semi-simple algebraic groups $G$, more precisely the semi-direct product of $T$ and the subgroups $P_\alpha$ isomorphic to the additive group $\G_A$, invariant under $T$ and correspond to the roots of $G$ relative to $T$, shows that the eigenspace of $\g$ relative to the unit character is none other than $\t$, and the other eigenspaces are of dimension $1$, with associated weights being the roots of $G$ relative to $T$. As the reductive center of $G$ is the intersection of kernels of roots for $T$ (cf. \cref{scheme smooth affine diagonalizable maximal torus reductive center char}), we see that $G$ is adjoint semi-simple if and only if the roots generate $M=\Hom(T,\G_m)$. But it is well-known that every root is part of a system of roots simple, therefore from a basis of the group generated by the roots, and consequently from a basis of the dual $M$ of $T$. We then conclude:
\begin{corollary}
If $G$ is an adjoint semi-simple algebraic group over an algebraically closed field $k$ and $T$ is a maximal torus of $G$, then for any root $\alpha:T\to\G_m$ of $G$ relative to $T$, the corresponding homomorphism $\alpha_*:\t\to k$ is nonzero.
\end{corollary}
This theorem is essentially equivalent to \cref{scheme group adjoint semi-simple subgroup type (C) is Cartan}, because for $t\in\t$, $\ad(t)$ is semi-simple and its eigenvalues over $\g/\t$ is none other than $\alpha_*(t)$, hence is injective if and only if $\alpha_*(t)\neq 0$ for any root $\alpha$, and there exists such $t\in\t$ if and only if the $\alpha_*$ are nonzero.
\end{proof}

\subsection{Relations between subgroups of type (C) and Cartan subalgebras}
\paragraph{A digression on Borel subgroups}
Let $G$ be an algebraic group over an algebraically closed field. A \textbf{Borel subgroup} of $G$ is defined to be a smooth connected solvable subgroup of $G$ which is maximal for this property. If $G$ is affine, this coincides with the terminology introduces in (\cite{Chevalley1958} 6, def.1), and we note that the Borel subgroup of $G$ are those of $G^0$. If $Z$ is a connected smooth subgroup of $G$, contained in the center (or more generally, solvable and normal) of $G$, then for any Borel subgroup $B$ of $G$, the image $BZ$ of $B\times Z$ under the morphism $(b,z)\mapsto bz$ from $B\times Z$ to $G$ is a smooth connected solvable subgroup of $G$ containing $B$, whence equal to $B$, and so $B$ contains $Z$. Therefore, $B$ is the inverse image of an algebraic group $B'$ of $G'=G/Z$, and it is immediate that $B'$ is a Borel subgroup of $G'$. Putting $Z=Z(G^0)$, we see that $G'=G/Z$ is affine (\cref{scheme alg group center is closed and quotient affine}, since $Z$ has finite index in $Z(G)$), so the Borel subgroups of $G'$ are conjugate and for such a subgroup $B'$, $G'/B'$ is a projective variety (\cite{Chevalley1958} 6 th.4). We thus have the following:

\begin{proposition}\label{scheme alg group Borel subgroup prop}
Let $G$ be an algebraic group over an algebraically closed field. Then the subgroups of $G$ are conjugate. If $B$ is a Borel subgroup then $G/B$ is a projective variety. The maximal tori of $B$ (resp. Cartan subgroups of $B$, $G$ being connected) are the maximal tori of $G$ (resp. Cartan subgroups of $G$).
\end{proposition}
\begin{proof}
It remains to prove the last assertion, and we can evidently assume that $G$ is connected. For Cartan subgroups, this follows from the analogous assertion for $G'$ (\cite{Chevalley1958} 6 th.4. cor.4) and (\cite{SGA3-2} \Rmnum{12} 6.6(e)), and this implies the assertion for maximal tori in view of (\cite{SGA3-2} \Rmnum{12} 6.6(c)).
\end{proof}

\begin{corollary}\label{scheme alg group Borel subgroup union equal G}
Suppose that $G$ is connected, then any element of $G$ is contained in a Borel subgroup of $G$.
\end{corollary}
\begin{proof}
We are reduced to the same assertion for $G'$, which is well-known (\cite{Chevalley1958} 6 th.5 (d)).
\end{proof}

\begin{corollary}\label{scheme alg group Borel subgroup normalizer of Cartan subgroup}
Let $B$ be a Borel subgroup of $G$, $C$ be a Cartan subgroup of $B$ and $N=N_G(C)$ be its normalizer in $G$. Then $N\cap B=C$.
\end{corollary}
\begin{proof}
In fact, $N\cap B=N_G(C)$, so we are reduced to show that if $G$ is connected and solvable, then any Cartan subgroup $C$ is equal to its connected normalizer. With the preceding notations, $C$ is the inverse image of a Cartan subgroup $C'$ of $G'$, so we may assume that $G$ is affine. As the normalizer of a Cartan subgroup is smooth ($C$ being equal to its connected normalizer, cf. \cite{SGA3-2} \Rmnum{12} 6.6(c)), it suffices to see that $C$ and $N$ have the same points with values in $k$, which is none other than (\cite{Chevalley1958} 6 th.6 (d)).
\end{proof}

Let $G$ be a smooth group scheme of finite presentation over $S$. A \textbf{Borel subgroup} of $G$ is defined to be a smooth subgroup $B$ of $G$ of finite presentation, such that for any $s\in S$, the geometric fiber $B_{\bar{s}}$ is a Borel subgroup of $G_{\bar{s}}$. This is therefore, as we immediately verify, a notion stable under base change, and local for the fpqc topology (because if $k'$ is an algebraically closed field extension of an algebraically closed field $k$, then an algebraic subgroup $B$ of the smooth algebraic group $G$ over $k$ is a Borel subgroup of $G$ if and only if $B_{k'}$ is one of $G_{k'}$). It follows from this definition that if $G$ is a smooth algebraic group over any field $k$ and $B$ a Borel subgroup of $G$, then $G/B$ is a projective variety, any maximal torus $T$ of $B$ is a maximal torus of $G$, its normalizer in $B$ is identical to its centralizer $C$, and is a Cartan subgroup of $G$ when $G$ is connected.

\begin{proposition}\label{scheme group smooth Borel group self-normalizing closed}
Let $G$ be a smooth $S$-group of finite presentation with connected fibers and $B$ be a Borel subgroup of $G$. Then $B$ is self-normalizing, and it a closed subscheme of $G$.
\end{proposition}
\begin{proof}
In view of (\cite{SGA3-2} \Rmnum{12} 7.10), we are reduced to prove that over an algebraically closed field $k$, any element $G(k)$ which normalizes $B$ is in $B(k)$. The case for $G$ affine is a fundamental result of Chevalley (\cite{Chevalley1958} 9 th.1), and the general case follows by the reduction utilized in \cref{scheme alg group Borel subgroup prop}.
\end{proof}

\begin{remark}
We can generalize the above definition by introducing equally the notion of parabolic subgroups of $G$: a smooth subgroup $P$ of $G$ of finite presentation over $S$ is called \textbf{parabolic} if for any $s\in S$, its geometric fiber $P_{\bar{s}}$ is a parabolic subgroup of $G_{\bar{s}}$, i.e. contains a Borel subgroup of $G_{\bar{s}}$. The result of \cref{scheme group smooth Borel group self-normalizing closed} is valid (with the same reduction) for parabolic subgroups of $G$. We note the following consequence of this result (cf. \cite{SGA3-2} \Rmnum{16} 2.5): If $P$ is a parabolic subgroup of $G$, then $G/P$ is representable by a smooth projective scheme of finite presentation over $S$ (suppose that $G$ have connected fibers). Moreover $G/P$ is obviously smooth over $S$, and has connected and proper geometric fibers, from which one can easily conclude, using (\cite{EGA4-3} 15.7.10), that $D=G/P$ is in fact proper, hence projective, over $S$. Moreover, if its relative dimension is $n$, it is known that the invertible sheaf $\Omega_{D/S}^n$ is such that its inverse induces ample sheaves over the geometric fibers of $D/S$, therefore (\cite{EGA3-1} 4.7.1) the sheaf $(\Omega_{D/S}^n)^{-1}$ is ample over $D$ relatively to $S$. 
\end{remark}

It is easy to see, by reduction to the affine case and to the case of an algebraic closed base field, that if $u:G\to G'$ is an epimorphism of smooth algebraic groups, then for any Borel subgroup $B$ of $G$, $u(B)=B'$ is a Borel subgroup of $G'$. We are particular interested in the case where we can obtain a one-to-one correspondence between Borel subgroups of $G$ and $G'$:

\begin{proposition}\label{scheme group smooth epimorphism Borel subgroup correspond}
Let $G,G'$ be smooth $S$-groups of finite presentation with connected fiber, $u:G\to G'$ be a faithfully flat homomorphism of groups. Suppose that we are in one of the following cases ($N=\ker u$):
\begin{enumerate}
    \item[(a)] $N$ is central in $G$.
    \item[(b)] $S$ is the spectrum of a field $k$, and if $\bar{k}$ denotes its algebraic closure, $N_{\bar{k}}=\ker u_{\bar{k}}$ is contained in the radical of $G_{\bar{k}}$, i.e. in the largest smooth connected normal solvable subgroup of $G_{\bar{k}}$.
\end{enumerate}
Then the map $B'\mapsto u^{-1}(B')$ induces a bijection from the set of Borel subgroups of $G'$ to that of $G$.
\end{proposition}

\begin{corollary}\label{scheme group smooth epimorphism Borel subgroup correspond Lie algebra equality}
With the notations of \cref{scheme group smooth epimorphism Borel subgroup correspond}, if $B'$ and $B=u^{-1}(B')$ are Borel subgroups of $G'$ and $B$, we have
\[\b=\mathfrak{Lie}(u)^{-1}(\b')\]
where $\mathfrak{Lie}(u):\g\to\g'$ is the induced homomorphism, and $\b,\b'$ are Lie algebras of $B$ and $B'$.
\end{corollary}
\begin{proof}
This follows trivially from the definitions and the relation $u^{-1}(B')=B$.
\end{proof}

We now prove the principal result of this paragraph:

\begin{theorem}\label{scheme alg group smooth Lie algebra union of that of Borel subgroup}
Let $G$ be a smooth algebraic group over an algebraically closed field $k$, $\g$ be its Lie algebra. Then $\g$ is equal to the union of Lie algebras $\b$ of its Borel subgroups $B$ of $G$.
\end{theorem}
\begin{proof}
We can evidently suppose that $G$ is connected. Let $R$ be the radical of $G$ and $G'=G/R$. Then \cref{scheme group smooth epimorphism Borel subgroup correspond}~(b) and \cref{scheme group smooth epimorphism Borel subgroup correspond Lie algebra equality} reduce us to prove \cref{scheme alg group smooth Lie algebra union of that of Borel subgroup} for $G'$, i.e. we can suppose that $G$ is semi-simple. Let $Z$ be the reductive center of $G$, and let $G'=G/Z$. The same reasoning (using \cref{scheme group smooth epimorphism Borel subgroup correspond}~(a)) reduces the theorem for $G'$, i.e. we can suppose that $G$ is adjoint semi-simple. Let $B$ be a Borel subgroup of $G$, $T$ be a maximal torus of $B$, hence of $G$, and let $\b$, $\t$ be the Lie algebras. In view of \cref{scheme group adjoint semi-simple subgroup type (C) is Cartan}, $T$ is a subgroup of type (C) of $G$, i.e. $\t$ is a Cartan subalgebra of $\g$, hence the union of conjugate of $\t$ is dense in $\g$ ((a1)$\Rightarrow$(b3) of \cref{scheme alg group smooth Lie subalgebra contain Cartan iff}). A fortiori, the union of conjugation of $\b$ is dense in $\g$. Now let $X$ be the closed subscheme of $G/B\times W(\g)$ whose values at $k$ is the couples $(g,x)$ such that $x\in\Ad(g)\cdot\b$. Then the morphism $\psi:X\to W(\g)$ induces by the second projection is proper since $G/B$ is proper over $k$, but it is dominant by the preceding arguments, hence surjective.
\end{proof}

\begin{corollary}\label{scheme alg group smooth representation minimum of nullity attained}
Let $k$ be an infinite field, $G$ be a smooth algebraic group over $k$, $T$ be a maximal torus of $G$, $\g$ and $\t$ be the Lie algebras, $u:G\to\GL(V)$ be a linear representation (with $V$ finite dimensional over $k$), whence an induced representation $\rho:\g\to\gl(V)$. Then the minimum of the nullity (i.e. the dimension of the nilspace) of $\rho(x)$ ($x\in\g$) is attained for an element $t\in\t$.
\end{corollary}
\begin{proof}
We are easily reduced to the case where $k$ is algebraically closed and $G$ is reduced, and by quotient $G$ by $(\ker u)_\red$, we can suppose that $G$ is affine. Using \cref{scheme alg group smooth Lie algebra union of that of Borel subgroup} and the conjugation theorem of maximal tori of $G$ (\cite{SGA3-2} 6.6 (a)), we are reduced to the case where $G$ is solvable. Then $G$ is a semi-product $T\cdot V$, where $V$ is the unipotent part of $G$, which is a smooth connected unipotent group (\cite{Chevalley1958} 6, th.3), hence $\g$ is the direct sum of subalgebras $\t$ and $\mathfrak{v}=\mathfrak{Lie}(V)$. In view of Lie-Kolchin theorem (\cite{Chevalley1958} 6, th.1), $\mathfrak{v}$ admits a composition series by stable subspaces $\mathfrak{v}_i$, such that $\mathfrak{v}_i/\mathfrak{v}_{i+1}=\mathfrak{w}_i$ are of dimension $1$. Then for each $i$, we have an induced representation $u_i:G\to\GL(\mathfrak{w}_i)=\G_m$ and the induced homomorphism of Lie algebras $\rho_i:\g\to k$, so that for any $x\in\g$, the nullity of $\rho(x)$ is equal to the number of $i$ such that $\rho_i(x)=0$. Now as $V$ is unipotent, the $u_i$ are trivial over $V$ (as morphisms to $\G_m$), so $\rho_i$ are trivial on $\mathfrak{v}$, which implies that if $x=t+v$ ($t\in\t$, $v\in\mathfrak{v}$), then $\rho_i(x)=\rho_i(t)$ for any $i$, hence the nullity of $\rho(x)$ is equal to that of $\rho(t)$. The assertion of \cref{scheme alg group smooth representation minimum of nullity attained} therefore follows.
\end{proof}

\paragraph{Relation between Cartan subgroups and Cartan subalgebras}
Applying \cref{scheme alg group smooth representation minimum of nullity attained} to the adjoint representation of $G$, we then obtain the following theorem:

\begin{theorem}\label{scheme alg group Lie algebra of maximal torus contains regular element}
Let $G$ be a smooth algebraic group over an infinite field, $T$ be a maximal torus of $G$, $\g\sups\t$ be the Lie algebras. Then $\t$ contains a regular element of $\g$.
\end{theorem}
\begin{proof}
In fact, the nullity of $\ad(x)$ ($x\in\g$) is attained if and only if $x$ is regular in $\g$.
\end{proof}

\begin{corollary}\label{scheme group smooth subgroup contained in subgroup of type (C) if}
Let $G$ be a smooth $S$-group of finite presentation and $\g$ be its Lie algebra.
\begin{enumerate}
    \item[(a)] The conditions (C0)--(C2) of \cref{scheme Lie algebra nilpotent rank locally constant iff} over $\g$ are equivalent, in particular if the infinitesimal rank of the fibers of $\g$ is locally constant over $S$, then $\g$ admits locally for the \'etale topology a Cartan subalgebra, hence (by \cref{scheme group smooth subgroup of type (C) prop}) $G$ admits locally for the \'etale topology a subgroup of type (C).
    \item[(b)] Let $H$ be a smooth subgroup of $G$ of finite presentation over $S$, with connected fibers and the same reductive rank as $G$ at each $s\in S$ (for example, $H$ is a maximal torus or a Cartan subgroup of $G$), $D$ be a subgroup of type (C) of $G$, then $H\sups D$ if and only if $\h\sups\d$.
    \item[(c)] Suppose that the condition (C0) is satisfied, i.e. the infinitesimal rank of $G$ is locally constant. Let $H$ be a subgroup of $G$ of finite presentation over $S$, with nilpotent connected fibers and same reductive rank as $G$ at each point of $S$ (for example, $H$ is a maximal torus or a Cartan subgroup of $G$). Then $H$ is contained in a unique subgroup $D$ of type (C) of $G$.
    \item[(d)] Suppose that $G$ admits locally for the fpqc topology a Cartan subgroup, then the same is true for any subgroup $D$ of type (C) of $G$.
\end{enumerate}
\end{corollary}
\begin{proof}
Suppose that condition (C0) is satisfied, and we prove (C2), i.e. that any quasi-regular section $a$ of $\g$ is regular. We are as usual reduced to the case where $S$ is affine Noetherian, and as the question is infinitesimal (\cref{scheme Lie algebra condition (C2) Cartan subalgebra regular stalk extension prop}~(b)), to the case where $S$ is local Artinian (then (C0) is trivially). We can also suppose that the residue field $k$ of $S$ is infinite. Note that in view of \cref{scheme group smooth Lie algebra quasi-regular is regular}, it suffices to show that $\g$ admits a Cartan subalgebra. Let $T$ be a maximal torus of $G$ (exists by \cref{scheme alg group admits maximal torus}); in view of \cref{scheme alg group Lie algebra of maximal torus contains regular element}, there exists a quasi-regular element $t$ contained in the Lie algebra $\t$ of $T$, so it suffices to show that it is regular. Consider the linear representation of $T$ on $\g$ induced by the adjoint representation of $G$; there then exists a finite set $(u_i)_{i\in I}$ of characters of $T$ such that $\g$ decomposes into direct sum of submodules $\g_i$ stable under $T$ and $T$ acting on $\g_i$ by $u_i$. Let $\rho_i:\t\to A$ be the induced homomorphism of algebras from $u_i:T\to\G_m$, and consider the homomorphisms $u_{i,0}$ and $\rho_{i,0}$ deduced from the preceding by passing to fiber, i.e. by base change $A\to k$. Let $I'$ be the subset of $i\in I$ such that $\rho_{i,0}\neq 0$, and let $I''=I-I'$. The fact that $t$ is quasi-regular is expressed by the conditions $\rho_{i,0}(t_0)\neq 0$ for any $i\in I'$, hence $\rho_i(t)$ is invertible for $i\in I'$. The Cartan subalgebra $\g_0$ defined by $t$, i.e. the kernel of the semi-simple endomorphism $\ad(t_0)$ of $\g_0$, is hence $\sum_{i\in I''}(\g_i)_0$. Consider 
\[\d=\sum_{i\in I''}\g_i,\]
then $\ad(t)$ is nilpotent on $\d$ (since $A$ is Artinian). On the other hand, $\ad(t)$ is an automorphism on $\g/\d\cong\sum_{i\in I'}\g_i$ by our construction. In view of \cref{scheme Lie algebra unique Cartan subalgebra containing quasi-regular section}, this implies that $t$ is regular.\par
Now (b) follows from \cref{scheme group smooth subgroup contain iff Lie algbera}~(a), as $H$ verifies the hypothesis that any geometric fiber of $\h$ contains a regular element of $\g$, thanks to \cref{scheme alg group Lie algebra of maximal torus contains regular element}, and (d) is a particular case of (\cite{SGA3-2} \Rmnum{12} 7.9 (d)). To prove (c), we note that that in view of (b), it suffices to show that the Lie algebra $\h$ of $H$ is contained in a unique Cartan subalgebra $\d$ of $\g$. As condition (C2) is satisfied in this case (thanks to (a)), we are hence reduced to the following lemma:
\begin{lemma}\label{scheme Lie algebra contains unique Cartan subalgebra if geometric fiber regular}
Let $\g$ be a Lie algebra over $S$ which is locally free of finite type and satisfies condition (C2) of \cref{scheme Lie algebra nilpotent rank locally constant iff}. Let $\h$ be a subalgebra of $\g$ satisfying the following conditions: $\h$ is locally a direct factor of $\g$, is locally nilpotent, and for any $s\in S$, the geometric fiber $\h_{\bar{s}}$ contains a regular element of $\g_{\bar{s}}$. Then $\h$ contains a unique Cartan subalgebra of $\g$.
\end{lemma}
(In the case of (b), $\h$ has nilpotent fibers because $H$ has nilpotent fibers, and the condition on regular elements is satisfied by \cref{scheme alg group Lie algebra of maximal torus contains regular element}). As \cref{scheme Lie algebra contains unique Cartan subalgebra if geometric fiber regular} is local for the fpqc topology, it suffices to prove that at a point $s\in S$ such that $\kappa(s)$ is infinite, there exists an open neighborhood $U$ of $s$ such that the existence and uniqueness are true for any change of basis $S'\to S$ factorizing through $U$. Now take a regular element of $\g\otimes\kappa(s)$ contained in $\h\otimes\kappa(s)$, extend it into a section $a$ of $\h$ over an open neighborhood $U$ of $s$; thanks to (C2), we can suppose that this section is regular (\cref{scheme Lie algebra section regular locus is open}), and we can also assume that $U=S$. A Cartan subalgebra of $\g$ which contains $\h$ must contain $a$, so is identical to $\d=\g^0(a)$ (\cref{scheme Lie algebra unique Cartan subalgebra containing quasi-regular section}), whence the uniqueness. Moreover, as $h$ is locally nilpotent, we have $\h\sub\d$, which proves existence.
\end{proof}

\begin{corollary}\label{scheme alg group smooth Lie algebra contain Cartan conjugate number prop}
Let $G$ be a smooth connected algebraic group over an algebraically closed field $k$, $H$ be a connected subgroup such that $\h$ contains a Cartan subalgebra of $\g$, i.e. that $H$ contains a subgroup of type (C) of $G$. Then the number of conjugates of $H$ containing a regular element of $G(k)$ is equal to the number of conjugates of $\h$ containing a regular element of $\g$.
\end{corollary}

Let $G$ be a smooth $S$-group of finite presentation, $\g$ be its Lie algebra, and suppose that the infinitesimal rank of fibers of $G$ are constant (condition (C0)). Then thanks to \cref{scheme group smooth subgroup contained in subgroup of type (C) if}~(c), we have a homomorphism of functors $\mathscr{C}\to\mathscr{D}$, where
\begin{gather*}
\mathscr{C}(S')=\{\text{set of Cartan subgroups of $G_{S'}$}\},\\
\mathscr{D}(S')=\{\text{set of subgroups of type (C) of $G_{S'}$}\}.
\end{gather*}
In view of \cref{scheme Lie group functor of Cartan subalgebra representable} and \cref{scheme group smooth subgroup contained in subgroup of type (C) if}~(a), $\mathscr{D}$ is representable by a quasi-projective smooth scheme over $S$. Consider the "universal" subgroup $D$ of type (C) of $G_\mathscr{D}$; we can also define a functor $\mathscr{C}_D:\Sch_{/\mathscr{D}}^{\op}\to\Set$ in the same way as $\mathscr{C}$ (i.e. the functor of Cartan subgroups of $D$). We then have the following result:

\begin{proposition}\label{scheme group smooth functor of Cartan subgroup and universal type (C)}
Under the preceding conditions, consider $\mathscr{C}$ as a functor over the scheme $\mathscr{D}$, then $\mathscr{C}$ is $\mathscr{D}$-isomorphic to the functor $\mathscr{C}_D$ of Cartan subgroups of $D$.
\end{proposition}
\begin{proof}
As any Cartan subgroup of $G$ is contained in a unique subgroup of type (C) in view of \cref{scheme group smooth subgroup contained in subgroup of type (C) if}~(c), we see that the proposition follows from the following result:
\begin{corollary}\label{scheme group smooth Cartan subgroup of type (C) iff Cartan subgroup}
Let $G$ be a smooth $S$-group of finite presentation and $D$ be a subgroup of type (C) of $G$. Then there exists a bijective correspondence between Cartan subgroups of $G$ contained in $D$ and Cartan subgroups of $D$ (i.e. a subgroup $H$ of $G$ is a Cartan subgroup of $G$ if and only if it is a Cartan subgroup of $D$).
\end{corollary}
In fact, this is a special case of (\cite{SGA3-2} \Rmnum{12} 7.9 (c)), as over an algebraically closed field, a subgroup of type (C) of $G$ contains a Cartan subgroup of $G$ (\cref{scheme group smooth subgroup contained in subgroup of type (C) if}~(a)).
\end{proof}

\paragraph{Applications to the structure of algebraic groups}
\begin{theorem}\label{scheme alg group smooth scheme of Cartan subgroup is rational}
Let $G$ be a smooth algebraic group over a field $k$. Consider the scheme $\mathscr{T}$ of maximal tori of $G$, isomorphic to the scheme $\mathscr{C}$ of Cartan subgroups of $G$, which is a homogeneous space under $G^0$ and affine smooth connected (\cite{SGA3-2} \Rmnum{12} 7.1 (d)). Then $\mathscr{C}$ is a rational variety, i.e. the function field of $\mathscr{C}$ is a purely transcendental extension over $k$.
\end{theorem}
\begin{proof}
We first demonstrate the case where $k$ is infinite. We can evidently suppose that $G$ is connected, because $\mathscr{T}$ (therefore $\mathscr{C}$) does not change by replacing $G$ with $G^0$. Also, by virtue of (\cite{SGA3-2} \Rmnum{12} 7.6), $\mathscr{C}$ does not change by dividing $G$ by a central subgroup. This allows us, by dividing $G$ by its center, to suppose that $G$ is affine, and then, by dividing its reductive center (\cref{scheme alg group affine reductive center exist}), to suppose that the reductive center of $G$ is trivial (\cref{scheme smooth affine connected fiber reductive center exist if}~(c)). We now proceed by induction on $n=\dim(G)$, assuming the theorem has been proved for the dimensions $n'<n$. If $G$ is nilpotent, then $\mathscr{C}$ reduces to a rational point on $k$, and the assertion is trivial. Otherwise, the Lie algebra $\g$ of $G$ is not nilpotent (\cref{scheme alg group reductive center trivial and nilpotent Lie alg is unipotent}), so the Cartan subalgebras of $\g$ are of dimension $n'<n$, and the subgroups of type (C) of $G$ are of dimension $n'<n$. Consider then the morphism $\mathscr{C}\to\mathscr{D}$ in \cref{scheme group smooth functor of Cartan subgroup and universal type (C)}. We see by \cref{scheme group smooth fp functor of subgroup of type (C) representable} that $\mathscr{D}$ is a rational variety, i.e. the rational function field $K$ of $\mathscr{D}$ is a purely transcendental extension of $k$. Consider the fiber of $\mathscr{C}$ at the generic point $\eta$ of $\mathscr{D}$, which is in view of \cref{scheme group smooth functor of Cartan subgroup and universal type (C)} the scheme of Cartan subgroups of a cartain smooth connected algebraic group $D_\eta$ over $K=\kappa(\eta)$ (namely, $D_\eta$ is the generic subgroup of type (C) of $G$). The rational function field $L$ of $\mathscr{C}$ is hence isomorphic to the rational function field of $\mathscr{C}_{D_\eta}$, which by the induction hypothesis (as $\dim(D_\eta)<n'<n$) is a purely transcendental extension of $K$. By transitivity, $L$ is then a purely transcendental extension of $k$.\par
If $k$ is finite, we need a different proof. We can still assume that $G$ is affine and connected. Note that as $k$ is perfect, it follows immediately from Galois descent that the radical $\widebar{R}$ of $G_{\bar{k}}$ is defined over $k$, i.e. comes from a subgroup $R$ of $G$. Suppose first that $R\neq G$, i.e. $G$ is not solvable, and let
\[u:G\to G'=G/R\]
be the canonical morphism. Consider the corresponding morphism
\[v:\mathscr{C}_G\to\mathscr{C}_{G'},\quad C\mapsto u(C)\]
(whose definition is immediate in view of (\cite{SGA3-2} 7.1 (e))). Let $\eta$ be the generic point of $\mathscr{C}_{G'}$, then the fiber $v^{-1}(\eta)$ is identified with the scheme of Cartan subgroups $C$ of $G_K$ (where $K=\kappa(\eta)$) whose image in $G'_{K}$ is a certain Cartan subgroup $C'_{\eta}$ (namely, the generic Cartan subgroup of $G'$, where $C'$ is the universal Cartan subgroup of $G'$). This is therefore also the scheme of Cartan subgroups of $H=u_K^{-1}(C')$ (\cite{SGA3-2} \Rmnum{12} 7.9 (c)) and as $K$ is an infinite field over $k$, it follows from the already proved part of \cref{scheme alg group smooth scheme of Cartan subgroup is rational} that the rational function field $L$ of $\mathscr{C}_G$, equal to that of $\mathscr{C}_H$, is a purely transcendental extension of $K$. To prove that it is a purely transcendental extension of $k$, it then suffices to prove that so is $K$, i.e. we are reduced to the case where $G$ is semi-simple. We can also suppose that $G$ is adjoint (by dividing $G$ by its reductive center). But then by \cref{scheme group adjoint semi-simple subgroup type (C) is Cartan}, we have $\mathscr{C}_G\cong\mathscr{D}_G$, and in view of \cref{scheme group smooth fp functor of subgroup of type (C) representable}, it suffices to show that $\g$ admits a regular element, which is an unpublished result of Chevalley (cf. \cite{SGA3-2} \Rmnum{14} Appendice).\par
Now it only remains to prove the theorem in the case where $k$ finite and $G$ is affine connected solvable. In fact, we have the following general result:
\begin{corollary}\label{scheme alg group smooth solvable scheme of Cartan subgroup isomorphic to A^N}
Let $G$ be a smooth solvable algebraic group over a field $k$, then the variety $\mathscr{C}$ of Cartan subgroups of $G$ is isomorphic to an affine space $\A_k^N$.
\end{corollary}
To prove this corollary, we can still suppose that $G$ is connected and affine. Let $G_\infty$ be the smalless subgroup appearing in the descent central series of $G$ (i.e. $G_{i+1}=[G,G_i]$); this is then the smallest normal subgroup of $G$ such that $G/G_\infty$ is nilpotent. Let $C$ be a Cartan subgroup of $G$ (exists in view of \cref{scheme alg group admits maximal torus}), then the image of $C$ in $G/G_\infty$ is a Cartan subgroup, hence is equal to $G/G_\infty$. It follows that we have $C\cdot G_\infty=G$, so the conjugate class of $C$ under $G$ is equal to that under $G_\infty$; in other words, the morphism $G_\infty\to\mathscr{C},g\mapsto\inn(g)\cdot C$ is an epimorphism, and identifies $\mathscr{C}$ with a homogeneous space $G_\infty/N\cap G_\infty$, where $N$ is the normalizer of $C$ in $G$. Note that $U=G_\infty$ is evidently a smooth connected unipotent subgroup (because over the algebraic closure of $k$, it is contained in the unipotent part of $G$, in view of the structure theorem of smooth affine solvable groups, cf. \cite{Chevalley1958} 6, th.3). If $k$ is perfect (the only necessary case in order to prove \cref{scheme alg group smooth scheme of Cartan subgroup is rational}), it then follows easily that $U$ is $k$-unipotent, i.e. admits a composition series by subgroups $U_i$ such that $U_i/U_{i+1}$ is isomorphic to $\G_a$\footnote{In fact, Rosenlicht have proved this result is valid for a group of the form $G_\infty$ as above, without any restriction on $k$, cf. \cite{Rosenlicht1963}}. It now suffices to apply the following theorem due to Rosenlicht (for a proof using the language of modern algebraic geometry, one can consult \cite{Brion2021})\footnote{In fact, Rosenlicht proved that (cf. \cite{Rosenlicht1963} th.5) for any homogeneous space $X=U/V$ under a connected $k$-split solvable algebraic group (i.e. admits a composition series whose successive quotient are isomorphic to $\G_a$ or $\G_m$) is $k$-isomorphic to a product of a certain number of copies of $\G_a$ and $\G_m$, where the number of $\G_a$'s (resp. $\G_m$'s) is the number occurring as successive quotient groups of a composition series of $U$ minus that of $V$. In particular, if $U$ is $k$-unipotent, then $X$ must be a product of $\G_a$, hence is isomorphic to an affine space.}:
\begin{theorem}[\cite{Rosenlicht1963}]\label{scheme alg group unipotent rational homogeneous space isomorphic to A^N}
Let $U$ be a smooth connected algebraic group over a field $k$, $X=U/V$ be a homogeneous space under $U$. Suppose that $U$ is $k$-unipotent, then $X$ is isomorphic to an affine space $\Spec(k[t_1,\dots,t_N])$.
\end{theorem}
To prove this theorem, let $(U_i)_{0\leq i\leq n}$ be a composition series of normal subgroups of $U$ by smooth conencted subgroups whose successive quotients are isomorphic to $\G_a$. Then the $K_i=U_iV$ are subgroups of $U$, and $K_{i+1}$ is normal in $K_i$. We have a canonical epimorphism $U_i/U_{i+1}\to K_i/K_{i+1}$, which proves that $K_i/K_{i+1}$ is either trivial or isomorphic to $\G_a$ (\cref{scheme alg group quotient of elementary unipotent isomorphic}). Put $X_i=U/K_i$, we prove by induction on $i$ that $X_i$ is isomorphic to an affine space. The case $i=0$ is trivial, so suppose that this is true for $X_i$. If $K_i/K_{i+1}=e$, then $X_i=X_{i+1}$, and the assertion is trivial. Otherwise, $X_{i+1}$ is a torsor over $X_i$ with structure group $K_i/K_{i+1}\cong\G_a$. As $X_i$ is affine, it follows from $H^1(X_i,\G_a)=H^1(X_i,\mathscr{O}_{X_i})=0$ that $X_{i+1}$ is trivial, i.e. isomorphic to $X_i\times\G_a$. This shows that $X_{i+1}$ is isomorphic to an affine space, and completes the proof.
\end{proof}

\begin{corollary}\label{scheme alg group smooth Cartan subgroup union dense}
Let $G$ be a smooth algebraic group over an infinite field $k$. Then the set of rational points of $\mathscr{C}$ over $k$ is dense for the Zariski topology. The union of Cartan subgroups of $G$ is dense in $G$.
\end{corollary}
\begin{proof}
The first assertion is valid for any unirational variety over an infinite field. The second one follows from the first and the density theorem \cref{scheme alg smooth connected subgroup contain Cartan iff}.
\end{proof}

\begin{corollary}\label{scheme alg group smooth connected scheme of semi-simple regular is unirational}
Let $G$ be a smooth connected algebraic group over $k$. Then the variety $Z$ of semi-simple regular points of $G$ (\cref{scheme group smooth sp functor of regular section in torus representable}) is a unirational variety. In particular, if $k$ is infinite, the set of rational points of $Z$ over $k$ is dense in $Z$. 
\end{corollary}
\begin{proof}
In fact, $Z$ is an open subset of a torus over $\mathscr{C}$ (namely, the universal maximal torus of $G$), hence its rational function field $L$ is that of a torus defined over the function field $K$ of $\mathscr{C}$ (namely the generic maximal torus of $G$), which is a unirational extension of $K$ by \cref{scheme group torus function field is unirational}. As $K$ is also a purely transcendental extension of $k$, $L$ is also a unirational extension of $k$.
\end{proof}

\begin{corollary}\label{scheme alg group smooth subgroup of semi-simple regular is unirational}
Let $G$ be a smooth connected algebraic group over $k$ and $H$ be the subgroup of $G$ generated by the subscheme $Z$ of semi-simple regular points, i.e. (\cite{SGA3-2} \Rmnum{12} 8.2) the smallest normal subgroup of $G$ such that $G/H$ is of zero reductive rank. Then $H$ is a unirational variety. In particular, if $G=H$, i.e. (\cite{SGA3-2} \Rmnum{12} 8.2) if $G$ is affine and over the algebraic closure $\bar{k}$, there is no nontrivial homomorphism from $G_{\bar{k}}$ to $\G_a$, then $G$ is a unirational variety, hence if $k$ is infinite, the set of rational points of $G$ over $k$ is dense.
\end{corollary}
\begin{proof}
This follows easily from \cref{scheme alg group smooth connected scheme of semi-simple regular is unirational}, because it is immediate that if we have smooth connected $k$-schemes $Z_i$ which are unirational and morphisms $u_i:Z_i\to G$, then the subgroup of $G$ generated by the $u_i$ (\cite{SGA3-1} $\Rmnum{6}_B$ 7.1) is a unirational variaty.
\end{proof}

As a special case of \cref{scheme alg group smooth subgroup of semi-simple regular is unirational}, we obtain the following:

\begin{corollary}\label{scheme alg group smooth affine zero unipotent rank is unirational}
Let $G$ be an affine smooth connected algebraic group of zero unipotent rank, then $G$ is a unirational variety.
\end{corollary}
\begin{proof}
In fact, in this case every Cartan subgroup is a maximal torus, and there does not exist a nontrivial homomorphism from it to $\G_a$. But the union of Cartan subgroups is dense in $G$ (\cref{scheme alg group smooth Cartan subgroup union dense}), so there does not exist a nontrivial homomorphism from $G$ to $\G_a$, i.e. $G=H$.
\end{proof}

We can also precise \cref{scheme alg group smooth Cartan subgroup union dense} in the following way:

\begin{corollary}\label{scheme alg group smooth Cartan conjugate by semi-simple regular dense}
Let $G$ be a smooth connected algebraic group over an infinite field $k$, and $C$ be a Cartan subgroup of $G$. Then the union of conjugates of $C$ under semi-simple regular elements of $G(k)$ is dense in $G$.
\end{corollary}
\begin{proof}
This follows easily from \cref{scheme alg group smooth Cartan subgroup union dense} and \cref{scheme group smooth sp conjugate by regular in torus dominant}, which says that the morphism $\varphi:Z\times C\to G,(t,c)\mapsto\inn(t)c$ is dominant. 
\end{proof}

\begin{corollary}\label{scheme alg group smooth subgroup of same rank unirational if}
Let $G$ be a smooth connected algebraic group over $k$, $H$ be a smooth connected subgroup of $G$ such that $H$ has the same reductive rank and nilpotent rank as $G$ (i.e. over the algebraic closure $\bar{k}$, $H_{\bar{k}}$ contains a Cartan subgroup of $G_{\bar{k}}$). If $H$ is a unirational variety, so if $G$. If $H(k)$ is dense in $H$, $G(k)$ is dense in $G$.
\end{corollary}
\begin{proof}
The morphism $\varphi:Z\times H\to G$ defined by $\varphi(t,h)=\inn(t)(h)$ is dominant by \cref{scheme group smooth sp conjugate by regular in torus dominant}. Now in view of \cref{scheme alg group smooth connected scheme of semi-simple regular is unirational}, $Z$ is a unirational variety, and by hypothesis so is $H$, whence $Z\times H$ and the first assertion. The second one can be deduced in the same way.
\end{proof}

\begin{corollary}\label{scheme alg group smooth affine over perfect is unirational}
Let $G$ be an affine smooth connected algebraic group over a perfect field. Then $G$ is a unirational variety, hence if $k$ is infinite, $G(k)$ is dense in $G$.
\end{corollary}
\begin{proof}
In view of \cref{scheme alg group admits maximal torus}, $G$ admits a Cartan subgroup $C$, and by \cref{scheme alg group smooth subgroup of same rank unirational if}, it suffices to prove that this is a unirational variety. Now as $k$ is perfect, we immediately see by Galois descent from the algebraically closed case (\cite{Chevalley1958} 6 th.2) that we have $C=T\times C_u$, where $T$ is the maximal torus of $C$ and $C_u$ is a smooth connected unipotent group. We have seen that $T$ is a unirational variety (\cref{scheme alg group smooth connected scheme of semi-simple regular is unirational}), so it suffices to show that the same is true for $C_u$. Now $k$ being perfect, $C_u$ is itself $k$-unipotent, and we can apply \cref{scheme alg group unipotent rational homogeneous space isomorphic to A^N}.
\end{proof}

\section{Unipotent algebraic groups}
In this section, we only consider algebraic groups defined over a field $k$. The number $p\geq 0$ always denote the characteristic of $k$, $\F_p$ denotes the prime field of $p$ elements if $p>0$, $\bar{k}$ is the algebraic closure of $k$, and $q$ is a prime number distinct from $p$.

\subsection{Definition of unipotent groups}
An algebraic group $G$ defined over an algebraically closed field $k$ is called \textbf{unipotent} if $G$ admits a composition series whose successive quotients are isomorphic to subgroups of $\G_a$.

\begin{proposition}\label{scheme alg group unipotent iff extension to ac field}
Let $k$ be an algebraically closed field, $K$ be an algebrically closed field extension of $K$, and $G$ be an algebraic group defined over $K$. Then, for $G$ to be unipotent, it is necessary and sufficient that $G_K$ is unipotent.
\end{proposition}
\begin{proof}
The necessity of this condition is clear since a composition series gives a decomposition seies by base extension. The proof of sufficiency is standard: $K$ is the inductive limit of subalgebras of finite type. By (\cite{SGA3-1} $\Rmnum{6}_B$ \S 10), we can find a sub-$k$-algebra of finite type $A$ of $K$, a composition series $G_i$ of $G_S$ ($S=\Spec(A)$) and immersions $u_i:H_i=G_i/G_{i+1}\hookrightarrow\G_{a,S}$. To prove that $G$ is unipotent, it suffices to take a $k$-base change $A\to k$, which is possible because $\Hom_{k}(A,k)$ is nonempty, $A$ being a nonzero $k$-algebra of finite type over an algebraically closed field $k$.
\end{proof}

Now let $G$ be an algebraic group defined over a field $k$ (not necessarily algebraically closed). We say that $G$ is \textbf{unipotent} if there exists an algebraically closed field extension $\bar{k}$ of $k$ such that $G_{\bar{k}}$ is unipotent. In view of \cref{scheme alg group unipotent iff extension to ac field}, this definition is independent of the chosen field $\bar{k}$.\par
Let $H$ be an algebraic group defined over an extension $k'$ of $k$. We recall that $H$ is called a \textbf{form} of $G$ over $k'$ if the algebraic groups $G_{k'}$ and $H_{k'}$ are isomorphic over $\bar{k}'$ (as usual we see that this property is independent of the algebraically closed extension $\bar{k}'$ of $k'$). In this case, we also say that $H$ is a \textbf{twisted $G$-group}. We are now able to describe all subgroups of $\G_a$.

\begin{proposition}\label{scheme alg subgroup of G_a char}
Let $k$ be a field with characterisric $p\geq 0$, $H$ be a subgroup of $\G_{a}$, and $F$ be the Frobenius endomorphism on $\G_{a}$. 
\begin{enumerate}
    \item[(\rmnum{1})] If $p=0$, then $H=0$ or $\G_a$.
    \item[(\rmnum{2})] If $p>0$, then there exists an endomorphism
    \[f=a_rF^r+a_{r+1}F^r+\cdots+a_sF^s,\quad a_r,\dots,a_s\in k,a_r\neq 0,a_s\neq 0\]
    of $\G_a$ such that the sequence
    \[\begin{tikzcd}
    1\ar[r]&H\ar[r]&\G_a\ar[r,"f"]&\G_a\ar[r]&1
    \end{tikzcd}\]
    is exact. We then have $H^0\cong\bm{\alpha}_{p^r}$ and $\pi_0(H)=\ker(a_r\id+a_{r+1}F+\cdots+a_sF^{s-r})$, and the latter is a form of $(\Z/p\Z)^{s-r}$. Finally, if $k$ is perfect, then $H$ is a direct product of $\pi_0(H)$ and $H^0$.
\end{enumerate}    
\end{proposition}
\begin{proof}
We know that $H$ is defined by a Hopf ideal $\mathfrak{I}$ of $k[X]=\mathscr{O}(\G_a)$ (endowed with the coproduct $\Delta(X)=X\otimes 1+1\otimes X$ and $\eps(X)=0$), which is generated by a single polynomial $P$ as $k[X]$ is a PID. As we have $\eps(\mathfrak{I})=0$ and $\Delta(\mathfrak{I})\sub\mathfrak{I}\otimes k[X]+k[X]\otimes\mathfrak{I}$, we have $P(0)=0$ and there exists $A,B\in k[X,Y]$ such that
\[P(X+Y)-P(X)-P(Y)=A(X,Y)P(X)+B(X,Y)P(Y).\]
Dividing $A(X,Y)$ by $P(Y)$, we can suppose that the degree of $A$ in $Y$ i strictly smaller than that of $P$. By comparing the degrees in $Y$ of the both sides, we then deduce that $B=0$, whence $A=0$ and $P(X+Y)=P(X)+P(Y)$. If $f:\G_a\to\G_a$ is the endomorphism defined by $x\mapsto P(x)$, we then have $H=\ker f$, and $f$ is an epimorphism by (\cite{SGA3-1} $\Rmnum{6}_B$ 11.14). This proves the first assertion in (\rmnum{2}), and thus (\rmnum{1}).\par
Now assume that $p>0$, we then have $P=a_rX^{p^r}+\cdots+a_sX^{p^s}$, and $f=a_rF^r+\cdots+a_sF^s$, with $a_i\in k$ and $a_r,a_s\neq 0$. If we put $Q=a_rX+a_{r+1}X^p+\cdots+a_sX^{p^{s-r}}$ and $g=a_r\id+a_{r+1}F+\cdots+a_sF^{s-r}$, then $f=gF^r$. As $F^r$ is an epimorphism on $\G_a$, we then deduce an exact sequence
\[\begin{tikzcd}
1\ar[r]&\ker F^r\ar[r]&\ker f\ar[r]&\ker g\ar[r]&1
\end{tikzcd}\]
But $\ker g=\Spec(k[X]/Q)$ is smooth over $k$ and of dimension $0$, hence \'etale over $k$. As $\ker F=\bm{\alpha}_{p^r}$, we then conclude that $\ker F^r=\bm{\alpha}_{p^r}=H^0$ and $\ker g=\pi_0(H)$ (cf. \cref{scheme alg group pi_0 universal prop}). As $\pi_0(H)$ is \'etale, its base change to $\bar{k}$ is constant, so $\pi_0(H)\otimes_k{\bar{k}}\cong\Gamma_{\bar{k}}$, where $\Gamma$ is a finite subgroup of $\G_a(\bar{k})=\bar{k}$, which must be of the form $(\Z/p\Z)^n$. By comparing the rank of $\pi_0(H)\otimes_k\bar{k}$ and $\Gamma_{\bar{k}}$ over $\bar{k}$, we see that $n=s-r$.\par
It remains to consider the case where $k$ is perfect, for which we need the following general lemma:
\begin{lemma}\label{scheme alg group extension of etale by radiciel trivial if perfect field}
Let $k$ be a perfect field. Then any extension $H$ of an \'etale algebraic group $H''$ be an radiciel group $H'$ is trivial. Moreover there exists a unique lifting from $H''$ into $H$, namely $H_{\red}$.
\end{lemma}
In fact, as $k$ is perfect, the $k$-scheme $H_\red$ is an algebraic subgroup of $H$, and smooth over $k$ (\cref{scheme alg group flat smooth iff at unit}), hence \'etale, $H$ being of dimension $0$. To see that the canonical projection $H_\red\to H''$ is an isomorphism, it suffices to consider a base change $k\to\bar{k}$, and then it suffices to prove that we have an isomorphism on $\bar{k}$-values points, which is clear. The last assertion follows from the fact that any lifting of $H''$ into $H$, being \'etale over $k$, is reduced, hence is necessarily contained in $H_\red$.
\end{proof}

\begin{corollary}\label{scheme alg group quotient of elementary unipotent isomorphic}
Let $H$ be an elementary unipotent group over a field $k$, then any quotient group $H'$ of $H$ is either trivial or isomorphic to $H$.
\end{corollary}

We note that $\bm{\alpha}_{p^n}$ can be considered as a successive extension of $\bm{\alpha}_p$ by itself, so we conclude from \cref{scheme alg subgroup of G_a char} the following characterization:

\begin{corollary}\label{scheme alg group unipotent iff composition series quotient G_a Z/p a_p}
For an algebraic group $G$ defined over an algebraically closed field $k$ to be unipotent, it is necessary and sufficient that it possesses a composition series whose successive quotients are isomorphic to $\G_a$ if $p=0$, or one of the groups $\G_a$, $\Z/p\Z$, $\bm{\alpha}_p$ if $p>0$ (these are called \textbf{elementary unipotent groups}). 
\end{corollary}

\begin{corollary}\label{scheme alg group smooth connected 1-dim unipotent is G_a over perfect}
Suppose that $k$ is perfect. Then any one-dimensional smooth connected unipotent algebraic group over $k$ is isomorphic to $\G_a$.
\end{corollary}
\begin{proof}
If $p=0$, then $G$ possesses a composition series with quotients isomorphic to $\G_a$, hence it is isomorphic to $\G_a$. Now assume that $p>0$ and $k$ is algebraically closed. Then we have a composition series
\[G=G_0\sups G_1\sups\cdots\sups G_n=1\]
whose successive quotients are elementary unipotent groups (\cref{scheme alg group unipotent iff composition series quotient G_a Z/p a_p}). We now prove by induction on $i$ that $G/G_i$ is isomorphic to $\G_a$ for each $i$. The case $i=0$ is immediate: $G/G_1$ is smooth and connected, so it is isomorphic to $\G_a$. Suppose that $G/G_i$ is isomorphic to $\G_a$, so that we have an exact sequence
\[\begin{tikzcd}
1\ar[r]&G_i\ar[r]&G\ar[r]&\G_a\ar[r]&1
\end{tikzcd}\]
and $G_i$ is of dimension $0$, hence isomorphic to $\bm{\alpha}_p$ or $(\Z/p\Z)_k$. As $G$ is smooth and connected, it is then isomorphic to $\G_a$ by (\cite{DG} \Rmnum{3} \S 6, n5, 5.5). In the general case ($p>0$), we shall use the following lemma:
\begin{lemma}\label{scheme k-group separable extension isomorphic to G_a}
Let $k'$ be a separable extension of $k$, and $G$ be a $k$-group such that $G\otimes_kk'$ is isomorphic to $\G_{a,k'}$. Then $G$ is isomorphic to $\G_{a,k}$.
\end{lemma}
By hypothesis, $G$ is a form of $\G_a$ over $k$, and such forms are classified by the cohomology group $H^1(k'/k,\Aut_{\Grp}(\G_a))=H^1(k'/k,\G_m)$, which is trivial by by Hilbert 90 (as $\Pic(\Spec(k'))$ is trivial). Therefore $G$ is isomorphic to $\G_{a,k}$.
\end{proof}

\begin{proposition}[\textbf{Properties of Unipotent Groups}]\label{scheme alg group unipotent permanence prop}
\mbox{}
\begin{enumerate}
    \item[(\rmnum{1})] The property of unipotency is stable under base field change.
    \item[(\rmnum{2})] Any subgroup of a unipotent algebraic group is unipotent.
    \item[(\rmnum{3})] Any quotient group of a unipotent algebraic group is unipotent.
    \item[(\rmnum{4})] Any extension of a unipotent algebraic group by a unipotent algebraic group is unipotent. 
\end{enumerate}
\end{proposition}
\begin{proof}
We note that assertion (\rmnum{1}) follows immediately from the definition and \cref{scheme alg group unipotent iff extension to ac field}. To establish the other properties, we may assume that $k$ is algebraically closed. Then (\rmnum{2}) is evident from our definition.\par
Let $G$ be a unipotent algebraic group, $G'$ be a subgroup of $G$, $G''$ be a quotient group of $G$, and $(G_i)_{0\leq i\leq n}$ be a composition series of $G$ such that $H_i=G_i/G_{i+1}$ is an elementary unipotent group. To prove (\rmnum{2}), consider the composition series of $G'$ induced by that of $G$: $G_i'=G_i\cap G'$. The group $G'_i/G'_{i+1}$ is identified with a subgroup of $H_i$, hence is isomorphic to a subgroup of $\G_a$, therefore $G'$ is unipotent.\par
For (\rmnum{3}), we consider the composition series $(G_i'')$ of $G''$ induced by image of that of $G$. The group $G_i''/G_{i+1}''$ is then a quotient of $H_i$, and it suffices to apply \cref{scheme alg group quotient of elementary unipotent isomorphic}.
\end{proof}

\begin{proposition}\label{scheme alg group unipotent is affine}
A unipotent algebraic group over a field $k$ is affine.
\end{proposition}
\begin{proof}
By fpqc descent of affine morphisms, it suffices to prove the proposition under the assumption that $k$ is algebraically closed. In this case, $G$ is, by definition, multiple extensions of affine algebraic groups, hence is affine by \cref{scheme group fpqc quotient of monomorphism representable prop}~(\rmnum{8}) applied to affine morphisms.
\end{proof}

\begin{corollary}\label{scheme alg group unipotent if and only if}
Let $G$ be an algebriaic group over a field $k$. Then $G$ is unipotent if and only if it is affine and for any nontrivial subgroup $\widebar{H}$ of $G_{\bar{k}}$, there exists a nontrivial homomorphism $\widebar{H}\to\G_a$.
\end{corollary}
\begin{proof}
In view of the definitions, we may assume that $k$ is algebraically closed. If $G$ is unipotent, then any nontrivial subgroup $H$ is also unipotent (\cref{scheme alg group unipotent permanence prop}), so it has a nontrivial homomorphism to $\G_a$ (from its composition series). Conversely, if $G$ is affine and satisfies this condition, then there exists a nontrivial homomorphism $G\to\G_a$. By induction, we then obtain a sequence of subgroups $(G_i)$ of $G$ such that we have a nontrivial homomorphism $G_i\to\G_a$, and $G_{i+1}$ is the kernel of this homomorphism. As $G$ is Noetherian, this sequence stable for $i\gg 0$, and we then obtain a composition series of $G$ whose successive quotients are subgroups of $\G_a$, therefore $G$ is unipotent.
\end{proof}

\begin{example}\label{scheme alg group constant unipotent iff}
Let $G=M_k$ be a constant $k$-group. If $p=0$, then $G$ is unipotent if and only if $G$ is the \textit{trivial group}: In fact, $G$ is affine if and only if $M$ is finite (since an affine scheme must be quasi-compact), but $\G_a(k)=k$ does not contain any finite group in this case. If $p>0$, then $G$ is unipotent if and only if $M$ is a \textit{finite $p$-group}, i.e. the order of $M$ is a power of $p$. To see this, we note that any subgroup of $M_k$ is of the form $N_k$, where $N\sub M$ is a subgroup. If $M$ is a finite $p$-group and $N$ is nontrivial, then there exists an epimorphism $N\to\Z/p\Z$, hence a nonzero homomorphism $N_k\to\G_a$, so $M_k$ is unipotent in view of \cref{scheme alg group unipotent if and only if}. Conversely, if $M_k$ is unipotent, then $M$ is finite. If $x\in M$ has order coprime to $p$, then any homomorphism $(\Z x)_k\to \G_a$ is trivial, which is a contradiction.
\end{example}

\begin{proposition}\label{scheme alg group Hom of unipotent and multiplicative trivial}
Let $k$ be a field, $G$ be a $k$-group of multiplicative finite type, and $U$ be a unipotent algebraic group over $k$. Then
\[\sHom_{k\dash\Grp}(M,U)=\underline{e},\quad \Hom_{k\dash\Grp}(U,M)=e.\]
\end{proposition}
\begin{proof}
With the hypothesis, for a group homomorphism $u:M\to U$ to be trivial, it suffices that the restriction to fibers of $M$ over the points of $S$ is the trivial homomorphism (\cref{scheme group multiplicative morphism trivial on fiber lifting}). We are therefore reduced to the case where $S$ is the spectrum of a field, which can be assumed to be algebraically closed. In view of the definition of unipotent group, we it suffices to assume that $U=\G_a$, in which case the property has already been demonstrated (\cref{scheme group multiplicative morphism to G_a trivial}).\par
We now consider the assertion for $\Hom_{k\dash\Grp}(U,M)$, so let $u:U\to M$ a $k$-morphism of groups. The image $u(U)$ is representable by an algebraic subgroup $U'$ of $M$ (\cite{SGA3-1} $\Rmnum{6}_B$ 5.4), which is unipotent as the quotient of a unipotent group (\cref{scheme alg group unipotent permanence prop}~(\rmnum{3})) and is of multiplicative type as a subgroup of a group of multiplicative type (cf. \cref{scheme k-group multiplicative quotient subgroup is multiplicative}), so $U'$ is the trivial group according to the first assertion.
\end{proof}

\begin{remark}
Note that it is not true in general that $\sHom_{k\dash\Grp}(U,M)=\underline{e}$ (with the notations of \cref{scheme alg group Hom of unipotent and multiplicative trivial}). For example, take a scheme $S$ such that $\Gamma(S,\mathscr{O}_S)$ contains a nonzero square zero element $\eps$ (for example the spectrum of the algebra of dual numbers of a ring $A$). For any $S'$ over $S$, the maps $u\mapsto 1+\eps_{S'}u$ defines a functorial homomorphism from $\Gamma(S',\mathscr{O}_{S'})$ to $\Gamma(S',\mathscr{O}_{S'}^\times)$, whence defines an $S$-group morphism $\G_{a,S}\to\G_{m,S}$. As $\eps\neq 0$, this morphism is nontrivial.
\end{remark}

\begin{example}\label{scheme alg group affine commutative unipotent iff multiplicative trivial}
An affine commutative algebraic group $G$ is unipotent if and only if any multiplicative subgroup of $G$ is trivial. In fact, if $G$ is unipotent, then any such subgroup $H$ has a nontrivial homomorphism to $\G_a$, which is a contradiction by \cref{scheme alg group Hom of unipotent and multiplicative trivial}. Conversely, if $G$ satisfies this condition, any nontrivial subgroup $H$ of $G$ is not multiplicative, hence has a nontrivial homomorphism to $\G_a$ (\cref{scheme k-group abelian multiplicative iff}~(\rmnum{4})).
\end{example}

\subsection{Linear representations of unipotent groups}
Recall that if $S$ is a scheme and $\mathscr{M}$ is an $\mathscr{O}_S$-module, we denote by $\mathbf{W}(\mathscr{M})$ the functor $\Sch_{/S}^{\op}\to\Set$ defined by the condition $\mathbf{W}(\mathscr{M})(S')=\Gamma(S',\mathscr{M}\otimes_{\mathscr{O}_S}\mathscr{O}_{S'})$. Further, recall that for a functor $F$ acted by an $S$-group $G$, we denote by $F^G$ the $S$-functor of invariants of $F$ under $G$, whose value at $S'$ is the set of points $x\in F(S')$ such that $x_{S''}$ is fixed by $G(S'')$ for any $S''\to S'$.

\begin{lemma}\label{scheme alg group representation invariant isomorphic to Gamma}
Let $S$ be a scheme, $G$ be an affine group scheme over $S$, defined by a quasi-coherent $\mathscr{O}_S$-algebra $\mathscr{A}$. Suppose that $G$ acts on a quasi-coherent $\mathscr{O}_S$-module $\mathscr{M}$, and let $\mu:\mathscr{M}\to\mathscr{A}\otimes_{\mathscr{O}_S}\mathscr{M}$ be the comodule morphism. Consider the morphism
\[\nu:M\to\mathscr{A}\otimes_{\mathscr{O}_S}\mathscr{M},\quad x\mapsto\mu(x)-1\otimes x.\]
\begin{enumerate}
    \item[(a)] $\mathscr{M}^G(S)=\Gamma(S,\ker\nu)$.
    \item[(b)] If $S$ is the spectrum of a field $k$, then $\mathscr{M}^G$ is of the form $\mathbf{W}(\mathscr{N})$, where $\mathscr{N}$ is a subspace of $\mathscr{M}$.
    \item[(c)] If $S$ is the spectrum of a field $k$, any element $x\in\mathscr{M}$ is contained in a subspace of $M$, of finite dimension over $k$ and stable under the action of $G$.
\end{enumerate}
\end{lemma}
\begin{proof}
The last assertion (c) is already proved in (\cite{SGA3-1} $\Rmnum{6}_B$ 11.2). It is clear that $\mathscr{M}^G(S)$ is contained in $\Gamma(S,\ker\nu)$, to prove the converse, we can assume that $S$ is affine with ring $B$, and $\mathscr{A}=\widetilde{A}$. Let $m\in\mathscr{M}^G(S)$, then for any $B$-algebra $B'$ and any $u\in\Hom_{B\dash\Alg}(A,B')=G(\Spec(B'))$, we have
\[(u\otimes 1_M)\nu(m)=0\in B'\otimes_BM.\]
In particular, putting $B'=A$ and $u=\id_A$, we see that $\nu(m)=0$, so $m\in\Gamma(S,\ker\nu)$.\par
Now assume that $S=\Spec(k)$, where $k$ is a field. Let $\mathscr{N}$ be the kernel of $\nu$, whose global section is equal to $\mathscr{M}^G(k)$ by (a). Any $k$-scheme $S$ is flat over $k$, so we have
\[M^G(S)=\Gamma(S,\ker(\nu\otimes_kS))=\Gamma(S,\mathscr{N}\otimes_kS),\]
so $\mathscr{M}^S$ is isomorphic to the functor $\mathbf{W}(\mathscr{N})$.
\end{proof}

\begin{proposition}\label{scheme alg group representation invariant nonzero}
Let $G$ be a unipotent algebraic group over a field $k$, acting on a $k$-vector space $V$. If $V\neq 0$, then $V^G\neq 0$.
\end{proposition}
\begin{proof}
In view of \cref{scheme alg group representation invariant isomorphic to Gamma}~(b) and (c), we can suppose that $k$ is algebraically closed and $V$ is finite-dimensional over $k$. Let $1\to G'\to G\to G''\to 1$ be an exact sequence of algebraic groups and suppose that $G$ acts over a sheaf $F$ (for the fpqc topology). Then we have $F^G\sub F^G{'}$, and the quotient sheaf $G/G'$ acts naturally on $F^{G'}$. But $F^{G'}$ is a sheaf (as the kernel of a pair of morphisms $V\rightrightarrows\sHom(G',V)$), so the associated sheaf $G/G'$, that is, $G''$, acts on $F^{G'}$, and it is immediate that $(F^{G'})^{G''}=(F^{G'})^{G/G'}$. This reduces us to the case where $G$ is an elementary unipotent group.\par
If $p=0$ and $G=\G_a$, it follows from (\cite{Chevalley1958} 4 prop.4) that a morphism $\G_a\to\GL(V)$ is fiven by a exponential map
\[T\mapsto\sum_{i=0}^{\infty}\frac{n^i}{i!}\]
where $n$ is a nilpotent endomorphism of $V$. But then $V\neq 0$ implies $\ker n\neq 0$, and it is clear that any vector of $V$ annihilated by $n$ is also fixed by $G$.\par
Now suppose that $p>0$. If $G=\bm{\alpha}_p$, then as $\bm{\alpha}_p$ is a radiciel group of height $1$ (\cite{SGA3-2} $\Rmnum{7}_A$ \S 7), any representation of $\bm{\alpha}_p$ over $V$ is given by a representation of the Lie $p$-algebra of $\bm{\alpha}_p$ over $V$, that is, by an element $x\in\End(V)$ such that $x^p=0$ (cf. \cite{SGA3-2} $\Rmnum{17}$ App \Rmnum{2}. B2.1). But then $V\neq 0$ implies $W=\ker x\neq 0$, and we see by (cf. \cite{SGA3-2} $\Rmnum{17}$ App \Rmnum{2}. B2.2) that $W=V^{\bm{\alpha}_p}$. On the other hand, if $G=\Z/p\Z$, then a representation of $G$ over $V$ is determined by an endomorphism $x\in\Aut(V)$ such that $x^p=1$, i.e. $(x-1)^p=0$. Therefore $x=1+n$ for a nilpotent endomorphism of $V$, and $W=\ker n$ is fixed by $x$.\par
Finally, let $G=\G_a$. Consider an increasing filtered family $(G_i)$ of \'etale subgroups of $\G_a$, which are isomorphic to $(\Z/p\Z)^{r_i}$ (cf. \cref{scheme alg subgroup of G_a char}) and let $V_i=V^{G_i}$. As $V$ is of dinite nonzero dimension and $V_i$ are nonzero, the decreasing filtered family $V_i$ is stable, and $W=\bigcap_iV_i\neq 0$. We claim that $(G_i)_{i\in I}$ is schematically dense in $G$, so for $w\in W$, its stabilizer in $G$ is then a subgroup of $\G_a$ dominating $G_i$ for each $i\in I$, hence is equal to $\G_a$ and therefore we have $W=V^G$. To completes the proof, it remains to prove the following lemma:
\begin{lemma}\label{scheme etale subgroup of G_a is schematically dense}
The family of \'etale subgroups of $\G_{a,S}$ ($S$ of characteristic $p>0$) is schematically dense in $G$.
\end{lemma}
By \cref{scheme Noe flat schematically dense if fiber is}, it suffices to prove this lemma for $S=\Spec(\F_p)$. In this case, it then suffices to consider the family of \'etale subgroups $G_n=\Spec(k[X]/(X^{p^n}-X))$ ($n\geq 1$), which are schematically dense in $\G_{a,\F_p}$ since they contain any closed point of $\G_{a,\F_p}$.
\end{proof}

Using \cref{scheme alg group representation invariant nonzero}, we can now give the following characterization of unipotent algebraic groups:

\begin{theorem}\label{scheme alg group unipotent if and only if representation}
Let $G$ be an algebraic group defined over a field $k$. Then the following conditions are equivalent:
\begin{enumerate}
    \item[(\rmnum{1})] $G$ is unipotent.
    \item[(\rmnum{2})] $G$ possesses a (central) composition series whose successive quotients are isomorphic to $\G_a$ if $p=0$, and to $\bm{\alpha}_p$, $\G_a$, or twisted $(\Z/p\Z)^r$ if $p>0$. In particular, it is nilpotent.
    \item[(\rmnum{3})] $G$ possesses a characteristic composition series whose successive quotients are isomorphic to $(\G_a)^r$ if $p=0$, and to $(\bm{\alpha}_p)^r$, $(\G_a)^s$, or twisted $(\Z/p\Z)^r$, taken precisely in this order, if $p>0$.
    \item[(\rmnum{4})] $G$ is affine and for any finite-dimensional linear representation $V$ of $G$, we have $V^G\neq 0$ if $V\neq 0$.
    \item[(\rmnum{5})] $G$ is affine and for any finite-dimensional linear representation $V$ of $G$, there exists a flag
    \[0=V_0\subset V_1\subset\cdots\subset V_n=V\]
    stable under $G$ and such that $G$ acts trivially on the quotients $V_{i+1}/V_i$. We can also suppose that each $V_{i+1}/V_i$ is one dimensional.
    \item[(\rmnum{6})] There exists a faithful representation $G\to\GL(V)$ and a flag $0=V_0\subset V_1\subset\cdots\subset V_n=V$ stable under $G$ such that $G$ acts trivially on the quotients $V_{i+1}/V_i$.
    \item[(\rmnum{7})] $G$ is isomorphic to a subgroup of the group $\mathbb{U}_n$ of strict upper triangular matrices, for some $n>0$.
\end{enumerate}
\end{theorem}
\begin{proof}
We have seen that (\rmnum{1})$\Rightarrow$(\rmnum{4}) in \cref{scheme alg group representation invariant nonzero}, and (\rmnum{4})$\Rightarrow$(\rmnum{5}). In fact, we can then define $V_i$ inductively such that $V_{i+1}/V_i=(V/V_i)^G$. If $\dim(V_{i+1}/V_i)\neq 1$ for some $i$, we can then refine the inclusion $V_i\sub V_{i+1}$ into a sequence whose successive quotients have dimension $1$ (it is clear that $G$ still acts trivial on each quotient). To see that (\rmnum{5})$\Rightarrow$(\rmnum{6}), we can apply (\cite{SGA3-1} $\Rmnum{6}_B$ 11.11), since then $G$ admits a faithful linear representation.\par
Now (\rmnum{6})$\Rightarrow$(\rmnum{7}): by refining the flag, we may assume that $\dim_k(V_{i+1}/V_i)=1$ for each $i$. For any $i\geq 1$, let $e_i\in V_i-V_{i-1}$, then for a $k$-algebra $R$ and $g\in G(R)$, the automorphism $\rho(g)$ is represented by a matrix $u(g)\in\mathbb{U}_n(R)$ under the bases $e_1\otimes 1,\dots,e_n\otimes 1$ of $V\otimes_kR$, so $g\mapsto u(g)$ gives the desired embedding. We also recall that the algebraic group $\mathbb{U}_n$ admits a canonical central composition series (cf. standard composition series of $\mathbb{U}_n$ \cref{*}), with successive quotients $\G_a$, so (\rmnum{7})$\Rightarrow$(\rmnum{2}) in view of \cref{scheme alg subgroup of G_a char}. Finally, in view of the definition, it is clear that (\rmnum{2})$\Rightarrow$(\rmnum{1}).
\end{proof}

The implication (\rmnum{1})$\Rightarrow$(\rmnum{3}) in \cref{scheme alg group unipotent if and only if representation} is a little bit complicated, so we will deal with it later. For now, let us note some concequences of \cref{scheme alg group unipotent if and only if representation}:

\begin{corollary}\label{scheme alg affine orbit under unipotent is closed}
Let $G$ be a unipotent algebraic group acting on an affine algebraic $k$-scheme $X$. For any $x\in G(k)$, the orbit $G\cdot x$ (i.e. the image of the morphism $G\to X,g\mapsto g\cdot x$) is closed in $X$, hence is affine.
\end{corollary}
\begin{proof}
We may suppose that $k$ is algebraically closed (cf. \cite{SGA1} \Rmnum{8} 4.6), then $G_\red$ is a subgroup of $G$ acting on $X_\red$, and $(G\cdot x)_\red=G_\red\cdot x$; we can hence suppose that $G$ and $X$ are reduced. Let $Y=G\cdot x$, and $\widebar{Y}$ be the closure of $Y$, endowed with the reduced induced structure; then any point $g\in G(k)$ induces an automorphism of $X$ preserving $Y$, hence $\widebar{Y}$, so $\widebar{Y}$ is stable under $G$. Replacing $X$ by $\widebar{Y}$, we can therefore assume that $Y$ is dense in $X$, hence open in $X$ by the orbit lemma (\cref{scheme alg group closed orbit lemma}). Let $\mathfrak{I}$ be the functions $f\in A=\Gamma(X,\mathscr{O}_X)$ vanishing over the closed subset $|X|-|Y|$ (which is also stable under $G$). Then $\mathfrak{I}$ is a sub-$k[G]$-module of $A$; as $G$ is unipotent and $\mathfrak{I}\neq 0$, we have $\mathfrak{I}^G\neq 0$ by \cref{scheme alg group unipotent if and only if representation}. If $0\neq f\in\mathfrak{I}^G$, then we have $f(gx)=f(x)$ for $g\in G(k)$, so $f$ is constant on $Y(k)$, hence over $X(k)$, and over $X$ by Nullstellensatz. This ensures that $\mathfrak{I}$ consists of constant functions, so $Y=X$, and this proves the corollary.
\end{proof}

\begin{corollary}
Let $G$ be an affine algebraic group and $H$ be a unipotent subgroup of $G$. Then $G/H$ is isomorphic to a subscheme of an affine scheme, and is affine if $G$ is unipotent.
\end{corollary}
\begin{proof}
By (\cite{SGA3-1} $\Rmnum{6}_B$), there exists a finite-dimensional linear representation $G\to\GL(V)$ such that $H$ is the normalizer of a one-dimensional subspace $D$ of $V$. By \cref{scheme alg group representation invariant nonzero}, $D^H\neq 0$, hence $D=D^H$ and $Z_G(D)=H$. Let $0\neq x\in D$, then by \cref{scheme alg group closed orbit lemma}, $g\mapsto gx$ defines an immersion $G/H\to\mathbf{W}(V)$, which proves the first assertion. If $G$ is unipotent, then the image of this immersion is closed by \cref{scheme alg affine orbit under unipotent is closed}, so $G/H\cong G\cdot x$ is affine.
\end{proof}

\begin{corollary}\label{scheme alg group unipotent simple module is k}
If $G$ is a unipotent algebraic group over $k$, then up to isomorphisms, $k$ is the only simple $G$-module.
\end{corollary}
\begin{proof}
If $G$ is unipotent, then in view of \cref{scheme alg group representation invariant nonzero}, any simple $G$-module is one dimensional and acted trivially by $G$, so it is isomorphic to $k$.
\end{proof}

\begin{corollary}\label{scheme alg group connected ker Ad quotient Z(G) is unipotent}
Let $G$ be a connected algebriac group over $k$. Let $\Ad:G\to\GL(\g)$ be the adjoint representation of $G$, then the quotient group $(\ker\Ad)/Z(G)$ is unipotent.
\end{corollary}
\begin{proof}
Let $\mathscr{O}_{G,e}$ be the local ring of $G$ at $e$, and $\m_e$ be the maximal ideal at $e$. Let $V_n=\Spec(\mathscr{O}_e/\m_e^{n+1})$ be the $n$-th neighborhood of the unit section. It is clear that $V_n$ is stable under the operation of $G$ on $G$ by inner automorphisms (the construction of $V_n$ commutes with base change), so $G$ acts on $\mathscr{O}_e/\m_e^{n+1}$ by automorphism of algebras. Moreover, $\ker\Ad$ acts trivially over $\m_e^i/\m_e^{i+1}$, so by \cref{scheme alg group unipotent if and only if representation}~(\rmnum{6}), the group $(\ker\Ad)/Z_G(V_n)$ is unipotent.\par
It remains to show that $Z(G)=Z_G(V_n)$ for $n$ large enough: in fact, let $H=\bigcap_nZ_G(V_n)$, which clearly contains $Z(G)$; as $G$ is Noetherian, we evidently have an integer $n_0\geq 0$ such that $H=Z_G(V_n)$ for $n\geq n_0$. Let $\mathscr{I}$ be the ideal of $\mathscr{O}_G$ defining $Z_G(H)$. As $H$ centralizes $V_n$, the latter is dominated by $Z_G(H)$, so we have $\mathscr{I}_e\sub\m_e^{n+1}$ for any $n$, whence $\mathscr{I}_e=0$. Since $\mathscr{I}$ is of finite type, there then exists an open neighborhood $U$ of $e$ such that $\mathscr{I}|_U=0$ (cf. \cref{sheaf of module ft local prop}), so $Z_G(H)$ contains $U\cdot U$, and is therefore equal to $G$ (\cref{scheme A-group product of open dense is G} and \cref{scheme alg group identity component prop}). It then follows that $H=Z(G)$, and this completes the proof.
\end{proof}

Recall that if $G$ is an algebraic group over a field $k$ of characterstic $p$, its Lie algebra is naturally endowed with a structure of a Lie $p$-algebra (cf. \cite{SGA3-1} $\Rmnum{7}_A$ \S 6). This Lie $p$-algebra $\g$ is called \textbf{unipotent} if the map $x\mapsto x^{[p]}$ is nilpotent, that is, if for any $x\in\g$, there exists an integer $n>0$ such that $x^{[p^n]}=0$.

\begin{corollary}\label{scheme alg group unipotent is nilpotent Lie algebra}
A unipotent algebraic group $G$ is nilpotent; its Lie algebra is nilpotent and is isomorphic to a subalgebra of $\gl(V)$ formed by nilpotent endomorphisms, where $V$ is a finite dimensional vector space. In characteristic $p>0$, $\g$ is a unipotent Lie $p$-algebra.
\end{corollary}
\begin{proof}
By \cref{scheme alg group unipotent if and only if representation}~(\rmnum{7}), it suffices to prove \cref{scheme alg group unipotent is nilpotent Lie algebra} for $G=\mathbb{U}_n$. We have already used the fact that $\mathbb{U}_n$ is a nilpotent algebraic group. Moreover, the Lie algebra $\mathfrak{u}$ of $\mathbb{U}_n$ is formed by upper triangular endomorphisms of $V$ which have zeros on the main diagonal. They are therefore nilpotent, and hence $\mathfrak{u}$ is nilpotent. If $p>0$, as the $p$-th power in the Lie $p$-algebra $\gl(V)=\End(V)$ coincides with the $p$-th power of the endomorphisms of $V$ (\cite{SGA3-1} $\Rmnum{7}_A$ 6.4.4), we see that $\mathfrak{u}$ is unipotent.
\end{proof}

\begin{corollary}\label{scheme alg group smooth affine unipotent iff G(k) unipotent}
Let $k$ be an algebriacally closed field and $G$ be an affine smooth algebraic group over $k$. Then $G$ is unipotent if and only if $G(k)$ consists of unipotent elements (\cite{Chevalley1958} 4 prop.4 cor.1).
\end{corollary}
\begin{proof}
If $G$ is unipotent, then it is isomorphic to a subgroup of $\mathbb{U}_n$, hence $G(k)$ is formed by unipotent elements. Conversely, assume that $G(k)$ is formed by unipotent elements, and let $G^0$ be the identity component of $G$. The maximal tori of $G^0$ are then formed by unipotent elements, hence are trivial, so $G^0$ is a Cartan subgroup of itself. Therefore, $G^0$ is solvable (\cite{Chevalley1958} 6 th.6), hence triangularizable (\cite{Chevalley1958} 6 th.1). In other words, $G^0$ is a subgroup of $\mathbb{U}_n$, so it is unipotent.\par
The group $(G/G^0)(k)$ is a finite group formed by unipotent elements (as $k$ is algebraically closed and $(G/G^0)$ is \'etale), so is it zero if $p=0$ and equals to a finite $p$-group if $p>0$ (\cite{Chevalley1958} 4 prop.4). But then $G/G^0$ is unipotent in view of \cref{scheme alg group constant unipotent iff}, so $G$ is unipotent by \cref{scheme alg group unipotent permanence prop}.
\end{proof}

Now we return to the proof of \cref{scheme alg group unipotent if and only if representation}. It is clear that (\rmnum{3})$\Rightarrow$(\rmnum{1}), so we only need to derive (\rmnum{3}) from (\rmnum{1}). If the characteristic $p>0$, then consider the increasing sequence of subgroups of $G$:
\[\{1\}\subset{_FG}\subset{_{F^2}G}\subset\cdots\subset{_{F^n}}G\sub G^0\sub G,\]
where $F$ is the Frobenius morphism. This is a characteristic composition series of $G$ (\cite{SGA3-2} \Rmnum{17} App. \Rmnum{2} 1), and for $n\gg 0$, $G/_{F^n}G$ is smooth (\cite{SGA3-2} \Rmnum{17} App. \Rmnum{2} 3.1), so that the successive quotients are (in order):
\begin{enumerate}
    \item[(a)] radiciel groups of height $1$,
    \item[(b)] a smooth and connected group,
    \item[(c)] an \'etale group.
\end{enumerate}
The proof of (\rmnum{1})$\Rightarrow$(\rmnum{3}) then reduces to that of the following lemma:

\begin{lemma}\label{scheme alg group unipotent over p>0 char composition series}
Let $G$ be a unipotent algebraic group defined over a field $k$ of characteristic $p>0$. Then $G$ possesses a characteristic composition series whose successive quotients are isomorphic to
\begin{enumerate}
    \item[(\rmnum{1})] $(\bm{\alpha}_p)^r$ if $G$ is radiciel.
    \item[(\rmnum{2})] Twisted $(\G_a)^r$ if $G$ is smooth and connected.
    \item[(\rmnum{3})] Twisted $(\Z/p\Z)^r$ if $G$ is \'etale. 
\end{enumerate}
\end{lemma}
\begin{proof}
The proof of \cref{scheme alg group unipotent over p>0 char composition series}~(\rmnum{1}): If $G$ is radiciel, then by filtering $G$ by the subgroups ${_{F^n}G}$, we are reduced to the case where $G$ is radiciel of height $1$. As $G$ is nilpotent (\cref{scheme alg group unipotent if and only if representation}~(\rmnum{2})) and its is representable (\cref{scheme subfunctor Weil restriction example}), we can consider the ascendent central series of $G$, which is evidently characteristic in $G$. This reduces us to the case where $G$ is commutative.\par
Let $\g=\mathfrak{Lie}(G)$. The morphism $\pi$ of $p$-th power is additive on $\g$ (\cite{SGA3-1} $\Rmnum{7}_A$ \S 6), and we shall see we can reduce to the case where it is zero. For any scheme $S$ over $k$, put $\g_S=\g\otimes_k\mathscr{O}_S$ and $\h_S$ be the image sheaf of $\pi_S$ in $\g_S$. Let $\widebar{\h}_S$ be the sub-$\mathscr{O}_S$-modules of $\g_S$ generated by $\h_S$. It is clear that $\widebar{\h}_S=\h_k\otimes_k\mathscr{O}_S$ and that $\widebar{\h}_S$ is a characteristic Lie sub-$p$-algebra of $\g_S$ (that is, stable under $\sAut_{p\dash\Lie}(\g)$). It then follows from (\cite{SGA3-2} \Rmnum{17} App. \Rmnum{2} 2.2) that $\widebar{\h}_k$ is the Lie algebra of a characteristic subgroup $H$ of $G$.\par
In view of \cref{scheme alg group unipotent is nilpotent Lie algebra} and \cref{scheme Lie p-algebra subalgebra of p-th power proper} below, if $G\neq\{e\}$, $H$ is distinct from $G$ because $\widebar{\h}_k\neq\g$. Set $G''=G/H$, we then have $\mathfrak{Lie}(G'')=\g/\widebar{\h}_k$ (\cite{SGA3-2} \Rmnum{17} App. \Rmnum{2} 2.2), so the $p$-th power is zero on $\mathfrak{Lie}(G'')$. Preceding by induction on $\dim(\g)$, we are then reduced to the case where $\g$ is a Lie $p$-algebra with trivial $p$-th power. But then $\g$ is isomorphic to $\mathfrak{Lie}(\bm{\alpha}_p^r)$ for some $r\geq 0$ (\cite{SGA3-2} \Rmnum{17} App. \Rmnum{2} 2.1), so $G''$ is isomorphic to $(\bm{\alpha}_p)^r$ (\cite{SGA3-2} \Rmnum{17} App. \Rmnum{2} 2.2). It remains to prove the following lemma:
\begin{lemma}\label{scheme Lie p-algebra subalgebra of p-th power proper}
Let $k$ be a field of characteristic $p>0$, $\g$ be a commutative unipotent Lie $p$-algebra of finite dimension over $k$, and $\h$ be a sub-$p$-algebra of $\g$ generated by the $p$-th powers in $\g$. Then if $\g\neq 0$, we have $\h\neq\g$.
\end{lemma}
In fact, as $\g$ is commutative, $\h$ is simply the subspace of $\g$ generated by the $x^{[p]}$, ($x\in\g$). If $\g\neq 0$ and is unipotent, there exists $0\neq x\in\g$ such that $x^{[p]}=0$. Let $x_1,\dots,x_n$ be a basis of a supplementary in $\g$ of the line $k\cdot x$. The subalgebra $\h$ is then the subspace of $\g$ generated by $x_1^{[p]},\dots,x_n^{[p]}$, hence is of dimension at most $n-1$.\par
The proof of \cref{scheme alg group unipotent over p>0 char composition series}~(\rmnum{2}): Assume that $G$ is smooth connected and $k$ is algebraically closed. In this case, the descending central series of $G$ is representable by smooth connected characteristic subgroups $G_i$ of $G$ (\cite{SGA3-1} $\Rmnum{6}_B$ 8.3 et 7.4), and $G_i=0$ for $i\gg 0$ since $G$ is nilpotent. It suffices to prove \cref{scheme alg group unipotent over p>0 char composition series} for the groups $G_i/G_{i+1}$, so we are reduced to the case where $G$ is also commutative. For any integer $n>0$, let $G_n$ be the image of $G$ under the $p^n$-th power morphism (that is, the morphism $x\mapsto x^{p^n}$, if we write $G$ multiplicatively). The group $G_n$ is hence smooth, connected and characteristic. As we can embedd $G$ into some $\mathbb{U}_N$ by \cref{scheme alg group unipotent if and only if representation}, we see that $G_n=0$ for $n\gg 0$\footnote{Note that for $g\in\U_n(k)$, we have $(g-1)^{p^n}=g^{p^n}-1=0$ because $(g-1)$ is a nilpotent endomorphism over $k^n$.}. Replacing $G$ with $G_n/G_{n+1}$, we can also suppose that $G$ is annihilated by the $p$-th power morphism. But then by (\cite{Serre_GACC} \Rmnum{7}, prop.11), $G$ is of the form $(\G_a)^r$, for some $r\geq 0$.\par
The proof of \cref{scheme alg group unipotent over p>0 char composition series}~(\rmnum{3}): Preceding in the same way, we may assume that $G$ is \'etale, commutative, and annihilated by $p$-th power. But then $G_{\bar{k}}$ is isomorphic to the constant group $(\Z/p\Z)^r_{\bar{k}}$, so $G$ is a twisted $(\Z/p\Z)^r$ group. This completes the proof of \cref{scheme alg group unipotent over p>0 char composition series}.
\end{proof}

It remains to prove (\rmnum{1})$\Rightarrow$(\rmnum{3}) of \cref{scheme alg group unipotent if and only if representation} for $p=0$. In this case, the group $G$ is automatically smooth connected (as a multiple extension of $\G_a$), and preceding as in \cref{scheme alg group unipotent over p>0 char composition series}~(\rmnum{2}), we may assume that $G$ is also commutative. We then have the following more precise result:

\begin{lemma}\label{scheme alg group unipotent commutative isomorphic to exp(W(g))}
Let $k$ be a field of characteristic $0$ and $G$ be a unipotent commutative algebraic group over $k$, $\g=\mathfrak{Lie}(G)$. Then there exists a canonical isomorphicm
\[\exp:W(\g)\stackrel{\sim}{\to} G\]
More over, $\exp$ is the unique homomorphism $W(\g)\to G$ which induces the identity on Lie algebras.
\end{lemma}
\begin{proof}
As $G$ is unipotent, it can be embedded into the group $\mathbb{U}_n$ for some $n$. The choice of such an embedding asslows us to identity $\g$ as a subalgebra of $\gl_n$, formed by nilpotent endomorphisms. We then have a $k$-morphism
\[\exp:W(\g)\to\GL_n,\quad T\mapsto\sum_{i=0}^{\infty}\frac{T^i}{i!}.\]
As $G$ is commutative, the morphism $\exp$ is a group homomorphism. Let $G'$ be the image of $W(\g)$ under $\exp$. If we identity the Lie algebra of $W(\g)$ with $\g$, then the tangent map of $\exp$ is simple the injection $\g\to\gl_n$. Therefore $\mathfrak{Lie}(G\cap G')=\g\cap\g=\g$. As $G\cap G'$ is automatically smooth and $G$ is connected, we necessarily have $G=G'$. The kernel $\ker\exp$ is a unipotent \'etale group, hence is trivial and $\exp$ is therefore an isomorphism from $W(\g)$ to $G$.\par
If $W(\g)\to G$ is another homomorphism such that $\mathfrak{Lie}(h)$ is the identity on $\g$, then the morphism $h-\exp$ is a group homomorphism ($G$ being commutative) with trivial tangent map. As $k$ has characteristic zero and $W(\g)$ is connected, this implies that $h=\exp$, whence the last assertion.
\end{proof}

In particular, by choosing a basis for $\g$, we see that $G$ is $k$-isomorphic to the vectorial group $(\G_a)^r$, and therefore satisfies condition (\rmnum{3}). This proves (\rmnum{1})$\Rightarrow$(\rmnum{3}) and completes the proof of \cref{scheme alg group unipotent if and only if representation}.\par

We have seen that for a unipotent group $G$, any homomorphism from a multiplicative group to $G$ is necessarily trivial (\cref{scheme alg group Hom of unipotent and multiplicative trivial}). In fact, such groups are characterized by the following proposition:

\begin{proposition}\label{scheme alg group morphism from multiplicative zero iff}
Let $G$ be an algebraic group over an algebraically closed field $k$. Then the following conditions are equivalent:
\begin{enumerate}
    \item[(\rmnum{1})] Any morphism from a multiplicative group $M$ to $G$ is trivial.
    \item[(\rmnum{1}')] $G$ does not possesses any non-trivial subgroup of multiplicative type.
    \item[(\rmnum{2})] (If $p=0$) $G(k)$ does not contain points of finite order other than $e$.
    \item[(\rmnum{2}')] (If $p>0$) Any subgroup of $G(k)$ is a $p$-group and for any $x\in\g=\mathfrak{Lie}(G)$ such that $x^{[p]}=x$, we have $x=0$. 
\end{enumerate}
\end{proposition}

In the next subsection, we shall see that the converse of this is also true, i.e. $G$ is unipotent if and only if it satisfies the equivalent conditions of \cref{scheme alg group morphism from multiplicative zero iff}.

\subsection{A characterization of unipotent groups}
As we have announced, we are going to show that an algebraic group $G$ defined over an algebraically closed field $k$, which does not contain any nontrivial subgroup of multiplicative type, is unipotent. In fact, it is enough that it does not contain subgroups of very particular "elementary" multiplicative type, which depends on the assumptions made on $G$. Before stating the general theorem, let us study in detail some particular cases.
\paragraph{Case of affine smooth connected algebraic groups}
\begin{proposition}\label{scheme alg group smooth affine connected unipotent iff}
Let $k$ be a field, $G$ be an affine smooth connected algebraic group over $k$, $\g=\mathfrak{Lie}(G)$. Then the following propoties are equivalent:
\begin{enumerate}
    \item[(\rmnum{1})] $G$ is unipotent.
    \item[(\rmnum{2})] $G$ possesses a central composition series whose successive quotients are forms of $\G_a$.
    \item[(\rmnum{3})] $G$ possesses a central characteristic composition series whose successive quotients are forms of $(\G_a)^r$.
    \item[(\rmnum{4})] There exists an integer $n>1$ such that $G_{\bar{k}}$ does not contains any subgroup isomorphic to $\bm{\mu}_n$.
    \item[(\rmnum{5})] Any maximal torus of $G$ is trivial.
\end{enumerate}
Suppose that $G$ is a subgroup of $\GL_n$, then the preceding conditions are also equivalent to:
\begin{enumerate}
    \item[(\rmnum{6})] $\g\sub\gl_n$ is formed by nilpotent endomorphisms.
    \item[(\rmnum{7})] $\g$ is nilpotent and its center does not contain nontrivial semi-simple endomorphisms. 
\end{enumerate}
\end{proposition}
The implication (\rmnum{2})$\Rightarrow$(\rmnum{1}) is clear and (\rmnum{1})$\Rightarrow$(\rmnum{3}) follows from \cref{scheme alg group unipotent over p>0 char composition series}. The implication (\rmnum{3})$\Rightarrow$(\rmnum{2}) follows from the following lemma:
\begin{lemma}\label{scheme alg group form of (G_a)^r prop}
Let $k$ be a field, $G$ be an algebraic group over $k$ which is a form of $(\G_a)^r$.
\begin{enumerate}
    \item[(a)] $G$ is isomorphic to a subgroup of $(\G_a)^n$ for some $n$.
    \item[(b)] $G$ possesses a composition series whose successive quotients are isomorphic to $\G_a$.
\end{enumerate}
\end{lemma}
\begin{proof}
In fact, by hypothesis, there exists an extension $k'$ of $k$ such that $G_{k'}$ is isomorphic to $(\G_{a,k'})^r$. By the principal of finite extension (cf. \cite{EGA4-3} 9.1.4), we can suppose that $k'$ is a finite extension of $k$. But then for (a), it suffices to consider the canonical closed immersion
\[G\to\Res_{k'/k}(G_{k'})\stackrel{\sim}{\to}(\G_{a,k})^n,\quad n=r[k':k].\]
To prove (b), in view of (a) we can suppose that $G$ is a closed subgroup of $G'=(\G_a)^n$. If $G\neq 0$, there exists a hyperplane $\h$ of $\g'=\mathfrak{Lie}(G')$ which is not contained in $\g=\mathfrak{Lie}(G)$. Let $H$ be the vector subgroup $W(\h)$ of $W(\g')=G'$. As $H$ is defined by a single equation in $G'$, $H\cap G$ is defined by a single equation in $G$ and we have the inequalities
\begin{align*}
\dim(G)-1&\leq\dim(G\cap H)\leq\dim_k(\mathfrak{Lie}(G\cap H))=\dim_k(\g\cap\h)=\dim_k(\g)-1=\dim(G)-1,
\end{align*}
whence $\dim(G\cap H)=\dim_k(\mathfrak{Lie}(G\cap H))$ and therefore $G\cap H$ is smooth. The group $G_1=(G\cap H)^0$ is a smooth connected subgroup of $G$ such that $G/G_1$ is smooth connected of dimension $1$, hence is a form of $\G_a$ (by (\rmnum{1})$\Rightarrow$(\rmnum{3}) of \cref{scheme alg group smooth affine connected unipotent iff}, since $G'$ is unipotent). We then completes the proof of induction on the dimension of $G$.
\end{proof}

Before continuing the proof, we note that the equivalence (\rmnum{1})$\Leftrightarrow$(\rmnum{2}) and \cref{scheme alg group smooth connected 1-dim unipotent is G_a over perfect} leads to the following corollary:

\begin{corollary}\label{scheme alg group smooth connected over perfect unipotent iff extension by G_a}
Let $k$ be a perfect field and $G$ be a smooth connected algebraic group over $k$. Then $G$ is unipotent if and only if it possesses a composition series whose successive quotients are isomorphic to $\G_a$.
\end{corollary}

Now return to the proof of \cref{scheme alg group smooth affine connected unipotent iff}. We note that (\rmnum{1})$\Rightarrow$(\rmnum{4}) by \cref{scheme alg group Hom of unipotent and multiplicative trivial}. On the other hand, assume that (\rmnum{4}) is satisfied. By \cref{scheme alg group admits maximal torus}, $G$ possesses a maximal torus defined over $k$. Now if $r=\dim(T)$, then $({_nT})_{\bar{k}}$ is isomorphic to $(\bm{\mu}_n)^r$, hence $r=0$ and $T$ is trivial. Finally, the implication (\rmnum{5})$\Rightarrow$(\rmnum{1}) can be deduced as in \cref{scheme alg group smooth affine unipotent iff G(k) unipotent}.\par
If $G$ is a subgroup of $\GL_n$, then by \cref{scheme alg group unipotent if and only if representation}, it can be embedded into a subgroup of $\GL_n$ isomorphic to $\mathbb{U}_n$ if $G$ is unipotent, so (\rmnum{1})$\Rightarrow$(\rmnum{6}). On the other hand, if $\g\sub\gl_n$ is formed by nilpotent elements, then by Engle's theorem $\g$ is nilpotent, and any semi-simple endomorphism in $\g$ has zero eigenvalues, hence is zero.\par
Let us prove that (\rmnum{7})$\Rightarrow$(\rmnum{5}), which then completes the proof. Let $T$ be a maximal torus of $G$ (\cref{scheme alg group admits maximal torus}), $\t$ be its Lie algebra. The embedding of $G$ into $\GL_n$ defines a representation of $T$ on $k^n$, which is necessarily semi-simple (\cref{scheme k-group abelian multiplicative iff}). Hence if $x\in\t$, $x$ is a semi-simple endomorphism in $\gl_n$. We immediately see that the map
\[\ad(x):\gl_n\to\gl_n,\quad y\mapsto[x,y]\]
is a semi-simple endomorphism of $\gl_n$, hence of $\g$. As this endomorphism is nilpotent ($\g$ being nilpotent), we conclude that $\ad(x)=0$, hence $x$ is central. But then $\t$ is central and formed by semi-simple endomorphisms, hence is zero by hypothesis. A fortiori, we see that $T$ is trivial.

\begin{remark}\label{scheme (G_a)^r form over perfect field trivial}
If $k$ is perfect, then any $k$-algebraic $G$ which is a form of $(\G_a)^r$ is isomorphic to $(\G_a)^r$. In fact, this is true for $p=0$, and follows from \cref{scheme alg group smooth connected 1-dim unipotent is G_a over perfect} if $r=1$. In the general case, by the principle of finite extension (cf. \cite{EGA4-3} 9.1.4), we can find a finite extension $k'$ of $k$ such that $G_{k'}\cong(\G_a)^r$. But we know that the $k'/k$-forms of $(\G_a)^r$ are classified by $H^1(k'/k,\GL_n)$, which also classifies locally free modules of rank $n$ over $\Spec(k')$. Sicne the latter is obviously trivial, we conclude that $G$ is a tirival form of $(\G_a)^r$, i.e. $G\cong(\G_a)^r$.
\end{remark}

\paragraph{Case of radiciel groups}
\begin{proposition}\label{scheme alg group radiciel unipotent iff}
Let $G$ be a radiciel algebraic group defined over a field $k$ of characteristic $p>0$. Then the following conditions are equivalent:
\begin{enumerate}
    \item[(\rmnum{1})] $G$ is unipotent.
    \item[(\rmnum{2})] $G$ possesses a central composition series whose successive quotients are isomorphic to $\bm{\alpha}_p$. 
    \item[(\rmnum{3})] $G$ possesses a central characteristic composition series whose successive quotients are isomorphic to $(\bm{\alpha}_p)^r$.
    \item[(\rmnum{4})] $G_{\bar{k}}$ does not contain any subgroup isomorphic to $\bm{\mu}_p$.
    \item[(\rmnum{5})] $\g=\mathfrak{Lie}(G)$ is a unipotent Lie $p$-algebra.   
\end{enumerate}
\end{proposition}
The implication (\rmnum{3})$\Rightarrow$(\rmnum{2})$\Rightarrow$(\rmnum{1}) is clear, (\rmnum{1})$\Rightarrow$(\rmnum{3}) is \cref{scheme alg group unipotent over p>0 char composition series}~(\rmnum{1}), and (\rmnum{1})$\Rightarrow$(\rmnum{5}) follows from \cref{scheme alg group Hom of unipotent and multiplicative trivial}. For the rest of the proof, we will need the following lemma on abelian Lie $p$-algebras, whose proof is immediate\footnote{In fact, we can take $\u$ to be the kernel of a suitable power of $x\mapsto x^{[p^n]}$, and $\r$ be its image (cf. \cite{Bourbaki_LieI} \Rmnum{1} \S 1, exercice 23)}:

\begin{lemma}\label{scheme Lie p-algebra abelian reductive unipotent part}
Let $\g$ be an abelian Lie $p$-algebra of finite dimension over a perfect field $k$. Then $\g$ can be written uniquely as a direct sum of a sub-$p$-algebra $\r$ over which the $p$-th power is bijective (called the \textbf{reductive part} of $\g$) and a unipotent sub-$p$-algebra $\u$ (called the unipotent part of $\g$). The formation of $\r$ and $\u$ commutes with base field extension. If $k$ is algebraically closed, then $\r$ admits a basis $e_i$ such that $e_i^{[p]}=e_i$.
\end{lemma}

We now prove the implication (\rmnum{4})$\Rightarrow$(\rmnum{5}). By taking a base change, we may assume that $k$ is algebraically closed. Let $x\in\g$ and $\h$ be the sub-$p$-algebra genrated by $x$ in $\g$. The algebra $\h$ is evidently commutative and its reductive part (\cref{scheme Lie p-algebra abelian reductive unipotent part}) is zero: otherwise $\h$ would contain a nonzero element $y$ such that $y^{[p]}=y$, and therefore (\cite{SGA3-2} \Rmnum{17} App. \Rmnum{2} 2.1 et 2.2) $G$ contains a subgroup isomorphic to $\bm{\mu}_p$, which contradicts the hypothesis. Hence $\h$, and therefore $\g$, is a unipotent Lie $p$-algbebra.\par
Finally, we deal with the implication (\rmnum{5})$\Rightarrow$(\rmnum{1}), which is the most nontrivial part of \cref{scheme alg group radiciel unipotent iff}. For simplicity, we first consider the case where $G$ is of height $1$. As $G$ is radiciel, it is affine, hence isomorphic to a subgroup of $\GL(V)$. This embedding identifies $\g$ with a sub-$p$-algebra of $\gl(V)$, whose $p$-th power is given by the $p$-th power of the endomorphism $x$ (\cite{SGA3-1} $\Rmnum{7}_A$ 6.4.4). As $\g$ is unipotent by hypothesis, it is formed by nilpotent endomorphisms of $V$, so by Engle's theorem is a subalgebra of the Lie algebra $\u$ of the group of strictly upper triangular matrices $\mathbb{U}_n$ relative to a complete flag of $V$. As $G$ is of height $1$, we then deduced from (\cite{SGA3-2} \Rmnum{17} App. \Rmnum{2} 2.2) that $G$ itself is a subgroup of $\mathbb{U}_n$, hence is unipotent.\par
In the general case, we proceed by induction on the height $h$ of $G$. The case $h=1$ has been proved, so suppose that $h>1$, and put $G'={_FG}$, $G''=G/G'$. The group $G'$ is of height $1$ and $\mathfrak{Lie}(G)=\mathfrak{Lie}(G')$ is unipotent, hence $G'$ is unipotent by our induction hypothesis. To show that $G$ is unipotent, it then suffices to prove that $G''$ is unipotent (\cref{scheme alg group unipotent permanence prop}). But $G''$ has height $h-1$, so by the induction hypothesis, it suffices to show that $\mathfrak{Lie}(G'')$ is unipotent. As (\rmnum{4})$\Rightarrow$(\rmnum{5}), we can also show that $G''_{\bar{k}}$ does not contain any subgroup isomorphic to $\bm{\mu}_p$. Now if there exists such a subgroup in $G''_{\bar{k}}$, let $H$ be its inverse image in $G_{\bar{k}}$. As $G'$ is unipotent, the extension is necasseily trivial (we will prove this in (\cite{SGA3-2} \Rmnum{17} \S 5)):
\[\begin{tikzcd}
1\ar[r]&G_{\bar{k}}'\ar[r]&H\ar[r]&\bm{\mu}_p\ar[r]&1
\end{tikzcd}\]
so $\bm{\mu}_p$ can be lifted to $H$. But since $\bm{\mu}_p$ is of height $1$, it is necessarily contained in $G'_{\bar{k}}={_FG_{\bar{k}}}$, whence a contradiction since $G'$ is unipotent.

\paragraph{Case of affine connected groups in charactersitic \texorpdfstring{$p>0$}{P}}
\begin{proposition}\label{scheme alg group affine connected over char p unipotent iff}
Let $G$ be an affine connected algebraic group over a field $k$ of characteristic $p>0$. Then the following conditions are equivalent:
\begin{enumerate}
    \item[(\rmnum{1})] $G$ is unipotent.
    \item[(\rmnum{2})] $G$ possesses a composition series whose successive quotients are isomorphic to $\bm{\alpha}_p$ and $\G_a$ (precesely in this order).
    \item[(\rmnum{3})] $G$ admits a characteristic composition series whose successive quotients are isomorphic to $(\bm{\alpha}_p)^r$ and $(\G_a)^s$ (precesely in this order).
    \item[(\rmnum{4})] $G_{\bar{k}}$ does not contain any subgroup isomorphic to $\bm{\mu}_p$.
    \item[(\rmnum{5})] $\g=\mathfrak{Lie}(G)$ is unipotent.
    \item[(\rmnum{6})] $\g$ is nilpotent, and the reductive part of the center $\z(\g)$ is trivial.
    \item[(\rmnum{6}')] $\g$ is nilpotent, and any subgroup of multiplicative type of the identity component of $Z(G)$ is trivial.
    \item[(\rmnum{6}'')] $G$ is nilpotent, and any subgroup of multiplicative type of the identity component of $Z(G)$ is trivial.
\end{enumerate}
\end{proposition}
It is clear that (\rmnum{3})$\Rightarrow$(\rmnum{2})$\Rightarrow$(\rmnum{1}). To establish (\rmnum{1})$\Rightarrow$(\rmnum{3}), we shall use the following lemma:
\begin{lemma}\label{scheme over radial of height n restriction by Frobenius}
Let $k$ be a field of characteristic $p>0$, $n\in\N$, and $k'$ be a radiciel extension of $k$ such that $(k')^{p^n}$ is contained in $k$. For any $k$-scheme $X$ (resp. any $k'$-scheme $X'$), denote by $X^{(p^n)}$ (resp. $X'_\varphi$) the $k$-scheme deduced from $X$ (resp. $X'$) by the base change
\[F^n:k\to k,\quad x\mapsto x^{p^n},\quad(\text{resp.}\quad\varphi:k'\to k,\quad x'\mapsto x'^{p^n}).\]
Then for any $k$-scheme $X$, there exists a functorial isomorphicm
\[(X_{k'})_\varphi\stackrel{\sim}{\to} X^{(p^n)}.\]
Therefore, if $X$ and $Y$ are $k$-schemes such that there exists a $k'$-isomorphism $u':X_{k'}\stackrel{\sim}{\to} Y_{k'}$, then there exists an $k$-isomorphicm $u:X^{(p^n)}\stackrel{\sim}{\to}Y^{(p^n)}$. If $X$ and $Y$ are also endowed with group structures and $u'$ is a group homomorphism, then $u$ is a group homomorphism. 
\end{lemma}
\begin{proof}
This follows simply from the transitivity of base change and the fact that the composition morphism $k\to k'\stackrel{\varphi}{\to}k$ is equal to $F^n$.
\end{proof}

We now prove (\rmnum{1})$\Rightarrow$(\rmnum{3}). We proceed by induction on $\dim(G)$. If $\dim(G)=0$, as $G$ is connected, it is radiciel and we can apply \cref{scheme alg group radiciel unipotent iff}(\rmnum{3}). If $\dim(G)>0$, then there exists an integer $m\geq 0$ such that the quotient $G/{_{F^m}G}$ is smooth (\cite{SGA3-1} $\Rmnum{7}_A$ 8.3), which is evidently connected and nonzero. Applying \cref{scheme alg group smooth affine connected unipotent iff}~(\rmnum{3}), we see that there exists a connected and characteristic subgroup $G'$ of $G$ such that $G''=G/G'$ is a form of $(\G_a)^r$ ($r>0$). By \cref{scheme (G_a)^r form over perfect field trivial}, if $K$ is a perfect closure of $k$, we have $G_K''\cong(\G_{a,K})^r$. As $G''$ is of finite type over $k$, there then exists a finite radiciel extension $k'$ of $k$ such that $G''_{k'}\cong(\G_{a,k'})^r$ (\cite{SGA3-2} $\Rmnum{6}_B$, 10.2). Let $n>0$ be such that $(k')^{p^n}\sub k$, then with the notations of \cref{scheme over radial of height n restriction by Frobenius}, we deduce an isomorphism of algebraic $k$-groups
\[(\G_{a,k})^r=(\G_{a,k'})^r_\varphi\stackrel{\sim}{\to}(G'')^{(p^n)}.\]
Consider then the relative Frobenius homomorphism of $G''$ (\cite{SGA3-1} $\Rmnum{7}_A$ \S 4)
\[F^n:G''\to(G'')^{(p^n)}.\]
As $G''$ is smooth over $k$, $F^n$ is an epimorphism for the fpqc topology (\cite{SGA3-1} $\Rmnum{7}_A$ 8.3.1), so that $(G'')^{(p^n)}$ is identified with $G''/{_{F^n}G''}$. We have shown that $G''/{_{F^n}G''}$ is isomorphic to $(\G_a)^r$. The inverse image $G_n'$ of ${_{F^n}G''}$ in $G$ is then a subgroup of $G$, connected, characteristic, of dimension strictly lower than that of $G$, to which we can apply the induction hypothesis.\par
We note that (\rmnum{1})$\Rightarrow$(\rmnum{4}) by \cref{scheme alg group Hom of unipotent and multiplicative trivial}. Conversely, assume (\rmnum{4}), and consider $G$ as the extension of a smooth connected group $G''$ by a radiciel group $G'$ (\cite{SGA3-2} \Rmnum{17} App. \Rmnum{2} 3.1). The group $G'$ is unipotent by \cref{scheme alg group radiciel unipotent iff}~(\rmnum{4}), so it suffices to see that $G''$ is unipotent and for this we only need to show that $G''_{\bar{k}}$ does not contain any subgroup isomorphic to $\bm{\mu}_p$ (\cref{scheme alg group smooth affine connected unipotent iff}~(\rmnum{4})). Now if $G''_{\bar{k}}$ contains a subgroup $\bm{\mu}_p$, then it can be lifted into $G_{\bar{k}}$ by the result of (\cite{SGA3-2} \Rmnum{17} 5.1), whence a contradiction with (\rmnum{4}).\par
We have seen (\rmnum{1})$\Rightarrow$(\rmnum{5}) in \cref{scheme alg group unipotent is nilpotent Lie algebra}, and (\rmnum{5})$\Rightarrow$(\rmnum{6}). In fact, as $\ad(x)^{p^r}=\ad(x^{[p^r]})$ (\cite{SGA3-1} $\Rmnum{7}_A$ 5.2), $\ad(x)$ is nilpotent if $\g$ is unipotent, and hence $\g$ is nilpotent by Engel's theorem. Moreover, if $\g$ is unipotent, so is its center and the reductive center of $\z(\g)$ is then trivial. On the other hand, (\rmnum{6})$\Rightarrow$(\rmnum{4}) because if $G_{\bar{k}}$ contains a subgroup isomorphic to $\bm{\mu}_p$, then there exists a nonzero element $x\in\g$ such that $x^{[p]}=x$ (\cite{SGA3-2} \Rmnum{17} App. \Rmnum{2} 2.1), hence $x^{[p^r]}=x$ for any $r>0$. As $\ad(x)$ is nilpotent ($\g$ being nilpotent) and $\ad(x)^{p^r}=\ad(x^{[p^r]})$, we have necessarily $\ad(x)=0$, hence $x$ belongs to the reductive part of the center $\z(\g)$, which contradicts with (\rmnum{6}).\par
It remains to prove the equivalence of (\rmnum{1}) and (\rmnum{6}), (\rmnum{6}'), (\rmnum{6}''). We first note that (\rmnum{1})$\Rightarrow$(\rmnum{6}'') by \cref{scheme alg group Hom of unipotent and multiplicative trivial} and \cref{scheme alg group unipotent if and only if representation}~(\rmnum{2}), and (\rmnum{6}'')$\Rightarrow$(\rmnum{6}') because if $G$ is nilpotent, then so is the subgroup ${_FG}$, and it follows from (\cite{SGA3-2} \Rmnum{17} App. \Rmnum{2} 2.2) that ${_FG}$ is nilpotent if and only if $\mathfrak{Lie}({_FG})=\mathfrak{Lie}(G)$ is nilpotent.\par
Finally, we prove that (\rmnum{6}')$\Rightarrow$(\rmnum{6}). Let $Z$ be the identity component of $Z(G)$ and $\r$ be the reductive part of the center $\z(\g)$ of $\g$; we show that $\r=0$. Now it is immediate that $\r$ is a characteristic sub-$p$-algebra of $\g=\mathfrak{Lie}({_FG})$, so $\r$ is the Lie algebra of a characteristic subgroup $R$ of ${_FG}$ (\cite{SGA3-2} \Rmnum{17} App. \Rmnum{2} 2.2). On the other hand, it follows from the last assertion of \cref{scheme Lie p-algebra abelian reductive unipotent part} and (\cite{SGA3-2} \Rmnum{17} App. \Rmnum{2} 2.2) that $R$ is of the form $(\bm{\mu}_p)^r$. As ${_FG}$ is also characteristic in $G$ (\cite{SGA3-2} \Rmnum{17} App. \Rmnum{2} 1), $R$ is normal in $G$, hence is central ($G$ being connected, cf. \cref{scheme group multiplicative normal subgroup is central}). By the hypothesis (\rmnum{6}'), $R$ is therefore trivial, and hence so is $\r$.

\paragraph{Case of finite \'etale groups}
\begin{proposition}\label{scheme alg group finite etale unipotent iff}
Let $G$ be a finite \'etale group over a field $k$. Then $G$ is unipotent if and only if for any integer $q$ coprime to $p=\char(k)$, $G_{\bar{k}}$ does not contain subgroups isomorphic to $\bm{\mu}_q$.
\end{proposition}
\begin{proof}
In view of the definitions, we may assume that $k$ is algebraically closed, so $G\cong M_k$ is constant, where $M$ is an ordinary finite group. By \cref{scheme alg group constant unipotent iff}, we know that $G$ is unipotent if and only if $M$ is a $p$-group, which by Sylow's theorem is equivalent to saying that $M$ does not contain any element $x$ of order coprime to $p$. Since $\bm{\mu}_q\cong(\Z/q\Z)_k$ over an algebraically closed field $k$, we then conclude the assertion.
\end{proof}

\paragraph{Case of abelian varieties}
\begin{proposition}\label{scheme abelian variety unipotent iff}
Let $G$ be an abelian variety defined over a field $k$. Then the following conditions are equivalent:
\begin{enumerate}
    \item[(\rmnum{1})] $G$ is unipotent
    \item[(\rmnum{2})] $G$ is the trivial $k$-group.
    \item[(\rmnum{3})] There exists an integer $n$, coprime to $p=\char(k)$, such that $G_{\bar{k}}$ does not contain subgroups isomorphic to $\bm{\mu}_n$.
\end{enumerate}
\end{proposition}
\begin{proof}
We may assume that $k$ is algebraically closed. Then if $G$ is an abelian variety of dimension $d$, we see that (\cite{Lang_AV} \Rmnum{4} \S 3, th.6) that the group ${_nG}(k)$ ($n$ coprime to $p$) is isomorphic to $(\Z/n\Z)^{2d}$, hence to $(\bm{\mu}_n)^{2d}$. This proves (\rmnum{3})$\Rightarrow$(\rmnum{2}), and (\rmnum{2})$\Rightarrow$(\rmnum{1})$\Rightarrow$(\rmnum{3}) is evident.
\end{proof}

\paragraph{The general case}
If $G$ and $H$ are algebraic groups defined over an algebraically closed field $k$, we denote by $\mathcal{P}(G,H)$ the following property: there does not exist subgroups of $G$ isomorphic to $H$. We then obtain the following characterization of unipotent algebraic groups:

\begin{theorem}\label{scheme alg group over ac field unipotent iff no subgroup}
Let $G$ be an algebraic group defined over an algebraically closed field $k$ of characteristic $p$.
\begin{enumerate}
    \item[(\rmnum{1})] If $G$ is smooth connected, then $G$ is unipotent if and only if there exists an integer $n$ coprime to $p$ such that $\mathcal{P}(G,\bm{\mu}_n)$ is valid.
    \item[(\rmnum{2})] If $G$ is smooth, then $G$ is unipotent if and only if for any integer $n$ coprime to $p$, $\mathcal{P}(G,\bm{\mu}_n)$ is valid.
    \item[(\rmnum{3})] If $G$ is connected and $p>0$, then $G$ is unipotent if and only if there exists an integer $n$ coprime to $p$ such that $\mathcal{P}(G,\bm{\mu}_n)$ and $\mathcal{P}(G,\bm{\mu}_p)$ are valid.
    \item[(\rmnum{4})] If $G$ is an arbitrary algebraic group, then $G$ is unipotent if and only if for any integer $n$ coprime to $p$, $P(G,\bm{\mu}_n)$ is valid.
\end{enumerate}
\end{theorem}
\begin{proof}
Let $G$ be an algebraic group. If $G$ is unipotent, then $\mathcal{P}(G,\bm{\mu}_n)$ is valid for any $n>1$ (\cref{scheme alg group Hom of unipotent and multiplicative trivial}). For the converse, let $G^0$ be the identity component of $G$ and assume that there exists an integer $n$ such that $\mathcal{P}(G,\bm{\mu}_n)$ is valid. If $G^0$ is smooth (i.e. $G$ is smooth), then it follows from Chevalley's structural theorem that $G^0$ is an extension of an abelian variety $A$ by a smooth connected affine group $L$. Since a fortiori $\mathcal{P}(L,\bm{\mu}_n)$ is satisfied, we see that $L$ is unipotent by \cref{scheme alg group affine connected over char p unipotent iff}. Now if $A$ is nonzero, then there exists a subgroup of $A$ isomorphic to $\bm{\mu}_n$ (\cref{scheme abelian variety unipotent iff}). As $L$ is unipotent, by (\cite{SGA3-1} \Rmnum{17} 5.1), this subgroup can be lifted into $G$, so $\mathcal{P}(G,\bm{\mu}_q)$ is not valid. We then have $A=0$, and $G^0=L$ is therefore unipotent. This already proves assertion (\rmnum{1}), and that $G^0$ is unipotent under the condition of (\rmnum{2}). To prove (\rmnum{2}), we note that if $G/G^0$ is a finite unipotent group. If it is not unipotent, then there exists an integer $n$ coprime to $p$ and a subgroup of $G/G^0$ isomorphic to $\bm{\mu}_n$ (\cref{scheme alg group finite etale unipotent iff}). As $G^0$ is unipotent, by (\cite{SGA3-1} \Rmnum{17} 5.1) this group can be lifted to a subgroup of $G$, which contradicts the hypothesis in (\rmnum{2}).\par
If $G$ is not smooth (so that $p>0$), there exists an integer $n>0$ such that $G''=G^0/{_{F^n}G}$ is smooth (\cite{SGA3-2} \Rmnum{17} App. 3.1). Then $G''$ is the extension of an abelian variety $A$ by a smooth connected affine group $L''$. Let $L$ be the inverse image of $L''$ in $G^0$, which is still affine and connected since ${_{F^n}G}$ is radiciel. We then obtain a composition series
\[0\sub L\sub G^0\sub G\]
where $L$ is affine connected, $G^0/L=A$ is an abelian variety, and $G/G^0$ is an \'etale group. If $\mathcal{P}(G,\bm{\mu}_n)$ is satisfied, a fortiori $\mathcal{P}(L,\bm{\mu}_p)$ is satisfied, so $L$ is unipotent by \cref{scheme alg group affine connected over char p unipotent iff}~(\rmnum{4}). By the same reasoning, we see that $A=0$ if $\mathcal{P}(G,\bm{\mu}_n)$ is satisfied for some integer $n$ coprime to $p$, so $G^0$ is unipotent in this case. This proves (\rmnum{3}), and for the general case (\rmnum{4}), we can apply the same reasoning as above, using \cref{scheme alg group finite etale unipotent iff} and (\cite{SGA3-1} \Rmnum{17} 5.1).
\end{proof}

\chapter{The theory of reductive group schemes}
\section{Generalities of reductive groups}
\subsection{Reminders of algebraic groups over an algerbaically closed field}
In this subsection, $k$ denotes an algebraically closed field. Let $G$ be an affine smooth connected $k$-group. The \textit{radical} of $G$ (\cite{Chevalley1958} \S 9.4, prop.2) is the reduced subgroup associated with the identity component of the intersection of Borel subgroups of $G$; it is also the largest smooth connected solvable normal subgroup of $G$; we will denote it by $\rad(G)$. The unipotent radical, on the other hand, is the unipotent part of the radical of $G$; this is also the largest smooth connected unipotent normal subgroup of $G$, we denote it by $\rad^u(G)$.\par
Now let $T$ be a torus of $G$. Then the centralizer $Z_G(T)$ of $T$ in $G$ is a smooth connected closed subgroup of $G$ (\cref{scheme group affine smooth Weyl group representable} and \cite{Chevalley1958} \S 6.6 th.6 (a) or \cite{Chevalley_Classification} \S 6.7, th.7 (a)). We have a fundamental relation
\begin{equation}\label{scheme k-group unipotent radical of centralizer of torus char}
\rad^u(Z_G(T))=\rad^u(G)\cap Z_G(T).
\end{equation}
First, $U=\rad^u(G)\cap Z_G(T)$ is a unipotent normal subgroup of $Z_G(T)$. If we act $T$ on $\rad^u(G)$ by inner automorphisms, then $U$ is none other than the invariant subscheme of this action. It then follows from following lemma that $U$ is smooth and connected:
\begin{lemma}\label{scheme torus act on sp smooth gorup invariant representable}
Let $S$ be a scheme, $T$ be an $S$-torus acting on a separated and smooth $S$-group $H$.
\begin{enumerate}
    \item[(a)] The functor of invariants $H^T$ is representable by a closed subscheme of $H$, which is of finite presentation over $H$ and smooth over $S$.
    \item[(b)] If $H$ is affine over $S$ and has connected fibers, then so is $H^T$.
\end{enumerate}
\end{lemma}
\begin{proof}

\end{proof}
Therefore, $U$ is a closed subgroup of $\rad^u(Z_G(T))$. On the other hand, by (\cite{Chevalley1958} 12, \S 3, cor. au th.1), we have $\rad^u(Z_G(T))(k)\sub \rad^u(G)(k)$, so the equality \cref{scheme k-group unipotent radical of centralizer of torus char} follows.\par

We note a particular case of the preceding result: if $G$ is an affine smooth connected $k$-group and $T$ is a maximal torus of $G$, then
\begin{equation}\label{scheme k-group centralizer of maximal torus decomposition}
Z_G(T)=T\cdot(Z_G(T)\cap \rad^u(G)).
\end{equation}
In fact, by (\cite{Chevalley1958} \S 6.6 th.6 (c) or \cite{Chevalley_Classification} \S 6.7, th.7 (c)), $Z_G(T)$ is a smooth connected solvable subgroup, hence a semi-direct product of its maximal tori and its unipotent radical.\par
By the density theorem (\cite{Chevalley1958} \S 6.5 th.5 (a) or \cite{Chevalley_Classification} \S 6.6, th.6 (a)), the union of $Z_G(T)$, for $T$ runs through maximal tori of $G$, contains an open dense subset of $G$. We therefore conclude that:

\begin{corollary}\label{scheme k-group smooth affine union of Trad_u open dense}
If $G$ is an affine smooth connected $k$-group, the union of $T\cdot \rad^u(G)$, where $T$ runs through maximal tori of $G$, contains an open dense subset of $G$.
\end{corollary}

We recall that the \textbf{reductive rank} of an affine smooth $k$-group $G$ is defined to be the dimension of the maximal tori of $G$, denoted by $\rho_r(G/k)$ or simply $\rho_r(G)$. For $\rho_r(G/k)=0$, it is necessary and sufficient that $G$ is unipotent, i.e. $G=\rad^u(G)$ (cf. \cite{Chevalley1958} \S 6.4 cor.1 du th.4 or \cite{Chevalley_Classification} \S 6.5 cor.1 au th.5).\par
If $H$ is a normal subgroup of $G$, the quotient $G/H$ is affine and smooth (cf. \cite{SGA3-1} $\Rmnum{6}_B$ 11.17 et \cite{EGA4-4} 17.7.7). Moreover, by (\cite{Chevalley1958} \S 7.3, th.3 (a) et (c)), we have
\[\rho_r(G)=\rho_r(G/H)+\rho_r(H).\] 
The $k$-group $G$ is called \textbf{reductive} if it is affine smooth connected and $\rad(G)$ is a torus, that is, if $\rad^u(G)=\{e\}$.

\begin{proposition}\label{scheme alg group reductive prop}
Let $G$ be a reductive algebraic group over $k$.
\begin{enumerate}
    \item[(a)] If $T$ is a torus of $G$, then $Z_G(T)$ is reductive.
    \item[(b)] In particular, if $T$ is a maximal torus of $G$, then $Z_G(T)=T$.
    \item[(c)] The center of $G$ is contained in any maximal torus of $G$, hence is diagonalizable.
    \item[(d)] The radical of $G$ is the unique maximal torus of $Z(G)$.  
\end{enumerate}
\end{proposition}
\begin{proof}
In fact, (a) follows from (\ref{scheme k-group unipotent radical of centralizer of torus char}), (b) follows from (\ref{scheme k-group centralizer of maximal torus decomposition}), and (c) follows from (b) as $T$ is diagonalizable\footnote{Any maximal torus $T$ splits over a finite separable extension of $k$, and since $k$ is algebraically closed, it therefore splits over $k$.}. Finally, the maximal torus of $Z(G)$ (that is, the identity component $Z(G)^0$) is a smooth connected solvabel normal subgroup of $G$, hence is contained in $\rad(G)$. Conversely, as $G$ is reductive, $\rad(G)$ is a normal torus in $G$, hence central (\cite{Chevalley1958} \S 4.3, cor. a la prop.2), whence (d).
\end{proof}

\begin{proposition}\label{scheme subgroup smooth connected radical equation}
Let $G$ be an affine smooth connected algebraic group over $k$ and $H$ be a smooth connected normal subgroup of $G$. Then we have
\begin{equation}\label{scheme subgroup smooth connected radical equation-1}
\rad(H)=(\rad(G)\cap H)_\red^0,\quad \rad^u(H)=(\rad^u(G)\cap H)_\red^0
\end{equation}
In particular, if $G$ is reductive, so is $H$.
\end{proposition}
\begin{proof}
In fact, as $(\rad(G)\cap H)_\red^0$ is a smooth connected normal subgroup of $H$, it is obviously contained in the radical of $H$. Conversely, if $K\sub H$ is a smooth connected normal subgroup, then $K\sub \rad(G)$, so $K\sub \rad(G)\cap H$. As $K$ is smooth and connected, we see that $K$ is in fact a subgroup of $(\rad(G)\cap H)_\red^0$, whence the first assertion. The second one can be proved similarly.
\end{proof}

\begin{proposition}\label{scheme group affine smooth unipotent radical under fp morphism}
Let $f:G\to G'$ be a faithfully flat morphism of affine smooth connected $k$-groups. Then
\begin{equation}\label{scheme group affine smooth unipotent radical under fp morphism-1}
f(\rad^u(G))=\rad^u(G').
\end{equation}
In particular, if $G$ is reductive, so is $G'$.
\end{proposition}
\begin{proof}
First, $f$ sends $\rad^u(G)$ into $\rad^u(G)$, as this image is connected smooth (since $f$ is faithfully flat, cf. \cite{EGA4-4} 17.7.7) and normal (\cref{site sheaf quotient by normal subgroup correspond}). We introduce $H=(f^{-1}(\rad^u(G')))_\red^0$, which contains $\rad^u(G)$. We have $\rad^u(H)=\rad^u(G)$, and we are reduced to the case where $G=H$, i.e. where $G'$ is unipotent. As the union of the $T\cdot \rad^u(G)$ ($T$ runs through maximal tori of $G$) is dense in $G$, the union of the $f(T)f(\rad^u(G))$ is dense in $G'$. But $f(T)$ consists of semi-simple elements, so $f(T)=\{e\}$, $G'$ being unipotent. This proves that $f(\rad^u(G))$ is dense in $G'$. Hence, by (\cite{Chevalley1958} \S 5.4 lemme 4 or \cite{Chevalley_Classification} \S 6.1 lemme 1), $f(\rad^u(G))$ is an open subgroup of $G'$. As it is connected, it follows that $f(\rad^u(G))=G'$ (\cref{scheme A-group product of open dense is G}).
\end{proof}

A $k$-group $G$ is called \textbf{semi-simple} if it is affine smooth connected and $\rad(G)=\{e\}$. If $G$ is an affine smooth connected $k$-group, then $G/\rad(G)$ is semi-simple (\cite{Chevalley1958} \S 9.4 prop.2), and $G/\rad^u(G)$ is reductive. We define the \textbf{semi-simple rank} of $G$, denoted by $\rho_s(G)$, to be the reductive rank of $G/\rad(G)$. If $G$ is reductive, we then have
\begin{equation}\label{scheme alg group rho_r rho_s and radical dimension}
\rho_r(G)=\rho_s(G)+\dim(\rad(G)).
\end{equation}

If $G$ is an affine smooth connected group over $k$ and $Q$ is a central subtorus of $G$, then $G/Q$ is semi-simple if and only if $G$ is reductive and $Q=\rad(G)$. In fact, one direction is trivial; conversely, if $G/Q$ is semi-simple, then $Q\sups \rad(G)$, so $\rad^u(G)$ is trivial and $G$ is then reductive. By \cref{scheme alg group reductive prop}~(d), $\rad(G)$ is the maximal torus of $Z(G)$, so $\rad(G)=Q$. Moreover, in this case, ($G/Q$ is reductive) and we have 
\begin{equation}\label{scheme alg group rho_s invariant under quotient by central torus}
\rho_s(G)=\rho_s(G/Q).
\end{equation}

If $k'$ is an algerbaically closed extension of $k$, then $G$ is reductive (resp. semi-simple) if and only if $G_{k'}$ is, and we have
\[\rho_r(G/k)=\rho_r(G_{k'}/k'),\quad \rho_s(G/k)=\rho_s(G_{k'}/k').\]

Let $G$ be a smooth connected $k$-group and $T$ be a torus of $G$. We denote by $\g$ and $\t$ the Lie algebras of $G$ and $T$, respectively. Then $\g$ decomposes under the adjoint action of $T$ into
\begin{equation}\label{scheme group smooth acted by torus Lie algebra decomposition-1}
\g=\g^0\oplus\bigoplus_{\alpha\in R}\g^\alpha,
\end{equation}
where $\alpha\in R$ are nontrivial characters of $T$ such that $\g^\alpha\neq 0$. These characters $\alpha\in R$ are called roots of $G$ relative to $T$. By \cref{scheme group normalizer and centralizer Lie prop}~(\rmnum{2}), we have
\begin{equation}\label{scheme group smooth acted by torus Lie algebra decomposition-2}
\g^0=\mathfrak{Lie}(Z_G(T)).
\end{equation}
In particular, as $Z_G(T)$ is connected (\cite{SGA3-2} \Rmnum{12} 6.6 (b)), we see that $T$ is self-centralizing if and only if $\g^0=\t$. This condition implies that $T$ is maximal and that $Z(G)\sub T$, hence by (\cite{SGA3-2} \Rmnum{12} 8.8 (d)) we have
\begin{equation}\label{scheme group smooth acted by torus Z(G) intersection of ker}
Z(G)=\ker(\Ad:T\to\GL(\g))=\bigcap_{\alpha\in R}\ker\alpha.
\end{equation}
Therefore $Z(G)$ is affine, hence the morphism $G\to G/Z(G)$ is affine. As the latter is affine by \cref{scheme alg group center is closed and quotient affine}, we see that $G$ is also affine.\par
If $G$ is reductive and $T$ is maximal, then the roots in the preceding sense coincide with those in (\cite{Chevalley1958} \S 12.2), and there are $\dim(G)-\dim(T)$ roots (\cite{Chevalley1958} 13.4 cor.2 au th.3). Moreover, if $\alpha$ is a root, then $-\alpha$ is also a root (\cite{Chevalley1958} 12.2 cor. a la prop.1). In particular, for $G$ reductive, the integer
\begin{equation}\label{scheme group reductive dim(G)-rho_r(G) is Card(R)}
\dim(G)-\rho_r(G)=\Card(R)
\end{equation}
is always even. Finally, the semi-simple rank of $G$ is equal to the dimension of the $\Q$-vector space $R\otimes_{\Z}\Q$.

\begin{lemma}\label{scheme alg group smooth reductive with ss-rank 1 iff}
Let $k$ be an algebraically closed field, $G$ be an affine smooth connected algebraic group over $k$, $T$ be a torus of $G$, $W(T)=N_G(T)/Z_G(T)$ be the Weyl group of $G$ relative to $T$. Then the following conditions are equivalent:
\begin{enumerate}
    \item[(\rmnum{1})] $G$ is reductive, $T$ is maximal and $\rho_s(G)=1$.
    \item[(\rmnum{2})] $G$ is reductive, $T$ is maximal, $G\neq T$, and there exists a sub-torus $Q$ of $T$ of codimension $1$ in $T$ which is central in $G$.
    \item[(\rmnum{3})] $G$ is not solvable and $\dim(G)-\dim(T)\leq 2$.
    \item[(\rmnum{4})] $W(T)\neq\{e\}$ and $\dim(G)-\dim(T)\leq 2$.
    \item[(\rmnum{5})] $W(T)=(\Z/2\Z)_k$ and $\dim(G)-\dim(T)=2$.    
\end{enumerate}
\end{lemma}

\begin{example}
Let $G=\GL_2$ be the general linear group of order $2$ defined over an algebraically closed field, and $T\sub G$ be the subgroup of diagonal matrices, which is a mximal torus of $G$. Then we see that $Z_G(T)=T$ and $N_G(T)\cong T\rtimes(\Z/2\Z)$ is isomorphic to the group of monomial matrices of order $2$ (cf. \cref{scheme alg group GL_2 regular element example}). Therefore $G$ is reductive, and as its center is equal to the subgroup of scalar matrices $D$, we conclude from \cref{scheme alg group reductive prop} that $\rad(G)=D$. In particular, the semi-simple rank of $G$ is $1$, so the conditions of \cref{scheme alg group smooth reductive with ss-rank 1 iff} is satisfied. 
\end{example}

\begin{proposition}\label{scheme alg group smooth connected acted by torus reductive iff}
Let $k$ be an algebraically closed field, $G$ be a smooth connected algebraic group over $k$, $T$ be a torus of $G$, $R$ be the set of roots of $G$ relative to $T$, and
\[\g=\g^0\oplus\bigoplus_{\alpha\in R}\g^\alpha\]
be the decomposition of $\g$ under $T$. For each $\alpha\in R$, let $T_\alpha$ be the maximal torus of $\ker\alpha$ and $G_\alpha=Z_G(T_\alpha)$. Then the following conditions are equivalent:
\begin{enumerate}
    \item[(\rmnum{1})] $G$ is affine reductive, and $T$ is maximal.
    \item[(\rmnum{2})] $\g^0=\t$ and each $G_\alpha$ ($\alpha\in R$) is reductive.
    \item[(\rmnum{3})] $\g^0=\t$, each $\g^\alpha$ ($\alpha\in R$) is of dimension $1$, and if $\alpha,\beta\in R$ is such that $\beta=q\alpha$ with $q\in\Q$, then $q=\pm 1$. Moreover, for each $\alpha\in R$, there exists $w_\alpha\in G(k)$ which normalizes $T$, centralizes $T_\alpha$, but does not centralizes $T$.
\end{enumerate}
Moreover, under these conditions, each $G_\alpha$ is of semi-simple rank $1$ and we have $\mathfrak{Lie}(G_\alpha)=\t\oplus\g^\alpha\oplus\g^{-\alpha}$.
\end{proposition}

\begin{corollary}\label{scheme alg group smooth connected acted by torus semi-simple iff}
With the notations of \cref{scheme alg group smooth connected acted by torus reductive iff}, the following conditions are equivalent:
\begin{enumerate}
    \item[(\rmnum{1})] $G$ is affine semi-simple and $T$ is maximal.
    \item[(\rmnum{2})] $\g^0=\t$, each $G_\alpha$ is reductive, and $\bigcap_{\alpha\in R}\ker\alpha$ is finite.
\end{enumerate}
\end{corollary}
\begin{proof}
In fact, as $\rad(G)$ is the maximal torus of $Z(G)$ if $G$ is reductive and $Z(G)$ is diagonalizable (\cref{scheme alg group reductive prop}), we see that $\rad(G)$ is trivial if and only if $Z(G)$ is finite, and the corollary then follows from the equality (\ref{scheme group smooth acted by torus Z(G) intersection of ker}) and \cref{scheme alg group smooth connected acted by torus reductive iff}.
\end{proof}

\subsection{Reductive group schemes}
In this subsection we extend the definition of reductive and semi-simple algebraic groups to group schemes over an arbitrary base. Let $S$ be a scheme and $G$ be a group scheme over $S$. We first note that the following properties are equivalent:
\begin{enumerate}
    \item[(\rmnum{1})] $G$ is affine and smooth over $S$, with connected fibers.
    \item[(\rmnum{2})] $G$ is affine and flat over $S$, of finite presentation over $S$, and has geometrically integral fibers.
\end{enumerate}
In fact, if (\rmnum{1}) is verified, then as $G$ is affine and smooth over $S$, it is of finite presentation over $S$; and as its fibers are smooth and connected, they are geometrically integral by \cref{scheme alg group identity component prop}. Conversely, if (\rmnum{2}) is verified, then the fibers of $G$ are geometrically reduced and connected, hence smooth, so $G$ is smooth over $S$. Of course, the above conditions are stable under base change, and local for the Zariski topology. Therefore, for $G$ to have the indicated properties, it suffices to verify that for $G_{S'}\to S'$, where $S'\to S$ is a faithfully flat and quasi-compact morphism.\par
Let $G$ be an $S$-group satisfying these properties, and $T$ be a torus of $G$. Then by (\cite{SGA3-2} \Rmnum{11}, 6.11 (a)) and \cref{scheme multiplicative to smooth transporter formally smooth}, $Z_G(T)$ is representable by a closed subscheme of $G$ (hence affine over $S$), of finite presentation and smooth over $S$. Moreover, as each geometric fiber $G_{\bar{s}}$ of $G$ is an affine smooth connected $\kappa(\bar{s})$-group, the centralizer of $T_{\bar{s}}$ in $G_{\bar{s}}$ is representable, and we have
\[Z_{G_{\bar{s}}}(T_{\bar{s}})=(Z_G(T))_{\bar{s}}.\]

\begin{lemma}\label{scheme group smooth affine locus of reductive rho_s=1 is open}
Let $G$ be an affine smooth group scheme over $S$, with connected fibers, $T$ be a torus of $G$. The set of $s\in S$ such that $G_{\bar{s}}$ is a reductive $\kappa(\bar{s})$-group of semi-simple rank $1$ and that $T_{\bar{s}}$ is maximal, is an open subset $U$ of $S$.
\end{lemma}
\begin{proof}
As $G$ and $T$ are smooth over $S$, the function
\[s\mapsto\dim(G_{\bar{s}})-\dim(T_{\bar{s}})=\dim(G_s)-\dim(T_s)\]
is locally constant over $S$ (\cite{EGA4-4} 17.10.2); let $U_1$ be the open subset of $s\in S$ such that it is equal to $2$. By \cref{scheme group smooth fp centralizer and normalizer of subtorus prop}, the Weyl group $W_G(T)=N_G(T)/Z_G(T)$ is representable by a \'etale and separated $S$-scheme, and the function
\[s\mapsto\Card(W_G(T)_{\bar{s}})\]
is lower semi-continuous. Let $U_2$ be the points of $s\in S$ where this function is $>1$, which is open. By \cref{scheme alg group smooth reductive with ss-rank 1 iff}, the set of $s\in S$ such that $G_{\bar{s}}$ is reductive of semi-simple rank $1$ and $T_{\bar{s}}$ is maximal is equal to $U=U_1\cap U_2$, and for any $s\in U$, $W_G(T)_{\bar{s}}$ has exactly two points.
\end{proof}

\begin{remark}\label{scheme group reductive locus Weyl group Z/2Z prop}
From the proof of \cref{scheme group affine smooth Weyl group fiber rank semicontinuous}, we see that $W_G(T)_U$ is \'etale and finite over $U$. It is in fact isomorphic to $(\Z/2\Z)_U$. To see this, as the functor of automorphisms of $(\Z/2\Z)_U$ is trivial, it suffices to verify this assertion locally. Now the hypothesis implies that the algebra $\mathscr{A}$ defining $W_G(T)_U$ is a locally free module of rank $2$, and as the augmentation ideal $\mathscr{I}$ is a direct factor of $\mathscr{A}$, by replacing $U$ with a small affine open $\Spec(R)$, we can suppose that $I=\Gamma(U,\mathscr{I})$ is a free $R$-module of rank $1$. If $e$ is a generator of $I$, then we have $e^2=ae$ for some $a\in R$, and by \cref{scheme group affine of order 2 char}, there exists $c\in R$ such that $ac=2$. We can then define a morphism of $R$-modules
\[R[\Z/2\Z]\cong R^2\to A,\quad (\lambda,\mu)\mapsto \lambda+\mu(1-ce).\]
Since $(1-ce)^2=1-2ce+c^2e^2=1$, we easily see that this is a homomorphism of rings, whence our assertion.
\end{remark}

Let $S$ be a scheme, $G$ be an $S$-group, $e:S\to G$ be the unit section of $G$. Recall that (\cref{scheme tangent bundle representable if}) the functor $\mathfrak{Lie}(G/S)$ is representable by the vector bundle $\V(\omega_{G/S}^1)$ (where $\omega_{G/S}^1=e^*(\Omega_{G/S}^1)$), and we denote by
\[\g=\sLie(G/S)=(\omega_{G/S}^1)^\vee\]
the sheaf of sections of this vector bundle. Suppose that $G$ is smooth over $S$, then $\omega_{G/S}^1$, and hence $\sLie(G/S)$ are locally free $\mathscr{O}_S$-modules of finite type (cf. \cref{scheme group omega_G/S differential module prop}), and we have (\cref{scheme group omega_G/S locally free construct Lie})
\[\mathfrak{Lie}(G/S)=\mathbf{W}(\sLie(G/S)),\]
that is, for any $S$-scheme $S'$,
\[\mathfrak{Lie}(G/S)(S')=\Gamma(S',\sLie(G/S)\otimes_{\mathscr{O}_S}\mathscr{O}_{S'}).\]
By \cref{scheme group condition (E) Ad is linear representation}, the adjoint action of $G$ endows $\mathfrak{Lie}(G/S)=\mathbf{W}(\sLie(G/S))$ a structure of $\mathbb{O}_S[G]$-module, so $\sLie(G/S)$ is an $\mathscr{O}_S[G]$-module. If $G$ is also affine over $S$, then this equivalent to saying that $\sLie(G/S)$ is an $\mathscr{A}(G)$-comodule\footnote{In practice, we often identity $\mathfrak{Lie}(G)$, $\Lie(G)$ and $\sLie(G)$, and simply use $\mathfrak{Lie}(G)$ to denote the Lie algebra of $G$.}.\par
If $T$ is a torus of $S$, we say that $T$ splits if it is isomorphic to $(\G_{m,S})^r$ for some integer $r\geq 0$, and we say that $T$ is trivialized if we have fixed such an isomorphism, or more generally, an isomorphism $T\cong D_S(M)$, where $M$ is a free abelian group of rank $r$. Finally, recall that $T$ is called \textbf{maximal} if for any $s\in S$, the geometric fiber $T_{\bar{s}}$ is a maximal torus in $G_{\bar{s}}$.

\begin{theorem}\label{scheme group fiber reductive fpqc nbhd lifting}
Let $S$ be a scheme, $G$ be an affine $S$-group of finite presentation over $S$, with connected fibers, and $s_0\in S$. Suppose that $G$ is flat over $S$ at $e(s_0)$ and the geometric fiber $G_{\bar{s}_0}$ is reductive (resp. semi-simple). Then there exists an open neighborhood $U$ of $s$ in $S$ and a surjective \'etale morphism $S'\to U$ such that
\begin{enumerate}
    \item[(a)] $G|_U$ is smooth over $S$, with reductive (resp. semi-simple) fibers, of constant reductive rank and semi-simple rank.
    \item[(b)] $G_{S'}$ possesses a split maximal torus $T$ and the Weyl group 
    \[W_{G_{S'}}(T)=N_{G_{S'}}(T)/Z_{G_{S'}}(T)=N_{G_{S'}}(T)/T\]
    is finite over $S'$. 
\end{enumerate}
\end{theorem}
\begin{proof}
Denote by $\pi:G\to S$ the structural morphism and $e:S\to G$ the unit section. As $G$ is flat over $S$ at $e(s_0)$ and $G_{\bar{s}_0}$ is reduced (in fact smooth over $\kappa(\bar{s}_0)$), we see that $G$ is smooth over $S$ at $e(s_0)$ (\cite{EGA4-4} 17.5.1), that is, there exists an open neighborhood $V$ of $e(s_0)$ such that $\pi|_V$ is smooth. Then $S'=e^{-1}(V)$ is an open subset of $S$, and $G_{S'}$ is smooth over $S'$ at points of $e(S')$. As $G$ has connected fibers, $G_{S'}$ is smooth over $S'$ by \cref{scheme group smooth at unit section iff}. Hence, by replacing $S$ with $S'$, we can suppose that $G$ is smooth over $S$.\par
By \cref{scheme group affine smooth functor of subgroup multiplicative representable}, the functor of subgroups of multiplicative type of $G$ is representable by an $S$-scheme $\mathscr{M}$, smooth and separated over $S$. Denote by $r_0$ the reductive rank of $G_{\bar{s}_0}$ and consider the open subscheme $\mathscr{M}_{r_0}$ of $\mathscr{M}$, which represents the functor of sub-tori of $G$ of rank $r_0$\footnote{This is also the locus of $\mathscr{M}$ where the universal subgroup of multiplicative type of $G$ has rank $r_0$, which is open as this subgroup is smooth over $\mathscr{M}$.}. Then $\mathscr{M}_{r_0}$ admits a rational point over a finite separable extension of $\kappa(s_0)$ (\cite{EGA4-4} 17.15.10 (\rmnum{3})), so by taking an \'etale base change, we can suppose that $G_{s_0}$ admits a torus of rank $r_0$. By \cref{scheme group affine smooth multiplicative fiber etale lifting}, we can lift this torus to an $S'$-torus $T'$ of $G_{S'}$, and in view of \cref{scheme group multiplicative ft is quasi-isotrivial} (see also \cref{scheme group smooth affine maximal tori self-centralizing at fiber prop}), there then exists an \'etale morphism $f:S''\to S'$ spliting $T$ and such that $f^{-1}(s_0')\neq 0$. As an \'etale morphism is open and the assertions of (a) are local for the fpqc topology, we can then suppose that $G$ admits a spliting torus $T$, which is maximal at $s_0$. Write $T=D_S(M)$ and let
\[\g=\bigoplus_{m\in M}\g^m\]
be the decomposition of $\g=\mathfrak{Lie}(G/S)$ under $\Ad(T)$. We put $\t=\mathfrak{Lie}(T)$ and, for any $m\in M$, we denote by $\g^m(s_0)=\g^m\otimes_{\mathscr{O}_S}\kappa(s_0)$. Let $R$ be the set of nonzero $m\in M$ such that $\g^m(s_0)\neq 0$. As $G_{\bar{s}_0}$ is reductive, we have $\g^0(s_0)$, hence
\[\g(s_0)=\t(s_0)\oplus\bigoplus_{\alpha\in R}\g^\alpha(s_0).\]
As the modules on both sides are locally free, we can, by restricting $S$, suppose that the $\g^\alpha$ are free and
\[\g=\t\oplus\bigoplus_{\alpha\in R}\g^\alpha.\]
We recall that (cf. \cref{scheme group multiplicative fy unique maximal torus}) a group of multiplicative type possesses a unique maximal torus. Let $T_\alpha$ be the maximal torus of $\ker\alpha$ (the latter is of multiplicative type, cf. \cref{scheme group multiplicative morphism factorization}) and $G_\alpha=Z_G(T_\alpha)$. By (\cite{SGA3-2} \Rmnum{11}, 6.11 (a)) and \cref{scheme multiplicative to smooth transporter formally smooth}, $G_\alpha$ is affine and smooth over $S$, with connected fibers, and by \cref{scheme alg group smooth connected acted by torus reductive iff} its fiber $(G_\alpha)_{\bar{s}_0}$ is reductive of semi-simple rank $1$, with $T_{\bar{s}_0}$ being maximal. From \cref{scheme group smooth affine locus of reductive rho_s=1 is open}, there then exists an open subset $U_\alpha$ of $S$ containing $s_0$ such that $G_\alpha|_{U_\alpha}$ has reductive fibers. Put $U=\bigcap_{\alpha\in R}U_\alpha$ (note that $R$ is a finite set). In view of \cref{scheme alg group smooth connected acted by torus reductive iff} (resp. \cref{scheme alg group smooth connected acted by torus semi-simple iff}), for any $s\in U$, $G_{\bar{s}}$ is then reductive (resp. semi-simple), with maximal torus $T_{\bar{s}}$ and the set of roots of $G_{\bar{s}}$ relative to $T_{\bar{s}}$ is identified with $R$. We then conclude that
\[\rho_r(G_{\bar{s}})=\dim(T)=\rank(M),\quad \rho_s(G_{\bar{s}})=\rank(R).\]
We have therefore proved (a) and the first assertion of (b). It remains to prove that the Weyl group $W_{G_U}(T_U)$ is finite over $U$, i.e. it has the same number of points at each geometric fiber (cf. the proof of \cref{scheme group affine smooth Weyl group fiber rank semicontinuous}). For this, it suffices to note that the geometric fiber of this group at $s\in U$ is determined by the situation $R\sub M$, as the constant group over $\kappa(\bar{s})$ associated with the abstract Weyl group of this root system, and in particular is independent of the point $s$ (cf. \cite{Chevalley1958} \S 11.3, th.2, see also \cite{SGA3-3} \Rmnum{22}, \S 3.4).
\end{proof}

\begin{corollary}\label{scheme group smooth affine rho_r and rho_s locally constant}
Let $G$ be an affine smooth $S$-group with connected fibers. The set of $s\in S$ such that $G_{\bar{s}}$ is reductive (resp. semi-simple) is an open subset $U$ of $S$ and the function
\[s\mapsto\rho_r(G_{\bar{s}})\quad (\text{resp.}\quad s\mapsto\rho_s(G_{\bar{s}}))\]
is locally constant on $U$.
\end{corollary}

An $S$-group $G$ is called \textbf{reductive} (resp. \textbf{semi-simple}) if it is affine smooth over $S$, with connected and reductive (resp. semi-simple) fibers. It is immediate that the fact of being reductive (resp. semi-simple) for an $S$-group $G$ is stable under base change and local for the fpqc topology. Let $G$ be a reductive $S$-group. For any torus (resp. maximal torus) $T$ of $G$, it then follows from \cref{scheme alg group reductive prop} that $C=Z_G(T)$ is reductive (resp. equal to $T$). Applying \cref{scheme group fiber reductive fpqc nbhd lifting} to $Z_G(T)$, we see that $T$ is contained (locally for the \'etale topology) in a maximal torus of $G$.

\subsection{Roots and root systems}
Let $S$ be a scheme, $T$ be an $S$-torus acting linearly on a locally free $\mathscr{O}_S$-module $\mathscr{F}$ of finite type. For any character $\alpha\in X(T)$ (that is, $\alpha\in\Hom_{S\dash\Grp}(T,\G_{m,S})$), we define a subfunctor $\mathbf{W}(\mathscr{F})$ by
\[\mathbf{W}(\mathscr{F})^\alpha(S')=\{x\in\mathbf{W}(\mathscr{F})(S'):\text{$t\cdot x=\alpha(t)x$ for any $t\in T(S'')$, $S''\to S'$}\}.\]

\begin{lemma}\label{scheme group torus action W(F) component exchange}
We have $\mathbf{W}(\mathscr{F})^\alpha=\mathbf{W}(\mathscr{F}^\alpha)$, where $\mathscr{F}^\alpha$ is a locally direct factor of $\mathscr{F}$, hence also locally free.
\end{lemma}
\begin{proof}
In fact, this assertion is local for the fpqc topology, and we can suppose that $T=D_S(M)$, where $M$ is a free abelian group of finite rank. Then $\alpha$ is identified with a locally constant function from $S$ to $M$ (\cref{scheme group D(M_S) to G_m morphism is locally constant}), and by restriction, we may assume that this function is constant. We are then reduced to \cref{scheme module over diagonalizable group cat equivalent to graded module}. 
\end{proof}

Let $S$ be a scheme, $G$ be a smooth $S$-scheme with connected fibers, $T$ be a sub-torus of $G$. We denote by $\g=\mathfrak{Lie}(G/S)$ and act $T$ over $\g$ by adjoint representation. We say that a character $\alpha$ of $T$ is a \textbf{root} of $G$ relative to $T$ if the following equivalent conditions are satisfied:
\begin{enumerate}
    \item[(\rmnum{1})] For each $s\in S$, $\alpha_{\bar{s}}$ is a root of $G_{\bar{s}}$ relative to $T_{\bar{s}}$.
    \item[(\rmnum{2})] $\alpha$ is nontrivial over each fiber and $\g^\alpha(s)\neq 0$ for each $s\in S$.
\end{enumerate}
Note that as for condition (\rmnum{2}), we have the following lemma:

\begin{lemma}\label{scheme group torus character nontrivial on fiber iff}
Let $S$ be a scheme, $T$ be an $S$-torus, $\alpha$ be a character of $T$. The following conditions are equivalent:
\begin{enumerate}
    \item[(\rmnum{1})] $\alpha$ is nontrivial on each fiber, that is, for any $s\in S$, $\alpha_{\bar{s}}$ is nontrivial over $T_{\bar{s}}$.
    \item[(\rmnum{2})] For any $S'\to S$, $S'\neq\emp$, $\alpha_{S'}$ is nontrivial over $T_{S'}$.
    \item[(\rmnum{3})] The morphism $\alpha$ is faithfully flat. 
\end{enumerate}
\end{lemma}
\begin{proof}
It is clear that (\rmnum{2})$\Rightarrow$(\rmnum{1}) and we easily see that (\rmnum{3})$\Rightarrow$(\rmnum{1}). We have (\rmnum{1})$\Rightarrow$(\rmnum{2}) because if $s'\in S'$ is lying over $s$ and if $\alpha_{s'}$ is trivial, then so is $\alpha_s$ by fpqc descent. Finally, to prove (\rmnum{1})$\Rightarrow$(\rmnum{3}), we are reduced to the case where $T$ is diagonalizable, and we then conclude by \cref{scheme group diagonalizable transpose mono epi iff}.
\end{proof}

Now assume that $G$ is reductive and $T$ is a maximal torus of $G$. Let $\alpha$ be a root of $G$ relative to $T$. Then by \cref{scheme alg group smooth connected acted by torus reductive iff}, $\g^\alpha$ is a locally free $\mathscr{O}_S$-module of rank $1$, and $-\alpha$ is also a root of $G$ relative to $T$. In particular, if $G$ is of semi-simple rank $1$, we have by \cref{scheme alg group smooth connected acted by torus reductive iff}:

\begin{lemma}\label{scheme group reductive rho_s=1 Lie algebra root decomposition}
Let $S$ be a scheme, $G$ be a reductive $S$-group of semi-simple rank $1$, and $T$ be a maximal torus of $G$. If $\alpha$ is a root of $G$ relative to $T$, then $-\alpha$ is also a root and we have
\[\g=\t\oplus\g^\alpha\oplus\g^{-\alpha},\]
where $\g^{\alpha}$ and $\g^{-\alpha}$ is locally free of rank $1$.
\end{lemma}

Let $R$ be a set of roots of $G$ relative to $T$. We say that $R$ is a \textbf{root system} of $G$ relative to $T$ if it satisfies the following equivalent conditions:
\begin{enumerate}
    \item[(\rmnum{1})] For each $s\in S$, $\alpha\mapsto\alpha_{\bar{s}}$ is an bijection from $R$ to the set of roots of $G_{\bar{s}}$ relative to $T_{\bar{s}}$.
    \item[(\rmnum{2})] The elements of $R$ are distinct over each fiber (i.e. if $\alpha,\beta\in R$ are distinct elements, then $\alpha-\beta$ is nontrivial over each fiber) and for each $s\in S$, we have
    \begin{equation}\label{scheme group reductive root system def-1}
    \dim(G_s)-\dim(T_s)=\Card(R).
    \end{equation}
    \item[(\rmnum{3})] We have $\g=\t\oplus\bigoplus_{\alpha\in R}\g^\alpha$. 
\end{enumerate}
In fact, it is clear that (\rmnum{3})$\Rightarrow$(\rmnum{1}), and (\rmnum{1})$\Leftrightarrow$(\rmnum{2}) because we have the equality (\ref{scheme group reductive root system def-1}) with $R$ replaced by the root system of $G_{\bar{s}}$. Finally, to see that (\rmnum{1})$\Rightarrow$(\rmnum{3}), we can consider the quotient $\g'=\g'/(\t\oplus\bigoplus_{\alpha\in R}\g^\alpha)$, whose fiber at each $s\in S$ trivial in view of condition (\rmnum{1}) and \cref{scheme alg group smooth connected acted by torus reductive iff}. We then conclude that $\g'=0$, so we obtain the decomposition of (\rmnum{3}). We also note that (\rmnum{3}) implies the following lemma:

\begin{lemma}\label{scheme group reductive root system locally equal}
Let $S$ be a scheme, $G$ be a reductive $S$-group, $T$ be a maximal torus of $G$, $R$ be a root system of $G$ relative to $T$. Then any root of $G$ relative to $T$ is locally over $S$ equal to an element of $R$.
\end{lemma}
\begin{proof}
To prove this, we may assume that $T=D_S(M)$ is diagonalizable, so that any root of $T$ corresponds to a locally constant function $f:S\to M$. But then by condition (\rmnum{1}) above, for any $s\in S$, $f(s)$ is equals to the value of an element of $R$ at $s$, hence is locally (over $S$) eauals to it.
\end{proof}

Put $\mathscr{M}=\sHom_{S\dash\Grp}(T,\G_{m,S})$, which is a twisted constant $S$-group (\cref{scheme group multiplicative D_S representable if quasi-isotrivial}). If $G$ admits a root system $R$ relative to $T$, then the inclusion $R\hookrightarrow\mathscr{M}(S)$ defines a morphism $R_S\to\mathscr{M}$, where $R_S$ is the constant $S$-scheme associated with $R$. Thanks to \cref{scheme group reductive root system locally equal}, we easily see that this morphism is an open and closed immersion whose image is none other than $\bigcup_{\alpha\in R}\alpha(S)$ (each $\alpha$ being considered as a section $S\to\mathscr{M}$).\par
Let $\mathscr{R}$ be the functor of roots of $G$ relative to $T$. By definition, $\mathscr{R}(S')$ is the set of roots of $G_{S'}$ relative to $T_{S'}$ for any $S'\to S$. If $S'=\emp$, we put $\mathscr{R}(\emp)=\{1\}$, and if $S'\neq\emp$, then the inclusion $R\hookrightarrow\mathscr{M}(S')$ ideitifies $R$ with a root system of $G_{S'}$ relative to $T_{S'}$, and hence by \cref{scheme group reductive root system locally equal}, we have
\[\mathscr{R}(S')=\Hom_{\mathrm{loc.const.}}(S',R)=R_S(S')\]
which proves that $\mathscr{R}$ is representable by $R_S$.\par
If now we do not suppose that $G$ possesses a root system relative to $T$, then $\mathscr{R}$ is still a subsheaf of $\mathscr{M}$ for the fpqc topology. Locally for this topology, $G$ possesses a root system relative to $T$ (take for example $T$ splits). By using \cref{site sheaf representable iff descent data effective} and fpqc descent of open and closed subschemes (\cite{SGA1} \Rmnum{8} 4.4), we then obtain the following proposition:

\begin{proposition}\label{scheme group functor of root system representable}
Let $S$ be a scheme, $G$ be a group scheme over $S$, and $T$ be a maximal torus of $G$. The functor $\mathscr{R}$ of roots of $G$ relative to $T$ is representable by a twisted constant finite $S$-scheme, which is an open and closed subscheme of $\sHom_{S\dash\Grp}(T,\G_{m,S})$. Moreover, for $R\sub\sHom_{S\dash\Grp}(T,\G_{m,S})$ to be a root system of $G$ relative to $T$, it is necessary and sufficient that the corresponding morphism $R_S\to\sHom_{S\dash\Grp}(T,\G_{m,S})$ induces an isomorphism $R_S\stackrel{\sim}{\to}\mathscr{R}$.
\end{proposition}

\begin{example}\label{scheme group reductive induced elementary system by root}
Let $S$ be a scheme, $G$ be a reductive $S$-group, $T$ be a maximal torus of $G$, and $\alpha$ be a root of $G$ relative to $T$ (i.e. a section of $\mathscr{R}$). Consider the kernel $\ker\alpha$, with (unique) maximal torus $T_\alpha$ and centralizer $G_\alpha=Z_G(T_\alpha)$. This is a closed subgroup of $G$, reductive (\cref{scheme alg group reductive prop}) of semi-simple rank $1$ (\cref{scheme alg group smooth connected acted by torus reductive iff}), and
\[\mathfrak{Lie}(G_\alpha/S)=\t\oplus\g^\alpha\oplus\g^{-\alpha},\]
hence $\{\alpha,-\alpha\}$ is a root system of $G_\alpha$ relative to $T$.
\end{example}

\subsection{Vectorial group schemes}
Let $S$ be a scheme and $\mathscr{F}$ be a locally free $\mathscr{O}_S$-module of finite type. The $S$-scheme $\mathbf{W}(\mathscr{F})$ is smooth over $S$, as it is represented by $\Spec(\bm{S}(\mathscr{F}^\vee))$. Its Lie algebra is canonically isomorphic to $\mathscr{F}$. In fact, we have a canonical isomorphism (\cref{scheme O_S-module Gamma is good} and \cref{scheme O_S-module good Lie and Lie' coincide})
\[\mathbf{W}(\mathscr{F})\stackrel{\sim}{\to} \mathfrak{Lie}(\mathbf{W}(F)/S)=\mathbf{W}(\mathfrak{Lie}(\mathbf{W}(\mathscr{F})/S)).\]
We thus identity $\mathscr{F}$ and $\mathfrak{Lie}(\mathbf{W}(\mathscr{F})/S)$.

\begin{lemma}\label{scheme smooth vector bundle exp morphism}
Let $S$ be a scheme and $V$ be a smooth vector bundle over $S$. Then there exists a unique isomorphism of $\mathbb{O}_S$-modules
\[\exp:\mathbf{W}(\mathfrak{Lie}(V/S))\to V\]
which induces the identity on Lie algebras.
\end{lemma}
\begin{proof}
In fact, we have $V=\V(\mathscr{F})$ for some quasi-coherent $\mathscr{O}_S$-module $\mathscr{F}$. Let $\pi:V\to S$ be the projection and $e:S\to V$ be the zero section. Then $\Omega_{V/S}^1=\pi^*(\mathscr{F})$, whence
\begin{equation*}
\omega_{V/S}^1:=e^*(\Omega_{V/S}^1)\cong e^*\pi^*(\mathscr{F})\cong\mathscr{F},
\end{equation*}
and hence $\mathfrak{Lie}(V/S)=(\omega_{V/S}^1)^\vee\cong\mathscr{F}^\vee$. Since $V$ is smooth over $S$, we then conclude that $\mathscr{F}$ is locally free of finite type, and hence
\[V=\V(\mathscr{F})\cong\mathbf{W}(\mathscr{F}^\vee)\cong\mathbf{W}(\mathfrak{Lie}(V/S)).\]
The uniquness of the isomorphic follows from the fully faithfulness of $\mathbf{W}$ (\cref{scheme Gamma module functor prop}). 
\end{proof}

If $V$ is a vector bundle over $S$, we denote by $V^\times$ the open subset of $V$ obtained by removing the zero section. We write the group law of $V$ multiplicatively. The operation of $\mathbb{O}_S$ over $V$ defining the module structure is then denoted exponentially:
\[\mathbb{O}_S\times_SV\to V,\quad (x,v)\mapsto v^x.\]
We then have 
\[(vw)^x=v^xw^x,\quad v^{x+y}=v^xv^y,\quad v^{xy}=(v^x)^y.\]
In particular, if we restrict to the action of $\G_{m,S}$, then $V^\times$ is stable under $\G_{m,S}$ and hence endowed with the structure of an $\G_{m,S}$-object. We denote this action by $(z,v)\mapsto v^z$.\par
If $\mathscr{L}$ is an invertible module over $S$ and $\mathbb{W}(\mathscr{L})$ is the associated vector bundle, then $\mathbb{W}(\mathscr{L})^\times$ is a principal homogeneous bundle under $\G_{m,S}$ (Zariski locally trivial). We then write $\Gamma(S,\mathscr{L})^\times=\mathbb{W}(\mathscr{L})^\times(S)$, and we note that there exists a bijective correspondence between isomorphisms of $\mathscr{O}_S$-modules $\mathscr{O}_S\cong\mathscr{L}$, isomorphisms of $\mathbb{O}_S$-modules $\mathbb{O}_S\cong\mathbf{W}(\mathscr{L})$, and sections $S\to\mathbf{W}(\mathscr{L})^\times$. This correspondence is realized by $f\mapsto\mathbf{W}(f)\mapsto\mathbf{W}(f)(1)$, which is compatible with base change. We can hence consider $\mathbf{W}(\mathscr{L})^\times$ as the "scheme of trivializations of $\mathbf{W}(\mathscr{L})$".\par
Let $S$ be a scheme, $T$ be a torus over $S$, $P$ be an $S$-group acted by $\G_{m,S}$ (for example a vector bundle over $S$), $\alpha$ be a character of $T$. We denote by $T\cdot_\alpha P$ the product $S$-scheme of $T$ and $P$, with a group law defined by
\[(t,x)\cdot_\alpha(t',x'):=(tt',\alpha(t')xx').\]

\begin{definition}\label{scheme group morphism from W(L) normalized def}
Let $G$ be a group scheme over $S$, $T$ be a subgroup of $G$, $\alpha$ be a character of $T$, $\mathscr{L}$ be a quasi-coherent $\mathscr{O}_S$-module. Let
\[p:\mathbf{W}(\mathscr{L})\to G\]
be a homorphism of $S$-functors in groups. We say that $p$ is \textbf{normalized by $T$ with multiplicator $\alpha$} if it satisfies the following equivalent condition:
\begin{enumerate}
    \item[(\rmnum{1})] $p$ is a morphism of $T$-objects, if we acts $T$ over $\mathbf{W}(\mathscr{L})$ by $\alpha$ and over $G$ by inner automorphisms. In other words, for any $S'\to S$ and any $t\in T(S')$, $x\in\mathbf{W}(\mathscr{L})(S')=\Gamma(S',\mathscr{L}\otimes_{\mathscr{O}_S}\mathscr{O}_{S'})$, we have
    \[\inn(t)p(x)=p(\alpha(t)x).\] 
    \item[(\rmnum{2})] The morphism $\varphi:T\cdot_\alpha\mathbf{W}(\mathscr{L})\to G$ defined by $(t,x)\mapsto t\cdot p(x)$ is a homomorphism of groups.
\end{enumerate}
\end{definition}
In fact, from the definition of the group law of $T\cdot_\alpha\mathbf{W}(\mathscr{L})$, we see that $\varphi$ is a morphism of groups if and only if (note that $p$ is a group morphism)
\[tp(x)t'p(x')=tt'\cdot p(\alpha(t')xx')=tt'\cdot p(\alpha(t')x)p(x'),\]
i.e. if and only if $t'p(x)t'^{-1}=p(\alpha(t')x)$. 

\begin{lemma}\label{scheme group morphism from W(L) normalized is root}
Unde the conditions of \cref{scheme group morphism from W(L) normalized def}, suppose that $G$ is smooth with connected fibers, $T$ is a maximal torus of $G$, and $\mathscr{L}$ is invertible. If $p$ is a monomorphism and $\alpha$ is nonzero on each fiber, then $\alpha$ is a root of $G$ relative to $T$.
\end{lemma}
\begin{proof}
In fact, $\mathfrak{Lie}(p):\mathscr{L}\to\g$ is then a monomorphism, which factors through $\g^\alpha$ in view of condition (\rmnum{1}).
\end{proof}

\begin{proposition}\label{scheme group morphism from W(L) image g^alpha if}
Under the conditions of \cref{scheme group morphism from W(L) normalized is root}, suppose that $G$ is reductive, and that $p$ is a monomorphism. Then $\alpha$ is a root of $G$ relative to $T$ and $\mathfrak{Lie}(p)$ induces an isomorphism
\[\mathfrak{Lie}(p):\mathscr{L}\stackrel{\sim}{\to}\g^\alpha.\]
\end{proposition}
\begin{proof}
In view of \cref{scheme group morphism from W(L) normalized is root} and the fact that $\g^\alpha$ is invertible, it suffices to prove that $\alpha$ is nonzero over each fiber. Let $s\in S$ such that $\alpha_{\bar{S}}=0$. If $x$ is a nonzero section of $\mathscr{L}_{\bar{s}}$, then by condition (\rmnum{1}), $p(x)$ is a nonzero unipotent element of $G(\bar{s})$ which centralizes $T_{\bar{s}}$. But this is impossible since the geometric fiber of $Z_G(T)=T$ at $\bar{s}$ consists of semi-simple elements.
\end{proof}

\begin{corollary}\label{scheme group reductive monomorphism W(g^alpha) to G}
Under the conditions of \cref{scheme group morphism from W(L) image g^alpha if}, then there exists a monomorphism of groups acted by $T$:
\[\mathbf{W}(\g^\alpha)\to G,\]
which induces the canonical morphism $\g^\alpha\to\g$ on Lie algebras. 
\end{corollary}

We shall see that \cref{scheme group reductive monomorphism W(g^alpha) to G} is in fact valid whenever $G$ is a reductive group and $\alpha$ is a root of $G$ relative to $T$ (i.e. without the hypothesis that there is a morphism $p:\mathbf{W}(\mathscr{L})\to G$ normalized by $T$ with multiplicator $\alpha$), and this morphism is unique.

\begin{remark}
Let $k$ be an algebraically closed field, $G$ be a reductive $k$-group, $T$ be a maximal torus of $G$, and $\alpha$ be a root of $G$ relative to $T$. Then there exists a monomorphism
\[p:\G_{a,k}\to G\]
normalized by $T$ with multiplicator $\alpha$ (cf. \cite{Chevalley1958} \S13.1, th.1).
\end{remark}

We conclude this subsection by a technical lemme which will be used later. Let $S$ be a scheme and $\mathscr{L}$ be an invertible $\mathscr{O}_S$-module. Let $q>0$ be an integer such that $x\mapsto x^q$ is an endomorphism of $S$-group $\G_{a,S}$ (if $S\neq\emp$, we have $q=1$, or $q=p^n$ where $p$ is a prime number nonzero over $S$; this follows easily from the following elementary fact: the prime gcd of the binomial coefficients $\binom{q}{i}$, for $i\neq 0,q$, is equal to $p$ ($p$ being prime) if $q=p^n$, and $1$ in the contrary case). The morphism defined by the $q$-th power
\[\mathscr{L}\to\mathscr{L}^{\otimes q}\]
is a morphism of sheaves of abelian groups. It then defines by base change a morphism of $S$-groups:
\[\mathbf{W}\to\mathbf{W}(\mathscr{L}^{\otimes q}).\]
In particular, if $\mathscr{L}'$ is another invertible module and if we have a morphism of $\mathscr{O}_S$-modules
\[h:\mathscr{L}^{\otimes q}\to\mathscr{L}',\]
we then deduce a morphism of $S$-groups
\[\mathbf{W}(\mathscr{L})\to\mathbf{W}(\mathscr{L}'),\quad x\mapsto h(x^q).\]
With these notations, we then have:

\begin{proposition}\label{scheme group bimorphism of invertible module under character prop}
Let $S$ be a scheme, $T$ (resp. $T'$) be an $S$-torus, $\mathscr{L}$ (resp. $\mathscr{L}'$) be an invertible $\mathscr{O}_S$-module, $\alpha$ (resp. $\alpha'$) be a character of $T$ (resp. $T'$). Let $f:T\to T'$ be a group homomorphism and $g:\mathbf{W}(\mathscr{L})\to\mathbf{W}(\mathscr{L}')$ be a morphisms of $S$-schemes (not necessarily a group homomorphism) verifying the following conditions:
\begin{equation}\label{scheme group bimorphism of invertible module under character prop-1}
g(\alpha(t)x)=\alpha'(f(t))g(x)
\end{equation}
for any $x\in\mathbf{W}(\mathscr{L})(S')$, $t\in T(S')$, $S'\to S$. Let $s_0\in S$ be such that $\alpha_{\bar{s}_0}\neq 0$.
\begin{enumerate}
    \item[(a)] Suppose that $g$ sends the zero section to the zero section and that for any integer $n>0$, we have $(\alpha'\circ f)_{\bar{s}_0}\neq n\alpha_{\bar{s}_0}$. Then $g=0$ in a neighborhood of $s_0$.
    \item[(b)] Suppose that $g$ is a group homomorphism such that $g_{\bar{s}_0}\neq 0$. Then there exists an open neighborhood $U$ of $s_0$ in $S$ and an integer $q>0$ such that $x\mapsto x^q$ is an endomorphism of $\G_{a,U}$ and that $(\alpha'\circ f)|_U=q\alpha_U$.
    \item[(c)] Suppose that $(\alpha'\circ f)_{\bar{s}_0}=q\alpha_{\bar{s}_0}$, where $q>0$ is an integer such that $x\mapsto x^q$ is an endomorphism of $\G_{a,S}$. Then there exists an open neighborhood $U$ of $s_0$ in $U$ and a unique morphism of $\mathscr{O}_S$-modules
    \[h:\mathscr{L}^{\otimes q}|_U\to\mathscr{L}'|_U\]
    such that $g_U$ is the composition morphism
    \[\mathbf{W}(\mathscr{L})_U\stackrel{x\mapsto x^q}{\longrightarrow}\mathbf{W}(\mathscr{L}^{\otimes q})_U\stackrel{\mathbf{W}(h)}{\longrightarrow}\mathscr{L}'_U.\] 
\end{enumerate}
\end{proposition}
\begin{proof}
To prove the assertion of (a), we note that as the conclusion is local over $S$, we can suppose that $\mathbf{W}(\mathscr{L})=\mathbf{W}(\mathscr{L}')=\G_{a,S}$ and hence that $g$ is expressed by a polynomial
\[g(X)=\sum_{n\geq 0}a_nX^n,\quad a_n\in\Gamma(S,\mathscr{O}_S).\]
Condition (\ref{scheme group bimorphism of invertible module under character prop-1}) is then written as an identity in $\Gamma(S',\mathscr{O}_{S'})[X]$:
\[\sum_{n\geq 0}a_n\alpha'(f(t))X^n=\sum_{n\geq 0}a_n\alpha(t)^nX^n;\]
that is, for any $n\geq 0$, any $S'\to S$ and $t\in T(S')$, we have
\[a_n(\alpha'(f(t))-\alpha(t)^n)=0.\]
For each $\ngeq 0$, let $S_n$ be the set of $s\in S$ such that $(\alpha'\circ f)_{\bar{s}}=n\alpha_{\bar{s}}$. By \cref{scheme group multiplicative morphism trivial locus prop}, we see that each $S_n$ is clopen and $(\alpha'\circ f)_{S_n}=n\alpha_{S_n}$. Moreover, as $\alpha_{\bar{s}_0}\neq 0$, we can, by restricting $S$, suppose that $\alpha$ is nonzero over each fiber, so that the $S_n$ are disjoint. By further restricting $S$, we can hence suppose that one of the following conditions is satisfied: there exists an integer $n\geq 0$ such that $S=S_n$, or any $S_n$ is empty.\par
Let $m\geq 0$ be such that $S_m=\emp$, we then claim that $a_m=0$, so that $g$ is trivial in a neighborhood of $s_0$. In fact, $\alpha'\circ f$ and $m\alpha$ are then distinct over each fiber over $S$, and we have:
\begin{lemma}\label{scheme torus distinct character fpqc local difference by 1}
Let $S$ be a scheme, $T$ be an $S$-torus, $\alpha,\alpha'$ be two characters distinct over each fiber. Then there exists a covering family $\{S_i\to S\}$ for the fpqc topology and $t_i\in T(S_i)$ for each $i$ such that $\alpha(t_i)-\alpha'(t_i)=1$. 
\end{lemma}
To prove this, by taking a fpqc cover, we can reduce to the case where $S$ is affine and $T$ is diagonalizable, and thus to $T=\G_{m,S}$. Then $\alpha,\alpha'$ are given by $x\mapsto x^n$ and $x\mapsto x^{n'}$, where $n,n'$ are distinct integers. It then suffices to find $t\in\Gamma(S',\mathscr{O}_{S'})^\times$ such that $t^n-t^{n'}=1$, where $S'\to S$ is a suitable faithfully flat and quasi-compact morphism, which is possible by taking a quotient of a polynomial ring.\par
Now return to the proof of \cref{scheme group bimorphism of invertible module under character prop}; we have just proved (a). In the cases (b) and (c), there exists an integer $n\geq 0$ such that $S=S_n$ ($n=q$ in (c)). By the preceding result, we then have $a_m=0$ for $m\neq n$, which proves that
\[g(X)=a_nX^n,\quad a_n\in\Gamma(S,\mathscr{O}_S).\]
This clearly implies the assertions in (c). In case (b), we see that $a_n(s_0)\neq 0$, hence we can suppose that $a_n$ is invertible over $S$, which implies that $x\mapsto x^n$ is an endomorphism of $\G_{a,S}$ (in view of the hypothesis of (b)), which proves the assertion.
\end{proof}

\subsection{An instructive example}
Let $k$ be an algebraically closed field of characteristic zero. Put $A=k[t]$, $S=\Spec(A)$, and consider the following Lie algebrao $\g$ over $\mathscr{O}_S$: as an $\mathscr{O}_S$-module, it is free of dimension $3$, with basis $\{X,Y,H\}$; the bracket of $\g$ is given by:
\[[X,Y]=2t\cdot H,\quad [H,X]=X,\quad [X,Y]=-Y.\]
For $s\in S$, $s\neq s_0$ ($s_0$ being the point defined by $t=0$), the fiber $\g(s)=\g\otimes_A\kappa(s)$ is isomorphic to the Lie algebra of the group $\PGL_{2,\kappa(s)}$. But for $s=s_0$, it is a solvable Lie algebra.\par
Let $G_1$ be the automorphism group of $\g$: for any $S'\to S$, $G_1(S')$ is the automorphism group of the Lie $\mathscr{O}_{S'}$-algebra $\g\otimes_{\mathscr{O}_S}\mathscr{O}_{S'}$. This is a closed subscheme of $\GL(\g)$: Let $S'\to S$ and $u\in\mathcal{M}_3(\Gamma(S',\mathscr{O}_{S'}))$, considered as an endomorphism of the $\mathscr{O}_{S}$-module $\g\otimes_{\mathscr{O}_S}\mathscr{O}_{S'}$:
\begin{align*}
u(X)&=aX+bY+eH,\\
u(Y)&=b'X+a'Y+e'H,\\
u(H)&=cX+c'Y+dH,
\end{align*}
then $u$ is a section of $G_1$ if and only if $\det(u)$ is invertible and we have the following relations\footnote{The equality $[u(X),u(Y)]=2tu(H)$ (resp. $[u(H),u(X)]=u(X)$, resp. $[u(H),u(Y)]=-u(Y)$) corresponds to the relations (4), (4') and (5) (resp. (1)--(3), resp. (1')--(3')).}:
\begin{alignat*}{2}
&a(d-1)\stackrel{\textup{(1)}}{=} ec,&\quad\quad&a'(d-1)\stackrel{\textup{(1')}}{=}e'c',\\
&b(d+1)\stackrel{\textup{(2)}}{=} ec',&\quad\quad&b'(d+1)\stackrel{\textup{(2')}}{=}e'c,\\
&e\stackrel{\textup{(3)}}{=} 2t(bc-ac'),&\quad\quad&e'\stackrel{\textup{(3')}}{=}2t(b'c'-a'c),\\
&2tc\stackrel{\textup{(4)}}{=} eb'-ae',&\quad\quad&2tc'\stackrel{\textup{(4')}}{=}be'-ea',
\end{alignat*}
\vspace*{-8.5mm}
\begin{align*}
2t(aa'-bb')\stackrel{\textup{(5)}}{=}2td.
\end{align*}

\begin{lemma}\label{scheme group Aut of PGL_2 deform prop}
The relations (1), (1'), (2), (2') imply
\begin{equation}\label{scheme group Aut of PGL_2 deform prop-1}
\det(u)=aa'(2-d)+bb'(2+d),\quad aa'-bb'=d\cdot\det(u).
\end{equation}
\end{lemma}
\begin{proof}
In fact, the first assertion follows easily from the relations (1), (1'), (2), (2') in the developement of $\det(u)$:
\begin{align*}
\det(u)&=aa'd+be'c+b'c'e-a'ec-ae'c'-bb'd\\
&=aa'd+bb'(d+1)+bb'(d+1)-aa'(d+1)-aa'(d-1)-bb'd-aa'(d-1)\\
&=aa'(d-d+1-d+1)+bb'(d+1+d+1-d)\\
&=aa'(2-d)+bb'(2+d).
\end{align*}
Multiplying this relation by $d$, we then obtain
\[d\cdot\det(u)=aa'(2d-d^2)+bb'(2d+d^2).\]
But the relation $\textup{(1)}\times\textup{(1')}=\textup{(2)}\times\textup{(2')}$ gives $aa'(d-1)^2=bb'(d+1)^2$, so we obtain the second formula in (\ref{scheme group Aut of PGL_2 deform prop-1}).
\end{proof}

Now we consider $G=G_1\cap\SL(\g)$. This is a closed subgroup of $G_1$ defined by the equation $\det(u)=1$, and hence is affine over $S$.

\begin{proposition}\label{scheme group SL_2 cap Aut PGL_2 deform is smooth}
The group $G$ is smooth over $S$.
\end{proposition}
To prove this proposition, we need the following lemmas:

\begin{lemma}\label{scheme group SL_2 cap Aut PGL_2 deform ad invertible char}
Let $U$ be the open subset of $\sEnd_A(\g)\cong\mathbf{W}(\mathcal{M}_3(\mathscr{O}_S))$ defined by the condition "the product $ad$ is invertible", i.e. the open subset $\sEnd_A(\g)_f$, where $f$ is the function defined by $f(u)=ad$. Then $U\cap G$ is the closed subscheme of $U$ defined by the following equations: (1), (2), (2'), (3), (3'), and
\[aa'-bb'\stackrel{\textup{(D)}}{=}d.\]
\end{lemma}

\begin{lemma}\label{scheme group SL_2 cap Aut PGL_2 deform smooth at unit}
$G$ is smooth over $S$ at the unit section. 
\end{lemma}
\begin{proof}
By \cref{scheme group SL_2 cap Aut PGL_2 deform ad invertible char} and (\cite{SGA1-} \Rmnum{2} 4.10), it suffices to prove that the differential of the functions
\begin{gather*}
a(d-1)-ec,\quad b(d+1)-ec',\quad b'(d+1)-e'c,\\
e-2t(bc-ac'),\quad e'-2t(b'c'-a'c),\quad aa'-bb'-d,
\end{gather*}
at the points of the unit section of $G$, are linear independent. Now, denoting by a capital letter the differential of the corresponding lower case, these are
\[D,\quad 2B,\quad 2B',\quad E+2tC',\quad E'+2tC,\quad A+A'-D,\]
which are easily seen to be linearly independent modulo any $(t-\lambda)$, $\lambda\in k$.
\end{proof}

\begin{lemma}\label{scheme group SL_2 cap Aut PGL_2 deform fiber semi-simple}
For $s\in S$, $s\neq s_0$, the fiber $G_s$ is connected and semi-simple.
\end{lemma}
\begin{proof}
In fact, as $s\neq s_0$, $\g(s)$ is isomorphic to the Lie algebra of $\PGL_{2,\kappa(s)}$, and we have $G_s=(G_1)_s$. But it is known that the automorphism group of the Lie algebra of $\PGL_2$ over a field of characteristic zero is $\PGL_2$ itself, which is connected and semi-simple.
\end{proof}

\begin{lemma}\label{scheme group SL_2 cap Aut PGL_2 deform fiber at origin solvable}
The fiber $G_{s_0}$ is solvable and has two connected components, which are of the following form:
\[G^0_{s_0}=\left\{\begin{pmatrix}
a&0&0\\
0&a^{-1}&0\\
c&c'&1
\end{pmatrix}\right\},\quad G_{s_0}^-=\left\{\begin{pmatrix}
0&b&0\\
b^{-1}&0&0\\
c&c'&-1
\end{pmatrix}\right\}.\]
In particular, $w=\big(\begin{smallmatrix}0&1&0\\1&0&0\\0&0&-1\end{smallmatrix}\big)$ is a section of $G$ over $S$, such that $w(s_0)\in G_{s_0}^-$.
\end{lemma}
\begin{proof}
At the point $s_0$, we have $t=0$, so $e=e'=0$. It is then easy to see that $G_{s_0}$ is defined by the following relations:
\[a(d-1)=0,\quad b(d+1)=0,\quad b'(d+1)=0,\quad aa'-bb'=d\]
from which the first assertion follows. The second one is immediate, since $w$ is clearly a section of $G$.
\end{proof}

\begin{proof}[\textbf{Proof of \cref{scheme group SL_2 cap Aut PGL_2 deform is smooth}}]
Denote by $G^0$ the union of identity components of fibers of $G$ (that is, the complementary of $G_{s_0}^-$ in $G$). As $G$ is smooth over $S$ along the unit section \cref{scheme group SL_2 cap Aut PGL_2 deform smooth at unit}, we see that $G^0$ is an open subgroup of $G$ which is smooth over $S$ (\cref{scheme group smooth at unit section iff}). Since, by applying the right translation $\ell_w$, we see that $G$ is smooth at points of $w(S)$, it is therefore smooth over $S$.
\end{proof}

Consider the morphism $\G_{m,S}\to G^0$ defined by
\[z\mapsto\begin{pmatrix}
z&0&0\\
0&z^{-1}&0\\
0&0&1
\end{pmatrix}\]
This is a monomorphism defining a torus $T$ of $G^0$. We claim that 
\[T=Z_G(T)=Z_{G^0}(T)\]
It suffices to verify the first equality, and since this is a smooth subgroup of $G$, it suffices to verify this at the geometric points. For the fibers at $s\neq s_0$, this results from the fact that $\PGL_{2,\kappa(s)}$ is reductive and that $T_s$ is a maximal torus (by dimension consideration, cf. \cref{scheme alg group smooth reductive with ss-rank 1 iff}). Over the fiber at $s_0$, this calculation is immedaite, using the fact that $t=0$. It follows in particular that $T$ is a maximal torus of $G$ and of $G^0$.\par
The section $w$ of $G$ defined in \cref{scheme group SL_2 cap Aut PGL_2 deform fiber at origin solvable} normalizes $T$, and it follows easily (cf. \cref{scheme group reductive locus Weyl group Z/2Z prop}) that the Weyl group of $G$ is isomorphic to $(\Z/2\Z)_S$, and in particular finite over $S$:
\[W_G(T)=N_G(T)/T=(\Z/2\Z)_S.\]
On the other hand, $W_{G^0}(T)$ is not finite over $S$: in the identity componenet $G_{s_0}^0$, the torus $T_{s_0}$ is self-normalizing (follows from a immediate calculation), so the fiber of $W_{G^0}(T)$ at $s_0$ is trivial. In other words, $W_{G^0}(T)$ is not isomorphic to $\Z/2\Z$ because it is "missing a point" over $s_0$.\par
The open immersion $G^0\to G$ is not a closed immersion (because $G^0$ is dense in $G$); this is however an affine morphism (and hence $G^0$ is affine oveer $S$). In fact, as $G_{s_0}^0$ is closed in $G_{s_0}$, which is closed in $G$, the complementary $U$ of $G_{s_0}^0$ in $G$ is open; $G^0$ and $U$ form an open covering of $G$, and it suffices to verify that the immersions $G^0\to G^0$ and $G^0\cap U\to U$ are affine. The first case is trivial, and for the second, we remark that $U\cap G^0$ is defined in $U$ by the equation $t\neq 0$, so the morphism $U\cap G^0\to U$ is affine.\par
We have therefore constructed an affine smooth $S$-group $G^0$, with connected fibers and possessing a maximal torus $T$ which is self-centralizing and whose Weyl group $W_{G^0}(T)$ is not finite over $S$ (compare \cref{scheme group fiber reductive fpqc nbhd lifting}).

\subsection{Local existence of maximal tori and Weyl group}
During the proof of \cref{scheme group fiber reductive fpqc nbhd lifting}, we have used a result of \autoref{scheme group affine smooth functor of multiplicative type subsection} on the local existence (for the etale topology) of maximal tori; the proof of \autoref{scheme group affine smooth functor of multiplicative type subsection} relies on the representability of the functor of subgroups of multiplicative type (namely \cref{scheme group affine smooth functor of subgroup multiplicative representable}). In the particular case which concerns us, we can give another proof, based on the ideas of (\cite{SGA3-2} \Rmnum{12} \S 7\autoref{*}).

\begin{proposition}\label{scheme group smooth affine maximal tori self-centralizing at fiber prop}
Let $S$ be a scheme, $G$ be an affine smooth $S$-group with connected fibers over $S$, $s_0$ be a point of $S$ such that the maximal tori of the geometric fiber $G_{\bar{s}_0}$ are self-centralizing. Then there exists an \'etale morphism $S'\to S$ passing through $s_0$, and a splitting maximal torus $T'$ of $G_{S'}$\footnote{In the proof, we will also see that the reductive rank of $G_{\bar{s}}$ is constant in a neighborhood of $s=s_0$.}.
\end{proposition}
\begin{proof}

\end{proof}

\begin{proposition}\label{scheme group smooth fp centralizer and normalizer of subtorus prop}
Let $S$ be a scheme, $G$ be a smooth $S$-group of finite presentation over $S$, and $T$ be a sub-torus of $G$.
\begin{enumerate}
    \item[(a)] $Z_G(T)$ and $N_G(T)$ are representable by closed subgroups of $G$, which are smooth (and hence of finite presentation) over $S$.
    \item[(b)] $Z_G(T)$ is a clopen subscheme of $N_G(T)$. The quotient $W_G(T)=N_G(T)/Z_G(T)$ is representable by an open subgroup of $\sAut_{S\dash\Grp}(T)$, this is hence a quasi-finite $S$-group, which is \'etale and separated over $S$.
    \item[(c)] For any $s\in S$, put
    \[w(s)=\Card(N_{G_{\bar{s}}}(T_{\bar{s}})/Z_{G_{\bar{s}}}(T_{\bar{s}})).\]
    Then $s\mapsto w(s)$ is lower semi-continuous, and is constant in a neighborhood of $s$ if and only if $W_G(T)$ is finite over a neighborhood of $s$.
\end{enumerate}
\end{proposition}
\begin{proof}
By (\cite{SGA3-2} \Rmnum{11} 6.11), $Z_G(T)$ and $N_G(T)$ are representable by closed subgroups of $G$, and is of finite presentation over $S$. There are smooth by \cref{scheme group smooth subgroup multiplicative transporter formally smooth} and \cref{scheme multiplicative to smooth transporter formally smooth}, which proves (a). The assertions (b) and (c) are then proved as in \cref{scheme group affine smooth Weyl group representable} and \cref{scheme group affine smooth Weyl group fiber rank semicontinuous}, whose proof in fact uses only (a).
\end{proof}

\section{Reductive groups of semi-simple rank \texorpdfstring{$1$}{a}}\label{scheme group reductive elementary system section}
\subsection{Elementary \texorpdfstring{$S$}{S}-systems}
Let $S=\Spec(k)$, where $k$ is an algebraically closed field, and let $G$ be a reductive $S$-group of semi-simple rank $1$, $T$ be a maximal torus (not necessarily splitting) of $G$. We then have
\begin{equation}\label{scheme alg group elementary system root decomposition}
\g=\t\oplus\g^\alpha\oplus\g^{-\alpha}
\end{equation}
where $\alpha$ and $-\alpha$ are the roots of $G$ relative to $T$. Moreover, there exists two monomorphisms of groups
\[p_\alpha:\G_{a,S}\to G,\quad p_{-\alpha}:\G_{a,S}\to G\]
such that for any $S'\to S$ and $t\in T(S')$, $x\in\G_a(S')$,
\[tp_\alpha(x)t^{-1}=p(\alpha(t)x),\quad tp_{-\alpha}(x)t^{-1}=p_{-\alpha}(\alpha^{-1}(t)x),\]
and that the morphism
\[\G_{a,S}\times_ST\times_S\G_{a,S}\to G,\quad (y,t,x)\mapsto p_{-\alpha}tp_\alpha(x)\]
is radiciel and dominant (\cite{Chevalley1958}, \S 13.4, cor.2 of th.3). Since the tangent map of $u$ at identity is bijective in view of (\ref{scheme alg group elementary system root decomposition}), this morphism is equally \'etale, whence a dominant open immersion (\cite{EGA4-4} 17.9.1).

\begin{lemma}\label{scheme group morphism from G_a twisted by root char}
Let $S$ be a scheme, $G$ be a group scheme over $S$, $T$ be a torus of $G$, $Q$ be a sub-torus of $T$, $\alpha$ be a character of $T$ inducing a character over $Q$ nontrivial over each fiber. Let $p_\alpha:\G_{a,S}\to G$ (resp. $p_{-\alpha}:\G_{a,S}\to G$) be a group homomorphism normalized by $T$ with multiplicator $\alpha$ (resp. $-\alpha$), and suppose that the morphism 
\[u:\G_{a,S}\times_ST\times_S\G_{a,S}\to G,\quad (y,t,x)\mapsto p_{-\alpha}(y)tp_\alpha(x)\]
is an open immersion. Finally, let $q\geq 0$ be an integer and $p:\G_{a,S}\to G$ be a group homorphism such that for any $S'\to S$, $t\in Q(S')$, $x\in\G_a(S')$, we have
\begin{equation}\label{scheme group morphism from G_a twisted by root char-1}
\inn(t)^q(p(x))=p(\alpha(t)x).
\end{equation}
then there exists a unique $\nu\in\G_a(S)$ such that $p(x)=p_\alpha(\nu x^q)$.
\end{lemma}
\begin{proof}
Let $\Omega$ be the image of $u$ and $U=p^{-1}(\Omega)$. This is an open subset of $\G_{a,S}$ which contains the zero section. For any section $t$ of $Q$, in view of (\ref{scheme group morphism from G_a twisted by root char-1}) and the hypothesis on $p_\alpha$, $p_{-\alpha}$, we see that the automorphism of $\G_{a,S}$ defined by the multiplication by $\alpha(t)$ fixes $U$. We claim that $U=\G_{a,S}$: it suffices to verify this for the case where $S$ is the spectrum of an algebraically closed field $k$. But then $\alpha:Q(k)\to k^\times$ is surjective, so $U(k)\sups k^\times$, whence $U=\G_{a,k}$. Now there then exists morphisms
\[a:\G_{a,S}\to\G_{a,S},\quad b:\G_{a,S}\to T,\quad c:\G_{a,S}\to\G_{a,S}\]
such that 
\[p(x)=p_{-\alpha}(a(x))b(x)p_\alpha(c(x)).\]
The condition made over $p$ is expressed by:
\begin{align*}
a(\alpha(t)x)=\alpha(t)^{-1}a(x),\quad b(\alpha(t)x)=b(x),\quad c(\alpha(t)x)=\alpha(t)^qc(x).
\end{align*}
By the same reasoning, we see that for any $S'\to S$ and any $z\in\G_{m}(S')$, 
\[a(zx)=z^{-1}a(x),\quad b(zx)=b(x),\quad c(zx)=z^qc(x),\]
whence
\[z^qa(z)=a(1),\quad b(z)=b(1),\quad c(z)=z^qc(1).\]
As $\G_{m,S}$ is schematically dense in $\G_{a,S}$, we then obtain that for any $S'\to S$, $x\in\G_{a}(S')$:
\begin{gather*}
x^qa(x)=a(0)=0,\quad b(x)=b(0)=e,\quad c(x)=x^qc(1)=\nu x^q,
\end{gather*}
where $\nu\in\G_a(S)$. This completes the proof.
\end{proof}

\begin{definition}
Let $S$ be a scheme, we define an \textbf{elementary $S$-system} to be a triple $(G,T,\alpha)$ satisfying the following conditions:
\begin{enumerate}[leftmargin=40pt]
    \item[(E1)] $G$ is a reductive $S$-group of semi-simple rank $1$.
    \item[(E2)] $T$ is a mximal torus of $G$.
    \item[(E3)] $\alpha$ is a root of $G$ relative to $T$.  
\end{enumerate}
We then have a decomposition of the Lie algebra $\g$ of $G$ (\cref{scheme group reductive rho_s=1 Lie algebra root decomposition}):
\[\g=\t\oplus\g^\alpha\oplus\g^{-\alpha},\]
where $\g^\alpha$ and $\g^{-\alpha}$ are free of rank $1$.
\end{definition}

Let $(G,T,\alpha)$ be an elementary $S$-system, then $(G_{S'},T_{S'},\alpha_{S'})$ is an elementary $S'$-system for any $S'\to S$, and $(G,T,-\alpha)$ is also an elementary $S$-system. We note that if $G$ is an arbitrary reductive group over $S$, $T$ is a maximal torus of $G$, and $\alpha$ is a root of $G$ relative to $T$, then $(G_\alpha,T,\alpha)$ is an elementary $S$-system (\cref{scheme group reductive induced elementary system by root}).\par
If $(G,T,\alpha)$ is an elementary $S$-system, the invertible module $\g^\alpha$ is canonically endowed with a structure of $T$-module. We then have a $T$-module structure over the vector bundle $\mathbf{W}(\g^\alpha)$. On the other hand, the inner automorphisms of $T$ defines an action of $T$ over $G$.

\begin{theorem}\label{scheme group elementary system exp morphism}
Let $(G,T,\alpha)$ be an elementary $S$-system.
\begin{enumerate}
    \item[(a)] There exists a unique $T$-equivariant group homomorphism
    \[\exp:\mathbf{W}(\g^\alpha)\to G\]
    which induces the canonical inclusion $\g^\alpha\to\g$ on Lie algebras. In other words, $\exp$ is the unique morphism verifying the following conditions: for any $S'\to S$, $t\in T(S')$, $X,Y\in\mathbf{W}(\g^\alpha)(S')$, we have
    \begin{align*}
    \exp(X+Y)&=\exp(X)\exp(Y),\\
    \inn(t)(\exp(X))&=\exp(\alpha(t)X),\\
    \mathfrak{Lie}(\exp)(X)&=X.
    \end{align*}  
    \item[(b)] If we define similarly (for the elementary $S$-system $(G,T,-\alpha)$) $\exp:\mathbf{W}(\g^{-\alpha})\to G$, then the morphism
    \begin{equation}\label{scheme group elementary system exp morphism-1}
    u:\mathbf{W}(\g^{-\alpha})\times_ST\times_S\mathbf{W}(\g^\alpha)\to G,\quad (Y,t,X)\mapsto \exp(Y)\cdot t\cdot \exp(X)
    \end{equation}
    is an open immersion.
\end{enumerate}
\end{theorem}
\begin{proof}
Suppose that we have already proved the existence of the morphism $\exp$ with the desired properties in the theorem, we now prove (b). As the two members in (b) are of finite presentation and flat over $S$, it suffices to verify this for $S$ being the spectrum of an algebraically closed field (\cite{SGA1} \Rmnum{1} 5.7 et \rmnum{8} 5.5), so let $S=\Spec(k)$. Let $X\in\Gamma(S,\g^\alpha)^\times$, $Y\in\Gamma(S,\g^{-\alpha})^\times$, it suffices to prove that the morphism
\[\G_{a,k}\times_kT\times_k\G_{a,k}\to G,\quad (y,t,x)\mapsto\exp(yY)\cdot t\cdot\exp(xX)\]
is an opem immersion. Now since $\exp$ satisfies the condition in \cref{scheme group morphism from G_a twisted by root char}, there exists $a,b\in k$ such that 
\[\exp(yY)=p_{-\alpha}(ay),\quad \exp(xX)=p_\alpha(bx).\]
Since $\exp$ induces a monomorphism on Lie algebras, we have $a,b\neq 0$, and the assertion follows from the remark at the begining of this subsection (i.e. the assertion is true for $a=b=1$). The uniquenss of $\exp$ can be proved locally over $S$, so we are reduced to the case where $\g^\alpha$ and $\g^{-\alpha}$ are free, and we can then apply \cref{scheme group morphism from G_a twisted by root char} (with $q=1$).
\end{proof}

The image of the canonical immersion
\[W(\g^{-\alpha})\times_ST\times_S\mathbf{W}(\g^\alpha)\to G\]
is denoted by $\Omega$. This is an open subset of $G$ containing the unit section. The image of $\mathbf{W}(\g^{-\alpha})$ (resp. $\mathbf{W}(\g^\alpha)$, resp. $\mathbf{W}(\g^{-\alpha})\times_ST$, resp. $T\times_S\mathbf{W}(\g^{\alpha})$) is denoted by $U_{-\alpha}$ (resp. $U_\alpha$, resp. $U_{-\alpha}\cdot T$, resp. $T\cdot U_\alpha$). Therefore, $U_\alpha$ (resp. $U_{-\alpha}$) is a subgroup of $G$ canonically endowed with a vector bundle structure, and we have
\[\inn(t)(x)=x^{\alpha(t)},\quad (\text{resp.}\quad \inn(t)(x)=x^{-\alpha(t)})\]
for any $S'\to S$, $t\in T(S')$, $x\in U_\alpha(S')$ (resp. $x\in U_{-\alpha}(S')$). Finally, we have canonical isomorphisms
\[T\cdot U_\alpha\cong T\cdot_\alpha U_\alpha,\quad T\cdot U_{-\alpha}=T\cdot_{-\alpha}U_{-\alpha},\]
and the open subset $\Omega$ is stable under $\inn(T)$:
\begin{equation}\label{scheme group elementary system exp morphism conjugate of Omega formula}
\inn(t')(y\cdot t\cdot x)=y^{-\alpha(t')}\cdot t\cdot x^{-\alpha(t')}.
\end{equation}

\begin{corollary}\label{scheme group elementary system Lie algebra of U_alpha prop}
We have $\mathfrak{Lie}(U_\alpha/S)=\g^\alpha$ and $\mathfrak{Lie}(U_{-\alpha}/S)=\g^{-\alpha}$, and the isomorphisms
\[\exp:\mathbf{W}(\g^\alpha)\stackrel{\sim}{\to} U_\alpha,\quad \exp:\mathbf{W}(\g^{-\alpha})\stackrel{\sim}{\to} U_{-\alpha}\]
are those of \cref{scheme smooth vector bundle exp morphism}.
\end{corollary}

\begin{corollary}\label{scheme group elementary system image Omega relative schematically dense}
The open subset $\Omega$ is relativelly schematically dense in $G$ (cf. \cite{SGA3-2} \Rmnum{18}, \S 1).
\end{corollary}
\begin{proof}
This follows from (\cite{SGA3-2} \Rmnum{18}, 1.3) as the fiber $\Omega_s$ for any $s\in S$ is dense in $G_s$ (\cite{Chevalley1958}, \S 13.4, cor.2 of th.3).
\end{proof}

\begin{corollary}\label{scheme group elementary system center is kernel of root}
The center of $G$ is $Z(G)=\ker\alpha$, which is therefore a closed subgroup of $G$, of multiplicative finite type.
\end{corollary}
\begin{proof}
The second assertion follows from the first one according to \cref{scheme group multiplicative morphism factorization}. To prove the first one, we note that the inner automorphisms defined by sections of $\ker\alpha$ acts trivially on $\Omega$ (the formula (\ref{scheme group elementary system exp morphism conjugate of Omega formula})), hence over $G$ by \cref{scheme group elementary system image Omega relative schematically dense}. Conversely, if $g\in G(S)$ centralizes $G$, then it centralizes $T$ and $U_\alpha$, hence is a section of $T$ (\cref{scheme alg group reductive prop}~(b)), where $\alpha$ vanishes (as $\exp$ is a monomorphism, $\alpha(g)$ then induces the trivial automorphism on $\G_{m,S}$, whence is trivial). Since this is true over any base change, we conclude that $Z(G)=\ker\alpha$.
\end{proof}

\begin{corollary}\label{scheme group elementary system monomorphism from G_a iff g^alpha free}
For that there exists a monomorphism $p_\alpha:\G_{a,S}\to G$ normalized by $T$ with multiplicator $\alpha$, it is necessary and sufficient that the $\mathscr{O}_S$-module $\g^\alpha$ is free. More precisely, we have a bijection fiven by
\[X_\alpha\mapsto (x\mapsto\exp(xX_\alpha))\And p_\alpha\mapsto\mathfrak{Lie}(p_\alpha)\]
between $\Gamma(S,\g^\alpha)^\times$ and the set of monomorphisms $p_\alpha$ as above (which is also the set of isomorphisms of vectorial groups $\G_{a,S}\stackrel{\sim}{\to} U_\alpha$)\footnote{In fact, $\mathfrak{Lie}(\mathscr{O}_S)=\mathscr{O}_S$ and $\mathfrak{Lie}(p_\alpha)$ is an element of $\Hom_{\mathscr{O}_S}(\mathscr{O}_S,\g^\alpha)=\Gamma(S,\g^\alpha)$.}.
\end{corollary}

\begin{corollary}\label{scheme group elementary system U_alpha U_-alpha noncommutative}
The subgroups $U_\alpha$ and $U_{-\alpha}$ of $G$ do not commute over each fiber.
\end{corollary}
\begin{proof}
If $(U_\alpha)_s$ and $(U_{-\alpha})_s$ commutes, then $\Omega_s$ is a subgroup of $G_s$ (hence closed in $G_s$), whence $\Omega_s=G_s$ (\cref{scheme group elementary system image Omega relative schematically dense}) and $G_s$ is solvable\footnote{An easy calculation, using the formula (\ref{scheme group elementary system exp morphism conjugate of Omega formula}), shows that the commutator of $G_s$ is contained in $U_\alpha\cdot U_{-\alpha}$, which is a commutative group.}, which contradicts the hypothesis that $G_s$ is reductive of semi-simple rank $1$. 
\end{proof}

\begin{example}\label{scheme group elementary system GL_2 example}
Let $G=\GL_{2,k}$ be the general linear group of order $2$ over an algebraically closed field $k$, which is reductive of semi-simple rank $1$. Let $T$ be the maximal torus of $G$ formed by diagonal matrices in $G$. There is then a root $\alpha$ of $G$ relative to $T$, defined by
\[\alpha\begin{pmatrix}
a&0\\
0&d
\end{pmatrix}=ad^{-1},\]
and the Lie algebra $\g=\mathfrak{Lie}(G)=\gl_2$ decomposes into a direct sum $\g=\t\oplus\g^\alpha\oplus\g^{-\alpha}$, where
\[\t=\left\{\begin{pmatrix}
a&0\\
0&d
\end{pmatrix}\right\},\quad \g^\alpha=\left\{\begin{pmatrix}
0&b\\
0&0
\end{pmatrix}\right\},\quad \g^{-\alpha}=\left\{\begin{pmatrix}
0&0\\
c&0
\end{pmatrix}\right\}.\]
The exponential morphisms $\exp:\mathbf{W}(\g^{-\alpha})\to G$ and $\exp:\mathbf{W}(\g^{\alpha})\to G$ are therefore given by\footnote{Note that these coincide with the exponential morphism of matrices over $\R$ or $\C$, defined by $X\mapsto\sum_{n=0}^{\infty}\frac{X^n}{n!}$.}
\[\exp\begin{pmatrix}
0&0\\
c&0
\end{pmatrix}=\begin{pmatrix}
1&0\\
c&1
\end{pmatrix},\quad \exp\begin{pmatrix}
0&1\\
0&0
\end{pmatrix}=\begin{pmatrix}
1&b\\
0&1
\end{pmatrix}\]
and it is immedaitely verified that the conditions of \cref{scheme group elementary system exp morphism} are verified. The image $\Omega$ is therefore generateed by matrices of the form
\[\begin{pmatrix}
1&0\\
c&1
\end{pmatrix}\begin{pmatrix}
a&0\\
0&d
\end{pmatrix}\begin{pmatrix}
1&b\\
0&1
\end{pmatrix}=\begin{pmatrix}
a&ab\\
ac&abc+d
\end{pmatrix}\]
where $ad\neq 0$. Finally, we note that the center of $G$ is equal to the kernel of $\alpha$, formed by scalar matrices in $G$, and that $U_\alpha$ and $U_{-\alpha}$ do not commute.
\end{example}

\subsection{Structure of elementary systems}
\begin{theorem}\label{scheme group elementary system canonical pair prop}
Let $S$ be a scheme, $(G,T,\alpha)$ be an elementary $S$-system. Then there exists a morphism of $\mathscr{O}_S$-modules
\[\g^\alpha\otimes_{\mathscr{O}_S}\g^{-\alpha}\to\mathscr{O}_S,\quad (X,Y)\mapsto\langle X,Y\rangle\]
and a morphism of $S$-groups (called the \textbf{coroot associated to $\alpha$})
\[\check{\alpha}:\G_{m,S}\to T\]
such that for any $S'\to S$ and any $X\in\Gamma(S',\g^\alpha\otimes_{\mathscr{O}_S}\mathscr{O}_{S'})$, $Y\in\Gamma(S',\g^{-\alpha}\otimes_{\mathscr{O}_S}\mathscr{O}_{S'}$), we have the equivalence\footnote{Compare this with the definition (\ref{scheme group elementary system exp morphism-1}), in which we have $\mathbf{W}(\g^{-\alpha})$ on the left, and $\mathbf{W}(\g^\alpha)$ on the right.}:
\[\exp(X)\cdot\exp(Y)\in\Omega(S')\iff 1+\langle X,Y\rangle\in\G_m(S'),\]
and under these conditions, we have the formula:
\begin{equation}\label{scheme group elementary system canonical pair prop-1}
\exp(X)\cdot\exp(Y)=\exp\Big(\frac{Y}{1+\langle X,Y\rangle}\Big)\check{\alpha}(1+\langle X,Y\rangle)\exp\Big(\frac{X}{1+\langle X,Y\rangle}\Big).
\end{equation}
Moreover, the morphisms $(X,Y)\mapsto\langle X,Y\rangle$ and $\check{\alpha}$ are uniquely determined, and $\langle\cdot\,,\cdot\rangle$ is nondegenerate, hence realizes $\g^\alpha$ and $\g^{-\alpha}$ under duality, and we have $\alpha\circ\check{\alpha}=2$\footnote{For morphisms $\alpha:T\to\G_{m,S}$ and $\beta:\G_{m,S}\to T$, the composition $\alpha\circ\beta$ is an endomorphism of $\G_{m,S}$, hence is determined by an integer $n\in\Z$. We often identity $\alpha\circ\beta$ with this integer, which is also determined by
\[\mathfrak{Lie}(\alpha\circ\beta)=n\cdot\id_{\G_a}.\]}.
\end{theorem}

Given the assertions of uniqueness of the theorem, it suffices to prove it locally over $S$. We can therefore assume that $\g^\alpha$ and $\g^{-\alpha}$ are free over $S$. We then take $X\in\Gamma(S,\g^\alpha)^\times$, $Y\in\Gamma(S,\g^{-\alpha})^\times$, and put 
\[p_\alpha(x)=\exp(xX),\quad p_{-\alpha}(y)=\exp(yY)\]
for $S'\to S$, $x,y\in\G_a(S')$. By \cref{scheme group elementary system exp morphism} and \cref{scheme group elementary system U_alpha U_-alpha noncommutative}, it then suffices to prove the following more general result:

\begin{proposition}\label{scheme group morphism G_a normalized by character induced cocharacter prop}
Let $S$ be a scheme, $G$ be an $S$-group, $T$ be a torus of $G$, $\alpha$ be a character of $T$ nontrivial over each fiber, and $p_\alpha:\G_{a,S}\to G$ (resp. $p_{-\alpha}:\G_{a,S}\to G$) be a monomorphism of groups normalized by $T$ with multiplicator $\alpha$ (resp. $-\alpha$). Suppose that:
\begin{enumerate}
    \item[(a)] The morphism $\G_{a,S}\times_ST\times_S\G_{a,S}\to G$ defined by $(y,t,x)\mapsto p_{-\alpha}(y)tp_\alpha(x)$ is an open immersion, we denote by $\Omega$ its image.
    \item[(b)] For any $s\in S$, $(p_\alpha)_s(\G_{a,\kappa(s)})$ and  $(p_\alpha)_s(\G_{a,\kappa(s)})$ do not commute.
\end{enumerate}
Then there exists $z\in\G_a(S)$ and $\check{\alpha}\in\Hom_{S\dash\Grp}(\G_{m,S},T)$, uniquely determined by the following properties: for any $S'\to S$ and any $x,y\in\G_a(S')$, we have
\[p_\alpha(x)p_{-\alpha}(y)\in\Omega(S')\iff 1+zxy\in\G_m(S'),\]
and under this condition, we have the formula
\[p_\alpha(x)p_{-\alpha}(y)=p_{-\alpha}\Big(\frac{y}{1+zxy}\Big)\check{\alpha}(1+zxy)p_\alpha\Big(\frac{x}{1+zxy}\Big).\]
Moreover, $z\in\G_m(S)$, the morphism $(X,Y)\mapsto z$ is bilinear, and $\alpha\circ\check{\alpha}=2$.
\end{proposition}

\begin{corollary}\label{scheme group elementary system product of Omega char}
Let $\exp(Y)\cdot t\cdot \exp(X)$ and $\exp(Y')\cdot t'\cdot\exp(X')$ be two elements of $\Omega(S')$. Then their product is in $\Omega(S')$ if and only if $u=1+\langle X,Y'\rangle$ is invertible, and we then have
\begin{equation}\label{scheme group elementary system product of Omega char-1}
\exp(Y)t\exp(X)\cdot \exp(Y')t'\exp(X')=\exp(Y+u^{-1}\alpha(t)^{-1}Y')\cdot tt'\check{\alpha}(u)\cdot\exp(u^{-1}\alpha(t')^{-1}X+X').
\end{equation}
\end{corollary}
\begin{proof}
In fact, the product of $\exp(Y)\cdot t\cdot \exp(X)$ and $\exp(Y')\cdot t'\cdot\exp(X')$ is in $\Omega(S')$ if and only if $\exp(X)\cdot\exp(Y')$, so the assertion follows from \cref{scheme group elementary system canonical pair prop}.
\end{proof}

We can also write the formula (\ref{scheme group elementary system canonical pair prop-1}) of \cref{scheme group elementary system canonical pair prop} without involving the morphisms $\exp$. In fact, transporting by the duality duality $\g^\alpha\otimes\g^{-\alpha}\to\mathscr{O}_S$, we obtain a canonical coupling of vector bundles
\[U_\alpha\times_SU_{-\alpha}\to\G_{a,S},\quad (x,y)\mapsto\langle x,y\rangle,\]
so that we have
\[\langle\exp(X),\exp(Y)\rangle=\langle X,Y\rangle.\]
If $x\in U_\alpha(S')$, $y\in U_{-\alpha}(S')$ and $1+\langle x,y\rangle\in\G_m(S')$, then
\begin{equation}\label{scheme group elementary system canonical pair prop-2}
x\cdot y=y^{(1+\langle x,y\rangle)^{-1}}\cdot\check{\alpha}(1+\langle x,y\rangle)\cdot x^{(1+\langle x,y\rangle)^{-1}}.
\end{equation}
Since the paring $\g^\alpha\times\g^{-\alpha}\to\mathscr{O}_S$ is nondegenerate, we easily see that the coupling on $\mathbf{W}(\g^\alpha)$ and $\mathbf{W}(\g^{-\alpha})$ is also nondegenerate. We therefore obtain the following corollary:

\begin{corollary}\label{scheme group elementary system coupling on vector bundle prop}
The coupling
\[\mathbf{W}(\g^\alpha)\times_S\mathbf{W}(\g^{-\alpha})\to \G_{a,S}\]
defines a coupling of pincipal fiber bundles under $\G_{m,S}$:
\[\mathbf{W}(\g^\alpha)^\times\times_S\mathbf{W}(\g^{-\alpha})^\times\to\G_{m,S}.\]
This coupling is also denoted by $(X,Y)\mapsto\langle X,Y\rangle$, or simply $(X,Y)\mapsto XY$. For any section $X\in\Gamma(S,\g^\alpha)^\times$, there then exists a unique section $X^{-1}$ of $\Gamma(S,\g^{-\alpha})^\times$ such that $XX^{-1}=1$ (called the \textbf{dual section} of $X$), and we have $(zX)^{-1}=z^{-1}X^{-1}$. The morphism
\[s:\mathbf{W}(\g^\alpha)^\times\to\mathbf{W}(\g^{-\alpha})^\times\]
thus defined is an isomorphism of schemes, compatible with the isomorphism $s:z\mapsto z^{-1}$ of $\G_{m,S}$.
\end{corollary}

\begin{remark}\label{scheme group elementary system coupling as multiplication}
As the paring $\langle\cdot\,,\cdot\rangle$ is nondegenerate and $\g^\alpha$, $\g^{-\alpha}$ are locally free of rank $1$, we may think $X^{-1}$ as the "inverse" of $X$: for example, we have 
\[XY^{-1}=(YX^{-1})^{-1},\quad XY^{-1}\cdot Y=X.\]
In particular, this allows us to endow $\g^\alpha$, $\g^{-\alpha}$ with a "multiplication" structure, and we can therefore think $X,Y$ as scalars (i.e. we can think them as sections in $\G_{m,S}$). This structure is compatible with the action of $\G_{m,S}$ since we have $(zX)^{-1}=z^{-1}X^{-1}$.
\end{remark}

\begin{remark}\label{scheme group elementary system contragredient of p_alpha char}
With the coupling of \cref{scheme group elementary system coupling on vector bundle prop}, let $X\in\Gamma(S,\g^\alpha)^\times$ and consider morphism $p_\alpha:\G_{a,S}\to\mathbf{W}(\g^\alpha)$ by
\[p_\alpha(x)=\exp(xX).\]
If we identity the dual of $\g^\alpha$ with $\g^{-\alpha}$ by the pairing $\langle\cdot\,,\cdot\rangle$ and consider the canonical pairing of $\G_a$ defined by $(x,y)\mapsto xy$, then the contragredient of $p_\alpha$ is given by
\[p_{-\alpha}(y)=\exp(yX^{-1}).\]
To see this, it suffices to note that we have the equality
\[\langle x,y\rangle=xy=\langle \exp(xX),\exp(yX^{-1})\rangle=\langle p_\alpha(x),p_{-\alpha}(y)\rangle,\]
which implies that $p_\alpha=p_{-\alpha}^\vee$.
\end{remark}

\begin{remark}
Applying \cref{scheme group elementary system product of Omega char} to $Y=0=X'$, $t=t'=1$ and $Y'=aX^{-1}$ (with $a\in\mathscr{O}_S(S)$), then $u=1+a$ and
\[u^{-1}Y'=u^{-1}(u-1)X^{-1}=(1-u^{-1})X^{-1},\]
and we obtain the following:
\begin{equation}\label{scheme group elementary system alpha^*(u) formula}
\check{\alpha}(u)=\exp\big((u^{-1}-1)X^{-1}\big)\exp(X)\exp((u-1)X^{-1})\exp(-u^{-1}X).
\end{equation}
Since any $u\in\Gamma(S,\mathscr{O}_S)^\times$ can be written as $u=1+a$, the above formula hold for any such $u$.
\end{remark}

\begin{remark}
If $(G,T,\alpha)$ is an elementary $S$-system, $(G,T,-\alpha)$ is also an elementary $S$-system, and we then obtain a duality between $\g^{-\alpha}$ and $\g^{-\alpha}$, and a coroot $(-\alpha)\rcheck$. By the inverse of formula (\ref{scheme group elementary system canonical pair prop-1}), we find that
\[\langle X,Y\rangle=\langle Y,X\rangle,\quad (-\alpha)\rcheck=-\check{\alpha}.\]
We now turn to the Lie algebra $\g$ of $G$. The root $\alpha$ and the coroot $\check{\alpha}$ define linear forms
\[\begin{tikzcd}
\mathscr{O}_S\ar[r,"\check{\alpha}_*"]&\t\ar[r,"\alpha_*"]&\mathscr{O}_S
\end{tikzcd}
\]
We write $H_\alpha=\check{\alpha}_*(1)$. The linear form $\alpha_*$ is called the \textbf{infinitesimal root associated with $\alpha$}, and $H_\alpha$ is called the corresponding \textbf{infinitesimal coroot}.
\end{remark}

\begin{proposition}\label{scheme group elementary system Ad action formula by exp}
Let $S'\to S$ and $X,X'\in\mathbf{W}(\g^\alpha)(S')$, $Y,Y'\in\mathbf{W}(\g^{-\alpha})(S')$, $H\in\mathbf{W}(\t)(S')$ and $t\in T(S')$. Then we have
\begin{gather}
\Ad(t)H=H,\quad \Ad(t)X=\alpha(t)X,\quad \Ad(t)Y=\alpha(t)^{-1}Y,\label{scheme group elementary system Ad action formula by exp-1}\\
\begin{cases*}
\Ad(\exp(X))H=H-\alpha_*(H)X,\quad \Ad(\exp(X))X'=X',\\
\Ad(\exp(X))Y=Y+\langle X,Y\rangle H_\alpha-\langle X,Y\rangle X.
\end{cases*}
\label{scheme group elementary system Ad action formula by exp-2}\\
\begin{cases*}
\Ad(\exp(Y))H=H-\alpha_*(H)X,\quad \Ad(\exp(Y))Y'=Y',\\
\Ad(\exp(Y))X=X+\langle X,Y\rangle H_{-\alpha}-\langle X,Y\rangle Y.
\end{cases*}
\label{scheme group elementary system Ad action formula by exp-3}\\
[H,X]=\alpha_*(H)X,\quad [H,Y]=-\alpha_*(H)Y,\quad [X,Y]=\langle X,Y\rangle H_\alpha,\label{scheme group elementary system Ad action formula by exp-4}\\
H_{-\alpha}=-H_\alpha,\label{scheme group elementary system Ad action formula by exp-5}\\
\alpha_*(H_\alpha)=2.\label{scheme group elementary system Ad action formula by exp-6}
\end{gather}
\end{proposition}
\begin{proof}
The proof of (\ref{scheme group elementary system Ad action formula by exp-1}) and (\ref{scheme group elementary system Ad action formula by exp-5}) is trivial, and (\ref{scheme group elementary system Ad action formula by exp-6}) follows from the equality $\alpha\circ\check{\alpha}=2$. On the other hand, the equality in (\ref{scheme group elementary system Ad action formula by exp-2}) or (\ref{scheme group elementary system Ad action formula by exp-3}) follows by taking differential from (\ref{scheme group elementary system exp morphism conjugate of Omega formula}), the second is trivial, and the last one follows by taking differential from (\ref{scheme group elementary system canonical pair prop-1}).
\end{proof}

\begin{corollary}\label{scheme group elementary system section dual iff bracket is H_alpha}
Suppose that $H_\alpha$ is nonzero over each fiber (which is in particular the case if $2$ is invertible over $S$, by (\ref{scheme group elementary system Ad action formula by exp-6})). Then $X_\alpha\in\Gamma(S,\g^\alpha)^\times$ and $X_{-\alpha}\in\Gamma(S,\g^{-\alpha})$ are dual with each other if and only if $[X_\alpha,X_{-\alpha}]=H_\alpha$.
\end{corollary}

\begin{example}
Consider the example \cref{scheme group elementary system GL_2 example}. Then the coroot $\check{\alpha}:\G_{m,S}\to T$ is given by
\[\check{\alpha}(z)=\begin{pmatrix}
z&0\\
0&z^{-1}
\end{pmatrix},\]
and the pairing on $\g^\alpha\times_k\g^{-\alpha}$ is
\[\left(\begin{pmatrix}
0&x\\
0&0
\end{pmatrix},\begin{pmatrix}
0&0\\
y&0
\end{pmatrix}\right)\mapsto xy.\]
In fact, the product $(\begin{smallmatrix}1&x\\0&1\end{smallmatrix})\cdot(\begin{smallmatrix}1&0\\y&1\end{smallmatrix})$ belongs to the image $\Omega$ if and only if $1+xy\neq 0$, and in this case we have
\[\begin{pmatrix}
1&x\\
0&1
\end{pmatrix}\begin{pmatrix}
1&0\\
y&1
\end{pmatrix}=\begin{pmatrix}
1+xy&x\\
y&1
\end{pmatrix}=\begin{pmatrix}
1&0\\
\frac{y}{1+xy}&1
\end{pmatrix}\begin{pmatrix}
1+xy&0\\
0&(1+xy)^{-1}
\end{pmatrix}\begin{pmatrix}
1&\frac{x}{1+xy}\\
0&1
\end{pmatrix}.\]
Also, the composition $\alpha\circ\check{\alpha}$ sends $z$ to $z^2$, which justifies \cref{scheme group elementary system canonical pair prop}. For a matrix $(\begin{smallmatrix}0&x\\0&0\end{smallmatrix})\in\g^\alpha$, its dual is then given by $(\begin{smallmatrix}0&0\\x^{-1}&0\end{smallmatrix})$. In this case, the formula (\ref{scheme group elementary system alpha^*(u) formula}) reads
\[\begin{pmatrix}
u&0\\
0&u^{-1}
\end{pmatrix}=\begin{pmatrix}
1&0\\
(u^{-1}-1)x^{-1}&1
\end{pmatrix}\begin{pmatrix}
1&x\\
0&1
\end{pmatrix}\begin{pmatrix}
1&0\\
(u-1)x^{-1}&1
\end{pmatrix}\begin{pmatrix}
1&-u^{-1}x\\
0&1
\end{pmatrix}\]
which can be justified by an immediate calculation.\par
Finally, the infinitesimal root $\alpha_*$ and $\check{\alpha}_*$ of $G$ have the following form:
\[\alpha_*\begin{pmatrix}
a&0\\
0&b
\end{pmatrix}=a-b,\quad \check{\alpha}_*(z)=\begin{pmatrix}
z&0\\
0&-z
\end{pmatrix}\] 
and in particular $H_\alpha=(\begin{smallmatrix}1&0\\0&-1\end{smallmatrix})$. The formulas in \cref{scheme group elementary system Ad action formula by exp} can then be varifed: For example, for $X=(\begin{smallmatrix}0&x\\
0&0\end{smallmatrix})$, $H=(\begin{smallmatrix}a&0\\0&b\end{smallmatrix})$ and $Y=(\begin{smallmatrix}0&0\\y&0\end{smallmatrix})$, we have $[X,Y]=(\begin{smallmatrix}
xy&0\\0&xy
\end{smallmatrix})=\langle X,Y\rangle H_\alpha$, and
\[\Ad(\exp(X))(H)=\begin{pmatrix}
a&(b-a)x\\
0&b
\end{pmatrix}=\begin{pmatrix}
a&0\\
0&b
\end{pmatrix}-\alpha_*\begin{pmatrix}
a&0\\
0&b
\end{pmatrix}\cdot \begin{pmatrix}
0&x\\
0&0
\end{pmatrix}=H-\alpha_*(H)X,\]
and we have $\alpha_*(H_\alpha)=2$.
\end{example}

Let $(G,T,\alpha)$ be an elementary $S$-system. We have seen in \cref{scheme group elementary system center is kernel of root} that the center of $G$ is $Z(G)=\ker\alpha$, which is of multiplicative finite type. If $Q$ is a subgroup of multiplicative type of $Z(G)$, the quotient $G'=G/Q$ is affine (\cref{scheme group mono multiplicative to affine prop}) and smooth (as $Q$ is smooth) over $S$, with reductive connected fibers of semi-simple rank $1$ (\ref{scheme alg group rho_s invariant under quotient by central torus}). Put $T'=T/Q$, which is a maximal torus of $G'$. The open subset $U_{-\alpha}\cdot T\cdot U_\alpha$ of $G$ is stable under $Q$ and we easily see that the quotient is isomorphic to $U_{-\alpha}\times_S(T/Q)\times_SU_\alpha$. If we denote by $\alpha'$ the character of $T'$ induced by $\alpha$, it then follows that the morphism derived from the canonical morphism $G\to G'$ induces isomorphisms
\[\g^\alpha\stackrel{\sim}{\to}\g'^\alpha,\quad \g^{-\alpha}\stackrel{\sim}{\to}\g'^{-\alpha}.\]
In particular, $\alpha'$ is a root of $G'$ relative to $T'$, hence, denote by $\alpha/Q$ the character $T/Q\to\G_{m,S}$ induced by $\alpha$, the triple $(G/Q,T/Q,\alpha/Q)$ is an elementary $S$-system. We also note that under the preceding conditions, the following diagrams are commutative:
\[\begin{tikzcd}
\mathbf{W}(\g^\alpha)\ar[r,"\exp"]\ar[d,"\sim"]&G&\mathbf{W}(\g^{-\alpha})\ar[l,swap,"\exp"]\ar[d,"\sim"]\\
\mathbf{W}(\g'^{\alpha'})\ar[r,"\exp"]&G'&\mathbf{W}(\g'^{-\alpha'})\ar[l,swap,"\exp"]
\end{tikzcd}\quad\quad \begin{tikzcd}
\g^\alpha\otimes\g^{-\alpha}\ar[r,"\sim"]\ar[d,"\sim"]&\mathscr{O}_S\ar[d,"\id"]\\
\g'^{\alpha'}\otimes\g'^{-\alpha'}\ar[r,"\sim"]&\mathscr{O}_S
\end{tikzcd}\quad\quad \begin{tikzcd}[row sep=6mm, column sep=6mm]
&T\ar[rd,"\alpha"]\ar[dd]&\\
\G_{m,S}\ar[ru,swap,"\check{\alpha}"]\ar[rd,swap,"\check{\alpha}'"]&&\G_{m,S}\\
&T'\ar[ru,swap,"\alpha'"]&
\end{tikzcd}\]

\subsection{The Weyl group}\label{scheme group elementary system Weyl group subsection}
If $(G,T,\alpha)$ is an elementary $S$-system, we put
\[N=N_G(T),\quad W=N_G(T)/T.\]
Then $N$ is a closed subgroup of $G$, smooth over $S$. We denote by $N^\times=N-T$ the open subscheme of $N$ induced over the complementary of $T$. Let $R$ be the (unique) maximal torus of $\ker\alpha$, and $T'$ be the image of $\check{\alpha}:\G_{m,S}\to T$, which is a subtorus of dimension $1$ of $T$. The morphism
\[T'\times_SR\to T\]
induced by the product in $T$ is surjective (hence faithfully flat); in fact, we are reduced to verify this over the geometric fibers, and this follows easily from the formula $\alpha\circ\check{\alpha}=2$.

\begin{theorem}\label{scheme group elementary system Weyl group prop}
With the preceding notations:
\begin{enumerate}
    \item[(\rmnum{1})] $W$ is isomorphic to the constant group $(\Z/2\Z)_S$.
    \item[(\rmnum{2})] $N^\times$ is a locally trivial principal fiber bundle under $T$ with left the action $(t,q)\mapsto tq$ (resp. the right action $(q,t)\mapsto qt$).
    \item[(\rmnum{3})] We have the formula
    \[\inn(w)t=t\cdot\check{\alpha}(\alpha(t)^{-1})\]
    for $w\in N^\times(S')$, $t\in T(S')$, $S'\to S$. In the decomposition $T_{S'}=T'_{S'}\cdot R_{S'}$, $\inn(w)$ induces the identity on $R_{S'}$ and the symmetry on $T'_{S'}$. We have the relations
    \[\alpha\circ\inn(w)=-\alpha,\quad \inn(w)\circ\check{\alpha}=-\check{\alpha}.\]
    \item[(\rmnum{4})] For any $X\in\mathbf{W}(\g^\alpha)^\times(S')$, $S'\to S$, put
    \[w_\alpha(X)=\exp(X)\exp(-X^{-1})\exp(X).\]
    Then $w_\alpha(X)\in N^\times(S')$ and the morphism $w_\alpha:\mathbf{W}(\g^\alpha)^\times\to N^\times$ thus defined verifies
    \[w_\alpha(zX)=\check{\alpha}(z)w_\alpha(X)=w_\alpha(X)\check{\alpha}(z)^{-1},\]
    for $z\in\G_m(S')$, $X\in\mathbf{W}(\g^\alpha)^\times(S')$, $S'\to S$.
    \item[(\rmnum{5})] For $X,X'\in\mathbf{W}(\g^\alpha)^\times$, we have the relation
    \[w_\alpha(X)w_\alpha(X')=\check{\alpha}(-XX'^{-1}).\]
    In particular,
    \[w_\alpha(X)^2=\check{\alpha}(-1)\in{_2T}(S)\cap Z(G)(S),\quad w_\alpha(X)^{-1}=w_\alpha(-X)=\check{\alpha}(-1)w_\alpha(X).\]
    \item[(\rmnum{6})] If we define similarly for $Y\in\mathbf{W}(\g^{-\alpha})^\times(S')$,
    \[w_{-\alpha}(Y)=\exp(Y)\exp(-Y^{-1})\exp(Y),\]
    we have 
    \[w_{-\alpha}(X^{-1})=w_\alpha(X)^{-1}=w_\alpha(-X),\quad w_\alpha(X)w_{-\alpha}(Y)=\check{\alpha}(XY).\]  
\end{enumerate}
\end{theorem}
\begin{proof}
The assertion in (\rmnum{1}) has been proved in \cref{scheme group reductive locus Weyl group Z/2Z prop}, and it follows easily that $N^\times$ is a locally trivial principal fiber bundle under $T$ for the actions defined in (\rmnum{2}). The fact that it is locally trivial follows in particular from the formula in (\rmnum{4}).\par
If $w\in N^\times(S)$, it is clear that $\alpha\circ\inn(w)$ is a root of $G$ relative to $T$, which is therefore locally equal to $\alpha$ or $-\alpha$. As over each fiber this is equal to $-\alpha$ (\cite{Chevalley1958} 12-05, d\'emonstration du cor. a la prop.1), we have $\alpha\circ\inn(w)=-\alpha$. By transporting the structure, we then deduce
\[-\check{\alpha}=\inn(w)^{-1}\circ\check{\alpha}=\inn(w)\circ\check{\alpha}\]
because $\inn(w)^2=\inn(w^2)$ and $w^2$ is a section of $T$. Therefore $\inn(w)$ induces the symmetry over $T'$; as $R$ is central, $\inn(w)$ induces the identity over $R$. The formula of (\rmnum{3}) defines a morphism $T\to T$ which verifies the same properties, which therefore coincides with $\inn(w)$.\par
To prove (\rmnum{4}), we note that
\begin{align*}
w_\alpha(X)tw_\alpha(X)^{-1}&=\exp(X)\exp(-X^{-1})\exp(X)t\exp(-X)\exp(X^{-1})\exp(-X)\\
&=\exp(X)\exp(-X^{-1})\exp(X-\alpha(t)X)\exp(\alpha(t)^{-1}X^{-1})\exp(-\alpha(t)X)t.
\end{align*}
By applying the formula (\ref{scheme group elementary system canonical pair prop-1}), we have
\[\exp(-X^{-1})\exp((1-\alpha(t))X)=\exp((\alpha(t)^{-1}-1)X)\check{\alpha}(\alpha(t)^{-1})\exp(-\alpha(t)^{-1}X^{-1}).\]
Plug this into the preceding relation, we find that
\begin{align*}
\inn(w_\alpha(X))t&=\exp(\alpha(t)^{-1}X)\check{\alpha}(\alpha(t)^{-1})\exp(-\alpha(t)X)t\\
&=\exp(aX)\check{\alpha}(\alpha(t)^{-1})t,
\end{align*}
where $a=\alpha(t)^{-1}-(\alpha\circ\check{\alpha})(\alpha(t)^{-1})\alpha(t)$. Since $\alpha\circ\check{\alpha}=2$, we see that $a=0$, so $w_\alpha(X)\in N^\times(S')$. As for the second assertion of (\rmnum{4}), we have\footnote{The first equality follows from \cref{scheme group elementary system exp morphism}~(a), which, combined with the equality $\alpha\circ\check{\alpha}=2$, gives the formula
\begin{equation}\label{scheme group elementary system Weyl group prop-1}
\check{\alpha}(z)\exp(X)\check{\alpha}(z)^{-1}=\exp(z^2X),\quad \check{\alpha}(z)\exp(X^{-1})\check{\alpha}(z)^{-1}=\exp(z^{-2}X^{-1}),
\end{equation}
the third equality follows from the formula (\ref{scheme group elementary system canonical pair prop-1}), and the foruth from (\ref{scheme group elementary system Weyl group prop-1}) again.
}
\begin{align*}
\check{\alpha}(z)w_\alpha(X)&=\exp(z^2X)\exp(-z^{-2}X^{-1})\exp(z^2X)\check{\alpha}(z)\\
&=\exp(zX)\exp((z^2-z)X)\exp(-z^{-2}X^{-1})\exp(z^2X)\check{\alpha}(z)\\
&=\exp(zX)\exp(-z^{-1}X^{-1})\check{\alpha}(z)^{-1}\exp((z^3-z^2)X)\exp(z^2X)\check{\alpha}\\
&=\exp(zX)\exp(-z^{-1}X^{-1})\exp(zX)=w_\alpha(zX).
\end{align*}
Finally, an analoguous calculation shows that $w_\alpha(X)\check{\alpha}(z^{-1})=w_\alpha(zX)$.\par
In view of the preceding results, the first formula in (\rmnum{5}) follows from the following computation\footnote{The second equality follows from the formula (\ref{scheme group elementary system product of Omega char-1}), and the third and fourth follows from (\ref{scheme group elementary system Weyl group prop-1}) and \cref{scheme group elementary system coupling as multiplication}.}:
\begin{align*}
w_\alpha(X)w_\alpha(X')&=\exp(X)\exp(-X^{-1})\exp(X+X')\exp(-X'^{-1})\exp(X')\\
&=\exp(X)\exp(-X^{-1}+(XX'^{-1})^{-1}X'^{-1})\check{\alpha}(-XX'^{-1})\exp(-(XX'^{-1})^{-1}(X+X')+X')\\
&=\exp(X)\exp(-XX'^{-1}(X+X')+(XX'^{-1})^2X')\check{\alpha}(-XX'^{-1})\\
&=\exp((1-XX'^{-1})X+((XX'^{-1})^2-(XX'^{-1}))X')\check{\alpha}(-XX'^{-1})=\check{\alpha}(-XX'^{-1}).
\end{align*}
This implies the other equalities in (\rmnum{5}).\par
Finally, we prove (\rmnum{6}). The first assertion is a parituclar case of the second one, so we only need to prove the second one. The two members in this formula define morphisms from $\mathbf{W}(\g^\alpha)^\times\times_S\mathbf{W}(\g^{-\alpha})^\times$ to $G$. To show that they coincide, it suffices to do this over a nonempty open subset of each fiber (\cite{SGA3-2} \Rmnum{18} 1.4); it then siffices to verify this relation for $1+XY$ invertible, and we then have:
\begin{align*}
w_\alpha(X)w_{-\alpha}(Y)&=\exp(X)\exp(-X^{-1})\exp(X)\exp(Y)\exp(-Y^{-1})\exp(Y)\\
&=\exp(X)\exp(-X^{-1})\exp\Big(\frac{Y}{1+XY}\Big)\check{\alpha}(1+XY)\exp\Big(\frac{X}{1+XY}\Big)\exp(-Y^{-1})\exp(Y)\\
&=\exp(X)\exp\Big(\frac{-X^{-1}}{1+XY}\Big)\check{\alpha}(1+XY)\exp\Big(\frac{-Y^{-1}}{1+XY}\Big)\exp(Y)\\
&=\exp\Big(-\frac{X^{-1}}{XY}\Big)\check{\alpha}\Big(\frac{XY}{1+XY}\Big)\exp\Big(\frac{X(1+XY)}{XY}\Big)\check{\alpha}(1+XY)\exp\Big(\frac{-Y^{-1}}{1+XY}\Big)\exp(Y)\\
&=\exp\Big(-\frac{X^{-1}}{XY}\Big)\check{\alpha}(XY)\exp\Big(\frac{X}{XY(1+XY)}\Big)\exp\Big(\frac{-Y^{-1}}{1+XY}\Big)\exp(Y)\\
&=\check{\alpha}(XY)\exp(-Y)\exp(Y)=\check{\alpha}(XY),
\end{align*}
and this completes the proof of \cref{scheme group elementary system Weyl group prop}.
\end{proof}

\begin{corollary}\label{scheme group elementary system section of N^times iff}
Let $n\neq 0$ be an integer. For any $w\in G(S)$, the following conditions are equivalent:
\begin{enumerate}
    \item[(\rmnum{1})] $w\in N^\times(S)$,
    \item[(\rmnum{2})] we have $\inn(w)\circ n\check{\alpha}=-n\check{\alpha}$.
\end{enumerate}
\end{corollary}
\begin{proof}
In view of \cref{scheme group elementary system Weyl group prop}~(\rmnum{3}), we have (\rmnum{1})$\Rightarrow$(\rmnum{2}). Conversely, we can suppose that $N^\times$ possesses a section, and we are then reduced to prove:
\begin{lemma}\label{scheme group elementary system centralizer of coroot is maximal torus}
We have $Z_G(n\check{\alpha})=T$ for $n\neq 0$.
\end{lemma}
In fact, the image $T'$ of $n\check{\alpha}$ is a sub-torus of $G$. It then follows from \cref{scheme alg group reductive prop}~(a) that $Z_G(n\check{\alpha})$ is a reductive subgroup of $G$, containing $T$. As over each fiber we have $Z_{G_{\bar{s}}}(n\check{\alpha}_{\bar{s}})\neq G_{\bar{s}}$, we conclude that $Z_{G_{\bar{s}}}(n\check{\alpha}_{\bar{s}})=T_{\bar{s}}$\footnote{In fact, the hypothesis $Z_{G_{\bar{s}}}(n\check{\alpha}_{\bar{s}})\neq G_{\bar{s}}$ implies that $\dim(Z_{G_{\bar{s}}}(\check{\alpha}_{\bar{s}}))-\dim(T_{\bar{s}})<2$, but this difference is even, in view of (\ref{scheme group reductive dim(G)-rho_r(G) is Card(R)}).}, hence $Z_G(n\check{\alpha})=T$, because it is a smooth subgroup of $G$.
\end{proof}

\begin{remark}
The construction of $w_\alpha$ and the fact that $w_\alpha(X)$ normalizes $T$ only reply on formula (\ref{scheme group elementary system canonical pair prop-1}). In particular, if $G$ is an $S$-group satisfying the conditions of \cref{scheme group morphism G_a normalized by character induced cocharacter prop}, then $N_G(T)$ is different from $T$ over each fiber. It follows that if $G$ is an affine $S$-group with connected fibers satisfying the conditions of \cref{scheme group morphism G_a normalized by character induced cocharacter prop}, then it is reductive of semi-simple rank $1$. In fact, it is smooth in a neighborhood of the unit section, therefore smooth and one can apply the criterion of \cref{scheme alg group smooth reductive with ss-rank 1 iff}.
\end{remark}

Before proceeding further, let us make a few remarks. We identify as usual $\g^{-\alpha}$ with $(\g^\alpha)^{-1}$. Similarly, we identity $\sHom_{\mathscr{O}_S}(\g^{-\alpha},\g^\alpha)$ to $(\g^\alpha)^{\otimes 2}$ and hence
\[\sIso_{\mathbb{O}_S}(\mathbf{W}(\g^{-\alpha}),\mathbf{W}(\g^\alpha))\cong\mathbf{W}((\g^\alpha)^{\otimes 2})^\times.\]
If $w\in N^\times(S)$, then $\Ad(w)$ permutes $\g^\alpha$ and $\g^{-\alpha}$ (\cref{scheme group elementary system Weyl group prop}~(\rmnum{3})), hence defines an isomorphism
\[a_\alpha(w):\g^{-\alpha}\stackrel{\sim}{\to}\g^\alpha,\]
which is therefore identified with a section $a_\alpha(w)\in\Gamma(S,(\g^\alpha)^{\otimes 2})^\times$. This construction is compatible with base change, and hence defines a morphism
\[a_\alpha:N^\times\to\mathbf{W}((\g^\alpha)^{\otimes 2})^\times,\]
such that $a_\alpha(w)(Y)=\Ad(w)Y$ for any $w\in N^\times(S')$, $Y\in\Gamma(S',\g^{-\alpha})^\times$, $S'\to S$.

\begin{theorem}\label{scheme group elementary system a_alpha morphism prop}
With the preceding notations:
\begin{enumerate}
    \item[(\rmnum{1})] For any $S'\to S$, $w\in N^\times(S')$, $Y\in\mathbf{W}(\g^{-\alpha})(S')$, we have
    \[\inn(w)\exp(Y)=\exp(a_\alpha(w)Y).\]
    \item[(\rmnum{2})] For any $S'\to S$, $w\in N^\times(S')$, $t\in T(S')$, we have
    \[a_\alpha(tw)=\alpha(t)a_\alpha(w),\quad a_\alpha(wt)=\alpha(t)^{-1}a_\alpha(w).\]
    \item[(\rmnum{3})] If we define similarly $a_{-\alpha}:N^\times\to W((\g^{-\alpha})^{\otimes 2})^\times$, then
    \[a_{-\alpha}(w)=a_\alpha(w)^{-1}.\]
    \item[(\rmnum{4})] For any $X\in\mathbf{W}(\g^\alpha)^\times(S')$, $S'\to S$, we have
    \[a_\alpha(w_\alpha(X))=-X^2.\]
\end{enumerate}
\end{theorem}
\begin{proof}
The first assertion is trivial, by the characterization of $\exp$ given in \cref{scheme group elementary system exp morphism}, and assertion (\rmnum{2}) is immediate since $\Ad(tw)=\Ad(t)\Ad(w)$, and so is (\rmnum{3}). We now prove (\rmnum{4}): let $X\in\Gamma(S',\g^\alpha)^\times$, $Z\in\Gamma(S',\g^\alpha)$; we have by definition:
\[a_\alpha(w_\alpha(X))^{-1}(Z)=\Ad(w_\alpha(X))(Z)=\Ad(\exp(X))\Ad(\exp(-X^{-1}))\Ad(\exp(X))Z.\]
Applying the formulas in \cref{scheme group elementary system Ad action formula by exp} and the fact that $XX^{-1}=1$, we see that the right hand side is equal to
\begin{align*}
\Ad(\exp(X))\Ad(\exp(-X^{-1}))(Z)&=\Ad(\exp(X))(Z+\langle X^{-1},Z\rangle(H_\alpha-X^{-1}))\\
&=Z+\langle X^{-1},Z\rangle(H_\alpha-2X-X^{-1}-H_\alpha+X)\\
&=Z-\langle X^{-1},Z\rangle X-\langle X^{-1},Z\rangle X^{-1}\\
&=Z-Z-X^{-2}Z=-X^{-2}Z.
\end{align*}
Therefore $a_\alpha(w_\alpha(X))^{-1}=-X^{-2}$, and hence $a_\alpha(w_\alpha(X))=-X^2$.
\end{proof}

\begin{corollary}\label{scheme group elementary system a_alpha exp prop}
In particular, we have
\[\inn(w_\alpha(X))\exp(X)=\exp(-X^{-1}),\]
whence (by the definition of $w_\alpha(X)$):
\[w_\alpha(X)\exp(X)w_\alpha(X)^{-1}=\exp(-X)w_\alpha(X)\exp(-X),\]
and $(w_\alpha(X)\exp(X))^3=e$ (cf. \cref{scheme group elementary system Weyl group prop}~(\rmnum{5})).
\end{corollary}

\begin{corollary}\label{scheme group elementary system (w exp(X))^3=e char}
Let $X\in\Gamma(S,\g^\alpha)^\times$ and $n\neq 0$ be an integer, then $w_\alpha(X)$ is the unique section that vefifies:
\begin{enumerate}
    \item[(a)] $\inn(w)\circ n\check{\alpha}=-n\check{\alpha}$.
    \item[(b)] $(w\exp(X))^3=e$.
\end{enumerate}
\end{corollary}
\begin{proof}
We have seen that $w_\alpha(X)$ satisfies these conditions. Conversely, let $w\in G(S)$ verify (a) and (b). By \cref{scheme group elementary system Weyl group prop}~(\rmnum{2}) and \cref{scheme group elementary system section of N^times iff}, we see that there exists $t\in T(S)$ such that $w=w_\alpha(X)t$. Put $u=\exp(X)$, we then have
\begin{align*}
wuw^{-1}&=w_\alpha(X)t\exp(X)t^{-1}w_\alpha(X)^{-1}=\inn(w_\alpha(X))\exp(\alpha(t)X)\\
&=\exp(a_\alpha(w_\alpha(X))\alpha(t)X)=\exp(-\alpha(t)X^{-1}).
\end{align*}
On the other hand, 
\begin{align*}
u^{-1}wu^{-1}&=\exp(-X)w_\alpha(X)t\exp(-X)\\
&=\exp(-X)w_\alpha(X)\exp(-\alpha(t)X)t\\
&=\exp(-X^{-1})\exp(X-\alpha(t)X)t.
\end{align*}
Now $(wu)^3=e$ if and only if $wuw^{-1}=u^{-1}wu^{-1}$, so by comparing the two decompositions of these elements in $U_{-\alpha}\cdot T\cdot U_\alpha$, we conclude that $t=e$.
\end{proof}

\begin{remark}
We can summarize the results of this subsection by the following diagram of homogeneous principal bundles (on the left):
\[\begin{tikzcd}
\mathbf{W}(\g^\alpha)^\times\ar[r,"w_\alpha"]&N^\times\ar[r,"a_\alpha"]&\mathbf{W}((\g^\alpha)^{\otimes 2})^\times\\
\G_{m,S}\ar[u]\ar[r,"\check{\alpha}"]&T\ar[r,"\alpha"]\ar[u]&\G_{m,S}\ar[u]
\end{tikzcd}\]
We remark that $a_\alpha$ is faithfully flat (since $\alpha$ is) and that $w_\alpha$ is a monomorphism if and only if $\check{\alpha}$ is a monomorphism. 
\end{remark}

\subsection{The existence theorem of morphisms}
\begin{proposition}\label{scheme group elementary system morphism preserving torus char}
Let $S$ be a scheme, $a\in\Z$, $q>0$ be an integer such that $x\mapsto x^q$ defines an endomorphism of $\G_{a,S}$, $(G,T,\alpha)$ and $(G',T',\alpha')$ be elementary $S$-systems, $f:G\to G'$ be a morphism of $S$-groups. Then the following conditions are equivalent:
\begin{enumerate}
    \item[(\rmnum{1})] The restriction of $f$ to $T$ factors into a morphism $f_T:T\to T'$ fitting into the following diagram:
    \[\begin{tikzcd}
    \G_{m,S}\ar[d,"q"]\ar[r,"\check{\alpha}"]&T\ar[r,"\alpha"]\ar[d,"f_T"]&\G_{m,S}\ar[d,"q"]\\
    \G_{m,S}\ar[r,"\check{\alpha}'"]&T'\ar[r,"\alpha'"]&\G_{m,S}
    \end{tikzcd}\]
    \item[(\rmnum{2})] There exists an (unique) isomorphism of $\mathscr{O}_S$-modules
    \[h:(\g^\alpha)^{\otimes q}\to \g'^{\alpha'}\]
    such that for any $S'\to S$, $X\in\mathbf{W}(\g^\alpha)(S')$ and $Y\in\mathbf{W}(\g^{-\alpha})(S')$, we have
    \[f(\exp(X))=\exp(h(X^q)),\quad f(\exp(Y))=\exp(h^\vee(Y^q)).\]
\end{enumerate}
\end{proposition}
\begin{proof}
We have (\rmnum{2})$\Rightarrow$(\rmnum{1}): in fact, by the formula (\ref{scheme group elementary system alpha^*(u) formula}), condition (\rmnum{2}) implies that $f\circ\check{\alpha}=q\check{\alpha}'$, hence by \cref{scheme group elementary system centralizer of coroot is maximal torus}, $f|_T$ factors through $T'$. It remains to prove that $\alpha'(f(t))=\alpha(t)^q$, which follows from the fact that $f$ induces a group homomorphism from $T\cdot U_\alpha$ to $T'\cdot U_{\alpha'}$. In fact, we have
\[\exp(\alpha(t)^qh(X^q))=f(\exp(\alpha(t)X))=f(t\cdot\exp(X))=f(t)\cdot\exp(h(X^q))=\exp(\alpha'(f(t))\cdot h(X^q)),\]
whence $\alpha'(f(t))=\alpha(t)^q$, since $\exp$ is a monomorphism.\par 
We now prove that (\rmnum{1})$\Rightarrow$(\rmnum{2}). Let $X\in\Gamma(S,\g^\alpha)$, $Y\in\Gamma(S,\g^{-\alpha})$, and put $p_+(x)=f(\exp(xX))$, $p_-(x)=f(\exp(yY))$; these are then homomorphisms of groups:
\[p_+,p_-:\G_{a,S}\to G.\]
If there exists a morphism $f_T$ satistying the condition (\rmnum{1}), then
\begin{align*}
\inn(\check{\alpha}'(z)^q)(p_+(x))&=\inn(f_T(\check{\alpha}(z)))(f(\exp(xX)))=f(\inn(\check{\alpha}(z))(\exp(xX)))\\
&=f(\exp(z^2xX))=p_+(z^2X).
\end{align*}
Applying \cref{scheme group morphism from G_a twisted by root char} (with $Q=\check{\alpha}'(\G_{m,S}))$, we see that there exists a section $X'\in\Gamma(S,\g^{\alpha'})$ such that
\[f(\exp(xX))=p_+(x)=\exp(x^qX').\]
Similarly, there exists a section $Y'\in\Gamma(S,\g^{-\alpha'})$ such that
\[f(\exp(yY))=\exp(y^qY').\]
Since $f$ is a morphism of groups, hence respects the formula (\ref{scheme group elementary system canonical pair prop-1}), we easily obtain that $X^qY^q=(XY)^q=X'Y'$, so the maps $X^q\mapsto X'$ and $Y^q\mapsto Y'$ define isomorphisms $h$ and $h^\vee$, as desired.
\end{proof}

\begin{proposition}\label{scheme group elementary system intersection of conjugate char}
Let $(G,T,\alpha)$ be an elementary $S$-system, $w\in N^\times(S)$, and put
\[\Omega_0=\Omega\cap\inn(w^{-1})(\Omega).\]
Let $d$ be the function on $\Omega$ defined by
\[d(\exp(Y)\cdot t\cdot\exp(X))=\alpha(t)^{-1}+XY.\]
Then $\Omega_0=\Omega_d$ (the locus where $d$ is invertible) and for $\exp(Y)\cdot y\cdot\exp(X)\in\Omega_0(S')$, we have the following formula:
\begin{equation}
\inn(w)(\exp(Y)\cdot t\cdot\exp(X))=\exp(z^{-1}a_\alpha(w)^{-1}X)\cdot t\check{\alpha}(z)\cdot\exp(z^{-1}a_\alpha(w)Y).
\end{equation}
Moreover, we have $d\circ\inn(w)=d^{-1}$ over $\Omega_0$.
\end{proposition}

\begin{theorem}\label{scheme group elementary system morphism existence if}
Let $S$ be a scheme, $q>0$ be an integer such that $x\mapsto x^q$ is an endomorphism of $\G_{a,S}$, $(G,T,\alpha)$ and $(G',T',\alpha')$ be elementary $S$-systems. Let
\[h:(\g^\alpha)^{\otimes q}\to\g'^{\alpha'},\quad h^\vee:(\g^{-\alpha})^{\otimes q}\to\g'^{-\alpha'}\]
be contragredient isomorphismsm and $f_T:T\to T'$ be a morphism of $S$-groups fitting into the following diagram:
\[\begin{tikzcd}
\G_{m,S}\ar[d,"q"]\ar[r,"\check{\alpha}"]&T\ar[r,"\alpha"]\ar[d,"f_T"]&\G_{m,S}\ar[d,"q"]\\
\G_{m,S}\ar[r,"\check{\alpha}'"]&T'\ar[r,"\alpha'"]&\G_{m,S}
\end{tikzcd}\]
Then there exists a unique morphism of $S$-groups $f:G\to G'$ which extends $f_T$ and satisfies
\[f(\exp(X))=\exp(h(X^q)),\]
and such a morphism $f$ also satisfies
\[\quad f(\exp(Y))=\exp(h^\vee(Y^q)),\quad f(w_\alpha(Z))=w_{\alpha'}(h(Z^q))\]
for any $S'\to S$, $X\in\mathbf{W}(\g^\alpha)(S')$, $Y\in\mathbf{W}(\g^{-\alpha})(S')$, $Z\in\Gamma(S',\g^\alpha)^\times$.
\end{theorem}
\begin{proof}
If $f:G\to G'$ extends $f_T$, then $f\circ\check{\alpha}=\check{\alpha}'^q$. If $f$ also satisfies $f(\exp(X))=\exp(h(X^q))$, then $f$ also satisfies the other conditions in \cref{scheme group elementary system morphism existence if}. In fact, we then have $f\circ\check{\alpha}=q\check{\alpha}'$, and since $f$ is a group homomorphism,
\begin{gather*}
\inn(f(w_\alpha(Z)))(q\check{\alpha}'(z))=f(\inn(w_\alpha(Z)))(f(\check{\alpha}(z)))=f(\inn(w_\alpha(Z))(\check{\alpha})(z))=-f(\check{\alpha}(z))=-q\check{\alpha}'(z),\\
(f(w_\alpha(Z))\exp(h(Z^q)))^3=f(w_\alpha(Z)\exp(Z))^3=f(e)=e.
\end{gather*}
In view of \cref{scheme group elementary system (w exp(X))^3=e char}, we see that $f(w_\alpha(Z))=w_{\alpha'}(h(Z^q))$, which then implies, by the first formula of \cref{scheme group elementary system a_alpha exp prop}, that
\begin{align*}
f(\exp(-Y))&=f(\inn(w_\alpha(Y^{-1}))\exp(Y^{-1}))=\inn(w_{\alpha'}(h(Y^{-q})))\exp(h(Y^{-q}))\\
&=\exp(-h(Y^{-q})^{-1})=\exp(-h^\vee(Y^q)),
\end{align*}
whence our assertion. Therefore, the morphism $f$ is in fact uniquely determined by the relation
\[f(\exp(Y)t\exp(X))=\exp(h^\vee(Y^q))f_T(t)\exp(h(X^q)).\]
As $\Omega$ is schematically dense in $G$, this proves the uniqueness of $G$. For the existence, it suffices, in view of (\cite{SGA3-2} \Rmnum{18} 2.3), to prove that the above formula defines a "generically multiplicative" morphism from $\Omega$ to $G'$. Now by \cref{scheme group elementary system product of Omega char}, it suffices to verify that $\alpha'\circ f=q\alpha$, which follows from the fact that $f$ extends $f_T$.
\end{proof}

\begin{remark}\label{scheme group elementary system and coroot pair equivalence}
We can also interprete \cref{scheme group elementary system morphism existence if} in the following way: consider the category $\mathcal{E}$ of elementary $S$-systems and the category $\mathcal{D}$ of couples
\[(\G_{m,S}\stackrel{\check{\alpha}}{\to}T\stackrel{\alpha}{\to}\G_{m,S},\mathscr{L}),\]
where $T$ is a torus, $\alpha$ and $\check{\alpha}$ are group homomorphisms such that $\alpha\circ\check{\alpha}=2$, and $\mathscr{L}$ is an invertible $\mathscr{O}_S$-module. We define a functor $\mathcal{C}\to\mathcal{D}$ by
\[(G,T,\alpha)\mapsto(\G_{m,S}\stackrel{\check{\alpha}}{\to}T\stackrel{\alpha}{\to}\G_{m,S},\g^\alpha).\]
The preceding theorem then says that this functor is faithfully flat (this is in fact an equivalence of categories).
\end{remark}

\begin{corollary}\label{scheme group elementary system isomorphism iff on torus}
If $q=1$ and $f_T$ is an isomorphism, then $f$ is an isomorphism.
\end{corollary}

\begin{corollary}\label{scheme group elementary system faithfully flat morphism prop}
If $q=1$ and $f_T$ is faithfully flat with kernel $Q$, then $f$ is faithfully flat and quasi-compact with kernel $Q$, hence identifies $G'$ with $G/Q$.
\end{corollary}
\begin{proof}
In fact, if $f_T$ is faithfully flat with kernel $Q$, then $Q=\ker(f_T)\sub\ker(f_T\circ\alpha')=\ker\alpha$. By considering the elementary system $(G/Q,T/Q,\alpha/Q)$, we are then reduced to proving that $f/Q$ induces an isomorphism from $G/Q$ to $G'$, which follows easily from \cref{scheme group elementary system isomorphism iff on torus}.
\end{proof}

\subsection{Examples and applications}
In this subsection, we give a generalization of \cref{scheme group elementary system GL_2 example}. Let $S$ be a scheme, $\mathscr{L}$ be an invertible $\mathscr{O}_S$-module. Consider the group $G_\mathscr{L}$ over $S$ defined by
\[G_\mathscr{L}(S')=\left\{\begin{pmatrix}
a&b\\
c&d
\end{pmatrix}:a,d\in\G_a(S'),b\in\mathbf{W}(\mathscr{L})(S'),c\in\mathbf{W}(\mathscr{L}^{-1})(S'),ad-bc\in\G_m(S')\right\},\]
endowed with the usual multiplication law of matrices. This is locally isomorphic to $\GL_{2,S}$, and is hence an affine smooth $S$-group with connected fibers.

\begin{remark}
If $\mathscr{L}'$ and $\mathscr{L}''$ are invertible sheaves over $S$ such that $\mathscr{L}=\mathscr{L}'\oplus\mathscr{L}''^{-1}$, then we have an isomorphism of $S$-groups
\[G_\mathscr{L}\stackrel{\sim}{\to}\GL(\mathscr{L}'\oplus\mathscr{L}'')\]
defined as follows: if $x$ (resp. $y$) is a section of $\mathscr{L}'$ (resp. $\mathscr{L}''$) over an open subset $V$ of $S$, we have
\[\begin{pmatrix}
a&b\\
c&d
\end{pmatrix}\begin{pmatrix}
x\\
y
\end{pmatrix}=\begin{pmatrix}
ax+by\\
cx+dy
\end{pmatrix}\]
\end{remark}

We denote by $S_\mathscr{L}$ the closed subgroup of $G_\mathscr{L}$ defined by the relation $ad-bc=1$. This is also an affine and smooth $S$-group, with connected fibers (locally isomorphic to $\SL_{2,S}$). Similarly, consider the morphisms $\G_{m,S}\to G_\mathscr{L}$ defined by $z\mapsto(\begin{smallmatrix}z&0\\0&z\end{smallmatrix})$. This is a central monomorphism; by passing to quotient, we then deduce a group $P_\mathscr{L}$, smooth and affine over $S$, with connected fibers (\cref{scheme group diagonaizable quotient by monomorphism prop}). By passing to quotient from the isomorphism in the preceding remark, we see that $P_\mathscr{L}$ is identified with the group of automorphisms of the projective bundle $\P(\mathscr{L}'\oplus\mathscr{L}'')$. We denote by $i$ and $p$ the canonical morphisms
\[\begin{tikzcd}
S_\mathscr{L}\ar[r,"i"]&G_\mathscr{L}\ar[r,"p"]&P_\mathscr{L}
\end{tikzcd} 
\]
where $i$ is a closed immersion and $p$ is faithfully flat and affine.\par

Consider the morphisms of groups
\begin{alignat*}{2}
t_G:\G_{m,S}^2\to G_\mathscr{L},&\quad&t_G(z,z')=\begin{pmatrix}
z&0\\
0&z'
\end{pmatrix},\\
t_S:\G_{m,S}^2\to S_\mathscr{L},&\quad&t_S(z)=\begin{pmatrix}
z&0\\
0&z^{-1}
\end{pmatrix},\\
t_P:\G_{m,S}^2\to P_\mathscr{L},&\quad&t_P(z)=p(t_G(z,1)).
\end{alignat*}
There are monomorphisms of groups, which define in each group a (splitting) torus of relative codimension $2$. For any $s\in S$, let
\[X\in\Gamma(\bar{s},\mathscr{L}\otimes\bar{s})^\times,\]
then the section $(\begin{smallmatrix}0&X\\X^{-1}&0\end{smallmatrix})$ of $G_{\mathscr{L},\bar{s}}$ normalizes $t_G(\G_{m,\bar{s}}^2)$ and does not centralizes it; we then conclude from \cref{scheme alg group smooth reductive with ss-rank 1 iff} that $G_\mathscr{L}$ is reductive, of semi-simple rank $1$, and maximal torus $t_G(\G_{m,S}^2)$. By the same reasoning, we see that the groups $S_\mathscr{L}$ and $P_\mathscr{L}$ are reductive, of semi-simple rank $1$, with maximal torus $t_G(\G_{m,S})$ (resp. $t_P(\G_{m,S})$).\par

By an easily calculation, we can determine the Lie algebra of these groups and the adjoint representations of tori. For example, $\mathfrak{Lie}(G_\mathscr{L}/S)$ is the Lie algebra of the following matrices:
\[\mathfrak{Lie}(G_\mathscr{L}/S)=\left\{\begin{pmatrix}
a&b\\
c&d
\end{pmatrix}:\text{$a,d$ are sections of $\mathscr{O}_S$, $b$ (resp. $c$) is a section of $\mathscr{L}$ (resp. $\mathscr{L}^{-1}$)}\right\}\]
with the usual bracket, and we have
\[\Ad(t_G(z,z'))\begin{pmatrix}
a&b\\
c&d
\end{pmatrix}=\begin{pmatrix}
a&zz'^{-1}b\\
z'z^{-1}&d
\end{pmatrix}.\]
Denote by $\g=\mathfrak{Lie}(G_\mathscr{L}/S)$, and $\alpha_G:t_G(\G_{m,S}^2)\to\G_{m,S}$ be the character defined by
\[\alpha_G(t_G(z,z'))=zz'^{-1}.\]
We easily see that $\alpha_G$ is a root of $G_\mathscr{L}$ relative to $t_G(\G_{m,S}^2)$ and that the morphisms
\[u:\mathscr{L}\to\g,\quad u_-:\mathscr{L}^{-1}\to\g\]
defined by $u(X)=(\begin{smallmatrix}0&X\\0&0\end{smallmatrix})$ and $u_-(X)=(\begin{smallmatrix}0&0\\X&0\end{smallmatrix})$, are isomorphism of $\mathscr{L}$ to $\g^{\alpha_G}$ and $\mathscr{L}^{-1}$ to $\g^{-\alpha_G}$. Therefore, we have shown that $(G_\mathscr{L},t_G(\G_{m,S}),\alpha_G)$ is an elementary $S$-system.\par

Similarly, we consider
\[\alpha_S(t_S(z))=z^2,\quad \alpha_P(t_P(z))=z\]
then $(S_\mathscr{L},t_S(\G_{m,S}),\alpha_S)$ and $(P_\mathscr{L},t_P(\G_{m,S}),\alpha_P)$ are elementary $S$-systems, and we obtain isomorphisms from $\mathscr{L}$ (resp. $\mathscr{L}^{-1}$) to the corresponding direct factors of the Lie algebras of $S_\mathscr{L}$ and $P_\mathscr{L}$.\par

PUt $\exp(\begin{smallmatrix}0&X\\0&0\end{smallmatrix})=(\begin{smallmatrix}1&X\\0&1\end{smallmatrix})$, we thus define a morphism
\[\mathbf{W}(\g^{\alpha_G})\to G_\mathscr{L},\]
which induces over Lie algebras the canonical inclusion, hencei s the unique morphism of type \cref{scheme group elementary system exp morphism}. Similarly, we put $\exp(\begin{smallmatrix}0&0\\Y&0\end{smallmatrix})=(\begin{smallmatrix}1&0\\Y&1\end{smallmatrix})$. Performing the explicit calculation of the formula (\ref{scheme group elementary system canonical pair prop-1}), we find that
\[\left\langle\begin{pmatrix}
0&X\\
0&0
\end{pmatrix},\begin{pmatrix}
0&0\\
Y&0
\end{pmatrix}\right\rangle=XY,\quad \check{\alpha}_G(z)=\begin{pmatrix}
z&0\\
0&z^{-1}
\end{pmatrix}=t_G(z,z^{-1}).\]
The open subset $N^\times=N^\times_G$ is
\[N_G^\times(S')=\left\{\begin{pmatrix}
0&P\\
Q&0
\end{pmatrix}:\text{$P\in\mathbf{W}(\g^\alpha)^\times(S')$, $Q\in\mathbf{W}(\g^{-\alpha})^\times(S')$}\right\},\]
the morphism $w_{\alpha_G}$ (defined in \cref{scheme group elementary system Weyl group prop}~(\rmnum{4})) is given, for any $X\in\mathbf{W}(\g^\alpha)^\times(S')$, by
\[w_{\alpha_G}(X)=\begin{pmatrix}
0&X\\
-X^{-1}&0
\end{pmatrix},\]
and the morphism $a_{\alpha_G}$ (cf. \cref{scheme group elementary system a_alpha morphism prop}) is given, for $w=(\begin{smallmatrix}0&P\\Q&0\end{smallmatrix})$, by:
\[a_{\alpha_G}(w)=PQ^{-1}\in\mathbf{W}((\g^\alpha)^{\otimes 2})^\times(S'),\]
that is, for any $Y\in\mathbf{W}(\g^{-\alpha})^\times$, we have $a_{\alpha_G}(w)(Y)=PQ^{-1}Y\in\mathbf{W}(\g^\alpha)^\times(S')$.

The calculation for $S_\mathscr{L}$ and $P_\mathscr{L}$ can be carried out similarly. For example, we find that the coroot of $S_{\mathscr{L}}$ and $P_\mathscr{L}$ are given by
\[\check{\alpha}_S(z)=t_S(z),\quad \check{\alpha}_P(z)=t_P(z^2).\]
Denote by $p_T$ the morphism induced by $p:G_\mathscr{L}\to P_\mathscr{L}$ over $t_S(\G_{m,S})$, that is,
\[p_T(t_S(z))=t_P(z^2).\]
We then have the following commutative diagram (note that $t_S$ and $t_P$ are isomorphisms):
\[\begin{tikzcd}[column sep=12mm,row sep=12mm]
&&\G_{m,S}\ar[lld,swap,"\id"]\ar[ld,"\check{\alpha}_S"]\ar[rd,swap,"\check{\alpha}_P"]\ar[rrd,"2"]&&\\
\G_{m,S}\ar[rrd,swap,"2"]\ar[r,"t_S"]&t_S(\G_{m,S})\ar[rd,"\alpha_S"]\ar[rr,"p_T"]&&t_P(\G_{m,S})\ar[ld,swap,"\alpha_P"]&\G_{m,S}\ar[l,swap,"t_P"]\ar[lld,"\id"]\\
&&\G_{m,S}&&
\end{tikzcd}\]
We note that the the central part ressembles the commutative diagram of \cref{scheme group elementary system morphism existence if} relative to the canonical morphism $p\circ i:S_\mathscr{L}\to P_\mathscr{L}$ (with $q=1$), which induces a morphism of these elementary $S$-systems.\par

Now let $(G,T,\alpha)$ be an arbitrary elementary $S$-system. Consider the commutative diagram:
\[\begin{tikzcd}
&\G_{m,S}\ar[d,"\check{\alpha}"]\ar[rd,"2"]\ar[ld,swap,"\id"]\\
\G_{m,S}\ar[rd,swap,"2"]\ar[r,"\check{\alpha}"]&T\ar[d,"\alpha"]\ar[r,"\alpha"]&\G_{m,S}\ar[ld,"\id"]\\
&\G_{m,S}&
\end{tikzcd}\]
Combine this with the precding diagram, we then obtain a commutative diagram
\[\begin{tikzcd}
&\G_{m,S}\ar[d,"\check{\alpha}"]\ar[rd,"\check{\alpha}_P"]\ar[ld,swap,"\check{\alpha}_S"]\\
t_S(\G_{m,S})\ar[rd,swap,"\alpha_S"]\ar[r,"\check{\alpha}\circ t_S^{-1}"]&T\ar[d,"\alpha"]\ar[r,"t_P\circ\alpha"]&t_P(\G_{m,S})\ar[ld,"\alpha_P"]\\
&\G_{m,S}&
\end{tikzcd}\]

Using \cref{scheme group elementary system morphism existence if}, we then conclude:

\begin{proposition}\label{scheme group elementary system morphism from SL and PGL prop}
Let $S$ be a scheme, $(G,T,\alpha)$ be an elementary $S$-system. Put $\mathscr{L}=\g^\alpha$ (and hence $\mathscr{L}^{-1}=\g^{-\alpha}$).
\begin{enumerate}
    \item[(\rmnum{1})] There exists a unique group morphism $f:S_\mathscr{L}\to G$ which satisfies the following equivalent conditions:
    \begin{enumerate}
        \item[(a)] $f(\begin{smallmatrix}z&0\\0&z^{-1}\end{smallmatrix})=\check{\alpha}(z)$, $f(\begin{smallmatrix}1&X\\0&1\end{smallmatrix})=\exp(X)$;
        \item[(b)] $f(\begin{smallmatrix}1&X\\0&1\end{smallmatrix})=\exp(X)$, $f(\begin{smallmatrix}1&0\\Y&1\end{smallmatrix})=\exp(Y)$;
        \item[(c)] $f(\begin{smallmatrix}1&X\\0&1\end{smallmatrix})=\exp(X)$, $f(\begin{smallmatrix}0&X\\-X^{-1}&0\end{smallmatrix})=w_\alpha(X)$;
    \end{enumerate}
    \item[(\rmnum{2})] There exists a unique group morphism $g:G\to P_\mathscr{L}$ which satisfies 
    \[g(t)=\begin{pmatrix}
    \alpha(t)&0\\
    0&1
    \end{pmatrix},\quad g(\exp(X))=p\begin{pmatrix}
    1&X\\
    0&1
    \end{pmatrix}.\]
    and we also have
    \[g(\exp(Y))=p\begin{pmatrix}
    1&0\\
    Y&1
    \end{pmatrix},\quad g(w_\alpha(X))=p\begin{pmatrix}
    0&X\\
    -X^{-1}&0
    \end{pmatrix},\]
    The morphism $g$ is faithfully flat and quasi-compact with kernel $\ker\alpha=Z(G)$, and $g\circ f$ is the canonical morphism $S_\mathscr{L}\to P_\mathscr{L}$.
\end{enumerate}
\end{proposition}

\begin{corollary}\label{scheme group elementary system subgroup U_alpha closed}
Let $(G,T,\alpha)$ be an elementary $S$-system. The subgroup $T\cdot U_\alpha$, $T\cdot U_{-\alpha}$, $U_\alpha$, $U_{-\alpha}$ are closed\footnote{In particular, the $\exp$ morphism is a closed immersion.}.
\end{corollary}
\begin{proof}
As $U_\alpha$ is a closed subgroup of $T\cdot U_\alpha$, it suffices to consider the latter group. By \cref{site sheaf quotient by normal N-subgroup correspond}, since closed immersions are fpqc descent, it suffices to prove that $(T\cdot U_\alpha)/\ker\alpha$ is a closed subgroup of $G/\ker\alpha$. In view of \cref{scheme group elementary system morphism from SL and PGL prop}, we are then reduced to prove that the subgroup of $P_\mathscr{L}$ (or $G_\mathscr{L}$) defined by $c=0$ is closed, which is immediate.
\end{proof}

Let $\mathscr{L}$ be an invertible $\mathscr{O}_S$-module and
\[\G_{m,S}\stackrel{\check{\alpha}}{\to}T\stackrel{\alpha}{\to} \G_{m,S}\]
be a diagram of groups such that $\alpha\circ\check{\alpha}=2$ ($T$ being a torus). Let $R$ be the maximal torus of $\ker\alpha$ and $K=\check{\alpha}^{-1}(R)$, then $K$ is a subgroup of multiplicative type of $\G_{m,S}$, and in view of $\alpha\circ\check{\alpha}^*=2$, it is also a subgroup of $\bm{\mu}_{2,S}$. In particular, the morphism
\[K\to S_\mathscr{L},\quad z\mapsto\begin{pmatrix}
z&0\\
0&z^{-1}
\end{pmatrix}\]
is central. We then have a central group monomorphism
\[K\to R\times S_\mathscr{L},\quad z\mapsto(\check{\alpha}(z),\begin{pmatrix}
z&0\\
0&z^{-1}
\end{pmatrix}).\]
Consider the group $G=(R\times S_\mathscr{L})/K$ obtained by passing to quotient. This is an affine smooth group over $S$ with connected fibers. It is immediate that the sequence
\[\begin{tikzcd}
1\ar[r]&K\ar[r]&R\times t_S(\G_{m,S})\ar[r]&T\ar[r]&1
\end{tikzcd}\]
where $u(x,t_S(z))=x\check{\alpha}(z)$ is exact. The image of $R\times t_S(\G_{m,S})$ in $G$ is then a torus $T'$ isomorphic to $T$. It is then easy to see that if $\alpha'$ is the character of $T'$ induced from $\alpha$ by the preceding isomorphism, then $(G,T',\alpha')$ is an elementary $S$-system, that $\g^{\alpha'}$ is isomorphic to $\mathscr{L}$, and that $\check{\alpha}'$ is obtained from $\check{\alpha}$ by the isomorphism $T\stackrel{\sim}{\to} T'$. We therefore obtain an elementary $S$-system $(G,T',\alpha')$ such that the corresponding object $(\G_{m,S}\stackrel{\check{\alpha}'}{\to} T'\stackrel{\alpha'}{\to}\G_{m,S},\g^{\alpha'})$ of the category defined in \cref{scheme group elementary system and coroot pair equivalence} is isomorphic to $(\G_{m,S}\stackrel{\check{\alpha}}{\to} T\stackrel{\alpha}{\to}\G_{m,S},\g^{\alpha'})$, so the functor in \cref{scheme group elementary system and coroot pair equivalence} is an equivalence of categories.

\section{Root datum}
In this section, we denote by $\Q_+$ the set of positive rational numbers; we have $\Z\cap\Q_+=\N$. Let $V$ be a $\Q$-vector space, if $A$ (resp. $B$) is a subset of $\Q$ (resp. of $V$), we denote by $A\cdot B$ the image of $A\otimes B$ under the morphism $\Q\otimes_{\Q}V\to V$, that is, the set of linear combinations of elements of $B$ with coefficients in $A$. We write $-B=\{-1\}B$, and by $E-F$ we mean the difference of two sets.
\subsection{Generalities}
Let $M$ and $\check{M}$ be two free $\Z$-modules of finite rank in duality. We put $V=M\otimes_{\Z}\Q$, $\check{V}=\check{M}\otimes_{\Z}\Q$, which are then $\Q$-vector spaces in duality. We can identity $M$ (resp. $\check{M}$) with the subset of $V$ (resp. $\check{V}$). The canonical bilinear form over $\check{M}\times M$ (resp. $\check{V}\times V$) will be denoted by $\langle\cdot\,,\cdot\rangle$.\par
Let $R$ be a \textbf{finite} subset of $M$, and assume that we are given a map $\alpha\mapsto\check{\alpha}$ from $R$ into $\check{M}$. The set of $\check{\alpha}$, for $\alpha\in R$, is denoted by $\check{R}$. To each $\alpha\in R$, we associate an endomorphism $s_\alpha$ (resp. $s_{\check{\alpha}}$) of $M$ and $V$ (resp. $\check{M}$ and $\check{V}$), given by the formulas
\begin{alignat}{2}
s_\alpha(x)&=x-\langle\check{\alpha},x\rangle\alpha,&\quad&\text{i.e.\quad$s_\alpha=\id-\check{\alpha}\otimes\alpha$};\label{root datum reflection def-1}\\
s_{\check{\alpha}}(u)&=u-\langle u,\alpha\rangle\check{\alpha},&\quad &\text{i.e.\quad $s_\alpha=\id-\alpha\otimes\check{\alpha}$}.\label{root datum reflection def-2}
\end{alignat}
We say that the couple $(R,\check{R})$ (or more precisely, the couple $(R,\check{R},\alpha\mapsto\check{\alpha})$) is a \textbf{root datum} in $(M,\check{M})$, or that $(M,\check{M},R,\check{R})$ is a root datum, if the following axioms are satisfied:
\begin{enumerate}[leftmargin=40pt]
    \item[(RD1)] For each $\alpha\in R$, $(\check{\alpha},\alpha)=2$.
    \item[(RD2)] For each $\alpha\in R$, $s_\alpha(R)\sub R$ and $s_{\check{\alpha}}(\check{R})\sub\check{R}$. 
\end{enumerate}
We say that $R$ is the root system of the root datum $\mathcal{R}=(M,\check{M},R,\check{R})$. The elements of $R$ (resp. $\check{R}$) are called the \textbf{roots} (resp. \textbf{coroots}) of the root datum.

\begin{remark}\label{root datum axiom imply symmetry nonzero}
The axiom (DR1) is equivalent to each one of the following properties:
\[\text{(1)\ \ $s_\alpha^2=\id$},\quad \text{(1')\ \ $s_{\check{\alpha}}^2=\id$},\quad \text{(2)\ \ $s_\alpha(\alpha)=-\alpha$},\quad \text{(2')\ \ $s_{\check{\alpha}}(\check{\alpha})=-\check{\alpha}$}.\]
Therefore, the axioms (DR1) and (DR2) imply that
\[R=-R,\quad \check{R}=-\check{R},\quad 0\notin R,\quad 0\notin\check{R}.\]
\end{remark}

\begin{lemma}\label{root datum coroot map bijection}
The map $R\to\check{R}$ is a bijction. More generally, if $\alpha,\beta\in R$ and $\langle\check{\alpha},x\rangle=\langle\check{\beta},x\rangle$ for any $x\in R$, then $\alpha=\beta$.
\end{lemma}
\begin{proof}
In fact, by (RD1) we then have $s_\beta(\alpha)=\alpha-2\beta$, $s_\alpha(\beta)=\beta-2\alpha$, from which we deduce
\[s_\beta s_\alpha(\alpha)=2\beta-\alpha=\alpha+2(\beta-\alpha),\quad s_\beta s_\alpha(\beta-\alpha)=s_\beta(\beta-\alpha)=\beta-\alpha,\]
whence $(s_\beta s_\alpha)^n(\alpha)=\alpha+2n(\beta-\alpha)\in R$ by (RD2). As $R$ is finite, we then have $\beta-\alpha=0$.
\end{proof}

\begin{corollary}
The inverse map $\check{R}\to R$ defines a root datum
\[\check{\mathcal{R}}=(\check{M},M,\check{R},R),\]
called the \textbf{dual} of $\mathcal{R}$.
\end{corollary}

We denote by $\Lambda_r(R)$ the subgroup of $M$ generated by $R$, and by $V(R)$ the $\Q$-vector subspace of $V$ generated by $R$, that is, $\Lambda_r(R)\otimes_{\Z}\Q$. Applying these definitions to $\check{\mathcal{R}}$, we smilarly construct $\Lambda_r(\check{R})$ and $V(\check{R})$. The \textbf{reductive rank} of $\mathcal{R}$ is defined to be the number
\[\rho_r(\mathcal{R})=\rank(M)=\dim(V)=\dim(\check{V})=\rank(\check{M})=\rho_r(\check{\mathcal{R}}),\]
and the \textbf{semi-simple rank} of $\mathcal{R}$ is
\[\rho_s(\mathcal{R})=\rank(R)=\rank(\Lambda_r(R))=\dim(V(R)).\]
We therefore have $\rho_s(\mathcal{R})\leq\rho_r(\mathcal{R})$, and $\dim(V(R))=\dim(V(\check{R}))$ in view of the positive nondegenerated form
\[(x,y)=\sum_{\alpha\in R}\langle\check{\alpha},x\rangle\langle\check{\alpha},y\rangle\]
over $V(R)$ (cf. \cref{root system invariant bilinear form}), so that $\mathcal{R}$ and $\check{\mathcal{R}}$ have the same semi-simple rank. We say that $\mathcal{R}$ is \textbf{semi-simple} (resp. \textbf{trivial}) if $\rho_s(\mathcal{R})=\rho_r(\mathcal{R})$ (resp. if $\rho_s(\mathcal{R})=0$). For $\mathcal{R}$ to be trivial, it is hence necessary and sufficient that $R=\emp$. The trivial root datum of zero reductive rank is denoted by $0=(\{0\},\{0\},\emp,\emp)$.\par
We define $W(\mathcal{R})$ to be the transformation group of $M$ generated by the $s_\alpha$ ($\alpha\in R$), called the \textbf{Weyl group} of $\mathcal{R}$. From the symmetric bilinear form $(\cdot,\cdot)$, we can define a linear map $\varrho:M\to\check{M}$ by
\[\varrho(x)=\sum_{\alpha\in R}\langle\check{\alpha},x\rangle\alpha,\]
which is an isomorphism from $V$ to $\check{V}$. From \cref{root system invariant bilinear form} we know that the bilinear form $(\cdot\,,\cdot)$ is invariant under $W(\mathcal{R})$, so by formula (\ref{orthogonal reflection formula}), $\check{\alpha}$ is given by
\[\check{\alpha}=\frac{2\varrho(\alpha)}{(\alpha,\alpha)}.\]

Finally, we note that, under the duality of $M$ and $\check{M}$, $W(\mathcal{R})$ and $W(\check{\mathcal{R}})$ are contragredient with each other:

\begin{proposition}\label{root datum s_alpha and inverse contragredient prop}
For any $\alpha\in R$, $x\in V$, $u\in \check{V}$, we have\footnote{If we assume that $0\notin R$ and $0\notin\check{R}$, then the formula (\ref{root datum s_alpha and inverse contragredient prop-1}) is in fact equivalent to (RD1).}
\begin{equation}\label{root datum s_alpha and inverse contragredient prop-1}
\langle s_{\check{\alpha}}(u),s_\alpha(x)\rangle=\langle u,x\rangle
\end{equation}
\end{proposition}
\begin{proof}
In fact, in view of the definitions (\ref{root datum reflection def-1}) and (\ref{root datum reflection def-2}), we find that the first member is equal to $\langle u,x\rangle+\langle u,\alpha\rangle\langle\check{\alpha},x\rangle(\langle\check{\alpha},\alpha\rangle-2)=\langle u,x\rangle$.
\end{proof}

\begin{corollary}\label{root datum Weyl group of root and coroot contragredient}
Denote by $h\mapsto h^\vee$ the contragredient isomorphism, then (\ref{root datum s_alpha and inverse contragredient prop-1}) implies that $(s_\alpha)^\vee=s_{\check{\alpha}}$. In particular, $W(\mathcal{R})$ and $W(\check{\mathcal{R}})$ are contragredient with each other.
\end{corollary}

Now $W(\mathcal{R})$ acts over $R$, $\Lambda_r(R)$, $V(R)$, $M$ and $V$. If $w\in W(\mathcal{R})$ and $x\in M$ (resp. $x\in V$), we have $wx-x\in\Lambda_r(R)$ (resp. $wx-x\in V(R)$), which is immediate from the formula (\ref{root datum reflection def-1}). The same result holds for $W(\check{\mathcal{R}})$.

\begin{remark}
In view of \cref{root datum Weyl group of root and coroot contragredient}, we may identity the Weyl groups $W(\mathcal{R})$ and $W(\check{\mathcal{R}})$ and use $W$ to denote them. Since for $w\in W$, $u\in \check{V}$ and $x\in V$, we have
\[\langle u,w(x)\rangle=\langle w^t(u),x\rangle=\langle (w^\vee)^{-1}(u),x\rangle,\]
we may then write $\langle u,w(x)\rangle=\langle w^{-1}(u),x\rangle$ and identify $w^{-1}$ with the dual morphism of $w$.
\end{remark}

\begin{corollary}\label{root datum Weyl group faithful over root}
The Weyl group $W=W(\mathcal{R})$ acts faithfully over $R$, and hence is finite.
\end{corollary}
\begin{proof}
Let $u\in\check{V}$, $w\in W$, and suppose that $w(\alpha)=\alpha$ for any $\alpha\in R$. We then have
\[\langle w(u)-u,\alpha\rangle=\langle w(u),\alpha\rangle-\langle u,\alpha\rangle=\langle u,w^{-1}(\alpha)\rangle-\langle u,\alpha\rangle=0.\]
But $w(u)-u\in V(\check{R})$, so it is zero.
\end{proof}

\begin{proposition}\label{root datum Weyl group respects coroot}
The operation of $W$ respects the correspondence of roots and coroots. In other words, for any $\alpha\in R$ and $w\in W$, we have
\[w(\check{\alpha})=w(\alpha)\rcheck.\]
\end{proposition}
\begin{proof}
It suffices to verify this for $w=s_\beta$, $\beta\in R$, that is, to verify the formula
\[s_{\check{\beta}}(\check{\alpha})=s_\beta(\alpha)\rcheck.\]
Now note that $(s_\beta(\alpha),s_\beta(\alpha))s_\beta(\alpha)\rcheck/2$ is equal to:
\begin{align*}
\varrho(s_\beta(\alpha))&=\sum_{u\in\check{R}}\langle u,s_\beta(\alpha)\rangle u=\sum_{u\in\check{R}}\langle s_{\check{\beta}}(u),\alpha\rangle u=\sum_{u\in\check{R}}\langle u,\alpha\rangle s_{\check{\beta}}(u)=s_{\check{\beta}}(\varrho(\alpha));
\end{align*}
as $(s_\beta(\alpha),s_\beta(\alpha))=(\alpha,\alpha)$, we then obtain that $s_\beta(\alpha)\rcheck=s_\beta(2\varrho(\alpha)/(\alpha,\alpha))=s_\beta(\check{\alpha})$.
\end{proof}

\begin{corollary}\label{root datum reflection conjugate by Weyl group char}
If $\alpha\in R$ and $w\in W$, we have
\[ws_\alpha w^{-1}=s_{w(\alpha)}.\]
\end{corollary}
\begin{proof}
In fact, $ws_\alpha w^{-1}(x)=x-\langle\check{\alpha},w^{-1}(x)\rangle w(\alpha)=x-\langle w(\check{\alpha}),x\rangle w(\alpha)$, and the latter equals, by \cref{root datum Weyl group respects coroot}, to
\[x-\langle w(\check{\alpha}),x\rangle w(\alpha)=s_{w(\alpha)}(x),\]
which proves our assertion.
\end{proof}

\section{Reductive groups: splittings, subgroups and quotients}
\subsection{Splitting groups and root datums}\label{scheme group splitting root datum subsection}
\begin{theorem}\label{scheme group reductive exp morphism and pairing exist}
Let $S$ be a scheme, $G$ be a reductive $S$-group, $T$ be a maximal torus of $G$, $\alpha$ be a root relative to $T$.
\begin{enumerate}
    \item[(a)] There exists a unique morphism of groups acted by $T$:
    \[\exp_\alpha:\mathbf{W}(\g^\alpha)\to G\]
    which induces the canonical inclusion $\g^\alpha\to\g$ over Lie algebras. This morphism is a closed immersion, and the corresponding morphism
    \[T\cdot_\alpha\mathbf{W}(\g^\alpha)\to G\]
    is also a closed immersion.\par
    If $p_\alpha:\G_{a,S}\to G$ is a monomorphism normalized by $T$ with multiplicator $\alpha$, there exists a unique $X_\alpha\in\Gamma(S,\g^\alpha)^\times$ such that
    \[p_\alpha(x)=\exp_\alpha(xX_\alpha);\]
    we have $\mathfrak{Lie}(p_\alpha)(1)=X_\alpha$, and the preceding formulas establish a bijection between $\Gamma(S,\g^\alpha)^\times$ and the set of monomorphisms $\G_{a,S}\to G$ normalized by $T$ with multiplicator $\alpha$.
    \item[(b)] There exists a unique duality (denoted by $(X,Y)\mapsto XY$):
    \[\g^\alpha\otimes_{\mathscr{O}_S}\g^{-\alpha}\stackrel{\sim}{\to}\mathscr{O}_S,\]
    and a unique group homomorphism $\check{\alpha}:\G_{m,S}\to T$ such that we have formula (\ref{scheme group elementary system canonical pair prop-1}), that $\alpha\circ\check{\alpha}=2$, $(-\alpha)\rcheck=-\check{\alpha}$, and that $\check{\alpha}$ is given by the formula (\ref{scheme group elementary system alpha^*(u) formula}).
\end{enumerate}
\end{theorem}
\begin{proof}
In fact, a morphism normalized by $T$ with multiplicator $\alpha$ necessarily factors through the closed subgroup $G_\alpha=Z_G(T_\alpha)$ of $G$ (cf. \cref{scheme group reductive induced elementary system by root}). Now $(G_\alpha,T,\alpha)$ is an elementary $S$-system, and we are then reduced to the results of \autoref{scheme group reductive elementary system section} (namely, \cref{scheme group elementary system exp morphism}, \cref{scheme group elementary system canonical pair prop} and \cref{scheme group elementary system subgroup U_alpha closed}).
\end{proof}

\begin{remark}
The statement of \cref{scheme group reductive exp morphism and pairing exist}~(a) is still valid if we only assume that $\alpha$ is a character of $T$, nontrivial over each fiber. In fact, we then have a decomposition $S=S'\coprod S''$, such that $\alpha|_{S'}$ is a root of $G_{S'}$ relative to $T_{S'}$ and $\g^\alpha|_{S''}=0$. If $S=S'$, we are then reduced to \cref{scheme group reductive exp morphism and pairing exist}; if $S=S''$ the result is trivial, so the general case follows.
\end{remark}

As in \autoref{scheme group reductive elementary system section}, we denot by $U_\alpha$ the image of $\mathbf{W}(\g^\alpha)$ under $\exp_\alpha$; this is a closed subgroup of $G$, endowed canonically with a vector bundle structure. We say that this is the vectorial group associated with the root $\alpha$. The morphism $\check{\alpha}$ is called the \textbf{coroot associated with $\alpha$}. Two sections $X_\alpha\in\Gamma(S,\g^\alpha)$ and $X_{-\alpha}\in\Gamma(S,\g^{-\alpha})$ are called \textbf{dual} if $X_\alpha X_{-\alpha}=1$. In this case, $X_\alpha\in\Gamma(S,\g^\alpha)^\times$, and similarly for $X_{-\alpha}$. The corresponding morphisms $p_\alpha$ and $p_{-\alpha}$ are contragredient with each other (cf. \cref{scheme group elementary system contragredient of p_alpha char}), and we have
\[p_\alpha(x)p_{-\alpha}(y)=p_{-\alpha}\Big(\frac{y}{1+xy}\Big)\check{\alpha}(1+xy)p_\alpha\Big(\frac{x}{1+xy}\Big).\]

\begin{proposition}\label{scheme group reductive exp morphism and pairing Weyl group action}
Under the conditions of \cref{scheme group reductive exp morphism and pairing exist}. let $w\in N_G(T)(S)$. Then $\beta=\alpha\circ\inn(w)^{-1}:T\to\G_{m,S}$ is a root of $G$ relative to $T$, $\check{\beta}=\inn(w)\circ\check{\alpha}$, is the corresponding coroot, and the following diagram is commutative:
\[\begin{tikzcd}
\mathbf{W}(\g^\alpha)\ar[r,"\exp_\alpha"]\ar[d,swap,"\Ad(w)"]&G\ar[d,"\inn(w)"]\\
\mathbf{W}(\g^\beta)\ar[r,"\exp_\beta"]&G
\end{tikzcd}\]
\end{proposition}
\begin{proof}
In view of the definition, it is clear that $\beta=\alpha\circ\inn(w)^{-1}$ is also a root of $\G_{m,S}$, and the rest then follows easily by transporting the structure.
\end{proof}

Under the conditions of \cref{scheme group reductive exp morphism and pairing exist}, we denote by $s_\alpha$ the automorphism of $T$ defined by
\begin{equation}\label{scheme group reductive root reflection s_alpha formula-1}
s_\alpha(t)=t\cdot\check{\alpha}(\alpha(t))^{-1}.
\end{equation}
We denote by $\langle\cdot\,,\cdot\rangle$ the canonical pairing
\[\sHom_{S\dash\Grp}(\G_{m,S},T)\times\sHom_{S\dash\Grp}(T,\G_{m,S})\to\sHom_{S\dash\Grp}(\G_{m,S},\G_{m,S})=\Z_S.\]
Then the action of $s_\alpha$ on $\sHom_{S\dash\Grp}(T,\G_{m,S})$ (reps. $\sHom_{S\dash\Grp}(\G_{m,S},T)$) is given by the following formulas, where $\chi$ (resp. $u$) denotes an arbitrary section of $\sHom_{S\dash\Grp}(T,\G_{m,S})$ (reps. $\sHom_{S\dash\Grp}(\G_{m,S},T)$)\footnote{To see this, we note that if $t\in T(S')$, then $s_\alpha(\chi)(t):=\chi(s_\alpha(t))=\chi(t)\cdot\chi(\check{\alpha}(\alpha(t)^{-1}))$. If we write this formula additively, we then get (\ref{scheme group reductive root reflection s_alpha formula-2}). The formula (\ref{scheme group reductive root reflection s_alpha formula-3}) can be obtained similarly.}:
\begin{align}
s_\alpha(\chi)&=\chi-\langle\check{\alpha},\chi\rangle\alpha,\label{scheme group reductive root reflection s_alpha formula-2}\\
s_\alpha(u)&=u-\langle u,\alpha\rangle\check{\alpha}.\label{scheme group reductive root reflection s_alpha formula-3}.
\end{align}
We then have $s_\alpha^2=\id$ and $s_{-\alpha}=\alpha$. If $X\in\Gamma(S,\g^\alpha)^\times$, then the interior automorphism of the section $w_\alpha(X)$ of $T$ defined by
\[w_\alpha(X)=\exp_\alpha(X)\exp_{-\alpha}(-X^{-1})\exp_\alpha(X)\]
coincides with $s_\alpha$ (cf. \cref{scheme group elementary system Weyl group prop}). By applying \cref{scheme group reductive exp morphism and pairing Weyl group action} to $w_\alpha(X)$ and using (\ref{scheme group reductive root reflection s_alpha formula-2}), (\ref{scheme group reductive root reflection s_alpha formula-3}), we then conclude the following:

\begin{corollary}\label{scheme group reductive root reflection is root}
Let $S$ be a scheme, $G$ be a reductive $S$-group, $T$ be a maximal torus of $G$, $\alpha,\beta$ be two roots of $G$ relative to $T$. Then
\[s_\alpha(\beta)=\beta-\langle\check{\alpha},\beta\rangle\alpha\]
is a root of $G$ relative to $T$, and the corresponding coroot is
\[s_\alpha(\beta)\rcheck=s_\alpha(\check{\beta})=\check{\beta}-\langle\check{\beta},\alpha\rangle\check{\alpha}.\]
\end{corollary}

\begin{corollary}\label{scheme group reductive root equal iff coroot equal}
Under the preceding conditions, $\check{\alpha}=\check{\beta}$ implies $\alpha=\beta$.
\end{corollary}
\begin{proof}
In fact, if $\check{\alpha}=\check{\beta}$, then since $\alpha\circ\check{\alpha}=2$, we have
\[s_\beta(\alpha)=\alpha-2\beta,\quad s_\alpha(\beta)=\beta-2\alpha,\]
and we then deduce by induction that 
\[(s_\beta s_\alpha)^n=\alpha+2n(\beta-\alpha).\]
If $\alpha\neq\beta$, then there exists $s\in S$ such that $\alpha_s\neq\beta_s$. But the preceding formula then shows that there exists infinitely many roots for $G_{\bar{s}}$ relative to $T_{\bar{s}}$, which is impossible.
\end{proof}

If $u:\G_{m,S}\to T$ is a group homomorphism, we say that $u$ is a \textbf{coroot} of $G$ relative to $T$ if there exists a root $\alpha$ of $G$ relative to $T$ such that $\check{\alpha}=u$. Consider the functor $\check{\mathscr{R}}$ of coroots of $G$ relative to $T$ defined as follows:
\[\check{\mathscr{R}}(S')=\{\text{set of coroots of $G_{S'}$ relative to $T_{S'}$}\}.\]
If $\mathscr{R}$ is the functor of roots of $G$ relative to $T$ (cf. \cref{scheme group functor of root system representable}), we then have a canonical morphism $\mathcal{R}\to\check{\mathcal{R}}$. In view of \cref{scheme group reductive root equal iff coroot equal} and \cref{scheme group functor of root system representable}, we have:

\begin{corollary}
The morphism $\mathscr{R}\to\check{\mathscr{R}}$ is an isomorphism. In particular, $\check{\mathscr{R}}$ is representable by a finite twisted constant $S$-scheme which is clopen in $\sHom_{S\dash\Grp}(\G_{m,S},T)$.
\end{corollary}

This leads to the following definition:
\begin{definition}
Let $S$ be a scheme, $T$ be an $S$-torus. A \textbf{twisted root datum} in $T$ is a triple $(\mathscr{R},\check{\mathscr{R}},\alpha\mapsto\check{\alpha})$, where $\mathscr{R}$ (resp. $\check{\mathscr{R}}$) is a finite subscheme of $\sHom_{S\dash\Grp}(T,\G_{m,S})$ (resp. $\sHom_{S\dash\Grp}(\G_{m,S},T)$) and $\alpha\mapsto\check{\alpha}$ is an isomorphism $\mathscr{R}\stackrel{\sim}{\to}\check{\mathscr{R}}$, such that the following conditions are satisfied: 
\begin{enumerate}[leftmargin=40pt]
    \item[(RD1)] For any $S'\to S$ and any $\alpha\in\mathscr{R}(S')$, $\alpha\circ\check{\alpha}=2$.
    \item[(RD2)] For any $S'\to S$ and any $\alpha,\beta\in\mathscr{R}(S')$,
    \[\alpha-\langle\check{\beta},\alpha\rangle\beta\in\mathscr{R}(S'),\quad \check{\alpha}-\langle\check{\alpha},\beta\rangle\check{\beta}\in\check{\mathscr{R}}(S').\]
\end{enumerate}
If $\alpha\in\mathscr{R}(S')$ ($S'\neq\emp$) implies $2\alpha\notin\mathscr{R}(S')$, we say that the twisted root data is \textbf{reduced}.
\end{definition}

\begin{proposition}\label{scheme group reductive twisted root datum def}
Let $S$ be a scheme, $G$ be a reductive $S$-group, $T$ be a maximal torus of $G$, $\mathscr{R}$ (resp. $\check{\mathscr{R}}$) be the scheme of roots (resp. coroots) of $G$ relative to $T$. Then $(\mathscr{R},\check{\mathscr{R}})$ is a reduced twisted root datum in $T$.
\end{proposition}
\begin{proof}
It remains to verify that this twisted root datum is reduced, which follows from easily from \cref{scheme alg group smooth connected acted by torus reductive iff}.
\end{proof}

Let $T=D_S(M)$ be a splitting torus. If we denote by $\check{M}$ the dual abelian group of $M$ (i.e. $\check{M}=\Hom_{\Grp}(M,\Z)$), we then have canonical isomorphisms (cf. \cref{scheme group diagonalizable of Hom isomorphism if ft}):
\begin{align*}
\sHom_{S\dash\Grp}(T,\G_{m,S})\stackrel{\sim}{\to} M_S,\\
\sHom_{S\dash\Grp}(\G_{m,S},T)\stackrel{\sim}{\to} \check{M}_S,
\end{align*}
whence canonical isomorphisms of groups:
\begin{align*}
\Hom_{S\dash\Grp}(T,\G_{m,S})\stackrel{\sim}{\to} \Hom_{\mathrm{loc.const.}}(S,M),\\
\Hom_{S\dash\Grp}(\G_{m,S},T)\stackrel{\sim}{\to} \Hom_{\mathrm{loc.const.}}(S,\check{M}).
\end{align*}
A character of $T$ (resp. a cocharacter $\G_{m,S}\to T$) is called \textbf{constant} (relative to the given trivialization of $T$) if the preceding isomorphism transforms it into a constant map from $S$ to $M$ (resp. $\check{M}$).\par
Under the same notations, let $(M,\check{M},R,\check{R})$ be a root datum. Then $(R_S,\check{R}_S)$ is a twisted root datum in $T=D_S(M)$. Conversely, if $(\mathscr{R},\check{\mathscr{R}})$ is a twisted root datum in a torus $T$, we define a splitting of $(\mathscr{R},\check{\mathscr{R}})$ to be the data of a usual root datum $(M,\check{M},R,\check{R})$ and an isomorphism $T\cong D_S(M)$ which sends $(\mathscr{R},\mathscr{R})$ to $(R_S,\check{R}_S)$. 

\begin{definition}\label{scheme group reductive splitting def}
Let $S$ be a scheme, $G$ be a reductive $S$-group, $T$ be a maximal torus of $G$. A \textbf{splitting of $G$ relative to $T$} to defined to be the following data:
\begin{enumerate}
    \item[(a)] an abelian group $M$ and an isomorphism $T\cong D_S(M)$,
    \item[(b)] a root system $R$ of $G$ relative to $T$.
\end{enumerate}
which satisfy the following conditions:
\begin{enumerate}[leftmargin=40pt]
    \item[(D1)] $S$ is nonempty and the roots $\alpha\in R$ (resp. the corresponding coroots) are identified with constant functions from $S$ to $M$ (resp. $\check{M}$)\footnote{Note that condition (D1) implies that $R$ (resp. $\check{R}$) is canonically identified with a subset of $M$ (resp. $\check{M}$).}.
    \item[(D2)] The $\g^\alpha$ ($\alpha\in R$) are free $\mathscr{O}_S$-modules. 
\end{enumerate}
\end{definition}

We say that $G$ is \textbf{splittable} relative to $T$ if there exists a splitting of $G$ relative to $T$, and a \textbf{splitting of $G$} is defined to be the data of a maximal torus $T$ of $G$ and a splitting of $G$ relative to $T$. We say that $G$ is \textbf{spliitable} if there exists a splitting of $G$. A reductive $S$-group $G$, endowed with a splitting, is called a \textbf{splitting $S$-group}, and often denoted by the symbol $(G,T,M,R)$. 

\begin{proposition}\label{scheme group splitting induced root datum}
Let $S$ be a (nonempty) scheme, $(G,T,M,R)$ be a splitting $S$-group, then
\[\mathcal{R}(G,T,M,R)=(M,\check{M},R,\check{R})\]
is a reduced root datum, which is a splitting of the twisted root datum of \cref{scheme group reductive twisted root datum def}.
\end{proposition}
\begin{proof}
This is a trivial concequence of \cref{scheme group reductive twisted root datum def} and \cref{scheme group reductive root system locally equal}.
\end{proof}

\begin{remark}\label{scheme group splitting under base change}
If $S$ is connected and nonempty (resp. if $\Pic(S)=0$), the condition (D1) (resp. (D2)) is automatically satisfied. Moreover, if $(G,T,M,R)$ is a splitting $S$-group, then for any $S'\to S$, $S'\neq\emp$, $(G_{S'},T_{S'},M,R)$ is a splitting $S'$-group and $\mathcal{R}(G,T,M,R)=\mathcal{R}(G_{S'},T_{S'},M,R)$.
\end{remark}

Let $T=D_S(M)$ be a splitting torus. The Lie algebra $\t$ of $T$ is canonically identified with (\cref{scheme diagonalizable group of free group Lie isomorphism})
\[\t\cong\check{M}\otimes_{\Z}\mathscr{O}_S.\]
For any group homomorphism $u:T\to\G_{m,S}$, $\mathfrak{Lie}(u)$ is a linear form
\[\mathfrak{Lie}(u):\t\to\mathscr{O}_S=\mathfrak{Lie}(\G_{m,S}/S).\]
In particular, if $u$ is defined by an element $\alpha\in M$, then $\mathfrak{Lie}(u)$ is the linear form $\alpha_*$ over $\check{M}\otimes_{\Z}\mathscr{O}_S$ defined by
\begin{equation}\label{scheme group splitting torus infinitesimal root formula}
\alpha_*(m\otimes x)=\langle m,\alpha\rangle x.
\end{equation}

On the other hand, for any group homomorphism $h:\G_{m,S}\to T$, $\mathfrak{Lie}(h)$ is an $\mathscr{O}_S$-morphism $\mathscr{O}_S=\mathfrak{Lie}(\G_{m,S}/S)\to\t$ defined canonically by the section
\[H=\mathfrak{Lie}(h)(1)\in\Gamma(S,\t).\]
In particular, if $h$ is defined by an element $m\in\check{M}$, then we have
\begin{equation}\label{scheme group splitting torus infinitesimal coroot formula}
H=\mathfrak{Lie}(h)(1)=m\otimes 1.
\end{equation}
Compare the two definitions, we find in particular that
\begin{equation}\label{scheme group splitting torus root coroot duality formula}
\alpha_*(H)=\langle m,\alpha\rangle\cdot 1\in\Gamma(S,\mathscr{O}_S).
\end{equation}

These definitions apply in particular to the case where $T$ is a maximal torus of a splitting $S$-group. Any root $\alpha\in R$ defines an \textbf{infinitesimal root} $\alpha_*\in\Hom_{\mathscr{O}_S}(\t,\mathscr{O}_S)$ with
\[\alpha_*(m\otimes x)=\langle m,\alpha\rangle x,\]
and any coroot $\alpha\in R$ defines an infinitesimal coroot $H_\alpha=\check{\alpha}\otimes 1\in\Gamma(S,\t)$. For $\alpha,\beta\in R$, we have the relation
\[\alpha_*(H_\beta)=\langle\check{\beta},\alpha\rangle\cdot 1,\]
and in particular $\alpha_*(H_\alpha)=2$. Therefore, if $2$ is invertible over $S$, then $\alpha_*$ and $H_\alpha$ are nonzero over each fiber.

\begin{proposition}\label{scheme group reductive locally splittable if}
Let $S$ be a scheme, $G$ be a reductive $S$-group, $T$ be a maximal torus of $G$. Suppose that $T$ is splitting, then $G$ is locally splittable relative to $T$: for any $s_0\in S$, there exists an open neighborhood $U$ of $s_0$ such that the $U$-group $G_U$ is splittable relative to $T_U$.
\end{proposition}
\begin{proof}
Write $T=D_S(M)$ and $\g=\bigoplus_{m\in M}\g^m$. Let $R=\{m\in M:m\neq 0,\g^m(s_0)\neq 0\}$. By replacing $S$ with an open neighborhood of $s_0$, we can suppose that the $\g^\alpha$, $\alpha\in R$, is free, and the $\g^m$, $m\neq 0$, $m\notin R$, are zero. We then have
\[\g=\t\oplus\bigoplus_{\alpha\in R}\g^\alpha\]
the $\g^\alpha$ being free of rank $1$. It follows that $R$ is a root system for $G$ relative to $T$. The coroots $\check{\alpha}$ correponding to $\alpha\in R$ are then identified with the locally constant function over $S$ with values in $\check{M}$. By restricting $S$, we can suppose that they are constant, which then completes the proof.
\end{proof}

\begin{proposition}\label{scheme group reductive splittable if over Pic=0}
Let $S$ be a connected nonempty scheme such that $\Pic(S)=0$, for example $\Spec(\Z)$ or a local scheme. If $G$ is a reductive $S$-group possessing a splitting maximal torus $T$, then $G$ is splittable relative to $T$.
\end{proposition}
\begin{proof}
In fact, this follows from the proof of \cref{scheme group reductive locally splittable if}, since we do not need to restrict $S$ in this case.
\end{proof}

Recall that a reductive group possesses locally for the \'etale topology a maximal torus (\cref{scheme group fiber reductive fpqc nbhd lifting}), so we obtain the following corollaries:

\begin{corollary}\label{scheme group reductive etale local splittable}
Let $S$ be a scheme, $G$ be a reductive $S$-group (resp. and $T$ be a maximal torus of $G$). Then $G$ is locally splittable (resp. locally splittable relative to $T$) for the \'etale topology over $S$.
\end{corollary}

\begin{corollary}
Let $k$ be a field, $G$ be a reductive $k$-group. There exists a finite separable extension $k'$ over $k$ such that $G_{k'}$ is splittable.
\end{corollary}

\begin{remark}
Using \cref{scheme group reductive locally splittable if} and (\cite{EGA4-3} \S 8), we easily prove the following result: let $G=(G,T,M,R)$ be a splitting $S$-group, there exists a covering of $S$ by $U_i$ such that each splitting group $G_U$ comes from a splittable group over a Noetherian ring (in fact a $\Z$-algebra of finite type) by base change. On the other hand, we shall see that any splitting group over $S$ comes from a splitting $\Z$-group.
\end{remark}

Let $k$ be an algebraically closed field and $G$ be a reductive $k$-group. We have seen that there exists a splitting of $G$ (\cref{scheme group reductive etale local splittable}); let $(G,T,M,R)$ and $(G,T',M',R')$ be two splittings. The root datum $\mathcal{R}(G,T,M,R)$ and $\mathcal{R}(G,T',M',R')$ are then isomorphism. In fact, we can reduce to the case where $T=T'$ (because there exists $g\in G(k)$ such that $T'=\inn(g)T$ (\cite{Chevalley1958} 6, th.4 (c)), and we easily verify that if we transport a splitting by an automorphism of $G$, we obtain an isomorphic root datum); but $S=\Spec(k)$ being connected, the isomorphism $D_k(M)\cong T\cong D_k(M')$ comes from a unique isomorphism $M\cong M'$. By the same reasoning, there exists at most one root system of $G$ relative to $T$.\par
If $G$ is a reductive group over an algebraically closed field $k$, we define the \textbf{type} of $G$ to be the isomorphism class of the root datum defined by an arbitrary splitting of $G$. If $G$ is a torus, of type $M$, then the type of $G$ as a reductive group is given by the trivial root datum $(M,\check{M},\emp,\emp)$. By \cref{scheme group splitting under base change}, it is clear that the type of $G$ is invariant under (algebraically closed) base field change.\par
If $G$ is a reductive $S$-group and $s\in S$, we define the \textbf{type of $G$ at $s$} to be the type of the reductive $\kappa(\bar{s})$-group $G_{\bar{s}}$. For any $S'\to S$ and any $s'\in S'$ lying over $s$, the type of $G_{S'}$ at $s'$ is equal to that of $G$ at $s$. If $G$ is splittable and $(G,T,M,R)$ is a splitting of $G$, then by \cref{scheme group splitting under base change}, the type of $G$ at $s$ is the isomorphism class of $\mathcal{R}(G,T,M,R)$. It then follows easily from \cref{scheme group reductive etale local splittable} and that:

\begin{proposition}\label{scheme group reductive type function locally constant}
Let $G$ be a reductive $S$-group ($S\neq\emp$), the function
\[s\mapsto\text{type of $G$ at $s$}\]
is locally constant over $S$. In particular, there exists a partition of $S$ into open nonempty subschemes such that over each piece $G$ is of constant type. More precisely, let $E$ be the set of types of fibers of $G$, for any $t\in E$, let $S_t$ be the set of points $s\in S$ where $G$ is of type $t$. Then $(S_t)_{t\in E}$ is a partition of $S$ and each $S_t$ is clopen (and nonempty).
\end{proposition}

\subsection{The Weyl group}\label{scheme group reductive Weyl group subsection}
Let $S$ be a scheme, $G$ be a reductive $S$-group, $T$ be a maximal torus of $G$. Then
\[W_G(T)=N_G(T)/T\]
is a finite \'etale $S$-group (\cref{scheme group smooth fp centralizer and normalizer of subtorus prop}). The morphism $n\mapsto\inn(n)$ induces by passing to quotient a canonical monomorphism (which is an open immersion)
\[W_G(T)\to\sAut_{S\dash\Grp}(T).\]
Suppose that $G$ is splittable relative to $T$. Choose a splitting, say $(G,T,M,R)$. We then have a canonical isomorphism (\cref{scheme group sHom of D(M) representable if ft})
\[\sAut_{S\dash\Grp}(T)\stackrel{\sim}{\to}(\Aut_{\Grp}(M))_S.\]
In particular, if $W$ is the Weyl group of the root datum $\mathcal{R}(G)$, we have a monomorphism
\[W_S\to\sAut_{S\dash\Grp}(T).\]

For each root $\alpha\in R$, the symmetry $s_\alpha\in W$ acts over $M$ by
\[s_\alpha(x)=x-\langle\check{\alpha},x\rangle\alpha,\]
hence over $T$ (by the perceding morphism) by
\[s_\alpha(t)=t\cdot\check{\alpha}(\alpha(t)^{-1}).\]
On the other hand, as $\g^\alpha$ is supposed to be free, there exists $X\in\Gamma(S,\g^\alpha)^\times$. Consider the section $w_\alpha(X)\in N_G(T)(S)$, we then have (\cref{scheme group elementary system Weyl group prop})
\[\inn(w_\alpha(X))(t)=s_\alpha(t).\]

As $W$ is generated by the $s_\alpha$, $\alpha\in R$, it follows from the preceding remarks that if we consider $W$ and $N_G(T)(S)/T(S)$ as automorphism groups of $T$, we then have
\[W\sub N_G(T)(S)/T(S)\sub W_G(T)(S).\]
By the definition the constant group $W_S$ assoicated with $S$, we then have a commutative diagram
\[\begin{tikzcd}[row sep=6mm,column sep=6mm]
W_S\ar[rr]\ar[rd]&&W_G(T)\ar[ld]\\
&\sAut_{S\dash\Grp}(T)
\end{tikzcd}\]

\begin{proposition}\label{scheme group splitting Weyl group scheme isomorphic to constant}
Let $S$ be a scheme, $(G,T,M,R)$ be a splitting $S$-group, $W$ be the Weyl group of the root datum $\mathcal{R}(G)$. Then the canonical monomorphism
\[W_S\to W_G(T)=N_G(T)/T\]
is an isomorphism.
\end{proposition}
\begin{proof}
These are \'etale groups over $S$, so it suffices to verify that for any $s\in S$, $W_S(\bar{s})\to W_G(T)(\bar{s})$ is an isomorphism (in fact, by \cite{EGA4-4} 17.3.4, the morphism $W_S\to W_G(T)$ is \'etale; if this is an isomorphism on each fiber, then it is a surjective open immersion by \cite{EGA4-4} 17.9.1, whence an isomorphism). Now this follows, for example, by (\cite{Chevalley1958} \S 11.3, th.2).
\end{proof}

\begin{remark}
Using \cref{scheme group reductive etale local splittable}, \cref{scheme group splitting Weyl group scheme isomorphic to constant} provides us a proof of the fact that the Weyl group $W_G(T)$ of a maximal torus of a reductive $S$-group $G$ is finite over $S$.
\end{remark}

Under the preceding notations, for any $w\in W_G(T)(S)$, we denote by $N_w$ the following fiber product:
\[\begin{tikzcd}
N_w\ar[r]\ar[d]&N_G(T)\ar[d]\\
S\ar[r,"w"]&W_G(T)
\end{tikzcd}\] 
This is a clopen subscheme of $N_G(T)$, which is a principal fiber bundle under $T$ on the left (resp. on the right) by the action $(t,q)\mapsto tq$ (resp. $(q,t)\mapsto qt$). If $n\in N_w(S)$, we have
\[N_{ww'}=n\cdot N_{w'},\quad N_{w'w}=N_{w'}\cdot n.\]
In particular, if $\alpha$ is a root of $G$ relative to $T$, then $N_{s_\alpha}$ is none other than the scheme $N^\times$ of \autoref{scheme group elementary system Weyl group subsection}. If $\g^\alpha$ is free over $S$, we then have $N_{s_\alpha}(S)\neq\emp$.

\begin{corollary}\label{scheme group splitting section of Weyl group inverse image exist}
Under the conditions of \cref{scheme group splitting Weyl group scheme isomorphic to constant}, the morphism
\[N_G(T)(S)\to W_G(T)(S)=\Hom_{\mathrm{loc.const.}}(S,W)\]
is surjective. In particular, for any $w\in W$, there exists $n_w\in N_G(T)(S)$ such that $\inn(n_w)|_T=w$.
\end{corollary}
\begin{proof}
This follows from \cref{scheme group splitting Weyl group scheme isomorphic to constant} and condition (D2) of a splitting.
\end{proof}

\subsection{Homomorphisms of splitting groups}
\paragraph{The big cell}\label{scheme group splitting big cell paragraoh}
Let $(G,T,M,R)$ be a splitting reductive $S$-group. Choose a positive root system $R_+$ in the root datum $\mathcal{R}(G)$ (with a total order), put $R_-=-R_+$, and consider the morphism induced by the product in $G$:
\[u:\prod_{\alpha\in R_-}U_\alpha\times_ST\times_S\prod_{\alpha\in R_+}\to G.\]
This is an open immersion. In fact, as the two members are flat and of finite presentation over $S$, it suffices to verify this over each geometric fiber (\cite{SGA1} \Rmnum{1} 5.7 et \Rmnum{8} 5.4). We are then reduced to the case where $S$ is the spectrum of an algebraically closed field, and by (\cite{Chevalley1958} \S 13.4, cor.2 au th.3), $u$ is radical and dominant. As the tangent map of $u$ at the identity is an isomorphism (definition of a root system), $u$ is equally \'etale, whence a dominant open immersion (\cite{EGA4-4} 17.9.1).\par
The iamge $\Omega$ of this open immersion $u$ is independent of the order chosen for $R_+$ (resp. $R_-$). Since this is a question of comparing open subsets of $G$, it suffices to to prove that they have the same geometric points, so we can still assume that $S$ is the spectrum of an algebraically closed field. But then the assertion is none other than (\cite{Chevalley1958}, \S 13, prop.1 (c) et th.1(a)). We have therefore proved the following:

\begin{proposition}\label{scheme group splitting big cell exist}
Let $(G,T,M,R)$ be a splitting $S$-group. Let $R_+$ be a positive root system of $R$. Then there exists an open subset $\Omega_{R_+}$ of $G$ such that for any total order over $R_+$ (resp. $R_-$), the morphism induced by product in $G$:
\[u:\prod_{\alpha\in R_-}U_\alpha\times_ST\times_S\prod_{\alpha\in R_+}U_\alpha\to G\]
is an open immersion with image $\Omega_{R_+}$. The open subset $\Omega_{R_+}$ is called the \textbf{big cell} corresponding to $R_+$.
\end{proposition}

\begin{remark}
By \cref{scheme group splitting big cell exist}, if we choose for each $\alpha\in R$ an isomorphism of vector bundles $p_\alpha:\G_{a,S}\stackrel{\sim}{\to} U_\alpha$, then the morphism (we put $N=\Card(R_+)=\Card(R_-)$)
\[\G_{a,S}^N\times_ST\times_S\G_{a,S}^N\to G,\quad ((x_\alpha)_{\alpha\in R_-},t,(x_\alpha)_{\alpha\in R_+})\mapsto\prod_{\alpha\in R_-}p_\alpha(x_\alpha)\cdot t\cdot\prod_{\alpha\in R_+}p_\alpha(x_\alpha)\]
is an open immersion, whose image only depends on $R_+$.
\end{remark}

\begin{proposition}\label{scheme group splitting big cell schematically dense}
The scheme $\Omega_{R_+}$ is of finite presentation over $S$ (hence retrocompact in $G$) and is universally schematically dense in $G$ relative to $S$.
\end{proposition}
\begin{proof}
It is clear that $\Omega_{R_+}$ is flat and of finite presentation over $S$ and contains the unit section, therefore cuts over each fiber of $G$ a non-empty (hence dense) open subset; the second assertion therefore follows from (\cite{SGA3-2} \Rmnum{18} 1.3).
\end{proof}

\begin{corollary}\label{scheme group splitting center is intersection of kernel}
Let $(G,T,M,R)$ be a splitting reductive $S$-group, then
\[Z(G)=\bigcap_{\alpha\in R}\ker\alpha.\]
Therefore, $Z(G)$ is representable by a closed subgroup of $G$ and is diagonalizable.
\end{corollary}
\begin{proof}
The second assertion follows easily from the first one (as $T$ is diagonalizable, each $\ker\alpha$ is diagonalizable). To prove the latter, we can recall \cref{scheme smooth affine diagonalizable maximal torus reductive center char} and \cref{scheme smooth affine zero unipotent rank reductive center is center}. We can also give a direct proof as follows: let $S'\to S$, if $t\in T(S')$ and $\alpha(t)=1$ for any $\alpha\in R$, then $\inn(t)$ induces the identity over $T_{S'}$, and over each $(U_\alpha)_{S'}$, $\alpha\in R$, hence also over $(\Omega_{R_+})_{S'}$, and finally over $G_{S'}$ by density, so $t\in Z(G)(S')$. Conversely, as $Z_{G'}(T_{S'})=T_{S'}$ (\cref{scheme alg group reductive prop}), if $g\in G(S')$ centralizes $T_{S'}$ and the $(U_\alpha)_{S'}$, then it is a section of $T_{S'}$ which annihilates $\alpha\in R$.
\end{proof}

\begin{corollary}\label{scheme group reductive center is intersection of maximal tori}
Let $S$ be a scheme, $G$ be a reductive $S$-group. Then the center of $G$ is representable by a closed subgroup of $G$, of multiplicative type, which is also the "intersection of maximal tori of $G$" in the following sense: for any $S'\to S$, $Z(G)(S')$ is the set of $g\in G(S')$ whose inverse image in $G(S'')$, for any $S''\to S'$, is contained in any $T(S'')$, where $T$ runs through maximal tori of $G_{S''}$.
\end{corollary}
\begin{proof}
In view of \cref{scheme group reductive etale local splittable}, the first assertion follows from \cref{scheme group splitting center is intersection of kernel} by descent. Now let $H$ be the "intersection of maximal tori" of $G$, in the preceding sense. We have evidently $Z(G)\sub H$, so by descent, it suffices to prove that $Z(G)=H$ in the case where $G$ is splitting. As $H$ is contained in the intersection of maximal tori of $G$ as the usual sense, this then follows from the following remark: if $(G,T,M,R)$ is a splitting, $\alpha\in R$ and $t\in H(S)$, then for any $X\in\Gamma(S,\g^\alpha)^\times$ and $t'\in T(S)$, we have $t\in\inn(\exp_\alpha(X))(T)(S)$, so
\[\inn(\exp_\alpha(X))(t')=\inn(t)(\inn(\exp_\alpha(X))(t'))=\inn(\exp_\alpha(\alpha(t)X))(t')\]
which implies that $\alpha(t)=1$, and therefore any section $t\in H(S)$ is contained in $Z(G)$ by \cref{scheme group splitting center is intersection of kernel}.
\end{proof}

In the following, for a splitting group $(G,T,M,R)$, we shall identity $T$ with $D_S(M)$. Then $Z(G)$ is none other than $D_S(M/\Lambda_r(R))$, where $\Lambda_r(R)$ is the subgroup of $M$ generated by $R$. If $\{\alpha_1,\dots,\alpha_n\}$ is a simply root system of $R$, we then have (cf. \cref{scheme group elementary system center is kernel of root}):
\[Z(G)=\bigcap\ker\alpha_i=\bigcap Z(Z_{\alpha_i}).\]

\begin{proposition}\label{scheme group splitting twisted morphism by torus}
Let $S$ be a scheme, $(G,T,M,R)$ be a splitting $S$-group, $Q$ be an $S$-torus, $\alpha_0$ be a character of $Q$, $\mathscr{L}$ be an invertible $\mathscr{O}_S$-module, and
\[f:Q\to T,\quad p:\mathbf{W}(\mathscr{L})\to G\]
be group morphisms verifying the following relation
\[p(\alpha_0(t),x)=\inn(f(t))p(x)\]
for any $t\in Q(S')$, $x\in\mathbf{W}(\mathscr{L})(S')$, $S'\to S$. Suppose that $f$ separates the elements of $R$ in the following sense: if $\alpha,\alpha'\in R$ and $m,m'\in\Z$, then $m\alpha\circ f=m'\alpha'\circ f$ implies $m\alpha=m'\alpha'$. Finally, let $s\in S$ be such that $(\alpha_0)_{\bar{s}}\neq 1$ and $p_{\bar{s}}\neq 1$. Then there exists an open neighborhood $U$ of $s$ in $S$, an integer $q>0$ such that $x\mapsto x^q$ is an endomorphism of $\G_{a,U}$, a root $\alpha\in R$ and an isomorphism of $\mathscr{O}_U$-modules
\[h:(\mathscr{L}|_U)^{\otimes q}\stackrel{\sim}{\to} \g^\alpha|_U\]
such that:
\begin{enumerate}
    \item[(\rmnum{1})] $(\alpha\circ f)|_U=(q\alpha_0)|_U$,
    \item[(\rmnum{2})] $p(X)=\exp_\alpha(h(X^q))$ for any $X\in\mathbf{W}(\mathscr{L})(S')$, $S'\to U$. 
\end{enumerate}
Moreover, once $U$ is chosen, $q$, $\alpha$ and $h$ are then uniquely determined.
\end{proposition}

\begin{proposition}\label{scheme group splitting big cell sheaf morphism prop}
Let $(G,T,M,R)$ be a splitting $S$-group, $R_+$ be a positive root system of $R$, $\Omega_{R_+}$ be the corresponding big cell.
\begin{enumerate}
    \item[(a)] Let $H$ be a sub-functor in groups which is separated for the fppf topology. If $f,g:G\rightrightarrows H$ are two group homomorphisms which coincide over $\Omega_{R_+}$, then $f=g$.
    \item[(b)] Let $H$ be an $S$-sheaf in groups for the fppf topology and $f:\Omega_{R_+}\to H$ be an $S$-morphism verifying the following condition: for any $S'\to S$ and any $x,y\in\Omega_{R_+}(S')$ such that $xy\in\Omega_{R_+}(S')$, we have $f(xy)=f(x)f(y)$. Then there exists a (unique) group homomorphism $\bar{f}:G\to H$ extending $f$.
\end{enumerate}
\end{proposition}
\begin{proof}
In fact, by \cref{scheme group splitting big cell schematically dense}, assertion (a) (resp. (b)) follows from (\cite{SGA3-2} \Rmnum{18} 2.2) (resp. (\cite{SGA3-2} \Rmnum{18} 2.3 et 2.4)).
\end{proof}

\begin{remark}
If $\alpha\in R_+$, then we have
\begin{equation}\label{scheme group splitting big cell intersection with Z_alpha}
\Omega_{R_+}\cap G_\alpha=U_{-\alpha}\cdot T\cdot U_\alpha.
\end{equation}
In fact, for any $S'\to S$, if $g=\prod_{\beta\in R_-}p_\beta(x_\beta)\cdot t\cdot\prod_{\beta\in R_+}p_\beta(x_\beta)$ is an element of $\Omega_{R_+}(S')$ and if $t'\in T_\alpha(S'')$, then
\[t'gt'^{-1}=\prod_{\beta\in R_-}p_\beta(\beta(t')x_\beta)t\cdot\prod_{\beta\in R_+}p_\beta(\beta(t')x_\beta).\]
Since $\alpha$ and $-\alpha$ are the only elements of $R$ which takes value $1$ over $T_\alpha$ (the unique maximal torus of $\ker\alpha$), we obtain that $g\in G_\alpha=Z_G(T_\alpha)$ if and only if $x_\beta=0$ for $\beta\neq\pm\alpha$.\par
By (\ref{scheme group splitting big cell intersection with Z_alpha}), we deduce from \cref{scheme group elementary system canonical pair prop} that if $X\in\Gamma(X,\g^\alpha)$ and $Y\in\Gamma(S,\g^{-\alpha})$, then
\[\exp_\alpha(X)\exp_\alpha(Y)\in\Omega_{R_+}(S)\iff\text{$1+XY$ is invertible}.\]
\end{remark}

\paragraph{Morphisms of splitting groups}
Let $S$ be a (nonempty) scheme, $(G,T,M,R)$ and $(G',T',M',R')$ be two splitting $S$-groups. We say that an $S$-group homomorphism $f:G\to G'$ is \textbf{compatible with the splittings}, or defines a \textbf{morphism of splitting groups}, if the restriction of $f$ to $T$ factors into a morphism $f_T:T\to T'$, which is of the form $f_T=D_S(h)$, where $h:M'\to M$ is a morphism of groups satisfying the following conditions: there exists a bijection $d:R\stackrel{\sim}{\to}R'$ and for each $\alpha\in R$ an integer $q(\alpha)>0$ such that $x\mapsto x^{q(\alpha)}$ is an endomorphism of $\G_{a,S}$ and that
\[h(d(\alpha))=q(\alpha)\alpha,\quad h^t(\check{\alpha})=q(\alpha)d(\alpha)\rcheck.\]

It is immediate that $h,d,q(\alpha),\alpha\in R$ are uniquely determined by $f$, We denote by $f=\mathcal{R}(f)$. The $q(\alpha)$ are called root exponents of $f$ (or of $h$).\par
Let $p$ be a prime number (if exists) which is zero over $S$, and put $p=1$ if there does not exists such number. Then $\mathcal{R}(f)$ is a $p$-morphism of root datum in the sense of (\cite{SGA3-3} \Rmnum{21} 6.8). We have hence defined a functor $\mathcal{R}$ from the category of splitting $S$-groups to that of reduced root datums (endowed with $p$-morphisms).

\paragraph{Central quotients of reductive groups}
In this paragraph, we discuss reductive quotient of a reductive $S$-group. First, let us consider the following particular case:
\begin{proposition}\label{scheme group splitting quotient by diagonalizable of center}
Let $S$ be a scheme, $(G,T,M,R)$ be a splitting $S$-group, $N$ be a subgroup of $M$ containing $R$, $Q=D_S(M/N)\sub Z(G)$\footnote{Recall that $Z(G)=D_S(M/\Lambda_r(R))$ in this case, so $Q\sub Z(G)$.}.
\begin{enumerate}
    \item[(\rmnum{1})] $G'=G/Q$ is a reductive $S$-group, $T'=T/Q$ is a maximal torus of $G'$.
    \item[(\rmnum{2})] If we identity $T'$ with $D_S(N)$, then $R\sub N$ is a root system of $G'$ relative to $T'$, $(G',T',N,R)$ is a splitting of $G$, and $\mathcal{R}(G')$ is canonically identified with the induced root datum $\mathcal{R}(G)_N$ (\cite{SGA3-3} \Rmnum{21} 6.5).
    \item[(\rmnum{3})] The canonical morphism $G\to G'$ is compatible with the splittings, with root exponent $1$, and gives by functoriality the canonical morphism $\mathcal{R}(G)_N\to\mathcal{R}(G)$.
\end{enumerate}
\end{proposition}

\begin{corollary}\label{scheme group reductive normal subgroup multiplicative is central quotient}
Let $S$ be a scheme, $G$ be a reductive $S$-group, $Q$ be a normal subgroup of multiplicative type of $G$. Then $Q$ is central in $G$, the quotient $G/Q$ is representable by a reductive $S$-group, and the canonical morphism $G\to G/Q$ is locally splittable for the \'etale topology (with root exponential equals to $1$).
\end{corollary}
\begin{proof}
The first assertion follows from \cref{scheme group multiplicative normal subgroup is central}, and the others are local for the \'etale topology, so we are reduced to \cref{scheme group splitting quotient by diagonalizable of center} by \cref{scheme group reductive etale local splittable}.
\end{proof}

\begin{definition}
Let $G$ be a reductive $S$-group. We say that $G$ is \textbf{adjoint} (resp. \textbf{simply connected}) if for any $s\in S$, the type of $G$ at $s$ is given by an adjoint (resp. simply connected) root datum (cf. \cite{SGA3-3} \Rmnum{21} 6.2.6), i.e. such that $M$ is generated by $R$ (resp. $\check{M}$ is generated by $\check{R}$).
\end{definition}

\begin{proposition}\label{scheme group reductive adjoint simply connected is semi-simple}
Let $G$ be an adjoint (resp. simply connected) reductive $S$-group.
\begin{enumerate}
    \item[(a)] $G$ is semi-simple.
    \item[(b)] If $T$ is a maximal torus of $G$ and $\alpha$ is a root of $G$ relative to $T$, then the infinitesimal root $\alpha_*$ is nonzero over each fiber (resp. $\check{\alpha}$ is a monomorphism and the infinitesimal coroot $H_\alpha$ is nonzero over each fiber).
\end{enumerate}
\end{proposition}
\begin{proof}
The first assertion is trivial by definition, since in both case we have $\rho_r(G)=\rho_s(G)$; assertion (b) can be verified over each geometric fiber, and hence follows from (cf. \cite{SGA3-3} \Rmnum{21} 6.2.8).
\end{proof}

\begin{proposition}\label{scheme group reductive adjoint iff Z(G)}
Let $G$ be a reductive $S$-group.
\begin{enumerate}
    \item[(a)] For $G$ to be adjoint, it is necessary and sufficient that $Z(G)=1$.
    \item[(b)] The quotient group $G/Z(G)$ is an adjoint reductive $S$-group.  
\end{enumerate}
\end{proposition}
\begin{proof}
We may assume that $G$ is splitting, then (a) is trivial (bacause $Z(G)=D_S(M/\Lambda_r(R))$), and (b) follows from \cref{scheme group splitting quotient by diagonalizable of center}.
\end{proof}

Let $G$ be a reductive $S$-group. We define the \textbf{adjoint group} $\ad(G)$ of $G$ to be the group $G/Z(G)$, which is adjoint reductive by \cref{scheme group reductive adjoint iff Z(G)}. The radical of $G$, denoted by $\rad(G)$, is defined to be the maximal torus of $Z(G)$ (unique by \cref{scheme group multiplicative fy unique maximal torus}). The quotient group $G^{\mathrm{ss}}:=G/\rad(G)$ is called the \textbf{semi-simple group associated with $G$}. It is clear that these definitions are compatible with base change. If $s\in S$, then $\rad(G)_{\bar{s}}$ is the radical of $G_{\bar{s}}$ in the usual sense (\cref{scheme alg group reductive prop}).\par
If $(G,T,M,R)$ is a splitting group, recall that we have $Z(G)=D_S(M/\Lambda_r(R))$. $\rad(G)=D_S(M/N)$, where $N=M\cap V(R)$, hence the semi-simple associated with $G$ is endowed with a canonical splitting (\cref{scheme group splitting quotient by diagonalizable of center}) and we have a diagram of splitting $S$-groups
\[\begin{tikzcd}
G\ar[r]&G^{\mathrm{ss}}\ar[r]&\ad(G)
\end{tikzcd}\]
corresponding to the canonical morphism of root datums (\cite{SGA3-3} \Rmnum{21} 6.5.5)
\[\begin{tikzcd}
\ad(\mathcal{R}(G))\ar[r]&\mathcal{R}(G)^{\mathrm{ss}}\ar[r]&\mathcal{R}(G)
\end{tikzcd}\]

\begin{remark}
Let $(G,T,M,R)$ be an adjoint (resp. simply connected) splitting $S$-group, $\Delta$ be a set of simple roots of $R$. Then the family $\{\alpha\}_{\alpha\in\Delta}$ (resp. $\{\check{\alpha}\}_{\alpha\in\Delta}$) induces an isomorphism
\[T\stackrel{\sim}{\to}(\G_{m,S})^\Delta,\quad (\text{resp.}\quad (\G_{m,S})^\Delta\stackrel{\sim}{\to} T).\]
In fact, in this case we have $M=\Lambda_r(R)$ (resp. $\check{M}=\Lambda_r(\check{R})$) and $\Delta$ (resp. $\check{\Delta}$) is a basis for the abelian group $\Lambda_r(R)$ (resp. $\Lambda_r(\check{R})$) (\cite{SGA3-3} \Rmnum{21} 3.1.8).
\end{remark}

\subsection{Subgroups of type (R)}
We are especially interested in reductive groups, but some of the results that we are going to establish are valid more generally for a broader class of groups: groups of type (RR).
\paragraph{Groups of type (RR)}
\begin{definition}
Let $S$ be a scheme, $G$ be an $S$-group. We say that $G$ is of type (RR) if it verifies the following conditions:
\begin{enumerate}
    \item[(\rmnum{1})] $G$ is smooth of finite presentation over $S$, with connected fibers.
    \item[(\rmnum{2})] $G$ possesses locally for the fpqc topology a maximal torus.
    \item[(\rmnum{3})] For any $s\in S$, any maximal torus $T$ of $G_{\bar{s}}$ and any root of $G_{\bar{s}}$ relative to $T_{\bar{s}}$, $\mathfrak{Lie}(G_{\bar{s}})^\alpha$ is of dimension $1$ (as a vector space over $\kappa(\bar{s})$).
    \item[(\rmnum{4})] For any $s\in S$ and any maximal torus $T$ of $G_{\bar{s}}$, denote by $R$ the set of roots of $G_{\bar{s}}$ relative to $T$ and $\Lambda_r(R)$ the subgroup of $X(T)$ generate by $R$, then the content\footnote{The content of a root $\alpha$ is the positive generator of the ideal $\{f(\alpha):f\in\Lambda_r(R)\rcheck\ \}$ of $\Z$; this is the largest integer $c>0$ such that $\alpha/c\in\Lambda_r(R)$.} of any root $\alpha\in R$ in the free abelian group $\Lambda_r(R)$ (which is a positive integer) is invertible over $\kappa(\bar{s})$\footnote{This condition is automatically verified if $S$ has characteristic zero.}.  
\end{enumerate}
\end{definition}

Recall that if $G$ is a smooth connected algebraic group over an algebraically closed field $k$, a Cartan subgroup of $G$ is the centralizer of a maximal torus of $G$, and such a subgroup is smooth and connected (\cite{Chevalley1958} \S 7.1, th.1) and (\cite{SGA3-2} \Rmnum{12} 6.6). If $S$ is an arbitrary scheme and $G$ is a smooth $S$-group of finite type, we define a Cartan subgroup of $G$ to be a smooth subgroup $C$ of $G$ such that, for any $s\in S$, $C_{\bar{s}}$ is a Cartan subgroup of $G_{\bar{s}}$ (cf. \cref{scheme group smooth ft Cartan subgroup def}). In particular, Cartan subgroups of $G$ have connected fibers.\par

\begin{remark}\label{scheme group type (RR) maximal tori Cartan subgroup prop}
In view of (\cite{SGA3-2} \Rmnum{12} 7.1) (where the hypothesis that $G$ is separated is verified since $G$ has connected fibers, cf. \cref{scheme group local fp connected fiber uo is separated and qc}), we see that (\rmnum{1}) and (\rmnum{2}) imply that $G$ possesses locally for the \'etale topology maximal tori (resp. Cartan subgroups), which are locally conjugate for the \'etale topology. Moreover, if $G$ is affine over $S$, then these conditions are equivalent to the following:
\begin{enumerate}
    \item[(\rmnum{1}')] $G$ is smooth and has connected fibers.
    \item[(\rmnum{2}')] The reductive ranks of fibers of $G$ is locally constant (\cref{scheme group affine smooth rho_r and rho_n and maximal tori prop}).
\end{enumerate}
We also note that condition (\rmnum{4}) is verified if any root is an indivisible element in the group gerated by roots. In particular, any reductive group is of type (RR) (cf. \cref{scheme group splitting induced root datum}). Finally, it is clear that the fact of being of type (RR) is stable under base change and local for the fpqc topology.
\end{remark}

\begin{remark}
Let $G$ be an $S$-group of type (RR). By (\cite{SGA3-2} \Rmnum{12} 8.8 (c) et (d)), $G$ possesses a reductive center $Z$ and for any $s\in S$, we have, with the notations of (\rmnum{4}), $Z_{\bar{s}}=\bigcap_{\alpha\in R}\ker\alpha$, whence
\begin{equation}\label{scheme group type (RR) quotient reductive center character char}
\Hom_{\bar{s}\dash\Grp}((T/Z)_{\bar{s}},\G_{m,\bar{s}})\cong\Lambda_r(R).
\end{equation}
Moreover, if $T$ is a maximal torus of $G$, we can apply \cref{scheme group smooth fp centralizer and normalizer of subtorus prop}. In particular, $W_G(T)$ is \'etale, quasi-finite and separated over $S$.
\end{remark}

\begin{proposition}\label{scheme group type (RR) quotient by central subgroup stable}
Let $S$ be a scheme, $G$ be an $S$-group of type (RR), $Q$ be a central subgroup of $G$ of finite presentation over $S$ such that the quotient $G/Q$ is representable (for example if $G$ is affine over $S$ and $Q$ is of multiplicative type, cf. \cref{scheme group multiplicative free action quotient prop}); then $G/Q$ is of type (RR).
\end{proposition}
\begin{proof}
In fact, $G/Q$ is smooth over $S$ (\cref{scheme group fpqc quotient of monomorphism representable prop}), of finite presentation over $S$ (\cite{EGA4-4} 17.7.5) and has connected fibers, so condition (\rmnum{1}) is verified. On the other hand, condition (\rmnum{2}) follows from (\cite{SGA3-2} \Rmnum{12} 7.6), so it remains to verify conditions (\rmnum{3}) and (\rmnum{4}). Put $G'=G/Q$, let $u:G\to G'$ be the canonical morphism, $T'=u(T)$ be the image of $T$, which is a maximal torus of $G'$ (cf. \cite{SGA3-2} \Rmnum{12} 7.1 (e)). For each $\alpha\in R$, we still write $\alpha$ for the character of $T'$ induced by $\alpha$ (we have $Q\cap T\sub\bigcap_{\alpha\in R}\ker\alpha$ by (\ref{scheme group type (RR) quotient reductive center character char})). We first prove the following lemma:
\begin{lemma}\label{scheme group type (RR) quotient by central subgroup Lie algebra decomposition}
Under the conditions of \cref{scheme group type (RR) quotient by central subgroup stable}, let $T=D_S(M)$ be a splitting maximal torus of $G$ and suppose that the decomposition of $\g=\mathfrak{Lie}(G)$ under $\Ad(T)$ is of the form
\[\g=\g^0\oplus\bigoplus_{\alpha\in R}\g^\alpha,\quad R\sub M-\{0\},\]
where for any $s\in S$, $\g^\alpha(s)\neq 0$ for any $\alpha\in R$ (hence $\g^\alpha$ is an invertible $\mathscr{O}_S$-module for any $\alpha\in R$ and $R$ is the set of roots of $G_{\bar{s}}$ relative to $T_{\bar{s}}$ for any $s\in S$). Then the Lie algebra $\g'$ of $G'$ decomposes under $\Ad(T')$ in the following way:
\[\g'=\g'^0\oplus\bigoplus_{\alpha\in R}\g'^\alpha\]
and $\mathfrak{Lie}(u)$ induces an isomorphism from $\g^\alpha$ to $\g'^\alpha$.
\end{lemma}
Let $p=\mathfrak{Lie}(u):\g\to\g'$, then we have $p(\g^\alpha)\sub\g'^\alpha$ for each $\alpha\in R$, and $p(\g^0)\sub\g'^0$. As
\[\ker p=\mathfrak{Lie}(Q)\sub\mathfrak{Lie}(Z_G(T))=\g^0\]
we conclude that $p$ induces a monomorphism from $\g^\alpha$ to $\g'^\alpha$, for each $\alpha\in R$. To prove the lemma, it suffices to consider the case where $S=\Spec(k)$, where $k$ is an algebraically closed field, and in view of the preceding remarks, it then suffices to prove that
\[\rank(\g')=\rank(\g'^0)+\Card(R).\]
Now put $C=Z_G(T)$, $C'=Z_{G'}(T')$; by (\cite{SGA3-2} \Rmnum{12} 7.1 (e)), $u$ induces a faithfully flat morphism $C\to C'$ with kernel $Q$, so we have
\[\dim(C')+\dim(Q)=\dim(C).\]
But $G$, $G'$, $C$ and $C'$ are smooth, so
\begin{align*}
\dim(G)&=\rank(\g)=\rank(\g^0)+\Card(R)=\dim(C)+\Card(R)\\
&=\dim(Q)+\dim(C')+\Card(R)=\dim(Q)+\rank(\g'^0)+\Card(R).
\end{align*}
Since $\rank(\g')=\dim(G')=\dim(G)-\dim(Q)$, we then deduce that 
\[\rank(\g')=\rank(\g'^0)+\Card(R),\]
which completes the proof of the lemma.\par
We now return to the proof of \cref{scheme group type (RR) quotient by central subgroup stable}. We can suppose that $S$ is the spectrum of an algebraically closed field. Let $T$ be a maximal torus of $G$; applying \cref{scheme group type (RR) quotient by central subgroup Lie algebra decomposition}, we then have condition (\rmnum{3}) and (\rmnum{4}) for $G/Q$.
\end{proof}

To utilize the preceding proposition, we introduce the following definition:

\begin{definition}
An $S$-group $G$ is said to be \textbf{of type (RA)} if it is of type (RR) and verifies the following condiiton (\rmnum{4}') (stronger than (\rmnum{4})):
\begin{enumerate}
    \item[(\rmnum{4}')] For any $s\in S$ and any maximal torus $T$ of $G_{\bar{s}}$, any root of $G_{\bar{s}}$ relative to $T$ has a content in $\Hom_{\bar{s}\dash\Grp}(T,\G_{m,\bar{s}})$ which is invertible over $S$.
\end{enumerate}
\end{definition}
It is clear from definition that the fact of being of type (RA) is stable under base change and is local for the fpqc topology.

\begin{remark}\label{scheme group type (RR) quotient by reductive center example}
In view of (\ref{scheme group type (RR) quotient reductive center character char}), any adjoint reductive $S$-group is of type (RA). More generally, let $G$ be an $S$-group of type (RR) and $Z$ be its reductive center. Suppose that $G/Z$ is representable (for example if $G$ is affine over $S$, or $S$ is Artinian), then $G/Z$ is of type (RA).
\end{remark}

\paragraph{Subgroups of type (R)}
Let $S$ be an $S$-scheme, $G$ be a smooth $S$-group of finite presentation with connected fibers, $H$ be a subgroup of $G$. We say that $H$ is of type (R) if:
\begin{enumerate}
    \item[(\rmnum{1})] $H$ is smooth, of finite presentation over $S$ and has connected fibers\footnote{The hypothesis that $G$ (resp. $H$) is of finite presentation over $S$ is automatically satisfied if $G$ (resp. $H$) is smooth and has connected fibers (\cref{scheme group local fp connected fiber uo is separated and qc}).}.
    \item[(\rmnum{2})] For any $s\in S$, $H_{\bar{s}}$ contains a Cartan subgroup of $G_{\bar{s}}$.
\end{enumerate}
This notion is clearly stable under base change and local for the fpqc topology. Moreover, by (\cite{SGA3-2} \Rmnum{12} 7.9), under the preceding conditions, we have $H=N_G(H)^0$, and if $G$ contains locally for the \'etale topology Cartan subgroups (resp. maximal tori), then so does $H$, and the Cartan subgroups (resp. maximal tori) of $H$ are those of $G$.

\begin{example}[\textbf{Examples of subgroups of type (R)}]\label{scheme subgroup of type (R) example}
\mbox{}
\begin{enumerate}
    \item[(a)] \textit{Borel subgroups}: a Borel subgroup of $G$ is a subgroup of type (R) of $G$ whose geometric fibers are Borel subgroups of those of $G$\footnote{This amounts to saying that $H$ is a smooth subgroup of $G$, whose geometric fibers $H_{\bar{s}}$ is a Borel subgroup of $G_{\bar{s}}$ (since any Borel subgroup of $G_{\bar{s}}$ is connected and contains a Cartan subgroup of $G_{\bar{s}}$).}.
    \item[(b)] \textit{Parabolic subgroups}: a parabolic subgroup of $G$ is a subgroup of type (R) whose geometric fibers contain Borel subgroups.
\end{enumerate}
\end{example}

\begin{remark}\label{scheme subgroup of type (R) contain centralizer}
If $H$ is a subgroup of type (R) of $G$ and $T$ is a maximal torus contained in $H$, then $H$ also contains $C=Z_G(T)$. In fact, by hypothesis $H$ contains $Z_G(T')$ for a certain maximal torus $T'$ of $G$. Then $T$ and $T'$ are both maximal tori of $H$, so they are conjugate in $H$, hence $H$ also contains $C=Z_G(T)$.
\end{remark}

\begin{proposition}\label{scheme subgroup of type (R) subgroup of type (R) iff}
Let $G$ be as above and $K\sub H$ be subgroups of $G$, with $H$ being of type (R). Then $K$ is a subgroup of type (R) of $H$ if and only if it is a subgroup of type (R) of $G$.
\end{proposition}
\begin{proof}
In fact, let $s\in S$, as $H$ is of type (R), any maximal torus of $H_{\bar{s}}$ is a maximal torus of $G_{\bar{s}}$, and hence the same is true for Cartan subgroups.
\end{proof}

\begin{proposition}\label{scheme group smooth subgroup of type (R) intersection with centralizer}
Let $G$ be a smooth $S$-group of finite presentation with connected fibers, $T$ be a maximal torus of $G$, $Q$ be a subtorus of $T$, $Z=Z_G(Q)$. If $H$ is a subgroup of type (R) of $G$ containing $T$, then $H\cap Z$ is a subgroup of type (R) of $Z$.
\end{proposition}
\begin{proof}
We first recall that $Z$ is a smooth closed subgroup of $G$ (\cref{scheme multiplicative to smooth transporter formally smooth}), of finite presentation over $S$ and has connected fibers (\cite{SGA3-2} \Rmnum{12} 6.6). Similarly, $H\cap Z$ is of finite presentation, smooth and has connected fibers (because $H\cap Z=Z_H(Q)$); moreover, $H\cap Z\sups Z_G(T)$, which proves the assertion.
\end{proof}

\begin{proposition}\label{scheme group type (RR) (RA) subgroup of type (R) prop}
Let $S$ be a scheme, $G$ be an $S$-group of type (RR) (resp. of type (RA)), and $H$ be a subgroup of type (R) of $G$. Then $H$ is a subgroup of type (RR) (resp. of type (RA)).
\end{proposition}
\begin{proof}
In fact (\rmnum{1}) and (\rmnum{2}) are clear, (\rmnum{3}) and (\rmnum{4}) (resp. (\rmnum{4}')) can be checked when $S$ is the spectrum of an algebraically closed field. Then $H$ contains a maximal torus $T$ of $G$ (and its centralizer $C=Z_G(T)$, cf. \cref{scheme subgroup of type (R) contain centralizer}), and the assertion follows easily from the following lemma:
\begin{lemma}\label{scheme group type (RR) subgroup of type (R) Lie algebra decomposition}
Let $S$ be a scheme, $G$ be an $S$-group of type (RR), $T$ be a maximal torus of $G$ endowed with a trivialization $T\cong D_S(M)$, and suppose that
\[\g=\g^0\oplus\bigoplus_{\alpha\in R}\g^\alpha\]
(the $\g^\alpha$ being invertible $\mathscr{O}_S$-modules). Let $H$ be a subgroup of type (R) containing $C=Z_G(T)$ (i.e. containing $T$). Then $\h=\mathfrak{Lie}(H/S)$ is locally over $S$ of the form
\[\g^0+\bigoplus_{\alpha\in R'}=\g_{R'}.\]
More precisely, for any $s\in S$, let $R'(s)=\{\alpha\in R:\g^\alpha(s)\sub\h(s)\}$. Then $R'(s)$ is a locally constant function on $s$; if $U$ is an open subset of $S$ over which $R'(s)\equiv R'$, then have
\[\h_U=\g^0|_U\oplus\bigoplus_{\alpha\in R'}\g_U^\alpha.\]
\end{lemma}
In fact, $\h$ is locally a direct factor of $\g$, containing $\g^0$ and stable under $T$.
\end{proof}

\paragraph{Strict transporter of subgroups of type (R)}
Let $S$ be a scheme, $G$ be a smooth $S$-group, $\g=\mathfrak{Lie}(G/S)$ and $\h$ be a locally direct factor of $\g$. The $\mathscr{O}_S$-algebra $\mathscr{A}=\bm{S}(\omega_{G/S}^1)$ is locally free, so the scheme $\Lie(G/S)=\mathbf{W}(\g)=\Spec(\mathscr{A})$ is essentially free in the sense of \autoref{sheme functor representability of Res subsection}. As $\mathbf{W}(\h)$ is a closed subscheme of $\Lie(G/S)$, of finite presentation over $\Lie(G/S)$, by \cref{scheme subfunctor Weil restriction example}~(a) we see that $N=N_G(\h)$ is representable by a closed subgroup of $G$, of finite presentation over $G$. On the other hand, by \cref{scheme group normalizer and centralizer of rep Lie prop}~(b), we have $\Lie(N/S)=N_{\Lie(G/S)}(\h)$, and by \cref{scheme group smooth at unit section iff}, if $N$ is smooth over $S$ at the unit section, then the subfunctor in groups $N^0$ is representable by an open subgroup of $N$, which is smooth over $S$.

\begin{proposition}\label{scheme subgroup type (R) normalizer smooth prop}
Let $S$ be a scheme, $G$ be an $S$-group of type (RA), $H$ be a subgroup of type (R) of $G$, and $\g\sups\h$ be their Lie algebras. Then $N_G(\h)$ is smooth over $S$ at the unit section and we have $N_G(\h)^0=H$.
\end{proposition}
\begin{proof}
Put $N=N_G(\h)$ and $\n=\mathfrak{Lie}(N/S)$; then $H\sub N$ and, by \cref{scheme group normalizer and centralizer of rep Lie prop}, for any $s\in S$ we have
\[\h(s)\sub\n(s)=N_{\g(s)}(\h(s)).\]
From \cref{scheme group type (RA) subgroup of type (R) Lie algebra self-normalizing} below, we see that $\h(s)=N_{\g(s)}(\h(s))$ for any $s\in S$, and as $H$ is smooth over $S$, we also have $\dim_{\kappa(s)}(\h(s))=\dim(H_s)$ (cf. \cite{DG} \S\Rmnum{2}.5, th.2.1). We then obtain that
\[\dim_{\kappa(s)}(\n(s))=\dim_{\kappa(s)}(\h(s))=\dim(H_s)\leq\dim(N_s)\]
whence $N_s^0=H_s^0=H_s$ ($H$ has connected fibers). It then follows that the subfunctor in groups $H^0$ is represented by the smooth $S$-group $H$.
\end{proof}
\begin{lemma}\label{scheme group type (RA) subgroup of type (R) Lie algebra self-normalizing}
Under the conditions of \cref{scheme group type (RR) subgroup of type (R) Lie algebra decomposition}, if $G$ is of type (RA), we have, for any $s\in S$, 
\[N_{\g(s)}(\h(s))=\h(s).\]
\end{lemma}
\begin{proof}
We may assume that $S$ is the spectrum of a field, hence $\h=\g_{R'}$ for a subset $R'\sub R$ (by \cref{scheme group type (RR) subgroup of type (R) Lie algebra decomposition}). If $h\in\t$ and $x\in\g^\alpha$, we have $[h,x]=\alpha_*(h)x$, where $\alpha_*:\t\to\mathscr{O}_S$ is the morphism induced by $\alpha$. Condition (\rmnum{4}') says precisely that $\alpha_*\neq 0$ for any $\alpha\in R$\footnote{We may assume that $T$ is splitting and $S=\Spec(k)$, where $k$ is an algebraically closed field. Then by the formula (\ref{scheme group splitting torus infinitesimal root formula}) (and its notations), it suffices to show that if the content of $\alpha$ in $M$ is invertible over $k$, then there exists an element $m\otimes x\in\check{M}\otimes_{\Z}k$ such that $\langle m,\alpha\rangle x\neq 0$. Now the hypothesis implies that the ideal $\{\langle m,\alpha\rangle:m\in\check{M}\}$ is coprime to the characteristic of $S$, so its image in $k$ is nonzero, whence the assertion.}, so $[h,x]\in\h$ if and only if $x\in\h$. This implies $N_{\g}(\h)=\h$ since the former is contained in $\STrans(\t,\h)$.
\end{proof}

\begin{corollary}\label{scheme group type (RA) subgroup of type (R) equal iff Lie algebra equal}
Let $S$ be a scheme, $G$ be an $S$-group of type (RA), $H$ and $H'$ be subgroups of type (R) of $G$, and $\h$, $\h'$ be their Lie algebras. Then $H=H'$ if and only if $\h=\h'$.
\end{corollary}
\begin{proof}
In fact, one direction is trivial, and the other direction follows from the equality $H=N_G(\h)^0$ of \cref{scheme subgroup type (R) normalizer smooth prop}. 
\end{proof}

\begin{corollary}\label{scheme group type (RA) subgroup of type (R) and stable Lie subalgebra correspond}
Under the conditions of \cref{scheme group type (RR) subgroup of type (R) Lie algebra decomposition}, with $G$ being of type (RA), the maps
\[H\mapsto\mathfrak{Lie}(H/S),\quad \h\mapsto N_G(\h)^0\]
give a bijective correspondence between the set of subgroups of type (R) of $G$ containing $T$ and the set of subalgebras of $\g$ containing $\g^0$, stable under $T$, and whose normalizer in $G$ is smooth over $S$ at the unit section.
\end{corollary}
\begin{proof}
Let $\h$ be a subalgebra of $\g$ verifying the properties above. Then as has been remarked at the begining of this paragraph, $H=N_G(\h)^0$ is a smooth $S$-group. Moreover, as $C=Z_G(T)$ has connected fibers (\cite{SGA3-2} 6.6) and stablizes each $\g^\alpha$\footnote{If $x\in\Gamma(S,\g^\alpha)$, $t\in T(S)$ and $g\in Z_G(T)(S)$, then have $\Ad(t)(\Ad(g)x)=\Ad(g)(\Ad(t)x)=\alpha(t)\Ad(g)x$, so $\Ad(t)$ stablizes $\g^\alpha$.}, we have $C\sub H$, so $H$ is a subgroup of type (R) of $G$. By \cref{scheme group normalizer and centralizer of rep Lie prop}~(b), we have $\mathfrak{Lie}(H)=N_\g(\h)$, which is equal to $\h$ by the proof of \cref{scheme group type (RA) subgroup of type (R) Lie algebra self-normalizing}.
\end{proof}

\begin{corollary}\label{scheme group type (RR) subgroup of type (R) equal iff Lie algebra equal}
Let $S$ be an $S$-scheme, $G$ be an $S$-group of type (RR), $T$ be a maximal torus of $G$, $H$ and $H'$ be subgroups of type (R) of $G$ containing $T$. Then $H=H'$ if and only if $\h=\h'$.
\end{corollary}
\begin{proof}
In view of the finite presentation hypothesis, we can reduce to the case where $S$ is Noetherian (cf. \cite{EGA4-3} \S 8 et \cite{SGA3-1} $\Rmnum{6}_B$ \S 10); it then suffices to verify that $\h=\h'$ implies $H_{S'}=H'_{S'}$ for any $S'$ being the spectrum of an Artinian quotient of a local ring of $S$\footnote{In fact, let $g\in G$, $s\in S$ be its image, $\m$ be the maximal ideal of $\mathscr{O}_{G,g}$, $\n$ be that of $\mathscr{O}_{S,s}$, and $\mathfrak{I}$ (resp. $\mathfrak{I}'$) be the kernel of the morphism $\mathscr{O}_{G,g}\to\mathscr{O}_{H,g}$ (resp. $\mathscr{O}_{G,g}\to\mathscr{O}_{H',g}$) (which is zero if $g\notin H$ (resp. $g\notin H'$)). As $\mathscr{O}_{G,g}$ is Noetherian, $\mathfrak{I}$ and $\mathfrak{I}'$ are separated for the $\m$-adic topology, hence a foriori for the $\n$-adic topology, it suffices to show that $\mathfrak{I}+\n^n\mathscr{O}_{G,g}=\mathfrak{I}'+\n^n\mathscr{O}_{G,g}$ for any $n\in\N$.}. We are then reduced to the case where $S'$ is Artinian, and where we can apply \cref{scheme group type (RR) quotient by reductive center example}. Let $u:G\to G'=G/Z$ be the canonical morphism and $T'=T/Z$ the maximal torus of $G'$ corresponding to $T$. In view of (\cite{SGA3-2} \Rmnum{12} 7.12), there exists subgroups $H_1$ and $H_1'$ of $G'$ of type (R), containing $T'$, such that $H=u^{-1}(H_1)$ and $H'=u^{-1}(H_1')$. It suffices to prove that $H_1=H_1'$, but by \cref{scheme group type (RR) subgroup of type (R) Lie algebra decomposition} and \cref{scheme group type (RR) quotient by central subgroup Lie algebra decomposition}, we have
\[\mathfrak{Lie}(H_1)=\mathfrak{Lie}(H_1')\]
and we are reduced to \cref{scheme group type (RA) subgroup of type (R) equal iff Lie algebra equal}.
\end{proof}

\begin{remark}\label{scheme group type (RR) maximal torus same Lie algebra example}
The fact that $H$ and $H'$ contain the same maximal torus $T$ is essential for the validity of \cref{scheme group type (RR) subgroup of type (R) equal iff Lie algebra equal}. For example, consider $G=\SL_{2,k}$, where $k$ is an algebraically closed field of characteristic $2$. Then any maximal tori of $G$ are conjugate under $G(k)$, and hence conjugate to the maximal torus $T$ of diagonal matrices:
\[T=\left\{\begin{pmatrix}
x&0\\
0&x^{-1}
\end{pmatrix}:x\in k^\times\right\}.\]
Now we note that if $g=(\begin{smallmatrix}a&b\\c&d\end{smallmatrix})\in G(k)$, then
\[T'(k)=gT(k)g^{-1}=\left\{\begin{pmatrix}
adx-bcx^{-1}&ab(x^{-1}-x)\\
cd(x-x^{-1})&adx^{-1}-bcx
\end{pmatrix}:x\in k^\times\right\},\]
so the Lie algebra of $T'$ is given by (note that $ad+bc=ad-bc=1$)
\[\mathfrak{Lie}(T')(k)=\left\{\begin{pmatrix}
(ad+bc)x&0\\
0&-(ad+bc)x
\end{pmatrix}:x\in k\right\}=\left\{\begin{pmatrix}
x&0\\
0&-x
\end{pmatrix}:x\in k\right\}=\mathfrak{Lie}(T)(k).\]
In other words, any maximal torus of $G$ has Lie algebra $k(\begin{smallmatrix}1&0\\0&-1\end{smallmatrix})\sub\mathcal{M}_2(k)$ (which is invariant under the adjoint action).
\end{remark}

\begin{corollary}\label{scheme group type (RR) subgroup of type (R) coincide locus clopen}
Let $S$ be a scheme, $G$ be an $S$-group of type (RR), $T$ be a maximal torus of $G$, $H$ and $H'$ be subgroups of type (R) containing $T$. Then the set $U$ of $s\in S$ such that $H_s=H'_s$ is clopen in $S$ and $H|_U=H'|_U$.
\end{corollary}
\begin{proof}
This follows easily from \cref{scheme group type (RR) subgroup of type (R) equal iff Lie algebra equal} and \cref{scheme group type (RR) subgroup of type (R) Lie algebra decomposition}.
\end{proof}

\begin{corollary}\label{scheme group type (RR) functor of subgroup of type (R) formally unramified}
The functor of subgroups of type (R) containing $T$, where $T$ is a maximal torus of an $S$-group $G$ of type (RR), is formally unramified.
\end{corollary}
\begin{proof}
Let $\mathscr{R}$ be the functor of subgroups of type (R) containing $T$, it suffices to prove that if $S$ is affine and $S_0$ is a closed subscheme of $S$ defined by a nilpotent ideal, then the map
\[\mathscr{R}(S)\to\mathscr{R}(S_0)\]
is injective. But this follows from \cref{scheme group type (RR) subgroup of type (R) coincide locus clopen}, since $S$ and $S_0$ have the same residue fields.
\end{proof}

\begin{theorem}\label{scheme group type (RR) strict transporter of subgroup of type (R) representable}
Let $G$ be an $S$-group of type (RR), $H$, $H'$ be two subgroups of type (R) of $G$. Let $\STrans_G(H,H')$ be the strict transporter of $H$ in $H'$ defined by
\[\STrans_G(H,H')(S')=\{g\in G(S'):\inn(g)H_{S'}=H'_{S'}\}.\]
Then $\STrans_G(H,H')$ is representable by a closed subscheme of $G$, which is smooth and of finite presentation over $S$.
\end{theorem}
\begin{proof}
The fact that $\STrans_G(H,H')$ is representable by a closed subscheme of $G$ and of finite presentation over $S$ follows from (\cite{SGA3-2} \Rmnum{11} 6.11 (a)). To prove that it is smooth over $S$, it suffices to show that if $S$ is affine, $S_0$ is a closed subscheme of $S$ defined by a nilpotent ideal $\mathfrak{J}$, and if $g_0\in G(S_0)$ and $\inn(g_0)H_0=H_0'$, then there exists $g\in G(S)$, projecting to $g_0$ and such that $\inn(g)H=H'$. As the question of smoothness is local for the fpqc topology (\cite{EGA4-4} 17.7.3), we can suppose that $H$ contains a maximal torus $T$ of $G$. Then $T_0$ is a maximal torus of $H_0$, so $\inn(g_0)T_0$ is a maximal torus of $H'_0$. By \cref{scheme group multiplicative morphism nilpotent lifting exist}, there exists a torus $T'$ of $H'$ such that $T'_0=\inn(g_0)T_0$, and by \cref{scheme group multiplicative morphism conjugation if nilpotent reduction}, there exists $g\in G(S)$, projecting to $g_0$ and such that $\inn(g)=T'$. By replacing $H$ with $\inn(g)H$, we can hence suppose that $H$ and $H'$ contain the same maximal torus $T$ and that $H_0=H'_0$. But then $H=H'$ by \cref{scheme group type (RR) subgroup of type (R) coincide locus clopen}.
\end{proof}

\begin{corollary}\label{scheme group type (RR) normalizer of subgroup of type (R) representable}
Let $G$ be an $S$-group of type (RR), $H$ be a subgroup of type (R) of $G$. Then $N_G(H)$ is representable by a closed subgroup of $G$, of finite presentation and smooth over $S$.
\end{corollary}

Now using the reasoning of deducing \cref{scheme group multiplicative to affine smooth morphism conjugate iff} from \cref{scheme multiplicative to affine smooth transporter representable}, we obtain:

\begin{corollary}\label{scheme group type (RR) subgroup of type (R) conjugate iff}
Under the hypotheses of \cref{scheme group type (RR) strict transporter of subgroup of type (R) representable}, the following conditions are equivalent:
\begin{enumerate}
    \item[(\rmnum{1})] $H$ and $H'$ are locally conjugate in $G$ for the \'etale topology.
    \item[(\rmnum{1}')] $H$ and $H'$ are locally conjugate in $G$ for the fpqc topology.
    \item[(\rmnum{2})] For any $s\in S$, $H_{\bar{s}}$ and $H'_{\bar{s}}$ are conjugate by an element of $G(\bar{s})$.
    \item[(\rmnum{2}')] The structural morphism $\STrans_G(H,H')\to S$ is surjective.
    \item[(\rmnum{3})] $\STrans_G(H,H')$ is a principal homogeneous bundle under the action of $N_G(H)$.    
\end{enumerate}
\end{corollary}

Now using (\cite{Chevalley1958} \S 6.4, th.4 et \S 9.3), we then obtain by \cref{scheme group type (RR) subgroup of type (R) conjugate iff} and \cref{scheme group type (RR) normalizer of subgroup of type (R) representable}:

\begin{corollary}\label{scheme group type (RR) Borel subgroup conjugate prop}
Let $G$ be an $S$-group of type (RR). The Borel subgroups of $G$ are closed in $G$, self-normalizing, and locally conjugate for the \'etale topology.
\end{corollary}

Let $S$ be a scheme and $G$ be a smooth $S$-group of finite presentation over $S$ with connected fibers. A \textbf{Killing couple} of $G$ is defined to be a couple $T\sub B$, where $T$ is a maximal torus of $G$ and $B$ is a Borel subgroup of $G$ containing $T$. Now using the conjugation of maximal tori in $B$ (\cref{scheme group type (RR) maximal tori Cartan subgroup prop}), we have:

\begin{corollary}\label{scheme group type (RR) Killing couple conjugate}
Let $G$ be an $S$-group of type (RR). The Killing couples of $G$ are locally conjugate for the \'etale topology.
\end{corollary}

\begin{corollary}\label{scheme group type (RR) functor of Borel subgroup containing torus is principal homogeneous}
Let $G$ be an $S$-group of type (RR), $T$ be a maximal torus of $G$, $W_G(T)=N_G(T)/Z_G(T)$ be the corresponding Weyl group (\cref{scheme group smooth fp centralizer and normalizer of subtorus prop}). Then the functor of Borel subgroups of $G$ containing $T$ is formally principal homogeneous under $W_G(T)$.
\end{corollary}
\begin{proof}
If $B$ is a Borel subgroup of $G$ containing $T$ and $B'$ is a conjugate of $B$ containing $T$, then by \cref{scheme group type (RR) Borel subgroup conjugate prop}, $B$ and $B'$ are locally conjugate for the \'etale topology by a section $g$ of $G$ (i.e. $B'=gBg^{-1}$ over an \'etale surjective morphism $S'\to S$). The conjugate $T'=gTg^{-1}$ by $g$ is then a maximal torus of $B'$, and hence is conjugate to $T$ by a \'etale local section $h\in B'(S'')$ ($S''\to S'$ being \'etale surjective, cf. \cite{SGA3-2} \Rmnum{12} 7.1). But then $hg\in N_G(T)(S'')$, and $B'=hgBg^{-1}h^{-1}$, so $N_G(T)$ acts transitively on Borel subgroups of $G$ containing $T$ (for the \'etale topology). Finally, the assertion follows from the fact that $N_G(B)=B$ for a Borel subgroup $B$ (\cref{scheme group type (RR) Borel subgroup conjugate prop}) and (cf. \cref{scheme alg group Borel subgroup normalizer of Cartan subgroup})
\begin{equation*}
N_G(T)\cap B=Z_G(T).\qedhere
\end{equation*}
\end{proof}

\begin{proposition}\label{scheme group type (RR) subgroup type (R) Weyl group exact sequence}
Let $G$ be an $S$-group of type (RR), $H$ be a subgroup of type (R), $N=N_G(H)$ be its normalizer. Let $T$ be a maximal torus of $H$, $W_H(T)$ and $W_N(T)$ be the corresponding Weyl groups (\'etale, quasi-finite and separated over $S$ by \cref{scheme group smooth fp centralizer and normalizer of subtorus prop}). Then we have an exact sequence of sheaves for the \'etale topology:
\[\begin{tikzcd}
1\ar[r]&W_H(T)\ar[r]&W_N(T)\ar[r]&N/H\ar[r]&1
\end{tikzcd}\]
\end{proposition}
\begin{proof}
These morphisms are induced by the morphisms $N_G(T)\to N_N(T)\to N/H$, noting that $Z_H(T)\sub Z_N(T)\sub Z_G(T)\sub H$ (\cref{scheme subgroup of type (R) contain centralizer}). Since $Z_H(T)=Z_N(T)\cap H$, we see that the morphism $W_H(T)\to W_N(T)$ is injective, so it suffices to show that the morphism $W_N(T)\to N/H$ is an epimorphism, so let $n\in N(S')$, $S'\to S$. The two maximal tori $T$ and $\inn(n)T$ of $H$ are locally conjugate in $H$ for the \'etale topology, so there exists a covering family $\{S_i'\to S'\}$ and for each $i$, a section $h_i\in H(S'_i)$ such that $\inn(h_i)T=\inn(n)T$. We then have $nh_i^{-1}\in N_N(T)$, which proves the assertion.
\end{proof}

\begin{remark}
The Weyl group $W_N(T)$ can also be descriped in the following way: suppose that we are in the situation of \cref{scheme group type (RR) subgroup of type (R) Lie algebra decomposition}, with $\h=\g_{R'}$. Then $W_N(T)$ equals to $N_W(R')$, the sheaf of sections of $W=W_G(T)$ which normalizes $R'$. In fact, by \cref{scheme group type (RR) subgroup of type (R) equal iff Lie algebra equal}, we have
\[N_N(T)=N_G(H)\cap N_G(T)=N_G(\h)\cap N_G(T).\]
\end{remark}

\begin{corollary}\label{scheme group type (RR) Weyl group finite subgroup type (R) closed iff}
Let $G$ be an $S$-group of type (RR), $H$ be a subgroup of type (R), and $N=N_G(H)$ be its normalizer. Suppose that the Weyl group of $G$ is finite, i.e. that for any $S'\to S$ and any maximal torus $T$ of $G_{S'}$, the \'etale $S'$-scheme $N_{G_{S'}}(T)/Z_{G_{S'}}(T)$ is finite. Then the following conditions are equivalent:
\begin{enumerate}
    \item[(\rmnum{1})] $H$ is closed in $G$.
    \item[(\rmnum{2})] $N/H$ is representable by a finite \'etale $S$-scheme.
    \item[(\rmnum{3})] The Weyl group of $H$ is finite. 
\end{enumerate}
\end{corollary}
\begin{proof}
Since the conditions are all local for the fpqc topology, we can suppose that $H$ possesses a maximal torus $T$. By \cref{scheme group type (RR) normalizer of subgroup of type (R) representable}, $N$ is closed in $G$, so $W_N(T)=N\cap W_G(T)$ is closed in $W_G(T)$ and hence finite over $S$. We have evidently (\rmnum{1})$\Rightarrow$(\rmnum{3}) since $W_H(T)=W_G(T)\cap H$, and (\rmnum{3})$\Rightarrow$(\rmnum{2}) follows from the exact sequence of \cref{scheme group type (RR) subgroup type (R) Weyl group exact sequence}. Finally, we have (\rmnum{2})$\Rightarrow$(\rmnum{1}) because if $N/H$ is finite, it is separated, so the unit section $\bar{e}$ in $N/H$ is closed. Since $H$ is equals to the inverse image of $\bar{e}$ under $N\to N/H$, we conclude that $H$ is closed in $N$ ($N\to N/H$ being faithfully flat and quasi-compact), and hence in $G$.
\end{proof}

\paragraph{Subgroups of type (R) of a splitting reductive group}
If $H$ is a subgroup of type (R) of a reductive group $G$, then $H$ contains locally for the \'etale topology a maximal torus of $G$ (\cite{SGA3-2} \Rmnum{12} 7.9). By \cref{scheme group reductive etale local splittable}, we can, locally for the \'etale topology, suppose that $G$ splits relative to this torus. Let $(G,T,M,R)$ be a splitting $S$-group, $H$ be a subgroup of type (R) of $G$ containing $T$. By \cref{scheme group type (RR) subgroup of type (R) equal iff Lie algebra equal}, such a subgroup is characterized by its Lie algebra, which is locally over $S$ of the form $\g_{R'}$ (\cref{scheme group type (RR) subgroup of type (R) Lie algebra decomposition}):
\[\g_{R'}=\t\oplus\bigoplus_{\alpha\in R'}\g^\alpha.\]

\begin{definition}
Let $(G,T,M,R)$ be a splitting $S$-group. A subset $R'$ of $R$ is said to be \textbf{of type (R)} if $\g_{R'}$ is the Lie algebra of a subgroup of type (R) of $G$ containing $T$. This group, uniquely determined by $R'$, is denoted by $H_{R'}$.
\end{definition}

\begin{proposition}\label{scheme group splitting subgroup of type (R) relation with U_alpha Z_alpha}
Under the preceding conditions, we have the following equivalences (cf.  \autoref{scheme group splitting root datum subsection}):
\begin{align*}
H\cap G_\alpha=T &\iff \alpha\notin R',-\alpha\notin R',\\
H\sups U_\alpha &\iff \alpha\in R',\\
H\cap U_\alpha=e &\iff \alpha\notin R',\\
H\sups G_\alpha &\iff \alpha\in R',-\alpha\in R'
\end{align*}
\end{proposition}
\begin{proof}
Since $G_\alpha=Z_G(T_\alpha)$, where $T_\alpha$ is the unique maximal torus of $\ker\alpha$, by \cref{scheme group smooth subgroup of type (R) intersection with centralizer} we see that $H\cap G_\alpha$ is a subgroup of type (R) of $G_\alpha$; but a subgroup of type (R) of $G_\alpha$, containing $T$, is locally equal to the following subgroups: $T$, $T\cdot U_\alpha$, $T\cdot U_{-\alpha}$, $G_\alpha$, by \cref{scheme group type (RR) subgroup of type (R) equal iff Lie algebra equal}.
\end{proof}

\begin{proposition}\label{scheme group splitting subset of type (R) Omega open immerison}
Let $(G,T,M,R)$ be a splitting $S$-group and $R'\sub R$ be a subset of type (R). Let $R_+$ be a positive system of roots and choose an order over $R_+$. Then the morphism
\[\Omega_{R_+,R'}=\prod_{\alpha\in R'\cap R_-}U_\alpha\times_ST\times_S\prod_{\alpha\in R'\cap R_+}U_\alpha\to G\]
induced by the product in $G$, induces an open immersion $\Omega_{R_+,R'}\to H_{R'}$.
\end{proposition}
\begin{proof}
By \cref{scheme group splitting subgroup of type (R) relation with U_alpha Z_alpha}, this morphism factors through $H_{R'}$ and hence induces an open immersion $\Omega_{R_+,R'}\to H_{R'}$. The assertion then follows as in \cref{scheme group splitting big cell exist}.
\end{proof}

\begin{proposition}\label{scheme group splitting intersection of subgroup of type (R) char}
Let $(G,T,M,R)$ be a splitting $S$-group and $R'$, $R''$ be subsets of $R$ of type (R).
\begin{enumerate}
    \item[(a)] $H_{R'}\cap H_{R''}$ is smooth at the unit section, $R'\cap R''$ is of type (R), and we have 
    \[(H_{R'}\cap H_{R''})^0=H_{R'\cap R''}.\]
    \item[(b)] $H_{R'}\sub H_{R''}$ if and only if $R'\sub R''$.
\end{enumerate}
\end{proposition}
\begin{proof}
In fact, (\rmnum{2}) follows from (\rmnum{1}), and to prove (\rmnum{1}), it suffices to prove that $H_{R'}\cap H_{R''}$ is smooth at the unit section: its identity component (\cref{scheme group smooth at unit section iff}) is then a subgroup of type (R) containing $T$ (in view of \cref{scheme subgroup of type (R) contain centralizer}, it contains the centralizer of $T$), hence equals to $H_{R'\cap R''}$. But $\Omega_{R_+,R'}\cap\Omega_{R_+,R''}=\Omega_{R_+,R'\cap R''}$ is an open subset of $H_{R'}\cap H_{R''}$ containing the unit section, and is smooth over $S$,
\end{proof}

\begin{corollary}\label{scheme group reductive subgroup of type (R) inclusion locus open}
Let $S$ be a scheme, $G$ be a reductive $S$-group, $T$ be a maximal torus of $G$, $s$ be a point of $S$. If $H$ and $H'$ are subgroups of type (R) of $G$ containing $T$ such that $H_s\sub H'_s$, then there exists an open neighborhood $U$ of $s$ in $S$ such that $H_U\sub H'_U$.
\end{corollary}
\begin{proof}
We can suppose that $G$ splits relative to $T$, and then the corollary follows from \cref{scheme group splitting intersection of subgroup of type (R) char}.
\end{proof}

We are now led to the question that which subset $R'$ of $R$ is of type (R). Since the reductive center of $G$ is diagonalizable (\cref{scheme group splitting center is intersection of kernel} and \cref{scheme smooth affine zero unipotent rank reductive center is center}), we can assume that the group is adjoint reductive (\cref{scheme group type (RR) quotient by reductive center example} and \cref{scheme group multiplicative free action quotient prop}), and it is then necessary to verify that $\g_{R'}$ is a Lie algebra and that its normalizer is smooth at the unit section (cf. \cref{scheme group type (RA) subgroup of type (R) and stable Lie subalgebra correspond}). The most important case is given by the following theorem:

\begin{theorem}\label{scheme group splitting closed subset is of type (R)}
Let $(G,T,M,R)$ be a splitting $S$-group. Then any closed\footnote{Recall that a subset $R'\sub R$ is called closed if $\alpha,\beta\in R'$ and $\alpha+\beta\in R$ implies $\alpha+\beta\in R'$.} subset $R'$ of $R$ is of type (R).
\end{theorem}

In fact, we shall see (\cite{SGA3-3} \Rmnum{23} 6.6) that if $2$ and $3$ are nonzero over $S$ (for example, if $S$ possesses a residue field with characteristic $\neq 2,3$), the fact that $\g_{R'}$ is a Lie algebra already implies that $R'$ is closed, so $R'$ is of type (R) if and only if it is closed.\par
In order to prove \cref{scheme group splitting closed subset is of type (R)}, let us first establish the following lemma:

\begin{lemma}\label{scheme group splitting conjugation coefficient lemma}
choose for each $\alpha\in R$ a section $X_\alpha\in\Gamma(S,\g^\alpha)^\times$. Let $\alpha,\beta\in R$, with $\alpha+\beta\neq 0$, and let $q$ be the largest integer $i$ such that $\alpha+i\beta\in R$. There exist uniquely determined sections $M_{\alpha,\beta,i}\in\Gamma(S,\mathscr{O}_S)$ such that
\[\Ad(\exp_\alpha(xX_\alpha))(X_\beta)=X_\beta+\sum_{i=1}^{q}M_{\alpha,\beta,i}X_{\beta+i\alpha},\]
for any $x\in\G_a(S')$, $S'\to S$.
\end{lemma}
\begin{proof}
In fact, $x\mapsto\Ad(\exp_\alpha(xX_\alpha))(X_\beta)$ defines a morphism $\G_{a,S}\to \mathbf{W}(\g)\cong\G_{a,S}^m$, so there exists uniquely determined sections $Y_n\in\Gamma(S,\g)$ such that\footnote{This follows from the fact that
\[\Hom_S(\G_{a,S},\mathbf{W}(\g))=\Hom_{\mathscr{O}_S\dash\Alg}(\bm{S}(\g^\vee),\mathscr{O}_S[T])=\Hom_{\mathscr{O}_S\dash\Mod}(\g^\vee,\mathscr{O}_S[T])=\g\otimes_{\mathscr{O}_S}\mathscr{O}_S[T],\]
so a morphism $\G_{a,S}\to \mathbf{W}(\g)$ is determined by a polynomial with coefficients in $\g$.}
\begin{equation}\label{scheme group splitting conjugation coefficient lemma-1}
\Ad(\exp_\alpha(xX_\alpha))(X_\beta)=\sum_{n\geq 0}x^nY_n.
\end{equation}
By acting by inner automorphism defined by a section $t$ of $T$, we find that
\[\Ad(t)(Y_n)=\beta(t)\alpha(t)^nY_n,\]
which implies that $Y_n\in\Gamma(S,\g^{\beta+n\alpha})$. As $\alpha$ and $\beta$ are not propotional, each $\beta+n\alpha$ is nonzero, so $Y_n=0$ for $n>q$ and $Y_n=M_{\alpha,\beta,n}X_{\beta+n\alpha}$ for $0\leq n\leq q$, where $M_{\alpha,\beta,n}\in\G_a(S)$ is uniquely determined. Putting $x=0$ in the formula, we then obtain that $Y_0=X_\beta$, whence the lemma.
\end{proof}

\begin{remark}\label{scheme group splitting bracket coefficient}
By taking derivative at $x=0$ in (\ref{scheme group splitting conjugation coefficient lemma-1}), we also find that
\[[X_\alpha,X_\beta]=\begin{cases*}
N_{\alpha,\beta}X_{\alpha+\beta}&\text{if $\alpha+\beta\in R$},\\
0&\text{otherwise},
\end{cases*}\]
where $N_{\alpha,\beta}=M_{\alpha,\beta,1}$.
\end{remark}

We now prove \cref{scheme group splitting closed subset is of type (R)}. If $R'$ is a closed subset of $R$, then $\g_{R'}$ is a subalgebra of $\g$ by \cref{scheme group splitting bracket coefficient} and (\ref{scheme group elementary system Ad action formula by exp-4}). By \cref{scheme group splitting conjugation coefficient lemma} and (\ref{scheme group elementary system Ad action formula by exp-4}), $U_\alpha$ normalizes $\g_{R'}$ for each $\alpha\in R'$. Now choose a positive root system $R_+$ and consider the open subset $\Omega_{R_+}$ of \cref{scheme group splitting big cell exist}; let $\Omega_{R_+,R'}$ be the closed subscheme of $\Omega_{R_+}$ defined as follows:
\[\Omega_{R_+,R'}=\prod_{\alpha\in R'\cap R_-}U_\alpha\cdot T\cdot \prod_{\alpha\in R'\cap R_+}U_\alpha.\]
The canonical immersion $\Omega_{R_+,R'}\to G$ then factors through $i:\Omega_{R_+,R'}\to N_G(\g_{R'})$. We can assume that $G$ is adjoint (by the remarks before \cref{scheme group splitting closed subset is of type (R)}), then the tangent map of $i$ at the unit section is bijective by \cref{scheme group type (RA) subgroup of type (R) Lie algebra self-normalizing}. In particular, $i$ is \'etale at the unit section, hence a local open immersion at the unit section\footnote{By (\cite{EGA4-4} 17.11.2), $i$ is \'etale at the unit section (and $N_G(\g_{R'})$ is smooth at the unit section). Moreover, let $V$ be the largets open subset of $\Omega_{R_+,R'}$ over which $i$ is \'etale; as $i$ is a monomorphism, $i_V$ is then an open immersion by (\cite{EGA4-4} 17.9.1).}. Since $\Omega_{R_+,R'}$ is smooth, we conclude that $N_G(\g_{R'})$ is smooth at the unit section, which proves our assertion.

\paragraph{Borel subgroups of a splitting reductive group}
\begin{proposition}\label{scheme group splitting Borel subgroup given by positive root}
Let $(G,T,M,R)$ be a splitting $S$-group. For any system of positive roots $R_+$ of $R$, $H_{R_+}$ (exists by \cref{scheme group splitting closed subset is of type (R)}) is a Borel subgrouop of $G$ and for any order over $R_+$, the induced morphism
\[T\times_S\prod_{\alpha\in R_+}U_\alpha\to G\]
is a closed immersion with image $H_{R_+}$ (we write $B_{R_+}=H_{R_+}$). 
\end{proposition}
\begin{proof}
By the definition of Borel subgroups, the first assertion can be verified by replacing $S$ with the spectrum of an algebraically closed field. Let $B$ be a Borel subgroup of $G$ containing $T$, which corresponds to a positive system of roots $R_+$ (\cite{Chevalley1958} \S 10.4, prop.9). Then the Lie algebra of $B$ is $\g_{R_+}$, so we have $B=H_{R_+}$ by \cref{scheme group type (RR) subgroup of type (R) equal iff Lie algebra equal}. Finally, the morphism in the proposition induces an open immersion $i:T\times_S\prod_{\alpha\in R_+}U_\alpha\to H_{R_+}$ by\cref{scheme group splitting subset of type (R) Omega open immerison}, which is surjective by (\cite{Chevalely1958} \S 15.1, cor.1 a la prop.1).
\end{proof}

\begin{corollary}\label{scheme group splitting conjugation coefficient of two roots}
Choose an order over $R_+$ and for each $\alpha\in R_+$ a section $X_\alpha\in\Gamma(S,\g^\alpha)^\times$. Let $\alpha,\beta\in R_+$; for each couple $(i,j)$ of positive integers such that $i\alpha+j\beta\in R$, there exists a unique section $C_{i,j,\alpha,\beta}\in\Gamma(S,\mathscr{O}_S)$ such that, for any $x,y\in\G_a(S')$, $S'\to S$, we have\footnote{Setting $y=0$ and compare with \cref{scheme group splitting conjugation coefficient lemma}, we find that $C_{i,1,\alpha,\beta}=M_{\alpha,\beta,i}$.}
\[\exp_\alpha(xX_\alpha)\exp_\beta(yX_\beta)\exp_\alpha(-xX_\alpha)=\exp_\beta(yX_\beta)\prod_{\substack{i,j\in\N^*\\ i\alpha+j\beta\in R}}\exp_{i\alpha+j\beta}(C_{i,j,\alpha,\beta}x^iy^jX_{i\alpha+j\beta}).\]
\end{corollary}
\begin{proof}
If $\alpha=\beta$, the assertion is trivial, so suppose that $\alpha\neq\beta$. Then in view of \cref{scheme group splitting Borel subgroup given by positive root}, there exists unique morphisms
\[F_0=\G_{a,S}^2\to T,\quad F_\gamma:\G_{a,S}^2\to\G_{a,S}\ (\gamma\in R_+)\]
such that we have
\[\exp(xX_\alpha)\exp(yX_\beta)\exp(-xX_\alpha)=F_0(x,y)\prod_{\gamma\in R_+}\exp(F_\gamma(x,y)X_\gamma).\]
Let $t\in T(S')$, $S'\to S$. By acting with $\inn(t)$ on this formula, we find the relations
\begin{align}
F_0(\alpha(t)x,\beta(t)y)&=F_0(x,y),\label{scheme group splitting conjugation coefficient of two roots-1}\\
F_\gamma(\alpha(t)x,\beta(y))&=\gamma(t)F_\gamma(x,y).\label{scheme group splitting conjugation coefficient of two roots-2}
\end{align}
As $\alpha$ and $\beta$ are linearly independent characters (over $\Q$) of $T$, we conclude as usual from the first relation that $F_0$ is constant, hence $F_0(x,y)=e$. Let's then write
\[F_\gamma(x,y)=\sum a_{ij}x^iy^j,\quad a_{ij}\in\Gamma(S,\mathscr{O}_S).\]
Carrying over to relation (\ref{scheme group splitting conjugation coefficient of two roots-2}) and identifying the polynomials of the two members, we find that
\[a_{ij}(\alpha(t)^i\beta(t)^j-\gamma(t))=0.\]
If $\gamma\neq i\alpha+j\beta$, we see in \cref{scheme torus distinct character fpqc local difference by 1} that there exists a faithfully flat and quasi-compact morphism $S'\to S$ and $t\in T(S')$ such that $\alpha(t)^i\beta(t)^j-\gamma(t)=1$. Then we have $a_{ij}=0$ over $S'$, whence over $S$. If $\gamma=i\alpha+j\beta$, then we put $a_{ij}=C_{i,j,\alpha,\beta}$. Putting $x=0$ (resp. $y=0$), also we find that $C_{0,1,\alpha,\beta}=1$ (Resp. $C_{1,0,\alpha,\beta}=0$).
\end{proof}

\begin{corollary}\label{scheme group reductive conjugation formula by exp of tensor}
Let $S$ be a scheme, $G$ be a reductive $S$-group, $T$ be a maximal torus of $G$, $\alpha\neq\beta$ be two roots of $G$ relative to $T$ such that $\alpha+\beta$ is nontrivial over each fiber. Choose an order over the set $i\alpha+j\beta$ ($i,j\in\N^*$). Then for any $i,j\in\N^*$ such that $i\alpha+j\beta\in R$, there exists a unqiue morphism of $\mathscr{O}_S$-modules
\[f_{\alpha,\beta,i,j}:(\g^\alpha)^i\otimes(\g^\beta)^j\to\g^{i\alpha+j\beta}\]
such that for any $S'\to $ and $X\in \mathbf{W}(\g^\alpha)(S')$, $Y\in \mathbf{W}(\g^\beta)(S')$, we have\footnote{If $\g^{i\alpha+j\beta}=0$ over a connected component of $S$, the corresponding exponential map takes value $1$ over this component.}
\[\exp_\alpha(X)\exp_\beta(Y)\exp_\alpha(-X)=\exp_\beta(Y)\prod_{i,j}\exp_{i\alpha+j\beta}(f_{\alpha,\beta,i,j}(X^i\otimes Y^j))\]
\end{corollary}
\begin{proof}
The assertion is local for the fpqc topology. By \cref{scheme group reductive etale local splittable}, we can hence suppose that $G$ splits relative to $T$, $\alpha$ and $\beta$ being constant in the splitting. As $\alpha+\beta\neq 0$, there exists a positive root system $R_+$ containing $\alpha$ and $\beta$ (\cite{SGA3-3} \Rmnum{21} 3.5.4), and we are then reduced to \cref{scheme group splitting conjugation coefficient of two roots}.
\end{proof}

\begin{corollary}\label{scheme group reductive etale local Borel subgroup prop}
Let $S$ be a scheme and $G$ be a reductive $S$-group.
\begin{enumerate}
    \item[(\rmnum{1})] $G$ possesses locally for the \'etale topology Borel subgroups. If $T$ is a maximal torus of $G$, then $G$ possesses locally for the \'etale topology Borel subgroups containing $T$.
    \item[(\rmnum{2})] If $T$ is a maximal torus of $G$, the functor of Borel subgroups of $G$ containing $T$ is representable by a principal fiber bundle under $W_G(T)$.
    \item[(\rmnum{3})] If $(G,T,M,R)$ is splitting, any Borel subgroup $B$ of $G$ containing $T$ is locally over $S$ of the form $B_{R_+}$, where $R_+$ is a positive root system of $R$.
    \item[(\rmnum{4})] If $T\sub B$ is a Killing couple, there exists a \'etale surjective family $\{S_i\to S\}$, for each $i$ a splitting $(G_{S_i},T_{S_i},M_i,R_i)$, and a positive root system $R_{i+}$ of $R_i$, such that $B_{S_i}=B_{R_{i+}}$.
\end{enumerate}
\end{corollary}
\begin{proof}
In fact, (\rmnum{1}) follows from \cref{scheme group splitting Borel subgroup given by positive root} and \cref{scheme group reductive etale local splittable}, (\rmnum{2}) from (\rmnum{1}) and \cref{scheme group type (RR) functor of Borel subgroup containing torus is principal homogeneous}, (\rmnum{3}) from (\rmnum{1}) and \cref{scheme group splitting Borel subgroup given by positive root}, and (\rmnum{4}) from (\rmnum{3}) and \cref{scheme group reductive etale local splittable}.
\end{proof}

Let $G=\GL_{n,k}$ where $k$ is an algebraically closed field. Then any Borel subgroup of $G$ is conjugate to the subgroup $B$ of upper triangular matrices of $G$, which can be written as a semi-direct product of $T$ (the subgroup of diagonal matrices) and $\mathbb{U}_n$ (the subgroup of strictly upper triangular matrices). We note that the subgroup $\mathbb{U}_n$ is unipotent and has a standard central decomposition series (\cite{*}). This result can be generalized to splitting groups over an arbitrary scheme, as in the following proposition:

\begin{proposition}\label{scheme group splitting Borel subgroup filtration induced by chain of roots}
Let $(G,T,M,R)$ be a splitting $S$-group and $\alpha_1\prec\cdots\prec\alpha_n$ be the elements of a positive root system $R_+$ of $R$ (endowed with an order). Consider the isomorphism
\[\varphi:T\times_SU_{\alpha_1}\times_S\cdots\times_SU_{\alpha_n}\to B_{R_+}\]
induced by product in $G$, and for each $i=1,\dots,n$, put
\[U_{\geq i}=\varphi(U_{\alpha_i}\times_S\cdots\times_SU_{\alpha_n}).\]
\begin{enumerate}
    \item[(\rmnum{1})] Each $U_{\geq i}$ is a closed normal subgroup of $B_{R_+}$.
    \item[(\rmnum{2})] For $1\leq i\leq n-1$, $U_{\geq i}$ is identified with the semi-direct product
    \[U_{\geq i}=U_{\alpha_i}\cdot U_{\geq i+1}.\]
    \item[(\rmnum{3})] $B_{R_+}$ is identified with the semi-direct product
    \[B_{R_+}=T\cdot U_{\geq 1}.\]
    \item[(\rmnum{4})] For $1\leq i\leq n-1$, the inner automorphisms of $U_{\geq 1}$ acts trivially on $U_{\geq i}/U_{\geq i+1}$ (which is identified with $U_{\alpha_i}$ by (\rmnum{2})). 
\end{enumerate}
\end{proposition}

\begin{proposition}\label{scheme group splitting chain of roots product of exp polynomial relation}
With the preceding notations, choose for each $1\leq i\leq n$ a section $X_i\in\Gamma(S,\g^{\alpha_i})^\times$ and consider the isomorphism
\[a:\G_{a,S}^n\to U_{\geq 1},\quad (x_1,\dots,x_n)\mapsto \exp_{\alpha_1}(x_1X_1)\cdots\exp_{\alpha_n}(x_nX_n).\]
Then there exists a unique family $(Q_i)_{1\leq i\leq n}$ of polynomials 
\[Q_i=Q_i(x_1,\dots,x_n,y_1,\dots,x_n)\]
with coefficients in $\Gamma(S,\mathscr{O}_S)$ such that we have 
\[a(x_1,\dots,x_n)a(y_1,\dots,y_n)=a(Q(x_1,\dots,y_n),\dots,Q_n(x_1,\dots,y_n)).\]
Moreover, the coefficients of $Q_i$ belong to the subring of $\Gamma(S,\mathscr{O}_S)$ generated by $C_{i,j,\alpha,\beta}$ of \cref{scheme group splitting conjugation coefficient of two roots}, and each $Q_i$ is of the form
\[Q_i(x_1,\dots,y_n)=x_i+y_i+Q_i'(x_1,\dots,x_{i-1},y_1,\dots,y_{i-1}).\]
\end{proposition}

We note that the preceding induction also proves the following result:

\begin{corollary}\label{scheme group splitting group homomorphism defined by chain of roots iff}
With the notations of \cref{scheme group splitting Borel subgroup filtration induced by chain of roots}, for each $1\leq i\leq n$, let $f_i:U_{\alpha_i}\to H$ be a group homomorphism, where $H$ is an $S$-functor in groups. For that the morphism
\[f:U_{\geq 1}\to H,\quad f(\exp(x_1X_1)\cdots\exp(x_nX_n))=f_1(\exp(x_1X_1))\cdots f_n(\exp(x_nX_n))\]
be a group homomorphism, it is necessary and sufficient that for any $i<j$, we have
\[f_j(\exp(x_jX_j))f_i(\exp(x_iX_i))f_j(\exp(-x_jX_j))=f\big(\exp(x_jX_j)\exp(x_iX_i)\exp(-x_jX_j)\big).\]
\end{corollary}

\paragraph{Subgroups of type (R) with solvable fibers}
\begin{proposition}\label{scheme group splitting H_R' solvable iff}
Let $(G,T,M,R)$ be a splitting $S$-group, $R'$ be a subset of type (R) of $R$, $H_{R'}$ be the corresponding subgroup of $G$. The following conditions are equivalent.
\begin{enumerate}
    \item[(\rmnum{1})] $H_{R'}$ has solvable geometric fibers.
    \item[(\rmnum{2})] There exists a poritive root system $R_+$ such that $R'\sub R_+$, whence $H_{R'}\sub B_{R_+}$.
    \item[(\rmnum{3})] $R'\cap -R'=\emp$.
    \item[(\rmnum{4})] For any order over $R'$, the morphism
    \[T\times_S\prod_{\alpha\in R'}U_\alpha\to H_{R'}\]
    induced by the product in $G$, is an isomorphism.
    \item[(\rmnum{5})] $H_{R'}\cap N_G(T)=T$.
    \item[(\rmnum{6})] For any subset $R''$ of type (R) of $R$, we have (cf. \cref{scheme group splitting intersection of subgroup of type (R) char})
    \[H_{R'}\cap N_G(H_{R''})=H_{R'\cap R''}.\]
\end{enumerate}
\end{proposition}
\begin{proof}
We shall prove the equivalence by establishing the following implications:
\[\begin{tikzcd}[row sep=6mm,column sep=6mm]
\textup{(\rmnum{3})}\ar[dd,Rightarrow]&\textup{(\rmnum{2})}\ar[l,Rightarrow]\ar[r,Rightarrow]\ar[d,Rightarrow]&\textup{(\rmnum{6})}\ar[dd,Rightarrow]\\
&\textup{(\rmnum{4})}\ar[rd,Rightarrow]&\\
\textup{(\rmnum{1})}\ar[ruu,Rightarrow]&&\textup{(\rmnum{5})}\ar[ll,Rightarrow]
\end{tikzcd}\]
We have evidently (\rmnum{2})$\Rightarrow$(\rmnum{3}) and (\rmnum{6})$\Rightarrow$(\rmnum{5}) (putting $R''=\emp$). By \cref{scheme group reductive subgroup of type (R) inclusion locus open}, it suffices to verify (\rmnum{1})$\Rightarrow$(\rmnum{2}) over geometric fibers; now if $S$ is the spectrum of an algebraically closed field, $H_{R'}$ is contained in a Borel subgroup containing $T$ (since it is solvable), hence of the form $H_{R_+}$ (\cref{scheme group reductive etale local Borel subgroup prop}~(\rmnum{3})).\par
Similarly (\rmnum{3})$\Rightarrow$(\rmnum{1}) can be verified over geometric fibers. Suppose that (\rmnum{3}) is verified; if $H_{R'}$ is not solvable, there exists a subtorus $Q$ of $T$, of codimension $1$ in $T$ such that $Z_{H_{R'}}(Q)$ is not solvable (\cite{Chevalley1958} \S 10.4, prop.8). But $Z_{H_{R'}}(Q)$ has Lie algebra $\g_{R''}$, where $R''$ is the set of roots of $R'$ vanishing over $Q$, hence $R''=\emp$ or $\{\alpha\}$ (in view of (\rmnum{3})). Therefore $Z_{H_{R'}}(Q)$, which is a subgroup of type (R) of $G$ (as it contains $T=Z_G(T)$), is $T$ or $T\cdot U_\alpha$, hence solvable, contradicting to the hypothesis.\par
The implication (\rmnum{2})$\Rightarrow$(\rmnum{4}) can also be verified over geometric fibers (because the $S$-schemes are flat and of finite presentation); by (\cite{Chevalley1958} \S 13.2, th.1 (d)), the morphism in (\rmnum{4}) is bijective, and it induces an isomorphism over tangent spaces at the unit section. We then conclude as in \ref{scheme group splitting big cell paragraoh} that it is an open immersion, whence an isomorphism.\par
We have (\rmnum{4})$\Rightarrow$(\rmnum{5}) by (\cite{SGA3-3} \Rmnum{22} 4.2.7). To prove that (\rmnum{5})$\Rightarrow$(\rmnum{1}), we are reduced to the case where $S$ is the spectrum of an algebraically closed field, and we then conclude by (\cite{Chevalley1958} \S 10.3, cor. a la prop.6 et \S 9.3, cor.3 au th.1).\par
It remains to prove that (\rmnum{2})$\Rightarrow$(\rmnum{6}), for which we can assume that $G$ is adjoint. We then have, by \cref{scheme group type (RR) subgroup of type (R) equal iff Lie algebra equal},
\[N_G(H_{R''})=N_G(\g_{R''})\sub\STrans_G(\t,\g_{R''}).\]
By \cref{scheme group splitting intersection of subgroup of type (R) char} we have $H_{R'\cap R''}\sub H_{R'}\cap N_G(H_{R''})$, so it suffices to show that
\begin{equation}\label{scheme group splitting H_R' solvable iff-1}
H_{R'}(S)\cap\STrans_{G(S)}(\t,\g_{R''})\sub H_{R'\cap R''}(S).
\end{equation}
For this, we first prove the following lemma:
\begin{lemma}\label{scheme group splitting chain of roots product Ad component}
With the notations of \cref{scheme group splitting chain of roots product of exp polynomial relation}, let
\[u=\exp(x_1X_1)\cdots\exp(x_nX_n)\]
where $x_i\in\G_a(S)$. Let $1\leq m\leq n$ be an integer such that $x_i=0$ for $i<m$.
\begin{enumerate}
    \item[(a)] If $H\in\Gamma(S,\t)$, the component of $\Ad(u)H$ over $\g^{\alpha_m}$ is $-\alpha_{m*}(H)x_mX_m$.
    \item[(b)] If $Y\in\Gamma(S,\g^{-\alpha_m})$, the component of $\Ad(u)Y$ over $\t$ is $x_m\langle X_m,Y\rangle H_{\alpha_m}$. 
\end{enumerate}
\end{lemma}
Now we return to the proof of (\ref{scheme group splitting H_R' solvable iff-1}). Suppose that there exists $h\in H_{R'}(S)$, $h\notin H_{R'\cap R''}(S)$, such that $\Ad(h)\t\sub\g_{R''}$. Then we can write
\[h=t\exp(x_1X_1)\cdots\exp(x_nX_n).\]
As $h\notin H_{R'\cap R''}(S)$, there exists a smallets integer $m$ such that
\[t\exp(x_1X_1)\cdots\exp(x_{m-1}X_{m-1})\in H_{R'\cap R''}(S),\quad \alpha_m\notin R'',x_m\neq 0.\]
Then $h'=\exp(x_mX_m)\cdots\exp(x_nX_n)$ also verifies the conditions imposed on $h$. But by \cref{scheme group splitting chain of roots product Ad component}, for any $H\in\Gamma(S,\t)$, the component of $\Ad(h')H$ over $\h^{\alpha_m}$ is $-\alpha_*(X)x_mX_m$. In view of the hypothesis over $h$ and $m$, we then have $\alpha_m(H)=0$ for any $H\in\Gamma(S,\t)$, which is impossible because $G$ is supposed to be adjoint and $\alpha_{m*}$ is nonzero over each fiber.
\end{proof}

\begin{remark}\label{scheme group splitting chain of roots product Ad trivial iff}
Retain the notations of \cref{scheme group splitting chain of roots product Ad component}. If $\Ad(u)$ is the identity over $\t$ and $\g^{-\alpha_m}$, then $x_m=0$. In fact, we have $x_m\alpha_{m*}=0$ and $x_mH_{\alpha_m}=0$. Since $\alpha_{m*}$ and $H_{\alpha_m}$ can not both be zero over each fiber, this implies $x_m=0$. It then follows that $u=e$ if $\Ad(u)$ acts trivially on $\g$.
\end{remark}

\begin{remark}\label{scheme reductive solvable subgroup of type (R) is closed}
If $H$ is a subgroup of type (R) of a reductive group $G$, with solvable geometric fibers, then by \cref{scheme group splitting H_R' solvable iff}~(\rmnum{5}), the Weyl group of $H$ is trivial, so $H$ is closed in $G$ and $N_G(H)/H$ is representable by a finite \'etale separated $S$-scheme (\cref{scheme group type (RR) Weyl group finite subgroup type (R) closed iff}).
\end{remark}

\begin{corollary}\label{scheme group splitting closed positive subset is in positive root}
Let $(G,T,M,R)$ be a splitting reductive $S$-group. If $R'\sub R$ is a closed subset and $R'\cap-R'=\emp$, then $R'$ is contained in a positive root system.
\end{corollary}
\begin{proof}
In fact, $R'$ is of type (R) by \cref{scheme group splitting closed subset is of type (R)}, hence the result follows from \cref{scheme group splitting H_R' solvable iff}.
\end{proof}

\begin{corollary}\label{scheme group splitting solvable subgroup of type (R) unipotent part}
Under the conditions of \cref{scheme group splitting H_R' solvable iff}, the product in $G$ induces an isomorphism
\[\prod_{\alpha\in R'}\stackrel{\sim}{\to} U_{R'},\]
where $U_{R'}$ is a closed subgroup of $G$, smooth over $S$, with connected and unipotent geometric fibers, and is independent of the choice of the order over $R'$. Moreover, $H_{R'}$ is the semi-direct product $T\cdot U_{R'}$ ($U_{R'}$ being normal)\footnote{In particular, $U_{R_+}$ is the subgroup $U_{\geq 1}$ of \cref{scheme group splitting Borel subgroup filtration induced by chain of roots}.}.
\end{corollary}
\begin{proof}
If $R'\sub R_+$, then $H_{R'}\cap U_{\geq 1}$ (with notations in \cref{scheme group splitting Borel subgroup filtration induced by chain of roots}) is a closed subgroup of $G$ of finit epresentation and normal in $H_{R'}$. By \cref{scheme group splitting H_R' solvable iff}~(\rmnum{4}), we have $H_{R'}=T\cdot U_{R'}$, and $U_{R'}$ has unipotent fiber in view of \cref{scheme group splitting Borel subgroup filtration induced by chain of roots}~(\rmnum{2}), since $U_{\alpha_i}\cong\G_{a,S}$ for each $i$.
\end{proof}

\begin{corollary}\label{scheme group splitting solvable subgroup of type (R) normalizer prop}
Let $(G,T,M,R)$ be a splitting $S$-group, $R'$ and $R''$ be subsets of type (R) of $R$, with $R'\cap-R'=\emp$.
\begin{enumerate}
    \item[(a)] We have $U_{R'}\cap N_G(H_{R''})=U_{R'\cap R''}$.
    \item[(b)] Suppose that $R'$ is closed. If for any $\alpha\in R'$, $\beta\in R''$ such that $\alpha+\beta\in R$, we have $\alpha+\beta\in R'$, then $H_{R''}$ normalizes $U_{R'}$. 
\end{enumerate}
\end{corollary}
\begin{proof}
The first assertion follows easily from \cref{scheme group splitting solvable subgroup of type (R) unipotent part} and \cref{scheme group splitting H_R' solvable iff}~(\rmnum{6}). To prove (b), it suffices, by \cref{scheme group splitting subset of type (R) Omega open immerison} and \cref{scheme group splitting big cell schematically dense}, to show that $T$ and each $U_\beta$, $\beta\in R''$, normalize $U_{R'}$. For $T$, this is trivial, and for $U_\beta$, this follows from \cref{scheme group splitting conjugation coefficient of two roots}. In fact, it suffices to seeing that, under the hypothesis of (b), if $\alpha\in R'$ and $\beta\in R''$, then any root of the form $i\alpha+j\beta$ (with $i,j\in\N^*$) belongs to $R'$. For this, we can use (\cite{SGA3-3} \Rmnum{21} 3.1.2).
\end{proof}

\begin{corollary}\label{scheme group splitting unipotent part of Borel avoiding simple root prop}
Let $(G,T,M,R)$ be a splitting $S$-group, $R_+$ be a positive root system, $\alpha$ be a simple root of $R_+$ (i.e. an element of $R_+$ such that $R_+-\{\alpha\}$ is closed). Let $U_{\hat{\alpha}}=U_{R_+-\{\alpha\}}$.
\begin{enumerate}
    \item[(\rmnum{1})] $U_{\hat{\alpha}}$ is a normal subgroup of $B_{R_+}$.
    \item[(\rmnum{2})] $U_{R_+}$ is the semi-direct product $U_\alpha\cdot U_{\hat{\alpha}}$.
    \item[(\rmnum{3})] $U_{-\alpha}$ and $G_\alpha$ normalizes $U_{\hat{\alpha}}$.
\end{enumerate}
If we define similarly $U_{-\hat{\alpha}}=U_{R_--\{-\alpha\}}$, we have
\[\Omega_{R_+}=U_{-\hat{\alpha}}\cdot U_{-\alpha}\cdot T\cdot U_\alpha\cdot U_{\hat{\alpha}}.\] 
\end{corollary}
\begin{proof}
In fact, (\rmnum{2}) follows from \cref{scheme group splitting solvable subgroup of type (R) unipotent part}, and (\rmnum{1}) from \cref{scheme group splitting solvable subgroup of type (R) normalizer prop}~(b). Similarly, the fact that $U_{-\alpha}$ normalizes $U_{\hat{\alpha}}$ follows from \cref{scheme group splitting conjugation coefficient of two roots} (in fact, if $\beta\in R_+$, $\beta\neq\alpha$, no combination $i(-\alpha)+j\beta$, with $i,j>0$, can be negative, bacause $\beta$ contains at least one simple root $\neq\alpha$). Then $G_\alpha$ normalizes $U_{\hat{\alpha}}$ because $U_{-\alpha}\cdot T\cdot U_\alpha$ is schematically dense in $G_\alpha$ (and $U_{\hat{\alpha}}$ is normal in $B_{R_+}$). Finally, the last assertion follows from (\rmnum{2}) and its analogue for $U_{R_-}$. 
\end{proof}

In the general situation (i.e. where $G$ is not assumed to be splitting), we can still define the unipotent part of a subgroup $H$ of type (R) of a reductive $S$-group $G$.

\begin{proposition}\label{scheme group reductive solvable subgroup type (R) unipotent part}
Let $S$ be a scheme, $G$ be a reductive $S$-group, and $H$ be a subgroup of type (R) with solvable geometric fibers.
\begin{enumerate}
    \item[(\rmnum{1})] $D_S(H)=\sHom_{S\dash\Grp}(H,\G_{m,S})$ is representable by a twisted constant $S$-group, whose type at $s\in S$ is $\Z^{\rho_r(G_s)}$. The biduality morphism (cf. (\ref{category D(D(G)) canonical morphism}))
    \[f:H\to D_S(D_S(H))\]
    is smooth and surjective. 
    \item[(\rmnum{2})] The kernel $H^u$ of $f$ is the largest closed smooth normal subgroup of $H$ with connected and unipotent geometric fibers, which is called the \textbf{unipotent part} of $H$, denoted also by $H^u=\rad^u(H)$. This is also the sheaf of commutators of $H$: any group homomorphism from $H$ into an $S$-presheaf in commutative groups, separated for the fppf topology, is trivial over $H^u$ and factors through $H/H^u=D_S(D_S(H))$. 
    \item[(\rmnum{3})] If $T$ is a maximal torus of $H$, the morphism $T\to H$ induces isomorphisms $D_S(H)\stackrel{\sim}{\to}D_S(T)$ and $T\stackrel{\sim}{\to}D_S(D_S(H))$. Moreover, $H$ is identified with the semi-direct product $T\cdot H^u$.
    \item[(\rmnum{4})] In the situation of \cref{scheme group splitting H_R' solvable iff}, if $H=H_{R'}$, then $H^u=U_{R'}$.
\end{enumerate}
\end{proposition}
\begin{proof}
The assertions of the proposition are local for the \'etale topology (\cref{scheme twisted constant group fppf effective descent}), so we can reduced to the case of \cref{scheme group splitting H_R' solvable iff}. We then see that $H_{R'}$ is the semi-direct product of $U_{R'}$ by $T$. To see that $U_{R'}$ is the commutator sheaf of $H_{R'}$, since $H_{R'}/U_{R'}=T$, it suffices to prove that any group homomorphism $\phi:H_{R'}\to V$ as in (\rmnum{2}) is trivial over $U_{R'}$. For this, it suffices to consider the morphism $\phi$ over each $U_\alpha$, $\alpha\in R'$. Now if $t\in T(S')$, $X\in\mathbf{W}(\g^\alpha)(S')$, we have
\[1=\phi(t\exp_\alpha(X)t^{-1}\exp_\alpha(-X))=\phi\big(\exp_\alpha((\alpha(t)-1)X)\big).\]
As $\alpha:T\to\G_{m,S}$ is faithfully flat, we conclude easily that $\phi$ is trivial over $U_\alpha$, so
\[\sHom_{S\dash\Grp}(H,V)=\sHom_{S\dash\Grp}(H/U_{R'},V)\]
for any $V$ as in (\rmnum{2}). Applying this to $V=\G_{m,S}$, we then deduce (\rmnum{1}) and (\rmnum{3}), whence (\rmnum{4}) and the second of (\rmnum{2}). It remains to prove the first assertion of (\rmnum{2}); the only nontrivial point is that any closed normal subgroup $U$ of $H$, smooth over $S$ with connected and unipotent fibers, is a subgroup of $H^u$. To see this, we first note that:

\begin{lemma}\label{scheme group reductive unipotent subgroup normalized by torus intersection prop}
Let $G$ be a reductive $S$-group, $T$ be a maximal torus, $U$ be a subgroup of $G$, smooth over $S$ with unipotent geometric fibers, and normalized by $T$. Then $U\cap T=e$.
\end{lemma}
In fact, as $T=Z_G(T)$, we have $U\cap T=U^T$ (invariants under $\inn(T)$). Applying \cref{scheme torus act on sp smooth gorup invariant representable}, we then deduce that $U\cap T$ is smooth over $S$. But it is also radiciel over $S$: for any $s\in S$, $U(\bar{s})\cap T(\bar{s})$ is formed by elements that are both unipotent and semi-simple; this proves the lemma.\par

Return to the proof of \cref{scheme group reductive solvable subgroup type (R) unipotent part}~(\rmnum{2}). If $U$ is a normal subgroup as above, then the semi-direct product $T\cdot U$ is a subgroup of type (R) of $G$, with solvable geometric fibers. We can then suppose that it is of the form $H_{R''}$, with $R''\sub R'$, and it suffices to prove that $U=U_{R''}$. By replacing $R''$ with $R'$, we are then reduced to the case $H=T\cdot U$; but the quotient $H/U$ is commutative, so $U$ is a subsheaf of the sheaf of commutators of $H$, which is $H^u$.
\end{proof}

Note that we have in fact proved the following result:

\begin{proposition}\label{scheme group reductive solvable subgroup type (R) and unipotent subgroup correspond}
Let $S$ be a scheme, $G$ be a reductive $S$-group, $T$ be a maximal torus of $G$. The maps
\[H\mapsto H^u,\quad U\mapsto T\cdot U\]
define a bijective correspondence between the set of subgroups $H$ of type (R), containing $T$ and have solvable geometric fibers, and the set of subgroups $U$ of $G$, smooth over $S$, normalized by $T$, with connected and unipotent connected geometric fibers\footnote{We have removed the assumption that $U$ is closed, which is automatically verified. Indeed, for such a subgroup $U$, we have $U\cap T=e$ by \cref{scheme group reductive unipotent subgroup normalized by torus intersection prop}, so the semi-direct product $H=T\cdot U$ is a subgroup of type (R) of $G$ with solvable geometric fibres. Therefore, by \cref{scheme reductive solvable subgroup of type (R) is closed}, $H$ is closed in $G$, and since $U$ is closed in $H$, it is closed in $G$.}. In particular, if $(G,T,M,R)$ is splitting, the groups $H_{R'}$ and $U_{R'}$ are corresponding to each other.
\end{proposition}

\begin{corollary}\label{scheme group splitting unipotent subgroup of type (R) char}
Let $S$ be a scheme, $(G,T,M,R)$ be a splitting $S$-group (resp. and $R_+$ be a positive root system of $R$ defining a Borel subgroup $B$). Any smooth subgroup of $G$, with connected unipotent geometric fibers (resp. any smooth subgroup of $B^u$), normalized by $T$, is locally over $S$ of the form $U_{R'}$, where $R'$ is a subset of $R$ contained in a positive root system (resp. a subset of $R_+$) of type (R).
\end{corollary}
\begin{proof}
It suffices to note that the geometric fibers of the given groups are unipotent and connected by (\cite{Chevalley1958} \S13.2, th.1 (d)).
\end{proof}

\begin{corollary}\label{scheme group reductive solvable subgroup of type (R) functor of maximal tori representable}
Let $S$ be a scheme, $G$ be a reductive $S$-group, $H$ be a subgroup of type (R) with solvable geometric fibers, and $\mathscr{T}_H$ be the functor of maximal tori of $H$:
\[\mathscr{T}_H(S')=\{\text{set of maximal tori of $H_{S'}$}\}.\]
Then $\mathscr{T}_H$ is representable by an affine and smooth $S$-scheme, which is a principal homogeneous under $H^u$ for the action $(h,T)\mapsto\inn(h)T$.
\end{corollary}
\begin{proof}
If $T$ and $T'$ are maximal tori of $H_{S'}$, there exists a unique section $h\in H^u(S')$ such that $\inn(h)T=T'$. The uniqueness of $h$ follows easily from the equality (cf. \cref{scheme group splitting H_R' solvable iff})
\[N_G(T)\cap H^u=e.\]
It then suffices to prove the existence of $h$ locally for the \'etale topology. By \cref{scheme group type (RR) (RA) subgroup of type (R) prop}, $H$ is a group of type (RR), so in view of \cref{scheme group type (RR) maximal tori Cartan subgroup prop}, we can suppose that $T$ and $T'$ are conjugate by a section of $H$, whence the conclusion since $H=T\cdot H^u$ by \cref{scheme group reductive solvable subgroup type (R) unipotent part}~(\rmnum{3}). This also ensures that $\mathscr{T}_H$ is a principal homogeneous sheaf under $H^u$, which is affine and smooth over $S$ by \cref{scheme group reductive solvable subgroup type (R) unipotent part} and \cref{scheme group type (RR) maximal tori Cartan subgroup prop}. The assertion then follows from (\cite{SGA1} \Rmnum{8} 2.1 et \cite{EGA4-4} 17.7.1).
\end{proof}

\paragraph{Bruhat's theorem}
Let $k$ be an algebraically closed field, $G$ be a reductive $k$-group, $B$ be a Borel subgroup of $G$, $U=B^u$ the unipotent part of $B$, $T$ be a maximal torus of $B$, $N=N_G(T)$. Then by Bruhat's theorem (\cite{Chevalley1958} \S13.4, cor.1 au th.3), we have
\[G(k)=B(k)N(k)B(k);\]
more precisely, with the notations of \autoref{scheme group reductive Weyl group subsection}, the sets
\[B(k)N_w(k)B(k)=U(k)N_w(k)U(k)\]
form, for $w$ runs through $(N/T)(k)$, a partition of $G(k)$. If $B'$ is another Borel subgroup containing $T$, the sets $B'(k)N_w(k)B(k)$ also form a partition of $G(k)$. In fact, if $y\in N(k)$ is such that $\inn(y)B=B'$, we have
\[yB(k)N_w(k)B(k)=B'(k)N_{yw}(k)B(k).\]

Now let $(G,T,M,R)$ be a splitting $S$-group, $R_+$ be a positive root system of $R$. For $w\in W$, we set
\[R_-^w=R_-\cap w(R_-),\quad U^-_w=U_{R_-^w}=\prod_{\alpha\in R_-^w}U_\alpha.\]
If $n_w\in N_G(T)(S)$ is a representative of $w$, we can also write
\[U^-_w=U^-\cap\inn(n_w)U^-.\]

\begin{proposition}\label{scheme group splitting BNB image representable}
Let $(G,T,M,R)$ be a splitting $S$-group, $R_+$ be a positive root system of $R$, $B$ (resp. $B^-$) be the Borel subgroup of $G$ defined by $R_+$ (resp. $R_-$). Let $w\in W$, and $N_w$, $B_w^u$ be the corresponding subschemes of $G$.
\begin{enumerate}
    \item[(a)] The sheaf $B^-\cdot N_w\cdot B$ is representable by a subscheme of $G$ (and in fact a closed subscheme of the open subscheme $n_w\Omega_{R_+}$).
    \item[(b)] The morphism
    \[U^-_w\times_SN_w\times U\to G\]
    induced by the product in $G$, is an immersion whose image is the subscheme in (a). 
\end{enumerate}
\end{proposition}
\begin{proof}
By definition, $\inn(n_w)^{-1}$ induces a closed immersion from $U^-_w$ into $U^-$, so the morphism $(u,b)\mapsto n_w^{-1}un_wb$ induces a closed immersion
\[U^-_w\times_SB\to\Omega_{R_+}.\]
This implies immediately that the morphism of (b) induces a closed immersion into the open subset $n_w\Omega_{R_+}$. To prove (a), it suffices to show that
\[B^-(S)N_w(S)B(S)=U^-_w(S)N_w(S)U(S).\]
Now if $\alpha\in R$, we have $\inn(n_w)U_\alpha(S)=U_{w(\alpha)}(S)$, hence if $w^{-1}(\alpha)\in R_+$,
\begin{align*}
U_\alpha(S)N_w(S)B(S)&=U_\alpha(S)n_wT(S)B^u(S)=n_wU_{w^{-1}(\alpha)}(S)T(S)B^u(S)=n_wB(S)=N_w(S)B^u(S).
\end{align*}
This easily implies, in view of the definition of $R_-^w$, the desired assertion.
\end{proof}

\begin{theorem}[\textbf{Bruhat's Theorem}]\label{scheme group splitting Bruhat decomposition}
Let $S$ be a scheme, $(G,T,M,R)$ be a splitting $S$-group, $B$ be the Borel subgroup defined by a positive root system $R_+$.
\begin{enumerate}
    \item[(a)] The schemes $U^-_w\cdot N_w\cdot U=B^-\cdot N_w\cdot B$ form, for $w$ runs through $W$, a partition of $G$.
    \item[(b)] For each $w\in W$, let $n_w$ be a representative of $w$ in $N_G(T)(S)$; then the open subsets $n_w\Omega=n_wU^-\cdot T\cdot U$ form, for $w$ runs through $W$, a covering of $G$.
\end{enumerate}
\end{theorem}
\begin{proof}
The two assertions can be verified over geometric fibers, and therefore follow from \cref{scheme group splitting BNB image representable} and (\cite{Chevalley1958} \S13.4, cor.1 au th.3).
\end{proof}

\begin{corollary}\label{scheme group splitting over local base section generated by U_alpha}
Let $\Delta$ be a system of simple roots of the splitting $S$-group $G$ over a local scheme $S$.
\begin{enumerate}
    \item[(a)] Then $G(S)$ is generated by $T(S)$ and the $U_\alpha(S)$, $\alpha\in\Delta\cup-\Delta$.
    \item[(b)] If $G$ is simply connected, $G(S)$ is already generated by the $U_\alpha(S)$, $\alpha\in\Delta\cup-\Delta$. 
\end{enumerate}
\end{corollary}

\begin{corollary}\label{scheme group splitting semi-simple rank 1 cover by U_alpha}
If $G$ is of semi-simple rank $1$, choose a section $u_\alpha\in U_\alpha^\times(S)$. Then $\Omega$ and $u_\alpha\Omega$ form a covering of $G$.
\end{corollary}
\begin{proof}
If $u_{-\alpha}$ is the section of $U_{-\alpha}$ dual to $u_\alpha$, we have, by \cref{scheme group elementary system Weyl group prop}~(\rmnum{4}) and \cref{scheme group splitting Bruhat decomposition}~(b),
\[G=\Omega\cup u_{-\alpha}^{-1}u_\alpha u_{-\alpha}^{-1}\Omega,\]
whence
\begin{equation*}
G=u_{-\alpha}G=u_{-\alpha}\Omega\cup u_\alpha u_{-\alpha}^{-1}\Omega=\Omega\cup u_\alpha\Omega.\qedhere
\end{equation*}
\end{proof}

\begin{corollary}\label{scheme group reductive essentially free}
Let $S$ be a scheme, $G$ be a reductive $S$-group. Then $G$ is essentially free over $S$.
\end{corollary}
\begin{proof}
The assertion is localy for the fpqc topology, so we can suppose that $G$ splits. Then by \cref{scheme group splitting Bruhat decomposition}~(b), $G$ admits a covering by open subschemes isomorphic to $\G_{a,S}^r\times_S\G_{m,S}^s$, hence is essentially free.
\end{proof}
