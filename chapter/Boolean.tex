\chapter{Boolean algebras}
\section{Boolean algebras}
\subsection{The set $2^X$ as a Boolean algebra.}
\begin{definition}
A \textbf{Boolean algebra} (or a \textbf{Boolean ring}) is a ring $A$ in which each element is idempotent.
\end{definition}
The most important and elementry example of a Boolean algebra is the power set of a set nonempty $X$.
\begin{example}\label{Bool ring 2^X}
Let $X$ be a nonempty set, consider the set $2^X$ with the operations of symmetric difference and intersection. 
To verify the distribution and association, we observe that
\[(A\cap B)\cap C)=A\cap(B\cap C).\]
and
\begin{align*}
A\Delta(B\Delta C)&=(A\cap(B\Delta C)^c)\cup(A^c\cap(B\Delta C))\\
&=[A\cap((B\cup C)\cap(B^c\cup C^c))^c]\cup[A^c\cap((B\cap C^c)\cap(B^c\cap C))]\\
&=[A\cap((B^c\cap C^c)\cup(B\cap C))]\cup[A^c\cap((B\cap C^c)\cap(B^c\cap C))]\\
&=(A\cap B^c\cap C^c)\cup(A\cap B\cap C)\cup(A^c\cap B\cap C^c)\cup(A^c\cap B^c\cap C).
\end{align*}
Note that last equation is invariant under $\mathfrak{S}_3$, therefore we get
\[A\Delta(B\Delta C)=(A\Delta B)\Delta C.\]
Also,
\begin{align*}
(A\cap B)\Delta(A\cap C)&=(A\cap B\cap(A\cap C)^c)\cup(A\cap C\cap(A\cap B)^c)\\
&=(A\cap B\cap(A^c\cup C^c))\cup(A\cap C\cap(A^c\cup B^c))\\
&=(A\cap B\cap C^c)\cup(A\cap C\cap B^c)\\
&=A\cap(B\Delta C).
\end{align*} 
Or we can use the characteristic function. We have
\begin{align*}
\chi_{A\Delta B}=\chi_A+\chi_{B},\quad \chi_{A\cap B}=\chi_A\chi_B.
\end{align*}

This means that $(2^X,\cap,\Delta)$ is a ring with $\cap$ being the multiplication and $\Delta$ the addition. In particular, the additive identity is $\emp$, while the multiplicative identity is $X$. 
\end{example}
\begin{proposition}
The ring $2^X$ is commutative and the additive inverse of a subset $A$ is itself. That is, $2^X$ has character $2$. 
\end{proposition}
\begin{proposition}
In the ring $2^X$, we have a order, namely the inclusion, such that for any subset $A,B$ we have
\[A\leq B\iff A\sub B\iff A\cap B=A.\]
Also, the upper and lower bounds of $A,B$ are given by
\[A\cup B=(A\Delta B)\Delta(A\cap B),\quad A\cap B.\]
Moreover, for any element $A$ we have a complement $A^c$ defined By $A^c:=X\Delta A$. It satisfies $A\cap A^c=\emp$.
\end{proposition}
\begin{remark}
Note that
\[\chi_{A\cup B}=\chi_A+\chi_B+\chi_A\chi_B,\quad \chi_{A^c}=1+\chi_A.\]
These will be useful when we consider general Boolean rings.
\end{remark}
\begin{remark}
The order in $2^X$ is compatible with multiplication, but not with addition.
\end{remark}
\subsection{Properties of a Boolean ring.}
\begin{proposition}
Let $A$ be a Boolean ring. Then 
\begin{itemize}
\item $2x=0$ for every $x\in A$.
\item $A$ is commutative.
\item every finitely generated ideal in $A$ is principal.
\end{itemize}
\end{proposition}
\begin{proof}
Note that 
\[x+y=(x+y)^2=x^2+y^2+xy+yx=x+y+xy+yx.\]
Therefore $xy+yx=0$. Take $y=1$ we know that $\char A=2$, and therefore $xy=yx$.\par
Let $I=(x,y)$ be an ideal in $A$, then clearly $(x+y+xy)\sub I$. We claim that $I=(x+y+xy)$. In fact, $x(x+y+xy)=x\in(x+y+xy)$ and $y(x+y+xy)=y\in(x+y+xy)$.
\end{proof}
\begin{proposition}\label{Bool ring cap cup}
Let $A$ be a Boolean ring. Then we can defined an order on $A$ by
\[x\leq y\iff xy=x\]
This order satisfies the following properties.
\begin{itemize}
\item[$(1)$] The least upper bound and greatest lower bound for any two elements in $A$ exist.
\item[$(2)$] Every non-empty finite subset $\{x_1,\dots,x_n\}$ of $A$ has a greatest lower bound $x_1\wedge\cdots\wedge x_n$ 
and a least upper bound $x_1\vee\cdots\vee x_n$.
\item[$(3)$] The operations $\wedge$ and $\vee$ thus defined on $A$ are associative and commutative. Moreover, $0$ is an identity element for the operation $\vee$ and 
an absorbing element for the operation $\wedge$; while $1$ is an identity elementfor the operation $\wedge$ and an absorbing
element for the operation $\vee$.
\item[$(4)$] Each of the operations $\wedge$ and $\vee$ is distributive over the other.
\end{itemize}
\end{proposition}
\begin{proof}
First we check that the definition above indeed gives an order. Let $x,y,z\in A$ such that $x\leq y$ and $y\leq z$, then $xy=x$ and $yz=y$, so that
\[xz=xyz=xy=x.\]
Therefore $x\leq z$. Also, if $x\leq y$ and $y\leq x$, then $x=yx=y$. Thus $\leq$ is indeed an order.\par
One can check that
\[x\wedge y=xy,\quad x\vee y=x+y+xy\]
gives an lower bound and an upper buond of $x$ and $y$. Moreover, if $z$ is a common lower bound for $x$ and $y$, we have $zx=z$ and $zy=y$, hence 
\[z(xy)=(zx)y=zy=z,\] 
which means that $z\leq xy$; thus $xy$ is the greatest of the common lower bounds for $x$ and $y$. Similarly, if $x\leq z$ and $y\leq z$, then $xz=x$ and $yz=y$, and
\[z(x+y+xy)=zx+zy+zxy=x+y+xy.\]
Therefore $x+y+xy$ is the least upper bound of $x$ and $y$. Similarly we can show that $xy$ is the greatest lower bound.\par
Finally, we have
\begin{align*}
x\wedge(y\vee z)&=x(y+z+yz)=xy+xz+xyz=xy+xz+xyxz\\
&=(xy)\vee(xz)=(x\wedge y)\vee(x\wedge z).
\end{align*}
and
\begin{align*}
(x\vee y)\wedge(x\vee z)&=(x+y+xy)(x+z+xz)\\
&=x+xz+xz+xy+yz+xyz+xy+xyz+xyz\\
&=x+yz+xyz=x\vee(yz)\\
&=x\vee(y\wedge z).
\end{align*}
Thus the claim follows.
\end{proof}
\begin{proposition}\label{Bool ring complement}
For every element $x$ in $A$, there is an element $x^c$ in $A$ called the complement of $x$, such that $x^c\vee x=1$ and $x\wedge x^c=0$. The 
map $x\mapsto 1+x$ from $A$ into $A$ is a involution that reverses the order.
\end{proposition}
\begin{proof}
Define $x^c=1+x$. Then we can see that $x\vee x^c=1$ and $x\wedge x^c=0$.\par
This map is an involution since, for every $x$, $1+(1+x)=x$. On the other hand, 
for any elements $x$ and $y$, we have
\[(1+x)\leq(1+y)\iff (1+x)(1+y)=1+x+y+xy=(1+x)\iff y=xy\iff y\leq x\]
as claimed.
\end{proof}
\begin{proposition}
We have the de Morgan's laws:
\[(x\wedge y)^c=x^c\vee y^c.\]
\[(x\vee y)^c=x^c\wedge y^c.\]
\end{proposition}
\begin{proof}
We only prove the first equality. We have
\[(x\wedge y)\wedge(x^c\vee y^c)=(x\wedge y\wedge x^c)\vee(x\wedge y\wedge y^c)=0\vee 0=0,\]
and
\[(x\wedge y)\vee(x^c\vee y^c)=(x\vee x^c\vee y^c)\wedge(y\vee x^c\vee y^c)=1\vee 1=1.\]
Therefore $(x\wedge y)^c=x^c\vee y^c$.
\end{proof}
\begin{proposition}\label{Bool ring mult as cap cup}
The multiplication in $A$ are given by
\[x+y=(x^c\wedge y)\vee(x\wedge y^c)=(x\vee y)\wedge(x^c\vee y),\]
\end{proposition}
\begin{proof}
We verify that
\begin{align*}
(x^c\wedge y)\vee(x\wedge y^c)&=((1+x)y)\vee(x(1+y))\\
&=y+xy+x+xy+xy(1+x)(1+y)\\
&=x+y+xy(1+x+y+xy)=x+y.
\end{align*}
Also,
\begin{align*}
(x\vee y)\wedge(x^c\vee y)&=(x+y+xy)(1+x+1+y+(1+x)(1+y))\\
&=(x+y+xy)(1+xy)\\
&=x+y+xy+xy+xy+xy=xy. 
\end{align*}
Therefore the claim holds.
\end{proof}
\begin{lemma}\label{Bool ring x leq 1+y}
For any elements $x$ and $y$ of $A$, we have $x\leq 1+y$ if and only if $xy=0$.
\end{lemma}
\begin{proof}
Note that
\[x\leq(1+y)\iff x(1+y)=x\iff xy=0\]
as claimed.
\end{proof}
\subsection{Boolean algebras as ordered sets}
\begin{definition}
An ordered set $(A,\leq)$ is called a \textbf{lattice} if 
\begin{itemize}
\item There is a least element (denoted by $0$)  and a greatest element (denoted by $1$) in $A$
\item Any two elements $x$ and $y$ have a least upper bound (denoted by $x\vee y$) and a greatest lower bound (denoted by $x\wedge y$).
\end{itemize}
It is called a \textbf{distributive lattice} if $\wedge$ and $\vee$ is distrihutive over the other, and a \textbf{complemented lattice} if for every element $x$ in $A$, 
there is at least one element $x^c$ in $A$ such that $x\vee x^c=1$ and $x\wedge x^c=0$.
\end{definition}
\begin{lemma}
Let $A$ be a complemented lattice, then the complement of an element $x$ is the unique.
\end{lemma}
\begin{proof}
Suppose that $x_1$ and $x_2$ are each complements of $x$ and consider the element $y=(x\wedge x_1)\vee x_2$. On one hand, $y$ is equal to $0\vee x_2=x_2$. 
On the other hand, distributivity leads us to
\[y=(x\vee x_2)\wedge(x_1\vee x_2)=1\wedge(x_1\vee x_2)=x_1\vee x_2.\]
So we have $x_2=x_1\vee x_2$, which means $x_1\leq x_2$. By interchanging the roles of $x_1$ and $x_2$ in this argument, we naturally obtain $x_2\leq x_1$, 
and, in the end, $x_1=x_2$.
\end{proof}
Now we can give our main theorem.
\begin{theorem}\label{Bool ring lattice}
Let $(A,\leq)$ be distributive and complemented lattice. Then $A$ can be given the structure of a Boolean algebra $(A,+,\times,0,1)$ in such a way that the 
given order $\leq$ on $A$ will coincide with the order that is associated with its Boolean algebra structure.
\end{theorem}
\begin{proof}
The multiplication and addition in $A$ are defined By
\[x+y:=(x\wedge y^c)\vee(x^c\wedge y),\quad x\cdot y:=x\wedge y.\]
Then $0$ is the identity of the addition:
\[x+0=(x\wedge 1)\vee(x^c\wedge 0)=x\vee 0=x,\]
and $1$ is the identity of multiplication:
\[x\cdot 1=x\wedge 1=x.\]
The association of the addition and the multiplication, together with the distribution can be verified just as in 
Example~\ref{Bool ring 2^X}, so that $(A,+,\times,0,1)$ is a Boolean ring. Finally, we check that
\[xy=x\iff x\wedge y=x\iff x\leq y\]
Therefore the order $\leq$ on $A$ coincides with the order of the Boolean algebra.
\end{proof}
\subsection{Atoms in a Boolean algebra}
\begin{definition}
An element $A$ in a Boolean algebra is called an \textbf{atom} if and only if it is non-zero and has no non-zero strict lower bound.
\end{definition}
In other words, $a$ is an atom if and only if $a\neq0$ and, for every element $b$ in $A$, 
if $b\leq a$, then either $b=a$ or $b=0$.
\begin{example}
In the Boolean algebra $2^X$ of subsets of the set $X$, the atoms are the singletons.
\end{example}
\begin{definition}
A Boolean algebra is \textbf{atomic} and only if every non-zero element has at least one atom below it.
\end{definition}
This is the situation, for example, with the Boolean algebra of all subsets of a
given set (every non-empty set contains at least one singleton).
\begin{theorem}\label{Bool ring finite atomic}
Every finite Boolean algebra is atomic.
\end{theorem}
\begin{proof}
Let $A$ be a finite Boolean algebra and let $x$ be a non-zero element of $A$. Denote by $m(x)$ the set of 
non-zero strict lower bounds of $x$ in $A$. If $m(x)$ is empty, then $x$ is an atom. If $m(x)$ is not empty, 
then because it is finite, at least one of its elements is minimal in the ordering $\leq$, i.e. no element of $m(x)$ is strictly below it. 
It is easy to see that such a minimal element is an atom of $A$ that is below $x$.
\end{proof}
Now we characterize atoms.
\begin{theorem}\label{Bool ring atom iff}
Let $A$ be a Boolean algebra. Then for every non-zero element $a$ of $A$ and for every integer $n\geq2$, the following properties are equivalent:
\begin{itemize}
\item[$(1)$] $a$ is a atom.
\item[$(2)$] For every element $x$ in $A$, either $a\leq x$ or $a\leq 1+x$.
\item[$(3)$] For all elements $x_1,\dots,x_n$ in $A$, if $a\leq x_1\vee\cdots\vee x_n$, then $a\leq x_i$ for some $i\in\{1,\dots,n\}$.
\end{itemize}
\end{theorem}
\begin{proof}
Note that $a\leq 1+x$ if and only if $ax=0$. Now if $a$ is a atom, then $ax\leq a$, so $ax=a$ or $ax=0$, which correspond to the two cases in $(2)$.\par
Assume that $(2)$ holds. If there are elements $x_1,\dots,x_n\in A$ such that $a\leq x_1\vee\cdots\vee x_n$ and non of $a\leq x_i$ is ture. Then by $(2)$,
\[a\leq(1+x_1)\wedge\cdots\wedge(1+x_n)=(x_1\vee\cdots\vee x_n)^c.\]
Now $a$ is less than $x_1\vee\cdots\vee x_n$ and its complement, which is impossible since $a$ is nonzero. Hence $(3)$ holds.\par
Now if $(3)$ holds and $b$ is a lower bound of $a$, then $a\leq b\vee(1+b)=1$, so that $a\leq b$ or $a\leq 1+b$. In the first case, we obtain $b=a$ and
in the second case, $b=ab=0$ (Lemma~\ref{Bool ring x leq 1+y}). We have thereby proved that $a$ is an atom.
\end{proof}
\subsection{Homomorphisms of Boolean algebras}
\begin{lemma}\label{Bool ring homomorphism prop}
Let $\varphi::A\to B$ be a Homomorphism between Boolean rings, then
\[\varphi(x\wedge y)=\varphi(x)\wedge\varphi(y),\quad \varphi(x\vee y)=\varphi(x)\vee\varphi(y),\quad \varphi(x^c)=\varphi(x)^c.\]
And
\[x\leq y\Rightarrow \varphi(x)\leq\varphi(y).\]
\end{lemma}
\begin{theorem}\label{Bool ring homomorphism iff}
Let $\varphi:A\to B$ be a map between two Boolean algebras. For $\varphi$ to be a homomorphism of Boolean algebras, it is necessary 
and sufficient that for all elements $x$ and $y$ in $A$, we have
\[\varphi(x\wedge y)=\varphi(x)\wedge\varphi(y),\quad \varphi(x^c)=\varphi(x)^c.\]
\end{theorem}
\begin{proof}
One direction is given in the previous lemma. Now assume that two equalities. We have
\begin{align*}
\varphi(xy)=\varphi(x\wedge y)=\varphi(x)\wedge\varphi(y)=\varphi(x)\varphi(y).
\end{align*}
To deal with the addition, we first note that
\begin{align*}
\varphi(x\vee y)&=\varphi((x^c)^c\vee (y^c)^c)=\varphi((x^c\wedge y^c)^c)\\
&=\varphi(x^c\wedge y^c)^c=(\varphi(x^c)\wedge\varphi(y^c)^c)^c\\
&=\varphi(x^c)^c\vee\varphi(y^c)^c=\varphi(x)\vee\varphi(y).
\end{align*}
Then
\[\varphi(x+y)=\varphi((x\wedge y^c)\vee(x^c\wedge y))=(\varphi(x)\wedge\varphi(y)^c)\vee(\varphi(x)^c\wedge\varphi(y))=\varphi(x)+\varphi(y).\]
With these, we can also get
\[\varphi(0)=\varphi(x-x)=\varphi(x)-\varphi(x)=0,\quad \varphi(1)=\varphi(0^c)=1.\]
Therefore $\varphi$ is a ring homomorphism.
\end{proof}
\begin{remark}
It is clear that in the statement of the preceding theorem, we could replace the operation $\wedge$ by the operation $\vee$ everywhere.
\end{remark}
\begin{theorem}\label{Bool ring isomomorphism iff}
Let $\varphi:A\to B$ be a surjective map between two Boolean algebras. For $\varphi$ to be an isomorphism 
of Boolean algebras, it is necessary and sufficient that
\begin{align}\label{Bool ring isomomorphism iff-1}
x\leq y\iff \varphi(x)\leq\varphi(y).
\end{align}
\end{theorem}
\begin{proof}
First let us suppose that $\varphi$ is an isomorphism and let $x$ and $y$ be elements of 
$A$. If $x\leq y$, then by Lemma~\ref{Bool ring homomorphism prop} $\varphi(x)\leq\varphi(y)$. If $\varphi(x)\leq\varphi(y)$, then 
by the definition of $\leq$ and because $\varphi$ is a homomorphism, $\varphi(x)=\varphi(x)\varphi(y)=\varphi(xy)$.
But since $\varphi$ is injective, this requires $x=xy$, which is to say, $x\leq y$. So $(\ref{Bool ring isomomorphism iff-1})$ is
satisfied.\par
For the converse, if $(\ref{Bool ring isomomorphism iff-1})$, then we can easily seen that
\[\varphi(x\wedge y)=\varphi(x)\wedge\varphi(y),\quad \varphi(x\vee y)=\varphi(x)\vee\varphi(y).\]
Let $u$ be an arbitrary element of $B$ and let $t$ be its unique preimage in $A$ under $\varphi$. In $A$, we have $0\leq t$ and $t\leq 1$. It 
follows, using $(\ref{Bool ring isomomorphism iff-1})$, that, in $B$ we have $\varphi(0)\leq u\leq\varphi(1)$. This shows that $\varphi(0)$ and 
$\varphi(1)$ are, respectively, the least and greatest elements of $B$, or in other words, that $\varphi(0)$ and $\varphi(1)$.\par
So for every element $x$ in $A$, we have
\[\varphi(x^c)\vee\varphi(x)=\varphi(x^c\vee x)=\varphi(1)=1,\quad \varphi(x^c)\wedge\varphi(x)=\varphi(x^c\wedge x)=\varphi(0)=0.\]
Therefore
\[\varphi(x^c)=\varphi(x)^c.\]
Now apply Theorem~\ref{Bool ring homomorphism iff} we get the claim.
\end{proof}
As an application of the previous theorem, we consider finite Boolean algebras.
\begin{theorem}\label{Bool ring finite}
Every finite Boolean algebra is isomorphic to the Boolean algebra of subsets of some set.
\end{theorem}
\begin{proof}
Let $A$ be a finite Boolean algebra and $E$ be the set of its atoms. We prove that $A\cong 2^E$ by define a map
\[\varphi:A\to 2^E,\quad \varphi(x)=\{a\in E:a\leq x\}.\]
\begin{itemize}
\item $\varphi$ is surjective: indeed, we first of all have $\varphi(0)=\emp$; as well, let $X=\{a_1,\dots,a_n\}$ 
be a non-empty subset of $E$ and set $M_X=a_1\vee\cdots\vee a_n$, we claim $\varphi(M_X)=X$: the inclusion $X\sub\varphi(M_x)$ 
follows immediately; the reverse inclusion is shown using Theorem~\ref{Bool ring atom iff}: if $a$ is an element of $\varphi(M_x)$, 
i.e. an atom which is below $M_X=a_1\vee\cdots\vee a_n$, then we have $a\leq a_i$ for at least one index $i$, but since $a$ and $a_i$ 
are atoms, this entails $a=a_i$, and so $a\in X$.\par
\item For all elements $x$ and $y$ of $A$, if $x\leq y$, then $\varphi(x)\leq\varphi(y)$. Conversely, if $\varphi(x)\leq\varphi(y)$, we show 
that $x\leq y$: indeed, if $x$ is not less that or equal to $y$, then $x(1+y)\neq 0$ (Lemma~\ref{Bool ring x leq 1+y}). As $A$ is finite, it is
atomic (Theorem~\ref{Bool ring finite atomic}) so we can find an atom $a\in E$ such that $a\leq x(1+y)$. The atom $a$ is thus below both $x$ and $1+y$; 
it cannot be below $y$ as well since it is non-zero. So we have $a\in \varphi(x)$ and $a\notin\varphi(y)$, which shows that $\varphi(x)$ is not
included in $\varphi(y)$.
\end{itemize}
We may now conclude, thanks to Theorem~\ref{Bool ring isomomorphism iff}, that $\varphi$ is an isomorphism of Boolean algebras from $A$ onto $2^E$.
\end{proof}
\subsection{Boolean subalgebras}
\begin{definition}
A subalgebra of a Boolean algebra is just a subring.
\end{definition}
\begin{proposition}
A subset of a Boolean algerba is a Boolean algebra if and only if it is the image of a homomorphism bewteen Boolean algebras.
\end{proposition}
\begin{theorem}\label{Bool ring subalg iff}
In a Boolean algebra $A$, for a subset $S$ to be a Boolean subalgebra, it is necessary and sufficient that $B$ contain $0$ and 
be closed under the operations $x\mapsto x^c$ and $(x,y)\mapsto x\wedge y$.
\end{theorem}
\begin{proof}
If $S$ is closed under the  operations, then the closure of $S$ under complementation and $\wedge$ guarantees its closure under $\vee$. 
Moreover, $1=0^c$ must belong to $S$. Since the operations $+$ and $\times$ can be defined exclusively in terms of $\wedge,\vee$, and 
complementation, we conclude that $S$ is closed under $+$ and $\times$ and that $S$ is a Boolean subalgebra of $A$.
\end{proof}
\begin{example}
Let $X$ be a topological space and let $\mathcal{B}(X)$ be the subset of $2^X$ consisting of subsets of $X$ 
that are both open and closed in the topology on $X$. This set $\mathcal{B}(X)$ is a Boolean subalgebra of 
the Boolean algebra of subsets of $X$.
\end{example}
\begin{example}
Let $A$ be a Boolean algebra and let $a$ be an atom in this algebra (we are assuming that one exists). Let us define 
a map $\varphi_a$ from $A$ into $\{0,1\}$ by
\[\varphi_a(x)=\begin{cases}
1&a\leq x,\\
0&a\leq 1+x.
\end{cases}\]
These two cases are mutually exclusive since $a$ is different from $0$, and there are no other cases since $a$ is an atom.\par
Let $x$ and $y$ be two elements of $A$. We have 
\begin{align*}
\varphi_a(x\wedge y)=1&\iff a\leq x\wedge y\iff a\leq x,a\leq y\\
&\iff \varphi_a(x)=\varphi_a(y)=1\\
&\iff \varphi_a(x)\wedge\varphi_a(y)=1.
\end{align*}
It follows that
\[\varphi_a(x\wedge y)=\varphi_a(x)\wedge\varphi_a(y).\]
Also,
\[\varphi_a(x^c)=1\iff a\leq x^c=1+x.\]
Since $\varphi_a$ only assumes the values $0$ or $1$, this means that $\varphi_a(x^c)=\varphi(x)^c$. Therefore $\varphi_a$ is a homomorphism from $A$ to $\{0,1\}$.
\end{example}
\section{Idelas and filters}
\subsection{Ideals}
\begin{proposition}
Let $A$ be a Boolean algebra and $I$ a subset of $A$. For $I$ to be an ideal, it is necessary and sufficient that the following three conditions be satisfied:
\begin{itemize}
\item $0\in I$, $1\notin I$.
\item for all elements $x$ and $y$ of $I$, $x\vee y\in I$.
\item for all $x\in I$ andfor all $y\in A$, if $y\leq x$, then $y\in I$.
\end{itemize}
\end{proposition}
\begin{proof}
Assume that conditions holds. If $x\in I$ and $y\in I$, then $x\vee y\in I$, but since $x+y\leq x\vee y$ (this is trivial to check). Since $0\in I$, we have all we need for $I$ to be a subgroup. 
Moreover, if $x\in I$ and $y\in A$, then since $xy\leq x$, we may conclude that $xy\in I$. The set $I$ is therefore an ideal.
\end{proof}
\begin{corollary}
If $I$ is an ideal in a Boolean algebra $A$, there is no element $x$ in $A$ that can satisfy both $x\in I$ and $1+x\in I$.
\end{corollary}
Here is a collection of ways to characterize maximal ideals in a Boolean algebra:
\begin{theorem}\label{Bool ring maximal ideal iff}
For every Boolean ring $A$, for every ideal $I$ in $A$, the following properties are equivalent:
\begin{itemize}
\item[$(1)$] $I$ is maximal.
\item[$(2)$] $A/I$ is isomorphic to $\{0,1\}$.
\item[$(3)$] For every element $x$ in $A$, $x\in I$ or $1+x\in I$.
\item[$(4)$] $I$ is prime.
\end{itemize}
\end{theorem}
\begin{proof}
Note that any Boolean integral domain is isomorphic to $\{0,1\}$ and hence a field.
\end{proof}
\subsection{Filters}
\begin{definition}
A \textbf{filter} in a Boolean algebra $A$ is a subset of $A$ such that $\{x\in A:x^c\in F\}$ is an ideal.
\end{definition}
\begin{proposition}
Let $A$ be a Boolean algebra and $F$ a subset of $A$. For $F$ to be afilter, it is necessary and sufficient that the following three conditions be satisfied:
\begin{itemize}
\item $1\in F$, $0\notin F$.
\item for all elements $x$ and $y$ of $F$, $x\wedge y\in F$.
\item for all $x\in F$ andfor all $y\in A$, if $y\geq x$, then $y\in F$.
\end{itemize}
\end{proposition}
\begin{definition}
In a Boolean algebra, an \textbf{ultrafilter} is a maximal filter, i.e. a filter which is not strictly included in any other filter.
\end{definition}
Therefore $F$ is an ultrafilter if and only if $F^c$ is a maximal ideal.
\begin{theorem}\label{Bool ring ultrafilter iff}
For every Boolean ring $A$, for every filter $F$ in $A$, the following properties are equivalent:
\begin{itemize}
\item[$(1)$] $F$ is an ultrafilter.
\item[$(2)$] there is a homomorphism $\varphi$ from $A$ into $\{0,1\}$ such that
\[F=\{x\in A:\varphi(x)=1\}.\] 
\item[$(3)$] for every element $x$ in $A$, $x\in F$ or $1+x\in F$.
\item[$(4)$] for all elements $x$ and $y$ of $A$, if $x\vee y\in F$, then $x\in F$ or $y\in F$.
\end{itemize}
\end{theorem}
\begin{example}
Here are some examples of filters.
\begin{itemize}
\item If $E$ is an infinite set, the set of all cofinite subsets of $E$ is a filter in the Boolean algebra $2^E$. 
This filter is called the \textbf{Frechet filter} on $E$. It is not an ultrafilter since there are subsets of $E$ 
which are infinite and whose complements are also infinite, so condition $(3)$ of Theorem~\ref{Bool ring ultrafilter iff} is not satisfied.
\item If $a$ is a non-zero element in a Boolean algebra $A$, the set $F_a=\{x\in A:x\geq a\}$ is a filter called the \textbf{principal filter generated by $\bm{a}$}. It is the dual
filter of the ideal generated by $1+a$.
\end{itemize}
\end{example}
\begin{theorem}
Let $A$ be a Boolean algebra and let $a$ be a non-zero element of $A$. For the principal filter generated by $a$ to 
be an ultrafilter, it is necessary and sufficient that $a$ be an atom.
\end{theorem}
\begin{proof}
This comes from Theorem~\ref{Bool ring atom iff}$(2)$ and Theorem~\ref{Bool ring ultrafilter iff}$(3)$.
\end{proof}
When the principal filter Fa generated by a non-zero element $a$ of $A$ is an ultrafilter (thus, when $a$ is an atom), we 
say that this is a \textbf{trivial ultrafilter}. The homomorphism $\varphi_a$ with values in $\{0,1\}$ that is associated with it is also called a trivial homomorphism.
\begin{lemma}
Let $A$ be a Boolean algebra and let $F$ be an ultrafilter on $A$. For $F$ to be trivial, it is necessary and sufficient that it contain at least one atom.
\end{lemma}
\begin{proof}
If $F$ contains an atom $b$, it also contains all elements greater than or equal to $b$. It follows that the principal filter $F_b$ generated by $b$ is included in $F$. 
But since $b$ is an atom, $F_b$ is maximal. Hence $F=F_b$ is a trivial ultrafilter.
\end{proof}
\begin{theorem}
Let $E$ be an infinite set and let $\mathcal{F}$ be an ultrafilter in the Boolean algebra $2^E$. For $\mathcal{F}$ to be non-trivial, it is necessary and sufficient that it includes the Frechet filter on $E$.
\end{theorem}
\begin{proof}
The atoms in $2^E$ are singletons, therefore if $\mathcal{F}$ is trivial, it clear does not include that Frechet filter. Conversely, If $\mathcal{F}$ does not include the Frechet filter, 
we can choose a cofinite subset $X$ of $E$ that does not belong to $\mathcal{F}$, and hence whose complement $E-X$ does belong to $\mathcal{F}$. As $E$ is the identity element for the Boolean 
algebra $2^E$, $E\in\mathcal{F}$; hence $X\neq E$. The complement of $X$ in $E$ is thus a non-empty finite subset of $E$: for example, $E-X=\{a_1,\cdots,a_n\}$. So we have 
$\{a_1,\dots,a_n\}\in\mathcal{F}$, which is to also say:
\[\{a_1\}\vee\cdots\vee\{a_n\}\in\mathcal{F}.\]
Now by Theorem~\ref{Bool ring ultrafilter iff} we have $a_i\in\mathcal{F}$ for some $i\in\{1,\dots,n\}$. Therefore $\mathcal{F}$ is trivial by the preceding lemma.
\end{proof}
\subsection{Filterbases}
\begin{definition}
In a Boolean algebra $A$, a \textbf{basis for a filter} (filterbase) is a subset of A that has the following property, known as the finite intersection property: every non-empty finite 
subset has a non-zero greatest lower bound.
\end{definition}
\begin{lemma}\label{Bool ring filter contain}
Let $A$ be a Boolean algebra and let $X$ be a subset of $A$. For the existence of a filter on $A$ that includes $X$, it is necessary and sufficient that $X$ be a filter base.
\end{lemma}
\begin{proof}
If $X$ is included in a filter $F$, and if $x_1,\dots,x_n$ are elements of $X$, then their greatest lower bound $x_1,\dots,x_n$ belongs to $F$, and as $0\notin F$, this greatest 
lower bound is non-zero; thus $X$ is a filterbase.\par
Now suppose that $X$ is a filterbase. We can use $X$ to generate a filter:
\[F_X:=\{x\in A:(\exists n\in\N)((\exists x_1,\dots,x_n\in X)x\geq x_1\wedge\cdots\wedge x_n\}.\]
\end{proof}
In the particular case of Boolean algebras, we can state Krull's theorem in terms of filters. It is then known as the ultrafilter theorem:
\begin{theorem}
In a Boolean algebra, every filter is included in at least one ultrafilter.
\end{theorem}
\begin{lemma}\label{Bool ring ultrafilter contain}
Let $A$ be a Boolean algebra and let $X$ be a subset of $A$. For the existence of an ultrafilter on $A$ that includes $X$, it is necessary and sufficient that $X$ be afilterbase.
\end{lemma}
\section{Boolean spaces}
\begin{definition}
Let $X$ be a topological space.
\begin{itemize}
\item We say $X$ is \textbf{totally disconnected} if the only connected subsets of $X$ are single points.
\item We say $X$ is \textbf{totally separated} if, whenever $x$ and $y$ are distinct points of $X$, there is a clopen subset of $X$ containing $x$ but not $y$.
\item We say $X$ is \textbf{zero-dimensional} if the clopen subsets of $X$ form a base for the topology.
\end{itemize}
\end{definition}
The implications between these concepts and the separation axioms are trivial:
\begin{lemma}
\begin{itemize}
\item A totally disconnected space is $T_1$.
\item A totally separated space is Hausdorff and totally disconnected.
\item A zero-dimensional $T_0$-space is regular and totally separated.
\end{itemize}
\end{lemma}
Now we have the following equivalences.
\begin{theorem}
The following conditions on a space $X$ are equivalent:
\begin{itemize}
\item[$(1)$] $X$ is compact, Hausdorff and totally disconnected.
\item[$(2)$] $X$ is compact and totally separated.
\item[$(3)$] $X$ is compact, $T_0$ and zero-dimensional.
\item[$(4)$] $X$ is Hausdorff and spectral.
\end{itemize}
\end{theorem}
\begin{proof}
The implication $(3)\Rightarrow(2)\Rightarrow(1)$ follows from the previous lemma.\par
$(1)\Rightarrow(2)$: Let $x$ be a point of $X$, and consider the set $C(x)$ of all points of
$X$ which cannot be separated from $x$ by a clopen set. First we note that $C(x)$ is closed, for if 
$y\in X-C(x)$ then there is a clopen set $U$ containing $y$ but not $x$, and $U$ is then a neighbourhood 
of $y$ contained in $X-C(x)$.\par
Suppose $C(x)$ contains more than one point; then it is disconnected, i.e. we can write $C(x)=F_1\cup F_2$ 
where $F_1$ and $F_2$ are disjoint closed subsets of $C(x)$ (and hence of $X$). Since $X$ is compact Hausdorff, 
it is normal; so we can find an open set $U\sups F_1$ such that the closure of $U$ does not meet $F_2$. Consider the 
boundary $\partial U=\widebar{U}\cap(X-U)$; it is clear that $\partial U$ does not meet $C(x)$, since $X-U$ does not 
meet $F_1$ and $\widebar{U}$ does not meet $F_2$. So for each $y\in\partial U$, we can find a clopen set $V_y$ 
containing $y$ but not $x$. But $\partial U$ is closed in $X$ and therefore compact; so we can cover it by a
finite number of sets $V_{y_1},\dots,V_{y_n}$. Let $V$ be the clopen set $V=\bigcup_{i=1}^{n}V_{y_i}$, then
$V$ contains $\partial U$, but $x\notin V$ and so $V$ does not meet $C(x)$. Now it is clear that the set
\[W:=U-V=\widebar{U}-V\]
is both open and closed; but $W\cap C(x)=U\cap C(x)=F_1$, so both $W$ and $X-W$ meet $C(x)$. But this contradicts 
the definition of $C(x)$, since they cannot both contain $x$. So $C(x)$ consists of the single point $x$, i.e. $X$ is
totally separated.\par
$(2)\Rightarrow(3)$: Let $U$ be open in $X$, $x\in U$. We have to find a clopen $V$ with $x\in V\sub U$. But for each $y\in X-U$ 
we can find a clopen $V_y$ with $x\in V_y$ and $y\notin V_y$; now the sets $\{X-V_y:y\in X-U\}$ form an open cover of the closed 
(hence compact) set $X-U$, so there is a finite subcover $X-V_{y_1},\dots,X-V_{y_n}$. Then $V:=\bigcap_{i=1}^{n}V_{y_i}$ is the required
clopen set.\par
$(3)\Rightarrow(4)$: In a compact Hausdorff space, the compact open subsets are precisely the clopen subsets; so they are automatically 
closed under finite intersections, and they form a base for the topology iff the clopen sets do. And any Hausdorff space is sober.\par
$(4)\Rightarrow(1)$: If $X$ is spectral then it is compact. If it is Hausdorff, then it is totally disconnected.
\end{proof}
A space satisfying the conditions of the Theorem will be called a \textbf{Boolean space}.
\section{Stone's duality}
We will refer to the duality Stone established between Boolean algebras and certain topological spaces as Stone duality. In the following when we speak of a clopen
set, we will mean of course a closed and open set.
\subsection{Stone space of a Boolean algebra}
\begin{definition}
Let $A$ be a Boolean algebra. The set of homomorphisms of a Boolean algebra $A$ into $\{0,1\}$ is denoted 
by $\mathcal{S}(A)$ and is called the \textbf{Stone space} of $A$. We endow $\mathcal{S}(A)$ with the subspace topology of $\{0,1\}^A$.
\end{definition}
Since any homomorphism from $A$ to $\{0,1\}$ is associated with an ultrafilter of $A$, $\mathcal{S}(A)$ can also be regarded as the collection of all ultrafilter of $A$. With this in mind, we define
\[\Delta_a=\{\varphi\in\mathcal{S}(A):\varphi(a)=1\}=\{\varphi\in\mathcal{S}(A):a\in F_\varphi\}\]
where $F_\varphi$ is the ultrafilter associated with $\varphi$. First we observe that
\begin{lemma}\label{Bool ring delta_a}
For any $a,b\in A$, we have
\[\Delta_a\cup\Delta_b=\Delta_{a\vee b},\quad \Delta_a\cap\Delta_b=\Delta_{a\wedge b},\quad \Delta_{1+a}=(\Delta_a)^c.\]
Moreover, if $a\neq b$, then $\Delta_a\neq\Delta_b$.
\end{lemma}
\begin{proof}
We have
\begin{align*}
\varphi\in\Delta_a\cup\Delta_{b}\iff \varphi(a)=1\text{ or }\varphi(b)=1\iff\varphi(a\vee b)=\varphi(a)\vee\varphi(b)=1.
\end{align*}
Therefore $\Delta_{a}\cup\Delta_b=\Delta_{a\vee b}$. The rest are proved similarly.\par
Now if $a\neq b$, then $a+b\neq 0$, so we may consider the principal filter generated by $a+b$ and, in view of the ultrafilter theorem, 
an ultrafilter that includes this filter. To such an ultrafilter, there is an associated homomorphism $\varphi$ from $A$ into to $\{0,1\}$ 
that satisfies $\varphi(a+b)=1$, or again, $\varphi(a)+\varphi(b)=1$, which in particular means that one and only one of the two elements
$\varphi(a)$ and $\varphi(b)$ is equal to $1$. This proves that $\Delta_a\neq\Delta_b$.
\end{proof}
With these observations, we can show that $\{\Delta_a:a\in A\}$ forms a basis of $\mathcal{S}(A)$.
\begin{proposition}\label{Bool ring basic open}
For a subset $U$ of $\mathcal{S}(A)$ to be a basic open set, it is necessary and sufficient that there exist an element $a$ in $A$ such that $U=\Delta_a$. 
In particular, $\{\Delta_a:a\in A\}$ is a basis for $\mathcal{S}(A)$.
\end{proposition}
\begin{proof}
A basic open set in $\mathcal{S}(A)$ is of the forms
\[U=\{0,1\}^{A-\{i_1,\dots,i_n\}}\times\{\eps_{i_1}\}\times\cdots\times\{\eps_{i_n}\}\cap\mathcal{S}(A)=\{\varphi\in\mathcal{S}(A):\varphi(a_1)=\eps_1,\dots,\varphi(a_n)=\eps_n.\}\]
For each $i\in\{1,\dots,n\}$ we set
\[b_i=\begin{cases}
a_i&\eps_i=1,\\
1+a_i&\eps_i=0.
\end{cases}\]
Then $U$ can be written as
\[U=\{\varphi\in\mathcal{S}(A):\varphi(b_1)=\cdots=\varphi(b_n)=1\}=\Delta_{b_1}\cap\cdots\cap\Delta_{b_n}=\Delta_{b_1\wedge\cdots\wedge b_n}.\]
Therefore the claim follows.
\end{proof}
Now we prove an important result.
\begin{proposition}
The space $\mathcal{S}(A)$ is a Boolean space.
\end{proposition}
\begin{proof}
Since $\{0,1\}^A$ is Hausdorff, $\mathcal{S}(A)$ is also Hausdorff. From corollary~\ref{Bool ring S(A) clopen} we see that $\mathcal{S}(A)$ has a basis of 
clopen subsets, so we only need to show that $\mathcal{S}(A)$ is compact.\par
We must show that from any family of closed subsets of $\mathcal{S}(A)$ whose intersection is empty, we can extract a finite subfamily whose intersection is already
empty. But we have seen that it suffices to do this for families of basic closed sets.\par
Now, as we have just seen, the basic closed sets coincide with the basic open sets. So consider an family $\{\Gamma_i\}_{i\in I}$ of basic open subsets 
of $\mathcal{S}(A)$ such that $\bigcap_{i\in I}\Gamma_i=\emp$. By the preceding corollary, there exists, for each $i\in I$, a unique element $x_i$ in $A$ such 
that $\Gamma_i=\Delta_{x_i}$. Set $X=\{x_i:i\in I\}$. To say that the intersection of the family $\{\Gamma_i\}_{i\in I}$ is empty is to say that there is no 
homomorphism of Boolean algebras from $A$ into $\{0,1\}$ that assumes the value $1$ for all elements of $X$, or again, that there is no ultrafilter on $A$ that 
contains $X$. This means (Lemma~\ref{Bool ring ultrafilter contain}) that $X$ is not a filterbase. So there exists a finite subset $\{x_{i_1},\dots,x_{i_n}\}\sub X$ 
such that
\[x_{i_1}\wedge\cdots\wedge x_{i_n}=0.\]
Then no ultrafilter on $A$ could simultaneously contain $x_{i_1},\dots,x_{i_n}$. In other words, no homomorphism from $A$ into $\{0,1\}$ can simultaneously assume the value $1$ at the points $x_{i_1},\dots,x_{i_n}$. 
This amounts to saying that
\[\Gamma_{i_1}\cap\cdots\cap\Gamma_{i_n}=\emp.\]
We thus have a finite subfamily of the family $\{\Gamma_{i}\}_{i\in I}$ whose intersection is empty.
\end{proof}
\begin{corollary}\label{Bool ring S(A) clopen}
The set of clopen subsets of $\mathcal{S}(A)$ coincides with the set of its basic open sets.
\end{corollary}
\begin{proof}
A clopen subset $\Gamma$ of $\mathcal{S}(A)$ is compact, hence is a finite union of basic open subsets. But by Proposition~\ref{Bool ring basic open} and Lemma~\ref{Bool ring delta_a} we see that $\Gamma$ is itself 
a basic open set.
\end{proof}
Now we are ready to state the Stone's theorem.
\begin{definition}
Let $X$ be a Boolean space, the Boolean subalgebra of clopen subsets of $X$ is denoted by $\mathcal{B}(X)$. 
\end{definition}
\begin{theorem}[\textbf{Stone}]
Let $A$ be a Boolean algebra. Then $A$ is isomorphic to $\mathcal{B}\mathcal{S}(A)$ under the mapping
\[a\mapsto\Delta_a.\]
\end{theorem}
\begin{proof}
We already see that this map is surjective (Corollary~\ref{Bool ring S(A) clopen}), so we only need to show that
\[x\leq y\iff \Delta_x\sub\Delta_y\]
in view of Theorem~\ref{Bool ring isomomorphism iff}.\par
So let $x$ and $y$ be two elements of $A$. If $x$ is less than or equal to $y$, then for any homomorphism $\varphi(x)$ that satisfies $\varphi(x)=\varphi(xy)=1$, we must also have $\varphi(y)=1$, 
which means that $\Delta_x$ is a subset of $\Delta_y$. If $x$ is not less than or equal to $y$, then $x(1+y)\neq0$ (Lemma~\ref{Bool ring x leq 1+y}). So we may consider the 
principal filter generated by $x(1+y)$, an ultrafilter that includes it (by the ultrafilter theorem) and the homomorphism $\varphi$ associated with this ultrafilter. We have $\varphi(x(1+y))=1$,
hence $\varphi(x)=1$ and $\varphi(1+y)=1$, i.e. $\varphi(y)=0$. We conclude that $\varphi\in\Delta_x-\Delta_y$, and so $\Delta_x$ is not included in $\Delta_y$.
\end{proof}
Stone's theorem allows us to give a very simple proof of Theorem~\ref{Bool ring finite}.
\begin{corollary}
Everyfinite Boolean algebra is isomorphic to a Boolean algebra of subsets of some set.
\end{corollary}
\begin{proof}
If $A$ is a finite Boolean algebra, then the space $\{0,1\}^A$ is discrete, hence so is $\mathcal{S}(A)$. Therefore $\mathcal{B}\mathcal{S}(A)$ coincides with $2^{\mathcal{S}(A)}$ and $A$ is isomorphic to $2^{\mathcal{S}(A)}$.
\end{proof}
\subsection{Boolean spaces are Stone spaces}
With each Boolean algebra, we have associated a Boolean topological space: its Stone space $\mathcal{S}(A)$, and we have seen that $A$ is isomorphic to the Boolean algebra
of clopen subsets of this Boolean space. It is therefore natural to study the case in which $A$ is given as the Boolean algebra of clopen subsets of some Boolean topollogical 
space $X$. The problem that then arises is to compare $X$ and $\mathcal{S}\mathcal{B}(X)$.
Let $X$ be a Boolean space. For each $x\in X$, let $\varphi_x$ denote the map from $\mathcal{B}(X)$ into $\{0,1\}$ defined by
\[\varPhi_x(\Omega)=\begin{cases}
1&x\in\Omega\\
0&x\notin\Omega
\end{cases}\]
\begin{lemma}
The map $\varPhi_x$ is a homomorphism from $\mathcal{B}(X)$ to $\{0,1\}$, and for $x\neq y$ we have $\varPhi_x\neq\varPhi_y$.
\end{lemma}
\begin{proof}
For any clopen subsets $U$ and $V$ of $X$, we have $\varPhi_x(U\cap V)=1$ if and only if $x\in U\cap V$, i.e. $x\in U$ and $x\in V$. We conclude from this that 
$\varPhi_x(U\cap V)=\varPhi_x(U)\wedge\varPhi_x(V)$. Also, $\varPhi_x(\Omega^c)=1+\varPhi_x(\Omega)$. 
Therefore $\varPhi_x$ is a homomorphism.\par
For $x\neq y$, since $X$ is Hausdorff, we see that $\varPhi_x\neq\varPhi_y$.
\end{proof}
\begin{lemma}
Every homomorphism from $\mathcal{B}(X)$ to $\{0,1\}$ is of the form $\varPhi_x$ for some $x\in X$.
\end{lemma}
\begin{proof}
Let $\psi$ be a homomorphism from $\mathcal{B}(X)$ into $\{0,1\}$. The ultrafilter on $\mathcal{B}(X)$ associated with $\varphi$ is
\[\mathcal{U}=\{\Omega\in\mathcal{B}(X):\varphi(\Omega)=1\}.\]
Since $\mathcal{U}$ has the finite intersection property, and the elements of $\mathcal{U}$ are closed, and since the topological space $X$ is compact, we may assert that the intersection 
of all the elements of $\mathcal{U}$ is non-empty. Let $x$ be an element of this intersection. We claim that $\varphi=\varPhi_x$.\par
For every clopen set $\Omega\in\mathcal{S}$, we have: either $\Omega\in\mathcal{U}$, in which case $\varPhi_x(\Omega)=1$ and $\varphi(\Omega)=1$, or else $\Omega^c\in\mathcal{U}$, 
in which case $\varPhi_x(\Omega)=0$ and $\varphi(\Omega)=0$. Thus it follows that $\varphi=\varPhi_x$.
\end{proof}
\begin{lemma}
The map $\varPhi:X\to\mathcal{S}\mathcal{B}(X)$ is open and continuous.
\end{lemma}
\begin{proof}
Let $G$ be an open set belonging to the basis of clopen subsets of $\mathcal{S}\mathcal{B}(X)$. According to Lemma~\ref{Bool ring basic open}, there exists a unique element $\Omega$ in 
$\mathcal{B}(X)$ such that
\[G=\Delta_{\Omega}=\{\varphi\in\mathcal{S}\mathcal{B}(X):\varphi(\Omega)=1\}.\]
Note that
\begin{align*}
\varPhi^{-1}(G)=\{x\in X:\varPhi_x\in G\}=\{x\in X:\varPhi_x(\Omega)=1\}=\{x\in X:x\in\Omega\}=\Omega.
\end{align*}
Therefore the map $\varPhi$ is continuous. Moreover,
\begin{align*}
\varPhi(\Omega)=\{\varphi\in\mathcal{S}\mathcal{B}(X):\varphi=\varPhi_x\text{ for some }x\in\Omega\}.
\end{align*}
Now for every $x\in\Omega$ we have $\varPhi_x(\Omega)=1$, so $\varPhi(\Omega)\sub\Delta_{\Omega}$. On the other hand, by the previous lemma we have $\varphi=\varPhi_x$ with some $x\in X$ for each $\varphi\in\mathcal{S}\mathcal{B}(X)$. If $\varphi\in\Delta_{\Omega}$, then $x\in\Omega$, so 
$\varphi\in\varPhi(\Omega)$. This proves $\Delta_{\Omega}=\varPhi(\Omega)$ and therefore $\varPhi$ is open.
\end{proof}
The therorem is now obvious.
\begin{theorem}
Every Boolean topological space $X$ is homeomorphic to the Stone space $\mathcal{S}\mathcal{B}(X)$ of the Boolean algebra of clopen subsets of $X$.
\end{theorem}
\subsection{Equivalence of the category of Boolean spaces and algebras}
Let $\mathsf{Ring}_{Bool}$ and $\mathsf{Top}_{Bool}$ be the categories of Boolean spaces and algebras, respectively. We now define two functors
\[\mathcal{S}:\mathsf{Ring}_{Bool}\to\mathsf{Top}_{Bool},\quad \mathcal{B}:\mathsf{Top}_{Bool}\to\mathsf{Ring}_{Bool}\]
and show that they gives an Equivalence of categories. Since we already have the isomorphisms $X\cong\mathcal{S}\mathcal{B}(X)$ and $A\cong\mathcal{B}\mathcal{S}(A)$, we only need to consider the morphisms.
\begin{definition}
Let $\varphi:A\to B$ be a homomorphism between Boolean algerbas, we define 
\[\mathcal{S}(\varphi):\mathcal{S}(B)\to\mathcal{S}(A),\quad \psi\mapsto\psi\circ\varphi\]
to be the included map between stone spaces.
\end{definition}
\begin{definition}
Let $f:X\to Y$ be a continuous map between Boolean spaces, we define
\[\mathcal{B}(f):\mathcal{B}(Y)\to\mathcal{B}(X),\quad U\mapsto f^{-1}(U)\]
to be the included map between stone spaces.
\end{definition}
First we check that
\begin{lemma}
The map $\mathcal{S}(\varphi)$ is continuous, and the map $\mathcal{B}(f)$ is a homomorphism.
\end{lemma}
\begin{proof}
Note that for $b\in B$,
\[\mathcal{S}(\varphi)^{-1}(\Delta_b)=\{\psi\in\mathcal{S}(A)\mid\psi\circ\varphi(b)=1\}=\Delta_{\varphi(b)}.\]
Therefore $\mathcal{S}(\varphi)$ is continuous. Also,
\[\mathcal{B}(f)(U\cap V)=f^{-1}(U\cap V)=f^{-1}(U)\cap f^{-1}(V)=\mathcal{B}(f)(U)\cap\mathcal{B}(f)(V),\]
\[\mathcal{B}(f)(U\Delta V)=f^{-1}(U\Delta V)=f^{-1}(U)\Delta f^{-1}(V)=\mathcal{B}(f)(U)\Delta\mathcal{B}(f)(V)\]
so $\mathcal{B}(f)$ is a homomorphism.
\end{proof}
\begin{theorem}
The functor $\mathcal{S}:\mathsf{Ring}_{Bool}\to\mathsf{Top}_{Bool}$ is an equivalence of categories, so is the functor $\mathcal{B}:\mathsf{Top}_{Bool}\to\mathsf{Ring}_{Bool}$.
\end{theorem}
\begin{proof}
We verify the naturality of the isomorphisms $\Delta:A\to\mathcal{B}\mathcal{S}(A)$ and $\varPhi:X\to\mathcal{S}\mathcal{B}(X)$.\par
First, let $\varphi:A\to B$ be a homomorphism 
of Boolean algebras. We need to establish the commutativity of the diagram
\[\begin{tikzcd}
A\ar[r,"\Delta"]\ar[d,swap,"\varphi"]&\mathcal{B}\mathcal{S}(A)\ar[d,"\mathcal{B}\mathcal{S}(f)"]\\
B\ar[r,"\Delta"]&\mathcal{B}\mathcal{S}(B)
\end{tikzcd}\]
For any $a\in A$ we see that
\[\mathcal{B}\mathcal{S}(\varphi)(\Delta_a)=\mathcal{S}(\varphi)^{-1}(\Delta_a)=\Delta_{\varphi(a)}.\]

Then, let $f:X\to Y$ be a continuous map between Boolean spaces, consider the diagram
\[\begin{tikzcd}
X\ar[r,"\varPhi"]\ar[d,swap,"f"]&\mathcal{S}\mathcal{B}(Y)\ar[d,"\mathcal{S}\mathcal{B}(f)"]\\
Y\ar[r,"\varPhi"]&\mathcal{S}\mathcal{B}(Y)
\end{tikzcd}\]then
\[\mathcal{S}\mathcal{B}(f)(\varPhi_x)(\Omega)=(\varPhi_x\circ\mathcal{B}(f))(\Omega)=\varPhi_x(f^{-1}(\Omega))=\begin{cases}
1&x\in f^{-1}(\Omega)\\
0&x\notin f^{-1}(\Omega)
\end{cases}=\varPhi_{f(x)}(\Omega).\]
Therefore the claim follows.
\end{proof}
Next we establish some properties of the functors $\mathcal{S}$ and $\mathcal{B}$.
\begin{lemma}
Let $\varphi:A\to B$ be a homomorphism between Boolean algebras. Then $\varphi$ is injective (respectively, surjective) if and only if $\mathcal{S}(\varphi)$ is surjective 
(respectively, injective).
\end{lemma}
\begin{lemma}
Let $f:X\to Y$ be a continuous map between Boolean spaces. If $f$ is injective (respectively, surjective), then $\mathcal{B}(f)$ is surjective (respectively, injective).
\end{lemma}
\subsection{Spectra of a Boolean ring}
\begin{proposition}
Let $A$ be a Boolean ring. Then the space $\Spec(A)$ is homeomorphic to $\mathcal{S}(A)$.
\end{proposition}