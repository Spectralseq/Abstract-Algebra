\chapter{\texorpdfstring{$D$}{D}-modules and perverse sheaves}
\section{Elementary properties of \texorpdfstring{$D$}{D}-modules}
In this section we introduce several standard operations for $D$-modules over smooth algebraic varieties over $\C$ and present some fundamental results concerning them such as Kashiwara's equivalence theorem. Our main reference is \cite{hotta_Dmodule}.
\subsection{Differential operators}
Let $X$ be a smooth (non-singular) algebraic variety over the complex number field $\C$ and $\mathscr{O}_X$ be the sheaf of rings of regular functions (structure sheaf) on it. We denote by $\Theta_X$ the \textbf{sheaf of vector fields} (\textbf{tangent sheaf}) on $X$:
\begin{align*}
\Theta_X&=\sDer_{\C_X}(\mathscr{O}_X)=\sHom_{\mathscr{O}_X}(\Omega_X^1,\mathscr{O}_X)\\
&=\{P\in\sEnd_{\C_X}(\mathscr{O}_X):\text{$P(fg)=P(f)g+fP(g)$ for $f,g\in\mathscr{O}_X$}\}.
\end{align*}
Hereafter, if there is no risk of confusion, we use the notation $f\in\mathscr{O}_X$ for a local section $f$ of $\mathscr{O}_X$. Since $X$ is smooth, the sheaf $\Theta_X$ is locally free of rank $n=\dim(X)$ over $\mathscr{O}_X$. We will identify $\mathscr{O}_X$ with a subsheaf of $\sEnd_{\C_X}(\mathscr{O}_X)$ by identifying $f\in\mathscr{O}_X$ with $()\in\sEnd_{\C_X}(\mathscr{O}_X)$. We define the sheaf $\mathscr{D}_X$ of \textbf{differential operators} on $X$ as the subalgebra of $\sEnd_{\C_X}(\mathscr{O}_X)$ generated by $\mathscr{O}_X$ and $\Theta_X$. For any point of $X$ we can take its affine open neighborhood $U$ and a local coordinate system $\{x_i,\partial_i\}$ on it satisfying
\[x_i\in\mathscr{O}_X(U),\quad \Theta_U=\bigoplus_{i=1}^{n}\mathscr{O}_U\partial_i,\quad [\partial_i,\partial_j]=0,\quad [\partial_i,x_j]=\delta_{ij}.\]
Hence we have 
\[\mathscr{D}_U=\mathscr{D}_X|_U=\bigoplus_{\alpha\in\N^n}\mathscr{O}_U\partial^\alpha.\]
The ring $\mathscr{D}_X$ is generated by $\mathscr{O}_X$ and $\Theta_X$, and their fundamental relations are the following:
\begin{itemize}
    \item $\mathscr{O}_X\to\mathscr{D}_X$ is a ring homomorphism.
    \item $\Theta_X\to\mathscr{D}_X$ is left $\mathscr{O}_X$-linear.
    \item $\Theta_X\to\mathscr{D}_X$ is a Lie algebra homomorphism.
    \item $[v,f]=v(f)$ for $v\in\Theta_X$ and $f\in\mathscr{O}_X$.
\end{itemize}
where we denote by $v(f)$ the element of $\mathscr{O}_X$ obtained by differentiating $f$ with respect to $v$. To be more precise, we formulate the above as follows:

\begin{proposition}\label{D-module ring D_X generating relation}
Let $\mathscr{A}$ be a sheaf of rings on $X$, and $\iota:\mathscr{O}_X\to\mathscr{A}$, $\varphi:\Theta_X\to\mathscr{A}$ be sheaf morphisms such that
\begin{enumerate}
    \item[(a)] $\iota:\mathscr{O}_X\to\mathscr{A}$ is a ring homomorphism;
    \item[(b)] $\varphi:\Theta_X\to\mathscr{A}$ is left $\mathscr{O}_X$-linear;
    \item[(c)] $\varphi:\Theta_X\to\mathscr{A}$ is a Lie algebra homomorphism;
    \item[(d)] $[\varphi(v),\iota(f)]=\iota(v(f))$ for $v\in\Theta_X$, $f\in\mathscr{O}_X$.
\end{enumerate}
Then there exists a unique ring homomorpism $\Phi:\mathscr{D}_X\to\mathscr{A}$ such that the following diagram is commutative:
\[\begin{tikzcd}
\mathscr{O}_X\ar[d]\ar[rd,"\iota"]\\
\mathscr{D}_X\ar[r,"\Phi"]&\mathscr{A}\\
\Theta_X\ar[u]\ar[ru,swap,"\varphi"]
\end{tikzcd}\]
\end{proposition}
\begin{proof}
Since the question is local, we can choose local coordinate system, so by the assumptions on $\iota$ and $\varphi$, we have
\[\Phi(\sum c_\alpha\partial^\alpha)=\sum_\alpha \iota(c_\alpha)\varphi(\partial_1^{\alpha_1})\cdots\varphi(\partial_n^{\alpha_n}).\]
Conversely, if we define $\Phi$ as above, then it is easy to see that $\Phi$ is a ring homomorphism.
\end{proof}

The ring $\mathscr{D}_X$ is non-commutative, so the study of its structure is more complicated than algebraic geometry (the study of commutative rings). However, we can derive objects in the theory of commutative algrbra from $\mathscr{D}_X$ as follows.\par
Let $U$ be an affine open subset of $X$ with local coordinates $\{x_i,\partial_i\}$. For a differential operator $P=\sum a_\alpha\partial^\alpha$, the \textbf{total symbol} of $P$ is defined to be
\[\sigma(P)(x,\xi):=\sum a_\alpha(x)\xi^\alpha.\]
This is a function in $(x,\xi)=(x_1,\dots,x_n,\xi_1,\dots,\xi_n)$, which depends on the choice of a coordinate system. Let $Q=\sum b_\alpha \partial^\alpha$ be another differential operator and $\sigma(Q)(x,\xi)=\sum b_\alpha(x)\xi^\alpha$ be its total symbol, then the total symbol of $PQ$ is given by the formula
\begin{equation}\label{D-module total symbol of composition Leibniz rule}
\sigma(PQ)(x,\xi)=\sum_\alpha\frac{1}{\alpha!}\partial^\alpha_\xi\sigma(P)(x,\xi)\cdot\partial^\alpha_x\sigma(Q)(x,\xi).
\end{equation}
Since $\sigma(P)(x,\xi)$ is a polynomial in $\xi$, we have $\partial^\alpha_\xi\sigma(P)(x,\xi)=0$ except for finitely many $\alpha$, and thus the right hand side of (\ref{D-module total symbol of composition Leibniz rule}) is a finite sum over $\alpha$, and it is easily deduced from Leibniz's rule
\begin{equation}\label{D-module higher differential Leibniz rule}
\partial^\alpha(fg)=\sum_\beta\binom{\alpha}{\beta}(\partial^\beta f)(\partial^{\alpha-\beta}g).
\end{equation}
We now define an order filtration $F_p(\mathscr{D}_X)$ of $\mathscr{D}_X$ by
\[F_p(\mathscr{D}_X)=\{P\in\mathscr{D}_X:\text{$P=\sum_{|\alpha|\leq p}a_\alpha \partial^\alpha$}\}.\]
Although this definition uses a coordinate system, the following proposition shows that it does not depend on the choice of this coordinate system.
\begin{proposition}\label{D-module ring D_X filtration prop}
We have $F_p(\mathscr{D}_X)=0$ for $p<0$, and for $p\geq 0$, 
\[F_p(\mathscr{D}_X)=\{P\in\mathscr{D}_X:[P,\mathscr{O}_X]\sub F_{p-1}(\mathscr{D}_X)\}.\]
Moreover, $F_0(\mathscr{D}_X)=\mathscr{O}_X$ and $F_1\mathscr{D}_X=\mathscr{O}_X\oplus\Theta_X$.
\end{proposition}
\begin{proof}
If locally we write $P=\sum_\alpha a_\alpha\partial^\alpha$ and $f\in\mathscr{O}_X$, then the operator $[P,f]$ is given by
\begin{align*}
[P,f](g)&=P(fg)-fP(g)=\sum_\alpha\Big[a_\alpha\sum_\beta\binom{\alpha}{\beta}(\partial^\beta f)(\partial^{\alpha-\beta}g)-fa_\alpha(\partial^\alpha g)\Big]\\
&=\sum_{\alpha}a_\alpha\sum_{0\prec \beta\preceq\alpha}\binom{\alpha}{\beta}(\partial^\beta f)(\partial^{\alpha-\beta}g),
\end{align*}
which shows that $[P,\mathscr{O}_X]$ belongs to $F_{p-1}\mathscr{D}_X$ if and only if $P\in F_p(\mathscr{D}_X)$.
\end{proof} 

\begin{proposition}\label{D-module ring D_X filtration commutator prop}
The collection $\{F_p(\mathscr{D}_X)\}$ is an exhaustive increasing filtration of $\mathscr{D}_X$ and each $F_p(\mathscr{D}_X)$ is a locally free module over $\mathscr{O}_X$. Moreover, we have
\[(F_p(\mathscr{D}_X))(F_q(\mathscr{D}_X))=F_{p+q}(\mathscr{D}_X),\quad [F_p(\mathscr{D}_X),F_q(\mathscr{D}_X)]\sub F_{p+q-1}\mathscr{D}_X.\]
\end{proposition}
\begin{proof}
The first assertion is clear, and the second one can be checked locally. In fact, let $P\in F_p(\mathscr{D}_X)$ and $Q\in F_q(\mathscr{D}_X)$ be two differential operators and write $\sigma_j(W)=\sum_{|\alpha|=j}c_\alpha(x)\xi^\alpha$ for a differential operator $W$. We then have
\begin{equation}\label{D-module ring D_X filtration commutator prop-1}
\begin{aligned}
\sigma(PQ)=&\sigma_p(P)\sigma_q(Q)+\Big(\sigma_{p-1}(P)\sigma_q(Q)+\sigma_p(P)\sigma_{q-1}(Q)+\sum_i\frac{\partial \sigma_p(P)}{\partial\xi_i}\frac{\partial \sigma_q(Q)}{\partial x_i}\Big)\\
&+\text{terms of degree less than $p+q-1$ in $\xi$},
\end{aligned}
\end{equation}
whence the assertions.
\end{proof}

Let us now consider the graded ring
\[\gr(\mathscr{D}_X)=\bigoplus_{p=0}^{\infty}\gr_p(\mathscr{D}_X).\]
By \cref{D-module ring D_X filtration commutator prop}, this is a sheaf of commutative algebras of finite type over $\mathscr{O}_X$. Since $\gr_0(\mathscr{D}_X)=\mathscr{O}_X$ and $\gr_1(\mathscr{D}_X)=\Theta_X$, we obtain an $\mathscr{O}_X$-algebra homomorphism
\begin{equation}\label{D-module ring D_X graded ring isomorphism}
\bm{S}_{\mathscr{O}_X}(\Theta_X)\to\gr(\mathscr{D}_X)
\end{equation}
Take an affine chart $U$ with a coordinate system $\{x_i,\partial_i\}$ and set $\xi_i:=\partial_i$ mod $F_0\mathscr{D}_U=\mathscr{O}_U$, we then have
\[\gr(\mathscr{D}_U)=\mathscr{O}_U[\xi_1,\dots,\xi_n],\]
so the homomorphism (\ref{D-module ring D_X graded ring isomorphism}), which is given by $\xi^\alpha\mapsto\partial^\alpha$, is an isomorphism. We denote by $\sigma_p$ the homomorphism defined by
\[\sigma_p:F_p(\mathscr{D}_X)\to \gr_p(\mathscr{D}_X)\sub \gr(\mathscr{D}_X)\cong \bm{S}_{\mathscr{O}_X}(\Theta_X).\]
By using local coordinates, we have
\[\sigma_p(P)=\sum_{|\alpha|=p}a_\alpha(x)\xi^\alpha\in \bm{S}_{\mathscr{O}_X}^p(\Theta_X)\]
for $P=\sum_\alpha a_\alpha\partial^\alpha$. The corresponding section $\sigma_p(P)$ is called the \textbf{principal symbol} of $P$.

\begin{proposition}\label{D-module section ring of D_X is Noe}
Assume that $A=\mathscr{D}_X(U)$ for some affine open subset $U$ of $X$ or $A=\mathscr{D}_{X,x}$ for some $x\in X$. Then $A$ is a left (and right) Noetherian ring.
\end{proposition}
\begin{proof}
We have seen that $A$ admits a Hausdorff and complete filtration so that the associated graded ring is (commutative and) Noetherian. The assertion thus follows from \cref{filtration complete ring gr(A) Noe imply A Noe}.
\end{proof}

We have thus succeeded to derive an element of the commutative algebra $\bm{S}_{\mathscr{O}_X}(\Theta_X)$, namely the principal symbol, from a differential operator. Although the principal symbol is only a part of a differential operator (indeed, it is the part of the highest degree of the total symbol), it carries a great deal of information on $\mathscr{D}_X$, as seen in the following.\par

Let $\pi:T^*X\to X$ denote the cotangent bundle of $X$. Then we may regard $\xi_1,\dots,\xi_n$ as the coordinate system of the cotangent space $\bigoplus_{i=1}^{n}\C dx_i$, and hence $\bm{S}_{\mathscr{O}_X}(\Theta_X)$ is canonically identified with the sheaf $\pi_*(\mathscr{O}_{T^*X})$. A section of $\bm{S}_{\mathscr{O}_X}(\Theta_X)$ can therefore be regarded as a function on $T^*X$. If $P\in D_p\mathscr{D}_X$ and $Q\in D_q\mathscr{D}_X$, then $[P,Q]\in D_{p+q-1}\mathscr{D}_X$, so the commutator induces a multiplication map on $\gr(\mathscr{D}_X)$. This can be explicitly calculated by equation (\ref{D-module ring D_X filtration commutator prop-1}) in local coordinates:
\begin{equation}\label{D-module ring D_X graded ring multiplication-1}
\sigma_{p+q-1}([P,Q])=\sum_i\Big(\frac{\partial \sigma_p(P)}{\partial\xi_i}\frac{\partial\sigma_q(Q)}{\partial x_i}-\frac{\partial \sigma_q(Q)}{\partial\xi_i}\frac{\partial\sigma_p(P)}{\partial x_i}\Big).
\end{equation}
Now for two function $f,g$ in $x$ and $\xi$, the \textbf{Possion bracket} of $f$ and $g$ is defined by
\begin{equation}\label{D-module ring D_X graded ring multiplication-2}
\{f,g\}=\sum_i\Big(\frac{\partial f}{\partial\xi_i}\frac{\partial g}{\partial x_i}-\frac{\partial g}{\partial\xi_i}\frac{\partial f}{\partial x_i}\Big).
\end{equation}
With the notation, we can then write (\ref{D-module ring D_X graded ring multiplication-1}) into the following form:
\begin{equation}\label{D-module ring D_X graded ring multiplication-3}
\sigma_{p+q-1}([P,Q])=\{\sigma_p(P),\sigma_q(Q)\}.
\end{equation}
Considering $f$ and $g$ as functions on $T^*X$, we see that $\{\cdot,\cdot\}$ is independent of the choice of a local coordinate system as below. We have a canonical $1$-form $\omega_X$ on $T^*X$. For every point $p\in T^*X$, a $1$-form $\omega_p$ at a point $\pi(p)$ of $X$ is determined by the definition of $T^*X$. The canonical $1$-form $\omega_X$ is then defined by $\omega_X(p)=\pi^*\omega_p$. In local coordinates, we have
\[\omega_X=\sum_i\xi_idx_i.\]
At each point $p$, the $2$-form $\theta_X=d\omega_X$ gives an anti-symmetric bilinear form on $T_p(T^*X)$, which is nondegenerate and induces an isomorphism $H:T_p^*(T^*X)\cong T_p(T^*)X$ by
\[\theta_X(v,H(\eta))=\langle\eta,v\rangle,\quad \eta\in T_p^*(T^*X),v\in T_p(T^*X).\]
Explicitly in local coordinates, the isomorphism $H$ is given by
\begin{equation}\label{D-module Hamiltonian on T^*X formula}
H:T_p^*(T^*X)\cong T_p(T^*)X,\quad d\xi_i\mapsto\partial/\partial_i,dx_i\mapsto -\partial/\partial\xi_i.
\end{equation}
In particular, $H_f=H(df)$ is a vector field on $T^*X$ for any function $f$ on $T^*X$; this is called the \textbf{Hamiltonian} of $f$.

\begin{definition}
For functions $f,g$ on $T^*X$, the \textbf{Poisson bracket} of $f$ and $g$ is defined to be $\{f,g\}=H_f(g)$.
\end{definition}

By (\ref{D-module Hamiltonian on T^*X formula}), the Possion bracket $\{\cdot,\cdot\}$ is expressed as (\ref{D-module ring D_X graded ring multiplication-2}) in local coordinates so it is determined by $(T^*X,\theta_X)$. A pair $(M,\theta)$ of a manifold $M$ and a closed $2$-form $\theta$ on $M$ is called a \textbf{symplectic manifold} if $\theta$ is a nondegenerate anti-symmetric bilinear form on $T_pM$ for every $p$. For such a manifold, we can define a Possion bracket in the same way as above, and this is a notion depending on the symplectic structure of $M$. By tracing back the above arguments, we can determine a $2$-form from the Poisson bracket.\par
In the formula (\ref{D-module ring D_X graded ring multiplication-3}), the commutator of $\mathscr{D}_X$ expresses the noncommutatitity of $\mathscr{D}_X$. Hence, symbolically speaking, the noncommutatitity of $\mathscr{D}_X$ determines a symplectic structure of $T^*X$.

\subsection{\texorpdfstring{$D$}{D}-modules}
Let $X$ be a smooth algebraic variety. We say that a sheaf $\mathscr{M}$ on $X$ is a left $\mathscr{D}_X$-module if $\Gamma(U,\mathscr{M})$ is endowed with a left $\Gamma(U,\mathscr{D}_X)$-module structure for each open subset $U$ of $X$ and these actions are compatible with restriction morphisms. Note that $\mathscr{O}_X$ is a left $\mathscr{D}_X$-module via the canonical action of $\mathscr{D}_X$. We have the following very easy (but useful) interpretation of the notion of left $\mathscr{D}_X$-modules.
\begin{lemma}\label{D-module iff flat connection}
Let $\mathscr{M}$ be an $\mathscr{O}_X$-module. Giving a left $\mathscr{D}_X$-module structure on $\mathscr{M}$ extending the $\mathscr{O}_X$-module structure is equivalent to giving a $\C$-linear morphism
\[\nabla:\Theta_X\to\sEnd_\C(\mathscr{M}),\quad v\mapsto\nabla_v\]
satisfying the following conditions:
\begin{enumerate}
    \item[(a)] $\nabla_{fv}(s)=f\nabla_v(s)$ for $f\in\mathscr{O}_X$, $v\in\Theta_X$, $s\in M$;
    \item[(b)] $\nabla_v(fs)=v(f)s+f\nabla_v(s)$ for $f\in\mathscr{O}_X$, $v\in\Theta_X$, $s\in M$;
    \item[(c)] $\nabla_{[v,w]}(s)=[\nabla_v,\nabla_w](s)$ for $v,w\in\Theta_X$, $s\in M$.
\end{enumerate}
In terms of $\nabla$ the left $\mathscr{D}_X$-module structure on $\mathscr{M}$ is given by $v\cdot s=\nabla_v(s)$ for $v\in\Theta_X$, $s\in\mathscr{M}$.
\end{lemma}
\begin{proof}
The proof is immediate, because $\mathscr{D}_X$ is generated by $\mathscr{O}_X$, $\Theta_X$ and satisfies the relation $[v,f]=v(f)$.
\end{proof}

For a locally free left $\mathscr{O}_X$-module $\mathscr{M}$ of finite rank, a $\C$-linear morphism $\nabla:\Theta_X\to\sEnd_\C(\mathscr{M})$ satisfying the conditions (a), (b) is usually called a \textbf{connection} (of the corresponding vector bundle). If it also satisfies the condition (c), it is called an \textbf{integrable} (or \textbf{flat}) \textbf{connection}. Hence we may regard a (left) $\mathscr{D}_X$-module as an integrable connection of an $\mathscr{O}_X$-module which is not necessarily locally free of finite rank. We say that a $\mathscr{D}_X$-module $\mathscr{M}$ is an \textbf{integrable connection} if it is locally free of finite rank over $\mathscr{O}_X$, and we denote by $\Conn(X)$ the category of integrable connections on $X$. Integrable connections are the most elementary left $D$-modules. Nevertheless, they are especially important because they generate (in a categorical sense) the category of holonomic systems, as we see later.

\begin{example}[\textbf{Ordinary Differential Equations}]
Consider an ordinary differential operator
\[P=a_n(x)\partial^n+\cdots+a_0(x),\quad \partial=d/dx,a_i\in\mathscr{O}_\C\]
on $\C$ and the corrresponding $\mathscr{D}_\C$-module $\mathscr{M}=\mathscr{D}_\C/\mathscr{D}_\C P=\mathscr{D}_\C u$, where $u\equiv 1$ mod $\mathscr{D}_\C P$, and hence $Pu=0$. Then on $U=\{x\in\C:a_n(x)\neq 0\}$ we have $\mathscr{M}|_U\cong\bigoplus_{i=0}^{n-1}\mathscr{O}_Uu_i$ (where $u_i=\partial^iu$), so $\mathscr{M}$ is an integrable connection of rank $n$ on $U$. 
\end{example}
\paragraph{Differential homomorphisms}
Let $X$ be a smooth algebraic variety and $\mathscr{M},\mathscr{N}$ be $\mathscr{O}_X$-modules. A $\C$-linear sheaf homomorphism $\varphi:\mathscr{M}\to\mathscr{N}$ is called a \textbf{differential homomorphism} if for every $s\in\mathscr{M}$ there exists finitely many $P_i\in\mathscr{D}_X$ and $v_i\in\mathscr{N}$ such that
\[\varphi(fs)=\sum_iP_i(f)v_i\]
for any $f\in\mathscr{O}_X$. In other words, $\varphi$ is a differential homomorphism if it can be generated by differential operators. Let $\sDiff(\mathscr{M},\mathscr{N})$ be the the sheaf of differential homomorphisms from $\mathscr{M}$ to $\mathscr{N}$. From our definition, it is clear that $\mathscr{D}_X=\sDiff(\mathscr{O}_X,\mathscr{O}_X)$.

\begin{lemma}\label{D-module differential homomorphism composition}
Let $\mathscr{M},\mathscr{N},\mathscr{H}$ be $\mathscr{O}_X$-modules and $\varphi:\mathscr{M}\to\mathscr{N}$, $\psi:\mathscr{N}\to\mathscr{H}$ be differential homomorphisms. Then $\psi\circ\varphi:\mathscr{M}\to\mathscr{N}$ is also a differential homomorphism.
\end{lemma}
\begin{proof}
By definition, for every $s\in\mathscr{M}$, there exists $P_i,Q_j\in\mathscr{D}_X$ and $v_i\in\mathscr{N}$, $w_{ij}\in\mathscr{H}$ such that
\[\varphi(fs)=\sum_iP_i(f)v_i,\quad \psi(gv_i)=\sum_jQ_j(g)w_{ij}.\]
It then follows that
\[(\psi\circ\varphi)(fs)=\psi\Big(\sum_iP_i(f)v_i\Big)=\sum_ijQ_i(P_i(f))w_{ij}\]
so $\psi\circ\varphi$ is also a differential homomorphism.
\end{proof}

Let us consider the right $\mathscr{D}_X$-module $\mathscr{N}\otimes_{\mathscr{O}_X}\mathscr{D}_X$ for an $\mathscr{O}_X$-module $\mathscr{N}$. The right $\mathscr{D}_X$-module structure gives $\mathscr{N}\otimes_{\mathscr{O}_X}\mathscr{D}_X$ an $\mathscr{O}_X$-module structure. By tensoring $\mathscr{N}$ with the left $\mathscr{O}_X$-linear homomorphism\footnote{This is in fact the projection from $\mathscr{D}_X$ to $\mathscr{O}_X$, which is the identity on $\mathscr{O}_X$ and zero on $\Theta_X$.}
\[\mathscr{D}_X\to\mathscr{O}_X,\quad P\mapsto P(1)\in\mathscr{O}_X,\]
we obtain a $\C$-linear homomorphism (which is not $\mathscr{O}_X$-linear)
\[P_\mathscr{N}:\mathscr{N}\otimes_{\mathscr{O}_X}\mathscr{D}_X\to\mathscr{N}\]
which induces a map
\[\sHom_{\mathscr{O}_X}(\mathscr{M},\mathscr{N}\otimes_{\mathscr{O}_X}\mathscr{D}_X)\to \sHom_{\C}(\mathscr{M},\mathscr{N}),\quad \varphi\mapsto P_\mathscr{N}\circ\varphi.\]
If $\varphi(s)=\sum_iv_i\otimes P_i$ for $s\in\mathscr{M}$, then we have
\[P_\mathscr{N}\circ\varphi(fs)=\sum_iP_i(f)v_i.\]
Therefore, we see that $P_\mathscr{N}\circ\varphi$ is a differential homomorphism.

\begin{proposition}\label{D-module differential module char}
The homomorphism $\sHom_{\mathscr{O}_X}(\mathscr{M},\mathscr{N}\otimes_{\mathscr{O}_X}\mathscr{D}_X)\to\sDiff(\mathscr{M},\mathscr{N})$ is an isomorphism.
\end{proposition}
\begin{proof}
We first prove the proposition for $\mathscr{M}=\mathscr{O}_X$. By taking a coordinate system $\{x_i,\partial_i\}$, we have $\mathscr{N}\otimes_{\mathscr{O}_X}\mathscr{D}_X\cong\bigoplus_\alpha\mathscr{N}\otimes\partial^\alpha$, and
\[\sHom_{\mathscr{O}_X}(\mathscr{O}_X,\mathscr{N}\otimes_{\mathscr{O}_X}\mathscr{D}_X)\cong \bigoplus_\alpha\mathscr{N}\otimes\partial^\alpha.\]
Let $\mathscr{K}$ be the kernel of the homomorphism in question, and put
\[F_p\mathscr{K}=\{\sum v_\alpha\otimes\partial^\alpha\in\mathscr{K}:\text{$v_\alpha=0$ for $|\alpha|>p$}\}.\]
Then we have $\mathscr{K}=\bigcup_pF_p\mathscr{K}$, so it suffices to prove inductively that $F_p\mathscr{K}=0$ for each $p\geq 0$. Let $\sum v_\alpha\otimes\partial^\alpha\in F_p\mathscr{K}$; then we have $\sum(\partial^\alpha f)v_\alpha=0$ for any $f\in\mathscr{O}_X$. First we see that $F_0\mathscr{K}=0$ for $f=1$, so assume that $p>0$ and $F_{p-1}\mathscr{K}=0$. For each fixed integer $i$, by replacing $f$ with $x_if$, we have
\begin{align*}
0=\sum \partial^\alpha(x_if)v_\alpha-\sum x_i(\partial^\alpha f)v_\alpha=\sum_{\alpha_i>0}(\partial^{\alpha-\delta_i}f)v_\alpha,
\end{align*}
where $\delta_i$ is the $i$-th unit vector $(0,\dots,1,\dots,0)$. The induction hypothesis leads to $v_\alpha=0$ for $\alpha_i>0$. Fianlly, we also see that $v_0=0$, since $v_0\otimes\partial^0\in F_0\mathscr{K}=0$.\par
On the other hand, if $\varphi\in\sDiff(\mathscr{O}_X,\mathscr{N})$ and $\varphi(f)=\sum_iP_i(f)v_i$, then $\varphi=P_\mathscr{N}\circ\tilde{\varphi}$ with $\tilde{\varphi}(1)=\sum v_i\otimes P_i$.\par
We have thus proved the proposition in the case when $\mathscr{M}=\mathscr{O}_X$. For the general case, let $\varphi\in\sDiff(\mathscr{M},\mathscr{N})$; since the map $\mathscr{O}_X\ni f\mapsto\varphi(fs)$ belongs to $\sDiff(\mathscr{O}_X,\mathscr{N})$ for every $s\in\mathscr{M}$, there exists a unique homomorphism $\tilde{\varphi}(s)\in\mathscr{N}\otimes_{\mathscr{O}_X}\mathscr{D}_X$ satisfying
\[P_\mathscr{N}(\tilde{\varphi}(s)f)=\varphi(fs),\quad f\in\mathscr{O}_X.\]
It is immediate that $\tilde{\varphi}\in\sHom(\mathscr{M},\mathscr{N}\otimes_{\mathscr{O}_X}\mathscr{D}_X)$ is $\mathscr{O}_X$-linear. If $\psi\in\sHom(\mathscr{M},\mathscr{N}\otimes_{\mathscr{O}_X}\mathscr{D}_X)$ satisfies $P_\mathscr{N}\circ\psi=0$, then for every $s\in\mathscr{M}$ we have $P_\mathscr{N}\psi(fs)=0$ for $f\in\mathscr{O}_X$, and therefore $\psi(s)=0$.
\end{proof}

\begin{corollary}\label{D-module differential module char by Hom of tensor}
There is a canonical isomorphism
\[\sHom_{\mathscr{D}_X^{\op}}(\mathscr{M}\otimes_{\mathscr{O}_X}\mathscr{D}_X,\mathscr{N}\otimes_{\mathscr{O}_X}\mathscr{D}_X)\stackrel{\sim}{\to} \sDiff(\mathscr{M},\mathscr{N}).\]
\end{corollary}
\begin{proof}
It suffices to note that
\begin{equation*}
\sHom_{\mathscr{D}_X^{\op}}(\mathscr{M}\otimes_{\mathscr{O}_X}\mathscr{D}_X,\mathscr{N}\otimes_{\mathscr{O}_X}\mathscr{D}_X)\cong \sHom_{\mathscr{O}_X}(\mathscr{M},\mathscr{N}\otimes_{\mathscr{O}_X}\mathscr{D}_X).\qedhere
\end{equation*}
\end{proof}

\begin{remark}\label{D-module differential module char inverse map}
The inverse map of \cref{D-module differential module char} is given as follows. Let $\varphi\in\sDiff(\mathscr{M},\mathscr{N})$ be a differential homomorphism. Then for each $s\in\mathscr{M}$, there exists $P_i\in\mathscr{D}_X$ and $v_i\in\mathscr{N}$ such that
\begin{equation}\label{D-module differential module char inverse map-1}
\varphi(fs)=\sum_iP_i(f)v_i\for f\in\mathscr{O}_X.
\end{equation}
The inverse image $\tilde{\varphi}\in\sHom_{\mathscr{O}_X}(\mathscr{M},\mathscr{N}\otimes_{\mathscr{O}_X}\mathscr{D}_X)$ is then defined to be
\[\tilde{\varphi}(s)=\sum_iv_i\otimes P_i\in\mathscr{N}\otimes\mathscr{D}_X.\]
In fact, it is easy to verify that $P_\mathscr{N}\circ\tilde{\varphi}=\varphi$ from equation (\ref{D-module differential module char inverse map-1}).
\end{remark}

\begin{proposition}
Let $\mathscr{M}$, $\mathscr{N}$ and $\mathscr{H}$ be $\mathscr{O}_X$-modules. Then the diagram
\begin{small}
\[\begin{tikzcd}[column sep=4mm]
\sHom_{\mathscr{D}_X^{\op}}(\mathscr{M}\otimes_{\mathscr{O}_X}\mathscr{D}_X,\mathscr{N}\otimes_{\mathscr{O}_X}\mathscr{D}_X)\otimes_{\C}\sHom_{\mathscr{D}_X^{\op}}(\mathscr{M}\otimes_{\mathscr{O}_X}\mathscr{D}_X,\mathscr{N}\otimes_{\mathscr{O}_X}\mathscr{D}_X)\ar[r]\ar[d]&\sDiff(\mathscr{M},\mathscr{N})\otimes_{\C}\sDiff(\mathscr{N},\mathscr{H})\ar[d]\\
\sHom_{\mathscr{D}_X^{\op}}(\mathscr{M}\otimes_{\mathscr{O}_X}\mathscr{D}_X,\mathscr{H}\otimes_{\mathscr{O}_X}\mathscr{D}_X)\ar[r]&\sDiff(\mathscr{M},\mathscr{H})
\end{tikzcd}\]
is commutative, where the vertical arrows are the homomorphisms obtained by composition.
\end{small}
\end{proposition}
\begin{proof}
This follows from \cref{D-module differential module char}, since the isomorphsim in it is compatible with compositions.
\end{proof}

Let us apply the above argument to the de Rham complex
\[\begin{tikzcd}
\cdots\ar[r]&0\ar[r]&\Omega_X^0\ar[r]&\Omega_X^1\ar[r]&\cdots\ar[r]&\Omega_X^n\ar[r]&0\ar[r]&\cdots
\end{tikzcd}\]
Take a coordinate system $\{x_i,\partial_i\}$, we then have
\[d(f\omega)=df\wedge\omega+fd\omega=\sum_i\frac{\partial f}{\partial x_i}dx_i\wedge\omega+fd\omega,\]
so the exterior differential $d$ is a differential homomorphism, and \cref{D-module differential module char by Hom of tensor} gives a complex of right $\mathscr{D}_X$-modules
\begin{equation}\label{D-module de Rham complex of D_X-1}
\begin{tikzcd}
\cdots\ar[r]&0\ar[r]&\Omega_X^0\otimes_{\mathscr{O}_X}\mathscr{D}_X\ar[r]&\cdots\ar[r]&\Omega_X^n\otimes_{\mathscr{O}_X}\mathscr{D}_X\ar[r]&0\ar[r]&\cdots
\end{tikzcd}
\end{equation}
By definition, the differential in this complex is given by
\begin{equation}\label{D-module de Rham complex of D_X-2}
d(\omega\otimes P)=\sum_{i=1}^{n}dx_i\wedge \omega\otimes\partial_iP+d\omega\otimes P
\end{equation}  
where $\omega\in\Omega_X^\bullet$, $P\in\mathscr{D}_X$. In particular, choosing $\omega=1\in\Omega_X^0$, we obtain
\begin{gather}
dP=\sum_{i=1}^{n}dx_i\otimes\partial_iP,\label{D-module de Rham complex of D_X-3}\\
d(\omega\wedge\eta)=d\omega\wedge\eta+(-1)^p\omega\wedge d\eta,\label{D-module de Rham complex of D_X-4}
\end{gather}
where $\omega\in\Omega_X^p$ and $\eta\in\Omega_X^q\otimes_{\mathscr{O}_X}\mathscr{D}_X$.\par 

Conversely, these two formule characterize the differential $d$ of $\Omega_X^\bullet\otimes_{\mathscr{O}_X}\mathscr{D}_X$. For every left $\mathscr{D}_X$-module $\mathscr{M}$, we then obtain a complex
\[\begin{tikzcd}
\cdots\ar[r]&0\ar[r]&\Omega_X^0\otimes_{\mathscr{O}_X}\mathscr{M}\ar[r]&\cdots\ar[r]&\Omega_X^n\otimes_{\mathscr{O}_X}\mathscr{M}\ar[r]&0\ar[r]&\cdots
\end{tikzcd}\]
by applying the functor $-\otimes_{\mathscr{D}_X}\mathscr{M}$ to (\ref{D-module de Rham complex of D_X-1}). In other words, we have
\begin{gather}
du=\sum_{i=1}^{n}dx_i\otimes\partial_iu,\label{D-module de Rham complex of module-1}\\
d(\omega\wedge\eta)=d\omega\wedge\eta+(-1)^p\omega\wedge d\eta,\label{D-module de Rham complex ofmodule-2}
\end{gather}
where $u\in\mathscr{M}$, $\omega\in\Omega_X^p$, $\eta\in\Omega_X^q\otimes\mathscr{M}$. The complex $(\Omega_X^\bullet\otimes_{\mathscr{O}_X}\mathscr{M},d)$ is called the \textbf{de Rham complex} of $\mathscr{M}$ and denoted by $DR_X(\mathscr{M})$.\par

By applying the functor
\[\sHom_{\mathscr{D}_X^{\op}}(-,\mathscr{D}_X):\Mod(\mathscr{D}_X^{\op})\to\Mod(\mathscr{D}_X)^{\op}\]
to (\ref{D-module de Rham complex of D_X-1}), we obtain a complex of left $\mathscr{D}_X$-modules
\begin{equation}\label{D-module Spencer resolution of O_X}
\begin{tikzcd}
\cdots\ar[r]&0\ar[r]&\mathscr{D}_X\otimes_{\mathscr{O}_X}\bigw^n\Theta_X\ar[r]&\cdots\ar[r]&\mathscr{D}_X\otimes_{\mathscr{O}_X}\bigw^0\Theta_X\ar[r]&0\ar[r]&\cdots
\end{tikzcd}
\end{equation}
since we have 
\[\sHom_{\mathscr{D}_X^{\op}}(\Omega_X^p\otimes_{\mathscr{O}_X}\mathscr{D}_X,\mathscr{D}_X)\cong\sHom_{\mathscr{O}_X}(\Omega_X^p,\mathscr{D}_X)\cong\mathscr{D}_X\otimes_{\mathscr{O}_X}\bigw^p\Theta_X.\]
Explicitly, the differential $\delta$ is given by
\begin{align*}
\delta(P\otimes v_1\wedge\cdots\wedge v_p)&=\sum_i(-1)^{i-1}Pv_i\otimes v_1\wedge\cdots\wedge\hat{v}_i\wedge\cdots\wedge v_p\\
&+\sum_{i<j}(-1)^{i+j}P\otimes[v_i,v_j]\wedge v_1\wedge\cdots\wedge\hat{v}_i\wedge\cdots\wedge\hat{v}_j\wedge\cdots\wedge v_p.
\end{align*}

\begin{proposition}\label{D-module resolution of Omega_X and O_X}
We have the following locally free resolutions of the left $\mathscr{D}_X$-module $\mathscr{O}_X$ and the right $\mathscr{D}_X$-module $\Omega_X$.
\begin{equation}
\begin{tikzcd}
0\ar[r]&\Omega_X^0\otimes_{\mathscr{O}_X}\mathscr{D}_X\ar[r]&\cdots\ar[r]&\Omega_X^n\otimes_{\mathscr{O}_X}\mathscr{D}_X\ar[r]&\Omega_X\ar[r]&0
\end{tikzcd}
\end{equation}
\vspace*{-4mm}
\begin{equation}
\begin{tikzcd}
0\ar[r]&\mathscr{D}_X\otimes_{\mathscr{O}_X}\bigw^n\Theta_X\ar[r]&\cdots\ar[r]&\mathscr{D}_X\otimes_{\mathscr{O}_X}\bigw^0\Theta_X\ar[r]&\mathscr{O}_X\ar[r]&0
\end{tikzcd}
\end{equation}
\end{proposition}
\begin{proof}
The assertion for $\Omega_X$ follows from the one for $\mathscr{O}_X$ using the side-changing operation. Later we will show that the complex (\ref{D-module de Rham complex of D_X-1}) is a resolution of $\mathscr{O}_X$ in a general setting (cf. \cref{D-module relative D-module resolution}).
\end{proof}

\paragraph{Correspondence between left and right \texorpdfstring{$D$}{D}-modules}
Take a local coordinate system $\{x_i,\partial_i\}$ on an affine open subset $U$ of $X$. For $P=\sum_\alpha a_\alpha\partial^\alpha\in\mathscr{D}_U$, consider its formal adjoint
\[P^*:=\sum_\alpha(-1)^{|\alpha|}\partial^\alpha a_\alpha(x)\in\mathscr{D}_U.\]
Then we have $(PQ)^*=Q^*P^*$, and an anti-automorphism $P\mapsto P^*$ of $\mathscr{D}_U$. Therefore, for a left $\mathscr{D}_U$-module $\mathscr{M}$ we can define a right action of $\mathscr{D}_U$ on $\mathscr{M}$ by $sP:=P^*s$ for $s\in\mathscr{M}$, and obtain a right $\mathscr{D}_U$-module $\mathscr{M}^*$. However, this notion depends on the choice of a local coordinate. In order to globalize this correspondence to arbitrary smooth algebraic variety $X$ we need to use the canonical sheaf $\Omega_X:=\Omega_X^n$, where $n=\dim(X)$, since the formal adjoint of a differential operator naturally acts on $\Omega_X$.\par
Recall that there are two natural actions of $\Theta$ on the sheaf $\Omega_X^\bullet=\bigoplus_i\Omega_X^i$. For $v\in\Theta_X$, its \textbf{inner derivation} $i_v\in\sEnd_\C(\Omega_X^\bullet)$ is characterized by the following properties:
\begin{enumerate}[leftmargin=40pt]
    \item[(I1)] $i_{fv}=fi_v=i_vf$ for $f\in\mathscr{O}_X$, $v\in\Theta_X$;
    \item[(I2)] $i_v(\omega\wedge\eta)=(i_v\omega)\wedge\eta+(-1)^{\deg(\omega)}\omega\wedge i_v\eta$ for $\omega,\eta\in\Omega_X^\bullet$;
    \item[(I3)] $i_v(\mathscr{O}_X)=0$;
    \item[(I4)] $i_v(\omega)=\langle v,\omega\rangle\in\mathscr{O}_X$ for $\omega\in\Omega_X^1$.
\end{enumerate}
In contrast, the Lie derivative $\mathfrak{L}_v\in\sEnd_\C(\Omega_X^\bullet)$ is characterized by the following properties:
\begin{enumerate}[leftmargin=40pt]
    \item[(L1)] $\mathfrak{L}_v(\omega\wedge\eta)=\mathfrak{L}_v\omega\wedge\eta+\omega\wedge\mathfrak{L}_v\eta$ for $\omega,\eta\in\Omega_X^\bullet$;
    \item[(L2)] $\mathfrak{L}_vf=v(f)$ for $f\in\mathscr{O}_X$,
    \item[(L3)] $d\mathfrak{L}_v=\mathfrak{L}_vd$. 
\end{enumerate}
Hence $\mathfrak{L}_V$ is an operator of degree $0$ on $\Omega_X^\bullet$, and locally it is given by
\begin{align*}
\mathfrak{L}_v(\omega)(v_1,\dots,v_n)=v(\omega(v_1,\dots,v_n))-\sum_{i=1}^{n}\omega(v_1,\dots,[v,v_i],\dots,v_n).
\end{align*}
The Lie derivatives also satisfy $[\mathfrak{L}_v,\mathfrak{L}_w]=\mathfrak{L}_{[v,w]}$ for $v,w\in\Theta_X$, and these two derivatives are related by
\[\mathfrak{L}_v=di_v+i_vd.\]

\begin{lemma}\label{D-module on Omega Lie derivative linear}
For every $f\in\mathscr{O}_X$ and $v\in\Theta_X$, we have
\begin{gather*}
\mathfrak{L}_{fv}(\omega)=\mathfrak{L}_v(f\omega)=f\mathfrak{L}_v(\omega)+v(f)\omega.
\end{gather*}
where $\omega\in\Omega_X:=\Omega_X^n$.
\end{lemma}
\begin{proof}
For $\omega\in\Omega_X^\bullet$, we have
\begin{align*}
\mathfrak{L}_{fv}(\omega)&=d(fi_v\omega)+fi_v(d\omega)=df\wedge i_v\omega+fdi_v\omega+fi_vd\omega=df\wedge i_v\omega+f\mathfrak{L}_v(\omega).
\end{align*}
Since $i_v(df\wedge\omega)=i_v(df)\wedge\omega-df\wedge i_v\omega=v(f)\omega-df\wedge i_v\omega$, we then conclude that
\[\mathfrak{L}_{fv}(\omega)=f\mathfrak{L}_v(\omega)+v(f)\omega-i_v(df\wedge\omega)=\mathfrak{L}_v(f\omega)-i_v(df\wedge\omega),\]
and this gives the desired result since $df\wedge\omega=0$ for $\omega\in\Omega_X$.
\end{proof}

\cref{D-module on Omega Lie derivative linear} shows that the map $\varphi:\Theta_X\to\sEnd_\C(\Omega_X)^{\op}$, $v\mapsto -\mathfrak{L}_v$ is left $\mathscr{O}_X$-linear, so by \cref{D-module ring D_X generating relation} we obtain an extended homomorphism $\mathscr{D}_X\to\sEnd_{\C}(\Omega_X)^{\op}$, which gives a right $\mathscr{D}_X$-module structure on $\Omega_X$. By definition, the equality
\[\omega v=-\mathfrak{L}_v(\omega)\]
holds for $v\in\Theta_X$ and $\omega\in\Omega_X$.

\begin{remark}
The action of $\Omega_X$ is related to integration by parts. Namely, $f\in\mathscr{O}_X$, $\omega\in\Omega_X$ and $P\in\mathscr{D}_X$ formally satisfy
\[\int_X(\omega P)f=\int_X\omega P(f).\]
That is, there exists a differential form $\eta$ of degree $(n-1)$ such that
\[(\omega P)f-\omega P(f)=d\eta.\]
For example, if $P=v\in\Theta_X$, then it is easy to see that
\[(\omega v)f-\omega v(f)=-\mathfrak{L}_v(\omega)f-\omega v(f)=-d(fi_v\omega).\]
\end{remark}

\begin{remark}\label{D-module Omega_X module structure expression}
In terms of an affine open $U$ and a local coordinate system $\{x_i,\partial_i\}$, we have
\begin{equation}\label{D-module Omega_X module structure expression-1}
(fdx_1\wedge\cdots\wedge dx_n)P=P^*f\,dx_1\wedge\cdots\wedge dx_n,
\end{equation}
where $f\in\mathscr{O}_X$ and $P\in\mathscr{D}_U$. To see this, it suffices to note that if $v=\sum_iv_i\partial_i$, then we have
\begin{align*}
(fdx_1\wedge\cdots\wedge dx_n)v&=(fdx_1\wedge\cdots\wedge dx_n)(\sum_iv_i\partial_i)=-\sum_iv_i\mathfrak{L}_{\partial_i}(fdx_1\wedge\cdots dx_n)\\
&=-\sum_iv_if\mathfrak{L}_{\partial_i}(x_1\wedge\cdots\wedge x_n)-\sum_iv_i\frac{\partial f}{\partial x_i}x_1\wedge\cdots\wedge x_n=v^*f\,dx_1\wedge\cdots\wedge dx_n.
\end{align*}
\end{remark}

For an invertible $\mathscr{O}_X$-module $\mathscr{L}$, we denote by $\mathscr{L}^{\otimes -1}$ the dual $\sHom_{\mathscr{O}_X}(\mathscr{L},\mathscr{O}_X)$. For $t\in\mathscr{L}^{\otimes -1}$ and $s\in\mathscr{L}$, we denote by $\langle t,s\rangle\in\mathscr{O}_X$ the image of $t\otimes s$ under the isomorphism $\mathscr{L}^{\otimes -1}\otimes_{\mathscr{O}_X}\mathscr{L}\cong\mathscr{O}_X$. Let $\mathscr{A}$ be an $\mathscr{O}_X$-algebra, that is, a sheaf of rings with a ring homomorphism $\mathscr{O}_X\to\mathscr{A}$. Then there exists a natural ring structure on $\mathscr{L}\otimes_{\mathscr{O}_X}\mathscr{A}\otimes_{\mathscr{O}_X}\mathscr{L}^{\otimes -1}$, given by
\[(s_1\otimes a_1\otimes t_1)\cdot(s_2\otimes a_2\otimes t_2)=s_1\otimes a_1\langle t_1,s_2\rangle a_2\otimes t_2\]
where $s_i\in\mathscr{L}$, $t_i\in\mathscr{L}^{\otimes -1}$, $a_i\in\mathscr{A}$. If $\mathscr{M}$ is an $\mathscr{A}$-module, then $\mathscr{L}\otimes_{\mathscr{O}_X}\mathscr{M}$ is a left $\mathscr{L}\otimes_{\mathscr{O}_X}\mathscr{A}\otimes_{\mathscr{O}_X}\mathscr{L}^{\otimes -1}$-module, whose action is given by
\[(s\otimes a\otimes t)\cdot(s'\otimes u)=s\otimes a\langle t,s'\rangle u.\]
In view of the isomorphism $\mathscr{L}^{\otimes -1}\otimes_{\mathscr{O}_X}\mathscr{L}\cong\mathscr{O}_X$, it is clear that we have the following proposition:
\begin{proposition}\label{D-module equivalence of twisted module category}
Let $\mathscr{L}$ be an invertible $\mathscr{O}_X$-module, and $\mathscr{A}$ an $\mathscr{O}_X$-algebra. Then the category $\Mod(\mathscr{A})$ of left $\mathscr{A}$-modules and the category $\Mod(\mathscr{L}\otimes_{\mathscr{O}_X}\mathscr{A}\otimes_{\mathscr{O}_X}\mathscr{L}^{\otimes -1})$ of left $\mathscr{L}\otimes_{\mathscr{O}_X}\mathscr{A}\otimes_{\mathscr{O}_X}\mathscr{L}^{\otimes -1}$-modules are equivalent to each other by
\[\Mod(\mathscr{A})\to \Mod(\mathscr{L}\otimes_{\mathscr{O}_X}\mathscr{A}\otimes_{\mathscr{O}_X}\mathscr{L}^{\otimes -1}),\quad \mathscr{M}\mapsto \mathscr{L}\otimes_{\mathscr{O}_X}\mathscr{M}.\]
\end{proposition}

We now apply this result to the invertible sheaf $\Omega_X$ over $X$. For thism, we note the following result.
\begin{proposition}\label{D-module D_X is twisted ring by Omega}
We have a canonical isomorphism
\[\mathscr{D}_X^{\op}\cong\Omega_X\otimes_{\mathscr{O}_X}\mathscr{D}_X\otimes_{\mathscr{O}_X}\Omega_X^{\otimes -1}.\]
\end{proposition}
\begin{proof}
The right $\mathscr{D}_X$-module structure on $\Omega_X$ gives a homomorphism $\mathscr{D}_X^{\op}\to\sEnd_\C(\Omega_X)$, and its image is in $\sDiff(\Omega_X,\Omega_X)=\Omega_X\otimes_{\mathscr{O}_X}\mathscr{D}_X\otimes_{\mathscr{O}_X}\Omega_X^{\otimes -1}$ in view of (\ref{D-module Omega_X module structure expression-1}). We therefore obtain a homomorphism
\[\varphi:\mathscr{D}_X^{\op}\to\Omega_X\otimes_{\mathscr{O}_X}\mathscr{D}_X\otimes_{\mathscr{O}_X}\Omega_X^{\otimes -1}.\]
Note that $\Omega_X^{\otimes -1}\otimes_{\mathscr{O}_X}\mathscr{D}_X^{\op}\otimes_{\mathscr{O}_X}\Omega_X=(\Omega_X\otimes_{\mathscr{O}_X}\mathscr{D}_X\otimes_{\mathscr{O}_X}\Omega_X^{\otimes -1})^{\op}$, so we also obtain a homomorphism
\[\psi:=(\Omega_X^{\otimes -1}\otimes\varphi\otimes\Omega_X)^{\op}:\Omega_X\otimes_{\mathscr{O}_X}\mathscr{D}_X\otimes_{\mathscr{O}_X}\to \mathscr{D}_X^{\op}.\]
Using \cref{D-module Omega_X module structure expression}, it is easy to check that $\psi$ and $\varphi$ are inverses of each other.
\end{proof}

\begin{remark}
In an affine open $U$ and a local coordinate system $\{x_i,\partial_i\}$, from the proof of \cref{D-module D_X is twisted ring by Omega} we see that the isomorphism of \cref{D-module D_X is twisted ring by Omega} is explicitly given by
\[\mathscr{D}_X^{\op}\to \Omega_X\otimes_{\mathscr{O}_X}\mathscr{D}_X\otimes_{\mathscr{O}_X}\Omega_X^{\otimes -1},\quad P\mapsto dx\otimes P^*\otimes dx^{\otimes -1}\]
where $dx=dx_1\wedge\cdots dx_n$ and $P^*$ is the formal adjoint of $P$.  
\end{remark}

We can identify $\Mod(\mathscr{D}_X^{\op})$ with the category of right $\mathscr{D}_X$-modules. Using tensor products and Hom sheaf, we have the following construction between left and right $\mathscr{D}_X$-modules.
\begin{proposition}\label{D-module tensor and Hom module structure def}
Let $\mathscr{M},\mathscr{N}$ be left $\mathscr{D}_X$-modules and $\mathscr{M}',\mathscr{N}'$ be right $\mathscr{D}_X$-modules. Then we have the following induced $\mathscr{D}_X$-module structures:
\begin{enumerate}
    \item[(\rmnum{1})] $\mathscr{M}\otimes_{\mathscr{O}_X}\mathscr{N}$ has a left $\mathscr{D}_X$-module defined by $v(s\otimes t)=vs\otimes t+s\otimes vt$.
    \item[(\rmnum{2})] $\mathscr{M}'\otimes_{\mathscr{O}_X}\mathscr{N}$ has a right $\mathscr{D}_X$-module defined by $(s'\otimes t)v=s'v\otimes t-s'\otimes vt$.
    \item[(\rmnum{3})] $\sHom_{\mathscr{O}_X}(\mathscr{M},\mathscr{N})$ has a left $\mathscr{D}_X$-module defined by $(v\varphi)(s)=v(\varphi(s))-\varphi(v(s))$.
    \item[(\rmnum{4})] $\sHom_{\mathscr{O}_X}(\mathscr{M}',\mathscr{N}')$ has a left $\mathscr{D}_X$-module defined by $(v\varphi)(s)=-v(\varphi(s))-\varphi(v(s))$.
    \item[(\rmnum{5})] $\sHom_{\mathscr{O}_X}(\mathscr{M},\mathscr{N}')$ has a right $\mathscr{D}_X$-module defined by $(v\varphi)(s)=v(\varphi(s))+\varphi(v(s))$. 
\end{enumerate}
\end{proposition}
The verifications of \cref{D-module tensor and Hom module structure def} are straightforward, for which we just need to check the conditions of \cref{D-module ring D_X generating relation}. A good way to memorize these results is by using the corespondence "left" $\leftrightarrow 0$, "right" $\leftrightarrow 1$, and $\sHom(\bullet,\bigstar)=-\bullet+\bigstar$.

\begin{proposition}\label{D-module Kronecker tensor and base change}
Given a left $\mathscr{D}_X$-module $\mathscr{M}$ and an $\mathscr{O}_X$-module $\mathscr{N}$, there exists a canonical isomorphism of left $\mathscr{D}_X$-modules
\[\mathscr{D}_X\otimes_{\mathscr{O}_X}(\mathscr{N}\otimes_{\mathscr{O}_X}\mathscr{M})\stackrel{\sim}{\to} (\mathscr{D}_X\otimes_{\mathscr{O}_X}\mathscr{N})\otimes_{\mathscr{O}_X}\mathscr{M},\]
where the left $\mathscr{D}_X$-module structure on $\mathscr{D}_X\otimes_{\mathscr{O}_X}-$ is induced by left multiplication on $\mathscr{D}_X$.
\end{proposition}
\begin{proof}
An $\mathscr{O}_X$-module homomorphism
\[\mathscr{N}\otimes_{\mathscr{O}_X}\mathscr{M}\to (\mathscr{D}_X\otimes_{\mathscr{O}_X}\mathscr{N})\otimes_{\mathscr{O}_X}\mathscr{M},\quad s\otimes u \mapsto (1\otimes s)\otimes u\]
can be extended to a $\mathscr{D}_X$-module homomorphism
\[\varphi:\mathscr{D}_X\otimes_{\mathscr{O}_X}(\mathscr{N}\otimes_{\mathscr{O}_X}\mathscr{M})\to (\mathscr{D}_X\otimes_{\mathscr{O}_X}\mathscr{N})\otimes\mathscr{M}.\]
It is easy to see that the image of $F_p(\mathscr{D}_X\otimes_{\mathscr{O}_X}(\mathscr{N}\otimes_{\mathscr{O}_X}\mathscr{M})):=F_p(\mathscr{D}_X)\otimes_{\mathscr{O}_X}(\mathscr{N}\otimes_{\mathscr{O}_X}\mathscr{M})$ under $\varphi$ is contained in $F_p((\mathscr{D}_X\otimes_{\mathscr{O}_X}\mathscr{N})\otimes\mathscr{M}):=(F_p(\mathscr{D}_X)\otimes_{\mathscr{O}_X}\mathscr{N})\otimes_{\mathscr{O}_X}\mathscr{M}$, so $\varphi$ induces a homomorphism from
\[\gr_p(\varphi):\gr_p(\mathscr{D}_X\otimes_{\mathscr{O}_X}(\mathscr{N}\otimes_{\mathscr{O}_X}\mathscr{M}))\to \gr_p((\mathscr{D}_X\otimes_{\mathscr{O}_X}\mathscr{N})\otimes_{\mathscr{O}_X}\mathscr{M}).\]
It is easy to check that this is the identity, using the fact that $\gr_p(\mathscr{D}_X)\cong\bm{S}_{\mathscr{O}_X}^p(\Theta_X)$, so by induction on $p$, it is easy to see that $\varphi$ is an isomorphism.
\end{proof}

\begin{remark}
Take a local coordinate system $\{x_i,\partial_i\}$. Then the homomorphism $\varphi$ in \cref{D-module Kronecker tensor and base change} is given by
\[\varphi(\partial^\alpha\otimes(s\otimes u))=\partial^\alpha((1\otimes s)\otimes u)=\sum_{\beta}\binom{\alpha}{\beta}(\partial^\beta\otimes s)\otimes\partial^{\alpha-\beta}u.\]
Therefore, the inverse map $\psi$ of $\varphi$ is given by
\[\psi((\partial^\alpha\otimes s)\otimes u)=\sum(-1)^{|\beta|}\binom{\alpha}{\beta}\partial^{\alpha-\beta}\otimes (s\otimes\partial^\beta u).\]
\end{remark}

\begin{remark}
By setting $\mathscr{N}=\mathscr{O}_X$ in \cref{D-module Kronecker tensor and base change}, we see that the two left $\mathscr{D}_X$-module structures on $\mathscr{D}_X\otimes_{\mathscr{O}_X}\mathscr{M}$ are isomorphic. This is a nontrivial fact since the isomorphism of these two modules is not the identity map.
\end{remark}

\begin{proposition}\label{D-module tensor with Kronecker tensor isomorphism}
Let $\mathscr{N}$ be a right $\mathscr{D}_X$-module and $\mathscr{M}_1$, $\mathscr{M}_2$ be left $\mathscr{D}_X$-modules. Then
\[\mathscr{N}\otimes_{\mathscr{D}_X}(\mathscr{M}_1\otimes_{\mathscr{O}_X}\mathscr{M}_2)\cong (\mathscr{N}\otimes_{\mathscr{O}_X}\mathscr{M}_1)\otimes_{\mathscr{D}_X}\mathscr{M}_2.\]
\end{proposition}
\begin{proof}
Each side is a quotient of $\mathscr{N}\otimes_{\mathscr{O}_X}\mathscr{M}_1\otimes_{\mathscr{O}_X}\mathscr{M}_2$. The left hand side is the one divided by the submodule generated by
\[sv\otimes (t_1\otimes t_2)-s\otimes v(t_1\otimes t_2)=sv\otimes t_1\otimes t_2-s\otimes vt_1\otimes t_2-s\otimes t_1\otimes vt_2,\]
and the right hand side is the one divided by the submodule generated by
\[(s\otimes t_1)v\otimes t_2-(s\otimes t_1)\otimes vt_2=sv\otimes t_1\otimes t_2-s\otimes vt_1\otimes t_2-s\otimes t_1\otimes vt_2,\]
so they are isomorphic.
\end{proof}

\begin{proposition}\label{D-module Hom and Kronecker tensor prop}
Let $\mathscr{M}_1,\mathscr{M}_2$ and $\mathscr{M}_3$ be left $\mathscr{D}_X$-modules. Then we have a canonical homomorphism of left $\mathscr{D}_X$-modules
\[\mathscr{M}_1\otimes_{\mathscr{O}_X}\sHom_{\mathscr{O}_X}(\mathscr{M}_1,\mathscr{M}_2)\to\mathscr{M}_2,\]
and an isomorphism
\begin{equation}\label{D-module Hom and Kronecker tensor prop-1}
\sHom_{\mathscr{D}_X}(\mathscr{M}_1\otimes\mathscr{M}_2,\mathscr{M}_3)\cong\sHom_{\mathscr{D}_X}(\mathscr{M}_1,\sHom_{\mathscr{O}_X}(\mathscr{M}_2,\mathscr{M}_3)).
\end{equation}
\end{proposition}
\begin{proof}
It is clear that we have a canonical homomorphism $\Phi:\mathscr{M}_1\otimes_{\mathscr{O}_X}\sHom_{\mathscr{O}_X}(\mathscr{M}_1,\mathscr{M}_2)\to\mathscr{M}_2$ which is $\mathscr{O}_X$-linear. To see that it is $\mathscr{D}_X$-linear, we note that
\[\Phi(v(s_1\otimes\varphi))=\Phi(vs_1\otimes\varphi+s_1\otimes v\varphi)=\varphi(vs_1)+(v\varphi)(s_1)=v(\varphi(s_1))=v(\Phi(s_1\otimes\varphi)).\]
where $v\in\Theta_X$, $s_i\in\mathscr{M}_i$, $\varphi\in\sHom_{\mathscr{O}_X}(\mathscr{M}_1,\mathscr{M}_2)$. This proves the first claim, and the second one can be proved similarly, by a detailed computation.
\end{proof}

\begin{proposition}\label{D-module side change by Omega_X}
The correspondence
\[\Omega_X\otimes_{\mathscr{O}_X}\bullet:\Mod(\mathscr{D}_X)\to\Mod(\mathscr{D}_X^{\op}).\]
gives an equivalence of categories, whose quasi-inverse is given by
\[\Omega_X^{\otimes-1}\otimes_{\mathscr{O}_X}\bullet=\sHom_{\mathscr{O}_X}(\Omega_X,\bullet):\Mod(\mathscr{D}_X)^{\op}\to\Mod(\mathscr{D}_X).\]
These operations are called \textbf{side-changing operations} of $\mathscr{D}_X$-modules.
\end{proposition}
\begin{proof}
By \cref{D-module Hom and Kronecker tensor prop}, we have a canonical homomorphism $\Omega_X^{\otimes-1}\otimes_{\mathscr{O}_X}\Omega_X\otimes_{\mathscr{O}_X}\mathscr{M}\to\mathscr{M}$, which is an isomorphism since $\Omega_X$ is locally free. The other direction follows from (\ref{D-module Hom and Kronecker tensor prop-1}) since $\sHom_{\mathscr{O}_X}(\mathscr{O}_X,\mathscr{M})\cong\mathscr{M}$.
\end{proof}

\paragraph{Quasi-coherent and coherent \texorpdfstring{$D$}{D}-modules}
On algebraic varieties, the category of quasi-coherent sheaves (over $\mathscr{O}_X$) is sufficiently wide and suitable for various algebraic operations. Since our sheaf $\mathscr{D}_X$ is locally free over $\mathscr{O}_X$, it is quasi-coherent over $\mathscr{O}_X$. We mainly deal with $\mathscr{D}_X$-modules which are quasi-coherent over $\mathscr{O}_X$. For a smooth algebraic variety $X$, the category of $\mathscr{D}_X$-modules that are quasi-coherent over $\mathscr{O}_X$ is denoted by $\Qcoh(\mathscr{O}_X)$. This is clearly an abelian category.\par
It is well known that for affine algebraic varieties $X$, the global section functor $\Gamma(X,-)$ is exact and $\Gamma(X,\mathscr{M})=0$ if and only if $\mathscr{M}=0$ for $\mathscr{M}\in\Qcoh(\mathscr{M})$. Replacing $\mathscr{O}_X$ by $\mathscr{D}_X$, we come to the following notion.
\begin{definition}
A smooth algebraic variety $X$ is called \textbf{$\bm{D}$-affine} if the following conditions are satisfied:
\begin{enumerate}[leftmargin=40pt]
    \item[(A1)] The global section functor $\Gamma(X,-):\Qcoh(\mathscr{D}_X)\to\Mod(\Gamma(X,\mathscr{D}_X))$ is exact.
    \item[(A2)] If $\Gamma(X,\mathscr{M})=0$ for $\mathscr{M}\in\Qcoh(\mathscr{D}_X)=0$, then $\mathscr{M}=0$.
\end{enumerate}
\end{definition}

It is clear that any smooth affine algebraic variety is $D$-affine. As in the case of quasi-coherent $\mathscr{O}_X$-modules on affine varieties we have the following.
\begin{proposition}\label{D-module global section equivalence for D-affine}
Assume that $X$ is a smooth algebraic variety that is $D$-affine.
\begin{enumerate}
    \item[(a)] Any $\mathscr{M}\in\Qcoh(\mathscr{D}_X)$ is generated over $\mathscr{D}_X$ by its global sections.
    \item[(b)] The functor $\Gamma(X,-):\Qcoh(\mathscr{D}_X)\to\Mod(\Gamma(X,\mathscr{D}_X))$ is an equivalence of categories.
\end{enumerate}
\end{proposition}

\begin{remark}
The D-affinity holds also for certain non-affine varieties. For example, we will see that projective spaces are $D$-affine. We will also show that flag manifolds for semi-simple algebraic groups are $D$-affine. This fact was one of the key points in the settlement of the Kazhdan-Lusztig conjecture.
\end{remark}

\begin{remark}
If $X$ is affine, we can replace $D_X$ with $\mathscr{D}_X^{\op}$ in the above argument. In other words, smooth affine varieties are also $D^{\op}$-affine. Note that $D$-affine varieties are not necessarily $D^{\op}$-affine in general. For example, $\P^1$ is not $D^{\op}$-affine by $\Gamma(\P^1,\Omega_{\P^1})=0$, but it is $D$-affine as we shall see.
\end{remark}

Recall that an $\mathscr{A}$-module $\mathscr{M}$ over a ringed space $(X,\mathscr{A})$ is called \textit{coherent} if it is of finite type over $\mathscr{A}$ and any finite generating relation of $\mathscr{M}$ is finitely presented. We now consider coherent $\mathscr{D}_X$-modules over $X$.
\begin{proposition}\label{D-module coherent D_X-module iff qcoh and ft}
Let $X$ be a smooth algebraic variety. Then $\mathscr{D}_X$ is a coherent sheaf of rings. Moreover, a $\mathscr{D}_X$-module is coherent if and only if it is quasi-coherent over $\mathscr{O}_X$ and of finite type over $\mathscr{D}_X$.
\end{proposition}
\begin{proof}
The first statement follows from the second one, so we only need to prove the second one. If $\mathscr{M}$ is a coherent $\mathscr{D}_X$-module, then $\mathscr{M}$ is of finite type over $\mathscr{D}_X$. Moreover, $\mathscr{M}$ is quasi-coherent over $\mathscr{O}_X$ since it is locally finitely presented as a $\mathscr{D}_X$-module and $\mathscr{D}_X$ is quasi-coherent over $\mathscr{O}_X$. Conversely, assume that $\mathscr{M}$ is of finite type over $\mathscr{D}_X$ and quasi-coherent over $\mathscr{O}_X$. To see that $\mathscr{M}$ is coherent over $\mathscr{D}_X$, it suffices to show that for any affine open subset $U$ of $X$, the kernel of any homomorphism $\varphi:\mathscr{D}_U^m\to\mathscr{M}|_U$ of $\mathscr{D}_U$ is finitely presented over $\mathscr{D}_U$. Since $\mathscr{D}_U(U)$ is a left Noetherian ring, the kernel of $\mathscr{D}_U(U)^m\to\mathscr{M}(U)$ is a finitely generated $\mathscr{D}_U(U)$-module, and this proves the assertion in view of \cref{D-module global section equivalence for D-affine}.
\end{proof}

\begin{theorem}\label{D-module coherent over O_X iff locally free}
A $\mathscr{D}_X$-module is coherent over $\mathscr{O}_X$ if and only if it is an integrable connection.
\end{theorem}

\begin{proposition}\label{D-module D-affine coherent equivalent to finite module}
Let $X$ be a smooth algebraic variety that is $D$-affine. Then the global section functor $\Gamma(X,-)$ induces the equivalence
\[\Coh(\mathscr{D}_X)\stackrel{\sim}{\to} \Mod_f(\Gamma(X,\mathscr{D}_X))\]
where for a ring $A$, we denote by $\Mod_f(A)$ the category of finitely generated $A$-modules.
\end{proposition}

\begin{proposition}\label{D-module qcoh module flasque resolution}
Any $\Qcoh(\mathscr{D}_X)$ is embedded into an injective object $\mathscr{I}$ of $\Qcoh(\mathscr{D}_X)$ which is flasque.
\end{proposition}

\begin{corollary}\label{D-module cohomology trivial for D-affine}
If $X$ is $D$-affine, then for any $\mathscr{M}\in\Qcoh(\mathscr{D}_X)$, we have $H^i(X,\mathscr{M})=0$ for $i>0$.
\end{corollary}

\begin{proposition}\label{D-module quotient of locally free on quasi-proj}
Let $X$ be a smooth quasi-projective algebraic variety.
\begin{enumerate}
    \item[(a)] Any $\mathcal{M}\in\Qcoh(\mathscr{D}_X)$ is a quotient of a locally free $\mathscr{D}_X$-module.
    \item[(b)] Any $\mathcal{M}\in\Coh(\mathscr{D}_X)$ is a quotient of a locally free $\mathscr{D}_X$-module of finite rank.
\end{enumerate}
\end{proposition}

\begin{corollary}\label{D-module resolution by locally free and projective on quasi-proj}
Let $X$ be a smooth quasi-projective algebraic variety. Then any $\mathscr{M}\in\Qcoh(\mathscr{D}_X)$ admits a resolution by locally free $\mathscr{D}_X$-modules and a finite resolution by locally projective $\mathscr{D}_X$-modules. If $\mathscr{M}$ is a coherent $\mathscr{D}_X$-module, then we can assume that each term of these resolutions has finite rank.
\end{corollary}

\subsection{Characteristic varieties}
Recall that we have obtained from $\mathscr{D}_X$, first a commutative object $\bm{S}_{\mathscr{O}_X}(\Theta_X)$ by taking graded ring, next the cotangent bundle $T^*X$ as its geometric object, and finally the symplectic structure on $T^*X$ as a reflection of the noncommutativity of $\mathscr{D}_X$.\par
Now we consider $\mathscr{D}_X$-modules and derive a commutative object from each of them. Let $\mathscr{M}$ be a $\mathscr{D}_X$-module which is quasi-coherent over $\mathscr{O}_X$. We consider an \textit{increasing} filtration of $\mathscr{M}$ by quasi-coherent $\mathscr{O}_X$-submdules $F_i(\mathscr{M})$ satisfying the following conditions:
\begin{enumerate}[leftmargin=40pt]
    \item[(F1)] $\mathscr{M}=\bigcup_iF_i(\mathscr{M})$.
    \item[(F2)] $F_i(\mathscr{D}_X)F_j(\mathscr{M})\sub F_{i+j}(\mathscr{M})$.
    \item[(F3)] $F_i(\mathscr{M})=0$ for $i\ll 0$.  
\end{enumerate}
In this case, we say that $M$ is a \textbf{filtered $\mathscr{D}_X$-module}. The associated graded module
\[\gr(\mathscr{M})=\bigoplus_{i\in\Z}F_i(\mathscr{M})/F_{i-1}(\mathscr{M})\]
is then a graded module over $\gr(\mathscr{D}_X)=\pi_*(\mathscr{O}_{T^*X})$, which is clearly quasi-coherent over $\mathscr{O}_X$.

\begin{proposition}\label{D-module good filtration iff}
Let $M$ be a filtered $\mathscr{D}_X$-module. Then the following conditions are equivalent:
\begin{enumerate}
    \item[(\rmnum{1})] $\gr(\mathscr{M})$ is coherent over $\pi_*(\mathscr{O}_{T^*X})$.
    \item[(\rmnum{2})] There exist locally a surjective $\mathscr{D}_X$-linear homorphism $\Phi:\mathscr{D}_X^{\oplus r}\to\mathscr{M}$ and integers $n_j$ ($j=1,\dots,r$) such that for each $i\in\Z$,
    \[\Phi\Big(\bigoplus_{j=1}^{r}F_{i-n_j}(\mathscr{D}_X)\Big)=F_i(\mathscr{M}).\]
    \item[(\rmnum{3})] $F_i(\mathscr{M})$ is coherent over $\mathscr{O}_X$ for each $i$, and there exists an integer $i_0$ such that locally we have
    \[F_j(\mathscr{D}_X)F_i(\mathscr{M})=F_{i+j}(\mathscr{M})\for i\geq i_0,j\geq 0.\]
\end{enumerate}
\end{proposition}
\begin{proof}
Since the question is local, we can reduce to affine case, from which the equivalence of (\rmnum{1}) and (\rmnum{2}) is clear. Now it is easily checked that (\rmnum{2}) holds if and only if $F_i(\mathscr{M})$ is coherent over $\mathscr{O}_X$ and one can find $i_0$ as in (\rmnum{3}) locally on $X$.
\end{proof}

A filtration $\{F_i(\mathscr{M})\}$ is called \textbf{good} if it satisfies the conditions of \cref{D-module good filtration iff}. By \cref{D-module good filtration iff}, a good filtration induces a coherent module $\gr(\mathscr{M})$ over $\pi_*(\mathscr{O}_{T^*X})$.

\begin{theorem}\label{D-module good filtration iff coh}
Let $X$ be a smooth algebraic variety.
\begin{enumerate}
    \item[(a)] Any coherent $\mathscr{D}_X$-module admits a (locally defined) good filtration. Conversely, a quasi-coherent $\mathscr{D}_X$-module endowed with a good filtration is coherent.
    \item[(b)] Let $F,F'$ be two filtrations of a $\mathscr{D}_X$-module $\mathscr{M}$ and assume that $F$ is good. Then there exists an integer $i_0$ such that locally we have
    \[F_i(\mathscr{M})\sub F'_{i+i_0}(\mathscr{M})\for i\in\Z.\]
    If, moreover, $F'$ is also a good filtration, there exists $i_0$ such that locally
    \[F'_{i-i_0}(\mathscr{M})\sub F_i(\mathscr{M})\sub F'_{i+i_0}(\mathscr{M})\for i\in\Z.\] 
\end{enumerate}
\end{theorem}
\begin{proof}
If $\mathscr{M}$ is a coherent $\mathscr{D}_X$-module, then $\mathscr{M}$ is locally generated by a finite number of sections $u_1,\dots,u_N$, and we can define $F_i(\mathscr{M})$ by
\[F_i(\mathscr{M})=\sum_{\nu=1}^{N}F_i(\mathscr{D}_X)u_\nu.\]
It is easy to see that this is a good filtration of $\mathscr{M}$. Conversely, if $\mathscr{M}$ admits a good filtration, then it is locally generated by finitely many sections, hence coherent. The second assertion is local, and hence follows from the corresponding result for graded modules.
\end{proof}

Let $\mathscr{M}$ be a coherent $\mathscr{D}_X$-module with a good filtration. Let $\pi:T^*X\to X$ be the cotangent bundle of $X$. Since we have $\gr(\mathscr{D}_X)\cong\pi_*(\mathscr{O}_{T^*X})$, the graded module $\gr(\mathscr{M})$ is a coherent module over $\pi_*(\mathscr{O}_{T^*X})$ by \cref{D-module good filtration iff}. The support of the coherent $\mathscr{O}_{T^*X}$-module
\[\widetilde{\gr(\mathscr{M})}:=\mathscr{O}_{T^*X}\otimes_{\pi^{-1}(\mathscr{O}_{T^*X})}\pi^*(\gr(\mathscr{M}))\]
is called the \textbf{characteristic variety} of $\mathscr{M}$ and denoted by $\Ch(M)$ (it is sometimes called the \textbf{singular support} of $M$). Since $\gr(\mathscr{M})$ is a graded module over the graded ring $\gr(\mathscr{D}_X)$, we see that $\Ch(M)$ is a closed conic (i.e., stable by the scalar multiplication of complex numbers on the fibers) algebraic subset in $T^*X$.

\begin{theorem}\label{D-module coherent module char variety independent filtration}
The characteristic variety of a coherent $\mathscr{D}_X$-module $\mathscr{M}$ does not depend on the choice of a good filtration.
\end{theorem}
\begin{proof}
We say two good filtrations $F$ and $G$ are "adjacent" if they satisfy the condition
\[F_i(\mathscr{M})\sub G_i(\mathscr{M})\sub F_{i+1}(\mathscr{M})\for i\in\Z.\]
We first show the assertion in this case. Consider the natural homomorphism
\[\varphi_i:F_i(\mathscr{M})/F_{i-1}(\mathscr{M})\to G_i(\mathscr{M})/G_{i-1}(\mathscr{M}).\]
Then we have $\ker\varphi_i\cong G_{i-1}(\mathscr{M})/F_{i-1}(\mathscr{M})\cong\coker\varphi_{i-1}$, so the homorphism $\varphi:\gr^F(\mathscr{M})\to \gr^G(\mathscr{M})$ entails an isomorphism $\ker\varphi\cong\coker\varphi$. Consider the exact sequence
\[\begin{tikzcd}
0\ar[r]&\ker\varphi\ar[r]&\gr^F(\mathscr{M})\ar[r]&\gr^G(\mathscr{M})\ar[r]&\gr^G(\mathscr{M})\ar[r]&\coker\varphi\ar[r]&0
\end{tikzcd}\]
of coherent $\gr^F(\mathscr{D}_X)$-modules. From this we obtain
\begin{gather*}
\supp(\gr^F(\mathscr{M}))=\supp(\ker\varphi)\cup\supp(\im\varphi),\\
\supp(\gr^G(\mathscr{M}))=\supp(\im\varphi)\cup\supp(\coker\varphi).
\end{gather*}
Hence $\ker\varphi\cong\coker\varphi$ implies $\supp(\gr^F(\mathscr{M}))=\supp(\gr^G(\mathscr{M}))$, so the assertion is proved for adjacent good filtrations.\par
Let us consider the general case. Namely, assume that $F$ and $G$ are arbitrary good filtrations of $M$. For $k\in\Z$ set
\[F_i^{(k)}(\mathscr{M})=F_i(\mathscr{M})+G_{i+1}(\mathscr{M}),\quad i\in\Z.\]
By \cref{D-module good filtration iff coh}, $F^{(k)}$ is a good filtration of $\mathscr{M}$, and $F^{(k)}$, $F^{(k+1)}$ are adjacent for each $k\in\Z$. Since $F^{(k)}=F$ for $k\ll 0$ and $F^{(k)}=G[k]$ for $k\gg 0$, we conclude the assertion from the adjacent case.
\end{proof}

Let $U$ be an affine open subset of $X$. Then $T^*U$ is an affine open subset of $T^*X$, and $\Ch(\mathscr{M})\cap T^*U$ coincides with the support of the coherent $\mathscr{O}_{T^*U}$-module associated to the finitely generated $\gr(\Gamma(U,\mathscr{D}_U))$-module $\gr(\Gamma(U,\mathscr{M}))$. We then have
\[\Ch(\mathscr{M})\cap T^*U=\{p\in T^*U:\text{$f(p)=0$ for $f\in\mathfrak{J}_{\Gamma(U,\mathscr{M})}$}\},\]
where $\mathfrak{J}_{\Gamma(U,\mathscr{M})}$ is the characteristic ideal of $\Gamma(U,\mathscr{M})$, defined by
\[\mathfrak{J}_{\Gamma(U,\mathscr{M})}=\sqrt{\Ann(\gr(\Gamma(U,\mathscr{M})))}=\bigcap_{\p\in\SS_0(\Gamma(U,\mathscr{M}))}\p.\]
The decomposition of $\Ch(\mathscr{M})\cap T^*U$ into irreducible components is given by
\[\Ch(\mathscr{M})\cap T^*U=\bigcup_{\p\in\SS_0(\Gamma(U,\mathscr{M}))}\{p\in T^*U:\text{$f(p)=0$ for $f\in\p$}\}.\]

\begin{proposition}\label{D-module coh characteristic variety prop}
Let $\mathscr{M}$ be a coherent $\mathscr{O}_X$-module and $\pi:T^*X\to X$ be the cotangent bundle.
\begin{enumerate}
    \item[(a)] $\supp(\mathscr{M})=\pi(\Ch(\mathscr{M}))$.
    \item[(b)] $\Ch(\mathscr{M})$ is a closed conic and algebraic subset.
    \item[(c)] If $0\to\mathscr{M}'\to\mathscr{M}\to\mathscr{M}''\to 0$ is an exact sequence of coherent $\mathscr{D}_X$-modules, then
    \[\Ch(\mathscr{M})=\Ch(\mathscr{M}')\cup\Ch(\mathscr{M}'').\] 
\end{enumerate}
\end{proposition}
\begin{proof}
Since $\mathscr{O}_{T^*X}$ is faithfully flat over $\pi^{-1}(\gr(\mathscr{D}_X))$, we see that $\pi(\Ch(\mathscr{M}))=\supp(\gr(\mathscr{M}))$, so the first assertion follows from the easily seen fact $\supp(\mathscr{M})=\supp(\gr(\mathscr{M}))$. Now we have remarked (b), and to prove (c), we may assume that $\mathscr{M}$ has a good filtration $F$. If we endow $\mathscr{M}'$ and $\mathscr{M}''$ the induced filtration, then they are good in view of \cref{D-module good filtration iff coh}, and the sequence
\[\begin{tikzcd}
0\ar[r]&\gr(\mathscr{M}')\ar[r]&\gr(\mathscr{M})\ar[r]&\gr(\mathscr{M}'')\ar[r]&0
\end{tikzcd}\]
The assertion then follows from the corresponding property of $\supp$.
\end{proof}

Let $X$ be a smooth algebraic variety and assume that we are given a coherent $\mathscr{O}_X$-module $\mathscr{G}$. Then we can define an algebraic cycle $\Cyc(\mathscr{G})$ associated to $\mathscr{G}$ as follows. For each irreducible component $V$ of $\supp(\mathscr{G})$, with generic point $\eta$, the local ring $\mathscr{O}_{X,\eta}$ is an Artinian ring, and we define the \textbf{multiplicity of $\mathscr{G}$ along $\bm{V}$} to be
\[\mult_X(\mathscr{G}):=\ell_{\mathscr{O}_{X,\eta}}(\mathscr{G}_\eta).\]
For irreducible subvariety $V$ with $V\nsubseteq\supp(\mathscr{G})$, we set $\mult_V(\mathscr{G})=0$; we then define the formal sum
\[\Cyc(\mathscr{G})=\sum \mult_V(\mathscr{G})\cdot V\]
which is called the associated cycle of $\mathscr{G}$.\par
Let $\mathscr{M}$ be a coherent $\mathscr{D}_X$-module. By choosing a good filtration of $\mathscr{M}$, we can consider the coherent $\mathscr{O}_{T^*X}$-module $\widetilde{\gr(\mathscr{M})}$. From the proof of \cref{D-module coherent module char variety independent filtration}, it is easy to see that the cycle $\Cyc(\gr(\mathscr{M}))$ is independent of the choice of the filtration of $\mathscr{M}$.

\begin{definition}
For a coherent $\mathscr{D}_X$-module $\mathscr{M}$ we define the \textbf{characteristic cycle} of $\mathscr{M}$ by
\[\Cyc(\mathscr{M}):=\Cyc(\widetilde{\gr(\mathscr{M})})=\sum_V\mult_V(\widetilde{\gr(\mathscr{M})})\cdot V.\]
For $d\in\N$, we denote its degree $d$ part by
\[\Cyc_d(\mathscr{M}):=\sum_{\dim(V)=d}\mult_V(\widetilde{\gr(\mathscr{M})})\cdot V.\]
\end{definition}

\begin{proposition}\label{D-module coh associated cycle additive}
If $0\to\mathscr{M}'\to\mathscr{M}\to\mathscr{M}''\to 0$ is an exact sequence of coherent $\mathscr{D}_X$-modules, then for any irreducible component $V$ of $\Ch(\mathscr{M})$, we have
\[\mult_V(\widetilde{\gr(\mathscr{M})})=\mult_V(\widetilde{\gr(\mathscr{M}')})+\mult_V(\widetilde{\gr(\mathscr{M}'')}).\]
In particular, for $d=\dim(\Ch(\mathscr{M}))$, we have
\[\Cyc_d(\mathscr{M})=\Cyc_d(\mathscr{M}')+\Cyc_d(\mathscr{M}'').\]
\end{proposition}
\begin{proof}
The first assertion follows from the above remarks, and the second one follows from this.
\end{proof}

\begin{proposition}\label{D-module coh associated cycle of coker of monomorphism prop}
Let $\mathscr{M}$ be a coherent $\mathscr{D}_X$-module, and $f:\mathscr{M}\to\mathscr{M}$ be a monomorphism of $\mathscr{D}_X$-modules. Then $\Ch(\mathscr{M}/f(\mathscr{M}))$ is a nowhere dense subset of $\Ch(\mathscr{M})$.
\end{proposition}
\begin{proof}
We choose a good filtration on $\mathscr{N}:=\mathscr{M}/f(\mathscr{M})$. By the exact sequence $0\to\mathscr{M}\to\mathscr{M}\to\mathscr{N}\to 0$, we have
\[\mult_V(\mathscr{N})=\mult_V(\mathscr{M})-\mult_V(\mathscr{M})=0\]
for any irreducible component $V$ of $\Ch(\mathscr{M})$, so the multiplicity of $\widetilde{\gr(\mathscr{N})}$ on $V$ is $0$. It follows that the support of $\widetilde{\gr(\mathscr{N})}$ cannot contain $V$, so $\Ch(\mathscr{M}/f(\mathscr{M}))$ is nowhere dense.
\end{proof}

\begin{example}
Let $\mathscr{M}$ be an integrable connection of rank $r>0$ on $X$. We can then define a good filtration on $\mathscr{M}$ by setting
\[F_i(\mathscr{M})=\begin{cases}
0&i<0,\\
\mathscr{M}&i\geq 0,
\end{cases}\]
and we have $\gr(\mathscr{M})\cong\mathscr{M}\cong\mathscr{O}_X^r$ locally. Moreover, since $\Theta_X$ annihilates $\gr(\mathscr{M})$ by degree consideration, we get $\Ch(\mathscr{M})=T^*_XX=s(X)\cong X$ (the zero section of $T^*X$), and $\Cyc(\mathscr{M})=r\cdot T^*_XX$.
\end{example}

\begin{proposition}
For a non-zero coherent $\mathscr{D}_X$-module $\mathscr{M}$ the following three conditions are equivalent:
\begin{enumerate}
    \item[(\rmnum{1})] $\mathscr{M}$ is an integrable connection.
    \item[(\rmnum{2})] $\mathscr{M}$ is coherent over $\mathscr{O}_X$.
    \item[(\rmnum{3})] $\Ch(\mathscr{M})=T^*_XX\cong X$ (the zero section of $T^*X$).
\end{enumerate}
\end{proposition}
\begin{proof}
We have seen that (\rmnum{1}) is equivalent to (\rmnum{2}) (\cite{hotta_Dmodule} Theorem 1.4.10), so it remains to prove that (\rmnum{3})$\Rightarrow$(\rmnum{2}). Since the question is local, we may assume that $X$ is affine with local coordinate system $\{x_i,\partial_i\}$, so that we have $T^*X=X\times\C^n$. Assume that $\Ch(\mathscr{M})=T^*_XX$, then for a good filtration of $\mathscr{M}$ we have
\[\sqrt{\Ann_{\mathscr{O}_X[\xi_1,\dots,\xi_n]}(\gr(\mathscr{M}))}=\sum_{i=1}^{n}\mathscr{O}_X[\xi]\xi_i.\]
Here we denote by $\xi_i$ the principal symbol of $\partial_i$, and we identify $\pi_*(\mathscr{O}_{T^*X})$ with $\mathscr{O}_X[\xi_1,\dots,\xi_n]$. Note that if we set $\mathfrak{I}=\sum_{i=1}^{n}\mathscr{O}_X[\xi]\xi_i$, then since $\mathscr{O}_X[\xi_1,\dots,\xi_n]$ is Noetherian, there exists an integer $m>0$ such that we have
\[\mathfrak{I}^m\sub\Ann_{\mathscr{O}_X[\xi_1,\dots,\xi_n]}(\gr(\mathscr{M}))\sub\mathfrak{I}\]
for $m\gg 0$. Since the monomials $\xi^\alpha$ of degree $m$ generate $\mathfrak{I}^m$, we conclude that
\[\partial^\alpha F_j(\mathscr{M})\sub F_{j+m-1}(\mathscr{M}),\quad |\alpha|=m,\quad j\in\Z.\]
On the other hand, since the filtration is good, we have $F_i(\mathscr{D}_X)F_j(\mathscr{M})=F_{i+j}(\mathscr{M})$ for $j\gg 0$, so it follows that
\begin{align*}
F_{m+j}(\mathscr{M})&=F_m(\mathscr{D}_X)F_j(\mathscr{M})=\sum_{|\alpha|\leq m}\mathscr{O}_X\partial^\alpha F_j(\mathscr{M})\sub F_{j+m-1}(\mathscr{M}).
\end{align*}
This implies $F_{j+1}(\mathscr{M})=F_j(\mathscr{M})=\mathscr{M}$ for $j\gg 0$, which means $\mathscr{M}$ is coherent over $\mathscr{O}_X$ (since each $F_i(\mathscr{M})$ is coherent over $\mathscr{O}_X$).
\end{proof}

\begin{example}
For a coherent $\mathscr{D}_X$-module $\mathscr{M}=D_Xu\cong\mathscr{D}_X/\mathscr{I}$, where $\mathscr{I}=\Ann(u)$, consider the good filtration $F_i(\mathscr{M})=F_i(\mathscr{D}_X)u$ on $\mathscr{M}$. If we define a filtration on $\mathscr{I}$ by $F_i(\mathscr{I})=F_i(\mathscr{D}_X)\cap\mathscr{I}$, then we have $\gr(\mathscr{M})\cong\gr(\mathscr{D}_X)/\gr(\mathscr{I})$. In this case, the graded ideal $\gr(\mathscr{I})$ is generated by the principal symbols $\sigma(P)$ of $P\in\mathfrak{I}$. Therefore, for an arbitrary set $\{\sigma(P_i)\}$ of generators of $\gr(\mathscr{I})$, we have $\mathfrak{I}=\sum_i\mathscr{D}_XP_i$ and
\[\Ch(\mathscr{M})=\{(x,\xi)\in T^*X:\text{$\sigma(P_i)(x,\xi)=0$ for each $i$}\}.\]
However, for a set $\{Q_i\}$ of generators of $\mathfrak{I}$, the similar equality does not always hold. In general, we have only the inclusion
\[\Ch(\mathscr{M})\sub\{(x,\xi)\in T^*X:\text{$\sigma(Q_i)(x,\xi)=0$ for each $i$}\}.\]
This failure is due to the fact that in general we do note have
\[\gr(\mathscr{I})=\sum\gr(\mathscr{D}_X)\sigma_i(P_i).\]
\end{example}

One of the most fundamental results in the theory of $D$-modules is the following result about the characteristic varieties of coherent $D$-modules.

\begin{theorem}\label{D-module characteristic variety involutive}
The characteristic variety of any coherent $\mathscr{D}_X$-module is involutive (or coisotropic) with respect to the symplectic structure of the cotangent bundle $T^*X$.
\end{theorem}

Let us admit this theorem for the time being and proceed with our arguments. Since the dimension of any involutive closed analytic subset is greater than or equal to $\dim(X)$, we obtain the following proposition.

\begin{proposition}\label{D-module characteristic variety dimension geq n}
For every coherent $\mathscr{D}_X$-module $\mathscr{M}$, the dimension of $\Ch(\mathscr{M})$ at every point is greater than or equal to $\dim(X)$.
\end{proposition}

A coherent $\mathscr{D}_X$-module $\mathscr{M}$ is called a \textbf{holonomic $\mathscr{D}_X$-module} (or a holonomic system, or a \textbf{maximally overdetermined system}) if it satisfies $\dim(\Ch(\mathscr{M}))=\dim(X)$. By \cref{D-module characteristic variety involutive}, characteristic varieties of holonomic $\mathscr{D}_X$-modules are $\C^\times$-invariant Lagrangian subset of $T^*X$. Holonomic $\mathscr{D}_X$-modules are the coherent $\mathscr{D}_X$-modules whose characteristic variety has minimal possible dimension. Assume that the dimension of the characteristic variety $\Ch(M)$ is "small", then this means that the ideal defining the corresponding system of differential equations is "large", and hence the space of the solutions should be "small". In fact, we will see later that the holonomicity is related to the finite dimensionality of the solution space.

\subsection{Codimension filtration}

\begin{theorem}\label{D-module sExt with D_X prop}
Let $\mathscr{M}$ be a coherent $\mathscr{D}_X$-module.
\begin{enumerate}
    \item[(a)] $\sExt_{\mathscr{D}_X}^i(\mathscr{M},\mathscr{D}_X)=0$ for $i<\codim(\Ch(\mathscr{M}))$.
    \item[(b)] $\codim(\Ch(\sExt_{\mathscr{D}_X}^i(\mathscr{M},\mathscr{D}_X)))\geq i$.
    \item[(c)] $\Ch(\sExt_{\mathscr{D}_X}^i(\mathscr{M},\mathscr{D}_X))\sub\Ch(\mathscr{M})$.
\end{enumerate}
\end{theorem}
\begin{proof}

\end{proof}

\subsection{Global dimension of \texorpdfstring{$\mathscr{D}_X$}{D}}
Recall that we have defined a functor
\[\sHom_{\mathscr{O}_X}:\Mod(\mathscr{D}_X)^{\op}\times\Mod(\mathscr{D}_X)\to\Mod(\mathscr{D}_X),\]
whose derived functor is denoted by
\[R\!\sHom_{\mathscr{O}_X}:D^-(\mathscr{D}_X)^{\op}\times D^+(\mathscr{D}_X)\to D^+(\mathscr{D}_X).\]
Since injective $\mathscr{D}_X$-modules are injective $\mathscr{O}_X$-modules, the diagram
\[\begin{tikzcd}[column sep=15mm]
D^-(\mathscr{D}_X)^{\op}\times D^+(\mathscr{D}_X)\ar[r,"R\!\sHom_{\mathscr{O}_X}"]\ar[d]&D^+(\mathscr{D}_X)\ar[d]\\
D^-(\mathscr{O}_X)^{\op}\times D^+(\mathscr{O}_X)\ar[r,"R\!\sHom_{\mathscr{O}_X}"]&D^+(\mathscr{O}_X)
\end{tikzcd}\]
is commutative.

\begin{lemma}\label{D-module derived Hom and Kronecker tensor adjoint}
There is a canonical isomorphism
\[R\!\sHom_{\mathscr{D}_X}(\mathscr{O}_X,R\!\sHom_{\mathscr{O}_X}(\mathscr{M},\mathscr{N}))\cong R\!\sHom_{\mathscr{D}_X}(\mathscr{M},\mathscr{N}).\]
\end{lemma}

Let $\mathscr{M},\mathscr{N}\in\Mod(\mathscr{D}_X)$ and $d_X=\dim(X)$. Since $H^iR\!\sHom_{\mathscr{O}_X}(\mathscr{M},\mathscr{N})=0$ for $i>d_X+1$ by Golovin's theorem (\cite{Golovin}), $R\!\sHom_{\mathscr{O}_X}(\mathscr{M},\mathscr{N})$ can be represented by a complex $\mathscr{K}^\bullet$ with $\mathscr{K}^i=0$ for $i>d_X+1$ and $i<0$. We therefore obtain an isomorphism
\[R\!\sHom_{\mathscr{D}_X}(\mathscr{M},\mathscr{N})\cong R\!\sHom_{\mathscr{D}_X}(\mathscr{O}_X,\mathscr{K}^\bullet)\cong\sHom_{\mathscr{D}_X}(\mathscr{D}_X\otimes_{\mathscr{O}_X}\bigw^\bullet\Theta_X,\mathscr{K}^\bullet).\]
This leads us to the following theorem about the global dimension of $\mathscr{D}_X$.

\begin{theorem}\label{D-module global dimension 2n+1}
The global dimension of $\mathscr{D}_X$ is bounded by $2d_X+1$.
\end{theorem}
\begin{proof}
Since the complex $\mathscr{D}_X\otimes_{\mathscr{O}_X}\bigw^\bullet\Theta_X$ is concentrated at degree $[0,d_X]$ and $\mathscr{K}^\bullet$ at degree $[0,1+d_X]$, we conclude that $H^iR\!\sHom_{\mathscr{D}_X}(\mathscr{O}_X,\mathscr{K}^\bullet)=0$ for $i>2d_X+1$, which proves our assertion.
\end{proof}

\begin{remark}
Let $X=\C^n$ with $n>1$ and $I$ be an infinite set. Since $R\!\sHom_{\mathscr{D}_X}(\mathscr{O}_X,\mathscr{D}_X^{\oplus I})\cong\mathscr{O}_X^{\oplus I}[-n]$, we have
\[\Ext_{\mathscr{D}_X}^{2n+1}(\mathscr{O}_{X,0},\mathscr{D}_X^{\oplus I})\cong H_{\{0\}}^{n+1}(X,\mathscr{O}_X^{\oplus I}).\]
We prove that this does not vanish by induction on $n$. Suppose that $n>1$, and let $p:X\to Y=\C^{n-1}$ be the projection map. Then we have an exact sequence
\[\begin{tikzcd}
0\ar[r]&p^{-1}(\mathscr{O}_Y^{\oplus I})\ar[r]&\mathscr{O}_X^{\oplus I}\ar[r,"\partial_n"]&\mathscr{O}_X^{\oplus I}\ar[r]&0
\end{tikzcd}\]
and accordingly an exact sequence
\[\begin{tikzcd}
H_{\{0\}}^{n+1}(X,\mathscr{O}_X^{\oplus I})\ar[r]&H_{\{0\}}^{n+2}(X,p^{-1}(\mathscr{O}_Y^{\oplus I}))\ar[r]&H_{\{0\}}^{n+2}(X,\mathscr{O}_X^{\oplus I})=0
\end{tikzcd}\]
Note that $H^i_{\{0\}}(X,p^{-1}(\mathscr{F}))=H^{i-2}_{\{0\}}(Y,\mathscr{F})$ for any sheaf $\mathscr{F}$ over $Y$, so we conclude that
\[H^{n+2}_{\{0\}}(X,p^{-1}(\mathscr{O}_Y^{\oplus I}))=H^n_{\{0\}}(Y,\mathscr{O}_Y^{\oplus I})\neq 0\]
and therefore $H^{n+1}_{\{0\}}(X,\mathscr{O}_X^{\oplus I})\neq 0$.
\end{remark}

We shall next consider the global dimension of $\mathscr{D}_{X,x}$ for any point $x\in X$. For this, it suffices to consider the following lemma.
\begin{lemma}\label{Noe ring finite module projdim finite if Ext bounded}
Suppose that the projective dimension of a finitely generated module $M$ over a Noetherian ring $A$ is finite, and that $\Ext_A^i(M,A)=0$ for $i>r$. Then the projective dimension of $M$ is less than or equal to $r$.
\end{lemma}
\begin{proof}

\end{proof}

\begin{theorem}\label{D-module global dimension of stalk n}
Let $X$ be an $n$-dimensional smooth algebra variety. Then for any affine open subset $U$ and $x\in X$, the global dimension of $\Gamma(U,\mathscr{D}_X)$ and $\mathscr{D}_{X,x}$ are equal $n$.
\end{theorem}
\begin{proof}
By \cref{Noe ring finite module projdim finite if Ext bounded}, it suffices to show that $\sExt_{\mathscr{D}_X}^i(\mathscr{M},\mathscr{D}_X)=0$ for $i>n$ and any coherent $\mathscr{D}_X$-module. By \cref{D-module sExt with D_X prop}~(b), we have
\[\codim(\Ch(\sExt_{\mathscr{D}_X}^i(\mathscr{M},\mathscr{D}_X)))\geq i>n\]
so $\sExt_{\mathscr{D}_X}^i(\mathscr{M},\mathscr{D}_X)=0$ by \cref{D-module characteristic variety dimension geq n}. We therefore conclude that $\gldim(\mathscr{D}_{X,x})\leq n$ and $\gldim(\Gamma(U,\mathscr{D}_X))$, and the equality follows from the fact that $\Ext_{\Gamma(U,\mathscr{D}_X)}^n(\Gamma(U,\mathscr{O}_{X}),\Gamma(U,\mathscr{D}_{X}))=\Gamma(U,\Omega_X)\neq 0$ (cf. \cref{D-module dual of coh char by sheaf dual}).
\end{proof}

\subsection{Duality functor}
Let $D(\mathscr{D}_X)$ be the derived category of $\Mod(\mathscr{D}_X)$, and $D^*(\mathscr{D}_X)$ ($*\in\{+,-,b\}$) denote the full subcategory of $D(\mathscr{D}_X)$ consisting of complexes bounded above, below, and bounded, respectively. Let $D^b_{\coh}(\mathscr{D}_X)$ denote the full subcategory of $\mathscr{M}\in D^b(\mathscr{D}_X)$ such that $H^i(\mathscr{M})$ are coherent $\mathscr{D}_X$-modules. Then since $\Coh(\mathscr{D}_X)$ is a Serre subcategory of $\Mod(\mathscr{D}_X)$, we see that $D^b_{\coh}(\mathscr{D}_X)$ is a triangulated subcategory of $D^b(\mathscr{D}_X)$.\par
We now try to find heuristically a candidate for the "dual" of a left $\mathscr{D}$-module. Let $\mathscr{M}$ be a left $\mathscr{D}_X$-module. Since $\mathscr{D}_X$ is a $(\mathscr{D}_X,\mathscr{D}_X)$-bimodule, we see $\sHom_{\mathscr{D}_X}(\mathscr{M},\mathscr{D}_X)$ is a right $\mathscr{D}_X$-module by right multiplication of $\mathscr{D}_X$ on $\mathscr{D}_X$. By the side-changing functor $\otimes_{\mathscr{O}_X}\Omega_X^{\otimes -1}$, we then obtain a left $\mathscr{D}_X$-module $\sHom_{\mathscr{D}_X}(\mathscr{M},\mathscr{D})\otimes_{\mathscr{O}_X}\Omega_X^{\otimes-1}$. Since the functor $\sHom_{\mathscr{D}_X}(-,\mathscr{D}_X)$ is not exact, a more natural choice is the complex $R\!\sHom_{\mathscr{D}_X}(\mathscr{M},\mathscr{D}_X)\otimes_{\mathscr{O}_X}\Omega_X^{\otimes-1}$ of left $\mathscr{D}_X$-modules.

\begin{example}
Let $X=\C$ (or an open subset of $\C$) and $\mathscr{M}=\mathscr{D}_X/\mathscr{D}_XP$ ($P\neq 0$). By applying the functor $\sHom_{\mathscr{D}_X}(-,\mathscr{D}_X)$ to the exact sequence
\[\begin{tikzcd}
0\ar[r]&\mathscr{D}_X\ar[r,"\cdot P"]&\mathscr{D}_X\ar[r]&\mathscr{M}\ar[r]&0
\end{tikzcd}\]
of left $\mathscr{D}_X$-modules, we get an exact sequence
\[\begin{tikzcd}
0\ar[r]&0\ar[r]&\sHom_{\mathscr{D}_X}(\mathscr{M},\mathscr{D}_X)\ar[r]&\mathscr{D}_X\ar[r,"P\cdot"]&\mathscr{D}_X
\end{tikzcd}\]
Hence in this case, we have 
\[\sExt_{\mathscr{D}_X}^0(\mathscr{M},\mathscr{D}_X)=\sHom_{\mathscr{D}_X}(\mathscr{M},\mathscr{D}_X)=\ker(P:\mathscr{D}_X\to\mathscr{D}_X)=0,\]
and the only non-vanishing cohomology group is the first one
\[\sExt_{\mathscr{D}_X}^1(\mathscr{M},\mathscr{D}_X)=\mathscr{D}_X/P\mathscr{D}_X.\]
The left DX-module obtained by the side changing $\otimes\Omega_X^{\otimes-1}$ is isomorphic to
\[\sExt_{\mathscr{D}_X}^1(\mathscr{M},\mathscr{D}_X)\otimes_{\mathscr{O}_X}\Omega_X^{\otimes-1}\cong \mathscr{D}_X/\mathscr{D}_XP^*,\]
where $P^*$ is the formal adjoint of $P$. From this calculation, we see that $\sExt^1$ is more suited than $\sExt^0$ to be called a "dual" of $\mathscr{M}$. More generally, if $d_X=\dim(X)$ and $\mathscr{M}$ is a holonomic $\mathscr{D}_X$-module, then we can (and will) prove that only the term $\sExt_{\mathscr{D}_X}^n(\mathscr{M},\mathscr{D}_X)$ survives and the resulting left $\mathscr{D}_X$-module $\sExt_{\mathscr{D}_X}^n(\mathscr{M},\mathscr{D}_X)\otimes_{\mathscr{O}_X}\Omega_X^{\otimes-1}$ is also holonomic. Hence the correct definition of the dual $D_X(\mathscr{M})$ of a holonomic $\mathscr{D}_X$-module $\mathscr{M}$ is given by $D_X(\mathscr{M})=\sExt_{\mathscr{D}_X}^n(\mathscr{M},\mathscr{D}_X)\otimes_{\mathscr{O}_X}\Omega_X^{\otimes-1}$. For a non-holonomic $\mathscr{D}_X$-module, one may have other non-vanishing cohomology groups, and hence the duality functor should be defined as follows for the derived categories.
\end{example}

We now define a contravariant functor $D_X:D^-(\mathscr{D}_X)\to D^+(\mathscr{D}_X)$ by
\[D_X(\mathscr{M})=R\!\sHom_{\mathscr{D}_X}(\mathscr{M},\mathscr{D}_X)\otimes_{\mathscr{O}_X}\Omega_X^{\otimes -1}[d_X]=R\!\sHom_{\mathscr{D}_X}(\mathscr{M},\mathscr{D}_X\otimes_{\mathscr{O}_X}\Omega_X^{\otimes -1})[d_X].\]
The shift $[d_X]$ is added so that $D_X$ sends $\mathscr{O}_X$ to itself. Since the cohomological dimension of $\mathscr{D}_X$ is finite, $D_X$ preserves $D^b(\mathscr{D}_X)$.

\begin{example}
We have 
\[H^i(D_X(\mathscr{D}_X))=\begin{cases}
\mathscr{D}_X\otimes_{\mathscr{O}_X}\Omega_X^{\otimes-1}&i=-d_X,\\
0&i\neq -d_X.
\end{cases}\]
\end{example}

\begin{lemma}\label{D-module sExt compatible with restriction}
Let $\mathscr{M}$ be a coherent $\mathscr{D}_X$-module. Then for any affine open subset $U$ of $X$, we have
\[\Gamma(U,\sExt_{\mathscr{D}_X}^i(\mathscr{M},\mathscr{D}_X))=\Ext_{\Gamma(U,\mathscr{D}_X)}^i(\Gamma(U,\mathscr{M}),\Gamma(U,\mathscr{D}_X)).\]
\end{lemma}
\begin{proof}

\end{proof}

\begin{proposition}\label{D-module dual functor involutive prop}
The functor $D_X$ sends $D_{\coh}^b(\mathscr{D}_X)$ to $D^b_{\coh}(\mathscr{D}_X)^{\op}$ and $D_X^2\cong\id$ on $D_{\coh}^b(\mathscr{D}_X)$. In particular, $D_X$ is fully faithful on $D_{\coh}^b(\mathscr{D}_X)$.
\end{proposition}
\begin{proof}
We see from \cref{D-module sExt compatible with restriction} that $H^i(D_X(\mathscr{M}))$ is coherent for each $i$, whence the first claim. Now we construct a canonical morphism $\mathscr{M}\to D_X^2(\mathscr{M})$ for $\mathscr{M}\in D^b(\mathscr{D}_X)$. First note that
\[D_X^2(\mathscr{M})\cong R\sHom_{\mathscr{D}_X^{\op}}(R\sHom_{\mathscr{D}_X}(\mathscr{M},\mathscr{D}_X),\mathscr{D}_X),\]
where $R\sHom_{D_X}(\mathscr{M},\mathscr{D}_X)$ and $\mathscr{D}_X$ are regareded as objects of $D^b(\mathscr{D}_X^{\op})$ byy the right multiplication of $\mathscr{D}_X$, and the left $\mathscr{D}_X$-action on the right hand side is induced from the left multiplication of $\mathscr{D}_X$. Set $\mathscr{H}^\bullet=R\sHom_{\mathscr{D}_X}(\mathscr{M},\mathscr{D}_X)\in D^b_{\coh}(\mathscr{D}_X^{\op})$. Then we have an isomorphism (note that $\otimes_\C$ is exact, hence has no derived functors)
\[R\sHom_{\mathscr{D}_X\otimes_{\C}\mathscr{D}_X^{\op}}(\mathscr{M}\otimes_{\C}\mathscr{H}^\bullet,\mathscr{D}_X)\cong R\sHom_{\mathscr{D}_X}(\mathscr{M},R\sHom_{\mathscr{D}_X^{\op}}(\mathscr{H}^\bullet,\mathscr{D}_X)).\]
Applying $H^0(R\Gamma(X,-))$, we then obtain
\[\Hom_{\mathscr{D}_X\otimes_{\C}\mathscr{D}_X^{\op}}(\mathscr{M}\otimes_{\C}\mathscr{H}^\bullet,\mathscr{D}_X)\cong \Hom_{\mathscr{D}_X}(\mathscr{M},R\sHom_{\mathscr{D}_X^{\op}}(\mathscr{H}^\bullet,\mathscr{D}_X)).\]
Hence the canonical homomorphism $\mathscr{M}\otimes_{\C}\mathscr{H}^\bullet\to\mathscr{D}_X$ gives rise to a canonical morphism
\[\mathscr{M}\to R\sHom_{\mathscr{D}_X^{\op}}(\mathscr{H}^\bullet,\mathscr{D}_X).\]
To show that this morphism is an isomorphism for $\mathscr{M}\in D_{\coh}^b(\mathscr{D}_X)$, we may assume that $X$ is affine. Then we can replace $\mathscr{M}$ with $\mathscr{D}_X$ by a five lemma argument, and the claim is then clear.
\end{proof}

\begin{proposition}\label{D-module dual of holonomic prop}
Let $\mathscr{M}$ be a coherent $\mathscr{D}_X$-module.
\begin{enumerate}
    \item[(a)] $H^i(D_X(\mathscr{M}))=0$ unless $\codim(\Ch(\mathscr{M}))-d_X\leq i\leq 0$.
    \item[(b)] $\codim(\Ch(H^i(D_X(\mathscr{M}))))\geq d_X+i$.
    \item[(c)] $\mathscr{M}$ is holonomic if and only if $H^i(D_X(\mathscr{M}))=0$ for $i\neq 0$.
    \item[(d)] If $\mathscr{M}$ is holonomic, then $D_X(\mathscr{M})\cong H^0(D_X(\mathscr{M}))$ is also holonomic.
\end{enumerate}
\end{proposition}
\begin{proof}
Since $D_X(\mathscr{M})\cong R\sHom_{\mathscr{D}_X}(\mathscr{M},\mathscr{D}_X)[d_x]$ as $\mathscr{O}_X$-modules, the first two assertions follow from \cref{D-module sExt with D_X prop}. The statement (d) and the only if part of (c) follow follow from (a), (b) and \cref{D-module characteristic variety dimension geq n}, so it suffices to prove that if part of (c). Assume that $H^i(D_X(\mathscr{M}))=0$ for $i\neq 0$, that is, $D_X(\mathscr{M})\cong H^0(D_X(\mathscr{M}))$. If we write $\mathscr{M}^*=H^0(D_X(\mathscr{M}))$, then $D_X(\mathscr{M}^*)=D^2_X(\mathscr{M})\cong\mathscr{M}$ and $H^0(D_X(\mathscr{M}^*))\cong \mathscr{M}$. On the other hand, by (b) we have $\codim(\Ch(H^0(D_X(\mathscr{M}^*))))\geq d_X$, and hence $D_X(\mathscr{M}^*)\cong\mathscr{M}$ is a holonomic $\mathscr{D}_X$-module.
\end{proof}

\begin{proposition}\label{D-module dual of coh char by sheaf dual}
Let $\mathscr{M}$ be an integrable connection. Then
\[D_X(\mathscr{M})\cong \sHom_{\mathscr{O}_X}(\mathscr{M},\mathscr{O}_X).\]
\end{proposition}
\begin{proof}
We consider the locally free resolution
\begin{equation}\label{D-module dual of coh char by sheaf dual-1}
\begin{tikzcd}
0\ar[r]&\mathscr{D}_X\otimes_{\mathscr{O}_X}\bigw^{d_X}\Theta_X\ar[r]&\cdots\ar[r]&\mathscr{D}_X\otimes_{\mathscr{O}_X}\Theta_X\ar[r]&\mathscr{D}_X\ar[r]&\mathscr{O}_X\ar[r]&0
\end{tikzcd}
\end{equation}
of $\mathscr{O}_X$. Since $\mathscr{M}$ is locally free over $\mathscr{O}_X$, $\mathscr{D}_X\otimes_{\mathscr{O}_X}\bigw^\bullet\Theta_X\otimes_{\mathscr{O}_X}\mathscr{M}$ is a locally free resolution of $\mathscr{M}$. Using this resolution, we can calculate $D_X(\mathscr{M})$ by the complex
\begin{align*}
\sHom_{\mathscr{D}_X}&(\mathscr{D}_X\otimes_{\mathscr{O}_X}(\bigw^\bullet\Theta_X\otimes_{\mathscr{O}_X}\mathscr{M}),\mathscr{D}_X\otimes_{\mathscr{O}_X}\Omega_X^{\otimes-1})[d_X]\\
&\cong \sHom_{\mathscr{O}_X}(\bigw^\bullet\Theta_X\otimes_{\mathscr{O}_X}\mathscr{M},\mathscr{D}_X\otimes_{\mathscr{O}_X}\Omega_X^{\otimes-1})[d_X]\\
&\cong \sHom_{\mathscr{O}_X}(\Omega_X\otimes_{\mathscr{O}_X}\bigw^\bullet\Theta_X\otimes_{\mathscr{O}_X}\mathscr{M},\Omega_X\otimes_{\mathscr{O}_X}\mathscr{D}_X\otimes_{\mathscr{D}_X}\Omega_X^{\otimes-1})[d_X].
\end{align*}
Since $\Omega_X\otimes_{\mathscr{O}_X}\bigw^\bullet\Theta_X\cong\Omega_X^\bullet$, this is isomorphic to
\begin{align*}
\sHom_{\mathscr{O}_X}&(\Omega_X^\bullet\otimes_{\mathscr{O}_X}\mathscr{M},\mathscr{O}_X)\otimes_{\mathscr{O}_X}(\Omega_X\otimes_{\mathscr{O}_X}\mathscr{D}_X\otimes_{\mathscr{D}_X}\Omega_X^{\otimes-1})[d_X]\\
&\cong (\bigw^\bullet\Theta_X\otimes_{\mathscr{O}_X}\sHom_{\mathscr{O}_X}(\mathscr{M},\mathscr{O}_X))\otimes_{\mathscr{O}_X}(\Omega_X\otimes_{\mathscr{O}_X}\mathscr{D}_X\otimes_{\mathscr{D}_X}\Omega_X^{\otimes-1})[d_X]\\
&\cong \mathscr{D}_X\otimes(\bigw^\bullet\Theta_X\otimes_{\mathscr{O}_X}\sHom_{\mathscr{O}_X}(\mathscr{M},\mathscr{O}_X))[d_X].
\end{align*}
This complex is quasi-isomorphic to $\sHom_{\mathscr{O}_X}(\mathscr{M},\mathscr{O}_X)$ in view of (\ref{D-module dual of coh char by sheaf dual-1}).
\end{proof}

For a complex $\mathscr{M}\in D^b(\mathscr{D}_X)$, we define the support of $\mathscr{M}$ by
\[\supp(\mathscr{M}):=\bigcup_i\supp(H^i(\mathscr{M})).\]
and for $\mathscr{M}\in D^b_{\coh}(\mathscr{D}_X)$, we set
\[\Ch(\mathscr{M}):=\bigcup_i\Ch(H^i(\mathscr{M})).\]
\begin{proposition}\label{D-module dual of coh characteristic variety char}
For a coherent $\mathscr{M}\in D_{\coh}^b(\mathscr{D}_X)$, we have
\[\Ch(D_X(\mathscr{M}))=\Ch(\mathscr{M}).\]
\end{proposition}
\begin{proof}
The proof of the inclusion $\Ch(D_x(\mathscr{M}))\sub\Ch(\mathscr{M})$ is reduced to the case $\mathscr{M}\in\Mod(\mathscr{D}_X)$, which is nothing but \cref{D-module sExt with D_X prop}. Then, by applying this to $D_X(\mathscr{M})$, we obtain the inverse inclusion, since $D_X^2(\mathscr{M})\cong\mathscr{M}$.
\end{proof}

In the rest of this paragraph, we give a description of $R\sHom_{\mathscr{D}_X}(\mathscr{M},\mathscr{N})$ for $M\in D^b_{\coh}(\mathscr{D}_X)$, $N\in D^b(\mathscr{D}_X)$, in terms of the duality functor.

\begin{lemma}\label{D-module Hom and tensor with dual isomorphic for coh}
For $M\in D^b_{\coh}(\mathscr{D}_X)$ and $N\in D^b(\mathscr{D}_X)$, we have
\[R\sHom_{\mathscr{D}_X}(\mathscr{M},\mathscr{N})\cong R\sHom_{\mathscr{D}_X}(\mathscr{M},\mathscr{D}_X)\otimes_{\mathscr{D}_X}^L\mathscr{N}.\]
\end{lemma}
\begin{proof}
Note that there is a canonical morphism
\[R\sHom_{\mathscr{D}_X}(\mathscr{M},\mathscr{N})\to  R\sHom_{\mathscr{D}_X}(\mathscr{M},\mathscr{D}_X)\otimes_{\mathscr{D}_X}^L\mathscr{N}.\]
Hence we may assume that $\mathscr{M}=\mathscr{D}_X$, in which case the assertion is obvious since both sides are isomorphic to $\mathscr{N}$.
\end{proof}

\begin{proposition}\label{D-module Hom of coh isomorphism by duality}
For $M\in D^b_{\coh}(\mathscr{D}_X)$ and $N\in D^b(\mathscr{D}_X)$, we have
\begin{align}
R\sHom_{\mathscr{D}_X}(\mathscr{M},\mathscr{N})&\cong(\Omega_X\otimes_{\mathscr{O}_X}^LD_X(\mathscr{M}))\otimes_{\mathscr{D}_X}^L\mathscr{N}[-d_X]\label{D-module Hom of coh isomorphism by duality-1}\\
&\cong \Omega_X\otimes_{\mathscr{D}_X}^L(D_X(\mathscr{M})\otimes_{\mathscr{O}_X}^L\mathscr{N})[-d_X]\label{D-module Hom of coh isomorphism by duality-2}\\
&\cong R\sHom_{\mathscr{D}_X}(\mathscr{O}_X,D_X(\mathscr{M})\otimes_{\mathscr{O}_X}^L\mathscr{N}) \label{D-module Hom of coh isomorphism by duality-3}
\end{align}
in $D^b(\C_X)$. In particular, we have 
\begin{equation}\label{D-module Hom of coh isomorphism by duality-4}
R\sHom_{\mathscr{D}_X}(\mathscr{O}_X,\mathscr{N})\cong\Omega_X\otimes_{\mathscr{O}_X}^L\mathscr{N}[-d_X].
\end{equation}
\end{proposition}
\begin{proof}
We first prove (\ref{D-module Hom of coh isomorphism by duality-4}). By \cref{D-module Hom and tensor with dual isomorphic for coh}, we may assume that $\mathscr{N}=\mathscr{D}_X$. In this case, by \cref{D-module resolution of Omega_X and O_X} we have
\begin{align*}
R\sHom_{\mathscr{D}_X}(\mathscr{O}_X,\mathscr{D}_X)&\cong \sHom_{\mathscr{D}_X}(\mathscr{D}_X\otimes_{\mathscr{O}_X}\bigw^\bullet\Theta_X,\mathscr{D}_X)\cong\sHom_{\mathscr{O}_X}(\bigw^\bullet\Theta_X,\mathscr{D}_X)\\
&\cong\Omega_X^\bullet\otimes_{\mathscr{O}_X}\mathscr{D}_X\cong\Omega_X[-d_X].
\end{align*}
Now in the general case, by \cref{D-module Hom and tensor with dual isomorphic for coh} we have
\begin{align*}
R\sHom_{\mathscr{D}_X}(\mathscr{M},\mathscr{N})\cong R\sHom_{\mathscr{D}_X}(\mathscr{M},\mathscr{D}_X)\otimes_{\mathscr{D}_X}^L\mathscr{N}\cong(\Omega_X\otimes_{\mathscr{O}_X}^LD_X(\mathscr{M}))\otimes_{\mathscr{D}_X}^L\mathscr{N}[-d_X].
\end{align*}
The second and the third isomorphisms follow from the derived version of \cref{D-module tensor with Kronecker tensor isomorphism} and (\ref{D-module Hom of coh isomorphism by duality-4}), respectively.
\end{proof}

\section{Functorial properties of \texorpdfstring{$D$}{D}-modules}
\subsection{Inverse images of \texorpdfstring{$D$}{D}-modules}
Let $f:X\to Y$ be a morphism of smooth algebraic varieties. We want to construct a $\mathscr{D}_X$-module by lifting a $\mathscr{D}_Y$-module. Let $\mathscr{M}$ be a (left) $\mathscr{D}_Y$-module and consider its inverse image
\[f^*(\mathscr{M})=\mathscr{O}_X\otimes_{f^{-1}(\mathscr{O}_Y)}f^{-1}(\mathscr{M}).\]
We can endow $f^*(\mathscr{M})$ with a (left) $\mathscr{D}_X$-module structure as follows. First, note that we have a canonical $\mathscr{O}_X$-linear homomorphism
\begin{equation}\label{D-module inverse image def-1}
\Theta_X\to f^*(\Theta_Y)=\mathscr{O}_X\otimes_{f^{-1}(\mathscr{O}_Y)}f^{-1}(\Theta_Y),\quad v\mapsto\tilde{v}
\end{equation}
obtained by taking the $\mathscr{O}_X$-dual of the homomorphism $\mathscr{O}_X\otimes_{f^{-1}(\mathscr{O}_Y)}f^{-1}(\Omega_Y^1)\to\Omega_X^1$. Then we can define a left $\mathscr{D}_X$-module structure on $f^*(\mathscr{M})$ by
\begin{equation}\label{D-module inverse image def-2}
v(a\otimes s)=v(a)\otimes s+a\tilde{v}(s),\quad v\in\Theta_X,a\in\mathscr{O}_X,s\in\mathscr{M}.
\end{equation}
Here, if we write $\tilde{v}=\sum_ia_i\otimes w_i$ with $a_i\in\mathscr{O}_X$ and $w_i\in\Theta_Y$, then
\begin{equation}\label{D-module inverse image def-3}
\tilde{v}(s)=\sum_ia_i\otimes w_i(s).
\end{equation}
This definition is independent of the choice of $\sum_ia_i\otimes w_i$. Indeed, since we can define a morphism
\[(\mathscr{O}_X\otimes_{f^{-1}(\mathscr{O}_Y)}f^{-1}(\Theta_Y))\otimes_{\C}f^{-1}(\mathscr{M})\to \mathscr{O}_X\otimes_{f^{-1}(\mathscr{O}_Y)}f^{-1}(\mathscr{M}),\quad (f\otimes v)\otimes s\mapsto f\otimes vs\]
we obtain from (\ref{D-module inverse image def-1}) a homomorphism
\[\Theta_X\otimes_{\C}(\mathscr{O}_X\otimes_{\C}f^{-1}(\mathscr{M}))\to \mathscr{O}_X\otimes_{f^{-1}(\mathscr{O}_Y)}f^{-1}(\mathscr{M})\]
which can be checked to be given by 
\[v\otimes a\otimes s\mapsto \sum_iaa_i\otimes w_i(s).\]
Moreover, it is easily seen that $v(af^*(b)\otimes s)=v(a\otimes bs)$ for $b\in\mathscr{O}_Y$, so this provides us a homomorphism $\Theta_X\otimes_{\C}f^*(\mathscr{M})\to f^*(\mathscr{M})$. Since this action satisfies the conditions of \cref{D-module ring D_X generating relation}, it can be extended to an action of $\mathscr{D}_X$. If we are given a local coordinate system $\{y_i,\partial_i\}$ of $Y$, then the action of $v\in\Theta_X$ can be written more explicitly as
\begin{equation}\label{D-module inverse image def-4}
v(a\otimes s)=v(a)\otimes s+a\sum_{i=1}^{n}v(y_i\circ f)\otimes\partial_is.
\end{equation}

Regarding $\mathscr{D}_Y$ as a left $\mathscr{D}_Y$-module by the left multiplication, we obtain a left $\mathscr{D}_X$-module $f^*(\mathscr{D}_Y)=\mathscr{O}_X\otimes_{f^{-1}(\mathscr{O}_Y)}f^{-1}(\mathscr{D}_Y)$. Then the right multiplication of $\mathscr{D}_Y$ on $\mathscr{D}_Y$ induces a right $f^{-1}(\mathscr{D}_Y)$-module structure on $f^*(\mathscr{D}_Y)$:
\[(a\otimes P)Q=a\otimes PQ,\quad a\in\mathscr{O}_X,P,Q\in\mathscr{D}_Y,\]
and it is immediate that these two actions commute, so $f^*(\mathscr{D}_Y)$ turns out to be a $(\mathscr{D}_X,f^{-1}(\mathscr{D}_Y))$-bimodule, which is denote by $\mathscr{D}_{X\to Y}$. Let $1_{X\to Y}$ denote the canonical element $1\otimes 1$ of $\mathscr{D}_{X\to Y}$. Then for $b\in\mathscr{O}_Y$, we have
\begin{equation}\label{D-module inverse image def-5}
v1_{X\to Y}=\sum_ia_i1_{X\to Y}w_i,\quad 1_{X\to Y}b=f^*(b)1_{X\to Y}.
\end{equation}

Therefore, if $\mathscr{M}$ is a left $\mathscr{D}_Y$-module, we have an isomorphism
\[f^*(\mathscr{M})\cong \mathscr{D}_{X\to Y}\otimes_{f^{-1}(\mathscr{D}_Y)}f^{-1}(\mathscr{M})\]
of left $\mathscr{D}_X$-modules. We have thus defined a functor
\[f^*:\Mod(\mathscr{D}_Y)\to\Mod(\mathscr{D}_X),\quad \mathscr{M}\mapsto \mathscr{D}_{X\to Y}\otimes_{f^{-1}(\mathscr{D}_Y)}f^{-1}(\mathscr{M}).\]

\begin{example}
Let $X=\C$ and $Y=\C$ with coordinates $\{x,\partial_x\}$ and $\{y,\partial_y\}$, respectively. Let $f$ be the morphism given by
\[f:X\to Y,\quad x\mapsto x^2.\]
Then
\[\mathscr{D}_Y=\bigoplus\mathscr{O}_Y\partial^n_y,\quad \mathscr{D}f^*(\mathscr{D}_Y)=\mathscr{D}_{X\to Y}=\bigoplus_{n\geq 0}\mathscr{O}_X\otimes_{\C}\C\partial_y^n\]
and we have
\[\partial_x(a\otimes\partial_y^n)=\frac{\partial a}{\partial x}\otimes\partial_y^n+2ax\otimes\partial_y^{n+1}.\]
While $\mathscr{D}_{X\to Y}$ is isomorphic to $\mathscr{D}_X$ in $X\setminus\{0\}$, we see that it is not coherent in a neighborhood of $x=0$. In fact, we have
\[\mathscr{D}_{X\to Y}=\bigoplus_{n\geq 0}\mathscr{O}_X(x^{-1}\partial_x)^n\sub\mathscr{D}_X[x^{-1}].\]
\end{example}

\begin{example}\label{D-module closed immersion relative D-module}
Assume that $i:X\to Y$ is a closed immersion of smooth algebraic varieties. At each point $x\in X$, we can choose a local coordinate $\{y_i,\partial_{y_i}\}$ on an affine open subset of $Y$ such that $y_{r+1}=\cdots=y_n=0$ gives a local defining equation of $X$. We set $x_i=y_i\circ i$ for $i=1,\dots,r$, which gives a local coordinate $\{x_i,\partial_{x_i}\}$ of an affine open subset of $X$. The canonical morphism $\Theta_X\to\mathscr{O}_X\otimes_{i^{-1}(\mathscr{O}_Y)}i^{-1}(\Theta_Y)$ is then given by $\partial_{x_i}\mapsto\partial_{y_i}$ for $i=1,\dots,r$. Now consider
\[\mathscr{D}=\bigoplus_{\alpha_1,\dots,\alpha_r}\mathscr{O}_Y\partial_{y_1}^{\alpha_1}\cdots\partial_{y_r}^{\alpha_r}\sub\mathscr{D}_Y.\]
Since $[\partial_{y_i},\partial_{y_j}]=0$, $\mathscr{D}$ is a subring of $\mathscr{D}_Y$, and we have $\mathscr{D}_Y\cong\mathscr{D}\otimes_{\C}\C[\partial_{y_{r+1}},\dots,\partial_{y_n}]$ as a left $\mathscr{D}$-module. We therefore conclude that
\[\mathscr{D}_{X\to Y}\cong \mathscr{O}_X\otimes_{i^{-1}(\mathscr{O}_Y)}i^{-1}(\mathscr{D})\otimes_{\C}\C[\partial_{y_{r+1}},\dots,\partial_{y_n}].\]
On the other hand, it is easily seen that $\mathscr{D}_X$ is isomorphic to the submodule $\mathscr{O}_X\otimes_{i^{-1}(\mathscr{O}_Y)}i^{-1}(\mathscr{D})$ of $\mathscr{D}_{X\to Y}$, so we conclude that
\begin{equation}\label{D-module closed immersion relative D-module-1}
\mathscr{D}_{X\to Y}\cong\mathscr{D}_X\otimes_{\C}\C[\partial_{y_{r+1}},\dots,\partial_{y_n}].
\end{equation}
as a left $\mathscr{D}_X$-module. In particular, $\mathscr{D}_{X\to Y}$ is a locally free $\mathscr{D}_X$-module of infinite rank (unless $r=n$).
\end{example}

\begin{proposition}\label{D-module smooth morphism relative D-module}
Let $f:X\to Y$ be a smooth morphism. Then $\mathscr{D}_{X\to Y}$ is generated by $1_{X\to Y}$ as a left $\mathscr{D}_X$-module.
\end{proposition}
\begin{proof}
We can locally write $f(x,y)=y$ in appropriate coordinate systems $y=(y_1,\dots,y_m)$ of $Y$ and $(x,y)=(x_1,\dots,x_n,y_1,\dots,y_m)$ of $X$. It then follows from (\ref{D-module inverse image def-4}) that
\[\partial_{x_i}(a\otimes P)=0,\quad \partial_{y_i}(a\otimes P)=a\otimes\partial_{y_i}P\for a\in\mathscr{O}_X,P\in\mathscr{D}_Y.\]
In particular, we see that $1_{X\to Y}\partial_{y_i}=\partial_{y_i}1_{X\to Y}$, so $\mathscr{D}_{X\to Y}$ is generated by $1_{X\to Y}$ as a left $\mathscr{D}_X$-module.
\end{proof}

Let $f:X\to Y$ be a smooth morphism of smooth algebraic varieties and $\Omega_{X/Y}^\bullet$ denote the sheaf of differential forms relative to $f$, that is,
\begin{align}
\Omega_{X/Y}^1&=\coker(f^*\Omega_Y^1\to\Omega_X^1),\label{D-module relative differential form def-1} \\
\Omega_{X/Y}^p&=\bigw^p\Omega_{X/Y}^1=\Omega_X^p/\big(\im(f^*(\Omega_Y^1)\to \Omega_X^1)\wedge \Omega_X^{p-1}\big).\label{D-module relative differential form def-2}
\end{align}
We then have the relative de Rham complex
\[\begin{tikzcd}
\Omega_{X/Y}^\bullet:\cdots\ar[r]&0\ar[r]&\Omega_{X/Y}^0\ar[r,"d^0_{X/Y}"]&\Omega_{X/Y}^1\ar[r]&\cdots\ar[r]&\Omega_{X/Y}^n\ar[r]&0\ar[r]&\cdots 
\end{tikzcd}
\]
($n$ is the relative dimension of $f$.) Since the differential $d_{X/Y}$ are differential homomorphisms of degree $1$, we obtain a complex of right $\mathscr{D}_X$-modules
\[\begin{tikzcd}[column sep=6mm]
\Omega_{X/Y}^\bullet\otimes_{\mathscr{O}_X}\mathscr{D}_X:\cdots\ar[r]&0\ar[r]&\Omega_{X/Y}^0\otimes_{\mathscr{O}_X}\mathscr{D}_X\ar[r]&\cdots\ar[r]&\Omega_{X/Y}^n\otimes_{\mathscr{O}_X}\mathscr{D}_X\ar[r]&0\ar[r]&\cdots 
\end{tikzcd}
\]
by \cref{D-module differential module char by Hom of tensor}. Applying $\sHom_{\mathscr{D}_X}(-,\mathscr{D}_X)$ to this complex, we get
\[\begin{tikzcd}[column sep=6mm]
\mathscr{D}_X\otimes_{\mathscr{O}_X}\bigw^\bullet\Theta_{X/Y}:\cdots\ar[r]&\mathscr{D}_X\otimes_{\mathscr{O}_X}\bigw^n\Theta_{X/Y}\ar[r]&\cdots\ar[r]&\mathscr{D}_X\otimes_{\mathscr{O}_X}\bigw^1\Theta_{X/Y}\ar[r]&\cdots 
\end{tikzcd}
\]
where 
\[\Theta_{X/Y}=\sHom_{\mathscr{O}_X}(\Omega_{X/Y}^1,\mathscr{O}_X)=\ker(\Theta_X\to f^*(\Theta_Y)).\]

\begin{proposition}\label{D-module relative D-module resolution}
Let $f:X\to Y$ be a smooth morphism of relative dimension $n$ between smooth algebraic varieties. Then the following sequence is exact 
\begin{equation}\label{D-module relative D-module resolution-1}
\begin{tikzcd}[column sep=6mm]
0\ar[r]&\mathscr{D}_X\otimes_{\mathscr{O}_X}\bigw^n\Theta_{X/Y}\ar[r]&\cdots\ar[r]&\mathscr{D}_X\otimes_{\mathscr{O}_X}\bigw^0\Theta_{X/Y}\ar[r]&\mathscr{D}_{X\to Y}\ar[r]&0
\end{tikzcd}
\end{equation}
\end{proposition}
\begin{proof}
We define a good filtration of $\mathscr{D}_X\otimes_{\mathscr{O}_X}\bigw^k\Theta_{X/Y}$ by
\[F_i(\mathscr{D}_X\otimes_{\mathscr{O}_X}\bigw^k\Theta_{X/Y})=F_{i-k}(\mathscr{D}_X)\otimes_{\mathscr{O}_X}\bigw^k\Theta_{X/Y},\]
and equip $\mathscr{D}_{X\to Y}$ with the quotient filtration from $\mathscr{D}_X$. Then (\ref{D-module relative D-module resolution-1}) is a complex of filtered modules. Taking associated graded modules, we obtain a complex of $\gr(\mathscr{D}_X)$-modules
\begin{equation}\label{D-module relative D-module resolution-2}
\begin{tikzcd}[column sep=6mm]
0\ar[r]&\gr(\mathscr{D}_X\otimes_{\mathscr{O}_X}\bigw^n\Theta_{X/Y})\ar[r]&\cdots\ar[r]&\gr(\mathscr{D}_X\otimes_{\mathscr{O}_X}\bigw^0\Theta_{X/Y})\ar[r]&\gr(\mathscr{D}_{X\to Y})\ar[r]&0
\end{tikzcd}
\end{equation}
If we write locally $\Theta_{X/Y}=\bigoplus_{i=1}^{n}\mathscr{O}_Xv_i$, then (\ref{D-module relative D-module resolution-2}) is the Koszul complex with respect to $(v_1,\dots,v_n)$, which is exact since the codimension of the zero set of $\{v_1,\dots,v_n\}$ in $T^*X$ equals to $n$. We then conclude that (\ref{D-module relative D-module resolution-1}) is exact.
\end{proof}

The language of derived categories is most suitable for the systematic study of inverse images of $D$-modules. Let $f:X\to Y$ be a morphism of smooth algebraic varieties, then the functor
\[f^*=\mathscr{O}_X\otimes_{f^{-1}(\mathscr{O}_Y)}f^{-1}(-):\Mod(\mathscr{D}_Y)\to\Mod(\mathscr{D}_X)\]
is a right exact functor, hence extends, via left derivation, to a triangulated functor
\[Lf^*:D^-(\mathscr{D}_Y)\to D^-(\mathscr{D}_X).\]
Since the flat dimension of $\mathscr{O}_Y$ is finite, this functor sends $D^b(\mathscr{D}_Y)$ to $D^b(\mathscr{D}_X)$, and hence induces a functor
\[Lf^*:D^b(\mathscr{D}_Y)\to D^b(\mathscr{D}_X).\]
For a bounded complex $\mathscr{N}$ of $\mathscr{D}_Y$-modules, by taking a bounded flat resolution $\mathscr{L}\to\mathscr{N}$, we can express $Lf^*(\mathscr{N})$ as $f^*(\mathscr{L})$.

\begin{proposition}\label{D-module inverse image of composition prop}
Let $f:X\to Y$ and $g:Y\to Z$ be morphisms of smooth algebraic varieties, then
\[Lf^*\circ Lg^*\cong L(g\circ f)^*.\]
\end{proposition}
\begin{proof}
We first note that
\begin{align*}
Lf^*(\mathscr{D}_{Y\to Z})&=\mathscr{O}_X\otimes_{f^{-1}(\mathscr{O}_Y)}^Lf^{-1}(\mathscr{D}_{Y\to Z})\\
&=\mathscr{O}_X\otimes_{f^{-1}(\mathscr{O}_Y)}^Lf^{-1}(\mathscr{O}_Y\otimes_{g^{-1}(\mathscr{O}_Z)}g^{-1}(\mathscr{D}_Z))\\
&=\mathscr{O}_X\otimes_{f^{-1}(\mathscr{O}_Y)}^Lf^{-1}(\mathscr{O}_Y)\otimes_{(g\circ f)^{-1}(\mathscr{O}_Z)}^L(g\circ f)^{-1}(\mathscr{D}_Z)\\
&=\mathscr{O}_X\otimes_{(g\circ f)^{-1}(\mathscr{O}_Z)}^L(g\circ f)^{-1}(\mathscr{D}_Z)=\mathscr{D}_{X\to Z}
\end{align*}
where we have used the fact that $\mathscr{D}_Z$ is a locally free $\mathscr{O}_Z$-module. We thus obtain isomorphisms
\[\mathscr{D}_{X\to Z}\cong\mathscr{D}_{X\to Y}\otimes_{f^{-1}(\mathscr{D}_Y)}^Lf^{-1}(\mathscr{D}_{Y\to Z})\cong \mathscr{D}_{X\to Y}\otimes_{f^{-1}(\mathscr{D}_Y)}^Lf^{-1}(\mathscr{D}_{Y\to Z})\]
and therefore
\begin{align*}
L(g\circ f)^*(\mathscr{M})&=\mathscr{D}_{X\to Z}\otimes_{(g\circ f)^{-1}(\mathscr{D}_Y)}^L(g\circ f)^{-1}(\mathscr{M})\\
&\cong (\mathscr{D}_{X\to Y}\otimes_{f^{-1}(\mathscr{D}_Y)}f^{-1}(\mathscr{D}_{Y\to Z}))\otimes_{f^{-1}(g^{-1}(\mathscr{D}_Y))}^Lf^{-1}(g^{-1}(\mathscr{M}))\\
&\cong \mathscr{D}_{X\to Y}\otimes_{f^{-1}(\mathscr{D}_Y)}^Lf^{-1}(\mathscr{D}_{Y\to Z}\otimes_{g^{-1}(\mathscr{D}_Y)}^Lg^{-1}(\mathscr{M}))\cong Lf^*(Lg^*(\mathscr{M})),
\end{align*}
whence our claim.
\end{proof}

\begin{proposition}\label{D-module inverse image of qcoh is qcoh}
Let $f:X\to Y$ be a morphism of smooth algebraic varieties, then $Lf^*$ sends $D_{\qcoh}^b(\mathscr{D}_Y)$ to $D_{\qcoh}^b(\mathscr{D}_X)$.
\end{proposition}
\begin{proof}
Let $\mathscr{M}\in D_{\qcoh}^b(\mathscr{D}_Y)$, then as a complex of $\mathscr{O}_X$-modules, we have
\begin{align*}
Lf^*(\mathscr{M})=\mathscr{O}_X\otimes_{f^{-1}(\mathscr{O}_Y)}^Lf^{-1}(\mathscr{M}).
\end{align*}
The assertion then follows from the corresponding result for the derived tensor product on the category $\Qcoh(\mathscr{O}_X)$.
\end{proof}

\begin{remark}
We note that $Lf^*(\mathscr{D}_Y)=\mathscr{D}_{X\to Y}\otimes_{f^{-1}(\mathscr{D}_Y)}^Lf^{-1}(\mathscr{D}_Y)=\mathscr{D}_{X\to Y}$. If $f$ is a closed immersion with $\dim(X)<\dim(Y)$, then the $\mathscr{D}_X$-module $\mathscr{D}_{X\to Y}$ is locally of infinite rank (cf. \cref{D-module closed immersion relative D-module}). Therefore, the functor $Lf^*$ does not necessarily send $D_{\coh}^b(\mathscr{D}_Y)$ to $D_{\coh}^b(\mathscr{D}_X)$.
\end{remark}

\begin{proposition}\label{D-module derived inverse image of smooth morphism prop}
Let $f:X\to Y$ be a smooth morphism between smooth algebraic varieties.
\begin{enumerate}
    \item[(a)] If $\mathscr{M}$ is a left $\mathscr{D}_Y$-module, then $L^if^*(\mathscr{M})=0$ for $i\neq 0$ (hence we can write $f^*$ instead of $Lf^*$).
    \item[(b)] If $\mathscr{M}$ is a coherent $\mathscr{D}_Y$-module, then $f^*(\mathscr{M})$ is a coherent $\mathscr{D}_X$-module.
\end{enumerate}
\end{proposition}
\begin{proof}
The first assertion follows from the flatness of $\mathscr{O}_X$ over $f^{-1}(\mathscr{O}_Y)$, since $f$ is flat. To see that $f^*(\mathscr{M})$ is a coherent $\mathscr{D}_X$-module if $\mathscr{M}$ is coherent, we use \cref{D-module coherent D_X-module iff qcoh and ft}, so let $\mathscr{D}_X^{\oplus m}\to\mathscr{M}$ be a surjective homomorphism. Then by applying $f^*$, we obtain a surjective homomorphism $\mathscr{D}_{X\to Y}^{\oplus m}\to f^*(\mathscr{M})$, and it suffices to note that there is a surjective homomorphism $\mathscr{D}\to\mathscr{D}_{X\to Y}$ given by $P\mapsto P1_{X\to Y}$ (cf. \cref{D-module smooth morphism relative D-module}).
\end{proof}

\subsection{External tensor product}
Let $X$ and $Y$ be two smooth algebraic varieties and $\pr_1:X\times Y\to X$, $\pr_2:X\times Y\to Y$ be the canonical projections. For an $\mathscr{O}_X$-module $\mathscr{F}$ and an $\mathscr{O}_Y$-module $\mathscr{G}$, we can define an $\mathscr{O}_{X\times Y}$-module $\mathscr{F}\boxtimes\mathscr{G}$ by
\begin{align*}
\mathscr{F}\boxtimes\mathscr{G}&:=(\mathscr{O}_{X\times Y}\otimes_{\pr_1^{-1}(\mathscr{O}_X)}\mathscr{F})\otimes_{\pr_2^{-1}(\mathscr{O}_Y)}\pr_2^{-1}(\mathscr{G})\\
&=\mathscr{O}_{X\times Y}\otimes_{\pr_1^{-1}(\mathscr{O}_X)\otimes_\C \pr_2^{-1}(\mathscr{O}_Y)}(\pr_1^{-1}(\mathscr{F})\otimes_{\C}\pr_2^{-1}(\mathscr{G})).
\end{align*}
It is well known that this functor is exact with respect to $\mathscr{F}$ and $\mathscr{G}$, hence extends to a functor
\[(-)\boxtimes(-):D^b(\mathscr{O}_X)\times D^b(\mathscr{O}_Y)\to D^b(\mathscr{O}_{X\times Y}).\]
Furthermore, we have $\supp(\mathscr{F}\boxtimes\mathscr{G})=\supp(\mathscr{F})\times\supp(\mathscr{G})$.\par

We note that the external tensor product of $\mathscr{D}_X$ and $\mathscr{D}_Y$ is given by
\[\mathscr{D}_X\boxtimes\mathscr{D}_Y=\mathscr{O}_{X\times Y}\otimes_{\pr_1^{-1}(\mathscr{O}_X)\otimes_{\C}\pr_2^{-1}(\mathscr{O}_Y)}(\pr_1^{-1}(\mathscr{D}_X)\otimes_{\C}\pr_2^{-1}(\mathscr{D}_Y))\cong\mathscr{D}_{X\times Y},\]
so for a $\mathscr{D}_X$-module $\mathscr{M}$ and a $\mathscr{D}_Y$-module $\mathscr{N}$, we have
\begin{align*}
\mathscr{M}\boxtimes\mathscr{N}&=\mathscr{O}_{X\times Y}\otimes_{\pr_1^{-1}(\mathscr{O}_X)\otimes_{\C}\pr_2^{-1}(\mathscr{O}_Y)}(\pr_1^{-1}(\mathscr{M})\otimes_\C\pr_2^{-1}(\mathscr{N}))\\
&\cong \mathscr{D}_{X\times Y}\otimes_{\pr_1^{-1}(\mathscr{D}_X)\otimes_{\C}\pr_2^{-1}(\mathscr{D}_Y)}(\pr_1^{-1}(\mathscr{M})\otimes_\C\pr_2^{-1}(\mathscr{N})).
\end{align*}
This means the $\mathscr{O}_X$-module $\mathscr{M}\boxtimes\mathscr{N}$ is canoncially endowed with a left $\mathscr{D}_{X\times Y}$-module structure, so we obtain a bifunctor
\[(-)\boxtimes(-):D^b(\mathscr{D}_X)\times D^b(\mathscr{D}_Y)\to D^b(\mathscr{D}_{X\times Y})\]
for derived categories such that the following diagram is commutative:
\[\begin{tikzcd}[column sep=15mm]
D^b(\mathscr{D}_X)\otimes D^b(\mathscr{D}_Y)\ar[d]\ar[r,"(-)\boxtimes(-)"]&D^b(\mathscr{D}_{X\times Y})\ar[d]\\
D^b(\mathscr{O}_X)\otimes D^b(\mathscr{O}_Y)\ar[r,"(-)\boxtimes(-)"]&D^b(\mathscr{O}_{X\times Y})
\end{tikzcd}\]
It is easily seen that the functor $(-)\boxtimes(-)$ sends $D^b_{\qcoh}(\mathscr{D}_X)\times D^b_{\qcoh}(\mathscr{D}_Y)$ (resp $D^b_{\coh}(\mathscr{D}_X)\times D^b_{\coh}(\mathscr{D}_Y)$) to $D^b_{\qcoh}(\mathscr{D}_{X\times Y})$ (resp. $D^b_{\coh}(\mathscr{D}_{X\times Y})$), and we have
\[\pr_1^*(\mathscr{M})\cong\mathscr{M}\boxtimes\mathscr{O}_Y,\quad \pr_2^*(\mathscr{N})\cong \mathscr{O}_X\boxtimes\mathscr{N}.\]

Let $X$ be a smooth algebraic variety and $\Delta_X:X\to X\times X$ be the diagonal morphism. For $\mathscr{M},\mathscr{N}\in\Mod(\mathscr{D}_X)$, we easily seen that $\mathscr{M}\otimes_{\mathscr{O}_X}\mathscr{N}$ is isomorphic to $\Delta_X^*(\mathscr{M}\boxtimes\mathscr{N})$ as a $\mathscr{D}_X$-module. Moreover, the external tensor products of flat modules are flat, so for $\mathscr{M},\mathscr{N}\in D^b(\mathscr{D}_X)$ we have a canonical isomorphism
\[\mathscr{M}\otimes_{\mathscr{O}_X}^L\mathscr{N}\cong L\Delta_X^*(\mathscr{M}\boxtimes\mathscr{N}).\]

\begin{proposition}\label{D-module inverse image and tensor prop}
Let $f:X\to Y$ and $f':X'\to Y'$ be morphisms of smooth algebraic varieties. Then for $\mathscr{M},\mathscr{N}\in D^b(\mathscr{D}_Y)$ and $\mathscr{M}'\in D^b(\mathscr{D}_{Y'})$, we have
\begin{gather*}
L(f\times f')^*(\mathscr{M}\boxtimes\mathscr{M}')\cong Lf^*(\mathscr{M})\boxtimes Lf'^*(\mathscr{M}'),\label{D-module inverse image and tensor prop-1}\\
Lf^*(\mathscr{M}\otimes_{\mathscr{O}_Y}\mathscr{N})\cong Lf^*(\mathscr{M})\otimes_{\mathscr{O}_X}^LLf^*(\mathscr{N}).\label{D-module inverse image and tensor prop-2}
\end{gather*}
\end{proposition}
\begin{proof}
The first statement follows from $(f\times f')^*(\mathscr{M}\boxtimes\mathscr{N}')\cong f^*(\mathscr{M})\boxtimes f'^*(\mathscr{M}')$ for $\mathscr{M}\in\Mod(\mathscr{D}_Y)$, $\mathscr{M}'\in\Mod(\mathscr{D}_Y)$. The second one then follows as follows:
\begin{equation*}
\begin{aligned}
Lf^*(\mathscr{M}\otimes_{\mathscr{O}_Y}\mathscr{N})&\cong Lf^*L\Delta_Y^*(\mathscr{M}\boxtimes\mathscr{N})\cong L\Delta_X^*L(f\times f)^*(\mathscr{M}\boxtimes\mathscr{N})\\
&\cong L\Delta_X^*(Lf^*(\mathscr{M})\boxtimes Lf^*(\mathscr{N}))\cong Lf^*(\mathscr{M})\otimes_{\mathscr{O}_X}^LLf^*(\mathscr{N}).
\end{aligned}
\end{equation*}
where we have used the equality $(f,f)\circ\Delta_X=\Delta_Y\circ f$.
\end{proof}

\begin{proposition}\label{D-module derived tensor and base change}
Let $\mathscr{M},\mathscr{N}\in D^b(\mathscr{D}_X)$ and $\mathscr{L}\in D^b(\mathscr{D}_X^{\op})$. Then we have canonical isomorphisms of $\C_X$-modules
\[(\mathscr{L}\otimes_{\mathscr{O}_X}^L\mathscr{N})\otimes_{\mathscr{D}_X}^L\mathscr{M}\cong\mathscr{L}\otimes_{\mathscr{D}_X}^L(\mathscr{M}\otimes_{\mathscr{O}_X}^L\mathscr{N})\]
\end{proposition}
\begin{proof}
By taking flat resolutions, we may assume that $\mathscr{M},\mathscr{N}\in\Mod(\mathscr{D}_X)$ and $\mathscr{L}\in\Mod(\mathscr{D}_X^{\op})$. The assertion then follows from \cref{D-module tensor with Kronecker tensor isomorphism}.
\end{proof}

\begin{proposition}\label{D-module characteristic variety of external tensor char}
For a coherent $\mathscr{D}_X$-module $\mathscr{M}$ and a coherent $\mathscr{D}_Y$-module $\mathscr{N}$, we have
\[\Ch(\mathscr{M}\boxtimes\mathscr{N})=\Ch(\mathscr{M})\times\Ch(\mathscr{N}).\]
\end{proposition}
\begin{proof}
Let $F(\mathscr{M})$ and $F(\mathscr{N})$ be good filtrations of $\mathscr{M}$ and $\mathscr{N}$, respectively. We define a good filtration of $\mathscr{M}\boxtimes\mathscr{N}$ by
\[F_k(\mathscr{M}\boxtimes\mathscr{N})=\sum_{i+j=k}F_i(\mathscr{M})\boxtimes F_j(\mathscr{N}).\]
Then we have $\gr(\mathscr{M}\boxtimes\mathscr{N})=\gr(\mathscr{M})\boxtimes\gr(\mathscr{N})$, hence
\[\widetilde{\gr(\mathscr{M}\boxtimes\mathscr{N})}=\widetilde{\gr(\mathscr{M})}\boxtimes\widetilde{\gr(\mathscr{N})}.\]
This proves the assertion, as we have remarked.
\end{proof}

\subsection{Direct image of \texorpdfstring{$D$}{D}-modules}
Suppose that $v(y)=\int u(x,y)dx$ makes sense for a function $u(x,y)$, let us consider how to derive differential equations for $v(y)$ from those for $u(x,y)$. Supposing, in addtion, that Stokes's theorem $\int\frac{\partial u(x,y)}{\partial x_i}dx=0$ holds, we obtain
\[\int\frac{\partial}{\partial x_i}S(x,y,\partial_x,\partial_y)u(x,y)=0\]
for all differential operators $S(x,y,\partial_x,\partial_y)$, so for
\[Q(y,\partial_y)=\sum_i\partial_{x_i}S_i(x,y,\partial_x,\partial_y)+P(x,y,\partial_x,\partial_y),\]
we have
\[Q(y,\partial_y)v(y)=\int P(x,y,\partial_x,\partial_y)u(x,y)dx.\]
Furthermore, if $P(x,y,\partial_x,\partial_y)u=0$, then $Q(y,\partial_y)v(y)=0$.\par
We now describe the above consideration in the languate of $D$-modules. Let $X$ be a smooth algebraic variety with a coordinate system $(x,y)$ and $Y$ be a submanifold with a coordinate system $y$. Then the above consideration means that we can associate a $\mathscr{D}_X$-module $\mathscr{M}$ with a $\mathscr{D}_Y$-module
\[\mathscr{M}/(\sum_i\partial_{x_i}\mathscr{M})=(\mathscr{D}_X/\sum_i\partial_{x_i}\mathscr{D}_X)\otimes_{\mathscr{D}_X}\mathscr{M}.\]
This can be generalized to an arbitrary morphism, which is the subject of this paragraph.\par

Let $f:X\to Y$ be a morphisms of smooth algebraic varieties. The right $\mathscr{D}_Y$-module structure of $\mathscr{D}_Y$ gives via side-changing a left $\mathscr{D}_Y$-module structure on $\mathscr{D}_Y\otimes_{\mathscr{O}_Y}\Omega_Y^{\otimes-1}$, whose inverse image by $f$ is the left $\mathscr{D}_X$-module
\[f^{-1}(\mathscr{D}_Y\otimes_{\mathscr{O}_Y}\Omega_Y^{\otimes-1})=f^{-1}(\mathscr{D}_Y\otimes_{\mathscr{O}_Y}\Omega_Y^{\otimes-1})\otimes_{f^{-1}(\mathscr{O}_Y)}\mathscr{O}_X.\]
By side-changing again, we then obtain a right $\mathscr{D}_X$-module
\begin{align*}
f^{-1}(\mathscr{D}_Y\otimes_{\mathscr{O}_Y}\Omega_Y^{\otimes-1})\otimes_{f^{-1}(\mathscr{O}_Y)}\Omega_X&=f^{-1}(\mathscr{D}_Y)\otimes_{f^{-1}(\mathscr{O}_Y)}f^{-1}(\Omega_Y^{\otimes-1})\otimes_{f^{-1}(\mathscr{O}_Y)}\Omega_X\\
&\cong f^{-1}(\mathscr{D}_Y)\otimes_{f^{-1}(\mathscr{O}_Y)}\Omega_{X/Y}
\end{align*}
where we write $\Omega_{X/Y}=\Omega_X\otimes_{f^{-1}(\mathscr{O}_Y)}f^{-1}(\Omega_Y^{\otimes-1})$. Since the right and left action of $\mathscr{D}_Y$ commute, we see that $f^{-1}(\mathscr{D}_Y)\otimes_{f^{-1}(\mathscr{O}_Y)}\Omega_{X/Y}$ is an $(f^{-1}(\mathscr{D}_Y),\mathscr{D}_X)$-bimodule, which is denoted by $\mathscr{D}_{Y\leftarrow X}$.\par

Recall tht we have defined a $(\mathscr{D}_X,f^{-1}(\mathscr{D}_Y))$-bimodule $\mathscr{D}_{X\to Y}=\mathscr{O}_X\otimes_{f^{-1}(\mathscr{O}_Y)}f^{-1}(\mathscr{D}_Y)$. Since the category of left $\mathscr{D}$-modules and that of right $\mathscr{D}_X$-modules are equivalent, we can switch $\mathscr{D}_{X\to Y}$ to a $(f^{-1}(\mathscr{D}_Y),\mathscr{D}_X)$-bimodule, which is explicitly given by
\begin{align*}
\Omega_X\otimes_{\mathscr{O}_X}\mathscr{D}_{X\to Y}\otimes_{f^{-1}(\mathscr{O}_Y)}f^{-1}(\Omega_Y^{\otimes-1})&=\Omega_X\otimes_{f^{-1}(\mathscr{O}_Y)}f^{-1}(\mathscr{D}_Y)\otimes_{f^{-1}(\mathscr{O}_Y)}f^{-1}(\Omega_Y^{\otimes-1})\\
&\cong f^{-1}(\mathscr{D}_Y)\otimes_{f^{-1}(\mathscr{O}_Y)}\Omega_{X/Y}\cong\mathscr{D}_{X\leftarrow Y}.
\end{align*}
In other words, the bimodule $\mathscr{D}_{Y\leftarrow X}$ is obtained by side-changing from $\mathscr{D}_{X\to Y}$.\par 

Let us explicitly describe the action of $\mathscr{D}_X$ on $\mathscr{D}_{Y\leftarrow X}$. The action of 