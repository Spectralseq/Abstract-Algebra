\chapter{The theory of topos and \'etale cohomology of schemes}
\section{Presheaves of sets}\label{category presheaf section}
In this section, we consider the category of presheaves of sets over a category $\mathcal{C}$, and prove some of its properties. In order to avoid set-theoretic issues, we fix once for all a universe $\mathscr{U}$ which has an element with infinite cardinality. A set is said to be \textbf{$\mathscr{U}$-small} (or simply \textbf{small} if there is no confusion) if it is isomorphic to an element of $\mathscr{U}$. We also use the following terminology: small group, small ring, small category. We often assume that the schemes, topological spaces, sets of indices, with which we work are $\mathscr{U}$-small, or at least have cardinality belonging to $\mathscr{U}$. A category $\mathcal{C}$ is called a \textbf{$\mathscr{U}$-category} if for any objects $x,y$ in $\mathcal{C}$, the set $\Hom_\mathcal{C}(x,y)$ is $\mathscr{U}$-small, and is called $\mathscr{U}$-small if the set $\Ob(\mathcal{D})$ is also contained in the universe $\mathscr{U}$. For two categories $\mathcal{C}$, $\mathcal{D}$, we denote by $\sHom(\mathcal{C},\mathcal{D})$ the category of (covariant) functors from $\mathcal{C}$ to $\mathcal{D}$. It is then easy to verify the following two conditions:
\begin{itemize}
\item If $\mathcal{C}$ and $\mathcal{D}$ are elements of $\mathscr{U}$ (resp. $\mathscr{U}$-small), then $\sHom(\mathcal{C},\mathcal{D})$ is an element of $\mathscr{U}$ (resp. $\mathscr{U}$-small).
\item If $\mathcal{C}$ is a $\mathscr{U}$-small category and $\mathcal{D}$ is a $\mathscr{U}$-category, $\sHom(\mathcal{C},\mathcal{D})$ is a $\mathscr{U}$-category.
\end{itemize}
However, note that if $\mathcal{D}$ is a $\mathscr{U}$-small category and $\mathcal{C}$ is a $\mathscr{U}$-category, then $\sHom(\mathcal{C},\mathcal{D})$ is not $\mathscr{U}$-small in general. For example, the category $\sHom(\mathcal{C},\mathscr{U}\text{-}\mathbf{Set})$. It should be noted that $\mathscr{U}$-smallness is really a restrictive condition for categories, and there are many interesting examples where this condition is not satisfied in general.
\subsection{The category of presheaves of sets}
Let $\mathcal{C}$ be a category. We define the \textbf{category of presheaves of sets over $\mathcal{C}$ relative to the universe $\mathscr{U}$} (or, if there is no confusion, the category of presheaves of sets over $\mathcal{C}$) to be the category of contravariant functors from $\mathcal{C}$ to the category of $\mathscr{U}$-sets, and denote it by $\PSh(\mathcal{C})_{\mathscr{U}}$ (or simply $\PSh(\mathcal{C})$ if there is no risk of confusion). The objects of $\PSh(\mathcal{C})_{\mathscr{U}}$ are called \textbf{$\mathscr{U}$-presheaves} (of simply presheaves) over $\mathcal{C}$. If $\mathcal{C}$ is $\mathscr{U}$-small, then $\PSh(\mathcal{C})_{\mathscr{U}}$ is a $\mathscr{U}$-category. However, this is not true in general if $\mathcal{C}$ is only assumed to be a $\mathscr{U}$-category.\par
Let $x$ be an object of a $\mathscr{U}$-category $\mathcal{C}$. We can associate with $x$ a presheaf $h_x:\mathcal{C}^{\op}\to\mathscr{U}\text{-}\mathbf{Set}$, defined in the following way:
\begin{itemize}
\item If $\Hom_\mathcal{C}(y,x)$ is an element of $\mathscr{U}$, then we set $h_x(y)=\Hom_\mathcal{C}(y,x)$.
\item Suppose that $\Hom_\mathcal{C}(y,x)$ is not an element of $\mathscr{U}$ and let $R(Z)$ be the relation "the set $Z$ is the target of an isomorphism $\Hom_\mathcal{C}(y,x)\stackrel{\sim}{\to}Z$". We then put $h_x(y)=\tau_Z(R(Z))$.
\end{itemize}
Let $R'(u)$ be the relation "$u$ is a bijection from $\Hom_\mathcal{C}(y,x)$ to $h_x(y)$" and set $\varphi(y,x)=\tau_u(R'(u))$. Then in both cases, we have a canonical isomorphism
\[\varphi(y,x):\Hom_\mathcal{C}(y,x)\stackrel{\sim}{\to}h_y(x).\]
Now let $u:y\to y'$ be a morphism of $\mathcal{C}$. Then by composition, $u$ defines a map
\[\Hom_{\mathcal{C}}(u,x):\Hom_\mathcal{C}(y',x)\to\Hom_\mathcal{C}(y,x)\]
and we define $h_x(u)$ to be the composition
\[h_x(u)=\varphi(y,x)\Hom_\mathcal{C}(x,u)\varphi(y,x)^{-1}.\]
It is immediate to verify that $h_x$ then defines a functor $\mathcal{C}^{\op}\to\mathscr{U}\text{-}\mathbf{Set}$.
\subsection{Projective limits and inductive limits}
\subsection{Exactness properties of the category of presheaves}
\section{Grothendieck topologies and sheaves}
\subsection{Topologies and basis}
Let $\mathcal{C}$ be a category. A full subcategory $\mathcal{D}$ of $\mathcal{C}$ is called a \textbf{sieve} of $\mathcal{C}$ if it satisfies the following property: any object of $\mathcal{C}$ such that there exists a morphism from this object to an object of $\mathcal{D}$ is in $\mathcal{D}$. If $X$ is an object of $\mathcal{C}$, then the sieves of the category $\mathcal{C}_{/X}$ are also called \textbf{sieves of $\bm{X}$}.\par
Let $\mathscr{U}$ be a universe such that $\mathcal{C}$ is a $\mathscr{U}$-category, and $\PSh(\mathcal{C})$ be the corresponding category of presheaves. For any sieve of $X$, we can define a sub-object of $X$ in $\PSh(\mathcal{C})$ by associating any object $Y$ in $\mathcal{C}$ with the set of morphisms $f:Y\to X$ such that the object $(Y,f)$ of $\mathcal{C}_{/X}$ belongs to the sieve. In this way, we obtain a correspondence between sieves and sub-objects:
\begin{proposition}\label{category sieve and subobject correspondence}
The map defined above is a correspondence between the set of sieves of $X$ and the set of sub-objects of $X$ in $\PSh(\mathcal{C})$.
\end{proposition}
\begin{proof}
It suffices to establish a converse process from sub-objects of $X$ in $\PSh(\mathcal{C})$ to sieves of $X$. For this, we associate each sub-functor $R$ of $X$ with the category $\mathcal{C}_{/R}$ of objects lying over $R$. Since $R\sub X$, it is immediate that $\mathcal{C}_{/R}$ is a sieve of $\mathcal{C}{/X}$, whose corresponding sub-object of $X$ is $R$.
\end{proof}
Let $\mathcal{C}$ be a $\mathscr{U}$-category. In view of {category sieve and sub-object correspondence}, we shall identify sieves of $X$ with sub-objects of $X$ in the category $\PSh(\mathcal{C})$. In this way, for any presheaf $F$ and any sieve $R$ of $X$, we can then define $\Hom_{\PSh(\mathcal{C})}(R,F)$ to be the set of morphisms of functors $R\to F$, and we have a functorial isomorphism (SGA4, \Rmnum{1}, 3.5)
\begin{align}\label{category Hom of functors char by limit}
\Hom_{\PSh(\mathcal{C})}(R,F)\stackrel{\sim}{\to}\llim_{Y\in\mathcal{C}_{/R}}F(Y).
\end{align}
Therefore, \cref{category sieve and subobject correspondence} allows us to transport the usual operations on functors to sieves; here are a few examples.
\begin{itemize}
\item \textbf{Base change}. Let $R$ be a sieve of $X$ and $f:Y\to X$ be a morphism in $\mathcal{C}$. The fiber product $R\times_XY$ is then a sieve of $Y$, which is called the \textbf{base change} of $R$ to $Y$. The corresponding subcategory of $\mathcal{C}_{/Y}$ is then the inverse image of the subcategories of $\mathcal{C}_{/X}$ defined by $R$ under the morphism $f$.
\item \textbf{Order relation, intersection, union}. The inclusion of sub-functors of $X$ is an order relation, which then defines an order relation on sieves of $X$. We define the union and intersection of a family of sieves as the supremum and infimum of the corresponding family of sub-presheaves.
\item \textbf{Image, generated sieve}. Let $(F_\alpha)_{\alpha\in I}$ be a family of presheaves and $f_\alpha:F_\alpha\to X$ be a morphism for each $\alpha$, where $X$ is an object of $\mathcal{C}$. We define the image of this family of morphisms to be the union of the images of $f_\alpha$, which is then a sieve of $X$. In particular, if the $F_\alpha$ are objects of $\mathcal{C}$, the image sieve is called the \textbf{sieve generated by the morphisms $\bm{f_\alpha}$}. The corresponding subcategory of $\mathcal{C}_{/X}$ is by definition the full subcategory of $\mathcal{C}_{/X}$ formed by objects $Y\to X$ such that there exists an $X$-morphism of $Y$ into one of the $F_\alpha$.
\end{itemize}

The use of sieves allows us to define a topology on a category $\mathcal{C}$. Roughly speaking, this amounts to associate for each object $X$ of $\mathcal{C}$ a class of sieves of $X$. A sieve of $X$ in this class can then be considered as a \textit{covering} of $X$, or a way to "localize" $X$.
\begin{definition}
Let $\mathcal{C}$ be a category. A \textbf{topology} $\mathcal{T}$ on $\mathcal{C}$ is the assignment to each object $X$ of $\mathcal{C}$ of a set $\mathcal{T}(X)$ of sieves of $X$, so that the following conditions are satisfied:
\begin{enumerate}[leftmargin=40pt]
\item[(T1)] (\textbf{Stable under base change}). For any object $X$ of $\mathcal{C}$, any sieve $R\in\mathcal{T}(X)$ and any morphism $f:X\to Y$ in $\mathcal{C}$, the sieve $R\times_XY$ of $Y$ belongs to $\mathcal{T}(Y)$.
\item[(T2)] (\textbf{Local characterization}). Let $R$ and $R'$ be two sieves of $X$ and assume that $R\in\mathcal{T}(X)$. If for any $Y\in\Ob(\mathcal{C})$ and any morphism $Y\to R$, the sieve $R'\times_XY$ belongs to $\mathcal{T}(Y)$, then $R'$ belongs to $\mathcal{T}(X)$.
\item[(T3)] For any object $X$ of $\mathcal{C}$, $X$ belongs to $\mathcal{T}(X)$.
\end{enumerate}
\end{definition}
The sieves in $\mathcal{T}(X)$ are called \textbf{covering sieves of $\bm{X}$}, or simply the \textbf{coverings of $\bm{X}$}. If $R\in\mathcal{T}(X)$, the inclusion $R\hookrightarrow X$ is also called a \textbf{refinement} of $X$. From the above axioms, we immediately deduce that the set of coverings of $X$ is stable under finite intersection and that any sieve containing a covering sieve is a covering sieve. For example, if $R'$ contains a covering sieve of $X$, then for any morphism $Y\to R$, we have the following Cartesian diagram
\[\begin{tikzcd}
R\times_XY\ar[r,hook]\ar[d]&R'\times_XY\ar[d]\ar[r,hook]&Y\ar[d]\\
R\ar[d]\ar[r]&R\ar[r]\ar[d]&R\ar[d,hook]\\
R\ar[r,hook]&R'\ar[r,hook]&X
\end{tikzcd}\]
We then conclude that $R\times_XY$ (and a fortiori $R'\times_XY$) is equal to $Y$, so $R'\in J(X)$ by axiom (T2). The set $\mathcal{T}(X)$ of covering sieves of $X$, ordered by the inclusion relation, is therefore directed.\par
Let $\mathcal{C}$ be a category and $\mathcal{T},\mathcal{T}'$ be topologies over $\mathcal{C}$. The topology $\mathcal{T}$ is called \textbf{finer than} $\mathcal{T}'$ (or equivalently, $\mathcal{T}'$ is \textbf{coarser than} $\mathcal{T}$) if for any object $X$ of $\mathcal{C}$, any covering of $X$ for the topology $\mathcal{T}'$ is also a covering for the topology $\mathcal{T}$. In this way, we define an order relation on the set of topologies over $\mathcal{C}$.
\begin{example}\label{G-topo intersection and union}
Let $(\mathcal{T}_i)_{i\in I}$ be a family of topologies over $\mathcal{C}$. Then for any object $X$ of $\mathcal{C}$, the intersection of the sets $\mathcal{T}_i(X)$ is easily verified to satisfy the axioms (T1), (T2) and (T3), so it defines a topology $\mathcal{T}$ over $\mathcal{C}$, called the \textbf{intersection} of the $\mathcal{T}_i$. This is the finest topology on $\mathcal{C}$ that coarser than any of the $\mathcal{T}_i$, and is clearly the infimum of the topologies $(\mathcal{T}_i)_{i\in I}$. On the other hand, the family $(\mathcal{T}_i)_{i\in I}$ also admits a supremum: the intersection of the topologies than are finer than each of the $\mathcal{T}_i$.
\end{example}
\begin{example}\label{G-topo discrete}
The topology $\mathcal{T}$ such that $\mathcal{T}(X)$ is the set of sieves of $X$, is clearly the finest topology on $\mathcal{C}$, which is called the \textbf{discrete topology} on $\mathcal{C}$. On the other hand, the coarsest topology on $\mathcal{C}$ is given by $\mathcal{T}(X)=\{X\}$ for any object $X$ of $\mathcal{C}$, which is called the \textbf{trivial topology} on $\mathcal{C}$.
\end{example}
A category $\mathcal{C}$, endowed with a topology, is called a \textbf{site}. The category $\mathcal{C}$ is called the underlying category of the site.
\begin{definition}
Let $\mathcal{C}$ be a site and $X$ be an object of $\mathcal{C}$. A family $\{f_\alpha:X_\alpha\to X\}$ is called a \textbf{covering of $\bm{X}$} if the sieve generated by the $f_\alpha$ is a covering sieve of $X$.
\end{definition}
Let $\mathcal{C}$ be a category. If for each object $X$ of $\mathcal{C}$ we are given a set of families of morphisms with target $X$, then there exists a coarsest topology $\mathcal{T}$ on $\mathcal{C}$ for which the given families of morphisms are coverings, namely the intersection of all these topologies. This topology is called the \textbf{topology generated by the set of families of morphisms}. In general, it is difficult to describe all the coverings in this topology, but the situation is this topology is generated by a \textit{basis}:
\begin{definition}
Let $\mathcal{C}$ be a category. A \textbf{basis} (for a topology) on $\mathcal{C}$ is the assignment to each object $X$ of $\mathcal{C}$ of a set $\Cov(X)$ of families of morphisms with target $X$ (called \textbf{coverings} of $X$), which satisfies the following conditions:
\begin{enumerate}[leftmargin=40pt]
\item[(PT1)] (\textbf{Stable under base change}). For any object $X$ of $C$, any covering $\{X_\alpha\to X\}$ of $X$, and any morphism $Y\to X$ in $\mathcal{C}$, the fiber products $X_\alpha\times_XY$ exists in $\mathcal{C}$ and the family $\{X_\alpha\times_XY\to Y\}$ is a covering of $X$.
\item[(PT2)] (\textbf{Stable under composition}). If $\{X_\alpha\to X\}$ is a covering of $X$ and for each $\alpha$, $\{X_{\beta\alpha}\to X_\alpha\}$ is a covering of $X_\alpha$, then the composite family $\{X_{\beta\alpha}\to X_\alpha\to X\}$ is a covering of $X$.
\item[(PT3)] The family $\{\id_X:X\to X\}$ is a covering of $X$.
\end{enumerate}
\end{definition}
For any given basis on $\mathcal{C}$, we can consider the topology over $\mathcal{C}$ generated by this basis. Note that if $\mathcal{C}$ is a category where fiber products exist, then any topology $\mathcal{T}$ of $\mathcal{C}$ can be defined by a basis, namely the one for which $\Cov(X)$ is formed by the covering families of $X$ for the topology $\mathcal{T}$.
\begin{proposition}\label{G-topo generated by basis sieve char}
Let $\mathcal{C}$ be a category, $\mathcal{B}$ be a basis on $\mathcal{C}$, $\mathcal{T}$ the topology generated by $\mathcal{B}$, $X$ an object of $\mathcal{C}$. Denote by $\mathcal{T}_{\mathcal{B}}(X)$ the set of sieves of $X$ generated by the families of morphisms of the basis, and by $\mathcal{T}(X)$ the set of covering sieves of $X$ for the topology $\mathcal{T}$. Then $\mathcal{T}_{\mathcal{B}}(X)$ is cofinal in $\mathcal{T}(X)$, in other words, for a sieve $R$ of $X$ to belong to $\mathcal{T}(X)$, it is necessary and sufficient that there exists a sieve $R'$ of $\mathcal{T}_\mathcal{B}(X)$ such that $R'\sub R$. 
\end{proposition}
\begin{proof}
For any object $X$ of $\mathcal{C}$, let $\mathcal{J}(X)$ be the set of sieves of $X$ that contain a sieve of $\mathcal{T}_{\mathcal{B}}(X)$. We have evidently $\mathcal{J}(X)\sub\mathcal{T}(X)$, and to prove the converse inclusion, it suffices to show that the sets $\mathcal{J}(X)$ defines a topology over $\mathcal{C}$.\par
It is clear that the $\mathcal{J}(X)$ satisfy the axioms (T1) and (T3), and it remains to verify (T2). To this end, let $R',R$ be sieves of $X$ with $R\in\mathcal{T}_\mathcal{B}(X)$ such that for any morphism $Y\to R$, the sieve $R'\times_XY$ of $Y$ belongs to $\mathcal{J}(Y)$. By definition, $R$ contains a sieve $T$ which is generated by a covering $\{X_\alpha\to X\}$ of $X$. For each $\alpha$, we have a canonical morphism $X_\alpha\to T$, defined by
\[\Hom(Y,X_\alpha)\to T(Y),\quad (Y\to X_\alpha)\mapsto(Y\to X_\alpha\to X),\]
so by our hypothesis the fiber product sieve $R'\times_XX_\alpha$ belongs to $\mathcal{J}(X_\alpha)$, hence contains a sieve generated by a covering $\{X_{\beta\alpha}\to X_\alpha\}$ of $X_\alpha$. Unwinding the definitions, this means for any $\alpha$ and any morphism $f:Y\to X_{\beta\alpha}\to X_\alpha$, there exists a morphism $g:Y\to X$ in $R'(Y)$ such that $g$ is equal to the composition of $f$ with $X_\alpha\to X$. We therefore deduce that $R'$ contains the sieve generated by the composite family $\{X_{\beta\alpha}\to X\}$, so by axiom (PT2), $R'$ contains a sieve of $\mathcal{J}_\mathcal{B}(X)$ and hence belongs to $\mathcal{J}(X)$.
\end{proof}
In practice, we usually give basis to generate a topology on the category $\mathcal{C}$. In this case, by a topology, we in fact mean the topology generated by the basis defined by these arrows\footnote{In fact, many authors define a Grothendieck topology to be a collection of families of morphisms satisfying axioms (PT1), (PT2) and (PT3).}. Here are some examples of topologies generated by basis, where for a family $\{f_\alpha:X_\alpha\to X\}$ of maps is called \textbf{jointly surjective} if the induced map $\coprod_\alpha X_\alpha\to X$ is surjective (in other words, the union of the images of $f_\alpha$ equal to $X$).
\begin{example}[\textbf{The site of a topological space}]
Let $X$ be a topological space and let $X_{\cl}$ be the category of open subsets of $X$, with morphisms given by inclusions. Then we get a topology on $X_{\cl}$ by associating with each open subset $U\sub X$ the set of open coverings of $U$, whence a generated topology on $X_{\cl}$. In this case, if $U_1\to U$ and $U_2\to U$ are arrows, the fiber product $U_1\times_UU_2$ is the intersection $U_1\cap U_2$.
\end{example}
\begin{example}[\textbf{The global classical topology}]
Let $\mathcal{C}=\mathbf{Top}$ be the category of topological spaces. If $U$ is a topological space, then a covering of $U$ will be a jointly surjective collection of open embeddings $U_i\to U$.
\end{example}
\begin{example}[\textbf{The small \'etale site for a scheme}]
Let $X$ be a scheme and consider the full subcategory $X_{\et}$ of $\mathbf{Sch}_{/X}$, consisting of \'etale morphisms $U\to X$. Note that any morphism $U\to V$ of objects in $X_{\et}$ is necessarily \'etale, and a covering of $U\to X$ is a jointly surjective collection of morphisms $U_i\to U$.
\end{example}
\begin{example}[\textbf{The topologies on $\mathbf{Sch}_{/S}$}]
Let $\mathbf{Sch}_{/S}$ be the category of schemes over a fixed scheme $S$. We can define several topologies on $\mathbf{Sch}_{/S}$:
\begin{itemize}
\item The \textbf{Zariski topology} on $\mathbf{Sch}_{/S}$ is defined by collections of open coverings $\{U_i\to U\}$ of $U$.
\item The \textbf{global \'etale topology} on $\mathbf{Sch}_{/S}$ is defined by jointly surjective collections of \'etale morphisms in $\mathbf{Sch}_{/S}$.
\item The \textbf{fppf topology} on $\mathbf{Sch}_{/S}$ is defined by jointly surjective collections of flat morphisms that are locally of finite presentation (the abbreviation "fppf" stands for "fid\`element plat et de pr\'esentation finie").
\end{itemize}
\end{example}

\subsection{Sheaves over a site}
Let $\mathcal{C}$ be a site whose underlying category is a $\mathscr{U}$-category and $F:\mathcal{C}^{\op}\to\mathbf{Set}$ be a presheaf over $\mathcal{C}$. The functor $F$ is called \textbf{separated} (resp. a \textbf{sheaf}) if for any object $X$ of $\mathcal{C}$ and any covering sieve $R$ of $X$, the map
\[\Hom_{\PSh(\mathcal{C})}(X,F)\to\Hom_{\PSh(\mathcal{C})}(R,F)\]
is injective (resp. bijective). The full subcategory of $\PSh(\mathcal{C})$ of sheaves over $\mathcal{C}$ is call the \textbf{category of sheaves of sets over $\mathcal{C}$}, and denoted by $\Sh(\mathcal{C})$. If there is no risk of ambiguity, this is simply called the category of sheaves over $\mathcal{C}$.
\begin{proposition}\label{G-topo generated by family of presheaf}
Let $C$ be a $\mathscr{U}$-category and $\mathfrak{F}=(F_i)_{i\in I}$ be a family of presheaves over $\mathcal{C}$. For each object $X$ of $\mathcal{C}$, let $\mathcal{T}_{\mathfrak{F}}(X)$ be the set of sieves $R$ of $X$ such that for any morphism $Y\to X$ in $\mathcal{C}$ and any $i\in I$, the map
\[\Hom_{\PSh(\mathcal{C})}(Y,F_i)\to\Hom_{\PSh(\mathcal{C})}(R\times_XY,F_i)\]
is injective (resp. bijective). Then the sets $\mathcal{T}_{\mathfrak{F}}(X)$ define a topology $\mathcal{T}$ over $\mathcal{C}$, which is the finest topology for which each $F_i$ is a separated presheaf (resp. a sheaf).
\end{proposition}
\begin{proof}

\end{proof}
\begin{corollary}\label{G-topo generated by morphism sheaf iff}
Let $\mathcal{C}$ be a $\mathscr{U}$-category, and for any object $X$ of $\mathcal{C}$, let $K(X)$ be a set of sieves of $X$ satisfying axiom (T1). For a presheaf $F$ over $\mathcal{C}$ to be a separated presheaf (resp. a sheaf) for the topology generated by the $K(X)$, it is necessary and sufficient that for any object $X$ of $\mathcal{C}$ and any sieve $R\in K(X)$, the map
\begin{align}\label{G-topo generated by morphism sheaf iff-1}
\Hom_{\PSh(\mathcal{C})}(X,F)\to\Hom_{\PSh(\mathcal{C})}(R,F)
\end{align}
is injective (resp. bijective).
\end{corollary}
\begin{proof}
In fact, let $\mathcal{T}_F(X)$ be the set of sieves $R$ of $X$ such that for any morphism $Y\to X$ in $\mathcal{C}$ and any $i\in I$, the map
\[\Hom_{\PSh(\mathcal{C})}(Y,F)\to\Hom_{\PSh(\mathcal{C})}(R\times_XY,F_i)\]
is bijective (resp. injective), and $\mathcal{T}_F$ be the topology generated by $\mathcal{T}_F(X)$. Since $K(X)$ satisfies (T1), the set $K(X)$ is contained in $\mathcal{T}_F(X)$ with the condition of the corollary, so the topology $\mathcal{T}$ generated by $K(X)$ is coarser than $\mathcal{T}_F$. By \cref{G-topo generated by family of presheaf}, it then follows that $F$ is a separated presheaf (resp. a sheaf) for the topology $\mathcal{T}$.
\end{proof}
\begin{corollary}\label{G-topo generated by basis sheaf iff}
Let $\mathcal{C}$ be a $\mathscr{U}$-category endowed with a basis. For a presheaf $F$ to be a separated presheaf (resp. a sheaf), it is necessary and sufficient that for any object $X$ of $\mathcal{C}$ and for any covering $\{X_\alpha\to X\}$ of $X$, the following sequence
\[\begin{tikzcd}
F(X)\ar[r]&\prod_{\alpha}F(X_\alpha)\ar[r,shift left=3pt]\ar[r,shift right=3pt]&\prod_{\alpha,\beta}F(X_\alpha\times_XX_\beta)
\end{tikzcd}\]
is exact (resp. the map $F(X)\to\prod_\alpha F(X_\alpha)$ is injective.)
\end{corollary}
\begin{proof}
By (SGA4, \Rmnum{1}, 3.5), we have a functorial isomorphism 
\[\Hom_{\PSh(\mathcal{C})}(F,R)=\llim_{(X,u)\in\mathcal{C}_{/R}}F(X).\]
so the corollary follows from \cref{G-topo generated by morphism sheaf iff} and (SGA4, \Rmnum{1}, 2.12).
\end{proof}
Let $\mathcal{C}$ be a $\mathscr{U}$-category. We define the \textbf{canonical topology} of $\mathcal{C}$ to be the finest topology such that all representable functors are sheaves. A covering sieve of $X$ for the canonical topology is then called \textbf{universally effective-epimorphic}. In other words, this means for any object $Z$ of $\mathcal{C}$, the canonical map
\[\Hom_{\PSh(\mathcal{C})}(X,Z)\to\Hom_{\PSh(\mathcal{C})}(R,Z)\]
is bijective. More generally, a topology $\mathcal{T}$ over $\mathcal{C}$ such that all representable functors are sheaves is called \textbf{subcanonical}, so the canonical topology is the finest subcanonical topology on $\mathcal{C}$. In most cases, the topology considered over a category $\mathcal{C}$ is subcanonical, so the covering sieves of $\mathcal{C}$ are universally effective-epimorphic. The only exception is the site $\PSh(\mathcal{C})$ obtained from a site $\mathcal{C}$, whose topology is finer, and often strictly finer, than the canonical topology.
\begin{proposition}\label{G-topo sieve universal effective epi iff}
For a sieve $R$ of an object $X$ of $\mathcal{C}$ to be universally effective-epimorphic, it is necessary and sufficient that for any object $Y\to X$ of $\mathcal{C}_{/X}$ and any object $Z$ of $\mathcal{C}$, the map
\[\Hom_{\mathcal{C}}(Y,Z)\to\llim_{U\in\mathcal{C}_{/(Y\times_XR)}}\Hom_\mathcal{C}(U,Z)\]
is a bijection.
\end{proposition}
\begin{proof}
By definition of canonical topology and \cref{G-topo generated by family of presheaf}, $R$ is universally effective-epimorphic if and only if for any for any object $Z$ of $\mathcal{C}$ and any morphism $Y\to X$ in $\mathcal{C}$, the map
\[\Hom_{\PSh(\mathcal{C})}(Y,Z)\to\Hom_{\PSh(\mathcal{C})}(R\times_XY,Z)\]
is bijective. On the other hand, by (SGA4, \Rmnum{1}, 3.5), we have a functorial isomorphism 
\[\Hom_{\PSh(\mathcal{C})}(R\times_XY,Z)=\llim_{U\in\mathcal{C}_{/R\times_XY}}h_Z(U)=\llim_{U\in\mathcal{C}_{/R\times_XY}}\Hom_{\mathcal{C}}(U,Z),\]
whence the proposition.
\end{proof}
\begin{corollary}\label{G-topo generated by basis sieve universal effective epi iff}
Let $\mathcal{C}$ be a $\mathscr{U}$-category endowed with a basis. Then for a sieve $R$ of $X$ defined by a covering $\{X_\alpha\to X\}$ to be universally effective-epimorphic, it is necessary and sufficient that for any object $Z$ of $\mathcal{C}$, the sequence
\begin{equation}\label{G-topo generated by basis sieve universal effective epi iff-1}
\begin{tikzcd}
\Hom_{\mathcal{C}}(X,Z)\ar[r]&\prod_{\alpha}\Hom_{\mathcal{C}}(X_\alpha,Z)\ar[r,shift left=3pt]\ar[r,shift right=3pt]&\prod_{\alpha,\beta}\Hom_{\mathcal{C}}(X_\alpha\times_XX_\beta,Z)
\end{tikzcd}
\end{equation}
is exact. In particular, the covering families of the canonical topology of $\mathcal{C}$ are universally effective-epimorphic families.
\end{corollary}
\begin{proof}
The first assertion follows from the proof of \cref{G-topo generated by basis sheaf iff} and \cref{G-topo sieve universal effective epi iff}. As for the second one, it suffices to note that a family $\{X_\alpha\to X\}$ of morphisms in $\mathcal{C}$ is universally effective-epimorphic if and only if the sequence (\ref{G-topo generated by basis sieve universal effective epi iff-1}) is exact.
\end{proof}
\begin{remark}
\cref{G-topo sieve universal effective epi iff} gives a characterization of universally effective epimorphisms of a category $\mathcal{C}$, which is independent of the universe in which the presheaves take their values, with the sole condition that the sets of morphisms $\Hom_{\mathcal{C}}(X,Y)$ of $\mathcal{C}$ belong to this universe. It therefore permits thus to define the canonical topology for any category $\mathcal{C}$, without specifying the universes we are considering.
\end{remark}
\subsection{Sheafifications of presheaves}
Let $\mathcal{C}$ be a site. We define a \textbf{topological generating family} (or simply a \textbf{generating family} of $\mathcal{C}$ is there is no risk of confusion) of $\mathcal{C}$ to be a set $G$ of objects of $\mathcal{C}$ such that any object of $\mathcal{C}$ is the target of a family of covering morphisms of $\mathcal{C}$ whose sources are elements of $G$. A site $\mathcal{C}$ is called a \textbf{$\mathscr{U}$-site} (where $\mathscr{U}$ is the fixed universe) if the underlying category is a $\mathscr{U}$-category and $\mathcal{C}$ has a $\mathscr{U}$-small topological generating family. If $\mathcal{C}$ is a category, we define a \textbf{$\mathscr{U}$-topology} over $\mathcal{C}$ to be a topology with which $\mathcal{C}$ is a $\mathscr{U}$-site. The site $\mathcal{C}$ is called \textbf{$\mathscr{U}$-small} (or by abusing of language, small) if the underlying category is small. It follows immediately from definition that any topology finer than a $\mathscr{U}$-topology is a $\mathscr{U}$-topology, and any small site is a $\mathscr{U}$-site.
\begin{proposition}\label{site small generated sieve set prop}
Let $\mathcal{C}$ be a $\mathscr{U}$-site and $G$ be a small topological generating family of $\mathcal{C}$. For any object $X$ of $\mathcal{C}$, we denote by $\mathcal{T}_G(X)$ the set of covering sieves of $X$ generated by the families of morphisms $\{u_\alpha:Y_\alpha\to X\}$, where $Y_\alpha\in G$. Then
\begin{enumerate}
\item[(a)] The set $\mathcal{T}_G(X)$ is $\mathscr{U}$-small and cofinal in the set $\mathcal{T}(X)$ of covering sieves of $X$, ordered by inclusion.
\item[(b)] For any $R\in\mathcal{T}_G(X)$, there exists a $\mathscr{U}$-small epimorphic family $\{u_\alpha:Y\to R\}$ with $Y\in G$.
\end{enumerate}
\end{proposition}
\begin{proof}
For a presheaf $F\in\PSh(\mathcal{C})$, we define
\[A(F)=\coprod_{Y\in G}\Hom(Y,F),\]
which is a $\mathscr{U}$-small set by our hypothesis. For any object $X$ of $\mathcal{C}$, we have $|\mathcal{T}_G(X)|\leq 2^{|A(X)|}$, so the set $\mathcal{T}_G(X)$ is $\mathscr{U}$-small. Now for a sieve $R\in\mathcal{T}(X)$, let $R'$ be the sieve of $X$ generated by the family $\{u:Y\to R\}$, where $u\in A(R)$. We then have $R'\sub R$, and it suffices to prove that $R'$ is a covering sieve. By axiom (T2), we only need to show that for any morphism $Z\to R$, where $Z$ is an object of $\mathcal{C}$, the sieve $R'\times_XZ$ of $Z$ is a covering sieve. But $R'\times_XZ$ contains the sieve generated by the family of morphisms $Y\to Z$, where $Y\in G$, which are coverings by hypothesis, so it is also a covering sieve by axiom (T2) again; this completes the proof of (a). Finally, we note that for any $R\in \mathcal{T}_G(X)$, the family $\{u:Y\to R\}_{u\in A(R)}$ is by hypothesis epimorphic, and is $\mathscr{U}$-small.
\end{proof}
Let $\mathcal{C}$ be a $\mathscr{U}$-site, $\mathscr{V}$ be a universe containing $\mathscr{U}$ such that $\mathcal{C}$ is $\mathscr{V}$-small, and $G$ be a $\mathscr{V}$-small topological generating family of $\mathcal{C}$. The category $\PSh(\mathcal{C})$ of presheaves of $\mathscr{U}$-sets over $\mathcal{C}$ is then a $\mathscr{V}$-category. Let $X$ be an object of $\mathcal{C}$; the set $\mathcal{T}(X)$ of covering sieves of $X$ is $\mathscr{V}$-small, and directed under inclusion. For any $\mathscr{U}$-presheaf $F$, the inductive limit
\begin{align}\label{site small generated sheafification-1}
\rlim_{R\in\mathcal{T}(X)}\Hom_{\PSh(\mathcal{C})}(R,F)
\end{align}
is then represented by an element of $\mathscr{V}$ (SGA4 \Rmnum{1}, 2.4.1). Moreover, it follows from \cref{site small generated sieve set prop}(b) and (\ref{category Hom of functors char by limit}) that, for any $R\in\mathcal{T}(X)$, $\Hom_{\PSh(\mathcal{C})}(R,F)$ is $\mathscr{U}$-small, and as $\mathcal{T}_G(X)$ is a $\mathscr{U}$-small cofinal set in $\mathcal{T}(X)$, it follows from (SGA4 \Rmnum{1}, 2.4.2) that the limit (\ref{site small generated sheafification-1}) is $\mathscr{U}$-small. We then choose, for any $F$ and for any $X$, an element of $\mathscr{U}$ that represents this inductive limit and let
\begin{align}\label{site small generated sheafification-2}
LF(X)=\rlim_{R\in\mathcal{T}(X)}\Hom_{\PSh(\mathcal{C})}(R,F).
\end{align}
If $g:Y\to X$ is a morphism in $\mathcal{C}$, the base change functor $g^*:\mathcal{T}(X)\to\mathcal{T}(Y)$ then defines a map
\[LF(g):LF(X)\to LF(Y)\]
so that the construction $X\mapsto LF(X)$ is a presheaf on $\mathcal{C}$. We also note that since the family $\{\id_X:X\to X\}$ is an element of $\mathcal{T}(X)$, for any object $X$ of $\mathcal{C}$, we have a map
\[\ell_F(X):F(X)\to LF(X),\]
from which we obtain a morphism of functors $\ell_F:F\to LF$, and therefore a morphism
\[\ell:\id\to L.\]
Now let $R\hookrightarrow X$ be a refinement of $X$. The definition of $LF(X)$ and Yoneda's lemma then give a map
\[Z_R:\Hom_{\PSh(\mathcal{C})}(R,F)\to\Hom_{\PSh(\mathcal{C})}(X,LF),\]
and for any morphism $g:Y\to X$ in $\mathcal{C}$, we have the following commutative diagram:
\begin{equation}\label{site small generated sheafification-3}
\begin{tikzcd}
\Hom_{\PSh(\mathcal{C})}(R,F)\ar[d]\ar[r,"Z_R"]&\Hom_{\PSh(\mathcal{C})}(X,LF)\ar[d]\\
\Hom_{\PSh(\mathcal{C})}(R\times_XY,F)\ar[r,"Z_{R\times_XY}"]&\Hom_{\PSh(\mathcal{C})}(Y,LF)
\end{tikzcd}
\end{equation} 
where the vertical arrows are induced by the obvious morphisms.
\begin{lemma}\label{site small generated sheafification lemma}
Let $LF$ be the presheaf defined above and $\ell:\id\to L$ be the induced morphism.
\begin{enumerate}
\item[(a)] For any refinement $i_R:R\hookrightarrow X$ and any morphism $u:R\to F$, the diagram
\begin{equation}\label{site small generated sheafification-4}
\begin{tikzcd}[row sep=10mm, column sep=10mm]
R\ar[r,hook,"i_R"]\ar[d,swap,"u"]&X\ar[d,"Z_R(u)"]\\
F\ar[r,"\ell_F"]&LF
\end{tikzcd}
\end{equation}
is commutative.
\item[(b)] For any morphism $v:X\to LF$, there exists a refinement $R$ of $X$ and a morphism $u:R\to F$ such that $Z_R(u)=v$. 
\item[(c)] Let $Y$ be an object of $C$ and $u,v:Y\rightrightarrows F$ be two morphisms such that $\ell_F\circ u=\ell_F\circ v$. Then ther kernel of the couple $(u,v)$ is a refinement of $Y$.
\item[(d)] Let $R,R'$ be refinements of $X$ and $u:R\to F$, $u':R'\to F$ be morphisms. For that $Z_R(u)=Z_{R'}(u')$, it is necessary and sufficient that $u$ and $u'$ coincide on a refinement $R''\hookrightarrow R\times_XR'$.
\end{enumerate}
\end{lemma}
\begin{proof}
In fact, any morphism $v:X\to LF$ corresponds to an element of the inductive limit $\rlim_{R\in\mathcal{T}(X)}\Hom_{\PSh(\mathcal{C})}(R,F)$, which is the canonical image of an element of $\Hom_{\PSh(\mathcal{C})}(R,F)$, and two such images coincide if and only if their canonical images to a further refinement are equal; this proves (b) and (d). Now if $Y$ is an object of $C$ and $u,v:Y\to F$ are two morphisms such that $\ell_F\circ u=\ell_F\circ v$, then by our preceding remarks, there exists a refinement $R$ of $Y$ such that $u$ and $v$ coincides on $R$. Since the kernel of $(u,v)$ must contains $R$, it is therefore a refinement of $Y$.\par
Finally, in view of Yoneda's lemma, to prove (a), it suffices to show that the compositions of $Z_R(u)\circ i_R$ and $\ell_F\circ u$ with any morphism $g:Y\to R$ ($Y$ being an object of $\mathcal{C}$) are equal. For this, we consider $f=i_R\circ g$ and the fiber product $R\times_XY$:
\[\begin{tikzcd}
R\times_XY\ar[r,hook,"i'"]\ar[d,swap,"f'"]&Y\ar[ld,swap,"g"]\ar[d,"f"]\\
R\ar[r,hook,"i_R"]\ar[d,swap,"u"]&X\ar[d,"Z_R(u)"]\\
F\ar[r,"\ell_F"]&LF
\end{tikzcd}\]
By the definition of $\ell_F$, the morphism $Z_{R\times_XY}(u\circ f')$ is equal to $\ell_F\circ u\circ g$. On the other hand, the commutative diagram (\ref{site small generated sheafification-3}) shows that $Z_{R\times_XY}(u\circ f')=Z_R(u)\circ f$, so we conclude that
\begin{equation*}
\ell_F\circ u\circ g=Z_R(u)\circ f=Z_R(u)\circ i_R\circ g.\qedhere
\end{equation*}.
\end{proof}
\begin{proposition}\label{site small generated sheafification functor prop}
Let $L$ be the functor on $\PSh(\mathcal{C})$ defined by (\ref{site small generated sheafification-2}).
\begin{enumerate}
\item[(a)] The functor $L$ is left exact.
\item[(b)] For any presheaf $F$, $LF$ is a separated presheaf.
\item[(c)] The presheaf $F$ is separated if and only if the morphism $\ell_F:F\to LF$ is a monomorphism, and in this case $LF$ is a sheaf.
\item[(d)] The morphism $\ell_F:F\to LF$ is an isomorphism if and only if $F$ is a sheaf. 
\end{enumerate} 
\end{proposition}
\begin{proof}

\end{proof}
\begin{remark}
If $\mathcal{J}(X)$ is a cofinal subset of $\mathcal{T}(X)$, we then have
\[LF(X)=\rlim_{R\in\mathcal{J}(X)}\Hom_{\PSh(\mathcal{C})}(R,F).\]
In particular, if the topology of $\mathcal{C}$ is defined by a basis $X\mapsto\Cov(X)$, the functor $L$ can be then described by the covering families of $\Cov(X)$. 
\end{remark}
From \cref{site small generated sheafification functor prop}, we then deduce the following theorem on the eixstence of sheafification functor on $\PSh(\mathcal{C})$:
\begin{theorem}\label{site small generated sheafification exist}
For a $\mathscr{U}$-site $\mathcal{C}$, the inclusion functor $i:\Sh(\mathcal{C})\to\PSh(\mathcal{C})$ admits a left adjoint $\#$:
\[\begin{tikzcd}[column sep=12mm]
\Sh(\mathcal{C})\ar[r,shift left=3pt,"i"]&\PSh(\mathcal{C})\ar[l,shift left=3pt,"(-)^\#"]
\end{tikzcd}\]
The functor $i\circ\#$ is canonically isomorphic to $L\circ L$, and the sheaf $F^\#$ is called the \textbf{sheafification} of a presheaf $F$. For a presheaf $F$, the adjunction morphism $F\to i(F^\#)$ is induced by the morphism $\ell_{LF}\ell_F:F\to L\circ L(F)$.
\end{theorem}
\begin{remark}
We note that, since $L$ is left exact and the functor $\#=L\circ L$ is a left adjoint, the sheafification functor $\#$ is in fact exact on $\PSh(\mathcal{C})$. Recall that this result is obtained in the classical situation by the observation that sheafification does not change the stalks.
\end{remark}
\subsection{Exactness properties of the category of sheaves}\label{site category sheaf exactness property subsection}
The exactness properties of the category of sheaves is then induced from that of the category of presheaves via \cref{site small generated sheafification exist}. In this paragraph, we explain this philosophy by single out some of the most useful standard statements.
\subsection{The induced topology on \texorpdfstring{$\PSh(\mathcal{C})$}{Sh}}
\begin{proposition}\label{site morphism of presheaf covering def}
Let $\mathcal{C}$ be a $\mathscr{U}$-site and $f:H\to K$ be a morphism in $\PSh(\mathcal{C})$. Then the following conditions are equivalent:
\begin{enumerate}
\item[(\rmnum{1})] For any morphism $X\to K$, where $X\in\Ob(\mathcal{C})$, the corresponding morphism $H\times_KX\to X$ generates a covering sieve of $X$.
\item[(\rmnum{2})] The morphism $f^\#:H^\#\to K^\#$ of sheafifications is an epimorphism in $\Sh(\mathcal{C})$.
\item[(\rmnum{2}')] For any sheaf $F$ over $\mathcal{C}$, the map $f^*:\Hom(K,F)\to\Hom(H,F)$ is injective.
\end{enumerate}
\end{proposition}
\begin{proof}
By \cref{site small generated sheafification exist}, for any sheaf $F$ over $\mathcal{C}$ we have $\Hom(K,F)=\Hom(K^\#,F)$, so it is clear that (\rmnum{2}) is equivalent to (\rmnum{2}'). On the other hand, in view of (SGA4, \Rmnum{2}, 4.4), condition (\rmnum{1}) signifies that $u:(H\times_KX)^\#\to X^\#$ is an epimorphism. Since epimorphisms in $\Sh(\mathcal{C})$ are stable under base change and $u$ is induced from $f^\#$ by base change, this proves (\rmnum{2})$\Rightarrow$(\rmnum{1}).\par
Conversely, since the family $\{X\to K:X\in\Ob(\mathcal{C})\}$ is epimorphic in $\PSh(\mathcal{C})$, so is the induced family $\{X^\#\to K^\#:X\in\Ob(\mathcal{C})\}$ in $\Sh(\mathcal{C})$ (SGA4 \Rmnum{2}, 4.1). Therefore, the morphism $f^\#:H^\#\to K^\#$ in $\Sh(\mathcal{C})$ is an epimorphism if and only if its base change $f^\#:H^\#\times_{K^\#}X^\#\to X^\#$ is an epimorphism for any $X\to K$, $X\in\Ob(\mathcal{C})$. This proves the implication (\rmnum{1})$\Rightarrow$(\rmnum{2}) in view of (SGA4, \Rmnum{2}, 4.1).
\end{proof}
A morphism $f:H\to K$ satisfying the equivalent conditions of \cref{site morphism of presheaf covering def} is called a \textbf{covering} morphism. A family $\{f_i:H_i\to K\}$ is called \textbf{covering} if the induced morphism $f:\coprod_iH_i\to K$ is covering. A morphism $f:H\to K$ is called \textbf{bicovering} if it is covering and the diagonal morphism $H\to H\times_KH$ is covering. Similarly, a family $\{f_i:H_i\to K\}$ is called \textbf{bicovering} if the induced morphism $f:\coprod_iH_i\to K$ is bicovering\footnote{We note that, in view of condition (\rmnum{2}) of \cref{site morphism of presheaf covering def}, the condition of being covering (resp. bicovering) for a family of morphisms in $\PSh(\mathcal{C})$ is stable under base change.}.\par
By condition (\rmnum{1}) of \cref{site morphism of presheaf covering def}, to say that a family $\{f_i:H_i\to K\}$ is a covering signifies that for any morphism $X\to K$, where $X\in\Ob(\mathcal{C})$, the family $\{H_i\times_KX\to X\}$ generates a covering sieve of $X$. Or equivalently, by (\rmnum{2}), the family $\{f_i^\#:H_i^\#\to K^\#\}$ in $\Sh(\mathcal{C})$ is epimorphic (since the functor $\#$ commutes with direct sums).
\begin{proposition}\label{site morphism of presheaf bicovering iff}
Let $\mathcal{C}$ be a $\mathscr{U}$-site and $f:H\to K$ be a morphism in $\PSh(\mathcal{C})$. The following conditions are equivalent:
\begin{enumerate}
\item[(\rmnum{1})] The morphism $f$ is bicovering.
\item[(\rmnum{1}')] The morphism $f$ is covering and for any object $X$ of $\mathcal{C}$ and any couple of morphisms $u,v:X\rightrightarrows H$ such that $fu=fv$, the kernel $(u,v)$ is a covering sieve of $X$.
\item[(\rmnum{2})] The morphism $f^\#:H^\#\to K^\#$ is an isomorphism in $\Sh(\mathcal{C})$.
\item[(\rmnum{2}')] For any sheaf $F$ over $\mathcal{C}$, the map $\Hom(K,F)\to\Hom(H,F)$ is a bijection.
\end{enumerate}
\end{proposition}
\begin{proof}
The equivalence (\rmnum{1})$\Leftrightarrow$(\rmnum{1}') follows from condition (\rmnum{1}) of \cref{site morphism of presheaf covering def}, applied to the diagonal morphism $H\to H\times_KH$ (recall that $\ker(u,v)$ is given by the base change of the diagonal morphism $H\to H\times_KH$ along $(u,v)_K:X\to H\times_KH$), and that of (\rmnum{2}) and (\rmnum{2}') is trivial. We now prove that (\rmnum{1})$\Rightarrow$(\rmnum{2}), so let $f:H\to K$ be a bicovering morphism in $\PSh(\mathcal{C})$. The morphism $f^\#:H^\#\to K^\#$ is then an epimorphism by \cref{site morphism of presheaf covering def}, and the diagonal morphism $H^\#\to H^\#\times_{K^\#}H^\#$ is an epimorphism since the functor $\#$ commutes with fiber products. As the diagonal morphism always a monomorphism, it is then an isomorphism (SGA4, \Rmnum{2}, 4.2), so $f^\#$ is a monomorphism, whence an isomorphism.\par
Conversely, if $f^\#$ is an isomorphism, then the morphism $f$ and the diagonal $H\to H\times_KH$ are both isomorphisms (since $\#$ commutes with fiber products). In particular, their image under $\#$ are isomorphisms, so the morphism $f$ is covering.
\end{proof}
\begin{remark}
It follows from \cref{site morphism of presheaf bicovering iff} and the fact that $\#$ commutes with direct sums that a family $\{f_i:H_i\to K\}$ of morphisms in $\PSh(\mathcal{C})$ is bicovering if and only if the $f_i$ induce an isomorphism $\coprod_iH_i^\#\to K^\#$, of equivalently, if and only if for any sheaf $F$, the map
\[\Hom(K,F)\to\prod_i\Hom(H_i,F)\]
is bijective.
\end{remark}
\begin{proposition}\label{site induced topology on PSh char}
Let $\mathcal{C}$ be a $\mathscr{U}$-site. Then there exists a (unique) topology on $\PSh(\mathcal{C})$ such that a family $H_i\to K$ of morphisms in $\PSh(\mathcal{C})$ is covering for this topology if and only if it is covering (in the sense of \cref{site morphism of presheaf covering def}). This is also the coarsest topology $\mathcal{T}$ on $\PSh(\mathcal{C})$ satisfying the following conditions:
\begin{enumerate}
\item[(a)] $\mathcal{T}$ is finer than the canonical topology of $\PSh(\mathcal{C})$.
\item[(b)] Any covering family in $\mathcal{C}$ is covering in $\PSh(\mathcal{C})$.
\end{enumerate}
\end{proposition}
\begin{proof}
We first show that the covering families in the sense of \cref{site morphism of presheaf covering def} generates a topology $\mathcal{T}_\mathcal{C}$ on $\PSh(\mathcal{C})$.\par 
The covering family of the canonical topology over $\PSh(\mathcal{C})$ are the universally effective-epimorphic families of $\PSh(\mathcal{C})$ (\cref{G-topo generated by basis sieve universal effective epi iff}). As $\#$ commutes with inductive limits, the topology $\mathcal{T}_\mathcal{C}$ is then finer than the canonical topology of $\PSh(\mathcal{C})$. Moreover, the covering families of $\mathcal{C}$ are also that of $\mathcal{T}_\mathcal{C}$ (\cref{site morphism of presheaf covering def}(\rmnum{1})), so if $\mathcal{T}'$ denote the coarsest topology on $\PSh(\mathcal{C})$ satisfying conditions (a), (b), then $\mathcal{T}_\mathcal{C}$ is fiber than $\mathcal{T}'$. 
\end{proof}
\begin{remark}\label{site topology on PSh finer than canonical char}
The proof of \cref{site induced topology on PSh char} in fact proves that any topology $\mathcal{T}'$ over $\PSh(\mathcal{C})$, finer than the canonical topology, is the coarsest topology $\mathcal{T}$ of $\PSh(\mathcal{C})$ with the following properties:
\begin{itemize}
\item[(a)] $\mathcal{T}$ is fiber than the canonical topology of $\PSh(\mathcal{C})$.
\item[(b)] Any covering family for $\mathcal{T}'$ of the form $\{u_i:X_i\to X\}$, where $X_i$ and $X$ are objects of $\mathcal{C}$, is covering for $\mathcal{T}$.
\end{itemize}
In particular, any topology over $\PSh(\mathcal{C})$, finer than the canonical topology, is completely determined by its covering families of morphisms in $\mathcal{C}$.\par
Conversely, one can easily show that for any topology $\mathcal{T}'$ over $\PSh(\mathcal{C})$, finer than the canonical topology, the set of covering families of morphisms $\{X_i\to X\}$ in $\mathcal{C}$ is the set of covering families of a topology on $\mathcal{C}$. Therefore, we obtain a correspondence between topologies on $\mathcal{C}$ and topologies on $\PSh(\mathcal{C})$ finer than the canonical topology.
\end{remark}
Let $\mathcal{C}$ be a small category. We denote by $\mathrm{Caf}$ the set of strictly full subcategories (any object isomorphic to an object of the subcategory is an object of it) of $\PSh(\mathcal{C})$ whose inclusion functor to $\PSh(\mathcal{C})$ admits a left adjoint which commutes with finite projective limits, and by $\mathscr{T}$ the set of topologies on $\mathcal{C}$. In view of \cref{site small generated sheafification exist}, we have a map
\[\Phi:\mathscr{T}\to\mathrm{Caf}\]
which sends a topology over $\mathcal{C}$ to the sheaf category it defines. We now construct an inverse of the map $\Phi$. For this, we need to associate for each element
\[
\begin{tikzcd}
\mathcal{C}'\ar[r,shift left=3pt,"i'"]&\PSh(\mathcal{C})\ar[l,shift left=3pt,"j'"]
\end{tikzcd}
\]
(where $j'$ is the adjoint of $i'$) of $\mathrm{Caf}$ a topology $\mathcal{T}_e$ over $\mathcal{C}$. For any object $X$ of $\mathcal{C}$, we define $\mathcal{T}_e(X)$ to be the set of sub-objects of $X$ whose canonical inclusion to $X$ is transformed by $j'$ into an isomorphism. It is immediately verified, using the assumptions made on $j'$, that this defines a topology $\mathcal{T}_e$ over $\mathcal{c}$. We have therefore defined an application $\Psi:\mathrm{Caf}\to\mathscr{T}$.
\begin{theorem}[\textbf{J. Giraud}]\label{site topology correspond to subcategory}
The map $\Phi$ is an bijection with inverse map $\Psi$.
\end{theorem}
\begin{proof}
The map $\Psi\circ\Phi$ is the identity in view of \cref{site morphism of presheaf bicovering iff} and the axioms (T1), (T2). Conversely, let $i':\mathcal{C}'\to\PSh(\mathcal{C})$ be an element of $\mathrm{Caf}$ with left adjoint $j'$, $\mathcal{T}_e$ be the corresponding topology, and $\mathcal{C}_e$ be the sheaf category defined by $\mathcal{T}_e$. By using the definition of $\mathcal{T}_e$ and bicovering morphisms, we can then prove the equivalence of the following conditions for a morphism $u$ of $\PSh(\mathcal{C})$:
\begin{enumerate}
\item[(a)] $u$ is bicovering for the topology $\mathcal{T}_e$.
\item[(b)] $u$ is transformed by $j'$ to an isomorphism.
\end{enumerate}
It is clear that $\mathcal{C}'$ is a full subcategory of $\mathcal{C}_e$, so it suffices to prove that any sheaf $F$ for the topology $\mathcal{T}_e$ is an object of $\mathcal{C}'$. However, by the above equivalence, the morphism $F\to i'\circ j'(F)$ is bicovering ($j'\circ i'$ is isomorphic to the identity functor since $j'$ is exact), whose source and target are sheaves. We deduce from \cref{site morphism of presheaf bicovering iff}(\rmnum{2}) that $F\to i'\circ j'(F)$ is an isomorphism, so $F\in\mathcal{C}'$ (recall that $\mathcal{C}'$ is strictly full).
\end{proof}
\section{Functoriality of categories of sheaves}
In \cref{category presheaf section}, we studied the behavior of the categories of presheaves with respect to the functors between the underlying categories. In this section, we extend these results to sites and categories of sheaves.
\subsection{Continuous functors}
Let $\mathcal{C}$ and $\mathcal{D}$ be two $\mathscr{U}$-sites. A functor $u:\mathcal{C}\to\mathcal{D}$ on the underlying categories is called \textbf{continuous} if for any sheaf of sets $F$ on $\mathcal{D}$, the presheaf $X\mapsto F(u(X))$ over $\mathcal{C}$ is a sheaf. Equivalenyly, this means there exists a functor $u_s:\Sh(\mathcal{D})\to\Sh(\mathcal{C})$ such that we have a commutative diagram
\begin{equation}\label{site small continuous functor iff-1}
\begin{tikzcd}
\Sh(\mathcal{D})\ar[d,swap,"i_\mathcal{D}"]\ar[r,"u_s"]&\Sh(\mathcal{C})\ar[d,"i_\mathcal{C}"]\\
\PSh(\mathcal{D})\ar[r,"u^*"]&\PSh(\mathcal{C})
\end{tikzcd}
\end{equation}
where if $i_\mathcal{C}:\PSh(\mathcal{C})\to\Sh(\mathcal{C})$ and $i_\mathcal{D}:\PSh(\mathcal{D})\to\Sh(\mathcal{D})$ are the canonical inclusion functors.
\begin{proposition}\label{site small continuous functor iff}
Let $\mathcal{C}$ be a small site, $\mathcal{D}$ be a $\mathscr{U}$-site and $u:\mathcal{C}\to\mathcal{D}$ be a functor on the underlying categories. The following properties are equivalent:
\begin{enumerate}
\item[(\rmnum{1})] The functor $u$ is continuous.
\item[(\rmnum{2})] For any object $X$ of $C$ and any covering sieve $R$ of $X$, the morphism $u_!(R)\hookrightarrow u(X)$ is bicovering in $\PSh(\mathcal{D})$.
\item[(\rmnum{3})] For any bicovering family $\{H_i\to K\}$ of $\PSh(\mathcal{C})$, the family $\{u_!(H_i)\to u_!(K)\}$ is bicovering in $\PSh(\mathcal{D})$.
\item[(\rmnum{4})] There exists a functor $u^s:\Sh(\mathcal{D})\to\Sh(\mathcal{D})$ which commutes with inductive limits and extends $u$, i.e. such that the following diagram is commutative up to isomorphisms:
\begin{equation}\label{site small continuous functor iff-2}
\begin{tikzcd}
\mathcal{C}\ar[r,"u"]\ar[d,swap,"\eps_\mathcal{D}"]&\mathcal{D}\ar[d,"\eps_\mathcal{D}"]\\
\Sh(\mathcal{C})\ar[r,"u^s"]&\Sh(\mathcal{D})
\end{tikzcd}
\end{equation}
\end{enumerate}
Moreover, if $\mathcal{C}$ is only assumed to be a $\mathscr{U}$-site, we still have the equivalence (\rmnum{1})$\Leftrightarrow$(\rmnum{4}), and the functor $u^s$ of (\rmnum{4}) is necessarily a left adjoint of the functor $u_s:\Sh(\mathcal{D})\to\Sh(\mathcal{C})$ (therefore uniquely determined up to isomorphisms).
\end{proposition}
\begin{proof}
The proof of (\rmnum{1})$\Leftrightarrow$(\rmnum{4}) when $\mathcal{C}$ is a $\mathscr{U}$-site will be done in \cref{*}, so let us suppose that $\mathcal{C}$ is small (so that the functor $u_!$ is \textit{defined}, cf. SGA4, \Rmnum{1}, 5.0). Let $X$ be an object of $X$ and $R\hookrightarrow X$ be a covering sieve. Then for any sheaf $F$ over $\mathcal{D}$, by the adjunction property of $u_!$ and $u^*$, we have a commutative diagram
\[\begin{tikzcd}
\Hom(u_!(X),F)\ar[r]\ar[d]&\Hom(u_!(R),F)\ar[d]\\
\Hom(X,u^*(F))\ar[r]&\Hom(R,u^*(F))
\end{tikzcd}\]
where the vertical are isomorphisms. In view of \cref{site morphism of presheaf bicovering iff}, the equivalence of (\rmnum{1}), (\rmnum{2}), (\rmnum{3}) is then easily deduced.\par
We now prove (\rmnum{1})$\Rightarrow$(\rmnum{4}), so assume that $u$ is a continuous functor. For any sheaf $G$ on $\mathcal{C}$, we define $u^s(G)=(u_!(i_\mathcal{C}(G)))^\#$ for the functor $u^s$. Then for any sheaf $F$ on $\mathcal{D}$, we have canoical isomorphisms
\begin{align*}
\Hom(G,u_s(F))&\cong\Hom(i_\mathcal{C}(G),i_\mathcal{C}(u_s(F)))\cong\Hom(i_\mathcal{C}(G),u^*(i_\mathcal{D}(F)))\\
&=\Hom(u_!(i_\mathcal{C}(G)),i_\mathcal{D}(F))=\Hom((u_!(i_\mathcal{C}(G)))^\#,F)
\end{align*}
(the first isomorphism follows from the fact that $i_\mathcal{C}$ is fully faithful, the second one is the commutative diagram (\ref{site small continuous functor iff-1}), and the third one follows from adjunction). Therefore the functor $u^s$ is left adjoint to $u_s$, and it clearly commutes with inductive limits. For any presheaf $K$ over $\mathcal{C}$ and sheaf $F$ over $\mathcal{D}$, we also have the following canonical isomorphisms
\begin{align*}
\Hom(u^s(K^\#),F)&\cong \Hom(K^\#,u_s(F))\cong\Hom(K,i_\mathcal{C}(u_s(F)))\cong\Hom(K,u^*(i_\mathcal{D}(F)))\\
&\cong \Hom(u_!(K),i_\mathcal{D}(F))\cong\Hom((u_!(K))^\#,F).
\end{align*}
In particular, if $K$ is representable, we then obtain a commutative diagram
\[\begin{tikzcd}
\mathcal{C}\ar[r,"u"]\ar[d,swap,"\eps_\mathcal{D}"]&\mathcal{D}\ar[d,"\eps_\mathcal{D}"]\\
\Sh(\mathcal{C})\ar[r,"u^s"]&\Sh(\mathcal{D})
\end{tikzcd}\]
where we use the commutative diagram (SGA4 \Rmnum{1}, 5.4 (3)).\par
Conversely, assume that we are given a functor $u^s:\Sh(\mathcal{D})\to\Sh(\mathcal{D})$ satsisfying the conditions of (\rmnum{4}). Consider the diagram
\[\begin{tikzcd}
\mathcal{C}\ar[r,"u"]\ar[d,swap,"h"]&\mathcal{D}\ar[d,"h"]\\
\PSh(\mathcal{C})\ar[r,"u_!"]\ar[d,swap,"(-)^\#"]&\PSh(\mathcal{D})\ar[d,"(-)^\#"]\\
\Sh(\mathcal{C})\ar[r,"u^s"]&\Sh(\mathcal{D})
\end{tikzcd}\]
By the definition of $u_!$, the upper square is commutative, and the vertical compositions are $\eps_\mathcal{C}$ and $\eps_\mathcal{D}$, respectively. For any object $K$ in $\PSh(\mathcal{C})$, we have a canonical isomorphism (SGA4, \Rmnum{1}, 3.4)
\[\rlim_{(X\to K)\in\mathcal{C}_{/K}}h(X)\stackrel{\sim}{\to}K.\]
As the functors $u_!$, $u^s$ and $\#$ commutes with inductive limits, we then deduce the following isomorphisms
\begin{gather*}
\rlim_{(X\to K)\in\mathcal{C}_{/K}}u^s(X^\#)\stackrel{\sim}{\to}u^s(K^\#),\quad \rlim_{(X\to K)\in\mathcal{C}_{/K}}(u_!(X))^\#\stackrel{\sim}{\to}(u_!(K))^\#.
\end{gather*}
Since $u_!h=hu$, the commutativity of (\ref{site small continuous functor iff-2}) then provides an isomorphism $u^s(K^\#)\cong(u_!(K))^\#$, which is immediately verified to be funcotrial on $K$. The diagram
\[\begin{tikzcd}
\PSh(\mathcal{C})\ar[r,"u_!"]\ar[d,swap,"(-)^\#"]&\PSh(\mathcal{D})\ar[d,"(-)^\#"]\\
\Sh(\mathcal{C})\ar[r,"u^s"]&\Sh(\mathcal{D})
\end{tikzcd}\]
is then commutative up to isomorphisms. As $\#\circ i$ is isomorphic to the identity functor, we obtain an isomorphism $u^s\cong(u_!i)^\#$, whence the uniqueness of $u^s$. Now let $v:H\to K$ be a bicovering of $\PSh(\mathcal{C})$. Then $v^\#$ is an isomorphism by \cref{site morphism of presheaf bicovering iff}, so $(u_!(v))^\#$ (isomorphic to $u^s(v^\#)$) is also an isomorphism, which means $u_!(v)$ is bicovering. This proves (\rmnum{4})$\Rightarrow$(\rmnum{2}) and completes the proof.
\end{proof}
The functor $u^s:\Sh(\mathcal{C})\to\Sh(\mathcal{D})$ of \cref{site small continuous functor iff} can be intepreted as the "inverse image" functor induced by the functor $u$. Its properties are summarized in the following proposition:
\begin{proposition}\label{site continuous functor sheaf inverse image prop}
Let $u:\mathcal{C}\to\mathcal{D}$ be a continuous functor from a small site $\mathcal{C}$ to a $\mathscr{U}$-site $\mathcal{D}$.
\begin{enumerate}
\item[(a)] The functor $u^s$ is left adjoint to $u_s$.
\item[(b)] We have canonical isomorphisms $u^s\cong (u_!i)^\#$ and $u^s\#\cong \# u_!$.
\item[(c)] The functor $u^s$ commutes with inductive limits.
\item[(d)] If the functor $u_!$ is left exact, then $u^s$ is also left exact. More generally, the functor $u^s$ commutes with finite projective limit if the functor $u_!$ does.
\end{enumerate}
\end{proposition}
\begin{proof}
The first three assertions are proved in \cref{site small continuous functor iff}, and (d) follows from the isomorphism $u^s\cong (u_!i)^\#$ in (b), since $i$ and $\#$ commute with finite projective limits.
\end{proof}
\begin{remark}
Combing (SGA4 \Rmnum{1}, 5.4) and \cref{site continuous functor sheaf inverse image prop}, we obtain a diagram
\[\begin{tikzcd}
\mathcal{C}\ar[r,"u"]\ar[d,swap,"h"]&\mathcal{D}\ar[d,"h"]\\
\PSh(\mathcal{C})\ar[r,"u_!"]\ar[d,swap,"(-)^\#"]&\PSh(\mathcal{D})\ar[d,"(-)^\#"]\\
\Sh(\mathcal{C})\ar[r,"u^s"]&\Sh(\mathcal{D})
\end{tikzcd}\]
where the upper square is commutative and the lower square commutes up to isomorphisms. But the functors $\#$, $u_!$ and $u^s$ are only defined up to isomorphism. One can check that we can choose them such that:
\begin{enumerate}
\item[(a)] The composition $h^\#$ is injective on the set of objects.
\item[(b)] The diagram
\[\begin{tikzcd}
\mathcal{C}\ar[r,"u"]\ar[d,swap,"h^\#"]&\mathcal{D}\ar[d,"h^\#"]\\
\Sh(\mathcal{C})\ar[r,"u^s"]&\Sh(\mathcal{D})
\end{tikzcd}\]
is commutative.
\end{enumerate}
More precisely, we can choose the sheafification functors on $\PSh(\mathcal{C})$ and $\PSh(\mathcal{D})$ to satisfy condition (a). Then we can choose the functor $u^s$ in such a way that condition (b) is fulfilled.
\end{remark}
\section{Grothendieck Topoi}
We have seen in \cref{site category sheaf exactness property subsection} that various exactness properties for the category of sheaves over $\mathcal{C}$, where $\mathcal{C}$ is a small site, which can be expressed by saying that in many respects such a category inherit familiar properties from the category $\mathbf{Set}$ of (small) sets. In this section, we consider categories that can be realized as the category of sheaves of sets over a site. As we shall see, topoi behave much like the category of sets and possess a notion of localization.
\subsection{Definition and characterization of topos}
A category $\mathcal{X}$ is called a \textbf{$\mathscr{U}$-topos}, or simply a \textbf{topos}, if there exists a small site $\mathcal{C}$ such that $\mathcal{X}$ is equivalent the the category $\Sh(\mathcal{C})$ of sheaves of sets over $\mathcal{C}$. Let $\mathcal{X}$ be a topos. We will always endow $\mathcal{X}$ with the canonical topology, so that it is a site (and in fact a $\mathscr{U}$-site as we shall see). Unless futher explaination, we will not consider any other topology on $\mathcal{X}$.\par
We have seen in (SGA4 \Rmnum{2}, 4.8), (SGA4 \Rmnum{2}, 4.10) and (SGA4 \Rmnum{2}, 4.11) that a $\mathscr{U}$-topos $\mathcal{X}$ is a $\mathscr{U}$-category satisfying the following properties (these are called \textbf{Giraud's conditions}):
\begin{enumerate}
\item[(a)] Finite projective limits exist in $\mathcal{X}$.
\item[(b)] Small direct sums exist in $\mathcal{X}$ and are disjoint.
\item[(c)] All equivalence relations in $\mathcal{X}$ are effective.
\item[(d)] $\mathcal{X}$ admits a small set of generators.
\end{enumerate}
In fact, we will see that these intrinsic properties characterize $\mathscr{U}$-topoi:
\begin{theorem}[\textbf{J. Giraud}]\label{topos Giraud characterization}
Let $\mathcal{X}$ be a $\mathscr{U}$-category. The following properties are equivalent:
\begin{enumerate}
\item[(\rmnum{1})] $\mathcal{X}$ is a $\mathscr{U}$-topos.
\item[(\rmnum{1}')] There exists a small site $\mathcal{C}$ endowed with a subcanonical topology such that projective limits exist in $\mathcal{C}$ and $\mathcal{X}$ is equivalent to the category of sheaves over $\mathcal{C}$.
\item[(\rmnum{2})] $\mathcal{X}$ satisfies Giraud's conditions.
\item[(\rmnum{3})] The sheaves over $\mathcal{X}$ (endowed with the canonical topology) are representable, and $\mathcal{X}$ admits a small set of generators.
\item[(\rmnum{4})] There exists a small category $\mathcal{C}$ and a fully faithful functor $i:\mathcal{X}\to\PSh(\mathcal{C})$ which admits an exact left adjoint.
\end{enumerate}
\end{theorem}
\begin{proof}
The equivalence of (\rmnum{1}) and (\rmnum{4}) follows from \cref{site topology correspond to subcategory}, and we have remarked that (\rmnum{1})$\Rightarrow$(\rmnum{2}). As (\rmnum{1}') clearly implies (\rmnum{1}), it remains to prove that (\rmnum{2})$\Rightarrow$(\rmnum{3}) and (\rmnum{3})$\Rightarrow$(\rmnum{1}').\par
We first deal with the implication (\rmnum{3})$\Rightarrow$(\rmnum{1}'), so let $\X=(X_i)_{i\in I}$ be a small family of generators of $\mathcal{X}$ and assume that any sheaf over $\mathcal{X}$ is representable. As $\mathcal{X}$ is a $\mathscr{U}$-category, the set of isomorphism classes of finite diagrams in $\mathcal{X}$ with objects belonging to $\X$ is $\mathscr{U}$-small. Therefore, the smallest set $\X'$ of objects of $\mathcal{X}$ containing the finite projective limits of objects of $\X^{(0)}$ is a countable union of small sets, hence is small. By redefining $\X^{(n+1)}=(\X^{(n)})'$ and $\X=\bigcup_n\X^{(n)}$, we see that, by adding generators if necessary, we can suppose that $\X$ is stable under finite projective limits.
\end{proof}
\begin{corollary}\label{topos subcategory sheaf cat equivalent iff generating}
Let $\mathcal{X}$ be a $\mathscr{U}$-topos and $\mathcal{C}$ be a full subcategory of $\mathcal{X}$ endowed with the induced topology (SGA4 \Rmnum{3}, 3.1). Consider the functor
\[\mathcal{C}\to\Sh(\mathcal{C})\]
which associates an object $X$ of $\mathcal{X}$ the restriction to $\mathcal{C}$ of the sheaf represented by $X$. This functor is an equivalence of categories if and only if $\Ob(\mathcal{C})$ is a topological generating family of $\mathcal{X}$.
\end{corollary}
\begin{proof}
It suffices to note that the considered functor factors into $\mathcal{X}\to\Sh(\mathcal{X})\to\Sh(\mathcal{C})$, where the first one is an equivalence by \cref{topos Giraud characterization}(\rmnum{3}). The question is then reduced to determining whether $\Sh(\mathcal{X})\to\Sh(\mathcal{C})$ is an equivalence, which can be deduced from the comparision lemma (SGA4 \Rmnum{3}, 4.1).
\end{proof}
\section{Cohomology of topoi}

\subsection{Complements on abelian categories}
In this paragraph, we recall some definitions and results concerning abelian categories which will be used later in the discussion of cohomology of topos. The material we give here can all be found on the famuous papar "Tohoku".

\begin{proposition}\label{abelian cat with generator AB5 iff}
Let $\mathcal{A}$ be an abelian category with a generator. The following conditions are equivalent:
\begin{enumerate}
    \item[(\rmnum{1})] The category $\mathcal{A}$ verifies the axiom (AB5): $\mathcal{A}$ possesses small direct sums and if $(X_i)_{i\in I}$ is a filtered small family of sub-objects of an object $X$ of $\mathcal{A}$ and $Y$ is a sub-object of $X$, then
    \[(\sup_iX_i)\cap Y=\sup_i(X_i\cap Y).\]
    \item[(\rmnum{2})] Small filtered limits exist in $\mathcal{A}$ and commute with finite projective limits.
\end{enumerate}
\end{proposition}

An abelian category $\mathcal{A}$ possessing a generator and satisfying
the axiom (AB5) is called a \textbf{Grothendieck category}. It can be proved that any Grothendieck category has enough injectives i.e. any object embeds itself in a injective object. Moreover, according to the result already cited, small products are representable in $\mathcal{A}$.

\begin{proposition}\label{abelian cat adjoint functor exact iff injective to injective}
Let $\mathcal{A}$ and $\mathcal{B}$ be abelian categories and $F\dashv G:\mathcal{A}\to\mathcal{B}$ be an adjoint pair of additive functors. Consider the following properties:
\begin{enumerate}
    \item[(\rmnum{1})] The functor $F$ is exact.
    \item[(\rmnum{2})] The functor $G$ transforms injective object in $\mathcal{B}$ to injective objects in $\mathcal{A}$.
\end{enumerate}
Then we always have (\rmnum{1})$\Rightarrow$(\rmnum{2}). If any nonzero object of $\mathcal{B}$ is the source of a nonzero morphism into an injective object (this is true for example if $\mathcal{B}$ has enough injectives), then (\rmnum{2})$\Rightarrow$(\rmnum{1}).
\end{proposition}

\begin{proposition}\label{abelian cat adjoint functor engouh injective if}
Let $\mathcal{A}$ and $\mathcal{B}$ be abelian categories and $F\dashv G:\mathcal{A}\to\mathcal{B}$ be an adjoint pair of additive functors. Suppose that:
\begin{enumerate}
    \item[(a)] the category $\mathcal{B}$ has enough injectives;
    \item[(b)] the equivalent conditions of \cref{abelian cat adjoint functor exact iff injective to injective} are satisfied;
    \item[(c)] the functor $F$ is faithful. 
\end{enumerate}
Then the category $\mathcal{A}$ has enough injectives.
\end{proposition}

\begin{remark}
The category of abelian groups has enough injectives, so by applying \cref{abelian cat adjoint functor engouh injective if}, we deduce that any category of modules over a ring has enough injectives. Then applying the result of [5] (used in the proof of \cref{abelian cat with generator AB5 iff}) and \cref{abelian cat adjoint functor exact iff injective to injective}, we deduce that any Grothendieck category has enough injectives, which provides a new proof of this fact.
\end{remark}

\begin{proposition}\label{abelian category G spectral sequence iff injective to acyclic}
Let $\mathcal{A}$, $\mathcal{B}$, $\mathcal{C}$ be abelian categories and $F:\mathcal{A}\to\mathcal{B}$, $G:\mathcal{B}\to\mathcal{C}$ be two left exact additive functors. Suppose that $\mathcal{A}$ and $\mathcal{B}$ have enough injectives. Then the following conditions are equivalent:
\begin{enumerate}
    \item[(\rmnum{1})] There exists a spectral functor
    \[E_2^{p,q}=R^pG\circ R^qF\Rightarrow R^{p+q}(G\circ F).\]
    \item[(\rmnum{2})] The functor $F$ transforms injective objects to $G$-acyclic objects. 
\end{enumerate}
\end{proposition}
\begin{proof}
The implication (\rmnum{1})$\Rightarrow$(\rmnum{2}) is trivial because it suffices to apply this spectral sequence to an injective object of $\mathcal{A}$. The converse implication is the famous Grothendieck spectral sequence.
\end{proof}

\begin{proposition}\label{abelian category acyclic collection if}
Let $F:\mathcal{A}\to\mathcal{B}$ be a left exact additive functor of abelian categories. Let $\mathfrak{M}$ be a collection of object of $\mathcal{A}$ possessing the following properties:
\begin{enumerate}
    \item[(a)] Any object of $\mathcal{A}$ can be embedded into an element of $\mathfrak{M}$.
    \item[(b)] If $X\oplus Y$ belongs to $\mathfrak{M}$, then $X$ and $Y$ belong to $\mathfrak{M}$.
    \item[(c)] If we have an exact sequence $0\to X'\to X\to X''\to 0$ is where $X'$ and $X$ belong to $\mathfrak{M}$, then $X''$ belongs to $\mathfrak{M}$ and the sequence
    \[\begin{tikzcd}
        0\ar[r]&F(X')\ar[r]&F(X)\ar[r]&F(X'')\ar[r]&0
    \end{tikzcd}\]
    is exact. Moreover, the zero object belongs to $\mathfrak{M}$.
\end{enumerate}
Then any injective object belongs to $\mathfrak{M}$, and the objects of $\mathfrak{M}$ are $F$-acyclic, i.e. for any $p>0$ and any $X\in\mathfrak{M}$ we have $R^pF(X)=0$. In particular, the resolutions by objects of $\mathfrak{M}$ computes the derived functors of $F$.
\end{proposition}

\begin{proposition}\label{abelian category generaring family in universe prop}
Let $\mathscr{U}\sub\mathscr{V}$ be universes, $\mathcal{A}$ (resp. $\mathcal{B}$) be an abelian $\mathscr{U}$-category (resp. $\mathscr{V}$-category) satisfying (AB5) and possesses a $\mathscr{U}$-small (resp. $\mathscr{V}$-small) topological generating family. Let $\eps:\mathcal{A}\to\mathcal{B}$ be a fully faithful and exact functor. The following conditions are equivalent:
\begin{enumerate}
    \item[(\rmnum{1})] There exists a generaring family $(X_i)_{i\in I}$ of $\mathcal{A}$ such that $(\eps(X_i))_{i\in I}$ is a generating family of $\mathcal{B}$.
    \item[(\rmnum{1}')] Any object of $\mathcal{B}$ is isomorphic to a quotient of an object of the form $\bigoplus_{\alpha\in A}\eps(Y_\alpha)$, where $A$ is $\mathscr{V}$-small.
\end{enumerate}
Under these equivalent conditions, we have:
\begin{enumerate}
    \item[(\rmnum{2})] the functor $\eps$ transforms $\mathscr{U}$-small products to products (hence commutes with $\mathscr{U}$-small projective limits),
\end{enumerate}
and the following conditions are equivalent:
\begin{enumerate}
    \item[(a)] For any object $Y$ of $\mathcal{A}$, any sub-object of $\eps(Y)$ is isomorphic to the image under $\eps$ of a sub-object of $Y$.
    \item[(a')] There exists a generating family $(X_i)_{i\in I}$ of $\mathcal{A}$ such that the family $(\eps(X_i))_{i\in I}$ is generating in $\mathcal{B}$ and that for any $i\in I$, any sub-object of $\eps(X_i)$ is isomorphic to the image under $\eps$ of a sub-object of $X_i$.
    \item[(b)] Any object of $\mathcal{B}$ is isomorphic to a sub-object of an object of the form $\prod_{\alpha\in A}\eps(Y_\alpha)$, where $A$ is $\mathscr{V}$-small.
    \item[(c)] $\eps$ commutes with $\mathscr{U}$-small direct sums (hence commutes with $\mathscr{U}$-small inductive limits).
\end{enumerate}
Finally, if these conditions are satisfied, the functor $\eps$ transforms injective objects to injective objects.
\end{proposition}

\begin{remark}
If $\mathscr{U}=\mathscr{V}$, then the conditions (\rmnum{1}) and (a') imply that $\eps$ is an equivalence of categories, since we then have conditions (b), (\rmnum{2}) and (b).
\end{remark}

\subsection{Flat modules on ringed topoi}
Let $(\mathcal{X},A)$ be a ringed topos. A right (resp. left) $A$-module $M$ is called \textbf{flat} if the functor $M\otimes_A(-)$ (resp. $(-)\otimes_AM$) from the category of left (resp. right) $A$-modules to the category of abelian sheaves on $\mathcal{X}$ is exact.

\begin{proposition}\label{ringed topos flat module prop}
Let $M$ be a $(B,A)$-bimodule.
\begin{enumerate}
    \item[(a)] The following properties are equivalent:
    \begin{enumerate}
        \item[(\rmnum{1})] The right $A$-module $M$ is flat.
        \item[(\rmnum{2})] For any injective $B$-module $I$, the right $A$-module $\sHom_B(M,I)$ is injective.
    \end{enumerate}
    \item[(b)] A filtered limit of flat modules is flat.
    \item[(c)] If $M^\bullet$ is an exact complex of flat modules, then for any module $F$, the complex $M^\bullet\otimes_AF$ is exact.
\end{enumerate}
\end{proposition}
\begin{proof}
By the adjunction (SGA4 \Rmnum{4}, 12.12), we have a canonical isomorphism
\begin{align}\label{ringed topos flat module prop-1}
\Hom_B(M\otimes_A(-),-)\stackrel{\sim}{\to}\Hom_A(-,\sHom_B(M,-)).
\end{align}
To establish the equivalence of (\rmnum{1}) and (\rmnum{2}), it then suffices to apply \cref{abelian cat adjoint functor exact iff injective to injective}. The isomorphism (\ref{ringed topos flat module prop-1}) also shows that tensor products commutes with inductive limits, so the fact the filtered inductive limits are exact (\cref{abelian cat with generator AB5 iff}) implies the second assertion. To see that the complex $M^\bullet\otimes_AF$ is exact if $M^\bullet$ is an exact flat complex, it suffices to show that for any injective abelian sheaf $I$, the complex $\Hom^\bullet_\Z(M^\bullet\otimes_AF,I)$ is exact. This complex is isomorphic, in view of the adjunction formula, to the complex $\Hom^\bullet_A(F,\sHom(M^\bullet,I))$, and by the equivalence (\rmnum{1})$\Rightarrow$(\rmnum{2}), the complex $\sHom_\Z(M^\bullet,I)$ is an exact complex whose objects are injective, whence our conclusion.
\end{proof}

\begin{proposition}\label{ringed topos flat module shrink is flat}
Let $(\mathcal{X},A)$ be a ringed topos, $X$ be an object of $\mathcal{X}$, $j:\mathcal{X}_{/X}\to\mathcal{X}$ be the localiztion functor, and $M$ be a flat $A|_X$-module. Then $j_!(M)$ is a flat $A$-module. In particular, $A_X$ is a flat $A$-module
\end{proposition}
\begin{proof}
Suppose that $M$ is a right $A|_X$-module. For any left $A$-module $N$, we have a canonical isomorphism (SGA4 \Rmnum{4}, 12)
\[N\otimes_Aj_!(M)\stackrel{\sim}{\to}j_!(N\otimes_{A|_X}M).\]
The functors $j_!$ and $j^*$ are exact by (SGA4 \Rmnum{4}, 11.3.1) and (SGA4 \Rmnum{4}, 11.12.2), and by hypothesis the functor $(-)\otimes_{A|_X}M$ is exact. We then conclude that the functor $(-)\otimes_Aj_!(M)$ is exact, so $j_!(M)$ is exact.
\end{proof}

\begin{proposition}[\textbf{Projection formula for closed immersions}]\label{ringed topos closed immersion projection formula}
Let $(\mathcal{X},A)$ be a ringed topos, $i:\mathcal{Z}\to \mathcal{X}$ be a closed subtopos of $\mathcal{X}$, and put $A_{/\mathcal{Z}}=i^*(A)$. Then for any right $A_{/\mathcal{Z}}$-module $M$ and any left $A$-module $N$, we have a canonical isomorphism
\begin{align}\label{ringed topos closed immersion projection formula-1}
i_*(M\otimes_{A_{/\mathcal{Z}}}i^*(N))\stackrel{\sim}{\to}i_*(M)\otimes_AN.
\end{align}
\end{proposition}
\begin{proof}
Let $\mathcal{U}$ be the open complement of $\mathcal{Z}$ and $j:\mathcal{U}\to\mathcal{X}$ be the canonical open immersion. We have $j_*(i_*(M)\otimes_AN)\stackrel{\sim}{\to}0\otimes_{A_{/U}}j^*(N)$ (SGA4 \Rmnum{4}, 12), so $i_*(M)\otimes_AN$ is supported in $\mathcal{Z}$, and the corresponding unit morphism
\[i_*(M)\otimes_AN\to i_*i^*(i_*(M)\otimes_AN)\]
is an isomorphism (SGA4 \Rmnum{4}, 14). We have $i^*(i_*(M)\otimes_AN)\cong i^*i_*(M)\otimes_{A_{/\mathcal{Z}}}i^*(N)$ by (SGA4 \Rmnum{4}, 12), and $i^*i_*(M)\cong M$ since $i$ is a closed immersion; whence the canonical isomorphism.
\end{proof}

\begin{corollary}\label{ringed topos closed immersion flat module direct image flat}
Let $(\mathcal{X},A)$ be a ringed topos, $i:\mathcal{Z}\to \mathcal{X}$ be a closed subtopos of $\mathcal{X}$, and put $A_{/\mathcal{Z}}=i^*(A)$. For any flat $A_{/\mathcal{Z}}$-module $M$, the $A$-module $i_*(M)$ is flat.
\end{corollary}
\begin{proof}
It follows from \cref{ringed topos closed immersion projection formula} and (SGA4 \Rmnum{4} 14) that the functor $N\mapsto i_*(M)\otimes_AN$ is exact, so $i_*(M)$ is flat.
\end{proof}

Let $(\mathcal{X},A)$ be a ringed topos, $x:\mathcal{P}\to \mathcal{X}$ be a point of $\mathcal{X}$ (SGA \Rmnum{4}, 6.1), and $\mathfrak{U}(x)$ be the category of neighborhoods of $x$ (SGA \Rmnum{4}, 6.8). For any object $V$ of $\mathfrak{U}(x)$, we can associates an object of $\mathcal{X}$ (still denote by $V$), and a point $x_V:\mathcal{P}\to \mathcal{X}/V$ of $\mathcal{X}/V$. Moreover, any morphism $u:V\to W$ in $\mathfrak{U}(x)$ corresponds to a commutative diagram of topos (SGA \Rmnum{4}, 6.7)
\[\begin{tikzcd}[row sep=12mm,column sep=12mm]
\mathcal{P}\ar[rd,swap,"x_W"]\ar[r,"x_V"]&\mathcal{X}/V\ar[d,"j_u"]\\
&\mathcal{X}/W
\end{tikzcd}\]

As an application of the materials given in this paragraph, we construct the \v{C}ech complex for a family of morphisms in E and prove its exactness under certain circumstances, that is, for epimorphic families. Now let $\mathcal{X}$ be a topos and $\mathfrak{U}=\{U_i\to X\}_{i\in I}$ be a small family of morphisms. For any ordered set $[n]=\{0,\dots,n\}$, we define
\[S_n(\mathfrak{U})=\coprod_{f:[n]\to I}U_f\]
where the direct sum is taken over all maps $f:[n]\to I$, and for such a map $f$ we define $U_f$ to be
\[U_f:=U_{f(1)}\times_XU_{f(2)}\times_X\cdots\times_XU_{f(n)}.\]
For any nondecreasing map $g:[m]\to[n]$, we have a morphism
\begin{align}\label{ringed topos Cech complex for covering-1}
s(g):S_n(\mathfrak{U})\to S_m(\mathfrak{U})
\end{align}
defined in the following way: for any map $f:[n]\to I$, the restriction of $s(g)$ to the component $U_f$ is the composition morphism
\[\begin{tikzcd}
U_f\ar[r,"s_f(g)"]&U_{fg}\ar[r,hook]&S_m(\mathfrak{U})
\end{tikzcd}\]
where $s_f(g):U_f\to U_{fg}$ is the unique morphism such that for any $i\in[m]$, we have 
\begin{align}\label{ringed topos Cech complex for covering-2}
\pr_{f(g(i))}s_f(g)=\pr_{f(g(i))},
\end{align}
where $\pr_j$ is the $j$-th projection. We therefore obtain a contravariant functor $[n]\mapsto S_n$ from the category of finite sets to $\mathcal{X}$, or in other words, a semi-simplicial object $S_\bullet(\mathfrak{U})$ of $\mathcal{X}$. Note that this complex is canonically augmented by $X$. Any functor of $\mathcal{X}$ into a category $\mathcal{C}$ transforms $S_\bullet(\mathfrak{U})$ into a simplicial object of $\mathcal{C}$. In particular, if $A$ is a ring object of $\mathcal{X}$, the "free $A$-module functor" transforms $S_\bullet(\mathfrak{U})$ into a simplicial complex of $A$-bimodules augmented by $A_X$, which is denoted by $A_\bullet(\mathfrak{U})$. We have
\begin{equation}\label{ringed topos Cech complex for covering-3}
A_n(\mathfrak{U})=\bigoplus_{f:[n]\to I}A_{U_{fg}}.
\end{equation}
Let $\{d_i:S_n(\mathcal{C})\to S_{n-1}(\mathcal{C})\}_{0\leq i\leq n}$ be the face maps of $S_\bullet(\mathcal{C})$, then the complex $A_\bullet(\mathfrak{U})$ has the following form
\begin{equation}\label{ringed topos Cech complex for covering-4}
\begin{tikzcd}[column sep=12mm]
\cdots\ar[r,shift left=8pt,"s_0"description]\ar[r,"s_1"description]\ar[r,shift right=8pt,"s_2"description]
&\bigoplus_{i,j}A_{U_i\times_XU_j}\ar[r,shift left=4pt,"s_0"description]\ar[r,shift right=4pt,"s_1"description]
&\bigoplus_iA_{U_i}\ar[r]
&A_X
\end{tikzcd}
\end{equation}
For such a complex, we can define a differential complex, augmented by $A_X$, by simply setting $d=\sum_i(-1)^is_i$:
\begin{equation}\label{ringed topos Cech complex for covering-5}
\begin{tikzcd}
\cdots\ar[r,"d"]&\bigoplus_{i,j}A_{U_i\times_XU_j}\ar[r,"d"]&A_X
\end{tikzcd}
\end{equation}

\begin{proposition}\label{ringed topos Cech complex resolution for epimorphic}
If the famly $\mathfrak{U}$ is epimorphic, the differential complex (\ref{ringed topos Cech complex for covering-5}) is exact and hence a resolution of $A_X$.
\end{proposition}
\begin{proof}
We denote by $\Z$ the constant sheaf with values $\Z$. By the definition of the "free $A$-module functor", we have, for any object $Y$ of $\mathcal{X}$,
\[A_Y\cong\Z_Y\otimes_\Z A,\]
whence an isomorphism
\[A_\bullet(\mathfrak{U})\cong\Z_\bullet(\mathfrak{U})\otimes_\Z A.\]
As the components of $\Z_\bullet(\mathfrak{U})$ are all flat $\Z$-modules by \cref{ringed topos flat module shrink is flat}, it suffices to prove the proposition for $A=\Z$.\par
Suppose first that $\mathcal{X}$ is the topos of sets. Then the augmented complex $S_\bullet(\mathfrak{U})$ is the direct sum of augmented complexed of the form
\[\begin{tikzcd}
\cdots\ar[r,shift left=6pt]\ar[r,shift left=2pt]\ar[r,shift right=2pt]\ar[r,shift right=6pt]
&S\times S\times S\ar[r,shift left=4pt]\ar[r,shift right=4pt]\ar[r]
&S\times S\ar[r,shift left=2pt]\ar[r,shift right=2pt]
&S\ar[r]&\ast
\end{tikzcd}\]
where $\ast$ is the set with a single element. Since each of these complexes is homotopically trivial, we then conclude that $S_\bullet(\mathfrak{U})$ is a homotopically trivial augmented complex, whence the proposition in this case.\par
Now let $p:\mathbf{Set}\to\mathcal{X}$ be a point of $\mathcal{X}$. As the formation of the complex $\Z_\bullet(\mathfrak{U})$ commutes with inverse image functors between topos, $p^*(\Z_\bullet(\mathfrak{U}))\cong\Z_\bullet(p^*(\mathfrak{U}))$ is a resolution of $\Z_{p^*(X)}\cong p^*(\Z_X)$, which proves the proposition if $\mathcal{X}$ possesses enough stalk functors (SGA4 \Rmnum{4} 4.6). This is the case in particular if $\mathcal{X}$ is the topos of presheaves over a small site $\mathcal{C}$, because for any object $X$ of $\mathcal{C}$, $\Gamma(X,-)$ is a stalk functor. In the general case, $\mathcal{X}$ is equivalent to the topos of sheaves over a small site $\mathcal{C}$ (\cref{topos Giraud characterization}), and the epimorphic family $\mathfrak{U}$ is the image, under the sheafification functor, of an epimorphic family $\mathfrak{U}'=\{U_i'\to X'\}$. Therefore we have $\Z_\bullet(\mathfrak{U})=(\Z_\bullet(\mathfrak{U}'))^\#$, which is a resolusion of $(\Z_{X'})^\#\cong\Z_X$.
\end{proof}

\subsection{\v{C}ech cohomology}
\paragraph{The general notion of cohomology}\label{ringed topos cohomology def paragarph}
Let $(\mathcal{X},A)$ be a ringed topos, $M$, $N$ be two $A$-modules (say left modules). We denote by $\Ext_A^p(\mathcal{X};M,N)$ (or simply $\Ext_A^p(M,N)$ if there is no risk of confusion) the value of the $p$-th right derived functor of the functor $\Hom_A(M,-)$ at $N$. In other words,
\[\Ext_A^p(\mathcal{X};M,N):=R^p\Hom_A(M,-)(N).\]
The functors $\Ext^p_A(\mathcal{X};M,N)$ then form a $\delta$-functor on the variable $N$, and is also a contravariant functor on the variable $M$.\par
Let $X$ be an object of $\mathcal{X}$. If $M=A_X$ is the free $A$-module generated by $X$ (SGA4 \Rmnum{4}, 12), we then write
\[\Ext_A^p(\mathcal{X};A_X,N)=H^p(X,N).\]
Note that in this notation, the ring $A$ no longer appears. This leads to no confusion because we will show that the formation of $H^p(X,-)$ commutes to the restriction of scalars, and the functor $H^p(X,-)$ is the $p$-th right derived functor of the functor $\Hom_A(A_X,-)=\Hom_\mathcal{X}(X,-)$, which is again denoted by $\Gamma(X,-)$. In particular, if $X$ is the final object of $\mathcal{X}$, then $A_X=A$ and we write
\[\Ext_A^p(\mathcal{X};A,N)=H^p(\mathcal{X},N).\]
Let $X$ be an object of $\mathcal{X}$ and $j:\mathcal{X}_{/X}\to\mathcal{X}$ be the localization morphism (SGA4 \Rmnum{4}, 8). The functor $j^*$ is exact on $A$-modules and admits a left adjoint functor $j_!$. Therefore $j^*$ transforms injective modules to injective modules (\cref{abelian cat adjoint functor exact iff injective to injective}), and for any $A$-module $N$ and $A|_X$-module $M$, we have a canonical isomorphism
\begin{align}\label{ringed topos Ext localization to object}
\Ext_{A|_X}^p(\mathcal{X}_{/X};M,j^*(N))\stackrel{\sim}{\to} \Ext_A^p(\mathcal{X};j_!(M),N).
\end{align}
In particular, by setting $M=A$, we obtain canonical isomorphisms
\begin{align}\label{ringed topos cohomology localization to object}
H^p(\mathcal{X}_{/X},j^*(N))\stackrel{\sim}{\to} H^p(X,N).
\end{align}
For any object $X$ of $\mathcal{X}$ and any couple $M$, $N$ of $A$-modules, we put
\begin{align}\label{ringed topos Ext on object def}
\Ext_A^p(X;M,N):=\Ext_A^p(\mathcal{X}_{/X};M|_X,N|_X)
\end{align}
From the above remarks, the functors $\Ext_A^p(X;M,-)$ are the derived functors of the functors $\Hom_{A|_X}(M|_X,(-)|_X)$, and the functors $(M,N)\mapsto\Ext_A^p(X;M,N)$ form an $\delta$-functor with respect to each of the variables.

\paragraph{Cohomology for topos of presheaves}\label{ringed topos presheaf cohomology paragarph}
Let $\mathcal{C}$ be a small category endowed with a presheaf of rings $A$, $\PSh(\mathcal{C})$ be the topos of presheaves over $\mathcal{C}$. We divide the computation of cohomology groups into two cases:
\begin{itemize}
    \item Let $X$ be a representable object of $\PSh(\mathcal{C})$. The functor which associates an $A$-module $M$ with the group $\Gamma(X,M)=M(X)$ is then exact by (SGA4 \Rmnum{1}, 3), so we have $H^p(X,M)=0$ for any $p>0$ and any $A$-module $M$. In particular, since $M(X)=\Hom_A(A_X,M)$, we conclude that $A_X$ is a projective $A$-module.
    \item Let $S$ be a presheaf over $\mathcal{C}$. We have a canonical isomorphism for any $A$-module $M$ (SGA4 \Rmnum{1}, 2):
    \[\Gamma(S,M)=\Hom(S,M)=\llim_{U\in\mathcal{C}_{/S}}M(U).\]
    Moreover, for any injective $A$-module $M$, the $A|_S$-module $M|_S$ is injective (SGA4 2.2). Therefore, the group $H^p(S,M)$ is the value at $M|_S$ of the $p$-th right derived functor of the functor $\llim_{U\in\mathcal{C}_{/S}}\Gamma(U,-)$. Denote by $\llim_{\mathcal{C}_{/S}}^p$ this derived functor, we then have canonical isomorphism
    \begin{align}\label{ringed topos presheaf cohomology is limit-1}
        H^p(S,M)\stackrel{\sim}{\to}\llim\nolimits_{\mathcal{C}_{/S}}^pM.
    \end{align}
    In particular, if $S$ is the final object in $\mathcal{S}$, we then obtain a canonical isomorphism
    \begin{align}\label{ringed topos presheaf cohomology is limit-2}
        H^p(\PSh(\mathcal{C}),M)\cong\llim\nolimits_{\mathcal{C}}^pM.
    \end{align}
\end{itemize}

We now turn to the computation of \v{C}ech cohomologies of $\PSh(\mathcal{C})$. Let $X$ be an object of $\mathcal{C}$ and $\mathfrak{U}=\{U_i\to X\}_{i\in I}$ be a family of squarable morphisms in $\mathcal{C}$. We denote by $A_\bullet$ the simplicial complex (\ref{ringed topos Cech complex for covering-4}):
\[A_\bullet:\begin{tikzcd}[column sep=12mm]
\cdots\ar[r,shift left=8pt,"s_0"description]\ar[r,"s_1"description]\ar[r,shift right=8pt,"s_2"description]
&\bigoplus_{i,j}A_{U_i\times_XU_j}\ar[r,shift left=4pt,"s_0"description]\ar[r,shift right=4pt,"s_1"description]
&\bigoplus_iA_{U_i}\ar[r]
&A_X
\end{tikzcd}\]
For any $A$-module $M$, we denote by $C^\bullet(\mathfrak{U},M)$ the complex obtained by applying the functor $\Hom_A(A_\bullet,M)$:
\[C^\bullet(\mathfrak{U},M):\begin{tikzcd}[column sep=12mm]
\prod_{i}M(U_i)\ar[r,shift left=2pt]\ar[r,shift right=2pt]
&\prod_{i,j}M(U_i\times_XU_j)\ar[r,shift left=4pt]\ar[r]\ar[r,shift right=4pt]
&\cdots
\end{tikzcd}\]
The cohomology of this complex of abelian groups is denoted by $H^p(\mathfrak{U},M)=H^p(C^\bullet(\mathfrak{U},M))$.

\begin{proposition}\label{ringed topos presheaf Cech cohomology char}
With the above notions, let $R\hookrightarrow X$ be the sieve generated by $\mathfrak{U}$. Then we have a canonical isomorphism
\begin{align}\label{ringed topos presheaf Cech cohomology char-1}
H^p(\mathfrak{U},M)\stackrel{\sim}{\to}H^p(R,M)
\end{align}
Moreover, the functors $H^p(\mathfrak{U},-)$ commutes with restrictions of scalars.
\end{proposition}
\begin{proof}
As $R$ is a sub-object of $X$ in $\PSh(\mathcal{C})$, the fiber products $U_{i_1}\times_R\cdots\times_RU_{i_p}$ and $U_{i_1}\times_X\cdots\times_XU_{i_p}$ are canonical isomorphic, so it follows from \cref{ringed topos Cech complex resolution for epimorphic} that the complex $A_\bullet$ is a resolution of $A_R$. Now recall from our previous discussion that the components of $A_\bullet$ are projective $A$-modules, so by definition, the cohomology groups of $C^\bullet(\mathfrak{U},M)$ are then canonically isomorphic to $\Ext_A^p(A_R,M)$, whence the isomorphism (\ref{ringed topos presheaf Cech cohomology char-1}). The second assertion follows immediately from the description of the complex $C^\bullet(\mathfrak{U},M)$.
\end{proof}

\begin{corollary}\label{ringed topos presheaf Cech cohomology refinement prop}
Let $\mathfrak{U}=\{U_i\to X\}$ and $\mathfrak{V}=\{V_i\to X\}$ be two families of morphisms with target $X$ and
\[\phi=(\phi:I\to J,f_i:U_i\to V_{\phi(i)}),\quad \phi=(\psi:I\to J,g_i:U_i\to V_{\phi(i)})\]
be morphisms (lying over $X$) from $\mathfrak{U}$ to $\mathfrak{V}$. Then $\phi$ and $\psi$ induce equal morphisms $H^p(\mathfrak{U},M)\to H^p(\mathfrak{U},M)$. In particular, if the families $\mathfrak{U}$ and $\mathfrak{V}$ are equivalent (i.e. there exists a morphism from $\mathfrak{U}$ to $\mathfrak{V}$ and a morphism from $\mathfrak{V}$ to $\mathfrak{U}$), then the $A$-modules $H^p(\mathfrak{U},M)$ and $H^p(\mathfrak{V},M)$ are canonically isomorphic.
\end{corollary}

\paragraph{Cohomology for small sites}\label{ringed small site cohomology paragraph}
Let $(\mathcal{C},A)$ be a ringed $\mathscr{U}$-site, $\Sh(\mathcal{C})$ be the topos of sheaves over $\mathcal{C}$, and $\eps:\mathcal{C}\to\Sh(\mathcal{C})$ be the canonical functor which associated an object of $\mathcal{C}$ with the associated sheaf. By abusing of languages, for any object $X$ of $\mathcal{C}$ and any sheaf of $A$-modules $M$, we define $H^p(X,M)$ to be the $p$-th derived functor of the functor $\Gamma(X,-)$. Recall that if the topology on $\mathcal{C}$ is subcanonical, then the functor $\eps$ is fully faithful and we can identify $\mathcal{C}$ with a subcategory of $\Sh(\mathcal{C})$.\par
Now consider the inclusion functor $\mathcal{H}^0:\Sh(\mathcal{C}_A)\to\PSh(\mathcal{C}_A)$ from the category of sheaves of $A$-modules to the category of presheaves of $A$-modules. For any sheaf of $A$-modules $M$ and any object $X$ of $\mathcal{C}$, we have by definition
\[\mathcal{H}^0(M)(X)=H^0(X,M)=M(X).\]
Since functor $\mathcal{H}^0$ is obviously left exact, we can define its right derived functors, which are denoted by $\mathcal{H}^p$. Note that as for any object $X$ of $\mathcal{C}$, the functor $\Gamma(X,-)$ is exact on the category of presheaves, we have
\begin{align}\label{ringed small site sheaf cohomology sheaf section char}
\mathcal{H}^p(M)(X)=H^p(X,M)
\end{align}
for any sheaf of $A$-modules $M$, so the presheaf $\mathcal{H}^p(M)$ is defined by $X\mapsto H^p(X,M)$.\par
The definition of the cohomology group $H^p(X,M)$ is simple, but hard to compute. Because of this, it is necessary to introduce another cohomology group, the \v{C}ech cohomology group, which are much easily to handle. Suppose that $(\mathcal{C},A)$ is a small ringed site, so that $\PSh(\mathcal{C})$ is a topos and we can apply the results of \ref{ringed topos presheaf cohomology paragarph}. Let $X$ be an object of $\mathcal{C}$ and $R\hookrightarrow X$ be a covering sieve. For any presheaf of $A$-modules $G$, the groups $H^p(R,G)$ (which are computed in the topos $\PSh(\mathcal{C})$) are then called the \textbf{\v{C}ech cohomology groups of the presheaf $\bm{G}$ relative to the covering sieve $\bm{R}$}. If $R\hookrightarrow X$ is generated by a covering family $\mathfrak{U}=\{U_i\to X\}$, these groups are then computed by the \v{C}ech complex $C^\bullet(\mathfrak{U},G)$, and called the \textbf{\v{C}ech cohomology groups of the presheaf $\bm{G}$ relative to the covering family $\mathfrak{U}$} (denoted by $H^p(\mathfrak{U},G)=H^p(R,G)$). If $M$ is a sheaf of $A$-modules over $\mathcal{C}$, the groups $H^p(\mathfrak{U},\mathcal{H}^0(M))$ are then denoted by $H^p(\mathfrak{U},M)$, and called the \v{C}ech cohomology groups of the sheaf $M$ relative to the covering family $\mathfrak{U}$.\par
The cohomology group $H^p(R,M)$ thus defined is inadequate to reflect the cohomological natures of $M$, and in fact differs from the group $H^p(X,M)$. To fix this, we must apply a limit process as the case of classical \v{C}ech cohomologies. Now let $\check{\mathcal{H}}^0:\PSh(\mathcal{C}_A)\to\PSh(\mathcal{C}_A)$ be the natrual extension of the functor $\mathcal{H}^0$ to the category of presheaves of $A$-modules (composed with the functor $L$). We then have, by (\ref{site small generated sheafification-2}), for any presheaf $G$ and any object $X$ of $\mathcal{C}$:
\begin{align}\label{ringed small site presheaf 0-cohomology sheaf section char}
\check{\mathcal{H}}^0(G)(X)=\rlim_{R\hookrightarrow X}G(R)
\end{align}
where the inductive limit is taken over all covering sieves of $X$. From this, we see that the functor $\check{\mathcal{H}}^0$ is left exact, and we denote its right derived functors by $\check{\mathcal{H}}^p$. As the section functor $\Gamma(X,-)$ and taking filtered limits are both exact, it follows from (\ref{ringed small site presheaf 0-cohomology sheaf section char}) that
\begin{align}\label{ringed small site presheaf p-cohomology sheaf section char}
\check{\mathcal{H}}^p(G)(X)=\rlim_{R\hookrightarrow X}H^p(R,G),
\end{align}
The presheaves $\check{\mathcal{H}}^p(G)$ are then called the \textbf{presheaves of \v{C}ech chomologies of $\bm{G}$}. For any object $X$ of $\mathcal{C}$, the \textbf{\v{C}ech cohomology groups of $G$} are defined to be
\begin{align}\label{ringed small site presheaf Cech cohomology group def}
\check{H}^p(X,G):=\check{\mathcal{H}}^p(G)(X).
\end{align}
If the topology of $\mathcal{C}$ is defined by a basis, which is most of the case in practice, we then have, in view of \cref{ringed topos presheaf Cech cohomology char},
\begin{align}\label{ringed small site presheaf Cech p-cohomology sheaf section char}
\check{H}^p(X,G)=\rlim_\mathfrak{U}H^p(\mathfrak{U},G)
\end{align}
where the inductive limit is taken over all covering families $\mathfrak{U}$ of $X$, ordered by refinements. If $M$ is a sheaf of $A$-modules, then by abusing of languages, we write
\begin{align}\label{ringed small site sheaf Cech cohomology group def}
\check{H}^p(X,M)=\check{H}^p(X,\mathcal{H}^0(M))=\check{\mathcal{H}}^p(\mathcal{H}^0(M))(X).
\end{align}
The groups $\check{H}^p(X,M)$ are called the \textbf{\v{C}ech cohomology groups of the sheaf $\bm{M}$}. Note that although the functors $\check{H}^p$ are derived functors on the category of presheaves, they do not, in general, form a $\delta$-functor on the category of sheaves.
\paragraph{\v{C}ech cohomology for \texorpdfstring{$\mathscr{U}$}{U}-sites}\label{site cohomology change universe paragraph}
Let $(\mathcal{C},A)$ be a ringed $\mathscr{U}$-site and $\mathscr{V}$ be a universe containing $\mathscr{U}$. Then the site $(\mathcal{C},A)$ is also a $\mathscr{V}$-site, and we have a $\mathscr{U}$-topos $\Sh(\mathcal{C})_\mathscr{U}$, a $\mathscr{V}$-topos $\Sh(\mathcal{C})_\mathscr{V}$, and a canonical inclusion functor $\eps:\Sh(\mathcal{C})_\mathscr{U}\to\Sh(\mathcal{C})_\mathscr{V}$. The functor $\eps$ is exact and fully faithful over the category of modules and transforms injective modules to injective modules (\cref{abelian category generaring family in universe prop}).

\subsection{The Cartan-Laray spectral sequence}
The classical Leray spectral sequence for a covering $\mathfrak{U}$ (also called the \v{C}ech-to-derive spectral sequence) of a topological space $X$ relates the cohomology sheaf and \v{C}ech cohomology into a spectral sequence of the form
\begin{equation*}
E_2^{p,q}=\check{H}^p(\mathfrak{U},\mathcal{H}^q(X,\mathscr{F}))\Rightarrow H^{p+q}(X,\mathscr{F})
\end{equation*}
where $\mathscr{F}$ is a sheaf on $X$. This spectral sequence has many useful concequences. For example, if the cohomology vanishes for any finite intersections of the covering $\mathfrak{U}$, then the $E_2$-term degenerates and the edge morphisms yield an isomorphism of \v{C}ech cohomology for this covering to sheaf cohomology. This provides a method of computing sheaf cohomology using \v{C}ech cohomology: for instance, this happens if $\mathscr{F}$ is a quasi-coherent sheaf on a scheme and each element of $\mathfrak{U}$ is an open affine subscheme such that all finite intersections are again affine (e.g. if the scheme is separated).\par
In this paragraph we provide a direct genralization of the Leray spectral sequence for cohomology of topos. As we shall see, the language of derived functors and the Grothendieck spectral sequence can be used to give an easy proof of such generalizations. A relative version of this, which relates the sheaf cohomology with higher direct images of a morphism, will also be given after we introduce the notion of flasque sheaves.

\begin{proposition}\label{ringed site change universe cohomology prop}
Let $(\mathcal{C},A)$ be a ringed $\mathscr{U}$-site and $\mathscr{V}$ be a universe containing $\mathscr{U}$. Then the functor $\mathcal{H}^0:\Sh(\mathcal{C}_A)\to\PSh(\mathcal{C}_A)_\mathscr{V}$ transforms injective $A$-modules to injective presheaves. For any integer $p>0$ and any $A$-module $M$, the sheaf associated with the presheaf $\mathcal{H}^p(M)$ is zero.
\end{proposition}
\begin{proof}
We denote by $(-)^\#_\mathscr{V}$ the sheafification functor on $\mathscr{V}$-presheaves, and $\eps:\Sh(\mathcal{C}_A)\to\Sh(\mathcal{C}_A)_\mathscr{V}$ the inclusion functor. Since we have $(\mathcal{H}^0)^\#_\mathscr{V}=\eps$ by \cref{site small generated sheafification functor prop} and the functors $(-)^\#_\mathscr{V}$ and $\eps$ are exact, we conclude that $(\mathcal{H}^p)^\#=0$ for any integer $p>0$. Now for any $\mathscr{U}$-sheaf $M$ and any $\mathscr{V}$-presheaf $N$, we have a functorial isomorphism
\[\Hom_{\PSh(\mathcal{C}_A)}(N,\mathcal{H}^0(M))\stackrel{\sim}{\to}\Hom_{\Sh(\mathcal{A})_\mathscr{V}}(N^\#_\mathscr{V},\eps(M)).\]
If $M$ is injective, then $\eps(M)$ is injective (\cref{abelian cat adjoint functor exact iff injective to injective}) and as the functor $(-)^\#_\mathscr{V}$ is exact, the functor $\Hom_{\PSh(\mathcal{C}_A)}(-,\mathcal{H}^0(M))$ is exact, therefore $\mathcal{H}^0(M)$ is injective.
\end{proof}

\begin{theorem}\label{ringed site Leray spectral sequence for presheaf functor}
Let $(\mathcal{C},A)$ be a ringed $\mathscr{U}$-site, $R$ be a $\mathscr{U}$-presheaf of sets over $\mathcal{C}$, $M$ be a sheaf of $A$-modules. Then there exists a canonical spectral sequence
\begin{equation}\label{ringed site Leray spectral sequence for presheaf functor-1}
E_2^{p,q}=H^p(R,\mathcal{H}^q(M))\Rightarrow H^{p+q}(R^\#,M).
\end{equation}
(If $\mathcal{C}$ is not $\mathscr{U}$-small, the term $H^p(R,\mathcal{H}^q(M))$ should be considered as the cohomology of the presheaf $\mathcal{H}^p(M)$ in the topos $\PSh(\mathcal{C}_A)_\mathscr{V}$, where $\mathscr{V}$ is a universe containing $\mathscr{U}$ such that $\mathcal{C}$ is $\mathscr{V}$-small.)
\end{theorem}
\begin{proof}
By the definition of the functor $\#$, we have an isomorphism of functors
\[H^0(R^\#,M)\stackrel{\sim}{\to}H^0(R,\mathcal{H}^0(M)).\]
The functor $\mathcal{H}^0$ transforms injective objects to injective objects, so we conclude the spectral sequence from \cref{abelian category G spectral sequence iff injective to acyclic}.
\end{proof}

\begin{corollary}\label{ringed site Leray spectral sequence for covering sieve}
Let $X$ be an object of $\mathcal{C}$ and $\mathfrak{U}=\{U_i\to X\}$ be a covering family of $X$. Then we have the following Cartan-Leray spectral sequence
\begin{align}\label{ringed site Leray spectral sequence for covering sieve-1}
E_2^{p,q}=H^p(\mathfrak{U},\mathcal{H}^p(M))\Rightarrow H^{p+q}(X,M).
\end{align}
\end{corollary}
\begin{proof}
Let $R\hookrightarrow X$ be the sieve generated by $\mathfrak{U}$. As the sieve is covering, the sheaf associated with $R$ is the sheaf associated with $X$ (\cref{site morphism of presheaf bicovering iff}), so we have $H^{p+q}(R^\#,M)=H^{p+q}(X,M)$ in view of the definition of $H^p$. The corollary then follows from \cref{ringed topos Cech complex resolution for epimorphic}. 
\end{proof}

\begin{corollary}\label{ringed site Cech to derive spectral sequence}
There exists a canonical spectral sequence on sheaves $M$ and the objects $X$ of $\mathcal{C}$:
\begin{align}\label{ringed site Cech to derive spectral sequence-1}
E_2^{p,q}=\check{H}^p(X,\mathcal{H}^p(M))\Rightarrow H^{p+q}(X,M).
\end{align}
As $X$ varies in $\mathcal{C}$, this spectral sequence gives a spectral sequence of presheaves
\begin{align}\label{ringed site Cech to derive spectral sequence-2}
E_2^{p,q}=\check{\mathcal{H}}^p(\mathcal{H}^p(M))\Rightarrow \mathcal{H}^{p+q}(M),
\end{align}
which gives canonical edge morphisms
\begin{align}
\phi^p(M)&:\check{\mathcal{H}}^p(M)\to\mathcal{H}^p(M),\label{ringed site Cech to derive spectral sequence-3}\\
\phi^p_X(M)&:\check{H}^p(X,M)\to H^p(X,M).\label{ringed site Cech to derive spectral sequence-4}
\end{align}
The morphisms $\phi^p(M)$ and $\phi_X^p(M)$ are isomorphisms for $p=0,1$, and are monomorphisms for $p=2$. In general, if the presheaf $\mathcal{H}^i(M)$ is zero for $0<i<n$, then the morphisms $\phi^p(M)$ and $\phi^p_X(M)$ are isomorphic for $0\leq p\leq n$ and monomorphic for $p=n+1$.
\end{corollary}
\begin{proof}
The first spectral sequence are obtained by passing to inductive limits in the spectral sequence (\ref{ringed site Leray spectral sequence for covering sieve-1}) over covering sieves $R\hookrightarrow X$, and the second one is induced in view of (\ref{ringed small site presheaf Cech cohomology group def}). By \cref{ringed site Leray spectral sequence for presheaf functor}, the sheaf associated with the presheaf $\mathcal{H}^p(M)$ is zero if $p>0$, which implies $\check{\mathcal{H}}^0\mathcal{H}^q(M)=0$ for $p>0$ (\cref{site small generated sheafification functor prop}). The assertions on the induced morphisms $\phi^p$ and $\phi^p_X$ therefore follows.
\end{proof}

\begin{corollary}\label{ringed site sheaf cohomology restriction of scalar}
Let $(\mathcal{X},A)$ be a ringed topos and $M$ be a (left) $A$-module. Denote by $\mathscr{M}$ the underlying abelian group of $M$. Then the functor $M\mapsto\mathscr{M}$ is exact and for any object $X$ of $\mathcal{X}$, we have a canonical isomorphism
\[H^0(X,M)\stackrel{\sim}{\to}H^0(X,\mathscr{M})\]
which extendes to isomorphisms
\begin{align}\label{ringed site sheaf to underlying group exact-1}
H^p(X,M)\stackrel{\sim}{\to}H^p(X,\mathscr{M})\for p\geq 0.
\end{align}
\end{corollary}
\begin{proof}
For any object $Y$ of $\mathcal{X}$, we have
\[\check{H}^p(Y,M)=\rlim_\mathfrak{U}H^p(\mathfrak{U},M),\quad \check{H}^p(Y,\mathscr{M})=\rlim_\mathfrak{U}H^p(\mathfrak{U},\mathscr{M})\]
where the limit is taken over covering families $\mathfrak{U}$. Since the cohomology $H^p(\mathfrak{U},-)$ commutes with restriction of scalars (\cref{ringed topos Cech complex resolution for epimorphic}(b)), we conclude that the canonical homomorphism $\check{H}^p(Y,M)\to\check{H}^p(Y,\mathscr{M})$ is an isomorphism. Now if $M$ is an injective $A$-module, we have have $\check{\mathcal{H}}^p(\mathscr{M})=0$ for $p>0$, whence $\mathcal{H}^p(\mathscr{M})=0$ by induction on $p$ and use \cref{ringed site Cech to derive spectral sequence}. It then follows that $H^p(X,\mathscr{M})=0$ for $p>0$, so the functor $M\mapsto\mathscr{M}$ transforms injective objects to acyclic objects for the functor $H^0(X,-)$, and we can apply \cref{abelian category G spectral sequence iff injective to acyclic} to get the isomorphisms (\ref{ringed site sheaf to underlying group exact-1}).
\end{proof}

\begin{example}
Let $G$ be a topological group and $\mathcal{B}G$ be the classifying topos. Let $E_G$ be the left regular representation of $G$ (given by left translations of $G$), which is an object of $\mathcal{B}G$. The canonical morphism from $E_G$ to the final object $e_G$ of $B_G$ is easily seen to be an epimorphism, so it gives a covering $\mathfrak{U}=\{E_G\to e_G\}$ and, for any abelian sheaf $F$ on $\mathcal{B}G$, a spectral sequence
\begin{equation}\label{topos BG spectral sequence of E_G to e_G}
E_2^{p,q}=H^p(\mathfrak{U},\mathcal{H}^p(F))\Rightarrow H^{p+q}(\mathcal{B}G,F).
\end{equation}
By definition, the $E_2$ terms of this spectral sequence is computed as the cohomology of the following sequence
\[\begin{tikzcd}
H^p(E_G,F)\ar[r]&H^p(E_G\times E_G,F)\ar[r]&H^p(E_G\times E_G\times E_G,F)\ar[r]&\cdots
\end{tikzcd}\]
Let $\Topos(G)$ denote the topos of sheaves over the big site $\mathbf{Top}_{/G}$ associated with $G$. We claim that for each integer $n>0$, there is a canonical equivalence
\[\mathcal{B}G_{/(E_G)^{\times n}}\stackrel{\sim}{\to}\Topos(G^{\times(n-1)})\]
which associates each $G$-set $X$ over $(E_G)^{\times n}$ with its orbit set $X/G$. To see this is an equivalence, it suffices to note that any $G$-set $X$ over $(E_G)^{\times n}$ has a faithful action by $G$, hence isomorphic to a product of coplies of $G$. By passing to quotient we then obtain a morphism.
\end{example}

\subsection{Acyclic sheaves}
Let $(\mathcal{X},A)$ be a ringed topos, $F$ be an $A$-module, $S$ be a topological generating family of $\mathcal{X}$. The sheaf $F$ is said to be \textbf{$\bm{S}$-acyclic} if for any object $X$ of $S$ and any integer $p>0$, we have $H^p(X,F)=0$, and \textbf{$\mathcal{C}$-acyclic} if $S$ is the family of sheaves assoicated with the objects of $\mathcal{C}$. If $S$ is equal to $\Ob(\mathcal{X})$, the $S$-acyclic sheaves are then called \textbf{flasque sheaves}.
\begin{proposition}\label{ringed site C-acyclic sheaf iff}
Let $(\mathcal{C},A)$ be a ringed $\mathscr{U}$-site, $F$ be a sheaf of $A$-modules. Denote by $\mathcal{H}^0:\Sh(\mathcal{C}_A)\to\PSh(\mathcal{C}_A)$ the canonical inclusion functor. The following conditions are equivalent:
\begin{enumerate}
    \item[(\rmnum{1})] $F$ is $\mathcal{C}$-acyclic.
    \item[(\rmnum{2})] For any object $X$ of $\mathcal{C}$ and any covering sieve $R\hookrightarrow X$, we have $H^p(R,\mathcal{H}^0(F))=0$ for $p>0$.
    \item[(\rmnum{3})] For any object $X$ of $\mathcal{C}$, we have $\check{H}^p(X,F)=0$ for $p>0$.
\end{enumerate}
\end{proposition}
\begin{proof}
If $F$ is $\mathcal{C}$-acyclic, the presheaf $\mathcal{H}^p(F)$ is zero for $p>0$, so the spectral sequence (\ref{ringed site Leray spectral sequence for presheaf functor-1}) gives an isomorphism $H^p(R,\mathcal{H}^0(F))\cong H^p(X,F)$ for $p>0$, which implies (\rmnum{2}). By passing to inductive limit on $R$, it is immediate that (\rmnum{2})$\Rightarrow$(\rmnum{3}). Conversely, if $\check{H}^p(X,F)=0$ for $p>0$, then by induction and \cref{ringed site Cech to derive spectral sequence} we conclude that $\mathcal{H}^p(F)=0$ for $p>0$, whence (\rmnum{1}).
\end{proof}

It follows from \cref{ringed site C-acyclic sheaf iff}(\rmnum{2}) and \cref{ringed site sheaf cohomology restriction of scalar} that the property of $S$-acyclicity only depends on the underlying abelian sheaf. In particular, a sheaf of $A$-modules is flasque if and only if the underlying abelian sheaf is flasque.

\begin{corollary}\label{ringed topos flasque sheaf iff Cech for epimorphic}
Let $(\mathcal{X},A)$ be a ringed topos and $F$ be an $A$-module. The following properties are equivalent:
\begin{enumerate}
\item[(\rmnum{1})] $F$ is flasque;
\item[(\rmnum{2})] for any epimorphic family $\mathfrak{U}=\{X_i\to X\}$, $H^p(\mathfrak{U},F)=0$ for $p>0$.
\end{enumerate}
\end{corollary}
\begin{proof}
In the definition, we can take $\mathcal{C}$ to be the topos $\mathcal{X}$, so the corollary follows from the equivalence (\rmnum{1})$\Leftrightarrow$(\rmnum{2}) of \cref{ringed site C-acyclic sheaf iff}.
\end{proof}

Any injective sheaf is by definition flasque, and flasque sheaves are $S$-acyclic for any topological generating family $S$. Note that a flasque sheaf is not necessarily injective (for example consider the topos of sets). An $S$-acyclic sheaf is also not necessarily flasque.

\begin{proposition}\label{ringed topos flasque sheaf restriction surjective}
Let $(\mathcal{X},A)$ be a topos, $F$ be a flasque $A$-module, $X$ be an object of $\mathcal{X}$. Then for any sub-object $Y$ of $X$, the canonical homomorphism $H^0(X,F)\to H^0(Y,F)$ is surjective.
\end{proposition}
\begin{proof}
Let $Y$ be a sub-object of $X$ such that the morphism $H^0(X,F)\to H^0(Y,F)$ is not surjective, and $Z$ be the object obtained by glueing two copies of $X$ along $Y$. The object $Z$ is then covered by two sub-objects $X_1$ and $X_2$ isomorphic to $X$, and we have $X_1\times_ZX_2=Y$. If $\mathfrak{U}=\{X_1,X_2\}$ is the corresponding covering of $Z$, we then have $H^1(\mathfrak{U},F)\neq 0$, which is a contradiction.
\end{proof}

The criterion of \cref{ringed topos flasque sheaf restriction surjective} is not sufficient to characterize, in the case of general topos, flasque sheaves. However, it characterizes them in the case of topos generated by their open sets and in particular in the case of topos associated with topological spaces. Therefore, our terminology adopted here coincides with the terminology for classical flasque sheaves over topological spaces.

\begin{proposition}\label{ringed topos morphism on acyclic sheaf}
Let $f:(\mathcal{X},A)\to(\mathcal{Y},B)$ be a morphism of ringed topoi.
\begin{enumerate}
    \item[(a)] The functor $f_*$ transforms flasque $A$-modules into flasque $B$-modules.
    \item[(b)] Let $S$ (resp. $T$) be a topological generating family of $\mathcal{X}$ (resp. $\mathcal{Y}$) such that $f^*(T)\sub S$. Then the functor $f_*$ transforms $S$-acyclic $A$-modules to $T$-acyclic $T$-modules.
    \item[(c)] If $f$ is a flat morphism, the functor $f_*$ transforms injective $A$-modules to injective $B$-modules.
\end{enumerate}
\end{proposition}
\begin{proof}
Let $Y$ be an object of $\mathcal{Y}$, $\mathfrak{V}=\{Y_i\to Y\}$ be an epimorphic family, $F$ be a flasque $A$-module, $C^\bullet(\mathfrak{U},f_*(F))$ be the \v{C}ech complex of covering $\mathfrak{V}$. By using the adjunction of $f_*$ and $f^*$ and the fact that $f^*$ commutes with fiber products, we then obtain a canonical isomorphism
\[C^\bullet(\mathfrak{V},f_*(F))\cong C^\bullet(f^*(\mathfrak{V}),F).\]
Since $f^*$ also commutes with inductive limits, $f^*(\mathfrak{V})$ is an epimorphic family, and we then conclude that $H^p(f^*(\mathfrak{V}),F)=0$ for $p>0$ since $F$ is flasque, whence
\[H^p(\mathfrak{V},f_*(F))=H^p(C^\bullet(\mathfrak{V},f_*(F)))=H^p(C^\bullet(f^*(\mathfrak{V}),F))=H^p(f^*(\mathfrak{V}),F)=0\for p>0.\]
We then conclude that $f_*(F)$ is flasque (\cref{ringed topos flasque sheaf iff Cech for epimorphic}), whence the first assertion. The second assertion can be done similarly if the family $T$ is stable under fiber products. In the general case, we can use the spectral sequence (\ref{ringed topos Leray spectral sequence for morphism-2}) of the morphism $f$, since its proof only depends on assertion (a). Let $F$ be $S$-acyclic sheaf. Then $R^pf_*(F)$ are the sheaves associated with the presheaves $Y\mapsto H^p(f^*(Y),F)$. As $T$ is a topologically generating family and $F$ is $S$-acyclic, we have $R^pf_*(F)=0$ for $p>0$. The spectral sequence (\ref{ringed topos Leray spectral sequence for morphism-2}) then provides a cannical isomorphism $H^p(Y,f_*(F))\cong H^p(f^*(Y),F)$ for $Y\in\Ob(\mathcal{Y})$, so we conclude that $H^p(Y,f_*(F))=0$ for $p>0$ and $Y\in T$, which means $F$ is $T$-acyclic. Finally, if $f$ is flat, then the functor $f^*$ is exact on modules, so $f_*$ transforms injective objects to injective objects (\cref{abelian category G spectral sequence iff injective to acyclic}).
\end{proof}

\begin{proposition}\label{ringed topos injective sheaf sHom is exact}
Let $F$ be an $A$-module over a ringed topos $(\mathcal{X},A)$ and $I$ be an injective $A$-module.
\begin{enumerate}
\item[(a)] The functor $M\mapsto\sHom_A(M,G)$ is exact.
\item[(b)] The abelian sheaf $\sHom_A(F,G)$ is flasque.
\end{enumerate}
\end{proposition}
\begin{proof}
To prove the first assertion, we consider an exact sequence
\[\begin{tikzcd}
0\ar[r]&F'\ar[r]&F\ar[r]&F''\ar[r]&0
\end{tikzcd}\]
which induces a sequence
\[\begin{tikzcd}
0\ar[r]&\sHom_A(F'',G)\ar[r]&\sHom_A(F,G)\ar[r]&\sHom_A(F',G)\ar[r]&0
\end{tikzcd}\]
To prove that this sequence is exact, it suffices to consider, for any object $H$ of $\mathcal{X}$, the following sequence
\[\begin{tikzcd}[column sep=3mm]
0\ar[r]&\Hom_\mathcal{X}(H,\sHom_A(F'',G))\ar[r]&\Hom_\mathcal{X}(H,\sHom_A(F,G))\ar[r]&\Hom_\mathcal{X}(H,\sHom_A(F',G))\ar[r]&0
\end{tikzcd}\]
which is isomorphic to the sequence
\[\begin{tikzcd}[column sep=5mm]
0\ar[r]&\Hom_A(A_H\otimes_AF'',G)\ar[r]&\Hom_A(A_H\otimes_AF,G)\ar[r]&\Hom_A(A_H\otimes_AF',G)\ar[r]&0
\end{tikzcd}\]
Since the $A$-module $A_H$ is flat (\cref{ringed topos flat module shrink is flat}), this sequence is exact, whence the desired assertion. Now to prove that $\sHom_A(F,G)$ is flasque, we consider an epimorphic family $\mathfrak{U}=\{X_i\to X\}$ and the complex $\Z_\bullet(\mathfrak{U})$ defined in (\ref{ringed topos Cech complex for covering-5}). This complex is a flat resolution of the object $\Z_X$, which is also flat by (\cref{ringed topos flat module shrink is flat}). From the definition of \v{C}ech cohomology, we then have
\[H^p(\mathfrak{U},\sHom_A(F,G))\stackrel{\sim}{\to}H^p(\Hom_\Z(\Z_\bullet(\mathfrak{U}),\Hom_A(F,G)))\stackrel{\sim}{\to}H^p(\Hom_A(\Z_\bullet(\mathfrak{U})\otimes_\Z F,G)).\]
The complex $Z_\bullet(\mathfrak{U})\otimes_\Z F$ is exact at degree $\neq 0$ because $\Z_\bullet(\mathfrak{U})$ is a flat resolution of a flat module, so $\Hom(\Z_\bullet(\mathfrak{U})\otimes_\Z F,G)$ is acyclic at degree $\neq 0$, which proves our assertion.
\end{proof}

\begin{proposition}\label{ringed topos flasque injective sheaf restriction prop}
Let $(\mathcal{X},A)$ be a ringed topos, $F$ be a flasque (resp. injective) sheaf of $A$-modules.
\begin{enumerate}
\item[(a)] For any object $X$ of $\mathcal{X}$, the $A|_X$-module $j_X^*(F)$ is flasque (resp. injective).
\item[(b)] For any closed $\mathcal{Z}$ of $\mathcal{X}$, the sheaf of sections of $F$ supported in $\mathcal{Z}$ is flasque (resp. injective).
\end{enumerate}
\end{proposition}
\begin{proof}
The first assertion follows from \cref{ringed topos morphism on acyclic sheaf} if $F$ is flasque. The functor $j_X^*$ admits a left adjoint $(j_X)_!$, so if $F$ is injective, $j_X^*(F)$ is also injective (\cref{abelian cat adjoint functor exact iff injective to injective}). As for (b), let $i:\mathcal{Z}\to \mathcal{X}$ be the inclusion morphism. The sheaf of sections of $F$ with support in $\mathcal{Z}$ is then given by $i^*i^!(F)$ (SGA4 \Rmnum{4}, 14). Since $i_*i^!$ is right adjoint to $i_*i^*$, which is exact by (SGA4 \Rmnum{4}, 14), we conclude that it transforms injective sheaves to injective sheaves. Now let $\mathcal{U}$ be the open complement of $\mathcal{Z}$, $j:\mathscr{U}\to\mathcal{X}$ be the inclusion morphism, and $F$ be a flasque sheaf. We then have an exact sequence (SGA4 \Rmnum{4}, 14)
\begin{equation}\label{ringed topos flasque injective sheaf restriction prop-1}
\begin{tikzcd}
0\ar[r]&i_*i^!(F)\ar[r]&F\ar[r]&j_*j^*(F)
\end{tikzcd}
\end{equation}
For any object $X$ of $\mathcal{X}$, we have $j_*j^*(F)(X)=F(X\times\mathcal{U})$ and the induced morphism $F(X)\to j_*j^*(F)(X)$ is given by the canonical inclusion $X\times\mathcal{U}\hookrightarrow X$. As $F$ is flasque, this morphism is surjective by \cref{ringed topos flasque sheaf restriction surjective}, so the last arrow of (\ref{ringed topos flasque injective sheaf restriction prop-1}) is an epimorphism of presheaves. For any object $X$ of $\mathcal{X}$, the long exact sequence induced by \cref{ringed topos flasque sheaf restriction surjective} shows that $H^p(X,i_*i^!(F))=0$ for $p>0$, so $i_*i^!(F)$ is flasque.
\end{proof}

\begin{example}
Let $X$ be a locally compact space and $F$ be a $c$-soft sheaf on $X$. 
\end{example}

\begin{example}
Let $G$ be a discrete group and $\mathcal{B}G$ be the classifying topos. Prove that for any abelian sheaf $F$ and any monomorphism $X\hookrightarrow Y$, the homomorphism $F(Y)\to F(X)$ is surjective. Let $E_G$ be the group $G$ considered as a $G$-set with left translations of $G$. Then the topos $\mathcal{B}G_{/E_G}$ can be identified with the pointed topos. The morphism $E_G\to e_G$ ($e_G$ being the final object of $\mathcal{B}G$) is an epimorphism. For any abelian sheaf $F$ of $\mathcal{B}G$, the sheaf $F|_{E_G}$ is flasque. Deduce that the property of being flasque (or injective) is not a local property.
\end{example}

\begin{example}
We say a topos $\mathcal{X}$ is \textbf{generated by its opens} if the opens of $\mathcal{X}$ (i.e. the sub-objects of the final object $e$ of $E$) form a generating family. Such a topos has the following property:
\begin{enumerate}
\item[(P)] Any epimorphic family $\{X_i\to X\}$ is dominated by an epimorphic family $\{U_j\to X\}$, where the $U_j\to X$ are monomorphisms.
\end{enumerate}
\end{example}

As an application of flasque sheaves, we now consider a morphism $f:(\mathcal{X},A)\to(\mathcal{Y},B)$ of ringed topoi and the direct image functor $f_*:\mathcal{X}\to\mathcal{Y}$ induced by $f$. The functor $f_*$ is left exact on the category of (left) modules, let $R^pf_*$ denote its right derived functors. The higher direct images $R^pf_*$ behaves much like the classical case, and we also have a Leray spectral sequence relating $R^pf_*$ and sheaf cohomology $H^p$.

\begin{proposition}\label{ringed topos morphism higher direct image prop}
Let $f:(\mathcal{X},A)\to(\mathcal{Y},B)$ be a morphism of ringed topoi.
\begin{enumerate}
\item[(a)] For any $A$-module $M$, $R^pf_*(M)$ is the sheaf associated with the presheaf $Y\mapsto H^p(f^*(Y),M)$ for $Y\in\Ob(\mathcal{Y})$.
\item[(b)] The formation of $R^pf_*$ commutes with restriction of scalars.
\item[(c)] The formation $R^pf_*$ commutes with localization. More precisely, for any object $Y$ of $\mathcal{X}$, if we denote by $f_{/X}:\mathcal{X}_{/X}\to\mathcal{Y}_{/Y}$ the induced morphism under localization, where $X=f^*(Y)$, we have, for any $A$-module $M$, a canonical isomorphism
\begin{equation}\label{ringed topos morphism higher direct image prop-1}
R^p(f_{/X})_*(M|_X)\stackrel{\sim}{\to} R^pf_*(M)|_Y\for p\geq 0.
\end{equation}
\end{enumerate}
\end{proposition}
\begin{proof}
We denote by $\hat{f}_*:\PSh(\mathcal{X})\to\PSh(\mathcal{Y})$ the direct image functor for $\mathscr{U}$-presheaves (that is, $\hat{f}_*(M)=M\circ f^*$). As $f^*$ and $f_*$ are adjoints, we have an isomorphism $f_*=(\hat{f}_*)^\#$. But the functors $\#$ and $\hat{f}_*$ are exact, so this implies
\[R^pf_*\cong(\hat{f}_*)^\#\mathcal{H}^p,\]
which is equivalent to assertion (a). Assertion (b) then follows from (a) and \cref{ringed site sheaf cohomology restriction of scalar}. To prove (c), let $Y$ be an object of $\mathcal{Y}$ and consider the induced morphism $f_{/X}:\mathcal{X}_{/X}\to \mathcal{Y}_{/Y}$, where $X=f^*(Y)$. By the definition of the morphism $f_{/X}$, we have the canonical isomorphism
\[(f_{/X})_*(M|_{X})\stackrel{\sim}{\to}f_*(M)|_Y.\]
The case general for higher direct images is then deduced by noting that the localization functors are exact and transform injective objects into injective objects (\cref{abelian cat adjoint functor exact iff injective to injective}).
\end{proof}

\begin{proposition}\label{ringed topos morphism acyclic direct image prop}
Let $f:\mathcal{X}\to\mathcal{Y}$ be a morphism of topoi and $T$ be a topological generating family of $\mathcal{Y}$. Then the sheaves $M$ acyclic for the functors $H^0(f^*(Y),-)$, $Y\in T$ are acyclic for the functor $f_*$. In particular, the flasque sheaves are acyclic for $f_*$.
\end{proposition}
\begin{proof}
This follows from the assertion of \cref{ringed topos morphism higher direct image prop}(a).
\end{proof}

\begin{proposition}\label{ringed topos Leray spectral sequence for morphism}
Let $f:\mathcal{X}\to\mathcal{Y}$ be a morphism of topoi and $M$ be an abelian sheaf on $\mathcal{X}$. Then we have a spectral sequence
\begin{equation}\label{ringed topos Leray spectral sequence for morphism-1}
E_2^{p,q}=H^p(\mathcal{Y},R^pf_*(M))\Rightarrow H^{p+q}(\mathcal{X},M).
\end{equation}
More generally, for any object $Y$ of $\mathcal{Y}$, we have a spectral sequence
\begin{equation}\label{ringed topos Leray spectral sequence for morphism-2}
E_2^{p,q}=H^p(Y,R^pf_*(M))\Rightarrow H^{p+q}(f^*(Y),M).
\end{equation}
\end{proposition}
\begin{proof}
By the definitions of the direct image and inverse image functors, we have a canonical isomorphism $H^0(Y,f_*(M))\stackrel{\sim}{\to}H^0(f^*(Y),M)$. The functor $f_*$ transforms injective objects to flasque objects (\cref{ringed topos morphism on acyclic sheaf}), so the spectral sequence follows from \cref{abelian category G spectral sequence iff injective to acyclic}.
\end{proof}

\begin{remark}[\textbf{Flasque sheaves and changing the universe}]
Let $\mathcal{C}$ be a $\mathscr{U}$-site (for example a $\mathscr{U}$-topos) and $\mathscr{V}$ be a universe containing $\mathscr{U}$. Let $\eps:\Sh(\mathcal{C})_\mathscr{U}\to\Sh(\mathcal{C})_\mathscr{V}$ be the canonical injection functor. Let $F$ be an abelian $\mathscr{U}$-sheaf that is flasque over $\mathcal{C}$. Then for any object $X$ of $\Sh(\mathcal{C})_\mathscr{U}$, we have $H^p(\eps(X),\eps(F))=0$ for $p>0$. Since any object $Y$ in $\Sh(\mathcal{C})_\mathscr{V}$ admits an epimorphic family $\{\eps(X_i)\to Y\}$ where $\eps(X_i)\to Y$ are monomorphisms, we then conclude that
\[H^p(Y,\eps(F))=\llim\nolimits_{\eps(Y)\hookrightarrow Y}^pF(X).\]
By reducing to the topos $\Sh(\Ouv(\mathcal{C}_{/Y}))_\mathscr{V}$ and use the fact that in this topos, a sheaf is flasque if it is locally flasque, we conclude that $\eps(F)$ is flasque.
\end{remark}

\subsection{Local \texorpdfstring{$\Ext$}{Ext} and cohomology with closed support}

Let $(\mathcal{X},A)$ be a ringed topos and $M$ be a (left) $A$-module. The functor $N\mapsto\Hom_A(M,N)$ from the category of left $A$-modules to the category of abelian groups is left exact, and its derived functors are then denoted by $\sExt_A^p(M,N)$. In particular, we have
\[\sExt_A^0(M,N)=\sHom_A(M,N).\]
By definition, for any object $X$ of $\mathcal{X}$, we have the following canonical isomorphisms
\[H^0(X,\sExt_A^0(M,N))=\Ext_A^0(X;M,N)=\Hom_{A|_X}(M|_X,N|_X).\]

\begin{proposition}\label{ringed topos sExt porp}
Let $(\mathcal{X},A)$ be a ringed topos and $M,N$ be left $A$-modules.
\begin{enumerate}
    \item[(a)] The formation of $\sExt_A^p$ commutes with localizations. More precisely, for any object $X$ of $\mathcal{X}$, we have a functorial isomorphism
    \begin{align}\label{ringed topos sExt porp-1}
        \sExt_A^p(M,N)|_X\stackrel{\sim}{\to}\sExt_{A|_X}^p(M|_X,N|_X).
    \end{align}
    \item[(b)] The sheaf $\sExt_A^p(M,N)$ is the sheaf associated with the presheaf $X\mapsto\Ext_A^p(X;M,N)$.
    \item[(c)] There is a spectral sequence
    \begin{align}\label{ringed topos sExt porp-2}
        E_2^{p,q}=H^p(\mathcal{X},\sExt_A^p(F,G))\Rightarrow \Ext_A^p(\mathcal{X};M,N).
    \end{align}
    More generally, for any object $X$ of $\mathcal{X}$, we have a functorial spectral sequence
    \begin{align}\label{ringed topos sExt porp-3}
        E_2^{p,q}=H^p(X,\sExt_A^p(F,G))\Rightarrow \Ext_A^p(X;M,N).
    \end{align}
\end{enumerate}
\end{proposition}
\begin{proof}
By the definition of $\sHom$, we have the isomorphism (\ref{ringed topos sExt porp-1}) for $p=0$, so the general case follows from the fact that the localization functor is exact and transforms injective modules to injective modules (\ref{abelian cat adjoint functor exact iff injective to injective}). The functor $N\mapsto\sHom_A(M,N)=\sExt_A^0(M,N)$ transforms injective modules to flasque sheaves (hence flasque for $H^0(X,-)$), so the spectral sequences (\ref{ringed topos sExt porp-2}) and (\ref{ringed topos sExt porp-3}) follow from \cref{abelian category G spectral sequence iff injective to acyclic}. If $X$ varies in $\mathcal{X}$, the spectral sequence (\ref{ringed topos sExt porp-3}) becomes a spectral sequence for presheaves, whence by passing to sheafification, a spectral sequence for presheaves. As the sheaf associated with the presheaf $X\mapsto H^p(X,-)$ is zero for $p>0$ (\cref{ringed site change universe cohomology prop}), this spectral sequence degenerates and we obtain the isomorphism in (b).
\end{proof}

\begin{proposition}\label{ringed topos sExt is delta-functor}
The functors $(M,N)\mapsto\sExt_A^p(M,N)$ form a $\delta$-functor on the variable $M$ or the variable $N$, and the same is true for the functors $(M,N)\mapsto\Ext_A^p(X;M,N)$ for any object $X$ of $\mathcal{X}$.
\end{proposition}
\begin{proof}
This follows from the general property of the functors $\Ext_A^p$ that for any object $X$ of $\mathcal{X}$, the functors $(M,N)\mapsto\Ext_A^p(X;M,N)$ form a $\delta$-functor on each of its variable, whence our assertion in view of \cref{ringed topos sExt porp}(b).
\end{proof}
