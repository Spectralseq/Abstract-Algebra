\chapter{Category}
\section{Functors}
\subsection{Definition of Functors}
Let $\mathcal{C},\mathcal{D}$ be two categories. A \textbf{covariant functor}
\[F:\mathcal{C}\to\mathcal{D}\]
is an assignment of an object $F(A)\in\Ob(\mathcal{D})$ for every $A\in\Ob(\mathcal{C})$ and of a function
\[\Mor_{\mathcal{C}}(A,B)\to\Mor_{\mathcal{D}}(F(A),F(B))\]
for every pair of objects $A,B$ in $\mathcal{C}$ such that $\forall f\in\Mor_{\mathcal{C}}(A,B),g\in\Mor_{\mathcal{C}}(B,C)$ we have
\[F(\id_A)=\id_{F_{A}},\quad F(g\circ f)=F(g)\circ F(f).\]

\begin{example}
If $R$ is a ring, we have denoted by $R^{\times}$ the group of units in $R$; every ring homomorphism $R\to S$ induces a group homomorphism $R^{\times}\to S^{\times}$, and this assignment is compatible with compositions; therefore this operation defines a covariant functor $\mathbf{Ring}\to\mathbf{Ab}$.
\end{example}
\subsection{Equivalence of Category}
The structure of an object in a category is adequately carried by its isomorphism class, and a natural notion of equivalence of categories should aim at matching isomorphism classes, rather than individual objects. The \textit{morphisms} are a more essential piece of information; the quality of a functor is first of all measured on how it acts on morphisms.
\begin{definition}
Let $\mathcal{C},\mathcal{D}$ be two categories. Let $F:\mathcal{C}\to\mathcal{D}$ be a covariant functor.
\begin{enumerate}
\item[(a)] $F$ is \textbf{faithful} if for all objects $A,B$ of $\mathcal{C}$, the induced function
\[\Mor_{\mathcal{C}}(A,B)\to\Mor_{\mathcal{D}}(F(A),F(D))\]
is injective.
\item[(b)] $F$ is \textbf{full} if this function is surjective, for all objects $A,B$.
\item[(c)] $F$ is called essentially surjective if for any object $B\in\Ob(\mathcal{B})$ there exists an object $A\in\Ob(\mathcal{A})$ such that $F(A)$ is isomorphic to $B$ in $\mathcal{B}$.
\end{enumerate}
\end{definition}
\begin{lemma}\label{fully faithful fuctor}
Let $F:\mathcal{C}\to\mathcal{D}$ be a fully faithful functor. If $A,B$ are objects in $\mathcal{C}$, then $A\cong B$ in $\mathcal{C}$ if and only if $F(A)\cong F(B)$ in $\mathcal{D}$.
\end{lemma}
\begin{proof}
Assume $F$ is covariant. Since we have $F(f\circ g)=F(f)\circ G(g)$, if $A\cong B$, we have $F(A)\cong\mathscr{B}$. Conversely, if $F(A)\cong F(B)$, assume $f$, $g$ are isomorphisms between them with $g=f^{-1}$. Since $F$ is full, there are two morphisms $f',g'$ such that $F(f')=f,F(g')=g$. Then we have
\[F(f'\circ g')=f\circ g=\mathrm{id},\quad F(g'\circ f')=g\circ f=\id\]
Since $F(\id)=\id$, and $F$ is faithful, we get $f'\circ g'=g'\circ f'=\id$, so $A\cong B$ follows.
\end{proof}
\begin{definition}
Let $\mathcal{C}, \mathcal{D}$ be categories, and let $F$, $G$ be functors $\mathcal{C}\to\mathcal{D}$. A \textbf{natural transformation} $\nu:F\to G$ is the datum of a morphism $\nu_X:F(X)\to G(X)$ in $\mathcal{D}$ for every object $X$ in $\mathcal{C}$, such that $\forall\alpha:X\to Y$ in $\mathcal{C}$ the diagram
\[\begin{tikzcd}
F(X)\ar[r,"\nu_X"]\ar[d,swap,"F(\alpha)"]&G(X)\ar[d,"G(\alpha)"]\\
F(Y)\ar[r,"\nu_Y"]&G(Y)
\end{tikzcd}\]
commutes. A \textbf{natural isomorphism} is a natural transformation $\mu$ such that $\mu_X$ is an isomorphism for every $X$.
\end{definition}
A natural transformation is often written as
\[\begin{tikzcd}
\mathcal{C}\ar[r,bend left=50,"F",""'{name=F,below}]\ar[r,bend right=50,swap,"G",""'{name=G,above}]&\mathcal{D}
\arrow[Rightarrow,from=F,to=G,"\nu"]
\end{tikzcd}\]
In addition, given a morphism of functors $\mu:F\to G$ and a morphism of functors $\nu:\mathscr{E}\to F$ then the composition $\nu\circ\mu$ is defined by the rule
\[(\nu\circ\mu)_X:=\nu_X\circ\mu_X.\]
for $X\in\Ob(\mathcal{A})$. It is easy to verify that this is indeed a morphism of functors from $\mathscr{E}$ to $G$. In this way, given categories $\mathcal{A}$ and $\mathcal{B}$ we obtain a new category, namely
the category of functors between $\mathcal{A}$ and $\mathcal{B}$.
\begin{definition}
An \textbf{equivalence of categories} $F:\mathcal{A}\to\mathcal{B}$ is a functor such that there exists a functor $G:\mathcal{B}\to\mathcal{A}$ such that the compositions $F\circ G$ and $G\circ F$ are isomorphic to the identity functors $\id_{\mathcal{B}}$, respectively $\id_{\mathcal{A}}$. In this case we say that $G$ is a \textbf{quasi-inverse} to $F$.
\end{definition}
\begin{proposition}\label{equiv category iff}
Let $F:\mathcal{A}\to\mathcal{B}$ be a fully faithful functor. Suppose for every $X\in\Ob(\mathcal{B})$ we are given an object $G(X)$ of $A$ and an isomorphism $i_X:X\to F(G(X))$. Then there is a unique functor $G:\mathcal{B}\to\mathcal{A}$ such that $G$ extends the rule on objects, and the isomorphisms $i_X$ define an isomorphism of functors $\id_{\mathcal{B}}\to F\circ G$. Moreover, $G$ and $F$ are quasi-inverse equivalences of categories.
\end{proposition}
\begin{proof}
The action of $G$ on objects is defined. For $X,Y\in\Ob(\mathcal{B})$ and $f\in\Mor_{\mathcal{B}}(X,Y)$, we have the diagram
\[\begin{tikzcd}
X\ar[r,"f"]\ar[d,"i_X"]&Y\ar[d,"i_Y"]\\
F(G(X))\ar[r,"i_Y\circ f\circ i_X^{-1}"]&F(G(Y))\\
G(X)\ar[r,dashed]\ar[u,"F"]&G(Y)\ar[u,"F"]
\end{tikzcd}\]
where the dashed map is induced by the bijection
\[\Mor_{\mathcal{A}}(X,Y)\cong\Mor_{\mathcal{B}}(F(X),F(Y))\]
by the functoriality, this defines a functor $G:\mathcal{B}\to\mathcal{A}$. Moreover, the upper half of the diagram means $i_X$ define an isomorphism of functors $\id_{\mathcal{B}}\to F\circ G$.\par
For $X\in\Ob(\mathcal{A})$, we have an isomorphism $i_{F(X)}:F(X)\to F\circ G\circ F(X)$. Since $F$ is full and faithful, there is an isomorphism $\mu_{X}:X\to G\circ F(X)$ by Lemma~\ref{fully faithful fuctor}. The naturality of $\mu_X$ follows from that of $i_{F(X)}$ and the faithfulness of $F$. Thus $\mu$ is an isomorphism $\id_{\mathcal{A}}\to G\circ F$.
\end{proof}
\begin{corollary}
A functor is an equivalence of categories if and only if it is both fully faithful and essentially surjective.
\end{corollary}
\begin{proof}
Let $F:\mathcal{A}\to\mathcal{B}$ be essentially surjective and fully faithful. As by convention all categories are small and as $F$ is essentially surjective we can, using the axiom of choice, choose for every $X\in\Ob(\mathcal{B})$ an object $G(X)$ of $\mathcal{A}$ and an isomorphism $i_X:X\to F(G(X))$. Then we apply Proposition~\ref{equiv category iff}.
\end{proof}
\subsection{Yoneda's Lemma}
\begin{definition}
Given a category $\mathcal{C}$ the opposite category $\mathcal{C}^{op}$ is the category with the same objects as $\mathcal{C}$ but all morphisms reversed.
\end{definition}
\begin{definition}
Let $\mathcal{A},\mathcal{B}$ be categories. A contravariant functor $F$ from $\mathcal{A}$ to $\mathcal{B}$ is a functor $\mathcal{A}^{op}\to\mathcal{B}$.\par
Concretely, a contravariant functor $F$ satisfies the property that, given another morphism $f:X\to Y$ and $g:Y\to Z$, we have 
\[F(g\circ f)=F(g)\circ F(f)\]
as morphism from $F(Z)$ to $F(X)$.
\end{definition}
\begin{definition}
Let $\mathcal{C}$ be a category. A \textbf{presheaf of sets on $\mathcal{C}$} or simply a presheaf is a contravariant functor $F$ from $\mathcal{C}$ to $\mathbf{Set}$. The category of presheaves is denoted $\mathbf{Psh}(\mathcal{C})$.
\end{definition}
\begin{example}[\textbf{Functor of points}]
For any $U\in\Ob(\mathcal{C})$ there is a contravariant functor
\[h_X:\mathcal{C}\to\mathbf{Set},\quad Y\mapsto\Mor_{\mathcal{C}}(Y,X).\]
In other words $h_X$ is a presheaf. We will always denote this presheaf $h_X:\mathcal{C}^{op}\to\mathbf{Set}$. It is called the \textbf{representable presheaf} associated to $X$.\par
Note that given a morphism $s:X\to Y$ in $\mathcal{C}$ we get a corresponding natural transformation of functors $h(s):h_X\to h_Y$ defined simply by composing with the morphism $U\to V$. It is trivial to see that this turns composition of morphisms in $\mathcal{C}$ into composition of transformations of functors. In other words we get a functor
\[h:\mathcal{C}\to\Fun(\mathcal{C}^{op},\mathbf{Set})=:\widehat{\mathcal{C}}.\]
\end{example}
\begin{lemma}[\textbf{Yoneda lemma}]
The functor $h$ is fully faithful. More generally, given any contravariant functor $F$ and any object $X$ of $\mathcal{C}$ we have a natural bijection
\[\Mor_{\widehat{\mathcal{C}}}(h_X,F)\to F(X),\quad \alpha\mapsto \alpha_X(\id_X).\]
\end{lemma}
\begin{proof}
An element $f\in h_X(Y)=\Mor_{\mathcal{C}}(Y,X)$ can be viewed as a morphism $f^*:\Mor_{\mathcal{C}}(X,X)\to\Mor_{\mathcal{C}}(Y,X)$. Note that $f^*(\id_X)=f$, so if there is a natural transformation $\alpha:h_X\to F$, then from the diagram
\[\begin{tikzcd}
\Mor_{\mathcal{C}}(X,X)\ar[r,"\alpha_X"]\ar[d,"f^*"]&F(X)\ar[d,"F(f)"]\\
\Mor_{\mathcal{C}}(Y,X)\ar[r,"\alpha_X"]&F(Y)
\end{tikzcd}\]
we obtain
\[\alpha_Y(f)=\alpha_Y\big(f^*(\id_X)\big)=F(f)\big(\alpha_X(\id_X)\big).\]
That is, $\alpha$ is simply determined by $\alpha_X(\id_X)$. Conversely, given $\xi\in F(X)$, we can define a natural transformation by the formula above:
\[\beta_Y:h_X(Y)\to F(Y),\quad \beta_Y(f)=F(f)(\xi).\]
It follows that these two map are inverse of each other.
\end{proof}
\begin{definition}
A contravariant functor $F:\mathcal{C}\to\mathbf{Set}$ is said to be \textbf{representable} if it is isomorphic to the functor of points $h_X$ for some object $X$ of $\mathcal{C}$.
\end{definition}
Let $\mathcal{C}$ be a category and let $F:\mathcal{C}^{op}\to\mathbf{Set}$ be a representable functor. Choose an object $X$ of $\mathcal{C}$ and an isomorphism $\alpha:h_X\to F$. The Yoneda lemma guarantees that the pair $(X,\alpha)$ is unique up to unique isomorphism. The object $X$ is called an object representing $F$.
\subsection{Limits and colimits}
The various universal properties encountered along the way are all particular cases of the notion of categorical \textbf{limit}, which is worth mentioning explicitly. Let $F:I\to C$ be a \textit{covariant functor}, where one thinks of $\mathcal{I}$ as a category of indices. The \textbf{limit} of $F$ is an object $L$ of $\mathcal{C}$, endowed with morphisms $\lambda_I:L\to F(I)$ for all objects $I$ of $\mathcal{I}$, satisfying the
following properties:
\begin{enumerate}
\item If $\alpha:I\to J$ is a morphism in $\mathcal{I}$, then $\lambda_J=F(\alpha)\circ\lambda_I$:
\[\begin{tikzcd}
&L\ar[ld,swap,"\lambda_I"]\ar[rd,"\lambda_J"]&\\
F(I)\ar[rr,swap,"F(\alpha)"]&&F(J)
\end{tikzcd}\]
\item $L$ is final with respect to this property: that is, if $M$ is another object, endowed with morphisms $\mu_I$, also satisfying the previous requirement, then there exists a unique morphism $M\to L$ making all relevant diagrams commute
\end{enumerate}
\begin{example}[\textbf{Products}]
Let $\mathcal{I}$ be the discrete category consisting of two objects $\bm{1}$, $\bm{2}$, with only identity morphisms, and let $\mathscr{A}$ be a functor from $\mathcal{I}$ to any category $\mathcal{C}$; let $A_1=\mathscr{A}(\bm{1})$, $A_2=\mathscr{A}(\bm{2})$ be the two objects of $\mathcal{C}$ indexed by $\mathcal{I}$. Then $\llim\mathscr{A}$ is nothing but the product of $A_1$ and $A_2$ in $\mathcal{C}$: a limit exists if and only if a product of $A_1$ and $A_2$ exists in $\mathcal{C}$.\par 
We can similarly define the product of any $($possibly infinite$)$ family of objects in a category as the limit over the corresponding discrete indexing category, provided of course that this limit exists.
\end{example}
The limit notion is a little more interesting if the indexing category $\mathcal{I}$ carries more structure.
\begin{example}[\textbf{Equalizers and kernels}]
Let $\mathcal{I}$ again be a category with two objects $\bm{1}$, $\bm{2}$, but assume that morphisms look like this:
\[\begin{tikzcd}
\bm{1}\ar[in=210, out=150,looseness=8]\ar[r,bend left,"\alpha"]\ar[r,swap,bend right,"\beta"]&\bm{2}\ar[in=-30, out=30,looseness=8]
\end{tikzcd}\]
That is, add to the discrete category two parallel morphisms $\alpha,\beta$ from one of the objects to the other. A functor $\mathscr{K}:\mathcal{I}\to\mathcal{C}$ amounts to the choice of two objects $A_1$, $A_2$ in $\mathcal{C}$ and two parallel morphisms between them. Limits of such functors are called \textbf{equalizers}. For a concrete example, assume $\mathcal{C}=R$-$\mathbf{Mod}$ is the category of $R$-modules for some ring $R$; let $\varphi:A_2\to A_1$ be a homomorphism, and choose $\mathscr{K}$ as above, with $\mathscr{K}(\alpha)=\varphi$ and $\mathscr{K}(\beta)=$ the zero-morphism. Then $\llim\mathscr{K}$ is nothing but the kernel of $\varphi$.
\end{example}
\begin{example}[\textbf{Limits over chains}]
In another typical situation, $\mathcal{I}$ may consist of a totally ordered set, for example:
\[\begin{tikzcd}
\cdots\ar[r]&\bm{4}\ar[r]&\bm{3}\ar[r]&\bm{2}\ar[r]&\bm{1}
\end{tikzcd}\]
$($that is, the objects are $\bm{i}$, for all positive integers $i$, and there is a unique morphism
$\bm{i}\to \bm{j}$ whenever $i\geq j$; we are only drawing the morphisms $\bm{j}+\bm{1}\to \bm{j}$$)$. Choosing $F:\mathcal{I}\to\mathcal{C}$ is equivalent to choosing objects $A_i$ of $\mathcal{C}$ for all positive integers $i$ and morphisms $\varphi_{ji}:A_i\to A_j$ for all $i\geq j$, with the requirement that $\varphi_{ii}=1_{A_i}$, and $\varphi_{kj}\circ\varphi_{ji}=\varphi_{ki}$ for all $i\geq j\geq k$. That is, the choice of $F$ amounts to the choice of a diagram
\[\begin{tikzcd}
\cdots\ar[r,"\varphi_{45}"]&A_4\ar[r,"\varphi_{34}"]&A_3\ar[r,"\varphi_{23}"]&A_2\ar[r,"\varphi_{12}"]&A_1
\end{tikzcd}\]
in $\mathcal{C}$. An inverse limit $\llim F$ $($which may also be denoted $\llim_iA_i$, when the morphisms $\varphi_{ji}$ are evident from the context$)$ is then an object $A$ endowed with morphisms $\varphi_i:A\to A_i$ such that the whole diagram
\[\begin{tikzcd}
&&&&A\ar[lllldd,swap,dashed,"\cdots"]\ar[lldd,"\varphi_4"]\ar[dd,"\varphi_3" description]\ar[rrdd,swap,"\varphi_2"]\ar[rrrrdd,"\varphi_1"]&&&&\\
&&&&\\
\cdots\ar[rr,"\varphi_{45}"]&&A_4\ar[rr,"\varphi_{34}"]&&A_3\ar[rr,"\varphi_{23}"]&&A_2\ar[rr,"\varphi_{12}"]&&A_1
\end{tikzcd}\]
commutes and such that any other object satisfying this requirement factors uniquely through $A$.\par
Such limits exist in many standard situations. For example, let $C=R$-$\mathbf{Mod}$ be
the category of left-modules over a fixed ring $R$, and let $A_i$, $\varphi_{ji}$ be as above.
\begin{proposition}
The limit $\llim_iA_i$ exists in $R$-$\mathbf{Mod}$.
\end{proposition}
\begin{proof}
The product $\prod_iA_i$ consists of arbitrary sequences $(a_i)_{i>0}$ of elements $a_i\in A_i$. Say that a sequence $(a_i)_{i>0}$ is \textit{coherent} if for all $i>0$ we have $a_i=\varphi_{i,i+1}(a_{i+1})$. Coherent sequences form an $R$-submodule $A$ of $\prod_iA_i$; the canonical projections restrict to $R$-module homomorphisms $\varphi_i:A\to A_i$. The reader will check that $A$ is a limit $\llim_iA_i$.
\end{proof}
This example easily generalizes to families indexed by more general posets.
\end{example}
The dual notion to limit is the \textbf{colimit} of a functor $F:\mathcal{I}\to\mathcal{C}$. The colimit is an object $C$ of $\mathcal{C}$, endowed with morphisms $\gamma_I:F(I)\to\mathcal{C}$ for all objects $I$ of $\mathcal{I}$, such that $\gamma_I=\gamma_J\circ F(\alpha)$ for all $\alpha:I\to J$ and that $C$ is \textit{initial} with respect to this requirement.\par
\begin{example}

For a typical situation consider again the case of a totally ordered set $\mathcal{I}$, for example:
\[\begin{tikzcd}
\bm{1}\ar[r]&\bm{2}\ar[r]&\bm{3}\ar[r]&\bm{4}\ar[r]&\cdots
\end{tikzcd}\]
A functor $F:\mathcal{I}\to\mathcal{C}$ consists of the choice of objects and morphisms
\[\begin{tikzcd}
A_1\ar[r,"\psi_{12}"]&A_2\ar[r,"\psi_{23}"]&A_3\ar[r,"\psi_{34}"]&A_4\ar[r,"\psi_{45}"]&\cdots
\end{tikzcd}\]
and the direct limit $\rlim_iA_i$ will be an object $A$ with morphisms $\psi_i:A_i\to A$ such
that the diagram
\[\begin{tikzcd}
A_1\ar[rrrrdd,"\psi_1"]\ar[rr,"\psi_{12}"]&&A_2\ar[rr,"\psi_{23}"]\ar[rrdd,"\psi_2"]&&A_3\ar[dd,"\psi_3" description]\ar[rr,"\psi_{34}"]&&A_4\ar[rr,"\psi_{45}"]\ar[lldd,swap,"\psi_4"]&&\cdots\ar[lllldd,dashed,swap,"\cdots"]\\
&&&&\\
&&&&A&&&&
\end{tikzcd}\]
commutes and such that $A$ is initial with respect to this requirement.
\end{example}
\begin{example}
If $\mathcal{C }=\mathbf{Set}$ and all the $\psi_{ij}$ are injective, we are talking about a nested sequence of sets:
\[A_1\sub A_2\sub A_3\sub\cdots\]
the direct limit of this sequence would be the infinite union $\bigcup_iA_i$.
\end{example}
\subsection{Exact functors}
\begin{definition}
Let $F:\mathcal{A}\to\mathcal{B}$ be a functor
\begin{enumerate}
\item[(a)] Suppose all finite limits exist in $\mathcal{A}$. We say $F$ is \textbf{left exact} if it commutes with all finite limits.
\item[(b)] Suppose all finite colimits exist in $\mathcal{A}$. We say $F$ is \textbf{right exact} if it commutes with all finite colimits.
\item[(c)] We say $F$ is \textbf{exact} if it is both left and right exact.
\end{enumerate}
\end{definition}
\begin{proposition}
Let $F:\mathcal{A}\to\mathcal{B}$ be a functor. Suppose all finite limits exist in $\mathcal{A}$. The following are equivalent:
\begin{enumerate}
\item[(a)] $F$ is left exact,
\item[(b)] $F$ commutes with finite products and equalizers.
\item[(c)] $F$ transforms a final object of $\mathcal{A}$ into a final object of $\mathcal{B}$, and commutes with fibre products.
\end{enumerate}
\end{proposition}
\subsection{Adjunction}
\begin{definition}
Let $\mathcal{C},\mathcal{D}$ be categories, and let $F:\mathcal{C}\to\mathcal{D}$, $G:\mathcal{D}\to\mathcal{C}$ be functors. We say that $F$ and $G$ are \textbf{adjoint} $($and we say that $G$ is right-adjoint to $F$ and $F$ is left-adjoint to $G$$)$ if there are natural isomorphisms
\[\tau_{XY}:\Mor_{\mathcal{C}}(X,G(Y))\stackrel{\sim}{\longrightarrow}\Mor_{\mathcal{D}}(F(X),Y)\]
for all objects $X$ of $\mathcal{C}$ and $Y$ of $\mathcal{D}$. More precisely, there should be a natural isomorphism of bifunctors \[\mathcal{C}^{op}\times\mathcal{D}\to\mathbf{Set}:\Mor_{\mathcal{C}}(-,G(-))\stackrel{\sim}{\to}\Mor_{\mathcal{D}}(F(-),-)\]
\end{definition}
\begin{proposition}
For each $Y$ there is a map $\eta_Y:FG(Y)\to Y$ so that for any for any $f:X\to G(Y)$, the corresponding map $\tau_{XY}(f):F(X)\to Y$ is given by the composition
\[\begin{tikzcd}
F(X)\ar[r,"F(f)"]&FG(Y)\ar[r,"\eta_Y"]&Y
\end{tikzcd}\]
Similarly, there is a map $\theta_X:X\to GF(X)$ for each $X$ so that $g:F(X)\to Y$, the corresponding $\tau^{-1}_{XY}(g):X\to G(Y)$ is given by the composition
\[\begin{tikzcd}
X\ar[r,"\theta_X"]&GF(Y)\ar[r,"G(g)"]&G(Y)
\end{tikzcd}\]
So the information of $\tau_{XY}$ is the same as these two maps.
\end{proposition}
\begin{proof}
We deal with the first case. Let $f:X\to G(Y)$ be a map, consider the follwing diagram
\[\begin{tikzcd}
\Mor_{\mathcal{C}}(X,G(Y))\ar[r,"\tau_{XY}"]&\Mor_{\mathcal{D}}(F(X),Y)\\
\Mor_{\mathcal{C}}(G(Y),G(Y))\ar[r,"\tau_{G(Y)Y}"]\ar[u,"f^*"]&\Mor_{\mathcal{D}}(FG(Y),Y)\ar[u,"F(f)^*"]
\end{tikzcd}\]
Set $\eta_Y$ to be the image of $\mathrm{id}_{G(Y)}$ under $\tau_{G(Y)Y}$ we get the claim. The second can be done similarly.
\end{proof}
\begin{proposition}
Let $F$ be a left adjoint to $G$. Then
\begin{enumerate}
\item[(a)] $F$ is fully faithful if and only if $\id_{\mathcal{C}}\cong G\circ F$.
\item[(b)] $G$ is fully faithful if and only if $F\circ G\cong\id_{\mathcal{D}}$.
\end{enumerate}
\end{proposition}
\begin{proof}
Assume $F$ is fully faithful. We have to show the adjunction map $X\to G\circ F(X)$ is an isomorphism. Let $X'\to G\circ F(X)$ be any morphism. By adjointness this corresponds to a morphism $F(X')\to F(X)$. By fully faithfulness of $F$ this corresponds to a morphism $X'\to X$. Thus we see that $X\mapsto F\circ G(X)$ defines a bijection \[\Mor_{\mathcal{C}}(X',X)\to\Mor(X',GF(X))\]
Hence it is an isomorphism. Conversely, if $\id_{\mathcal{C}}\cong G\circ F$ then $F$ has to be fully faithful, as $G$ defines an left-inverse on morphism sets. The other case is the dual part.
\end{proof}
\begin{proposition}
Let $F:\mathcal{C}\to\mathcal{D}$ be a functor between categories. If for each $Y\in\Ob(\mathcal{D})$ the functor $\Mor_{\mathcal{D}}(F(-),Y)$ is representable, then $F$ has a right adjoint.
\end{proposition}
\begin{proof}
For each $Y$ we choose an object $G(Y)$ and an isomorphism 
\[\Mor_{\mathcal{C}}(-,G(Y)) \stackrel{\sim}{\to} \Mor_{\mathcal{D}}(F(-),Y)\]
of functors. By Yoneda's lemma for any morphism $g:Y\to Y'$ the transformation of functors
\[\begin{tikzcd}
\Mor_{\mathcal{C}}(-,G(Y))\ar[r,"\sim"]&\Mor_{\mathcal{D}}(F(-),Y)\ar[r]&\Mor_{\mathcal{D}}(F(-),Y')\ar[r,"\sim"]&\Mor_{\mathcal{C}}(-,G(Y'))
\end{tikzcd}\]
corresponds to a unique morphism $G(g):G(Y)\to G(Y')$. The functoriality of $G$ comes from that of $F$.
\end{proof}
\begin{example}
The construction of the free group on a given set is concocted so that giving a set-function from a set $A$ to a group $G$ is the same as giving a group homomorphism from $F(A)$ to $G$. What this really means is that for all sets $A$ and all groups $G$ there are natural identifications
\[\Mor_{\mathbf{Set}}(A,S(G))\stackrel{\sim}{\longrightarrow}\Mor_{\mathbf{Grp}}(F(A),G)\]
where $S(G)$ forgets the group structure of $G$. That is, the functor $F:\mathbf{Set}\to\mathbf{Grp}$ constructing free groups is left-adjoint to the forgetful functor $S:\mathbf{Grp}\to\mathbf{Set}$. This of course applies to every other construction of free objects we have encountered: the free functor is, as a rule, left-adjoint to the forgetful functor.
\end{example}
In fact, the very fact that a functor has an adjoint will endow that functor with convenient features. We say that $F$ is a \textbf{left-adjoint functor} if it has a right adjoint, and that $G$ is a \textbf{right-adjoint functor} if it has a left-adjoint.
\begin{theorem}\label{radjoint limit}
Let $F$ be a left adjoint to $G$.
\begin{enumerate}
\item[(a)] Suppose that $\mathscr{A}:\mathcal{I}\to\mathcal{C}$ is a diagram, and suppose that $\llim\mathscr{A}$ exists in $\mathcal{C}$. Then 
\[G(\llim\mathscr{A})=\llim(G\circ\mathscr{A})\]
In other words, $G$ commutes with limits.
\item[(b)] Suppose that $\mathscr{A}:\mathcal{I}\to\mathcal{C}$ is a diagram, and suppose that $\rlim\mathscr{A}$ exists in $\mathcal{C}$. Then 
\[F(\rlim\mathscr{A})=\rlim(F\circ\mathscr{A})\]
In other words, $F$ commutes with colimits.
\end{enumerate}
\end{theorem}
\begin{proof}
A morphism from a colimit into an object is the same as a compatible system of morphisms from the constituents of the limit into the object, so
\begin{align*}
\Mor_{\mathcal{C}}(X,G(\llim\mathscr{A}))&\cong\Mor_{\mathcal{D}}(F(X),\llim\mathscr{A})=\llim\Mor_{\mathcal{D}}(F(X),\mathscr{A}_i)=\llim\Mor_{\mathcal{D}}(X,G(\mathscr{A}_i))
\end{align*}
proves that $G(\llim\mathscr{A})$ is the limit we are looking for. A similar argument works for the other statement.
\end{proof}
\begin{corollary}
Let $F$ be a left adjoint to $G$.
\begin{enumerate}
\item[(a)] If $\mathcal{C}$ has finite colimits, then $F$ is right exact. 
\item[(b)] If $\mathcal{D}$ has finite limits, then $G$ is right exact. 
\end{enumerate}
\end{corollary}
\subsection{Exercise}
\begin{exercise}
Let $F:\mathcal{C}\to\mathcal{D}$ be a covariant functor, and assume that both $\mathcal{C}$ and $\mathcal{D}$ have products. Prove that for all objects $A$, $B$ of $\mathcal{C}$, there is a unique morphism $F(A\times B)\to F(A)\times F(B)$ such that the relevant diagram involving natural
projections commutes.\par
If $\mathcal{D}$ has coproducts $($denoted $\amalg$$)$ and $G:\mathcal{C}\to\mathcal{D}$ is contravariant, prove that there is a unique morphism $G(A)\amalg G(B)\to G(A\times B)$ $($again, such that an appropriate diagram commutes$)$.
\end{exercise}
\begin{proof}
Apply the functor $F$ yields:
\[\begin{tikzcd}
&A\times B\ar[ld]\ar[rd]&\\
A&&B
\end{tikzcd}\stackrel{F}{\Longrightarrow}\begin{tikzcd}
&F(A\times B)\ar[ld]\ar[rd]&\\
F(A)&&F(B)
\end{tikzcd}\]
by the universal property of $F(A)\times F(B)$, there is a unique morphism:
\[F(A\times B)\stackrel{\exists !}{\longrightarrow}F(A)\times F(B)\]
Similar for coproducts:
\[\begin{tikzcd}
&A\times B\ar[ld]\ar[rd]&\\
A&&B
\end{tikzcd}\stackrel{G}{\Longrightarrow}\begin{tikzcd}
&G(A\times B)&\\
G(A)\ar[ru]&&G(B)\ar[lu]
\end{tikzcd}\]
By the universal property of $G(A)\amalg G(B)$, we get
\[G(A)\amalg G(B)\to G(A\times B)\]
\end{proof}
\begin{exercise}
Let $\mathcal{C}$ be a small category. Prove that $\mathcal{C}$ is equivalent to the subcategory of representable functors in $\mathbf{Set}^{\mathcal{C}^{\circ}}$. Thus, every $(small)$ category is equivalent to a subcategory of a functor category.
\end{exercise}
\begin{proof}
For $\varphi:A\to B$, there is an induced natural transformation:
\[\varphi:\Hom_{\mathcal{C}}(-,A)\to\Hom_{\mathcal{C}}(-,B)\]
from Yoneda lemma, there is a bijection from $\Hom(h_A,h_B)$ to $h_B(A)=\Hom(A,B)$. So $h$ is a fully faithful covariant functor. For any representable $F$, there is a functor $h_X$ and a natural isomorphism $F\cong h_X$. This shows $h$ is a equivalence of categories. 
\end{proof}
\begin{exercise}\label{adic comple}
Let $R$ be a commutative ring, and let $I\sub R$ be an ideal. Note that $I^n\sub I^m$ if $n\geq m$, and hence we have natural homomorphisms $\varphi_{mn}:R/I^n\to R/I^m$ for $n\geq m$.
\begin{enumerate}
\item Prove that the inverse limit $\widehat{R}_I:=\llim_nR/I^n$ exists as a commutative ring. This
is called the $I$-adic completion of $R$.
\item By the universal property of inverse limits, there is a unique homomorphism $R\to\widehat{R}_I$. Prove that the kernel of this homomorphism is $\bigcap_nI^n$.
\item Let $I=(x)$ in $R[x]$. Prove that the completion $\widehat{R[x]}_I$ is isomorphic to the power series ring $R[[x]]$.
\end{enumerate}
\end{exercise}
\begin{proof}
We first prove that limit exists in $\mathbf{Ring}$. Let $\mathcal{I}$ be a poset $(\mathcal{I},\leq)$. Choose $\{R_i\}_{i\in\mathcal{I}}$ and $\{\varphi_{ij}:R_i\to R_j\}$ such that
\[i\geq j\geq k\Rightarrow\varphi_{jk}\circ\varphi_{ij}=\varphi_{ik}\]
A sequence $(r_i)_{i\in\mathcal{I}}$ is coherent if $\varphi_{ij}(r_i)=r_j$. Define the limit $\llim_{i}R_i$ to be
\[\llim_iR_i:=\{(r_i)_{i\in\mathcal{I}}\mid (r_i)\text{ is coherent}\}.\]

Since $I^n\sub I^m$ for $n\geq m$, there is a well defined quotient homomorphism:
\[\varphi_{nm}:R/I^n\to R/I^m,\quad a+I^n\mapsto a+I^m\]
So the limit $\llim_iR/I^n$ is well defined.\par
For the homomorphism $\psi_n:R\to R/I^n$, it is clear that
\[\varphi_{nm}\circ\psi_n=\psi_m\]
so we get the unique homomorphsim $\psi:R\to\widehat{R}_I$, defined by $\psi(r)=(\psi_i(r))_{i\in\mathcal{I}}$. It follows that
\[\psi(r)=0\iff \psi_i(r)=0\iff r\in\bigcap_nI^n.\]

Finally, if $I=(x)$, then $i^n=(x^n)$. So 
\[\widehat{R[x]}_I=\{(r_i)_{i\in\N}:\deg r_i<i,r_i=r_{i-1}+a_{i-1}x^{i-1}\}\]
this set equals $R[[x]]$.
\end{proof}
\begin{exercise}
An important example of the construction presented in Exercise~\ref{adic comple} is the ring $\Z_p$ of $p$-adic integers: this is the limit $\llim_r\Z/p^r\Z$, for a positive prime integer $p$.\par
The field of fractions of $\Z_p$ is denoted $\Q_p$; elements of $\Q_p$ are called \textbf{$p$-adic numbers}.
\begin{enumerate}
\item Show that giving a $p$-adic integer $A$ is equivalent to giving a sequence of integers $A_r, r\geq 1$, such that $0\leq A_r<p^r$, and that $A_s\equiv A_r$ mod $p^s$ if $s\leq r$.
\item Equivalently, show that every $p$-adic integer has a unique infinite expansion $A=a_0+a_1\cdot p+a_2\cdot p^2+a_3\cdot p^3+\cdots$, where $0\leq a_i\leq p-1$. The arithmetic of $p$-adic integers may be carried out with these expansions in precisely the same way as ordinary arithmetic is carried out with ordinary decimal expansions.
\item With notation as in the previous point, prove that $A\in\Z_p$ is invertible if and only if $a_0\neq 0$.
\item Prove that $\Z_p$ is a local domain, with maximal ideal generated by $($the image in $\Z_p$ of$)$ $p$.
\item Prove that $\Z_p$ is a DVR. $($There is an evident valuation on $\Q_p$.$)$
\end{enumerate}
\end{exercise}
\begin{proof}
Every $A_r$ is in $\Z/p^r\Z$, so $0\leq A_r<p^r$. From the construction, $A_s+p^s=\varphi_{rs}(A_r+p^r)=A_r+p^s$ for $s\leq r$, we see that $A_r\equiv A_s$ mod $p^s$ if $s\leq r$.\par
Similar to the example $R[[x]]$. Giving a sequence is the same as giving a truncation of a series. And the third point is the same as series $R[[x]]$. From the previous point, we find that $\Z_p/p\Z_p$ is a field, so $p\Z_p$ is a maximal ideal.\par
Define a function $v_p(x):=\sup\{n:x\in p^n\Z_p\}=\inf\{n:x_n\neq 0\}$. Then for any ideal $I\sub\Z_p$, let $n:=\min\{v_p(x):x\in I\}$, then $I\sub p^n\Z_p$. Now let $y=p^nx\in I$, then $x$ is invertible, so $(p^nx)=p^n\Z_p\sub I$. This shows every ideal in $\Z_p$ has the form $p^n\Z_p$, and $p\Z_p$ is the unique maximal ideal.
\end{proof}
\begin{exercise}\label{completion of Z}
If $m,n$ are positive integers and $m\mid n$, then $(n)\sub (m)$, and there is an onto ring homomorphism $\Z/n\Z\twoheadrightarrow\Z/m\Z$. The limit ring $\llim\Z/n\Z$ exists and is denoted by $\widehat{\Z}$. Prove that $\widehat{\Z}=\End_{\mathbf{Ab}}(\Q/\Z)$. 
\end{exercise}
\begin{proof}
Every $f\in\End_{\mathbf{Ab}}(\Q/\Z)$ is uniquely determined by $f(\frac{1}{n})$. Since $n\cdot f(\frac{1}{n})=f(1)=f(0)=0$, $f(\frac{1}{n})=\frac{g(n)}{n}$ for some integer $g(n)$. Since we are deal with $\Q/\Z$, we may choose $0\leq g(n)<n$.\par
For $m\mid n$, we have $n=am$, so
\[f(\dfrac{1}{n})=f(\dfrac{1}{am})=\dfrac{g(n)}{am},\quad f(\dfrac{1}{m})=\dfrac{g(m)}{m}\]
and
\[a\cdot f(\dfrac{1}{n})=\dfrac{g(n)}{m}=f(\dfrac{1}{m})=\dfrac{g(m)}{m}\]
so we have $g(n)\equiv g(n)$ mod $m$. This means the sequence
\[(f(\dfrac{1}{n}))_{i\in\N}\]
is an element of $\widehat{\Z}$. Conversely, any element in $\widehat{\Z}$ uniquely defines an endomorphism of $\Q/\Z$. So we have $\widehat{\Z}=\End_{\mathbf{Ab}}(\Q/\Z)$.
\end{proof}
\begin{exercise}
Let $\widehat{\Z}$ be as in Exercise~\ref{completion of Z}.
\begin{enumerate}
\item If $R$ is a commutative ring endowed with homomorphisms $R\to\Z/p^r\Z$ for all primes $p$ and all $r$, compatible with all projections $\Z/p^r\Z\to\Z/p^s\Z$ for $s\leq r$, prove that there are ring homomorphisms $R\to\Z/n\Z$ for all $n$, compatible with all projections $\Z/n\Z\to\Z/m\Z$ for $m\mid n$.
\item Deduce that $\widehat{\Z}$ satisfies the universal property for the product of $\Z_p$, as $p$ ranges over all positive prime integers.
\end{enumerate}
It follows that $\prod_p\Z_p\cong\widehat{\Z}\cong\End_{\mathbf{Ab}}(\Q/\Z)$.
\end{exercise}
\begin{proof}
For any $n=p_1^{r_1}\cdots p_i^{r_i}$, $m=p_1^{r'_1}\cdots p_i^{r'_i}$ with $m\mid n$, by Chinese remainder theorem we have a commutative diagram
\[\begin{tikzcd}
\prod_{i}\Z/p_i^{r_i}\Z\ar[r,"\sim"]\ar[d, twoheadrightarrow]&\Z/n\Z\ar[d, twoheadrightarrow]\\
\prod_{i}\Z/p_i^{r'_i}\Z\ar[r,"\sim"]&\Z/m\Z
\end{tikzcd}\]
so we get the first result.\par
Note that giving a morphism from $R$ to $\Z_p$ is the same as giving morphisms from $R$ to $\Z/p^r\Z$ fro all $r$. So if there is a ring $R$ with morphisms to $\Z_p$ for all prime $p$, then we get morphisms to $\Z/p^r\Z$ for all prime $p$, all $r$. From the previous point, there are morphisms $R\to\Z/n\Z$ for all $n$, compatible with all projection $\Z/n\Z\to\Z m\Z$. From the definition of $\widehat{\Z}$, there is a unique morphism from $R$ to $\widehat{\Z}$. So $\widehat{\Z}$ satisfies the universal property of $\prod_p\Z_p$.
\end{proof}
\begin{exercise}
Let $R,S$ be rings. An additive covariant functor $F:R$-$\mathbf{Mod}\to S$-$\mathbf{Mod}$ is \textbf{faithfully exact} if \[\begin{tikzcd}
F(A)\ar[r,"F(\varphi)"]&F(B)\ar[r,"F(\psi)"]&F(C)
\end{tikzcd}\] 
is exact in $S$-$\mathbf{Mod}$ if and only if 
\[\begin{tikzcd}
A\ar[r,"\varphi"]&B\ar[r,"\psi"]&C
\end{tikzcd}\]
is exact in $R$-$\mathbf{Mod}$. Prove that an exact functor $F:R$-$\mathbf{Mod}\to S$-$\mathbf{Mod}$ is faithfully exact if and only if $F(M)\neq0$ for every nonzero $R$-module $M$, if and only if $F(\varphi)\neq0$ for every nonzero morphism $\varphi$ in $R$-Mod.
\end{exercise}
\begin{proof}
\mbox{}
\begin{enumerate}
\item One direction is easy: If $F$ is faithfully exact. Assume $F(M)=0$, then the sequence $0\to F(M)\to0$ is exact, but $0\to M\to 0$ is not exact unless $M=0$, so we find $M=0$. If $F(\varphi)=0$, then 
\[\begin{tikzcd}
F(M)\ar[r,"F(\varphi)"]&F(N)\ar[r,"id_{F(N)}"]&F(N)
\end{tikzcd}\] 
is exact, but 
\[\begin{tikzcd} 
M\ar[r,"\varphi"]&N\ar[r,"id_N"]&N 
\end{tikzcd}\] 
is exact only if $\varphi=0$.
\item Then we show that $F$ reflects zero objects if and only if $F$ reflects zero morphisms: If $F$ reflects zero morphisms, assume $F(X)=0$, then $\id_{F(X)}=0$, so $\id_X=0$. But $id_X=0$ if and only if $X=0$, so $X=0$.\par 
Now assume $F$ reflects zero objects. For $\varphi:A\to B$ such that $F(\varphi)=0$. Consider the exact sequence:
\[\begin{tikzcd}
A\ar[r,"\varphi"]&\im\varphi\ar[r,"\psi"]&\coker\varphi
\end{tikzcd}\]
since $F$ is exact, we also have a exact sequence
\[\begin{tikzcd}
F(A)\ar[r,"F(\varphi)"]&F(\im\varphi)\ar[r,"F(\psi)"]&F(\coker\varphi)
\end{tikzcd}\]
Note that $F(\varphi)=0$, so $F(\psi)$ is monic. But $\psi=0$ so $F(\psi)=0$, we conclude that $F(\im\varphi)=0$. This means $\im\varphi=0$, so we have $\varphi=0$.
\item Now we show the last direction. Let $F$ reflects zero morphisms, first we show that $F$ reflects monomorphisms and epimorphisms. In deed, suppose
\[\begin{tikzcd}
0\ar[r]&X\ar[r]&Y\ar[r]&Z
\end{tikzcd}\]
is exact; then
\[\begin{tikzcd}
0\ar[r]&F(X)\ar[r]&F(Y)\ar[r]&F(Z)
\end{tikzcd}\]
is exact. If $F(Y)\to F(Z)$ is monic, then $F(X)=0$, so $X=0$. This shows $F$ reflects monomorphisms, the dual argument shows that $F$ reflects epimorphisms. Now, in an abelian category, $f$ is an isomorphism if and only if $f$ is both monic and epic, so this implies $F$ reflects isomorphisms. Now suppose $X\to Y\to Z$ is given and
\[\begin{tikzcd}
0\ar[r]&F(X)\ar[r,"F(f)"]&F(Y)\ar[r,"F(g)"]&F(Z)
\end{tikzcd}\]
is exact. Since $F(X)\to\ker F(g)$ is an isomorphism, and $\ker F(g)=F(\ker g)$, $X\to\ker g$ is also an isomorphism, so $X\to Y\to Z$ is exact.
\end{enumerate}
\end{proof}
\begin{exercise}\label{locali exact}
Prove that localization is an exact functor.\par
In fact, prove that localization preserves homology: if
\[\begin{tikzcd}
M_{\bullet}:&\cdots\ar[r]&M_{i+1}\ar[r,"d_{i+1}"]&M_i\ar[r,"d_i"]&M_{i-1}\ar[r]&\cdots
\end{tikzcd}\]
is a complex of $R$-modules and $S$ is a multiplicative subset of $R$, then the localization of the $i$-th homology of $M_{\bullet}$ is the $i$-th homology $H_i(S^{-1}M_{\bullet})$ of the localized complex
\[\begin{tikzcd}
S^{-1}M_{\bullet}:&\cdots\ar[r]&S^{-1}M_{i+1}\ar[r,"S^{-1}d_{i+1}"]&S^{-1}M_i\ar[r,"S^{-1}d_i"]&S^{-1}M_{i-1}\ar[r]&\cdots
\end{tikzcd}\]
\end{exercise}
\begin{proof}
Since 
\[d_i(\dfrac{m}{s})=d(\dfrac{ms'}{ss'})\]
we have
\[\ker S^{-1}d_i=\{\dfrac{m}{s}\mid \exists r\in S, rm\in\ker d_i\},\quad\im S^{-1}d_{i+1}=\{\dfrac{m}{s}\mid\exists r\in s, rm\in\im d_{i+1}\}\]
Concerning the quotient, first we observet that, in the construction of $S^{-1}H_i(M_{\bullet})$:
\begin{align*}
\dfrac{a+\im d_{i+1}}{s}=\dfrac{a'+\im d_{i+1}}{s'}&\iff (\exists r\in S)\quad r[(a+\im d_{i+1})s'-(a'+\im d_{i+1})s]=0\text{ in $H_i(M_{\bullet})$ }\\
&\iff (\exists r\in S)\quad r(as'-a's)\in\im d_{i+1}\end{align*}
While in the quotient $H_i(S^{-1}M_{\bullet})$:
\[\dfrac{a}{s}+\im S^{-1}d_{i+1}=\dfrac{a'}{s'}+\im S^{-1}d_{i+1}\iff \dfrac{a}{s}-\dfrac{a'}{s'}\in\im S^{-1}d_{i+1}\iff (\exists r\in S)\  r(as'-a's)\in\in d_{i+1}\]
So there is a natural homomorphism:
\[\psi:S^{-1}H_i(M_{\bullet})\to H_i(S^{-1}M_{\bullet}),\quad \dfrac{a+\im d_{i+1}}{s}\mapsto\dfrac{a}{s}+\im S^{-1}d_{i+1}\]
this is an isomorphism from the observation above.
\end{proof}
\begin{exercise}
Suppose $M$ is a finitely presented $R$-module and $N$ is an arbitrary $R$-module. Show the followsing holds
\[S^{-1}\Hom_R(M,N)\stackrel{\sim}{\longrightarrow}\Hom_{S^{-1}R}(S^{-1}M,S^{-1}N)\]
But note that this does not holds for any module.
\end{exercise}
\begin{proof}
First we have
\[S^{-1}\Hom_R(R,N)\stackrel{\sim}{\longrightarrow}\Hom_{S^{-1}R}(S^{-1}R,S^{-1}N)\]
for any $N$. And we have a natural homomorphism 
\[S^{-1}\Hom_A(M,N)\to \Hom_{S^{-1}A}(S^{-1}M,S^{-1}N).\] 
Consider the diagram:
\[\begin{tikzcd}[column sep=small]
0\ar[r]&S^{-1}\Hom_{R}(M,N)\ar[r]\ar[d]&S^{-1}\Hom_{R}(R^m,N)\ar[d]\ar[r]&S^{-1}\Hom_{R}(R^n,N)\ar[d]\\
0\ar[r]&\Hom_{S^{-1}R}(S^{-1}M,S^{-1}N)\ar[r]&\Hom_{S^{-1}R}(S^{-1}R^m,S^{-1}N)\ar[r]&\Hom_{S^{-1}R}(S^{-1}R^n,S^{-1}N)
\end{tikzcd}\]
The right two vertical maps are isomorphisms, so we get the isomorphism.\par
For $R=N=\Z$, $M=\Q$, $S=\Z-\{0\}$, we have
\[S^{-1}\Hom_\Z(\Q,\Z)=S^{-1}\{0\}=0,\quad \Hom_{S^{-1}\Z}(S^{-1}\Q,S^{-1}\Z)=\Hom_{\Q}(\Q,\Q)=\Q\]
\end{proof}
\begin{exercise}
Suppose $F:\mathcal{A}\to\mathcal{B}$ is a covariant functor of abelian categories, and $C^\bullet$ is a complex in $\mathcal{A}$.
\begin{enumerate}
\item[(a)]If $F$ is right-exact, describe a natural morphism $FH^\bullet\to H^\bullet F$.
\item[(b)]If $F$ is right-exact, describe a natural morphism $FH^\bullet\leftarrow H^\bullet F$.
\item[(c)]If $F$ is exact, show that the morphisms of (a) and (b) are inverses and thus isomorphisms.
\end{enumerate}
\end{exercise}
\begin{proof}
First we recall that if $F$ is right-exact, then $F$ commutes with cokernels: For we have the following exact sequence
\[\begin{tikzcd}
0\ar[r]&F(C^i)\ar[r,"F(d^{i})"]&F(C^{i+1})\ar[r]&F(\coker d^i)\ar[r]&0
\end{tikzcd}\]
which is obtained from the corresponding short exact sequence. Hence
\[F(\coker d^i)\cong \coker F(d^i)\]
\begin{enumerate}
\item[(a)]Consider the exact sequence
\[\begin{tikzcd}
0\ar[r]&\im d^i\ar[r]&C^{i+1}\ar[r]&\coker d^i\ar[r]&0
\end{tikzcd}\]
Applying $F$ on this gives us
\[\begin{tikzcd}
F\im d^i\ar[r]&F(C^{i+1})\ar[r]&F\coker d^i\ar[r]&0
\end{tikzcd}\]
Together with the similar sequence in $F(C^\bullet)$ we get a diagram
\[\begin{tikzcd}
&F\im d^i\ar[r]\ar[d,dashed,"\alpha"]&F(C^{i+1})\ar[r]\ar[d,equal]&F\coker d^i\ar[d,"\cong"]\ar[r]&0\\
0\ar[r]&\im F(d^i)\ar[r]&F(C^{i+1})\ar[r]&F\coker d^i\ar[r]&0
\end{tikzcd}\]
Then we can show there is an induced map $\alpha:F\im d^i\to\im f(d^i)$. Further, by the snake lemma, this induced map $\alpha$ is an epimorphism.
Now consider another sequence 
\[\begin{tikzcd}
0\ar[r]&H^i(C^\bullet)\ar[r]&\coker d^{i-1}\ar[r]&\im d^i\ar[r]&0
\end{tikzcd}\]
Applying $F$ gives 
\[\begin{tikzcd}
FH^i(C^\bullet)\ar[r]&F\coker d^{i-1}\ar[r]&F\im d^i\ar[r]&0
\end{tikzcd}\]
Similarly, with the counterpart in $F(C^\bullet)$, there is a diagram
\[\begin{tikzcd}
&FH^i(C^\bullet)\ar[r]\ar[d,dashed,"\beta"]&F\coker d^{i-1}\ar[r]\ar[d,"\cong"]&F\im d^i\ar[r]\ar[d,"\alpha"]&0\\
0\ar[r]&H^iF(C^\bullet)\ar[r]&\coker F(d^{i-1})\ar[r]&\im F(d^i)\ar[r]&0
\end{tikzcd}\]
Together with $\alpha$, we get our desired map 
\[\beta:FH^i(C^\bullet)\to H^iF(C^\bullet)\]
\item[(b)]Instead of (a), we may use the sequence for kernels:
\[\begin{tikzcd}
0\ar[r]&\ker d^i\ar[r]&C^i\ar[r]&\im d^i\ar[r]&0
\end{tikzcd}\]
and
\[\begin{tikzcd}
0\ar[r]&\im d^{i-1}\ar[r]&\ker d^i\ar[r]&H^i(C^\bullet)\ar[r]&0
\end{tikzcd}\]
with the identification
\[\ker F(d^i)\cong F(\ker d^i)\]
\item[(c)]With the exactness hypothesis, the map we obtained all becomes isomorphisms.
\end{enumerate}
\end{proof}
\section{Presheaves of sets}\label{category presheaf section}
In this section, we consider the category of presheaves of sets over a category $\mathcal{C}$, and prove some of its properties. In order to avoid set-theoretic issues, we fix once for all a universe $\mathscr{U}$ which has an element with infinite cardinality. A set is said to be \textbf{$\mathscr{U}$-small} (or simply \textbf{small} if there is no confusion) if it is isomorphic to an element of $\mathscr{U}$. We also use the following terminology: small group, small ring, small category. We often assume that the schemes, topological spaces, sets of indices, with which we work are $\mathscr{U}$-small, or at least have cardinality belonging to $\mathscr{U}$. A category $\mathcal{C}$ is called a \textbf{$\mathscr{U}$-category} if for any objects $x,y$ in $\mathcal{C}$, the set $\Hom_\mathcal{C}(x,y)$ is $\mathscr{U}$-small, and is called $\mathscr{U}$-small if the set $\Ob(\mathcal{D})$ is also contained in the universe $\mathscr{U}$. For two categories $\mathcal{C}$, $\mathcal{D}$, we denote by $\sHom(\mathcal{C},\mathcal{D})$ the category of (covariant) functors from $\mathcal{C}$ to $\mathcal{D}$. It is then easy to verify the following two conditions:
\begin{itemize}
\item If $\mathcal{C}$ and $\mathcal{D}$ are elements of $\mathscr{U}$ (resp. $\mathscr{U}$-small), then $\sHom(\mathcal{C},\mathcal{D})$ is an element of $\mathscr{U}$ (resp. $\mathscr{U}$-small).
\item If $\mathcal{C}$ is a $\mathscr{U}$-small category and $\mathcal{D}$ is a $\mathscr{U}$-category, $\sHom(\mathcal{C},\mathcal{D})$ is a $\mathscr{U}$-category.
\end{itemize}
However, note that if $\mathcal{D}$ is a $\mathscr{U}$-small category and $\mathcal{C}$ is a $\mathscr{U}$-category, then $\sHom(\mathcal{C},\mathcal{D})$ is not $\mathscr{U}$-small in general. For example, the category $\sHom(\mathcal{C},\mathscr{U}\text{-}\mathbf{Set})$. It should be noted that $\mathscr{U}$-smallness is really a restrictive condition for categories, and there are many interesting examples where this condition is not satisfied in general.
\subsection{The category of presheaves of sets}
Let $\mathcal{C}$ be a category. We define the \textbf{category of presheaves of sets over $\mathcal{C}$ relative to the universe $\mathscr{U}$} (or, if there is no confusion, the category of presheaves of sets over $\mathcal{C}$) to be the category of contravariant functors from $\mathcal{C}$ to the category of $\mathscr{U}$-sets, and denote it by $\PSh(\mathcal{C})_{\mathscr{U}}$ (or simply $\PSh(\mathcal{C})$ if there is no risk of confusion). The objects of $\PSh(\mathcal{C})_{\mathscr{U}}$ are called \textbf{$\mathscr{U}$-presheaves} (of simply presheaves) over $\mathcal{C}$. If $\mathcal{C}$ is $\mathscr{U}$-small, then $\PSh(\mathcal{C})_{\mathscr{U}}$ is a $\mathscr{U}$-category. However, this is not true in general if $\mathcal{C}$ is only assumed to be a $\mathscr{U}$-category.\par
Let $x$ be an object of a $\mathscr{U}$-category $\mathcal{C}$. We can associate with $x$ a presheaf $h_x:\mathcal{C}^{\op}\to \mathscr{U}\text{-}\mathbf{Set}$, defined in the following way:
\begin{itemize}
\item If $\Hom_\mathcal{C}(y,x)$ is an element of $\mathscr{U}$, then we set $h_x(y)=\Hom_\mathcal{C}(y,x)$.
\item Suppose that $\Hom_\mathcal{C}(y,x)$ is not an element of $\mathscr{U}$ and let $R(Z)$ be the relation "the set $Z$ is the target of an isomorphism $\Hom_\mathcal{C}(y,x) \stackrel{\sim}{\to } Z$". We then put $h_x(y)=\tau_Z(R(Z))$.
\end{itemize}
Let $R'(u)$ be the relation "$u$ is a bijection from $\Hom_\mathcal{C}(y,x)$ to $h_x(y)$" and set $\varphi(y,x)=\tau_u(R'(u))$. Then in both cases, we have a canonical isomorphism
\[\varphi(y,x):\Hom_\mathcal{C}(y,x) \stackrel{\sim}{\to } h_y(x).\]
Now let $u:y\to y'$ be a morphism of $\mathcal{C}$. Then by composition, $u$ defines a map
\[\Hom_{\mathcal{C}}(u,x):\Hom_\mathcal{C}(y',x)\to \Hom_\mathcal{C}(y,x)\]
and we define $h_x(u)$ to be the composition
\[h_x(u)=\varphi(y,x)\Hom_\mathcal{C}(x,u)\varphi(y,x)^{-1}.\]
It is immediate to verify that $h_x$ then defines a functor $\mathcal{C}^{\op}\to \mathscr{U}\text{-}\mathbf{Set}$.
\subsection{Projective limits and inductive limits}
\subsection{Exactness properties of the category of presheaves}
\subsection{The functors \texorpdfstring{$\sHom$}{Hom} and \texorpdfstring{$\sIso$}{Iso}}
Let $\mathcal{C}$ be a category and $F,G$ be objects of $\PSh(\mathcal{C})$. We define an object $\sHom(F,G)$ of $\PSh(\mathcal{C})$ in the following way:
\[\sHom(F,G)(S)=\Hom_{\PSh(\mathcal{C}_{/S})}(F_S,G_S)\cong\Hom_{\PSh(\mathcal{C}_{/S})}(F\times h_S,G\times h_S)\cong\Hom_{\PSh(\mathcal{C})}(F\times h_S,G).\]
It is easy to verify that $\sHom(F,G)$ possesses the following properties:
\begin{itemize}
    \item $\sHom(e,G)\cong G$,
    \item If $E$ is an object of $\PSh(\mathcal{C})$, then
    \begin{equation}\label{category presheaf Hom functor prop-1}
    \sHom(E,F\times G)\cong \sHom(E,F)\times \sHom(E,G).
    \end{equation}
    \item The functor $\mathrm{Hom}$ commutes with base change:
    \begin{equation}\label{category presheaf Hom functor prop-2}
    \sHom(F_S,G_S)\cong \sHom(F,G)_S.
    \end{equation}
    \item $(F,G)\mapsto\sHom(F,G)$ is a bifunctor which is contravariant on $F$ and covariant on $G$.
\end{itemize}

Now we consider an object $E$ of $\PSh(\mathcal{C})$. Let $\phi:E\times F\to G$ be a morphism, we want to associates with $\phi$ a morphism from $E$ into $\mathrm{Hom}(F,G)$. For this, consider a morphism $S'\to S$ of $\mathcal{C}$. We then have the following induced maps:
\[E(S)\times F(S')\to E(S')\times F(S')\stackrel{\phi(S')}{\longrightarrow}G(S').\]
Any element $e$ of $E(S)$ therefore defines a map $F(S')\to G(S')$, which is functorial on $S'$; that is, an element $\theta_\phi(e)$ of $\sHom(F,G)(S)$. We therefore obtain a map
\[\Hom(E\times F,G)\to \Hom(E,\sHom(F,G)),\quad \phi\mapsto\theta_\phi\]
which is functorial on $E$.

\begin{proposition}\label{category presheaf Hom functor adjoint prop}
Let $E,F,G$ be objects of $\PSh(\mathcal{C})$. Then the map $\phi\mapsto\theta_\phi$ is a bijection:
\begin{equation}\label{category presheaf Hom functor adjoint prop-1}
\Hom_{\PSh(\mathcal{C})}(E\times F,G)\stackrel{\sim}{\to} \Hom_{\PSh(\mathcal{C})}(E,\sHom(F,G)),
\end{equation}
and we obtain an isomorphism of functors
\begin{equation}\label{category presheaf Hom functor adjoint prop-2}
\sHom(E\times F,G)\stackrel{\sim}{\to} \sHom(E,\sHom(F,G)).
\end{equation}
\end{proposition}
\begin{proof}
We consider the two members of (\ref{category presheaf Hom functor adjoint prop-1}) as functors of $E$. The first assertion is then valid for $E=h_X$, which follows directly from the definition of $\sHom(F,G)$. On the other hand, since the two functors both transforms inductive limits to projective limits and any object of $\PSh(\mathcal{C})$ can be written as an inductive limits of $h_X$, where $X$ runs through $\mathcal{C}_{/E}$, we conclude that (\ref{category presheaf Hom functor adjoint prop-1}) is a bijection.\par
We can also give a direct proof of (\ref{category presheaf Hom functor adjoint prop-1}). To any element $\theta\in\Hom(E,\sHom(F,G))$, we associate an element $\phi_\theta$ of $\Hom(E\times F,G)$ as follows. For any $S\in\mathcal{C}$, we have a map
\[\theta(S):E(S)\to \sHom(F,G)(S)=\Hom(F\times S,G)\]
which is functorial on $S$. If $(e,f)\in E(S)\times F(S)$, then $f$ can be considered as a morphism $S\to F$, so $f\times\id_S$ is a morphism $S\to F\times S$. On the other hand, $\theta(S)(e)$ is a morphism $F\times S\to G$, so by composing we obtain a morphism
\[\theta(S)(e)\circ(f\times\id_S):S\to G,\]
which is an element $\phi_\theta(e,f)$ of $G(S)$. We verify immediately that the correspondence $S\mapsto\phi_\theta(S)$ is functorial on $S$, so we get a morphism $\phi_\theta:E\times F\to G$. It then remains to check that $\theta\mapsto\phi_\theta$ and $\phi\mapsto\theta_\phi$ are inverses of each other, which is straightforward from definition.\par
We now prove the isomorphism (\ref{category presheaf Hom functor adjoint prop-2}). If $S\in\mathcal{C}$, then by (\ref{category presheaf Hom functor prop-2}) and (\ref{category presheaf Hom functor adjoint prop-1}) applied to $\mathcal{C}_{/S}$, we have
\begin{align*}
\sHom(E,\sHom(F,G))(S)&\cong\Hom_S(E_S,\sHom_S(F_S,G_S))\cong\Hom_S(E_S\times_SF_S,G_S)\\
&\cong \Hom(E\times F\times S,G)\cong\sHom(E\times F,G)(S),
\end{align*}
and these isomorphisms are functorial on $S$.
\end{proof}

\begin{corollary}
We have the following isomorphisms:
\begin{align}
\Hom(E,\sHom(F,G))&\cong \Hom(F,\sHom(E,G))\label{category presheaf Hom functor adjoint prop-3},\\
\sHom(E,\sHom(F,G))&\cong \sHom(F,\sHom(E,G))\label{category presheaf Hom functor adjoint prop-4}.
\end{align}
\end{corollary}
\begin{proof}
The first isomorphism follows from (\ref{category presheaf Hom functor adjoint prop-1}) and the fact that $E\times F\cong F\times E$, and the second one follows from (\ref{category presheaf Hom functor prop-2}).
\end{proof}

In particular, if $E=e$ is the final object, then since $\sHom(e,G)\cong G$, we have
\[\Gamma(\sHom(F,G))=\Hom(e,\sHom(F,G))\cong\Hom(F,\sHom(e,G))\cong\Hom(F,G).\]
We also note that the composition of $\Hom$ induces a functorial morphsim
\[\circ:\sHom(F,G)\times\sHom(G,H)\to\sHom(F,H).\]
In other words, with the operation $\sHom$ and $\times$, the category $\PSh(\mathcal{C})$ is self-enriched.\par

If $F$ and $G$ are objects of $\PSh(\mathcal{C})$, we denote by $\Iso(F,G)$ the subset of $\Hom(F,G)$ formed by isomorphisms from $F$ to $G$. We define a subobject $\sIso(F,G)$ of $\sHom(F,G)$ by
\[\sIso(F,G)(S)=\Iso(F_S,G_S).\]
We then have the following isomorphisms
\[\Gamma(\sIso(F,G))\cong\Iso(F,G),\quad \Iso(F,G)\cong\Iso(G,F).\]
In the particular case where $F=G$, we put
\begin{alignat*}{3}
\sEnd(F)&=\sHom(F,F),&\quad\quad &&\End(F)&=\Hom(F,F)\cong\Gamma(\sEnd(F)),\\
\sAut(F)&=\sIso(F,F),&\quad\quad &&\Aut(F)&=\Iso(F,F)\cong\Gamma(\sAut(F)).
\end{alignat*}
It is clear that the formations of $\sIso$, $\sAut$, $\sEnd$ also commutes with base changes.
\begin{remark}
Note that we can construct an object isomorphic to $\Iso(F,G)$ in the following way: we have a morphism
\[\sHom(F,G)\times \sHom(G,F)\to\sEnd(F);\]
By permuting $F$ and $G$, we then deduce a morphism
\[\Hom(F,G)\times\Hom(G,F)\to\sEnd(F)\times\sEnd(G).\]
On the other hand, the identity morphism of $F$ is an element of $\End(F)$ and defines a morphism $e\to\sEnd(F)$. By composition, we then obtain a morphism
\[e\mapsto\sEnd(F)\times\sEnd(G).\]
It it then immediate to see that the fiber product of $e$ and $\sHom(F,G)\times\sHom(G,F)$ is isomorphic to $\Iso(F,G)$.
\end{remark}
The definitions above are applicable in particular if $F=h_X$ and $G=h_Y$. In the case where $\sHom(h_X,h_Y)$ is representable by an object of $\mathcal{C}$, we denote this object by $\sHom(X,Y)$. It possesses the following property: if $Z\times X$ is representable, then 
\[\Hom(Z,\sHom(X,Y))\cong \Hom(Z\times X,Y).\]
We can also define the objects $\sIso(X)$, $\sEnd(X)$ and $\sAut(X)$. The preceding argumants also applies to the categories of the form $\mathcal{C}_{/S}$, and in this case, the corresponding objects are denoted by $\sHom_S$, $\sIso_S$, etc.
\section{Abelian Category}
\subsection{Additive categories}
\subsubsection{Preaditive category}
\begin{definition}
A category $\mathcal{A}$ is called \textbf{preadditive} if each morphism set $\Mor_{\mathcal{A}}(X,Y)$ is endowed with the structure of an abelian group such that the compositions
\[\Mor_{\mathcal{A}}(X,Y)\times \Mor_{\mathcal{A}}(Y,Z)\to\Mor_{\mathcal{A}}(X,Z)\]
are bilinear. A functor $F:\mathcal{A}\to\mathcal{B}$ of preadditive categories is called an \textbf{additive functor} if and only if 
\[F:\Mor_{\mathcal{A}}(X,Y)\to\Mor_{\mathcal{B}}(F(X),F(Y))\] 
is a homomorphism of abelian groups for all $X,Y\in\Ob(\mathcal{A})$.
\end{definition}
In particular for every $X,Y$ there exists at least one morphism $X\to Y$, namely the \textbf{zero map}.
\begin{lemma}\label{preadd cat id=0}
Let $\mathcal{A}$ be a preadditive category. Let $X$ be an object of $\mathcal{A}$. The following are equivalent:
\begin{enumerate}
\item[(a)] $X$ is a initial object.
\item[(b)] $X$ is a final object.
\item[(c)] $\id_X=0$ in $\Mor_{\mathcal{A}}(X,X)$.
\end{enumerate}
Furthermore, if such an object $0$ exists, then a morphism $f:X\to Y$ factors through $0$ if and only if $f=0$.
\end{lemma}
\begin{proof}
Clearly if $X$ is a final or initial object, then $\id_X=0$ is the unique morphism $X\to X$. Now assume $\id_X=0$ holds, then \[f\in\Mor_{\mathcal{A}}(X,Y)\Rightarrow f=f\circ\id_X=0,\And g\in\Mor_{\mathcal{A}}(Y,X)\Rightarrow g=\id_X\circ g=0.\] 
Thus $X$ is final and initial.
\end{proof}
\begin{definition}
In a preadditive category $\mathcal{A}$ we call \textbf{zero object}, and we denote it $0$ any final and initial object as in Lemma~\ref{preadd cat id=0} above.
\end{definition}
\begin{proposition}\label{preadd cat prod coprod}
Let $\mathcal{A}$ be a preadditive category. Let $X,Y\in\Ob(\mathcal{A})$. Then the product $X\times Y$ exists if and only if the coproduct $X\amalg Y$ exists. In this case $X\times Y\cong X\amalg Y$.
\end{proposition}
\begin{proof}
Suppose that $X\times Y$ exists with projections $\pi_1:X\times Y\to X$ and $\pi_2:X\times Y\to Y$. Denote $i_1:X\to X\times Y$ the morphism corresponding to $(0,1)$:
\[\begin{tikzcd}
&X\ar[ldd,bend right=20pt,swap,"1"]\ar[d,dashed,"i_1"]\ar[rdd,bend left=20pt,"0"]&\\
&X\times Y\ar[ld,swap,"\pi_1"]\ar[rd,"\pi_2"]&\\
X&&Y
\end{tikzcd}\]
Similarly, denote $i_2:Y\to X\times Y$ the morphism corresponding to $(0,1)$. Thus we have the commutative diagram
\[\begin{tikzcd}
X\ar[rr,"1"]\ar[rd,"i_1"]&&X\\
&X\times Y\ar[ru,"\pi_1"]\ar[rd,"\pi_2"]&\\
Y\ar[rr,"1"]\ar[ru,"i_2"]&&Y
\end{tikzcd}\]
where the diagonal compositions are zero. It follows that $i_1\circ \pi_1+i_2\circ\pi_2$ is the identity since it is a morphism which upon composing with $\pi_1$ gives $\pi_1$ and upon composing with $\pi_2$ gives $\pi_2$. Suppose given morphisms $f:X\to Z$ and $g:Y\to Z$. Then we can form the map $f\circ\pi_1+g\circ\pi_2:X\times Y\to Z$. In this way we get a bijection $\Mor_{\mathcal{A}}(X\times Y,Z)=\Mor_{\mathcal{A}}(X,Z)\times\Mor_{\mathcal{A}}(Y,Z)$ which show that $X\times Y\cong X\amalg Y$. The coproduet case can be done similarly.
\end{proof}

\begin{definition}
Given a pair of objects $X,Y$ in a preadditive category $\mathcal{A}$ we call \textbf{direct sum}, and we denote it $X\oplus Y$ the product $X\times Y$ endowed with the morphisms $\pi_1,\pi_2,i_1,i_2$ as in Proposition~\ref{preadd cat prod coprod} above.
\end{definition}
\begin{proposition}
Let $\mathcal{A},\mathcal{B}$ be preadditive categories. Let $F:\mathcal{A}\to\mathcal{B}$ be an additive functor. Then $F$ transforms direct sums to direct sums and zero to zero.
\end{proposition}
\begin{proof}
Suppose $F$ is additive. A direct sum $Z$ of $X$ and $Y$ is characterized by having morphisms 
\[i_1:X\to Z,\ i_2:Y\to Z,\ \pi_1:Z\to X,\ \pi_2:Z\to Y\]
such that
\[\pi_1\circ i_1=\id_X,\pi_2\circ i_2=\id_Y,\pi_2\circ i_1=0,\pi_1\circ i_2=0\And i_1\circ\pi_1+i_2\circ\pi_2=\id_Z.\]
Clearly $F(X)$, $F(Y)$, $F(Z)$ and the morphisms $F(i_1),F(i_2),F(\pi_1),F(\pi_1)$ satisfy exactly the same relations (by additivity) and we see that $F(Z)$ is a direct sum of $F(X)$ and $F(Y)$.
\end{proof}
\subsubsection{Additive category}
\begin{definition}
A category $\mathcal{A}$ is called \textbf{additive} if it is preadditive and finite
products exist, in other words it has a zero object and direct sums.
\end{definition}
Namely the empty product is a finite product and if it exists, then it is a final object.
\begin{definition}
Let $\varphi:A\to B$ be a morphism in an additive category $\mathcal{A}$. A morphism $\iota:K\to A$ is a \textbf{kernel} of $\varphi$ if $\varphi\circ\iota=0$ and for all morphisms $\zeta:Z\to A$ such that $\varphi\circ\zeta=0$ there exists a unique $\widetilde{\zeta}:Z\to K$ making the diagram
\[\begin{tikzcd}
Z\ar[rd,dashed,swap,"\exists !\widetilde{\zeta}"]\ar[r,swap,"\zeta"]\ar[rr,bend left,"0"]&A\ar[r,swap,"\varphi"]&B\\
&K\ar[u,"\iota"]&
\end{tikzcd}\]
commute.\par
A morphism $\psi:B\to C$ is a \textbf{cokernel} of $\varphi$ if $\psi\circ\varphi=0$ and for all morphisms $\beta:B\to Z$ such that $\beta\circ\varphi=0$ there exists a unique $\widetilde{\beta}:C\to Z$ making the diagram
\[\begin{tikzcd}
&C\ar[rd,dashed,"\exists !\widetilde{\beta}"]&\\
A\ar[r,"\varphi"]\ar[rr,swap,bend right,"0"]&B\ar[r,"\beta"]\ar[u,"\psi"]&Z
\end{tikzcd}\]
commute.
\end{definition}
\begin{definition}
If a kernel of $\varphi:A\to B$ exists, then a \textbf{coimage} of $\varphi$ is a cokernel for the morphism $\ker\varphi\to A$. If a cokernel of $\varphi:A\to B$ exists, then the \textbf{image} of $\varphi$ is a kernel of the morphism $B\to\coker\varphi$.
\end{definition}
\begin{lemma}\label{ker is mono}
In any additive category, kernels are monomorphisms and cokernels are epimorphisms.
\end{lemma}
\begin{proof}
Let $\varphi:A\to B$ be a morphism in an additive category $\mathcal{A}$, and let $\ker\varphi:K\to A$ be its kernel. Let $\zeta:Z\to K$ be a morphism such that $\ker\varphi\circ\zeta=0$. Then the composition $\varphi\circ(\ker\varphi\circ\zeta)$ is $0$ and by definition of kernel, $\ker\varphi\circ\zeta$ factors uniquely through $K$:
\[\begin{tikzcd}
Z\ar[r,bend left=20,"\zeta"]\ar[r,dashed,swap,bend right=20,"\exists !"]&K\ar[r,"\ker\varphi"]&A\ar[r,"\varphi"]&B
\end{tikzcd}\]
since $\ker\varphi\circ\zeta=0=\ker\varphi\circ 0$, the uniqueness of the decomposition gives $\zeta=0$.\par
The proof that cokernels are epimorphisms is analogous.
\end{proof}
Now we relate the direct sum to kernels as follows.
\begin{proposition}
Let $\mathcal{A}$ be a preadditive category. Let $X\oplus Y$ with morphisms as in Propostion~\ref{preadd cat prod coprod} be a direct sum in $\mathcal{A}$. Then $i_1:X\to X\oplus Y$ is a kernel of $\pi_2:X\oplus Y\to Y$. Dually, $\pi_1$ is a cokernel for $i_2$.
\end{proposition}
\begin{proof}
Let $f:Z\to X\oplus Y$ be a morphism such that $\pi_2\circ f=0$. We have to show that there exists a unique morphism $g:Z\to X$ such that $f=i_1\circ g$:
\[\begin{tikzcd}
X\ar[rr,"1"]\ar[rd,"i_1"]&&X\\
Z\ar[r,"f"]&X\times Y\ar[ru,"\pi_1"]\ar[rd,"\pi_2"]&\\
Y\ar[rr,"1"]\ar[ru,"i_2"]&&Y
\end{tikzcd}\] Since $i_1\circ\pi_1+i_2\circ\pi_2$ is the identity on $X\oplus Y$ we see that
\[f=(i_1\circ\pi_1+i_2\circ\pi_2)\circ f=i_1\circ\pi_1\circ f\]
and hence $g=\pi_1\circ f$ works. Uniquess holds because $\pi_1\circ i_1$ is the identity on $X$. The proof of the second statement is dual.
\end{proof}
\begin{theorem}\label{preadditive coim im}
Let $\varphi:A\to B$ be a morphism in a preadditive category such that
the kernel, cokernel, image and coimage all exist. Then $\varphi$ can be factored uniquely
\[\begin{tikzcd}
A\ar[rrr,bend left=20pt,"\varphi"]\ar[r]&\coim\varphi\ar[r]&\im\varphi\ar[r]&B
\end{tikzcd}\]
\end{theorem}
\begin{proof}
There is a canonical morphism $\coim\varphi\to B$ because $\ker\varphi\to A\to B$ is zero,
\[\begin{tikzcd}
&&\im\varphi&\\
\ker\varphi\ar[r]&A\ar[r,"\varphi"]\ar[d]&B\ar[r]&\coker\varphi\\
&\coim\varphi\ar[ru,dashed]
\end{tikzcd}\]
The composition $\coim\varphi\to B\to\coker\varphi$ is zero, because it is the unique morphism which gives rise to the morphism $A\to B\to\coker\varphi$ which is zero. Hence $\coim\varphi\to B$ factors uniquely through $\im\varphi\to B$, which gives us the desired map.
\end{proof}
\subsection{Abelian categories}
An abelian category is a category satisfying just enough axioms so the snake lemma holds. An axiom is that the canonical map $\coim\varphi\to\im\varphi$ of Theorem~\ref{preadditive coim im} is always an isomorphism.
\begin{definition}\label{ab cat def}
A category $\mathcal{A}$ is \textbf{abelian} if it is additive, if all kernels and cokernels exist, and if the natural map $\coim\varphi\to\im\varphi$ is an isomorphism for all morphisms $\varphi$ of $\mathcal{A}$.
\end{definition}
\begin{definition}
Let $\varphi:A\to B$ be a morphism in an abelian category.
\begin{enumerate}
\item[(a)] We say $\varphi$ is \textbf{injective} if $\ker\varphi=0$.
\item[(b)] We say $\varphi$ is \textbf{surjective} if $\coker\varphi=0$.
\end{enumerate}
\end{definition}
\begin{proposition}\label{mono epi iff ker coker}
Let $\varphi:A\to B$ be a morphism in an abelian category. Then
\begin{enumerate}
\item[(a)] $\varphi$ is \textbf{injective} if and only if $f$ is a monomorphism.
\item[(b)] $\varphi$ is \textbf{surjective} if and only if $f$ is a epimorphism.
\end{enumerate}
\end{proposition}
\begin{proof}
The condition for monomorphism can be interpreted as: If $\psi:Z\to A$ is any morphism such that $\varphi\circ\psi=0$, then $\psi$ factors through $0\to A$. So $\varphi$ is a monomorphism if and only if $0\to A$ is its kernel. The same holds for epimorphism.
\end{proof}
\begin{proposition}
Let $\mathcal{A}$ be an abelian category. All finite limits and finite colimits
exist in $\mathcal{A}$.
\end{proposition}
\begin{proof}
To show that finite limits exist it suffices to show that finite products and
equalizers exist. Finite products exist by definition and the equalizer of $f,g:X\to Y$ is the kernel of $a-b$. The argument for finite colimits is similar but dual to this.
\end{proof}
\begin{example}
Let $\mathcal{A}$ be an abelian category. Pushouts and fibre products in $\mathcal{A}$ have the following simple descriptions:
\begin{enumerate}
\item[(a)] If $f:X\to Y,g:Z\to Y$ are morphisms in $\mathcal{A}$, then we have the fibre product: $X\times_YZ=\ker((f,-g):X\oplus Z\to Y)$.
\item[(b)] If $f:Y\to X,g:Y\to Z$ are morphisms in $\mathcal{A}$, then we have the pushout: $X\amalg_YZ=\coker((f,-g):Y\oplus X\to Z)$.
\end{enumerate}
\end{example}
\begin{lemma}\label{ker is ker of coker}
In an abelian category $\mathcal{A}$, every kernel is the kernel of its cokernel; every cokernel is the cokernel of its kernel.
\end{lemma}
\begin{proof}
Let $\varphi:K\to A$ be the kernel of some morphism $A\to B$; since $\mathcal{A}$ is abelian, $\varphi$ has a cokernel $\psi:A\to C$. The composition $K\to A\to B$ is $0$, so $A\to B$ factors
through $\psi$ by definition of cokernel:
\[\begin{tikzcd}
C\ar[rd,dashed]&\\
A\ar[r]\ar[u,"\psi"]&B\\
K\ar[u,"\varphi"]&
\end{tikzcd}\]
Now let $Z\to A$ be a morphism such that the composition $Z\to A\to C$ is the zero-morphism; then so is the composition $Z\to A\to B$. Therefore $Z\to A$ factors through a unique morphism $Z\to K$,
\[\begin{tikzcd}
&C\ar[rd,dashed]&\\
Z\ar[r]\ar[rd,dashed]&A\ar[r]\ar[u,"\psi"]&B\\
&K\ar[u,"\varphi"]&
\end{tikzcd}\]
since $\varphi$ is the kernel of $A\to B$. But this shows that $\varphi:A\to B$ satisfies the property defining the kernel of its cokernel $A\to C$, as stated.
\end{proof}
\begin{proposition}\label{mono epi iff inverse}
Let $\varphi:A\to B$ be a morphism in an abelian category $\mathcal{A}$.
\begin{enumerate}
\item[(a)] $\varphi$ is a monomorphism if and only if $\varphi$ has a left-invere.
\item[(b)] $\varphi$ is a epimorphism if and only if $\varphi$ has a right-invere.
\end{enumerate}
Thus $\varphi$ is an isomorphism if and only if it is a monomorphism and a epimorphism.
\end{proposition}
\begin{proof}
If $\varphi$ has a left-inverse, then clearly it is monic. Conversely, if $\varphi$ is a monomorphis, then the kernel of $\varphi$ is $0\to A$. Further, $\varphi$ is the cokernel of $0\to A$. Now consider the identity $A\to A$:
\[\begin{tikzcd}
0\ar[r]&A\ar[r,"\varphi"]\ar[d,swap,"\id_A"]&B\\
&A
\end{tikzcd}\]
Since $0\to A\to A$ is the zero morphism and $\varphi$ is the cokernel of $0\to A$, we obtain a unique morphism $\psi:B\to A$ making the diagram commute:
\[\begin{tikzcd}
0\ar[r]&A\ar[r,"\varphi"]\ar[d,swap,"\id_A"]&B\ar[ld,"\psi"]\\
&A
\end{tikzcd}\]
As $\psi\circ\varphi=\id_A$, this shows that $\varphi$ has a right-inverse. The part (b) can be done similarly.
\end{proof}
\subsection{Exact sequence in Abelian category}
\begin{definition}
Let $\mathcal{A}$ be an additive category. We say a sequence of morphisms
\[\begin{tikzcd}
\cdots\ar[r]&A\ar[r,"\varphi"]&B\ar[r,"\psi"]&C\ar[r]&\cdots
\end{tikzcd}\]
in $\mathcal{A}$ is a \textbf{complex} if the composition of any two arrows is zero. If $\mathcal{A}$ is abelian then we say a sequence as above is \textbf{exact at $\bm{B}$} if $\im\psi=\ker\varphi$. We say it is exact if it is exact at every object. A \textbf{short exact sequence} is an exact complex of the form
\[\begin{tikzcd}
0\ar[r]&A\ar[r,"\varphi"]&B\ar[r,"\psi"]&C\ar[r]&0
\end{tikzcd}\]
\end{definition}
\begin{proposition}
Let $\mathcal{A}$ be an abelian category. Let $0\to M_1\to M_2\to M_3\to0$ be a complex of $\mathcal{A}$.
\begin{enumerate}
\item[(a)] $M_1\to M_2\to M_3\to 0$ is exact if and only if
\[\begin{tikzcd}
0\ar[r]&\Hom_{\mathcal{A}}(M_3,N)\ar[r]&\Hom_{\mathcal{A}}(M_2,N)\ar[r]&\Hom_{\mathcal{A}}(M_1,N)
\end{tikzcd}\]
is an exact sequence of abelian groups for all objects $N$ of $\mathcal{A}$.
\item[(b)] $0\to M_1\to M_2\to M_3$ is exact if and only if
\[\begin{tikzcd}
\Hom_{\mathcal{A}}(N,M_1)\ar[r]&\Hom_{\mathcal{A}}(N,M_2)\ar[r]&\Hom_{\mathcal{A}}(N,M_3)\ar[r]&0
\end{tikzcd}\]
is an exact sequence of abelian groups for all objects $N$ of $\mathcal{A}$.
\end{enumerate}
\end{proposition}
\begin{example}\label{fibered diagram}
For a slightly more interesting example, consider a diagram
\[\begin{tikzcd}
D\ar[d,swap,"\psi'"]\ar[r,"\varphi'"]&B\ar[d,"\psi"]\\
A\ar[r,swap,"\varphi"]&C
\end{tikzcd}\]
and the associated sequence
\[\begin{tikzcd}
D\ar[r,"{(\psi',\varphi')}"]&A\oplus B\ar[r,"{(\varphi,-\psi)}"]&C
\end{tikzcd}\]
obtained by letting $A\oplus B$ play both roles of product and coproduct. Then
\begin{enumerate}
\item the diagram is commutative if and only if this sequence is a complex;
\item the sequence obtained by adding a $0$ to the left,
\[\begin{tikzcd}
0\ar[r]&D\ar[r]&A\oplus B\ar[r]&C
\end{tikzcd}\]
is exact if and only if $D$ may be identified with the fibered product $A\times_{C}B$. If this holds, we say the diagram is \textbf{cartesian}.
\item likewise, the sequence
\[\begin{tikzcd}
D\ar[r]&A\oplus B\ar[r]&C\ar[r]&0
\end{tikzcd}\]
is exact if and only if $C$ may be identified with the fibered coproduct $A\amalg_DB$. If this holds, we say the diagram is \textbf{cocartesian}.
\end{enumerate}
\end{example}
\begin{lemma}\label{pull bak lem}
Let $\mathcal{A}$ be an abelian category. Let
\[\begin{tikzcd}
D\ar[d,swap,"\psi'"]\ar[r,"\varphi'"]&B\ar[d,"\psi"]\\
A\ar[r,swap,"\varphi"]&C
\end{tikzcd}\]
be a commutative diagram.
\begin{enumerate}
\item[(a)] If the diagram is cartesian, then the morphism $\ker\varphi'\to\ker\varphi$ induced by $\psi'$ is an isomorphism.
\item[(b)] If the diagram is cocartesian, then the morphism $\coker\varphi'\to\coker\varphi$ induced by $\psi$ is an isomorphism.
\end{enumerate}
\end{lemma}
\begin{proof}
Suppose the diagram is cartesian. Let $\epsilon:\ker\varphi'\to\ker\varphi$ be induced by $\psi'$. Let $i:\ker\varphi\to A$ and $j:\ker\varphi'\to D$ be the canonical injections. Consider the map $\alpha:\ker\varphi\to D$ determined by the morphisms $(i,0)$:
\[\psi'\circ\alpha=i,\quad \varphi'\circ\alpha=0.\]
Then there is an induced morphism $\gamma:\ker\varphi\to\ker\varphi'$:
\[\begin{tikzcd}
\ker\varphi'\ar[r,"j"]\ar[d,"\epsilon"]&D\ar[d,swap,"\psi'"]\ar[r,"\varphi'"]&B\ar[d,"\psi"]\\
\ker\varphi\ar[r,"i"]\ar[ru,"\alpha"]\ar[u,bend left=30pt,"\gamma"]&A\ar[r,swap,"\varphi"]&C
\end{tikzcd}\]
It follows that
\[\psi'\circ j\circ\gamma\circ\epsilon=\psi'\circ\alpha\circ\epsilon=i\circ\epsilon=\psi'\circ j\And\varphi'\circ j\circ\gamma\circ\epsilon=\varphi'\circ\alpha\circ\epsilon=0=\psi'\circ j.\]
By the universal property of pull back, we claim $j\circ\gamma\circ\epsilon=j$. Since $j$ is a monomorphism, this means $\gamma\circ\eps=\id_{\ker\varphi'}$.\par
Furthermore, we have 
\[i\circ\epsilon\circ\gamma=\psi'\circ j\circ\gamma=\psi'\circ\alpha=i.\]
Since $i$ is a monomorphism this implies $\epsilon\circ\gamma=\id_{\ker\varphi}$. This proves (a). Now, (b) follows by duality.
\end{proof}
\begin{lemma}
Let $\mathcal{A}$ be an abelian category. Let
\[\begin{tikzcd}
D\ar[d,swap,"\psi'"]\ar[r,"\varphi'"]&B\ar[d,"\psi"]\\
A\ar[r,swap,"\varphi"]&C
\end{tikzcd}\]
be a commutative diagram.
\begin{enumerate}
\item[(a)] If the diagram is cartesian and $\varphi$ is an epimorphism, then the diagram is cocartesian and $\varphi'$ is an epimorphism.
\item[(b)] If the diagram is cocartesian and $\varphi'$ is an monomorphism, then the diagram is cartesian and $\varphi$ is an epimorphism.
\end{enumerate}
\end{lemma}
\begin{proof}
Suppose the diagram is cartesian and $\varphi$ is an epimorphism. Let $\alpha=(\psi',\varphi'):D\to A\oplus B$ and let $\beta=(\varphi,-\psi):A\oplus B\to C$. As $\varphi$ is an epimorphism, $\alpha$ is an epimorphism, too. Therefore by Example~\ref{fibered diagram} the diagram is cocartesian. Finally, $\varphi'$ is an epimorphism by Lemma~\ref{pull bak lem}. This proves $(1)$, and $(2)$ follows by duality.
\end{proof}
\begin{corollary}
Let $\mathcal{A}$ be an abelian category.
\begin{enumerate}
\item[(a)] If $X\to Y$ is surjective, then for every $Z\to Y$ the projection $X\times_YZ\to Z$ is surjective.
\item[(b)] If $X\to Y$ is injective, then for every $X\to Z$ the morphism $Z\to Z\amalg_XY$ is injective.
\end{enumerate}
\end{corollary}
\begin{lemma}\label{exact iff}
Let \begin{tikzcd}X'\ar[r,"f"]&X\ar[r,"g"]&X''\end{tikzcd} be a complex. Then the conditions below are equivalent:
\begin{enumerate}
\item[$(\rmnum{1})$] the complex \begin{tikzcd}X'\ar[r,"f"]&X\ar[r,"g"]&X''\end{tikzcd} is exact.
\item[$(\rmnum{2})$] the induced morphism $X'\to\ker g$ is an epimorphism.
\item[$(\rmnum{3})$] for any morphism $h:S\to X$ such that $g\circ h=0$, there exist an epimorphism $f':S'\twoheadrightarrow S$ and a commutative diagram
\[\begin{tikzcd}
S'\ar[d]\ar[r,twoheadrightarrow,"f'"]&S\ar[d,"h"]\ar[rd,"0"]&\\
X'\ar[r,"f"]&X\ar[r,"g"]&X''
\end{tikzcd}\]
\end{enumerate}
\end{lemma}
\begin{proof}
$(\rmnum{1})\iff(\rmnum{2})$: the exactness is saying $\ker g=\im f$. If $X'\to\ker g$ is epic, by Exercise~\ref{epi mono decop}, $\ker g=\im f$ as needed. Conversely, if $\ker g=\in f$, by Lemma~\ref{im decomp}, $X'\to\ker g$ is epic.\par
$(\rmnum{1})\Rightarrow(\rmnum{3})$: It is enough to choose $X'\times_{\ker g}S$ as $S'$. Since $X'\to\ker g$ is an epimorphism, $S'\to S$ is an epimorphism by Lemma~\ref{pull bak lem}.\par
$(\rmnum{3})\Rightarrow(\rmnum{2})$: Choose $S=\ker g$. Then the diagram becomes
\[\begin{tikzcd}
S'\ar[d]\ar[r,twoheadrightarrow,"f'"]&\ker g\ar[d]\ar[rd,"0"]&\\
X'\ar[ru,dashed]\ar[r,"f"]&X\ar[r,"g"]&X''
\end{tikzcd}\]
since $g\circ f=0$, by the universal property of $\ker g$, there is a unique morphism $X'\to\ker g$. It follows that the composition $S'\to X'\to\ker g$ is an epimorphism. Hence $X'\to\ker g$ is an epimorphism.
\end{proof}
\subsection{Exercise}
\begin{exercise}\label{epi mono decop}
Let $\varphi:A\to B$ be a morphism in an abelian category, and assume $\varphi$ decomposes as an epimorphism $\pi$ followed by a monomorphism $i$:
\[\begin{tikzcd}
A\ar[rr,swap,bend right,"\varphi"]\ar[r,twoheadrightarrow,"\pi"]&C\ar[r,rightarrowtail,"i"]&B
\end{tikzcd}\]
Prove that necessarily $\pi=\coim\varphi$ and $i=\im\varphi$.
\end{exercise}
\begin{proof}
From the universal property of image and coimage, we have the following commutative diagram:
\[\begin{tikzcd}
&\coim\varphi\ar[rd,bend left,rightarrowtail]&\\
A\ar[ru,twoheadrightarrow,bend left]\ar[rd,twoheadrightarrow,bend right]\ar[r,twoheadrightarrow,"\pi"]&C\ar[u,dashed,"\exists!\nu"]\ar[r,rightarrowtail,"i"]&B\\
&\im\varphi\ar[u,dashed,"\exists!\mu"]\ar[ru,bend right,rightarrowtail]&
\end{tikzcd}\]
By simple observation, we find $\mu$ and $\nu$ are both monomorphic and epimorphic, hence are isomorphisms.
\end{proof}

\section{Triangulated categories}
\subsection{Localization of categories}
Consider a category $\mathcal{C}$ and a family $\mathcal{S}$ of morphisms in $\mathcal{C}$. The aim of localization is to find a new category $\mathcal{C}_\mathcal{S}$ and a functor $Q:\mathcal{C}\to\mathcal{C}_\mathcal{S}$ which sends the morphisms belonging to $\mathcal{C}$ to isomorphisms in $\mathcal{C}_\mathcal{S}$, $(\mathcal{C}_\mathcal{S},Q)$ being "universal" for such a property. We also discuss with some details the localization of functors. When considering a functor $F$ from $\mathcal{C}$ to a category $\mathcal{A}$ which does not necessarily send the morphisms in $\mathcal{S}$ to isomorphisms in $\mathcal{A}$, it is possible to define the right (resp. left) localization of $F$, a functor $R_\mathcal{S}F$ (resp. $L_\mathcal{S}F$) from $\mathcal{C}_\mathcal{S}$ to $\mathcal{A}$. Such a right localization always exists if $\mathcal{A}$ admits filtrant inductive limits.\par
Let $\mathcal{C}$ be a category and $\mathcal{S}$ be a family of morphisms in $\mathcal{C}$. A \textbf{localization} of $\mathcal{C}$ by $\mathcal{S}$ is the data of a category $\mathcal{C}_\mathcal{S}$ and a functor $Q:\mathcal{C}\to\mathcal{C}_\mathcal{S}$ satisfying:
\begin{enumerate}[leftmargin=40pt]
    \item[(L1)] for all $s\in\mathcal{S}$, $Q(s)$ is an isomorphism;
    \item[(L2)] for any category $\mathcal{A}$ and any functor $F:\mathcal{C}\to\mathcal{A}$ such that $F(s)$ is an isomorphism for all $s\in\mathcal{S}$, there exist a functor $F_\mathcal{S}:\mathcal{C}_\mathcal{S}\to\mathcal{A}$ and an isomorphism $F\cong F_\mathcal{S}\circ Q$ visualized by the diagram
    \[\begin{tikzcd}
    \mathcal{C}\ar[r,"F"]\ar[d,swap,"Q"]&\mathcal{A}\\ 
    \mathcal{C}_\mathcal{S}\ar[ru,dashed,swap,"F_\mathcal{S}"]
    \end{tikzcd}\]
    \item[(L3)] if $G$ and $G'$ are two functors from $\mathcal{C}_\mathcal{S}$ to $\mathcal{A}$, then the natural map
    \begin{equation}\label{category localization def-1}
    \Hom_{\Fun(\mathcal{C}_\mathcal{S},\mathcal{A})}(G,G')\to\Hom_{\Fun(\mathcal{C},\mathcal{A})}(G\circ Q,G'\circ Q)
    \end{equation}
    is bijective.
\end{enumerate}
Note that condition (L3) means that the functor $Q^\star:\Fun(\mathcal{C}_\mathcal{S},A)\to\Fun(\mathcal{C},A)$ induced by composition is fully faithful. In particular, this implies that $F_\mathcal{S}$ in (L2) is unique up to isomorphism.

\begin{proposition}\label{category localization opposite char}
Let $\mathcal{C}$ be a category and $\mathcal{S}$ be a family of morphisms in $\mathcal{C}$.
\begin{enumerate}
    \item[(a)] If $\mathcal{C}_\mathcal{S}$ exists, it is unique up to equivalence of categories.
    \item[(b)] If $\mathcal{C}_\mathcal{S}$ exists, then, denoting by $\mathcal{S}^{\op}$ the image of $\mathcal{S}$ in $\mathcal{C}^{\op}$, $(\mathcal{C}^{\op})_{\mathcal{S}^{\op}}$ exists and there is an equivalence of categories $(\mathcal{C}_\mathcal{S})^{\op}\cong(\mathcal{C}^{\op})_{\mathcal{S}^\op}$.
\end{enumerate}
\end{proposition}
\begin{proof}
If $(\mathcal{C}_\mathcal{S},Q)$ and $(\mathcal{C}_\mathcal{S}',Q')$ are two localizations of $\mathcal{C}$ by $\mathcal{S}$, then since $Q'(s)$ is an isomorphism for any $s\in\mathcal{S}$, we obtain a funcor $G:\mathcal{C}_\mathcal{S}\to\mathcal{C}'_\mathcal{S}$ such that $GQ\cong Q'$; similarly, there is a functor $G':\mathcal{C}_\mathcal{S}'\to\mathcal{C}_\mathcal{S}$ such that $G'Q'\cong Q$. Since $G'GQ\cong Q$, we conclude from (L3) that $G'G\cong\id_{\mathcal{C}_\mathcal{S}}$, and similarly $GG'\cong\id_{\mathcal{C}'_\mathcal{S}}$. The second assertion follows immediately from (a) and a easy verification by reversing the arrows.
\end{proof}

The existence of the localization $\mathcal{C}_\mathcal{S}$ is generally true, since we can construct $\mathcal{C}_\mathcal{S}$ by adding virtue inverses to $\mathcal{C}$ (like the construction of free groups). More precisely, we have $\Ob(\mathcal{C}_\mathcal{S})=\Ob(\mathcal{C})$, and the morphisms in $\mathcal{C}_\mathcal{S}$ are of the form
\[\cdots bt^{-1}as^{-1}\cdots=\left(\begin{tikzcd}[row sep=5mm, column sep=5mm]
\cdots\ar[rd]&&Y\ar[rd,"a"]\ar[ld,swap,"s"]&&W\ar[ld,swap,"t"]\ar[rd,"b"]&&\cdots\ar[ld]\\
&X&&Z&&U&
\end{tikzcd}\right)\]
where $a,b\in\Mor(\mathcal{C})$ and $s,t\in\mathcal{S}$, with the composition map defined in the obvious way subject to the relations
\[s^{-1}t^{-1}=(ts)^{-1},\quad ss^{-1}=\id,\quad s^{-1}s=\id.\]
The problem is that the equivalence relation in $\mathcal{C}_\mathcal{S}$ is now hard to manipulate: we can not tell which morphisms $f,g$ in $\mathcal{C}$ satisfy $Q(f)=Q(g)$. Due to this failure, we now impose some additional conditions on the system $\mathcal{S}$, so that the resulting localization $\mathcal{C}_\mathcal{S}$ is way more easiler to describe.

\begin{definition}
The family $\mathcal{S}$ is called a \textbf{right multiplicative system} if it satisfies the following axioms:
\begin{enumerate}[leftmargin=40pt]
    \item[(S1)] For any object $X$ of $\mathcal{C}$, $\id_X$ belongs to $\mathcal{S}$.
    \item[(S2)] If two morphisms $f:X\to Y$ and $g:Y\to Z$ belong to $\mathcal{S}$, then $g\circ f$ belongs to $\mathcal{S}$.
    \item[(S3)] Given two morphisms $f:X\to Y$ and $s:X\to X'$ with $s\in\mathcal{S}$, there exist $t:Y\to Y'$ and $g:X'\to Y'$ with $t\in\mathcal{S}$ such that the following diagram commutes:
    \[\begin{tikzcd}
    X\ar[r,"f"]\ar[d,swap,"s",tail]&Y\ar[d,dashed,"t",tail]\\
    X'\ar[r,dashed,"g"]&Y'
    \end{tikzcd}\]
    \item[(S4)] Let $f,g:X\rightrightarrows Y$ be two morphisms in $\mathcal{C}$. If there exists a morphism $s:Z\to X$ in $\mathcal{S}$ such that $fs=gs$, then there exists $t:Y\to W$ in $\mathcal{S}$ such that $tf=tg$. This is visualized by the diagram:
    \[\begin{tikzcd}
    Z\ar[r,"s",tail]&X\ar[r,shift left=3pt,"f"]\ar[r,shift right=3pt,swap,"g"]&Y\ar[r,dashed,"t",tail]&W
    \end{tikzcd}\]
\end{enumerate}
\end{definition}

\begin{remark}
Axioms (S1)-(S2) asserts that there exists a half-full subcategory $\widetilde{\mathcal{S}}$ of $\mathcal{C}$ with $\Ob(\widetilde{\mathcal{S}})=\Ob(\mathcal{C})$ and $\Mor(\widetilde{\mathcal{S}})=\mathcal{S}$. With these axioms, the notion of a right multiplicative system is stable by equivalence of categories.
\end{remark}
\begin{remark}
The notion of a \textbf{left multiplicative system} is defined similarly by reversing the arrows. This means that the condition (S3) and (S4) are replaced by the conditions (S3') and (S4') below:
\begin{enumerate}[leftmargin=40pt]
    \item[(S3')] Given two morphisms $f:X\to Y$ and $t:Y'\to Y$ with $t\in\mathcal{S}$, there exist $s:X'\to X$ and $g:X'\to Y'$ with $s\in\mathcal{S}$ such that the following diagram commutes:
    \[\begin{tikzcd}
    X'\ar[r,dashed,"g"]\ar[d,swap,dashed,"s",tail]&Y'\ar[d,"t",tail]\\
    X\ar[r,"f"]&Y
    \end{tikzcd}\]
    \item[(S4')] Let $f,g:X\rightrightarrows Y$ be two morphisms in $\mathcal{C}$. If there exists a morphism $t:Y\to W$ in $\mathcal{S}$ such that $tf=tg$, then there exists $s:Z\to X$ in $\mathcal{S}$ such that $fs=gs$. This is visualized by the diagram:
    \[\begin{tikzcd}
    Z\ar[r,dashed,"s",tail]&X\ar[r,shift left=3pt,"f"]\ar[r,shift right=3pt,swap,"g"]&Y\ar[r,"t",tail]&W
    \end{tikzcd}\]
\end{enumerate}
A collection $\mathcal{S}$ is simply called a \textbf{multiplicative system} if it is both a left multiplicative system and a right multiplicative system.
\end{remark}

Let $\mathcal{S}$ be a system of morphisms of $\mathcal{C}$ satisfying axioms (S1)-(S2) and $X\in\Ob(\mathcal{C})$. We define $\mathcal{S}_{/X}$ (resp. $\mathcal{S}_{X/}$) to be the full subcategory of $\mathcal{C}_{/X}$ (resp. $\mathcal{C}_{X/}$) with objects (morphisms in $\mathcal{C}$) belonging to $\mathcal{S}$.

\begin{proposition}\label{category localization comma is filtered}
If $\mathcal{S}$ is a left (resp. right) multiplicative system. Then the category $\mathcal{S}_{/X}$ (resp. $\mathcal{S}_{X/}$) is cofiltrant (resp. filtrant).
\end{proposition}
\begin{proof}
Note that $(\mathcal{S}^{\op})_{X/}=(\mathcal{S}_{/X})^{\op}$, so we only need to consider right multiplicative systems. For any objects $s:X\to Z$ and $s':X\to Z'$ of $\mathcal{S}_{X/}$, by (S3) we have a commutative diagram
\[\begin{tikzcd}
X\ar[r,"s'"]\ar[d,swap,"s"]&Z'\ar[d,"t'"]\\
Z\ar[r,"t"]&Y
\end{tikzcd}\]
with $t\in S$. Then $ts\in\mathcal{S}$ by (S2) and the composition $ts:X\to Y$ belongs to $\mathcal{S}_{X/}$. Now consider two morphisms $f,g:Z\rightrightarrows Z'$ such that $fs=gs=s'$. Then by (S4) there exists $t:Z'\to W$ such that $tf=tg$, so $t\circ s':X\to W$ belongs to $\mathcal{S}_{X/}$ and the compositions
\[\begin{tikzcd}
(Z,s)\ar[r,shift left=3pt,"f"]\ar[r,shift right=3pt,swap,"g"]&(Z',s')\ar[r,"t"]&(W,t\circ s')
\end{tikzcd}\]
coincides; this completes the proof.
\end{proof}

Let $X,Y$ be objects of $\mathcal{C}$. For left (resp. right) multiplicative system $\mathcal{S}$, we define a collection $M_{X,Y}^l$ (resp. $M_{X,Y}^r$), which will be considered to be the "morphisms" from $X$ to $Y$ in our localization category.
\begin{itemize}
    \item If $\mathcal{S}$ is a left multiplicative system, we denote by $M_{X,Y}^l$ the collection of diagrams of the form
    \[\begin{tikzcd}[row sep=5mm, column sep=5mm]
    &Z\ar[ld,swap,"s"]\ar[rd,"a"]\\
    X&&Y
    \end{tikzcd}\]
    where $s\in\mathcal{S}$ (such a diagram will be denoted by $(Z;s,a)$).
    \item If $\mathcal{S}$ is a right multiplicative system, we denote by $M_{X,Y}^r$ the collection of diagrams of the form
    \[\begin{tikzcd}[row sep=5mm, column sep=5mm]
    X\ar[rd,swap,"a"]&&Y\ar[ld,"s"]\\
    &Z
    \end{tikzcd}\]
    where $s\in\mathcal{S}$ (such a diagram will be denoted by $(Z;a,s)$).
\end{itemize} 

We define an equivalence relation on $M_{X,Y}^l$ (resp. $M_{X,Y}^r$) as follows: $(Z;s,a)\sim(Z';s',a')$ (resp. $(Z;a,s)\sim(Z';a',s')$) if there eixsts a commutative diagram of the form
\[\begin{tikzcd}
&Z\ar[ld,swap,"s",tail]\ar[rd,"a"]\\
X&W\ar[l,tail]\ar[r]\ar[u]\ar[d]&Y\\
&Z'\ar[lu,"s'",tail]\ar[ru,swap,"a'"]&
\end{tikzcd}\quad\quad \text{resp.}
\begin{tikzcd}
&Z\ar[d]\\
X\ar[ru,"a"]\ar[rd,swap,"a'"]\ar[r]&W&Y\ar[ld,"s'",tail]\ar[l,tail]\ar[lu,swap,"s",tail]\\
&Z'\ar[u]&
\end{tikzcd}\]
(we use $\rightarrowtail$ to indicates a morphism in $\mathcal{S}$.) We also note that for any morphism $f:X\to Y$, axiom (S1) implies that $(X;\id_X,f)\in M_{X,Y}^l$ and $(X;f,\id_Y)\in M_{X,Y}^r$.

\begin{lemma}\label{category localization Hom set char by colim}
Let $\mathcal{S}$ be a left (resp. right) multiplicative system. For any object $X,Y$ of $\mathcal{C}$, we have a bijection
\begin{gather*}
M_{X,Y}^l/\sim \stackrel{\sim}{\to} \rlim_{(Z\to X)\in\Ob(\mathcal{S}_{/X}^{\op})}\Hom(Z,Y),\quad [Z;s,a]\mapsto [a:Z\to Y],\\
M_{X,Y}^r/\sim \stackrel{\sim}{\to} \rlim_{(Y\to Z)\in\Ob(\mathcal{S}_{Y/})}\Hom(X,Z),\quad [Z;a,s]\mapsto [a:X\to Z].
\end{gather*}
\end{lemma}
\begin{proof}
We consider the functor $\alpha:\mathcal{S}_{/X}^{\op}\to\mathbf{Set}$ given by $(Z\to X)\mapsto\Hom(Z,Y)$. Then by definition we have
\[M_{X,Y}^l=\coprod_{Z\to X}\alpha(Z\to X).\]
On the other hand, it is not hard to see that the equivalence relation $\sim$ on $M_{X,Y}^l$ is induced from the limit $\rlim\Hom(Z,Y)$, so the claim follows.
\end{proof}

For a left multiplicative system $\mathcal{S}$, any objects $X,Y$ of $\mathcal{C}$ and $(U;s,a)\in M_{X,Y}^l$ and $(V;t,b)\in M_{Y,Z}^l$, by axiom (S3) we have a commutative diagram
\begin{equation}\label{category localization composition law def}
\begin{tikzcd}[row sep=5mm, column sep=5mm]
&&W\ar[ld,swap,"r",tail]\ar[rd,"c"]&&\\
&U\ar[ld,swap,"s",tail]\ar[rd,"a"]&&V\ar[ld,swap,"t",tail]\ar[rd,"b"]&\\
X&&Y&&Z
\end{tikzcd}
\end{equation}
We now define the composition of $(U;s,a)$ and $(V;t,b)$ to be the equivalent class of $(W;sr,bc)$.

\begin{proposition}\label{category localization composition law}
The composition law defined above is associative and only depends on the equivalent class of $(U;s,a)$ and $(V;t,b)$. Also, a similar result holds if $\mathcal{S}$ is a right multiplicative system.
\end{proposition}
\begin{proof}
We first fix the diagram $(V;t,b)$. In view of the definition, it suffices to prove that in the following diagram
\begin{equation}\label{category localization composition law-1}
\begin{tikzcd}
&U\ar[ld,swap,"s",tail]\ar[rd,"a"]&W\ar[l,swap,"r",tail]\ar[rd,"c"]&&\\
X&&Y&V\ar[l,swap,"t",tail]\ar[r,"b"]&Z\\
&U'\ar[lu,"s'",tail]\ar[ru,swap,"a'"]\ar[uu,"x"]&W'\ar[l,"r'",tail]\ar[ru,swap,"c'"]&&
\end{tikzcd}
\end{equation}
we have $(W;sr,bc)\sim(W';s'r',bc')$. For this, we apply axiom (S3) twice to obtain the following solid diagram:
\begin{equation}\label{category localization composition law-2}
\begin{tikzcd}[row sep=7mm, column sep=8mm]
&U\ar[rd,"a"]&W\ar[l,swap,"r",tail]\ar[rd,"c"{anchor=south}]&&\\
&&Y&V\ar[l,swap,"t",tail]&\\
&U'\ar[ru,swap,"a'"{anchor=north},pos=0.6]\ar[uu,"x"]&W'\ar[l,"r'",tail]\ar[ru,swap,"c'"{anchor=north}]&&\\
&&&&R\ar[lluuu,bend right=35pt,"q",dashed,pos=0.4]\ar[lllu,"p",tail,bend left=15pt,swap,dashed]&Q\ar[l,"h",tail,swap,dashed]\ar[lllu,"k",dashed,bend right=15pt]&P\ar[l,dashed,swap,"w",tail]
\end{tikzcd}
\end{equation}
From the diagram (\ref{category localization composition law-1}), we then conclude that $tc'k=a'r'k$, so by (S4) there is a morphism $w:P\to Q$ in $\mathcal{S}$ such that $c'kw=cqhw$. Now, it is not hard to verify that the following diagram commutes:
\[\begin{tikzcd}[row sep=15mm, column sep=15mm]
&W\ar[ld,swap,"sr",tail]\ar[rd,"bc"]&\\
X&P\ar[r,"bc'kw"description]\ar[l,swap,"s'phw"description,tail]\ar[u,"qhw"description]\ar[d,"kw"description]&Z\\
&W'\ar[lu,"s'r'",tail]\ar[ru,swap,"bc'"]&
\end{tikzcd}\]
so $(W;sr,bc)\sim(W';s'r',bc')$. A similar argument proves the case where $(U;s,a)$ is fixed, and the same result holds for right multiplicative systems.\par
We now verify that associativity, so let $(A;s,a)\in M_{X,Y}^l$, $(B;t,b)\in M_{Y,Z}^l$, and $(C;u,c)\in M_{Z,W}^l$. Apply axiom (S3) three times, we obtain a diagram
\[\begin{tikzcd}[row sep=5mm, column sep=5mm]
&&&\bullet\ar[rd]\ar[ld,tail]&&&\\
&&\bullet\ar[ld,tail]\ar[rd]&&\bullet\ar[rd]\ar[ld,tail]\\
&A\ar[ld,swap,tail,"s"]\ar[rd,"a"]&&B\ar[rd,"b"]\ar[ld,swap,"t"]&&C\ar[rd,"c"]\ar[ld,swap,"u",tail]&\\
X&&Y&&Z&&W
\end{tikzcd}\]
which can be considered as an element of $M_{X,W}^l$ and witnesses the associativity:
\begin{equation*}
[C;u,c]\circ([B;t,b]\circ[A;s,a])=([C;u,c]\circ[B;t,c])\circ[A;s,a].\qedhere
\end{equation*}
\end{proof}

\begin{definition}
Let $\mathcal{S}$ be a left multiplicative system of a category $\mathcal{C}$. We define $\mathcal{C}_\mathcal{S}^l$ to be the (big) category with objects $\Ob(\mathcal{C})$, and morphisms from $X$ to $Y$ given by $M_{X,Y}^l/\sim$. The identity morphism of $X$ is given by $[X;\id_X,\id_X]$, and the composition law is determined by \cref{category localization composition law}. Moreover, we define a functor $Q^l:\mathcal{C}\to\mathcal{C}_\mathcal{S}^l$ such that it is the identity on objects and sends a morphism $f:X\to Y$ to $[X;\id_X,f]$. Similarly, if $\mathcal{S}$ be a right multiplicative system, we can define a functor $Q^r:\mathcal{C}\to\mathcal{C}_\mathcal{S}^r$.
\end{definition}

It then remains to check that $(\mathcal{C}_\mathcal{S}^l,Q^l)$ (resp. $(\mathcal{C}_\mathcal{S}^r,Q^r)$) is our desired localization of $\mathcal{C}$ with respect to $\mathcal{S}$. For this, we shall use the following lemma:
\begin{lemma}\label{category localization faithful on functor lemma}
Let $Q:\mathcal{C}\to\mathcal{C}'$ and $G:\mathcal{C}'\to\mathcal{A}$ be two functors. Assume that for any $X\in\Ob(\mathcal{C}')$, there exist $Y\in\Ob(\mathcal{C})$ and a morphism $s:X\to Q(Y)$ which satisfy the following two properties:
\begin{enumerate}
    \item[(a)] $G(s)$ is an isomorphism;
    \item[(b)] for any $Y'\in\mathcal{C}$ and any morphism $t:X\to Q(Y')$, there exists $Y''\in\mathcal{C}$ and morphisms $s':Y'\to Y''$, $t':Y\to Y''$ in $\mathcal{C}$ such that $G(Q(s'))$ is an isomorphism and the following diagram commutes:
    \[\begin{tikzcd}
    X\ar[r,"s"]\ar[d,swap,"t"]&Q(Y)\ar[d,"Q(t')"]\\
    Q(Y')\ar[r,"Q(s')"]&Q(Y'')
    \end{tikzcd}\]
\end{enumerate}
Then the canonical map
\begin{equation}\label{category localization faithful on functor lemma-1}
\Hom_{\Fun(\mathcal{C}',\mathcal{A})}(F,G)\to\Hom_{\Fun(\mathcal{C},\mathcal{A})}(F\circ Q,G\circ Q)
\end{equation}
is bijective for any functor $F:\mathcal{C}'\to\mathcal{A}$.
\end{lemma}
\begin{proof}
Let $\theta_1$ and $\theta_2$ be two morphisms from $F$ to $G$ and assume that $\theta_1(Q(Y))=\theta_2(Q(Y))$ for all $Y\in\mathcal{C}$. For $X\in\mathcal{C}'$, choose a morphism $s:X\to Q(Y)$ such that $G(s)$ is an isomorphism, and consider the commutative diagram for $i=1,2$:
\[\begin{tikzcd}
F(X)\ar[r,"\theta_i(X)"]\ar[d,swap,"F(s)"]&G(X)\ar[d,"G(s)"]\\
F(Q(Y))\ar[r,"\theta_i(Q(Y))"]&G(Q(Y))
\end{tikzcd}\]
Since $G(s)$ is an isomorphism, we conclude from the hypothesis that $\theta_1(X)=\theta_2(X)$, so (\ref{category localization faithful on functor lemma-1}) is injective (it is given by horizontal composition).\par
Now let $\theta:F\circ Q\to G\circ Q$ be a morphism of functors. For each $X\in\mathcal{C}'$, we choose a morphism $s:X\to Q(Y)$ satisfying conditions (a) and (b), and define a morphism $\gamma(X):F(X)\to G(X)$ by
\[\gamma(X)=(G(s))^{-1}\circ \theta(Y)\circ F(s).\]
Let us prove that this construction is functorial, and in particular, does not depend on the choice of the morphism $s:X\to Q(Y)$ (take $f=\id_X$ in the proof). Once this is done, we obtain a morphism $\gamma:F\to G$ of functors which satisfies $\gamma(Q(Y))=\theta(Y)$ (since we can choose $s=\id_{Q(Y)}$ in this case), so (\ref{category localization faithful on functor lemma-1}) is surjective.\par
To this end, let $f:X_1\to X_2$ be a morphism in $\mathcal{C}'$. For any choice of morphisms $s_1:X_1\to Q(Y_1)$ and $s_2:X_2\to Q(Y_2)$ satisfying the given conditions, we can apply (b) to the morphisms $s_1:X_1\to Q(Y_1)$ and $s_2\circ f:X_1\to Q(Y_2)$; we then obtain morphisms $t_1:Y_1\to Y_3$ and $t_2:Y_2\to Y_3$ such that $G(Q(t_2))$ is an isomorphism and $Q(t_1)\circ s_1=Q(t_2)\circ s_2\circ f$. We then obtain a commutative diagram
\[\begin{tikzcd}[row sep=4mm, column sep=4mm]
F(X_1)\ar[rd,swap,"F(s_1)"]\ar[dddd,"F(f)"]\ar[rrrrr,"\gamma(X_1)"]&&&&&G(X_1)\ar[ld,swap,"G(s_1)","\sim"']\ar[dddd,"G(f)"]\\
&F(Q(Y_1))\ar[rd,"F(Q(t_1))"]\ar[rrr,"\theta(Y_1)"]&&&G(Q(Y_1))\ar[ld,swap,"G(Q(t_1))"]&\\
&&F(Q(Y_3))\ar[r,"\theta(Y_3)"]&G(Q(Y_3))&&\\
&F(Q(Y_2))\ar[ru,swap,"F(Q(t_2))"]\ar[rrr,"\theta(Y_2)"]&&&G(Q(Y_2))\ar[lu,"G(Q(t_2))"]&\\
F(X_2)\ar[ru,swap,"F(s_2)"]\ar[rrrrr,"\gamma(X_2)"]&&&&&G(X_2)\ar[lu,"G(s_2)","\sim"']
\end{tikzcd}\]
Since all the internal diagrams commute, the outer square also commutes, which proves our assertion.
\end{proof}

\begin{theorem}[\textbf{P. Gabriel, M. Zisman}]
Assume that $\mathcal{S}$ is a left multiplicative system. Then $(\mathcal{C}_\mathcal{S}^l,Q^l)$ (resp. $(\mathcal{C}_\mathcal{S}^r,Q^r)$) define a localization of $\mathcal{C}$ with respect to $\mathcal{S}$.
\end{theorem}
\begin{proof}
It suffices to verify the universal properties for $(\mathcal{C}_\mathcal{S}^l,Q^l)$. Write $Q=Q^l$, then if $f:X\to Y$ belongs to $\mathcal{S}$, the diagram
\[\begin{tikzcd}
&X\ar[ld,swap,"f",tail]\ar[d,"f"description]\ar[rd,"f"]&\\
Y\ar[r,equal]&Y\ar[r,equal]&Y
\end{tikzcd}\]
suggests that $[X;f,f]=[Y,\id_Y,\id_Y]$; on the other hand, the following diagram
\[\begin{tikzcd}[row sep=5mm, column sep=5mm]
&&X\ar[rd,equal]\ar[ld,equal]&&\\
&X\ar[ld,equal]\ar[rd,"f"]&&X\ar[ld,swap,"f",tail]\ar[rd,equal]&\\
X&&Y&&X
\end{tikzcd}\quad 
\begin{tikzcd}[row sep=5mm, column sep=5mm]
&&X\ar[rd,equal]\ar[ld,equal]&&\\
&X\ar[rd,equal]\ar[ld,swap,"f"]&&X\ar[rd,"f",tail]\ar[ld,equal]&\\
X&&Y&&X
\end{tikzcd}
\]
proves that $[X;f,\id_X]=[X;\id_X,f]^{-1}$, so the functor $Q$ sends the elements in $\mathcal{S}$ to isomorphisms.\par
Now consider a functor $F:\mathcal{C}\to\mathcal{A}$ such that $F(s)$ is an isomorphism for $s\in\mathcal{S}$. For any $X\in\Ob(\mathcal{C}_\mathcal{S})=\Ob(\mathcal{C})$, we define $F_\mathcal{S}(X)=F(X)$, and consider the morphisms
\begin{align*}
F_\mathcal{S}:\Hom_{\mathcal{C}_\mathcal{S}^l}(X,Y)\to \Hom_\mathcal{A}(F(X),F(Y)),\quad [U;s,a]\mapsto F(a)(F(s))^{-1}.
\end{align*}
It is clear that $F_\mathcal{S}$ sends identities to identities, and by applying $F$ to the diagram (\ref{category localization composition law def}), we see that $F_\mathcal{S}$ preserves compositions, so we obtain a functor $F_\mathcal{S}:\mathcal{C}_\mathcal{S}\to\mathcal{A}$. It is clear that $F_\mathcal{S}\circ Q\cong F$, and $F_\mathcal{S}$ is unique up to isomorphisms.\par
Finally, with the notations of \cref{category localization faithful on functor lemma}, we can choose $Y\in\Ob(\mathcal{C})$ such that $X=Q(Y)$ and $s=\id_{Q(Y)}$. Then any morphism $t:Q(Y)\to Q(Y')$ is given by morphisms
\[\begin{tikzcd}
Y\ar[r,"t'"]&Y''&Y'\ar[l,swap,"s'",tail]
\end{tikzcd}\]
and the diagram in \cref{category localization faithful on functor lemma} commutes.
\end{proof}

\begin{remark}
If $\mathcal{S}$ is both a left multiplicative system and right multiplicative system, then by \cref{category localization opposite char}, the two localizations of $\mathcal{C}$ are equivalent, and we simply denote it by $\mathcal{C}_\mathcal{S}$.
\end{remark}

\begin{corollary}\label{category localization morphism image equal iff}
Let $\mathcal{S}$ be a left (resp. right) multiplicative system, then two morphisms $f,g:X\rightrightarrows Y$ satisfy $Q^l(f)=Q^l(g)$ (resp. $Q^r(f)=Q^r(g)$) if and only if there exists $s\in\mathcal{S}$ such that $fs=gs$ (resp. $sf=sg$).
\end{corollary}
\begin{proof}
Let $\mathcal{S}$ be a left multiplicative system. Then if $Q^l(f)=Q^l(g)$, we have a commutative diagram
\[\begin{tikzcd}
&X\ar[rd,"f"]\ar[ld,equal]&\\
X&U\ar[l,"s",tail,swap]\ar[u]\ar[d]\ar[r,"a"]&Y\\
&X\ar[ru,swap,"g"]\ar[lu,equal]&
\end{tikzcd}\]
It then follows that $fs=a=gs$, whence the corollary.
\end{proof}

\begin{corollary}\label{category localization functor mono and epi}
Let $\mathcal{S}$ be a left (resp. right) multiplicative system. Then the functor $Q^l$ (resp. $Q^r$) sends monomorphisms to monomorphisms, and epimorphisms to epimorphisms.
\end{corollary}
\begin{proof}
We only consider a left multiplicative system $\mathcal{S}$. Let $f:X\to Y$ be a monomorphism in $\mathcal{C}$ and $\alpha,\beta:Q^l(W)\to Q^l(X)$ be two morphisms in $\mathcal{C}_\mathcal{S}^l$ such that $(Q^l(f))\alpha=(Q^l(f))\beta$. Then by (S2) and (S3), we can write $\alpha=(U;s,a)$ and $\beta=(U;s,b)$, and it then follows that $Q^l(fa)=Q^l(fb)$, so by \cref{category localization morphism image equal iff}, there exists $t\in\mathcal{S}$ such that $fat=fbt$, whence $at=bt$ and $\alpha=\beta$.
\end{proof}

\begin{remark}
We note that the category $\mathcal{C}_\mathcal{S}^l$ (resp. $\mathcal{C}_\mathcal{S}^r$) may not be small, since the collection $M_{X,Y}^l$ (resp. $M_{X,Y}^r$) is too big. However, if the collection $\mathcal{S}$ has a cofinal subset, then by \cref{category localization Hom set char by colim}, we can restrict the inductive limit to this set, and then $\mathcal{C}_\mathcal{S}^l$ (resp. $\mathcal{C}_\mathcal{S}^r$) will be small. This is the case if $\mathcal{C}$ itself is already small. 
\end{remark}

We now give some properties of the localization functor $Q$. For this, assume that $\mathcal{S}$ is a left (resp. right) multiplicative system and let $X\in\mathcal{C}$. We define a functor
\[\theta_{/X}:\mathcal{S}_{/X}\to\mathcal{C}_{Q(X)/}\quad (\text{resp.\ } \theta_{X/}:\mathcal{S}_{X/}\to\mathcal{C}_{/Q(X)})\]
by associating a morphism $s:Y\to X$ in $\mathcal{S}_{/X}$ (resp. a morphism $s:X\to Y$ in $\mathcal{S}_{X/}$) with the morphism $Q(s)^{-1}:Q(X)\to Q(Y)$ in $\mathcal{C}_{Q(X)/}$ (resp. the morphism $Q(s)^{-1}:Q(Y)\to Q(X)$ in $\mathcal{C}_{Q(X)/}$).

\begin{lemma}\label{category localization functor theta cofinal}
Assume that $\mathcal{S}$ is a left (resp. right) multiplicative system and let $X\in\mathcal{C}$. Then the functor $\theta_{/X}^{\op}$ (resp. $\theta_{X/}$) is cofinal.
\end{lemma}
\begin{proof}
We only consider right multiplicative systems. 
\end{proof}

\begin{proposition}\label{category localization functor exactness prop}
Let $\mathcal{S}$ be a left (resp. right) multiplicative system and $Q:\mathcal{C}\to\mathcal{C}_\mathcal{S}$ be the corresponding localization functor.
\begin{enumerate}
    \item[(a)] The functor $Q$ is left (resp. right) exact.
    \item[(b)] Let $\alpha:I\to\mathcal{C}$ be a projective (resp. inductive) system in $\mathcal{C}$ indexed by a finite category $I$. Assume that $\llim\alpha$ (resp. $\rlim\alpha$) exists in $\mathcal{C}$, then $\llim(Q\circ\alpha)$ (resp. $\rlim(Q\circ\alpha)$) exists in $\mathcal{C}_S$ and is isomorphic to $Q(\llim\alpha$) (resp. $Q(\rlim\alpha)$).
    \item[(c)] Assume that $\mathcal{C}$ admits kernels (resp. cokernels). Then $\mathcal{C}_\mathcal{S}$ admits kernels (resp. cokernels) and $Q$ commutes with kernels (resp. cokernels).
    \item[(d)] Assume that $\mathcal{C}$ admits finite products (resp. coproducts). Then $\mathcal{C}_\mathcal{S}$ admits finite products (resp. coproducts) and $Q$ commutes with finite products (resp. coproducts).
    \item[(e)] If $\mathcal{C}$ admits finite projective (resp. inductive) limits, then so does $\mathcal{C}_\mathcal{S}$.
\end{enumerate}
\end{proposition}

\begin{proposition}\label{category localization of subcategory prop}
Let $\mathcal{C}$ be a category, $\mathcal{I}$ be a full subcategory, $\mathcal{S}$ be a left (resp. right) multiplicative system in $\mathcal{C}$, and $\mathcal{T}$ be the family of morphisms in $\mathcal{I}$ which belong to $\mathcal{S}$.
\begin{enumerate}
    \item[(a)] Assume that $\mathcal{T}$ is a left (resp. right) multiplicative system in $\mathcal{I}$. Then there is a well-defined functor $\mathcal{I}^l_\mathcal{T}\to\mathcal{C}^l_\mathcal{S}$ (resp. $\mathcal{I}^r_\mathcal{T}\to\mathcal{C}^r_\mathcal{S}$).
    \item[(b)] Assume that for every $f:X\to Y$ in $\mathcal{S}$ with $Y\in\mathcal{I}$ (resp. $X\in\mathcal{I}$), there exist a morphism $g:W\to X$ with $W\in\mathcal{I}$ and $fg\in\mathcal{S}$ (resp. a morphism $g:Y\to W$ with $W\in\mathcal{I}$ and $gf\in\mathcal{S}$). Then $\mathcal{T}$ is a left (resp. right) multiplicative system and the functor $\mathcal{I}^l_\mathcal{T}\to\mathcal{C}^l_\mathcal{S}$ (resp. $\mathcal{I}^r_\mathcal{T}\to\mathcal{C}^r_\mathcal{S}$) is fully faithful.
\end{enumerate}
\end{proposition}
\begin{proof}
Assertion (a) is clear from the definition, and as for (b), it is easy to verify that $\mathcal{T}$ is a left multiplicative system under the corresonding assumption. For $X\in\mathcal{I}$, we define the category $\mathcal{T}_{/X}$ as the full subcategory of $\mathcal{S}_{/X}$ whose objects are morphisms $s:Y\to X$ with $Y\in\mathcal{I}$. The hypothesis in (b) then amounts to saying that the functor $\mathcal{T}_{/X}\to\mathcal{S}_{/X}$ is cofinal, so the result follows from \cref{*}.
\end{proof}

\begin{corollary}\label{category localization of subcategory equivalent if}
Let $\mathcal{C}$ be a category, $\mathcal{I}$ a full subcategory, $\mathcal{S}$ be a left (resp. right) multiplicative system in $\mathcal{C}$, $\mathcal{T}$ the family of morphisms in $\mathcal{I}$ which belong to $\mathcal{S}$. Assume that for any $X\in\mathcal{C}$ there exists a morphism $s:I\to X$ with $I\in\mathcal{I}$ and $s\in\mathcal{S}$ (resp. a morphism $s:X\to I$ with $I\in\mathcal{I}$ and $s\in\mathcal{S}$). Then $\mathcal{T}$ is a left (resp. right) multiplicative system and $\mathcal{I}_\mathcal{T}^l$ (resp. $\mathcal{I}_\mathcal{T}^r$) is equivalent to $\mathcal{C}_\mathcal{S}^l$ (resp. $\mathcal{C}_\mathcal{S}^r$).
\end{corollary}
\begin{proof}
The natural functor $\mathcal{I}^l_\mathcal{T}\to\mathcal{C}^l_\mathcal{S}$ (resp. $\mathcal{I}^r_\mathcal{T}\to\mathcal{C}^r_\mathcal{S}$) is fully faithful by \cref{category localization of subcategory prop}, and essentially surjective by hypothesis.
\end{proof}

\begin{theorem}\label{category localization additive prop}
Let $\mathcal{C}$ be a pre-additive category and $\mathcal{S}$ be a left (resp. right) multiplicative system.
\begin{enumerate}
    \item[(a)] The localization $\mathcal{C}_\mathcal{S}^l$ (resp. $\mathcal{C}_\mathcal{S}^r$) has a canonical structure of a pre-additive category, so that $Q^l$ (resp. $Q^r$) is an additive functor.
    \item[(b)] If $\mathcal{C}$ is additive and $\mathcal{S}$ be a multiplicative system, then $\mathcal{C}_\mathcal{S}$ is an additive category.  
\end{enumerate}
The same result is true if we replace additive by $k$-linear, where $k$ is a commutative ring.
\end{theorem}
\begin{proof}
As for (a), it suffices to consider right multiplicative systems. We now define an addition for the Hom set of $\mathcal{C}_\mathcal{S}$. If $f,g\in\Hom_{\mathcal{C}_\mathcal{S}}(X,Y)$, then, since $\mathcal{S}_{/X}^{\op}$ is filtrant, there exist $s:U\rightarrowtail X$ and $a_1,a_2:U\to Y$ such that $f=[U;s,a_1]$ and $g=[U;s,a_2]$. We can therefore define $f+g$ by
\[f+g:=[U;s,a_1+a_2]\in\Hom_{\mathcal{C}_\mathcal{S}}(X,Y).\]
In particular, the zero morphism can be written as $[U;s,0]$, and $-[U;s,a]=[U;s,-a]$. It is then a simple matter to show that this definition is independent of the choices of $a_1$ and $a_2$, which follows easily from the filtrant property of $\mathcal{S}_{/X}^{\op}$. Finally, with this definition, it is then easy to check that $Q$ is an additive functor, and the second assertion follows from \cref{category localization functor exactness prop}.
\end{proof}

\subsection{Kan extensions along a localization}
Let $\mathcal{C}$ be a category, $\mathcal{S}$ a (right, say) multiplicative system in $\mathcal{C}$ and $F:\mathcal{C}\to\mathcal{A}$ a functor. We consider the existence of the following factorization diagram:
\[\begin{tikzcd}
\mathcal{C}\ar[rd,bend left=25pt,"F"]\ar[d,swap,"Q"]&\\
\mathcal{C}_\mathcal{S}\ar[r,dashed,"\exists ?"]&\mathcal{A}
\end{tikzcd}\]
In general, $F$ does not send morphisms in $\mathcal{S}$ to isomorphisms in $\mathcal{A}$, so it does not factorize through $\mathcal{C}_\mathcal{S}$. It is however possible in some cases to define a localization of $F$ as a "best approximation", in the following sense:
\begin{definition}
Let $\mathcal{S}$ be a family of morphisms in $\mathcal{C}$ and assume that the localization $Q:\mathcal{C}\to\mathcal{C}_\mathcal{S}$ exists.
\begin{enumerate}
    \item[(a)] We say that $F$ is \textbf{right localizable} if the left Kan extension $\Lan_QF$ of $F$ with respect to $Q$ exists. In such a case, we say that $\Lan_QF$ is a \textbf{right localization} of F and we denote it by $R_\mathcal{S}F$. In other words, the \textbf{right localization} of $F$ is a functor $R_\mathcal{S}F:\mathcal{C}_\mathcal{S}\to\mathcal{A}$ together with a morphism of functors $\eta:F\to R_\mathcal{S}F\circ Q$ such that for any functor $G:\mathcal{C}_\mathcal{S}\to\mathcal{A}$, the map
    \[\Hom_{\Fun(\mathcal{C}_\mathcal{S},\mathcal{A})}(R_\mathcal{S}F,G)\to\Hom_{\Fun(\mathcal{C},\mathcal{A})}(F,G\circ Q)\]
    is bijective (This map is given by composing with $\eta$). 
    \item[(b)] We say that $F$ is universally right localizable if for any functor $K:\mathcal{A}\to\mathcal{B}$, the functor $K\circ F$ is localizable and $R_\mathcal{S}(K\circ F)\stackrel{\sim}{\to} K\circ R_\mathcal{S}F$.
\end{enumerate}
\end{definition}
We can similarly define left localizations of $F$ by right Kan extensions, and consider universally left localizable functors. That is, the left localization of $F$ is a functor $L_\mathcal{S}F:\mathcal{C}_\mathcal{S}\to\mathcal{A}$ together with a morphism $\eps:L_\mathcal{S}F\circ Q\to F$ such that for any functor $G:\mathcal{C}_\mathcal{S}\to\mathcal{A}$, $\eps$ induces a bijection
\[\Hom_{\Fun(\mathcal{C}_\mathcal{S},\mathcal{A})}(G,L_\mathcal{S}F)\to\Hom_{\Fun(\mathcal{C},\mathcal{A})}(G\circ Q,F).\]

One should be aware that even if $F$ admits both a right and a left localization, the two localizations are not isomorphic in general. However, when the localization $Q:\mathcal{C}\to\mathcal{C}_\mathcal{S}$ exists and $F$ is right and left localizable, the canonical morphisms of functors $L_\mathcal{S}F\circ Q\to F\to R_\mathcal{S}F\circ Q$ together with the isomorphism $\Hom(L_\mathcal{S}F\circ Q,R_\mathcal{S}F\circ Q)\cong\Hom(L_\mathcal{S}F,R_\mathcal{S}F)$ in (L3) gives a canonical morphism of functors $L_\mathcal{S}F\to R_\mathcal{S}F$. From now on, we shall concentrate on right localizations.

\begin{proposition}\label{category localization of functor exist if}
Let $\mathcal{C}$ be a category, $\mathcal{I}$ be a full subcategory, $\mathcal{S}$ be a left (resp. right) multiplicative system in $\mathcal{S}$, $\mathcal{T}$ be the family of morphisms in $\mathcal{I}$ which belong to $\mathcal{S}$. Let $F:\mathcal{C}\to\mathcal{A}$ be a functor. Assume that the following "resolution condition" is satisfied:
\begin{enumerate}
    \item[(a)] for any $X\in\mathcal{C}$, there exists $s:I\to X$ (resp. $s:X\to I$) with $I\in\mathcal{I}$ and $s\in\mathcal{S}$;
    \item[(b)] for any $t\in\mathcal{T}$, $F(t)$ is an isomorphism.
\end{enumerate}
Then $F$ is universally left (resp. right) localizable and the composition 
\[\begin{tikzcd}[column sep=12mm]
\mathcal{I}\ar[r]&\mathcal{C}\ar[r,"Q"]\ar[r]&\mathcal{C}_\mathcal{S}\ar[r,"\text{$L_\mathcal{S}F$ or $R_\mathcal{S}F$}"]&\mathcal{A}
\end{tikzcd}\]
is isomorphic to the restriction of $F$ to $\mathcal{I}$. Moreover, we have canonical isomorphisms
\begin{gather}
(L_\mathcal{S}F)(Q(X)) \stackrel{\sim}{\to} \llim_{[Y\rightarrowtail X]\in\Ob(\mathcal{S}_{/X}^{\op})}F(Y),\label{category localization of functor exist if-1}\\
(R_\mathcal{S}F)(Q(X)) \stackrel{\sim}{\to} \rlim_{[X\rightarrowtail Y]\in\Ob(\mathcal{S}_{X/})}F(Y).\label{category localization of functor exist if-2}
\end{gather}
and the morphism $\eps:L_\mathcal{S}F\circ Q\to F$ (resp. $\eta:F\to R_\mathcal{S}F\circ Q$) is given by projecting to the term $F(X)$ corresponding to the identity morphism $\id_X\in\Ob(\mathcal{S}_{/X}^{\op})$ (resp. $\id_X\in\Ob(\mathcal{S}_{X/})$).
\end{proposition}
\begin{proof}
It suffices to consider right multiplicative systems. Denote by $\iota:\mathcal{I}\to\mathcal{C}$ the natural functor. By condition (a) and \cref{category localization of subcategory equivalent if}, $\iota_Q:\mathcal{I}_\mathcal{T}\to\mathcal{C}_\mathcal{S}$ is an equivalence, and condition (b) implies that the localization $F_\mathcal{T}$ of $F\circ\iota$ exists. We consider the solid diagram
\[\begin{tikzcd}
&\mathcal{C}\ar[rd,swap,"Q_\mathcal{S}"]\ar[rrd,bend left=10pt,"F"]&&\\
\mathcal{I}\ar[ru,swap,"\iota"]\ar[rd,"Q_\mathcal{T}"]&&\mathcal{C}_\mathcal{S}\ar[r,dashed,"RF"]&\mathcal{A}\\
&\mathcal{I}_\mathcal{T}\ar[ru,"\iota_Q"]\ar[rru,swap,bend right=10pt,"F_\mathcal{T}"]
\end{tikzcd}\]
Denote by $\iota_Q^{-1}$ a quasi-inverse of $\iota_Q$ and set $RF=F_\mathcal{T}\circ\iota_Q^{-1}$. Then the above diagram commutes, except the triangle $(\mathcal{C},\mathcal{C}_\mathcal{S},\mathcal{A})$. We now prove that $RF$ is the right localization of $F$.\par
Let $G:\mathcal{C}_\mathcal{S}\to\mathcal{A}$ be a functor; we have the chain of a morphism and isomorphisms:
\begin{equation}\label{category localization of functor exist if-3}
\begin{aligned}
\Hom_{\Fun(\mathcal{C},\mathcal{A})}(F,G\circ Q_\mathcal{S})& \stackrel{\lambda}{\to} \Hom_{\Fun(\mathcal{I},\mathcal{A})}(F\circ\iota,G\circ Q_\mathcal{S}\circ\iota)\\
&\cong\Hom_{\Fun(\mathcal{I},\mathcal{A})}(F_\mathcal{T}\circ Q_\mathcal{T},G\circ\iota_Q\circ Q_\mathcal{T})\\
&\cong\Hom_{\Fun(\mathcal{I}_\mathcal{T},\mathcal{A})}(F_\mathcal{T},G\circ\iota_Q)\\
&\cong\Hom_{\Fun(\mathcal{C}_\mathcal{S},\mathcal{A})}(F_\mathcal{T}\circ\iota_Q^{-1},G)\\
&\cong\Hom_{\Fun(\mathcal{C}_\mathcal{S},\mathcal{A})}(RF,G).
\end{aligned}
\end{equation}
The second isomorphism follows from the fact that $Q_\mathcal{T}$ satisfies axiom (L3). To conclude, it remains to prove that the morphism $\lambda$ is bijective. Let us check that \cref{category localization faithful on functor lemma} applies to $\iota:\mathcal{I}\to\mathcal{C}$ and $Q_\mathcal{S}:\mathcal{C}\to\mathcal{C}_\mathcal{S}$, and hence to $\iota:\mathcal{I}\to\mathcal{C}$ and $G\circ Q_\mathcal{S}:\mathcal{C}\to\mathcal{A}$. Let $X\in\Ob(\mathcal{C})$; by hypothesis, there exists $Y\in\mathcal{I}$ and $s:X\to\iota(Y)$ with $s\in\mathcal{S}$. Then $F(s)$ is an isomorphism and condition (a) of \cref{category localization faithful on functor lemma} is satisfied. On the other hand, condition (b) follows from axiom (S3') and the fact that $\iota$ is fully faithful. Finally, to see that the limit of (\ref{category localization of functor exist if-2}) exists, we can assume that $X\in\Ob(\mathcal{I})$, but the limit is then isomorphic to $F(X)$, since $\id_X$ is initial in $\Ob(\mathcal{S}_{X/})$. In view of the general construction of $\Lan_QF$ and \cref{category localization functor theta cofinal}, it follows that $R_\mathcal{S}F$ is isomorphic to the limit in (\ref{category localization of functor exist if-2}).\par
If $K:\mathcal{A}\to\mathcal{A}'$ is another functor, $K\circ F(t)$ will be an isomorphism for any $t\in\mathcal{T}$. Hence, $K\circ F$ is localizable and we have
\begin{equation*}
R_\mathcal{S}(K\circ F)\cong (K\circ F)_\mathcal{T}\circ\iota_Q^{-1}\cong K\circ F_\mathcal{T}\circ\iota_Q^{-1}\cong K\circ R_\mathcal{S}F.\qedhere
\end{equation*}
\end{proof}

\begin{corollary}
Let $\mathcal{A}$ be a category which admits small filtrant inductive limits. Let $\mathcal{S}$ be a left (resp. right) multiplicative system and assume that for each $X\in\mathcal{C}$, the category $\mathcal{S}_{/X}^{\op}$ (resp. $\mathcal{S}_{X/}$) is cofinally small.
\begin{enumerate}
    \item[(a)] $\mathcal{C}_\mathcal{S}$ is a $\mathscr{U}$-category.
    \item[(b)] The functor $Q^\star$ admits a right adjoint $_\star Q$ (resp. left adjoint functor $Q_\star$).
    \item[(c)] Any functor $F:\mathcal{C}\to\mathcal{A}$ is left (resp. right) localizable and
    \begin{gather*}
    (L_\mathcal{S}F)(Q(X)) \stackrel{\sim}{\to} \llim_{[Y\rightarrowtail X]\in\Ob(\mathcal{S}_{/X}^{\op})}F(Y),\quad (R_\mathcal{S}F)(Q(X)) \stackrel{\sim}{\to} \rlim_{[X\rightarrowtail Y]\in\Ob(\mathcal{S}_{X/})}F(Y).
    \end{gather*}
\end{enumerate}
\end{corollary}
\begin{proof}
Assertion (a) is obvious and (b), (c) follow from \cref{category localization functor theta cofinal}, since we may apply \cref{category localization of functor exist if} to construct $_\star Q$ (resp. $Q_\star$).
\end{proof}

\subsection{Triangulated categories}
Triangulated categories are additive categories with a collection of distingushied triangles. They arise naturally from the derived category of an abelian category and is important for the study of properties of derived categories. To begin with, we first consider categories with a translation functor.
\begin{definition}
A \textbf{category with translation} $(\mathcal{D},T)$ is a category $\mathcal{D}$ endowed with an equivalence of categories $T:\mathcal{D}\to\mathcal{D}$. The functor $T$ is called the \textbf{translation functor}.
\begin{itemize}
    \item A functor of categories with translation $F:(\mathcal{D},T)\to (\mathcal{D}',T')$ is a functor $F:\mathcal{D}\to\mathcal{D}'$ together with an isomorphism $F\circ T\cong T'\circ F$. If $\mathcal{D}$ and $\mathcal{D}'$ are additive categories and $F$ is additive, we say that $F$ is a functor of additive categories with translation.
    \item Let $F,F':(\mathcal{D},T)\to (\mathcal{D}',T')$ be two functors of categories with translation. A morphism $\theta:F\to F'$ of functors of categories with translation is a morphism of functors such that the diagram below commutes
    \[\begin{tikzcd}
    F\circ T\ar[r,"\theta\circ T"]\ar[d,swap,"\sim"]&F'\circ T\ar[d,"\sim"]\\
    T'\circ F\ar[r,"T'\circ\theta"]&T'\circ F'
    \end{tikzcd}\]
    \item A subcategory with translation $(\mathcal{D}',T')$ of $(\mathcal{D},T)$ is a category with translation such that $\mathcal{D}'$ is a subcategory of $\mathcal{D}$ and the translation functor $T'$ is the restriction of $T$.
    \item Let $(\mathcal{D},T)$, $(\mathcal{D}',T')$ and $(\mathcal{D}'',T'')$ be additive categories with translation. A bifunctor of additive categories with translation $F:\mathcal{D}\times\mathcal{D}'\to\mathcal{D}''$ is an additive bifunctor endowed with functorial isomorphisms
    \[\theta_{X,Y}:F(T(X),Y) \stackrel{\sim}{\to} T''(F(X,Y)),\quad \lambda_{X,Y}:F(X,T'(Y))\stackrel{\sim}{\to} T''(F(X,Y))\]
    for $(X,Y)\in\mathcal{D}\times\mathcal{D}'$ such that the diagram below anti-commutes:
    \[\begin{tikzcd}
    F(T(X),T'(Y))\ar[r,"\theta_{X,T'(Y)}"]\ar[d,swap,"\lambda_{T(X),Y}"]&T''(F(X,T'(Y)))\ar[d,"T''(\lambda_{X,Y})"]\\
    T''(F(T(X),Y))\ar[r,"T''(\theta_{X,Y})"]&T''^2(F(X,Y))
    \end{tikzcd}\]
\end{itemize}
\end{definition}

If $(\mathcal{D},T)$ is a category with translation, we shall denote by $T^{-1}$ a quasi-inverse of $T$. Then $T^n$ is well defined for $n\in\Z$. These functors are unique up to unique isomorphism. If there is no risk of confusion, we shall write $\mathcal{D}$ instead of $(\mathcal{D},T)$ and $X[1]$ (resp. $X[-1]$) instead of $T(X)$ (resp. $T^{-1}(X)$).
\begin{example}
Let $\mathcal{A}$ be an additive category and $\Ch(A)$ be the category of chain complexes of $\mathcal{A}$. Then we have the shift functor $T:X\mapsto X[1]$ defined by $X[1]^n=X^{n+1}$ and $d[1]^n=-d^{n+1}$, so $(\Ch(\mathcal{A}),T)$ is an additive category with translation. 
\end{example}

\begin{definition}
Let $(\mathcal{D},T)$ be an additive category with translations. A \textbf{triangle} in $\mathcal{D}$ is a sequence of morphisms
\[\begin{tikzcd}
X\ar[r,"f"]&Y\ar[r,"g"]&Z\ar[r,"h"]&X[1]
\end{tikzcd}\]
A morphism of triangles is a commutative diagram
\[\begin{tikzcd}
X\ar[r,"f"]\ar[d,"\alpha"]&Y\ar[r,"g"]\ar[d,"\beta"]&Z\ar[r,"h"]\ar[d,"\gamma"]&X[1]\ar[d,"{\alpha[1]}"]\\
X'\ar[r,"f'"]&Y'\ar[r,"g'"]&Z'\ar[r,"h'"]&X'[1]
\end{tikzcd}\]
\end{definition}

\begin{remark}
Let $(\mathcal{D},T)$ be a $k$-linear category with translations and
\[\begin{tikzcd}
X\ar[r,"f"]&Y\ar[r,"g"]&Z\ar[r,"h"]&X[1]
\end{tikzcd}\]
be a triangle. Let $\eps,\zeta,\eta\in k^\times$. If $\eps\zeta\eta=1$, then the original triangle is isomorphic to the following:
\[\begin{tikzcd}
X\ar[r,"\eps f"]&Y\ar[r,"\zeta g"]&Z\ar[r,"\eta h"]&X[1]
\end{tikzcd}\]
In fact, we have a commutative diagram
\[\begin{tikzcd}
X\ar[r,"f"]\ar[d,equal]&Y\ar[r,"g"]\ar[d,"\eps"]&Z\ar[r,"h"]\ar[d,"\eps\zeta"]&X[1]\ar[d,"\eps\zeta\eta"]\\
X\ar[r,"\eps f"]&Y\ar[r,"\zeta g"]&Z\ar[r,"\eta h"]&X[1]
\end{tikzcd}\]
\end{remark}

\begin{definition}
A \textbf{triangulated category} is an additive category $(\mathcal{D},T)$ with translation endowed with a family of triangles, called \textbf{distinguished triangles}, satisfying the axioms below:
\begin{enumerate}[leftmargin=40pt]
    \item[(TR0)] A triangle isomorphic to a distinguished triangle is a distinguished triangle.
    \item[(TR1)] The triangle $X\stackrel{\id_X}{\longrightarrow}X\to 0\to X[1]$ is a distinguished triangle.
    \item[(TR2)] For any morphism $f:X\to Y$, there exists a distinguished triangle $X\stackrel{f}{\to} Y\to Z\to X[1]$.
    \item[(TR3)] A triangle $X\stackrel{f}{\to} Y\stackrel{g}{\to} Z\stackrel{h}{\to} X[1]$ is a distinguished triangle if and only if its "rotation"
    \[\begin{tikzcd}
    Y\ar[r,"-g"]&Z\ar[r,"-h"]&X[1]\ar[r,"{-f[1]}"]&Y[1]
    \end{tikzcd}\]
    is a distinguished triangle.
    \item[(TR4)] Given a solid diagram
    \[\begin{tikzcd}
    X\ar[r]\ar[d,"\alpha"]&Y\ar[r]\ar[d,"\beta"]&Z\ar[r]\ar[d,dashed,"\gamma"]&X[1]\ar[d,dashed,"{\alpha[1]}"]\\
    X'\ar[r]&Y'\ar[r]&Z'\ar[r]&X'[1]
    \end{tikzcd}\]
    with both rows being distinguished triangles, there exists a morphism $\gamma:Z\to Z'$ giving rise to a morphisms of distinguished triangles.
    \item[(TR5)] Given three distinguished triangles
    \begin{gather*}
    X\stackrel{f}{\to} Y\to Z'\to X[1],\\
    Y\stackrel{g}{\to} Z\to X'\to Y[1],\\
    X\stackrel{gf}{\to} Z\to Y'\to X[1],
    \end{gather*}
    there exists a distinguished triangle $Z'\to Y'\to X'\to Z'[1]$ making the following diagram commutative:
    \begin{equation}\label{triangle cat octahedron diagram}
    \begin{tikzcd}
    X\ar[r,"f"]\ar[d,equal]&Y\ar[r]\ar[d,"g"]&Z'\ar[r]\ar[d,dashed]&X[1]\ar[d,equal]\\
    X\ar[r,"gf"]\ar[d,"f"]&Z\ar[r]\ar[d,equal]&Y'\ar[r]\ar[d,dashed]&X[1]\ar[d]\\
    Y\ar[r,"g"]\ar[d]&Z\ar[r]\ar[d]&X'\ar[r]\ar[d,equal]&Y[1]\ar[d]\\
    Z'\ar[r,dashed]&Y'\ar[r,dashed]&X'\ar[r,dashed]&Z'[1]
    \end{tikzcd}
    \end{equation}
    Diagram (\ref{triangle cat octahedron diagram}) is often called the \textbf{octahedron diagram}. Indeed, it can be written using the vertices of an octahedron:
    \[\begin{tikzcd}
    &&Y'&&\\
    Z'&&&&X'\\
    X&&&&Z\\
    &&Y&&
    \arrow[from=2-1,to=1-3,dashed]
    \arrow[from=1-3,to=2-5,dashed]
    \arrow[from=2-1,to=3-1,swap,"+1"]
    \arrow[from=3-5,to=2-5]
    \arrow[from=2-5,to=4-3,swap,"+1",yshift=1ex]
    \arrow[from=4-3,to=2-1]
    \arrow[from=4-3,to=3-5,swap,"g"]
    \arrow[from=2-5,to=2-1,swap,"+1"]
    \arrow[from=3-5,to=1-3,crossing over]
    \arrow[from=3-1,to=4-3,swap,"f"]
    \arrow[from=1-3,to=3-1,crossing over]
    \arrow[from=3-1,to=3-5,crossing over]
    \end{tikzcd}\]
    Here we use $X'\stackrel{+1}{\to} Y$ to denote a morphism $X'\to Y[1]$.
\end{enumerate}
\end{definition}
An additive category $(\mathcal{D},T)$ satisfying (TR0)--(TR4) is called a \textbf{pretriangulated category}. One should note that the morphism $\gamma$ in (TR4) is not unique, and is unique up to \textit{non-unique} isomorphisms.

\begin{definition}
A \textbf{triangulated functor} of triangulated categories $F:(\mathcal{D},T)\to(\mathcal{D}',T')$ is a functor of additive categories with translation sending distinguished triangles to distinguished triangles. If moreover $F$ is an equivalence of categories, $F$ is called an \textbf{equivalence of triangulated categories}. If $F,F':(\mathcal{D},T)\to(\mathcal{D}',T')$ are triangulated functors, a morphism $\theta:F\to F'$ of triangulated functors is a morphism of functors of additive categories with translation.\par
A triangulated subcategory $(\mathcal{D}',T')$ of $(\mathcal{D},T)$ is an additive subcategory with translation of $\mathcal{D}$ (i.e., the functor $T'$ is the restriction of $T$) such that it is triangulated and that the inclusion functor is triangulated.
\end{definition}

\begin{remark}
A triangle $X\stackrel{f}{\to} Y\stackrel{g}{\to} Z\stackrel{h}{\to} X[1]$ is called \textbf{anti-distinguished} if the triangle $X\stackrel{f}{\to} Y\stackrel{g}{\to} Z\stackrel{-h}{\to} X[1]$ is distinguished. Then $(\mathcal{D},T)$ endowed with the family of anti-distinguished triangles is triangulated. If we denote by $(\mathcal{D}^{ant},T)$ this triangulated category, then $(\mathcal{D}^{ant},T)$ and $(\mathcal{D},T)$ are equivalent as triangulated categories.
\end{remark}

\begin{remark}
Consider the contravariant functor $\op:\mathcal{D}\to\mathcal{D}^{\op}$, and define
\[T^{\op}=\op\circ T^{-1}\circ\op:\mathcal{D}^{\op}\to\mathcal{D}^{\op}\]
(we use the fact that $\op^2=\id_\mathcal{D}$.) A triangle $X\stackrel{f}{\to} Y\stackrel{g}{\to} Z\stackrel{h}{\to} T^{\op}(X)$ in $\mathcal{D}^{\op}$ is called distinguished if its image
\[\begin{tikzcd}
Z^{\op}\ar[r,"g^{\op}"]&Y^{\op}\ar[r,"f^{\op}"]&X^{\op}\ar[r,"{T(h^{\op})}"]&T(Z^{\op})
\end{tikzcd}
\]
by $\op$ is distinguished. With this definition, it is easy to check that $(\mathcal{D}^{\op},T^{\op})$ is a triangulated category.
\end{remark}

\begin{proposition}\label{triangle cat dt composition zero}
If $X\stackrel{f}{\to} Y\stackrel{g}{\to} Z\to X[1]$ is a distinguished triangle, then $gf=0$.
\end{proposition}
\begin{proof}
Applying (TR1) and (TR4) we get a commutative diagram
\[\begin{tikzcd}
X\ar[r,equal]\ar[d,equal]&X\ar[d,"f"]\ar[r]&0\ar[d]\ar[r]&X[1]\ar[d,equal]\\
X\ar[r,"f"]&Y\ar[r,"g"]&Z\ar[r]&X[1]
\end{tikzcd}\]
Then $gf$ factorizes through $0$.
\end{proof}

\begin{definition}
Let $(\mathcal{D},T)$ be a pretriangulated category and $\mathcal{C}$ an abelian category. An additive functor $F:\mathcal{D}\to\mathcal{C}$ is called \textbf{cohomological} if for any distinguished triangle $X\to Y\to Z\to X[1]$ in $\mathcal{D}$, the sequence $F(X)\to F(Y)\to F(Z)$ is exact in $\mathcal{C}$.
\end{definition}

If $F$ is a cohomological functor $F:\mathcal{D}\to\mathcal{C}$, then for any distinguished triangle $X\to Y\to Z\to X[1]$ in $\mathcal{D}$, by rotating the triangle by (TR3), we obtain a long exact sequence
\[\begin{tikzcd}
\cdots\ar[r]&F(Z[-1])\ar[r]&F(X)\ar[r]&F(Y)\ar[r]&F(Z)\ar[r]&F(X[1])\ar[r]&\cdots
\end{tikzcd}\]

A basic example of cohomological functors is the Hom functor:
\begin{proposition}\label{triangle cat Hom functor cohomological}
Let $(\mathcal{D},T)$ be a pretriangulated category and $S$ be an object of $\mathcal{D}$. Then the functors $\Hom_\mathcal{D}(S,-)$ and $\Hom_\mathcal{D}(-,S)$ are cohomological.
\end{proposition}
\begin{proof}
Let $X\to Y\to Z\to X[1]$ be a distinguished triangle. We want to show that
\[\begin{tikzcd}
\Hom(S,X)\ar[r,"f_*"]&\Hom(S,Y)\ar[r]&\Hom(S,Z)
\end{tikzcd}\]
is exact, i.e. for any morphism $\varphi:S\to Y$ such that $g\circ\varphi=0$, there exists a morphism $\psi:S\to X$ such that $\varphi=f\circ\psi$. This is equivalent to say that the solid diagram below may be completed:
\[\begin{tikzcd}
S\ar[d,dashed]\ar[r,equal]&S\ar[r]\ar[d]&0\ar[r]\ar[d]&S[1]\ar[d,dashed]\\
X\ar[r,"f"]&Y\ar[r,"g"]&Z\ar[r]&X[1]
\end{tikzcd}\]
and this follows from (TR4) and (TR3). By replacing $\mathcal{D}$ with $\mathcal{D}^{\op}$, we obtain the assertion for $\Hom_\mathcal{D}(-,S)$.
\end{proof}
\begin{corollary}\label{triangle cat zero dt isomorphism}
For a distinguished triangle $X\stackrel{f}{\to} Y\to 0\to X[1]$ in a pretriangulated category, $f$ must be an isomorphism.
\end{corollary}
\begin{proof}
For every object $S$ of $\mathcal{D}$, by \cref{triangle cat Hom functor cohomological} we have an exact sequence
\[\begin{tikzcd}
\Hom(S,0[-1])=0\ar[r]&\Hom(S,X)\ar[r,"f_*"]&\Hom(S,Y)\ar[r]&\Hom(S,0)=0
\end{tikzcd}\]
so $f_*$ is an isomorphism, which means $f$ is an isomorphism.
\end{proof}

\begin{proposition}\label{triangle cat morphism dt isomorphism 2 of 3}
Let $(\mathcal{D},T)$ be a pretriangulated category and consider a morphism of distinguished triangle:
\[\begin{tikzcd}
X\ar[r]\ar[d,"\alpha"]&Y\ar[r]\ar[d,"\beta"]&Z\ar[r]\ar[d,dashed,"\gamma"]&X[1]\ar[d,dashed,"{\alpha[1]}"]\\
X'\ar[r]&Y'\ar[r]&Z'\ar[r]&X'[1]
\end{tikzcd}\]
If two of $\alpha,\beta,\gamma$ are isomorphisms, then so is the third one.
\end{proposition}
\begin{proof}
By rotating the triangle, we may assume that $\alpha,\gamma$ are isomorphisms. To show that $\beta$ is an isomorphism, it suffices to show that for any object $S$ of $\mathcal{D}$, the map $\beta_*:\Hom(S,Y)\to\Hom(S,Y')$ is an isomorphism. Now by \cref{triangle cat Hom functor cohomological} we have a commutative diagram with exact rows:
\[\begin{tikzcd}[column sep=5mm]
\Hom(S,Z[-1])\ar[r]\ar[d,"\sim"]&\Hom(S,X)\ar[r]\ar[d,"\sim"]&\Hom(S,Y)\ar[r]\ar[d,"\beta_*"]&\Hom(S,Z)\ar[r]\ar[d,"\sim"]&\Hom(S,X[1])\ar[d,"\sim"]\\
\Hom(S,Z'[-1])\ar[r]&\Hom(S,X')\ar[r]&\Hom(S,Y')\ar[r]&\Hom(S,Z')\ar[r]&\Hom(S,X'[1])
\end{tikzcd}\]
so the claim follows from five lemma.
\end{proof}

\begin{corollary}\label{triangle cat dt of subcategory prop}
Let $\mathcal{D}'$ be a full pretriangulated subcategory of $\mathcal{D}$.
\begin{enumerate}
    \item[(a)] Consider a triangle $X\stackrel{f}{\to} Y\to Z\to X[1]$ in $\mathcal{D}'$ and assume that this triangle is distinguished in $\mathcal{D}$. Then it is distinguished in $\mathcal{D}'$.
    \item[(b)] Consider a distinguished triangle $X\to Y\to Z\to X[1]$ in $\mathcal{D}$ with $X,Y$ in $\mathcal{D}'$. Then $Z$ is isomorphic to an object of $\mathcal{D}'$.
\end{enumerate}
\end{corollary}
\begin{proof}
In the situation of (a), there exists a distinguished triangle $X\stackrel{f}{\to} Y\to Z'\to X[1]$ in $\mathcal{D}'$, and $X\stackrel{f}{\to} Y\to Z\to X[1]$ is isomorphic to it in $\mathcal{D}$ in view of axiom (TR4) and \cref{triangle cat morphism dt isomorphism 2 of 3}. The second assertion follows from (a).
\end{proof}

\begin{corollary}\label{triangle cat morphism dt unique prop}
The distinguished triangle $X\stackrel{f}{\to} Y\to Z\to X[1]$ in (TR2) is unique up to (non-canonical) isomorphisms. 
\end{corollary}
\begin{proof}
For distinguished triangles $X\stackrel{f}{\to} Y\to Z\to X[1]$ and $X\stackrel{f}{\to} Y\to Z\to X[1]$, axiom (TR4) gives a morphism $\gamma$ such that the diagram
\[\begin{tikzcd}
X\ar[r]\ar[d,equal]&Y\ar[r]\ar[d,equal]&Z\ar[r]\ar[d,dashed,"\gamma"]&X[1]\ar[d,equal]\\
X'\ar[r]&Y'\ar[r]&Z'\ar[r]&X'[1]
\end{tikzcd}\]
is commutative. It then suffices to apply \cref{triangle cat morphism dt isomorphism 2 of 3}. 
\end{proof}

By \cref{triangle cat morphism dt unique prop}, we see that the object $Z$ given in (TR2) is unique up to isomorphism. As already mentioned, the fact that this isomorphism is not unique is the source of many difficulties (e.g., gluing problems in sheaf theory). Let us give a criterion which ensures, in some very special cases, the uniqueness of the third term of a distinguished triangle.

\begin{proposition}\label{triangle cat TR4 unique morphism if}
In the situation of (TR4), assume that $\Hom_\mathcal{D}(Y,X')=0$ and $\Hom_\mathcal{D}(X[1],Y')=0$. Then $\gamma$ is unique.
\end{proposition}
\begin{proof}
We may replace $\alpha$ and $\beta$ by the zero morphisms and prove that in this case, $\gamma$ is zero:
\[\begin{tikzcd}
X\ar[r]\ar[d,"0"]&Y\ar[r,"f"]\ar[d,"0"]&Z\ar[r,"g"]\ar[d,"\gamma"]&X[1]\ar[d,"0"]\\
X'\ar[r,"f'"]&Y'\ar[r,"g'"]&Z'\ar[r,"h'"]&X'[1]
\end{tikzcd}\]
Since $h'\gamma=0$, by \cref{triangle cat Hom functor cohomological} the morphism $\gamma$ factorizes through $g'$, i.e. there exists $u:Z\to Y'$ with $\gamma=g'\circ u$. Similarly, since $\gamma g=0$, $\gamma$ factorizes through $h$ so there exists $v:X[1]\to Z'$ with $\gamma=vh$. By (TR4), there then exists a morphism $w:Y[1]\to X'[1]$ defining a morphism
\[\begin{tikzcd}
Y\ar[r,"g"]&Z\ar[r,"h"]\ar[ld,swap,"u"]&X[1]\ar[r,"{-f[1]}"]\ar[ld,swap,"v"]&Y[1]\ar[ld,swap,"w"]\\
Y'\ar[r,"g'"]&Z'\ar[r,"h'"]&X'[1]\ar[r]&Y'[1]
\end{tikzcd}\]
By hypothesis we have $w=0$, so $v$ factorizes through $Y'$, and this implies $v=0$ by our hypothesis, whence $\gamma=0$.
\end{proof}

\begin{proposition}\label{triangle cat triangulated functor exact}
Let $F:\mathcal{D}\to\mathcal{D}'$ be a triangulated functor between pretriangulated categories. Then $F$ is exact.
\end{proposition}
\begin{proof}
Since $F^{\op}:\mathcal{D}'^{\op}\to\mathcal{D}^{\op}$ is also a triangulated functor between pretriangulated categories, it suffices to prove that $F$ is left exact, that is, for any $X\in\mathcal{D}'$, the category $\mathcal{D}_{/X}$ is filtrant.\par
The category $\mathcal{D}_{/X}$ is nonempty since it contains the object $0\to X$, and if $(Y_1,s_1)$ and $(Y_2,s_2)$ are two objects of $\mathcal{D}_{/X}$ with $Y_i\in\mathcal{D}$ and $s_i:F(Y_i)\to X$, $i=1,2$, we obtain a morphism $s:F(Y_1\oplus Y_2)\to X$, whence morphisms $(Y_i,s_i)\to (Y_1\oplus Y_2,s)$ for $i=1,2$. Finally, consider morphisms $f,g:(Y,s)\rightrightarrows (Y',s')$ in $\mathcal{D}_{/X}$. We can embed $f-g:Y\to Y'$ into a distinguished triangle
\[\begin{tikzcd}
Y\ar[r,"f-g"]&Y'\ar[r,"h"]&Z\ar[r]&Y[1]
\end{tikzcd}\]
Since $s'F(f)=s'F(g)$, \cref{triangle cat Hom functor cohomological} implies that the morphism $s':F(Y')\to X$ factorizes as $F(Y')\to F(Z)\stackrel{t}{\to} X$, so the compositions $(Y,s)\rightrightarrows (Y',s')\to (Z,t)$ coincide, and this proves that $\mathcal{D}_{/X}$ is filtrant. 
\end{proof}

\begin{proposition}\label{triangle cat sum product of dt}
Let $\mathcal{D}$ be a pretriangulated category which admits direct sums (resp. products) indexed by a set $I$. Then direct sums indexed by $I$ commute with the translation functor $T$, and a direct sum (resp. products) of distinguished triangles indexed by $I$ is a distinguished triangle.
\end{proposition}
\begin{proof}
The first assertion is obvious since $T$ is an equivalence of categories. Now let $D_i:X_i\to Y_i\to Z_i\to X_i[1]$ be a family of distinguished triangles indexed by $I$, and $D$ be the triangle
\[\bigoplus_iD_i:\bigoplus_iX_i\to \bigoplus_iY_i\to \bigoplus_iZ_i\to \bigoplus_iX_i[1].\]
By (TR2) there exists a distinguished triangle $D':\bigoplus_iX_i\to \bigoplus_iY_i\to Z\to (\bigoplus_iX_i)[1]$, and by (TR3) there exists morphisms of triangles $D_i\to D'$ such that they induces a morphism $D\to D'$. Let $S\in\mathcal{D}$, we show that the morphism $\Hom_\mathcal{D}(D',S)\to\Hom_\mathcal{D}(D,S)$ is an isomorphism, which then implies the isomorphism $D\cong D'$. Consider the commutative diagram
\begin{small}
\[\begin{tikzcd}[column sep=3.3mm]
\Hom((\bigoplus_iY_i)[1],S)\ar[r]\ar[d]&\Hom((\bigoplus_iX_i)[1],S)\ar[r]\ar[d]&\Hom(Z,S)\ar[r]\ar[d]&\Hom(\bigoplus_iY_i,S)\ar[r]\ar[d]&\Hom(\bigoplus_iX_i,S)\ar[d]\\
\Hom(\bigoplus_iY_i[1],S)\ar[r]&\Hom(\bigoplus_iX_i[1],S)\ar[r]&\Hom(\bigoplus_iZ_i,S)\ar[r]&\Hom(\bigoplus_iY_i,S)\ar[r]&\Hom(\bigoplus_iX_i,S)
\end{tikzcd}\]
\end{small}
The first row is exact since the functor $\Hom$ is cohomological, and the second row is isomorphic to
\begin{small}
\[\begin{tikzcd}[column sep=4mm]
\prod_i\Hom(Y_i[1],S)\ar[r]&\prod_i\Hom(X_i[1],S)\ar[r]&\prod_i\Hom(Z_i,S)\ar[r]&\prod_i\Hom(Y_i,S)\ar[r]&\prod_i\Hom(X_i,S)
\end{tikzcd}\]
\end{small}
Since the functor $\prod_i$ is exact on $\mathbf{Ab}$, this complex is exact. Now the vertical arrows except the middle one are isomorphisms, so the middle one is an isomorphism by five lemma.
\end{proof}

\begin{corollary}\label{triangle cat split exact is dt}
Let $\mathcal{D}$ be a pretriangulated category. Then a triangle of the form $X\stackrel{\iota_1}{\to} X\oplus Y\stackrel{p_2}{\to} Y\stackrel{0}{\to} X[1]$ is distinguished. Conversely, if a morphisn in a distinguished triangle is zero, then this triangle comes from a direct sum.
\end{corollary}
\begin{proof}
To prove the first assertion, it suffices to apply \cref{triangle cat sum product of dt} to the distinguished triangles $X\stackrel{\id_X}{\longrightarrow} X\to 0\to X[1]$ and $0\to Y\stackrel{\id_Y}{\longrightarrow} Y\to 0$. Now consider the second assertion; by rotating the triangle, it suffices to consider a distinguished triangle of the form
\[X\to M\to Y\stackrel{0}{\to} X[1].\]
By (TR4), we then obtain a morphism of distinguished triangles:
\[\begin{tikzcd}
X\ar[r,"\iota_1"]\ar[d,equal]&X\oplus Y\ar[d,"\alpha"]\ar[r,"p_2"]&Y\ar[r,"0"]\ar[d,equal]&X[1]\ar[d,equal]\\
X\ar[r]&M\ar[r]&Y\ar[r,"0"]&X[1]
\end{tikzcd}\]
and it follows from \cref{triangle cat morphism dt isomorphism 2 of 3} that $\alpha$ is an isomorphism.
\end{proof} 

\begin{corollary}\label{triangle cat functor is additive}
Let $F:(\mathcal{D},T)\to(\mathcal{D}',T')$ be a functor between pretriangulated categories such that $F$ sends distinguished triangles to distinguished triangles. Then $F$ is additive, so it is a triangulated functor.
\end{corollary}
\begin{proof}
From the distinguished triangle $0\to 0\to 0\to 0$, we obtain a distinguished triangle $F(0)\stackrel{\id}{\to} F(0)\stackrel{\id}{\to} F(0)\stackrel{\id}{\to} F(0)$ in $\mathcal{D}'$, so \cref{triangle cat dt composition zero} implies that $\id_{F(0)}=\id_{F(0)}\circ\id_{F(0)}=0$, and therefore $F(0)\cong 0$. This also shows that $F$ sends zero morphisms to zero morphisms.\par
Now consider a distinguished triangle $X\stackrel{\iota_1}{\to} X\oplus Y\stackrel{p_2}{\to} Y\stackrel{0}{\to} X[1]$ in $\mathcal{D}$. By applying $F$, we obtain a distinguished triangle
\[\begin{tikzcd}
F(X)\ar[r,"F(\iota_1)"]&F(X\oplus Y)\ar[r]&F(Y)\ar[r,"0"]&T(F(X))
\end{tikzcd}\]
in $\mathcal{D}'$. From \cref{triangle cat split exact is dt} and its proof, it is easy to see that the canonical morphism $F(X)\oplus F(Y)\to F(X\oplus Y)$ is an isomorphism, so $F$ is additive.
\end{proof}

\begin{proposition}[\textbf{Verdier's Exercise}]
Let $\mathcal{D}$ be a triangulated category. Then any commutative diagram
\[\begin{tikzcd}
X\ar[d]\ar[r]&Y\ar[d]\\
X'\ar[r]&Y'
\end{tikzcd}\]
can be extended into a diagram
\[\begin{tikzcd}
X\ar[d]\ar[r]&Y\ar[d]\ar[r]&Z\ar[r]\ar[d]&X[1]\ar[d]\\
X'\ar[r]\ar[d]&Y'\ar[r]\ar[d]&Z'\ar[r]\ar[d]&X'[1]\ar[d]\\
X''\ar[r]\ar[d]&Y''\ar[r]\ar[d]&Z''\ar[r]\ar[rd,phantom,"\ast"]\ar[d]&X''[1]\ar[d]\\
X[1]\ar[r]&Y[1]\ar[r]&Z[1]\ar[r]&X[2]
\end{tikzcd}\]
so that every rows and columns are distinguished triangles and every square is commutative except the one labeled by $\ast$, which is anti-commutative.
\end{proposition}
\subsection{Verdier quotient}
Let $\mathcal{D}$ be a triangulated category and $\mathcal{N}$ a full saturated subcategory. Recall that $\mathcal{N}$ is saturated if $X\in\mathcal{D}$ belongs to $\mathcal{N}$ whenever $X$ is isomorphic to an object of $\mathcal{N}$.

\begin{lemma}\label{triangle cat null system def lemma}
Let $\mathcal{N}$ be a full saturated triangulated subcategory of $\mathcal{D}$. Then $\Ob(\mathcal{N})$ satisfies the following conditions:
\begin{enumerate}[leftmargin=40pt]
    \item[(N1)] $0\in\mathcal{N}$.
    \item[(N2)] $X\in\mathcal{N}$ if and only if $X[1]\in\mathcal{N}$.
    \item[(N3)] If $X\to Y\to Z\to X[1]$ is a distinguished triangle in $\mathcal{D}$ and $X,Z\in\mathcal{N}$, then $Y\in\mathcal{N}$.
\end{enumerate}
Conversely, let $\mathcal{N}$ be a full saturated subcategory of $\mathcal{D}$ and assume that $\Ob(\mathcal{N})$ satisfies conditions (N1)--(N3) above. Then the restriction of $T$ and the collection of distinguished triangles $X\to Y\to Z\to X[1]$ with $X,Y,Z$ in $\mathcal{N}$ make $\mathcal{N}$ a full saturated triangulated subcategory of $\mathcal{D}$. Moreover it satisfies
\begin{enumerate}[leftmargin=40pt]
    \item[(N3')] If $X\to Y\to Z\to X[1]$ is a distinguished triangle in $\mathcal{D}$ and two objects among $X,Y,Z$ belong to $\mathcal{N}$, then so does the third one.
\end{enumerate}
\end{lemma}
\begin{proof}
Assume that $\mathcal{N}$ is a full saturated triangulated subcategory of $\mathcal{D}$. Then (N1) and (N2) are clearly satisfied. Moreover, (N3) follows from \cref{triangle cat dt of subcategory prop} and the hypothesis that $\mathcal{N}$ is saturated.\par
Conversely, let $\mathcal{N}$ be a full subcategory of $\mathcal{D}$ satisfying (N1)--(N3); then (N3') follows from (N2) and (N3) by rotating the triangle. We now show that $\mathcal{N}$ is saturated, so let $f:X\to Y$ be an isomorphism with $X\in\mathcal{N}$. The triangle $X\stackrel{f}{\to} Y\to 0\to X[1]$ is then isomorphic to the distinguished triangle $X\stackrel{\id_X}{\longrightarrow} X\to 0\to X[1]$:
\[\begin{tikzcd}
X\ar[r,"\id_X"]\ar[d,equal]&X\ar[r]\ar[d,"f","\sim"']&0\ar[r]\ar[d,equal]&X[1]\ar[d,equal]\\
X\ar[r,"f"]&Y\ar[r]&0\ar[r]&X[1]
\end{tikzcd}\]
and hence is distinguished. It then follows from (N3) that $Y\in\mathcal{N}$. On the other hand, since $X\to X\oplus Y\to Y\stackrel{0}{\to} X[1]$ is a distinguished triangle for $X,Y\in\mathcal{N}$, we find that $X\oplus Y\in\mathcal{N}$, so $\mathcal{N}$ is a full additive subcategory of $\mathcal{D}$. The axioms of triangulated categories are then easily checked.
\end{proof}

A \textbf{null system} in $\mathcal{D}$ is a full saturated subcategory N such that $\Ob(\mathcal{N})$ satisfies the conditions (N1)--(N3) in \cref{triangle cat null system def lemma}. By \cref{triangle cat null system def lemma}, $\mathcal{N}$ can be then considered as a triangulated subcategory of $\mathcal{D}$. We associate a family of morphisms to a null system as follows:
\begin{equation}\label{triangle cat null system associated multiplicative system-1}
\mathcal{N}Q=\{f:X\to Y:\text{there exists a distinguished triangle $X\to Y\to Z\to X[1]$ with $Z\in\mathcal{N}$}\}.
\end{equation}
The morphisms in $\mathcal{N}Q$ turn out to form a multiplicative system of $\mathcal{C}$ that is compatible with the distinguished triangles in $\mathcal{D}$. To make this precise, we introduce the following definition:
\begin{definition}
Let $\mathcal{S}$ be a multiplicative system of a triangulated category $\mathcal{D}$. Then $\mathcal{D}$ is said to be \textbf{compatible with the distinguished triangles} in $\mathcal{D}$ if it satisfies the following conditions:
\begin{enumerate}[leftmargin=40pt]
    \item[(ST1)] For any morphism $s:X\to Y$ in $\mathcal{D}$, $s\in\mathcal{S}$ if and only if $s[1]\in\mathcal{S}$.
    \item[(ST2)] Consider a solid diagram
    \[\begin{tikzcd}
    X\ar[r]\ar[d,"\alpha"]&Y\ar[r]\ar[d,"\beta"]&Z\ar[r]\ar[d,dashed,"\gamma"]&X[1]\ar[d,dashed]\\
    X'\ar[r]&Y'\ar[r]&Z'\ar[r]&X'[1]
    \end{tikzcd}\]
    if $\alpha,\beta\in\mathcal{S}$, then there exists a morphism $\gamma\in\mathcal{S}$ giving rise to a morphisms of distinguished triangles.
\end{enumerate}
\end{definition}
The importance of the compatibility of $\mathcal{S}$ with distinguished triangles is contained in the following proposition:
\begin{proposition}\label{triangle cat localization by compatible multiplicative system}
Let $\mathcal{S}$ be a multiplicative system of $\mathcal{D}$ that is compatible with the distinguished triangles. Then the localization $\mathcal{D}_\mathcal{S}$ has a uniquely determined triangulated category structure so that the localization functor $Q:\mathcal{D}\to\mathcal{D}_\mathcal{S}$ is triangulated.
\end{proposition}
\begin{proof}
Since $\mathcal{D}$ is additive, it follows from \cref{category localization additive prop} that the localization $\mathcal{D}/\mathcal{N}$ is additive, and $Q:\mathcal{D}\to\mathcal{D}/\mathcal{N}$ is an additive functor. As for the uniqueness of the triangulated category structure, it suffices to note that for any distinguished triangle $X\stackrel{f}{\to} Y\to Z\to X[1]$, by adjusting using isomorphisms in $\mathcal{C}_\mathcal{S}$, we can assume that $f:X\to Y$ is a morphism in $\mathcal{D}$. But it then follows from \cref{triangle cat morphism dt isomorphism 2 of 3} that this triangle is isomorphic to a distinguished triangle in $\mathcal{D}$.\par
We now define the distinguished triangles of $\mathcal{C}_\mathcal{S}$ as the images of that of $\mathcal{D}$ under $Q$. Axioms (TR0)--(TR3) follow directly from that of $\mathcal{D}$, so let's prove (TR4). With the notations of (TR3) and (Exercise), we may assume that there exists a commutative diagram in $\mathcal{D}$ of solid arrows, with horizontal arrows belong to $\mathcal{D}$:
\[\begin{tikzcd}
X\ar[r,"f"]\ar[d,"\tilde{\alpha}"]&Y\ar[r,"g"]\ar[d,"\tilde{\beta}"]&Z\ar[r]\ar[d,dashed,"\tilde{\gamma}"]&X[1]\ar[d,dashed,"{\tilde{\alpha}[1]}"]\\
A\ar[r]&B\ar[r,dashed]&C\ar[r,dashed]&A[1]\\
X'\ar[r,"f'"]\ar[u,swap,"s",tail]&Y'\ar[r,"g'"]\ar[u,swap,"t",tail]&Z'\ar[r]\ar[u,dashed,swap,"u",tail]&X'[1]\ar[u,dashed,swap,"{s[1]}",tail]
\end{tikzcd}\]
Now by applying (TR2) to the morphism $A\to B$, we obtain a distinguished triangle $A\to B\to C\to A[1]$, and by (ST2) there is a morphism $u:Z'\to C$ in $\mathcal{S}$ completing the lower square. Also, by (TR4) there is a morphism $\tilde{\gamma}:Z\to C$ completing the upper square, and we have construct the desired morphism of distinguished triangles in $\mathcal{D}_\mathcal{S}$. Finally, consider two morphisms $f:X\to Y$ and $g:Y\to Z$ in $\mathcal{D}_\mathcal{S}$, which we may assume to belong to $\mathcal{D}$. Then by applying (TR5) and take the iamge in $\mathcal{D}_\mathcal{S}$ of the octahedron diagram, we conclude that (TR5) holds for $\mathcal{D}_\mathcal{S}$.
\end{proof}

\begin{theorem}[\textbf{Verdier}]\label{triangle cat null system associated multiplicative system}
Let $\mathcal{N}$ be a null system in a triangulated category $\mathcal{D}$.
\begin{enumerate}
    \item[(\rmnum{1})] $\mathcal{N}Q$ is a multiplicative system compatible with distinguished triangles in $\mathcal{D}$.
    \item[(\rmnum{2})] Denote by $\mathcal{D}/\mathcal{N}$ the localization of $\mathcal{D}$ by $\mathcal{N}Q$ and by $Q:\mathcal{D}\to\mathcal{D}/\mathcal{N}$ the localization functor. Then $\mathcal{D}/\mathcal{N}$ is an additive category endowed with an automorphism (the image of $T$, still denoted by $T$), and there is a canonical triangulated structure on $\mathcal{D}_\mathcal{S}$ so that $\mathcal{D}/\mathcal{Q}$ is a triangulated category and $Q$ is a triangulated functor.
    \item[(\rmnum{3})] For a morphism $f:X\to Y$ in $\mathcal{D}$, we have $Q(f)=0$ if and only if $f$ factorizes through an object of $\mathcal{N}$. In particular, $Q(N)=0$ for $N\in\Ob(\mathcal{N})$.
    \item[(\rmnum{4})] For any pretriangulated category $\mathcal{D}'$ and any triangulated functor $F:\mathcal{D}\to\mathcal{D}'$ such that $F(X)\cong 0$ for any $X\in\mathcal{N}$, $F$ factors uniquely through $Q$.
    \item[(\rmnum{5})] For any abelian category $\mathcal{A}$ and any cohomological functor $H:\mathcal{D}\to\mathcal{A}$, if $H(N)=0$ for $N\in\Ob(\mathcal{N})$, then $H$ factors uniquely through $Q$.
\end{enumerate}
\end{theorem}
\begin{proof}
Since the opposite category of $\mathcal{D}$ is again triangulated and $\mathcal{N}^{\op}$ is a null system in $\mathcal{D}^{\op}$, it is enough to check that $\mathcal{N}Q$ is a right multiplicative system.
\begin{enumerate}[leftmargin=40pt]
    \item[(S1)] If $X\in\Ob(\mathcal{D})$, then $X\stackrel{\id_X}{\longrightarrow} X\to 0\to X[1]$ is distinguished in $\mathcal{D}$ by (TR1), so $\id_X\in\mathcal{N}Q$.
    \item[(S2)] Let $f:X\to Y$ and $g:Y\to Z$ be in $\mathcal{N}Q$. By (TR3), there are distinguished triangles
    \begin{gather*}
    X\stackrel{f}{\to} Y\to Z'\to X[1],\\
    Y\stackrel{g}{\to} Z\to X'\to Y[1],\\
    X\stackrel{gf}{\to} Z\to Y'\to X[1],
    \end{gather*}
    where we can assume that $Z',X'\in\mathcal{N}$. By (TR5), there exists a distinguished triangle $Z'\to Y'\to X'\to Z'[1]$, so $Y'\in\mathcal{N}$ in view of (N3).
    \item[(S3')] Let $f:X\to Y$ and $s:X\to X'$ be two morphisms with $s\in\mathcal{N}Q$. Then there exists a distinguished triangle $W\stackrel{h}{\to} X\stackrel{s}{\to} X'\to W[1]$ with $W\in\mathcal{N}$. By (TR2), there also exists a distinguished triangle $W\stackrel{fh}{\to} Y\stackrel{t}{\to} Z\to W[1]$, and we obtain a commutative diagram in view of (TR4):
    \[\begin{tikzcd}
    W\ar[r,"h"]\ar[d,equal]&X\ar[r,"s"]\ar[d,"f"]&X'\ar[r]\ar[d]&W[1]\ar[d,equal]\\
    W\ar[r,"fh"]&Y\ar[r,"t"]&Z\ar[r]&W[1]
    \end{tikzcd}\]
    Since $W\in\mathcal{N}$, we conclude that $t\in\mathcal{N}Q$.
    \item[(S4')] Replacing $f$ by $f-g$, it is enough to check that if there exists $s\in\mathcal{N}Q$ with $fs=0$, then there exists $t\in\mathcal{N}Q$ with $tf=0$. Consider the solid diagram
    \[\begin{tikzcd}
    X'\ar[r,"s"]&X\ar[r]\ar[rd,"f"]&Z\ar[r]\ar[d,dashed,"h"]&X'[1]\\
    &&Y\ar[d,dashed,"t"]&\\
    &&Y'
    \end{tikzcd}\]
    where the row is a distinguished triangle with $Z\in\mathcal{N}$. Since $fs=0$, the morphism $f$ factors thorugh $Z$ in view of \cref{triangle cat Hom functor cohomological}. There then exists a distinguished triangle $Z\to Y\stackrel{h}{\to} Y'\to Z[1]$ by (TR2), and we obtain that $t\in\mathcal{N}Q$ since $Z\in\mathcal{N}$. Finally, $th=0$ implies that $tf=0$ (cf. \cref{triangle cat dt composition zero}).
\end{enumerate}
It remains to see that $\mathcal{N}Q$ is compatible with distinguished triangles. For this, since $\mathcal{N}$ is closed under $T$, it is easy to see that $\mathcal{N}Q$ satisfies (ST1). Moreover, consider the diagram of (ST2) and assume that $\alpha,\beta\in\mathcal{N}Q$. Then by Verdier's Exercise, we have a commutative diagram
\[\begin{tikzcd}
{}&{}&{}\\
X''\ar[r]\ar[u,"+1"]&Y''\ar[r]\ar[u,"+1"]&Z''\ar[r,"+1"]\ar[u,"+1"]&{}\\
X'\ar[r]\ar[u]&Y'\ar[r]\ar[u]&Z'\ar[r,"+1"]\ar[u]&{}\\
X\ar[r]\ar[u,"\alpha"]&Y\ar[r]\ar[u,"\beta"]&Z\ar[u,"\gamma"]\ar[r,"+1"]&{}
\end{tikzcd}\]
so that every rows and columns are distinguished triangles and every square is commutative. By the saturality of $\mathcal{N}$ and \cref{triangle cat morphism dt unique prop}, we see that $X'',Y''\in\Ob(\mathcal{N})$, so it follows from (N3') that $Z''\in\mathcal{N}$, whence $\gamma\in\mathcal{N}Q$. Now from \cref{triangle cat localization by compatible multiplicative system}, we see that $\mathcal{D}/\mathcal{N}$ has a canonical structure of a triangulated category, and $Q:\mathcal{D}\to\mathcal{D}/\mathcal{N}$ is a triangulated functor.\par
As for (\rmnum{3}), consider a distinguished triangle $0\to N\to N\to 0$, where $N\in\mathcal{N}$. Then the morphism $0\to X$ belongs to $\mathcal{N}Q$, and hence is an isomorphism under $Q$. In particular, if $f:X\to Y$ can be decomposed into $X\to N\to Y$ with $N\in\mathcal{N}$, then $Q(f)=0$. Conversely, if $Q(f)=0$, then there exists a morphism $s:M\to X$ such that $s\in\mathcal{N}Q$ and $fs=0$ (\cref{category localization morphism image equal iff}). From the definition of $\mathcal{N}Q$, we have a solid commutative diagram
\[\begin{tikzcd}
M\ar[r,"s"]\ar[d]&X\ar[r]\ar[d,"f"]&N\ar[r]\ar[d,dashed]&M[1]\ar[d,dashed]\\
0\ar[r]&Y\ar[r,"\id_Y"]&Y\ar[r]&0
\end{tikzcd}\]
By (TR4), there exists a morphism $N\to Y$ giving rise to the commutative diagram, and we then obtain a decomposition $X\to N\to Y$ of $f$.\par
Now let $F:\mathcal{D}\to\mathcal{D}'$ be a triangulated functor, where $\mathcal{D}'$ is a pretriangulated category. Then for $s\in\mathcal{N}Q$, we have a distinguished triangle $X\stackrel{s}{\to} Y\to N\to X[1]$ such that $N\in\Ob(\mathcal{N})$, whence a distinguished triangle $F(X)\stackrel{F(s)}{\to} F(Y)\to 0\to F(X)[1]$ in $\mathcal{D}'$. By \cref{triangle cat zero dt isomorphism}, we conclude that $F(s)$ is an isomorphism, so there is a uniquely determined factorization $F=\widebar{F}\circ Q$, where $\widebar{F}$ is an additive functor. From the description of distinguished triangles in $\mathcal{N}Q$, it is easy to see that $\widebar{F}$ is a triangulated functor.\par
Finally, let $H:\mathcal{D}\to\mathcal{A}$ be a cohomological functor. By considering a distinguished triangle $X\stackrel{s}{\to} Y\to N\to X[1]$ such that $N\in\Ob(\mathcal{N})$, we obtain an exact sequence
\[\begin{tikzcd}
0=H(N[-1])\ar[r]&H(X)\ar[r,"H(s)"]&H(Y)\ar[r]&H(N)=0
\end{tikzcd}\]
so $H(s)$ is an isomorphism and we obtain a uniquely determined factorization $H=\widebar{H}\circ Q$. Similarly, it is immediate to check that $\widebar{H}$ is a cohomological functor.
\end{proof}

Now consider a full triangulated subcategory $\mathcal{I}$ of $\mathcal{D}$. We shall write $\mathcal{N}\cap\mathcal{I}$ for the full subcategory whose objects are $\Ob(\mathcal{N})\cap\Ob(\mathcal{I})$. This is clearly a null system in $\mathcal{I}$.

\begin{proposition}\label{triangle cat localization subcategory functor prop}
Let $\mathcal{D}$ be a triangulated category, $\mathcal{N}$ a null system, $\mathcal{I}$ a full triangulated subcategory of $\mathcal{D}$. Assume that one of the following conditions is true:
\begin{enumerate}
    \item[(a)] any morphism $Y\to Z$ with $Y\in\mathcal{I}$ and $Z\in\mathcal{N}$ factorizes as $Y\to Z'\to Z$ with $Z'\in\mathcal{N}\cap\mathcal{I}$;
    \item[(b)] any morphism $Z\to Y$ with $Y\in\mathcal{I}$ and $Z\in\mathcal{N}$ factorizes as $Z\to Z'\to Y$ with $Z'\in\mathcal{N}\cap\mathcal{I}$.
\end{enumerate}
Then $\mathcal{I}/(\mathcal{N}\cap\mathcal{I})\to\mathcal{D}/\mathcal{N}$ is fully faithful.
\end{proposition}
\begin{proof}
We may assume (b), the case (a) being deduced by considering $\mathcal{D}^{\op}$. We shall apply \cref{category localization of subcategory prop}. Let $f:X\to Y$ is a morphism in $\mathcal{N}Q$ with $X\in\mathcal{I}$, we show that there exists $g:Y\to W$ with $W\in\mathcal{I}$ and $g\circ f\in\mathcal{N}Q$. By definition, the morphism $f$ is embedded in a distinguished triangle $X\to Y\to Z\to X[1]$ with $Z\in\mathcal{N}$, and the hypothesis implies that the morphism $Z\to X[1]$ factorizes through an object $Z'\in\mathcal{N}\cap\mathcal{I}$. We may embed $Z'\to TX$ in a distinguished triangle in $\mathcal{I}$ and obtain a commutative diagram of distinguished triangles by (TR4):
\[\begin{tikzcd}
X\ar[r,"f"]\ar[d,equal]&Y\ar[r]\ar[d,dashed,"g"]&Z\ar[r]\ar[d]&X[1]\ar[d,equal]\\
X\ar[r,"g\circ f"]&W\ar[r]&Z'\ar[r]&X[1]
\end{tikzcd}\]
Since $Z'$ belongs to $\mathcal{N}$, we conclude that $gf\in\mathcal{N}Q\cap\Mor(\mathcal{I})$.
\end{proof}

\begin{proposition}\label{triangle cat localization equivalence if resolution}
Let $\mathcal{D}$ be a triangulated category, $\mathcal{N}$ be a null system, $\mathcal{I}$ be a full triangulated subcategory of $\mathcal{D}$, and assume that one of the following conditions is true:
\begin{enumerate}
    \item[(a)] for any $X\in\Ob(\mathcal{D})$, there exists a morphism $X\to Y$ in $\mathcal{N}Q$ with $Y\in\mathcal{I}$;
    \item[(b)] for any $X\in\Ob(\mathcal{D})$, there exists a morphism $Y\to X$ in $\mathcal{N}Q$ with $Y\in\mathcal{I}$;
\end{enumerate}
Then $\mathcal{I}/(\mathcal{N}\cap\mathcal{I})\to\mathcal{D}/\mathcal{N}$ is an equivalence of categories.
\end{proposition}
\begin{proof}
Apply \cref{category localization of subcategory equivalent if}.
\end{proof}

\begin{example}
Let $H$ be a cohomological functor on $\mathcal{D}$. We define $\mathcal{N}_H$ to be the collection of objects $X\in\mathcal{D}$ such that $H(X[n])=0$ for $n\in\Z$. Then it is easy to verify that $\mathcal{N}_H$ satisfies conditions (N1)--(N3), so we can form the localization $\mathcal{D}/\mathcal{N}_H$.
\end{example}

\begin{proposition}\label{triangle cat localization and direct sum}
Let $\mathcal{D}$ be a triangulated category admitting direct sums indexed by a set $I$ and let $\mathcal{N}$ be a null system closed by such direct sums. Let $Q:\mathcal{D}\to\mathcal{D}/\mathcal{N}$ denote the localization functor. Then $\mathcal{D}/\mathcal{N}$ admits direct sums indexed by $I$ and the localization functor $Q:\mathcal{D} \to\mathcal{D}/\mathcal{N}$ commutes with such direct sums.
\end{proposition}
\begin{proof}
Let $\{X_i\}_{i\in I}$ be a family of objects in $\mathcal{D}$. It is enough to show that $Q(\bigoplus_iX_i)$ is the direct sum of the family $Q(X_i)$, i.e., the map
\[\Hom_{\mathcal{D}/\mathcal{N}}(Q(\bigoplus_iX_i),Y)\to\prod_i\Hom_{\mathcal{D}/\mathcal{N}}(Q(X_i),Y)\]
is bijective for any $Y\in\mathcal{D}$. To this end, we first consider morphisms $u_i\in\Hom_{\mathcal{D}/\mathcal{N}}(Q(X_i),Y)$. Then $u_i$ is represented by a pair $(X'_i;s,u'_i)$, where $u'_i:X'_i\to Y$ is a morphism in $\mathcal{D}$ and we have a distinguished triangle
\[\begin{tikzcd}
X'_i\ar[r,"s"]&X_i\ar[r]&Z_i\ar[r]&X'_i[1]
\end{tikzcd}\]
in $\mathcal{D}$ with $Z_i\in\mathcal{N}$. We then get a morphism $\bigoplus_iX'_i\to Y$ and a distinguished triangle $\bigoplus_iX'_i\to\bigoplus_iX_i\to\bigoplus_iZ_i\to (\bigoplus_iX'_i)[1]$ in $\mathcal{D}$ with $\bigoplus_iZ_i\in\mathcal{N}$.\par
Now assume that the composition $Q(X_i)\to Q(\bigoplus_iX_i)\stackrel{u}{\to} Q(Y)$ is zero for each $i\in I$. By definition, the morphism $u$ is represented by a pair $(Y';s,u')$, where $u':\bigoplus_iX_i\to Y'$ is a morphism in $\mathcal{D}$ and $s:Y\to Y'$ is a morphism in $\mathcal{N}Q$. Using the result of \cref{triangle cat null system associated multiplicative system}(\rmnum{3}), we can find $Z_i\in\mathcal{N}$ such that $u'_i:X_i\to Y'$ factroizes as $X_i\to Z_i\to Y'$. Then the induced morphism $\bigoplus_iX_i\to Y'$ factorizes as $\bigoplus_iX_i\to\bigoplus_iZ_i\to Y'$. Since $\bigoplus_iZ_i\in\mathcal{N}$, we conclude that $Q(u)=0$, whence the proposition.
\end{proof}

\subsection{Localization of triangulated functors}
Let $F:\mathcal{D}\to\mathcal{D}'$ be a functor of triangulated categories, $\mathcal{N}$ and $\mathcal{N}'$ be null systems in $\mathcal{D}$ and $\mathcal{D}'$, respectively. The left (resp. right) localization of $F$ (when it exists) is then defined, by replacing "functor" by "triangulated functor". In the sequel, $\mathcal{D}$ (resp. $\mathcal{D}'$, $\mathcal{D}''$) is a triangulated category and $\mathcal{N}$ (resp. $\mathcal{N}'$, $\mathcal{N}''$) is a null system in this category. We denote by $Q:\mathcal{D}\to\mathcal{D}/\mathcal{N}$ (resp. $Q':\mathcal{D}'\to\mathcal{D}'/\mathcal{N}'$, $Q'':\mathcal{D}''\to\mathcal{D}''/\mathcal{N}''$) the localization functor and by $\mathcal{N}Q$ (resp. $\mathcal{N}'Q$, $\mathcal{N}''Q$) the family of morphisms in $\mathcal{D}$ (resp. $\mathcal{D}'$, $\mathcal{D}''$) defined in (\ref{triangle cat null system associated multiplicative system-1}).

\begin{definition}
A triangulated functor $F:\mathcal{D}\to\mathcal{D}'$ is called \textbf{left (resp. right) localizable with respect to $(\mathcal{N},\mathcal{N}')$} if $Q'\circ F:\mathcal{D}\to\mathcal{D}'/\mathcal{N}'$ is universally left (resp. right) localizable with respect to the multiplicative system $\mathcal{N}Q$. If there is no risk of confusion, we simply say that $F$ is left (resp. right) localizable or that $LF$ (resp. $RF$) exists.
\end{definition}

\begin{definition}
Let $F:\mathcal{D}\to\mathcal{D}'$ be a triangulated functor of triangulated categories, $\mathcal{N}$ and $\mathcal{N}'$ null systems in $\mathcal{D}$ and $\mathcal{D}'$, and $\mathcal{I}$ a full triangulated subcategory of $\mathcal{D}$. Consider the following conditions:
\begin{enumerate}
    \item[(a)] For any $X\in\mathcal{D}$, there exists a morphism $X\to Y$ in $\mathcal{N}Q$ with $Y\in\mathcal{I}$.
    \item[(b)] For any $X\in\mathcal{D}$, there exists a morphism $Y\to X$ in $\mathcal{N}Q$ with $Y\in\mathcal{I}$.
    \item[(c)] For any $Y\in\mathcal{N}\cap\mathcal{I}$, $F(Y)\in\mathcal{N}'$. 
\end{enumerate}
Then if (a) and (b) (resp. (b) and (c)) are satisfied, we say that the subcategory $\mathcal{I}$ is \textbf{$\bm{F}$-injective} (resp. \textbf{$\bm{F}$-projective}) with respect to $\mathcal{N}$ and $\mathcal{N}'$. If there is no risk of confusion, we often omit "with respect to $\mathcal{N}$ and $\mathcal{N}'$".
\end{definition}
Note that if $F(\mathcal{N})\sub\mathcal{N}'$, then $D$ is both $F$-injective and $F$-projective.

\begin{proposition}\label{triangle cat localization of functor exist if}
Let $F:\mathcal{D}\to\mathcal{D}'$ be a triangulated functor of triangulated categories, $\mathcal{N}$ and $\mathcal{N}'$ null systems in $\mathcal{D}$ and $\mathcal{D}'$, and $\mathcal{I}$ a full triangulated category of $\mathcal{D}$.
\begin{enumerate}
    \item[(a)] If $\mathcal{I}$ is $F$-injective with respect to $\mathcal{N}$ and $\mathcal{N}$, then $F$ is right localizable and its right localization is a triangulated functor.
    \item[(b)] If $\mathcal{I}$ is $F$-projective with respect to $\mathcal{N}$ and $\mathcal{N}'$, then $F$ left localizable and
    its left localization is a triangulated functor.
\end{enumerate}
\end{proposition}
\begin{proof}
By \cref{category localization of functor exist if}, the existence of the localizations is clear. To verify that they are triangulated, it suffices to apply \cref{triangle cat null system associated multiplicative system} to check this in $\mathcal{D}$ and $\mathcal{D}'$, and this follows from the hypothesis on $F$.
\end{proof}

We denote by $R_\mathcal{N}^{\mathcal{N}'}F:\mathcal{D}/\mathcal{N}\to \mathcal{D}'/\mathcal{N}'$ (resp. $L_\mathcal{N}^{\mathcal{N}'}F$) the right (resp. left) localization of $F$ with respect to $(\mathcal{N},\mathcal{N}')$. If there is no risk of confusion, we simply write $RF$ (resp. $LF$) instead of $R_\mathcal{N}^{\mathcal{N}'}F$ (resp. $L_\mathcal{N}^{\mathcal{N}'}F$). If $\mathcal{I}$ is $F$-injective, then $RF$ can be defined by the diagram
\[\begin{tikzcd}
&\mathcal{D}\ar[r]&\mathcal{D}/\mathcal{N}\ar[dd,dashed,"R_\mathcal{N}^{\mathcal{N}'}F"]\\
\mathcal{I}\ar[r]\ar[ru]\ar[rrd]&\mathcal{I}/(\mathcal{I}\cap\mathcal{N})\ar[ru,"\sim"]\ar[rd]&\\
&&\mathcal{D}'/\mathcal{N}'
\end{tikzcd}\]
and we have
\begin{equation}\label{triangle cat localization functor expression-1}
R_\mathcal{N}^{\mathcal{N}'}F(X)\cong F(Y)\quad\text{for $(X\to Y)\in\mathcal{N}Q$ with $Y\in\mathcal{I}$}.
\end{equation}
Similarly, if $\mathcal{I}$ is $F$-projective, then the diagram above defines $LF$ and we have
\begin{equation}\label{triangle cat localization functor expression-2}
L_\mathcal{N}^{\mathcal{N}'}F(X)\cong F(Y)\quad\text{for $(Y\to X)\in\mathcal{N}Q$ with $Y\in\mathcal{I}$}.
\end{equation}

\begin{proposition}\label{triangle cat localization functor composition}
Let $F:\mathcal{D}\to\mathcal{D}'$ and $F':\mathcal{D}'\to\mathcal{D}''$ be triangulated functors of triangulated categories and let $\mathcal{N}$, $\mathcal{N}'$ and $\mathcal{N}''$ be null systems in $\mathcal{D}$, $\mathcal{D}'$ and $\mathcal{D}''$, respectively.
\begin{enumerate}
    \item[(a)] Assume that $R_\mathcal{N}^{\mathcal{N}'}F$, $R_\mathcal{N}^{\mathcal{N}'}F$ and $R_\mathcal{N}^{\mathcal{N}'}F$ exist. Then there is a canonical morphism of functors:
    \begin{equation}\label{triangle cat localization functor composition-1}
    R_\mathcal{N}^{\mathcal{N}''}(F'\circ F) \to R_{\mathcal{N}'}^{\mathcal{N}''}F'\circ R_\mathcal{N}^{\mathcal{N}'}F.
    \end{equation}
    \item[(b)] Let $\mathcal{I}$ and $\mathcal{I}'$ be full triangulated subcategories of $\mathcal{D}$ and $\mathcal{D}'$, respectively. Assume that I is F-injective with respect to $\mathcal{N}$ and $\mathcal{N}'$, $\mathcal{I}'$ is $F'$-injective with respect to $\mathcal{N}'$ and $\mathcal{N}''$, and $F(\mathcal{I})\sub\mathcal{I}'$. Then $\mathcal{I}$ is $(F'\circ F)$-injective with respect to $\mathcal{N}$ and $\mathcal{N}''$ and (\ref{triangle cat localization functor composition-1}) is an isomorphism
\end{enumerate}
\end{proposition}
\begin{proof}
By definition, there exists a bijection
\[\Hom(R_\mathcal{N}^{\mathcal{N}''}F'\circ F,R_{\mathcal{N}'}^{\mathcal{N}''}F'\circ R_{\mathcal{N}}^{\mathcal{N}'}F)\cong\Hom(Q''\circ F'\circ F,R_\mathcal{N'}^{\mathcal{N}''}F'\circ R_{\mathcal{N}}^{\mathcal{N}'}F\circ Q),\]
and the natural morphism of functors
\[Q''\circ F'\to R_{\mathcal{N}'}^{\mathcal{N}''}F'\circ Q',\quad Q'\circ F\to R_{\mathcal{N}}^{\mathcal{N}'}F\circ Q.\]
We then deduce the canonical morphisms
\[Q''\circ F'\circ F\to R_{\mathcal{N}'}^{\mathcal{N}''}F'\circ Q'\circ F\to R_{\mathcal{N}'}^{\mathcal{N}''}F'\circ R_{\mathcal{N}}^{\mathcal{N}'}F\circ Q\]
whence the morphism in (a). Now assume the conditions in (b); the fact that $\mathcal{I}$ is $(F'\circ F)$-injective follows immediately from the definition. Let $X\in\mathcal{D}$ and consider a morphism $X\to Y$ in $\mathcal{N}Q$ with $Y\in\mathcal{I}$. Then $R_{\mathcal{N}}^{\mathcal{N}'}F(X)\cong F(Y)$ by (\ref{triangle cat localization functor expression-1}) and $F(Y)\in\mathcal{I}$' by our hypothesis. It then follows from (\ref{triangle cat localization functor expression-1}) that $(R_{\mathcal{N}'}^{\mathcal{N}''}F')(F(Y))\cong F'(F(Y))$, and we conclude that
\[(R_{\mathcal{N}'}^{\mathcal{N}''}F)(R_{\mathcal{N}}^{\mathcal{N}'}F(X))\cong F'(F(Y)).\]
On the other hand, $R_\mathcal{N}^{\mathcal{N}'}(F'\circ F)(X)\cong F'(F(Y))$ by (\ref{triangle cat localization functor expression-1}), since $\mathcal{I}$ is $(F'\circ F)$-injective.
\end{proof}

We now restrict our notations of localizations to triangulated bifunctors. Let $(\mathcal{D},T)$, $(\mathcal{D}',T')$ and $(\mathcal{D}'',T'')$ be triangulated categories. A \textbf{triangulated bifunctor} $F:\mathcal{D}\times\mathcal{D}'\to\mathcal{D}''$ is a bifunctor of additive categories with translations which sends distinguished triangles in each arguments to distinguished triangles.

\begin{definition}
Let $\mathcal{D}$, $\mathcal{D}'$ and $\mathcal{D}''$ be triangulated categories and $\mathcal{N}$, $\mathcal{N}'$ and $\mathcal{N}''$ be null systems in $\mathcal{D}$, $\mathcal{D}'$ and $\mathcal{D}''$, respectively. We say that a triangulated bifunctor $F:\mathcal{D}\times\mathcal{D}'\to\mathcal{D}''$ is \textbf{right (resp. left) localizable with respect to $(\mathcal{N},\mathcal{N},\mathcal{N}'')$} if $Q''\circ F:\mathcal{D}\times\mathcal{D}'\to\mathcal{D}''/\mathcal{N}''$ is universally right (resp. left) localizable with respect to the multiplicative system $\mathcal{N}Q\times\mathcal{N}'Q$. If there is no risk of confusion, we simply say that $F$ is right (resp. left) localizable.\par
If $\mathcal{I},\mathcal{I}'$ are full triangulated subcategories of $\mathcal{D}$ and $\mathcal{D}'$, respectively, then the pair $(\mathcal{I},\mathcal{I}')$ is \textbf{$F$-injective} with respect to $(\mathcal{N},\mathcal{N}',\mathcal{N}'')$ if
\begin{enumerate}
    \item[(\rmnum{1})] $\mathcal{I}'$ is $F(Y,-)$-injective with respect to $\mathcal{N}'$ and $\mathcal{N}''$ for any $Y\in\mathcal{I}$.
    \item[(\rmnum{2})] $\mathcal{I}$ is $F(-,Y')$-injective with respect to $\mathcal{N}$ and $\mathcal{N}''$ for any $Y'\in\mathcal{I}'$.
\end{enumerate}
Equivalently, this amounts to saying that
\begin{enumerate}
    \item[(a)] for any $X\in\mathcal{D}$, there exists a morphism $X\to Y$ in $\mathcal{N}Q$ with $Y\in\mathcal{I}$;
    \item[(b)] for any $X'\in\mathcal{D}'$, there exists a morphism $X'\to Y'$ in $\mathcal{N}'Q$ with $Y'\in\mathcal{I}'$;
    \item[(c)] $F(X,X')$ belongs to $\mathcal{N}''$ for $X\in\mathcal{I}$, $X'\in\mathcal{I}'$ as soon as $X$ belongs to $\mathcal{N}$ or $X'$ belongs to $\mathcal{N}'$.
\end{enumerate}
The property for $(\mathcal{I},\mathcal{I}')$ of being \textbf{$F$-projective} is defined similarly.
\end{definition}

We denote by $R_{\mathcal{N}\times\mathcal{N}'}^{\mathcal{N}''}$ the right localization of $F$ with respect to $(\mathcal{N}\times\mathcal{N}',\mathcal{N}'')$ if it exists. If there is no risk of confusion, we simply write $RF$. We use similar notations for the left localization.

\begin{proposition}\label{triangle cat localization bifunctor exists if}
Let $\mathcal{D}$, $\mathcal{D}'$ and $\mathcal{D}''$ be triangulated categories and $\mathcal{N}$, $\mathcal{N}'$ and $\mathcal{N}''$ be null systems in $\mathcal{D}$, $\mathcal{D}'$ and $\mathcal{D}''$, respectively. Let $F:\mathcal{D}\times\mathcal{D}'\to\mathcal{D}$ be a triangulated bifunctor and $\mathcal{I},\mathcal{I}'$ be full triangulated subcategories of $\mathcal{D}$ and $\mathcal{D}'$ such that $(\mathcal{I},\mathcal{I}')$ is $F$-injective with respect to $(\mathcal{N},\mathcal{N}')$. Then $F$ is right localizable, its right localization $R_{\mathcal{N}\times\mathcal{N}'}^{\mathcal{N}''}F$ is a triangulated bifunctor 
\[R_{\mathcal{N}\times\mathcal{N}'}^{\mathcal{N}''}:\mathcal{D}/\mathcal{N}\times\mathcal{D}'/\mathcal{N}'\to\mathcal{D}''/\mathcal{N}'',\]
and moreover,
\begin{equation}\label{triangle cat localization bifunctor exists if-1}
R_{\mathcal{N}\times\mathcal{N}'}^{\mathcal{N}''}F(X,X')\cong F(Y,Y')
\end{equation}
for $(X\to Y)\in\mathcal{N}Q$ and $(X'\to Y')\in\mathcal{N}'Q$ with $Y\in\mathcal{I}$, $Y'\in\mathcal{I}'$. There exists a similar result by replacing "injective" with "projective" and reversing the arrows.
\end{proposition}
\begin{proof}
By definition, $Q''\circ F$ sends $\mathcal{N}Q\cap\Mor(\mathcal{I})\times(\mathcal{N}'Q\cap\Mor(\mathcal{I}'))$ to isomorphisms in $\mathcal{D}''$, so it follows from \cref{category localization of functor exist if} that the right localization $R_{\mathcal{N}\times\mathcal{N}'}^{\mathcal{N}''}$ exists. The fact that $R_{\mathcal{N}\times\mathcal{N}'}^{\mathcal{N}''}$ is triangulated follows from the hypothsis on $F$, in view of \cref{triangle cat null system associated multiplicative system}. The last equation is a concequence of \cref{triangle cat localization functor composition}.
\end{proof}

\begin{corollary}\label{triangle cat localization bifunctor if exact one variable}
Retain the notations of \cref{triangle cat localization bifunctor exists if} and assume that
\begin{enumerate}
    \item[(a)] $F(\mathcal{I},\mathcal{N}')\sub\mathcal{N}''$;
    \item[(b)] for any $X'\in\mathcal{D}'$, $\mathcal{I}$ is $F(-,X')$-injective with respect to $\mathcal{N}$.
\end{enumerate}
Then $F$ is right localizable and we have
\[R_{\mathcal{N}\times\mathcal{N}'}^{\mathcal{N}''}F(X,X')\cong R_{\mathcal{N}}^{\mathcal{N}''}F(-,X')(X).\]
Again, there is a similar statement by replacing "injective" with "projective".
\end{corollary}
\begin{proof}
Under our hypothesis, for any fixed object $X'\in\mathcal{D}'$, the functor $F(-,X')$ is right localizable, and the last claim follows from (\ref{triangle cat localization functor composition-1}) and (\ref{triangle cat localization bifunctor exists if-1}).
\end{proof}

\section{Derived categories}
In this section, we apply the previous results of triangulated categories on the derived category of an abelian category $\mathcal{A}$, which is defined to be the localization of $K(\mathcal{A})$ with respect to quasi-isomorphisms. Our main refrence will be \cite{kashiwara_SAC}.
\subsection{Derived categories}
Let $(\mathcal{A},T)$ be an abelian category with translation. Recall that the cohomology functor $H:\mathcal{A}_c\to\mathcal{A}$ induces a cohomological functor
\[H:K_c(\mathcal{A})\to\mathcal{A}.\]
Let $\mathcal{N}$ be the full subcategory of $K_c(\mathcal{A})$ consisting of objects $X$ such that $H(X)\cong 0$, that is, $X$ is quasi-isomorphic to $0$. Since $H$ is cohomological, the category $\mathcal{N}$ is a triangulated subcategory of $K_c(\mathcal{A})$. We denote by $D_c(\mathcal{A})$ the category $K_c(\mathcal{A})/\mathcal{N}$, and call it the \textbf{derived category} of $(\mathcal{A},T)$. Note that $D_c(\mathcal{A})$ is triangulated by \cref{*}. By the properties of the localization, a quasi-isomorphism in $K_c(\mathcal{A})$ (or in $\mathcal{A}_c$) becomes an isomorphism in $D_c(\mathcal{A})$. One shall be aware that the category $D_c(\mathcal{A})$ may be a big category.\par
From now on, we shall restrict our study to the case where $\mathcal{A}_c$ is the category of complexes of an abelian category $\mathcal{A}$. Recall that the categories $C^*(\mathcal{A})$ are defined for $\ast\in\{+,-,b,\emp\}$, and we have full subcategories $K^*(\mathcal{A})$ of $K(\mathcal{A})$. For $\ast\in\{+,-,b,\emp\}$, we define
\[N^*(\mathcal{A})=\{X\in K^*(\mathcal{A}):\text{$H^i(X)\cong 0$ for all $i$}\}.\]
Clearly, $N^*(\mathcal{A})$ is a null system in $K^*(\mathcal{C})$.

\begin{definition}
The triangulated categories $D^*(\mathcal{A})$ are defined as $K^*(\mathcal{A})/N^*(\mathcal{A})$ and are called the \textbf{derived categories} of $\mathcal{A}$.
\end{definition}

Recall that to a null system $\mathcal{N}$ we have associated in (\ref{triangle cat null system associated multiplicative system-1}) a multiplicative system denoted by $\mathcal{N}Q$. It will be more intuitive to use here another notation for $\mathcal{N}Q$ when $\mathcal{N}=N(\mathcal{A})$:
\[\Qis=\{f\in\Mor(K(\mathcal{A})):\text{$f$ is a quasi-isomorphism}\}.\]
With this notation, we then have
\begin{align*}
\Hom_{D(\mathcal{A})}(X,Y)&\cong \rlim_{(X'\to X)\in\Qis}\Hom_{K(\mathcal{A})}(X',Y)\cong \rlim_{(Y'\to Y)\in\Qis}\Hom_{K(\mathcal{A})}(X,Y')\\&\cong \rlim_{\substack{(X'\to X)\in\Qis\\(Y\to Y')\in\Qis}}\Hom_{K(\mathcal{A})}(X',Y').
\end{align*}

\begin{remark}
Let $X\in K(\mathcal{A})$, and let $Q(X)$ denote its image in $D(\mathcal{A})$. Then it follows from our definition of $\mathcal{N}(\mathcal{A})$ that $Q(X)=0$ if and only if $H^n(X)=0$ for all $n\in\Z$. Also, if $f:X\to Y$ is a morphism in $\Ch(\mathcal{A})$, then by \cref{triangle cat null system associated multiplicative system}, $f=0$ in $D(\mathcal{A})$ if and only if there exist $X'$ and a quasi-isomorphism $g:X'\to X$ such that $fg$ is homotopic to $0$, or else, if and only if there exist $Y'$ and a quasi-isomorphism $h:Y\to Y'$ such that $hf$ is homotopic to $0$.
\end{remark}

\begin{proposition}\label{derived category of abelian cat prop}
Let $\mathcal{A}$ be an abelian category and $D(\mathcal{A})$ be its derived category.
\begin{enumerate}
    \item[(a)] For $n\in\Z$, the functor $H^n:D(\mathcal{A})\to\mathcal{A}$ is well defined and is a cohomological functor.
    \item[(b)] A morphism $f:X\to Y$ in $D(\mathcal{A})$ is an isomorphism if and only if $H^n(f):H^n(X)\to H^n(Y)$ is an isomorphism for all $n\in\Z$.
    \item[(c)] For $n\in\Z$, the functors $\tau_{\leq n},\tau^{\leq n}:D(\mathcal{A})\to D^-(\mathcal{A})$, as well as the functors $\tau_{\geq n},\tau^{\geq n}:D(\mathcal{A})\to D^+(\mathcal{A})$, are well defined and isomorphic.
    \item[(d)] For $n\in\Z$, the functor $\tau^{\leq n}$ induces a functor $D^+(\mathcal{A})\to D^b(\mathcal{A})$ and $\tau^{\geq n}$ induces a functor $D^-(\mathcal{A})\to D^b(\mathcal{A})$.
\end{enumerate}
\end{proposition}
\begin{proof}
Since $H^n(X)=0$ for $X\in N(\mathcal{A})$, the first assertion is clear, and the second one follows from \cref{triangle cat null system associated multiplicative system} and the definition of $\mathcal{N}Q$ for $\mathcal{N}=N(\mathcal{A})$: in fact, if $Q(f)$ is an isomorphism, then from the following commutative diagram
\[\begin{tikzcd}
Q(X)\ar[r,"Q(f)"]\ar[d,"Q(f)"]&Q(Y)\ar[r]\ar[d,equal]&Q(M(f))\ar[d]\ar[r]&X[1]\ar[d]\\
Q(Y)\ar[r,equal]&Q(Y)\ar[r]&0\ar[r]&Q(Y)[1]
\end{tikzcd}\]
we conclude that $Q(M(f))\to 0$ is an isomorphism (\cref{triangle cat morphism dt isomorphism 2 of 3}), so $H^n(f)$ is an isomorphism for each $n\in\Z$.\par
Now if $f:X\to Y$ is a quasi-isomorphism in $K(\mathcal{A})$, then $\tau^{\leq n}(f)$ and $\tau^{\geq n}(f)$ are quasi-isomorphism. Moreover, for $X\in K(\mathcal{A})$, the morphisms $\tau^{\leq n}(X)\to\tau_{\leq n}(X)$ and $\tau^{\geq n}(X)\to\tau_{\geq n}(X)$ are also quasi-isomorphism (\cref{*}), so assertion (c) follows from (d), and (d) is then obvious.
\end{proof}

To a distinguished triangle $X\stackrel{f}{\to} Y\stackrel{g}{\to} Z\to X[1]$ in $\mathcal{D}(\mathcal{A})$, the cohomological functor $H^0$ associates a long exact sequence in $\mathcal{A}$:
\[\begin{tikzcd}
\cdots\ar[r]&H^i(X)\ar[r]&H^i(Y)\ar[r]&H^i(Z)\ar[r]&H^{i+1}(X)\ar[r]&\cdots
\end{tikzcd}\]
For $X\in K(\mathcal{A})$, recall that the categories $\Qis_{/X}$ and $\Qis_{X/}$ are filtrant (or cofiltrant) categories of $K(\mathcal{C})_{/X}$ and $K(\mathcal{A})_{X/}$, respectively. If $\mathcal{J}$ is a subcategory of $K(\mathcal{C})_{/X}$, we denote by $\Qis_{/X}\cap\mathcal{J}$ the full subcategory of $\Qis_{/X}$ consisting of objects which belong to $\mathcal{J}$. We use similar notations for $\Qis_{X/}$ and $K(\mathcal{C})_{X/}$.

\begin{lemma}\label{derived category Qis cofinal truncation}
Let $\mathcal{A}$ be an abelian category and $n$ be an integer.
\begin{enumerate}
    \item[(a)] For $X\in K^{\leq n}(\mathcal{A})$, the categories $\Qis_{/X}\cap K^{\leq n}(\mathcal{A})_{/X}$ and $\Qis_{/X}\cap K^{-}(\mathcal{A})_{/X}$ are co-cofinal to $\Qis_{/X}$.
    \item[(b)] For $X\in K^{\geq n}(\mathcal{A})$, the categories $\Qis_{X/}\cap K^{\geq n}(\mathcal{A})_{X/}$ and $\Qis_{X/}\cap K^{+}(\mathcal{A})_{X/}$ are co-cofinal to $\Qis_{X/}$.
\end{enumerate}
\end{lemma}
\begin{proof}
The two statements are equivalent by reversing the arrows, so we only prove (b). The category $\Qis_{X/}\cap K^{\geq n}(\mathcal{A})_{X/}$ is a full subcategory of a filtrant category $\Qis_{X/}$, and for any object $(X\to Y)$ in $\Qis_{X/}$, there exists a canonical morphism $(X\to Y)\to(X\to\tau^{\geq n}Y)$.
\end{proof}

\begin{proposition}\label{derived category truncation Hom of leqgeq prop}
Let $n\in\Z$ and $X\in K^{\leq n}(\mathcal{A})$, $Y\in K^{\geq n}(\mathcal{A})$. Then we have
\begin{equation}
\Hom_{D(\mathcal{A})}(X,Y)\cong \Hom_{\mathcal{C}}(H^n(X),H^n(Y))
\end{equation}
\end{proposition}
\begin{proof}
The map $\Hom_{\Ch(\mathcal{A})}(X,Y)\to\Hom_{K(\mathcal{A})}(X,Y)$ is an isomorphism by our hypothesis and
\begin{align*}
\Hom_{\Ch(\mathcal{A})}(X,Y)&\cong\{f\in\Hom_{\mathcal{A}}(X^n,Y^n):u\circ d_X^{n-1}=0,d_Y^n\circ f=0\}\\
&\cong \Hom_\mathcal{A}(\coker d_X^{n-1},\ker d_Y^n)\cong\Hom_\mathcal{A}(H^n(X),H^n(Y))
\end{align*}
so we conclude that $\Hom_{K(\mathcal{A})}(X,Y)\cong\Hom_\mathcal{A}(H^n(X),H^n(Y))$. On the other hand, in view of \cref{derived category Qis cofinal truncation}, we have
\begin{align*}
\Hom_{D(\mathcal{A})}(X,Y)\cong\rlim_{(Y\to Y')\in\Qis\cap K^{\geq n}(\mathcal{A})}\Hom_{K(\mathcal{A})}(X,Y')\cong\Hom_{\mathcal{A}}(H^n(X),H^n(Y))
\end{align*}
so the proposition follows.
\end{proof}

For $-\infty\leq a\leq b\leq+\infty$, we denote by $D^{[a,b]}(\mathcal{A})$ the full subcategory of $D(\mathcal{A})$ consisting of objects $X$ satisfying $H^i(X)=0$ for $i\notin[a,b]$. With this notation, we set $D^{\leq a}(\mathcal{A}):=D^{[-\infty,a]}(\mathcal{A})$ and $D^{\geq a}(\mathcal{A}):=D^{[a,+\infty]}(\mathcal{A})$.

\begin{proposition}\label{derived category truncation subcategory prop}
Let $\mathcal{A}$ be an abelian category.
\begin{enumerate}
    \item[(a)] For $\ast\in\{+,-,b\}$, the triangulated category $D^*(\mathcal{A})$ is equivalent to the full triangulated subcategory of $\mathcal{D}(\mathcal{A})$ consisting of objects $X$ satisfying $H^i(X)=0$ for $i\ll 0$ (resp. $i\gg 0$, resp. $|i|\gg 0$).
    \item[(b)] For $-\infty\leq a\leq b\leq+\infty$, the canonical functor $Q:K^{[a,b]}(\mathcal{A})\to D^{[a,b]}(\mathcal{A})$ is essentially surjective.
    \item[(c)] The category $\mathcal{A}$ is equivalent to the full subcategory $D^{\leq 0}(\mathcal{A})\cap D^{\geq 0}(\mathcal{A})$.
    \item[(d)] For any $n\in\Z$ and $X,Y\in D(\mathcal{C})$, we have 
    \[\Hom_{D(\mathcal{A})}(\tau^{\leq n}X,\tau^{\geq n}Y)\cong\Hom_{\mathcal{A}}(H^n(X),H^n(Y)).\]
    In particular, $\Hom_{D(\mathcal{A})}(\tau^{\leq n}X,\tau^{\geq n+1}Y)=0$. 
\end{enumerate}
\end{proposition}
\begin{proof}
As for assertion (a), let us treat the case $*=+$, the other cases being similar. For $Y\in K^{\geq a}(\mathcal{A})$ and $Z\in N(\mathcal{A})$, any morphism $Z\to Y$ in $K(\mathcal{A})$ factors through $\tau^{\geq n}Z\in N(\mathcal{A})\cap K^{\geq n}(\mathcal{A})$. Applying \cref{category localization of subcategory prop}, we find that the natural functor $D^+(\mathcal{A})\to D(\mathcal{A})$ is fully faithful, and it is clear that if $Y\in D(\mathcal{A})$ belongs to the image of the functor $D^+(\mathcal{A})\to D(\mathcal{A})$, then $H^i(X)=0$ for $i\ll 0$. Conversely, let $X\in K(\mathcal{A})$ with $H^i(X)=0$ for $i<a$. Then $\tau^{\geq a}X\in K^+(\mathcal{A})$ and the morphism $X\to\tau^{\geq a}X$ in $K(\mathcal{A})$ is a quasi-isomorphism, whence an isomorphism in $D(\mathcal{A})$. We therefore conclude (a), and the proof of (b) can be done similarly. Finally, by \cref{derived category truncation Hom of leqgeq prop}, the functor $\mathcal{A}\to D(\mathcal{A})$ is fully faithful and essentially surjective by (b); this proves (c), and (d) follows from (b) and \cref{derived category truncation Hom of leqgeq prop}.
\end{proof}

\begin{proposition}\label{derived category exact sequence is dt}
Let $0\to X\stackrel{f}{\to} Y\stackrel{g}{\to} Z\to 0$ be an exact sequence in $\Ch(\mathcal{A})$. Then there exists a distinguished triangle $X\stackrel{f}{\to} Y\stackrel{g}{\to} Z\to X[1]$ and $Z$ is isomorphic to $M(f)$ in $D(\mathcal{A})$.
\end{proposition}
\begin{proof}
We define a morphism $\varphi:M(f)\to Z$ in $\Ch(\mathcal{A})$ by $\varphi^n=(0,g^n)$. By \cref{*}, $\varphi$ is then a quasi-isomorphism, whence an isomorphism in $D(\mathcal{A})$.
\end{proof}

\begin{remark}
Let $0\to X\to Y\to Z\to 0$ be an exact sequence in $\mathcal{A}$. By \cref{derived category exact sequence is dt}, we then get a morphism $\gamma:Z\to X[1]$ in $D(\mathcal{A})$. The morphism $H^i(\gamma):H^i(Z)\to H^{i+1}(X)$ is zero for all $i\in\Z$, although $\gamma$ is not the zero morphism in $D(\mathcal{A})$ in general (this happens only if the short exact sequence splits). The morphism $\gamma$ may be described in $K(\mathcal{A})$ by the morphisms with $\varphi$ a quasi-isomorphism:
\[X[1]\stackrel{\beta(f)}{\longleftarrow} M(f)\stackrel{\varphi}{\longrightarrow} Z.\]
\end{remark}

\begin{proposition}\label{derived category dt for truncation functor}
If $X\in D(\mathcal{A})$, there are distinguished triangles in $D(\mathcal{A})$:
\begin{gather}
\tau^{\leq n}X \longrightarrow X \longrightarrow \tau^{\geq n+1}X \stackrel{+1}{\longrightarrow},\label{derived category dt for truncation functor-1}\\
\tau^{\leq n-1}X \longrightarrow \tau^{\leq n}X \longrightarrow H^n(X)[-n] \stackrel{+1}{\longrightarrow},\label{derived category dt for truncation functor-2}\\
H^n(X)[-n] \longrightarrow \tau^{\geq n}X \longrightarrow \tau^{\geq n+1}X \stackrel{+1}{\longrightarrow},\label{derived category dt for truncation functor-3}
\end{gather}
Moreover, we have canonical isomorphisms
\begin{equation}\label{derived category dt for truncation functor-4}
H^n(X)[-n]\cong\tau^{\leq n}\tau^{\geq n}(X)\cong \tau^{\geq n}\tau^{\leq n}X.
\end{equation}
\end{proposition}
\begin{proof}
This is a direct concequence of \cref{*} and \cref{*}.
\end{proof}

\begin{proposition}\label{derived category truncation functor adjoint prop}
The functor $\tau^{\leq n}:D(\mathcal{A})\to D^{\leq n}(\mathcal{A})$ is a right adjoint to the natural inclusion $D^{\leq n}(\mathcal{A})\to D(\mathcal{A})$ and $\tau^{\geq n}:D(\mathcal{A})\to D^{\geq n}(\mathcal{A})$ is a left adjoint to the natural functor $D^{\geq n}(\mathcal{A})\to D(\mathcal{A})$. In other words, there are functorial isomorphisms
\begin{gather*}
\Hom_{D(\mathcal{A})}(X,Y)\cong \Hom_{D^{\leq n}(\mathcal{A})}(X,\tau^{\leq n}Y)\for X\in D^{\leq n}(\mathcal{A}), Y\in D(\mathcal{A}),\\
\Hom_{D(\mathcal{A})}(X,Y)\cong \Hom_{D^{\geq n}(\mathcal{A})}(\tau^{\geq n}X,Y)\for X\in D(\mathcal{A}), Y^{\geq n}\in D(\mathcal{A}).
\end{gather*}
\end{proposition}
\begin{proof}
Let $X\in D^{\leq n}(\mathcal{A})$, then by the distinguished triangle (\ref{derived category dt for truncation functor-1}) for $Y$, we have an exact sequence
\begin{small}
\begin{equation}\label{derived category truncation functor adjoint prop-1}
\begin{tikzcd}[column sep=5mm]
\Hom_{D(\mathcal{A})}(X,\tau^{\geq n+1}Y[-1])\ar[r]&\Hom_{D(\mathcal{A})}(X,\tau^{\leq n}Y)\ar[r]&\Hom_{D(\mathcal{A})}(X,Y)\ar[r]&\Hom_{D(\mathcal{A})}(X,\tau^{\geq n+1}Y)
\end{tikzcd}
\end{equation}
\end{small}
Since $\tau^{\geq n+1}Y[-1]$ and $\tau^{\geq n+1}Y$ belong to $D^{\geq n+1}(\mathcal{A})$, the first and fourth terms of (\ref{derived category truncation functor adjoint prop-1}) are zero by \cref{derived category truncation subcategory prop}~(d). The second isomorphism follows by reversing the arrows.
\end{proof}

\subsection{Resolutions}
The derived category $D^*(\mathcal{A})$ is often a big category and this causes many problems. In this paragraph, by considering resolutions in the category $\Ch(\mathcal{A})$, we show that in some case $D^*(\mathcal{A})$ is equivalent to the homotopy category of a subcategory of $\mathcal{A}$, and hence a $\mathscr{U}$-category, where $\mathscr{U}$ is the chosen universe.

\begin{lemma}\label{derived category resolution by subcat if}
Let $\mathcal{J}$ be a full additive subcategory of $\mathcal{A}$ and $X^\bullet\in \Ch^{\geq n}(\mathcal{A})$ for some $n\in\Z$. Assume that one of the following conditions holds:
\begin{enumerate}
    \item[(a)] $\mathcal{J}$ is cogenerating in $\mathcal{A}$ (i.e. for any $Y\in\mathcal{A}$ there exists a monomorphism $Y\to I$ with $I\in\mathcal{J}$);
    \item[(b)] $\mathcal{J}$ is closed under extensions and cokernels of monomorphisms, and for any monomorphism $I'\to Y$ in $\mathcal{A}$ with $I'\in\mathcal{J}$, there exists a morphism $Y\to I$ with $I\in\mathcal{J}$ such that the composition $I'\to I$ is a monomorphism. Moreover, $H^i(X^\bullet)\in\mathcal{J}$ for all $i\in\Z$.
\end{enumerate}
Then there exists $Y^\bullet\in\Ch^{\geq n}(\mathcal{J})$ and a quasi-isomorphism $X^\bullet\to Y^\bullet$.
\end{lemma}
\begin{proof}

\end{proof}

Let $\mathcal{J}$ be a full additive subcategory of $\mathcal{A}$. It is clear that for $*\in\{+,-,b,\emp\}$, the $N^*(\mathcal{J}):=N(\mathcal{A})\cap K^*(\mathcal{J})$ is a null system in $K^*(\mathcal{J})$. We say that $\mathcal{A}$ has \textbf{finite $\mathcal{J}$-dimension} if there exists a non-negative integer $d$ such that, for any exact sequence
\[\begin{tikzcd}
Y_d\ar[r]&\cdots\ar[r]&Y_1\ar[r]&Y\ar[r]&0
\end{tikzcd}\]
with $Y_i\in\mathcal{J}$ for $1\leq i\leq d$, we have $Y\in\mathcal{J}$.

\begin{proposition}\label{derived category resolution of cogenerating prop}
Assume that $\mathcal{J}$ is cogenerating in $\mathcal{C}$, then the natural functor
\[\theta^+:K^+(\mathcal{J})/N^+(\mathcal{J})\to D^+(\mathcal{C})\]
is an equivalence of categories. If $\mathcal{A}$ has finite $\mathcal{J}$-dimension, then $\theta^b:K^b(\mathcal{J})/N^b(\mathcal{J})\to D^b(\mathcal{C})$ is also an equivalence of categories.
\end{proposition}
\begin{proof}
Let $X\in K^+(\mathcal{A})$. By \cref{derived category resolution by subcat if}, there exists $Y\in K^+(\mathcal{J})$ and a quasi-isomorphism $X\to Y$, so the first assertion follows from \cref{triangle cat localization subcategory functor prop}. Now assume that $\mathcal{A}$ has finite $\mathcal{J}$-dimension and that $X^i=0$ for $i\geq n$, where $n\in\Z$. Then $\tau^{\leq i}Y\to Y$ is a quasi-isomorphism for $i\geq n$ and the hypothesis implies that $\tau^{\leq i}Y$ belongs to $K^b(\mathcal{J})$ for $i>n+d$. This proves the second assertion in view of \cref{triangle cat localization subcategory functor prop}.
\end{proof}

Let us apply the preceding proposition to the full subcategory of injective objects: $\mathcal{I}_\mathcal{A}=\{X\in\mathcal{A}:\text{$X$ is injective}\}$.
\begin{proposition}\label{derived category enough injective equivalence}
Assume that $\mathcal{A}$ admits enough injectives. Then the functor $K^+(\mathcal{I}_\mathcal{A})\to D^+(\mathcal{A})$ is an equivalence of categories. If moreover $\mathcal{A}$ has finite injective dimension, then $K^b(\mathcal{I}_\mathcal{A})\to D^b(\mathcal{A})$ is an equivalence of categories.
\end{proposition}
\begin{proof}
By \cref{derived category resolution of cogenerating prop}, it is enough to prove that if $X^\bullet\in\Ch^+(\mathcal{I}_\mathcal{A})$ is quasi-isomorphic to $0$, then $X^\bullet$ is homotopic to $0$. This is a particular case of the lemma below (choose $f=\id_{X^\bullet}$ in the lemma).
\end{proof}

\begin{lemma}\label{abelian cat injective qis zero is null-homotopy}
Let $f:X^\bullet\to I^\bullet$ be a morphism in $\Ch(\mathcal{A})$. Assume that $I^\bullet$ belongs to $\Ch^+(\mathcal{I}_\mathcal{A})$ and $X^\bullet$ is exact. Then $f$ is homotopic to $0$.
\end{lemma}


\begin{corollary}\label{derived category enough injective U-cat}
Let $\mathcal{A}$ be an abelian $\mathscr{U}$-category with enough injectives. Then $D^+(\mathcal{A})$ is a $\mathscr{U}$-category.
\end{corollary}

\begin{proposition}
Let $\mathcal{J}$ be a full additive subcategory of $\mathcal{A}$ and assume that $\mathcal{J}$ is cogenerating and $\mathcal{A}$ has finite $\mathcal{J}$-dimension. Then for any $X\in\Ch(\mathcal{A})$, there exists $Y\in\Ch(\mathcal{J})$ and a quasi-isomorphism $X\to Y$. In particular, there is an equivalence of triangulated categories $K(\mathcal{J})/N(\mathcal{J})\stackrel{\sim}{\to} D(\mathcal{A})$.
\end{proposition}

An important class of examples are given by Serre subcategories of $\mathcal{A}$: let $\mathcal{T}$ be a weak Serre subcategory of $\mathcal{A}$. For $*\in\{+,-,b,\emp\}$, we denote by $D^*_\mathcal{T}(\mathcal{A})$ the full additive subcategory of $D^*(\mathcal{A})$ consisting of objects $X$ such that $H^i(X)\in\mathcal{T}$ for all $i\in\Z$. This is clearly a triangulated subcategory of $D(\mathcal{A})$, and there is a natural functor
\begin{equation}\label{derived category truncation by Serre subcat prop-1}
\delta^*:D^*(\mathcal{T})\to D^*_\mathcal{T}(\mathcal{A}).
\end{equation}

\begin{theorem}\label{derived category truncation by Serre subcat prop}
Let $\mathcal{T}$ be a Serre subcategory of $\mathcal{A}$ and assume that for any monomorphism $Y\to X$, with $Y\in\mathcal{T}$, there exists a morphism $X\to Y'$ with $Y'\in\mathcal{T}$ such that the composition $Y\to Y'$ is a monomorphism. Then the functors $\delta^+$ and $\delta^b$ in (\ref{derived category truncation by Serre subcat prop-1}) are equivalences of categories.
\end{theorem}
\begin{proof}
The result for $\delta^+$ is an immediate consequence of \cref{category localization of subcategory equivalent if} and \cref{derived category resolution by subcat if}~(b). The case of $\delta^b$ follows since $D^b(\mathcal{T})$ is equivalent to the full subcategory of $D^+(\mathcal{T})$ of objects with bounded cohomology, and similarly for $D^b_\mathcal{T}(\mathcal{A})$.
\end{proof}
Note that, by reversing the arrows in \cref{derived category truncation by Serre subcat prop}, the functors $\delta^-$ and $\delta^b$ in (\ref{derived category truncation by Serre subcat prop-1}) are equivalences of categories if for any epimorphism $X\to Y$ with $Y\in\mathcal{T}$, there exists a morphism $Y'\to X$ with $Y'\in\mathcal{T}$ such that the composition $Y'\to Y$ is an epimorphism.
\subsection{Bounded functors and the way-out lemma}
We now introduce an important result on how a triangulated functor on derived categories is determined by its values on the underlying abelian category. This is useful when one want to show that some natural map is a functorial isomorphism.\par
In this paragraph, we consider abelian categories $\mathcal{A}$ and $\mathcal{A}'$, and additive functors between subcategories of $D(\mathcal{A})$ and $D(\mathcal{A}')$. We choose a weak Serre subcategory $\mathcal{T}$ of $\mathcal{A}$ and denote by $D_\mathcal{T}^*(\mathcal{A})$ the full additive subcategory of $D^*(\mathcal{A})$ consisting of objects $X$ such that $H^i(X)\in\mathcal{T}$ for all $i\in\Z$. If $\mathcal{E}$ is a subcategory of $D(\mathcal{A})$, we write $\mathcal{E}^{\geq n}$ (resp. $\mathcal{E}^{\geq n}$) for the subcategories of $\mathcal{E}$ whose objects are complexes $X$ such that $H^i(X)=0$ for $i<n$ (resp. $i>n$).
\begin{definition}
Let $\mathcal{E}$ be a subcategory of $D(\mathcal{A})$ and let $F:\mathcal{E}\to D(\mathcal{A}')$ be an additive functor. The \textbf{upper dimension} $\dim^+$ and \textbf{lower dimension} $\dim^-$ of the functor $F$ are defined by
\begin{gather*}
\dim^+(F):=\inf\{d\in\Z:\text{$F(\mathcal{E}^{\leq n})\sub D^{\leq n+d}(\mathcal{A}')$ for all $n\in\Z$}\},\\
\dim^-(F):=\inf\{d\in\Z:\text{$F(\mathcal{E}^{\geq n})\sub D^{\geq n-d}(\mathcal{A}')$ for all $n\in\Z$}\}.
\end{gather*}
The functor $F$ is called \textbf{bounded above} (resp. \textbf{bounded below}) if $\dim^+(F)<+\infty$ (resp. $\dim^-(F)<+\infty$), and \textbf{bounded} if it is both bounded above and bounded below.
\end{definition}

\begin{remark}
If $F:\mathcal{E}\to D(\mathcal{A}')$ is compatible with the translation functors of $D(\mathcal{A})$ and $D(\mathcal{A}')$, then we see that $F(\mathcal{E}^{\geq n})\sub D^{\geq n+d}(\mathcal{A}')$ holds for all $n\in\Z$ as soon as it holds for one single $n$, for example $n=0$. Therefore, in this case we can also define $\dim^+(F)$ to be the smallest integer $d$ such that $F(\mathcal{E}^{\leq 0})\sub D^{\leq d}(\mathcal{A}')$, and similarly for $\dim^-(F)$.
\end{remark}

\begin{example}\label{derived category functor dim leq d iff truncation quasi-isomorphism}
If $\mathcal{E}$ is a triangulated subcategory of $D(\mathcal{A})$ such that $\tau^{\geq n}(\mathcal{E})\sub\mathcal{E}$ and $\tau^{\leq n}(\mathcal{E})\sub\mathcal{E}$ (for example, if $\mathcal{E}=D_\mathcal{T}^*(\mathcal{A})$), and if $F$ is a triangulated functor, then $\dim^+(F)\leq d$ if and only if for any $X\in\mathcal{E}$, $n\in\Z$, and $i\geq n+d$, the canonical morphism
\[H^i(F(X))\to H^iF(\tau^{\geq n}X)\]
is an isomorphism. In fact, the implication $\Rightarrow$ follows from the exact sequence induced from the distinguished triangle (\ref{derived category dt for truncation functor-1}), since we have $H^i(F(\tau^{n-1}X))=0$ in this case. The converse implication is obtained by taking $X$ to be an arbitray complex in $\mathcal{E}^{\leq n-1}$. An equivalent condition is that if $f:X\to Y$ is a morphism in $\mathcal{E}$ such that $H^i(f)$ is an isomorphism for all $i\geq n$, (that is, if $f$ induces an isomorphism $\tau^{\geq n}X\to \tau^{\geq n}Y$), then $H^i(F(f))$ is an isomorphism for all $i\geq n+d$. Similarly, we have $\dim^-(F)\leq d$ if and only if the canonical morphism
\[H^i(F(\tau^{\leq n}X))\to H^iF(X)\]
is an isomorphism.\par
In particular, if $\mathcal{E}=\mathcal{T}$ is a weak Serre subcategory of $\mathcal{A}$ (also considered as a subcategory of $D(\mathcal{A})$), then $\mathcal{E}^{\geq 0}=\mathcal{E}=\mathcal{E}^{\leq 0}$, and we have
\begin{gather*}
\dim^+(F)\leq d\Leftrightarrow \text{$H^i(F(X))=0$ for all $i>d$ and $X\in\mathcal{T}$},\\
\dim^-(F)\leq d\Leftrightarrow \text{$H^i(F(X))=0$ for all $i<-d$ and $X\in\mathcal{T}$}.
\end{gather*}
\end{example}

\begin{proposition}\label{derived category functor dim on Serre subcat prop}
If $\mathcal{E}=D^*_\mathcal{T}(\mathcal{A})$ and $F$ is a triangulated functor, then
\[\dim^+(F)=\dim^+(F_0),\quad \dim^-(F)=\dim^-(F_0),\]
where $F_0$ is the restriction of $F$ to $\mathcal{T}$.
\end{proposition}
\begin{proof}
We deal with the case for $\dim^+(F)$, the case for $\dim^-(F)$ can be done similarly. First, we note that $\dim^+(F_0)\leq\dim^+(F)$ since $F'(\mathcal{E}^{\leq n})\sub F'(\mathcal{E}^{\leq n})$ for each $n\in\Z$. For the reverse inequality, we assume that $\dim^+(F_0)\leq d<+\infty$ and fix an integer $n\in\Z$. We prove that $H^i(F(X))=0$ for any $X\in\mathcal{E}^{\leq n}$ and $i>n+d$ by induction on the number $\nu=\nu(X)$ of non-vanishing cohomology objects of $X$. Since the case $\nu=0$ is trivial, we may assume that $\nu\geq 1$. If $\nu=1$, say $H=H^m(X)\neq 0$ for some $m\leq n$, and we have
\[X\cong\tau^{\leq m}\tau^{\geq m}X\cong H[-m]\]
by (\ref{derived category dt for truncation functor-4}). Since $F$ is a triangulated functor, we conclude that $F(X)\cong F(H)[-m]$, so by definition of $\dim^+(F_0)$,
\[H^i(F(X))\cong H^{i-m}(F(H))=H^{i-m}(F_0(H))=0\for i-m>d\]
whence the conclusion. If $\nu>1$, we choose an integer $s$ such that there exists integers $p<s\leq q$ with $H^p(X)\neq 0$ and $H^q(X)\neq 0$. Then $\nu(\tau^{\leq s-1}X)<\nu(X)$ and $\nu(\tau^{\geq s}X)<\nu(X)$, so by applying the induction hypothesis, we have
\[H^i(F(\tau^{\leq s-1}X))=H^i(\tau^{\geq s}X)=0\for i>n+d.\]
The inductive step then follows from the long exact sequence induced by the distinguished triangle (\ref{derived category dt for truncation functor-1}).
\end{proof}

\begin{proposition}[\textbf{Way-out Lemma}]\label{derived category way-out lemma}
Let $\mathcal{T}$ be a weak Serre subcategory of $\mathcal{A}$ and $*\in\{+,b,\emp\}$. Consider triangulated functors $F,G:D^*_\mathcal{T}(\mathcal{A})\to D(\mathcal{A}')$ and a morphism of functors $\eta:F\to G$, so that $\eta(X):F(X)\to G(X)$ is an isomorphism for any $X\in\mathcal{T}$. If one of the following conditions holds, then $\eta$ is an isomorphism:
\begin{enumerate}
    \item[(\rmnum{1})] $*=b$;
    \item[(\rmnum{2})] $*=+$ and $F,G$ are bounded below;
    \item[(\rmnum{3})] $*=-$ and $F,G$ are bounded above;
    \item[(\rmnum{4})] $*=\emp$ and $F,G$ are bounded.
\end{enumerate}
Moreover, if $\mathfrak{I}$ (resp. $\mathfrak{P}$) is a subset of $\Ob(\mathcal{T})$ such that for any $X\in\mathcal{T}$ there exists a monomorphism $X\hookrightarrow I$ with $I\in\mathfrak{I}$ (resp. an epimorphism $P\to X$ with $P\in\mathfrak{P}$), then $\eta$ is an isomorphism if $\eta(X):F(X)\to G(X)$ is an isomorphism for each $X\in\mathfrak{I}$, and one of conditions (\rmnum{1}), (\rmnum{2}) (resp. (\rmnum{1}), (\rmnum{3})) is satisfied.
\end{proposition}
\begin{proof}
We first deal with the case $*=b$. Since $\eta$ is a morphism of triangulated functors, we see by induction on $|n|$ that $\eta(X[n])$ is an isomorphism for each $X\in\mathcal{T}$ and $n\in\Z$. To see that $\eta(X)$ is an isomorphism for any $X\in D_\mathcal{T}^*(\mathcal{A})$, we may replace $X$ with the isomorphic complex $\tau^{\leq n}(X)$ with some integer $n$ large enough. From (\ref{derived category dt for truncation functor-2}), we obtain a morphism of triangles, induced by $\eta$:
\[\begin{tikzcd}[column sep=5mm]
F(H^n(X)[-n-1])\ar[r]\ar[d]&F(\tau^{\leq n-1}X)\ar[r]\ar[d]&F(\tau^{\leq n}X)\ar[r]\ar[d]&F(H^n(X)[-n])\ar[d]\\
G(H^n(X)[-n-1])\ar[r]&G(\tau^{\leq n-1}X)\ar[r]&G(\tau^{\leq n}X)\ar[r]&G(H^n(X)[-n])
\end{tikzcd}\]
and then we can conclude the proposition by \cref{triangle cat morphism dt isomorphism 2 of 3} and induction on the number of non-vanishing cohomology objects of $X$ (a number which is less for $\tau^{\leq n-1}X$ than for $X$ whenever $n$ is finite).\par
As for the case of (\rmnum{2}), by \cref{derived category of abelian cat prop}, it suffices to show that $\eta(X)$ induces an isomorphism from $H^i(F(X))$ to $H^i(G(X))$ for any $X\in D_\mathcal{T}^+(\mathcal{A})$ and all $i\in\Z$. For this, we may apply \cref{derived category functor dim leq d iff truncation quasi-isomorphism} to replace $X$ by $\tau^{\leq i+d}X\in D^b_\mathcal{T}(\mathcal{A})$ for $d\geq\max\{\dim^-(F),\dim^-(G)\}$, and then we can apply the conclusion of (\rmnum{1}). The case for (\rmnum{3}) can be proved similarly.\par
We now consider the case where $*=\emp$. In view of (\ref{derived category dt for truncation functor-1}), we have a morphism of triangles, induced by $\eta$:
\[\begin{tikzcd}[column sep=5mm]
F(\tau^{\geq 0}(X)[-1])\ar[r]\ar[d]&F(\tau^{\leq-1}X)\ar[r]\ar[d]&F(X)\ar[r]\ar[d]&F(\tau^{\geq 0}(X))\ar[d]\\
G(\tau^{\geq 0}(X)[-1])\ar[r]&G(\tau^{\leq-1}X)\ar[r]&G(X)\ar[r]&G(\tau^{\geq 0}(X))
\end{tikzcd}\]
and by induction on the number of non-vanishing cohomology objects, we may assume that the vertical morphisms, except $F(X)\to G(X)$, are all isomorphisms. It then follows from \cref{triangle cat morphism dt isomorphism 2 of 3} that $F(X)\to G(X)$ is also an isomorphism.\par
Finally, as for the last assertion (we consider the case for $\mathfrak{I}$, the other case can be proved similarly), it suffices to show that $\eta(X)$ is an isomorphism for any $X\in\mathcal{T}$. We take an exact sequence $0\to X\to I^0\to I^1\to\cdots$ so that each $I^i\in\mathfrak{I}$. Then gives rise to a quasi-isomorphism $X\to I$, so it remains to show that $\eta(I)$ is an isomorphism. For this, we consider the stupid truncations $\sigma^{\leq n}I$ and $\sigma^{\geq n+1}I$, which fit into an exact sequence
\[\begin{tikzcd}
0\ar[r]&\sigma^{\geq n+1}\ar[r]&I\ar[r]&\tau^{\leq n}I\ar[r]&0
\end{tikzcd}\]
and we have a corresponding distinguished triangle in $D(\mathcal{A})$; it then suffices to mimic the proof of (\rmnum{2}).
\end{proof}
\subsection{Derived functors}
Let $\mathcal{A}$, $\mathcal{A}'$ and $\mathcal{A}''$ be abelian categories and $F:\mathcal{A}\to\mathcal{A}'$ be an additive functor. Then $F$ defines naturally a triangulated functor
\[K^*(F):K^*(\mathcal{A})\to K^*(\mathcal{A}').\]
For short, we often write $F$ instead of $K^*(F)$. We shall denote by $Q:K^*(\mathcal{A})\to D^*(\mathcal{A})$ the localization functor, and similarly with $Q'$, $Q''$, when replacing $\mathcal{A}$ with $\mathcal{A}'$, $\mathcal{A}''$.

\begin{definition}
Let $*\in\{+,b,\emp\}$. We say that the functor $F$ is \textbf{right derivable} (or $F$ admits a \textbf{right derived functor}) on $K^*(\mathcal{A})$ if the triangulated functor $K^*(F):K^*(\mathcal{A})\to K^*(\mathcal{A}')$ is universally right localizable with respect to $N^*(\mathcal{A})$ and $N^*(\mathcal{A}')$. In such a case the localization of $F$ is denoted by $R^*F$ and $H^n\circ R^*F$ is denoted by $R^nF$. The functor $R^*F:D^*(\mathcal{A})\to D^*(\mathcal{A}')$ is called the \textbf{right derived functor} of $F$ and $R^nF$ the \textbf{$\bm{n}$-th derived functor} of $F$.
\end{definition}

By definition, the functor $F$ admits a right derived functor on $K^*(\mathcal{A})$ if the ind-object
\[\rlim_{\substack{(X\to X')\in\Qis\\ X'\in K^*(\mathcal{A})}}Q'\circ K(F)(X')\]
is representable in $D^*(\mathcal{A}')$ for all $X\in K^*(\mathcal{A})$. In such a case, this object is isomorphic to $R^*F(X)$. Note that $R^*F$ is a triangulated functor from $D^*(\mathcal{A})$ to $D^*(\mathcal{A})$ if it exists, and $R^nF$ is a cohomological functor from $D^*(\mathcal{A})$ to $\mathcal{A}'$. Morever, if $RF$ exists, then $R^+F$ exists and $R^+F$ is the restriction of $RF$ to $D^+(\mathcal{A})$.

\begin{definition}
Let $\mathcal{J}$ be a full additive subcategory of $\mathcal{A}$. We say for short that $\mathcal{J}$ is \textbf{$\bm{F}$-injective} if the subcategory $K^+(\mathcal{J})$ of $K^+(\mathcal{A})$ is $K^+(F)$-injective with respect to $N^+(\mathcal{A})$ and $N^+(\mathcal{A}')$. We shall also say that $\mathcal{J}$ is injective with respect to $F$. We define similarly the notion of an $F$-projective subcategory.
\end{definition}
By the definition, $\mathcal{J}$ is $F$-injective if and only if for any $X\in K^+(\mathcal{A})$, there exists a quasi-isomorphism $X\to Y$ with $Y\in K^+(\mathcal{J})$ and $F(Y)$ is exact for any exact complex $Y\in K^+(\mathcal{J})$. If $F$ is right (resp. left) derivable, an object $X$ of $\mathcal{A}$ such that $R^nF(X)=0$ (resp. $L^nF(X)=0$) for all $n\neq 0$ is called \textbf{right $\bm{F}$-acyclic} (resp. \textbf{left $\bm{F}$-acyclic}). If $\mathcal{J}$ is an $F$-injective subcategory, then any object of $\mathcal{J}$ is right $F$-acyclic.\par
From \cref{triangle cat localization functor composition}, it is immediate that we have the following result concerning composition of derived functors:
\begin{proposition}\label{derived category functor derived composition prop}
Let $F:\mathcal{A}\to\mathcal{A}'$ and $F':\mathcal{A}'\to\mathcal{A}''$ be additive functors of abelian categories. Let $*\in\{+,b,\emp\}$ and assume that the right derived functors $R^*F$, $R^*F'$ and $R^*(F'\circ F)$ exist. Then there is a canonical morphism of functors
\begin{equation}\label{derived category functor derived composition prop-1}
R^*(F'\circ F)\to R^*(F')\circ R^*(F).
\end{equation}
Assume that there exist full additive subcategories $\mathcal{J}\sub\mathcal{A}$ and $\mathcal{J}'\sub\mathcal{A}'$ such that $\mathcal{A}$ is $F$-injective, $\mathcal{J}'$ is $F'$ injective and $F(\mathcal{J})\sub\mathcal{J}'$. Then $\mathcal{J}$ is $F'\circ F$-injective and (\ref{derived category functor derived composition prop-1}) induces an isomorphism
\[R^+(F'\circ F)\stackrel{\sim}{\to} R^+F'\circ R^+F.\]
\end{proposition}
Note that in many cases (even if $F$ is exact), $F$ may not send injective objects of $\mathcal{A}$ to injective objects of $\mathcal{A}'$. This is a reason why the notion of an "$F$-injective" category is useful. 

\begin{proposition}\label{derived category F-injective subcat iff}
Let $F:\mathcal{A}\to\mathcal{A}'$ be an additive functor of abelian categories and let $\mathcal{J}$ be a full additive subcategory of $\mathcal{A}$.
\begin{enumerate}
    \item[(a)] If $\mathcal{J}$ is $F$-injective, then $R^+F:D^+(\mathcal{A})\to D^+(\mathcal{A}')$ exists and $R^+F(X)\cong F(Y)$ for any quasi-isomorphism $X\to Y$ with $Y\in K^+(\mathcal{J})$.
    \item[(b)] If $F$ is left exact, then $\mathcal{J}$ is $F$-injective if and only if it satisfies the following conditions:
    \begin{enumerate}
        \item[(\rmnum{1})] the category $\mathcal{J}$ is cogenerating in $\mathcal{A}$;
        \item[(\rmnum{2})] for any exact sequence $0\to X'\to X\to X''\to 0$, the sequence $0\to F(X')\to F(X)\to F(X'')\to 0$ is exact as soon as $X\in\mathcal{J}$ and there exists an exact sequence
        \[\begin{tikzcd}
        0\ar[r]&Y^0\ar[r]&\cdots\ar[r]&Y^n\ar[r]&X'\ar[r]&0
        \end{tikzcd}\]
        with $Y^i\in\mathcal{J}$.
    \end{enumerate}
\end{enumerate}
\end{proposition}
\begin{proof}
The first assertion follows from \cref{triangle cat localization of functor exist if} and (\ref{triangle cat localization functor expression-1}), so assume that $F$ is left exact. If $\mathcal{J}$ is $F$-injective, then for $X\in\mathcal{A}$, there exists a quasi-isomorphism $X\to Y$ with $Y\in K^+(\mathcal{J})$. The composition $X\to\ker d_Y^0\to H^0(Y)$ is then an isomorphism, so $X\to Y^0$ is a monomorphism and this proves that $\mathcal{J}$ is cogenerating in $\mathcal{A}$. By \cref{derived category resolution by subcat if}, there then exists an exact sequence $0\to X''\to Z^0\to Z^1\to\cdots$ with $Z^i\in\mathcal{J}$ for all $i$. The sequence
\[\begin{tikzcd}
0\ar[r]&Y^0\ar[r]&\cdots\ar[r]&Y^n\ar[r]&X\ar[r]&Z^0\ar[r]&Z^1\ar[r]&\cdots
\end{tikzcd}\]
is then exact and belongs to $K^+(\mathcal{J})$, so $F(X)\to F(Z^0)\to F(Z^1)$ is exact. Since $F$ is left exact, $F(X'')\cong \ker(F(Z^0)\to F(Z^1))$ and this implies that $F(X)\to F(X'')$ is an epimorphism.\par
Conversely, assume the two conditions in (b). By \cref{derived category resolution by subcat if}, for any $X\in K^+(\mathcal{A})$ there exists a quasi-isomorphism $X\to Y$ with $Y\in K^+(\mathcal{J})$, so it suffices to show that $F(X)$ is exact if $X\in K^+(\mathcal{J})$ is exact. To this end, we note that for each $n\in\Z$, the sequences
\[\begin{tikzcd}
\cdots\ar[r]&X^{n-2}\ar[r]&X^{n-1}\ar[r]&\ker d^n_X\ar[r]&0
\end{tikzcd}\]
\vspace*{-5mm}
\[\begin{tikzcd}
0\ar[r]&\ker d^n_X\ar[r]&X^n\ar[r]&\ker d^{n+1}_X\ar[r]&0
\end{tikzcd}\]
are exact, so by condition (\rmnum{2}), the sequence $0\to F(\ker d^n_X)\to F(X^n)\to F(\ker d^{n+1}_X)\to 0$ is exact, and this proves that $F(X)$ is exact.
\end{proof}

\begin{remark}
Note that for $X\in\mathcal{A}$, $R^nF(X)=0$ for $n<0$ and assuming that $F$ is left exact, $R^0F(X)\cong F(X)$. Indeed, for $X\in\mathcal{A}$ and any quasi-isomorphism $X\to Y$, the composition $X\to Y\to \tau^{\geq 0}Y$ is a quasi-isomorphism.
\end{remark}
\begin{example}
If $\mathcal{A}$ has enough injectives, then the full subcategory $\mathcal{I}_\mathcal{A}$ of injective objects of $\mathcal{A}$ is $F$-injective for any additive functor $F:\mathcal{A}\to\mathcal{A}'$. Indeed, any exact complex in $\Ch^+(\mathcal{I})$ is homotopic to zero by \cref{abelian cat injective qis zero is null-homotopy}, whence its image under $F$. In particular, $R^+F:D^+(\mathcal{A})\to D^+(\mathcal{A}')$ exists in this case.
\end{example}

We shall now give a sufficient condition in order that $\mathcal{J}$ is $F$-injective, which is especially useful if the category $\mathcal{A}$ does not have enough injectives.

\begin{theorem}\label{derived category F-injective subcat if factor through monomorphism}
Let $\mathcal{J}$ be a full additive subcategory of $\mathcal{A}$ and let $F:\mathcal{A}\to\mathcal{A}'$ be a left exact functor. Assume that
\begin{enumerate}
    \item[(a)] $\mathcal{J}$ is cogenerating in $\mathcal{A}$;
    \item[(b)] for any monomorphism $Y'\hookrightarrow X$ with $Y'\in\mathcal{J}$, there exists an exact sequence $0\to Y'\to Y\to Y''\to 0$ with $Y,Y''\in\mathcal{J}$ such that $Y'\to Y$ factors through $X$ and the sequence $0\to F(Y')\to F(Y)\to F(Y'')\to 0$ is exact. 
\end{enumerate}
Then $\mathcal{J}$ is $F$-injective.
\end{theorem}
Condition (b) of \cref{derived category F-injective subcat if factor through monomorphism} may be visualized as
\[\begin{tikzcd}
0\ar[r]&Y'\ar[d,equal]\ar[r]&X\ar[d,dashed]\\
0\ar[r]&Y'\ar[r,dashed]&Y\ar[r,dashed]&Y''\ar[r]&0
\end{tikzcd}\]
Since this condition is rather intricate, the often consider the following particular case of \cref{derived category F-injective subcat if factor through monomorphism}, which is sufficient for most applications.
\begin{corollary}\label{derived category F-injective subcat if exact on cokernel of monomorphism}
Let $\mathcal{J}$ be a full additive subcategory of $\mathcal{A}$ and let $F:\mathcal{A}\to\mathcal{A}'$ be a left exact functor. Assume that
\begin{enumerate}
    \item[(a)] $\mathcal{J}$ is cogenerating in $\mathcal{A}$;
    \item[(b)] $\mathcal{J}$ is closed under cokernels of monomorphisms;
    \item[(c)] for any exact sequence $0\to X'\to X\to X''\to 0$ in $\mathcal{A}$ with $X',X\in\mathcal{J}$, the sequence $0\to F(X')\to F(X)\to F(X'')\to 0$ is exact.
\end{enumerate}
Then $\mathcal{J}$ is $F$-injective.
\end{corollary}
\begin{proof}
For any monomorphism $Y'\to X$ with $Y'\in\mathcal{J}$, we can embedd $X$ into an object $Y\in\mathcal{J}$ and take $Y''$ to be the cokernel of $Y$ by $Y'$:
\[\begin{tikzcd}
0\ar[r]&Y'\ar[d,equal]\ar[r]&X\ar[d,hook]\\
0\ar[r]&Y'\ar[r]&Y\ar[r]&Y''\ar[r]&0
\end{tikzcd}\]
By hypothesis, we have $Y''\in\mathcal{J}$, and the sequence $0\to F(X')\to F(X)\to F(X'')\to 0$ is exact, so we can apply \cref{derived category F-injective subcat if factor through monomorphism}.
\end{proof}
\begin{corollary}\label{derived category F-acyclic subcat is F-injective if}
Let $F:\mathcal{A}\to\mathcal{A}'$ be a left exact functor of abelian categories and let $\mathcal{J}$ be an $F$-injective full subcategory of $\mathcal{A}$. Let $\mathcal{J}_F$ be the full subcategory of $\mathcal{A}$ consisting of right $F$-acyclic objects, then $\mathcal{J}_F$ contains $\mathcal{J}$ and $\mathcal{J}_F$ satisfies the conditions of \cref{derived category F-injective subcat if exact on cokernel of monomorphism}. In particular, $\mathcal{J}_F$ is $F$-injective.
\end{corollary}
\begin{proof}
Since $\mathcal{J}_F$ contains $\mathcal{J}$, $\mathcal{J}_F$ is cogenerating. Consider an exact sequence $0\to X'\to X\to X''\to 0$ in $\mathcal{A}$ with $X',X\in\mathcal{J}_F$. The exact sequences $R^iF(X)\to R^iF(X'')\to R^{i+1}F(X')$ for $i\geq 0$ imply that $R^iF(X'')=0$ for $i>0$. Moreover, there is an exact sequence $0\to F(X')\to F(X)\to F(X'')\to 0$ since $R^1F(X')=0$.
\end{proof}
Therefore, a full additive subcategory $\mathcal{J}$ of $\mathcal{A}$ is $F$-injective if and only if it is cogenerating and any object of $\mathcal{J}$ is $F$-acyclic (assuming the right derivability of $F$). Note that even if $F$ is right derivable, there may not exist an $F$-injective subcategory, since we do not know that whether the subcategory of $F$-acyclic objects is cogenerating.
\begin{example}
Let $A$ be a ring and let $N$ be a right $A$-module. The full additive subcategory of $\mathbf{Mod}(A)$ consisting of flat $A$-modules is $(N\otimes_A-)$-projective. In fact, this subcategory satisfies the dual conditions of \cref{derived category F-injective subcat if exact on cokernel of monomorphism}.
\end{example}

We now turn to the proof of \cref{derived category F-injective subcat if factor through monomorphism}, which we decompose into several lemmas.
\begin{lemma}\label{derived category F-acyclic if factor through monomorphism}
With the assumptions of \cref{derived category F-injective subcat if factor through monomorphism}, let $0\to Y'\to X\to X''$ be an exact sequence in $\mathcal{A}$ with $Y'\in\mathcal{J}$. Then the sequence $0\to F(Y')\to F(X)\to F(X'')$ is exact.
\end{lemma}
\begin{proof}
Choose an exact sequence $0\to Y'\to Y\to Y''$ as in \cref{derived category F-injective subcat if factor through monomorphism}. We get the commutative exact diagram:
\[\begin{tikzcd}
0\ar[r]&Y'\ar[d,equal]\ar[r]&X\ar[r]\ar[d]&X''\ar[r]\ar[d]&0\\
0\ar[r]&Y'\ar[r]&Y\ar[r]&Y''\ar[r]&0
\end{tikzcd}\]
where the right square is Cartesian. Since $F$ is left exact, it transforms this square to a Cartesian square and the bottom row to an exact row. Hence, the result follows from (\cite{kashiwara_SAC} lemma 8.3.11).
\end{proof}
\begin{lemma}\label{derived category truncation resolution if factor through monomorphism}
With the assumptions of \cref{derived category F-injective subcat if factor through monomorphism}, let $X^\bullet\in\Ch^+(\mathcal{A})$ be an exact complex, and assume $X^i=0$ for $i<n$ and $X^n\in\mathcal{J}$. There exist an exact complex $Y^\bullet\in\Ch^+(\mathcal{J})$ and a morphism $f:X^\bullet\to Y^\bullet$ such that $Y^i=0$ for $i<n$, $f^n:X^n\to Y^n$ is an isomorphism, and $\im d^i_Y\in\mathcal{J}$ for all $i$.
\end{lemma}
\begin{proof}
We argue by induction. By the hypothesis of \cref{derived category F-injective subcat if factor through monomorphism}, there exists a commutative
exact diagram:
\[\begin{tikzcd}
0\ar[r]&X^n\ar[r]\ar[d,"\sim"]&X^{n+1}\ar[d]\\
0\ar[r]&Y^n\ar[r]&Y^{n+1}\ar[r]&Z^{n+2}\ar[r]&0
\end{tikzcd}\]
with $Y^{n+1},Z^{n+2}$ in $\mathcal{J}$. Now suppose that we have already constructed a diagram
\[\begin{tikzcd}
0\ar[r]&X^n\ar[r]\ar[d,"\sim"]&\cdots\ar[r]&X^m\ar[d]\\
0\ar[r]&Y^n\ar[r]&\cdots\ar[r]&Y^m\ar[r]&Z^{m+1}\ar[r]&0
\end{tikzcd}\]
where the bottom row is exact and belongs to $\Ch^+(\mathcal{J})$, and $\im d_Y^i$ belongs to $\mathcal{J}$ for $n\leq i\leq m-1$. Define $W^{m+1}=X^{m+1}\oplus_{\coker d_X^{m-1}}Z^{m+1}$, so that we have a Cartesian exact diagram
\[\begin{tikzcd}
0\ar[r]&\coker d_X^{m-1}\ar[d]\ar[r]&X^{m+1}\ar[d]\\
0\ar[r]&Z^{m+1}\ar[r]&W^{m+1}
\end{tikzcd}\]
By the hypotheses, there exists an exact commutative diagram
\[\begin{tikzcd}
0\ar[r]&Z^{m+1}\ar[d,equal]\ar[r]&W^{m+1}\ar[d]\\
0\ar[r]&Z^{m+1}\ar[r]&Y^{m+1}\ar[r]&Z^{m+2}\ar[r]&0
\end{tikzcd}\]
with $Y^{m+1}$ and $Z^{m+2}$ in $\mathcal{J}$. If we define $d^m_Y$ to be the composition $Y^m\to Z^{m+1}\to Y^{m+1}$, then $\im d^m_Y\cong Z^{m+1}\in\mathcal{J}$, and this completes the induction process.
\end{proof}

\begin{proof}[\textbf{Proof of \cref{derived category F-injective subcat if factor through monomorphism}}]
Let $X^\bullet\in\Ch^+(\mathcal{J})$ be an exact complex, we have to prove that $F(X^\bullet)$ is exact. Let us show by induction on $m-n$ that $H^m(F(X^\bullet))=0$ if $X\in\Ch^{\geq n}(\mathcal{J})$. If $m<n$, this is clear, so we may assume that $m\geq n$. By \cref{derived category truncation resolution if factor through monomorphism}, there exists a morphism of complexes $f:X^\bullet\to Y^\bullet$ in $\Ch^+(\mathcal{J})$ such that $Y^\bullet\in\Ch^{\geq n}(\mathcal{J})$, $f^n:X^n\to Y^n$ is an isomorphism and $F(Y^\bullet)$ is exact. Let $\sigma^{\geq n+1}$ denote the stupid truncated complexes and $W$ denote the mapping cone of the morphism
\[\sigma^{\geq n+1}(f):\sigma^{\geq n+1}X^\bullet\to \sigma^{\geq n+1}Y^\bullet.\]
Then $W^i=(\sigma^{\geq n+1}X^\bullet)^{i+1}\oplus(\sigma^{\geq n+1}Y^\bullet)^i=0$ for $i<n$, and we have a distinguished triangle in $K(\mathcal{J})$:
\[\begin{tikzcd}
W\ar[r]&M(f)\ar[r]&M(X^n[-n]\to Y^n[-n])\ar[r]&W[1].
\end{tikzcd}\]
Since $X^n\to Y^n$ is an isomorphism, the mapping cone $M(X^n[-n]\to Y^n[-n])$ is exact and therefore $W\to M(f)$ is an isomorphism in $K(\mathcal{A})$. Applying the functor $F$, we then obtain an isomorphism $F(W)\cong F(M(f))$ in $K(\mathcal{A}')$, so $H^i(F(W))\cong H^i(F(M(f)))$ for each $i$. On the other hand, there is a distinguished triangle in $K^+(\mathcal{A}')$:
\[\begin{tikzcd}
F(X)\ar[r]&F(Y)\ar[r]&F(M(f))\ar[r]&F(X)[1]
\end{tikzcd}\]
and $F(Y)$ is exact by our hypothesis, whence $H^m(F(X))\cong H^{m-1}(F(M(f)))\cong H^{m-1}(F(W))$. Since $W$ is an exact complex and belongs to $\Ch^{\geq n}(\mathcal{J})$, the induction hypothesis implies that $H^{m-1}(F(W))=0$, so we conclude that $H^m(F(X))=0$. 
\end{proof}

\paragraph{Derived projective limit}
As an application of \cref{derived category F-injective subcat if factor through monomorphism} we shall discuss the existence of the derived functor of projective limits. Let $\mathcal{A}$ be an abelian $\mathscr{U}$-category. Recall that $\Pro(\mathcal{A})$ is an abelian category admitting small projective limits, and small filtrant projective limits as well as small products are exact. Assume that $\mathcal{A}$ admits small projective limits. Then the natural exact functor $\mathcal{A}\to\Pro(\mathcal{A})$ admits a right adjoint
\[\pi_\mathcal{A}:\Pro(\mathcal{A})\to\mathcal{A}\]
defined as follows: if $\beta:I^{\op}\to\mathcal{A}$ is a functor with $I$ small and filtrant, then $\pi_\mathcal{A}$ transforms $\text{"$\llim$"}\beta$ (as a pro-object) to the limit $\llim\beta$ in $\mathcal{A}$. The functor $\pi_\mathcal{A}$ is left exact and we shall give a condition in order that it is right derivable.\par
For a full additive subcategory $\mathcal{J}$ of $\mathcal{A}$, we define a full additive subcategory $\mathcal{J}_{\pro}$ of $\Pro(\mathcal{A})$ by
\[\mathcal{J}_{\pro}=\{X\in\Pro(\mathcal{A}):\text{$X\cong\underset{i\in I}{\text{"$\prod$"}}X_i$ for a small set $I$ and $X_i\in\mathcal{J}$}\}.\]
Here the product "$\prod$" is taken in the category $\Pro(\mathcal{A})$, so for $X_i,Y\in\mathcal{A}$, we have a canonical bijection
\[\Hom_{\Pro(\mathcal{A})}(\underset{i\in I}{\text{"$\prod$"}}X_i,Y)\stackrel{\sim}{\to} \bigoplus_{i\in I}\Hom_{\mathcal{A}}(X_i,Y).\]

\begin{proposition}\label{derived category derived functor of proj limit}
Let $\mathcal{A}$ be an abelian category admitting small projective limits and let $\mathcal{J}$ be a full additive subcategory of $\mathcal{A}$ satisfying:
\begin{enumerate}
    \item[(a)] $\mathcal{J}$ is cogenerating in $\mathcal{A}$;
    \item[(b)] $\mathcal{J}$ is closed under cokernels of monomorphisms;
    \item[(c)] if $0\to Y_i'\to Y_i\to Y''_i\to 0$ is a family of exact sequences in $\mathcal{J}$ indexed by a small set $I$, then the sequence $0\to Y_i'\to Y_i\to Y''_i\to 0$ is exact.
\end{enumerate}
Then the category $\mathcal{J}_{\pro}$ is $\pi_\mathcal{A}$-injective and the left exact functor $\pi_\mathcal{A}$ admits a right derived functor
\[R^+\pi_\mathcal{A}:D^+(\Pro(\mathcal{A}))\to D^+(\mathcal{A})\]
which satisfies $R^n\pi_\mathcal{A}(\underset{i}{\text{"$\prod$"}}X_i)=0$ for $n>0$ and $X_i\in\mathcal{J}$. Moreover, the composition
\[\begin{tikzcd}
D^+(\mathcal{A})\ar[r]&D^+(\Pro(\mathcal{A}))\ar[r,"R^+\pi_\mathcal{A}"]&D^+(\mathcal{A})
\end{tikzcd}\]
is isomorphic to the identity.
\end{proposition}
\begin{proof}
We shall verify the conditions of \cref{derived category F-injective subcat if factor through monomorphism}. The category $\mathcal{J}_{\pro}$ is cogenerating in $\Pro(\mathcal{A})$ since for $A=\text{"$\llim$\!"}\!_{i\in I}\hspace*{1pt}\alpha(i)\in\Pro(\mathcal{A})$, we obtain a monomorphism $A\hookrightarrow\text{"$\prod$"\!}_iX_i$ by choosing a monomorphism $\alpha(i)\hookrightarrow X_i$ with $X_i\in\mathcal{J}$ for each $i\in I$. Now consider a monomorphism $Y\hookrightarrow A$ in $\Pro(\mathcal{A})$ with $A\in\Pro(\mathcal{A})$ and $Y=\text{"$\prod$"\!}_iY_i$, $Y_i\in\mathcal{J}$. Applying the dual version of (\cite{kashiwara_SAC} proposition 8.6.9), for each $i$ we can find $X_i\in\mathcal{A}$ and a commutative exact diagram
\[\begin{tikzcd}
0\ar[r]&Y\ar[r]\ar[d]&A\ar[d]\\
0\ar[r]&Y_i\ar[r]&X_i
\end{tikzcd}\]
Since $\mathcal{J}$ is cogenerating, we may assume that $X_i\in\mathcal{J}$. Let $Z_i=\coker(Y_i\to X_i)$, which is in $\mathcal{J}$ by hypothesis. By hypothesis, functor $\prod$ is exact on $\mathcal{J}$, so we obtain a commutative diagram
\[\begin{tikzcd}
0\ar[r]&Y\ar[r]\ar[d,equal]&A\ar[d]\\
0\ar[r]&\text{"$\prod$"\!}_iY_i\ar[r]&\text{"$\prod$"\!}_iX_i\ar[r]&\text{"$\prod$"\!}_iZ_i\ar[r]&0
\end{tikzcd}\]
with exact rows. Applying $\pi_\mathcal{A}$ to the second row, we then obtain the sequence $0\to \prod_iY_i\to \prod_iX_i\to\prod_iZ_i\to 0$ in $\mathcal{A}$, which is exact by hypothesis (c). By \cref{derived category F-injective subcat if factor through monomorphism}, we then conclude that $\mathcal{J}_{\pro}$ is $\pi_\mathcal{A}$-injective, so $\pi_\mathcal{A}$ admits a right derived functor (\cref{triangle cat localization of functor exist if}), and we have $R^n\pi_\mathcal{A}(\text{"$\prod$"\!}_iX_i)=0$ for $n>0$ and $X_i\in\mathcal{J}$.\par
Finally, by assumption $\mathcal{J}$ is injective with respect to the exact functor $\mathcal{A}\to\Pro(\mathcal{A})$. Since the functor $\mathcal{A}\to\Pro(\mathcal{A})$ sends $\mathcal{J}$ to $\mathcal{J}_{\pro}$, the last assertion follows from \cref{derived category functor derived composition prop}.
\end{proof}

\begin{corollary}
Let $\mathcal{J}$ be a full additive subcategory of $\mathcal{A}$ satisfying the conditions of \cref{derived category derived functor of proj limit}. If $(X_n)$ is a projective system in $\mathcal{J}$ indexed by $\N$ and $X=\text{"$\llim$\!"}X_n$, then $R^p\pi_\mathcal{A}(X)=0$ for $p>1$, and we have a canonical isomorphism
\[R^1\pi_\mathcal{A}(X)\stackrel{\sim}{\to} \coker\Big(\prod_nX_n\stackrel{\Delta_X}{\longrightarrow}\prod_nX_n\Big)\]
where $\Delta_X:=T_X-\id:\prod_nX_n\to\prod_nX_n$.
\end{corollary}
\begin{proof}
We have an exact sequence in $\Pro(\mathcal{A})$:
\[\begin{tikzcd}
0\ar[r]&\text{"$\llim$\!"\!}_nX_n\ar[r]&\text{"$\prod$"\!}_nX_n\ar[r,"\Delta_X"]&\text{"$\prod$"\!}_nX_n\ar[r]&0
\end{tikzcd}\]
Applying the functor $R^+\pi_\mathcal{A}$, we then get a long exact sequence and the results follows since $R^p\pi_\mathcal{A}(\prod_nX_n)=0$ for $p>0$.
\end{proof}

\begin{example}
If $A$ is a ring and $\mathcal{A}=\mathbf{Mod}(A)$, we may choose $\mathcal{J}=\mathcal{A}$ in \cref{derived category derived functor of proj limit}. In fact, for a family of objects $(X_i)_{i\in I}$ in $\mathcal{A}$, the product $\text{"$\prod$"\!}_iX_i$ can be considered as the limit of the functor $\alpha:I\to\mathcal{A}$ with $I$ being considered as a discrete category.
\end{example}

\paragraph{Derived bifunctors}
We shall now apply the previous results to bifunctors between abelian categories. The most important examples in mind will be $\Hom$ and tensor functors.
\begin{theorem}\label{derived category R^0Hom is Hom in D(A)}
Let $\mathcal{A}$ be an abelian category and $X,Y\in D(\mathcal{A})$. Assume that the functor
\[\Hom_\mathcal{A}^\bullet:K(\mathcal{A})^{\op}\times K(\mathcal{A})\to K(\mathbf{Mod}(\Z)),\quad (X',Y')\mapsto \Hom_\mathcal{A}^{\bullet}(X',Y')\]
is right localizable at $(X,Y)$, then for any $n\in\Z$, we have
\[R^n\Hom_\mathcal{A}(X,Y)\cong\Hom_{D(\mathcal{A})}(X,Y[n]).\]
\end{theorem}
\begin{proof}
By hypothesis, we have
\[R\Hom_\mathcal{A}(X,Y)\stackrel{\sim}{\to} \rlim_{\substack{(X'\to X)\in\Qis,\\(Y\to Y')\in\Qis}}\Hom_\mathcal{A}^\bullet(X',Y').\]
Applying the functor $H^n$ and recalling that $\rlim$ commutes with $H^n$, we conclude from (\cite{kashiwara_SAC} Proposition 11.7.3) that
\begin{equation*}
R^n\Hom_\mathcal{A}(X,Y)\cong \rlim H^n(\Hom_\mathcal{A}^\bullet(X',Y'))\cong \rlim \Hom_{K(\mathcal{A})}(X',Y'[n])\cong\Hom_{D(\mathcal{A})}(X,Y[n]).\qedhere
\end{equation*}
\end{proof}

Consider now three abelian categories $\mathcal{A}$, $\mathcal{A}'$, $\mathcal{A}''$ and an additive bifunctor
\[F:\mathcal{A}\times\mathcal{A}'\to\mathcal{A}''.\]
By (\cite{kashiwara_SAC} Proposition 11.6.3), the triangulated functor
\[K^+F:K^+(\mathcal{A})\times K^+(\mathcal{A}')\to K^+(\mathcal{A}'')\]
is naturally defined by setting
\[K^+F(X,X')=\Tot(F(X,X')).\]
Similarly to the case of functors, if the triangulated bifunctor $K^+F$ is universally right localizable with respect to $(N^+(\mathcal{A})\times N^+(\mathcal{A}'),N^+(\mathcal{A}''))$, then $F$ is said to be \textbf{right derivable} and its localization is denoted by $R^+F$. We set $R^nF=H^n\circ R^+F$ and call it the $n$-th \textbf{derived bifunctor} of $F$.

\begin{definition}
Let $\mathcal{J}$ and $\mathcal{J}'$ be full additive subcategories of $\mathcal{A}$ and $\mathcal{A}'$ respectively. We say for short that $(\mathcal{J},\mathcal{J}')$ is \textbf{$\bm{F}$-injective} if $(K^+(\mathcal{J}),K^+(\mathcal{J}'))$ is $K^+F$-injective.
\end{definition}

\begin{proposition}\label{derived category bifunctor derive exists if}
Let $\mathcal{J}$ and $\mathcal{J}'$ be full additive subcategories of $\mathcal{A}$ and $\mathcal{A}'$ respectively. Assume that $(\mathcal{J},\mathcal{J}')$ is $F$-injective, then $F$ is right derivable and for $(X,X')\in D^+(\mathcal{A})\times D^+(\mathcal{A}')$ we have
\[R^+F(X,X')\cong Q''\circ K^+F(Y,Y')\]
for $(X\to Y)\in\Qis$ and $(X'\to Y')\in\Qis$ with $Y\in K^+(\mathcal{J})$ and $Y'\in K^+(\mathcal{J}')$.
\end{proposition}
\begin{proof}
It suffices to apply \cref{triangle cat localization bifunctor exists if} to the functor $Q''\circ K^+F$.
\end{proof}

\begin{proposition}\label{derived category bifunctor F-injective if separate}
Let $\mathcal{J}$ and $\mathcal{J}'$ be full additive subcategories of $\mathcal{A}$ and $\mathcal{A}'$ respectively. Assume that 
\begin{enumerate}
    \item[(a)] for any $Y\in\mathcal{J}$, $\mathcal{J}'$ is $F(Y,-)$-injective;
    \item[(b)] for any $Y'\in\mathcal{J}'$, $\mathcal{J}$ is $F(-,Y')$-injective.
\end{enumerate}
Then $(\mathcal{J},\mathcal{J}')$ is $F$-injective.
\end{proposition}
\begin{proof}
Let $(Y,Y')\in K^+(\mathcal{J})\times K^+(\mathcal{J}')$. If either $Y$ or $Y'$ is quasi-isomorphic to zero, then $\Tot(F(Y,Y'))$ is quasi-isomorphic to zero by (\cite{kashiwara_SAC} Proposition 12.5.5), so $(\mathcal{J},\mathcal{J}')$ is $F$-injective.
\end{proof}

\begin{corollary}\label{derived category bifunctor localization if exact one variable}
Let $\mathcal{J}$ be a full additive cogenerating subcategory of $\mathcal{A}$ and assume:
\begin{enumerate}
    \item[(a)] for any $X\in\mathcal{J}$, $F(X,-)$ is exact;
    \item[(b)] for any $X'\in\mathcal{A}'$, $\mathcal{J}$ is $F(-,X')$-injective.
\end{enumerate}
Then $F$ is right derivable and for $X\in K^+(\mathcal{A})$, $X'\in K^+(\mathcal{A}')$,
\[R^+F(X,X')\cong Q''\circ K^+F(Y,X')\]
for any $(X\to Y)\in\Qis$ with $Y\in K^+(\mathcal{J})$. In particular, for $X\in\mathcal{A}$ and $X'\in\mathcal{A}'$, $R^+F(X,X')$ is the derived functor of $F(-,X')$ calculated at $X$, that is, we have
\[R^+F(X,X')=(R^+F(-,X'))(X).\]
\end{corollary}
\begin{proof}
The first assertion follows from \cref{derived category bifunctor F-injective if separate} by setting $\mathcal{J}'=\mathcal{A}'$, and the second one follows from \cref{triangle cat localization bifunctor if exact one variable}.
\end{proof}

\begin{corollary}\label{derived category bounded RHom exists if}
Let $\mathcal{A}$ be an abelian category and assume that there are subcategories $\mathcal{P},\mathcal{J}$ in $\mathcal{A}$ such that $(\mathcal{P}^{\op},\mathcal{J})$ is injective with respect to the functor $\Hom_\mathcal{A}$. Then the functor $\Hom_\mathcal{A}$ admits a right derived functor $R^+\Hom_\mathcal{A}:D^-(\mathcal{A})^{\op}\times D^+(\mathcal{A})\to D^+(\Z)$. In particular, $D^b(\mathcal{A})$ is a $\mathscr{U}$-category.
\end{corollary}
\begin{proof}
The first assertion follows from \cref{derived category bifunctor F-injective if separate}, and the second one is a concequence of \cref{derived category R^0Hom is Hom in D(A)}, since $R^0\Hom$ takes its values in $\mathscr{U}$-sets.
\end{proof}

\begin{example}\label{derived category bounded RHom expression eg}
Assume that $\mathcal{A}$ has enough injectives. Then by \cref{derived category bifunctor localization if exact one variable}, the derived Hom functor
\[R^+\Hom_\mathcal{A}:D^-(\mathcal{A})^{\op}\times D^+(\mathcal{A})\to D^+(\Z)\]
exists and may be calculated as follows. Let $X\in D^-(\mathcal{A})$ and $Y\in D^+(\mathcal{A})$. Then there exists a quasi-isomorphism $Y\to I$ in $K^+(\mathcal{A})$, with the $I^i$'s being injective, and
\[R^+\Hom_\mathcal{A}(X,Y)\cong\Hom_\mathcal{A}^\bullet(X,I).\]
If $\mathcal{A}$ has enough projectives, then $R^+\Hom_\mathcal{A}$ also exists, and for a quasi-isomorphism $P\to X$ with $P^i$'s being projective, we have
\[R^+\Hom_\mathcal{A}(X,Y)\cong\Hom_\mathcal{A}^\bullet(P,Y).\]
These isomorphisms hold in $D^+(\Z)$, which means $R^+\Hom_\mathcal{A}(X,Y)\in D^+(\Z)$ is represented by the simple complex associated with the double complex $\Hom_\mathcal{A}^{\bullet,\bullet}(X,I)$ or $\Hom_\mathcal{A}^{\bullet,\bullet}(P,Y)$.
\end{example}
\begin{example}\label{derived category bounded Ltensor expression eg}
Let $A$ be a $k$-algebra, with $k$ being a ring. Since the category $\mathbf{Mod}(A)$ has enough projectives, the left derived functor of the functor $-\otimes_A-$ is well defined:
\[-\otimes_A^L-:D^-(A^{\op})\times D^-(A)\to D^-(k).\]
This functor may by calculated as follows:
\[N\otimes_A^LM\cong\Tot(N\otimes_AP)\cong\Tot(Q\otimes_AM)\cong\Tot(Q\otimes_AP)\]
where $P$ is a complex of projective $A$-modules quasi-isomorphic to $M$ and $Q$ is a complex of projective $A^{\op}$-modules quasi-isomorphic to $N$. A classical notation is $\Tor_n^A(N,M):=H_n(N\otimes_A^LM)$.
\end{example}

\begin{proposition}\label{derived category bounded derived adjointness}
Let $F:\mathcal{A}\to\mathcal{A}':G$ be an adjoint pair of additive functors. Assume that $\mathcal{A}$ has enough projectives and $\mathcal{A}'$ has enough injective objects, then there exists a canonical isomorphism in $D^+(\Z)$:
\[R\Hom_{\mathcal{A}'}(L^-F(X),Y)\stackrel{\sim}{\to} R\Hom_{\mathcal{A}'}(X,R^+G(Y)),\]
where $X\in D^-(\mathcal{A})$ and $Y\in D^+(\mathcal{A}')$. In particular, we have canonical isomorphisms
\[\Hom_{D(\mathcal{A}')}(L^-F(X),Y)\stackrel{\sim}{\to} \Hom_{D(\mathcal{A})}(X,R^+G(Y))\]
\end{proposition}
\begin{proof}
We take a projective resolution $P\to X$ and an injective resolution $Y\to I$. By \cref{derived category bounded RHom expression eg}, we have
\[R\Hom_{\mathcal{A}'}(L^-F(X),Y)\cong \Hom^\bullet(K^-F(P),I).\]
By the adjointness, in view of the definition of $K^-F$ and the $\Hom$ complex, the RHS is isomorphic to $\Hom^\bullet(P,K^+G(I))$, which is $R\Hom_\mathcal{A}(X,R^+G(Y))$. The last assertion follows from \cref{derived category R^0Hom is Hom in D(A)} by taking $H^0$.
\end{proof}
Note that the functors $L^-F$ and $R^+G$ are not adjoint functors, since they are functors between different pair of categories. This problem shall be resolved after we introduce the unbounded version of derived functors.

\section{Unbounded derived categories}
In this section we study the unbounded derived categories of Grothendieck categories. We prove the existence of enough homotopically injective objects in order to define unbounded right derived functors, and we prove that these triangulated categories satisfy the hypotheses of the Brown representability theorem. We also study unbounded derived functors in particular for pairs of adjoint functors. We start this study in the framework of abelian categories with translation, then we apply it to the case of the categories of unbounded complexes in abelian categories.

\subsection{Derived categories of abelian categories with translation}
Let $(\mathcal{A},T)$ be an abelian category with translation. Recall  that, denoting by $\mathcal{N}$ the triangulated subcategory of the homotopy category $K_c(\mathcal{A})$ consisting of objects $X$ quasi-isomorphic to $0$, the derived category $D_c(\mathcal{A})$ of $(\mathcal{A},T)$ is the localization $K_c(\mathcal{A})/\mathcal{N}$. Also recall that an object $X$ is quasi-isomorphic to $0$ if and only the sequence
\[\begin{tikzcd}
T^{-1}(X)\ar[r,"{T^{-1}(d_X)}"]&X\ar[r,"d_X"]&T(X)
\end{tikzcd}\]
is exact.\par
For $X\in\mathcal{A}_c$, the differential $d_X:X\to T(X)$ is a morphism in $\mathcal{A}_c$, so its cohomology $H(X)$ is regarded as an object of $\mathcal{A}_c$ and similarly for $\ker d_X$ and $\im d_X$. Note that their differentials vanish.

\begin{proposition}\label{abelian translation derive direct sum}
Assume that $\mathcal{A}$ admits direct sums indexed by a set $I$ and that such direct sums are exact. Then $\mathcal{A}_c$, $K_c(\mathcal{A})$ and $D_c(\mathcal{A})$ admit such direct sums and the two functors $\mathcal{A}_c\to K_c(\mathcal{A})$ and $K_c(\mathcal{A})\to D_c(\mathcal{A})$ commute with such direct sums.
\end{proposition}
\begin{proof}
The result concerning $\mathcal{A}_c$ and $K_c(\mathcal{A})$ is immediate, and that concerning $D_c(\mathcal{A})$ follows from \cref{triangle cat localization and direct sum}.
\end{proof}

For an object $X$ of $\mathcal{A}$, we denote by $M(X)$ the mapping cone of $\id_{T^{-1}(X)}$, regarding $T^{-1}(X)$ as an object of $\mathcal{A}_c$ with zero differential. Hence $M(X)$ is the object $X\oplus T^{-1}(X)$ of $\mathcal{A}_c$ with the differential
\[d_{M(X)}=\begin{pmatrix}
0&0\\
\id_X&0
\end{pmatrix}:X\oplus T^{-1}(X)\to T(X)\oplus X.\]
The functor $M:\mathcal{A}\to\mathcal{A}_c$ is easily seen to be exact, and $M$ is a left adjoint of the forgetful functor $\mathcal{A}_c\to\mathcal{A}$, as seen by the following lemma.

\begin{lemma}\label{abelian translation adjoint of forgetful}
For $Y\in\mathcal{A}$ and $X\in\mathcal{A}_c$, we have the isomorphism
\[\Hom_{\mathcal{A}_c}(M(Y),X)\stackrel{\sim}{\to} \Hom_\mathcal{A}(Y,X)\]
\end{lemma}
\begin{proof}
Any morphism $(u,v):M(Y)\to X$ in $\mathcal{A}_c$ satisfies $d_X\circ (u,v)=T(u,v)\circ d_{M(X)}$, which reads as $d_X\circ u=Tv$ and $d_X\circ v=0$. Therefore, it is completely determined by $u:Y\to X$.
\end{proof}

\begin{proposition}\label{abelian translation complex cat is Grothendieck}
Let $\mathcal{A}$ be a Grothendieck category. Then $\mathcal{A}_c$ is again a Grothendieck category.
\end{proposition}
\begin{proof}
The category $\mathcal{A}_c$ is abelian and admits small inductive limits, and small filtrant inductive limits in $\mathcal{A}_c$ are clearly exact. Moreover, if $G$ is a generator in $\mathcal{A}$, then $M(G)$ is a generator in $\mathcal{A}_c$ by \cref{abelian translation adjoint of forgetful}.
\end{proof}

\begin{definition}
Let $(\mathcal{A},T)$ be an abelian category with translation.
\begin{itemize}
    \item An object $I\in K_c(\mathcal{A})$ is called \textbf{homotopically injective} if $\Hom_{K_c(\mathcal{A})}(X,I)=0$ for any $X\in K_c(\mathcal{A})$ that is quasi-isomorphic to $0$.
    \item An object $P\in K_c(\mathcal{A})$ is called \textbf{homotopically projective} if $\Hom_{K_c(\mathcal{A})}(P,X)=0$ for any $X\in K_c(\mathcal{A})$ that is quasi-isomorphic to $0$.
\end{itemize}
\end{definition}

We shall denote by $K_{c,hi}(\mathcal{A})$ the full subcategory of $K_c(\mathcal{A})$ consisting of homotopically injective objects and by $\iota:K_{c,hi}(\mathcal{A})\to K_c(\mathcal{A})$ the inclusion functor. Similarly, we denote by $K_{c,hp}(\mathcal{A})$ the full subcategory of $K_c(\mathcal{A})$ consisting of homotopically projective objects. Note that $K_{c,hi}(\mathcal{A})$ is obviously a full triangulated subcategory of $K_c(\mathcal{A})$.

\begin{lemma}\label{abelian translation K-injective Hom K is D}
Let $(\mathcal{A},T)$ be an abelian category with translation. If $I\in K_c(\mathcal{A})$ is homotopically injective, then
\[\Hom_{K_c(\mathcal{A})}(X,I)\stackrel{\sim}{\to} \Hom_{D_c(\mathcal{A})}(X,I)\]
for all $X\in K_c(\mathcal{A})$.
\end{lemma}
\begin{proof}
Let $X\in K_c(\mathcal{A})$ and $X'\to X$ be a quasi-isomorphism. Then for $I\in K_{c,hi}(\mathcal{A})$, the morphism
\[\Hom_{K_c(\mathcal{A})}(X,I)\to\Hom_{K_c(\mathcal{A})}(X',I)\]
is an isomorphism since there is a distinguished triangle $X'\to X\to N\to T(X)$ with $N$ quasi-isomorphic to zero and
\[\Hom_{K_c(\mathcal{A})}(N,I)\cong \Hom_{K_c(\mathcal{A})}(T^{-1}(N),I)=0.\]
Therefore, for any $X\in K_c(\mathcal{A})$ and $I\in K_{c,hi}(\mathcal{A})$, we have
\begin{equation*}
\Hom_{D_c(\mathcal{A})}(X,I)\cong\rlim_{(X'\to X)\in\Qis}\Hom_{K_c(\mathcal{A})}(X',I)\cong\Hom_{K_c(\mathcal{A})}(X,I).\qedhere
\end{equation*}
\end{proof}

We now introduce the following notation:
\[QM=\{f\in\Mor(\mathcal{A}_c):\text{$f$ is both a quasi-isomorphism and a monomorphism}\}.\]
Recall that an object $I$ of $\mathcal{A}_c$ is called $QM$-injective if for any morphism $f:X\to Y$ in $QM$, the induced map
\[f^*:\Hom_{\mathcal{A}_c}(Y,I)\to\Hom_{\mathcal{A}_c}(X,I)\]
is surjective.

\begin{proposition}\label{abelian translation QM-injective iff}
Let $I\in\mathcal{A}_c$, then $I$ is $QM$-injective if and only if it satisfies the following conditions:
\begin{enumerate}
    \item[(a)] $I$ is homotopically injective,
    \item[(b)] $I$ is injective as an object of $\mathcal{A}$.
\end{enumerate}
\end{proposition}
\begin{proof}

\end{proof}

We shall now prove the following theorem, which asserts that any Grothendieck category has enough $QM$-injective objects.
\begin{theorem}\label{abelian translation Grothendieck enough QM-injective}
Let $(\mathcal{A},T)$ be an abelian category with translation where $\mathcal{A}$ is a Grothendieck category. Then, for any $X\in\mathcal{A}_c$, there exists $u:X\to I$ such that $u\in QM$ and $I$ is $QM$-injective.
\end{theorem}

\begin{corollary}\label{abelian translation Grothendieck enough K-injective}
Let $(\mathcal{A},T)$ be an abelian category with translation where $\mathcal{A}$ is a Grothendieck category. Then for any $X\in\mathcal{A}_c$, there exists a quasi-isomorphism $X\to I$ such that $I$ is homotopically injective.
\end{corollary}

\begin{corollary}\label{abelian translation Grothendieck derived cat prop}
Let $(\mathcal{A},T)$ be an abelian category with translation where $\mathcal{A}$ is a Grothendieck category. Then
\begin{enumerate}
    \item[(a)] the localization functor $Q:K_c(\mathcal{A})\to D_c(\mathcal{A})$ induces an equivalence $Q\circ\iota:K_{c,hi}(\mathcal{A})\stackrel{\sim}{\to} D_c(\mathcal{A})$;
    \item[(b)] the category $D_c(\mathcal{A})$ is a $\mathscr{U}$-category;
    \item[(c)] the functor $Q:K_c(\mathcal{A})\to D_c(\mathcal{A})$ admits a right adjoint $R:D_c(\mathcal{A})\to K_c(\mathcal{A})$ so that $Q\circ R\cong\id$, and $R$ is the composition of $\iota:K_{c,hi}(\mathcal{A})\to K_c(\mathcal{A})$ and a quasi-inverse of $Q\circ\iota$;
    \item[(d)] for any triangulated category $D$, any triangulated functor $F:K_c(\mathcal{A})\to\mathcal{D}$ admits a right localization $RF:D_c(\mathcal{A})\to\mathcal{D}$, and $RF\cong F\circ R$.
\end{enumerate}
\end{corollary}
\begin{proof}
The functor $Q:K_{c,hi}(\mathcal{A})$ is fully faithful by \cref{abelian translation K-injective Hom K is D} and essentially surjective by \cref{abelian translation Grothendieck enough K-injective}, whence assertion (a), and (b), (c) follow immediately: in fact, for $X\in K_c(\mathcal{A})$ and $Y\in D_c(\mathcal{A})$, we have $(Q\circ\iota)^{-1}(Y)\in K_{c,hi}(\mathcal{A})$, so by \cref{abelian translation K-injective Hom K is D},
\begin{align*}
\Hom_{K_c(\mathcal{A})}(X,\iota\circ(Q\circ\iota)^{-1}(Y))&\cong \Hom_{K_c(\mathcal{A})}(X,(Q\circ\iota)^{-1}(Y))\\
&\cong \Hom_{D_c(\mathcal{A})}(Q(X),(Q\circ\iota)\circ(Q\circ\iota)^{-1}(Y))\\
&\cong\Hom_{D_c(\mathcal{A})}(Q(X),Y).
\end{align*}
Finally, (d) follows from \cref{category localization of functor exist if} and (c).
\end{proof}

\subsection{The Brown representability theorem}
We shall show that the hypotheses of the Brown representability theorem are satisfied for $D_c(\mathcal{A})$ when $\mathcal{A}$ is a Grothendieck abelian category with translation. Note that $D_c(\mathcal{A})$ admits small direct sums and the localization functor $Q:K_c(\mathcal{A})\to D_c(\mathcal{A})$ commutes with such direct sums by \cref{abelian translation derive direct sum}.
\begin{theorem}\label{abelian translation Grothendieck derived Brown representability}
Let $(\mathcal{A},T)$ be an abelian category with translation where $\mathcal{A}$ is a Grothendieck category. Then the triangulated category $D_c(\mathcal{A})$ admits small direct sums and a system of $t$-generators.
\end{theorem}
Applying (\cite{kashiwara_SAC} corollary 10.5.2 and corollary 10.5.3), we then obtain the following corollaries.
\begin{corollary}\label{abelian translation Grothendieck derived presheaf representable if}
Let $(\mathcal{A},T)$ be an abelian category with translation where $\mathcal{A}$ is a Grothendieck category. Let $G:D^c(\mathcal{A})^{\op}\to\mathbf{Mod}(\Z)$ be a cohomological functor which commutes with small products, then $G$ is representable.
\end{corollary}
\begin{corollary}\label{abelian translation Grothendieck derived triangulated functor adjoint if}
Let $(\mathcal{A},T)$ be an abelian category with translation where $\mathcal{A}$ is a Grothendieck category. Let $\mathcal{D}$ be a triangulated category and $F:D_c(\mathcal{A})\to\mathcal{D}$ be a triangulated functor. Assume that $F$ commutes with small direct sums, then $F$ admits a right adjoint.
\end{corollary}

We shall prove a slightly more general statement than \cref{abelian translation Grothendieck derived Brown representability}. Let $\mathcal{I}$ be a full subcategory of $\mathcal{A}$ closed by subobjects, quotients and extensions in $\mathcal{A}$, and also by small direct sums. Let us denote by $D_{c,\mathcal{I}}(\mathcal{A})$ the full subcategory of $D_c(\mathcal{A})$ consisting of objects $X\in D_c(\mathcal{A})$ such that $H(X)\in\mathcal{I}$. Then $D_{c,\mathcal{I}}(\mathcal{A})$ is a full triangulated subcategory of $D_c(\mathcal{A})$ closed by small direct sums.

\begin{proposition}\label{abelian translation Grothendieck t-generator if}
The triangulated category $D_{c,\mathcal{I}}(\mathcal{A})$ admits a system of $t$-generators.
\end{proposition}


\subsection{Unbounded derived category}
From now on and until the end of this section, we consider abelian categories $\mathcal{A}$, $\mathcal{A}'$, etc. We shall apply the results in the preceding paragraphs to the abelian category with translation given by shifting. Then we have $\mathcal{A}_c\cong\Ch(\mathcal{A})$, $K_c(\mathcal{A})\cong K(\mathcal{A})$ and $D_c(\mathcal{A})\cong D(\mathcal{A})$. Assume that $\mathcal{A}$ admits direct sums indexed by a set $I$ and that such direct sums are exact. Then, clearly, $\mathcal{A}_c$ has the same properties, so it follows from \cref{abelian translation derive direct sum} that $\Ch(\mathcal{A})$, $K(\mathcal{A})$ and $D(\mathcal{A})$ also admit such direct sums and the two functors $\Ch(\mathcal{A})\to K(\mathcal{A})$ and $K(\mathcal{A})\to D(\mathcal{A})$ commute with such direct sums.\par
We shall write $K_{hi}(\mathcal{A})$ for $K_{c,hi}(\mathcal{A})$, so $K_{hi}(\mathcal{A})$ is the full subcategory of $K(\mathcal{A})$ consisting of homotopically injective objects. Let us denote by $\iota:K_{hi}(\mathcal{A})\to K(\mathcal{A})$ the inclusion functor. Similarly we denote by $K_{hp}(\mathcal{A})$ the full subcategory of $K(\mathcal{A})$ consisting of homotopically projective objects. Recall that $I\in K(\mathcal{A})$ is homotopically injective if and only if $\Hom_{K(\mathcal{A})}(X,I)=0$ for all $X\in K(\mathcal{A})$ that is quasi-isomorphism to $0$. Note that an object $I\in K^+(\mathcal{A})$ whose components are all injective is homotopically injective in view of \cref{abelian cat injective qis zero is null-homotopy}.\par
Let $\mathcal{A}$ be a Grothendieck abelian category. Then $\Ch(\mathcal{A})$ is also a Grothendieck category, so applying \cref{abelian translation Grothendieck enough K-injective} and \cref{abelian translation Grothendieck derived Brown representability}, we get the following theorem.
\begin{theorem}\label{abelian Grothendieck unbounded derived category prop}
Let $\mathcal{A}$ be a Grothendieck category.
\begin{enumerate}
    \item[(\rmnum{1})] if $I\in K(\mathcal{A})$ is homotopically injective, then we have an isomorphism
    \[\Hom_{K(\mathcal{A})}(X,I)\stackrel{\sim}{\to} \Hom_{D(\mathcal{A})}(X,I)\]
    for any $X\in K(\mathcal{A})$;
    \item[(\rmnum{2})] for any $X\in\Ch(\mathcal{A})$, there exists a quasi-isomorphism $X\to I$ such that $I$ is homotopically injective;
    \item[(\rmnum{3})] the localization functor $Q:K(\mathcal{A})\to D(\mathcal{A})$ induces an equivalence $K_{hi}(\mathcal{A})\stackrel{\sim}{\to} D(\mathcal{A})$;
    \item[(\rmnum{4})] the category $D(\mathcal{A})$ is a $\mathscr{U}$-category;
    \item[(\rmnum{5})] the functor $Q:K(\mathcal{A})\to D(\mathcal{A})$ admits a right adjoint $R:D(\mathcal{A})\to K(\mathcal{A})$ such that $Q\circ R\cong\id$ and $R$ is the composition of $\iota:K_{hi}(\mathcal{A})\to K(\mathcal{A})$ and a quasi-inverse of $Q\circ\iota$;
    \item[(\rmnum{6})] for any triangulated category $\mathcal{D}$, any triangulated functor $F:K(\mathcal{A})\to\mathcal{D}$ admits a right localization $RF:D(\mathcal{A})\to\mathcal{D}$ and $RF\cong F\circ R$;
    \item[(\rmnum{7})] the triangulated category $D(\mathcal{A})$ admits small direct sums and a system of $t$-generators;
    \item[(\rmnum{8})] any cohomological functor $G:D(\mathcal{A})^{\op}\to\mathbf{Mod}(\Z)$ is representable as soon as $G$ commutes with small products;
    \item[(\rmnum{9})] for any triangulated category $\mathcal{D}$, any triangulated functor $F:D(\mathcal{A})\to\mathcal{D}$ admits a right adjoint as soon as $F$ commutes with small direct sum.
\end{enumerate}
\end{theorem}

\begin{corollary}\label{abelian Grothendieck unbounded derived category RHom}
Let $k$ be a commutative ring and let $\mathcal{A}$ be a Grothendieck
k-abelian category. Then $(K(\mathcal{A})^{\op},K_{hi}(\mathcal{A}))$ is $\Hom_{\mathcal{A}}$-injective and the functor $\Hom_\mathcal{A} $admits a right derived functor $R\Hom_\mathcal{A}:D(\mathcal{A})^{\op}\times D(\mathcal{A})\to D(k)$. Moreover, we have
\[H^n(R\Hom_\mathcal{A}(X,Y))\cong \Hom_{D(\mathcal{A})}(X,Y)\]
for $X,Y\in D(\mathcal{A})$.
\end{corollary}
\begin{proof}
The functor $\Hom_\mathcal{A}$ induces a functor $\Hom_\mathcal{A}^\bullet:K(\mathcal{A})^{\op}\times K(\mathcal{A})\to K(k)$ and by (\cite{kashiwara_SAC}, proposition 11.7.3) we have
\[H^n(\Hom_\mathcal{A}^\bullet(X,Y))=\Hom_{K(\mathcal{A})}(X,Y[n]).\]
Let $I\in K_{hi}(\mathcal{A})$; if $X\in K(\mathcal{A})$ is quasi-isomorphic to zero, then $\Hom_{K(\mathcal{A})}(X,I)=0$. Moreover, if $I\in K_{hi}(\mathcal{A})$ is quasi-isomorphic to zero, then $I$ is isomorphic to zero. Therefore $(K(\mathcal{A})^{\op},K_{hi}(\mathcal{A}))$ is $\Hom_{\mathcal{A}}$-injective and we can apply \cref{derived category bifunctor derive exists if}. The last assertion follows from \cref{derived category R^0Hom is Hom in D(A)}.
\end{proof}

\begin{remark}
Let $\mathcal{I}$ be a full subcategory of a Grothendieck category $\mathcal{A}$ and assume that $\mathcal{I}$ is closed by subobjects, quotients and extensions in $\mathcal{A}$, and also by small direct sums. Then by \cref{abelian translation Grothendieck t-generator if}, the triangulated category $D_\mathcal{I}(\mathcal{A})$ admits small direct sums and a system of $t$-generators. Hence $D_\mathcal{I}(\mathcal{A})\to D(\mathcal{A})$ has a right adjoint.
\end{remark}

\subsection{Left derived functors}
We now give a criterion for the existence of the left derived functor $LG:D(\mathcal{A})\to D(\mathcal{A}')$ of an additive functor $G:\mathcal{A}\to\mathcal{A}'$ of abelian categories, assuming that $G$ admits a right adjoint. For this, we shall assume throughout this paragraph that $\mathcal{A}$ admits small direct sums and small direct sums are exact in $\mathcal{A}$. Hence, by \cref{abelian translation derive direct sum}, $\Ch(\mathcal{A})$, $K(\mathcal{A})$ and $D(\mathcal{A})$ admit small direct sums. Note that Grothendieck categories satisfy these conditions.

\section{Truncations and recollements}
\subsection{Abelian subcategories}
Let $\mathcal{D}$ be a triangulated category. For $X,Y\in\mathcal{D}$ and $i\in\Z$, we denote by $\Hom^i(X,Y):=\Hom(X,Y[i])$ the morphisms from $X$ to $Y$ of \textbf{degree $\bm{i}$}.
\begin{proposition}\label{triangle cat morphism extension to triangle iff}
Let $(X,Y,Z)$ and $(X',Y',Z')$ be distinguished triangles, and $\beta:Y\to Y'$ be a morphism such that we have a solid commutative diagram
\[\begin{tikzcd}[row sep=12mm,column sep=12mm]
X\ar[r,"f"]\ar[rd,phantom,font=\small,"(A)"]\ar[d,dashed,"\alpha"]&Y\ar[r,"g"]\ar[d,"\beta"]\ar[rd,phantom,"(B)"]&Z\ar[r,"h"]\ar[d,dashed,"\gamma"]&X[1]\\
X\ar[r,"f'"]&Y\ar[r,"g'"]&Z\ar[r,"h'"]&X'[1]
\end{tikzcd}\]
Then the following conditions are equivalent:
\begin{enumerate}
    \item[(\rmnum{1})] $g'\beta f=0$;
    \item[(\rmnum{2})] there eixsts a morphism $\alpha:X\to X'$ rendering the commutative square (A);
    \item[(\rmnum{3})] there eixsts a morphism $\gamma:Z\to Z'$ rendering the commutative square (B);
    \item[(\rmnum{4})] there exists a morphism of triangles $(\alpha,\beta,\gamma)$. 
\end{enumerate}
If these conditions are verified and $\Hom^{-1}(X,Z')=0$, then the morphisms $\alpha$, $\beta$ are unique.
\end{proposition}
\begin{proof}
The exactness of the sequence
\[\begin{tikzcd}
\Hom^{-1}(X,Z')\ar[r]&\Hom(X,X')\ar[r]&\Hom(X,Y')\ar[r]&\Hom(X,Z')
\end{tikzcd}\]
applied to $\beta f\in\Hom(X,Y')$, proves the equivalence of (\rmnum{1}) and (\rmnum{2}), with the uniqueness of $\alpha$ if $\Hom^{-1}(X,Z)=0$. The implication (\rmnum{2})$\Rightarrow$(\rmnum{4}) follows from (TR2): if $\alpha$ satisfies (\rmnum{2}), then there exists $\gamma:Z\to Z'$ such that $(\alpha,\beta,\gamma)$ is a morphism of triangles; the converse of this is trivial. Finally, a dual argument shows that (\rmnum{1})$\Leftrightarrow$(\rmnum{3}), and the uniqueness of $\gamma$ if $\Hom^{-1}(X,Z')=0$.
\end{proof}

\begin{corollary}\label{triangle cat Hom^-1 zero prop}
Let $X\stackrel{f}{\to} Y\stackrel{g}{\to} Z\stackrel{h}{\to} X[1]$ be a distinguished triangle. If $\Hom^{-1}(X,Z)=0$, then
\begin{enumerate}
    \item[(a)] the cone of $f$ is unique up to unique isomorphisms;
    \item[(b)] the morphism $h:Z\to X[1]$ is the unique morphism such that the triangle $X\stackrel{f}{\to} Y\stackrel{g}{\to} Z\to X[1]$ is distinguished. 
\end{enumerate}
\end{corollary}
\begin{proof}
If in \cref{triangle cat morphism extension to triangle iff}, we set $X=X'$, $Y=Y'$ and let $f,g$ be the identity, then $Z$ is isomorphic to $Z'$ and $\Hom^{-1}(X,Z')=0$, so (a) follows from the uniqueness of $\gamma$. For (b), we can apply {triangle cat morphism extension to triangle iff} to the diagram
\[\begin{tikzcd}
X\ar[r,"f"]\ar[d,equal]&Y\ar[r,"g"]\ar[d,equal]&Z\ar[r,"h"]&X[1]\\
X\ar[r,"f"]&Y\ar[r,"g"]&Z\ar[r,"h'"]&X[1]
\end{tikzcd}\]
We necessarily have $\gamma=\id_Z$, whence $h=h'$.
\end{proof}

Let $\mathcal{C}$ be a (fixed) full subcategory of $\mathcal{D}$ such that $0\in\mathcal{C}$. We say that $\mathcal{C}$ is \textbf{right leaning} if $\Hom^i_{\mathcal{D}}(X,Y):=\Hom_{\mathcal{D}}(X,Y[i])$ is null for any $i<0$ and $X,Y\in\mathcal{C}$. For such a category $\mathcal{C}$, by \cref{triangle cat Hom^-1 zero prop}, we see that a morphism $(\alpha,\beta,\gamma)$ of distinguished triangles in $\mathcal{C}$ is uniquely determined by the morphism $\beta$.
\begin{example}
Let $\mathcal{D}$ be the derived category of an abelian category $\mathcal{A}$ and $\mathcal{C}=\mathcal{A}$, identified with a full subcategory of $\mathcal{D}$. It is clear that for $X,Y\in\mathcal{A}$ we have $\Hom(X,Y[i])=0$ unless $i=0$, so $\mathcal{A}$ is right leaning in $\mathcal{D}$.
\end{example}

\begin{proposition}\label{triangle cat abelian subcat kernel cokernel prop}
Let $f:X\to Y$ be a morphism in $\mathcal{C}$. Complete $f$ into a distinguished triangle $X\stackrel{f}{\to}Y\to S$ and suppose that there is a distinguished triangle $N[1]\to S\to C$ with $N,C\in\mathcal{C}$, so that we have a commutative diagram
\begin{equation}\label{triangle cat abelian subcat kernel cokernel prop-1}
\begin{tikzcd}[row sep=6mm,column sep=6mm]
N[1]\ar[rd]\ar[dd,swap,"\alpha"]&&C\ar[ll,swap,"+1"]\\
&S\ar[ld,"+1"]\ar[ru]&\\
X\ar[rr,swap,"f"]&&Y\ar[lu]\ar[uu,swap,"\beta"]
\end{tikzcd}
\end{equation}
Then $\alpha[-1]:N\to X$ is a kernel of $f$ in $\mathcal{C}$ and $\beta:Y\to C$ is a cokernel of $f$ in $\mathcal{C}$.
\end{proposition}
\begin{proof}
For $Z\in\mathcal{C}$, the long exact sequence of $\Hom$ gives the following exact sequences
\[\begin{tikzcd}
0\ar[r]&\Hom^{-1}(Z,S)\ar[r]&\Hom(Z,X)\ar[r]&\Hom(Z,Y)
\end{tikzcd}\]
\vspace*{-4mm}
\[\begin{tikzcd}
0\ar[r]&\Hom(Z,N)\ar[r]&\Hom^{-1}(Z,S)\ar[r]&\Hom^{-1}(Z,C)=0
\end{tikzcd}\]
which proves that $(N,\alpha[-1])$ is a kernel of $f$ (recall the hypothesis on $\mathcal{C}$). A dual argument proves that $(C,\beta)$ is a cokernel of $f$.
\end{proof}

\begin{example}\label{triangle cat abelian subcat as derived eg}
Let $\mathcal{A}$ be an abelian category and $\mathcal{D}=D(\mathcal{A})$. Then the cone $S$ of $f:X\to Y$ is the complex $X\stackrel{f}{\to} Y$, with $X$ at degree $-1$ and $Y$ at degree $0$. It has a subcomplex $(\ker f)[1]=H^{-1}(S)[1]$ and the quotient $X/\ker f\to Y$ is quasi-isomorphic to the complex $\coker f=H^0(S)$, placed at degree $0$; we thus obtain the diagram (\ref{triangle cat abelian subcat kernel cokernel prop-1}).
\end{example}

\begin{example}\label{triangle cat abelian subcat exact sequence eg}
In the situation of \cref{triangle cat abelian subcat kernel cokernel prop}, if $f$ is a monomorphism, then we have $N=0$, so $S\cong C$ and (\ref{triangle cat abelian subcat kernel cokernel prop-1}) is reduced to the distinguished triangle $(X,Y,C)$. On the other hand, if $f$ is an epimorphic, then $C=0$, so $N[1]\cong S$ and (\ref{triangle cat abelian subcat kernel cokernel prop-1}) is reduced to the distinguished triangle $(N,X,Y)$.
\end{example}

A morphism $f:X\to Y$ in $\mathcal{C}$ is called \textbf{$\mathcal{C}$-admissible}, or simply \textbf{admissible} (if there is no ambiguity on $\mathcal{C}$), if it is the base of a diagram (\ref{triangle cat abelian subcat kernel cokernel prop-1}). For any distinguished triangle
\[\begin{tikzcd}
X\ar[r,"f"]&Y\ar[r,"g"]&Z\ar[r]&X[1]
\end{tikzcd}\]
with $X,Y,Z$ in $\mathcal{C}$ and $f,g$ admissible, we see that $f$ is a kernel of $g$ and $g$ is a cokernel of $f$. By \cref{triangle cat Hom^-1 zero prop}, the morphism $Z\to X[1]$ is uniquely determined by $f$ and $g$.\par
A sequence $X\to Y\to Z$ in $\mathcal{C}$ is called an \textbf{admissible short exact sequence} if it is deduced from a distinguished triangle by removing the arrow of degree $1$. In other words, $X\to Y\to Z$ is admissible if it can be extended into a distinguished triangle in $\mathcal{D}$.

\begin{proposition}\label{triangle cat abelian subcat iff morphism admissible}
Suppose that $\mathcal{C}$ is stable under finite direct sums. Then the following conditions are equivalent:
\begin{enumerate}
    \item[(\rmnum{1})] $\mathcal{C}$ is an abelian category and every short exact sequence is admissible.
    \item[(\rmnum{2})] Any morphism of $\mathcal{C}$ is $\mathcal{C}$-admissible.
\end{enumerate}
\end{proposition}
\begin{proof}
We first assume that any morphism in $\mathcal{C}$ is admissible. By \cref{triangle cat abelian subcat kernel cokernel prop}, any morphism of $\mathcal{C}$ has a kernel and cokernel, so to prove that $\mathcal{C}$ is abelian, it suffices to verify that $\coim f\cong\im f$ for any morphism $f:X\to Y$ in $\mathcal{C}$. Regarding (\ref{triangle cat abelian subcat kernel cokernel prop-1}) as the cap of an octahedron and apply (TR4), we obtain an octahedron
\[\begin{tikzcd}
C\ar[dd,swap,"+1"]&&Y\ar[ll,swap,"\beta"]\ar[ld]\\
&S\ar[rd,"+1"]\ar[lu]&\\
N[1]\ar[ru]\ar[rr,swap,"{\alpha[1]}"]&&X\ar[uu,swap,"f"]
\end{tikzcd}\quad\quad 
\begin{tikzcd}
C\ar[dd,swap,"+1"]\ar[rd]&&Y\ar[ll,swap,"\beta"]\\
&I\ar[ld,"+1"]\ar[ru]&\\
N[1]\ar[rr,swap,"{\alpha[1]}"]&&X\ar[uu,swap,"f"]\ar[lu]
\end{tikzcd}\]
By \cref{triangle cat abelian subcat kernel cokernel prop}, we see that $\beta$ is an (admissible) epimorphism (as the cokernel of $f$). Since the triangle $(I,Y,C)$ is distinguished, we conclude from \cref{triangle cat abelian subcat exact sequence eg} that $I\in\mathcal{C}$ and it is the the kernel of $\beta$, i.e. the image of $f$. Dually, the distinguished triangle $(N,X,I)$, obtained by rotating, shows that $I$ is the coimage of $f$. Finally, by \cref{triangle cat abelian subcat exact sequence eg}, we see that any short exact sequence is admissible.\par
Conversely, assume the codition (\rmnum{1}). The kernel $N$, cokernel $C$ and image $I$ of $f:X\to Y$ then give short exact sequences
\[\begin{tikzcd}
0\ar[r]&N\ar[r]&X\ar[r]&I\ar[r]&0
\end{tikzcd}\quad\quad \begin{tikzcd}
0\ar[r]&I\ar[r]&Y\ar[r]&C\ar[r]&0
\end{tikzcd}\]
By hypothesis, these two sequences are admissible, and the two thus obtained triangles form the upper cap of an octahedron. We can then apply (TR4) to obtain an octahedron of the above form, so $f$ is admissible.
\end{proof}

A full subcategory $\mathcal{C}$ of $\mathcal{D}$ is called \textbf{admissibly abelian} if it is right leaning and satisfies the equivalent conditions of \cref{triangle cat abelian subcat iff morphism admissible}. By \cref{triangle cat abelian subcat as derived eg}, we see that if $\mathcal{D}$ is the derived category of an abelian category $\mathcal{A}$, then $\mathcal{A}$ is admissibly abelian in $\mathcal{D}$.

\subsection{\texorpdfstring{$t$}{t}-structures}
Let $\mathcal{D}$ be a triangulated category. A \textbf{$\bm{t}$-structure} on $\mathcal{D}$ is a pair of strictly full subcategories $(\mathcal{D}^{\leq 0},\mathcal{D}^{\geq 0}$) satisfying the following conditions: If we put $\mathcal{D}^{\leq n}=T^{-n}(\mathcal{D}^{\leq 0})$ and $\mathcal{D}^{\geq n}=T^{-n}(\mathcal{D}^{\geq 0})$, then
\begin{enumerate}
    \item[(t1)] $\mathcal{D}^{\leq 0}\sub\mathcal{D}^{\leq 1}$ and $\mathcal{D}^{\geq 0}\sups\mathcal{D}^{\geq 1}$;
    \item[(t2)] $\Hom_{\mathcal{D}}(X,Y)=0$ for $X\in\mathcal{D}^{\leq 0}$ and $Y\in\mathcal{D}^{\geq 1}$; 
    \item[(t3)] for any $X\in\mathcal{D}$, there exists a distinguished triangle $(A,X,B)$ with $A\in\mathcal{D}^{\leq 0}$ and $B\in\mathcal{D}^{\geq 1}$.
\end{enumerate}
The \textbf{core} (or \textbf{heart}) of the $t$-structure $(\mathcal{D}^{\leq 0},\mathcal{D}^{\geq 0})$ is defined to be the full subcategory $\mathcal{C}=\mathcal{D}^{\leq 0}\cap\mathcal{D}^{\geq 0}$.

\begin{example}[\textbf{Examples of $t$-structures}]
\mbox{}
\begin{enumerate}
    \item[(a)] Let $\mathcal{A}$ be an abelian category and $\mathcal{D}=\mathcal{D}(A)$. The natural $t$-structure on $\mathcal{D}(A)$ is then defined so that $X\in\mathcal{D}^{\leq n}$ (resp. $X\in\mathcal{D}^{\geq n}$) if and only if $H^i(X)=0$ for $i>n$ (resp. for $i<n$). To verify (t3), we note that for any complex $X$, the truncations $\tau^{\leq 0}X$ and $\tau^{\geq 1}X$ are in $\mathcal{D}^{\leq 0}$ and $\mathcal{D}^{\geq 1}$, respectively, and we have a distinguished triangle $(\tau^{\leq 0}X,X,\tau^{\geq 1}X)$. 
    \item[(b)] If $(\mathcal{D}^{\leq 0},\mathcal{D}^{\geq 0})$ is a $t$-structure on $\mathcal{D}$, then for any integer $n$, $(\mathcal{D}^{\leq 0},\mathcal{D}^{\geq 0})$ is also a $t$-structure. We say that this $t$-structure is induced from the previous one by \textit{translation}.
    \item[(c)] If $(\mathcal{D}^{\leq 0},\mathcal{D}^{\geq 0})$ is a $t$-structure on $\mathcal{D}$, then $((\mathcal{D}^{\leq 0})^{\op},(\mathcal{D}^{\geq 0})^{\op})$ is a $t$-structure on $\mathcal{D}^{\op}$, called the dual $t$-structure.
\end{enumerate}
\end{example}
A triangulated category $\mathcal{D}$, endowed with a $t$-structure $(\mathcal{D}^{\leq 0},\mathcal{D}^{\geq 0})$, is called a \textbf{$\bm{t}$-category}.

\begin{proposition}\label{triangle cat t-structure truncation Hom prop}
Let $\mathcal{D}$ be a $t$-category. Then for integers $m<n$, we have $\Hom_{\mathcal{D}}(X,Y)=0$ if $X\in\mathcal{D}^{\leq m}$ and $Y\in\mathcal{D}^{\geq n}$.
\end{proposition}
\begin{proof}

\end{proof}

\begin{proposition}\label{triangle cat t-structure truncation functor prop}
Let $\mathcal{D}$ be a $t$-category.
\begin{enumerate}
    \item[(a)] The inclusion functor $\mathcal{D}^{\leq n}\to\mathcal{D}$ admits a right adjoint $\tau^{\leq n}$, and $\mathcal{D}^{\geq n}\to\mathcal{D}$ admits a left adjoint $\tau^{\geq n}$.
    \item[(b)] For any $X\in\mathcal{D}$, there eixsts a unique morphism in $\Hom^1(\tau^{\geq 1}X,\tau^{\leq 0}X)$ such that the triangle
    \[\begin{tikzcd}
    \tau^{\leq 0}X\ar[r]&X\ar[r]&\tau^{\geq 1}X\ar[r,"+1"]&{}
    \end{tikzcd}\]
    is distinguished. Moreover, up to isomorphisms, this is the unique distinguished $(A,X,B)$ such that $A\in\mathcal{D}^{\leq 0}$ and $B\in\mathcal{D}^{\geq 1}$.
\end{enumerate}
\end{proposition}
\begin{proof}
By duality and translation, it suffices to prove (a) for $\mathcal{D}^{\leq 0}$. For each $X\in\mathcal{D}$, we need to find $A\in\mathcal{D}^{\leq 0}$ with a morphism $A\to X$ (the value of $\tau^{\leq 0}$ at $X$), such that for any $T\in\mathcal{D}^{\leq 0}$ we have an isomorphism $\Hom(T,A)\cong\Hom(T,X)$. Let $(A,X,B)$ be a triangle as in (t3). The long exact sequence of $\Hom$, together with (t1), (t2), shows that $\Hom(T,A)\cong\Hom(T,X)$, so we set $A=\tau^{\leq 0}X$. A similar argument shows that $B=\tau^{\geq 1}X$, so there is a distinguished triangle $(\tau^{\leq 0}X,X,\tau^{\geq 1}X)$ and any distinguished triangle in (t3) is isomorphic to this triangle. The uniqueness of these isomorphisms follows from (t2) and \cref{triangle cat morphism extension to triangle iff}.
\end{proof}

The distinguished triangle $(\tau^{\leq 0}X,X,\tau^{\geq 1}X)$ shows that the following conditions are equivalent:
\begin{enumerate}
    \item[(\rmnum{1})] $\tau^{\leq 0}X=0$;
    \item[(\rmnum{2})] $\Hom(T,X)=0$ for $T\in\mathcal{D}^{\leq 0}$;
    \item[(\rmnum{3})] $X\cong\tau^{\geq 1}X$.
\end{enumerate}
The equivalence (\rmnum{2})$\Leftrightarrow$(\rmnum{3}) means that $\mathcal{D}^{\geq 1}$ is the right orthogonal of $\mathcal{D}^{\leq 0}$, which shows that $\mathcal{D}^{\geq 1}$ is stable under exensions\footnote{We recall that an object $Y$ is called an extensin of $Z$ by $X$ if there is a distinguished triangle $(X,Y,Z)$ in $\mathcal{D}$, and a subcategory $\mathcal{D}'$ of $\mathcal{D}$ is stable under extensions if for any distinguished triangle $(X,Y,Z)$ with $X,Z\in\mathcal{D}'$, we have $Y\in\mathcal{D}'$.}. Dually, we see that $\tau^{\geq 1}X=0$ if and only if $X\in\mathcal{D}^{\leq 0}$, and $\mathcal{D}^{\leq 0}$ is the left orthogonal of $\mathcal{D}^{\geq 1}$, which is stable under extensions. In particular, $\mathcal{D}^{\leq 0}$ and $\mathcal{D}^{\geq 1}$ are stable under finite direct sums.\par

For $a\leq b$, we have $\mathcal{D}^{\leq a}\sub\mathcal{D}^{\leq b}$, so there exists a unique morphism $\tau^{\leq a}X\to\tau^{\leq b}X$ so that we have a commutative diagram
\[\begin{tikzcd}[row sep=6mm,column sep=6mm]
\tau^{\leq a}\ar[rd]\ar[rr]&&\tau^{\leq b}X\ar[ld]\\
&X&
\end{tikzcd}\]
which identifies $\tau^{\leq a}X$ with $\tau^{\leq a}\tau^{\leq b}X$. Dually, we have $\tau^{\geq a}X\to\tau^{\geq b}X$, which identifies $\tau^{\geq b}X$ with $\tau^{\geq b}\tau^{\geq a}X$.\par
For any integer $a$, we write $\tau^{>a}$ for $\tau^{\geq a+1}$ and $\tau^{<a}$ for $\tau^{<a-1}$. We deduce by translation that $X\in\mathcal{D}^{\leq a}$ if and only if $\tau^{>a}X=0$. If $b>a$, then we have $\tau^{\geq b}X=0$ in this case, and for $b\leq a$ we have $\tau^{>a}\tau^{\geq b}X=\tau^{>a}X=0$, and hence $\tau^{\geq b}$ sends $\mathcal{D}^{\leq a}$ into itself.

\begin{proposition}\label{triangle cat t-structure truncation iteration prop}
Let $a\leq b$ be integers. For $X\in\mathcal{D}$, there exists a unique isomorphism $\tau^{\geq a}\tau^{\leq b}X\to\tau^{\leq b}\tau^{\geq a}X$ rendering the following commutative diagram
\begin{equation}\label{triangle cat t-structure truncation iteration prop-1}
\begin{tikzcd}
\tau^{\leq b}X\ar[d]\ar[r]&X\ar[r]&\tau^{\geq a}X\\
\tau^{\geq a}\tau^{\leq b}X\ar[rr]&&\tau^{\leq b}\tau^{\geq a}X\ar[u]
\end{tikzcd}
\end{equation}
\end{proposition}
\begin{proof}
Since $\tau^{\geq a}X\in\mathcal{D}^{\geq a}$, we see that the canonical morphism $\tau^{\leq b}X\to\tau^{\geq a}X$ is obtained by the composition $\tau^{\leq b}X\to\tau^{\geq a}\tau^{\leq b}X\to\tau^{\geq a}X$. Also, since $\tau^{\geq a}\tau^{\leq b}X\in\mathcal{D}^{\leq b}$, the morphism $\tau^{\leq b}X\to\tau^{\geq a}X$ factors through $\tau^{\leq b}\tau^{\geq a}X$, so we obtain the diagram (\ref{triangle cat t-structure truncation iteration prop-1}). Applying (TR4) to $\tau^{<a}X\to\tau^{\leq b}X\to X$, we obtain an octahedron
\vspace*{-2mm}
\begin{equation}\label{triangle cat t-structure truncation iteration prop-2}
\begin{tikzcd}[row sep=2pt,column sep=4pt]
&&&{}&\\
&&&{}&\\
&&Y\ar[ruu]\ar[rdd]&&\\
&&&{}&{}\\
&\tau^{\leq b}X\ar[rd]\ar[ruu]&&\tau^{\geq a}X\ar[ru]\ar[rddd]\\
&&X\ar[ru]\ar[rrdd]&&\\
&&&&\\
\tau^{<a}X\ar[ruuu]\ar[rruu]&&&&\tau^{>b}X\ar[rdd]\ar[rrd]\\
&&&&&&{}\\
&&&&&{}&
\end{tikzcd}
\vspace*{-2mm}
\end{equation}
In this octahedron, $Y$ is both isomorphic to $\tau^{\geq a}\tau^{\leq b}X$ (because of the distinguished triangle $(\tau^{<a}X,\tau^{\leq b}X,Y)$, in which $\tau^{<a}X=\tau^{<a}\tau^{\leq b}X$) and to $\tau^{\geq b}\tau^{\leq a}X$ (because of the distinguished triangle $(Y,\tau^{\geq a}X,\tau^{>b}X)$).
\end{proof}

\begin{corollary}\label{triangle cat t-structure truncation orthogonal prop}
Let $m,n\in\Z$ with $m<n$. Then we have $\tau^{\leq m}\tau^{\geq n}=\tau^{\geq n}\tau^{\leq m}=0$.
\end{corollary}
\begin{proof}
This follows from \cref{triangle cat t-structure truncation Hom prop}.
\end{proof}

Because of \cref{triangle cat t-structure truncation iteration prop}, we write $\tau^{[a,b]}X:=\tau^{\geq a}\tau^{\leq b}X\cong\tau^{\leq b}\tau^{\geq a}X$. It is clear that we thus obtain a functor $\tau^{[a,b]}:\mathcal{D}\to\mathcal{D}^{[a,b]}:=\mathcal{D}^{\leq b}\cap\mathcal{D}^{\geq a}$.\par

Let $\mathcal{A}$ be an abelian category and $\mathcal{D}=\mathcal{D}(A)$ be the derived category of $\mathcal{A}$. Then the natural $t$-structure $(\mathcal{D}^{\leq 0},\mathcal{D}^{\geq 0})$ has the property that $\mathcal{A}\cong\mathcal{D}^{\leq 0}\cap\mathcal{D}^{\geq 0}$. We now prove that this is true for any $t$-structure of a triangulated category $\mathcal{D}$. That is, the heart $\mathcal{C}=\mathcal{D}^{\leq 0}\cap\mathcal{D}^{\geq 0}$ is an abelian category.

\begin{theorem}\label{triangle cat t-structure heart admissible abelian}
The heart $\mathcal{C}=\mathcal{D}^{\leq 0}\cap\mathcal{D}^{\geq 0}$ of a $t$-category $\mathcal{D}$ is an admissible abelian category of $\mathcal{D}$, which is stable under extensions. Moreover, the functor $H^0:=\tau^{\geq 0}\tau^{\leq 0}:\mathcal{D}\to\mathcal{C}$ is a cohomological functor.
\end{theorem}
\begin{proof}
Let $X,Y\in\mathcal{C}$ and $f:X\to Y$ be a morphism with cone $S$. The distinguished triangle $(Y,S,X[1])$ then shows that $S$ is in $\mathcal{D}^{\leq 0}\cap\mathcal{D}^{\geq -1}$. The truncations $\tau^{\geq 0}S$ and $\tau^{\leq -1}S$ are hence in $\mathcal{C}$ and $\mathcal{C}[1]$, respectively, and the distinguished triangle $(\tau^{\leq-1}S,S,\tau^{\geq 0}S)$ fits into a diagram (\ref{triangle cat abelian subcat kernel cokernel prop}). This proves that $\mathcal{C}$ is admissibly abelian, and it is stable under extensions as we have remarked.\par
It remains to prove that for any distinguished triangle $(X,Y,Z)$, the sequence $H^0(X)\to H^0(Y)\to H^0(Z)$ is exact. To this end, we first assume that $X,Y$ and $Z$ are in $\mathcal{D}^{\leq 0}$. For $U\in\mathcal{D}^{\leq 0}$ and $V\in\mathcal{D}^{\geq 0}$, we note that $H^0(U)=\tau^{\geq 0}U$ and $H^0(V)=\tau^{\leq 0}V$, so there are isomorphisms
\[\Hom_{\mathcal{D}}(H^0(U),H^0(V))\stackrel{\sim}{\to} \Hom_{\mathcal{D}}(U,H^0(V))\stackrel{\sim}{\to}\Hom_{\mathcal{D}}(U,V).\] 
For $T\in\mathcal{D}^{\geq 0}$, the long exact sequence of $\Hom$ and (t2) then give an exact sequence
\[\begin{tikzcd}
0\ar[r]&\Hom(Z,T)\ar[r]&\Hom(Y,T)\ar[r]&\Hom(X,T)
\end{tikzcd}\]
so the sequence $0\to\Hom(H^0(Z),T)\to\Hom(H^0(Y),T)\to\Hom(H^0(X),T)$ is exact. Since this is true for any $T$, we conclude that $H^0(X)\to H^0(Y)\to H^0(Z)\to 0$ is exact.\par
We now show that the above conclusion is still valid if we only assume that $X\in\mathcal{D}^{\leq 0}$. For this, we note that for $T\in\mathcal{D}^{\geq 1}$, the long exact sequence of $\Hom$ gives an isomorphism $\Hom(Z,T)\cong\Hom(Y,T)$, so we have $\tau^{\geq 1}Y\cong\tau^{\geq 1}Z$. Apply (TR4) to $Y\to Z\to\tau^{\geq 1}Z$,
\vspace*{-3mm}
\begin{equation*}
\begin{tikzcd}[row sep=2pt,column sep=4pt]
&&&{}&\\
&&&{}&\\
&&\tau^{\leq 0}Z\ar[ruu]\ar[rdd]&&\\
&&&{}&{}\\
&\tau^{\leq 0}Y\ar[rd]\ar[ruu]&&Z\ar[ru]\ar[rddd]\\
&&Y\ar[ru]\ar[rrdd]&&\\
&&&&\\
X\ar[ruuu]\ar[rruu]&&&&\tau^{\geq 1}Z\ar[rdd]\ar[rrd]\\
&&&&&&{}\\
&&&&&{}&
\end{tikzcd}
\vspace*{-3mm}
\end{equation*}
we then obtain a distinguished triangle $(X,\tau^{\leq 0}Y,\tau^{\leq 0}Z)$, on which we can apply the preceding arguments to conclude that $H^0(X)\to H^0(\tau^{\leq 0}Y)\to H^0(\tau^{\leq 0}Z)\to 0$ is exact. This proves our claim since $H^0\tau^{\leq 0}\cong H^0$. Dually, we conclude that if $Z\in\mathcal{D}^{\leq 0}$, then the sequence $0\to H^0(X)\to H^0(Y)\to H^0(Z)$ is exact.\par
Finally, we deal with the general case. By (TR4) we have an octahedron
\vspace*{-3mm}
\begin{equation*}
\begin{tikzcd}[row sep=2pt,column sep=4pt]
&&&{}&\\
&&&{}&\\
&&\tau^{\geq 1}X\ar[ruu]\ar[rdd]&&\\
&&&{}&{}\\
&X\ar[rd]\ar[ruu]&&U\ar[ru]\ar[rddd]\\
&&Y\ar[ru]\ar[rrdd]&&\\
&&&&\\
\tau^{\leq 0}X\ar[ruuu]\ar[rruu]&&&&Z\ar[rdd]\ar[rrd]\\
&&&&&&{}\\
&&&&&{}&
\end{tikzcd}
\vspace*{-3mm}
\end{equation*}
From the distinguished triangle $(\tau^{\leq 0}X,Y,U)$, we see that the sequence $H^0(X)\to H^0(Y)\to H^0(U)\to 0$ is exact, and $(U,Z,\tau^{\geq 1}X[1])$ shows that $0\to H^0(U)\to H^0(Z)$ is exact. We then conclude the exactness of $H^0(X)\to H^0(Y)\to H^0(Z)$, which completes the proof.
\end{proof}

\begin{proposition}\label{triangle cat t-structure truncation and cohomology prop}
Let $X$ be an object of $\mathcal{D}$ and $n\in\Z$.
\begin{enumerate}
    \item[(a)] $H^p(i):H^p(\tau^{\leq n}X)\to H^p(X)$ is an isomorphism for $p\leq n$ and $H^p(\tau^{\leq n}X)=0$ for $p>0$.
    \item[(b)] $H^p(j):H^p(X)\to H^p(\tau^{\geq n}X)$ is an isomorphism for $p\geq n$ and $H^p(\tau^{\geq n}X)=0$ for $p<n$.
\end{enumerate}
\end{proposition}
\begin{proof}
By duality, it suffices to prove (a). If $p>n$, then by \cref{triangle cat t-structure truncation orthogonal prop} we have
\[H^p(\tau^{\leq n}X)=\tau^{\leq p}\tau^{\geq p}\tau^{\leq n}X[p]=0.\]
On the other hand, if $p\leq n$, then $\tau^{\leq p}\tau^{\leq n}X\to\tau^{\leq p}X$ is an isomorphism, and therefore
\[H^p(\tau^{\leq n}X)=\tau^{\geq p}\tau^{\leq p}\tau^{\leq n}X[p]\to\tau^{\geq p}\tau^{\leq p}X[p]=H^p(X)[p]\]
is an isomorphism.
\end{proof}

\begin{corollary}\label{triangle cat t-structure cohomology of D^geq n}
Let $n\in\Z$ and $X$ be an object in $\mathcal{D}^{\leq n}$ (resp. $\mathcal{D}^{\geq n}$). Then $H^p(X)=0$ for $p>n$ (resp. $p<n$).
\end{corollary}
\begin{proof}
If $X$ is an object in $\mathcal{D}^{\leq n}$, then $\tau^{\leq n}X\to X$ is an isomorphism, so $H^p(\tau^{\leq n}X)\to H^p(X)$ is an isomorphism for all $p\in\Z$. On the other hand, by \cref{triangle cat t-structure truncation and cohomology prop}, $H^p(\tau^{\leq n}X)=0$ for $p>n$. The other part of the corollary follows by duality.
\end{proof}

We say the $t$-structure of $\mathcal{D}$ is \textbf{non-degenerate} if the intersection of the $\mathcal{D}^{\leq n}$, and that of the $\mathcal{D}^{\geq n}$, both reduce to the zero object. For each integer $i\in\Z$, we put $H^i(X):=H^0(X[i])$.

\begin{proposition}\label{triangle cat t-structure non-degenerate D^leq 0 char by H^i}
If the $t$-structure of $\mathcal{D}$ is non-degenerate, then the system of functors $H^i$ is conservative, and for an object $X\in\mathcal{D}$ to belong to $\mathcal{D}^{\leq 0}$ (resp. $\mathcal{D}^{\geq 0}$), it is necessary and sufficient that $H^i(X)=0$ for $i>0$ (resp. $i<0$).
\end{proposition}
\begin{proof}
Let $X\in\mathcal{D}$. We first prove that $H^i(X)=0$ for all $i\in\Z$ if and only if $X=0$. If $X\in\mathcal{D}^{\leq 0}$, the hypothesis $H^0(X)=0$ shows that $\tau^{\geq 0}X=0$, whence $X\in\mathcal{D}^{\leq 1}$. Inductively, we conclude that $X\in\bigcap_n\mathcal{D}^{\leq n}$, whence is zero by the hypothesis. Dually, we also conclude that if $X\in\mathcal{D}^{\geq 0}$, then $X=0$ if and only if $H^i(X)=0$ for any $i\in\Z$. For the general case, the values of $\tau^{\leq 0}X$ and $\tau^{\geq 1}X$ under $H^i$ are all zero, hence they are zero. We then conclude the claim by the distinguished triangle $(\tau^{\leq 0}X,X,\tau^{\geq 1}X)$.\par
If a morphism $f:X\to Y$, with cone $Z$, induces isomorphisms $H^i(X)\cong H^i(Y)$ for each $i$, then the long exact sequence of $H^i$ shows that $H^i(Z)=0$ for all $i$, so $Z=0$ and $f$ is an isomorphism. Finally, if $H^i(X)=0$ for $i>0$, then $H^i(\tau^{>0}X)=0$ for any $i\in\Z$, so $\tau^{>0}X=0$ and we conclude that $X\in\mathcal{D}^{\leq 0}$. The dual argument shows that $X\in\mathcal{D}^{\geq 0}$ if $H^i(X)=0$ for $i<0$.
\end{proof}

\begin{proposition}\label{triangle cat t-structure bounded iff}
Let $\mathcal{D}$ be a $t$-category. Then the following conditins are equivalent:
\begin{enumerate}
    \item[(\rmnum{1})] the union of the $\mathcal{D}^{\leq n}$ and that of the $\mathcal{D}^{\geq n}$ both equal to $\mathcal{D}$;
    \item[(\rmnum{2})] the $t$-structure is non-degenerate and for any $X\in\mathcal{D}$, $H^p(X)$ are nonzero for finitely many $p\in\Z$.
\end{enumerate}
The $t$-structure $\mathcal{D}$ is called \textbf{bounded} if it satisfies the above conditions.
\end{proposition}
\begin{proof}
Assume the conditions of (\rmnum{1}) and let $X$ be an object of $\mathcal{D}$ such that $H^p(X)=0$ for all $p\in\Z$. By assumption, there exists $n,m\in\Z$ such that $X\in\mathcal{D}^{\leq n}\cap\mathcal{D}^{\geq m}$. By considering the distinguished triangle $(\tau^{\leq p-1}X,X,\tau^{\geq p}X)$ for any $p\in\Z$, we conclude that $X\in\mathcal{D}^{\leq n}\cap\mathcal{D}^{\geq n}$ for all $n\in\Z$. In particular, $X\in\mathcal{D}^{\leq-1}\cap\mathcal{D}^{\geq 0}$, which means $\Hom(X,X)=0$ and therefore $X=0$. This shows that the $t$-structure on $\mathcal{D}$ is non-degenerate. Let $X$ be an arbitrary object in $\mathcal{D}$. Then $X\in\mathcal{D}^{\leq n}\cap\mathcal{D}^{\geq m}$ for some $m,n\in\Z$. By \cref{triangle cat t-structure truncation and cohomology prop}, $H^p(X)=0$ for $p>n$ and $p<m$, so $H^p(X)\neq 0$ for finitely many $p\in\Z$.\par
Conversely, let $X$ be an object in $\mathcal{D}$. Then there exists $n\in\N$ such thar $H^p(X)=0$ for $|p|>n$. By , this implies that $H^p(\tau^{\leq -n}X)=0$ and $H^p(\tau^{\geq n}X)=0$ for all $p\in\Z$. Since the $t$-structure is non-degenerate, $\tau^{\leq -n}X=\tau^{\geq n}X=0$, so $X\in\mathcal{D}^{\geq -n+1}$ and $\mathcal{D}^{\leq n-1}$.
\end{proof}
\begin{example}
Let $\mathcal{A}$ be an abelian category. Then the standard $t$-structure on the bounded derived category Db(A) is bounded. The standard t-structures on $D^+(\mathcal{A})$, $D^-(\mathcal{A})$ and $D(\mathcal{A})$ are not bounded.
\end{example}
Let $\mathcal{D}$ be a triangulated category with a nondegenerate $t$-structure. Let Db be the full subcategory consisting of all $X$ in $\mathcal{D}$ such that $H^p(X)\neq 0$ for finitely many $p\in\Z$. Clearly, $\mathcal{D}^b$ is strictly full subcategory. Assume that $(X,Y,Z)$ is a distinguished triangle in $\mathcal{D}$ and that two of its vertices are in $\mathcal{D}^b$. Then, from the long exact sequence of cohomology we see that the third vertex is also in $\mathcal{D}^b$. Therefore, $\mathcal{D}^b$ is a triangulated subcategory. Let $X$ be an object in $\mathcal{D}^b$. Then, by \cref{triangle cat t-structure truncation and cohomology prop}, $\tau^{\leq n}X$ and $\tau^{\geq n}X$ are also in Db for all $n\in\Z$. This implies that $(\mathcal{D}^b\cap\mathcal{D}^{\leq 0},\mathcal{D}^b\cap\mathcal{D}^{\geq 0})$ is a $t$-structure on $\mathcal{D}^b$. Clearly, the truncation functors and the cohomology functor $H^0$ for this $t$-structure are the restrictions of the corresponding functors on $\mathcal{D}$. Also, from the above result we see that this $t$-structure on $\mathcal{D}^b$ is bounded. We call $\mathcal{D}^b$ the subcategory of \textbf{cohomologically bounded objects} in $\mathcal{D}$.\par

Let $\mathcal{D}$ be a triangulated category and denote by $\Iso(\mathcal{D})$ the collection of sets of isomorphism classes of objects of $\mathcal{D}$ (for $X\in\mathcal{D}$, we denote by $[X]$ its isomorphism class). We define an operation $\ast$ on $\Iso(\mathcal{D})$ as follows: for $A,B\in\Iso(\mathcal{D})$, $A\ast B$ is defined to be 
\[A\ast B=\{[X]:\text{there exists a distinguished triangle $(U,X,V)$ with $[U]\in A$ and $[V]\in B$}\}.\]

\begin{lemma}\label{triangle cat extension operator associative}
The operation $\ast$ is associative.
\end{lemma}
\begin{proof}
It suffices to prove that for $X,Y,Z\in\mathcal{D}$, we have
\[(\{[X]\}\ast\{[Y]\})\ast\{[Z]\}=\{[X]\}\ast(\{[Y]\}\ast\{[Z]\}).\]
For that $[T]$ belongs to the left side (resp. right side), it is necessary and sufficient that $T$ fits into a diagram of the upper cap (resp. lower cap)
\[\begin{tikzcd}
Z\ar[rd,"+1"]\ar[dd]&&T\ar[ll]\\
&U\ar[ld]\ar[ru]\ar[ru]&\\
Y\ar[rr,swap,"+1"]&&X\ar[lu]\ar[uu]
\end{tikzcd}\quad\quad\text{resp.}\quad
\begin{tikzcd}
Z\ar[dd,swap,"+1"]&&T\ar[ll]\ar[ld]\\
&V\ar[rd,"+1"]\ar[lu]&\\
Y\ar[rr]\ar[ru]&&X\ar[uu]
\end{tikzcd}\]
The lemma then follows from axiom (TR4) and the inverse of (TR4').
\end{proof}

\cref{triangle cat extension operator associative} permits us to define the $\ast$-product $A_1\ast\cdots\ast A_p$ for a sequence of elements in $\Iso(\mathcal{D})$, without using the parentheses. It will be convenient for us to define the $\ast$-product of the empty sequence as being $\{[0]\}$.

\begin{example}\label{triangle cat extension closure of subcategory}
Let $\mathcal{A}$ be a strictly full subcategory of $\mathcal{A}$ and $E\mathcal{A}$ be the smallest strictly full subcategory of $\mathcal{D}$ containing $\mathcal{A}$, the zero object, and is stable under extensions. We have
\begin{equation}\label{triangle cat extension closure of subcategory-1}
[E\mathcal{A}]=\bigcup_{n\geq 0}\underbrace{[\mathcal{A}]\ast\cdots\ast[\mathcal{A}]}_{\text{$n$-factors}}.
\end{equation}
We also note that the condition that any morphisms in $\mathcal{A}$ are $\mathcal{A}$-admissible can be reformulated as
\begin{equation}\label{triangle cat extension closure of subcategory-2}
[\mathcal{A}]\ast[\mathcal{A}[1]]\sub[\mathcal{A}[1]]\ast[\mathcal{A}].
\end{equation}
\end{example}

We now consider an admissible abelian subcategory $\mathcal{C}$ of $\mathcal{D}$ which is stable under extensions. Let $\mathcal{D}^b$ (resp. $\mathcal{D}^{b,\leq 0}$, resp. $\mathcal{D}^{b,\geq 0}$, resp. $\mathcal{D}^{b,I}$, where $I$ is an integer of $\Z$) be the smallest strictly full subcategory of $\mathcal{D}$ containing the $\mathcal{C}[n]$ for $n\in\Z$ (resp. $-n\leq 0$, resp. $-n\geq 0$, resp. $-n\in I$) which is stable under extensions\footnote{If $I$ is the empty interval, it is more natural to define $\mathcal{D}^{b,I}$ to be the category of zero objects.}.

\begin{proposition}\label{triangle cat t-structure by admissible abelian}
The pari $(\mathcal{D}^{b,\leq 0},\mathcal{D}^{b,\geq 0}$) is a bounded $t$-structure on $\mathcal{D}^{b}$. For $m\leq n$, we have $\mathcal{D}^{b,[m,n]}=\mathcal{D}^{b,\geq m}\cap\mathcal{D}^{b,\leq n}$. In particular, $\mathcal{C}=\mathcal{D}^{b,\leq 0}\cap\mathcal{D}^{b,\geq 0}$.
\end{proposition}
\begin{proof}
The axiom (t1) is trivially satisfied, and (t2) follows from the long exact sequence of $\Hom$ and $\Hom^i(X,Y)=0$ for $i<0$ and $X,Y\in\mathcal{C}$. Since $\mathcal{C}$ is stable under extension, for any interver $I=[m,n]$ of $\Z$ ($m\leq n$), we have
\begin{equation}\label{triangle cat t-structure by admissible abelian-1}
[\mathcal{D}^{b,I}]=[\mathcal{C}[-m]]\ast\cdots\ast[\mathcal{C}[-n]].
\end{equation}
In view of this, for intervals $J,K$ such that $I=J\cup K$, we then have
\begin{equation}\label{triangle cat t-structure by admissible abelian-2}
[\mathcal{D}^{b,I}]=[\mathcal{D}^{b,J}]\ast[\mathcal{D}^{b,K}].
\end{equation}
In particular, for $J=[m,0]$ and $K=[1,n]$, we then conclude that any object of $\mathcal{D}^{b,I}$ is the extension of an object of $\mathcal{D}^{b,K}\sub\mathcal{D}^{b,\geq 1}$ by an object of $\mathcal{D}^{b,J}\sub\mathcal{D}^{b,\leq 0}$. Since $\mathcal{D}^b$ is the union of $\mathcal{D}^{b,I}$, we conclude axiom (t3).\par
By (\ref{triangle cat t-structure by admissible abelian-1}), if $X$ is in $\mathcal{D}^{b,I}$, there exists a sequence of distinguished triangles $(X_i,X_{i+1},A_{i+1})$ ($m\leq i\leq n$) with $X_m\in\mathcal{C}[-m]$, $A_i\in\mathcal{C}[-j]$, and $X_n=X$. We may also set $X_{m-1}=0$ and $A_m=X_m$ and obtain a distinguished triangle $(X_{m-1},X_m,A_m)=(0,X_m,X_m)$. Apply the long exact sequence of cohomology to these triangles, we see by recurrence over $j$ that, for any $m\leq i\leq n$, we have
\[H^i(X_j)=\begin{cases}
0&i\notin[m,j],\\
A_i[i]&i\in[m,j].
\end{cases}\]
In particular, $[X]\in\{[H^m(X)[-m]]\}\ast\cdots\ast\{[H^n(X)[-n]]\}$; this proves the boundedness of the $t$-structure (if $H^i(X)=0$ for all $i$, then $X=0$) and the second statement of the proposition (if $X\in\mathcal{C}^{b,\geq m}\cap\mathcal{D}^{b,\leq n}$, the $H^i(X)=0$ for $i\notin[m,n]$ and therefore $X$ is in $\mathcal{D}^{b,[m,n]}$).
\end{proof}

\begin{remark}
Let $\mathcal{D}$ be a triangulated category and $\mathcal{C}$ be a right leaning full subcategory such that any morphism in $\mathcal{C}$ is admissible. The proof of \cref{triangle cat abelian subcat iff morphism admissible} shows that any morphism of $\mathcal{C}$ has a kernel and a cokernel, that $\im f=\coim f$ and that any short exact sequence of $\mathcal{C}$ is admissible. For $\mathcal{C}$ to be abelian, it is necessary and sufficient that $\mathcal{C}$ admits finite direct sums.\par
Let $\mathcal{C}'$ be the category $E\mathcal{C}$ of successive extensions of objects of $\mathcal{C}$. The long exact sequence of $\Hom$ shows that $\mathcal{C}'$ is right leaning. Moreover, we deduce from (\ref{triangle cat extension closure of subcategory-1}), (\ref{triangle cat extension closure of subcategory-2}) and the associativity of $\ast$ that $[\mathcal{C}']\ast[C'[1]]\sub[\mathcal{C}'[1]]\ast[\mathcal{C}']$, i.e. that every morphism of $\mathcal{C}$ is admissible. Therefore $\mathcal{C}$ is an admissible abelian subcategory of $\mathcal{D}$. Applying \cref{triangle cat t-structure by admissible abelian} to $\mathcal{C}'$, we conclude that, for $\mathcal{D}^b$, $\mathcal{D}^{b,\leq 0}$ and $\mathcal{D}^{b,\leq 0}$ defined as above, $(\mathcal{D}^{b,\leq 0},\mathcal{D}^{b,\geq 0})$ is a $t$-structure of $\mathcal{D}^b$ with heart $\mathcal{C}'$.
\end{remark}

\paragraph{\texorpdfstring{$t$}{t}-exact functors}
\begin{proposition}\label{triangle cat t-structure truncation <0 and leq 0}
Let $\mathcal{D}$ be a $t$-category, $K\in\mathcal{D}$, and consider an short exact sequence in $\mathcal{C}$:
\[\begin{tikzcd}
0\ar[r]&A\ar[r]&H^0(K)\ar[r]&B\ar[r]&0
\end{tikzcd}\]
\begin{enumerate}
    \item[(a)] There exists $K'\in\mathcal{D}^{\leq 0}$, with a morphism $i:K'\to K$, such that, over $\mathcal{D}^{\leq 0}$, $K'$ represents the functor
    \[L\mapsto\ker(\Hom(L,K)\to\Hom(H^0(L),B)).\]
    Moreover, for a couple $(K',i)$ to represents this functor, it is necessary and sufficient that for $K'\in\mathcal{D}^{\leq 0}$, $\tau^{<0}K'\cong\tau^{<0}K$ and $H^0(K')\cong A$.
    \item[(b)] Dually, we obtain a morphhism $j:K\to K''$ with $K''\in\mathcal{D}^{\geq 0}$ and the couple $(K'',j)$ represents the functor 
    \[L\mapsto\coker(\Hom(L,K)\to\Hom(H^0(L),B)).\]
    Moreover, the couple $(K'',j)$ is characterized by $K''\in\mathcal{D}^{\geq 0}$, $\tau^{>0}K\cong\tau^{>0}K''$ and $H^0(K'')=B$.
    \item[(c)] There exists a unique morphism $d:K''\to K'$ such that the triangle $(K',K,K'')$ is distinguished.
\end{enumerate}
\end{proposition}
\begin{proof}

\end{proof}

Let $\mathcal{D}_i$ ($i=1,2$) be $t$-categories, $\mathcal{C}_i$ be the heart of $\mathcal{D}_i$, and denote by $\eps:\mathcal{C}_i\to\mathcal{D}_i$ the inclusion functor. Let $T:\mathcal{D}_1\to\mathcal{D}_2$ be an exact functor of triangulated categories (in the usual sense for triangulated categories, that is, up to a natural equivalence it commutes with translation and preserves distinguished triangles); we say that $T$ is \textbf{right $\bm{t}$-exact} if $T(\mathcal{D}^{\leq 0})\sub D^{\leq 0}$, \textbf{left $\bm{t}$-exact} if $T(\mathcal{D}^{\geq 0})\sub D^{\geq 0}$, and $t$-exact if it is both left $t$-exact and right $t$-exact.

\begin{proposition}\label{triangle cat t-exact functor prop}
Let $T:\mathcal{D}_1\to\mathcal{D}_2$ be an exact functor.
\begin{enumerate}
    \item[(a)] If $T$ is left (resp. right) $t$-exact, the functor $^pT:=H^0\circ T\circ\eps:\mathcal{C}_1\to\mathcal{C}_2$ is left (resp. right) exact.
    \item[(b)] If $T$ is left (resp. right) $t$-exact and $K\in\mathcal{D}_1^{\geq 0}$ (resp. $K\in\mathcal{D}_1^{\leq 0}$), we have $^pT(H^0(K))\cong H^0(T(K))$ (resp. $H^0(T(K))\cong {^pT}(H^0(K))$).
    \item[(c)] Let $T^*:\mathcal{D}_2\to\mathcal{D}_1:T_*$ be a pair of adjoint functors. For $T^*$ to be right $t$-exact, it is necessary and sufficient that $T_*$ is left $t$-exact, and in this case $({^pT^*},{^pT_*})$ is an adjoint pair.
    \item[(d)] If $T_1:\mathcal{D}_1\to\mathcal{D}_2$ and $T_2:\mathcal{D}_2\to\mathcal{D}_3$ are left (resp. right) $t$-exact functors, then $T_2\circ T_1$ is also left (resp. right) $t$-exact and ${^p(T_2\circ T_1)}={^pT_2}\circ {^pT_1}$.  
\end{enumerate}
\end{proposition}
\begin{proof}
If $T$ is left $t$-exact, for any short exact sequence $0\to X\to Y\to Z\to 0$ in $\mathcal{C}_1$, the long exact sequence on cohomology of the distinguished triangle $(T(X),T(Y),T(Z))$ gives an exact sequence
\[0\to H^0(T(X))\to H^0(T(Y))\to H^0(T(Z)),\]
since $T(Z)$ is in $\mathcal{D}_2^{\geq 0}$, so ${^pT}$ is left exact.\par
For $K\in\mathcal{D}_1^{\geq 0}$, the distinguished triangle $(H^0(K),K,\tau^{>0}K)$ gives a distinguished triangle $(T(H^0(K)),T(K),T(\tau^{>0}K))$ with $T(\tau^{>0}K)\in\mathcal{D}_2^{>0}$, so the long exact sequence on cohomology shows that $H^0(T(H^0(K)))\cong H^0(T(K))$. This (and the dual argument) proves the assertions of (a) and (b).\par
Let $T^*:\mathcal{D}_2\to\mathcal{D}_1:T_*$ be a pair of adjoint functors. If $T_*$ is left $t$-exact, for $U\in\mathcal{D}_1^{>0}$ and $V\in\mathcal{D}_2^{\leq 0}$, we have $\Hom(T^*(V),U)=\Hom(V,T_*(U))=0$. Since this is valid for any $U$, we have $\tau^{>0}T^*(V)=0$, i.e. $T^*(V)$ is in $\mathcal{D}_1^{\leq 0}$, so $T^*$ is right $t$-exact. For $A\in\mathcal{C}_1$ and $B\in\mathcal{C}_2$, we then have $H^0(T^*(B))=\tau^{\geq 0}T^*(B)$ and $H^0(T_*(A))=\tau^{\leq 0}T_*(A)$, whence a functorial isomorphism
\[\Hom(H^0(T^*(B)),A)\stackrel{\sim}{\to} \Hom(T^*(B),A)=\Hom(A,T_*(B))\stackrel{\sim}{\leftarrow} \Hom(A,H^0(T_*(B))).\]
This (resp. its dual form) prove the assertions of (c).\par
Finally, if $T_1$ and $T_2$ are left $t$-exact and that $A\in\mathcal{C}_1$, we have $T_1(A)\in\mathcal{D}_2^{\geq 0}$ and
\[{^p(T_2\circ T_1)}(A)=H^0(T_2(T_1(A)))=H^0(T_2(H^0(T_1(A))))\]
in view of (b). This, together with its dual form, proves (d).
\end{proof}

\begin{remark}
Let $\mathcal{D}_1^+=\bigcup_n\mathcal{D}_1^{\geq n}$ and $\mathcal{D}_2^-=\bigcup_n\mathcal{D}_2^{\leq n}$. Then the result of \cref{triangle cat t-exact functor prop}~(c) is still valid for functors $T^*:\mathcal{D}_2^{-}\to\mathcal{D}_1$ and $T_*:\mathcal{D}_1^{+}\to\mathcal{D}_2$ which are adjoint in the sense that $\Hom(T^*(V),U)=\Hom(V,T_*(U))$ for $V\in\mathcal{D}_2^-$ and $U\in\mathcal{D}_2^+$. The proof is the same.
\end{remark}

\begin{remark}\label{triangle cat adjoint functor ^pT diagram}
In the situation of \cref{triangle cat t-exact functor prop}~(c), for $A\in\mathcal{C}_1$ and $B\in\mathcal{C}_2$, the adjoint morphisms of $(T^*,T_*)$ and $({^pT^*},{^pT_*})$ fit into the commmutative diagram
\[\begin{tikzcd}
T^*{^pT_*}(A)\ar[d]\ar[r]&{^pT^*}{^pT_*}(A)\ar[d]\\
T^*T_*(A)\ar[r]&A
\end{tikzcd}\quad\quad
\begin{tikzcd}
B\ar[r]\ar[d]&T_*T^*(B)\ar[d]\\
{^pT_*}{^pT^*}(B)\ar[r]&T_*{^pT^*}(B)
\end{tikzcd}\]
\end{remark}

\begin{example}\label{triangle cat t-structure subcategory eg}
Let $T:\mathcal{D}'\to\mathcal{D}$ be a fully faithful exact functor between triangulated categories. For a triangle $(X,Y,Z)$ of $\mathcal{D}'$ to be distinguished, it is necessary and sufficient that its image under $T$ is diatinguished: by (TR2) we can find a distinguished triangle $(X,Y,Z')$ of $\mathcal{D}'$, whose image under $T$ is then isomorphic to the distinguished triangle $(T(X),T(Y),T(Z))$ by \cref{triangle cat morphism dt isomorphism 2 of 3}. Since $T$ is fully faithful, this implies that $(X,Y,Z)$ is isomorphic to $(X,Y,Z')$, so it is distinguished.\par
Suppose that $\mathcal{D}$ and $\mathcal{D}'$ are endowed with $t$-structures and that $T$ is $t$-exact. For $X\in\mathcal{D}'$ to belong to $\mathcal{D}'^{\leq 0}$ (resp. $\mathcal{D}'^{\geq 0}$), it is it is necessary and sufficient that $T(X)$ is in $\mathcal{D}^{\leq 0}$ (resp. $\mathcal{D}^{\geq 0}$). This follows from the fact that $X\in\mathcal{D}'^{\leq 0}$ if and only if $\tau^{>0}X=0$, and that $T$ commutes with $\tau^{<0}$ (resp. the dual argument).\par
Conversely, if $\mathcal{D}'$ is a full triangulated subcategory of a triangulated category $\mathcal{D}$ and that $(\mathcal{D}^{\leq 0},\mathcal{D}^{\geq 0}$) is a $t$-structure over $\mathcal{D}$, for that $(\mathcal{D}'^{\leq 0},\mathcal{D}'^{\geq 0}):=(\mathcal{D}'\cap\mathcal{D}^{\leq 0},\mathcal{D}'\cap\mathcal{D}^{\geq 0})$ is a $t$-structure over $\mathcal{D}'$, it is necessary and sufficient that $\mathcal{D}'$ is stable under the functor $\tau^{\leq 0}$. If this conditions is satisfied, this $t$-structure over $\mathcal{D}'$ is called the \textbf{induced $t$-structure}. For $\mathcal{D}'$ endowed with the induced $t$-structure, the inclusion functor $\mathcal{D}'\to\mathcal{D}$ is then $t$-exact: we have $\mathcal{C}'=\mathcal{D}'\cap\mathcal{C}$, and the restriction of the functors $\tau^{\leq n}$, $\tau^{\geq n}$ or $H^p$ of $\mathcal{D}$ is identified with the same functors of $\mathcal{D}'$.
\end{example}

Let $(\mathcal{D}_i)_{i\in I}$ be a finite family of triangulated categories and $T:\prod_i\mathcal{D}_i\to\mathcal{D}$ be an exact multifunctor to a triangulated category $\mathcal{D}$. Suppose that the $\mathcal{D}_i$ and $\mathcal{D}$ are endowed with $t$-structures. We say that $T$ is \textbf{left $t$-exact} (resp. \textbf{right $t$-exact}) if it sends the product of the $\mathcal{D}_i^{\geq 0}$ (resp. $\mathcal{D}_i^{\leq 0}$) into $\mathcal{D}^{\geq 0}$ (resp. $\mathcal{D}^{\leq 0}$), and \textbf{$t$-exact} if it is both left $t$-exact and right $t$-exact. If $T$ is left $t$-exact (resp. right $t$-exact, resp. $t$-exact) and fix certain variables to be an object of $\mathcal{D}_i^{\geq 0}$ (resp. $\mathcal{D}_i^{\leq 0}$, resp. $\mathcal{C}_i$, the herat of $\mathcal{D}_i$), the functor obtained in the remaining variables is still left $t$-exact (resp. right $t$-exact, resp. $t$-exact). This allows us to apply some proven results for functors of one variable. For example: let $\eps_i:\mathcal{C}_i\to\mathcal{D}_i$ be the inclusion functor. Put ${^pT}=H^0\circ T\circ(\eps_i)_{i\in I}$. If $T$ is left (resp. right) $t$-exact, the additive multifunctor ${^pT}$ is then left (resp. right) exact.\par
Suppose that $T$ is left $t$-exact. For $(K_i)\in\prod_i\mathcal{D}_i^{\geq 0}$, we conclude from \cref{triangle cat t-exact functor prop}~(b) that ${^pT}(H^0(K_i))\cong H^0(T(K_i))$. Therefore, for $(K_i)\in\prod_i\mathcal{D}_i$, the morphisms $K_i\to\tau^{\geq 0}K_i$ then give a morphism
\begin{equation}\label{triangle cat multifunctor t-exact-1}
H^0(T(K_i))\to H^0(T(\tau^{\geq 0}K_i))\stackrel{\sim}{\leftarrow} {^pT}(H^0(K_i))
\end{equation}
By translating, we then obtain, for $\sum n_i=n$, a morphism
\begin{equation}\label{triangle cat multifunctor t-exact-2}
H^n(T(K_i))\to {^pT}(H^{n_i}(K_i))
\end{equation}
There is a problem of signs here, which does not appear if we consider instead
\begin{equation}\label{triangle cat multifunctor t-exact-3}
H^n(T(K_i))\to H^nT((H^{n_i}(K_i)[-n_i]))
\end{equation}

For $T$ right $t$-exact, we have $H^0T(K_i)\cong {^pT}H^0(K_i)$ for $(K_i)\in\prod_i\mathcal{D}_i^{\leq 0}$, so the morphisms $\tau^{\leq 0}K_i\to K_i$ provides a morphism ${^pT}(H^0(K_i))\to H^0(T(K_i))$, and by translating, we obtain morphisms
\begin{equation}\label{triangle cat multifunctor t-exact-4}
H^nT((H^{n_i}(K_i)[-n_i]))\to H^nT(K_i).
\end{equation}
where $n=\sum_in_i$.\par

If $T$ is $t$-exact, we have both (\ref{triangle cat multifunctor t-exact-3}) and (\ref{triangle cat multifunctor t-exact-4}). Applying $H^nT$ to the commutative diagram
\[\begin{tikzcd}
\tau^{\leq n_i}K_i\ar[d]\ar[r]&H^{n_i}K_i[-n_i]\ar[d]\\
K_i\ar[r]&\tau^{\geq n_i}K_i
\end{tikzcd}\]
we conclude that the composition of (\ref{triangle cat multifunctor t-exact-3}) and (\ref{triangle cat multifunctor t-exact-4}) is the identity. Now if $\sum_in_i=\sum_im_i=n$ and that $(n_i)_{i\in I}\neq (m_i)_{i\in I}$, there exists $i\in I$ such that $n_i<m_i$. The composition $\tau^{\leq n_i}K_i\to K_i\to\tau^{\geq m_i}K_i$ is zero, and we deduce that the composition of (\ref{triangle cat multifunctor t-exact-3}) for $(m_i)$ and (\ref{triangle cat multifunctor t-exact-4}) for $n_i$ is zero.

\begin{proposition}\label{triangle cat multifunctor t-exact cohomology char}
If $T$ is $t$-exact and $(K_i)\in\prod_i\mathcal{D}_i^b$, the morphisms (\ref{triangle cat multifunctor t-exact-3}) and (\ref{triangle cat multifunctor t-exact-4}) are inverses of each other, so that we have an isomorphism
\begin{equation}\label{triangle cat multifunctor t-exact cohomology char-1}
H^n(T(K_i))=\bigoplus H^nT((H^{n_i}K_i)[-n_i])
\end{equation}
where the sum is taken over $\sum_in_i=n$.
\end{proposition}
\begin{proof}
We have already seen that these morphisms make the second member a direct factor of the first. Both members of (\ref{triangle cat multifunctor t-exact cohomology char-1}) are cohomological functors in each $K_i$ (for the right-hand side, thanks to the exactness of ${^pT}$) and the morphisms (\ref{triangle cat multifunctor t-exact-3}) and (\ref{triangle cat multifunctor t-exact-4}) are morphisms of cohomoloaic functors. By d\'evissage, we are reduced to assume each $K_i\in\mathcal{C}_i$, where the assertion is trivial.
\end{proof}

\subsection{Recollement}
\paragraph{Six functors for topological spaces}\label{triangle cat recollement topo space paragraph}
For $X$ a topological space, endowed with a sheaf of rings $\mathscr{O}_X$, we denote by $D(X,\mathscr{O}_X)$ the derived category of the abelian category $\Mod(\mathscr{O}_X)$ of sheaf of left $\mathscr{O}_X$-modules over $X$. As usual, $D^+(X,\mathscr{O}_X)$ is the full subcategory consisting of lower bounded complexes.\par
Let $U$ be an open subset of $X$ and $Z$ be its complement. We denote by $j:U\to X$ and $i:Z\to X$ the canonical inclusions, and by $\mathscr{O}_U$, $\mathscr{O}_Z$ the inverse image of the structural sheaf $\mathscr{O}_X$ over $U$ and $Z$, respectively. We now describe the construction of glueing a $t$-structure over $D^+(U,\mathscr{O}_U)$ and that of $D^+(Z,\mathscr{O}_Z)$.\par
The categories $\Mod(\mathscr{O}_X)$, $\Mod(\mathscr{O}_U)$, and $\Mod(\mathscr{O}_Z)$ are related by the functors
\[\begin{tikzcd}[column sep=12mm]
\Mod(\mathscr{O}_U)\ar[r,bend left=40pt,"j_!"]\ar[r,bend right=40pt,"j_*"]&\Mod(\mathscr{O}_X)\ar[l,swap,"j^*=j^!"]\ar[r,bend left=40pt,"i^*"]\ar[r,bend right=40pt,"i^!"]&\Mod(\mathscr{O}_Z)\ar[l,swap,"i_*=i_!"]
\end{tikzcd}\]
so that we have adjoint pairs $(j_!,j^*,j_*)$ and $(i^*,i_*,i^!)$. We also have identities
\begin{gather}
j^*j_*=1,\quad j^*j_!=1,\quad i^!i_*=1,\quad i^*i_*=1,\label{triangle cat recollement-1}\\
j^*i_*=0,\quad i^*j_!=0,\quad i^!j_*=0.\label{triangle cat recollement-2}
\end{gather}
and for a sheaf $\mathscr{F}$ over $X$, the adjoint morphisms fit into exact sequences
\begin{equation}\label{triangle cat recollement-3}
\begin{tikzcd}
0\ar[r]&j_!j^*(\mathscr{F})\ar[r]&\mathscr{F}\ar[r]&i_*i^*(\mathscr{F})\ar[r]&0
\end{tikzcd}
\end{equation}
\vspace*{-4mm}
\begin{equation}\label{triangle cat recollement-4}
\begin{tikzcd}
0\ar[r]&i_*i^!(\mathscr{F})\ar[r]&\mathscr{F}\ar[r]&j_*j^*(\mathscr{F})
\end{tikzcd}
\end{equation}
where the last arrow is surjective if $\mathscr{F}$ is injective.\par

For an adjoint pair $(T^*,T_*)$, the adjunction morphism $T^*T_*\to\id$ (resp. $\id\to T_*T^*$) is an isomorphism if and only if $T_*$ (resp. $T^*$) is fully faithful. The assertion (\ref{triangle cat recollement-1}) is then equivalent to that $i_*$, $j_*$ and $j_!$ are fully faithful.\par

For an exact functor $T$ between abelian categories, it trivially extends to the derived categories. We shall use the same symbol for this extension. This extension coincides with the left derived functor $LT$ and the right derived functor $RT$, whenever they are defined.\par
The functors described above induce functors over $D^+(X,\mathscr{O}_X)$, $D^+(U,\mathscr{O}_U)$ and $D^+(Z,\mathscr{O}_Z)$, which form adjoint pairs $(j_!,j^*,Rj_*)$ and $(i^*,i_*,Ri^!)$. We have $j^*i_*=0$, whence by adjunction $i^*j_!=0$ and $Ri^!Rj_*=0$. For $K\in K^+(X,\mathscr{O}_X)$, the exact sequences (\ref{triangle cat recollement-3}) and (\ref{triangle cat recollement-3}) then gives distinguished triangles $(j_!j^*K,K,i_*i^*K)$ and $(i_*Ri^!K,K,Rj_*j^*K)$. Finally, for $K\in D^+(Z,\mathscr{O}_Z)$ (resp. $K\in D^+(U,\mathscr{O}_U)$), by (\ref{triangle cat recollement-1}) we have isomorphisms
\[i_*i^*K\cong K\cong Ri^!i_*K,\quad\quad(\text{resp.}\quad j^*Rj_*K\cong K\cong j^*j_!K.)\]
In fact, $j_*$ and $i_*$ transform injectives to injectives, so we can apply Grothendieck's spectral sequence.\par

The properties listed above are all that we need to glue $t$-structures, and we meet them in various contexts: for example, for $\ell$-adic derived categories, which do not come strictly within the scope of \ref{triangle cat recollement topo space paragraph}. To cover these cases, we will place ourselves in a more general framework. In this context, the sheaf categories no longer appear (only the triangulated categories appear) and we take advantage of this to lighten the notation by simply writing $j_*$ and $i^*$ for the derived functors $Rj_*$ and $Ri^!$.

\paragraph{Recollement of \texorpdfstring{$t$}{t}-structures}\label{triangle cat t-structure recollement paragraph}
Let $\mathcal{T}$ be a triangulated category endowed with strictly full subcategories $\mathcal{U}$, $\mathcal{V}$, which are stable under translations. Suppose that for $U\in\mathcal{U}$ and $V\in\mathcal{V}$ we have $\Hom(U,V)=0$, and that any $X\in\mathcal{T}$ fits into a distinguished triangle $(U,X,V)$, with $U\in\mathcal{U}$ and $V\in\mathcal{V}$. The pair $(\mathcal{U},\mathcal{V})$ is then a $t$-structure over $\mathcal{T}$. By \cref{triangle cat t-structure truncation Hom prop}, $\mathcal{V}$ is the right orthogonal of $\mathcal{U}$ and $\mathcal{U}$ is the left orthogonal of $\mathcal{V}$. In particular, $\mathcal{U}$ and $\mathcal{V}$ are Serre subcategories of $\mathcal{T}$. The hypothesis (\rmnum{1}), (\rmnum{2}) of (\cite{tohoku} 6.4, p.25) are then satisfied and by (\cite{tohoku} 6.4, p.23-p.26), the projection $\mathcal{T}\to\mathcal{T}/\mathcal{U}$ admits a fully faithful right adjoint, with image $\mathcal{V}$, and the projection $\mathcal{T}\to\mathcal{T}/\mathcal{V}$ admits a fully faithful left adjoint, with image $\mathcal{U}$. In other words, the inclusion $u:\mathcal{U}\to\mathcal{T}$ (resp. $v:\mathcal{V}\to\mathcal{T}$) has a right adjoint $u_\bullet$ (resp. a left adjoint $v^\bullet$) and the sequences
\[\begin{tikzcd}
0\ar[r]&\mathcal{U}\ar[r,"u"]&\mathcal{T}\ar[r,"v^\bullet"]&\mathcal{V}\ar[r]&0
\end{tikzcd}\]
\vspace*{-4mm}
\[\begin{tikzcd}
0\ar[r]&\mathcal{V}\ar[r,"v"]&\mathcal{T}\ar[r,"u_\bullet"]&\mathcal{U}\ar[r]&0
\end{tikzcd}\]
are "exact" in the sense that $v^\bullet$ (resp. $u_\bullet$) identified $\mathcal{V}$ (resp. $\mathcal{U}$) with the quotient of $\mathcal{T}$ by the Serre subcategory $\mathcal{U}$ (resp. $\mathcal{V}$).\par

Now consider triangulated categories $\mathcal{D}$, $\mathcal{D}_U$ and $\mathcal{D}_Z$ such that we have exact functors
\[\begin{tikzcd}
\mathcal{D}_Z\ar[r,"i_*"]&\mathcal{D}\ar[r,"j^*"]&\mathcal{D}_U
\end{tikzcd}\]
It is convenient to put $i_!:=i_*$ and $j^!=j^*$. Suppose that the following \textit{recollement conditions} are satisfied\footnote{We note that this formalism is selfdual by exchanging $j_!$ with $j_*$ and $i^*$ with $i^!$.}:
\begin{enumerate}[leftmargin=40pt]
    \item[(R1)] $i_*$ admits a left adjoint functor $i^*$ and an exact right adjoint functor $i^!$.
    \item[(R2)] $j^*$ admits a right adjoint functor $j_*$ and an exact left adjoint functor $j_!$.
    \item[(R3)] $j^*i_*=i^*j_!=i^!j_*=0$, and therefore for $A\in\mathcal{D}_Z$ and $B\in\mathcal{D}_U$,
    \[\Hom(j_!(B),i_*(A))=\Hom(i_*(A),j_*(B))=0.\]
    \item[(R4)] There are distinguished triangles $(j_!j^*(K),K,i_*i^*(K))$ and $(i_*i^!(K),K,j_*j^*(K))$ for $K\in\mathcal{D}$.
    \item[(R5)] $i^*i_*\to\id\to i^!i_*$ and $j^*j_*\to\id\to j^*j_!$ are isomorphisms, or equivalently, $i_*$, $j_!$ and $j_*$ are fully faithful. 
\end{enumerate}
Applying the preceding arguments to $\mathcal{T}=\mathcal{D}$, and choose for $(\mathcal{U},\mathcal{V})$ the pairs of subcategories $(i_*\mathcal{D}_Z,j_*\mathcal{D}_U)$ and $(j_!\mathcal{D}_U,i_*\mathcal{D}_Z)$, we obtain the following exact sequences
\[\begin{tikzcd}
0&\mathcal{D}_Z\ar[l]&\mathcal{D}\ar[l,swap,"i^*"]&\mathcal{D}_U\ar[l,swap,"j_!"]&0\ar[l]
\end{tikzcd}\]
\vspace*{-4mm}
\[\begin{tikzcd}
0\ar[r]&\mathcal{D}_Z\ar[r,"i_*"]&\mathcal{D}\ar[r,"j^*"]&\mathcal{D}_U\ar[r]&0
\end{tikzcd}\]
\vspace*{-4mm}
\[\begin{tikzcd}
0&\mathcal{D}_Z\ar[l]&\mathcal{D}\ar[l,swap,"i^!"]&\mathcal{D}_U\ar[l,swap,"j_*"]&0\ar[l]
\end{tikzcd}\]

Since the functor $i_*$ is fully faithful, the composition of the adjunction morphisms $i_*i^!\to\id\to i_*i^*$ gives a unique morphism of functors
\begin{equation}\label{triangle cat recollement i^! to i^* morphism}
i^!\to i^*.
\end{equation}
If we apply this to $i_*(X)$ and identify $i^!i_*(X)$ and $i^*i_*(X)$ with $X$, we obtain the identity morphism of $X$.\par

The functor $j^*$ being a quotient functor (it identifies $\mathcal{D}_U$ with the quotient category), the composition of the adjunction morphisms $j_!j^*\to\id\to j_*j^*$ defines a unique morphism of functors
\begin{equation}\label{triangle cat recollement j_! to j_* morphism}
j_!\to j_*.
\end{equation}
If we identify $j^*j_!$ and $j^*j_*$ with the identity functor, then by applying $j^*$ to (\ref{triangle cat recollement j_! to j_* morphism}), we obtain the identity morphism of the idnetity functor.\par

For $X\in\mathcal{D}_U$, the cone of $j_!(X)\to j_*(X)$ is then annihilated by $j^*$, so it belongs to $i_*\mathcal{D}_F$. By condition (R3) and \cref{triangle cat morphism extension to triangle iff}, the distinguished triangle with base $j_!(X)\to j_*(X)$ is uniquely determined up to unique isomorphisms, so we obtain a functor $j_*/j_!:\mathcal{D}_U\to\mathcal{D}_F$ which fits into a functorial distinguished triangle
\begin{equation}\label{triangle cat recollement j_*/j_! dt-1}
(j_!,j_*,i_*(j_*/j_!)).
\end{equation}
The dual construction provides a functor $T:\mathcal{D}_U\to\mathcal{D}_F$, which is characterized by a distinguished triangle $(i_*T,j_!,j_*)$. A triangle of this type is deduced from (\ref{triangle cat recollement j_*/j_! dt-1}) by rotation, so we have an isomorphism $T=(j_*/j_!)[-1]$. Applying $i^*$ and $i^!$ to the triangle (\ref{triangle cat recollement j_*/j_! dt-1}) (and its rotation), and noting that $i^*j_!=i^!j_*=0$, we obtain isomorphisms
\begin{equation}\label{triangle cat recollement j_*/j_! isomorphic}
i^*j_*\stackrel{\sim}{\to} j_*/j_!\stackrel{\sim}{\to} i^!j_![1].
\end{equation}

Let $X\in\mathcal{D}$ and apply (TR4) to the adjuction morphims $j_!j^*(X)\to X\to j_*j^*(X)$, we obtain an octahedron
\begin{equation*}
\begin{tikzcd}[row sep=2pt,column sep=4pt]
&&&{}&\\
&&A\ar[ru]\ar[rdd]&&\\
&&&{}&{}&{}\\
&X\ar[rd]\ar[ruu]&&B\ar[rru]\ar[rrddd]\\
&&j_*j^*(X)\ar[ru]\ar[rrrdd]&&\\
&&&&\\
j_!j^*(X)\ar[ruuu]\ar[rruu]&&&&&C\ar[rdd]\ar[rrd]\\
&&&&&&&{}\\
&&&&&&{}&
\end{tikzcd}
\vspace*{-2mm}
\end{equation*}
By (R3), (R4) and \cref{triangle cat morphism extension to triangle iff}, there exists a unique isomorphism $A\cong i_*i^*(X)$, which identified $X\to A$ with the adjunction morphism. It also identifies $(j_!j^*X,X,A)$ to the distinguished triangle $(j_!j^*(X),X,i_*i^*(X))$ of (R4).\par
The same argument, applied to $j_*j^*(X)$, whose image under $j^*$ is $X$, we see that $B$ is identified with $i_*i^*j_*j^*(X)=i_*(j_*/j_!)j^*(X)$ (cf. (\ref{triangle cat recollement j_*/j_! isomorphic})) and $j_*j^*(X)\to B$ is identified with the adjunction morphism of $(i^*,i_*)$. By \cref{triangle cat morphism extension to triangle iff}, there is a unique morphism $A\to B$ rendering the upper cap of an octahedron, i.e. a morphism $i_*i^*(X)\to i_*i^*j_*j^*(X)$; this is then deduced from the morphism $X\to j_*j^*(X)$ by adjunction.\par
Dually, there exists a unique isomorphism from $C$ to $i_*i^!(X)[1]$, which identified the morphism $j_*j^*(X)\to C$ with the $+1$ shift of the adjunction morphism $i_*i^!(X)\to X$ (the triangle $(X,j_*j^*(X),C)$ is obtained by rotating $(i_*i^!(X),X,j_*j^*(X))$). The morphism $B\to C$, which is the unique morphism rendering the commutative square $(B,C,j_!j^*(X),X)$, is then identified, via the isomorphism (\ref{triangle cat recollement j_*/j_! isomorphic}), with the morphism
\[i_*(j_*/j_!)j^*(X)=i_*i^!j_!j^*(X)[1]\to i_*i^!(X)[1]\]
induced from $j_!j^*(X)\to X$ by functoriality.\par
We have thus determined all the vertices, and all the arrows of the octahedron ($C\to A$ is the composition $C\to X\to A$), and proves its functoriality. If we replace $A$, $B$, $C$ by their values, the octahedron is then written as 
\vspace*{-3mm}
\begin{equation}\label{triangle cat recollement octahedron}
\begin{tikzcd}[row sep=2pt,column sep=2pt]
&&&{}&\\
&&i_*i^*(X)\ar[ru]\ar[rdd]&&\\
&&&{}&{}\\
&X\ar[rd]\ar[ruu]&&i_*(j_*/j_!)j^*(X)\ar[ru]\ar[rddd]\\
&&j_*j^*(X)\ar[ru]\ar[rrdd]&&\\
&&&&\\
j_!j^*(X)\ar[ruuu]\ar[rruu]&&&&i_*i^!(X)[1]\ar[rdd]\ar[rrd]\\
&&&&&&{}\\
&&&&&{}&
\end{tikzcd}
\vspace*{-3mm}
\end{equation}
Since $i_*$ is fully faithful, the distinguished triangle $(i_*i^*(X),i_*(j_*/j_!)j^*(X),i_*i^!(X)[1])$ is then the image under $i_*$ of a distinguished triangle $(i^*(X),(j_*/j_!)j^*(X),i^!(X)[1])$. The image under $i_*$ of the $+1$ morphism of this triangle is the composition $i_*i^!X[1]\to X\to i_*i^*(X)$, so $d:i^!(X)[1]\to i^*(X)[1]$ is the morphism (\ref{triangle cat recollement i^! to i^* morphism}) for $X[1]$ (the translation of (\ref{triangle cat recollement i^! to i^* morphism}) for $X$). Rotating the triangle, with a sign change for the morphism of degree $+1$ and erasing $i_*$ (cf. \cref{triangle cat t-structure subcategory eg}), we obtain a functorial distinguished triangle
\begin{equation}\label{triangle cat recollement j_*/j_! dt-2}
(i^!,i^*,(j_*/j_!)j^*).
\end{equation}

Now let $(\mathcal{D}_U^{\leq 0},\mathcal{D}_U^{\geq 0})$ be a $t$-structure over $\mathcal{D}_U$ and $(\mathcal{D}_Z^{\leq 0},\mathcal{D}_Z^{\geq 0})$ be a $t$-structure over $\mathcal{D}_Z$. We define subcategories of $\mathcal{D}$ by
\begin{align*}
\mathcal{D}^{\leq 0}&:=\{K\in\mathcal{D}:\text{$j^*(K)\in\mathcal{D}_U^{\leq 0}$ and $i^*(K)\in\mathcal{D}_Z^{\leq 0}$}\},\\
\mathcal{D}^{\geq 0}&:=\{K\in\mathcal{D}:\text{$j^*(K)\in\mathcal{D}_U^{\geq 0}$ and $i^!(K)\in\mathcal{D}_Z^{\geq 0}$}\}.
\end{align*}

\begin{theorem}\label{triangle cat recollement theorem}
With the preceding hypotheses, $(\mathcal{D}^{\leq 0},\mathcal{D}^{\geq 0})$ is a $t$-structure over $\mathcal{D}$.
\end{theorem}
\begin{proof}
Let $X\in\mathcal{D}^{\leq 0}$ and $Y\in\mathcal{D}^{\geq 1}$. The triangle $(j_!j^*(X),X,i_*i^*(X))$ gives the exact sequence
\begin{small}
\[\Hom(i^*(X),i^!(Y))=\Hom(i_*i^*(X),Y)\to \Hom(X,Y)\to\Hom(j_!j^*(X),Y)=\Hom(j^*(X),j^*(Y)).\]
\end{small}
We then conclude from axiom (t2) of $\mathcal{D}_U$ and $\mathcal{D}_Z$ that $\Hom(X,Y)=0$. Also, in view of the definition above, the axiom (t1) for $\mathcal{D}$ follows from that of $\mathcal{D}_U$ and $\mathcal{D}_Z$.\par
For $X\in\mathcal{D}$, to verify axiom (t3), we choose objects $Y$ and $A$ so that we have distinguished triangles $(Y,X,j_*\tau^{>0}j^*(X))$ and $(A,Y,i_*\tau^{>0}i^*(Y))$, and apply (TR4):
\vspace*{-2mm}
\begin{equation*}
\begin{tikzcd}[row sep=2pt,column sep=4pt]
&&&{}&\\
&&i_*\tau^{>0}i^*(Y)\ar[ru]\ar[rdd]&&\\
&&&{}&{}&{\hspace*{8mm}}\\
&Y\ar[rd]\ar[ruu]&&B\ar[rru]\ar[rrddd]\\
&&X\ar[ru]\ar[rrrdd]&&\\
&&&&\\
A\ar[ruuu]\ar[rruu]&&&&&j_*\tau^{>0}j^*(X)\ar[rdd]\ar[rrd]\\
&&&&&&&{}\\
&&&&&&{}&
\end{tikzcd}
\vspace*{-3mm}
\end{equation*}
Applying $j^*$, $i^*$, $i^!$ to the distinguished triangles in this octahedron, we conclude from (R3) and (R5) that
\begin{alignat*}{2}
&j^*(i_*\tau^{>0}i^*(Y),B,j_*\tau^{>0}j^*(X))=(0,j^*(B),\tau^{>0}j^*(X))&\quad\quad&\text{whence $j^*(B)\stackrel{\sim}{\to} \tau^{>0}j^*(X)$},\\
&j^*(A,X,B)=(j^*(A),j^*(X),\tau^{>0}j^*(X))&\quad\quad&\text{whence $j^*(A)\stackrel{\sim}{\to} \tau^{\leq 0}j^*(X)$},\\
&i^*(A,Y,i_*\tau^{>0}i^*(Y))=(i^*(A),i^*(Y),\tau^{>0}i^*(Y))&\quad\quad&\text{whence $i^*(A)\stackrel{\sim}{\to} \tau^{\leq 0}i^*(Y)$},\\
&i^!(i_*\tau^{>0}i^*(Y),B,j_*\tau^{>0}j^*(X))=(\tau^{>0}i^*(Y),i^!(B),0)&\quad\quad&\text{whence $\tau^{>0}i^*(Y)\stackrel{\sim}{\to} i^!(B)$}.
\end{alignat*}
We then conclude that $A\in\mathcal{D}^{\leq 0}$ and $B\in\mathcal{D}^{\geq 1}$, and this proves axiom (t3).
\end{proof}

\begin{proposition}\label{triangle cat recollement t-structure iff}
Under the hypothesis and notations of \cref{triangle cat recollement theorem}, let $(\mathcal{D}^{\leq 0},\mathcal{D}^{\geq 0})$ be a $t$-structure over $\mathcal{D}$. Then the following conditions are equivalent:
\begin{enumerate}
    \item[(\rmnum{1})] $j_!j^*$ is right $t$-exact;
    \item[(\rmnum{1}')] $j_*j^*$ is left $t$-exact;
    \item[(\rmnum{2})] the $t$-structure of $\mathcal{D}$ is obtained by glueing.
\end{enumerate}
\end{proposition}
\begin{proof}
The equivalence of (\rmnum{1}) and (\rmnum{1}') follows from \cref{triangle cat t-exact functor prop}~(c), and (\rmnum{1}), (\rmnum{1}') are equivalent to axiom (t2) for $(j^*\mathcal{D}^{\leq 0},j^*\mathcal{D}^{\geq 0})$. The distinguished triangle $(j_!j^*,\id,i_*i^*)$ and $(i_*i^!,\id,j_*j^*)$ shows respectively that (\rmnum{1}) and (\rmnum{1}') implies that $i_*i^*$ is right $t$-exact and $i_*i^!$ is left $t$-exact, which are equivalent by \cref{triangle cat t-exact functor prop}~(c), and signifies that $(i^*\mathcal{D}^{\leq 0},i^!\mathcal{D}^{\geq 0})$ verifies axiom (t2).\par
It is clear that (\rmnum{2})$\Rightarrow$(\rmnum{1}), (\rmnum{1}'), and that the $t$-structure over $\mathcal{D}_U$ and $\mathcal{D}_Z$ are such that $\mathcal{D}$ is the glueing of $(j^*\mathcal{D}^{\leq 0},j^*\mathcal{D}^{\geq 0})$ over $\mathcal{D}_U$ and $(i^*\mathcal{D}^{\leq 0},i^!\mathcal{D}^{\geq 0})$ over $\mathcal{D}_Z$. Conversely, if (\rmnum{1}) and (\rmnum{1}') are satisfied, we verify successively that
\begin{enumerate}
    \item[(a)] $(j^*\mathcal{D}^{\leq 0},j^*\mathcal{D}^{\geq 0})$ is a $t$-structure over $\mathcal{D}_U$. This follows from the fact that $j^*$ is essentially surjective.
    \item[(b)] $(i^*\mathcal{D}^{\leq 0},i^!\mathcal{D}^{\geq 0})$ is a $t$-structure over $\mathcal{D}_Z$. Only axiom (t3) is nontrivial: for $X\in\mathcal{D}_Z$, the $t$-exactness of $j^*$ shows that $j^*\tau^{\leq 0}i_*(X)=\tau^{\leq 0}j^*i_*(X)=0$, and that similarly $j^*\tau^{>0}i_*(X)=0$\footnote{The truncations functor are that for the $t$-structure $(j^*\mathcal{D}^{\leq 0},j^*\mathcal{D}^{\geq 0})$ over $\mathcal{D}_U$.}. The truncations $\tau^{\leq 0}i_*(X)$ and $\tau^{>0}i_*(X)$ are therefore in $i_*\mathcal{D}_Z$, and we obtain a distinguished triangle $(i^*\tau^{\leq 0}i_*(X),X,i^*\tau^{>0}i_*(X))$, which proves axiom (t3).
    \item[(c)] The identity functor of $\mathcal{D}$, endowed with the $t$-structure over $\mathcal{D}$ and the $t$-structure glueing $(j^*\mathcal{D}^{\leq 0},j^*\mathcal{D}^{\geq 0})$ and $(i^*\mathcal{D}^{\leq 0},i^!\mathcal{D}^{\geq 0})$, is $t$-exact.
\end{enumerate}
By \cref{triangle cat t-structure subcategory eg}, the original $t$-structure of $\mathcal{D}$ is therefore obtained by glueing $(j^*\mathcal{D}^{\leq 0},j^*\mathcal{D}^{\geq 0})$ and $(i^*\mathcal{D}^{\leq 0},i^!\mathcal{D}^{\geq 0})$.
\end{proof}

Suppose that we are given a $t$-structure over $\mathcal{D}_Z$, and apply \cref{triangle cat recollement theorem} to the degenerate $t$-structure $(\mathcal{D}_U,0)$ over $\mathcal{D}_U$ and the given $t$-structure over $\mathcal{D}_Z$. The functor $\tau^{\leq n}$ relative to the $t$-structure obtained over $\mathcal{D}$ is then denoted by $\tau^{\leq n}_Z$, which is right adjoint to the inclusion of the full subcategory of $\mathcal{D}$ formed by $X\in\mathcal{D}$ such that $i^*(X)\in\mathcal{D}_Z^{\leq n}$. The proof of axiom (t3) in \cref{triangle cat recollement theorem} shows that we have a distinguished triangle
\begin{equation}\label{triangle cat recollement truncation tau_Z^leq dt}
(\tau^{\leq n}_ZX,X,i_*\tau^{>n}i^*(X))
\end{equation}
(with the notations of \cref{triangle cat recollement theorem}, we have $X=Y$ since $j_*\tau^{>0}j^*(X)=0$). The cohomology functor $H^n$ for this $t$-structure is then given by $i_*H^ni^*(X)$.\par
Dually, we define $\tau^{\geq n}_Z$ via the degenerate $t$-structure $(0,\mathcal{D}_U)$ over $\mathcal{D}_U$, which is left adjoint to the inclusion of the full subcategory of $\mathcal{D}$ formed by $X\in\mathcal{D}$ such that $i^!(X)\in\mathcal{D}_Z^{\geq n}$. We have a distinguished triangle
\begin{equation}\label{triangle cat recollement truncation tau_Z^geq dt}
(i_*\tau^{<n}i^!(X),X,\tau^{\geq n}_ZX),
\end{equation}
and the functor $H^n$ is given by $i_*H^ni^!(X)$.\par
Similarly, if we are given a $t$-structure over $\mathcal{D}_U$, and we endow $\mathcal{D}_Z$ with the $t$-structure $(\mathcal{D}_Z,0)$ (resp. $(0,\mathcal{D}_Z)$), we can define $t$-structure over $\mathcal{D}$, and truncation functor $\tau^{\leq n}_U$ (resp. $\tau^{\geq n}_U$), which fit into distinguished triangles
\begin{equation}\label{triangle cat recollement truncation tau_U dt}
(\tau^{\leq n}_UX,X,j_*\tau^{>n}j^*(X)),\quad\quad (\text{resp.}\quad (j_!\tau^{<n}j^*(X),X,\tau^{\geq n}_UX)).
\end{equation}
The cohomology functor $H^n$ is given by $j_*H^nj^*(X)$ (resp. $j_!H^nj^*(X)$).\par
The proof of axiom (t3) in \cref{triangle cat recollement theorem} shows that $\tau^{\leq 0}=\tau^{\leq 0}_Z\tau^{\leq 0}_U$. By translating and duality, we obtain
\begin{equation}\label{triangle cat recollement truncation functor decomposition}
\tau^{\leq n}=\tau^{\leq n}_Z\tau^{\leq n}_U,\quad \tau^{\geq n}=\tau^{\geq n}_Z\tau^{\geq n}_U.
\end{equation}

\begin{example}\label{triangle cat recollement truncation tau_Z for topo space}
In the situation of \ref{triangle cat recollement topo space paragraph} and for the natural $t$-structure of $D^+(Z,\mathscr{O}_Z)$, the functor $\tau_Z^{\leq n}$ is deduced from the functor of the category of complexes of sheaves into itself which to a complex $K$ associates the subcomplex which coincides with $K$ over $U$, and with the subcomplex $\tau^{\leq n}K$ over $Z$.
\end{example}

A \textbf{prolongation} of an object $Y$ of $\mathcal{D}_U$ is defined to be an object $X$ of $\mathcal{D}$ endowed with an isomorphism $j^*(X)\cong Y$. Such an isomorphism gives by adjunction morphisms $j_!(Y)\to X\to j_*(Y)$. If $n\in\Z$ is an integer, then from the distinguished triangle (\ref{triangle cat recollement truncation tau_Z^leq dt}) (resp. (\ref{triangle cat recollement truncation tau_Z^geq dt})) and (R3), (R5), we see that $\tau^{\geq n}_Zj_!(Y)$ (resp. $\tau_Z^{\leq n}j_*(Y)$) is a prolongation of $Y$. If a prolongation $X$ is isomorphic as a prolongation to $\tau^{\geq n}_Zj_!(Y)$ (resp. $\tau_Z^{\leq n}j_*(Y)$), the isomorphism $j^*(X)\cong Y$ is then uniquely determined, and we simply write $X=\tau^{\geq n}_Zj_!(Y)$ (resp. $\tau_Z^{\leq n}j_*(Y)$).

\begin{proposition}\label{triangle cat recollement prolongation exist}
Let $Y\in\mathcal{D}_U$ and $n$ be an integer. There exists, up to unique isomorphisms, a unique prolongation $X$ of $Y$ such that $i^*(X)\in\mathcal{D}_Z^{\leq n-1}$ and $i^!(X)\in\mathcal{D}_Z^{\geq n+1}$. This prolongation is given by $\tau^{\leq n-1}_Zj_*(Y)$, and we have $\tau^{\leq n-1}_Zj_*(Y)=\tau^{\geq n+1}_Zj_!(Y)$.
\end{proposition}
\begin{proof}
Let $X$ be a prolongation of $Y$. The distinguished triangle $(i^*(X),(j_*/j_!)(Y),i^!(X)[1])$ obtained by rotating (\ref{triangle cat recollement j_*/j_! dt-2}) shows that the following conditions are equivalent:
\begin{enumerate}
    \item[(\rmnum{1})] $i^*(X)\in\mathcal{D}_Z^{\leq n-1}$ and $i^!(X)\in\mathcal{D}_Z^{\geq n+1}$;
    \item[(\rmnum{2})] $i^!(X)[1]=\tau^{\geq n}(j_*/j_!)(Y)=\tau^{\geq n}i^*j_*(Y)$;
    \item[(\rmnum{2}')] $i^*(X)=\tau^{\leq n-1}(j_*/j_!)(Y)=\tau^{\leq n}i^!j_!(Y)[1]$. 
\end{enumerate}
The distinguished triangles $(X,j_*(Y),i_*i^!(X)[1])$ of (\ref{triangle cat recollement octahedron}) and $(\tau^{\leq n-1}_Z,\id,i_*\tau^{>n-1}i^*)$ imply that condition (\rmnum{2}) is equivalent to $X=\tau^{\leq n-1}_Fj_*(Y)$, and the triangles $(j_!(Y),X,i_*i^*(X))$ of (\ref{triangle cat recollement octahedron}) and $(i_*\tau^{<n+1}i^!,\id,\tau_Z^{\geq n+1})$ imply that condition (\rmnum{2}') is equivalent to $X=\tau_Z^{\geq n+1}j_!(Y)$; we therefore conclude the proposition.
\end{proof}

\begin{remark}\label{triangle cat recollement prolongation equivalence by j^*}
Let $n\in\Z$ be an integer and $\mathcal{D}_m$ be the full subcategory of $\mathcal{D}$ formed by objects $X$ such that $i^*(X)\in\mathcal{D}_Z^{\leq n-1}$ and $i^!(X)\in\mathcal{D}_Z^{\geq n+1}$. The functor $j^*$ then induces an equivalence of categories $\mathcal{D}_m\stackrel{\sim}{\to}\mathcal{D}_U$, and it admits $\tau^{\leq n-1}_Zj_*$ as a quasi-inverse, which we often denote by ${^pj_{!*}}$.
\end{remark}

Let $\mathcal{C}$, $\mathcal{C}_U$ and $\mathcal{C}_Z$ be the hearts of the $t$-categories $\mathcal{D}$, $\mathcal{D}_U$ and $\mathcal{D}_Z$, respectively, where $\mathcal{D}_U$ and $\mathcal{D}_Z$ are endowed with given $t$-structures, and $\mathcal{D}$ with the glueing $t$-structure. Denote by $\eps$ the inclusions of $\mathcal{C}$, $\mathcal{C}_U$ or $\mathcal{C}_Z$ into $\mathcal{D}$, $\mathcal{D}_U$ or $\mathcal{D}_Z$, and for $T$ the functors $j_!,j^*,j_*,i^*,i_*,i^!$, we denote by $^{pT}$ the functor $H^0\circ T\circ\eps$. By the definition of the $t$-structure of $\mathcal{D}$, $j^*$ is $t$-exact, $i^*$ is right $t$-exact, and $i^!$ is left $t$-exact. Applying \cref{triangle cat t-exact functor prop}~(c) and (d), we obtain the following result:
\begin{proposition}\label{triangle cat recollement functors composition and exact sequence}
Let $\mathcal{D}$, $\mathcal{D}_U$ and $\mathcal{D}_Z$ be triangulated categories as above.
\begin{enumerate}
    \item[(a)] The functors $j_!$ and $i^*$ (resp. $j_*$ and $i^!$, resp. $j^*$ and $i_*$) are right $t$-exact (resp. left $t$-exact, resp. $t$-exact), and we have adjoint triples $({^pj_!},{^pj^*},{^pj_*})$ and $({^pi^*},{^pi_*},{^pi^!})$.
    \item[(b)] The composition ${^pj^*}\circ{^pi_*}$, ${^pi^*}\circ{^pj_!}$ and ${^pi^!}\circ{^pj_*}$ are zero. For $A\in\mathcal{C}_Z$ and $B\in\mathcal{C}_U$, we have
\[\Hom({^pj_!}(B),{^pi_*}(A))=\Hom({^pi_*}(A),{^pj_*}(B))=0.\]
    \item[(c)] For $A\in\mathcal{C}$, the sequences
    \[\begin{tikzcd}
        {^pj_!}{^pj^*}(A)\ar[r]&A\ar[r]&{^pi_*}{^pi^*}(A)\ar[r]&0
    \end{tikzcd}\]
    \vspace*{-4mm}
    \[\begin{tikzcd}
        0\ar[r]&{^pi_*}{^pi^!}(A)\ar[r]&A\ar[r]&{^pj_*}{^pj^*}(A)
    \end{tikzcd}\]
    are exact.
    \item[(d)] The functors ${^pi_*}$, ${^pj_!}$ and ${^pj_*}$ are fully faithful, or equivalently, ${^pi^*}{^pi_*}\to\id\to{^pi^!}{^pi_*}$ and ${^pj^*}{^pj_*}\to\id\to{^pj^*}{^pj_!}$ are isomorphisms. 
\end{enumerate}
\end{proposition}
\begin{proof}
Only the exact sequences in (c) are nontrivial. For this, we note that by \cref{triangle cat t-exact functor prop} we have ${^pT_2}\circ{^pT_1}=H^0(T_2\circ T_1)$, so for $A\in\mathcal{C}$ we have a long exact sequence
\[\begin{tikzcd}
\cdots\ar[r,"+1"]&H^0(j_!j^*(A))={^pj_!}{^pj^*}(A)\ar[r]&A\ar[r]&{^pi_*}{^pi^*}(A)=H^0(i_*i^*(A))\ar[r,"+1"]&\cdots
\end{tikzcd}\]
To see that the morphism $A\to H^0(i_*i^*(A))$ is surjective, we note that since $j_!$ is right $t$-exact and $j^*$ is $t$-exact, we have $j_!j^*(A)\in\mathcal{D}^{\leq 0}$, so $H^1(j_!j^*(A))=0$ and we obtain the first exact sequence in (c). The second one can be deduced similarly, using the fact that $j_*$ is left $t$-exact and $j^*$ is $t$-exact (so $j_*j^*(A)\in\mathcal{D}^{\geq 0}$).
\end{proof}

\begin{remark}\label{triangle cat recollement ^pi_* adjoint char}
By ${^pj^*}{^pi_*}=0$ and the exact sequence of (c), for $X\in\mathcal{C}$ to be in the essential image $\widebar{\mathcal{C}}_Z$ of ${^pi_*}$, it is necessary and sufficient that ${^pj^*}(X)=0$. Since the functor ${^pj^*}$ is exact, this essential image is a Serre subcategory of $\mathcal{C}$. If we identify $\mathcal{C}_Z$ with $\widebar{\mathcal{C}}_Z$ by the fully faithful functor ${^pi_*}$, the adjunctions $({^pi^*},{^pi_*})$ and $({^pi_*},{^pi_!})$ show that for $X\in\mathcal{C}$, ${^pi^*}(X)$ is the largest quotient of $X$ which is in $\mathcal{C}_Z$, and ${^pi^!}(X)$ is the largest sub-object of $X$ that is in $\mathcal{C}_Z$.
\end{remark}

\begin{proposition}\label{triangle cat recollement ^pj^* is quotient}
The functor ${^pj^*}$ identifies $\mathcal{C}_U$ with the quotient of $\mathcal{C}$ by the Serre subcategory $\mathcal{C}_Z$ (or, more precisely, its iamge $\widebar{\mathcal{C}}_Z$).
\end{proposition}
\begin{proof}
Let $Q:\mathcal{C}\to\mathcal{C}/\mathcal{C}_Z$ be the quotient functor. The exact functor ${^pj^*}$ admits a factorization $T\circ Q$:
\[\begin{tikzcd}
\mathcal{C}\ar[rd,swap,"Q"]\ar[rr,"{^pj^*}"]&&\mathcal{C}_U\\
&\mathcal{C}/\mathcal{C}_Z\ar[ru,swap,"T"]
\end{tikzcd}\]
We note that $T$ is faithful: if a morphism $f$ in $\mathcal{C}/\mathcal{C}_Z$ comes from a morphism $f_1$ of $\mathcal{C}$, and $f$ is killed by $T$, then $f_1$ is killed by ${^pj^*}$. Since ${^pj^*}$ is exact, this implies ${^pj^*}(\im(f_1))=\im({^pj^*}(f_1))=0$, so $\im(f_1)\in\widebar{\mathcal{C}}_Z$ and $f_1$ is killed by $Q$. Since $\id\stackrel{\sim}{\to}{^pj^*}{^pj_!}=T\circ Q\circ{^pj_!}$, we see that $T$ is essentially surjective, so it remains to verify that $T$ is fully faithful, whence an equivalence of categories.\par
We note that for $A\in\mathcal{C}$ there is an exact sequence
\[\begin{tikzcd}
0\ar[r]&{^pi_*}H^{-1}i^*(A)\ar[r]&{^pj_!}{^pj^*}(A)\ar[r]&A\ar[r]&{^pi_*}{^pi^*}(A)\ar[r]&0
\end{tikzcd}\]
(Note that $i_*$ is $t$-exact and $i^*$ is right $t$-exact, so we have $H^{-1}(i_*i^*(A))={^pi_*}H^{-1}i^*(A)$ by \cref{triangle cat t-exact functor prop}~(b).) The kernel and cokernel of the morphism ${^pj_!}{^pj^*}(A)\to A$ are therefore in the image of ${^pi_*}$, and any object of $\mathcal{C}/\widebar{\mathcal{C}}_Z$ is then contained in the essential image of ${^pj_!}$. For ${^pj_!}(X)$ and ${^pj_!}(Y)$ in this image, the map
\[T:\Hom(Q{^pj_!}(X),Q{^pj_!}(Y))\to\Hom(TQ{^pj_!}(X),TQ{^pj_!}(Y))=\Hom(X,Y)\]
admits a section $Q{^pj_!}$, and hence is surjective. This completes the proof.
\end{proof}

\begin{remark}
In the situation of \ref{triangle cat recollement topo space paragraph}, if we endow $D^+(U,\mathscr{O}_U)$ and $D^+(Z,\mathscr{O}_Z)$ the natural $t$-structures, then the glueing $t$-structure on $D^+(X,\mathscr{O}_X)$ is the natural $t$-structure, and the abelian categories $\mathcal{C}$, $\mathcal{C}_U$ and $\mathcal{C}_Z$ are $\Mod(\mathscr{O}_X)$, $\Mod(\mathscr{O}_U)$ and $\Mod(\mathscr{O}_Z)$. In the general case, however, the functor ${^pj_!}$ is only right exact, ${^pi_*}$ is only left exact, and the first sequence of \cref{triangle cat recollement functors composition and exact sequence}~(c) is not left exact.
\end{remark}

Since the functor ${^pi_*}$ is fully faithful, the composition of the adjunction morphisms ${^pi_*}{^pi^!}\to\id\to{^pi_*}{^pi^*}$ is the image under ${^pi_*}$ of a unique morphism of functors
\begin{equation}\label{triangle cat recollement i^! to i^* heart morphism-1}
{^pi^!}\to{^pi^*}.
\end{equation}
The diagrams of \cref{triangle cat adjoint functor ^pT diagram} for $(i^*,i_*)$ and $(i_*,i^!)$, and the $t$-exactness of $i_*$, imply that for $A\in\mathcal{C}$ we have a commutative diagram
\[\begin{tikzcd}
{^pi_*}{^pi_!}\ar[r]\ar[d,equal]&\id\ar[dd,equal]\ar[r]&{^pi_*}{^pi^*}\ar[d,equal]\\
i_*{^pi_!}\ar[d]&&i_*{^pi^*}\\
i_*i^!\ar[r]&\id\ar[r]&i_*i^*\ar[u]
\end{tikzcd}\]
Therefore, for $A\in\mathcal{C}$, the morphism ${^pi^!}(A)\to {^pi^*}(A)$ in (\ref{triangle cat recollement i^! to i^* heart morphism-1}) is given by the composition
\begin{equation}\label{triangle cat recollement i^! to i^* heart morphism-2}
{^pi^!}(A)\to i^!(A)\stackrel{(\ref{triangle cat recollement i^! to i^* morphism})}{\longrightarrow} i^*(A)\to {^pi^*}(A).
\end{equation}
By \cref{triangle cat recollement functors composition and exact sequence}(d), if we apply (\ref{triangle cat recollement i^! to i^* heart morphism-1}) to ${^pi_*}(A)$ (for $A\in\mathcal{C}_Z$), then we obtain the identity morphism of $A$.\par

On the other hand, since the functor ${^pj^*}$ is identified with the quotient functor (\cref{triangle cat recollement ^pj^* is quotient}), the composition of the adjunction morphisms ${^pj_!}{^pj^*}\to\id\to{^pj_*}{^pj^*}$ provides a unique morphism of functors
\begin{equation}\label{triangle cat recollement j_! to j_* heart morphism-1}
{^pj_!}\to{^pj_*}.
\end{equation}
Similarly, the diagrams of \cref{triangle cat adjoint functor ^pT diagram} for $(j^*,j_*)$ and $(j_!,j^*)$, together with the $t$-exactness of $j^*$, imply that for $A\in\mathcal{C}$ we have a commutative diagram
\[\begin{tikzcd}
{^pj_!}{^pj^*}\ar[r]\ar[d,equal]&\id\ar[dd,equal]\ar[r]&{^pj_*}{^pj^*}\ar[d,equal]\\
{^pj_!}j^*&&{^pj_*}j^*\ar[d]\\
j_!j^*\ar[r]\ar[u]&\id\ar[r]&j_*j^*
\end{tikzcd}\]
Therefore, for $B\in\mathcal{C}_U$, the morphism $j_!(B)\to j_*(B)$ of (\ref{triangle cat recollement j_! to j_* heart morphism-1}) is the composition
\begin{equation}\label{triangle cat recollement j_! to j_* heart morphism-2}
j_!(B)\to \tau^{\geq 0}j_!(B)={^pj_!}(B)\stackrel{(\ref{triangle cat recollement j_! to j_* morphism})}{\longrightarrow}{^pj_*}(B)=\tau^{\leq 0}j_*(B)\to j_*(B).
\end{equation}
By \cref{triangle cat recollement functors composition and exact sequence}(d), if we apply ${^pj_*}$ to (\ref{triangle cat recollement j_! to j_* heart morphism-1}), then we obtain the identity morphism. In particular, for $B\in\mathcal{C}_U$, the kernel and cokernel of ${^pj_!}(B)\to {^pj_*}(B)$ are in ${^pi_*}\mathcal{C}_Z$.

\begin{definition}
The functor $j_{!*}:\mathcal{C}_U\to\mathcal{C}$ is defined to be the functor which to $B\in\mathcal{C}_U$ associates the image of ${^pj_!}(B)$ into ${^pj_*}(B)$.
\end{definition}

For $B\in\mathcal{C}_U$, (\ref{triangle cat recollement j_! to j_* heart morphism-2}) shows that we have the following factorization of the morphism $j_!(B)\to j_*(B)$ of (\ref{triangle cat recollement j_! to j_* morphism}):
\begin{equation}\label{triangle cat recollement j_! to j_* heart morphism-3}
j_!(B)\to {^pj_!}(B)\to j_{!*}(B)\to {^pj_*}(B)\to j_*(B)
\end{equation}

\begin{proposition}\label{triangle cat recollement ^pj_! and ^pj_* is truncation}
For $B\in\mathcal{C}_U$, we have
\begin{align*}
{^pj_!}(B)&=\tau^{\geq 0}_Zj_!(B)=\tau^{\leq-2}_{Z}j_*(B),\\
j_{!*}(B)&=\tau^{\geq 1}_Zj_!(B)=\tau^{\leq-1}_Zj_*(B),\\
{^pj_*}(B)&=\tau^{\geq 2}_Zj_!(B)=\tau^{\leq 0}_Zj_*(B).
\end{align*}
More precisely, ${^pj_!}(B)$, endowed with the morphism $j_!(B)\to{^pj_!}(B)$, is isomorphic to $\tau^{\geq 0}_Zj_!(B)$, and so on.
\end{proposition}
\begin{proof}
Since $j^*j_!(B)\cong B$ is in $\mathcal{C}_U$, we see that $j_!(B)$ is stable under $\tau^{\geq 0}_U$. By (\ref{triangle cat recollement truncation functor decomposition}) and \cref{triangle cat recollement functors composition and exact sequence}~(a), we then have ${^pj_!}(B)=\tau^{\geq 0}j_!(B)=\tau^{\geq 0}_Zj_!(B)$, and \cref{triangle cat recollement prolongation exist} shows that $\tau^{\geq 0}_Zj_!(B)=\tau^{\leq -2}_Zj_*(B)$. Similarly, since $j^*j_*(B)\cong B$, we have ${^pj_*}(B)=\tau^{\leq 0}_Zj_*(B)=\tau^{\geq 2}_Zj_!(B)$.\par
The determination $H^n=i_*H^ni^!$ for the $t$-structure defining $\tau^{\geq n}_Z$ shows that we have a distinguished triangle
\[(i_*H^0i^!j_!(B),\tau^{\geq 0}_Zj_!(B),\tau^{\geq 1}_Zj_!(B))=(i_*H^0i_!j_!(B),{^pj_!}(B),\tau^{\geq 1}_Zj_!(B)),\]
which implies that $\tau_Z^{\geq 1}j_!(B)\in\mathcal{D}^{[-1,0]}$. A dual argument provides a distinguished triangle
\[(\tau^{\leq -1}_Zj_*(B),\tau^{\leq 0}_Fj_*(B),i_*H^0i^*j_*(B))=(\tau^{\leq -1}_Zj_*(B),{^pj_*}(B),i_*H^0i^*(B)),\]
which shows that $\tau^{\leq -1}_Zj_*(B)\in\mathcal{D}^{[0,1]}$. By \cref{triangle cat recollement prolongation exist}, we then conclude that $\tau^{\geq 1}_Zj_!(B)=\tau^{\leq -1}_Zj_*(B)$ belongs to $\mathcal{C}$, and the above triangles produce short exact sequences
\[\begin{tikzcd}
0\ar[r]&i_*H^0i^!j_!(B)\ar[r]&{^pj_!}(B)\ar[r]&\tau^{\geq 1}_Zj_!(B)\ar[r]&0
\end{tikzcd}\]
\vspace*{-4mm}
\[\begin{tikzcd}
0\ar[r]&\tau_Z^{\leq -1}j_*(B)\ar[r]&{^pj_*}(B)\ar[r]&i_*H^0i^*j_*(B)\ar[r]&0
\end{tikzcd}\]
These together show that $\tau^{\geq 1}_Zj_!(B)=\tau^{\leq -1}_Zj_*(B)$ is the image $j_{!*}(B)$ of ${^pj_!}(B)$ in ${^pj_*}(B)$.
\end{proof}

\begin{corollary}\label{triangle cat recollement j_!* prolongation char}
For $B\in\mathcal{C}_U$, $j_{!*}(B)$ is the unique prolongation $X$ of $B$ in $\mathcal{D}$ such that $i^*(X)\in\mathcal{D}_Z^{\leq -1}$ and $i^!(X)\in\mathcal{D}_Z^{\geq 1}$.
\end{corollary}
\begin{proof}
This follows from \cref{triangle cat recollement prolongation exist}. Similarly, ${^pj_!}(B)$ (resp. ${^pj_*}(B)$) is the unique prolongation $X$ such that $i^*(X)\in\mathcal{D}_Z^{\leq -2}$ (resp. $\mathcal{D}_Z^{\leq 0}$) and $i^!(X)\in\mathcal{D}_Z^{\geq 0}$ (Resp. $\mathcal{D}_Z^{\geq 2}$).
\end{proof}

\begin{corollary}\label{triangle cat recollement j_!* char by subobject quotient}
For $B\in\mathcal{C}_U$, $j_{!*}(B)$ is the unique prolongation $X$ of $B$ in $\mathcal{C}$ with no nontrivial sub-object or quotient in the essential image $\widebar{\mathcal{C}}_Z$ of $\mathcal{C}_Z$ under ${^pi}_*$.
\end{corollary}
\begin{proof}
By definition, $j_{!*}(B)$ is in $\mathcal{C}\sub\mathcal{D}$. For any prolongation $X\in\mathcal{C}$ of $B$, we have $i^*(X)\in\mathcal{D}_Z^{\leq 0}$, and $i^*(X)\in\mathcal{D}_Z^{\leq -1}$ if and only if ${^pi^*}(X)=0$. Dually, $i^!(X)\in\mathcal{D}_Z^{\geq 0}$, and $i^!(X)\in\mathcal{D}_Z^{\geq 1}$ if and only if ${^pi^!}(X)=0$. Identify $\mathcal{C}_Z$ with $\widebar{\mathcal{C}}_Z$ via ${^pi_*}$. Since ${^pi^*}(X)$ (resp. ${^pi^!}(X)$) is the largets quotient (resp. sub-object) of $X$ which is in $\mathcal{C}_Z$ (cf. \cref{triangle cat recollement ^pi_* adjoint char}), the characterization \cref{triangle cat recollement j_!* char by subobject quotient} of $j_{!*}(B)$ follows from \cref{triangle cat recollement j_!* prolongation char}.
\end{proof}

\begin{proposition}\label{triangle cat recollement simple object of heart char}
The simple objects of $\mathcal{C}$ are the ${^pi_*}(S)$, for $S$ simple in $\mathcal{C}_Z$, and the $j_{!*}(S)$, for $S$ simple in $\mathcal{C}_U$.
\end{proposition}
\begin{proof}
Since the essential image $\widebar{\mathcal{C}}_Z$ of $\mathcal{C}_Z$ under ${^pi_*}$ is a Serre subcategory of $\mathcal{C}$, for an object $X\in\mathcal{C}$ to be simple, it is necessary and sufficient that one of the following conditions is satisfied:
\begin{enumerate}
    \item[(a)] $X\in\widebar{\mathcal{C}}_Z$ and is simple in $\widebar{\mathcal{C}}_Z$;
    \item[(b)] the image of $X$ in $\mathcal{C}/\widebar{\mathcal{C}}_Z$ is simple, and $X$ has no nontrial sub-object or quotient in $\widebar{\mathcal{C}}_Z$.
\end{enumerate}
The case (a) corresponds to $X={^pi_*}(S)$, for $S$ simple in $\mathcal{C}_Z$, and by \cref{triangle cat recollement j_!* char by subobject quotient} and \cref{triangle cat recollement ^pj^* is quotient}, case (b) corresponds to $X=j_{!*}(S)$, for $S$ simple in $\mathcal{C}_U$.
\end{proof}

\section{Perverse sheaves}
\subsection{Stratified spaces}\label{triangle cat perverse sheaf stratified space subsection}
Let $X$ be a topological space endowed with a structural sheaf of rings $\mathscr{O}_X$, $\mathcal{S}$ be a finite partition of $X$ by locally closed subsets (a \textit{stratification}), and $p:\mathcal{S}\to\Z$ be a function (called the \textbf{perversity}). By definition, the stratum $S$ is nonempty. We further suppose that the closure of any strarum is a union of strata.\par
For a continuous map $f:X\to Y$, we shall simply write $f_!$, $f_*$, $f^!$, $f^*$ for the derived functors $Rf_!$, $Rf_*$, $Rf^!$, $Lf^*$. This notation is motivated by the fact that we often work with the derived categories, rather than the module categories. The corresponding functors on module categories will be denoted by ${^0f_!}$, ${^0f_*}$, ${^0f^!}$, ${^0f^*}$.

\begin{definition}
We denote by ${^pD^{\leq 0}}(X,\mathscr{O}_X)$ (resp. ${^pD^{\geq 0}}(X,\mathscr{O}_X)$) the subcategory of $D(X,\mathscr{O}_X)$ formed by complexes $\mathscr{F}\in D(X,\mathscr{O}_X)$ (resp. $\mathscr{F}\in D^+(X,\mathscr{O}_X)$) such that for any strata $S\in\mathcal{S}$, we have $H^n(i_S^*(\mathscr{F}))=0$ for $n>p(S)$ (resp. $H^n(i_S^!(\mathscr{F}))=0$ for $n<p(S)$), where $i_S:S\to X$ denote the inclusion map.
\end{definition}

The exactness of the functor ${^0i^*}$ allows us to give another definition for ${^pD^{\leq 0}(X,\mathscr{O}_X)}$: for $\mathscr{F}$ to be in ${^pD^{\leq 0}}(X,\mathscr{O}_X)$, it is necessary and sufficient that the restriction of $H^i(\mathscr{F})$ to $S$ is zero for $i>p(S)$. The truncation functors $\tau^{\leq n}$ and $\tau^{\geq n}$, relative to the natrual $t$-structure of $D(X,\mathscr{O}_X)$, therefore sends ${^pD^{\leq 0}}(X,\mathscr{O}_X)$ into itself.

\begin{remark}\label{triangle cat perverse condition finite cohomology dim}
If the functors ${^0i_S^!}$ have finite cohomological dimensions, then $i_S^!:D^+(X,\mathscr{O}_X)\to D^+(S,\mathscr{O}_S)$ has a natural extension $D^+(X,\mathscr{O}_X)\to D^+(S,\mathscr{O}_S)$, still denoted by $i_S^!$. In this case, the condition "$H^n(i_S^!(\mathscr{F}))=0$ for $n<p(S)$" makes sense for any complex $\mathscr{F}$ in $D(X,\mathscr{O}_X)$. In fact, this condition implies that $\mathscr{F}\in D^+(X,\mathscr{O}_X)$, or more precisely, that $\mathscr{F}\in D^{\geq n}(X,\mathscr{O}_X)$ for $p\geq n$. To see this, we shall prove by descendent induction on the strata $\mathcal{S}$ (for the order $S_1\sub\widebar{S}_2$) that the restriction of $H^i(\mathscr{F})$ to $S$ is zero for $i<n$. We first note that the distinguished triangle $(\tau^{<n}\mathscr{F},\mathscr{F},\tau^{\geq n}\mathscr{F})$ and the left exactness of ${^0i_S^!}$ imply that $H^i(i_S^!\tau^{<n}\mathscr{F})\stackrel{\sim}{\to} H^i(i_S^!(\mathscr{F}))$. The induction hypothesis shows that $H^j(\tau^{<n}\mathscr{F})$ is zero over the strata $T\neq S$ such that $S\sub\widebar{T}$, so $S$ admits an neighborhood in which $H^j(\tau^{<n}\mathscr{F})$ is supported in $S$. We then have
\[H^i(\mathscr{F})|_S=H^i(i_S^*(\tau^{<n}\mathscr{F}))=H^i(i_S^!(\tau^{<n}\mathscr{F})),\]
and the last member is zero by our hypothesis.\par
Without the hypothesis on cohomological dimension, the same argument is applicable for $\mathscr{F}\in D^+(X,\mathscr{O}_X)$, and for integers $a\leq p\leq b$, we have
\begin{align}
D^{\leq a}(X,\mathscr{O}_X)\sub &{^pD^{\leq 0}}(X,\mathscr{O}_X)\sub D^{\leq b}(X,\mathscr{O}_X),\label{triangle cat perverse condition inclusion-1}\\
D^{\geq a}(X,\mathscr{O}_X)\sups &{^pD^{\geq 0}}(X,\mathscr{O}_X)\sups D^{\geq b}(X,\mathscr{O}_X).\label{triangle cat perverse condition inclusion-2}
\end{align}
We denote by ${^pD^{+,\leq 0}}(X,\mathscr{O}_X)$ the intersection of $D^+(X,\mathscr{O}_X)$ with ${^pD^{\leq 0}}(X,\mathscr{O}_X)$, and similarly for $+$ replaced by $-,b$ and $0$ replaced by $n\in\Z$.
\end{remark}

\begin{proposition}\label{triangle cat perverse t-structure on D^+}
For any perversity $p$, the pair $({^pD^{+,\leq 0}}(X,\mathscr{O}_X),{^pD^{+,\geq 0}}(X,\mathscr{O}_X))$ is a $t$-structure over $D^+(X,\mathscr{O}_X)$.
\end{proposition}
\begin{proof}
We preceed by induction on the number $N$ of the stratum. If $N=0$, we have $X=\emp$, and the assertion is trivial. If $N=1$, then we obtain the natrual $t$-structure on $D^+(X,\mathscr{O}_X)$, translated by $p(X)$. For $N\geq 2$, let $Z$ be a proper closed subset of $X$ which is a union of strata, and $U$ be its complement. The induction hypothesis, applied to $Z$ and $U$, endowed with the induced stratification, gives $t$-structures over $D^+(U,\mathscr{O}_U)$ and $D^+(Z,\mathscr{O}_Z)$. The $t$-structure considered over $D^+(X,\mathscr{O}_X)$ is then obtained by glueing: this is clear for ${^pD^{+,\leq 0}}(X,\mathscr{O}_X)$, since $i_S^*$ is exact. As for ${^pD^{+,\geq 0}}(X,\mathscr{O}_X)$, we note that for $\mathscr{F}\in D^+(X,\mathscr{O}_X)$ and $S\in\mathcal{S}$, we have
\begin{align*}
H^n(i_{S\cap U}^!j^*(\mathscr{F}))&=H^n(i_{S\cap U}^!j^!(\mathscr{F}))=H^n(i_{S\cap U}^!(\mathscr{F})),\\
H^n(i_{S\cap Z}^!i^!(\mathscr{F}))&=H^n(i_{S\cap Z}^!(\mathscr{F})).
\end{align*}
On the other hand, since $Z$ is a union of strata, $S$ is either disjoint from $Z$ or contained in $Z$, so we conclude that $H^n(i_S^!(\mathscr{F}))=0$ if and only if $H^n(i_{S\cap U}^!j^*(\mathscr{F}))=H^n(i_{S\cap Z}^!i^!(\mathscr{F}))=0$. We can then apply \cref{triangle cat recollement theorem} to conclude the proposition.
\end{proof}

\begin{corollary}\label{triangle cat perverse t-structure on D(X)}
The pair $({^pD^{\leq 0}}(X,\mathscr{O}_X),{^pD^{\geq 0}}(X,\mathscr{O}_X))$ is a $t$-structure over $D(X,\mathscr{O}_X)$. It induces a $t$-structure over $D^*(X,\mathscr{O}_X)$ for $*\in\{+,-,b\}$.
\end{corollary}
\begin{proof}
Let $a,b$ be integers such that $a\leq p\leq b$. For $\mathscr{F}\in {^pD^{\leq 0}}(X,\mathscr{O}_X)$ and $\mathscr{G}\in{^pD^{\geq 0}}(X,\mathscr{O}_X)$, we have $\Hom(\tau^{\leq a}\mathscr{F},\mathscr{G})=0$ since $L$ is in $D^{>a}(X,\mathscr{O}_X)$, and $\Hom(\tau^{>a}\mathscr{F},\mathscr{G})=0$ since $\tau^{>a}\mathscr{F}$ is in ${^pD^{+,\leq 0}}(X,\mathscr{O}_X)$ and we can apply \cref{triangle cat perverse t-structure on D^+}. We then conclude from the long exact sequence of $\Hom$ induced from the distinguished triangle $(\tau^{\leq a}\mathscr{F},\mathscr{F},\tau^{>a}\mathscr{F})$ that $\Hom(\mathscr{F},\mathscr{G})=0$. Finally, for any $\mathscr{F}\in D(X,\mathscr{O}_X)$, by \cref{triangle cat perverse t-structure on D^+}, there exists a distinguished triangle $(\mathscr{M},\tau^{>a}\mathscr{F},\mathscr{N})$ with $\mathscr{M}\in {^pD^{\leq 0}}(X,\mathscr{O}_X)$ and $\mathscr{N}\in {^pD^{\geq 0}}(X,\mathscr{O}_X)$. Applying (TR4), we obtain two distinguished triangles $(\tau^{\leq a}\mathscr{F},\mathscr{G},\mathscr{M})$ and $(\mathscr{G},\mathscr{F},\mathscr{N})$. The first one shows that $\mathscr{G}$ is in ${^pD^{\leq 0}}(X,\mathscr{O}_X)$, and the second one proves axiom (t3). The axiom (t1) being trivial, we then obtain a $t$-structure over $D(X,\mathscr{O}_X)$, which is called the \textbf{$t$-structure of perversity $p$}.
\end{proof}

Let ${^p}\tau$ be the corresponding truncation functor of the perverse $t$-structure. Since we have ${^pD^{\leq 0}}(X,\mathscr{O}_X)\sub D^{\leq b}(X,\mathscr{O}_X)$ and ${^pD^{\geq 0}}(X,\mathscr{O}_X)\sub D^{\geq a}(X,\mathscr{O}_X)$, the (usual) cohomology long exact sequence of the distinguished triangle $({^p\tau^{\leq 0}}\mathscr{F},\mathscr{F},{^p\tau^{\geq 1}}\mathscr{F})$ shows that $H^i({^p\tau^{\leq 0}}\mathscr{F})=H^i(\mathscr{F})$ for $i<a$ and $H^i({^p\tau^{\leq 0}}\mathscr{F})=H^i(\mathscr{F})$ for $i>b$. It follows that ${^p\tau^{\leq 0}}$ and ${^p\tau^{\geq 0}}$ preserves $D^*(X,\mathscr{O}_X)$ for $*\in\{+,-,b\}$.

\begin{definition}
The heart $\Perv(X,\mathscr{O}_X,p)$ of the $t$-structure $({^pD^{\leq 0}}(X,\mathscr{O}_X),{^pD^{\geq 0}}(X,\mathscr{O}_X))$ is called the category of \textbf{sheaves of $p$-perverse $\mathscr{O}_X$-modules over $X$}. This is an admissible abelian subcategory of $D(X,\mathscr{O}_X)$.
\end{definition}

\begin{proposition}\label{triangle cat perverse t-structure t-exactness}
Let $W$ be a locally closed subset of $X$ which is a union of strata, and $j:W\to X$ be the inclusion. For any perversity $p$, the functors $j_!:D(W,\mathscr{O}_W)\to D(X,\mathscr{O}_X)$ and $j^*:D(X,\mathscr{O}_X)\to D(W,\mathscr{O}_W)$ are right $t$-exact, and $j^!:D(X,\mathscr{O}_X)\to D(W,\mathscr{O}_W)$ and $j_*:D(W,\mathscr{O}_W)\to D(X,\mathscr{O}_X)$ are left $t$-exact.
\end{proposition}
\begin{proof}
Let $\mathscr{F}\in {^pD^{\leq 0}}(W,\mathscr{O}_W)$ and $\mathscr{G}\in {^pD^{\leq 0}}(X,\mathscr{O}_X)$. We note that for any stratum $S$ contained in $W$, if $i_S^W:S\to W$ denotes the inclusion map, we have (cf. \cref{sheaf locally closed extension image char} and \cref{sheaf locally closed extension right adjoint})
\begin{align*}
i_S^*j_!(\mathscr{F})=(i_S^W)^*(j^*j_!(\mathscr{F}))\cong (i_S^W)^*(\mathscr{F}),\quad (i_S^W)^*j^*(\mathscr{G})=i_S^*(\mathscr{G})
\end{align*}
from which we conclude that $j_!(\mathscr{F})\in {^pD^{\leq 0}}(X,\mathscr{O}_X)$ and $j^*(\mathscr{G})\in {^pD^{\leq 0}}(W,\mathscr{O}_W)$. The rest of the proposition follows from \cref{triangle cat t-exact functor prop}~(c).
\end{proof}

To simplify the notation, we shall omit $(X,\mathscr{O}_X)$ and write $D$ for $D(X,\mathscr{O}_X)$. We also sometimes write $D^{\leq p}$ (resp. $D^{\geq p}$) for ${^pD^{\leq 0}}$ (resp. ${^pD^{\geq 0}}$). If $p$ has constant value $a$, then $D^{\leq p}=D^{\leq a}$ (the natural $t$-structure), and $D^{\geq p}=D^{\geq a}$. For any integer $n$, we have $D^{\leq p+n}={^pD^{\leq n}}$ and $D^{\geq p+n}={^pD^{\geq n}}$. Finally, for $p\leq q$, we have $D^{\leq p}\sub D^{\leq q}$ and $D^{\geq p}\sups D^{\geq q}$, which generalizes (\ref{triangle cat perverse condition inclusion-1}) and (\ref{triangle cat perverse condition inclusion-2}). Similarly, we write $\tau^{\leq p}$ and $\tau^{\geq p}$ for ${^p\tau^{\leq 0}}$ and ${^p\tau^{\geq 0}}$, and $H^p$ for $H^0$ in the sense of the $t$-structure of perversity $p$.\par
In the situation of \cref{triangle cat perverse t-structure t-exactness}, we denote simply by $j_!$, $j^!$, $j_*$ and $j^*$ the functors on derived categories. The functors deduced from them by passing to $p$-perverse sheaves will be denoted with $p$ as a left exponent. For example, for $\mathscr{A}\in\Perv(U,\mathscr{O}_U,p)$, we put ${^pj_!}(\mathscr{A})=\tau^{\geq p}j_!(\mathscr{A})=H^p(j_!(\mathscr{A}))$. By \cref{triangle cat t-exact functor prop}, $({^pj_!},{^pj^!})$ and $({^pj^*},{^pj_*})$ are adjoint pairs of functors. The induced functors on the category of usual sheaves of modules will be denoted with $0$ as a left exponent.\par
For $\mathscr{A}$ a $p$-perverse sheaf over $U$, $j_!(\mathscr{A})$ is in $D^{\leq p}(X,\mathscr{O}_X)$ and $j_*(\mathscr{A})$ is in $D^{\geq p}(X,\mathscr{O}_X)$. The natural morphism $\alpha:j_!(\mathscr{A})\to j_*(\mathscr{A})$ admits a factorization
\[j_!(\mathscr{A})\to{^pj_!}(\mathscr{A})\stackrel{\beta}{\to}{^pj_*}(\mathscr{A})\to j_*(\mathscr{A})\]
where $\beta={^pH^0}(\alpha)$. We then define the functor ${^pj_{!*}}$, or simply $j_{!*}$, by
\[j_{!*}(\mathscr{A})=\im({^pj_!}(\mathscr{A})\to {^pj_*}(\mathscr{A})).\]
On the other hand, for $\mathscr{B}$ a $p$-perverse sheaf over $X$, we define a canonical morphism ${^pj^!}(\mathscr{B})\to {^pj^*}(\mathscr{B})$ as the composition
\[{^pj^!}(\mathscr{B})\to j^!(\mathscr{B})\to j^*(\mathscr{B})\to {^pj^*}(\mathscr{B}).\]

If $k:U\to V$ and $j:V\to X$ are locally closed subsets which are unions of strata, the transititity formule for the functors $k_!,k^!,k_*,k^*$ and $j_!,j^!,j_*,j^*$ imply the similar fomule for the functors on $p$-perverse sheaves (\cref{triangle cat t-exact functor prop}~(d)). Applying ${^pj_!}$, ${^pj_*}$ and ${^pj_*}$ to the morphisms ${^pk_!}\twoheadrightarrow {^pk_{!*}}\hookrightarrow {^pk_*}$, we obtain a chain
\[{^p(jk)_!}={^pj_!}{^pk_!}\twoheadrightarrow {^pj_!}{^pk_{!*}}\twoheadrightarrow {^pj_{!*}}{^pk_{!*}}\hookrightarrow {^pj_*}{^pk_{!*}}\hookrightarrow {^pj_*}{^pk_*}={^p(jk)_*}\]
which gives an isomorphism of functors
\begin{equation}\label{triangle cat perverse t-structure j_!* transitivity}
{^p(jk)_{!*}}={^pj_{!*}}{^pk_{!*}}.
\end{equation}

\begin{remark}
Let $U$ be an open subset of $X$ which is a union of strata, and $Z$ be its complement, endowed with the induced stratifications. The $t$-structure of \cref{triangle cat perverse t-structure on D^+} over $D^+(X,\mathscr{O}_X)$ is then induced by that of $D^+(U,\mathscr{O}_U)$ and $D^+(Z,\mathscr{O}_Z)$ by glueing, and we can apply the formalism of \ref{triangle cat t-structure recollement paragraph}. Let $i:Z\to X$ and $j:U\to X$ be the inclusion maps, the functors $(j_!,j^*,j_*)$, $(i^*,i_*,i^!)$, and $j_{!*}$, thus coincide with the situation of \ref{triangle cat t-structure recollement paragraph}.
\end{remark}

\begin{proposition}\label{triangle cat perverse t-structure j_!* prolongation char}
For $\mathscr{B}\in\Perv(U,\mathscr{O}_U,p)$, $j_{!*}(\mathscr{B})$ is the unique prolongation $\mathscr{P}$ of $\mathscr{B}$ such that, for any stratum $S\sub Z$, we have $H^i(s^*(P))=0$ for $i\geq p(S)$ and $H^i(s^!(P))=0$ for $i\leq p(S)$, where $s:S\to X$ is the inclusion map.
\end{proposition}
\begin{proof}
This follows from \cref{triangle cat recollement ^pj_! and ^pj_* is truncation}. More precisely, it follows from \cref{triangle cat recollement prolongation equivalence by j^*} that if $\mathcal{D}'$ is the  subcategory of $D(X,\mathscr{O}_X)$ formed by $\mathscr{F}$ such that $H^i(s^*(\mathscr{F}))=0$ for $i\geq p(S)$ and $H^i(s^!(\mathscr{F}))=0$ for $i\leq p(S)$ for any strata $s:S\to Z$, then $j^*$ induces an equivalence $\mathcal{D}'\to D(U,\mathscr{O}_U)$. Its restriction to $\mathcal{D}'\cap\Perv(X,\mathscr{O}_X,p)$ is then an equivalence $\mathcal{D}'\cap\Perv(X,\mathscr{O}_X,p)\to \Perv(U,\mathscr{O}_U,p)$, with quasi-inverse $j_{!*}$. We have similar characterizations for ${^pj_!}(\mathscr{B})$ and ${^pj_*}(\mathscr{B})$ (cf. \cref{triangle cat recollement ^pj_! and ^pj_* is truncation}).
\end{proof}

Suppose that the perversity $p$ satisfies the following condition: if $S\sub\widebar{T}$, then $p(S)\geq p(T)$ (in this case, we say that $p$ is \textbf{decreasing}). For each $n\in\Z$, the union $Z_n$ (resp. $U_n$) of the strata $S$ such that $p(S)\geq n$ (resp. $p(S)\leq n$) is then closed (resp. open). We denote by $j_n:U_{n-1}\to U_n$ and $i_n:Z_{n-1}\to Z_n$ the inclusion maps.

\begin{proposition}\label{triangle cat perverse t-structure j_!* for open union char}
With the hypothesis and notations above, let $\mathscr{A}$ be a $p$-perverse sheaf over $U_k$ and $a\geq k$ be an integer such that $p\leq a$. If $j:U_k\to X=U_a$ denotes the inclusion map, then
\[j_{!*}(\mathscr{A})=\tau^{\leq a-1}j_{a,*}\cdots\tau^{\leq k}j_{k+1,*}(\mathscr{A}),\]
where the $\tau^{\leq i}$ are relative to the natural $t$-structure.
\end{proposition}
\begin{proof}
Applying (\ref{triangle cat perverse t-structure j_!* transitivity}), and noting that $j=j_a\cdots j_{k+1}$, it suffices to prove that $j_{k+1,!*}(\mathscr{A})=\tau^{\leq k}j_{k+1,*}(\mathscr{A})$ for each $k\in\Z$. If $Z=U_{k+1}-U_k$, then by \cref{triangle cat recollement ^pj_! and ^pj_* is truncation} we have $j_{k+1,!*}(\mathscr{A})={^p\tau^{\leq -1}_Z}j_{k+1,*}(\mathscr{A})$. On the other hand, over $Z$, the function $p$ is constant with value $k+1$, so ${^p\tau^{\leq-1}_Z}$ is none other than the functor $\tau^{\leq k}_Z$ (for the natural $t$-structure over $Z$). Since over $U_k$ we have $p\leq k$, $\mathscr{A}$ belongs to $D^{\leq k}(U,\mathscr{O}_U)$ (cf. (\ref{triangle cat perverse condition inclusion-1})) and we conclude from (\ref{triangle cat recollement truncation functor decomposition}) that $\tau^{\leq k}j_{k+1,*}(\mathscr{A})\cong\tau^{\leq k}_Zj_{k+1,*}(\mathscr{A})$.
\end{proof}

\begin{remark}
Let's explain the relationship between our notations and that of Mr. Goresky and R. MacPherson \cite{*}.
\begin{itemize}
    \item They work with cohomology with coefficients in a field $R$ (especially $R=\R$ or $\C$), i.e. they take for $\mathscr{O}_X$ the constant sheaf with value $R$.
    \item Their strata are topological manifolds, everywhere of the same dimension, and the stratifications satisfies a triviality conditions which ensures that for $j:S\to X$ a stratum and $\mathscr{F}$ a locally constant sheaf of $R$-vector spaces of finite dimension over $S$, the restrictions at each stratum of the $R^ij_*(\mathscr{F})$ are again locally constants of finite dimension. For $S$, $T$ two stratum with $S\sub\widebar{T}-T$, we have $\dim(S)<\dim(T)$.
    \item The space $X$ has an open dense stratum $U_0$ whose complement satisfies $\dim(U_0)-\dim(X-U_0)\geq 2$.
    \item The perversity function only depends on the dimension: we have a function $\bar{p}:\N\to\Z$ such that $p(S)=\bar{p}(\dim(S))$. The function $\bar{p}$ is assumed to be decreasing and positive, with $p(U_0)=\bar{p}(\dim(U_0))=0$.
\end{itemize}
Let $j:U_0\to X$ be the inclusion map, we are concerned with the object $j_{!*}(R)$ of $D^b(X,R)$. If we are given the description \cref{triangle cat perverse t-structure j_!* for open union char}, we are led to utilize the function $p'$ which satisfies $p'(U_0)=0$ and attach for each stratum $S\neq U_0$ with the integer $p(S)-1$. The operator $j_{!*}$ consists, starting from the constant sheaf $R$ over $U=U_0$, to successively take a direct image into the strata $S$ of dimension $\dim(U_0)-2$, $\dim(U_0)-3$, $\cdots$, truncated by $\tau^{\leq p'(S)}$. It is the functor $p'$ that they call perversty. From the point of view adopted here, this amounts to describing ${^pj_{!*}}$ as being ${^{p'}j_*}$, and it leads to shifts in the description of the phenomena of duality.\par
Goresky and MacPherson also assume that the function $p$ does not decrease too fast: $\bar{p}(n)-\bar{p}(n+1)=0$ or $1$. \cref{triangle cat constructible perverse structure refinement} below shows that this provides the independence of the perverse $t$-structure with the stratification.
\end{remark}

We now adapt the following assumption on the stratified space $(X,\mathscr{O}_X)$:
\begin{enumerate}[leftmargin=40pt]
    \item[(P1)] $\mathscr{O}_X$ is the constant sheaf with value $R$, for $R$ a left Noetherian ring.
    \item[(P2)] The strata are topological manifolds with equal dimension. If two strata $S,T$ satisfies $S\sub\widebar{T}$, then $\dim(S)<\dim(T)$.
    \item[(P3)] For $j:S\to X$ a stratum, the functor ${^0j_*}$ is of finite cohomological dimension over the category of $R$-modules. For any locally constant sheaf $\mathscr{F}$ of $R$-modules of finite type over $S$, the higher direct images $R^ij_*(\mathscr{F})$ are locally free of finite type over any stratum. 
\end{enumerate}

For $U$ a locally closed subset which is a union of strata, we denote by $D_c(U,R)$ (or $D_\mathcal{S}(U,R)$ if we want to stress the stratification $\mathcal{S}$) the triangulated full subcategory of $D(U,R)$ formed by constructible complexes $\mathscr{F}$\footnote{That is, for each $i\in\Z$, the cohomology sheaf $H^i(\mathscr{F})$ is constructible over $X$.} such that $H^i(\mathscr{F})$ is locally free and of finite type over each stratum of $\mathcal{S}$. We define similarly subcategories $D^*_c(U,R)$ for $*\in\{+,-,b\}$.

\begin{proposition}\label{triangle cat constructible preserve by ! and *}
For $j:U\to V$ with $U$ and $V$ being locally closed and are unions of strata, the functors $j_!$, $j^!$, $j_*$ and $j^*$ respects these subcategories.
\end{proposition}
\begin{proof}
In fact, the case for $j_!$ and $j^*$ are trivial, and for the case where $U$ is reduced to a single stratum, the assertion for $j_*$ follows from the spectral sequence $R^pj_*H^q(\mathscr{F})\to H^{p+q}(Rj_*(\mathscr{F}))$ and condition (P3). The general case can then be proved by induction on the number of strara in $U$: If $k:S\to U$ is the inclusion of an open stratum in $U$, and $\mathscr{G}$ is defined by the distinguished triangle $(\mathscr{F},k_*k^*(\mathscr{F}),\mathscr{G})$, then $\mathscr{G}$ is constructible because $k_*k^*(\mathscr{F})$ is, and is supported in $U'=U-S$. The induction hypothesis can be applied to the inclusion $U'\to X$, and $j_*(\mathscr{G})$ is then constructible; we deduce from the distinguished triangle $(j_*(\mathscr{F}),(jk)_*k^*(\mathscr{F}),j_*(\mathscr{G}))$ that $j_*(\mathscr{F})$ is constructible. Finally, the case for $j^!$ is reduced to the case where $j$ is a closed embedding, and this can deduced from the distinguished triangle $(j_!j^!(\mathscr{F}),\mathscr{F},k_*k^*(\mathscr{F}))$ and the assertion for $k_*$, where $k$ the embedding of the open complement.
\end{proof}

For $Z\sub U$ a closed subset which is a union of strata, by \cref{triangle cat recollement truncation tau_Z for topo space} the functor $\tau^{\leq a}_Z$ trivially respects $D_c(U,R)$. The proof of \cref{triangle cat recollement theorem} then shows that for any perversity $p$, $\tau^{\leq p}$ and $\tau^{\geq p}$ respects $D_c(X,R)$. The \textbf{$(p,\mathcal{S})$-perverse sheaves} over $X$ are then defined to be the $p$-perverse sheaves in $D_c(X,R)$.\par

\begin{proposition}\label{triangle cat constructible perverse structure refinement}
Let $\mathcal{T}$ be a stratification of $X$ which refines $\mathcal{S}$ and satisfies conditions (P2), (P3). Let $p$ be a perversity over $S$ and $q$ a perversity over $\mathcal{T}$. Suppose that for any stratum $S$ of $\mathcal{S}$ containing a stratum $T$ of $\mathcal{T}$, we have
\[p(S)\leq q(T)\leq p(S)+\dim(S)-\dim(T).\]
Then the $t$-structure of perversity $q$ over $D_\mathcal{T}(X,R)$ induces the $t$-structure of perversity $p$ over $D_\mathcal{S}(X,R)$.
\end{proposition}
\begin{proof}
It suffices to verify that $D_\mathcal{S}^{\leq p}(X,R)\sub D_\mathcal{T}^{\leq q}(X,R)$ and $D_\mathcal{S}^{\geq p}(X,R)\sub D_\mathcal{T}^{\geq q}(X,R)$. In fact, the inclusion $D_\mathcal{S}^{\leq p}(X,R)\sub D_\mathcal{T}^{\leq q}(X,R)$ implies $D_\mathcal{T}^{\geq q}(X,R)\cap D_\mathcal{S}(X,R)\sub D_\mathcal{S}^{\geq p}(X,R)$ since $\mathcal{F}\in D_\mathcal{S}^{\geq p}(X,R)$ if and only if $\Hom(\mathcal{F},\mathcal{G})=0$ for any $\mathcal{G}\in D_\mathcal{S}^{<p}(X,R)$, and similarly for $D_\mathcal{S}^{\leq p}(X,R)$.\par
The first inclusion follows immediately from that fact that $p(S)\leq q(T)$ for $S\sups T$. To see the second one, let $\mathscr{F}\in D_\mathcal{S}^{\geq p}(X,R)$, $T\in\mathcal{T}$, and $S\in\mathcal{S}$ containing $T$. Let $i:T\to S$ and $j:S\to X$ be the inclusion maps. We have $(ji)^!(\mathscr{F})=i^!j^!(\mathscr{F})$, and $H^n(j^!(\mathscr{F}))$ is locally constant over $S$, zero for $n<p(S)$. Let $\omega_T$ (resp. $\omega_S$) be the orientation sheaf over $T$ (resp. $S$), which is locally isomorphic to the constant sheaf $\Z$. Let $\omega=\sHom(\omega_T,i^*\omega_S)$ be the sheaf of normal orientations of $T$ in $S$, and put $d=\dim(S)-\dim(T)$. If $\mathscr{G}$ is a complex of sheaves over $S$, whose cohomology sheaves are locally constant, we then have $i^!(\mathscr{G})\cong i^*(\mathscr{G})\otimes_{\Z}\omega[-d]$. Put $\mathscr{G}=j^!(\mathscr{F})$, we see that $H^n(i^!j^!(\mathscr{F}))=0$ for $n<p(S)+d$, and the assertion follows since $q(T)\leq p(S)+d$ by hypothesis.
\end{proof}

\begin{corollary}\label{triangle cat constructible perverse refinement functor prop}
Any $(p,\mathcal{S})$-perverse sheaf is $(q,\mathcal{T})$-perverse, and the inclusion functor $D_\mathcal{S}(X,R)\to D_\mathcal{T}(X,R)$ is $t$-exact. For any locally closed subset $U$ which is a union of strata in $\mathcal{S}$, and $j:U\to X$ the inclusion map, the functors ${^qj_!}$, ${^qj^!}$, ${^qj_*}$, ${^qj^*}$, ${^qj_{!*}}$, restricted to the category of $(p,\mathcal{S})$-sheaves, coincide with the functors ${^pj_!}$, ${^pj^!}$, ${^pj_*}$, ${^pj^*}$, ${^pj_{!*}}$.
\end{corollary}

In addition to the situation of \cref{triangle cat constructible perverse structure refinement}, suppose that $R$ is a field and that $X$ admits a (locally finite) triangulation such that each stratum $S$ of $\mathcal{S}$ is a union of (open) simplices (for example, $X$ is an algebraic variety endowed with a Whitney stratification). We can then apply Verdier's duality, which is an involutive automorphism of $D_c(X,R)$, and for $j:U\to X$ locally closed which is a union of strata, this automorphism exchanges $j_!$ and $j_*$, as well as $j^!$ and $j^*$.\par
For any stratum $S$, with sheaf of orientation $\omega$, and of fimension $d$, the Verdier duality $D$ over $S$ ($\mathscr{F}\mapsto R\sHom(\mathscr{F},R\otimes\omega[d])$) satisfies for $\mathscr{F}\in D_\mathcal{S}(S,R)$ the equality
\[H^i(D(\mathscr{F}))=(H^{-d-i})^{\vee}\otimes\omega.\]
It is essential here that the cohomology sheaves of $\mathscr{F}$ are locally constant of finite rank, and that $R$ is a field (a local Artinian ring, for example $\Z/\ell^n\Z$, will sill do the trick, because we have an injective dualizing module $I$ ($\Z/\ell^n\Z$ in this case); the Verdier duality is then given by $\mathscr{F}\mapsto R\sHom(\mathscr{F},Rf^!(I))$, where $f$ is the projection to $\pt$.)\par
We define the \textbf{dual perversity} $p^*$ of a perversity $p$ by
\[p^*(S)=-p(S)-\dim(S).\]
The preceding arguments imply that $D$ exchanges $D^{\geq p}$ and $D^{\leq p^*}$ (and thus $D^{\leq p}$ with $D^{\geq p^*}$, since we have $p=p^{**}$). It in particular exchanges $p$-perverse sheaves and $p^*$-perverse sheaves, and ${^{p^*}H}^i$ and ${^pH^{-i}}$. For $j:U\to X$ the inclusion of a locally closed subset which is a union of strata, the functor $D$ also exchanges ${^pj_!}$ and ${^{p^*}j_*}$, ${^pj^!}$ and ${^{p^*}j^*}$, and ${^pj_{!*}}$ and ${^{p^*}j_{!*}}$.\par
In $D_c(X,R)$, the condition defining the $p$-perversity can then be rewritten as follows: for any stratum $j:S\to X$, we have $H^i(j^*(\mathscr{F}))=0$ for $i>p(S)$ and $H^i(j^*(D(\mathscr{F})))=0$ for $i>p^*(S)$. If every stratum of $\mathcal{S}$ is of even dimension, then there exists a \textbf{self-dual perversity} $p_{1/2}$, given by
\[p_{1/2}(S)=-\frac{1}{2}\dim(S).\]
A $p_{1/2}$-perverse sheaf over $X$ is also called a \textbf{self-dual perverse sheaf}.
\begin{proposition}\label{triangle cat constructible self-dual perverse prolongation char}
Under the hypothesis above, assume that every stratum of $\mathcal{S}$ is of even dimension and consider the self-dual perversity $p_{1/2}$. If $j:U\to X$ is an open subset of $X$ which is a union of strata and $\mathscr{A}$ is a $p_{1/2}$-perverse sheaf over $U$, then $j_{!*}(\mathscr{A})$ is the unique self-dual prolongation $\mathscr{P}$ of $\mathscr{A}$ (in $D_c(X,R)$) such that for any stratum $S\sub X-U$, the restriction of $H^i(\mathscr{P})$ to $S$ is zero for $i\geq-\frac{1}{2}\dim(S)$.  
\end{proposition}
\begin{proof}
That $j_{!*}(\mathscr{A})$ is self-dual follows from the self-duality of $\mathscr{A}$ and that of $j_{!*}$. This being the case, the proposition then follows from \cref{triangle cat perverse t-structure j_!* prolongation char}.
\end{proof}

\begin{remark}
If $U$ is orientable and smooth with pure dimension $d$, we can choose $\mathscr{A}$ to be the constant sheaf $R$, placed at degree $-d/2$. For this choice, the self-dual perverse sheaf $j_{!*}(\mathscr{A})$ is the "intersection complex" of $X$, denoted by $\mathrm{IC}_X$.
\end{remark}

Let $(X,\mathcal{S},\mathscr{O}_X)$ be as in the begining of this subsection, $\mathscr{O}_X^{\op}$ be the sheaf of opposite rings of $\mathscr{O}_X$, and $p,q$ be two perversities.
\begin{proposition}\label{triangle cat perverse structure and tensor Hom}
The functor $\otimes^L$ sends $D^{\leq p}\times D^{\leq q}$ into $D^{\leq p+q}$, and $R\sHom$ sends $D^{\leq p}\times D^{\geq q}$ into $D^{\geq (q-p)}$.
\end{proposition}
\begin{proof}
The first assertion is clear by definition, since $\otimes^L$ is compatible with restrictions. Let $\mathscr{F}\in D^{\leq p}$ and $\mathscr{G}\in D^{\geq q}$. A d\'evissage argument shows that we can suppose that $\mathscr{F}$ is of the form $j_!(\mathscr{A})[-n]$, where $\mathscr{A}$ is a sheaf over a stratum $S$, $j:S\to X$ is the inclusion, and $n\leq p(S)$. By adjunction, we then have
\[R\sHom(j_!(\mathscr{A})[-n],\mathscr{G})=j_*R\sHom(\mathscr{A}[-n],j^!(\mathscr{G})).\]
By hypothesis, we have $j^!(\mathscr{G})\in D^{\geq q(S)}(S,\mathscr{O}_S)$, so $R\sHom(\mathscr{A}[-n],j^!(\mathscr{G}))$ is in $D^{\geq q(S)-n}(S,\Z)\sub D^{\geq q(S)-p(S)}(S,\Z)$. The left $t$-exactness of $j_*$ then implies that $R\sHom(j_!(\mathscr{A})[-n],\mathscr{G})$ is in $D^{\geq q-p}$.
\end{proof}

\begin{corollary}\label{triangle cat perverse structure Hom of D^geq and D^leq}
For $\mathscr{F}\in D^{\leq p}(X,\mathscr{O}_X)$ and $\mathscr{G}\in D^{\geq p}(X,\mathscr{O}_X)$, we have $H^iR\sHom(\mathscr{F},\mathscr{G})=0$ for $i<0$.
\end{corollary}
\begin{proof}
This follows from \cref{triangle cat perverse structure and tensor Hom} by setting $p=q$. Alternatively, it can be deduced by localizing that $\Hom(\mathscr{F},\mathscr{G}[i])=0$ for $i<0$ (\cref{triangle cat perverse t-structure on D(X)}).
\end{proof}

\begin{corollary}\label{triangle cat perverse structure Hom define sheaf}
For $\mathscr{F}\in D^{\leq p}(X,\mathscr{O}_X)$ and $\mathscr{G}\in D^{\geq p}(X,\mathscr{O}_X)$, the presheaf
\[U\mapsto\Hom_{D(U,\mathscr{O}_U)}(\mathscr{F}|_U,\mathscr{G}|_U)\]
is a sheaf.
\end{corollary}
\begin{proof}
Since $H^iR\sHom(\mathscr{F},\mathscr{G})=0$ for $i<0$, the Grothendieck spectral sequence of $R\sHom$ and $\Gamma(U,-)$ implies that
\[\Hom_{D(U,\mathscr{O}_U)}(\mathscr{F}|_U,\mathscr{G}_U)=H^0(U,H^0R\sHom(\mathscr{F},\mathscr{G})),\]
and this proves our claim.
\end{proof}

Endow each open subset $U$ of $X$ the stratification induced by $\mathcal{S}$, and the perversity induced by $p$. The restriction of a $p$-perverse sheaf over $X$ to $U$ is then a $p$-perverse sheaf over $U$. We thus get a functor $U\mapsto\Perv(U,\mathscr{O}_U,p)$ on the category of open subset of $X$.
\begin{corollary}\label{triangle cat perverse sheaf is stack}
The functor $U\mapsto\Perv(U,\mathscr{O}_U,p)$ is a stack.
\end{corollary}
\begin{proof}
By \cref{triangle cat perverse structure Hom define sheaf}, the functor $U\mapsto\Perv(U,\mathscr{O}_U,p)$ is a prestack over $X$. The glueing property for objects follows from \cref{*} and \cref{triangle cat perverse structure Hom of D^geq and D^leq}.
\end{proof}

\subsection{Schemes}\label{triangle cat perverse sheaf scheme subsection}
\paragraph{The case of a complex algebraic variety}\label{triangle cat perverse sheaf complex variety paragraph}
As a model, we first consider the case of a complex algebraic variety. We want to apply the constructions of \autoref{triangle cat perverse sheaf stratified space subsection} to the space $X(\C)$ of rational points of $X$, endowed with the usual analytic topology.
\begin{itemize}
    \item We consider the perversity which $p(S)$ only depends on the dimension of $S$. For this, we need a function $p:\N\to\Z$. Let $p^*(n):-n-p(n)$, called the dual perversity of $p$. We assume that $p$ and $p^*$ are both decreasing: for $n\leq m$, we have
    \[0\leq p(n)-p(m)\leq m-n.\]
    \item Recall that any (algebraic) stratification of $X$ admits a Whitney refinement. We choose a Whitney stratification of equidimensional subvarieties of $X$, which satisfies conditions (P1), (P2) and (P3) of \autoref{triangle cat perverse sheaf stratified space subsection} (without further specifications, we therefore only consider Whitney stratifications of $X$). For any stratification $\mathcal{S}$, the perversity $p_\mathcal{S}$ of $\mathcal{S}$ is defined by
    \[p_\mathcal{S}(S)=p(2\dim_{\mathrm{alg}}(S))=p(\dim_{\mathrm{top}}(S)).\]
    We write $\dim(S)$ for the algebraic dimension $\dim_{\mathrm{alg}}(S)$ of $S$.
\end{itemize}

Let $R$ be a left Noetherian ring. The derived category $D_c(X(\C),R)$ is the full subcategory of $D(X(\C),R)$ formed by complexes $\mathcal{F}$ such that the cohomology sheaves $H^i(\mathscr{F})$ are constructible. The category $D^b_c(X(\C),R)=D^b(X(\C),R)\cap D_c(X(\C),R)$ is the (filtered) union of subcategories $D^b_\mathcal{S}(X(\C),R)$ of $D(X(\C),R)$ (where $\mathcal{S}$ runs over Whitney stratifications of $X$), over which the perversity $p_\mathcal{S}$ defines a $t$-structure. By \cref{triangle cat constructible perverse structure refinement} and the hypothesis on $p$, for $\mathcal{S}$ finer and finer, these $t$-structures are compatible with refinements, so by passing to limit we obtain a $t$-structure $(D^{b,\leq p}_c(X(\C),R),D^{b,\geq p}_c(X(\C),R))$ over $D^b_c(X(\C),R)$, called the \textbf{$t$-structure of perversity $p$}. Mimiking the proof of \cref{triangle cat perverse t-structure on D(X)}, we obtain a $t$-structure over $D_c(X(\C),R)$: let $a\leq p(2i)\leq b$ be integers where $0\leq i\leq\dim(X)$; we define $D_c^{\leq p}(X(\C),R)$ (resp. $D_c^{\geq p}(X(\C),R)$) as the subcategory of $D_c(X(\C),R)$ formed by complexes $\mathscr{F}$ such that $H^i(\mathscr{F})=0$ for $i>b$ (resp. $i<a$) and that $\tau^{[a,b]}\mathscr{F}$ is in $\mathcal{D}_c^{b,\leq p}(X(\C),R)$ (resp. $D_c^{b,\geq p}(X(\C),R)$).

\begin{proposition}\label{triangle cat perverse sheaf over scheme D^p char}
For $\mathscr{F}\in D_c(X(\C),R)$, the following conditions are equivalent:
\begin{enumerate}
    \item[(\rmnum{1})] $\mathscr{F}$ is in $D_c^{\leq p}(X(\C),R)$ (resp. $D_c^{\geq p}(X(\C),R)$).
    \item[(\rmnum{2})] Any irreducible subvariety $S'$ of $X$ has an open dense Zariski subset $S$ such that, if $i:S(\C)\to X(\C)$ is the inclusion morphism, we have $H^i(i^*(\mathscr{F}))=0$ for $i>p(S)$ (resp. $H^i(i^!(\mathscr{F}))=0$ for $i<p(S)$).
\end{enumerate}
If $\mathcal{T}$ is a finite family of smooth equidimensional locally closed subset (for the Zariski topology), which union $X$, such that for each $S\in\mathcal{T}$, the $H^i(i_S^*(\mathscr{F}))$ and $H^i(i_S^!(\mathscr{F}))$ are locally constant, then the above conditions are equivalent to
\begin{enumerate}
    \item[(\rmnum{3})] For any $S\in\mathcal{T}$, $H^i(i_S^*(\mathscr{F}))=0$ for $i>p(S)$ (resp. $H^i(i_S^!(\mathscr{F}))=0$ for $i<p(S)$).
\end{enumerate}
\end{proposition}
\begin{proof}
We first suppose that $\mathscr{F}$ is bounded. For any stratification $\mathcal{S}$ fine enough, we then have $\mathscr{F}\in D^b_\mathcal{S}$. In particular, there exists a stratification $\mathcal{T}$ with the required properties, and it suffices to show that (\rmnum{1})$\Rightarrow$(\rmnum{2})$\Rightarrow$(\rmnum{3})$\Rightarrow$(\rmnum{1}). The implication (\rmnum{2})$\Rightarrow$(\rmnum{3}) is trivial. To verify (\rmnum{1})$\Rightarrow$(\rmnum{2}), it suffices to choose $\mathcal{S}$ fine enough so that $\mathscr{F}$ is in $D^b_\mathcal{S}$ (hence in $D_\mathcal{S}^{b,\leq p}$ (resp. $D_\mathcal{S}^{b,\geq p}$)) and that $S'$ is a closure of strata. Finally, to see that (\rmnum{3})$\Rightarrow$(\rmnum{1}), let $\mathcal{S}$ be a refinement of $\mathcal{T}$ so that $\mathscr{F}$ is in $D_\mathcal{S}^b$. The hypothesis over $p$ and the proof of \cref{triangle cat constructible perverse structure refinement} show that $\mathscr{F}$ is in $D_\mathcal{S}^{b,\leq p}$ (resp. $D_\mathcal{S}^{b,\geq p}$), whence (\rmnum{1}).\par
In the general case, let $a,b$ be integers such that $a\leq p(2i)\leq b$ for $0\leq i\leq\dim(X)$. If in (\rmnum{1}), (\rmnum{2}) or (\rmnum{3}) we replace $\mathscr{F}$ by $\tau^{\geq a}\mathscr{F}$ (resp. $\tau^{\leq b}\mathscr{F}$), we obtain an equivalent condition and it remains to show that each of them implies $H^i(\mathscr{F})=0$ for $i>b$ (resp. $i<a$). For (\rmnum{1}), this is in the definition. For (\rmnum{3}), we start by replacing $\mathcal{T}$ by a finer stratification and recall the proof of \cref{triangle cat perverse condition finite cohomology dim}. For (\rmnum{2}), we prove by descendent induction on $\dim(S')$ that for each irreducible subvariety $S'$ and each $i>b$ (resp. $i<a$), there exists an open dense subset $S$ of $S'$ such that $H^i(\mathscr{F})$ is zero over $S$.
\end{proof}

If $U$ be a locally closed subset of $X$, for any stratification fine enough, $U$ is a union of strata. This allows us to apply \cref{triangle cat perverse t-structure t-exactness} to the inclusion map $j:U\to X$. We use the notations $\tau^{\leq p}$, $H^p$, ${^pj_!}$ and so on in \autoref{triangle cat perverse sheaf stratified space subsection}, and the corresponding results. For $U$ an open subset of $X$, with complement $Z$, the $t$-structure of perversity $p$ of $D_c(X(\C),R)$ is obtained by glueing that of $D_c(U(\C),R)$ and $D_c(Z(\C),R)$. We can therefore apply \cref{triangle cat recollement theorem}, as in \cref{triangle cat perverse t-structure j_!* prolongation char}. In particular, for a perverse sheaf $\mathscr{A}$ over $U$, we have (\cref{triangle cat recollement ^pj_! and ^pj_* is truncation})
\begin{equation}\label{triangle cat perverse sheaf over scheme j_!*}
j_{!*}(\mathscr{A})={^p\tau_Z^{\leq -1}}j_*(\mathscr{A})
\end{equation}
Suppose that $Z$ is of dimension $\leq d$ and put $t=p(2d)$. Since $\tau^{<t}i^*j_*(\mathscr{A})$ is in ${^pD_c^{<0}}$ and that the triangle $(\tau^{<t}i^*j_*(\mathscr{A}),i^*j_*(\mathscr{A}),\tau^{\geq t}i^*j_*(\mathscr{A}))$ is distinguished, if $\tau^{\geq t}i^*j_*(\mathscr{A})$ is in ${^pD_c^{\geq 0}}$, we have ${^p\tau^{<0}}i^*j_*(\mathscr{A})\cong\tau^{<t}i^*j_*(\mathscr{A})$ and hence ${^p\tau^{<0}}j_*(\mathscr{A})\cong\tau_Z^{<t}j_*(\mathscr{A})$\footnotetext{Here $\tau_Z^{\leq n}$ is the truncation functor induced by the natural $t$-structures, and ${^p\tau_Z^{\leq n}}$ denotes that induced by the $p$-perverse $t$-structure.}. In particular, we obtain the following:

\begin{proposition}\label{triangle cat perverse sheaf over scheme j_!* of j_*}
If $Z$ is smooth of dimension $d$ and that $H^i(j_*(\mathscr{A}))$ are locally constant over $Z$ for $i\geq t:=p(2d)$, we have
\begin{equation}\label{triangle cat perverse sheaf over scheme j_!* of j_*-1}
j_{!*}(\mathscr{A})=\tau_Z^{\leq t-1}j_*(\mathscr{A}).
\end{equation}
\end{proposition}
\begin{proof}
By \cref{triangle cat perverse sheaf over scheme D^p char}~(\rmnum{3}), applie to $Z$ and $\mathcal{T}=\{Z\}$, we conclude that $\tau^{\geq t}i^*j_*(\mathscr{A})$ is in ${^pD_c^{\geq 0}}$, so the claim follows.
\end{proof}

This, together with the transitivity of $j_{!*}$, is analogous to \cref{triangle cat perverse t-structure j_!* for open union char}. Similarly, under the same hypothesis over $Z$ and if the $H^i(i^*j_*(\mathscr{A}))$ are locally free for $i\geq t-1$ (resp. $t+1$), we have respectively, by (\cref{triangle cat recollement ^pj_! and ^pj_* is truncation}),
\begin{align}
{^pj_!}(\mathscr{A})&=\tau^{t-2}_Zj_*(\mathscr{A}),\label{triangle cat perverse sheaf over scheme ^pj_! char}\\
{^pj_*}(\mathscr{A})&=\tau^{\leq t}_Z.\label{triangle cat perverse sheaf over scheme ^pj_* char}
\end{align}

\begin{proposition}\label{triangle cat perverse sheaf over scheme quasi-finite morphism t-exact}
If $f:X\to Y$ is a quasi-finite morphism, the functors $f_!$ and $f^*$ are right $t$-exact, and $f^!$ and $f_*$ are left $t$-exact.
\end{proposition}
\begin{proof}
For $\mathscr{F}\in D_c(X(\C),R)$, the condition $\mathscr{F}\in {^pD_c^{\leq 0}}$ is equivalent to
\begin{equation}\label{triangle cat perverse sheaf over scheme quasi-finite morphism t-exact-1}
p(2\dim(\supp(H^i(\mathscr{F}))))\geq i.
\end{equation}
To see this, we choose a stratification $\mathcal{S}$ of $X$ so that $\supp(H^i(\mathscr{F}))$ is a union of strata of $\mathcal{S}$. If (\ref{triangle cat perverse sheaf over scheme quasi-finite morphism t-exact-1}) is satisfied, then for any stratum $S\in\mathcal{S}$ contained in $Y$, where $Y=\supp(H^i(\mathscr{F}))$, we have $p(S)\geq p(2\dim(Y))\geq i$, so $H^i(i_S^*(\mathscr{F}))=0$ if $i>p(S)$. Conversely, if $\mathscr{F}\in {^pD_c^{\leq 0}}$, then $p(S)\geq i$ for each stratum $S\in\mathcal{S}$ contained in $Y$, whence $p(2\dim(Y))\geq i$.\par
Now the functor ${^0f_!}$ being exact, we have $H^i(f_!(\mathscr{F}))={^0f_!}(H^i(\mathscr{F}))$, whence $\supp(f_!(\mathscr{F}))=\overline{f(\supp(H^i(\mathscr{F})))}$ and
\[\dim(\supp(H^i(f_!(\mathscr{F}))))=\dim(\supp(H^i(\mathscr{F}))).\] Similarly, $H^i(f^*(\mathscr{F}))={^0f^*}(H^i(\mathscr{F}))$, $\supp(H^i(f^*(\mathscr{F})))=f^{-1}(\supp(H^i(\mathscr{F})))$ and
\[\dim(\supp(H^i(f^*(\mathscr{F}))))\leq\dim(\supp(H^i(\mathscr{F}))).\] This proves the right $t$-exactness of $f_!$ and $f^*$, and the assertion for $f^!$ and $f_*$ follows from \cref{triangle cat t-exact functor prop}~(c).
\end{proof}

\begin{corollary}\label{triangle cat perverse sheaf over scheme finite etale morphism t-exact}
If $f$ is finite, $f_!=f_*$ is $t$-exact. If $f$ is \'etale, $f^!=f^*$ is $t$-exact.
\end{corollary}

For a quasi-finite morphism $f:X\to Y$ and a $p$-perverse sheaf $\mathscr{F}$ over $X$, by \cref{triangle cat perverse sheaf over scheme quasi-finite morphism t-exact}, $f_!(\mathscr{F})$ is in ${^pD^{\leq 0}}$ and $f_*(\mathscr{F})$ is in ${^pD^{\geq 0}}$. The natural morphism $f_!(\mathscr{F})\to f_*(\mathscr{F})$ then admits a factorization
\begin{equation}\label{triangle cat perverse sheaf over scheme f_! to f_*-1}
f_!(\mathscr{F})\to {^pf_!}(\mathscr{F})\to {^pf_*}(\mathscr{F})\to f_*(\mathscr{F}).
\end{equation}
We denote by $f_{!*}(\mathscr{F})$ the image of ${^pf_!}(\mathscr{F})$ in ${^pf_*}(\mathscr{F})$.\par

As in \cref{triangle cat perverse sheaf stratified space subsection}, if the perversities $p,q$ satisfy 
\[(p+q)(n)-(p+q)(m)\leq m-n\]
for $m\leq n$, $\otimes^L$ sends $D_c^{-,\leq p}\times D_c^{-,\leq q}$ into $D^{\leq p+q}_c$, and if $q-p$ is decreasing, $R\sHom$ sends $D_c^{-,\leq p}\times D_c^{+,\geq q}$ into $D_c^{+,\geq(q-p)}$. In particular, for $\mathscr{F}\in D_c^{\leq p}$ and $\mathscr{G}\in D_c^{\geq p}$, the chomology sheaves $H^iR\sHom(\mathscr{F},\mathscr{G})$ are zero for $i<0$, and $U\mapsto\Hom_{D(U)}(\mathscr{F}|_U,\mathscr{G}|_U)$ is a sheaf. As in \cref{triangle cat perverse sheaf is stack}, the $p$-perverse sheaves form a stack.\par
If $R$ is a field, the Verdier duality exchanges $D^{\geq p}$ and $D^{\leq p^*}$, $D^{\geq p}$ and $D^{\leq p^*}$, $p$-perverse sheaves and $p^*$-perverse sheaves, and ${^pH^j}$ and ${^{p^*}H^{-j}}$.

\paragraph{The case of an algberaic variety over a field \texorpdfstring{k}{k}}
Let $X$ be a scheme of finite type over a field $k$, and let $\ell$ be a prime number, which is coprime to the characteristic of $k$. Over $X$, we consider the sheaves for the \'etale topology, but locally closed subsets refer to Zariski topology. Our goal is to define $\Q_\ell$-perverse sheaves over $X$. The case of the sheaves of $\Z/\ell\Z$-modules being a bit easier, we start with them.\par
Due to the phenomena of wild ramification, we do not have stratifications playing the role of Whitney stratifications. We will have to consider pairs $(\mathcal{S},L)$ where
\begin{enumerate}
    \item[(a)] $\mathcal{S}$ is a finite partition of $X$ by locally closed subsets (the strata). The closure of any stratum is a union of strata, and over $\widebar{k}$ each stratrum is reduced and smooth, all of the same dimension.
    \item[(b)] $L$ associates for each stratum $\mathcal{S}$ to a \textit{finite} set $L(S)$ of isomorphy classes of locally constant sheaves of $\Z/\ell\Z$-modules over $S$, which is irreducible in the category of locally constant sheaves of $\Z/\ell\Z$-modules.
\end{enumerate}

A sheaf of $\Z/\ell\Z$-modules is called \textbf{$(\mathcal{S},L)$-constructible} if its restriction to any stratum $S$ of $\mathcal{S}$ is locally constant and a (finite) iterated extension of sheaves whose isomorphism classes are in $L(S)$. We denote by $D^b_{\mathcal{S},L}(X,\Z/\ell\Z)$ the full subcategory of $D^b(X,\Z/\ell\Z)$ formed by $(\mathcal{S},L)$-constructible complexes $\mathscr{F}$, i.e. such that $H^i(\mathscr{F})$ are $(\mathcal{S},L)$-constructible. We say that $(\mathcal{S}',L')$ refines $(\mathcal{S},L)$ if each stratum $S\in\mathcal{S}$ is a union of strata of $\mathcal{S}'$, and that any $\mathscr{F}\in L(S)$ is $(\mathcal{S}',L')$-constructible (i.e. $(\mathcal{S}'|_S,L'|_{\mathcal{S}'|_S})$-constructible).\par
Finally, we impose the following conditions:
\begin{enumerate}
    \item[(c)] For $S\in\mathcal{S}$ and $\mathscr{F}\in L(S)$. if $j:S\to X$ is the inclusion, the $R^nj_*(\mathscr{F})$ are $(\mathcal{S},L)$-constructible.
\end{enumerate}

It follows from the constructability theorem for $Rj_*$ (\cite{*}, 1.1) that for any system $(\mathcal{S}',L')$ verifying (a), (b), there exists a refinement $(\mathcal{S},L)$ of $(\mathcal{S}',L')$ which satisfies (a), (b), (c): if (a), (b), (c) are true with $X$ replaced by the union of strata of dimension $\geq n$, the constructability theorem makes it possible to refine $(\mathcal{S},L)$ into $(\mathcal{S}',L')$ without touching the strata of dimension $\geq n$, and to make (a), (b), (c) true with $X$ replaced by the union of strata of dimension $\geq n-1$. We can then iterate this construction to obtain the desired refinement.\par

Let $p:\mathcal{S}\to\Z$ be as \ref{triangle cat perverse sheaf complex variety paragraph}, defined by a perversity functor $p_\mathcal{S}:\mathcal{S}\to\Z, S\mapsto p(2\dim(S))$. Preceeding as in \autoref{triangle cat perverse sheaf stratified space subsection}, we obtain a $t$-structure over $D_{\mathcal{S},L}(X,\Z/\ell\Z)$. For $(\mathcal{S},L)$ finer and finer, these $t$-structures induce each other: the purity theorem of (\cite{SGA4-3}, \Rmnum{16}, 3.7) gives an analogue of \cref{triangle cat constructible perverse structure refinement}. By passing to limit, we obtain a $t$-structure $(D_c^{b,\leq p}(X,\Z/\ell\Z),D_c^{b,\geq p}(X,\Z/\ell\Z))$ over the filterred union $D_c^b(X,\Z/\ell\Z)$ of the $D_{\mathcal{S},L}^b(X,\Z/\ell\Z)$. We then deruce a $t$-stucture over $D_c(X,\Z/\ell\Z)$ as in \cref{triangle cat perverse t-structure on D(X)}, and \cref{triangle cat perverse sheaf over scheme D^p char} to \cref{triangle cat perverse sheaf over scheme finite etale morphism t-exact} are still valid, with similar arguments (the Verdier duality is replaced by the duality formalism of etale cohomology (\cite{SGA4-3}, \Rmnum{18})).\par
For a point $x$ of $X$ (closed or not), if $i_x:\{x\}\to X$ denotes the inclusion map, which factors into $x\stackrel{j}{\to}\widebar{\{x\}}\stackrel{i}{\to}X$, we define $i_x^!:=j^*i^!$. With this notation, the characterization of \cref{triangle cat perverse sheaf over scheme D^p char}~(\rmnum{2}) of $D_c^{\leq p}$ (resp. $D_c^{\geq p}$) admits the following formulation\footnote{This formulation leads to an extension of the construction of $t$-structures associated to perversity functions, avioding constructibility and the use of stratifications, see [Gab04].}:
\begin{enumerate}
    \item[(\rmnum{2}')] For any point $x$ of $X$ (closed or not), denote by $\dim(x)$ the dimension of $\widebar{\{x\}}$, we have $H^i(i_x^*(\mathscr{F}))=0$ for $i>p(2\dim(x))$ (resp. $H^i(i_x^!(\mathscr{F}))=0$ for $i<p(2\dim(X))$). 
\end{enumerate}

For the sheaves of $\Z/\ell^n\Z$-modules (or more generally $R$-modules, with $R$ finite over $\Z/\ell^n\Z$), we have similar definitions. It is convenient to continue to take for $L$ the datum, for each stratum, of a finite family of locally constant irreducible sheaves of $\Z/\ell\Z$-modules, and to say that $\mathscr{F}\in D^b(X,R)$ is $(\mathcal{S},L)$-constructible if $\ell^iH^*(\mathscr{F})/\ell^{i+1}H^*(\mathscr{F})$ are $(\mathscr{S},L)$-constructible (as sheaves of $\Z/\ell\Z$-modules).\par

For $\Z_\ell$-sheaves (resp. $\Q_\ell$-sheaves), we preceed with the same definition, once we can define triangulated categories $D_c^b(X,\Z_\ell)$ (resp. $D^b_c(X,\Q_\ell)$) obeying the usual variance formalism, and once we define their "natural" $t$-structures, through which we can consider the abelian category of constructible $\Z_\ell$-sheaves (resp. $\Q_\ell$-sheaves) on $X$ (\cite{SGA5}, \Rmnum{7}, 1.1). In the case of finite fields or algebraically closed fields (or more generally when for any finite extension $k'$ of $k$ the groups $H^i(\Gal(\widebar{k}/k'),\Z/\ell\Z)$ are finite), a solution is proposed in (\cite{*} 1.1.2): put
\begin{equation}\label{triangle cat Z_ell etale perverse sheaf over scheme-1}
D^b_c(X,\Z_\ell):=\rlim D^b_{\mathrm{ctf}}(X,\Z/\ell^n\Z)
\end{equation}
(ctf for "constructible with finite Tor-dimension). The transition functors are chosen to be the functors of extension of scalars $-\otimes_{\Z/\ell^n\Z}^L\Z/\ell^m\Z$; the restriction of having finite Tor-dimension is necessary for that they send $D^b$ into $D^b$. The following proposition, together with the theorem of finiteness in (\cite{*}), assure that we obtain a triangulated category.

\begin{proposition}\label{triangle cat 2-proj limit triangle if finite Hom}
Let $(\mathcal{D}_n)_{n\in\N}$ be a projective system of triangulated categories whose transition functors $T_{m,n}:\mathcal{D}_n\to\mathcal{D}_m$ are assumed to be exact. If for any $n\in\N$ and $K,L\in\mathcal{D}_n$, $\Hom(K,L)$ is finite, then the $2$-projective limit $\mathcal{D}$ of $\mathcal{D}_n$, endowed with the triangles whose image in each $\mathcal{D}_n$ are distinguished, is triangulated.
\end{proposition}
\begin{proof}
Let $u:X\to Y$ in $\mathcal{D}$, and $u_n:X_n\to Y_n$ be its image in $\mathcal{D}_n$. For
\end{proof}

Definition (\ref{triangle cat Z_ell etale perverse sheaf over scheme-1}) has the advantage that the variance formalism of $D^b_c(X,\Z_\ell)$ is obtained by passing to limit. The definition of the natural $t$-structure is more delicate, because the usual truncation functors $\tau^{\leq i}$ in $D^b_c(X,\Z/\ell^n\Z)$ do not respect finite Tor-dimension and do not commute with extensions of scalars. We define the functor $H^i$ of $D^b_c(X,\Z_\ell)$ in the abelian category of constructible $\Z_\ell$-sheaves by attaching to $\mathscr{F}$, defined by a system $\mathscr{F}_n\in D^b_c(X,\Z/\ell^n\Z)$, the projective limit of the $\Z_\ell$-sheaf $H^i(\mathscr{F}_n)$. We shall write $\mathscr{F}\otimes_{\Z_\ell}\Z/\ell^n\Z$ for $\mathscr{F}_n$, and we have the usual exact sequences
\begin{equation}\label{triangle cat Z_ell etale perverse sheaf over scheme-2}
\begin{tikzcd}
0\ar[r]&H^i(\mathscr{F})\otimes_{\Z_\ell}\Z/\ell^n\Z\ar[r]&H^i(\mathscr{F}\otimes_{\Z_\ell}\Z/\ell^n\Z)\ar[r]&\Tor_1(H^{i+1}(\mathscr{F}),\Z/\ell^n\Z)\ar[r]&0
\end{tikzcd}
\end{equation}
As $n$ varies, these form a projective system, and the projective limit of $\Tor_1$ is essentially zero. We define $D^{b,\leq 0}_c(X,\Z_\ell)$ (resp. $D^{b,\geq 0}_c(X,\Z_\ell)$) by the condition $H^i(\mathscr{F})=0$ for $i>0$ (resp. $i<0$). For that we obtain a $t$-structure, it is essential to construct the operators $\tau^{\leq i}$. This is done in (\cite{*}, 1.1.2). The functor $H^0$ induces an equivalenct of categories from $D^{b,\leq 0}_c(X,\Z_\ell)\cap D^{b,\geq 0}_c(X,\Z_\ell)$ to the category of constructible $\Z_\ell$-sheaves. The exact sequence (\ref{triangle cat Z_ell etale perverse sheaf over scheme-2}) shows that for $\mathscr{F}$ to be in $D^{b,\leq 0}_c(X,\Z_\ell)$, it is necessary and sufficient that for an (any) integer $n$, $\mathscr{F}\otimes_{\Z_\ell}\Z/\ell^n\Z$ is in $D^{b,\leq 0}(X,\Z/\ell^n\Z)$.\par

Let $D^b_{\mathcal{S},L}(X,\Z_\ell)$ be the full subcategory of $D^b_c(X,\Z_\ell)$ formed by complexes $\mathscr{F}$ verifying the following conditions (the equivalence follows from (\ref{triangle cat Z_ell etale perverse sheaf over scheme-2})):
\begin{enumerate}
    \item[(a)] for an (any) integer $n$, $\mathscr{F}\otimes_{\Z_\ell}\Z/\ell^n\Z$ is in $D^b_{\mathcal{S},L}(X,\Z/\ell^n\Z)$;
    \item[(b)] the $H^i(\mathscr{F})\otimes_{\Z_\ell}\Z/\ell^n\Z$ are $(\mathcal{S},L)$-constructible. 
\end{enumerate}
As $(\mathcal{S},L)$ varies, $D^b_c$ is the filtered union of $D_{\mathcal{S},L}^b$, and we define the $p$-perverse $t$-structure by passing to limit over $(\mathcal{S},L)$. Again, for $\mathscr{F}$ to be in $D_c^{\leq p}(X,\Z_\ell)$, it is necessary and sufficient that mod $\ell$ it is in $D_c^{\leq p}(X,\Z/\ell\Z)$.\par

We recall that by definition, $D^b_c(X,\Q_\ell):=D^b_c(X,\Z)\otimes_{\Z_\ell}\Q_\ell$, and that the abelian category of constructible $\Q_\ell$-sheaves is induced by constructible $\Z_\ell$-sheaves via tensoring with $\Q_\ell$. The full subcategory $D^b_{\mathcal{S},L}(X,\Q_\ell)$ (which is the image of $D^b_{\mathcal{S},L}(X,\Z_\ell)\otimes\Q_\ell$) is formed by the complexes $\mathscr{F}$ such that each $H^i(\mathscr{F})$ is tensoring with $\Q_\ell$ of an $(\mathcal{S},L)$-constructible, i.e. its mod $\ell$ reduction is $(\mathcal{S},L)$-constructible. We define the $p$-perverse structure over $D^b_c(X,\Q_\ell)$ by passing to limit over $(\mathcal{S},L)$ and glueing. For any interval $[a,b]$ of $\Z$, the natural functor
\[{^pD_c}^{[a,b]}(X,\Z_\ell)\otimes\Q_\ell\to {^pD}_c^{[a,b]}(X,\Q_\ell)\]
is an equivalence. In particular (for $a=b=0$), the abelian category of perverse $\Q_\ell$-sheaves is induced from that of $\Z_\ell$ by tensoring with $\Q_\ell$.\par

The proof of \cref{triangle cat perverse sheaf is stack} that the perverse sheaves form a stack no longer applies as $\Z_\ell$-sheaves and $\Q_\ell$-sheaves because it was written assuming that one is working inside of a derived category derived of a category of sheaves. Here is another proof. We already know, as in \cref{triangle cat perverse structure Hom define sheaf}, that the morphisms glue together, and it is a question of proving that if $(U_i)_{i\in I}$ is an \'etale covering of $X$, any family $\mathscr{A}_i$ of perverse sheaves endowed with a glueing datum, provides a perverse sheaf $\mathscr{A}$ over $X$. The given glueing datum permits us to define, for each \'etale morphism $j:V\to X$ which factors through one of the $U_i$, a perverse sheaf $\mathscr{A}_V$ over $V$, and these are compatible with inverse images $V'\to V$.\par
We also note that for any morphism $f:Y\to X$ and any irreducible subvariety $i:S'\to X$ of $X$, which give rise to a Cartesian diagram
\[\begin{tikzcd}
T'\ar[r,hook,"i"]\ar[d,"f"]&Y\ar[d,"f"]\\
S'\ar[r,hook,"i"]&X
\end{tikzcd}\]
we have $i^!f_*=f_*i^!$ as functors over derived categories. If $f$ is quasi-finite, $S'$ admits an open dense subset $S$ such that $T=f^{-1}(S)$ is finite over $S$, and is empty or purely of dimension $\dim(S)$. It then follows that $f_*$ sends $D_c^{>p}(Y,\Q_\ell)$ into $D_c^{>p}(X,\Q_\ell)$, and ${^pf_*}:=H^0f_*\eps$ is a left exact functor from perverse sheaves over $Y$ to perverse sheaves of $X$, whose left adjoint is ${^pf^*}:=H^0f^*\eps$ (\cref{triangle cat t-exact functor prop}).\par
Let $h_i$ (resp. $h_{ij}$) be the covering morphism from $U_i$ to $X$ (resp. $U_{ij}=U_i\times_XU_j$) into $X$. Put
\[\mathscr{A}:=\ker\Big(\prod_i{^ph_{i,*}(\mathscr{A}_{U_i})}\rightrightarrows\prod_{i,j}{^ph_{ij}}(\mathscr{A}_{U_{ij}})\Big).\]
It suffices to show that the morphism ${^ph_i^*}(\mathscr{A})\to\mathscr{A}_i=\mathscr{A}_{U_i}$ induced from $\mathscr{A}\to h_{i,*}(\mathscr{A}_{U_i})$ by adjunction is an isomorphism, which can be verified under a base change by any of the $h_i$. The direct images and inverse images commutes with such a localization, so we are reduced to the case where $U_i\to X$ admits a section. In this case, the $\mathscr{A}_i$ is the inverse image of an $\mathscr{A}_X$, and that $\mathscr{A}=\mathscr{A}_X$ follows from the usual homotopy argument which shows that two open covers which are refinements of each other give the same \v{C}ech cohomology.