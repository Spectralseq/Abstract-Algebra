\chapter{Category}
\section{Functors}
\subsection{Definition of Functors}
\begin{definition}
Let $\mathcal{C},\mathcal{D}$ be two categories. A \textbf{covariant functor}
\[\mathscr{F}:\mathcal{C}\to\mathcal{D}\]
is an assignment of an object $F\mathscr{F}(A)\in\mathrm{Obj}(\mathcal{D})$ for every $A\in\mathrm{Obj}(\mathcal{C})$ and of a function
\[\Mor_{\mathcal{C}}(A,B)\to\Mor_{\mathcal{D}}(\mathscr{F}(A),\mathscr{F}(B))\]
for every pair of objects $A,B$ in $\mathcal{C}$ such that $\forall f\in\Mor_{\mathcal{C}}(A,B),g\in\Mor_{\mathcal{C}}(B,C)$ we have
\[\mathscr{F}(\id_A)=\id_{\mathscr{F}_{A}},\quad\mathscr{F}(g\circ f)=\mathscr{F}(g)\circ\mathscr{F}(f).\]
\end{definition}
We have already mentioned briefly a few slightly more imaginative examples; here they are again.
\begin{example}
If $R$ is a ring, we have denoted by $R^{\times}$ the group of units in $R$; every ring homomorphism $R\to S$ induces a group homomorphism $R^{\times}\to S^{\times}$, and this assignment is compatible with compositions; therefore this operation defines a covariant functor $\mathsf{Ring}\to\mathsf{Ab}$.
\end{example}
\subsection{Equivalence of Category}
The structure of an object in a category is adequately carried by its isomorphism class, and a natural notion of equivalence of categories should aim at matching isomorphism classes, rather than individual objects. The \textit{morphisms} are a more essential piece of information; the quality of a functor is first of all measured on how it acts on morphisms.
\begin{definition}
Let $\mathcal{C},\mathcal{D}$ be two categories. Let $\mathscr{F}:\mathcal{C}\to\mathcal{D}$ be a covariant functor.
\begin{itemize}
\item[$(a)$] $\mathscr{F}$ is \textbf{faithful} if for all objects $A,B$ of $\mathcal{C}$, the induced function
\[\Mor_{\mathcal{C}}(A,B)\to\Mor_{\mathcal{D}}(\mathscr{F}(A),\mathscr{F}(D))\]
is injective.
\item[$(b)$] $\mathscr{F}$ is \textbf{full} if this function is surjective, for all objects $A,B$.
\item[$(c)$] $\mathscr{F}$ is called essentially surjective if for any object $B\in\mathrm{Obj}(\mathcal{B})$ there exists an object $A\in\mathrm{Obj}(\mathcal{A})$ such that $\mathscr{F}(A)$ is isomorphic to $B$ in $\mathcal{B}$.
\end{itemize}
\end{definition}
\begin{lemma}\label{fully faithful fuctor}
Let $\mathscr{F}:\mathcal{C}\to\mathcal{D}$ be a fully faithful functor. If $A,B$ are objects in $\mathcal{C}$, then $A\cong B$ in $\mathcal{C}$ if and only if $\mathscr{F}(A)\cong\mathscr{F}(B)$ in $\mathcal{D}$.
\end{lemma}
\begin{proof}
Assume $\mathscr{F}$ is covariant. Since we have $\mathscr{F}(f\circ g)=\mathscr{F}(f)\circ\mathscr{G}(g)$, if $A\cong B$, we have $\mathscr{F}(A)\cong\mathscr{B}$.\par
Conversely, if $\mathscr{F}(A)\cong\mathscr{F}(B)$, assume $f$, $g$ are isomorphisms between them with $g=f^{-1}$. Since $\mathscr{F}$ is full, there are two morphisms $f',g'$ such that $\mathscr{F}(f')=f,\mathscr{F}(g')=g$. Then we have
\[\mathscr{F}(f'\circ g')=f\circ g=\mathrm{id},\quad \mathscr{F}(g'\circ f')=g\circ f=\id\]
Since $\mathscr{F}(\id)=\id$, and $\mathscr{F}$ is faithful, we get $f'\circ g'=g'\circ f'=\id$, so $A\cong B$ follows.
\end{proof}
\begin{definition}
Let $\mathcal{C}, \mathcal{D}$ be categories, and let $\mathscr{F}$, $\mathscr{G}$ be functors $\mathcal{C}\to\mathcal{D}$. A \textbf{natural transformation} $\nu:\mathscr{F}\to\mathscr{G}$ is the datum of a morphism $\nu_X:\mathscr{F}(X)\to\mathscr{G}(X)$ in $\mathcal{D}$ for every object $X$ in $\mathcal{C}$, such that $\forall\alpha:X\to Y$ in $\mathcal{C}$ the diagram
\[\begin{tikzcd}
\mathscr{F}(X)\ar[r,"\nu_X"]\ar[d,swap,"\mathscr{F}(\alpha)"]&\mathscr{G}(X)\ar[d,"\mathscr{G}(\alpha)"]\\
\mathscr{F}(Y)\ar[r,"\nu_Y"]&\mathscr{G}(Y)
\end{tikzcd}\]
commutes. A \textbf{natural isomorphism} is a natural transformation $\mu$ such that $\mu_X$ is an isomorphism for every $X$.
\end{definition}
A natural transformation is often written as
\[\begin{tikzcd}
\mathcal{C}\ar[r,bend left=50,"\mathscr{F}",""'{name=F,below}]\ar[r,bend right=50,swap,"\mathscr{G}",""'{name=G,above}]&\mathcal{D}
\arrow[Rightarrow,from=F,to=G,"\nu"]
\end{tikzcd}\]
In addition, given a morphism of functors $\mu:\mathscr{F}\to\mathscr{G}$ and a morphism of functors $\nu:\mathscr{E}\to\mathscr{F}$ then the composition $\nu\circ\mu$ is defined by the rule
\[(\nu\circ\mu)_X:=\nu_X\circ\mu_X.\]
for $X\in\mathrm{Obj}(\mathcal{A})$. It is easy to verify that this is indeed a morphism of functors from $\mathscr{E}$ to $\mathscr{G}$. In this way, given categories $\mathcal{A}$ and $\mathcal{B}$ we obtain a new category, namely
the category of functors between $\mathcal{A}$ and $\mathcal{B}$.
\begin{definition}
An \textbf{equivalence of categories} $\mathscr{F}:\mathcal{A}\to\mathcal{B}$ is a functor such that there exists a functor $\mathscr{G}:\mathcal{B}\to\mathcal{A}$ such that the compositions $\mathscr{F}\circ\mathscr{G}$ and $\mathscr{G}\circ\mathscr{F}$ are isomorphic to the identity functors $\id_{\mathcal{B}}$, respectively $\id_{\mathcal{A}}$. In this case we say that $\mathscr{G}$ is a \textbf{quasi-inverse} to $\mathscr{F}$.
\end{definition}
\begin{proposition}\label{equiv category iff}
Let $\mathscr{F}:\mathcal{A}\to\mathcal{B}$ be a fully faithful functor. Suppose for every $X\in\mathrm{Obj}(\mathcal{B})$ we are given an object $\mathscr{G}(X)$ of $A$ and an isomorphism $i_X:X\to\mathscr{F}(\mathscr{G}(X))$. Then there is a unique functor $\mathscr{G}:\mathcal{B}\to\mathcal{A}$ such that $\mathscr{G}$ extends the rule on objects, and the isomorphisms $i_X$ define an isomorphism of functors $\id_{\mathcal{B}}\to\mathscr{F}\circ\mathscr{G}$. Moreover, $\mathscr{G}$ and $\mathscr{F}$ are quasi-inverse equivalences of categories.
\end{proposition}
\begin{proof}
The action of $\mathscr{G}$ on objects is defined. For $X,Y\in\mathrm{Obj}(\mathcal{B})$ and $f\in\Mor_{\mathcal{B}}(X,Y)$, we have the diagram
\[\begin{tikzcd}
X\ar[r,"f"]\ar[d,"i_X"]&Y\ar[d,"i_Y"]\\
\mathscr{F}(\mathscr{G}(X))\ar[r,"i_Y\circ f\circ i_X^{-1}"]&\mathscr{F}(\mathscr{G}(Y))\\
\mathscr{G}(X)\ar[r,dashed]\ar[u,"\mathscr{F}"]&\mathscr{G}(Y)\ar[u,"\mathscr{F}"]
\end{tikzcd}\]
where the dashed map is induced by the bijection
\[\Mor_{\mathcal{A}}(X,Y)\cong\Mor_{\mathcal{B}}(\mathscr{F}(X),\mathscr{F}(Y))\]
by the functoriality, this defines a functor $\mathscr{G}:\mathcal{B}\to\mathcal{A}$. Moreover, the upper half of the diagram means $i_X$ define an isomorphism of functors $\id_{\mathcal{B}}\to\mathscr{F}\circ\mathscr{G}$.\par
For $X\in\mathrm{Obj}(\mathcal{A})$, we have an isomorphism $i_{\mathscr{F}(X)}:\mathscr{F}(X)\to\mathscr{F}\circ\mathscr{G}\circ\mathscr{F}(X)$. Since $\mathscr{F}$ is full and faithful, there is an isomorphism $\mu_{X}:X\to\mathscr{G}\circ\mathscr{F}(X)$ by Lemma~\ref{fully faithful fuctor}. The naturality of $\mu_X$ follows from that of $i_{\mathscr{F}(X)}$ and the faithfulness of $\mathscr{F}$. Thus $\mu$ is an isomorphism $\id_{\mathcal{A}}\to\mathscr{G}\circ\mathscr{F}$.
\end{proof}
\begin{corollary}
A functor is an equivalence of categories if and only if it is both fully faithful and essentially surjective.
\end{corollary}
\begin{proof}
Let $\mathscr{F}:\mathcal{A}\to\mathcal{B}$ be essentially surjective and fully faithful. As by convention all categories are small and as $\mathscr{F}$ is essentially surjective we can, using the axiom of choice, choose for every $X\in\mathrm{Obj}(\mathcal{B})$ an object $\mathscr{G}(X)$ of $\mathcal{A}$ and an isomorphism $i_X:X\to\mathscr{F}(\mathscr{G}(X))$. Then we apply Proposition~\ref{equiv category iff}.
\end{proof}
\subsection{Yoneda's Lemma}
\begin{definition}
Given a category $\mathcal{C}$ the opposite category $\mathcal{C}^{op}$ is the category with the same objects as $\mathcal{C}$ but all morphisms reversed.
\end{definition}
\begin{definition}
Let $\mathcal{A},\mathcal{B}$ be categories. A contravariant functor $\mathscr{F}$ from $\mathcal{A}$ to $\mathcal{B}$ is a functor $\mathcal{A}^{op}\to\mathcal{B}$.\par
Concretely, a contravariant functor $\mathscr{F}$ satisfies the property that, given another morphism $f:X\to Y$ and $g:Y\to Z$, we have 
\[\mathscr{F}(g\circ f)=\mathscr{F}(g)\circ\mathscr{F}(f)\]
as morphism from $\mathscr{F}(Z)$ to $\mathscr{F}(X)$.
\end{definition}
\begin{definition}
Let $\mathcal{C}$ be a category. A \textbf{presheaf of sets on $\mathcal{C}$} or simply a presheaf is a contravariant functor $\mathscr{F}$ from $\mathcal{C}$ to $\mathsf{Set}$. The category of presheaves is denoted $\mathsf{Psh}(\mathcal{C})$.
\end{definition}
\begin{example}[\textbf{Functor of points}]
For any $U\in\mathrm{Obj}(\mathcal{C})$ there is a contravariant functor
\[h_X:\mathcal{C}\to\mathsf{Set},\quad Y\mapsto\Mor_{\mathcal{C}}(Y,X).\]
In other words $h_X$ is a presheaf. We will always denote this presheaf $h_X:\mathcal{C}^{op}\to\mathsf{Set}$. It is called the \textbf{representable presheaf} associated to $X$.\par
Note that given a morphism $s:X\to Y$ in $\mathcal{C}$ we get a corresponding natural transformation of functors $h(s):h_X\to h_Y$ defined simply by composing with the morphism $U\to V$. It is trivial to see that this turns composition of morphisms in $\mathcal{C}$ into composition of transformations of functors. In other words we get a functor
\[h:\mathcal{C}\to\mathrm{Func}(\mathcal{C}^{op},\mathsf{Set})=:\widehat{\mathcal{C}}.\]
\end{example}
\begin{lemma}[\textbf{Yoneda lemma}]
The functor $h$ is fully faithful. More generally, given any contravariant functor $\mathscr{F}$ and any object $X$ of $\mathcal{C}$ we have a natural bijection
\[\Mor_{\widehat{\mathcal{C}}}(h_X,\mathscr{F})\to\mathscr{F}(X),\quad \alpha\mapsto \alpha_X(\id_X).\]
\end{lemma}
\begin{proof}
An element $f\in h_X(Y)=\Mor_{\mathcal{C}}(Y,X)$ can be viewed as a morphism $f^*:\Mor_{\mathcal{C}}(X,X)\to\Mor_{\mathcal{C}}(Y,X)$. Note that $f^*(\id_X)=f$, so if there is a natural transformation $\alpha:h_X\to\mathscr{F}$, then from the diagram
\[\begin{tikzcd}
\Mor_{\mathcal{C}}(X,X)\ar[r,"\alpha_X"]\ar[d,"f^*"]&\mathscr{F}(X)\ar[d,"\mathscr{F}(f)"]\\
\Mor_{\mathcal{C}}(Y,X)\ar[r,"\alpha_X"]&\mathscr{F}(Y)
\end{tikzcd}\]
we obtain
\[\alpha_Y(f)=\alpha_Y\big(f^*(\id_X)\big)=\mathscr{F}(f)\big(\alpha_X(\id_X)\big).\]
That is, $\alpha$ is simply determined by $\alpha_X(\id_X)$. Conversely, given $\xi\in\mathscr{F}(X)$, we can define a natural transformation by the formula above:
\[\beta_Y:h_X(Y)\to\mathscr{F}(Y),\quad \beta_Y(f)=\mathscr{F}(f)(\xi).\]
It follows that these two map are inverse of each other.
\end{proof}
\begin{definition}
A contravariant functor $\mathscr{F}:\mathcal{C}\to\mathsf{Set}$ is said to be \textbf{representable} if it is isomorphic to the functor of points $h_X$ for some object $X$ of $\mathcal{C}$.
\end{definition}
Let $\mathcal{C}$ be a category and let $\mathscr{F}:\mathcal{C}^{op}\to\mathscr{Set}$ be a representable functor. Choose an object $X$ of $\mathcal{C}$ and an isomorphism $\alpha:h_X\to\mathscr{F}$. The Yoneda lemma guarantees that the pair $(X,\alpha)$ is unique up to unique isomorphism. The object $X$ is called an object representing $\mathscr{F}$.
\subsection{Limits and colimits}
The various universal properties encountered along the way are all particular cases of the notion of categorical \textbf{limit}, which is worth mentioning explicitly. Let $\mathscr{F}:I\to C$ be a \textit{covariant functor}, where one thinks of $\mathcal{I}$ as a category of indices. The \textbf{limit} of $\mathscr{F}$ is an object $L$ of $\mathcal{C}$, endowed with morphisms $\lambda_I:L\to\mathscr{F}(I)$ for all objects $I$ of $\mathcal{I}$, satisfying the
following properties:
\begin{itemize}
\item If $\alpha:I\to J$ is a morphism in $\mathcal{I}$, then $\lambda_J=\mathscr{F}(\alpha)\circ\lambda_I$:
\[\begin{tikzcd}
&L\ar[ld,swap,"\lambda_I"]\ar[rd,"\lambda_J"]&\\
\mathscr{F}(I)\ar[rr,swap,"\mathscr{F}(\alpha)"]&&\mathscr{F}(J)
\end{tikzcd}\]
\item $L$ is final with respect to this property: that is, if $M$ is another object, endowed with morphisms $\mu_I$, also satisfying the previous requirement, then there exists a unique morphism $M\to L$ making all relevant diagrams commute
\end{itemize}
\begin{example}[\textbf{Products}]
Let $\mathcal{I}$ be the discrete category consisting of two objects $\bm{1}$, $\bm{2}$, with only identity morphisms, and let $\mathscr{A}$ be a functor from $\mathcal{I}$ to any category $\mathcal{C}$; let $A_1=\mathscr{A}(\bm{1})$, $A_2=\mathscr{A}(\bm{2})$ be the two objects of $\mathcal{C}$ indexed by $\mathcal{I}$. Then $\llim\mathscr{A}$ is nothing but the product of $A_1$ and $A_2$ in $\mathcal{C}$: a limit exists if and only if a product of $A_1$ and $A_2$ exists in $\mathcal{C}$.\par 
We can similarly define the product of any $($possibly infinite$)$ family of objects in a category as the limit over the corresponding discrete indexing category, provided of course that this limit exists.
\end{example}
The limit notion is a little more interesting if the indexing category $\mathcal{I}$ carries more structure.
\begin{example}[\textbf{Equalizers and kernels}]
Let $\mathcal{I}$ again be a category with two objects $\bm{1}$, $\bm{2}$, but assume that morphisms look like this:
\[\begin{tikzcd}
\bm{1}\ar[in=210, out=150,looseness=8]\ar[r,bend left,"\alpha"]\ar[r,swap,bend right,"\beta"]&\bm{2}\ar[in=-30, out=30,looseness=8]
\end{tikzcd}\]
That is, add to the discrete category two parallel morphisms $\alpha,\beta$ from one of the objects to the other. A functor $\mathscr{K}:\mathcal{I}\to\mathcal{C}$ amounts to the choice of two objects $A_1$, $A_2$ in $\mathcal{C}$ and two parallel morphisms between them. Limits of such functors are called \textbf{equalizers}. For a concrete example, assume $\mathcal{C}=R$-$\mathsf{Mod}$ is the category of $R$-modules for some ring $R$; let $\varphi:A_2\to A_1$ be a homomorphism, and choose $\mathscr{K}$ as above, with $\mathscr{K}(\alpha)=\varphi$ and $\mathscr{K}(\beta)=$ the zero-morphism. Then $\llim\mathscr{K}$ is nothing but the kernel of $\varphi$.
\end{example}
\begin{example}[\textbf{Limits over chains}]
In another typical situation, $\mathcal{I}$ may consist of a totally ordered set, for example:
\[\begin{tikzcd}
\cdots\ar[r]&\bm{4}\ar[r]&\bm{3}\ar[r]&\bm{2}\ar[r]&\bm{1}
\end{tikzcd}\]
$($that is, the objects are $\bm{i}$, for all positive integers $i$, and there is a unique morphism
$\bm{i}\to \bm{j}$ whenever $i\geq j$; we are only drawing the morphisms $\bm{j}+\bm{1}\to \bm{j}$$)$. Choosing $\mathscr{F}:\mathcal{I}\to\mathcal{C}$ is equivalent to choosing objects $A_i$ of $\mathcal{C}$ for all positive integers $i$ and morphisms $\varphi_{ji}:A_i\to A_j$ for all $i\geq j$, with the requirement that $\varphi_{ii}=1_{A_i}$, and $\varphi_{kj}\circ\varphi_{ji}=\varphi_{ki}$ for all $i\geq j\geq k$. That is, the choice of $\mathscr{F}$ amounts to the choice of a diagram
\[\begin{tikzcd}
\cdots\ar[r,"\varphi_{45}"]&A_4\ar[r,"\varphi_{34}"]&A_3\ar[r,"\varphi_{23}"]&A_2\ar[r,"\varphi_{12}"]&A_1
\end{tikzcd}\]
in $\mathcal{C}$. An inverse limit $\llim\mathscr{F}$ $($which may also be denoted $\llim_iA_i$, when the morphisms $\varphi_{ji}$ are evident from the context$)$ is then an object $A$ endowed with morphisms $\varphi_i:A\to A_i$ such that the whole diagram
\[\begin{tikzcd}
&&&&A\ar[lllldd,swap,dashed,"\cdots"]\ar[lldd,"\varphi_4"]\ar[dd,"\varphi_3" description]\ar[rrdd,swap,"\varphi_2"]\ar[rrrrdd,"\varphi_1"]&&&&\\
&&&&\\
\cdots\ar[rr,"\varphi_{45}"]&&A_4\ar[rr,"\varphi_{34}"]&&A_3\ar[rr,"\varphi_{23}"]&&A_2\ar[rr,"\varphi_{12}"]&&A_1
\end{tikzcd}\]
commutes and such that any other object satisfying this requirement factors uniquely through $A$.\par
Such limits exist in many standard situations. For example, let $C=R$-$\mathsf{Mod}$ be
the category of left-modules over a fixed ring $R$, and let $A_i$, $\varphi_{ji}$ be as above.
\begin{proposition}
The limit $\llim_iA_i$ exists in $R$-$\mathsf{Mod}$.
\end{proposition}
\begin{proof}
The product $\prod_iA_i$ consists of arbitrary sequences $(a_i)_{i>0}$ of elements $a_i\in A_i$. Say that a sequence $(a_i)_{i>0}$ is \textit{coherent} if for all $i>0$ we have $a_i=\varphi_{i,i+1}(a_{i+1})$. Coherent sequences form an $R$-submodule $A$ of $\prod_iA_i$; the canonical projections restrict to $R$-module homomorphisms $\varphi_i:A\to A_i$. The reader will check that $A$ is a limit $\llim_iA_i$.
\end{proof}
This example easily generalizes to families indexed by more general posets.
\end{example}
The dual notion to limit is the \textbf{colimit} of a functor $\mathscr{F}:\mathcal{I}\to\mathcal{C}$. The colimit is an object $C$ of $\mathcal{C}$, endowed with morphisms $\gamma_I:\mathscr{F}(I)\to\mathcal{C}$ for all objects $I$ of $\mathcal{I}$, such that $\gamma_I=\gamma_J\circ\mathscr{F}(\alpha)$ for all $\alpha:I\to J$ and that $C$ is \textit{initial} with respect to this requirement.\par
\begin{example}

For a typical situation consider again the case of a totally ordered set $\mathcal{I}$, for example:
\[\begin{tikzcd}
\bm{1}\ar[r]&\bm{2}\ar[r]&\bm{3}\ar[r]&\bm{4}\ar[r]&\cdots
\end{tikzcd}\]
A functor $\mathscr{F}:\mathcal{I}\to\mathcal{C}$ consists of the choice of objects and morphisms
\[\begin{tikzcd}
A_1\ar[r,"\psi_{12}"]&A_2\ar[r,"\psi_{23}"]&A_3\ar[r,"\psi_{34}"]&A_4\ar[r,"\psi_{45}"]&\cdots
\end{tikzcd}\]
and the direct limit $\rlim_iA_i$ will be an object $A$ with morphisms $\psi_i:A_i\to A$ such
that the diagram
\[\begin{tikzcd}
A_1\ar[rrrrdd,"\psi_1"]\ar[rr,"\psi_{12}"]&&A_2\ar[rr,"\psi_{23}"]\ar[rrdd,"\psi_2"]&&A_3\ar[dd,"\psi_3" description]\ar[rr,"\psi_{34}"]&&A_4\ar[rr,"\psi_{45}"]\ar[lldd,swap,"\psi_4"]&&\cdots\ar[lllldd,dashed,swap,"\cdots"]\\
&&&&\\
&&&&A&&&&
\end{tikzcd}\]
commutes and such that $A$ is initial with respect to this requirement.
\end{example}
\begin{example}
If $\mathcal{C }=\mathsf{Set}$ and all the $\psi_{ij}$ are injective, we are talking about a nested sequence of sets:
\[A_1\sub A_2\sub A_3\sub\cdots\]
the direct limit of this sequence would be the infinite union $\bigcup_iA_i$.
\end{example}
\subsection{Exact functors}
\begin{definition}
Let $\mathscr{F}:\mathcal{A}\to\mathcal{B}$ be a functor
\begin{itemize}
\item[$(a)$] Suppose all finite limits exist in $\mathcal{A}$. We say $\mathscr{F}$ is \textbf{left exact} if it commutes with all finite limits.
\item[$(b)$] Suppose all finite colimits exist in $\mathcal{A}$. We say $\mathscr{F}$ is \textbf{right exact} if it commutes with all finite colimits.
\item[$(c)$] We say $\mathscr{F}$ is \textbf{exact} if it is both left and right exact.
\end{itemize}
\end{definition}
\begin{proposition}
Let $\mathscr{F}:\mathcal{A}\to\mathcal{B}$ be a functor. Suppose all finite limits exist in $\mathcal{A}$. The following are equivalent:
\begin{itemize}
\item[$(a)$] $\mathscr{F}$ is left exact,
\item[$(b)$] $\mathscr{F}$ commutes with finite products and equalizers.
\item[$(c)$] $\mathscr{F}$ transforms a final object of $\mathcal{A}$ into a final object of $\mathcal{B}$, and commutes with fibre products.
\end{itemize}
\end{proposition}
\subsection{Adjunction}
\begin{definition}
Let $\mathcal{C},\mathcal{D}$ be categories, and let $\mathscr{F}:\mathcal{C}\to\mathcal{D}$, $\mathscr{G}:\mathcal{D}\to\mathcal{C}$ be functors. We say that $\mathscr{F}$ and $\mathscr{G}$ are \textbf{adjoint} $($and we say that $\mathscr{G}$ is right-adjoint to $\mathscr{F}$ and $\mathscr{F}$ is left-adjoint to $\mathscr{G}$$)$ if there are natural isomorphisms
\[\tau_{XY}:\Mor_{\mathcal{C}}(X,\mathscr{G}(Y))\stackrel{\sim}{\longrightarrow}\Mor_{\mathcal{D}}(\mathscr{F}(X),Y)\]
for all objects $X$ of $\mathcal{C}$ and $Y$ of $\mathcal{D}$. More precisely, there should be a natural isomorphism of bifunctors \[\mathcal{C}^{op}\times\mathcal{D}\to\mathsf{Set}:\Mor_{\mathcal{C}}(-,\mathscr{G}(-))\stackrel{\sim}{\to}\Mor_{\mathcal{D}}(\mathscr{F}(-),-)\]
\end{definition}
\begin{proposition}
For each $Y$ there is a map $\eta_Y:\mathscr{F}\mathscr{G}(Y)\to Y$ so that for any for any $f:X\to \mathscr{G}(Y)$, the corresponding map $\tau_{XY}(f):\mathscr{F}(X)\to Y$ is given by the composition
\[\begin{tikzcd}
\mathscr{F}(X)\ar[r,"\mathscr{F}(f)"]&\mathscr{F}\mathscr{G}(Y)\ar[r,"\eta_Y"]&Y
\end{tikzcd}\]
Similarly, there is a map $\theta_X:X\to\mathscr{G}\mathscr{F}(X)$ for each $X$ so that $g:\mathscr{F}(X)\to Y$, the corresponding $\tau^{-1}_{XY}(g):X\to \mathscr{G}(Y)$ is given by the composition
\[\begin{tikzcd}
X\ar[r,"\theta_X"]&\mathscr{G}\mathscr{F}(Y)\ar[r,"\mathscr{G}(g)"]&\mathscr{G}(Y)
\end{tikzcd}\]
So the information of $\tau_{XY}$ is the same as these two maps.
\end{proposition}
\begin{proof}
We deal with the first case. Let $f:X\to\mathscr{G}(Y)$ be a map, consider the follwing diagram
\[\begin{tikzcd}
\Mor_{\mathcal{C}}(X,\mathscr{G}(Y))\ar[r,"\tau_{XY}"]&\Mor_{\mathcal{D}}(\mathscr{F}(X),Y)\\
\Mor_{\mathcal{C}}(\mathscr{G}(Y),\mathscr{G}(Y))\ar[r,"\tau_{\mathscr{G}(Y)Y}"]\ar[u,"f^*"]&\Mor_{\mathcal{D}}(\mathscr{F}\mathscr{G}(Y),Y)\ar[u,"\mathscr{F}(f)^*"]
\end{tikzcd}\]
Set $\eta_Y$ to be the image of $\mathrm{id}_{\mathscr{G}(Y)}$ under $\tau_{\mathscr{G}(Y)Y}$ we get the claim. The second can be done similarly.
\end{proof}
\begin{proposition}
Let $\mathscr{F}$ be a left adjoint to $\mathscr{G}$. Then
\begin{itemize}
\item[$(a)$] $\mathscr{F}$ is fully faithful $\iff\id_{\mathcal{C}}\cong\mathscr{G}\circ\mathscr{F}$.
\item[$(b)$] $\mathscr{G}$ is fully faithful $\iff\mathscr{F}\circ\mathscr{G}\cong\id_{\mathcal{D}}$.
\end{itemize}
\end{proposition}
\begin{proof}
Assume $\mathscr{F}$ is fully faithful. We have to show the adjunction map $X\to\mathscr{G}\circ\mathscr{F}(X)$ is an isomorphism. Let $X'\to\mathscr{G}\circ\mathscr{F}(X)$ be any morphism. By adjointness this corresponds to a morphism $\mathscr{F}(X')\to\mathscr{F}(X)$. By fully faithfulness of $\mathscr{F}$ this corresponds to a morphism $X'\to X$. Thus we see that $X\mapsto\mathscr{F}\circ\mathscr{G}(X)$ defines a bijection \[\Mor_{\mathcal{C}}(X',X)\to\Mor(X',\mathscr{G}\mathscr{F}(X))\]
Hence it is an isomorphism. Conversely, if $\id_{\mathcal{C}}\cong\mathscr{G}\circ\mathscr{F}$ then $\mathscr{F}$ has to be fully faithful, as $\mathscr{G}$ defines an left-inverse on morphism sets. The other case is the dual part.
\end{proof}
\begin{proposition}
Let $\mathscr{F}:\mathcal{C}\to\mathcal{D}$ be a functor between categories. If for each $Y\in\Obj(\mathcal{D})$ the functor $\Mor_{\mathcal{D}}(\mathscr{F}(-),Y)$ is representable, then $\mathscr{F}$ has a right adjoint.
\end{proposition}
\begin{proof}
For each $Y$ choose an object $\mathscr{G}(Y)$ and an isomorphism $\Mor_{\mathcal{C}}(-,\mathscr{G}(Y))\to\Mor_{\mathcal{D}}(\mathscr{F}(-),Y)$ of functors. By Yoneda's lemma for any morphism $g:Y\to Y'$ the transformation of functors
\[\begin{tikzcd}
\Mor_{\mathcal{C}}(-,\mathscr{G}(Y))\ar[r,"\sim"]&\Mor_{\mathcal{D}}(\mathscr{F}(-),Y)\ar[r]&\Mor_{\mathcal{D}}(\mathscr{F}(-),Y')\ar[r,"\sim"]&\Mor_{\mathcal{C}}(-,\mathscr{G}(Y'))
\end{tikzcd}\]
corresponds to a unique morphism $\mathscr{G}(g):\mathscr{G}(Y)\to\mathscr{G}(Y')$. The functoriality of $\mathscr{G}$ comes from that of $\mathscr{F}$.
\end{proof}
\begin{example}
The construction of the free group on a given set is concocted so that giving a set-function from a set $A$ to a group $G$ is the same as giving a group homomorphism from $F(A)$ to $G$. What this really means is that for all sets $A$ and all groups $G$ there are natural identifications
\[\Mor_{\mathsf{Set}}(A,S(G))\stackrel{\sim}{\longrightarrow}\Mor_{\mathsf{Grp}}(F(A),G)\]
where $S(G)$ forgets the group structure of $G$. That is, the functor $F:\mathsf{Set}\to\mathsf{Grp}$ constructing free groups is left-adjoint to the forgetful functor $S:\mathsf{Grp}\to\mathsf{Set}$. This of course applies to every other construction of free objects we have encountered: the free functor is, as a rule, left-adjoint to the forgetful functor.
\end{example}
In fact, the very fact that a functor has an adjoint will endow that functor with convenient features. We say that $\mathscr{F}$ is a \textbf{left-adjoint functor} if it has a right adjoint, and that $\mathscr{G}$ is a \textbf{right-adjoint functor} if it has a left-adjoint.
\begin{theorem}\label{radjoint limit}
Let $\mathscr{F}$ be a left adjoint to $\mathscr{G}$.
\begin{itemize}
\item[$(a)$] Suppose that $\mathscr{A}:\mathcal{I}\to\mathcal{C}$ is a diagram, and suppose that $\llim\mathscr{A}$ exists in $\mathcal{C}$. Then 
\[\mathscr{G}(\llim\mathscr{A})=\llim(\mathscr{G}\circ\mathscr{A})\]
In other words, $\mathscr{G}$ commutes with limits.
\item[$(b)$] Suppose that $\mathscr{A}:\mathcal{I}\to\mathcal{C}$ is a diagram, and suppose that $\rlim\mathscr{A}$ exists in $\mathcal{C}$. Then 
\[\mathscr{F}(\rlim\mathscr{A})=\rlim(\mathscr{F}\circ\mathscr{A})\]
In other words, $\mathscr{F}$ commutes with colimits.
\end{itemize}
\end{theorem}
\begin{proof}
A morphism from a colimit into an object is the same as a compatible system of morphisms from the constituents of the limit into the object, so
\begin{align*}
\Mor_{\mathcal{C}}(X,\mathscr{G}(\llim\mathscr{A}))&\cong\Mor_{\mathcal{D}}(\mathscr{F}(X),\llim\mathscr{A})=\llim\Mor_{\mathcal{D}}(\mathscr{F}(X),\mathscr{A}_i)=\llim\Mor_{\mathcal{D}}(X,\mathscr{G}(\mathscr{A}_i))
\end{align*}
proves that $\mathscr{G}(\llim\mathscr{A})$ is the limit we are looking for. A similar argument works for the other statement.
\end{proof}
\begin{corollary}
Let $\mathscr{F}$ be a left adjoint to $\mathscr{G}$.
\begin{itemize}
\item[$(a)$] If $\mathcal{C}$ has finite colimits, then $\mathscr{F}$ is right exact. 
\item[$(b)$] If $\mathcal{D}$ has finite limits, then $\mathscr{G}$ is right exact. 
\end{itemize}
\end{corollary}
\subsection{Exercise}
\begin{exercise}
Let $\mathscr{F}:\mathcal{C}\to\mathcal{D}$ be a covariant functor, and assume that both $\mathcal{C}$ and $\mathcal{D}$ have products. Prove that for all objects $A$, $B$ of $\mathcal{C}$, there is a unique morphism $\mathscr{F}(A\times B)\to\mathscr{F}(A)\times\mathscr{F}(B)$ such that the relevant diagram involving natural
projections commutes.\par
If $\mathcal{D}$ has coproducts $($denoted $\amalg$$)$ and $\mathscr{G}:\mathcal{C}\to\mathcal{D}$ is contravariant, prove that there is a unique morphism $\mathscr{G}(A)\amalg\mathscr{G}(B)\to \mathscr{G}(A\times B)$ $($again, such that an appropriate diagram commutes$)$.
\end{exercise}
\begin{proof}
Apply the functor $\mathscr{F}$ yields:
\[\begin{tikzcd}
&A\times B\ar[ld]\ar[rd]&\\
A&&B
\end{tikzcd}\stackrel{\mathscr{F}}{\Longrightarrow}\begin{tikzcd}
&\mathscr{F}(A\times B)\ar[ld]\ar[rd]&\\
\mathscr{F}(A)&&\mathscr{F}(B)
\end{tikzcd}\]
by the universal property of $\mathscr{F}(A)\times\mathscr{F}(B)$, there is a unique morphism:
\[\mathscr{F}(A\times B)\stackrel{\exists !}{\longrightarrow}\mathscr{F}(A)\times\mathscr{F}(B)\]
Similar for coproducts:
\[\begin{tikzcd}
&A\times B\ar[ld]\ar[rd]&\\
A&&B
\end{tikzcd}\stackrel{\mathscr{G}}{\Longrightarrow}\begin{tikzcd}
&\mathscr{G}(A\times B)&\\
\mathscr{G}(A)\ar[ru]&&\mathscr{G}(B)\ar[lu]
\end{tikzcd}\]
By the universal property of $\mathscr{G}(A)\amalg\mathscr{G}(B)$, we get
\[\mathscr{G}(A)\amalg\mathscr{G}(B)\to\mathscr{G}(A\times B)\]
\end{proof}
\begin{exercise}
Let $\mathcal{C}$ be a small category. Prove that $\mathcal{C}$ is equivalent to the subcategory
of representable functors in $\mathsf{Set}^{\mathcal{C}^{\circ}}$.\par
Thus, every $(small)$ category is equivalent to a subcategory of a functor category.
\end{exercise}
\begin{proof}
For $\varphi:A\to B$, there is an induced natural transformation:
\[\varphi:\Hom_{\mathcal{C}}(-,A)\to\Hom_{\mathcal{C}}(-,B)\]
from Yoneda lemma, there is a bijection from $\Hom(h_A,h_B)$ to $h_B(A)=\Hom(A,B)$. So $h$ is a fully faithful covariant functor. For any representable $\mathscr{F}$, there is a functor $h_X$ and a natural isomorphism $\mathscr{F}\cong h_X$. This shows $h$ is a equivalence of categories. 
\end{proof}
\begin{exercise}\label{adic comple}
Let $R$ be a commutative ring, and let $I\sub R$ be an ideal. Note that $I^n\sub I^m$ if $n\geq m$, and hence we have natural homomorphisms $\varphi_{mn}:R/I^n\to R/I^m$ for $n\geq m$.
\begin{itemize}
\item Prove that the inverse limit $\widehat{R}_I:=\llim_nR/I^n$ exists as a commutative ring. This
is called the $I$-adic completion of $R$.
\item By the universal property of inverse limits, there is a unique homomorphism $R\to\widehat{R}_I$. Prove that the kernel of this homomorphism is $\bigcap_nI^n$.
\item Let $I=(x)$ in $R[x]$. Prove that the completion $\widehat{R[x]}_I$ is isomorphic to the power series ring $R[[x]]$.
\end{itemize}
\end{exercise}
\begin{proof}
We first prove that limit exists in $\mathsf{Ring}$. Let $\mathcal{I}$ be a poset $(\mathcal{I},\leq)$. Choose $\{R_i\}_{i\in\mathcal{I}}$ and $\{\varphi_{ij}:R_i\to R_j\}$ such that
\[i\geq j\geq k\Rightarrow\varphi_{jk}\circ\varphi_{ij}=\varphi_{ik}\]
A sequence $(r_i)_{i\in\mathcal{I}}$ is coherent if $\varphi_{ij}(r_i)=r_j$. Define the limit $\llim_{i}R_i$ to be
\[\llim_iR_i:=\{(r_i)_{i\in\mathcal{I}}\mid (r_i)\text{ is coherent}\}.\]

Since $I^n\sub I^m$ for $n\geq m$, there is a well defined quotient homomorphism:
\[\varphi_{nm}:R/I^n\to R/I^m,\quad a+I^n\mapsto a+I^m\]
So the limit $\llim_iR/I^n$ is well defined.\par
For the homomorphism $\psi_n:R\to R/I^n$, it is clear that
\[\varphi_{nm}\circ\psi_n=\psi_m\]
so we get the unique homomorphsim $\psi:R\to\widehat{R}_I$, defined by $\psi(r)=(\psi_i(r))_{i\in\mathcal{I}}$. It follows that
\[\psi(r)=0\iff \psi_i(r)=0\iff r\in\bigcap_nI^n.\]

Finally, if $I=(x)$, then $i^n=(x^n)$. So 
\[\widehat{R[x]}_I=\{(r_i)_{i\in\N}:\deg r_i<i,r_i=r_{i-1}+a_{i-1}x^{i-1}\}\]
this set equals $R[[x]]$.
\end{proof}
\begin{exercise}
An important example of the construction presented in Exercise~\ref{adic comple} is the ring $\Z_p$ of $p$-adic integers: this is the limit $\llim_r\Z/p^r\Z$, for a positive prime integer $p$.\par
The field of fractions of $\Z_p$ is denoted $\Q_p$; elements of $\Q_p$ are called \textbf{$p$-adic numbers}.
\begin{itemize}
\item Show that giving a $p$-adic integer $A$ is equivalent to giving a sequence of integers $A_r, r\geq 1$, such that $0\leq A_r<p^r$, and that $A_s\equiv A_r$ mod $p^s$ if $s\leq r$.
\item Equivalently, show that every $p$-adic integer has a unique infinite expansion $A=a_0+a_1\cdot p+a_2\cdot p^2+a_3\cdot p^3+\cdots$, where $0\leq a_i\leq p-1$. The arithmetic of $p$-adic integers may be carried out with these expansions in precisely the same way as ordinary arithmetic is carried out with ordinary decimal expansions.
\item With notation as in the previous point, prove that $A\in\Z_p$ is invertible if and only if $a_0\neq 0$.
\item Prove that $\Z_p$ is a local domain, with maximal ideal generated by $($the image in $\Z_p$ of$)$ $p$.
\item Prove that $\Z_p$ is a DVR. $($There is an evident valuation on $\Q_p$.$)$
\end{itemize}
\end{exercise}
\begin{proof}
Every $A_r$ is in $\Z/p^r\Z$, so $0\leq A_r<p^r$. From the construction, $A_s+p^s=\varphi_{rs}(A_r+p^r)=A_r+p^s$ for $s\leq r$, we see that $A_r\equiv A_s$ mod $p^s$ if $s\leq r$.\par
Similar to the example $R[[x]]$. Giving a sequence is the same as giving a truncation of a series. And the third point is the same as series $R[[x]]$. From the previous point, we find that $\Z_p/p\Z_p$ is a field, so $p\Z_p$ is a maximal ideal.\par
Define a function $v_p(x):=\sup\{n:x\in p^n\Z_p\}=\inf\{n:x_n\neq 0\}$. Then for any ideal $I\sub\Z_p$, let $n:=\min\{v_p(x):x\in I\}$, then $I\sub p^n\Z_p$. Now let $y=p^nx\in I$, then $x$ is invertible, so $(p^nx)=p^n\Z_p\sub I$. This shows every ideal in $\Z_p$ has the form $p^n\Z_p$, and $p\Z_p$ is the unique maximal ideal.
\end{proof}
\begin{exercise}\label{completion of Z}
If $m,n$ are positive integers and $m\mid n$, then $(n)\sub (m)$, and there is an onto ring homomorphism $\Z/n\Z\twoheadrightarrow\Z/m\Z$. The limit ring $\llim\Z/n\Z$ exists and is denoted by $\widehat{\Z}$. Prove that $\widehat{\Z}=\End_{\mathsf{Ab}}(\Q/\Z)$. 
\end{exercise}
\begin{proof}
Every $f\in\End_{\mathsf{Ab}}(\Q/\Z)$ is uniquely determined by $f(\frac{1}{n})$. Since $n\cdot f(\frac{1}{n})=f(1)=f(0)=0$, $f(\frac{1}{n})=\frac{g(n)}{n}$ for some integer $g(n)$. Since we are deal with $\Q/\Z$, we may choose $0\leq g(n)<n$.\par
For $m\mid n$, we have $n=am$, so
\[f(\dfrac{1}{n})=f(\dfrac{1}{am})=\dfrac{g(n)}{am},\quad f(\dfrac{1}{m})=\dfrac{g(m)}{m}\]
and
\[a\cdot f(\dfrac{1}{n})=\dfrac{g(n)}{m}=f(\dfrac{1}{m})=\dfrac{g(m)}{m}\]
so we have $g(n)\equiv g(n)$ mod $m$. This means the sequence
\[(f(\dfrac{1}{n}))_{i\in\N}\]
is an element of $\widehat{\Z}$. Conversely, any element in $\widehat{\Z}$ uniquely defines an endomorphism of $\Q/\Z$. So we have $\widehat{\Z}=\End_{\mathsf{Ab}}(\Q/\Z)$.
\end{proof}
\begin{exercise}
Let $\widehat{\Z}$ be as in Exercise~\ref{completion of Z}.
\begin{itemize}
\item If $R$ is a commutative ring endowed with homomorphisms $R\to\Z/p^r\Z$ for all primes $p$ and all $r$, compatible with all projections $\Z/p^r\Z\to\Z/p^s\Z$ for $s\leq r$, prove that there are ring homomorphisms $R\to\Z/n\Z$ for all $n$, compatible with all projections $\Z/n\Z\to\Z/m\Z$ for $m\mid n$.
\item Deduce that $\widehat{\Z}$ satisfies the universal property for the product of $\Z_p$, as $p$ ranges over all positive prime integers.
\end{itemize}
It follows that $\prod_p\Z_p\cong\widehat{\Z}\cong\End_{\mathsf{Ab}}(\Q/\Z)$.
\end{exercise}
\begin{proof}
For any $n=p_1^{r_1}\cdots p_i^{r_i}$, $m=p_1^{r'_1}\cdots p_i^{r'_i}$ with $m\mid n$, by Chinese remainder theorem we have a commutative diagram
\[\begin{tikzcd}
\prod_{i}\Z/p_i^{r_i}\Z\ar[r,"\sim"]\ar[d, twoheadrightarrow]&\Z/n\Z\ar[d, twoheadrightarrow]\\
\prod_{i}\Z/p_i^{r'_i}\Z\ar[r,"\sim"]&\Z/m\Z
\end{tikzcd}\]
so we get the first result.\par
Note that giving a morphism from $R$ to $\Z_p$ is the same as giving morphisms from $R$ to $\Z/p^r\Z$ fro all $r$. So if there is a ring $R$ with morphisms to $\Z_p$ for all prime $p$, then we get morphisms to $\Z/p^r\Z$ for all prime $p$, all $r$. From the previous point, there are morphisms $R\to\Z/n\Z$ for all $n$, compatible with all projection $\Z/n\Z\to\Z m\Z$. From the definition of $\widehat{\Z}$, there is a unique morphism from $R$ to $\widehat{\Z}$. So $\widehat{\Z}$ satisfies the universal property of $\prod_p\Z_p$.
\end{proof}
\begin{exercise}
Let $R,S$ be rings. An additive covariant functor $\mathscr{F}:R$-$\mathsf{Mod}\to S$-$\mathsf{Mod}$ is \textbf{faithfully exact} if \[\begin{tikzcd}
\mathscr{F}(A)\ar[r,"\mathscr{F}(\varphi)"]&\mathscr{F}(B)\ar[r,"\mathscr{F}(\psi)"]&\mathscr{F}(C)
\end{tikzcd}\] 
is exact in $S$-$\mathsf{Mod}$ if and only if 
\[\begin{tikzcd}
A\ar[r,"\varphi"]&B\ar[r,"\psi"]&C
\end{tikzcd}\]
is exact in $R$-$\mathsf{Mod}$. Prove that an exact functor $\mathscr{F}:R$-$\mathsf{Mod}\to S$-$\mathsf{Mod}$ is faithfully exact if and only if $\mathscr{F}(M)\neq0$ for every nonzero $R$-module $M$, if and only if $\mathscr{F}(\varphi)\neq0$ for every nonzero morphism $\varphi$ in $R$-Mod.
\end{exercise}
\begin{proof}
\mbox{}
\begin{itemize}
\item One direction is easy: If $\mathscr{F}$ is faithfully exact. Assume $\mathscr{F}(M)=0$, then the sequence $0\to\mathscr{F}(M)\to0$ is exact, but $0\to M\to 0$ is not exact unless $M=0$, so we find $M=0$. If $\mathscr{F}(\varphi)=0$, then 
\[\begin{tikzcd}
\mathscr{F}(M)\ar[r,"\mathscr{F}(\varphi)"]&\mathscr{F}(N)\ar[r,"id_{\mathscr{F}(N)}"]&\mathscr{F}(N)
\end{tikzcd}\] 
is exact, but 
\[\begin{tikzcd} 
M\ar[r,"\varphi"]&N\ar[r,"id_N"]&N 
\end{tikzcd}\] 
is exact only if $\varphi=0$.
\item Then we show that $\mathscr{F}$ reflects zero objects if and only if $\mathscr{F}$ reflects zero morphisms: If $\mathscr{F}$ reflects zero morphisms, assume $\mathscr{F}(X)=0$, then $\id_{\mathscr{F}(X)}=0$, so $\id_X=0$. But $id_X=0\iff X=0$, so $X=0$.\par 
Now assume $\mathscr{F}$ reflects zero objects. For $\varphi:A\to B$ such that $\mathscr{F}(\varphi)=0$. Consider the exact sequence:
\[\begin{tikzcd}
A\ar[r,"\varphi"]&\im\varphi\ar[r,"\psi"]&\coker\varphi
\end{tikzcd}\]
since $\mathscr{F}$ is exact, we also have a exact sequence
\[\begin{tikzcd}
\mathscr{F}(A)\ar[r,"\mathscr{F}(\varphi)"]&\mathscr{F}(\im\varphi)\ar[r,"\mathscr{F}(\psi)"]&\mathscr{F}(\coker\varphi)
\end{tikzcd}\]
Note that $\mathscr{F}(\varphi)=0$, so $\mathscr{F}(\psi)$ is monic. But $\psi=0$ so $\mathscr{F}(\psi)=0$, we conclude that $\mathscr{F}(\im\varphi)=0$. This means $\im\varphi=0$, so we have $\varphi=0$.
\item Now we show the last direction. Let $\mathscr{F}$ reflects zero morphisms, first we show that $\mathscr{F}$ reflects monomorphisms and epimorphisms. In deed, suppose
\[\begin{tikzcd}
0\ar[r]&X\ar[r]&Y\ar[r]&Z
\end{tikzcd}\]
is exact; then
\[\begin{tikzcd}
0\ar[r]&\mathscr{F}(X)\ar[r]&\mathscr{F}(Y)\ar[r]&\mathscr{F}(Z)
\end{tikzcd}\]
is exact. If $\mathscr{F}(Y)\to\mathscr{F}(Z)$ is monic, then $\mathscr{F}(X)=0$, so $X=0$. This shows $\mathscr{F}$ reflects monomorphisms, the dual argument shows that $\mathscr{F}$ reflects epimorphisms. Now, in an abelian category, $f$ is an isomorphism if and only if $f$ is both monic and epic, so this implies $\mathscr{F}$ reflects isomorphisms. Now suppose $X\to Y\to Z$ is given and
\[\begin{tikzcd}
0\ar[r]&\mathscr{F}(X)\ar[r,"\mathscr{F}(f)"]&\mathscr{F}(Y)\ar[r,"\mathscr{F}(g)"]&\mathscr{F}(Z)
\end{tikzcd}\]
is exact. Since $\mathscr{F}(X)\to\ker\mathscr{F}(g)$ is an isomorphism, and $\ker\mathscr{F}(g)=\mathscr{F}(\ker g)$, $X\to\ker g$ is also an isomorphism, so $X\to Y\to Z$ is exact.
\end{itemize}
\end{proof}
\begin{exercise}\label{locali exact}
Prove that localization is an exact functor.\par
In fact, prove that localization preserves homology: if
\[\begin{tikzcd}
M_{\bullet}:&\cdots\ar[r]&M_{i+1}\ar[r,"d_{i+1}"]&M_i\ar[r,"d_i"]&M_{i-1}\ar[r]&\cdots
\end{tikzcd}\]
is a complex of $R$-modules and $S$ is a multiplicative subset of $R$, then the localization of the $i$-th homology of $M_{\bullet}$ is the $i$-th homology $H_i(S^{-1}M_{\bullet})$ of the localized complex
\[\begin{tikzcd}
S^{-1}M_{\bullet}:&\cdots\ar[r]&S^{-1}M_{i+1}\ar[r,"S^{-1}d_{i+1}"]&S^{-1}M_i\ar[r,"S^{-1}d_i"]&S^{-1}M_{i-1}\ar[r]&\cdots
\end{tikzcd}\]
\end{exercise}
\begin{proof}
Since 
\[d_i(\dfrac{m}{s})=d(\dfrac{ms'}{ss'})\]
we have
\[\ker S^{-1}d_i=\{\dfrac{m}{s}\mid \exists r\in S, rm\in\ker d_i\},\quad\im S^{-1}d_{i+1}=\{\dfrac{m}{s}\mid\exists r\in s, rm\in\im d_{i+1}\}\]
Concerning the quotient, first we observet that, in the construction of $S^{-1}H_i(M_{\bullet})$:
\begin{align*}
\dfrac{a+\im d_{i+1}}{s}=\dfrac{a'+\im d_{i+1}}{s'}&\iff (\exists r\in S)\quad r[(a+\im d_{i+1})s'-(a'+\im d_{i+1})s]=0\text{ in $H_i(M_{\bullet})$ }\\
&\iff (\exists r\in S)\quad r(as'-a's)\in\im d_{i+1}\end{align*}
While in the quotient $H_i(S^{-1}M_{\bullet})$:
\[\dfrac{a}{s}+\im S^{-1}d_{i+1}=\dfrac{a'}{s'}+\im S^{-1}d_{i+1}\iff \dfrac{a}{s}-\dfrac{a'}{s'}\in\im S^{-1}d_{i+1}\iff (\exists r\in S)\  r(as'-a's)\in\in d_{i+1}\]
So there is a natural homomorphism:
\[\psi:S^{-1}H_i(M_{\bullet})\to H_i(S^{-1}M_{\bullet}),\quad \dfrac{a+\im d_{i+1}}{s}\mapsto\dfrac{a}{s}+\im S^{-1}d_{i+1}\]
this is an isomorphism from the observation above.
\end{proof}
\begin{exercise}
Suppose $M$ is a finitely presented $R$-module and $N$ is an arbitrary $R$-module. Show the followsing holds
\[S^{-1}\Hom_R(M,N)\stackrel{\sim}{\longrightarrow}\Hom_{S^{-1}R}(S^{-1}M,S^{-1}N)\]
But note that this does not holds for any module.
\end{exercise}
\begin{proof}
First we have
\[S^{-1}\Hom_R(R,N)\stackrel{\sim}{\longrightarrow}\Hom_{S^{-1}R}(S^{-1}R,S^{-1}N)\]
for any $N$. And we have a natural homomorphism 
\[S^{-1}\Hom_A(M,N)\to \Hom_{S^{-1}A}(S^{-1}M,S^{-1}N).\] 
Consider the diagram:
\[\begin{tikzcd}[column sep=small]
0\ar[r]&S^{-1}\Hom_{R}(M,N)\ar[r]\ar[d]&S^{-1}\Hom_{R}(R^m,N)\ar[d]\ar[r]&S^{-1}\Hom_{R}(R^n,N)\ar[d]\\
0\ar[r]&\Hom_{S^{-1}R}(S^{-1}M,S^{-1}N)\ar[r]&\Hom_{S^{-1}R}(S^{-1}R^m,S^{-1}N)\ar[r]&\Hom_{S^{-1}R}(S^{-1}R^n,S^{-1}N)
\end{tikzcd}\]
The right two vertical maps are isomorphisms, so we get the isomorphism.\par
For $R=N=\Z$, $M=\Q$, $S=\Z-\{0\}$, we have
\[S^{-1}\Hom_\Z(\Q,\Z)=S^{-1}\{0\}=0,\quad \Hom_{S^{-1}\Z}(S^{-1}\Q,S^{-1}\Z)=\Hom_{\Q}(\Q,\Q)=\Q\]
\end{proof}
\begin{exercise}
Suppose $F:\mathcal{A}\to\mathcal{B}$ is a covariant functor of abelian categories, and $C^\bullet$ is a complex in $\mathcal{A}$.
\begin{itemize}
\item[$(a)$]If $F$ is right-exact, describe a natural morphism $FH^\bullet\to H^\bullet F$.
\item[$(b)$]If $F$ is right-exact, describe a natural morphism $FH^\bullet\leftarrow H^\bullet F$.
\item[$(c)$]If $F$ is exact, show that the morphisms of $(a)$ and $(b)$ are inverses and thus isomorphisms.
\end{itemize}
\end{exercise}
\begin{proof}
First we recall that if $F$ is right-exact, then $F$ commutes with cokernels: For we have the following exact sequence
\[\begin{tikzcd}
0\ar[r]&F(C^i)\ar[r,"F(d^{i})"]&F(C^{i+1})\ar[r]&F(\coker d^i)\ar[r]&0
\end{tikzcd}\]
which is obtained from the corresponding short exact sequence. Hence
\[F(\coker d^i)\cong \coker F(d^i)\]
\begin{itemize}
\item[$(a)$]Consider the exact sequence
\[\begin{tikzcd}
0\ar[r]&\im d^i\ar[r]&C^{i+1}\ar[r]&\coker d^i\ar[r]&0
\end{tikzcd}\]
Applying $F$ on this gives us
\[\begin{tikzcd}
F\im d^i\ar[r]&F(C^{i+1})\ar[r]&F\coker d^i\ar[r]&0
\end{tikzcd}\]
Together with the similar sequence in $F(C^\bullet)$ we get a diagram
\[\begin{tikzcd}
&F\im d^i\ar[r]\ar[d,dashed,"\alpha"]&F(C^{i+1})\ar[r]\ar[d,equal]&F\coker d^i\ar[d,"\cong"]\ar[r]&0\\
0\ar[r]&\im F(d^i)\ar[r]&F(C^{i+1})\ar[r]&F\coker d^i\ar[r]&0
\end{tikzcd}\]
Then we can show there is an induced map $\alpha:F\im d^i\to\im f(d^i)$. Further, by the snake lemma, this induced map $\alpha$ is an epimorphism.
Now consider another sequence 
\[\begin{tikzcd}
0\ar[r]&H^i(C^\bullet)\ar[r]&\coker d^{i-1}\ar[r]&\im d^i\ar[r]&0
\end{tikzcd}\]
Applying $F$ gives 
\[\begin{tikzcd}
FH^i(C^\bullet)\ar[r]&F\coker d^{i-1}\ar[r]&F\im d^i\ar[r]&0
\end{tikzcd}\]
Similarly, with the counterpart in $F(C^\bullet)$, there is a diagram
\[\begin{tikzcd}
&FH^i(C^\bullet)\ar[r]\ar[d,dashed,"\beta"]&F\coker d^{i-1}\ar[r]\ar[d,"\cong"]&F\im d^i\ar[r]\ar[d,"\alpha"]&0\\
0\ar[r]&H^iF(C^\bullet)\ar[r]&\coker F(d^{i-1})\ar[r]&\im F(d^i)\ar[r]&0
\end{tikzcd}\]
Together with $\alpha$, we get our desired map 
\[\beta:FH^i(C^\bullet)\to H^iF(C^\bullet)\]
\item[$(b)$]Instead of $(a)$, we may use the sequence for kernels:
\[\begin{tikzcd}
0\ar[r]&\ker d^i\ar[r]&C^i\ar[r]&\im d^i\ar[r]&0
\end{tikzcd}\]
and
\[\begin{tikzcd}
0\ar[r]&\im d^{i-1}\ar[r]&\ker d^i\ar[r]&H^i(C^\bullet)\ar[r]&0
\end{tikzcd}\]
with the identification
\[\ker F(d^i)\cong F(\ker d^i)\]
\item[$(c)$]With the exactness hypothesis, the map we obtained all becomes isomorphisms.
\end{itemize}
\end{proof}
\section{Abelian Category}
\subsection{Preadditive and additive categories}
\subsubsection{Preaditive category}
\begin{definition}
A category $\mathcal{A}$ is called \textbf{preadditive} if each morphism set $\Mor_{\mathcal{A}}(X,Y)$ is endowed with the structure of an abelian group such that the compositions
\[\Mor_{\mathcal{A}}(X,Y)\times \Mor_{\mathcal{A}}(Y,Z)\to\Mor_{\mathcal{A}}(X,Z)\]
are bilinear. A functor $\mathscr{F}:\mathcal{A}\to\mathcal{B}$ of preadditive categories is called an \textbf{additive functor} if and only if 
\[\mathscr{F}:\Mor_{\mathcal{A}}(X,Y)\to\Mor_{\mathcal{B}}(\mathscr{F}(X),\mathscr{F}(Y))\] 
is a homomorphism of abelian groups for all $X,Y\in\Obj(\mathcal{A})$.
\end{definition}
In particular for every $X,Y$ there exists at least one morphism $X\to Y$, namely the \textbf{zero map}.
\begin{lemma}\label{preadd cat id=0}
Let $\mathcal{A}$ be a preadditive category. Let $X$ be an object of $\mathcal{A}$. The following are equivalent:
\begin{itemize}
\item[$(a)$] $X$ is a initial object.
\item[$(b)$] $X$ is a final object.
\item[$(c)$] $\id_X=0$ in $\Mor_{\mathcal{A}}(X,X)$.
\end{itemize}
Furthermore, if such an object $0$ exists, then a morphism $f:X\to Y$ factors through $0$ if and only if $f=0$.
\end{lemma}
\begin{proof}
Clearly if $X$ is a final or initial object, then $\id_X=0$ is the unique morphism $X\to X$. Now assume $\id_X=0$ holds, then \[f\in\Mor_{\mathcal{A}}(X,Y)\Rightarrow f=f\circ\id_X=0,\And g\in\Mor_{\mathcal{A}}(Y,X)\Rightarrow g=\id_X\circ g=0.\] 
Thus $X$ is final and initial.
\end{proof}
\begin{definition}
In a preadditive category $\mathcal{A}$ we call \textbf{zero object}, and we denote it $0$ any final and initial object as in Lemma~\ref{preadd cat id=0} above.
\end{definition}
\begin{proposition}\label{preadd cat prod coprod}
Let $\mathcal{A}$ be a preadditive category. Let $X,Y\in\Obj(\mathcal{A})$. Then the product $X\times Y$ exists if and only if the coproduct $X\amalg Y$ exists. In this case $X\times Y\cong X\amalg Y$.
\end{proposition}
\begin{proof}
Suppose that $X\times Y$ exists with projections $\pi_1:X\times Y\to X$ and $\pi_2:X\times Y\to Y$. Denote $i_1:X\to X\times Y$ the morphism corresponding to $(0,1)$:
\[\begin{tikzcd}
&X\ar[ldd,bend right=20pt,swap,"1"]\ar[d,dashed,"i_1"]\ar[rdd,bend left=20pt,"0"]&\\
&X\times Y\ar[ld,swap,"\pi_1"]\ar[rd,"\pi_2"]&\\
X&&Y
\end{tikzcd}\]
Similarly, denote $i_2:Y\to X\times Y$ the morphism corresponding to $(0,1)$. Thus we have the commutative diagram
\[\begin{tikzcd}
X\ar[rr,"1"]\ar[rd,"i_1"]&&X\\
&X\times Y\ar[ru,"\pi_1"]\ar[rd,"\pi_2"]&\\
Y\ar[rr,"1"]\ar[ru,"i_2"]&&Y
\end{tikzcd}\]
where the diagonal compositions are zero. It follows that $i_1\circ \pi_1+i_2\circ\pi_2$ is the identity since it is a morphism which upon composing with $\pi_1$ gives $\pi_1$ and upon composing with $\pi_2$ gives $\pi_2$. Suppose given morphisms $f:X\to Z$ and $g:Y\to Z$. Then we can form the map $f\circ\pi_1+g\circ\pi_2:X\times Y\to Z$. In this way we get a bijection $\Mor_{\mathcal{A}}(X\times Y,Z)=\Mor_{\mathcal{A}}(X,Z)\times\Mor_{\mathcal{A}}(Y,Z)$ which show that $X\times Y\cong X\amalg Y$. The coproduet case can be done similarly.
\end{proof}

\begin{definition}
Given a pair of objects $X,Y$ in a preadditive category $\mathcal{A}$ we call \textbf{direct sum}, and we denote it $X\oplus Y$ the product $X\times Y$ endowed with the morphisms $\pi_1,\pi_2,i_1,i_2$ as in Proposition~\ref{preadd cat prod coprod} above.
\end{definition}
\begin{proposition}
Let $\mathcal{A},\mathcal{B}$ be preadditive categories. Let $\mathscr{F}:\mathcal{A}\to\mathcal{B}$ be an additive functor. Then $\mathscr{F}$ transforms direct sums to direct sums and zero to zero.
\end{proposition}
\begin{proof}
Suppose $\mathscr{F}$ is additive. A direct sum $Z$ of $X$ and $Y$ is characterized by having morphisms 
\[i_1:X\to Z,\ i_2:Y\to Z,\ \pi_1:Z\to X,\ \pi_2:Z\to Y\]
such that
\[\pi_1\circ i_1=\id_X,\pi_2\circ i_2=\id_Y,\pi_2\circ i_1=0,\pi_1\circ i_2=0\And i_1\circ\pi_1+i_2\circ\pi_2=\id_Z.\]
Clearly $\mathscr{F}(X)$, $\mathscr{F}(Y)$, $\mathscr{F}(Z)$ and the morphisms $\mathscr{F}(i_1),\mathscr{F}(i_2),\mathscr{F}(\pi_1),\mathscr{F}(\pi_1)$ satisfy exactly the same relations (by additivity) and we see that $\mathscr{F}(Z)$ is a direct sum of $\mathscr{F}(X)$ and $\mathscr{F}(Y)$.
\end{proof}
\subsubsection{Additive category}
\begin{definition}
A category $\mathcal{A}$ is called \textbf{additive} if it is preadditive and finite
products exist, in other words it has a zero object and direct sums.
\end{definition}
Namely the empty product is a finite product and if it exists, then it is a final object.
\begin{definition}
Let $\varphi:A\to B$ be a morphism in an additive category $\mathcal{A}$. A morphism $\iota:K\to A$ is a \textbf{kernel} of $\varphi$ if $\varphi\circ\iota=0$ and for all morphisms $\zeta:Z\to A$ such that $\varphi\circ\zeta=0$ there exists a unique $\widetilde{\zeta}:Z\to K$ making the diagram
\[\begin{tikzcd}
Z\ar[rd,dashed,swap,"\exists !\widetilde{\zeta}"]\ar[r,swap,"\zeta"]\ar[rr,bend left,"0"]&A\ar[r,swap,"\varphi"]&B\\
&K\ar[u,"\iota"]&
\end{tikzcd}\]
commute.\par
A morphism $\psi:B\to C$ is a \textbf{cokernel} of $\varphi$ if $\psi\circ\varphi=0$ and for all morphisms $\beta:B\to Z$ such that $\beta\circ\varphi=0$ there exists a unique $\widetilde{\beta}:C\to Z$ making the diagram
\[\begin{tikzcd}
&C\ar[rd,dashed,"\exists !\widetilde{\beta}"]&\\
A\ar[r,"\varphi"]\ar[rr,swap,bend right,"0"]&B\ar[r,"\beta"]\ar[u,"\psi"]&Z
\end{tikzcd}\]
commute.
\end{definition}
\begin{definition}
If a kernel of $\varphi:A\to B$ exists, then a \textbf{coimage} of $\varphi$ is a cokernel for the morphism $\ker\varphi\to A$. If a cokernel of $\varphi:A\to B$ exists, then the \textbf{image} of $\varphi$ is a kernel of the morphism $B\to\coker\varphi$.
\end{definition}
\begin{lemma}\label{ker is mono}
In any additive category, kernels are monomorphisms and cokernels are epimorphisms.
\end{lemma}
\begin{proof}
Let $\varphi:A\to B$ be a morphism in an additive category $\mathcal{A}$, and let $\ker\varphi:K\to A$ be its kernel. Let $\zeta:Z\to K$ be a morphism such that $\ker\varphi\circ\zeta=0$. Then the composition $\varphi\circ(\ker\varphi\circ\zeta)$ is $0$ and by definition of kernel, $\ker\varphi\circ\zeta$ factors uniquely through $K$:
\[\begin{tikzcd}
Z\ar[r,bend left=20,"\zeta"]\ar[r,dashed,swap,bend right=20,"\exists !"]&K\ar[r,"\ker\varphi"]&A\ar[r,"\varphi"]&B
\end{tikzcd}\]
since $\ker\varphi\circ\zeta=0=\ker\varphi\circ 0$, the uniqueness of the decomposition gives $\zeta=0$.\par
The proof that cokernels are epimorphisms is analogous.
\end{proof}
Now we relate the direct sum to kernels as follows.
\begin{proposition}
Let $\mathcal{A}$ be a preadditive category. Let $X\oplus Y$ with morphisms as in Propostion~\ref{preadd cat prod coprod} be a direct sum in $\mathcal{A}$. Then $i_1:X\to X\oplus Y$ is a kernel of $\pi_2:X\oplus Y\to Y$. Dually, $\pi_1$ is a cokernel for $i_2$.
\end{proposition}
\begin{proof}
Let $f:Z\to X\oplus Y$ be a morphism such that $\pi_2\circ f=0$. We have to show that there exists a unique morphism $g:Z\to X$ such that $f=i_1\circ g$:
\[\begin{tikzcd}
X\ar[rr,"1"]\ar[rd,"i_1"]&&X\\
Z\ar[r,"f"]&X\times Y\ar[ru,"\pi_1"]\ar[rd,"\pi_2"]&\\
Y\ar[rr,"1"]\ar[ru,"i_2"]&&Y
\end{tikzcd}\] Since $i_1\circ\pi_1+i_2\circ\pi_2$ is the identity on $X\oplus Y$ we see that
\[f=(i_1\circ\pi_1+i_2\circ\pi_2)\circ f=i_1\circ\pi_1\circ f\]
and hence $g=\pi_1\circ f$ works. Uniquess holds because $\pi_1\circ i_1$ is the identity on $X$. The proof of the second statement is dual.
\end{proof}
\begin{theorem}\label{preadditive coim im}
Let $\varphi:A\to B$ be a morphism in a preadditive category such that
the kernel, cokernel, image and coimage all exist. Then $\varphi$ can be factored uniquely
\[\begin{tikzcd}
A\ar[rrr,bend left=20pt,"\varphi"]\ar[r]&\coim\varphi\ar[r]&\im\varphi\ar[r]&B
\end{tikzcd}\]
\end{theorem}
\begin{proof}
There is a canonical morphism $\coim\varphi\to B$ because $\ker\varphi\to A\to B$ is zero,
\[\begin{tikzcd}
&&\im\varphi&\\
\ker\varphi\ar[r]&A\ar[r,"\varphi"]\ar[d]&B\ar[r]&\coker\varphi\\
&\coim\varphi\ar[ru,dashed]
\end{tikzcd}\]
The composition $\coim\varphi\to B\to\coker\varphi$ is zero, because it is the unique morphism which gives rise to the morphism $A\to B\to\coker\varphi$ which is zero. Hence $\coim\varphi\to B$ factors uniquely through $\im\varphi\to B$, which gives us the desired map.
\end{proof}
\subsection{Abelian categories}
An abelian category is a category satisfying just enough axioms so the snake lemma holds. An axiom is that the canonical map $\coim\varphi\to\im\varphi$ of Theorem~\ref{preadditive coim im} is always an isomorphism.
\begin{definition}\label{ab cat def}
A category $\mathcal{A}$ is \textbf{abelian} if it is additive, if all kernels and cokernels exist, and if the natural map $\coim\varphi\to\im\varphi$ is an isomorphism for all morphisms $\varphi$ of $\mathcal{A}$.
\end{definition}
\begin{definition}
Let $\varphi:A\to B$ be a morphism in an abelian category.
\begin{itemize}
\item[$(a)$] We say $\varphi$ is \textbf{injective} if $\ker\varphi=0$.
\item[$(b)$] We say $\varphi$ is \textbf{surjective} if $\coker\varphi=0$.
\end{itemize}
\end{definition}
\begin{proposition}\label{mono epi iff ker coker}
Let $\varphi:A\to B$ be a morphism in an abelian category. Then
\begin{itemize}
\item[$(a)$] $\varphi$ is \textbf{injective} if and only if $f$ is a monomorphism.
\item[$(b)$] $\varphi$ is \textbf{surjective} if and only if $f$ is a epimorphism.
\end{itemize}
\end{proposition}
\begin{proof}
The condition for monomorphism can be interpreted as: If $\psi:Z\to A$ is any morphism such that $\varphi\circ\psi=0$, then $\psi$ factors through $0\to A$. So $\varphi$ is a monomorphism if and only if $0\to A$ is its kernel. The same holds for epimorphism.
\end{proof}
\begin{proposition}
Let $\mathcal{A}$ be an abelian category. All finite limits and finite colimits
exist in $\mathcal{A}$.
\end{proposition}
\begin{proof}
To show that finite limits exist it suffices to show that finite products and
equalizers exist. Finite products exist by definition and the equalizer of $f,g:X\to Y$ is the kernel of $a-b$. The argument for finite colimits is similar but dual to this.
\end{proof}
\begin{example}
Let $\mathcal{A}$ be an abelian category. Pushouts and fibre products in $\mathcal{A}$ have the following simple descriptions:
\begin{itemize}
\item[$(a)$] If $f:X\to Y,g:Z\to Y$ are morphisms in $\mathcal{A}$, then we have the fibre product: $X\times_YZ=\ker((f,-g):X\oplus Z\to Y)$.
\item[$(b)$] If $f:Y\to X,g:Y\to Z$ are morphisms in $\mathcal{A}$, then we have the pushout: $X\amalg_YZ=\coker((f,-g):Y\oplus X\to Z)$.
\end{itemize}
\end{example}
\begin{lemma}\label{ker is ker of coker}
In an abelian category $\mathcal{A}$, every kernel is the kernel of its cokernel; every cokernel is the cokernel of its kernel.
\end{lemma}
\begin{proof}
Let $\varphi:K\to A$ be the kernel of some morphism $A\to B$; since $\mathcal{A}$ is abelian, $\varphi$ has a cokernel $\psi:A\to C$. The composition $K\to A\to B$ is $0$, so $A\to B$ factors
through $\psi$ by definition of cokernel:
\[\begin{tikzcd}
C\ar[rd,dashed]&\\
A\ar[r]\ar[u,"\psi"]&B\\
K\ar[u,"\varphi"]&
\end{tikzcd}\]
Now let $Z\to A$ be a morphism such that the composition $Z\to A\to C$ is the zero-morphism; then so is the composition $Z\to A\to B$. Therefore $Z\to A$ factors through a unique morphism $Z\to K$,
\[\begin{tikzcd}
&C\ar[rd,dashed]&\\
Z\ar[r]\ar[rd,dashed]&A\ar[r]\ar[u,"\psi"]&B\\
&K\ar[u,"\varphi"]&
\end{tikzcd}\]
since $\varphi$ is the kernel of $A\to B$. But this shows that $\varphi:A\to B$ satisfies the property defining the kernel of its cokernel $A\to C$, as stated.
\end{proof}
\begin{proposition}\label{mono epi iff inverse}
Let $\varphi:A\to B$ be a morphism in an abelian category $\mathcal{A}$.
\begin{itemize}
\item[$(a)$] $\varphi$ is a monomorphism if and only if $\varphi$ has a left-invere.
\item[$(b)$] $\varphi$ is a epimorphism if and only if $\varphi$ has a right-invere.
\end{itemize}
Thus $\varphi$ is an isomorphism if and only if it is a monomorphism and a epimorphism.
\end{proposition}
\begin{proof}
If $\varphi$ has a left-inverse, then clearly it is monic. Conversely, if $\varphi$ is a monomorphis, then the kernel of $\varphi$ is $0\to A$. Further, $\varphi$ is the cokernel of $0\to A$. Now consider the identity $A\to A$:
\[\begin{tikzcd}
0\ar[r]&A\ar[r,"\varphi"]\ar[d,swap,"\id_A"]&B\\
&A
\end{tikzcd}\]
Since $0\to A\to A$ is the zero morphism and $\varphi$ is the cokernel of $0\to A$, we obtain a unique morphism $\psi:B\to A$ making the diagram commute:
\[\begin{tikzcd}
0\ar[r]&A\ar[r,"\varphi"]\ar[d,swap,"\id_A"]&B\ar[ld,"\psi"]\\
&A
\end{tikzcd}\]
As $\psi\circ\varphi=\id_A$, this shows that $\varphi$ has a right-inverse. The part $(b)$ can be done similarly.
\end{proof}
\subsection{Exact sequence in Abelian category}
\begin{definition}
Let $\mathcal{A}$ be an additive category. We say a sequence of morphisms
\[\begin{tikzcd}
\cdots\ar[r]&A\ar[r,"\varphi"]&B\ar[r,"\psi"]&C\ar[r]&\cdots
\end{tikzcd}\]
in $\mathcal{A}$ is a \textbf{complex} if the composition of any two arrows is zero. If $\mathcal{A}$ is abelian then we say a sequence as above is \textbf{exact at $\bm{B}$} if $\im\psi=\ker\varphi$. We say it is exact if it is exact at every object. A \textbf{short exact sequence} is an exact complex of the form
\[\begin{tikzcd}
0\ar[r]&A\ar[r,"\varphi"]&B\ar[r,"\psi"]&C\ar[r]&0
\end{tikzcd}\]
\end{definition}
\begin{proposition}
Let $\mathcal{A}$ be an abelian category. Let $0\to M_1\to M_2\to M_3\to0$ be a complex of $\mathcal{A}$.
\begin{itemize}
\item[$(a)$] $M_1\to M_2\to M_3\to 0$ is exact if and only if
\[\begin{tikzcd}
0\ar[r]&\Hom_{\mathcal{A}}(M_3,N)\ar[r]&\Hom_{\mathcal{A}}(M_2,N)\ar[r]&\Hom_{\mathcal{A}}(M_1,N)
\end{tikzcd}\]
is an exact sequence of abelian groups for all objects $N$ of $\mathcal{A}$.
\item[$(b)$] $0\to M_1\to M_2\to M_3$ is exact if and only if
\[\begin{tikzcd}
\Hom_{\mathcal{A}}(N,M_1)\ar[r]&\Hom_{\mathcal{A}}(N,M_2)\ar[r]&\Hom_{\mathcal{A}}(N,M_3)\ar[r]&0
\end{tikzcd}\]
is an exact sequence of abelian groups for all objects $N$ of $\mathcal{A}$.
\end{itemize}
\end{proposition}
\begin{example}\label{fibered diagram}
For a slightly more interesting example, consider a diagram
\[\begin{tikzcd}
D\ar[d,swap,"\psi'"]\ar[r,"\varphi'"]&B\ar[d,"\psi"]\\
A\ar[r,swap,"\varphi"]&C
\end{tikzcd}\]
and the associated sequence
\[\begin{tikzcd}
D\ar[r,"{(\psi',\varphi')}"]&A\oplus B\ar[r,"{(\varphi,-\psi)}"]&C
\end{tikzcd}\]
obtained by letting $A\oplus B$ play both roles of product and coproduct. Then
\begin{itemize}
\item the diagram is commutative if and only if this sequence is a complex;
\item the sequence obtained by adding a $0$ to the left,
\[\begin{tikzcd}
0\ar[r]&D\ar[r]&A\oplus B\ar[r]&C
\end{tikzcd}\]
is exact if and only if $D$ may be identified with the fibered product $A\times_{C}B$. If this holds, we say the diagram is \textbf{cartesian}.
\item likewise, the sequence
\[\begin{tikzcd}
D\ar[r]&A\oplus B\ar[r]&C\ar[r]&0
\end{tikzcd}\]
is exact if and only if $C$ may be identified with the fibered coproduct $A\amalg_DB$. If this holds, we say the diagram is \textbf{cocartesian}.
\end{itemize}
\end{example}
\begin{lemma}\label{pull bak lem}
Let $\mathcal{A}$ be an abelian category. Let
\[\begin{tikzcd}
D\ar[d,swap,"\psi'"]\ar[r,"\varphi'"]&B\ar[d,"\psi"]\\
A\ar[r,swap,"\varphi"]&C
\end{tikzcd}\]
be a commutative diagram.
\begin{itemize}
\item[$(a)$] If the diagram is cartesian, then the morphism $\ker\varphi'\to\ker\varphi$ induced by $\psi'$ is an isomorphism.
\item[$(b)$] If the diagram is cocartesian, then the morphism $\coker\varphi'\to\coker\varphi$ induced by $\psi$ is an isomorphism.
\end{itemize}
\end{lemma}
\begin{proof}
Suppose the diagram is cartesian. Let $\epsilon:\ker\varphi'\to\ker\varphi$ be induced by $\psi'$. Let $i:\ker\varphi\to A$ and $j:\ker\varphi'\to D$ be the canonical injections. Consider the map $\alpha:\ker\varphi\to D$ determined by the morphisms $(i,0)$:
\[\psi'\circ\alpha=i,\quad \varphi'\circ\alpha=0.\]
Then there is an induced morphism $\gamma:\ker\varphi\to\ker\varphi'$:
\[\begin{tikzcd}
\ker\varphi'\ar[r,"j"]\ar[d,"\epsilon"]&D\ar[d,swap,"\psi'"]\ar[r,"\varphi'"]&B\ar[d,"\psi"]\\
\ker\varphi\ar[r,"i"]\ar[ru,"\alpha"]\ar[u,bend left=30pt,"\gamma"]&A\ar[r,swap,"\varphi"]&C
\end{tikzcd}\]
It follows that
\[\psi'\circ j\circ\gamma\circ\epsilon=\psi'\circ\alpha\circ\epsilon=i\circ\epsilon=\psi'\circ j\And\varphi'\circ j\circ\gamma\circ\epsilon=\varphi'\circ\alpha\circ\epsilon=0=\psi'\circ j.\]
By the universal property of pull back, we claim $j\circ\gamma\circ\epsilon=j$. Since $j$ is a monomorphism, this means $\gamma\circ\eps=\id_{\ker\varphi'}$.\par
Furthermore, we have 
\[i\circ\epsilon\circ\gamma=\psi'\circ j\circ\gamma=\psi'\circ\alpha=i.\]
Since $i$ is a monomorphism this implies $\epsilon\circ\gamma=\id_{\ker\varphi}$. This proves $(a)$. Now, $(b)$ follows by duality.
\end{proof}
\begin{lemma}
Let $\mathcal{A}$ be an abelian category. Let
\[\begin{tikzcd}
D\ar[d,swap,"\psi'"]\ar[r,"\varphi'"]&B\ar[d,"\psi"]\\
A\ar[r,swap,"\varphi"]&C
\end{tikzcd}\]
be a commutative diagram.
\begin{itemize}
\item[$(a)$] If the diagram is cartesian and $\varphi$ is an epimorphism, then the diagram is cocartesian and $\varphi'$ is an epimorphism.
\item[$(b)$] If the diagram is cocartesian and $\varphi'$ is an monomorphism, then the diagram is cartesian and $\varphi$ is an epimorphism.
\end{itemize}
\end{lemma}
\begin{proof}
Suppose the diagram is cartesian and $\varphi$ is an epimorphism. Let $\alpha=(\psi',\varphi'):D\to A\oplus B$ and let $\beta=(\varphi,-\psi):A\oplus B\to C$. As $\varphi$ is an epimorphism, $\alpha$ is an epimorphism, too. Therefore by Example~\ref{fibered diagram} the diagram is cocartesian. Finally, $\varphi'$ is an epimorphism by Lemma~\ref{pull bak lem}. This proves $(1)$, and $(2)$ follows by duality.
\end{proof}
\begin{corollary}
Let $\mathcal{A}$ be an abelian category.
\begin{itemize}
\item[$(a)$] If $X\to Y$ is surjective, then for every $Z\to Y$ the projection $X\times_YZ\to Z$ is surjective.
\item[$(b)$] If $X\to Y$ is injective, then for every $X\to Z$ the morphism $Z\to Z\amalg_XY$ is injective.
\end{itemize}
\end{corollary}
\begin{lemma}\label{exact iff}
Let \begin{tikzcd}X'\ar[r,"f"]&X\ar[r,"g"]&X''\end{tikzcd} be a complex. Then the conditions below are equivalent:
\begin{itemize}
\item[$(\rmnum{1})$] the complex \begin{tikzcd}X'\ar[r,"f"]&X\ar[r,"g"]&X''\end{tikzcd} is exact.
\item[$(\rmnum{2})$] the induced morphism $X'\to\ker g$ is an epimorphism.
\item[$(\rmnum{3})$] for any morphism $h:S\to X$ such that $g\circ h=0$, there exist an epimorphism $f':S'\twoheadrightarrow S$ and a commutative diagram
\[\begin{tikzcd}
S'\ar[d]\ar[r,twoheadrightarrow,"f'"]&S\ar[d,"h"]\ar[rd,"0"]&\\
X'\ar[r,"f"]&X\ar[r,"g"]&X''
\end{tikzcd}\]
\end{itemize}
\end{lemma}
\begin{proof}
$(\rmnum{1})\iff(\rmnum{2})$: the exactness is saying $\ker g=\im f$. If $X'\to\ker g$ is epic, by Exercise~\ref{epi mono decop}, $\ker g=\im f$ as needed. Conversely, if $\ker g=\in f$, by Lemma~\ref{im decomp}, $X'\to\ker g$ is epic.\par
$(\rmnum{1})\Rightarrow(\rmnum{3})$: It is enough to choose $X'\times_{\ker g}S$ as $S'$. Since $X'\to\ker g$ is an epimorphism, $S'\to S$ is an epimorphism by Lemma~\ref{pull bak lem}.\par
$(\rmnum{3})\Rightarrow(\rmnum{2})$: Choose $S=\ker g$. Then the diagram becomes
\[\begin{tikzcd}
S'\ar[d]\ar[r,twoheadrightarrow,"f'"]&\ker g\ar[d]\ar[rd,"0"]&\\
X'\ar[ru,dashed]\ar[r,"f"]&X\ar[r,"g"]&X''
\end{tikzcd}\]
since $g\circ f=0$, by the universal property of $\ker g$, there is a unique morphism $X'\to\ker g$. It follows that the composition $S'\to X'\to\ker g$ is an epimorphism. Hence $X'\to\ker g$ is an epimorphism.
\end{proof}
\subsection{Exercise}
\begin{exercise}\label{epi mono decop}
Let $\varphi:A\to B$ be a morphism in an abelian category, and assume $\varphi$ decomposes as an epimorphism $\pi$ followed by a monomorphism $i$:
\[\begin{tikzcd}
A\ar[rr,swap,bend right,"\varphi"]\ar[r,twoheadrightarrow,"\pi"]&C\ar[r,rightarrowtail,"i"]&B
\end{tikzcd}\]
Prove that necessarily $\pi=\coim\varphi$ and $i=\im\varphi$.
\end{exercise}
\begin{proof}
From the universal property of image and coimage, we have the following commutative diagram:
\[\begin{tikzcd}
&\coim\varphi\ar[rd,bend left,rightarrowtail]&\\
A\ar[ru,twoheadrightarrow,bend left]\ar[rd,twoheadrightarrow,bend right]\ar[r,twoheadrightarrow,"\pi"]&C\ar[u,dashed,"\exists!\nu"]\ar[r,rightarrowtail,"i"]&B\\
&\im\varphi\ar[u,dashed,"\exists!\mu"]\ar[ru,bend right,rightarrowtail]&
\end{tikzcd}\]
By simple observation, we find $\mu$ and $\nu$ are both monomorphic and epimorphic, hence are isomorphisms.
\end{proof}