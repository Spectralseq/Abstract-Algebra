\chapter{Category}
\section{Functors}
\subsection{Definition of Functors}
Let $\mathcal{C},\mathcal{D}$ be two categories. A \textbf{covariant functor}
\[F:\mathcal{C}\to\mathcal{D}\]
is an assignment of an object $F(A)\in\Ob(\mathcal{D})$ for every $A\in\Ob(\mathcal{C})$ and of a function
\[\Mor_{\mathcal{C}}(A,B)\to\Mor_{\mathcal{D}}(F(A),F(B))\]
for every pair of objects $A,B$ in $\mathcal{C}$ such that $\forall f\in\Mor_{\mathcal{C}}(A,B),g\in\Mor_{\mathcal{C}}(B,C)$ we have
\[F(\id_A)=\id_{F_{A}},\quad F(g\circ f)=F(g)\circ F(f).\]

\begin{example}
If $R$ is a ring, we have denoted by $R^{\times}$ the group of units in $R$; every ring homomorphism $R\to S$ induces a group homomorphism $R^{\times}\to S^{\times}$, and this assignment is compatible with compositions; therefore this operation defines a covariant functor $\mathbf{Ring}\to\mathbf{Ab}$.
\end{example}
\subsection{Equivalence of Category}
The structure of an object in a category is adequately carried by its isomorphism class, and a natural notion of equivalence of categories should aim at matching isomorphism classes, rather than individual objects. The \textit{morphisms} are a more essential piece of information; the quality of a functor is first of all measured on how it acts on morphisms.
\begin{definition}
Let $\mathcal{C},\mathcal{D}$ be two categories. Let $F:\mathcal{C}\to\mathcal{D}$ be a covariant functor.
\begin{enumerate}
\item[(a)] $F$ is \textbf{faithful} if for all objects $A,B$ of $\mathcal{C}$, the induced function
\[\Mor_{\mathcal{C}}(A,B)\to\Mor_{\mathcal{D}}(F(A),F(D))\]
is injective.
\item[(b)] $F$ is \textbf{full} if this function is surjective, for all objects $A,B$.
\item[(c)] $F$ is called essentially surjective if for any object $B\in\Ob(\mathcal{B})$ there exists an object $A\in\Ob(\mathcal{A})$ such that $F(A)$ is isomorphic to $B$ in $\mathcal{B}$.
\end{enumerate}
\end{definition}
\begin{lemma}\label{fully faithful fuctor}
Let $F:\mathcal{C}\to\mathcal{D}$ be a fully faithful functor. If $A,B$ are objects in $\mathcal{C}$, then $A\cong B$ in $\mathcal{C}$ if and only if $F(A)\cong F(B)$ in $\mathcal{D}$.
\end{lemma}
\begin{proof}
Assume $F$ is covariant. Since we have $F(f\circ g)=F(f)\circ G(g)$, if $A\cong B$, we have $F(A)\cong\mathscr{B}$. Conversely, if $F(A)\cong F(B)$, assume $f$, $g$ are isomorphisms between them with $g=f^{-1}$. Since $F$ is full, there are two morphisms $f',g'$ such that $F(f')=f,F(g')=g$. Then we have
\[F(f'\circ g')=f\circ g=\mathrm{id},\quad F(g'\circ f')=g\circ f=\id\]
Since $F(\id)=\id$, and $F$ is faithful, we get $f'\circ g'=g'\circ f'=\id$, so $A\cong B$ follows.
\end{proof}
\begin{definition}
Let $\mathcal{C}, \mathcal{D}$ be categories, and let $F$, $G$ be functors $\mathcal{C}\to\mathcal{D}$. A \textbf{natural transformation} $\nu:F\to G$ is the datum of a morphism $\nu_X:F(X)\to G(X)$ in $\mathcal{D}$ for every object $X$ in $\mathcal{C}$, such that $\forall\alpha:X\to Y$ in $\mathcal{C}$ the diagram
\[\begin{tikzcd}
F(X)\ar[r,"\nu_X"]\ar[d,swap,"F(\alpha)"]&G(X)\ar[d,"G(\alpha)"]\\
F(Y)\ar[r,"\nu_Y"]&G(Y)
\end{tikzcd}\]
commutes. A \textbf{natural isomorphism} is a natural transformation $\mu$ such that $\mu_X$ is an isomorphism for every $X$.
\end{definition}
A natural transformation is often written as
\[\begin{tikzcd}
\mathcal{C}\ar[r,bend left=50,"F",""'{name=F,below}]\ar[r,bend right=50,swap,"G",""'{name=G,above}]&\mathcal{D}
\arrow[Rightarrow,from=F,to=G,"\nu"]
\end{tikzcd}\]
In addition, given a morphism of functors $\mu:F\to G$ and a morphism of functors $\nu:\mathscr{E}\to F$ then the composition $\nu\circ\mu$ is defined by the rule
\[(\nu\circ\mu)_X:=\nu_X\circ\mu_X.\]
for $X\in\Ob(\mathcal{A})$. It is easy to verify that this is indeed a morphism of functors from $\mathscr{E}$ to $G$. In this way, given categories $\mathcal{A}$ and $\mathcal{B}$ we obtain a new category, namely
the category of functors between $\mathcal{A}$ and $\mathcal{B}$.
\begin{definition}
An \textbf{equivalence of categories} $F:\mathcal{A}\to\mathcal{B}$ is a functor such that there exists a functor $G:\mathcal{B}\to\mathcal{A}$ such that the compositions $F\circ G$ and $G\circ F$ are isomorphic to the identity functors $\id_{\mathcal{B}}$, respectively $\id_{\mathcal{A}}$. In this case we say that $G$ is a \textbf{quasi-inverse} to $F$.
\end{definition}
\begin{proposition}\label{equiv category iff}
Let $F:\mathcal{A}\to\mathcal{B}$ be a fully faithful functor. Suppose for every $X\in\Ob(\mathcal{B})$ we are given an object $G(X)$ of $A$ and an isomorphism $i_X:X\to F(G(X))$. Then there is a unique functor $G:\mathcal{B}\to\mathcal{A}$ such that $G$ extends the rule on objects, and the isomorphisms $i_X$ define an isomorphism of functors $\id_{\mathcal{B}}\to F\circ G$. Moreover, $G$ and $F$ are quasi-inverse equivalences of categories.
\end{proposition}
\begin{proof}
The action of $G$ on objects is defined. For $X,Y\in\Ob(\mathcal{B})$ and $f\in\Mor_{\mathcal{B}}(X,Y)$, we have the diagram
\[\begin{tikzcd}
X\ar[r,"f"]\ar[d,"i_X"]&Y\ar[d,"i_Y"]\\
F(G(X))\ar[r,"i_Y\circ f\circ i_X^{-1}"]&F(G(Y))\\
G(X)\ar[r,dashed]\ar[u,"F"]&G(Y)\ar[u,"F"]
\end{tikzcd}\]
where the dashed map is induced by the bijection
\[\Mor_{\mathcal{A}}(X,Y)\cong\Mor_{\mathcal{B}}(F(X),F(Y))\]
by the functoriality, this defines a functor $G:\mathcal{B}\to\mathcal{A}$. Moreover, the upper half of the diagram means $i_X$ define an isomorphism of functors $\id_{\mathcal{B}}\to F\circ G$.\par
For $X\in\Ob(\mathcal{A})$, we have an isomorphism $i_{F(X)}:F(X)\to F\circ G\circ F(X)$. Since $F$ is full and faithful, there is an isomorphism $\mu_{X}:X\to G\circ F(X)$ by Lemma~\ref{fully faithful fuctor}. The naturality of $\mu_X$ follows from that of $i_{F(X)}$ and the faithfulness of $F$. Thus $\mu$ is an isomorphism $\id_{\mathcal{A}}\to G\circ F$.
\end{proof}
\begin{corollary}
A functor is an equivalence of categories if and only if it is both fully faithful and essentially surjective.
\end{corollary}
\begin{proof}
Let $F:\mathcal{A}\to\mathcal{B}$ be essentially surjective and fully faithful. As by convention all categories are small and as $F$ is essentially surjective we can, using the axiom of choice, choose for every $X\in\Ob(\mathcal{B})$ an object $G(X)$ of $\mathcal{A}$ and an isomorphism $i_X:X\to F(G(X))$. Then we apply Proposition~\ref{equiv category iff}.
\end{proof}
\subsection{Yoneda's Lemma}
\begin{definition}
Given a category $\mathcal{C}$ the opposite category $\mathcal{C}^{op}$ is the category with the same objects as $\mathcal{C}$ but all morphisms reversed.
\end{definition}
\begin{definition}
Let $\mathcal{A},\mathcal{B}$ be categories. A contravariant functor $F$ from $\mathcal{A}$ to $\mathcal{B}$ is a functor $\mathcal{A}^{op}\to\mathcal{B}$.\par
Concretely, a contravariant functor $F$ satisfies the property that, given another morphism $f:X\to Y$ and $g:Y\to Z$, we have 
\[F(g\circ f)=F(g)\circ F(f)\]
as morphism from $F(Z)$ to $F(X)$.
\end{definition}
\begin{definition}
Let $\mathcal{C}$ be a category. A \textbf{presheaf of sets on $\mathcal{C}$} or simply a presheaf is a contravariant functor $F$ from $\mathcal{C}$ to $\mathbf{Set}$. The category of presheaves is denoted $\mathbf{Psh}(\mathcal{C})$.
\end{definition}
\begin{example}[\textbf{Functor of points}]
For any $U\in\Ob(\mathcal{C})$ there is a contravariant functor
\[h_X:\mathcal{C}\to\mathbf{Set},\quad Y\mapsto\Mor_{\mathcal{C}}(Y,X).\]
In other words $h_X$ is a presheaf. We will always denote this presheaf $h_X:\mathcal{C}^{op}\to\mathbf{Set}$. It is called the \textbf{representable presheaf} associated to $X$.\par
Note that given a morphism $s:X\to Y$ in $\mathcal{C}$ we get a corresponding natural transformation of functors $h(s):h_X\to h_Y$ defined simply by composing with the morphism $U\to V$. It is trivial to see that this turns composition of morphisms in $\mathcal{C}$ into composition of transformations of functors. In other words we get a functor
\[h:\mathcal{C}\to\Fun(\mathcal{C}^{op},\mathbf{Set})=:\widehat{\mathcal{C}}.\]
\end{example}
\begin{lemma}[\textbf{Yoneda lemma}]
The functor $h$ is fully faithful. More generally, given any contravariant functor $F$ and any object $X$ of $\mathcal{C}$ we have a natural bijection
\[\Mor_{\widehat{\mathcal{C}}}(h_X,F)\to F(X),\quad \alpha\mapsto \alpha_X(\id_X).\]
\end{lemma}
\begin{proof}
An element $f\in h_X(Y)=\Mor_{\mathcal{C}}(Y,X)$ can be viewed as a morphism $f^*:\Mor_{\mathcal{C}}(X,X)\to\Mor_{\mathcal{C}}(Y,X)$. Note that $f^*(\id_X)=f$, so if there is a natural transformation $\alpha:h_X\to F$, then from the diagram
\[\begin{tikzcd}
\Mor_{\mathcal{C}}(X,X)\ar[r,"\alpha_X"]\ar[d,"f^*"]&F(X)\ar[d,"F(f)"]\\
\Mor_{\mathcal{C}}(Y,X)\ar[r,"\alpha_X"]&F(Y)
\end{tikzcd}\]
we obtain
\[\alpha_Y(f)=\alpha_Y\big(f^*(\id_X)\big)=F(f)\big(\alpha_X(\id_X)\big).\]
That is, $\alpha$ is simply determined by $\alpha_X(\id_X)$. Conversely, given $\xi\in F(X)$, we can define a natural transformation by the formula above:
\[\beta_Y:h_X(Y)\to F(Y),\quad \beta_Y(f)=F(f)(\xi).\]
It follows that these two map are inverse of each other.
\end{proof}
\begin{definition}
A contravariant functor $F:\mathcal{C}\to\mathbf{Set}$ is said to be \textbf{representable} if it is isomorphic to the functor of points $h_X$ for some object $X$ of $\mathcal{C}$.
\end{definition}
Let $\mathcal{C}$ be a category and let $F:\mathcal{C}^{op}\to\mathbf{Set}$ be a representable functor. Choose an object $X$ of $\mathcal{C}$ and an isomorphism $\alpha:h_X\to F$. The Yoneda lemma guarantees that the pair $(X,\alpha)$ is unique up to unique isomorphism. The object $X$ is called an object representing $F$.
\subsection{Limits and colimits}
The various universal properties encountered along the way are all particular cases of the notion of categorical \textbf{limit}, which is worth mentioning explicitly. Let $F:I\to C$ be a \textit{covariant functor}, where one thinks of $\mathcal{I}$ as a category of indices. The \textbf{limit} of $F$ is an object $L$ of $\mathcal{C}$, endowed with morphisms $\lambda_I:L\to F(I)$ for all objects $I$ of $\mathcal{I}$, satisfying the
following properties:
\begin{enumerate}
\item If $\alpha:I\to J$ is a morphism in $\mathcal{I}$, then $\lambda_J=F(\alpha)\circ\lambda_I$:
\[\begin{tikzcd}
&L\ar[ld,swap,"\lambda_I"]\ar[rd,"\lambda_J"]&\\
F(I)\ar[rr,swap,"F(\alpha)"]&&F(J)
\end{tikzcd}\]
\item $L$ is final with respect to this property: that is, if $M$ is another object, endowed with morphisms $\mu_I$, also satisfying the previous requirement, then there exists a unique morphism $M\to L$ making all relevant diagrams commute
\end{enumerate}
\begin{example}[\textbf{Products}]
Let $\mathcal{I}$ be the discrete category consisting of two objects $\bm{1}$, $\bm{2}$, with only identity morphisms, and let $\mathscr{A}$ be a functor from $\mathcal{I}$ to any category $\mathcal{C}$; let $A_1=\mathscr{A}(\bm{1})$, $A_2=\mathscr{A}(\bm{2})$ be the two objects of $\mathcal{C}$ indexed by $\mathcal{I}$. Then $\llim\mathscr{A}$ is nothing but the product of $A_1$ and $A_2$ in $\mathcal{C}$: a limit exists if and only if a product of $A_1$ and $A_2$ exists in $\mathcal{C}$.\par 
We can similarly define the product of any $($possibly infinite$)$ family of objects in a category as the limit over the corresponding discrete indexing category, provided of course that this limit exists.
\end{example}
The limit notion is a little more interesting if the indexing category $\mathcal{I}$ carries more structure.
\begin{example}[\textbf{Equalizers and kernels}]
Let $\mathcal{I}$ again be a category with two objects $\bm{1}$, $\bm{2}$, but assume that morphisms look like this:
\[\begin{tikzcd}
\bm{1}\ar[in=210, out=150,looseness=8]\ar[r,bend left,"\alpha"]\ar[r,swap,bend right,"\beta"]&\bm{2}\ar[in=-30, out=30,looseness=8]
\end{tikzcd}\]
That is, add to the discrete category two parallel morphisms $\alpha,\beta$ from one of the objects to the other. A functor $\mathscr{K}:\mathcal{I}\to\mathcal{C}$ amounts to the choice of two objects $A_1$, $A_2$ in $\mathcal{C}$ and two parallel morphisms between them. Limits of such functors are called \textbf{equalizers}. For a concrete example, assume $\mathcal{C}=R$-$\mathbf{Mod}$ is the category of $R$-modules for some ring $R$; let $\varphi:A_2\to A_1$ be a homomorphism, and choose $\mathscr{K}$ as above, with $\mathscr{K}(\alpha)=\varphi$ and $\mathscr{K}(\beta)=$ the zero-morphism. Then $\llim\mathscr{K}$ is nothing but the kernel of $\varphi$.
\end{example}
\begin{example}[\textbf{Limits over chains}]
In another typical situation, $\mathcal{I}$ may consist of a totally ordered set, for example:
\[\begin{tikzcd}
\cdots\ar[r]&\bm{4}\ar[r]&\bm{3}\ar[r]&\bm{2}\ar[r]&\bm{1}
\end{tikzcd}\]
$($that is, the objects are $\bm{i}$, for all positive integers $i$, and there is a unique morphism
$\bm{i}\to \bm{j}$ whenever $i\geq j$; we are only drawing the morphisms $\bm{j}+\bm{1}\to \bm{j}$$)$. Choosing $F:\mathcal{I}\to\mathcal{C}$ is equivalent to choosing objects $A_i$ of $\mathcal{C}$ for all positive integers $i$ and morphisms $\varphi_{ji}:A_i\to A_j$ for all $i\geq j$, with the requirement that $\varphi_{ii}=1_{A_i}$, and $\varphi_{kj}\circ\varphi_{ji}=\varphi_{ki}$ for all $i\geq j\geq k$. That is, the choice of $F$ amounts to the choice of a diagram
\[\begin{tikzcd}
\cdots\ar[r,"\varphi_{45}"]&A_4\ar[r,"\varphi_{34}"]&A_3\ar[r,"\varphi_{23}"]&A_2\ar[r,"\varphi_{12}"]&A_1
\end{tikzcd}\]
in $\mathcal{C}$. An inverse limit $\llim F$ $($which may also be denoted $\llim_iA_i$, when the morphisms $\varphi_{ji}$ are evident from the context$)$ is then an object $A$ endowed with morphisms $\varphi_i:A\to A_i$ such that the whole diagram
\[\begin{tikzcd}
&&&&A\ar[lllldd,swap,dashed,"\cdots"]\ar[lldd,"\varphi_4"]\ar[dd,"\varphi_3" description]\ar[rrdd,swap,"\varphi_2"]\ar[rrrrdd,"\varphi_1"]&&&&\\
&&&&\\
\cdots\ar[rr,"\varphi_{45}"]&&A_4\ar[rr,"\varphi_{34}"]&&A_3\ar[rr,"\varphi_{23}"]&&A_2\ar[rr,"\varphi_{12}"]&&A_1
\end{tikzcd}\]
commutes and such that any other object satisfying this requirement factors uniquely through $A$.\par
Such limits exist in many standard situations. For example, let $C=R$-$\mathbf{Mod}$ be
the category of left-modules over a fixed ring $R$, and let $A_i$, $\varphi_{ji}$ be as above.
\begin{proposition}
The limit $\llim_iA_i$ exists in $R$-$\mathbf{Mod}$.
\end{proposition}
\begin{proof}
The product $\prod_iA_i$ consists of arbitrary sequences $(a_i)_{i>0}$ of elements $a_i\in A_i$. Say that a sequence $(a_i)_{i>0}$ is \textit{coherent} if for all $i>0$ we have $a_i=\varphi_{i,i+1}(a_{i+1})$. Coherent sequences form an $R$-submodule $A$ of $\prod_iA_i$; the canonical projections restrict to $R$-module homomorphisms $\varphi_i:A\to A_i$. The reader will check that $A$ is a limit $\llim_iA_i$.
\end{proof}
This example easily generalizes to families indexed by more general posets.
\end{example}
The dual notion to limit is the \textbf{colimit} of a functor $F:\mathcal{I}\to\mathcal{C}$. The colimit is an object $C$ of $\mathcal{C}$, endowed with morphisms $\gamma_I:F(I)\to\mathcal{C}$ for all objects $I$ of $\mathcal{I}$, such that $\gamma_I=\gamma_J\circ F(\alpha)$ for all $\alpha:I\to J$ and that $C$ is \textit{initial} with respect to this requirement.\par
\begin{example}

For a typical situation consider again the case of a totally ordered set $\mathcal{I}$, for example:
\[\begin{tikzcd}
\bm{1}\ar[r]&\bm{2}\ar[r]&\bm{3}\ar[r]&\bm{4}\ar[r]&\cdots
\end{tikzcd}\]
A functor $F:\mathcal{I}\to\mathcal{C}$ consists of the choice of objects and morphisms
\[\begin{tikzcd}
A_1\ar[r,"\psi_{12}"]&A_2\ar[r,"\psi_{23}"]&A_3\ar[r,"\psi_{34}"]&A_4\ar[r,"\psi_{45}"]&\cdots
\end{tikzcd}\]
and the direct limit $\rlim_iA_i$ will be an object $A$ with morphisms $\psi_i:A_i\to A$ such
that the diagram
\[\begin{tikzcd}
A_1\ar[rrrrdd,"\psi_1"]\ar[rr,"\psi_{12}"]&&A_2\ar[rr,"\psi_{23}"]\ar[rrdd,"\psi_2"]&&A_3\ar[dd,"\psi_3" description]\ar[rr,"\psi_{34}"]&&A_4\ar[rr,"\psi_{45}"]\ar[lldd,swap,"\psi_4"]&&\cdots\ar[lllldd,dashed,swap,"\cdots"]\\
&&&&\\
&&&&A&&&&
\end{tikzcd}\]
commutes and such that $A$ is initial with respect to this requirement.
\end{example}
\begin{example}
If $\mathcal{C }=\mathbf{Set}$ and all the $\psi_{ij}$ are injective, we are talking about a nested sequence of sets:
\[A_1\sub A_2\sub A_3\sub\cdots\]
the direct limit of this sequence would be the infinite union $\bigcup_iA_i$.
\end{example}
\subsection{Exact functors}
\begin{definition}
Let $F:\mathcal{A}\to\mathcal{B}$ be a functor
\begin{enumerate}
\item[(a)] Suppose all finite limits exist in $\mathcal{A}$. We say $F$ is \textbf{left exact} if it commutes with all finite limits.
\item[(b)] Suppose all finite colimits exist in $\mathcal{A}$. We say $F$ is \textbf{right exact} if it commutes with all finite colimits.
\item[(c)] We say $F$ is \textbf{exact} if it is both left and right exact.
\end{enumerate}
\end{definition}
\begin{proposition}
Let $F:\mathcal{A}\to\mathcal{B}$ be a functor. Suppose all finite limits exist in $\mathcal{A}$. The following are equivalent:
\begin{enumerate}
\item[(a)] $F$ is left exact,
\item[(b)] $F$ commutes with finite products and equalizers.
\item[(c)] $F$ transforms a final object of $\mathcal{A}$ into a final object of $\mathcal{B}$, and commutes with fibre products.
\end{enumerate}
\end{proposition}
\subsection{Adjunction}
\begin{definition}
Let $\mathcal{C},\mathcal{D}$ be categories, and let $F:\mathcal{C}\to\mathcal{D}$, $G:\mathcal{D}\to\mathcal{C}$ be functors. We say that $F$ and $G$ are \textbf{adjoint} $($and we say that $G$ is right-adjoint to $F$ and $F$ is left-adjoint to $G$$)$ if there are natural isomorphisms
\[\tau_{XY}:\Mor_{\mathcal{C}}(X,G(Y))\stackrel{\sim}{\longrightarrow}\Mor_{\mathcal{D}}(F(X),Y)\]
for all objects $X$ of $\mathcal{C}$ and $Y$ of $\mathcal{D}$. More precisely, there should be a natural isomorphism of bifunctors \[\mathcal{C}^{op}\times\mathcal{D}\to\mathbf{Set}:\Mor_{\mathcal{C}}(-,G(-))\stackrel{\sim}{\to}\Mor_{\mathcal{D}}(F(-),-)\]
\end{definition}
\begin{proposition}
For each $Y$ there is a map $\eta_Y:FG(Y)\to Y$ so that for any for any $f:X\to G(Y)$, the corresponding map $\tau_{XY}(f):F(X)\to Y$ is given by the composition
\[\begin{tikzcd}
F(X)\ar[r,"F(f)"]&FG(Y)\ar[r,"\eta_Y"]&Y
\end{tikzcd}\]
Similarly, there is a map $\theta_X:X\to GF(X)$ for each $X$ so that $g:F(X)\to Y$, the corresponding $\tau^{-1}_{XY}(g):X\to G(Y)$ is given by the composition
\[\begin{tikzcd}
X\ar[r,"\theta_X"]&GF(Y)\ar[r,"G(g)"]&G(Y)
\end{tikzcd}\]
So the information of $\tau_{XY}$ is the same as these two maps.
\end{proposition}
\begin{proof}
We deal with the first case. Let $f:X\to G(Y)$ be a map, consider the follwing diagram
\[\begin{tikzcd}
\Mor_{\mathcal{C}}(X,G(Y))\ar[r,"\tau_{XY}"]&\Mor_{\mathcal{D}}(F(X),Y)\\
\Mor_{\mathcal{C}}(G(Y),G(Y))\ar[r,"\tau_{G(Y)Y}"]\ar[u,"f^*"]&\Mor_{\mathcal{D}}(FG(Y),Y)\ar[u,"F(f)^*"]
\end{tikzcd}\]
Set $\eta_Y$ to be the image of $\mathrm{id}_{G(Y)}$ under $\tau_{G(Y)Y}$ we get the claim. The second can be done similarly.
\end{proof}
\begin{proposition}
Let $F$ be a left adjoint to $G$. Then
\begin{enumerate}
\item[(a)] $F$ is fully faithful if and only if $\id_{\mathcal{C}}\cong G\circ F$.
\item[(b)] $G$ is fully faithful if and only if $F\circ G\cong\id_{\mathcal{D}}$.
\end{enumerate}
\end{proposition}
\begin{proof}
Assume $F$ is fully faithful. We have to show the adjunction map $X\to G\circ F(X)$ is an isomorphism. Let $X'\to G\circ F(X)$ be any morphism. By adjointness this corresponds to a morphism $F(X')\to F(X)$. By fully faithfulness of $F$ this corresponds to a morphism $X'\to X$. Thus we see that $X\mapsto F\circ G(X)$ defines a bijection \[\Mor_{\mathcal{C}}(X',X)\to\Mor(X',GF(X))\]
Hence it is an isomorphism. Conversely, if $\id_{\mathcal{C}}\cong G\circ F$ then $F$ has to be fully faithful, as $G$ defines an left-inverse on morphism sets. The other case is the dual part.
\end{proof}
\begin{proposition}
Let $F:\mathcal{C}\to\mathcal{D}$ be a functor between categories. If for each $Y\in\Ob(\mathcal{D})$ the functor $\Mor_{\mathcal{D}}(F(-),Y)$ is representable, then $F$ has a right adjoint.
\end{proposition}
\begin{proof}
For each $Y$ we choose an object $G(Y)$ and an isomorphism 
\[\Mor_{\mathcal{C}}(-,G(Y)) \stackrel{\sim}{\to} \Mor_{\mathcal{D}}(F(-),Y)\]
of functors. By Yoneda's lemma for any morphism $g:Y\to Y'$ the transformation of functors
\[\begin{tikzcd}
\Mor_{\mathcal{C}}(-,G(Y))\ar[r,"\sim"]&\Mor_{\mathcal{D}}(F(-),Y)\ar[r]&\Mor_{\mathcal{D}}(F(-),Y')\ar[r,"\sim"]&\Mor_{\mathcal{C}}(-,G(Y'))
\end{tikzcd}\]
corresponds to a unique morphism $G(g):G(Y)\to G(Y')$. The functoriality of $G$ comes from that of $F$.
\end{proof}
\begin{example}
The construction of the free group on a given set is concocted so that giving a set-function from a set $A$ to a group $G$ is the same as giving a group homomorphism from $F(A)$ to $G$. What this really means is that for all sets $A$ and all groups $G$ there are natural identifications
\[\Mor_{\mathbf{Set}}(A,S(G))\stackrel{\sim}{\longrightarrow}\Mor_{\mathbf{Grp}}(F(A),G)\]
where $S(G)$ forgets the group structure of $G$. That is, the functor $F:\mathbf{Set}\to\mathbf{Grp}$ constructing free groups is left-adjoint to the forgetful functor $S:\mathbf{Grp}\to\mathbf{Set}$. This of course applies to every other construction of free objects we have encountered: the free functor is, as a rule, left-adjoint to the forgetful functor.
\end{example}
In fact, the very fact that a functor has an adjoint will endow that functor with convenient features. We say that $F$ is a \textbf{left-adjoint functor} if it has a right adjoint, and that $G$ is a \textbf{right-adjoint functor} if it has a left-adjoint.
\begin{theorem}\label{radjoint limit}
Let $F$ be a left adjoint to $G$.
\begin{enumerate}
\item[(a)] Suppose that $\mathscr{A}:\mathcal{I}\to\mathcal{C}$ is a diagram, and suppose that $\llim\mathscr{A}$ exists in $\mathcal{C}$. Then 
\[G(\llim\mathscr{A})=\llim(G\circ\mathscr{A})\]
In other words, $G$ commutes with limits.
\item[(b)] Suppose that $\mathscr{A}:\mathcal{I}\to\mathcal{C}$ is a diagram, and suppose that $\rlim\mathscr{A}$ exists in $\mathcal{C}$. Then 
\[F(\rlim\mathscr{A})=\rlim(F\circ\mathscr{A})\]
In other words, $F$ commutes with colimits.
\end{enumerate}
\end{theorem}
\begin{proof}
A morphism from a colimit into an object is the same as a compatible system of morphisms from the constituents of the limit into the object, so
\begin{align*}
\Mor_{\mathcal{C}}(X,G(\llim\mathscr{A}))&\cong\Mor_{\mathcal{D}}(F(X),\llim\mathscr{A})=\llim\Mor_{\mathcal{D}}(F(X),\mathscr{A}_i)=\llim\Mor_{\mathcal{D}}(X,G(\mathscr{A}_i))
\end{align*}
proves that $G(\llim\mathscr{A})$ is the limit we are looking for. A similar argument works for the other statement.
\end{proof}
\begin{corollary}
Let $F$ be a left adjoint to $G$.
\begin{enumerate}
\item[(a)] If $\mathcal{C}$ has finite colimits, then $F$ is right exact. 
\item[(b)] If $\mathcal{D}$ has finite limits, then $G$ is right exact. 
\end{enumerate}
\end{corollary}
\subsection{Exercise}
\begin{exercise}
Let $F:\mathcal{C}\to\mathcal{D}$ be a covariant functor, and assume that both $\mathcal{C}$ and $\mathcal{D}$ have products. Prove that for all objects $A$, $B$ of $\mathcal{C}$, there is a unique morphism $F(A\times B)\to F(A)\times F(B)$ such that the relevant diagram involving natural
projections commutes.\par
If $\mathcal{D}$ has coproducts $($denoted $\amalg$$)$ and $G:\mathcal{C}\to\mathcal{D}$ is contravariant, prove that there is a unique morphism $G(A)\amalg G(B)\to G(A\times B)$ $($again, such that an appropriate diagram commutes$)$.
\end{exercise}
\begin{proof}
Apply the functor $F$ yields:
\[\begin{tikzcd}
&A\times B\ar[ld]\ar[rd]&\\
A&&B
\end{tikzcd}\stackrel{F}{\Longrightarrow}\begin{tikzcd}
&F(A\times B)\ar[ld]\ar[rd]&\\
F(A)&&F(B)
\end{tikzcd}\]
by the universal property of $F(A)\times F(B)$, there is a unique morphism:
\[F(A\times B)\stackrel{\exists !}{\longrightarrow}F(A)\times F(B)\]
Similar for coproducts:
\[\begin{tikzcd}
&A\times B\ar[ld]\ar[rd]&\\
A&&B
\end{tikzcd}\stackrel{G}{\Longrightarrow}\begin{tikzcd}
&G(A\times B)&\\
G(A)\ar[ru]&&G(B)\ar[lu]
\end{tikzcd}\]
By the universal property of $G(A)\amalg G(B)$, we get
\[G(A)\amalg G(B)\to G(A\times B)\]
\end{proof}
\begin{exercise}
Let $\mathcal{C}$ be a small category. Prove that $\mathcal{C}$ is equivalent to the subcategory of representable functors in $\mathbf{Set}^{\mathcal{C}^{\circ}}$. Thus, every $(small)$ category is equivalent to a subcategory of a functor category.
\end{exercise}
\begin{proof}
For $\varphi:A\to B$, there is an induced natural transformation:
\[\varphi:\Hom_{\mathcal{C}}(-,A)\to\Hom_{\mathcal{C}}(-,B)\]
from Yoneda lemma, there is a bijection from $\Hom(h_A,h_B)$ to $h_B(A)=\Hom(A,B)$. So $h$ is a fully faithful covariant functor. For any representable $F$, there is a functor $h_X$ and a natural isomorphism $F\cong h_X$. This shows $h$ is a equivalence of categories. 
\end{proof}
\begin{exercise}\label{adic comple}
Let $R$ be a commutative ring, and let $I\sub R$ be an ideal. Note that $I^n\sub I^m$ if $n\geq m$, and hence we have natural homomorphisms $\varphi_{mn}:R/I^n\to R/I^m$ for $n\geq m$.
\begin{enumerate}
\item Prove that the inverse limit $\widehat{R}_I:=\llim_nR/I^n$ exists as a commutative ring. This
is called the $I$-adic completion of $R$.
\item By the universal property of inverse limits, there is a unique homomorphism $R\to\widehat{R}_I$. Prove that the kernel of this homomorphism is $\bigcap_nI^n$.
\item Let $I=(x)$ in $R[x]$. Prove that the completion $\widehat{R[x]}_I$ is isomorphic to the power series ring $R[[x]]$.
\end{enumerate}
\end{exercise}
\begin{proof}
We first prove that limit exists in $\mathbf{Ring}$. Let $\mathcal{I}$ be a poset $(\mathcal{I},\leq)$. Choose $\{R_i\}_{i\in\mathcal{I}}$ and $\{\varphi_{ij}:R_i\to R_j\}$ such that
\[i\geq j\geq k\Rightarrow\varphi_{jk}\circ\varphi_{ij}=\varphi_{ik}\]
A sequence $(r_i)_{i\in\mathcal{I}}$ is coherent if $\varphi_{ij}(r_i)=r_j$. Define the limit $\llim_{i}R_i$ to be
\[\llim_iR_i:=\{(r_i)_{i\in\mathcal{I}}\mid (r_i)\text{ is coherent}\}.\]

Since $I^n\sub I^m$ for $n\geq m$, there is a well defined quotient homomorphism:
\[\varphi_{nm}:R/I^n\to R/I^m,\quad a+I^n\mapsto a+I^m\]
So the limit $\llim_iR/I^n$ is well defined.\par
For the homomorphism $\psi_n:R\to R/I^n$, it is clear that
\[\varphi_{nm}\circ\psi_n=\psi_m\]
so we get the unique homomorphsim $\psi:R\to\widehat{R}_I$, defined by $\psi(r)=(\psi_i(r))_{i\in\mathcal{I}}$. It follows that
\[\psi(r)=0\iff \psi_i(r)=0\iff r\in\bigcap_nI^n.\]

Finally, if $I=(x)$, then $i^n=(x^n)$. So 
\[\widehat{R[x]}_I=\{(r_i)_{i\in\N}:\deg r_i<i,r_i=r_{i-1}+a_{i-1}x^{i-1}\}\]
this set equals $R[[x]]$.
\end{proof}
\begin{exercise}
An important example of the construction presented in Exercise~\ref{adic comple} is the ring $\Z_p$ of $p$-adic integers: this is the limit $\llim_r\Z/p^r\Z$, for a positive prime integer $p$.\par
The field of fractions of $\Z_p$ is denoted $\Q_p$; elements of $\Q_p$ are called \textbf{$p$-adic numbers}.
\begin{enumerate}
\item Show that giving a $p$-adic integer $A$ is equivalent to giving a sequence of integers $A_r, r\geq 1$, such that $0\leq A_r<p^r$, and that $A_s\equiv A_r$ mod $p^s$ if $s\leq r$.
\item Equivalently, show that every $p$-adic integer has a unique infinite expansion $A=a_0+a_1\cdot p+a_2\cdot p^2+a_3\cdot p^3+\cdots$, where $0\leq a_i\leq p-1$. The arithmetic of $p$-adic integers may be carried out with these expansions in precisely the same way as ordinary arithmetic is carried out with ordinary decimal expansions.
\item With notation as in the previous point, prove that $A\in\Z_p$ is invertible if and only if $a_0\neq 0$.
\item Prove that $\Z_p$ is a local domain, with maximal ideal generated by $($the image in $\Z_p$ of$)$ $p$.
\item Prove that $\Z_p$ is a DVR. $($There is an evident valuation on $\Q_p$.$)$
\end{enumerate}
\end{exercise}
\begin{proof}
Every $A_r$ is in $\Z/p^r\Z$, so $0\leq A_r<p^r$. From the construction, $A_s+p^s=\varphi_{rs}(A_r+p^r)=A_r+p^s$ for $s\leq r$, we see that $A_r\equiv A_s$ mod $p^s$ if $s\leq r$.\par
Similar to the example $R[[x]]$. Giving a sequence is the same as giving a truncation of a series. And the third point is the same as series $R[[x]]$. From the previous point, we find that $\Z_p/p\Z_p$ is a field, so $p\Z_p$ is a maximal ideal.\par
Define a function $v_p(x):=\sup\{n:x\in p^n\Z_p\}=\inf\{n:x_n\neq 0\}$. Then for any ideal $I\sub\Z_p$, let $n:=\min\{v_p(x):x\in I\}$, then $I\sub p^n\Z_p$. Now let $y=p^nx\in I$, then $x$ is invertible, so $(p^nx)=p^n\Z_p\sub I$. This shows every ideal in $\Z_p$ has the form $p^n\Z_p$, and $p\Z_p$ is the unique maximal ideal.
\end{proof}
\begin{exercise}\label{completion of Z}
If $m,n$ are positive integers and $m\mid n$, then $(n)\sub (m)$, and there is an onto ring homomorphism $\Z/n\Z\twoheadrightarrow\Z/m\Z$. The limit ring $\llim\Z/n\Z$ exists and is denoted by $\widehat{\Z}$. Prove that $\widehat{\Z}=\End_{\mathbf{Ab}}(\Q/\Z)$. 
\end{exercise}
\begin{proof}
Every $f\in\End_{\mathbf{Ab}}(\Q/\Z)$ is uniquely determined by $f(\frac{1}{n})$. Since $n\cdot f(\frac{1}{n})=f(1)=f(0)=0$, $f(\frac{1}{n})=\frac{g(n)}{n}$ for some integer $g(n)$. Since we are deal with $\Q/\Z$, we may choose $0\leq g(n)<n$.\par
For $m\mid n$, we have $n=am$, so
\[f(\dfrac{1}{n})=f(\dfrac{1}{am})=\dfrac{g(n)}{am},\quad f(\dfrac{1}{m})=\dfrac{g(m)}{m}\]
and
\[a\cdot f(\dfrac{1}{n})=\dfrac{g(n)}{m}=f(\dfrac{1}{m})=\dfrac{g(m)}{m}\]
so we have $g(n)\equiv g(n)$ mod $m$. This means the sequence
\[(f(\dfrac{1}{n}))_{i\in\N}\]
is an element of $\widehat{\Z}$. Conversely, any element in $\widehat{\Z}$ uniquely defines an endomorphism of $\Q/\Z$. So we have $\widehat{\Z}=\End_{\mathbf{Ab}}(\Q/\Z)$.
\end{proof}
\begin{exercise}
Let $\widehat{\Z}$ be as in Exercise~\ref{completion of Z}.
\begin{enumerate}
\item If $R$ is a commutative ring endowed with homomorphisms $R\to\Z/p^r\Z$ for all primes $p$ and all $r$, compatible with all projections $\Z/p^r\Z\to\Z/p^s\Z$ for $s\leq r$, prove that there are ring homomorphisms $R\to\Z/n\Z$ for all $n$, compatible with all projections $\Z/n\Z\to\Z/m\Z$ for $m\mid n$.
\item Deduce that $\widehat{\Z}$ satisfies the universal property for the product of $\Z_p$, as $p$ ranges over all positive prime integers.
\end{enumerate}
It follows that $\prod_p\Z_p\cong\widehat{\Z}\cong\End_{\mathbf{Ab}}(\Q/\Z)$.
\end{exercise}
\begin{proof}
For any $n=p_1^{r_1}\cdots p_i^{r_i}$, $m=p_1^{r'_1}\cdots p_i^{r'_i}$ with $m\mid n$, by Chinese remainder theorem we have a commutative diagram
\[\begin{tikzcd}
\prod_{i}\Z/p_i^{r_i}\Z\ar[r,"\sim"]\ar[d, twoheadrightarrow]&\Z/n\Z\ar[d, twoheadrightarrow]\\
\prod_{i}\Z/p_i^{r'_i}\Z\ar[r,"\sim"]&\Z/m\Z
\end{tikzcd}\]
so we get the first result.\par
Note that giving a morphism from $R$ to $\Z_p$ is the same as giving morphisms from $R$ to $\Z/p^r\Z$ fro all $r$. So if there is a ring $R$ with morphisms to $\Z_p$ for all prime $p$, then we get morphisms to $\Z/p^r\Z$ for all prime $p$, all $r$. From the previous point, there are morphisms $R\to\Z/n\Z$ for all $n$, compatible with all projection $\Z/n\Z\to\Z m\Z$. From the definition of $\widehat{\Z}$, there is a unique morphism from $R$ to $\widehat{\Z}$. So $\widehat{\Z}$ satisfies the universal property of $\prod_p\Z_p$.
\end{proof}
\begin{exercise}
Let $R,S$ be rings. An additive covariant functor $F:R$-$\mathbf{Mod}\to S$-$\mathbf{Mod}$ is \textbf{faithfully exact} if \[\begin{tikzcd}
F(A)\ar[r,"F(\varphi)"]&F(B)\ar[r,"F(\psi)"]&F(C)
\end{tikzcd}\] 
is exact in $S$-$\mathbf{Mod}$ if and only if 
\[\begin{tikzcd}
A\ar[r,"\varphi"]&B\ar[r,"\psi"]&C
\end{tikzcd}\]
is exact in $R$-$\mathbf{Mod}$. Prove that an exact functor $F:R$-$\mathbf{Mod}\to S$-$\mathbf{Mod}$ is faithfully exact if and only if $F(M)\neq0$ for every nonzero $R$-module $M$, if and only if $F(\varphi)\neq0$ for every nonzero morphism $\varphi$ in $R$-Mod.
\end{exercise}
\begin{proof}
\mbox{}
\begin{enumerate}
\item One direction is easy: If $F$ is faithfully exact. Assume $F(M)=0$, then the sequence $0\to F(M)\to0$ is exact, but $0\to M\to 0$ is not exact unless $M=0$, so we find $M=0$. If $F(\varphi)=0$, then 
\[\begin{tikzcd}
F(M)\ar[r,"F(\varphi)"]&F(N)\ar[r,"id_{F(N)}"]&F(N)
\end{tikzcd}\] 
is exact, but 
\[\begin{tikzcd} 
M\ar[r,"\varphi"]&N\ar[r,"id_N"]&N 
\end{tikzcd}\] 
is exact only if $\varphi=0$.
\item Then we show that $F$ reflects zero objects if and only if $F$ reflects zero morphisms: If $F$ reflects zero morphisms, assume $F(X)=0$, then $\id_{F(X)}=0$, so $\id_X=0$. But $id_X=0$ if and only if $X=0$, so $X=0$.\par 
Now assume $F$ reflects zero objects. For $\varphi:A\to B$ such that $F(\varphi)=0$. Consider the exact sequence:
\[\begin{tikzcd}
A\ar[r,"\varphi"]&\im\varphi\ar[r,"\psi"]&\coker\varphi
\end{tikzcd}\]
since $F$ is exact, we also have a exact sequence
\[\begin{tikzcd}
F(A)\ar[r,"F(\varphi)"]&F(\im\varphi)\ar[r,"F(\psi)"]&F(\coker\varphi)
\end{tikzcd}\]
Note that $F(\varphi)=0$, so $F(\psi)$ is monic. But $\psi=0$ so $F(\psi)=0$, we conclude that $F(\im\varphi)=0$. This means $\im\varphi=0$, so we have $\varphi=0$.
\item Now we show the last direction. Let $F$ reflects zero morphisms, first we show that $F$ reflects monomorphisms and epimorphisms. In deed, suppose
\[\begin{tikzcd}
0\ar[r]&X\ar[r]&Y\ar[r]&Z
\end{tikzcd}\]
is exact; then
\[\begin{tikzcd}
0\ar[r]&F(X)\ar[r]&F(Y)\ar[r]&F(Z)
\end{tikzcd}\]
is exact. If $F(Y)\to F(Z)$ is monic, then $F(X)=0$, so $X=0$. This shows $F$ reflects monomorphisms, the dual argument shows that $F$ reflects epimorphisms. Now, in an abelian category, $f$ is an isomorphism if and only if $f$ is both monic and epic, so this implies $F$ reflects isomorphisms. Now suppose $X\to Y\to Z$ is given and
\[\begin{tikzcd}
0\ar[r]&F(X)\ar[r,"F(f)"]&F(Y)\ar[r,"F(g)"]&F(Z)
\end{tikzcd}\]
is exact. Since $F(X)\to\ker F(g)$ is an isomorphism, and $\ker F(g)=F(\ker g)$, $X\to\ker g$ is also an isomorphism, so $X\to Y\to Z$ is exact.
\end{enumerate}
\end{proof}
\begin{exercise}\label{locali exact}
Prove that localization is an exact functor.\par
In fact, prove that localization preserves homology: if
\[\begin{tikzcd}
M_{\bullet}:&\cdots\ar[r]&M_{i+1}\ar[r,"d_{i+1}"]&M_i\ar[r,"d_i"]&M_{i-1}\ar[r]&\cdots
\end{tikzcd}\]
is a complex of $R$-modules and $S$ is a multiplicative subset of $R$, then the localization of the $i$-th homology of $M_{\bullet}$ is the $i$-th homology $H_i(S^{-1}M_{\bullet})$ of the localized complex
\[\begin{tikzcd}
S^{-1}M_{\bullet}:&\cdots\ar[r]&S^{-1}M_{i+1}\ar[r,"S^{-1}d_{i+1}"]&S^{-1}M_i\ar[r,"S^{-1}d_i"]&S^{-1}M_{i-1}\ar[r]&\cdots
\end{tikzcd}\]
\end{exercise}
\begin{proof}
Since 
\[d_i(\dfrac{m}{s})=d(\dfrac{ms'}{ss'})\]
we have
\[\ker S^{-1}d_i=\{\dfrac{m}{s}\mid \exists r\in S, rm\in\ker d_i\},\quad\im S^{-1}d_{i+1}=\{\dfrac{m}{s}\mid\exists r\in s, rm\in\im d_{i+1}\}\]
Concerning the quotient, first we observet that, in the construction of $S^{-1}H_i(M_{\bullet})$:
\begin{align*}
\dfrac{a+\im d_{i+1}}{s}=\dfrac{a'+\im d_{i+1}}{s'}&\iff (\exists r\in S)\quad r[(a+\im d_{i+1})s'-(a'+\im d_{i+1})s]=0\text{ in $H_i(M_{\bullet})$ }\\
&\iff (\exists r\in S)\quad r(as'-a's)\in\im d_{i+1}\end{align*}
While in the quotient $H_i(S^{-1}M_{\bullet})$:
\[\dfrac{a}{s}+\im S^{-1}d_{i+1}=\dfrac{a'}{s'}+\im S^{-1}d_{i+1}\iff \dfrac{a}{s}-\dfrac{a'}{s'}\in\im S^{-1}d_{i+1}\iff (\exists r\in S)\  r(as'-a's)\in\in d_{i+1}\]
So there is a natural homomorphism:
\[\psi:S^{-1}H_i(M_{\bullet})\to H_i(S^{-1}M_{\bullet}),\quad \dfrac{a+\im d_{i+1}}{s}\mapsto\dfrac{a}{s}+\im S^{-1}d_{i+1}\]
this is an isomorphism from the observation above.
\end{proof}
\begin{exercise}
Suppose $M$ is a finitely presented $R$-module and $N$ is an arbitrary $R$-module. Show the followsing holds
\[S^{-1}\Hom_R(M,N)\stackrel{\sim}{\longrightarrow}\Hom_{S^{-1}R}(S^{-1}M,S^{-1}N)\]
But note that this does not holds for any module.
\end{exercise}
\begin{proof}
First we have
\[S^{-1}\Hom_R(R,N)\stackrel{\sim}{\longrightarrow}\Hom_{S^{-1}R}(S^{-1}R,S^{-1}N)\]
for any $N$. And we have a natural homomorphism 
\[S^{-1}\Hom_A(M,N)\to \Hom_{S^{-1}A}(S^{-1}M,S^{-1}N).\] 
Consider the diagram:
\[\begin{tikzcd}[column sep=small]
0\ar[r]&S^{-1}\Hom_{R}(M,N)\ar[r]\ar[d]&S^{-1}\Hom_{R}(R^m,N)\ar[d]\ar[r]&S^{-1}\Hom_{R}(R^n,N)\ar[d]\\
0\ar[r]&\Hom_{S^{-1}R}(S^{-1}M,S^{-1}N)\ar[r]&\Hom_{S^{-1}R}(S^{-1}R^m,S^{-1}N)\ar[r]&\Hom_{S^{-1}R}(S^{-1}R^n,S^{-1}N)
\end{tikzcd}\]
The right two vertical maps are isomorphisms, so we get the isomorphism.\par
For $R=N=\Z$, $M=\Q$, $S=\Z-\{0\}$, we have
\[S^{-1}\Hom_\Z(\Q,\Z)=S^{-1}\{0\}=0,\quad \Hom_{S^{-1}\Z}(S^{-1}\Q,S^{-1}\Z)=\Hom_{\Q}(\Q,\Q)=\Q\]
\end{proof}
\begin{exercise}
Suppose $F:\mathcal{A}\to\mathcal{B}$ is a covariant functor of abelian categories, and $C^\bullet$ is a complex in $\mathcal{A}$.
\begin{enumerate}
\item[(a)]If $F$ is right-exact, describe a natural morphism $FH^\bullet\to H^\bullet F$.
\item[(b)]If $F$ is right-exact, describe a natural morphism $FH^\bullet\leftarrow H^\bullet F$.
\item[(c)]If $F$ is exact, show that the morphisms of (a) and (b) are inverses and thus isomorphisms.
\end{enumerate}
\end{exercise}
\begin{proof}
First we recall that if $F$ is right-exact, then $F$ commutes with cokernels: For we have the following exact sequence
\[\begin{tikzcd}
0\ar[r]&F(C^i)\ar[r,"F(d^{i})"]&F(C^{i+1})\ar[r]&F(\coker d^i)\ar[r]&0
\end{tikzcd}\]
which is obtained from the corresponding short exact sequence. Hence
\[F(\coker d^i)\cong \coker F(d^i)\]
\begin{enumerate}
\item[(a)]Consider the exact sequence
\[\begin{tikzcd}
0\ar[r]&\im d^i\ar[r]&C^{i+1}\ar[r]&\coker d^i\ar[r]&0
\end{tikzcd}\]
Applying $F$ on this gives us
\[\begin{tikzcd}
F\im d^i\ar[r]&F(C^{i+1})\ar[r]&F\coker d^i\ar[r]&0
\end{tikzcd}\]
Together with the similar sequence in $F(C^\bullet)$ we get a diagram
\[\begin{tikzcd}
&F\im d^i\ar[r]\ar[d,dashed,"\alpha"]&F(C^{i+1})\ar[r]\ar[d,equal]&F\coker d^i\ar[d,"\cong"]\ar[r]&0\\
0\ar[r]&\im F(d^i)\ar[r]&F(C^{i+1})\ar[r]&F\coker d^i\ar[r]&0
\end{tikzcd}\]
Then we can show there is an induced map $\alpha:F\im d^i\to\im f(d^i)$. Further, by the snake lemma, this induced map $\alpha$ is an epimorphism.
Now consider another sequence 
\[\begin{tikzcd}
0\ar[r]&H^i(C^\bullet)\ar[r]&\coker d^{i-1}\ar[r]&\im d^i\ar[r]&0
\end{tikzcd}\]
Applying $F$ gives 
\[\begin{tikzcd}
FH^i(C^\bullet)\ar[r]&F\coker d^{i-1}\ar[r]&F\im d^i\ar[r]&0
\end{tikzcd}\]
Similarly, with the counterpart in $F(C^\bullet)$, there is a diagram
\[\begin{tikzcd}
&FH^i(C^\bullet)\ar[r]\ar[d,dashed,"\beta"]&F\coker d^{i-1}\ar[r]\ar[d,"\cong"]&F\im d^i\ar[r]\ar[d,"\alpha"]&0\\
0\ar[r]&H^iF(C^\bullet)\ar[r]&\coker F(d^{i-1})\ar[r]&\im F(d^i)\ar[r]&0
\end{tikzcd}\]
Together with $\alpha$, we get our desired map 
\[\beta:FH^i(C^\bullet)\to H^iF(C^\bullet)\]
\item[(b)]Instead of (a), we may use the sequence for kernels:
\[\begin{tikzcd}
0\ar[r]&\ker d^i\ar[r]&C^i\ar[r]&\im d^i\ar[r]&0
\end{tikzcd}\]
and
\[\begin{tikzcd}
0\ar[r]&\im d^{i-1}\ar[r]&\ker d^i\ar[r]&H^i(C^\bullet)\ar[r]&0
\end{tikzcd}\]
with the identification
\[\ker F(d^i)\cong F(\ker d^i)\]
\item[(c)]With the exactness hypothesis, the map we obtained all becomes isomorphisms.
\end{enumerate}
\end{proof}
\section{Presheaves of sets}\label{category presheaf section}
In this section, we consider the category of presheaves of sets over a category $\mathcal{C}$, and prove some of its properties. In order to avoid set-theoretic issues, we fix once for all a universe $\mathscr{U}$ which has an element with infinite cardinality. A set is said to be \textbf{$\mathscr{U}$-small} (or simply \textbf{small} if there is no confusion) if it is isomorphic to an element of $\mathscr{U}$. We also use the following terminology: small group, small ring, small category. We often assume that the schemes, topological spaces, sets of indices, with which we work are $\mathscr{U}$-small, or at least have cardinality belonging to $\mathscr{U}$. A category $\mathcal{C}$ is called a \textbf{$\mathscr{U}$-category} if for any objects $x,y$ in $\mathcal{C}$, the set $\Hom_\mathcal{C}(x,y)$ is $\mathscr{U}$-small, and is called $\mathscr{U}$-small if the set $\Ob(\mathcal{D})$ is also contained in the universe $\mathscr{U}$. For two categories $\mathcal{C}$, $\mathcal{D}$, we denote by $\sHom(\mathcal{C},\mathcal{D})$ the category of (covariant) functors from $\mathcal{C}$ to $\mathcal{D}$. It is then easy to verify the following two conditions:
\begin{itemize}
\item If $\mathcal{C}$ and $\mathcal{D}$ are elements of $\mathscr{U}$ (resp. $\mathscr{U}$-small), then $\sHom(\mathcal{C},\mathcal{D})$ is an element of $\mathscr{U}$ (resp. $\mathscr{U}$-small).
\item If $\mathcal{C}$ is a $\mathscr{U}$-small category and $\mathcal{D}$ is a $\mathscr{U}$-category, $\sHom(\mathcal{C},\mathcal{D})$ is a $\mathscr{U}$-category.
\end{itemize}
However, note that if $\mathcal{D}$ is a $\mathscr{U}$-small category and $\mathcal{C}$ is a $\mathscr{U}$-category, then $\sHom(\mathcal{C},\mathcal{D})$ is not $\mathscr{U}$-small in general. For example, the category $\sHom(\mathcal{C},\mathscr{U}\text{-}\mathbf{Set})$. It should be noted that $\mathscr{U}$-smallness is really a restrictive condition for categories, and there are many interesting examples where this condition is not satisfied in general.
\subsection{The category of presheaves of sets}
Let $\mathcal{C}$ be a category. We define the \textbf{category of presheaves of sets over $\mathcal{C}$ relative to the universe $\mathscr{U}$} (or, if there is no confusion, the category of presheaves of sets over $\mathcal{C}$) to be the category of contravariant functors from $\mathcal{C}$ to the category of $\mathscr{U}$-sets, and denote it by $\PSh(\mathcal{C})_{\mathscr{U}}$ (or simply $\PSh(\mathcal{C})$ if there is no risk of confusion). The objects of $\PSh(\mathcal{C})_{\mathscr{U}}$ are called \textbf{$\mathscr{U}$-presheaves} (of simply presheaves) over $\mathcal{C}$. If $\mathcal{C}$ is $\mathscr{U}$-small, then $\PSh(\mathcal{C})_{\mathscr{U}}$ is a $\mathscr{U}$-category. However, this is not true in general if $\mathcal{C}$ is only assumed to be a $\mathscr{U}$-category.\par
Let $x$ be an object of a $\mathscr{U}$-category $\mathcal{C}$. We can associate with $x$ a presheaf $h_x:\mathcal{C}^{\op}\to \mathscr{U}\text{-}\mathbf{Set}$, defined in the following way:
\begin{itemize}
\item If $\Hom_\mathcal{C}(y,x)$ is an element of $\mathscr{U}$, then we set $h_x(y)=\Hom_\mathcal{C}(y,x)$.
\item Suppose that $\Hom_\mathcal{C}(y,x)$ is not an element of $\mathscr{U}$ and let $R(Z)$ be the relation "the set $Z$ is the target of an isomorphism $\Hom_\mathcal{C}(y,x) \stackrel{\sim}{\to } Z$". We then put $h_x(y)=\tau_Z(R(Z))$.
\end{itemize}
Let $R'(u)$ be the relation "$u$ is a bijection from $\Hom_\mathcal{C}(y,x)$ to $h_x(y)$" and set $\varphi(y,x)=\tau_u(R'(u))$. Then in both cases, we have a canonical isomorphism
\[\varphi(y,x):\Hom_\mathcal{C}(y,x) \stackrel{\sim}{\to } h_y(x).\]
Now let $u:y\to y'$ be a morphism of $\mathcal{C}$. Then by composition, $u$ defines a map
\[\Hom_{\mathcal{C}}(u,x):\Hom_\mathcal{C}(y',x)\to \Hom_\mathcal{C}(y,x)\]
and we define $h_x(u)$ to be the composition
\[h_x(u)=\varphi(y,x)\Hom_\mathcal{C}(x,u)\varphi(y,x)^{-1}.\]
It is immediate to verify that $h_x$ then defines a functor $\mathcal{C}^{\op}\to \mathscr{U}\text{-}\mathbf{Set}$.
\subsection{Projective limits and inductive limits}
\subsection{Exactness properties of the category of presheaves}
\subsection{The functors \texorpdfstring{$\sHom$}{Hom} and \texorpdfstring{$\sIso$}{Iso}}
Let $\mathcal{C}$ be a category and $F,G$ be objects of $\PSh(\mathcal{C})$. We define an object $\sHom(F,G)$ of $\PSh(\mathcal{C})$ in the following way:
\[\sHom(F,G)(S)=\Hom_{\PSh(\mathcal{C}_{/S})}(F_S,G_S)\cong\Hom_{\PSh(\mathcal{C}_{/S})}(F\times h_S,G\times h_S)\cong\Hom_{\PSh(\mathcal{C})}(F\times h_S,G).\]
It is easy to verify that $\sHom(F,G)$ possesses the following properties:
\begin{itemize}
    \item $\sHom(e,G)\cong G$,
    \item If $E$ is an object of $\PSh(\mathcal{C})$, then
    \begin{equation}\label{category presheaf Hom functor prop-1}
    \sHom(E,F\times G)\cong \sHom(E,F)\times \sHom(E,G).
    \end{equation}
    \item The functor $\mathrm{Hom}$ commutes with base change:
    \begin{equation}\label{category presheaf Hom functor prop-2}
    \sHom(F_S,G_S)\cong \sHom(F,G)_S.
    \end{equation}
    \item $(F,G)\mapsto\sHom(F,G)$ is a bifunctor which is contravariant on $F$ and covariant on $G$.
\end{itemize}

Now we consider an object $E$ of $\PSh(\mathcal{C})$. Let $\phi:E\times F\to G$ be a morphism, we want to associates with $\phi$ a morphism from $E$ into $\mathrm{Hom}(F,G)$. For this, consider a morphism $S'\to S$ of $\mathcal{C}$. We then have the following induced maps:
\[E(S)\times F(S')\to E(S')\times F(S')\stackrel{\phi(S')}{\longrightarrow}G(S').\]
Any element $e$ of $E(S)$ therefore defines a map $F(S')\to G(S')$, which is functorial on $S'$; that is, an element $\theta_\phi(e)$ of $\sHom(F,G)(S)$. We therefore obtain a map
\[\Hom(E\times F,G)\to \Hom(E,\sHom(F,G)),\quad \phi\mapsto\theta_\phi\]
which is functorial on $E$.

\begin{proposition}\label{category presheaf Hom functor adjoint prop}
Let $E,F,G$ be objects of $\PSh(\mathcal{C})$. Then the map $\phi\mapsto\theta_\phi$ is a bijection:
\begin{equation}\label{category presheaf Hom functor adjoint prop-1}
\Hom_{\PSh(\mathcal{C})}(E\times F,G)\stackrel{\sim}{\to} \Hom_{\PSh(\mathcal{C})}(E,\sHom(F,G)),
\end{equation}
and we obtain an isomorphism of functors
\begin{equation}\label{category presheaf Hom functor adjoint prop-2}
\sHom(E\times F,G)\stackrel{\sim}{\to} \sHom(E,\sHom(F,G)).
\end{equation}
\end{proposition}
\begin{proof}
We consider the two members of (\ref{category presheaf Hom functor adjoint prop-1}) as functors of $E$. The first assertion is then valid for $E=h_X$, which follows directly from the definition of $\sHom(F,G)$. On the other hand, since the two functors both transforms inductive limits to projective limits and any object of $\PSh(\mathcal{C})$ can be written as an inductive limits of $h_X$, where $X$ runs through $\mathcal{C}_{/E}$, we conclude that (\ref{category presheaf Hom functor adjoint prop-1}) is a bijection.\par
We can also give a direct proof of (\ref{category presheaf Hom functor adjoint prop-1}). To any element $\theta\in\Hom(E,\sHom(F,G))$, we associate an element $\phi_\theta$ of $\Hom(E\times F,G)$ as follows. For any $S\in\mathcal{C}$, we have a map
\[\theta(S):E(S)\to \sHom(F,G)(S)=\Hom(F\times S,G)\]
which is functorial on $S$. If $(e,f)\in E(S)\times F(S)$, then $f$ can be considered as a morphism $S\to F$, so $f\times\id_S$ is a morphism $S\to F\times S$. On the other hand, $\theta(S)(e)$ is a morphism $F\times S\to G$, so by composing we obtain a morphism
\[\theta(S)(e)\circ(f\times\id_S):S\to G,\]
which is an element $\phi_\theta(e,f)$ of $G(S)$. We verify immediately that the correspondence $S\mapsto\phi_\theta(S)$ is functorial on $S$, so we get a morphism $\phi_\theta:E\times F\to G$. It then remains to check that $\theta\mapsto\phi_\theta$ and $\phi\mapsto\theta_\phi$ are inverses of each other, which is straightforward from definition.\par
We now prove the isomorphism (\ref{category presheaf Hom functor adjoint prop-2}). If $S\in\mathcal{C}$, then by (\ref{category presheaf Hom functor prop-2}) and (\ref{category presheaf Hom functor adjoint prop-1}) applied to $\mathcal{C}_{/S}$, we have
\begin{align*}
\sHom(E,\sHom(F,G))(S)&\cong\Hom_S(E_S,\sHom_S(F_S,G_S))\cong\Hom_S(E_S\times_SF_S,G_S)\\
&\cong \Hom(E\times F\times S,G)\cong\sHom(E\times F,G)(S),
\end{align*}
and these isomorphisms are functorial on $S$.
\end{proof}

\begin{corollary}
We have the following isomorphisms:
\begin{align}
\Hom(E,\sHom(F,G))&\cong \Hom(F,\sHom(E,G))\label{category presheaf Hom functor adjoint prop-3},\\
\sHom(E,\sHom(F,G))&\cong \sHom(F,\sHom(E,G))\label{category presheaf Hom functor adjoint prop-4}.
\end{align}
\end{corollary}
\begin{proof}
The first isomorphism follows from (\ref{category presheaf Hom functor adjoint prop-1}) and the fact that $E\times F\cong F\times E$, and the second one follows from (\ref{category presheaf Hom functor prop-2}).
\end{proof}

In particular, if $E=e$ is the final object, then since $\sHom(e,G)\cong G$, we have
\[\Gamma(\sHom(F,G))=\Hom(e,\sHom(F,G))\cong\Hom(F,\sHom(e,G))\cong\Hom(F,G).\]
We also note that the composition of $\Hom$ induces a functorial morphsim
\[\circ:\sHom(F,G)\times\sHom(G,H)\to\sHom(F,H).\]
In other words, with the operation $\sHom$ and $\times$, the category $\PSh(\mathcal{C})$ is self-enriched.\par

If $F$ and $G$ are objects of $\PSh(\mathcal{C})$, we denote by $\Iso(F,G)$ the subset of $\Hom(F,G)$ formed by isomorphisms from $F$ to $G$. We define a subobject $\sIso(F,G)$ of $\sHom(F,G)$ by
\[\sIso(F,G)(S)=\Iso(F_S,G_S).\]
We then have the following isomorphisms
\[\Gamma(\sIso(F,G))\cong\Iso(F,G),\quad \Iso(F,G)\cong\Iso(G,F).\]
In the particular case where $F=G$, we put
\begin{alignat*}{3}
\sEnd(F)&=\sHom(F,F),&\quad\quad &&\End(F)&=\Hom(F,F)\cong\Gamma(\sEnd(F)),\\
\sAut(F)&=\sIso(F,F),&\quad\quad &&\Aut(F)&=\Iso(F,F)\cong\Gamma(\sAut(F)).
\end{alignat*}
It is clear that the formations of $\sIso$, $\sAut$, $\sEnd$ also commutes with base changes.
\begin{remark}
Note that we can construct an object isomorphic to $\Iso(F,G)$ in the following way: we have a morphism
\[\sHom(F,G)\times \sHom(G,F)\to\sEnd(F);\]
By permuting $F$ and $G$, we then deduce a morphism
\[\Hom(F,G)\times\Hom(G,F)\to\sEnd(F)\times\sEnd(G).\]
On the other hand, the identity morphism of $F$ is an element of $\End(F)$ and defines a morphism $e\to\sEnd(F)$. By composition, we then obtain a morphism
\[e\mapsto\sEnd(F)\times\sEnd(G).\]
It it then immediate to see that the fiber product of $e$ and $\sHom(F,G)\times\sHom(G,F)$ is isomorphic to $\Iso(F,G)$.
\end{remark}
The definitions above are applicable in particular if $F=h_X$ and $G=h_Y$. In the case where $\sHom(h_X,h_Y)$ is representable by an object of $\mathcal{C}$, we denote this object by $\sHom(X,Y)$. It possesses the following property: if $Z\times X$ is representable, then 
\[\Hom(Z,\sHom(X,Y))\cong \Hom(Z\times X,Y).\]
We can also define the objects $\sIso(X)$, $\sEnd(X)$ and $\sAut(X)$. The preceding argumants also applies to the categories of the form $\mathcal{C}_{/S}$, and in this case, the corresponding objects are denoted by $\sHom_S$, $\sIso_S$, etc.
\section{Abelian Category}
\subsection{Additive categories}
\subsubsection{Preaditive category}
\begin{definition}
A category $\mathcal{A}$ is called \textbf{preadditive} if each morphism set $\Mor_{\mathcal{A}}(X,Y)$ is endowed with the structure of an abelian group such that the compositions
\[\Mor_{\mathcal{A}}(X,Y)\times \Mor_{\mathcal{A}}(Y,Z)\to\Mor_{\mathcal{A}}(X,Z)\]
are bilinear. A functor $F:\mathcal{A}\to\mathcal{B}$ of preadditive categories is called an \textbf{additive functor} if and only if 
\[F:\Mor_{\mathcal{A}}(X,Y)\to\Mor_{\mathcal{B}}(F(X),F(Y))\] 
is a homomorphism of abelian groups for all $X,Y\in\Ob(\mathcal{A})$.
\end{definition}
In particular for every $X,Y$ there exists at least one morphism $X\to Y$, namely the \textbf{zero map}.
\begin{lemma}\label{preadd cat id=0}
Let $\mathcal{A}$ be a preadditive category. Let $X$ be an object of $\mathcal{A}$. The following are equivalent:
\begin{enumerate}
\item[(a)] $X$ is a initial object.
\item[(b)] $X$ is a final object.
\item[(c)] $\id_X=0$ in $\Mor_{\mathcal{A}}(X,X)$.
\end{enumerate}
Furthermore, if such an object $0$ exists, then a morphism $f:X\to Y$ factors through $0$ if and only if $f=0$.
\end{lemma}
\begin{proof}
Clearly if $X$ is a final or initial object, then $\id_X=0$ is the unique morphism $X\to X$. Now assume $\id_X=0$ holds, then \[f\in\Mor_{\mathcal{A}}(X,Y)\Rightarrow f=f\circ\id_X=0,\And g\in\Mor_{\mathcal{A}}(Y,X)\Rightarrow g=\id_X\circ g=0.\] 
Thus $X$ is final and initial.
\end{proof}
\begin{definition}
In a preadditive category $\mathcal{A}$ we call \textbf{zero object}, and we denote it $0$ any final and initial object as in Lemma~\ref{preadd cat id=0} above.
\end{definition}
\begin{proposition}\label{preadd cat prod coprod}
Let $\mathcal{A}$ be a preadditive category. Let $X,Y\in\Ob(\mathcal{A})$. Then the product $X\times Y$ exists if and only if the coproduct $X\amalg Y$ exists. In this case $X\times Y\cong X\amalg Y$.
\end{proposition}
\begin{proof}
Suppose that $X\times Y$ exists with projections $\pi_1:X\times Y\to X$ and $\pi_2:X\times Y\to Y$. Denote $i_1:X\to X\times Y$ the morphism corresponding to $(0,1)$:
\[\begin{tikzcd}
&X\ar[ldd,bend right=20pt,swap,"1"]\ar[d,dashed,"i_1"]\ar[rdd,bend left=20pt,"0"]&\\
&X\times Y\ar[ld,swap,"\pi_1"]\ar[rd,"\pi_2"]&\\
X&&Y
\end{tikzcd}\]
Similarly, denote $i_2:Y\to X\times Y$ the morphism corresponding to $(0,1)$. Thus we have the commutative diagram
\[\begin{tikzcd}
X\ar[rr,"1"]\ar[rd,"i_1"]&&X\\
&X\times Y\ar[ru,"\pi_1"]\ar[rd,"\pi_2"]&\\
Y\ar[rr,"1"]\ar[ru,"i_2"]&&Y
\end{tikzcd}\]
where the diagonal compositions are zero. It follows that $i_1\circ \pi_1+i_2\circ\pi_2$ is the identity since it is a morphism which upon composing with $\pi_1$ gives $\pi_1$ and upon composing with $\pi_2$ gives $\pi_2$. Suppose given morphisms $f:X\to Z$ and $g:Y\to Z$. Then we can form the map $f\circ\pi_1+g\circ\pi_2:X\times Y\to Z$. In this way we get a bijection $\Mor_{\mathcal{A}}(X\times Y,Z)=\Mor_{\mathcal{A}}(X,Z)\times\Mor_{\mathcal{A}}(Y,Z)$ which show that $X\times Y\cong X\amalg Y$. The coproduet case can be done similarly.
\end{proof}

\begin{definition}
Given a pair of objects $X,Y$ in a preadditive category $\mathcal{A}$ we call \textbf{direct sum}, and we denote it $X\oplus Y$ the product $X\times Y$ endowed with the morphisms $\pi_1,\pi_2,i_1,i_2$ as in Proposition~\ref{preadd cat prod coprod} above.
\end{definition}
\begin{proposition}
Let $\mathcal{A},\mathcal{B}$ be preadditive categories. Let $F:\mathcal{A}\to\mathcal{B}$ be an additive functor. Then $F$ transforms direct sums to direct sums and zero to zero.
\end{proposition}
\begin{proof}
Suppose $F$ is additive. A direct sum $Z$ of $X$ and $Y$ is characterized by having morphisms 
\[i_1:X\to Z,\ i_2:Y\to Z,\ \pi_1:Z\to X,\ \pi_2:Z\to Y\]
such that
\[\pi_1\circ i_1=\id_X,\pi_2\circ i_2=\id_Y,\pi_2\circ i_1=0,\pi_1\circ i_2=0\And i_1\circ\pi_1+i_2\circ\pi_2=\id_Z.\]
Clearly $F(X)$, $F(Y)$, $F(Z)$ and the morphisms $F(i_1),F(i_2),F(\pi_1),F(\pi_1)$ satisfy exactly the same relations (by additivity) and we see that $F(Z)$ is a direct sum of $F(X)$ and $F(Y)$.
\end{proof}
\subsubsection{Additive category}
\begin{definition}
A category $\mathcal{A}$ is called \textbf{additive} if it is preadditive and finite
products exist, in other words it has a zero object and direct sums.
\end{definition}
Namely the empty product is a finite product and if it exists, then it is a final object.
\begin{definition}
Let $\varphi:A\to B$ be a morphism in an additive category $\mathcal{A}$. A morphism $\iota:K\to A$ is a \textbf{kernel} of $\varphi$ if $\varphi\circ\iota=0$ and for all morphisms $\zeta:Z\to A$ such that $\varphi\circ\zeta=0$ there exists a unique $\widetilde{\zeta}:Z\to K$ making the diagram
\[\begin{tikzcd}
Z\ar[rd,dashed,swap,"\exists !\widetilde{\zeta}"]\ar[r,swap,"\zeta"]\ar[rr,bend left,"0"]&A\ar[r,swap,"\varphi"]&B\\
&K\ar[u,"\iota"]&
\end{tikzcd}\]
commute.\par
A morphism $\psi:B\to C$ is a \textbf{cokernel} of $\varphi$ if $\psi\circ\varphi=0$ and for all morphisms $\beta:B\to Z$ such that $\beta\circ\varphi=0$ there exists a unique $\widetilde{\beta}:C\to Z$ making the diagram
\[\begin{tikzcd}
&C\ar[rd,dashed,"\exists !\widetilde{\beta}"]&\\
A\ar[r,"\varphi"]\ar[rr,swap,bend right,"0"]&B\ar[r,"\beta"]\ar[u,"\psi"]&Z
\end{tikzcd}\]
commute.
\end{definition}
\begin{definition}
If a kernel of $\varphi:A\to B$ exists, then a \textbf{coimage} of $\varphi$ is a cokernel for the morphism $\ker\varphi\to A$. If a cokernel of $\varphi:A\to B$ exists, then the \textbf{image} of $\varphi$ is a kernel of the morphism $B\to\coker\varphi$.
\end{definition}
\begin{lemma}\label{ker is mono}
In any additive category, kernels are monomorphisms and cokernels are epimorphisms.
\end{lemma}
\begin{proof}
Let $\varphi:A\to B$ be a morphism in an additive category $\mathcal{A}$, and let $\ker\varphi:K\to A$ be its kernel. Let $\zeta:Z\to K$ be a morphism such that $\ker\varphi\circ\zeta=0$. Then the composition $\varphi\circ(\ker\varphi\circ\zeta)$ is $0$ and by definition of kernel, $\ker\varphi\circ\zeta$ factors uniquely through $K$:
\[\begin{tikzcd}
Z\ar[r,bend left=20,"\zeta"]\ar[r,dashed,swap,bend right=20,"\exists !"]&K\ar[r,"\ker\varphi"]&A\ar[r,"\varphi"]&B
\end{tikzcd}\]
since $\ker\varphi\circ\zeta=0=\ker\varphi\circ 0$, the uniqueness of the decomposition gives $\zeta=0$.\par
The proof that cokernels are epimorphisms is analogous.
\end{proof}
Now we relate the direct sum to kernels as follows.
\begin{proposition}
Let $\mathcal{A}$ be a preadditive category. Let $X\oplus Y$ with morphisms as in Propostion~\ref{preadd cat prod coprod} be a direct sum in $\mathcal{A}$. Then $i_1:X\to X\oplus Y$ is a kernel of $\pi_2:X\oplus Y\to Y$. Dually, $\pi_1$ is a cokernel for $i_2$.
\end{proposition}
\begin{proof}
Let $f:Z\to X\oplus Y$ be a morphism such that $\pi_2\circ f=0$. We have to show that there exists a unique morphism $g:Z\to X$ such that $f=i_1\circ g$:
\[\begin{tikzcd}
X\ar[rr,"1"]\ar[rd,"i_1"]&&X\\
Z\ar[r,"f"]&X\times Y\ar[ru,"\pi_1"]\ar[rd,"\pi_2"]&\\
Y\ar[rr,"1"]\ar[ru,"i_2"]&&Y
\end{tikzcd}\] Since $i_1\circ\pi_1+i_2\circ\pi_2$ is the identity on $X\oplus Y$ we see that
\[f=(i_1\circ\pi_1+i_2\circ\pi_2)\circ f=i_1\circ\pi_1\circ f\]
and hence $g=\pi_1\circ f$ works. Uniquess holds because $\pi_1\circ i_1$ is the identity on $X$. The proof of the second statement is dual.
\end{proof}
\begin{theorem}\label{preadditive coim im}
Let $\varphi:A\to B$ be a morphism in a preadditive category such that
the kernel, cokernel, image and coimage all exist. Then $\varphi$ can be factored uniquely
\[\begin{tikzcd}
A\ar[rrr,bend left=20pt,"\varphi"]\ar[r]&\coim\varphi\ar[r]&\im\varphi\ar[r]&B
\end{tikzcd}\]
\end{theorem}
\begin{proof}
There is a canonical morphism $\coim\varphi\to B$ because $\ker\varphi\to A\to B$ is zero,
\[\begin{tikzcd}
&&\im\varphi&\\
\ker\varphi\ar[r]&A\ar[r,"\varphi"]\ar[d]&B\ar[r]&\coker\varphi\\
&\coim\varphi\ar[ru,dashed]
\end{tikzcd}\]
The composition $\coim\varphi\to B\to\coker\varphi$ is zero, because it is the unique morphism which gives rise to the morphism $A\to B\to\coker\varphi$ which is zero. Hence $\coim\varphi\to B$ factors uniquely through $\im\varphi\to B$, which gives us the desired map.
\end{proof}
\subsection{Abelian categories}
An abelian category is a category satisfying just enough axioms so the snake lemma holds. An axiom is that the canonical map $\coim\varphi\to\im\varphi$ of Theorem~\ref{preadditive coim im} is always an isomorphism.
\begin{definition}\label{ab cat def}
A category $\mathcal{A}$ is \textbf{abelian} if it is additive, if all kernels and cokernels exist, and if the natural map $\coim\varphi\to\im\varphi$ is an isomorphism for all morphisms $\varphi$ of $\mathcal{A}$.
\end{definition}
\begin{definition}
Let $\varphi:A\to B$ be a morphism in an abelian category.
\begin{enumerate}
\item[(a)] We say $\varphi$ is \textbf{injective} if $\ker\varphi=0$.
\item[(b)] We say $\varphi$ is \textbf{surjective} if $\coker\varphi=0$.
\end{enumerate}
\end{definition}
\begin{proposition}\label{mono epi iff ker coker}
Let $\varphi:A\to B$ be a morphism in an abelian category. Then
\begin{enumerate}
\item[(a)] $\varphi$ is \textbf{injective} if and only if $f$ is a monomorphism.
\item[(b)] $\varphi$ is \textbf{surjective} if and only if $f$ is a epimorphism.
\end{enumerate}
\end{proposition}
\begin{proof}
The condition for monomorphism can be interpreted as: If $\psi:Z\to A$ is any morphism such that $\varphi\circ\psi=0$, then $\psi$ factors through $0\to A$. So $\varphi$ is a monomorphism if and only if $0\to A$ is its kernel. The same holds for epimorphism.
\end{proof}
\begin{proposition}
Let $\mathcal{A}$ be an abelian category. All finite limits and finite colimits
exist in $\mathcal{A}$.
\end{proposition}
\begin{proof}
To show that finite limits exist it suffices to show that finite products and
equalizers exist. Finite products exist by definition and the equalizer of $f,g:X\to Y$ is the kernel of $a-b$. The argument for finite colimits is similar but dual to this.
\end{proof}
\begin{example}
Let $\mathcal{A}$ be an abelian category. Pushouts and fibre products in $\mathcal{A}$ have the following simple descriptions:
\begin{enumerate}
\item[(a)] If $f:X\to Y,g:Z\to Y$ are morphisms in $\mathcal{A}$, then we have the fibre product: $X\times_YZ=\ker((f,-g):X\oplus Z\to Y)$.
\item[(b)] If $f:Y\to X,g:Y\to Z$ are morphisms in $\mathcal{A}$, then we have the pushout: $X\amalg_YZ=\coker((f,-g):Y\oplus X\to Z)$.
\end{enumerate}
\end{example}
\begin{lemma}\label{ker is ker of coker}
In an abelian category $\mathcal{A}$, every kernel is the kernel of its cokernel; every cokernel is the cokernel of its kernel.
\end{lemma}
\begin{proof}
Let $\varphi:K\to A$ be the kernel of some morphism $A\to B$; since $\mathcal{A}$ is abelian, $\varphi$ has a cokernel $\psi:A\to C$. The composition $K\to A\to B$ is $0$, so $A\to B$ factors
through $\psi$ by definition of cokernel:
\[\begin{tikzcd}
C\ar[rd,dashed]&\\
A\ar[r]\ar[u,"\psi"]&B\\
K\ar[u,"\varphi"]&
\end{tikzcd}\]
Now let $Z\to A$ be a morphism such that the composition $Z\to A\to C$ is the zero-morphism; then so is the composition $Z\to A\to B$. Therefore $Z\to A$ factors through a unique morphism $Z\to K$,
\[\begin{tikzcd}
&C\ar[rd,dashed]&\\
Z\ar[r]\ar[rd,dashed]&A\ar[r]\ar[u,"\psi"]&B\\
&K\ar[u,"\varphi"]&
\end{tikzcd}\]
since $\varphi$ is the kernel of $A\to B$. But this shows that $\varphi:A\to B$ satisfies the property defining the kernel of its cokernel $A\to C$, as stated.
\end{proof}
\begin{proposition}\label{mono epi iff inverse}
Let $\varphi:A\to B$ be a morphism in an abelian category $\mathcal{A}$.
\begin{enumerate}
\item[(a)] $\varphi$ is a monomorphism if and only if $\varphi$ has a left-invere.
\item[(b)] $\varphi$ is a epimorphism if and only if $\varphi$ has a right-invere.
\end{enumerate}
Thus $\varphi$ is an isomorphism if and only if it is a monomorphism and a epimorphism.
\end{proposition}
\begin{proof}
If $\varphi$ has a left-inverse, then clearly it is monic. Conversely, if $\varphi$ is a monomorphis, then the kernel of $\varphi$ is $0\to A$. Further, $\varphi$ is the cokernel of $0\to A$. Now consider the identity $A\to A$:
\[\begin{tikzcd}
0\ar[r]&A\ar[r,"\varphi"]\ar[d,swap,"\id_A"]&B\\
&A
\end{tikzcd}\]
Since $0\to A\to A$ is the zero morphism and $\varphi$ is the cokernel of $0\to A$, we obtain a unique morphism $\psi:B\to A$ making the diagram commute:
\[\begin{tikzcd}
0\ar[r]&A\ar[r,"\varphi"]\ar[d,swap,"\id_A"]&B\ar[ld,"\psi"]\\
&A
\end{tikzcd}\]
As $\psi\circ\varphi=\id_A$, this shows that $\varphi$ has a right-inverse. The part (b) can be done similarly.
\end{proof}
\subsection{Exact sequence in Abelian category}
\begin{definition}
Let $\mathcal{A}$ be an additive category. We say a sequence of morphisms
\[\begin{tikzcd}
\cdots\ar[r]&A\ar[r,"\varphi"]&B\ar[r,"\psi"]&C\ar[r]&\cdots
\end{tikzcd}\]
in $\mathcal{A}$ is a \textbf{complex} if the composition of any two arrows is zero. If $\mathcal{A}$ is abelian then we say a sequence as above is \textbf{exact at $\bm{B}$} if $\im\psi=\ker\varphi$. We say it is exact if it is exact at every object. A \textbf{short exact sequence} is an exact complex of the form
\[\begin{tikzcd}
0\ar[r]&A\ar[r,"\varphi"]&B\ar[r,"\psi"]&C\ar[r]&0
\end{tikzcd}\]
\end{definition}
\begin{proposition}
Let $\mathcal{A}$ be an abelian category. Let $0\to M_1\to M_2\to M_3\to0$ be a complex of $\mathcal{A}$.
\begin{enumerate}
\item[(a)] $M_1\to M_2\to M_3\to 0$ is exact if and only if
\[\begin{tikzcd}
0\ar[r]&\Hom_{\mathcal{A}}(M_3,N)\ar[r]&\Hom_{\mathcal{A}}(M_2,N)\ar[r]&\Hom_{\mathcal{A}}(M_1,N)
\end{tikzcd}\]
is an exact sequence of abelian groups for all objects $N$ of $\mathcal{A}$.
\item[(b)] $0\to M_1\to M_2\to M_3$ is exact if and only if
\[\begin{tikzcd}
\Hom_{\mathcal{A}}(N,M_1)\ar[r]&\Hom_{\mathcal{A}}(N,M_2)\ar[r]&\Hom_{\mathcal{A}}(N,M_3)\ar[r]&0
\end{tikzcd}\]
is an exact sequence of abelian groups for all objects $N$ of $\mathcal{A}$.
\end{enumerate}
\end{proposition}
\begin{example}\label{fibered diagram}
For a slightly more interesting example, consider a diagram
\[\begin{tikzcd}
D\ar[d,swap,"\psi'"]\ar[r,"\varphi'"]&B\ar[d,"\psi"]\\
A\ar[r,swap,"\varphi"]&C
\end{tikzcd}\]
and the associated sequence
\[\begin{tikzcd}
D\ar[r,"{(\psi',\varphi')}"]&A\oplus B\ar[r,"{(\varphi,-\psi)}"]&C
\end{tikzcd}\]
obtained by letting $A\oplus B$ play both roles of product and coproduct. Then
\begin{enumerate}
\item the diagram is commutative if and only if this sequence is a complex;
\item the sequence obtained by adding a $0$ to the left,
\[\begin{tikzcd}
0\ar[r]&D\ar[r]&A\oplus B\ar[r]&C
\end{tikzcd}\]
is exact if and only if $D$ may be identified with the fibered product $A\times_{C}B$. If this holds, we say the diagram is \textbf{cartesian}.
\item likewise, the sequence
\[\begin{tikzcd}
D\ar[r]&A\oplus B\ar[r]&C\ar[r]&0
\end{tikzcd}\]
is exact if and only if $C$ may be identified with the fibered coproduct $A\amalg_DB$. If this holds, we say the diagram is \textbf{cocartesian}.
\end{enumerate}
\end{example}
\begin{lemma}\label{pull bak lem}
Let $\mathcal{A}$ be an abelian category. Let
\[\begin{tikzcd}
D\ar[d,swap,"\psi'"]\ar[r,"\varphi'"]&B\ar[d,"\psi"]\\
A\ar[r,swap,"\varphi"]&C
\end{tikzcd}\]
be a commutative diagram.
\begin{enumerate}
\item[(a)] If the diagram is cartesian, then the morphism $\ker\varphi'\to\ker\varphi$ induced by $\psi'$ is an isomorphism.
\item[(b)] If the diagram is cocartesian, then the morphism $\coker\varphi'\to\coker\varphi$ induced by $\psi$ is an isomorphism.
\end{enumerate}
\end{lemma}
\begin{proof}
Suppose the diagram is cartesian. Let $\epsilon:\ker\varphi'\to\ker\varphi$ be induced by $\psi'$. Let $i:\ker\varphi\to A$ and $j:\ker\varphi'\to D$ be the canonical injections. Consider the map $\alpha:\ker\varphi\to D$ determined by the morphisms $(i,0)$:
\[\psi'\circ\alpha=i,\quad \varphi'\circ\alpha=0.\]
Then there is an induced morphism $\gamma:\ker\varphi\to\ker\varphi'$:
\[\begin{tikzcd}
\ker\varphi'\ar[r,"j"]\ar[d,"\epsilon"]&D\ar[d,swap,"\psi'"]\ar[r,"\varphi'"]&B\ar[d,"\psi"]\\
\ker\varphi\ar[r,"i"]\ar[ru,"\alpha"]\ar[u,bend left=30pt,"\gamma"]&A\ar[r,swap,"\varphi"]&C
\end{tikzcd}\]
It follows that
\[\psi'\circ j\circ\gamma\circ\epsilon=\psi'\circ\alpha\circ\epsilon=i\circ\epsilon=\psi'\circ j\And\varphi'\circ j\circ\gamma\circ\epsilon=\varphi'\circ\alpha\circ\epsilon=0=\psi'\circ j.\]
By the universal property of pull back, we claim $j\circ\gamma\circ\epsilon=j$. Since $j$ is a monomorphism, this means $\gamma\circ\eps=\id_{\ker\varphi'}$.\par
Furthermore, we have 
\[i\circ\epsilon\circ\gamma=\psi'\circ j\circ\gamma=\psi'\circ\alpha=i.\]
Since $i$ is a monomorphism this implies $\epsilon\circ\gamma=\id_{\ker\varphi}$. This proves (a). Now, (b) follows by duality.
\end{proof}
\begin{lemma}
Let $\mathcal{A}$ be an abelian category. Let
\[\begin{tikzcd}
D\ar[d,swap,"\psi'"]\ar[r,"\varphi'"]&B\ar[d,"\psi"]\\
A\ar[r,swap,"\varphi"]&C
\end{tikzcd}\]
be a commutative diagram.
\begin{enumerate}
\item[(a)] If the diagram is cartesian and $\varphi$ is an epimorphism, then the diagram is cocartesian and $\varphi'$ is an epimorphism.
\item[(b)] If the diagram is cocartesian and $\varphi'$ is an monomorphism, then the diagram is cartesian and $\varphi$ is an epimorphism.
\end{enumerate}
\end{lemma}
\begin{proof}
Suppose the diagram is cartesian and $\varphi$ is an epimorphism. Let $\alpha=(\psi',\varphi'):D\to A\oplus B$ and let $\beta=(\varphi,-\psi):A\oplus B\to C$. As $\varphi$ is an epimorphism, $\alpha$ is an epimorphism, too. Therefore by Example~\ref{fibered diagram} the diagram is cocartesian. Finally, $\varphi'$ is an epimorphism by Lemma~\ref{pull bak lem}. This proves $(1)$, and $(2)$ follows by duality.
\end{proof}
\begin{corollary}
Let $\mathcal{A}$ be an abelian category.
\begin{enumerate}
\item[(a)] If $X\to Y$ is surjective, then for every $Z\to Y$ the projection $X\times_YZ\to Z$ is surjective.
\item[(b)] If $X\to Y$ is injective, then for every $X\to Z$ the morphism $Z\to Z\amalg_XY$ is injective.
\end{enumerate}
\end{corollary}
\begin{lemma}\label{exact iff}
Let \begin{tikzcd}X'\ar[r,"f"]&X\ar[r,"g"]&X''\end{tikzcd} be a complex. Then the conditions below are equivalent:
\begin{enumerate}
\item[$(\rmnum{1})$] the complex \begin{tikzcd}X'\ar[r,"f"]&X\ar[r,"g"]&X''\end{tikzcd} is exact.
\item[$(\rmnum{2})$] the induced morphism $X'\to\ker g$ is an epimorphism.
\item[$(\rmnum{3})$] for any morphism $h:S\to X$ such that $g\circ h=0$, there exist an epimorphism $f':S'\twoheadrightarrow S$ and a commutative diagram
\[\begin{tikzcd}
S'\ar[d]\ar[r,twoheadrightarrow,"f'"]&S\ar[d,"h"]\ar[rd,"0"]&\\
X'\ar[r,"f"]&X\ar[r,"g"]&X''
\end{tikzcd}\]
\end{enumerate}
\end{lemma}
\begin{proof}
$(\rmnum{1})\iff(\rmnum{2})$: the exactness is saying $\ker g=\im f$. If $X'\to\ker g$ is epic, by Exercise~\ref{epi mono decop}, $\ker g=\im f$ as needed. Conversely, if $\ker g=\in f$, by Lemma~\ref{im decomp}, $X'\to\ker g$ is epic.\par
$(\rmnum{1})\Rightarrow(\rmnum{3})$: It is enough to choose $X'\times_{\ker g}S$ as $S'$. Since $X'\to\ker g$ is an epimorphism, $S'\to S$ is an epimorphism by Lemma~\ref{pull bak lem}.\par
$(\rmnum{3})\Rightarrow(\rmnum{2})$: Choose $S=\ker g$. Then the diagram becomes
\[\begin{tikzcd}
S'\ar[d]\ar[r,twoheadrightarrow,"f'"]&\ker g\ar[d]\ar[rd,"0"]&\\
X'\ar[ru,dashed]\ar[r,"f"]&X\ar[r,"g"]&X''
\end{tikzcd}\]
since $g\circ f=0$, by the universal property of $\ker g$, there is a unique morphism $X'\to\ker g$. It follows that the composition $S'\to X'\to\ker g$ is an epimorphism. Hence $X'\to\ker g$ is an epimorphism.
\end{proof}
\subsection{Exercise}
\begin{exercise}\label{epi mono decop}
Let $\varphi:A\to B$ be a morphism in an abelian category, and assume $\varphi$ decomposes as an epimorphism $\pi$ followed by a monomorphism $i$:
\[\begin{tikzcd}
A\ar[rr,swap,bend right,"\varphi"]\ar[r,twoheadrightarrow,"\pi"]&C\ar[r,rightarrowtail,"i"]&B
\end{tikzcd}\]
Prove that necessarily $\pi=\coim\varphi$ and $i=\im\varphi$.
\end{exercise}
\begin{proof}
From the universal property of image and coimage, we have the following commutative diagram:
\[\begin{tikzcd}
&\coim\varphi\ar[rd,bend left,rightarrowtail]&\\
A\ar[ru,twoheadrightarrow,bend left]\ar[rd,twoheadrightarrow,bend right]\ar[r,twoheadrightarrow,"\pi"]&C\ar[u,dashed,"\exists!\nu"]\ar[r,rightarrowtail,"i"]&B\\
&\im\varphi\ar[u,dashed,"\exists!\mu"]\ar[ru,bend right,rightarrowtail]&
\end{tikzcd}\]
By simple observation, we find $\mu$ and $\nu$ are both monomorphic and epimorphic, hence are isomorphisms.
\end{proof}

\section{Triangulated categories}
\subsection{Localization of categories}
Consider a category $\mathcal{C}$ and a family $\mathcal{S}$ of morphisms in $\mathcal{C}$. The aim of localization is to find a new category $\mathcal{C}_\mathcal{S}$ and a functor $Q:\mathcal{C}\to\mathcal{C}_\mathcal{S}$ which sends the morphisms belonging to $\mathcal{C}$ to isomorphisms in $\mathcal{C}_\mathcal{S}$, $(\mathcal{C}_\mathcal{S},Q)$ being "universal" for such a property. We also discuss with some details the localization of functors. When considering a functor $F$ from $\mathcal{C}$ to a category $\mathcal{A}$ which does not necessarily send the morphisms in $\mathcal{S}$ to isomorphisms in $\mathcal{A}$, it is possible to define the right (resp. left) localization of $F$, a functor $R_\mathcal{S}F$ (resp. $L_\mathcal{S}F$) from $\mathcal{C}_\mathcal{S}$ to $\mathcal{A}$. Such a right localization always exists if $\mathcal{A}$ admits filtrant inductive limits.\par
Let $\mathcal{C}$ be a category and $\mathcal{S}$ be a family of morphisms in $\mathcal{C}$. A \textbf{localization} of $\mathcal{C}$ by $\mathcal{S}$ is the data of a category $\mathcal{C}_\mathcal{S}$ and a functor $Q:\mathcal{C}\to\mathcal{C}_\mathcal{S}$ satisfying:
\begin{enumerate}[leftmargin=40pt]
    \item[(L1)] for all $s\in\mathcal{S}$, $Q(s)$ is an isomorphism;
    \item[(L2)] for any category $\mathcal{A}$ and any functor $F:\mathcal{C}\to\mathcal{A}$ such that $F(s)$ is an isomorphism for all $s\in\mathcal{S}$, there exist a functor $F_\mathcal{S}:\mathcal{C}_\mathcal{S}\to\mathcal{A}$ and an isomorphism $F\cong F_\mathcal{S}\circ Q$ visualized by the diagram
    \[\begin{tikzcd}
    \mathcal{C}\ar[r,"F"]\ar[d,swap,"Q"]&\mathcal{A}\\ 
    \mathcal{C}_\mathcal{S}\ar[ru,dashed,swap,"F_\mathcal{S}"]
    \end{tikzcd}\]
    \item[(L3)] if $G$ and $G'$ are two functors from $\mathcal{C}_\mathcal{S}$ to $\mathcal{A}$, then the natural map
    \begin{equation}\label{category localization def-1}
    \Hom_{\Fun(\mathcal{C}_\mathcal{S},\mathcal{A})}(G,G')\to\Hom_{\Fun(\mathcal{C},\mathcal{A})}(G\circ Q,G'\circ Q)
    \end{equation}
    is bijective.
\end{enumerate}
Note that condition (L3) means that the functor $Q^\star:\Fun(\mathcal{C}_\mathcal{S},A)\to\Fun(\mathcal{C},A)$ induced by composition is fully faithful. In particular, this implies that $F_\mathcal{S}$ in (L2) is unique up to isomorphism.

\begin{proposition}\label{category localization opposite char}
Let $\mathcal{C}$ be a category and $\mathcal{S}$ be a family of morphisms in $\mathcal{C}$.
\begin{enumerate}
    \item[(a)] If $\mathcal{C}_\mathcal{S}$ exists, it is unique up to equivalence of categories.
    \item[(b)] If $\mathcal{C}_\mathcal{S}$ exists, then, denoting by $\mathcal{S}^{\op}$ the image of $\mathcal{S}$ in $\mathcal{C}^{\op}$, $(\mathcal{C}^{\op})_{\mathcal{S}^{\op}}$ exists and there is an equivalence of categories $(\mathcal{C}_\mathcal{S})^{\op}\cong(\mathcal{C}^{\op})_{\mathcal{S}^\op}$.
\end{enumerate}
\end{proposition}
\begin{proof}
If $(\mathcal{C}_\mathcal{S},Q)$ and $(\mathcal{C}_\mathcal{S}',Q')$ are two localizations of $\mathcal{C}$ by $\mathcal{S}$, then since $Q'(s)$ is an isomorphism for any $s\in\mathcal{S}$, we obtain a funcor $G:\mathcal{C}_\mathcal{S}\to\mathcal{C}'_\mathcal{S}$ such that $GQ\cong Q'$; similarly, there is a functor $G':\mathcal{C}_\mathcal{S}'\to\mathcal{C}_\mathcal{S}$ such that $G'Q'\cong Q$. Since $G'GQ\cong Q$, we conclude from (L3) that $G'G\cong\id_{\mathcal{C}_\mathcal{S}}$, and similarly $GG'\cong\id_{\mathcal{C}'_\mathcal{S}}$. The second assertion follows immediately from (a) and a easy verification by reversing the arrows.
\end{proof}

The existence of the localization $\mathcal{C}_\mathcal{S}$ is generally true, since we can construct $\mathcal{C}_\mathcal{S}$ by adding virtue inverses to $\mathcal{C}$ (like the construction of free groups). More precisely, we have $\Ob(\mathcal{C}_\mathcal{S})=\Ob(\mathcal{C})$, and the morphisms in $\mathcal{C}_\mathcal{S}$ are of the form
\[\cdots bt^{-1}as^{-1}\cdots=\left(\begin{tikzcd}[row sep=5mm, column sep=5mm]
\cdots\ar[rd]&&Y\ar[rd,"a"]\ar[ld,swap,"s"]&&W\ar[ld,swap,"t"]\ar[rd,"b"]&&\cdots\ar[ld]\\
&X&&Z&&U&
\end{tikzcd}\right)\]
where $a,b\in\Mor(\mathcal{C})$ and $s,t\in\mathcal{S}$, with the composition map defined in the obvious way subject to the relations
\[s^{-1}t^{-1}=(ts)^{-1},\quad ss^{-1}=\id,\quad s^{-1}s=\id.\]
The problem is that the equivalence relation in $\mathcal{C}_\mathcal{S}$ is now hard to manipulate: we can not tell which morphisms $f,g$ in $\mathcal{C}$ satisfy $Q(f)=Q(g)$. Due to this failure, we now impose some additional conditions on the system $\mathcal{S}$, so that the resulting localization $\mathcal{C}_\mathcal{S}$ is way more easiler to describe.

\begin{definition}
The family $\mathcal{S}$ is called a \textbf{right multiplicative system} if it satisfies the following axioms:
\begin{enumerate}[leftmargin=40pt]
    \item[(S1)] For any object $X$ of $\mathcal{C}$, $\id_X$ belongs to $\mathcal{S}$.
    \item[(S2)] If two morphisms $f:X\to Y$ and $g:Y\to Z$ belong to $\mathcal{S}$, then $g\circ f$ belongs to $\mathcal{S}$.
    \item[(S3)] Given two morphisms $f:X\to Y$ and $s:X\to X'$ with $s\in\mathcal{S}$, there exist $t:Y\to Y'$ and $g:X'\to Y'$ with $t\in\mathcal{S}$ such that the following diagram commutes:
    \[\begin{tikzcd}
    X\ar[r,"f"]\ar[d,swap,"s",tail]&Y\ar[d,dashed,"t",tail]\\
    X'\ar[r,dashed,"g"]&Y'
    \end{tikzcd}\]
    \item[(S4)] Let $f,g:X\rightrightarrows Y$ be two morphisms in $\mathcal{C}$. If there exists a morphism $s:Z\to X$ in $\mathcal{S}$ such that $fs=gs$, then there exists $t:Y\to W$ in $\mathcal{S}$ such that $tf=tg$. This is visualized by the diagram:
    \[\begin{tikzcd}
    Z\ar[r,"s",tail]&X\ar[r,shift left=3pt,"f"]\ar[r,shift right=3pt,swap,"g"]&Y\ar[r,dashed,"t",tail]&W
    \end{tikzcd}\]
\end{enumerate}
\end{definition}

\begin{remark}
Axioms (S1)-(S2) asserts that there exists a half-full subcategory $\widetilde{\mathcal{S}}$ of $\mathcal{C}$ with $\Ob(\widetilde{\mathcal{S}})=\Ob(\mathcal{C})$ and $\Mor(\widetilde{\mathcal{S}})=\mathcal{S}$. With these axioms, the notion of a right multiplicative system is stable by equivalence of categories.
\end{remark}
\begin{remark}
The notion of a \textbf{left multiplicative system} is defined similarly by reversing the arrows. This means that the condition (S3) and (S4) are replaced by the conditions (S3') and (S4') below:
\begin{enumerate}[leftmargin=40pt]
    \item[(S3')] Given two morphisms $f:X\to Y$ and $t:Y'\to Y$ with $t\in\mathcal{S}$, there exist $s:X'\to X$ and $g:X'\to Y'$ with $s\in\mathcal{S}$ such that the following diagram commutes:
    \[\begin{tikzcd}
    X'\ar[r,dashed,"g"]\ar[d,swap,dashed,"s",tail]&Y'\ar[d,"t",tail]\\
    X\ar[r,"f"]&Y
    \end{tikzcd}\]
    \item[(S4')] Let $f,g:X\rightrightarrows Y$ be two morphisms in $\mathcal{C}$. If there exists a morphism $t:Y\to W$ in $\mathcal{S}$ such that $tf=tg$, then there exists $s:Z\to X$ in $\mathcal{S}$ such that $fs=gs$. This is visualized by the diagram:
    \[\begin{tikzcd}
    Z\ar[r,dashed,"s",tail]&X\ar[r,shift left=3pt,"f"]\ar[r,shift right=3pt,swap,"g"]&Y\ar[r,"t",tail]&W
    \end{tikzcd}\]
\end{enumerate}
A collection $\mathcal{S}$ is simply called a \textbf{multiplicative system} if it is both a left multiplicative system and a right multiplicative system.
\end{remark}

Let $\mathcal{S}$ be a system of morphisms of $\mathcal{C}$ satisfying axioms (S1)-(S2) and $X\in\Ob(\mathcal{C})$. We define $\mathcal{S}_{/X}$ (resp. $\mathcal{S}_{X/}$) to be the full subcategory of $\mathcal{C}_{/X}$ (resp. $\mathcal{C}_{X/}$) with objects (morphisms in $\mathcal{C}$) belonging to $\mathcal{S}$.

\begin{proposition}\label{category localization comma is filtered}
If $\mathcal{S}$ is a left (resp. right) multiplicative system. Then the category $\mathcal{S}_{/X}$ (resp. $\mathcal{S}_{X/}$) is cofiltrant (resp. filtrant).
\end{proposition}
\begin{proof}
Note that $(\mathcal{S}^{\op})_{X/}=(\mathcal{S}_{/X})^{\op}$, so we only need to consider right multiplicative systems. For any objects $s:X\to Z$ and $s':X\to Z'$ of $\mathcal{S}_{X/}$, by (S3) we have a commutative diagram
\[\begin{tikzcd}
X\ar[r,"s'"]\ar[d,swap,"s"]&Z'\ar[d,"t'"]\\
Z\ar[r,"t"]&Y
\end{tikzcd}\]
with $t\in S$. Then $ts\in\mathcal{S}$ by (S2) and the composition $ts:X\to Y$ belongs to $\mathcal{S}_{X/}$. Now consider two morphisms $f,g:Z\rightrightarrows Z'$ such that $fs=gs=s'$. Then by (S4) there exists $t:Z'\to W$ such that $tf=tg$, so $t\circ s':X\to W$ belongs to $\mathcal{S}_{X/}$ and the compositions
\[\begin{tikzcd}
(Z,s)\ar[r,shift left=3pt,"f"]\ar[r,shift right=3pt,swap,"g"]&(Z',s')\ar[r,"t"]&(W,t\circ s')
\end{tikzcd}\]
coincides; this completes the proof.
\end{proof}

Let $X,Y$ be objects of $\mathcal{C}$. For left (resp. right) multiplicative system $\mathcal{S}$, we define a collection $M_{X,Y}^l$ (resp. $M_{X,Y}^r$), which will be considered to be the "morphisms" from $X$ to $Y$ in our localization category.
\begin{itemize}
    \item If $\mathcal{S}$ is a left multiplicative system, we denote by $M_{X,Y}^l$ the collection of diagrams of the form
    \[\begin{tikzcd}[row sep=5mm, column sep=5mm]
    &Z\ar[ld,swap,"s"]\ar[rd,"a"]\\
    X&&Y
    \end{tikzcd}\]
    where $s\in\mathcal{S}$ (such a diagram will be denoted by $(Z;s,a)$).
    \item If $\mathcal{S}$ is a right multiplicative system, we denote by $M_{X,Y}^r$ the collection of diagrams of the form
    \[\begin{tikzcd}[row sep=5mm, column sep=5mm]
    X\ar[rd,swap,"a"]&&Y\ar[ld,"s"]\\
    &Z
    \end{tikzcd}\]
    where $s\in\mathcal{S}$ (such a diagram will be denoted by $(Z;a,s)$).
\end{itemize} 

We define an equivalence relation on $M_{X,Y}^l$ (resp. $M_{X,Y}^r$) as follows: $(Z;s,a)\sim(Z';s',a')$ (resp. $(Z;a,s)\sim(Z';a',s')$) if there eixsts a commutative diagram of the form
\[\begin{tikzcd}
&Z\ar[ld,swap,"s",tail]\ar[rd,"a"]\\
X&W\ar[l,tail]\ar[r]\ar[u]\ar[d]&Y\\
&Z'\ar[lu,"s'",tail]\ar[ru,swap,"a'"]&
\end{tikzcd}\quad\quad \text{resp.}
\begin{tikzcd}
&Z\ar[d]\\
X\ar[ru,"a"]\ar[rd,swap,"a'"]\ar[r]&W&Y\ar[ld,"s'",tail]\ar[l,tail]\ar[lu,swap,"s",tail]\\
&Z'\ar[u]&
\end{tikzcd}\]
(we use $\rightarrowtail$ to indicates a morphism in $\mathcal{S}$.) We also note that for any morphism $f:X\to Y$, axiom (S1) implies that $(X;\id_X,f)\in M_{X,Y}^l$ and $(X;f,\id_Y)\in M_{X,Y}^r$.

\begin{lemma}\label{category localization Hom set char by colim}
Let $\mathcal{S}$ be a left (resp. right) multiplicative system. For any object $X,Y$ of $\mathcal{C}$, we have a bijection
\begin{gather*}
M_{X,Y}^l/\sim \stackrel{\sim}{\to} \rlim_{(Z\to X)\in\Ob(\mathcal{S}_{/X}^{\op})}\Hom(Z,Y),\quad [Z;s,a]\mapsto [a:Z\to Y],\\
M_{X,Y}^r/\sim \stackrel{\sim}{\to} \rlim_{(Y\to Z)\in\Ob(\mathcal{S}_{Y/})}\Hom(X,Z),\quad [Z;a,s]\mapsto [a:X\to Z].
\end{gather*}
\end{lemma}
\begin{proof}
We consider the functor $\alpha:\mathcal{S}_{/X}^{\op}\to\mathbf{Set}$ given by $(Z\to X)\mapsto\Hom(Z,Y)$. Then by definition we have
\[M_{X,Y}^l=\coprod_{Z\to X}\alpha(Z\to X).\]
On the other hand, it is not hard to see that the equivalence relation $\sim$ on $M_{X,Y}^l$ is induced from the limit $\rlim\Hom(Z,Y)$, so the claim follows.
\end{proof}

For a left multiplicative system $\mathcal{S}$, any objects $X,Y$ of $\mathcal{C}$ and $(U;s,a)\in M_{X,Y}^l$ and $(V;t,b)\in M_{Y,Z}^l$, by axiom (S3) we have a commutative diagram
\begin{equation}\label{category localization composition law def}
\begin{tikzcd}[row sep=5mm, column sep=5mm]
&&W\ar[ld,swap,"r",tail]\ar[rd,"c"]&&\\
&U\ar[ld,swap,"s",tail]\ar[rd,"a"]&&V\ar[ld,swap,"t",tail]\ar[rd,"b"]&\\
X&&Y&&Z
\end{tikzcd}
\end{equation}
We now define the composition of $(U;s,a)$ and $(V;t,b)$ to be the equivalent class of $(W;sr,bc)$.

\begin{proposition}\label{category localization composition law}
The composition law defined above is associative and only depends on the equivalent class of $(U;s,a)$ and $(V;t,b)$. Also, a similar result holds if $\mathcal{S}$ is a right multiplicative system.
\end{proposition}
\begin{proof}
We first fix the diagram $(V;t,b)$. In view of the definition, it suffices to prove that in the following diagram
\begin{equation}\label{category localization composition law-1}
\begin{tikzcd}
&U\ar[ld,swap,"s",tail]\ar[rd,"a"]&W\ar[l,swap,"r",tail]\ar[rd,"c"]&&\\
X&&Y&V\ar[l,swap,"t",tail]\ar[r,"b"]&Z\\
&U'\ar[lu,"s'",tail]\ar[ru,swap,"a'"]\ar[uu,"x"]&W'\ar[l,"r'",tail]\ar[ru,swap,"c'"]&&
\end{tikzcd}
\end{equation}
we have $(W;sr,bc)\sim(W';s'r',bc')$. For this, we apply axiom (S3) twice to obtain the following solid diagram:
\begin{equation}\label{category localization composition law-2}
\begin{tikzcd}[row sep=7mm, column sep=8mm]
&U\ar[rd,"a"]&W\ar[l,swap,"r",tail]\ar[rd,"c"{anchor=south}]&&\\
&&Y&V\ar[l,swap,"t",tail]&\\
&U'\ar[ru,swap,"a'"{anchor=north},pos=0.6]\ar[uu,"x"]&W'\ar[l,"r'",tail]\ar[ru,swap,"c'"{anchor=north}]&&\\
&&&&R\ar[lluuu,bend right=35pt,"q",dashed,pos=0.4]\ar[lllu,"p",tail,bend left=15pt,swap,dashed]&Q\ar[l,"h",tail,swap,dashed]\ar[lllu,"k",dashed,bend right=15pt]&P\ar[l,dashed,swap,"w",tail]
\end{tikzcd}
\end{equation}
From the diagram (\ref{category localization composition law-1}), we then conclude that $tc'k=a'r'k$, so by (S4) there is a morphism $w:P\to Q$ in $\mathcal{S}$ such that $c'kw=cqhw$. Now, it is not hard to verify that the following diagram commutes:
\[\begin{tikzcd}[row sep=15mm, column sep=15mm]
&W\ar[ld,swap,"sr",tail]\ar[rd,"bc"]&\\
X&P\ar[r,"bc'kw"description]\ar[l,swap,"s'phw"description,tail]\ar[u,"qhw"description]\ar[d,"kw"description]&Z\\
&W'\ar[lu,"s'r'",tail]\ar[ru,swap,"bc'"]&
\end{tikzcd}\]
so $(W;sr,bc)\sim(W';s'r',bc')$. A similar argument proves the case where $(U;s,a)$ is fixed, and the same result holds for right multiplicative systems.\par
We now verify that associativity, so let $(A;s,a)\in M_{X,Y}^l$, $(B;t,b)\in M_{Y,Z}^l$, and $(C;u,c)\in M_{Z,W}^l$. Apply axiom (S3) three times, we obtain a diagram
\[\begin{tikzcd}[row sep=5mm, column sep=5mm]
&&&\bullet\ar[rd]\ar[ld,tail]&&&\\
&&\bullet\ar[ld,tail]\ar[rd]&&\bullet\ar[rd]\ar[ld,tail]\\
&A\ar[ld,swap,tail,"s"]\ar[rd,"a"]&&B\ar[rd,"b"]\ar[ld,swap,"t"]&&C\ar[rd,"c"]\ar[ld,swap,"u",tail]&\\
X&&Y&&Z&&W
\end{tikzcd}\]
which can be considered as an element of $M_{X,W}^l$ and witnesses the associativity:
\begin{equation*}
[C;u,c]\circ([B;t,b]\circ[A;s,a])=([C;u,c]\circ[B;t,c])\circ[A;s,a].\qedhere
\end{equation*}
\end{proof}

\begin{definition}
Let $\mathcal{S}$ be a left multiplicative system of a category $\mathcal{C}$. We define $\mathcal{C}_\mathcal{S}^l$ to be the (big) category with objects $\Ob(\mathcal{C})$, and morphisms from $X$ to $Y$ given by $M_{X,Y}^l/\sim$. The identity morphism of $X$ is given by $[X;\id_X,\id_X]$, and the composition law is determined by \cref{category localization composition law}. Moreover, we define a functor $Q^l:\mathcal{C}\to\mathcal{C}_\mathcal{S}^l$ such that it is the identity on objects and sends a morphism $f:X\to Y$ to $[X;\id_X,f]$. Similarly, if $\mathcal{S}$ be a right multiplicative system, we can define a functor $Q^r:\mathcal{C}\to\mathcal{C}_\mathcal{S}^r$.
\end{definition}

It then remains to check that $(\mathcal{C}_\mathcal{S}^l,Q^l)$ (resp. $(\mathcal{C}_\mathcal{S}^r,Q^r)$) is our desired localization of $\mathcal{C}$ with respect to $\mathcal{S}$. For this, we shall use the following lemma:
\begin{lemma}\label{category localization faithful on functor lemma}
Let $Q:\mathcal{C}\to\mathcal{C}'$ and $G:\mathcal{C}'\to\mathcal{A}$ be two functors. Assume that for any $X\in\Ob(\mathcal{C}')$, there exist $Y\in\Ob(\mathcal{C})$ and a morphism $s:X\to Q(Y)$ which satisfy the following two properties:
\begin{enumerate}
    \item[(a)] $G(s)$ is an isomorphism;
    \item[(b)] for any $Y'\in\mathcal{C}$ and any morphism $t:X\to Q(Y')$, there exists $Y''\in\mathcal{C}$ and morphisms $s':Y'\to Y''$, $t':Y\to Y''$ in $\mathcal{C}$ such that $G(Q(s'))$ is an isomorphism and the following diagram commutes:
    \[\begin{tikzcd}
    X\ar[r,"s"]\ar[d,swap,"t"]&Q(Y)\ar[d,"Q(t')"]\\
    Q(Y')\ar[r,"Q(s')"]&Q(Y'')
    \end{tikzcd}\]
\end{enumerate}
Then the canonical map
\begin{equation}\label{category localization faithful on functor lemma-1}
\Hom_{\Fun(\mathcal{C}',\mathcal{A})}(F,G)\to\Hom_{\Fun(\mathcal{C},\mathcal{A})}(F\circ Q,G\circ Q)
\end{equation}
is bijective for any functor $F:\mathcal{C}'\to\mathcal{A}$.
\end{lemma}
\begin{proof}
Let $\theta_1$ and $\theta_2$ be two morphisms from $F$ to $G$ and assume that $\theta_1(Q(Y))=\theta_2(Q(Y))$ for all $Y\in\mathcal{C}$. For $X\in\mathcal{C}'$, choose a morphism $s:X\to Q(Y)$ such that $G(s)$ is an isomorphism, and consider the commutative diagram for $i=1,2$:
\[\begin{tikzcd}
F(X)\ar[r,"\theta_i(X)"]\ar[d,swap,"F(s)"]&G(X)\ar[d,"G(s)"]\\
F(Q(Y))\ar[r,"\theta_i(Q(Y))"]&G(Q(Y))
\end{tikzcd}\]
Since $G(s)$ is an isomorphism, we conclude from the hypothesis that $\theta_1(X)=\theta_2(X)$, so (\ref{category localization faithful on functor lemma-1}) is injective (it is given by horizontal composition).\par
Now let $\theta:F\circ Q\to G\circ Q$ be a morphism of functors. For each $X\in\mathcal{C}'$, we choose a morphism $s:X\to Q(Y)$ satisfying conditions (a) and (b), and define a morphism $\gamma(X):F(X)\to G(X)$ by
\[\gamma(X)=(G(s))^{-1}\circ \theta(Y)\circ F(s).\]
Let us prove that this construction is functorial, and in particular, does not depend on the choice of the morphism $s:X\to Q(Y)$ (take $f=\id_X$ in the proof). Once this is done, we obtain a morphism $\gamma:F\to G$ of functors which satisfies $\gamma(Q(Y))=\theta(Y)$ (since we can choose $s=\id_{Q(Y)}$ in this case), so (\ref{category localization faithful on functor lemma-1}) is surjective.\par
To this end, let $f:X_1\to X_2$ be a morphism in $\mathcal{C}'$. For any choice of morphisms $s_1:X_1\to Q(Y_1)$ and $s_2:X_2\to Q(Y_2)$ satisfying the given conditions, we can apply (b) to the morphisms $s_1:X_1\to Q(Y_1)$ and $s_2\circ f:X_1\to Q(Y_2)$; we then obtain morphisms $t_1:Y_1\to Y_3$ and $t_2:Y_2\to Y_3$ such that $G(Q(t_2))$ is an isomorphism and $Q(t_1)\circ s_1=Q(t_2)\circ s_2\circ f$. We then obtain a commutative diagram
\[\begin{tikzcd}[row sep=4mm, column sep=4mm]
F(X_1)\ar[rd,swap,"F(s_1)"]\ar[dddd,"F(f)"]\ar[rrrrr,"\gamma(X_1)"]&&&&&G(X_1)\ar[ld,swap,"G(s_1)","\sim"']\ar[dddd,"G(f)"]\\
&F(Q(Y_1))\ar[rd,"F(Q(t_1))"]\ar[rrr,"\theta(Y_1)"]&&&G(Q(Y_1))\ar[ld,swap,"G(Q(t_1))"]&\\
&&F(Q(Y_3))\ar[r,"\theta(Y_3)"]&G(Q(Y_3))&&\\
&F(Q(Y_2))\ar[ru,swap,"F(Q(t_2))"]\ar[rrr,"\theta(Y_2)"]&&&G(Q(Y_2))\ar[lu,"G(Q(t_2))"]&\\
F(X_2)\ar[ru,swap,"F(s_2)"]\ar[rrrrr,"\gamma(X_2)"]&&&&&G(X_2)\ar[lu,"G(s_2)","\sim"']
\end{tikzcd}\]
Since all the internal diagrams commute, the outer square also commutes, which proves our assertion.
\end{proof}

\begin{theorem}[\textbf{P. Gabriel, M. Zisman}]
Assume that $\mathcal{S}$ is a left multiplicative system. Then $(\mathcal{C}_\mathcal{S}^l,Q^l)$ (resp. $(\mathcal{C}_\mathcal{S}^r,Q^r)$) define a localization of $\mathcal{C}$ with respect to $\mathcal{S}$.
\end{theorem}
\begin{proof}
It suffices to verify the universal properties for $(\mathcal{C}_\mathcal{S}^l,Q^l)$. Write $Q=Q^l$, then if $f:X\to Y$ belongs to $\mathcal{S}$, the diagram
\[\begin{tikzcd}
&X\ar[ld,swap,"f",tail]\ar[d,"f"description]\ar[rd,"f"]&\\
Y\ar[r,equal]&Y\ar[r,equal]&Y
\end{tikzcd}\]
suggests that $[X;f,f]=[Y,\id_Y,\id_Y]$; on the other hand, the following diagram
\[\begin{tikzcd}[row sep=5mm, column sep=5mm]
&&X\ar[rd,equal]\ar[ld,equal]&&\\
&X\ar[ld,equal]\ar[rd,"f"]&&X\ar[ld,swap,"f",tail]\ar[rd,equal]&\\
X&&Y&&X
\end{tikzcd}\quad 
\begin{tikzcd}[row sep=5mm, column sep=5mm]
&&X\ar[rd,equal]\ar[ld,equal]&&\\
&X\ar[rd,equal]\ar[ld,swap,"f"]&&X\ar[rd,"f",tail]\ar[ld,equal]&\\
X&&Y&&X
\end{tikzcd}
\]
proves that $[X;f,\id_X]=[X;\id_X,f]^{-1}$, so the functor $Q$ sends the elements in $\mathcal{S}$ to isomorphisms.\par
Now consider a functor $F:\mathcal{C}\to\mathcal{A}$ such that $F(s)$ is an isomorphism for $s\in\mathcal{S}$. For any $X\in\Ob(\mathcal{C}_\mathcal{S})=\Ob(\mathcal{C})$, we define $F_\mathcal{S}(X)=F(X)$, and consider the morphisms
\begin{align*}
F_\mathcal{S}:\Hom_{\mathcal{C}_\mathcal{S}^l}(X,Y)\to \Hom_\mathcal{A}(F(X),F(Y)),\quad [U;s,a]\mapsto F(a)(F(s))^{-1}.
\end{align*}
It is clear that $F_\mathcal{S}$ sends identities to identities, and by applying $F$ to the diagram (\ref{category localization composition law def}), we see that $F_\mathcal{S}$ preserves compositions, so we obtain a functor $F_\mathcal{S}:\mathcal{C}_\mathcal{S}\to\mathcal{A}$. It is clear that $F_\mathcal{S}\circ Q\cong F$, and $F_\mathcal{S}$ is unique up to isomorphisms.\par
Finally, with the notations of \cref{category localization faithful on functor lemma}, we can choose $Y\in\Ob(\mathcal{C})$ such that $X=Q(Y)$ and $s=\id_{Q(Y)}$. Then any morphism $t:Q(Y)\to Q(Y')$ is given by morphisms
\[\begin{tikzcd}
Y\ar[r,"t'"]&Y''&Y'\ar[l,swap,"s'",tail]
\end{tikzcd}\]
and the diagram in \cref{category localization faithful on functor lemma} commutes.
\end{proof}

\begin{remark}
If $\mathcal{S}$ is both a left multiplicative system and right multiplicative system, then by \cref{category localization opposite char}, the two localizations of $\mathcal{C}$ are equivalent, and we simply denote it by $\mathcal{C}_\mathcal{S}$.
\end{remark}

\begin{corollary}\label{category localization morphism image equal iff}
Let $\mathcal{S}$ be a left (resp. right) multiplicative system, then two morphisms $f,g:X\rightrightarrows Y$ satisfy $Q^l(f)=Q^l(g)$ (resp. $Q^r(f)=Q^r(g)$) if and only if there exists $s\in\mathcal{S}$ such that $fs=gs$ (resp. $sf=sg$).
\end{corollary}
\begin{proof}
Let $\mathcal{S}$ be a left multiplicative system. Then if $Q^l(f)=Q^l(g)$, we have a commutative diagram
\[\begin{tikzcd}
&X\ar[rd,"f"]\ar[ld,equal]&\\
X&U\ar[l,"s",tail,swap]\ar[u]\ar[d]\ar[r,"a"]&Y\\
&X\ar[ru,swap,"g"]\ar[lu,equal]&
\end{tikzcd}\]
It then follows that $fs=a=gs$, whence the corollary.
\end{proof}

\begin{corollary}\label{category localization functor mono and epi}
Let $\mathcal{S}$ be a left (resp. right) multiplicative system. Then the functor $Q^l$ (resp. $Q^r$) sends monomorphisms to monomorphisms, and epimorphisms to epimorphisms.
\end{corollary}
\begin{proof}
We only consider a left multiplicative system $\mathcal{S}$. Let $f:X\to Y$ be a monomorphism in $\mathcal{C}$ and $\alpha,\beta:Q^l(W)\to Q^l(X)$ be two morphisms in $\mathcal{C}_\mathcal{S}^l$ such that $(Q^l(f))\alpha=(Q^l(f))\beta$. Then by (S2) and (S3), we can write $\alpha=(U;s,a)$ and $\beta=(U;s,b)$, and it then follows that $Q^l(fa)=Q^l(fb)$, so by \cref{category localization morphism image equal iff}, there exists $t\in\mathcal{S}$ such that $fat=fbt$, whence $at=bt$ and $\alpha=\beta$.
\end{proof}

\begin{remark}
We note that the category $\mathcal{C}_\mathcal{S}^l$ (resp. $\mathcal{C}_\mathcal{S}^r$) may not be small, since the collection $M_{X,Y}^l$ (resp. $M_{X,Y}^r$) is too big. However, if the collection $\mathcal{S}$ has a cofinal subset, then by \cref{category localization Hom set char by colim}, we can restrict the inductive limit to this set, and then $\mathcal{C}_\mathcal{S}^l$ (resp. $\mathcal{C}_\mathcal{S}^r$) will be small. This is the case if $\mathcal{C}$ itself is already small. 
\end{remark}

We now give some properties of the localization functor $Q$. For this, assume that $\mathcal{S}$ is a left (resp. right) multiplicative system and let $X\in\mathcal{C}$. We define a functor
\[\theta_{/X}:\mathcal{S}_{/X}\to\mathcal{C}_{Q(X)/}\quad (\text{resp.\ } \theta_{X/}:\mathcal{S}_{X/}\to\mathcal{C}_{/Q(X)})\]
by associating a morphism $s:Y\to X$ in $\mathcal{S}_{/X}$ (resp. a morphism $s:X\to Y$ in $\mathcal{S}_{X/}$) with the morphism $Q(s)^{-1}:Q(X)\to Q(Y)$ in $\mathcal{C}_{Q(X)/}$ (resp. the morphism $Q(s)^{-1}:Q(Y)\to Q(X)$ in $\mathcal{C}_{Q(X)/}$).

\begin{lemma}\label{category localization functor theta cofinal}
Assume that $\mathcal{S}$ is a left (resp. right) multiplicative system and let $X\in\mathcal{C}$. Then the functor $\theta_{/X}^{\op}$ (resp. $\theta_{X/}$) is cofinal.
\end{lemma}
\begin{proof}
We only consider right multiplicative systems. 
\end{proof}

\begin{proposition}\label{category localization functor exactness prop}
Let $\mathcal{S}$ be a left (resp. right) multiplicative system and $Q:\mathcal{C}\to\mathcal{C}_\mathcal{S}$ be the corresponding localization functor.
\begin{enumerate}
    \item[(a)] The functor $Q$ is left (resp. right) exact.
    \item[(b)] Let $\alpha:I\to\mathcal{C}$ be a projective (resp. inductive) system in $\mathcal{C}$ indexed by a finite category $I$. Assume that $\llim\alpha$ (resp. $\rlim\alpha$) exists in $\mathcal{C}$, then $\llim(Q\circ\alpha)$ (resp. $\rlim(Q\circ\alpha)$) exists in $\mathcal{C}_S$ and is isomorphic to $Q(\llim\alpha$) (resp. $Q(\rlim\alpha)$).
    \item[(c)] Assume that $\mathcal{C}$ admits kernels (resp. cokernels). Then $\mathcal{C}_\mathcal{S}$ admits kernels (resp. cokernels) and $Q$ commutes with kernels (resp. cokernels).
    \item[(d)] Assume that $\mathcal{C}$ admits finite products (resp. coproducts). Then $\mathcal{C}_\mathcal{S}$ admits finite products (resp. coproducts) and $Q$ commutes with finite products (resp. coproducts).
    \item[(e)] If $\mathcal{C}$ admits finite projective (resp. inductive) limits, then so does $\mathcal{C}_\mathcal{S}$.
\end{enumerate}
\end{proposition}

\begin{proposition}\label{category localization of subcategory prop}
Let $\mathcal{C}$ be a category, $\mathcal{I}$ be a full subcategory, $\mathcal{S}$ be a left (resp. right) multiplicative system in $\mathcal{C}$, and $\mathcal{T}$ be the family of morphisms in $\mathcal{I}$ which belong to $\mathcal{S}$.
\begin{enumerate}
    \item[(a)] Assume that $\mathcal{T}$ is a left (resp. right) multiplicative system in $\mathcal{I}$. Then there is a well-defined functor $\mathcal{I}^l_\mathcal{T}\to\mathcal{C}^l_\mathcal{S}$ (resp. $\mathcal{I}^r_\mathcal{T}\to\mathcal{C}^r_\mathcal{S}$).
    \item[(b)] Assume that for every $f:X\to Y$ in $\mathcal{S}$ with $Y\in\mathcal{I}$ (resp. $X\in\mathcal{I}$), there exist a morphism $g:W\to X$ with $W\in\mathcal{I}$ and $fg\in\mathcal{S}$ (resp. a morphism $g:Y\to W$ with $W\in\mathcal{I}$ and $gf\in\mathcal{S}$). Then $\mathcal{T}$ is a left (resp. right) multiplicative system and the functor $\mathcal{I}^l_\mathcal{T}\to\mathcal{C}^l_\mathcal{S}$ (resp. $\mathcal{I}^r_\mathcal{T}\to\mathcal{C}^r_\mathcal{S}$) is fully faithful.
\end{enumerate}
\end{proposition}
\begin{proof}
Assertion (a) is clear from the definition, and as for (b), it is easy to verify that $\mathcal{T}$ is a left multiplicative system under the corresonding assumption. For $X\in\mathcal{I}$, we define the category $\mathcal{T}_{/X}$ as the full subcategory of $\mathcal{S}_{/X}$ whose objects are morphisms $s:Y\to X$ with $Y\in\mathcal{I}$. The hypothesis in (b) then amounts to saying that the functor $\mathcal{T}_{/X}\to\mathcal{S}_{/X}$ is cofinal, so the result follows from \cref{*}.
\end{proof}

\begin{corollary}\label{category localization of subcategory equivalent if}
Let $\mathcal{C}$ be a category, $\mathcal{I}$ a full subcategory, $\mathcal{S}$ be a left (resp. right) multiplicative system in $\mathcal{C}$, $\mathcal{T}$ the family of morphisms in $\mathcal{I}$ which belong to $\mathcal{S}$. Assume that for any $X\in\mathcal{C}$ there exists a morphism $s:I\to X$ with $I\in\mathcal{I}$ and $s\in\mathcal{S}$ (resp. a morphism $s:X\to I$ with $I\in\mathcal{I}$ and $s\in\mathcal{S}$). Then $\mathcal{T}$ is a left (resp. right) multiplicative system and $\mathcal{I}_\mathcal{T}^l$ (resp. $\mathcal{I}_\mathcal{T}^r$) is equivalent to $\mathcal{C}_\mathcal{S}^l$ (resp. $\mathcal{C}_\mathcal{S}^r$).
\end{corollary}
\begin{proof}
The natural functor $\mathcal{I}^l_\mathcal{T}\to\mathcal{C}^l_\mathcal{S}$ (resp. $\mathcal{I}^r_\mathcal{T}\to\mathcal{C}^r_\mathcal{S}$) is fully faithful by \cref{category localization of subcategory prop}, and essentially surjective by hypothesis.
\end{proof}

\begin{theorem}\label{category localization additive prop}
Let $\mathcal{C}$ be a pre-additive category and $\mathcal{S}$ be a left (resp. right) multiplicative system.
\begin{enumerate}
    \item[(a)] The localization $\mathcal{C}_\mathcal{S}^l$ (resp. $\mathcal{C}_\mathcal{S}^r$) has a canonical structure of a pre-additive category, so that $Q^l$ (resp. $Q^r$) is an additive functor.
    \item[(b)] If $\mathcal{C}$ is additive and $\mathcal{S}$ be a multiplicative system, then $\mathcal{C}_\mathcal{S}$ is an additive category.  
\end{enumerate}
The same result is true if we replace additive by $k$-linear, where $k$ is a commutative ring.
\end{theorem}
\begin{proof}
As for (a), it suffices to consider right multiplicative systems. We now define an addition for the Hom set of $\mathcal{C}_\mathcal{S}$. If $f,g\in\Hom_{\mathcal{C}_\mathcal{S}}(X,Y)$, then, since $\mathcal{S}_{/X}^{\op}$ is filtrant, there exist $s:U\rightarrowtail X$ and $a_1,a_2:U\to Y$ such that $f=[U;s,a_1]$ and $g=[U;s,a_2]$. We can therefore define $f+g$ by
\[f+g:=[U;s,a_1+a_2]\in\Hom_{\mathcal{C}_\mathcal{S}}(X,Y).\]
In particular, the zero morphism can be written as $[U;s,0]$, and $-[U;s,a]=[U;s,-a]$. It is then a simple matter to show that this definition is independent of the choices of $a_1$ and $a_2$, which follows easily from the filtrant property of $\mathcal{S}_{/X}^{\op}$. Finally, with this definition, it is then easy to check that $Q$ is an additive functor, and the second assertion follows from \cref{category localization functor exactness prop}.
\end{proof}

\subsection{Kan extensions along a localization}
Let $\mathcal{C}$ be a category, $\mathcal{S}$ a (right, say) multiplicative system in $\mathcal{C}$ and $F:\mathcal{C}\to\mathcal{A}$ a functor. We consider the existence of the following factorization diagram:
\[\begin{tikzcd}
\mathcal{C}\ar[rd,bend left=25pt,"F"]\ar[d,swap,"Q"]&\\
\mathcal{C}_\mathcal{S}\ar[r,dashed,"\exists ?"]&\mathcal{A}
\end{tikzcd}\]
In general, $F$ does not send morphisms in $\mathcal{S}$ to isomorphisms in $\mathcal{A}$, so it does not factorize through $\mathcal{C}_\mathcal{S}$. It is however possible in some cases to define a localization of $F$ as a "best approximation", in the following sense:
\begin{definition}
Let $\mathcal{S}$ be a family of morphisms in $\mathcal{C}$ and assume that the localization $Q:\mathcal{C}\to\mathcal{C}_\mathcal{S}$ exists.
\begin{enumerate}
    \item[(a)] We say that $F$ is \textbf{right localizable} if the left Kan extension $\Lan_QF$ of $F$ with respect to $Q$ exists. In such a case, we say that $\Lan_QF$ is a \textbf{right localization} of F and we denote it by $R_\mathcal{S}F$. In other words, the \textbf{right localization} of $F$ is a functor $R_\mathcal{S}F:\mathcal{C}_\mathcal{S}\to\mathcal{A}$ together with a morphism of functors $\eta:F\to R_\mathcal{S}F\circ Q$ such that for any functor $G:\mathcal{C}_\mathcal{S}\to\mathcal{A}$, the map
    \[\Hom_{\Fun(\mathcal{C}_\mathcal{S},\mathcal{A})}(R_\mathcal{S}F,G)\to\Hom_{\Fun(\mathcal{C},\mathcal{A})}(F,G\circ Q)\]
    is bijective (This map is given by composing with $\eta$). 
    \item[(b)] We say that $F$ is universally right localizable if for any functor $K:\mathcal{A}\to\mathcal{B}$, the functor $K\circ F$ is localizable and $R_\mathcal{S}(K\circ F)\stackrel{\sim}{\to} K\circ R_\mathcal{S}F$.
\end{enumerate}
\end{definition}
We can similarly define left localizations of $F$ by right Kan extensions, and consider universally left localizable functors. That is, the left localization of $F$ is a functor $L_\mathcal{S}F:\mathcal{C}_\mathcal{S}\to\mathcal{A}$ together with a morphism $\eps:L_\mathcal{S}F\circ Q\to F$ such that for any functor $G:\mathcal{C}_\mathcal{S}\to\mathcal{A}$, $\eps$ induces a bijection
\[\Hom_{\Fun(\mathcal{C}_\mathcal{S},\mathcal{A})}(G,L_\mathcal{S}F)\to\Hom_{\Fun(\mathcal{C},\mathcal{A})}(G\circ Q,F).\]

One should be aware that even if $F$ admits both a right and a left localization, the two localizations are not isomorphic in general. However, when the localization $Q:\mathcal{C}\to\mathcal{C}_\mathcal{S}$ exists and $F$ is right and left localizable, the canonical morphisms of functors $L_\mathcal{S}F\circ Q\to F\to R_\mathcal{S}F\circ Q$ together with the isomorphism $\Hom(L_\mathcal{S}F\circ Q,R_\mathcal{S}F\circ Q)\cong\Hom(L_\mathcal{S}F,R_\mathcal{S}F)$ in (L3) gives a canonical morphism of functors $L_\mathcal{S}F\to R_\mathcal{S}F$. From now on, we shall concentrate on right localizations.

\begin{proposition}\label{category localization of functor exist if}
Let $\mathcal{C}$ be a category, $\mathcal{I}$ be a full subcategory, $\mathcal{S}$ be a left (resp. right) multiplicative system in $\mathcal{S}$, $\mathcal{T}$ be the family of morphisms in $\mathcal{I}$ which belong to $\mathcal{S}$. Let $F:\mathcal{C}\to\mathcal{A}$ be a functor. Assume that the following "resolution condition" is satisfied:
\begin{enumerate}
    \item[(a)] for any $X\in\mathcal{C}$, there exists $s:I\to X$ (resp. $s:X\to I$) with $I\in\mathcal{I}$ and $s\in\mathcal{S}$;
    \item[(b)] for any $t\in\mathcal{T}$, $F(t)$ is an isomorphism.
\end{enumerate}
Then $F$ is universally left (resp. right) localizable and the composition 
\[\begin{tikzcd}[column sep=12mm]
\mathcal{I}\ar[r]&\mathcal{C}\ar[r,"Q"]\ar[r]&\mathcal{C}_\mathcal{S}\ar[r,"\text{$L_\mathcal{S}F$ or $R_\mathcal{S}F$}"]&\mathcal{A}
\end{tikzcd}\]
is isomorphic to the restriction of $F$ to $\mathcal{I}$. Moreover, we have canonical isomorphisms
\begin{gather}
(L_\mathcal{S}F)(Q(X)) \stackrel{\sim}{\to} \llim_{[Y\rightarrowtail X]\in\Ob(\mathcal{S}_{/X}^{\op})}F(Y),\label{category localization of functor exist if-1}\\
(R_\mathcal{S}F)(Q(X)) \stackrel{\sim}{\to} \rlim_{[X\rightarrowtail Y]\in\Ob(\mathcal{S}_{X/})}F(Y).\label{category localization of functor exist if-2}
\end{gather}
and the morphism $\eps:L_\mathcal{S}F\circ Q\to F$ (resp. $\eta:F\to R_\mathcal{S}F\circ Q$) is given by projecting to the term $F(X)$ corresponding to the identity morphism $\id_X\in\Ob(\mathcal{S}_{/X}^{\op})$ (resp. $\id_X\in\Ob(\mathcal{S}_{X/})$).
\end{proposition}
\begin{proof}
It suffices to consider right multiplicative systems. Denote by $\iota:\mathcal{I}\to\mathcal{C}$ the natural functor. By condition (a) and \cref{category localization of subcategory equivalent if}, $\iota_Q:\mathcal{I}_\mathcal{T}\to\mathcal{C}_\mathcal{S}$ is an equivalence, and condition (b) implies that the localization $F_\mathcal{T}$ of $F\circ\iota$ exists. We consider the solid diagram
\[\begin{tikzcd}
&\mathcal{C}\ar[rd,swap,"Q_\mathcal{S}"]\ar[rrd,bend left=10pt,"F"]&&\\
\mathcal{I}\ar[ru,swap,"\iota"]\ar[rd,"Q_\mathcal{T}"]&&\mathcal{C}_\mathcal{S}\ar[r,dashed,"RF"]&\mathcal{A}\\
&\mathcal{I}_\mathcal{T}\ar[ru,"\iota_Q"]\ar[rru,swap,bend right=10pt,"F_\mathcal{T}"]
\end{tikzcd}\]
Denote by $\iota_Q^{-1}$ a quasi-inverse of $\iota_Q$ and set $RF=F_\mathcal{T}\circ\iota_Q^{-1}$. Then the above diagram commutes, except the triangle $(\mathcal{C},\mathcal{C}_\mathcal{S},\mathcal{A})$. We now prove that $RF$ is the right localization of $F$.\par
Let $G:\mathcal{C}_\mathcal{S}\to\mathcal{A}$ be a functor; we have the chain of a morphism and isomorphisms:
\begin{equation}\label{category localization of functor exist if-3}
\begin{aligned}
\Hom_{\Fun(\mathcal{C},\mathcal{A})}(F,G\circ Q_\mathcal{S})& \stackrel{\lambda}{\to} \Hom_{\Fun(\mathcal{I},\mathcal{A})}(F\circ\iota,G\circ Q_\mathcal{S}\circ\iota)\\
&\cong\Hom_{\Fun(\mathcal{I},\mathcal{A})}(F_\mathcal{T}\circ Q_\mathcal{T},G\circ\iota_Q\circ Q_\mathcal{T})\\
&\cong\Hom_{\Fun(\mathcal{I}_\mathcal{T},\mathcal{A})}(F_\mathcal{T},G\circ\iota_Q)\\
&\cong\Hom_{\Fun(\mathcal{C}_\mathcal{S},\mathcal{A})}(F_\mathcal{T}\circ\iota_Q^{-1},G)\\
&\cong\Hom_{\Fun(\mathcal{C}_\mathcal{S},\mathcal{A})}(RF,G).
\end{aligned}
\end{equation}
The second isomorphism follows from the fact that $Q_\mathcal{T}$ satisfies axiom (L3). To conclude, it remains to prove that the morphism $\lambda$ is bijective. Let us check that \cref{category localization faithful on functor lemma} applies to $\iota:\mathcal{I}\to\mathcal{C}$ and $Q_\mathcal{S}:\mathcal{C}\to\mathcal{C}_\mathcal{S}$, and hence to $\iota:\mathcal{I}\to\mathcal{C}$ and $G\circ Q_\mathcal{S}:\mathcal{C}\to\mathcal{A}$. Let $X\in\Ob(\mathcal{C})$; by hypothesis, there exists $Y\in\mathcal{I}$ and $s:X\to\iota(Y)$ with $s\in\mathcal{S}$. Then $F(s)$ is an isomorphism and condition (a) of \cref{category localization faithful on functor lemma} is satisfied. On the other hand, condition (b) follows from axiom (S3') and the fact that $\iota$ is fully faithful. Finally, to see that the limit of (\ref{category localization of functor exist if-2}) exists, we can assume that $X\in\Ob(\mathcal{I})$, but the limit is then isomorphic to $F(X)$, since $\id_X$ is initial in $\Ob(\mathcal{S}_{X/})$. In view of the general construction of $\Lan_QF$ and \cref{category localization functor theta cofinal}, it follows that $R_\mathcal{S}F$ is isomorphic to the limit in (\ref{category localization of functor exist if-2}).\par
If $K:\mathcal{A}\to\mathcal{A}'$ is another functor, $K\circ F(t)$ will be an isomorphism for any $t\in\mathcal{T}$. Hence, $K\circ F$ is localizable and we have
\begin{equation*}
R_\mathcal{S}(K\circ F)\cong (K\circ F)_\mathcal{T}\circ\iota_Q^{-1}\cong K\circ F_\mathcal{T}\circ\iota_Q^{-1}\cong K\circ R_\mathcal{S}F.\qedhere
\end{equation*}
\end{proof}

\begin{corollary}
Let $\mathcal{A}$ be a category which admits small filtrant inductive limits. Let $\mathcal{S}$ be a left (resp. right) multiplicative system and assume that for each $X\in\mathcal{C}$, the category $\mathcal{S}_{/X}^{\op}$ (resp. $\mathcal{S}_{X/}$) is cofinally small.
\begin{enumerate}
    \item[(a)] $\mathcal{C}_\mathcal{S}$ is a $\mathscr{U}$-category.
    \item[(b)] The functor $Q^\star$ admits a right adjoint $_\star Q$ (resp. left adjoint functor $Q_\star$).
    \item[(c)] Any functor $F:\mathcal{C}\to\mathcal{A}$ is left (resp. right) localizable and
    \begin{gather*}
    (L_\mathcal{S}F)(Q(X)) \stackrel{\sim}{\to} \llim_{[Y\rightarrowtail X]\in\Ob(\mathcal{S}_{/X}^{\op})}F(Y),\quad (R_\mathcal{S}F)(Q(X)) \stackrel{\sim}{\to} \rlim_{[X\rightarrowtail Y]\in\Ob(\mathcal{S}_{X/})}F(Y).
    \end{gather*}
\end{enumerate}
\end{corollary}
\begin{proof}
Assertion (a) is obvious and (b), (c) follow from \cref{category localization functor theta cofinal}, since we may apply \cref{category localization of functor exist if} to construct $_\star Q$ (resp. $Q_\star$).
\end{proof}

\subsection{Triangulated categories}
Triangulated categories are additive categories with a collection of distingushied triangles. They arise naturally from the derived category of an abelian category and is important for the study of properties of derived categories. To begin with, we first consider categories with a translation functor.
\begin{definition}
A \textbf{category with translation} $(\mathcal{D},T)$ is a category $\mathcal{D}$ endowed with an equivalence of categories $T:\mathcal{D}\to\mathcal{D}$. The functor $T$ is called the \textbf{translation functor}.
\begin{itemize}
    \item A functor of categories with translation $F:(\mathcal{D},T)\to (\mathcal{D}',T')$ is a functor $F:\mathcal{D}\to\mathcal{D}'$ together with an isomorphism $F\circ T\cong T'\circ F$. If $\mathcal{D}$ and $\mathcal{D}'$ are additive categories and $F$ is additive, we say that $F$ is a functor of additive categories with translation.
    \item Let $F,F':(\mathcal{D},T)\to (\mathcal{D}',T')$ be two functors of categories with translation. A morphism $\theta:F\to F'$ of functors of categories with translation is a morphism of functors such that the diagram below commutes
    \[\begin{tikzcd}
    F\circ T\ar[r,"\theta\circ T"]\ar[d,swap,"\sim"]&F'\circ T\ar[d,"\sim"]\\
    T'\circ F\ar[r,"T'\circ\theta"]&T'\circ F'
    \end{tikzcd}\]
    \item A subcategory with translation $(\mathcal{D}',T')$ of $(\mathcal{D},T)$ is a category with translation such that $\mathcal{D}'$ is a subcategory of $\mathcal{D}$ and the translation functor $T'$ is the restriction of $T$.
    \item Let $(\mathcal{D},T)$, $(\mathcal{D}',T')$ and $(\mathcal{D}'',T'')$ be additive categories with translation. A bifunctor of additive categories with translation $F:\mathcal{D}\times\mathcal{D}'\to\mathcal{D}''$ is an additive bifunctor endowed with functorial isomorphisms
    \[\theta_{X,Y}:F(T(X),Y) \stackrel{\sim}{\to} T''(F(X,Y)),\quad \lambda_{X,Y}:F(X,T'(Y))\stackrel{\sim}{\to} T''(F(X,Y))\]
    for $(X,Y)\in\mathcal{D}\times\mathcal{D}'$ such that the diagram below anti-commutes:
    \[\begin{tikzcd}
    F(T(X),T'(Y))\ar[r,"\theta_{X,T'(Y)}"]\ar[d,swap,"\lambda_{T(X),Y}"]&T''(F(X,T'(Y)))\ar[d,"T''(\lambda_{X,Y})"]\\
    T''(F(T(X),Y))\ar[r,"T''(\theta_{X,Y})"]&T''^2(F(X,Y))
    \end{tikzcd}\]
\end{itemize}
\end{definition}

If $(\mathcal{D},T)$ is a category with translation, we shall denote by $T^{-1}$ a quasi-inverse of $T$. Then $T^n$ is well defined for $n\in\Z$. These functors are unique up to unique isomorphism. If there is no risk of confusion, we shall write $\mathcal{D}$ instead of $(\mathcal{D},T)$ and $X[1]$ (resp. $X[-1]$) instead of $T(X)$ (resp. $T^{-1}(X)$).
\begin{example}
Let $\mathcal{A}$ be an additive category and $\Ch(A)$ be the category of chain complexes of $\mathcal{A}$. Then we have the shift functor $T:X\mapsto X[1]$ defined by $X[1]^n=X^{n+1}$ and $d[1]^n=-d^{n+1}$, so $(\Ch(\mathcal{A}),T)$ is an additive category with translation. 
\end{example}

\begin{definition}
Let $(\mathcal{D},T)$ be an additive category with translations. A \textbf{triangle} in $\mathcal{D}$ is a sequence of morphisms
\[\begin{tikzcd}
X\ar[r,"f"]&Y\ar[r,"g"]&Z\ar[r,"h"]&X[1]
\end{tikzcd}\]
A morphism of triangles is a commutative diagram
\[\begin{tikzcd}
X\ar[r,"f"]\ar[d,"\alpha"]&Y\ar[r,"g"]\ar[d,"\beta"]&Z\ar[r,"h"]\ar[d,"\gamma"]&X[1]\ar[d,"{\alpha[1]}"]\\
X'\ar[r,"f'"]&Y'\ar[r,"g'"]&Z'\ar[r,"h'"]&X'[1]
\end{tikzcd}\]
\end{definition}

\begin{remark}
Let $(\mathcal{D},T)$ be a $k$-linear category with translations and
\[\begin{tikzcd}
X\ar[r,"f"]&Y\ar[r,"g"]&Z\ar[r,"h"]&X[1]
\end{tikzcd}\]
be a triangle. Let $\eps,\zeta,\eta\in k^\times$. If $\eps\zeta\eta=1$, then the original triangle is isomorphic to the following:
\[\begin{tikzcd}
X\ar[r,"\eps f"]&Y\ar[r,"\zeta g"]&Z\ar[r,"\eta h"]&X[1]
\end{tikzcd}\]
In fact, we have a commutative diagram
\[\begin{tikzcd}
X\ar[r,"f"]\ar[d,equal]&Y\ar[r,"g"]\ar[d,"\eps"]&Z\ar[r,"h"]\ar[d,"\eps\zeta"]&X[1]\ar[d,"\eps\zeta\eta"]\\
X\ar[r,"\eps f"]&Y\ar[r,"\zeta g"]&Z\ar[r,"\eta h"]&X[1]
\end{tikzcd}\]
\end{remark}

\begin{definition}
A \textbf{triangulated category} is an additive category $(\mathcal{D},T)$ with translation endowed with a family of triangles, called \textbf{distinguished triangles}, satisfying the axioms below:
\begin{enumerate}[leftmargin=40pt]
    \item[(TR0)] A triangle isomorphic to a distinguished triangle is a distinguished triangle.
    \item[(TR1)] The triangle $X\stackrel{\id_X}{\longrightarrow}X\to 0\to X[1]$ is a distinguished triangle.
    \item[(TR2)] For any morphism $f:X\to Y$, there exists a distinguished triangle $X\stackrel{f}{\to} Y\to Z\to X[1]$.
    \item[(TR3)] A triangle $X\stackrel{f}{\to} Y\stackrel{g}{\to} Z\stackrel{h}{\to} X[1]$ is a distinguished triangle if and only if its "rotation"
    \[\begin{tikzcd}
    Y\ar[r,"-g"]&Z\ar[r,"-h"]&X[1]\ar[r,"{-f[1]}"]&Y[1]
    \end{tikzcd}\]
    is a distinguished triangle.
    \item[(TR4)] Given a solid diagram
    \[\begin{tikzcd}
    X\ar[r]\ar[d,"\alpha"]&Y\ar[r]\ar[d,"\beta"]&Z\ar[r]\ar[d,dashed,"\gamma"]&X[1]\ar[d,dashed,"{\alpha[1]}"]\\
    X'\ar[r]&Y'\ar[r]&Z'\ar[r]&X'[1]
    \end{tikzcd}\]
    with both rows being distinguished triangles, there exists a morphism $\gamma:Z\to Z'$ giving rise to a morphisms of distinguished triangles.
    \item[(TR5)] Given three distinguished triangles
    \begin{gather*}
    X\stackrel{f}{\to} Y\to Z'\to X[1],\\
    Y\stackrel{g}{\to} Z\to X'\to Y[1],\\
    X\stackrel{gf}{\to} Z\to Y'\to X[1],
    \end{gather*}
    there exists a distinguished triangle $Z'\to Y'\to X'\to Z'[1]$ making the following diagram commutative:
    \begin{equation}\label{triangle cat octahedron diagram}
    \begin{tikzcd}
    X\ar[r,"f"]\ar[d,equal]&Y\ar[r]\ar[d,"g"]&Z'\ar[r]\ar[d,dashed]&X[1]\ar[d,equal]\\
    X\ar[r,"gf"]\ar[d,"f"]&Z\ar[r]\ar[d,equal]&Y'\ar[r]\ar[d,dashed]&X[1]\ar[d]\\
    Y\ar[r,"g"]\ar[d]&Z\ar[r]\ar[d]&X'\ar[r]\ar[d,equal]&Y[1]\ar[d]\\
    Z'\ar[r,dashed]&Y'\ar[r,dashed]&X'\ar[r,dashed]&Z'[1]
    \end{tikzcd}
    \end{equation}
    Diagram (\ref{triangle cat octahedron diagram}) is often called the \textbf{octahedron diagram}. Indeed, it can be written using the vertices of an octahedron:
    \[\begin{tikzcd}
    &&Y'&&\\
    Z'&&&&X'\\
    X&&&&Z\\
    &&Y&&
    \arrow[from=2-1,to=1-3,dashed]
    \arrow[from=1-3,to=2-5,dashed]
    \arrow[from=2-1,to=3-1,swap,"+1"]
    \arrow[from=3-5,to=2-5]
    \arrow[from=2-5,to=4-3,swap,"+1",yshift=1ex]
    \arrow[from=4-3,to=2-1]
    \arrow[from=4-3,to=3-5,swap,"g"]
    \arrow[from=2-5,to=2-1,swap,"+1"]
    \arrow[from=3-5,to=1-3,crossing over]
    \arrow[from=3-1,to=4-3,swap,"f"]
    \arrow[from=1-3,to=3-1,crossing over]
    \arrow[from=3-1,to=3-5,crossing over]
    \end{tikzcd}\]
    Here we use $X'\stackrel{+1}{\to} Y$ to denote a morphism $X'\to Y[1]$.
\end{enumerate}
\end{definition}
An additive category $(\mathcal{D},T)$ satisfying (TR0)--(TR4) is called a \textbf{pretriangulated category}. One should note that the morphism $\gamma$ in (TR4) is not unique, and is unique up to \textit{non-unique} isomorphisms.

\begin{definition}
A \textbf{triangulated functor} of triangulated categories $F:(\mathcal{D},T)\to(\mathcal{D}',T')$ is a functor of additive categories with translation sending distinguished triangles to distinguished triangles. If moreover $F$ is an equivalence of categories, $F$ is called an \textbf{equivalence of triangulated categories}. If $F,F':(\mathcal{D},T)\to(\mathcal{D}',T')$ are triangulated functors, a morphism $\theta:F\to F'$ of triangulated functors is a morphism of functors of additive categories with translation.\par
A triangulated subcategory $(\mathcal{D}',T')$ of $(\mathcal{D},T)$ is an additive subcategory with translation of $\mathcal{D}$ (i.e., the functor $T'$ is the restriction of $T$) such that it is triangulated and that the inclusion functor is triangulated.
\end{definition}

\begin{remark}
A triangle $X\stackrel{f}{\to} Y\stackrel{g}{\to} Z\stackrel{h}{\to} X[1]$ is called \textbf{anti-distinguished} if the triangle $X\stackrel{f}{\to} Y\stackrel{g}{\to} Z\stackrel{-h}{\to} X[1]$ is distinguished. Then $(\mathcal{D},T)$ endowed with the family of anti-distinguished triangles is triangulated. If we denote by $(\mathcal{D}^{ant},T)$ this triangulated category, then $(\mathcal{D}^{ant},T)$ and $(\mathcal{D},T)$ are equivalent as triangulated categories.
\end{remark}

\begin{remark}
Consider the contravariant functor $\op:\mathcal{D}\to\mathcal{D}^{\op}$, and define
\[T^{\op}=\op\circ T^{-1}\circ\op:\mathcal{D}^{\op}\to\mathcal{D}^{\op}\]
(we use the fact that $\op^2=\id_\mathcal{D}$.) A triangle $X\stackrel{f}{\to} Y\stackrel{g}{\to} Z\stackrel{h}{\to} T^{\op}(X)$ in $\mathcal{D}^{\op}$ is called distinguished if its image
\[\begin{tikzcd}
Z^{\op}\ar[r,"g^{\op}"]&Y^{\op}\ar[r,"f^{\op}"]&X^{\op}\ar[r,"{T(h^{\op})}"]&T(Z^{\op})
\end{tikzcd}
\]
by $\op$ is distinguished. With this definition, it is easy to check that $(\mathcal{D}^{\op},T^{\op})$ is a triangulated category.
\end{remark}

\begin{proposition}\label{triangle cat dt composition zero}
If $X\stackrel{f}{\to} Y\stackrel{g}{\to} Z\to X[1]$ is a distinguished triangle, then $gf=0$.
\end{proposition}
\begin{proof}
Applying (TR1) and (TR4) we get a commutative diagram
\[\begin{tikzcd}
X\ar[r,equal]\ar[d,equal]&X\ar[d,"f"]\ar[r]&0\ar[d]\ar[r]&X[1]\ar[d,equal]\\
X\ar[r,"f"]&Y\ar[r,"g"]&Z\ar[r]&X[1]
\end{tikzcd}\]
Then $gf$ factorizes through $0$.
\end{proof}

\begin{definition}
Let $(\mathcal{D},T)$ be a pretriangulated category and $\mathcal{C}$ an abelian category. An additive functor $F:\mathcal{D}\to\mathcal{C}$ is called \textbf{cohomological} if for any distinguished triangle $X\to Y\to Z\to X[1]$ in $\mathcal{D}$, the sequence $F(X)\to F(Y)\to F(Z)$ is exact in $\mathcal{C}$.
\end{definition}

If $F$ is a cohomological functor $F:\mathcal{D}\to\mathcal{C}$, then for any distinguished triangle $X\to Y\to Z\to X[1]$ in $\mathcal{D}$, by rotating the triangle by (TR3), we obtain a long exact sequence
\[\begin{tikzcd}
\cdots\ar[r]&F(Z[-1])\ar[r]&F(X)\ar[r]&F(Y)\ar[r]&F(Z)\ar[r]&F(X[1])\ar[r]&\cdots
\end{tikzcd}\]

A basic example of cohomological functors is the Hom functor:
\begin{proposition}\label{triangle cat Hom functor cohomological}
Let $(\mathcal{D},T)$ be a pretriangulated category and $S$ be an object of $\mathcal{D}$. Then the functors $\Hom_\mathcal{D}(S,-)$ and $\Hom_\mathcal{D}(-,S)$ are cohomological.
\end{proposition}
\begin{proof}
Let $X\to Y\to Z\to X[1]$ be a distinguished triangle. We want to show that
\[\begin{tikzcd}
\Hom(S,X)\ar[r,"f_*"]&\Hom(S,Y)\ar[r]&\Hom(S,Z)
\end{tikzcd}\]
is exact, i.e. for any morphism $\varphi:S\to Y$ such that $g\circ\varphi=0$, there exists a morphism $\psi:S\to X$ such that $\varphi=f\circ\psi$. This is equivalent to say that the solid diagram below may be completed:
\[\begin{tikzcd}
S\ar[d,dashed]\ar[r,equal]&S\ar[r]\ar[d]&0\ar[r]\ar[d]&S[1]\ar[d,dashed]\\
X\ar[r,"f"]&Y\ar[r,"g"]&Z\ar[r]&X[1]
\end{tikzcd}\]
and this follows from (TR4) and (TR3). By replacing $\mathcal{D}$ with $\mathcal{D}^{\op}$, we obtain the assertion for $\Hom_\mathcal{D}(-,S)$.
\end{proof}
\begin{corollary}\label{triangle cat zero dt isomorphism}
For a distinguished triangle $X\stackrel{f}{\to} Y\to 0\to X[1]$ in a pretriangulated category, $f$ must be an isomorphism.
\end{corollary}
\begin{proof}
For every object $S$ of $\mathcal{D}$, by \cref{triangle cat Hom functor cohomological} we have an exact sequence
\[\begin{tikzcd}
\Hom(S,0[-1])=0\ar[r]&\Hom(S,X)\ar[r,"f_*"]&\Hom(S,Y)\ar[r]&\Hom(S,0)=0
\end{tikzcd}\]
so $f_*$ is an isomorphism, which means $f$ is an isomorphism.
\end{proof}

\begin{proposition}\label{triangle cat morphism dt isomorphism 2 of 3}
Let $(\mathcal{D},T)$ be a pretriangulated category and consider a morphism of distinguished triangle:
\[\begin{tikzcd}
X\ar[r]\ar[d,"\alpha"]&Y\ar[r]\ar[d,"\beta"]&Z\ar[r]\ar[d,dashed,"\gamma"]&X[1]\ar[d,dashed,"{\alpha[1]}"]\\
X'\ar[r]&Y'\ar[r]&Z'\ar[r]&X'[1]
\end{tikzcd}\]
If two of $\alpha,\beta,\gamma$ are isomorphisms, then so is the third one.
\end{proposition}
\begin{proof}
By rotating the triangle, we may assume that $\alpha,\gamma$ are isomorphisms. To show that $\beta$ is an isomorphism, it suffices to show that for any object $S$ of $\mathcal{D}$, the map $\beta_*:\Hom(S,Y)\to\Hom(S,Y')$ is an isomorphism. Now by \cref{triangle cat Hom functor cohomological} we have a commutative diagram with exact rows:
\[\begin{tikzcd}[column sep=5mm]
\Hom(S,Z[-1])\ar[r]\ar[d,"\sim"]&\Hom(S,X)\ar[r]\ar[d,"\sim"]&\Hom(S,Y)\ar[r]\ar[d,"\beta_*"]&\Hom(S,Z)\ar[r]\ar[d,"\sim"]&\Hom(S,X[1])\ar[d,"\sim"]\\
\Hom(S,Z'[-1])\ar[r]&\Hom(S,X')\ar[r]&\Hom(S,Y')\ar[r]&\Hom(S,Z')\ar[r]&\Hom(S,X'[1])
\end{tikzcd}\]
so the claim follows from five lemma.
\end{proof}

\begin{corollary}\label{triangle cat dt of subcategory prop}
Let $\mathcal{D}'$ be a full pretriangulated subcategory of $\mathcal{D}$.
\begin{enumerate}
    \item[(a)] Consider a triangle $X\stackrel{f}{\to} Y\to Z\to X[1]$ in $\mathcal{D}'$ and assume that this triangle is distinguished in $\mathcal{D}$. Then it is distinguished in $\mathcal{D}'$.
    \item[(b)] Consider a distinguished triangle $X\to Y\to Z\to X[1]$ in $\mathcal{D}$ with $X,Y$ in $\mathcal{D}'$. Then $Z$ is isomorphic to an object of $\mathcal{D}'$.
\end{enumerate}
\end{corollary}
\begin{proof}
In the situation of (a), there exists a distinguished triangle $X\stackrel{f}{\to} Y\to Z'\to X[1]$ in $\mathcal{D}'$, and $X\stackrel{f}{\to} Y\to Z\to X[1]$ is isomorphic to it in $\mathcal{D}$ in view of axiom (TR4) and \cref{triangle cat morphism dt isomorphism 2 of 3}. The second assertion follows from (a).
\end{proof}

\begin{corollary}\label{triangle cat morphism dt unique prop}
The distinguished triangle $X\stackrel{f}{\to} Y\to Z\to X[1]$ in (TR2) is unique up to (non-canonical) isomorphisms. 
\end{corollary}
\begin{proof}
For distinguished triangles $X\stackrel{f}{\to} Y\to Z\to X[1]$ and $X\stackrel{f}{\to} Y\to Z\to X[1]$, axiom (TR4) gives a morphism $\gamma$ such that the diagram
\[\begin{tikzcd}
X\ar[r]\ar[d,equal]&Y\ar[r]\ar[d,equal]&Z\ar[r]\ar[d,dashed,"\gamma"]&X[1]\ar[d,equal]\\
X'\ar[r]&Y'\ar[r]&Z'\ar[r]&X'[1]
\end{tikzcd}\]
is commutative. It then suffices to apply \cref{triangle cat morphism dt isomorphism 2 of 3}. 
\end{proof}

By \cref{triangle cat morphism dt unique prop}, we see that the object $Z$ given in (TR2) is unique up to isomorphism. As already mentioned, the fact that this isomorphism is not unique is the source of many difficulties (e.g., gluing problems in sheaf theory). Let us give a criterion which ensures, in some very special cases, the uniqueness of the third term of a distinguished triangle.

\begin{proposition}\label{triangle cat TR4 unique morphism if}
In the situation of (TR4), assume that $\Hom_\mathcal{D}(Y,X')=0$ and $\Hom_\mathcal{D}(X[1],Y')=0$. Then $\gamma$ is unique.
\end{proposition}
\begin{proof}
We may replace $\alpha$ and $\beta$ by the zero morphisms and prove that in this case, $\gamma$ is zero:
\[\begin{tikzcd}
X\ar[r]\ar[d,"0"]&Y\ar[r,"f"]\ar[d,"0"]&Z\ar[r,"g"]\ar[d,"\gamma"]&X[1]\ar[d,"0"]\\
X'\ar[r,"f'"]&Y'\ar[r,"g'"]&Z'\ar[r,"h'"]&X'[1]
\end{tikzcd}\]
Since $h'\gamma=0$, by \cref{triangle cat Hom functor cohomological} the morphism $\gamma$ factorizes through $g'$, i.e. there exists $u:Z\to Y'$ with $\gamma=g'\circ u$. Similarly, since $\gamma g=0$, $\gamma$ factorizes through $h$ so there exists $v:X[1]\to Z'$ with $\gamma=vh$. By (TR4), there then exists a morphism $w:Y[1]\to X'[1]$ defining a morphism
\[\begin{tikzcd}
Y\ar[r,"g"]&Z\ar[r,"h"]\ar[ld,swap,"u"]&X[1]\ar[r,"{-f[1]}"]\ar[ld,swap,"v"]&Y[1]\ar[ld,swap,"w"]\\
Y'\ar[r,"g'"]&Z'\ar[r,"h'"]&X'[1]\ar[r]&Y'[1]
\end{tikzcd}\]
By hypothesis we have $w=0$, so $v$ factorizes through $Y'$, and this implies $v=0$ by our hypothesis, whence $\gamma=0$.
\end{proof}

\begin{proposition}\label{triangle cat triangulated functor exact}
Let $F:\mathcal{D}\to\mathcal{D}'$ be a triangulated functor between pretriangulated categories. Then $F$ is exact.
\end{proposition}
\begin{proof}
Since $F^{\op}:\mathcal{D}'^{\op}\to\mathcal{D}^{\op}$ is also a triangulated functor between pretriangulated categories, it suffices to prove that $F$ is left exact, that is, for any $X\in\mathcal{D}'$, the category $\mathcal{D}_{/X}$ is filtrant.\par
The category $\mathcal{D}_{/X}$ is nonempty since it contains the object $0\to X$, and if $(Y_1,s_1)$ and $(Y_2,s_2)$ are two objects of $\mathcal{D}_{/X}$ with $Y_i\in\mathcal{D}$ and $s_i:F(Y_i)\to X$, $i=1,2$, we obtain a morphism $s:F(Y_1\oplus Y_2)\to X$, whence morphisms $(Y_i,s_i)\to (Y_1\oplus Y_2,s)$ for $i=1,2$. Finally, consider morphisms $f,g:(Y,s)\rightrightarrows (Y',s')$ in $\mathcal{D}_{/X}$. We can embed $f-g:Y\to Y'$ into a distinguished triangle
\[\begin{tikzcd}
Y\ar[r,"f-g"]&Y'\ar[r,"h"]&Z\ar[r]&Y[1]
\end{tikzcd}\]
Since $s'F(f)=s'F(g)$, \cref{triangle cat Hom functor cohomological} implies that the morphism $s':F(Y')\to X$ factorizes as $F(Y')\to F(Z)\stackrel{t}{\to} X$, so the compositions $(Y,s)\rightrightarrows (Y',s')\to (Z,t)$ coincide, and this proves that $\mathcal{D}_{/X}$ is filtrant. 
\end{proof}

\begin{proposition}\label{triangle cat sum product of dt}
Let $\mathcal{D}$ be a pretriangulated category which admits direct sums (resp. products) indexed by a set $I$. Then direct sums indexed by $I$ commute with the translation functor $T$, and a direct sum (resp. products) of distinguished triangles indexed by $I$ is a distinguished triangle.
\end{proposition}
\begin{proof}
The first assertion is obvious since $T$ is an equivalence of categories. Now let $D_i:X_i\to Y_i\to Z_i\to X_i[1]$ be a family of distinguished triangles indexed by $I$, and $D$ be the triangle
\[\bigoplus_iD_i:\bigoplus_iX_i\to \bigoplus_iY_i\to \bigoplus_iZ_i\to \bigoplus_iX_i[1].\]
By (TR2) there exists a distinguished triangle $D':\bigoplus_iX_i\to \bigoplus_iY_i\to Z\to (\bigoplus_iX_i)[1]$, and by (TR3) there exists morphisms of triangles $D_i\to D'$ such that they induces a morphism $D\to D'$. Let $S\in\mathcal{D}$, we show that the morphism $\Hom_\mathcal{D}(D',S)\to\Hom_\mathcal{D}(D,S)$ is an isomorphism, which then implies the isomorphism $D\cong D'$. Consider the commutative diagram
\begin{small}
\[\begin{tikzcd}[column sep=3.3mm]
\Hom((\bigoplus_iY_i)[1],S)\ar[r]\ar[d]&\Hom((\bigoplus_iX_i)[1],S)\ar[r]\ar[d]&\Hom(Z,S)\ar[r]\ar[d]&\Hom(\bigoplus_iY_i,S)\ar[r]\ar[d]&\Hom(\bigoplus_iX_i,S)\ar[d]\\
\Hom(\bigoplus_iY_i[1],S)\ar[r]&\Hom(\bigoplus_iX_i[1],S)\ar[r]&\Hom(\bigoplus_iZ_i,S)\ar[r]&\Hom(\bigoplus_iY_i,S)\ar[r]&\Hom(\bigoplus_iX_i,S)
\end{tikzcd}\]
\end{small}
The first row is exact since the functor $\Hom$ is cohomological, and the second row is isomorphic to
\begin{small}
\[\begin{tikzcd}[column sep=4mm]
\prod_i\Hom(Y_i[1],S)\ar[r]&\prod_i\Hom(X_i[1],S)\ar[r]&\prod_i\Hom(Z_i,S)\ar[r]&\prod_i\Hom(Y_i,S)\ar[r]&\prod_i\Hom(X_i,S)
\end{tikzcd}\]
\end{small}
Since the functor $\prod_i$ is exact on $\mathbf{Ab}$, this complex is exact. Now the vertical arrows except the middle one are isomorphisms, so the middle one is an isomorphism by five lemma.
\end{proof}

\begin{corollary}\label{triangle cat split exact is dt}
Let $\mathcal{D}$ be a pretriangulated category. Then a triangle of the form $X\stackrel{\iota_1}{\to} X\oplus Y\stackrel{p_2}{\to} Y\stackrel{0}{\to} X[1]$ is distinguished. Conversely, if a morphisn in a distinguished triangle is zero, then this triangle comes from a direct sum.
\end{corollary}
\begin{proof}
To prove the first assertion, it suffices to apply \cref{triangle cat sum product of dt} to the distinguished triangles $X\stackrel{\id_X}{\longrightarrow} X\to 0\to X[1]$ and $0\to Y\stackrel{\id_Y}{\longrightarrow} Y\to 0$. Now consider the second assertion; by rotating the triangle, it suffices to consider a distinguished triangle of the form
\[X\to M\to Y\stackrel{0}{\to} X[1].\]
By (TR4), we then obtain a morphism of distinguished triangles:
\[\begin{tikzcd}
X\ar[r,"\iota_1"]\ar[d,equal]&X\oplus Y\ar[d,"\alpha"]\ar[r,"p_2"]&Y\ar[r,"0"]\ar[d,equal]&X[1]\ar[d,equal]\\
X\ar[r]&M\ar[r]&Y\ar[r,"0"]&X[1]
\end{tikzcd}\]
and it follows from \cref{triangle cat morphism dt isomorphism 2 of 3} that $\alpha$ is an isomorphism.
\end{proof} 

\begin{corollary}\label{triangle cat functor is additive}
Let $F:(\mathcal{D},T)\to(\mathcal{D}',T')$ be a functor between pretriangulated categories such that $F$ sends distinguished triangles to distinguished triangles. Then $F$ is additive, so it is a triangulated functor.
\end{corollary}
\begin{proof}
From the distinguished triangle $0\to 0\to 0\to 0$, we obtain a distinguished triangle $F(0)\stackrel{\id}{\to} F(0)\stackrel{\id}{\to} F(0)\stackrel{\id}{\to} F(0)$ in $\mathcal{D}'$, so \cref{triangle cat dt composition zero} implies that $\id_{F(0)}=\id_{F(0)}\circ\id_{F(0)}=0$, and therefore $F(0)\cong 0$. This also shows that $F$ sends zero morphisms to zero morphisms.\par
Now consider a distinguished triangle $X\stackrel{\iota_1}{\to} X\oplus Y\stackrel{p_2}{\to} Y\stackrel{0}{\to} X[1]$ in $\mathcal{D}$. By applying $F$, we obtain a distinguished triangle
\[\begin{tikzcd}
F(X)\ar[r,"F(\iota_1)"]&F(X\oplus Y)\ar[r]&F(Y)\ar[r,"0"]&T(F(X))
\end{tikzcd}\]
in $\mathcal{D}'$. From \cref{triangle cat split exact is dt} and its proof, it is easy to see that the canonical morphism $F(X)\oplus F(Y)\to F(X\oplus Y)$ is an isomorphism, so $F$ is additive.
\end{proof}

\begin{proposition}[\textbf{Verdier's Exercise}]
Let $\mathcal{D}$ be a triangulated category. Then any commutative diagram
\[\begin{tikzcd}
X\ar[d]\ar[r]&Y\ar[d]\\
X'\ar[r]&Y'
\end{tikzcd}\]
can be extended into a diagram
\[\begin{tikzcd}
X\ar[d]\ar[r]&Y\ar[d]\ar[r]&Z\ar[r]\ar[d]&X[1]\ar[d]\\
X'\ar[r]\ar[d]&Y'\ar[r]\ar[d]&Z'\ar[r]\ar[d]&X'[1]\ar[d]\\
X''\ar[r]\ar[d]&Y''\ar[r]\ar[d]&Z''\ar[r]\ar[rd,phantom,"\ast"]\ar[d]&X''[1]\ar[d]\\
X[1]\ar[r]&Y[1]\ar[r]&Z[1]\ar[r]&X[2]
\end{tikzcd}\]
so that every rows and columns are distinguished triangles and every square is commutative except the one labeled by $\ast$, which is anti-commutative.
\end{proposition}
\subsection{Verdier quotient}
Let $\mathcal{D}$ be a triangulated category and $\mathcal{N}$ a full saturated subcategory. Recall that $\mathcal{N}$ is saturated if $X\in\mathcal{D}$ belongs to $\mathcal{N}$ whenever $X$ is isomorphic to an object of $\mathcal{N}$.

\begin{lemma}\label{triangle cat null system def lemma}
Let $\mathcal{N}$ be a full saturated triangulated subcategory of $\mathcal{D}$. Then $\Ob(\mathcal{N})$ satisfies the following conditions:
\begin{enumerate}[leftmargin=40pt]
    \item[(N1)] $0\in\mathcal{N}$.
    \item[(N2)] $X\in\mathcal{N}$ if and only if $X[1]\in\mathcal{N}$.
    \item[(N3)] If $X\to Y\to Z\to X[1]$ is a distinguished triangle in $\mathcal{D}$ and $X,Z\in\mathcal{N}$, then $Y\in\mathcal{N}$.
\end{enumerate}
Conversely, let $\mathcal{N}$ be a full saturated subcategory of $\mathcal{D}$ and assume that $\Ob(\mathcal{N})$ satisfies conditions (N1)--(N3) above. Then the restriction of $T$ and the collection of distinguished triangles $X\to Y\to Z\to X[1]$ with $X,Y,Z$ in $\mathcal{N}$ make $\mathcal{N}$ a full saturated triangulated subcategory of $\mathcal{D}$. Moreover it satisfies
\begin{enumerate}[leftmargin=40pt]
    \item[(N3')] If $X\to Y\to Z\to X[1]$ is a distinguished triangle in $\mathcal{D}$ and two objects among $X,Y,Z$ belong to $\mathcal{N}$, then so does the third one.
\end{enumerate}
\end{lemma}
\begin{proof}
Assume that $\mathcal{N}$ is a full saturated triangulated subcategory of $\mathcal{D}$. Then (N1) and (N2) are clearly satisfied. Moreover, (N3) follows from \cref{triangle cat dt of subcategory prop} and the hypothesis that $\mathcal{N}$ is saturated.\par
Conversely, let $\mathcal{N}$ be a full subcategory of $\mathcal{D}$ satisfying (N1)--(N3); then (N3') follows from (N2) and (N3) by rotating the triangle. We now show that $\mathcal{N}$ is saturated, so let $f:X\to Y$ be an isomorphism with $X\in\mathcal{N}$. The triangle $X\stackrel{f}{\to} Y\to 0\to X[1]$ is then isomorphic to the distinguished triangle $X\stackrel{\id_X}{\longrightarrow} X\to 0\to X[1]$:
\[\begin{tikzcd}
X\ar[r,"\id_X"]\ar[d,equal]&X\ar[r]\ar[d,"f","\sim"']&0\ar[r]\ar[d,equal]&X[1]\ar[d,equal]\\
X\ar[r,"f"]&Y\ar[r]&0\ar[r]&X[1]
\end{tikzcd}\]
and hence is distinguished. It then follows from (N3) that $Y\in\mathcal{N}$. On the other hand, since $X\to X\oplus Y\to Y\stackrel{0}{\to} X[1]$ is a distinguished triangle for $X,Y\in\mathcal{N}$, we find that $X\oplus Y\in\mathcal{N}$, so $\mathcal{N}$ is a full additive subcategory of $\mathcal{D}$. The axioms of triangulated categories are then easily checked.
\end{proof}

A \textbf{null system} in $\mathcal{D}$ is a full saturated subcategory N such that $\Ob(\mathcal{N})$ satisfies the conditions (N1)--(N3) in \cref{triangle cat null system def lemma}. By \cref{triangle cat null system def lemma}, $\mathcal{N}$ can be then considered as a triangulated subcategory of $\mathcal{D}$. We associate a family of morphisms to a null system as follows:
\begin{equation}\label{triangle cat null system associated multiplicative system-1}
\mathcal{N}Q=\{f:X\to Y:\text{there exists a distinguished triangle $X\to Y\to Z\to X[1]$ with $Z\in\mathcal{N}$}\}.
\end{equation}
The morphisms in $\mathcal{N}Q$ turn out to form a multiplicative system of $\mathcal{C}$ that is compatible with the distinguished triangles in $\mathcal{D}$. To make this precise, we introduce the following definition:
\begin{definition}
Let $\mathcal{S}$ be a multiplicative system of a triangulated category $\mathcal{D}$. Then $\mathcal{D}$ is said to be \textbf{compatible with the distinguished triangles} in $\mathcal{D}$ if it satisfies the following conditions:
\begin{enumerate}[leftmargin=40pt]
    \item[(ST1)] For any morphism $s:X\to Y$ in $\mathcal{D}$, $s\in\mathcal{S}$ if and only if $s[1]\in\mathcal{S}$.
    \item[(ST2)] Consider a solid diagram
    \[\begin{tikzcd}
    X\ar[r]\ar[d,"\alpha"]&Y\ar[r]\ar[d,"\beta"]&Z\ar[r]\ar[d,dashed,"\gamma"]&X[1]\ar[d,dashed]\\
    X'\ar[r]&Y'\ar[r]&Z'\ar[r]&X'[1]
    \end{tikzcd}\]
    if $\alpha,\beta\in\mathcal{S}$, then there exists a morphism $\gamma\in\mathcal{S}$ giving rise to a morphisms of distinguished triangles.
\end{enumerate}
\end{definition}
The importance of the compatibility of $\mathcal{S}$ with distinguished triangles is contained in the following proposition:
\begin{proposition}\label{triangle cat localization by compatible multiplicative system}
Let $\mathcal{S}$ be a multiplicative system of $\mathcal{D}$ that is compatible with the distinguished triangles. Then the localization $\mathcal{D}_\mathcal{S}$ has a uniquely determined triangulated category structure so that the localization functor $Q:\mathcal{D}\to\mathcal{D}_\mathcal{S}$ is triangulated.
\end{proposition}
\begin{proof}
Since $\mathcal{D}$ is additive, it follows from \cref{category localization additive prop} that the localization $\mathcal{D}/\mathcal{N}$ is additive, and $Q:\mathcal{D}\to\mathcal{D}/\mathcal{N}$ is an additive functor. As for the uniqueness of the triangulated category structure, it suffices to note that for any distinguished triangle $X\stackrel{f}{\to} Y\to Z\to X[1]$, by adjusting using isomorphisms in $\mathcal{C}_\mathcal{S}$, we can assume that $f:X\to Y$ is a morphism in $\mathcal{D}$. But it then follows from \cref{triangle cat morphism dt isomorphism 2 of 3} that this triangle is isomorphic to a distinguished triangle in $\mathcal{D}$.\par
We now define the distinguished triangles of $\mathcal{C}_\mathcal{S}$ as the images of that of $\mathcal{D}$ under $Q$. Axioms (TR0)--(TR3) follow directly from that of $\mathcal{D}$, so let's prove (TR4). With the notations of (TR3) and (Exercise), we may assume that there exists a commutative diagram in $\mathcal{D}$ of solid arrows, with horizontal arrows belong to $\mathcal{D}$:
\[\begin{tikzcd}
X\ar[r,"f"]\ar[d,"\tilde{\alpha}"]&Y\ar[r,"g"]\ar[d,"\tilde{\beta}"]&Z\ar[r]\ar[d,dashed,"\tilde{\gamma}"]&X[1]\ar[d,dashed,"{\tilde{\alpha}[1]}"]\\
A\ar[r]&B\ar[r,dashed]&C\ar[r,dashed]&A[1]\\
X'\ar[r,"f'"]\ar[u,swap,"s",tail]&Y'\ar[r,"g'"]\ar[u,swap,"t",tail]&Z'\ar[r]\ar[u,dashed,swap,"u",tail]&X'[1]\ar[u,dashed,swap,"{s[1]}",tail]
\end{tikzcd}\]
Now by applying (TR2) to the morphism $A\to B$, we obtain a distinguished triangle $A\to B\to C\to A[1]$, and by (ST2) there is a morphism $u:Z'\to C$ in $\mathcal{S}$ completing the lower square. Also, by (TR4) there is a morphism $\tilde{\gamma}:Z\to C$ completing the upper square, and we have construct the desired morphism of distinguished triangles in $\mathcal{D}_\mathcal{S}$. Finally, consider two morphisms $f:X\to Y$ and $g:Y\to Z$ in $\mathcal{D}_\mathcal{S}$, which we may assume to belong to $\mathcal{D}$. Then by applying (TR5) and take the iamge in $\mathcal{D}_\mathcal{S}$ of the octahedron diagram, we conclude that (TR5) holds for $\mathcal{D}_\mathcal{S}$.
\end{proof}

\begin{theorem}[\textbf{Verdier}]\label{triangle cat null system associated multiplicative system}
Let $\mathcal{N}$ be a null system in a triangulated category $\mathcal{D}$.
\begin{enumerate}
    \item[(\rmnum{1})] $\mathcal{N}Q$ is a multiplicative system compatible with distinguished triangles in $\mathcal{D}$.
    \item[(\rmnum{2})] Denote by $\mathcal{D}/\mathcal{N}$ the localization of $\mathcal{D}$ by $\mathcal{N}Q$ and by $Q:\mathcal{D}\to\mathcal{D}/\mathcal{N}$ the localization functor. Then $\mathcal{D}/\mathcal{N}$ is an additive category endowed with an automorphism (the image of $T$, still denoted by $T$), and there is a canonical triangulated structure on $\mathcal{D}_\mathcal{S}$ so that $\mathcal{D}/\mathcal{Q}$ is a triangulated category and $Q$ is a triangulated functor.
    \item[(\rmnum{3})] For a morphism $f:X\to Y$ in $\mathcal{D}$, we have $Q(f)=0$ if and only if $f$ factorizes through an object of $\mathcal{N}$. In particular, $Q(N)=0$ for $N\in\Ob(\mathcal{N})$.
    \item[(\rmnum{4})] For any pretriangulated category $\mathcal{D}'$ and any triangulated functor $F:\mathcal{D}\to\mathcal{D}'$ such that $F(X)\cong 0$ for any $X\in\mathcal{N}$, $F$ factors uniquely through $Q$.
    \item[(\rmnum{5})] For any abelian category $\mathcal{A}$ and any cohomological functor $H:\mathcal{D}\to\mathcal{A}$, if $H(N)=0$ for $N\in\Ob(\mathcal{N})$, then $H$ factors uniquely through $Q$.
\end{enumerate}
\end{theorem}
\begin{proof}
Since the opposite category of $\mathcal{D}$ is again triangulated and $\mathcal{N}^{\op}$ is a null system in $\mathcal{D}^{\op}$, it is enough to check that $\mathcal{N}Q$ is a right multiplicative system.
\begin{enumerate}[leftmargin=40pt]
    \item[(S1)] If $X\in\Ob(\mathcal{D})$, then $X\stackrel{\id_X}{\longrightarrow} X\to 0\to X[1]$ is distinguished in $\mathcal{D}$ by (TR1), so $\id_X\in\mathcal{N}Q$.
    \item[(S2)] Let $f:X\to Y$ and $g:Y\to Z$ be in $\mathcal{N}Q$. By (TR3), there are distinguished triangles
    \begin{gather*}
    X\stackrel{f}{\to} Y\to Z'\to X[1],\\
    Y\stackrel{g}{\to} Z\to X'\to Y[1],\\
    X\stackrel{gf}{\to} Z\to Y'\to X[1],
    \end{gather*}
    where we can assume that $Z',X'\in\mathcal{N}$. By (TR5), there exists a distinguished triangle $Z'\to Y'\to X'\to Z'[1]$, so $Y'\in\mathcal{N}$ in view of (N3).
    \item[(S3')] Let $f:X\to Y$ and $s:X\to X'$ be two morphisms with $s\in\mathcal{N}Q$. Then there exists a distinguished triangle $W\stackrel{h}{\to} X\stackrel{s}{\to} X'\to W[1]$ with $W\in\mathcal{N}$. By (TR2), there also exists a distinguished triangle $W\stackrel{fh}{\to} Y\stackrel{t}{\to} Z\to W[1]$, and we obtain a commutative diagram in view of (TR4):
    \[\begin{tikzcd}
    W\ar[r,"h"]\ar[d,equal]&X\ar[r,"s"]\ar[d,"f"]&X'\ar[r]\ar[d]&W[1]\ar[d,equal]\\
    W\ar[r,"fh"]&Y\ar[r,"t"]&Z\ar[r]&W[1]
    \end{tikzcd}\]
    Since $W\in\mathcal{N}$, we conclude that $t\in\mathcal{N}Q$.
    \item[(S4')] Replacing $f$ by $f-g$, it is enough to check that if there exists $s\in\mathcal{N}Q$ with $fs=0$, then there exists $t\in\mathcal{N}Q$ with $tf=0$. Consider the solid diagram
    \[\begin{tikzcd}
    X'\ar[r,"s"]&X\ar[r]\ar[rd,"f"]&Z\ar[r]\ar[d,dashed,"h"]&X'[1]\\
    &&Y\ar[d,dashed,"t"]&\\
    &&Y'
    \end{tikzcd}\]
    where the row is a distinguished triangle with $Z\in\mathcal{N}$. Since $fs=0$, the morphism $f$ factors thorugh $Z$ in view of \cref{triangle cat Hom functor cohomological}. There then exists a distinguished triangle $Z\to Y\stackrel{h}{\to} Y'\to Z[1]$ by (TR2), and we obtain that $t\in\mathcal{N}Q$ since $Z\in\mathcal{N}$. Finally, $th=0$ implies that $tf=0$ (cf. \cref{triangle cat dt composition zero}).
\end{enumerate}
It remains to see that $\mathcal{N}Q$ is compatible with distinguished triangles. For this, since $\mathcal{N}$ is closed under $T$, it is easy to see that $\mathcal{N}Q$ satisfies (ST1). Moreover, consider the diagram of (ST2) and assume that $\alpha,\beta\in\mathcal{N}Q$. Then by Verdier's Exercise, we have a commutative diagram
\[\begin{tikzcd}
{}&{}&{}\\
X''\ar[r]\ar[u,"+1"]&Y''\ar[r]\ar[u,"+1"]&Z''\ar[r,"+1"]\ar[u,"+1"]&{}\\
X'\ar[r]\ar[u]&Y'\ar[r]\ar[u]&Z'\ar[r,"+1"]\ar[u]&{}\\
X\ar[r]\ar[u,"\alpha"]&Y\ar[r]\ar[u,"\beta"]&Z\ar[u,"\gamma"]\ar[r,"+1"]&{}
\end{tikzcd}\]
so that every rows and columns are distinguished triangles and every square is commutative. By the saturality of $\mathcal{N}$ and \cref{triangle cat morphism dt unique prop}, we see that $X'',Y''\in\Ob(\mathcal{N})$, so it follows from (N3') that $Z''\in\mathcal{N}$, whence $\gamma\in\mathcal{N}Q$. Now from \cref{triangle cat localization by compatible multiplicative system}, we see that $\mathcal{D}/\mathcal{N}$ has a canonical structure of a triangulated category, and $Q:\mathcal{D}\to\mathcal{D}/\mathcal{N}$ is a triangulated functor.\par
As for (\rmnum{3}), consider a distinguished triangle $0\to N\to N\to 0$, where $N\in\mathcal{N}$. Then the morphism $0\to X$ belongs to $\mathcal{N}Q$, and hence is an isomorphism under $Q$. In particular, if $f:X\to Y$ can be decomposed into $X\to N\to Y$ with $N\in\mathcal{N}$, then $Q(f)=0$. Conversely, if $Q(f)=0$, then there exists a morphism $s:M\to X$ such that $s\in\mathcal{N}Q$ and $fs=0$ (\cref{category localization morphism image equal iff}). From the definition of $\mathcal{N}Q$, we have a solid commutative diagram
\[\begin{tikzcd}
M\ar[r,"s"]\ar[d]&X\ar[r]\ar[d,"f"]&N\ar[r]\ar[d,dashed]&M[1]\ar[d,dashed]\\
0\ar[r]&Y\ar[r,"\id_Y"]&Y\ar[r]&0
\end{tikzcd}\]
By (TR4), there exists a morphism $N\to Y$ giving rise to the commutative diagram, and we then obtain a decomposition $X\to N\to Y$ of $f$.\par
Now let $F:\mathcal{D}\to\mathcal{D}'$ be a triangulated functor, where $\mathcal{D}'$ is a pretriangulated category. Then for $s\in\mathcal{N}Q$, we have a distinguished triangle $X\stackrel{s}{\to} Y\to N\to X[1]$ such that $N\in\Ob(\mathcal{N})$, whence a distinguished triangle $F(X)\stackrel{F(s)}{\to} F(Y)\to 0\to F(X)[1]$ in $\mathcal{D}'$. By \cref{triangle cat zero dt isomorphism}, we conclude that $F(s)$ is an isomorphism, so there is a uniquely determined factorization $F=\widebar{F}\circ Q$, where $\widebar{F}$ is an additive functor. From the description of distinguished triangles in $\mathcal{N}Q$, it is easy to see that $\widebar{F}$ is a triangulated functor.\par
Finally, let $H:\mathcal{D}\to\mathcal{A}$ be a cohomological functor. By considering a distinguished triangle $X\stackrel{s}{\to} Y\to N\to X[1]$ such that $N\in\Ob(\mathcal{N})$, we obtain an exact sequence
\[\begin{tikzcd}
0=H(N[-1])\ar[r]&H(X)\ar[r,"H(s)"]&H(Y)\ar[r]&H(N)=0
\end{tikzcd}\]
so $H(s)$ is an isomorphism and we obtain a uniquely determined factorization $H=\widebar{H}\circ Q$. Similarly, it is immediate to check that $\widebar{H}$ is a cohomological functor.
\end{proof}

Now consider a full triangulated subcategory $\mathcal{I}$ of $\mathcal{D}$. We shall write $\mathcal{N}\cap\mathcal{I}$ for the full subcategory whose objects are $\Ob(\mathcal{N})\cap\Ob(\mathcal{I})$. This is clearly a null system in $\mathcal{I}$.

\begin{proposition}\label{triangle cat localization subcategory functor prop}
Let $\mathcal{D}$ be a triangulated category, $\mathcal{N}$ a null system, $\mathcal{I}$ a full triangulated subcategory of $\mathcal{D}$. Assume that one of the following conditions is true:
\begin{enumerate}
    \item[(a)] any morphism $Y\to Z$ with $Y\in\mathcal{I}$ and $Z\in\mathcal{N}$ factorizes as $Y\to Z'\to Z$ with $Z'\in\mathcal{N}\cap\mathcal{I}$;
    \item[(b)] any morphism $Z\to Y$ with $Y\in\mathcal{I}$ and $Z\in\mathcal{N}$ factorizes as $Z\to Z'\to Y$ with $Z'\in\mathcal{N}\cap\mathcal{I}$.
\end{enumerate}
Then $\mathcal{I}/(\mathcal{N}\cap\mathcal{I})\to\mathcal{D}/\mathcal{N}$ is fully faithful.
\end{proposition}
\begin{proof}
We may assume (b), the case (a) being deduced by considering $\mathcal{D}^{\op}$. We shall apply \cref{category localization of subcategory prop}. Let $f:X\to Y$ is a morphism in $\mathcal{N}Q$ with $X\in\mathcal{I}$, we show that there exists $g:Y\to W$ with $W\in\mathcal{I}$ and $g\circ f\in\mathcal{N}Q$. By definition, the morphism $f$ is embedded in a distinguished triangle $X\to Y\to Z\to X[1]$ with $Z\in\mathcal{N}$, and the hypothesis implies that the morphism $Z\to X[1]$ factorizes through an object $Z'\in\mathcal{N}\cap\mathcal{I}$. We may embed $Z'\to TX$ in a distinguished triangle in $\mathcal{I}$ and obtain a commutative diagram of distinguished triangles by (TR4):
\[\begin{tikzcd}
X\ar[r,"f"]\ar[d,equal]&Y\ar[r]\ar[d,dashed,"g"]&Z\ar[r]\ar[d]&X[1]\ar[d,equal]\\
X\ar[r,"g\circ f"]&W\ar[r]&Z'\ar[r]&X[1]
\end{tikzcd}\]
Since $Z'$ belongs to $\mathcal{N}$, we conclude that $gf\in\mathcal{N}Q\cap\Mor(\mathcal{I})$.
\end{proof}

\begin{proposition}\label{triangle cat localization equivalence if resolution}
Let $\mathcal{D}$ be a triangulated category, $\mathcal{N}$ be a null system, $\mathcal{I}$ be a full triangulated subcategory of $\mathcal{D}$, and assume that one of the following conditions is true:
\begin{enumerate}
    \item[(a)] for any $X\in\Ob(\mathcal{D})$, there exists a morphism $X\to Y$ in $\mathcal{N}Q$ with $Y\in\mathcal{I}$;
    \item[(b)] for any $X\in\Ob(\mathcal{D})$, there exists a morphism $Y\to X$ in $\mathcal{N}Q$ with $Y\in\mathcal{I}$;
\end{enumerate}
Then $\mathcal{I}/(\mathcal{N}\cap\mathcal{I})\to\mathcal{D}/\mathcal{N}$ is an equivalence of categories.
\end{proposition}
\begin{proof}
Apply \cref{category localization of subcategory equivalent if}.
\end{proof}

\begin{example}
Let $H$ be a cohomological functor on $\mathcal{D}$. We define $\mathcal{N}_H$ to be the collection of objects $X\in\mathcal{D}$ such that $H(X[n])=0$ for $n\in\Z$. Then it is easy to verify that $\mathcal{N}_H$ satisfies conditions (N1)--(N3), so we can form the localization $\mathcal{D}/\mathcal{N}_H$.
\end{example}

\begin{proposition}\label{triangle cat localization and direct sum}
Let $\mathcal{D}$ be a triangulated category admitting direct sums indexed by a set $I$ and let $\mathcal{N}$ be a null system closed by such direct sums. Let $Q:\mathcal{D}\to\mathcal{D}/\mathcal{N}$ denote the localization functor. Then $\mathcal{D}/\mathcal{N}$ admits direct sums indexed by $I$ and the localization functor $Q:\mathcal{D} \to\mathcal{D}/\mathcal{N}$ commutes with such direct sums.
\end{proposition}
\begin{proof}
Let $\{X_i\}_{i\in I}$ be a family of objects in $\mathcal{D}$. It is enough to show that $Q(\bigoplus_iX_i)$ is the direct sum of the family $Q(X_i)$, i.e., the map
\[\Hom_{\mathcal{D}/\mathcal{N}}(Q(\bigoplus_iX_i),Y)\to\prod_i\Hom_{\mathcal{D}/\mathcal{N}}(Q(X_i),Y)\]
is bijective for any $Y\in\mathcal{D}$. To this end, we first consider morphisms $u_i\in\Hom_{\mathcal{D}/\mathcal{N}}(Q(X_i),Y)$. Then $u_i$ is represented by a pair $(X'_i;s,u'_i)$, where $u'_i:X'_i\to Y$ is a morphism in $\mathcal{D}$ and we have a distinguished triangle
\[\begin{tikzcd}
X'_i\ar[r,"s"]&X_i\ar[r]&Z_i\ar[r]&X'_i[1]
\end{tikzcd}\]
in $\mathcal{D}$ with $Z_i\in\mathcal{N}$. We then get a morphism $\bigoplus_iX'_i\to Y$ and a distinguished triangle $\bigoplus_iX'_i\to\bigoplus_iX_i\to\bigoplus_iZ_i\to (\bigoplus_iX'_i)[1]$ in $\mathcal{D}$ with $\bigoplus_iZ_i\in\mathcal{N}$.\par
Now assume that the composition $Q(X_i)\to Q(\bigoplus_iX_i)\stackrel{u}{\to} Q(Y)$ is zero for each $i\in I$. By definition, the morphism $u$ is represented by a pair $(Y';s,u')$, where $u':\bigoplus_iX_i\to Y'$ is a morphism in $\mathcal{D}$ and $s:Y\to Y'$ is a morphism in $\mathcal{N}Q$. Using the result of \cref{triangle cat null system associated multiplicative system}(\rmnum{3}), we can find $Z_i\in\mathcal{N}$ such that $u'_i:X_i\to Y'$ factroizes as $X_i\to Z_i\to Y'$. Then the induced morphism $\bigoplus_iX_i\to Y'$ factorizes as $\bigoplus_iX_i\to\bigoplus_iZ_i\to Y'$. Since $\bigoplus_iZ_i\in\mathcal{N}$, we conclude that $Q(u)=0$, whence the proposition.
\end{proof}

\subsection{Localization of triangulated functors}
Let $F:\mathcal{D}\to\mathcal{D}'$ be a functor of triangulated categories, $\mathcal{N}$ and $\mathcal{N}'$ be null systems in $\mathcal{D}$ and $\mathcal{D}'$, respectively. The left (resp. right) localization of $F$ (when it exists) is then defined, by replacing "functor" by "triangulated functor". In the sequel, $\mathcal{D}$ (resp. $\mathcal{D}'$, $\mathcal{D}''$) is a triangulated category and $\mathcal{N}$ (resp. $\mathcal{N}'$, $\mathcal{N}''$) is a null system in this category. We denote by $Q:\mathcal{D}\to\mathcal{D}/\mathcal{N}$ (resp. $Q':\mathcal{D}'\to\mathcal{D}'/\mathcal{N}'$, $Q'':\mathcal{D}''\to\mathcal{D}''/\mathcal{N}''$) the localization functor and by $\mathcal{N}Q$ (resp. $\mathcal{N}'Q$, $\mathcal{N}''Q$) the family of morphisms in $\mathcal{D}$ (resp. $\mathcal{D}'$, $\mathcal{D}''$) defined in (\ref{triangle cat null system associated multiplicative system-1}).

\begin{definition}
A triangulated functor $F:\mathcal{D}\to\mathcal{D}'$ is called \textbf{left (resp. right) localizable with respect to $(\mathcal{N},\mathcal{N}')$} if $Q'\circ F:\mathcal{D}\to\mathcal{D}'/\mathcal{N}'$ is universally left (resp. right) localizable with respect to the multiplicative system $\mathcal{N}Q$. If there is no risk of confusion, we simply say that $F$ is left (resp. right) localizable or that $LF$ (resp. $RF$) exists.
\end{definition}

\begin{definition}
Let $F:\mathcal{D}\to\mathcal{D}'$ be a triangulated functor of triangulated categories, $\mathcal{N}$ and $\mathcal{N}'$ null systems in $\mathcal{D}$ and $\mathcal{D}'$, and $\mathcal{I}$ a full triangulated subcategory of $\mathcal{D}$. Consider the following conditions:
\begin{enumerate}
    \item[(a)] For any $X\in\mathcal{D}$, there exists a morphism $X\to Y$ in $\mathcal{N}Q$ with $Y\in\mathcal{I}$.
    \item[(b)] For any $X\in\mathcal{D}$, there exists a morphism $Y\to X$ in $\mathcal{N}Q$ with $Y\in\mathcal{I}$.
    \item[(c)] For any $Y\in\mathcal{N}\cap\mathcal{I}$, $F(Y)\in\mathcal{N}'$. 
\end{enumerate}
Then if (a) and (b) (resp. (b) and (c)) are satisfied, we say that the subcategory $\mathcal{I}$ is \textbf{$\bm{F}$-injective} (resp. \textbf{$\bm{F}$-projective}) with respect to $\mathcal{N}$ and $\mathcal{N}'$. If there is no risk of confusion, we often omit "with respect to $\mathcal{N}$ and $\mathcal{N}'$".
\end{definition}
Note that if $F(\mathcal{N})\sub\mathcal{N}'$, then $D$ is both $F$-injective and $F$-projective.

\begin{proposition}\label{triangle cat localization of functor exist if}
Let $F:\mathcal{D}\to\mathcal{D}'$ be a triangulated functor of triangulated categories, $\mathcal{N}$ and $\mathcal{N}'$ null systems in $\mathcal{D}$ and $\mathcal{D}'$, and $\mathcal{I}$ a full triangulated category of $\mathcal{D}$.
\begin{enumerate}
    \item[(a)] If $\mathcal{I}$ is $F$-injective with respect to $\mathcal{N}$ and $\mathcal{N}$, then $F$ is right localizable and its right localization is a triangulated functor.
    \item[(b)] If $\mathcal{I}$ is $F$-projective with respect to $\mathcal{N}$ and $\mathcal{N}'$, then $F$ left localizable and
    its left localization is a triangulated functor.
\end{enumerate}
\end{proposition}
\begin{proof}
By \cref{category localization of functor exist if}, the existence of the localizations is clear. To verify that they are triangulated, it suffices to apply \cref{triangle cat null system associated multiplicative system} to check this in $\mathcal{D}$ and $\mathcal{D}'$, and this follows from the hypothesis on $F$.
\end{proof}

We denote by $R_\mathcal{N}^{\mathcal{N}'}F:\mathcal{D}/\mathcal{N}\to \mathcal{D}'/\mathcal{N}'$ (resp. $L_\mathcal{N}^{\mathcal{N}'}F$) the right (resp. left) localization of $F$ with respect to $(\mathcal{N},\mathcal{N}')$. If there is no risk of confusion, we simply write $RF$ (resp. $LF$) instead of $R_\mathcal{N}^{\mathcal{N}'}F$ (resp. $L_\mathcal{N}^{\mathcal{N}'}F$). If $\mathcal{I}$ is $F$-injective, then $RF$ can be defined by the diagram
\[\begin{tikzcd}
&\mathcal{D}\ar[r]&\mathcal{D}/\mathcal{N}\ar[dd,dashed,"R_\mathcal{N}^{\mathcal{N}'}F"]\\
\mathcal{I}\ar[r]\ar[ru]\ar[rrd]&\mathcal{I}/(\mathcal{I}\cap\mathcal{N})\ar[ru,"\sim"]\ar[rd]&\\
&&\mathcal{D}'/\mathcal{N}'
\end{tikzcd}\]
and we have
\begin{equation}\label{triangle cat localization functor expression-1}
R_\mathcal{N}^{\mathcal{N}'}F(X)\cong F(Y)\quad\text{for $(X\to Y)\in\mathcal{N}Q$ with $Y\in\mathcal{I}$}.
\end{equation}
Similarly, if $\mathcal{I}$ is $F$-projective, then the diagram above defines $LF$ and we have
\begin{equation}\label{triangle cat localization functor expression-2}
L_\mathcal{N}^{\mathcal{N}'}F(X)\cong F(Y)\quad\text{for $(Y\to X)\in\mathcal{N}Q$ with $Y\in\mathcal{I}$}.
\end{equation}

\begin{proposition}\label{triangle cat localization functor composition}
Let $F:\mathcal{D}\to\mathcal{D}'$ and $F':\mathcal{D}'\to\mathcal{D}''$ be triangulated functors of triangulated categories and let $\mathcal{N}$, $\mathcal{N}'$ and $\mathcal{N}''$ be null systems in $\mathcal{D}$, $\mathcal{D}'$ and $\mathcal{D}''$, respectively.
\begin{enumerate}
    \item[(a)] Assume that $R_\mathcal{N}^{\mathcal{N}'}F$, $R_\mathcal{N}^{\mathcal{N}'}F$ and $R_\mathcal{N}^{\mathcal{N}'}F$ exist. Then there is a canonical morphism of functors:
    \begin{equation}\label{triangle cat localization functor composition-1}
    R_\mathcal{N}^{\mathcal{N}''}(F'\circ F) \to R_{\mathcal{N}'}^{\mathcal{N}''}F'\circ R_\mathcal{N}^{\mathcal{N}'}F.
    \end{equation}
    \item[(b)] Let $\mathcal{I}$ and $\mathcal{I}'$ be full triangulated subcategories of $\mathcal{D}$ and $\mathcal{D}'$, respectively. Assume that I is F-injective with respect to $\mathcal{N}$ and $\mathcal{N}'$, $\mathcal{I}'$ is $F'$-injective with respect to $\mathcal{N}'$ and $\mathcal{N}''$, and $F(\mathcal{I})\sub\mathcal{I}'$. Then $\mathcal{I}$ is $(F'\circ F)$-injective with respect to $\mathcal{N}$ and $\mathcal{N}''$ and (\ref{triangle cat localization functor composition-1}) is an isomorphism
\end{enumerate}
\end{proposition}
\begin{proof}
By definition, there exists a bijection
\[\Hom(R_\mathcal{N}^{\mathcal{N}''}F'\circ F,R_{\mathcal{N}'}^{\mathcal{N}''}F'\circ R_{\mathcal{N}}^{\mathcal{N}'}F)\cong\Hom(Q''\circ F'\circ F,R_\mathcal{N'}^{\mathcal{N}''}F'\circ R_{\mathcal{N}}^{\mathcal{N}'}F\circ Q),\]
and the natural morphism of functors
\[Q''\circ F'\to R_{\mathcal{N}'}^{\mathcal{N}''}F'\circ Q',\quad Q'\circ F\to R_{\mathcal{N}}^{\mathcal{N}'}F\circ Q.\]
We then deduce the canonical morphisms
\[Q''\circ F'\circ F\to R_{\mathcal{N}'}^{\mathcal{N}''}F'\circ Q'\circ F\to R_{\mathcal{N}'}^{\mathcal{N}''}F'\circ R_{\mathcal{N}}^{\mathcal{N}'}F\circ Q\]
whence the morphism in (a). Now assume the conditions in (b); the fact that $\mathcal{I}$ is $(F'\circ F)$-injective follows immediately from the definition. Let $X\in\mathcal{D}$ and consider a morphism $X\to Y$ in $\mathcal{N}Q$ with $Y\in\mathcal{I}$. Then $R_{\mathcal{N}}^{\mathcal{N}'}F(X)\cong F(Y)$ by (\ref{triangle cat localization functor expression-1}) and $F(Y)\in\mathcal{I}$' by our hypothesis. It then follows from (\ref{triangle cat localization functor expression-1}) that $(R_{\mathcal{N}'}^{\mathcal{N}''}F')(F(Y))\cong F'(F(Y))$, and we conclude that
\[(R_{\mathcal{N}'}^{\mathcal{N}''}F)(R_{\mathcal{N}}^{\mathcal{N}'}F(X))\cong F'(F(Y)).\]
On the other hand, $R_\mathcal{N}^{\mathcal{N}'}(F'\circ F)(X)\cong F'(F(Y))$ by (\ref{triangle cat localization functor expression-1}), since $\mathcal{I}$ is $(F'\circ F)$-injective.
\end{proof}

We now restrict our notations of localizations to triangulated bifunctors. Let $(\mathcal{D},T)$, $(\mathcal{D}',T')$ and $(\mathcal{D}'',T'')$ be triangulated categories. A \textbf{triangulated bifunctor} $F:\mathcal{D}\times\mathcal{D}'\to\mathcal{D}''$ is a bifunctor of additive categories with translations which sends distinguished triangles in each arguments to distinguished triangles.

\begin{definition}
Let $\mathcal{D}$, $\mathcal{D}'$ and $\mathcal{D}''$ be triangulated categories and $\mathcal{N}$, $\mathcal{N}'$ and $\mathcal{N}''$ be null systems in $\mathcal{D}$, $\mathcal{D}'$ and $\mathcal{D}''$, respectively. We say that a triangulated bifunctor $F:\mathcal{D}\times\mathcal{D}'\to\mathcal{D}''$ is \textbf{right (resp. left) localizable with respect to $(\mathcal{N},\mathcal{N},\mathcal{N}'')$} if $Q''\circ F:\mathcal{D}\times\mathcal{D}'\to\mathcal{D}''/\mathcal{N}''$ is universally right (resp. left) localizable with respect to the multiplicative system $\mathcal{N}Q\times\mathcal{N}'Q$. If there is no risk of confusion, we simply say that $F$ is right (resp. left) localizable.\par
If $\mathcal{I},\mathcal{I}'$ are full triangulated subcategories of $\mathcal{D}$ and $\mathcal{D}'$, respectively, then the pair $(\mathcal{I},\mathcal{I}')$ is \textbf{$F$-injective} with respect to $(\mathcal{N},\mathcal{N}',\mathcal{N}'')$ if
\begin{enumerate}
    \item[(\rmnum{1})] $\mathcal{I}'$ is $F(Y,-)$-injective with respect to $\mathcal{N}'$ and $\mathcal{N}''$ for any $Y\in\mathcal{I}$.
    \item[(\rmnum{2})] $\mathcal{I}$ is $F(-,Y')$-injective with respect to $\mathcal{N}$ and $\mathcal{N}''$ for any $Y'\in\mathcal{I}'$.
\end{enumerate}
Equivalently, this amounts to saying that
\begin{enumerate}
    \item[(a)] for any $X\in\mathcal{D}$, there exists a morphism $X\to Y$ in $\mathcal{N}Q$ with $Y\in\mathcal{I}$;
    \item[(b)] for any $X'\in\mathcal{D}'$, there exists a morphism $X'\to Y'$ in $\mathcal{N}'Q$ with $Y'\in\mathcal{I}'$;
    \item[(c)] $F(X,X')$ belongs to $\mathcal{N}''$ for $X\in\mathcal{I}$, $X'\in\mathcal{I}'$ as soon as $X$ belongs to $\mathcal{N}$ or $X'$ belongs to $\mathcal{N}'$.
\end{enumerate}
The property for $(\mathcal{I},\mathcal{I}')$ of being \textbf{$F$-projective} is defined similarly.
\end{definition}

We denote by $R_{\mathcal{N}\times\mathcal{N}'}^{\mathcal{N}''}$ the right localization of $F$ with respect to $(\mathcal{N}\times\mathcal{N}',\mathcal{N}'')$ if it exists. If there is no risk of confusion, we simply write $RF$. We use similar notations for the left localization.

\begin{proposition}\label{triangle cat localization bifunctor exists if}
Let $\mathcal{D}$, $\mathcal{D}'$ and $\mathcal{D}''$ be triangulated categories and $\mathcal{N}$, $\mathcal{N}'$ and $\mathcal{N}''$ be null systems in $\mathcal{D}$, $\mathcal{D}'$ and $\mathcal{D}''$, respectively. Let $F:\mathcal{D}\times\mathcal{D}'\to\mathcal{D}$ be a triangulated bifunctor and $\mathcal{I},\mathcal{I}'$ be full triangulated subcategories of $\mathcal{D}$ and $\mathcal{D}'$ such that $(\mathcal{I},\mathcal{I}')$ is $F$-injective with respect to $(\mathcal{N},\mathcal{N}')$. Then $F$ is right localizable, its right localization $R_{\mathcal{N}\times\mathcal{N}'}^{\mathcal{N}''}F$ is a triangulated bifunctor 
\[R_{\mathcal{N}\times\mathcal{N}'}^{\mathcal{N}''}:\mathcal{D}/\mathcal{N}\times\mathcal{D}'/\mathcal{N}'\to\mathcal{D}''/\mathcal{N}'',\]
and moreover,
\begin{equation}\label{triangle cat localization bifunctor exists if-1}
R_{\mathcal{N}\times\mathcal{N}'}^{\mathcal{N}''}F(X,X')\cong F(Y,Y')
\end{equation}
for $(X\to Y)\in\mathcal{N}Q$ and $(X'\to Y')\in\mathcal{N}'Q$ with $Y\in\mathcal{I}$, $Y'\in\mathcal{I}'$. There exists a similar result by replacing "injective" with "projective" and reversing the arrows.
\end{proposition}
\begin{proof}
By definition, $Q''\circ F$ sends $\mathcal{N}Q\cap\Mor(\mathcal{I})\times(\mathcal{N}'Q\cap\Mor(\mathcal{I}'))$ to isomorphisms in $\mathcal{D}''$, so it follows from \cref{category localization of functor exist if} that the right localization $R_{\mathcal{N}\times\mathcal{N}'}^{\mathcal{N}''}$ exists. The fact that $R_{\mathcal{N}\times\mathcal{N}'}^{\mathcal{N}''}$ is triangulated follows from the hypothsis on $F$, in view of \cref{triangle cat null system associated multiplicative system}. The last equation is a concequence of \cref{triangle cat localization functor composition}.
\end{proof}

\begin{corollary}\label{triangle cat localization bifunctor if exact one variable}
Retain the notations of \cref{triangle cat localization bifunctor exists if} and assume that
\begin{enumerate}
    \item[(a)] $F(\mathcal{I},\mathcal{N}')\sub\mathcal{N}''$;
    \item[(b)] for any $X'\in\mathcal{D}'$, $\mathcal{I}$ is $F(-,X')$-injective with respect to $\mathcal{N}$.
\end{enumerate}
Then $F$ is right localizable and we have
\[R_{\mathcal{N}\times\mathcal{N}'}^{\mathcal{N}''}F(X,X')\cong R_{\mathcal{N}}^{\mathcal{N}''}F(-,X')(X).\]
Again, there is a similar statement by replacing "injective" with "projective".
\end{corollary}
\begin{proof}
Under our hypothesis, for any fixed object $X'\in\mathcal{D}'$, the functor $F(-,X')$ is right localizable, and the last claim follows from (\ref{triangle cat localization functor composition-1}) and (\ref{triangle cat localization bifunctor exists if-1}).
\end{proof}

\section{Derived categories}
In this section, we apply the previous results of triangulated categories on the derived category of an abelian category $\mathcal{A}$, which is defined to be the localization of $K(\mathcal{A})$ with respect to quasi-isomorphisms. Our main refrence will be \cite{kashiwara_SAC}.
\subsection{Derived categories}
Let $(\mathcal{A},T)$ be an abelian category with translation. Recall that the cohomology functor $H:\mathcal{A}_c\to\mathcal{A}$ induces a cohomological functor
\[H:K_c(\mathcal{A})\to\mathcal{A}.\]
Let $\mathcal{N}$ be the full subcategory of $K_c(\mathcal{A})$ consisting of objects $X$ such that $H(X)\cong 0$, that is, $X$ is quasi-isomorphic to $0$. Since $H$ is cohomological, the category $\mathcal{N}$ is a triangulated subcategory of $K_c(\mathcal{A})$. We denote by $D_c(\mathcal{A})$ the category $K_c(\mathcal{A})/\mathcal{N}$, and call it the \textbf{derived category} of $(\mathcal{A},T)$. Note that $D_c(\mathcal{A})$ is triangulated by \cref{*}. By the properties of the localization, a quasi-isomorphism in $K_c(\mathcal{A})$ (or in $\mathcal{A}_c$) becomes an isomorphism in $D_c(\mathcal{A})$. One shall be aware that the category $D_c(\mathcal{A})$ may be a big category.\par
From now on, we shall restrict our study to the case where $\mathcal{A}_c$ is the category of complexes of an abelian category $\mathcal{A}$. Recall that the categories $C^*(\mathcal{A})$ are defined for $\ast\in\{+,-,b,\emp\}$, and we have full subcategories $K^*(\mathcal{A})$ of $K(\mathcal{A})$. For $\ast\in\{+,-,b,\emp\}$, we define
\[N^*(\mathcal{A})=\{X\in K^*(\mathcal{A}):\text{$H^i(X)\cong 0$ for all $i$}\}.\]
Clearly, $N^*(\mathcal{A})$ is a null system in $K^*(\mathcal{C})$.

\begin{definition}
The triangulated categories $D^*(\mathcal{A})$ are defined as $K^*(\mathcal{A})/N^*(\mathcal{A})$ and are called the \textbf{derived categories} of $\mathcal{A}$.
\end{definition}

Recall that to a null system $\mathcal{N}$ we have associated in (\ref{triangle cat null system associated multiplicative system-1}) a multiplicative system denoted by $\mathcal{N}Q$. It will be more intuitive to use here another notation for $\mathcal{N}Q$ when $\mathcal{N}=N(\mathcal{A})$:
\[\Qis=\{f\in\Mor(K(\mathcal{A})):\text{$f$ is a quasi-isomorphism}\}.\]
With this notation, we then have
\begin{align*}
\Hom_{D(\mathcal{A})}(X,Y)&\cong \rlim_{(X'\to X)\in\Qis}\Hom_{K(\mathcal{A})}(X',Y)\cong \rlim_{(Y'\to Y)\in\Qis}\Hom_{K(\mathcal{A})}(X,Y')\\&\cong \rlim_{\substack{(X'\to X)\in\Qis\\(Y\to Y')\in\Qis}}\Hom_{K(\mathcal{A})}(X',Y').
\end{align*}

\begin{remark}
Let $X\in K(\mathcal{A})$, and let $Q(X)$ denote its image in $D(\mathcal{A})$. Then it follows from our definition of $\mathcal{N}(\mathcal{A})$ that $Q(X)=0$ if and only if $H^n(X)=0$ for all $n\in\Z$. Also, if $f:X\to Y$ is a morphism in $\Ch(\mathcal{A})$, then by \cref{triangle cat null system associated multiplicative system}, $f=0$ in $D(\mathcal{A})$ if and only if there exist $X'$ and a quasi-isomorphism $g:X'\to X$ such that $fg$ is homotopic to $0$, or else, if and only if there exist $Y'$ and a quasi-isomorphism $h:Y\to Y'$ such that $hf$ is homotopic to $0$.
\end{remark}

\begin{proposition}\label{derived category of abelian cat prop}
Let $\mathcal{A}$ be an abelian category and $D(\mathcal{A})$ be its derived category.
\begin{enumerate}
    \item[(a)] For $n\in\Z$, the functor $H^n:D(\mathcal{A})\to\mathcal{A}$ is well defined and is a cohomological functor.
    \item[(b)] A morphism $f:X\to Y$ in $D(\mathcal{A})$ is an isomorphism if and only if $H^n(f):H^n(X)\to H^n(Y)$ is an isomorphism for all $n\in\Z$.
    \item[(c)] For $n\in\Z$, the functors $\tau_{\leq n},\tau^{\leq n}:D(\mathcal{A})\to D^-(\mathcal{A})$, as well as the functors $\tau_{\geq n},\tau^{\geq n}:D(\mathcal{A})\to D^+(\mathcal{A})$, are well defined and isomorphic.
    \item[(d)] For $n\in\Z$, the functor $\tau^{\leq n}$ induces a functor $D^+(\mathcal{A})\to D^b(\mathcal{A})$ and $\tau^{\geq n}$ induces a functor $D^-(\mathcal{A})\to D^b(\mathcal{A})$.
\end{enumerate}
\end{proposition}
\begin{proof}
Since $H^n(X)=0$ for $X\in N(\mathcal{A})$, the first assertion is clear, and the second one follows from \cref{triangle cat null system associated multiplicative system} and the definition of $\mathcal{N}Q$ for $\mathcal{N}=N(\mathcal{A})$: in fact, if $Q(f)$ is an isomorphism, then from the following commutative diagram
\[\begin{tikzcd}
Q(X)\ar[r,"Q(f)"]\ar[d,"Q(f)"]&Q(Y)\ar[r]\ar[d,equal]&Q(M(f))\ar[d]\ar[r]&X[1]\ar[d]\\
Q(Y)\ar[r,equal]&Q(Y)\ar[r]&0\ar[r]&Q(Y)[1]
\end{tikzcd}\]
we conclude that $Q(M(f))\to 0$ is an isomorphism (\cref{triangle cat morphism dt isomorphism 2 of 3}), so $H^n(f)$ is an isomorphism for each $n\in\Z$.\par
Now if $f:X\to Y$ is a quasi-isomorphism in $K(\mathcal{A})$, then $\tau^{\leq n}(f)$ and $\tau^{\geq n}(f)$ are quasi-isomorphism. Moreover, for $X\in K(\mathcal{A})$, the morphisms $\tau^{\leq n}(X)\to\tau_{\leq n}(X)$ and $\tau^{\geq n}(X)\to\tau_{\geq n}(X)$ are also quasi-isomorphism (\cref{*}), so assertion (c) follows from (d), and (d) is then obvious.
\end{proof}

To a distinguished triangle $X\stackrel{f}{\to} Y\stackrel{g}{\to} Z\to X[1]$ in $\mathcal{D}(\mathcal{A})$, the cohomological functor $H^0$ associates a long exact sequence in $\mathcal{A}$:
\[\begin{tikzcd}
\cdots\ar[r]&H^i(X)\ar[r]&H^i(Y)\ar[r]&H^i(Z)\ar[r]&H^{i+1}(X)\ar[r]&\cdots
\end{tikzcd}\]
For $X\in K(\mathcal{A})$, recall that the categories $\Qis_{/X}$ and $\Qis_{X/}$ are filtrant (or cofiltrant) categories of $K(\mathcal{C})_{/X}$ and $K(\mathcal{A})_{X/}$, respectively. If $\mathcal{J}$ is a subcategory of $K(\mathcal{C})_{/X}$, we denote by $\Qis_{/X}\cap\mathcal{J}$ the full subcategory of $\Qis_{/X}$ consisting of objects which belong to $\mathcal{J}$. We use similar notations for $\Qis_{X/}$ and $K(\mathcal{C})_{X/}$.

\begin{lemma}\label{derived category Qis cofinal truncation}
Let $\mathcal{A}$ be an abelian category and $n$ be an integer.
\begin{enumerate}
    \item[(a)] For $X\in K^{\leq n}(\mathcal{A})$, the categories $\Qis_{/X}\cap K^{\leq n}(\mathcal{A})_{/X}$ and $\Qis_{/X}\cap K^{-}(\mathcal{A})_{/X}$ are co-cofinal to $\Qis_{/X}$.
    \item[(b)] For $X\in K^{\geq n}(\mathcal{A})$, the categories $\Qis_{X/}\cap K^{\geq n}(\mathcal{A})_{X/}$ and $\Qis_{X/}\cap K^{+}(\mathcal{A})_{X/}$ are co-cofinal to $\Qis_{X/}$.
\end{enumerate}
\end{lemma}
\begin{proof}
The two statements are equivalent by reversing the arrows, so we only prove (b). The category $\Qis_{X/}\cap K^{\geq n}(\mathcal{A})_{X/}$ is a full subcategory of a filtrant category $\Qis_{X/}$, and for any object $(X\to Y)$ in $\Qis_{X/}$, there exists a canonical morphism $(X\to Y)\to(X\to\tau^{\geq n}Y)$.
\end{proof}

\begin{proposition}\label{derived category truncation Hom of leqgeq prop}
Let $n\in\Z$ and $X\in K^{\leq n}(\mathcal{A})$, $Y\in K^{\geq n}(\mathcal{A})$. Then we have
\begin{equation}
\Hom_{D(\mathcal{A})}(X,Y)\cong \Hom_{\mathcal{C}}(H^n(X),H^n(Y))
\end{equation}
\end{proposition}
\begin{proof}
The map $\Hom_{\Ch(\mathcal{A})}(X,Y)\to\Hom_{K(\mathcal{A})}(X,Y)$ is an isomorphism by our hypothesis and
\begin{align*}
\Hom_{\Ch(\mathcal{A})}(X,Y)&\cong\{f\in\Hom_{\mathcal{A}}(X^n,Y^n):u\circ d_X^{n-1}=0,d_Y^n\circ f=0\}\\
&\cong \Hom_\mathcal{A}(\coker d_X^{n-1},\ker d_Y^n)\cong\Hom_\mathcal{A}(H^n(X),H^n(Y))
\end{align*}
so we conclude that $\Hom_{K(\mathcal{A})}(X,Y)\cong\Hom_\mathcal{A}(H^n(X),H^n(Y))$. On the other hand, in view of \cref{derived category Qis cofinal truncation}, we have
\begin{align*}
\Hom_{D(\mathcal{A})}(X,Y)\cong\rlim_{(Y\to Y')\in\Qis\cap K^{\geq n}(\mathcal{A})}\Hom_{K(\mathcal{A})}(X,Y')\cong\Hom_{\mathcal{A}}(H^n(X),H^n(Y))
\end{align*}
so the proposition follows.
\end{proof}

For $-\infty\leq a\leq b\leq+\infty$, we denote by $D^{[a,b]}(\mathcal{A})$ the full subcategory of $D(\mathcal{A})$ consisting of objects $X$ satisfying $H^i(X)=0$ for $i\notin[a,b]$. With this notation, we set $D^{\leq a}(\mathcal{A}):=D^{[-\infty,a]}(\mathcal{A})$ and $D^{\geq a}(\mathcal{A}):=D^{[a,+\infty]}(\mathcal{A})$.

\begin{proposition}\label{derived category truncation subcategory prop}
Let $\mathcal{A}$ be an abelian category.
\begin{enumerate}
    \item[(a)] For $\ast\in\{+,-,b\}$, the triangulated category $D^*(\mathcal{A})$ is equivalent to the full triangulated subcategory of $\mathcal{D}(\mathcal{A})$ consisting of objects $X$ satisfying $H^i(X)=0$ for $i\ll 0$ (resp. $i\gg 0$, resp. $|i|\gg 0$).
    \item[(b)] For $-\infty\leq a\leq b\leq+\infty$, the canonical functor $Q:K^{[a,b]}(\mathcal{A})\to D^{[a,b]}(\mathcal{A})$ is essentially surjective.
    \item[(c)] The category $\mathcal{A}$ is equivalent to the full subcategory $D^{\leq 0}(\mathcal{A})\cap D^{\geq 0}(\mathcal{A})$.
    \item[(d)] For any $n\in\Z$ and $X,Y\in D(\mathcal{C})$, we have 
    \[\Hom_{D(\mathcal{A})}(\tau^{\leq n}X,\tau^{\geq n}Y)\cong\Hom_{\mathcal{A}}(H^n(X),H^n(Y)).\]
    In particular, $\Hom_{D(\mathcal{A})}(\tau^{\leq n}X,\tau^{\geq n+1}Y)=0$. 
\end{enumerate}
\end{proposition}
\begin{proof}
As for assertion (a), let us treat the case $*=+$, the other cases being similar. For $Y\in K^{\geq a}(\mathcal{A})$ and $Z\in N(\mathcal{A})$, any morphism $Z\to Y$ in $K(\mathcal{A})$ factors through $\tau^{\geq n}Z\in N(\mathcal{A})\cap K^{\geq n}(\mathcal{A})$. Applying \cref{category localization of subcategory prop}, we find that the natural functor $D^+(\mathcal{A})\to D(\mathcal{A})$ is fully faithful, and it is clear that if $Y\in D(\mathcal{A})$ belongs to the image of the functor $D^+(\mathcal{A})\to D(\mathcal{A})$, then $H^i(X)=0$ for $i\ll 0$. Conversely, let $X\in K(\mathcal{A})$ with $H^i(X)=0$ for $i<a$. Then $\tau^{\geq a}X\in K^+(\mathcal{A})$ and the morphism $X\to\tau^{\geq a}X$ in $K(\mathcal{A})$ is a quasi-isomorphism, whence an isomorphism in $D(\mathcal{A})$. We therefore conclude (a), and the proof of (b) can be done similarly. Finally, by \cref{derived category truncation Hom of leqgeq prop}, the functor $\mathcal{A}\to D(\mathcal{A})$ is fully faithful and essentially surjective by (b); this proves (c), and (d) follows from (b) and \cref{derived category truncation Hom of leqgeq prop}.
\end{proof}

\begin{proposition}\label{derived category exact sequence is dt}
Let $0\to X\stackrel{f}{\to} Y\stackrel{g}{\to} Z\to 0$ be an exact sequence in $\Ch(\mathcal{A})$. Then there exists a distinguished triangle $X\stackrel{f}{\to} Y\stackrel{g}{\to} Z\to X[1]$ and $Z$ is isomorphic to $M(f)$ in $D(\mathcal{A})$.
\end{proposition}
\begin{proof}
We define a morphism $\varphi:M(f)\to Z$ in $\Ch(\mathcal{A})$ by $\varphi^n=(0,g^n)$. By \cref{*}, $\varphi$ is then a quasi-isomorphism, whence an isomorphism in $D(\mathcal{A})$.
\end{proof}

\begin{remark}
Let $0\to X\to Y\to Z\to 0$ be an exact sequence in $\mathcal{A}$. By \cref{derived category exact sequence is dt}, we then get a morphism $\gamma:Z\to X[1]$ in $D(\mathcal{A})$. The morphism $H^i(\gamma):H^i(Z)\to H^{i+1}(X)$ is zero for all $i\in\Z$, although $\gamma$ is not the zero morphism in $D(\mathcal{A})$ in general (this happens only if the short exact sequence splits). The morphism $\gamma$ may be described in $K(\mathcal{A})$ by the morphisms with $\varphi$ a quasi-isomorphism:
\[X[1]\stackrel{\beta(f)}{\longleftarrow} M(f)\stackrel{\varphi}{\longrightarrow} Z.\]
\end{remark}

\begin{proposition}\label{derived category dt for truncation functor}
If $X\in D(\mathcal{A})$, there are distinguished triangles in $D(\mathcal{A})$:
\begin{gather}
\tau^{\leq n}X \longrightarrow X \longrightarrow \tau^{\geq n+1}X \stackrel{+1}{\longrightarrow},\label{derived category dt for truncation functor-1}\\
\tau^{\leq n-1}X \longrightarrow \tau^{\leq n}X \longrightarrow H^n(X)[-n] \stackrel{+1}{\longrightarrow},\label{derived category dt for truncation functor-2}\\
H^n(X)[-n] \longrightarrow \tau^{\geq n}X \longrightarrow \tau^{\geq n+1}X \stackrel{+1}{\longrightarrow},\label{derived category dt for truncation functor-3}
\end{gather}
Moreover, we have canonical isomorphisms
\begin{equation}\label{derived category dt for truncation functor-4}
H^n(X)[-n]\cong\tau^{\leq n}\tau^{\geq n}(X)\cong \tau^{\geq n}\tau^{\leq n}X.
\end{equation}
\end{proposition}
\begin{proof}
This is a direct concequence of \cref{*} and \cref{*}.
\end{proof}

\begin{proposition}\label{derived category truncation functor adjoint prop}
The functor $\tau^{\leq n}:D(\mathcal{A})\to D^{\leq n}(\mathcal{A})$ is a right adjoint to the natural inclusion $D^{\leq n}(\mathcal{A})\to D(\mathcal{A})$ and $\tau^{\geq n}:D(\mathcal{A})\to D^{\geq n}(\mathcal{A})$ is a left adjoint to the natural functor $D^{\geq n}(\mathcal{A})\to D(\mathcal{A})$. In other words, there are functorial isomorphisms
\begin{gather*}
\Hom_{D(\mathcal{A})}(X,Y)\cong \Hom_{D^{\leq n}(\mathcal{A})}(X,\tau^{\leq n}Y)\for X\in D^{\leq n}(\mathcal{A}), Y\in D(\mathcal{A}),\\
\Hom_{D(\mathcal{A})}(X,Y)\cong \Hom_{D^{\geq n}(\mathcal{A})}(\tau^{\geq n}X,Y)\for X\in D(\mathcal{A}), Y^{\geq n}\in D(\mathcal{A}).
\end{gather*}
\end{proposition}
\begin{proof}
Let $X\in D^{\leq n}(\mathcal{A})$, then by the distinguished triangle (\ref{derived category dt for truncation functor-1}) for $Y$, we have an exact sequence
\begin{small}
\begin{equation}\label{derived category truncation functor adjoint prop-1}
\begin{tikzcd}[column sep=5mm]
\Hom_{D(\mathcal{A})}(X,\tau^{\geq n+1}Y[-1])\ar[r]&\Hom_{D(\mathcal{A})}(X,\tau^{\leq n}Y)\ar[r]&\Hom_{D(\mathcal{A})}(X,Y)\ar[r]&\Hom_{D(\mathcal{A})}(X,\tau^{\geq n+1}Y)
\end{tikzcd}
\end{equation}
\end{small}
Since $\tau^{\geq n+1}Y[-1]$ and $\tau^{\geq n+1}Y$ belong to $D^{\geq n+1}(\mathcal{A})$, the first and fourth terms of (\ref{derived category truncation functor adjoint prop-1}) are zero by \cref{derived category truncation subcategory prop}~(d). The second isomorphism follows by reversing the arrows.
\end{proof}

\subsection{Resolutions}
The derived category $D^*(\mathcal{A})$ is often a big category and this causes many problems. In this paragraph, by considering resolutions in the category $\Ch(\mathcal{A})$, we show that in some case $D^*(\mathcal{A})$ is equivalent to the homotopy category of a subcategory of $\mathcal{A}$, and hence a $\mathscr{U}$-category, where $\mathscr{U}$ is the chosen universe.

\begin{lemma}\label{derived category resolution by subcat if}
Let $\mathcal{J}$ be a full additive subcategory of $\mathcal{A}$ and $X^\bullet\in \Ch^{\geq n}(\mathcal{A})$ for some $n\in\Z$. Assume that one of the following conditions holds:
\begin{enumerate}
    \item[(a)] $\mathcal{J}$ is cogenerating in $\mathcal{A}$ (i.e. for any $Y\in\mathcal{A}$ there exists a monomorphism $Y\to I$ with $I\in\mathcal{J}$);
    \item[(b)] $\mathcal{J}$ is closed under extensions and cokernels of monomorphisms, and for any monomorphism $I'\to Y$ in $\mathcal{A}$ with $I'\in\mathcal{J}$, there exists a morphism $Y\to I$ with $I\in\mathcal{J}$ such that the composition $I'\to I$ is a monomorphism. Moreover, $H^i(X^\bullet)\in\mathcal{J}$ for all $i\in\Z$.
\end{enumerate}
Then there exists $Y^\bullet\in\Ch^{\geq n}(\mathcal{J})$ and a quasi-isomorphism $X^\bullet\to Y^\bullet$.
\end{lemma}
\begin{proof}

\end{proof}

Let $\mathcal{J}$ be a full additive subcategory of $\mathcal{A}$. It is clear that for $*\in\{+,-,b,\emp\}$, the $N^*(\mathcal{J}):=N(\mathcal{A})\cap K^*(\mathcal{J})$ is a null system in $K^*(\mathcal{J})$. We say that $\mathcal{A}$ has \textbf{finite $\mathcal{J}$-dimension} if there exists a non-negative integer $d$ such that, for any exact sequence
\[\begin{tikzcd}
Y_d\ar[r]&\cdots\ar[r]&Y_1\ar[r]&Y\ar[r]&0
\end{tikzcd}\]
with $Y_i\in\mathcal{J}$ for $1\leq i\leq d$, we have $Y\in\mathcal{J}$.

\begin{proposition}\label{derived category resolution of cogenerating prop}
Assume that $\mathcal{J}$ is cogenerating in $\mathcal{C}$, then the natural functor
\[\theta^+:K^+(\mathcal{J})/N^+(\mathcal{J})\to D^+(\mathcal{C})\]
is an equivalence of categories. If $\mathcal{A}$ has finite $\mathcal{J}$-dimension, then $\theta^b:K^b(\mathcal{J})/N^b(\mathcal{J})\to D^b(\mathcal{C})$ is also an equivalence of categories.
\end{proposition}
\begin{proof}
Let $X\in K^+(\mathcal{A})$. By \cref{derived category resolution by subcat if}, there exists $Y\in K^+(\mathcal{J})$ and a quasi-isomorphism $X\to Y$, so the first assertion follows from \cref{triangle cat localization subcategory functor prop}. Now assume that $\mathcal{A}$ has finite $\mathcal{J}$-dimension and that $X^i=0$ for $i\geq n$, where $n\in\Z$. Then $\tau^{\leq i}Y\to Y$ is a quasi-isomorphism for $i\geq n$ and the hypothesis implies that $\tau^{\leq i}Y$ belongs to $K^b(\mathcal{J})$ for $i>n+d$. This proves the second assertion in view of \cref{triangle cat localization subcategory functor prop}.
\end{proof}

Let us apply the preceding proposition to the full subcategory of injective objects: $\mathcal{I}_\mathcal{A}=\{X\in\mathcal{A}:\text{$X$ is injective}\}$.
\begin{proposition}\label{derived category enough injective equivalence}
Assume that $\mathcal{A}$ admits enough injectives. Then the functor $K^+(\mathcal{I}_\mathcal{A})\to D^+(\mathcal{A})$ is an equivalence of categories. If moreover $\mathcal{A}$ has finite injective dimension, then $K^b(\mathcal{I}_\mathcal{A})\to D^b(\mathcal{A})$ is an equivalence of categories.
\end{proposition}
\begin{proof}
By \cref{derived category resolution of cogenerating prop}, it is enough to prove that if $X^\bullet\in\Ch^+(\mathcal{I}_\mathcal{A})$ is quasi-isomorphic to $0$, then $X^\bullet$ is homotopic to $0$. This is a particular case of the lemma below (choose $f=\id_{X^\bullet}$ in the lemma).
\end{proof}

\begin{lemma}\label{abelian cat injective qis zero is null-homotopy}
Let $f:X^\bullet\to I^\bullet$ be a morphism in $\Ch(\mathcal{A})$. Assume that $I^\bullet$ belongs to $\Ch^+(\mathcal{I}_\mathcal{A})$ and $X^\bullet$ is exact. Then $f$ is homotopic to $0$.
\end{lemma}


\begin{corollary}\label{derived category enough injective U-cat}
Let $\mathcal{A}$ be an abelian $\mathscr{U}$-category with enough injectives. Then $D^+(\mathcal{A})$ is a $\mathscr{U}$-category.
\end{corollary}

\begin{proposition}
Let $\mathcal{J}$ be a full additive subcategory of $\mathcal{A}$ and assume that $\mathcal{J}$ is cogenerating and $\mathcal{A}$ has finite $\mathcal{J}$-dimension. Then for any $X\in\Ch(\mathcal{A})$, there exists $Y\in\Ch(\mathcal{J})$ and a quasi-isomorphism $X\to Y$. In particular, there is an equivalence of triangulated categories $K(\mathcal{J})/N(\mathcal{J})\stackrel{\sim}{\to} D(\mathcal{A})$.
\end{proposition}

An important class of examples are given by Serre subcategories of $\mathcal{A}$: let $\mathcal{T}$ be a weak Serre subcategory of $\mathcal{A}$. For $*\in\{+,-,b,\emp\}$, we denote by $D^*_\mathcal{T}(\mathcal{A})$ the full additive subcategory of $D^*(\mathcal{A})$ consisting of objects $X$ such that $H^i(X)\in\mathcal{T}$ for all $i\in\Z$. This is clearly a triangulated subcategory of $D(\mathcal{A})$, and there is a natural functor
\begin{equation}\label{derived category truncation by Serre subcat prop-1}
\delta^*:D^*(\mathcal{T})\to D^*_\mathcal{T}(\mathcal{A}).
\end{equation}

\begin{theorem}\label{derived category truncation by Serre subcat prop}
Let $\mathcal{T}$ be a Serre subcategory of $\mathcal{A}$ and assume that for any monomorphism $Y\to X$, with $Y\in\mathcal{T}$, there exists a morphism $X\to Y'$ with $Y'\in\mathcal{T}$ such that the composition $Y\to Y'$ is a monomorphism. Then the functors $\delta^+$ and $\delta^b$ in (\ref{derived category truncation by Serre subcat prop-1}) are equivalences of categories.
\end{theorem}
\begin{proof}
The result for $\delta^+$ is an immediate consequence of \cref{category localization of subcategory equivalent if} and \cref{derived category resolution by subcat if}~(b). The case of $\delta^b$ follows since $D^b(\mathcal{T})$ is equivalent to the full subcategory of $D^+(\mathcal{T})$ of objects with bounded cohomology, and similarly for $D^b_\mathcal{T}(\mathcal{A})$.
\end{proof}
Note that, by reversing the arrows in \cref{derived category truncation by Serre subcat prop}, the functors $\delta^-$ and $\delta^b$ in (\ref{derived category truncation by Serre subcat prop-1}) are equivalences of categories if for any epimorphism $X\to Y$ with $Y\in\mathcal{T}$, there exists a morphism $Y'\to X$ with $Y'\in\mathcal{T}$ such that the composition $Y'\to Y$ is an epimorphism.
\subsection{Bounded functors and the way-out lemma}
We now introduce an important result on how a triangulated functor on derived categories is determined by its values on the underlying abelian category. This is useful when one want to show that some natural map is a functorial isomorphism.\par
In this paragraph, we consider abelian categories $\mathcal{A}$ and $\mathcal{A}'$, and additive functors between subcategories of $D(\mathcal{A})$ and $D(\mathcal{A}')$. We choose a weak Serre subcategory $\mathcal{T}$ of $\mathcal{A}$ and denote by $D_\mathcal{T}^*(\mathcal{A})$ the full additive subcategory of $D^*(\mathcal{A})$ consisting of objects $X$ such that $H^i(X)\in\mathcal{T}$ for all $i\in\Z$. If $\mathcal{E}$ is a subcategory of $D(\mathcal{A})$, we write $\mathcal{E}^{\geq n}$ (resp. $\mathcal{E}^{\geq n}$) for the subcategories of $\mathcal{E}$ whose objects are complexes $X$ such that $H^i(X)=0$ for $i<n$ (resp. $i>n$).
\begin{definition}
Let $\mathcal{E}$ be a subcategory of $D(\mathcal{A})$ and let $F:\mathcal{E}\to D(\mathcal{A}')$ be an additive functor. The \textbf{upper dimension} $\dim^+$ and \textbf{lower dimension} $\dim^-$ of the functor $F$ are defined by
\begin{gather*}
\dim^+(F):=\inf\{d\in\Z:\text{$F(\mathcal{E}^{\leq n})\sub D^{\leq n+d}(\mathcal{A}')$ for all $n\in\Z$}\},\\
\dim^-(F):=\inf\{d\in\Z:\text{$F(\mathcal{E}^{\geq n})\sub D^{\geq n-d}(\mathcal{A}')$ for all $n\in\Z$}\}.
\end{gather*}
The functor $F$ is called \textbf{bounded above} (resp. \textbf{bounded below}) if $\dim^+(F)<+\infty$ (resp. $\dim^-(F)<+\infty$), and \textbf{bounded} if it is both bounded above and bounded below.
\end{definition}

\begin{remark}
If $F:\mathcal{E}\to D(\mathcal{A}')$ is compatible with the translation functors of $D(\mathcal{A})$ and $D(\mathcal{A}')$, then we see that $F(\mathcal{E}^{\geq n})\sub D^{\geq n+d}(\mathcal{A}')$ holds for all $n\in\Z$ as soon as it holds for one single $n$, for example $n=0$. Therefore, in this case we can also define $\dim^+(F)$ to be the smallest integer $d$ such that $F(\mathcal{E}^{\leq 0})\sub D^{\leq d}(\mathcal{A}')$, and similarly for $\dim^-(F)$.
\end{remark}

\begin{example}\label{derived category functor dim leq d iff truncation quasi-isomorphism}
If $\mathcal{E}$ is a triangulated subcategory of $D(\mathcal{A})$ such that $\tau^{\geq n}(\mathcal{E})\sub\mathcal{E}$ and $\tau^{\leq n}(\mathcal{E})\sub\mathcal{E}$ (for example, if $\mathcal{E}=D_\mathcal{T}^*(\mathcal{A})$), and if $F$ is a triangulated functor, then $\dim^+(F)\leq d$ if and only if for any $X\in\mathcal{E}$, $n\in\Z$, and $i\geq n+d$, the canonical morphism
\[H^i(F(X))\to H^iF(\tau^{\geq n}X)\]
is an isomorphism. In fact, the implication $\Rightarrow$ follows from the exact sequence induced from the distinguished triangle (\ref{derived category dt for truncation functor-1}), since we have $H^i(F(\tau^{n-1}X))=0$ in this case. The converse implication is obtained by taking $X$ to be an arbitray complex in $\mathcal{E}^{\leq n-1}$. An equivalent condition is that if $f:X\to Y$ is a morphism in $\mathcal{E}$ such that $H^i(f)$ is an isomorphism for all $i\geq n$, (that is, if $f$ induces an isomorphism $\tau^{\geq n}X\to \tau^{\geq n}Y$), then $H^i(F(f))$ is an isomorphism for all $i\geq n+d$. Similarly, we have $\dim^-(F)\leq d$ if and only if the canonical morphism
\[H^i(F(\tau^{\leq n}X))\to H^iF(X)\]
is an isomorphism.\par
In particular, if $\mathcal{E}=\mathcal{T}$ is a weak Serre subcategory of $\mathcal{A}$ (also considered as a subcategory of $D(\mathcal{A})$), then $\mathcal{E}^{\geq 0}=\mathcal{E}=\mathcal{E}^{\leq 0}$, and we have
\begin{gather*}
\dim^+(F)\leq d\Leftrightarrow \text{$H^i(F(X))=0$ for all $i>d$ and $X\in\mathcal{T}$},\\
\dim^-(F)\leq d\Leftrightarrow \text{$H^i(F(X))=0$ for all $i<-d$ and $X\in\mathcal{T}$}.
\end{gather*}
\end{example}

\begin{proposition}\label{derived category functor dim on Serre subcat prop}
If $\mathcal{E}=D^*_\mathcal{T}(\mathcal{A})$ and $F$ is a triangulated functor, then
\[\dim^+(F)=\dim^+(F_0),\quad \dim^-(F)=\dim^-(F_0),\]
where $F_0$ is the restriction of $F$ to $\mathcal{T}$.
\end{proposition}
\begin{proof}
We deal with the case for $\dim^+(F)$, the case for $\dim^-(F)$ can be done similarly. First, we note that $\dim^+(F_0)\leq\dim^+(F)$ since $F'(\mathcal{E}^{\leq n})\sub F'(\mathcal{E}^{\leq n})$ for each $n\in\Z$. For the reverse inequality, we assume that $\dim^+(F_0)\leq d<+\infty$ and fix an integer $n\in\Z$. We prove that $H^i(F(X))=0$ for any $X\in\mathcal{E}^{\leq n}$ and $i>n+d$ by induction on the number $\nu=\nu(X)$ of non-vanishing cohomology objects of $X$. Since the case $\nu=0$ is trivial, we may assume that $\nu\geq 1$. If $\nu=1$, say $H=H^m(X)\neq 0$ for some $m\leq n$, and we have
\[X\cong\tau^{\leq m}\tau^{\geq m}X\cong H[-m]\]
by (\ref{derived category dt for truncation functor-4}). Since $F$ is a triangulated functor, we conclude that $F(X)\cong F(H)[-m]$, so by definition of $\dim^+(F_0)$,
\[H^i(F(X))\cong H^{i-m}(F(H))=H^{i-m}(F_0(H))=0\for i-m>d\]
whence the conclusion. If $\nu>1$, we choose an integer $s$ such that there exists integers $p<s\leq q$ with $H^p(X)\neq 0$ and $H^q(X)\neq 0$. Then $\nu(\tau^{\leq s-1}X)<\nu(X)$ and $\nu(\tau^{\geq s}X)<\nu(X)$, so by applying the induction hypothesis, we have
\[H^i(F(\tau^{\leq s-1}X))=H^i(\tau^{\geq s}X)=0\for i>n+d.\]
The inductive step then follows from the long exact sequence induced by the distinguished triangle (\ref{derived category dt for truncation functor-1}).
\end{proof}

\begin{proposition}[\textbf{Way-out Lemma}]\label{derived category way-out lemma}
Let $\mathcal{T}$ be a weak Serre subcategory of $\mathcal{A}$ and $*\in\{+,b,\emp\}$. Consider triangulated functors $F,G:D^*_\mathcal{T}(\mathcal{A})\to D(\mathcal{A}')$ and a morphism of functors $\eta:F\to G$, so that $\eta(X):F(X)\to G(X)$ is an isomorphism for any $X\in\mathcal{T}$. If one of the following conditions holds, then $\eta$ is an isomorphism:
\begin{enumerate}
    \item[(\rmnum{1})] $*=b$;
    \item[(\rmnum{2})] $*=+$ and $F,G$ are bounded below;
    \item[(\rmnum{3})] $*=-$ and $F,G$ are bounded above;
    \item[(\rmnum{4})] $*=\emp$ and $F,G$ are bounded.
\end{enumerate}
Moreover, if $\mathfrak{I}$ (resp. $\mathfrak{P}$) is a subset of $\Ob(\mathcal{T})$ such that for any $X\in\mathcal{T}$ there exists a monomorphism $X\rightarrowtail I$ with $I\in\mathfrak{I}$ (resp. an epimorphism $P\to X$ with $P\in\mathfrak{P}$), then $\eta$ is an isomorphism if $\eta(X):F(X)\to G(X)$ is an isomorphism for each $X\in\mathfrak{I}$, and one of conditions (\rmnum{1}), (\rmnum{2}) (resp. (\rmnum{1}), (\rmnum{3})) is satisfied.
\end{proposition}
\begin{proof}
We first deal with the case $*=b$. Since $\eta$ is a morphism of triangulated functors, we see by induction on $|n|$ that $\eta(X[n])$ is an isomorphism for each $X\in\mathcal{T}$ and $n\in\Z$. To see that $\eta(X)$ is an isomorphism for any $X\in D_\mathcal{T}^*(\mathcal{A})$, we may replace $X$ with the isomorphic complex $\tau^{\leq n}(X)$ with some integer $n$ large enough. From (\ref{derived category dt for truncation functor-2}), we obtain a morphism of triangles, induced by $\eta$:
\[\begin{tikzcd}[column sep=5mm]
F(H^n(X)[-n-1])\ar[r]\ar[d]&F(\tau^{\leq n-1}X)\ar[r]\ar[d]&F(\tau^{\leq n}X)\ar[r]\ar[d]&F(H^n(X)[-n])\ar[d]\\
G(H^n(X)[-n-1])\ar[r]&G(\tau^{\leq n-1}X)\ar[r]&G(\tau^{\leq n}X)\ar[r]&G(H^n(X)[-n])
\end{tikzcd}\]
and then we can conclude the proposition by \cref{triangle cat morphism dt isomorphism 2 of 3} and induction on the number of non-vanishing cohomology objects of $X$ (a number which is less for $\tau^{\leq n-1}X$ than for $X$ whenever $n$ is finite).\par
As for the case of (\rmnum{2}), by \cref{derived category of abelian cat prop}, it suffices to show that $\eta(X)$ induces an isomorphism from $H^i(F(X))$ to $H^i(G(X))$ for any $X\in D_\mathcal{T}^+(\mathcal{A})$ and all $i\in\Z$. For this, we may apply \cref{derived category functor dim leq d iff truncation quasi-isomorphism} to replace $X$ by $\tau^{\leq i+d}X\in D^b_\mathcal{T}(\mathcal{A})$ for $d\geq\max\{\dim^-(F),\dim^-(G)\}$, and then we can apply the conclusion of (\rmnum{1}). The case for (\rmnum{3}) can be proved similarly.\par
We now consider the case where $*=\emp$. In view of (\ref{derived category dt for truncation functor-1}), we have a morphism of triangles, induced by $\eta$:
\[\begin{tikzcd}[column sep=5mm]
F(\tau^{\geq 0}(X)[-1])\ar[r]\ar[d]&F(\tau^{\leq-1}X)\ar[r]\ar[d]&F(X)\ar[r]\ar[d]&F(\tau^{\geq 0}(X))\ar[d]\\
G(\tau^{\geq 0}(X)[-1])\ar[r]&G(\tau^{\leq-1}X)\ar[r]&G(X)\ar[r]&G(\tau^{\geq 0}(X))
\end{tikzcd}\]
and by induction on the number of non-vanishing cohomology objects, we may assume that the vertical morphisms, except $F(X)\to G(X)$, are all isomorphisms. It then follows from \cref{triangle cat morphism dt isomorphism 2 of 3} that $F(X)\to G(X)$ is also an isomorphism.\par
Finally, as for the last assertion (we consider the case for $\mathfrak{I}$, the other case can be proved similarly), it suffices to show that $\eta(X)$ is an isomorphism for any $X\in\mathcal{T}$. We take an exact sequence $0\to X\to I^0\to I^1\to\cdots$ so that each $I^i\in\mathfrak{I}$. Then gives rise to a quasi-isomorphism $X\to I$, so it remains to show that $\eta(I)$ is an isomorphism. For this, we consider the stupid truncations $\sigma^{\leq n}I$ and $\sigma^{\geq n+1}I$, which fit into an exact sequence
\[\begin{tikzcd}
0\ar[r]&\sigma^{\geq n+1}\ar[r]&I\ar[r]&\tau^{\leq n}I\ar[r]&0
\end{tikzcd}\]
and we have a corresponding distinguished triangle in $D(\mathcal{A})$; it then suffices to mimic the proof of (\rmnum{2}).
\end{proof}
\subsection{Derived functors}
Let $\mathcal{A}$, $\mathcal{A}'$ and $\mathcal{A}''$ be abelian categories and $F:\mathcal{A}\to\mathcal{A}'$ be an additive functor. Then $F$ defines naturally a triangulated functor
\[K^*(F):K^*(\mathcal{A})\to K^*(\mathcal{A}').\]
For short, we often write $F$ instead of $K^*(F)$. We shall denote by $Q:K^*(\mathcal{A})\to D^*(\mathcal{A})$ the localization functor, and similarly with $Q'$, $Q''$, when replacing $\mathcal{A}$ with $\mathcal{A}'$, $\mathcal{A}''$.

\begin{definition}
Let $*\in\{+,b,\emp\}$. We say that the functor $F$ is \textbf{right derivable} (or $F$ admits a \textbf{right derived functor}) on $K^*(\mathcal{A})$ if the triangulated functor $K^*(F):K^*(\mathcal{A})\to K^*(\mathcal{A}')$ is universally right localizable with respect to $N^*(\mathcal{A})$ and $N^*(\mathcal{A}')$. In such a case the localization of $F$ is denoted by $R^*F$ and $H^n\circ R^*F$ is denoted by $R^nF$. The functor $R^*F:D^*(\mathcal{A})\to D^*(\mathcal{A}')$ is called the \textbf{right derived functor} of $F$ and $R^nF$ the \textbf{$\bm{n}$-th derived functor} of $F$.
\end{definition}

By definition, the functor $F$ admits a right derived functor on $K^*(\mathcal{A})$ if the ind-object
\[\rlim_{\substack{(X\to X')\in\Qis\\ X'\in K^*(\mathcal{A})}}Q'\circ K(F)(X')\]
is representable in $D^*(\mathcal{A}')$ for all $X\in K^*(\mathcal{A})$. In such a case, this object is isomorphic to $R^*F(X)$. Note that $R^*F$ is a triangulated functor from $D^*(\mathcal{A})$ to $D^*(\mathcal{A})$ if it exists, and $R^nF$ is a cohomological functor from $D^*(\mathcal{A})$ to $\mathcal{A}'$. Morever, if $RF$ exists, then $R^+F$ exists and $R^+F$ is the restriction of $RF$ to $D^+(\mathcal{A})$.

\begin{definition}
Let $\mathcal{J}$ be a full additive subcategory of $\mathcal{A}$. We say for short that $\mathcal{J}$ is \textbf{$\bm{F}$-injective} if the subcategory $K^+(\mathcal{J})$ of $K^+(\mathcal{A})$ is $K^+(F)$-injective with respect to $N^+(\mathcal{A})$ and $N^+(\mathcal{A}')$. We shall also say that $\mathcal{J}$ is injective with respect to $F$. We define similarly the notion of an $F$-projective subcategory.
\end{definition}
By the definition, $\mathcal{J}$ is $F$-injective if and only if for any $X\in K^+(\mathcal{A})$, there exists a quasi-isomorphism $X\to Y$ with $Y\in K^+(\mathcal{J})$ and $F(Y)$ is exact for any exact complex $Y\in K^+(\mathcal{J})$. If $F$ is right (resp. left) derivable, an object $X$ of $\mathcal{A}$ such that $R^nF(X)=0$ (resp. $L^nF(X)=0$) for all $n\neq 0$ is called \textbf{right $\bm{F}$-acyclic} (resp. \textbf{left $\bm{F}$-acyclic}). If $\mathcal{J}$ is an $F$-injective subcategory, then any object of $\mathcal{J}$ is right $F$-acyclic.\par
From \cref{triangle cat localization functor composition}, it is immediate that we have the following result concerning composition of derived functors:
\begin{proposition}\label{derived category functor derived composition prop}
Let $F:\mathcal{A}\to\mathcal{A}'$ and $F':\mathcal{A}'\to\mathcal{A}''$ be additive functors of abelian categories. Let $*\in\{+,b,\emp\}$ and assume that the right derived functors $R^*F$, $R^*F'$ and $R^*(F'\circ F)$ exist. Then there is a canonical morphism of functors
\begin{equation}\label{derived category functor derived composition prop-1}
R^*(F'\circ F)\to R^*(F')\circ R^*(F).
\end{equation}
Assume that there exist full additive subcategories $\mathcal{J}\sub\mathcal{A}$ and $\mathcal{J}'\sub\mathcal{A}'$ such that $\mathcal{A}$ is $F$-injective, $\mathcal{J}'$ is $F'$ injective and $F(\mathcal{J})\sub\mathcal{J}'$. Then $\mathcal{J}$ is $F'\circ F$-injective and (\ref{derived category functor derived composition prop-1}) induces an isomorphism
\[R^+(F'\circ F)\stackrel{\sim}{\to} R^+F'\circ R^+F.\]
\end{proposition}
Note that in many cases (even if $F$ is exact), $F$ may not send injective objects of $\mathcal{A}$ to injective objects of $\mathcal{A}'$. This is a reason why the notion of an "$F$-injective" category is useful. 

\begin{proposition}\label{derived category F-injective subcat iff}
Let $F:\mathcal{A}\to\mathcal{A}'$ be an additive functor of abelian categories and let $\mathcal{J}$ be a full additive subcategory of $\mathcal{A}$.
\begin{enumerate}
    \item[(a)] If $\mathcal{J}$ is $F$-injective, then $R^+F:D^+(\mathcal{A})\to D^+(\mathcal{A}')$ exists and $R^+F(X)\cong F(Y)$ for any quasi-isomorphism $X\to Y$ with $Y\in K^+(\mathcal{J})$.
    \item[(b)] If $F$ is left exact, then $\mathcal{J}$ is $F$-injective if and only if it satisfies the following conditions:
    \begin{enumerate}
        \item[(\rmnum{1})] the category $\mathcal{J}$ is cogenerating in $\mathcal{A}$;
        \item[(\rmnum{2})] for any exact sequence $0\to X'\to X\to X''\to 0$, the sequence $0\to F(X')\to F(X)\to F(X'')\to 0$ is exact as soon as $X\in\mathcal{J}$ and there exists an exact sequence
        \[\begin{tikzcd}
        0\ar[r]&Y^0\ar[r]&\cdots\ar[r]&Y^n\ar[r]&X'\ar[r]&0
        \end{tikzcd}\]
        with $Y^i\in\mathcal{J}$.
    \end{enumerate}
\end{enumerate}
\end{proposition}
\begin{proof}
The first assertion follows from \cref{triangle cat localization of functor exist if} and (\ref{triangle cat localization functor expression-1}), so assume that $F$ is left exact. If $\mathcal{J}$ is $F$-injective, then for $X\in\mathcal{A}$, there exists a quasi-isomorphism $X\to Y$ with $Y\in K^+(\mathcal{J})$. The composition $X\to\ker d_Y^0\to H^0(Y)$ is then an isomorphism, so $X\to Y^0$ is a monomorphism and this proves that $\mathcal{J}$ is cogenerating in $\mathcal{A}$. By \cref{derived category resolution by subcat if}, there then exists an exact sequence $0\to X''\to Z^0\to Z^1\to\cdots$ with $Z^i\in\mathcal{J}$ for all $i$. The sequence
\[\begin{tikzcd}
0\ar[r]&Y^0\ar[r]&\cdots\ar[r]&Y^n\ar[r]&X\ar[r]&Z^0\ar[r]&Z^1\ar[r]&\cdots
\end{tikzcd}\]
is then exact and belongs to $K^+(\mathcal{J})$, so $F(X)\to F(Z^0)\to F(Z^1)$ is exact. Since $F$ is left exact, $F(X'')\cong \ker(F(Z^0)\to F(Z^1))$ and this implies that $F(X)\to F(X'')$ is an epimorphism.\par
Conversely, assume the two conditions in (b). By \cref{derived category resolution by subcat if}, for any $X\in K^+(\mathcal{A})$ there exists a quasi-isomorphism $X\to Y$ with $Y\in K^+(\mathcal{J})$, so it suffices to show that $F(X)$ is exact if $X\in K^+(\mathcal{J})$ is exact. To this end, we note that for each $n\in\Z$, the sequences
\[\begin{tikzcd}
\cdots\ar[r]&X^{n-2}\ar[r]&X^{n-1}\ar[r]&\ker d^n_X\ar[r]&0
\end{tikzcd}\]
\vspace*{-5mm}
\[\begin{tikzcd}
0\ar[r]&\ker d^n_X\ar[r]&X^n\ar[r]&\ker d^{n+1}_X\ar[r]&0
\end{tikzcd}\]
are exact, so by condition (\rmnum{2}), the sequence $0\to F(\ker d^n_X)\to F(X^n)\to F(\ker d^{n+1}_X)\to 0$ is exact, and this proves that $F(X)$ is exact.
\end{proof}

\begin{remark}
Note that for $X\in\mathcal{A}$, $R^nF(X)=0$ for $n<0$ and assuming that $F$ is left exact, $R^0F(X)\cong F(X)$. Indeed, for $X\in\mathcal{A}$ and any quasi-isomorphism $X\to Y$, the composition $X\to Y\to \tau^{\geq 0}Y$ is a quasi-isomorphism.
\end{remark}
\begin{example}
If $\mathcal{A}$ has enough injectives, then the full subcategory $\mathcal{I}_\mathcal{A}$ of injective objects of $\mathcal{A}$ is $F$-injective for any additive functor $F:\mathcal{A}\to\mathcal{A}'$. Indeed, any exact complex in $\Ch^+(\mathcal{I})$ is homotopic to zero by \cref{abelian cat injective qis zero is null-homotopy}, whence its image under $F$. In particular, $R^+F:D^+(\mathcal{A})\to D^+(\mathcal{A}')$ exists in this case.
\end{example}

We shall now give a sufficient condition in order that $\mathcal{J}$ is $F$-injective, which is especially useful if the category $\mathcal{A}$ does not have enough injectives.

\begin{theorem}\label{derived category F-injective subcat if factor through monomorphism}
Let $\mathcal{J}$ be a full additive subcategory of $\mathcal{A}$ and let $F:\mathcal{A}\to\mathcal{A}'$ be a left exact functor. Assume that
\begin{enumerate}
    \item[(a)] $\mathcal{J}$ is cogenerating in $\mathcal{A}$;
    \item[(b)] for any monomorphism $Y'\rightarrowtail X$ with $Y'\in\mathcal{J}$, there exists an exact sequence $0\to Y'\to Y\to Y''\to 0$ with $Y,Y''\in\mathcal{J}$ such that $Y'\to Y$ factors through $X$ and the sequence $0\to F(Y')\to F(Y)\to F(Y'')\to 0$ is exact. 
\end{enumerate}
Then $\mathcal{J}$ is $F$-injective.
\end{theorem}
Condition (b) of \cref{derived category F-injective subcat if factor through monomorphism} may be visualized as
\[\begin{tikzcd}
0\ar[r]&Y'\ar[d,equal]\ar[r]&X\ar[d,dashed]\\
0\ar[r]&Y'\ar[r,dashed]&Y\ar[r,dashed]&Y''\ar[r]&0
\end{tikzcd}\]
Since this condition is rather intricate, the often consider the following particular case of \cref{derived category F-injective subcat if factor through monomorphism}, which is sufficient for most applications.
\begin{corollary}\label{derived category F-injective subcat if exact on cokernel of monomorphism}
Let $\mathcal{J}$ be a full additive subcategory of $\mathcal{A}$ and let $F:\mathcal{A}\to\mathcal{A}'$ be a left exact functor. Assume that
\begin{enumerate}
    \item[(a)] $\mathcal{J}$ is cogenerating in $\mathcal{A}$;
    \item[(b)] $\mathcal{J}$ is closed under cokernels of monomorphisms;
    \item[(c)] for any exact sequence $0\to X'\to X\to X''\to 0$ in $\mathcal{A}$ with $X',X\in\mathcal{J}$, the sequence $0\to F(X')\to F(X)\to F(X'')\to 0$ is exact.
\end{enumerate}
Then $\mathcal{J}$ is $F$-injective.
\end{corollary}
\begin{proof}
For any monomorphism $Y'\to X$ with $Y'\in\mathcal{J}$, we can embedd $X$ into an object $Y\in\mathcal{J}$ and take $Y''$ to be the cokernel of $Y$ by $Y'$:
\[\begin{tikzcd}
0\ar[r]&Y'\ar[d,equal]\ar[r]&X\ar[d,hook]\\
0\ar[r]&Y'\ar[r]&Y\ar[r]&Y''\ar[r]&0
\end{tikzcd}\]
By hypothesis, we have $Y''\in\mathcal{J}$, and the sequence $0\to F(X')\to F(X)\to F(X'')\to 0$ is exact, so we can apply \cref{derived category F-injective subcat if factor through monomorphism}.
\end{proof}
\begin{corollary}\label{derived category F-acyclic subcat is F-injective if}
Let $F:\mathcal{A}\to\mathcal{A}'$ be a left exact functor of abelian categories and let $\mathcal{J}$ be an $F$-injective full subcategory of $\mathcal{A}$. Let $\mathcal{J}_F$ be the full subcategory of $\mathcal{A}$ consisting of right $F$-acyclic objects, then $\mathcal{J}_F$ contains $\mathcal{J}$ and $\mathcal{J}_F$ satisfies the conditions of \cref{derived category F-injective subcat if exact on cokernel of monomorphism}. In particular, $\mathcal{J}_F$ is $F$-injective.
\end{corollary}
\begin{proof}
Since $\mathcal{J}_F$ contains $\mathcal{J}$, $\mathcal{J}_F$ is cogenerating. Consider an exact sequence $0\to X'\to X\to X''\to 0$ in $\mathcal{A}$ with $X',X\in\mathcal{J}_F$. The exact sequences $R^iF(X)\to R^iF(X'')\to R^{i+1}F(X')$ for $i\geq 0$ imply that $R^iF(X'')=0$ for $i>0$. Moreover, there is an exact sequence $0\to F(X')\to F(X)\to F(X'')\to 0$ since $R^1F(X')=0$.
\end{proof}
Therefore, a full additive subcategory $\mathcal{J}$ of $\mathcal{A}$ is $F$-injective if and only if it is cogenerating and any object of $\mathcal{J}$ is $F$-acyclic (assuming the right derivability of $F$). Note that even if $F$ is right derivable, there may not exist an $F$-injective subcategory, since we do not know that whether the subcategory of $F$-acyclic objects is cogenerating.
\begin{example}
Let $A$ be a ring and let $N$ be a right $A$-module. The full additive subcategory of $\mathbf{Mod}(A)$ consisting of flat $A$-modules is $(N\otimes_A-)$-projective. In fact, this subcategory satisfies the dual conditions of \cref{derived category F-injective subcat if exact on cokernel of monomorphism}.
\end{example}

We now turn to the proof of \cref{derived category F-injective subcat if factor through monomorphism}, which we decompose into several lemmas.
\begin{lemma}\label{derived category F-acyclic if factor through monomorphism}
With the assumptions of \cref{derived category F-injective subcat if factor through monomorphism}, let $0\to Y'\to X\to X''$ be an exact sequence in $\mathcal{A}$ with $Y'\in\mathcal{J}$. Then the sequence $0\to F(Y')\to F(X)\to F(X'')$ is exact.
\end{lemma}
\begin{proof}
Choose an exact sequence $0\to Y'\to Y\to Y''$ as in \cref{derived category F-injective subcat if factor through monomorphism}. We get the commutative exact diagram:
\[\begin{tikzcd}
0\ar[r]&Y'\ar[d,equal]\ar[r]&X\ar[r]\ar[d]&X''\ar[r]\ar[d]&0\\
0\ar[r]&Y'\ar[r]&Y\ar[r]&Y''\ar[r]&0
\end{tikzcd}\]
where the right square is Cartesian. Since $F$ is left exact, it transforms this square to a Cartesian square and the bottom row to an exact row. Hence, the result follows from (\cite{kashiwara_SAC} lemma 8.3.11).
\end{proof}
\begin{lemma}\label{derived category truncation resolution if factor through monomorphism}
With the assumptions of \cref{derived category F-injective subcat if factor through monomorphism}, let $X^\bullet\in\Ch^+(\mathcal{A})$ be an exact complex, and assume $X^i=0$ for $i<n$ and $X^n\in\mathcal{J}$. There exist an exact complex $Y^\bullet\in\Ch^+(\mathcal{J})$ and a morphism $f:X^\bullet\to Y^\bullet$ such that $Y^i=0$ for $i<n$, $f^n:X^n\to Y^n$ is an isomorphism, and $\im d^i_Y\in\mathcal{J}$ for all $i$.
\end{lemma}
\begin{proof}
We argue by induction. By the hypothesis of \cref{derived category F-injective subcat if factor through monomorphism}, there exists a commutative
exact diagram:
\[\begin{tikzcd}
0\ar[r]&X^n\ar[r]\ar[d,"\sim"]&X^{n+1}\ar[d]\\
0\ar[r]&Y^n\ar[r]&Y^{n+1}\ar[r]&Z^{n+2}\ar[r]&0
\end{tikzcd}\]
with $Y^{n+1},Z^{n+2}$ in $\mathcal{J}$. Now suppose that we have already constructed a diagram
\[\begin{tikzcd}
0\ar[r]&X^n\ar[r]\ar[d,"\sim"]&\cdots\ar[r]&X^m\ar[d]\\
0\ar[r]&Y^n\ar[r]&\cdots\ar[r]&Y^m\ar[r]&Z^{m+1}\ar[r]&0
\end{tikzcd}\]
where the bottom row is exact and belongs to $\Ch^+(\mathcal{J})$, and $\im d_Y^i$ belongs to $\mathcal{J}$ for $n\leq i\leq m-1$. Define $W^{m+1}=X^{m+1}\oplus_{\coker d_X^{m-1}}Z^{m+1}$, so that we have a Cartesian exact diagram
\[\begin{tikzcd}
0\ar[r]&\coker d_X^{m-1}\ar[d]\ar[r]&X^{m+1}\ar[d]\\
0\ar[r]&Z^{m+1}\ar[r]&W^{m+1}
\end{tikzcd}\]
By the hypotheses, there exists an exact commutative diagram
\[\begin{tikzcd}
0\ar[r]&Z^{m+1}\ar[d,equal]\ar[r]&W^{m+1}\ar[d]\\
0\ar[r]&Z^{m+1}\ar[r]&Y^{m+1}\ar[r]&Z^{m+2}\ar[r]&0
\end{tikzcd}\]
with $Y^{m+1}$ and $Z^{m+2}$ in $\mathcal{J}$. If we define $d^m_Y$ to be the composition $Y^m\to Z^{m+1}\to Y^{m+1}$, then $\im d^m_Y\cong Z^{m+1}\in\mathcal{J}$, and this completes the induction process.
\end{proof}

\begin{proof}[\textbf{Proof of \cref{derived category F-injective subcat if factor through monomorphism}}]
Let $X^\bullet\in\Ch^+(\mathcal{J})$ be an exact complex, we have to prove that $F(X^\bullet)$ is exact. Let us show by induction on $m-n$ that $H^m(F(X^\bullet))=0$ if $X\in\Ch^{\geq n}(\mathcal{J})$. If $m<n$, this is clear, so we may assume that $m\geq n$. By \cref{derived category truncation resolution if factor through monomorphism}, there exists a morphism of complexes $f:X^\bullet\to Y^\bullet$ in $\Ch^+(\mathcal{J})$ such that $Y^\bullet\in\Ch^{\geq n}(\mathcal{J})$, $f^n:X^n\to Y^n$ is an isomorphism and $F(Y^\bullet)$ is exact. Let $\sigma^{\geq n+1}$ denote the stupid truncated complexes and $W$ denote the mapping cone of the morphism
\[\sigma^{\geq n+1}(f):\sigma^{\geq n+1}X^\bullet\to \sigma^{\geq n+1}Y^\bullet.\]
Then $W^i=(\sigma^{\geq n+1}X^\bullet)^{i+1}\oplus(\sigma^{\geq n+1}Y^\bullet)^i=0$ for $i<n$, and we have a distinguished triangle in $K(\mathcal{J})$:
\[\begin{tikzcd}
W\ar[r]&M(f)\ar[r]&M(X^n[-n]\to Y^n[-n])\ar[r]&W[1].
\end{tikzcd}\]
Since $X^n\to Y^n$ is an isomorphism, the mapping cone $M(X^n[-n]\to Y^n[-n])$ is exact and therefore $W\to M(f)$ is an isomorphism in $K(\mathcal{A})$. Applying the functor $F$, we then obtain an isomorphism $F(W)\cong F(M(f))$ in $K(\mathcal{A}')$, so $H^i(F(W))\cong H^i(F(M(f)))$ for each $i$. On the other hand, there is a distinguished triangle in $K^+(\mathcal{A}')$:
\[\begin{tikzcd}
F(X)\ar[r]&F(Y)\ar[r]&F(M(f))\ar[r]&F(X)[1]
\end{tikzcd}\]
and $F(Y)$ is exact by our hypothesis, whence $H^m(F(X))\cong H^{m-1}(F(M(f)))\cong H^{m-1}(F(W))$. Since $W$ is an exact complex and belongs to $\Ch^{\geq n}(\mathcal{J})$, the induction hypothesis implies that $H^{m-1}(F(W))=0$, so we conclude that $H^m(F(X))=0$. 
\end{proof}

\paragraph{Derived projective limit}
As an application of \cref{derived category F-injective subcat if factor through monomorphism} we shall discuss the existence of the derived functor of projective limits. Let $\mathcal{A}$ be an abelian $\mathscr{U}$-category. Recall that $\Pro(\mathcal{A})$ is an abelian category admitting small projective limits, and small filtrant projective limits as well as small products are exact. Assume that $\mathcal{A}$ admits small projective limits. Then the natural exact functor $\mathcal{A}\to\Pro(\mathcal{A})$ admits a right adjoint
\[\pi_\mathcal{A}:\Pro(\mathcal{A})\to\mathcal{A}\]
defined as follows: if $\beta:I^{\op}\to\mathcal{A}$ is a functor with $I$ small and filtrant, then $\pi_\mathcal{A}$ transforms $\llim\beta$ (as a pro-object) to the limit $\llim\beta$ in $\mathcal{A}$. The functor $\pi_\mathcal{A}$ is left exact and we shall give a condition in order that it is right derivable.\par
For a full additive subcategory $\mathcal{J}$ of $\mathcal{A}$, we define a full additive subcategory $\mathcal{J}_{\pro}$ of $\Pro(\mathcal{A})$ by
\[\mathcal{J}_{\pro}=\{x\in\Pro(\mathcal{A}):\text{$X\cong\prod_{i\in I}X_i$ for a small set $I$ and $X_i\in\mathcal{J}$}\}.\]
Here the product $\prod_i$ is taken in the category $\Pro(\mathcal{A})$, so for $X_i,Y\in\mathcal{A}$, we have a canonical bijection
\[\Hom_{\Pro(\mathcal{A})}(\prod_{i\in I}X_i,Y)\stackrel{\sim}{\to} \bigoplus_{i\in I}\Hom_{\mathcal{A}}(X_i,Y).\]

\begin{proposition}\label{derived category derived functor of proj limit}
Let $\mathcal{A}$ be an abelian category admitting small projective limits and let $\mathcal{J}$ be a full additive subcategory of $\mathcal{A}$ satisfying:
\begin{enumerate}
    \item[(a)] $\mathcal{J}$ is cogenerating in $\mathcal{A}$;
    \item[(b)] $\mathcal{J}$ is closed under cokernels of monomorphisms;
    \item[(c)] if $0\to Y_i'\to Y_i\to Y''_i\to 0$ is a family of exact sequences in $\mathcal{J}$ indexed by a small set $I$, then the sequence $0\to Y_i'\to Y_i\to Y''_i\to 0$ is exact.
\end{enumerate}
Then the category $\mathcal{J}_{\pro}$ is $\pi_\mathcal{A}$-injective and the left exact functor $\pi_\mathcal{A}$ admits a right derived functor
\[R^+\pi_\mathcal{A}:D^+(\Pro(\mathcal{A}))\to D^+(\mathcal{A})\]
which satisfies $R^n\pi_\mathcal{A}(\prod_iX_i)=0$ for $n>0$ and $X_i\in\mathcal{J}$. Moreover, the composition
\[\begin{tikzcd}
D^+(\mathcal{A})\ar[r]&D^+(\Pro(\mathcal{A}))\ar[r,"R^+\pi_\mathcal{A}"]&D^+(\mathcal{A})
\end{tikzcd}\]
is isomorphic to the identity.
\end{proposition}
\begin{proof}
We shall verify the conditions of \cref{derived category F-injective subcat if factor through monomorphism}. The category $\mathcal{J}_{\pro}$ is cogenerating in $\Pro(\mathcal{A})$ since for $A=\llim_{i\in I}\alpha(i)\in\Pro(\mathcal{A})$, we obtain a monomorphism $A\rightarrowtail\prod_iX_i$ by choosing a monomorphism $\alpha(i)\rightarrowtail X_i$ with $X_i\in\mathcal{J}$ for each $i\in I$. Now consider a monomorphism $Y\rightarrowtail A$ in $\Pro(\mathcal{A})$ with $A\in\Pro(\mathcal{A})$ and $Y=\prod_iY_i$, $Y_i\in\mathcal{J}$. Applying the dual version of (\cite{kashiwara_SAC} proposition 8.6.9), for each $i$ we can find $X_i\in\mathcal{A}$ and a commutative exact diagram
\[\begin{tikzcd}
0\ar[r]&Y\ar[r]\ar[d]&A\ar[d]\\
0\ar[r]&Y_i\ar[r]&X_i
\end{tikzcd}\]
Since $\mathcal{J}$ is cogenerating, we may assume that $X_i\in\mathcal{J}$. Let $Z_i=\coker(Y_i\to X_i)$, which is in $\mathcal{J}$ by hypothesis. By hypothesis, functor $\prod$ is exact on $\mathcal{J}$, so we obtain a commutative diagram
\[\begin{tikzcd}
0\ar[r]&Y\ar[r]\ar[d,"\sim"]&A\ar[d]\\
0\ar[r]&\prod_iY_i\ar[r]&\prod_iX_i\ar[r]&\prod_iZ_i\ar[r]&0
\end{tikzcd}\]
with exact rows. Applying $\pi_\mathcal{A}$ to the second row, we then obtain the sequence $0\to \prod_iY_i\to \prod_iX_i\to\prod_iZ_i\to 0$ in $\mathcal{A}$, which is exact by hypothesis (c). By \cref{derived category F-injective subcat if factor through monomorphism}, we then conclude that $\mathcal{J}_{\pro}$ is $\pi_\mathcal{A}$-injective, so $\pi_\mathcal{A}$ admits a right derived functor (\cref{triangle cat localization of functor exist if}), and we have $R^n\pi_\mathcal{A}(\prod_iX_i)=0$ for $n>0$ and $X_i\in\mathcal{J}$.\par
Finally, by assumption $\mathcal{J}$ is injective with respect to the exact functor $\mathcal{A}\to\Pro(\mathcal{A})$. Since the functor $\mathcal{A}\to\Pro(\mathcal{A})$ sends $\mathcal{J}$ to $\mathcal{J}_{\pro}$, the last assertion follows from \cref{derived category functor derived composition prop}.
\end{proof}

\begin{corollary}
Let $\mathcal{J}$ be a full additive subcategory of $\mathcal{A}$ satisfying the conditions of \cref{derived category derived functor of proj limit}. If $(X_n)$ is a projective system in $\mathcal{J}$ indexed by $\N$ and $X=\llim_nX_n$, then $R^p\pi_\mathcal{A}(X)=0$ for $p>1$, and we have a canonical isomorphism
\[R^1\pi_\mathcal{A}(X)\stackrel{\sim}{\to} \coker\Big(\prod_nX_n\stackrel{\Delta_X}{\longrightarrow}\prod_nX_n\Big)\]
where $\Delta_X:=T_X-\id:\prod_nX_n\to\prod_nX_n$.
\end{corollary}
\begin{proof}
We have an exact sequence in $\Pro(\mathcal{A})$:
\[\begin{tikzcd}
0\ar[r]&\llim_nX_n\ar[r]&\prod_nX_n\ar[r,"\Delta_X"]&\prod_nX_n\ar[r]&0
\end{tikzcd}\]
Applying the functor $R^+\pi_\mathcal{A}$, we then get a long exact sequence and the results follows since $R^p\pi_\mathcal{A}(\prod_nX_n)=0$ for $p>0$.
\end{proof}

\begin{example}
If $A$ is a ring and $\mathcal{A}=\mathbf{Mod}(A)$, we may choose $\mathcal{J}=\mathcal{A}$ in \cref{derived category derived functor of proj limit}. In fact, for a family of objects $(X_i)_{i\in I}$ in $\mathcal{A}$, the product $\prod_iX_i$ can be considered as the limit of the functor $\alpha:I\to\mathcal{A}$ with $I$ being considered as a discrete category.
\end{example}

\paragraph{Derived bifunctors}
We now apply the previous results to bifunctors between abelian categories. The most important examples in mind will be $\Hom$ and tensor functors.
\begin{theorem}\label{derived category R^0Hom is Hom in D(A)}
Let $\mathcal{A}$ be an abelian category and $X,Y\in D(\mathcal{A})$. Assume that the functor
\[\Hom_\mathcal{A}^\bullet:K(\mathcal{A})^{\op}\times K(\mathcal{A})\to K(\mathbf{Mod}(\Z)),\quad (X',Y')\mapsto \Hom_\mathcal{A}^{\bullet}(X',Y')\]
is right localizable at $(X,Y)$, then for any $n\in\Z$, we have
\[R^n\Hom_\mathcal{A}(X,Y)\cong\Hom_{D(\mathcal{A})}(X,Y[n]).\]
\end{theorem}
\begin{proof}
By hypothesis, we have
\[R\Hom_\mathcal{A}(X,Y)\stackrel{\sim}{\to} \rlim_{\substack{(X'\to X)\in\Qis,\\(Y\to Y')\in\Qis}}\Hom_\mathcal{A}^\bullet(X',Y').\]
Applying the functor $H^n$ and recalling that $\rlim$ commutes with $H^n$, we conclude from (\cite{kashiwara_SAC} Proposition 11.7.3) that
\begin{equation*}
R^n\Hom_\mathcal{A}(X,Y)\cong \rlim H^n(\Hom_\mathcal{A}^\bullet(X',Y'))\cong \rlim \Hom_{K(\mathcal{A})}(X',Y'[n])\cong\Hom_{D(\mathcal{A})}(X,Y[n]).\qedhere
\end{equation*}
\end{proof}

Consider now three abelian categories $\mathcal{A}$, $\mathcal{A}'$, $\mathcal{A}''$ and an additive bifunctor
\[F:\mathcal{A}\times\mathcal{A}'\to\mathcal{A}''.\]
By (\cite{kashiwara_SAC} Proposition 11.6.3), the triangulated functor
\[K^+F:K^+(\mathcal{A})\times K^+(\mathcal{A}')\to K^+(\mathcal{A}'')\]
is naturally defined by setting
\[K^+F(X,X')=\Tot(F(X,X')).\]
Similarly to the case of functors, if the triangulated bifunctor $K^+F$ is universally right localizable with respect to $(N^+(\mathcal{A})\times N^+(\mathcal{A}'),N^+(\mathcal{A}''))$, then $F$ is said to be \textbf{right derivable} and its localization is denoted by $R^+F$. We set $R^nF=H^n\circ R^+F$ and call it the $n$-th \textbf{derived bifunctor} of $F$.

\begin{definition}
Let $\mathcal{J}$ and $\mathcal{J}'$ be full additive subcategories of $\mathcal{A}$ and $\mathcal{A}'$ respectively. We say for short that $(\mathcal{J},\mathcal{J}')$ is \textbf{$\bm{F}$-injective} if $(K^+(\mathcal{J}),K^+(\mathcal{J}'))$ is $K^+F$-injective.
\end{definition}

\begin{proposition}\label{derived category bifunctor derive exists if}
Let $\mathcal{J}$ and $\mathcal{J}'$ be full additive subcategories of $\mathcal{A}$ and $\mathcal{A}'$ respectively. Assume that $(\mathcal{J},\mathcal{J}')$ is $F$-injective, then $F$ is right derivable and for $(X,X')\in D^+(\mathcal{A})\times D^+(\mathcal{A}')$ we have
\[R^+F(X,X')\cong Q''\circ K^+F(Y,Y')\]
for $(X\to Y)\in\Qis$ and $(X'\to Y')\in\Qis$ with $Y\in K^+(\mathcal{J})$ and $Y'\in K^+(\mathcal{J}')$.
\end{proposition}
\begin{proof}
It suffices to apply \cref{triangle cat localization bifunctor exists if} to the functor $Q''\circ K^+F$.
\end{proof}

\begin{proposition}\label{derived category bifunctor F-injective if separate}
Let $\mathcal{J}$ and $\mathcal{J}'$ be full additive subcategories of $\mathcal{A}$ and $\mathcal{A}'$ respectively. Assume that 
\begin{enumerate}
    \item[(a)] for any $Y\in\mathcal{J}$, $\mathcal{J}'$ is $F(Y,-)$-injective;
    \item[(b)] for any $Y'\in\mathcal{J}'$, $\mathcal{J}$ is $F(-,Y')$-injective.
\end{enumerate}
Then $(\mathcal{J},\mathcal{J}')$ is $F$-injective.
\end{proposition}
\begin{proof}
Let $(Y,Y')\in K^+(\mathcal{J})\times K^+(\mathcal{J}')$. If either $Y$ or $Y'$ is quasi-isomorphic to zero, then $\Tot(F(Y,Y'))$ is quasi-isomorphic to zero by (\cite{kashiwara_SAC} Proposition 12.5.5), so $(\mathcal{J},\mathcal{J}')$ is $F$-injective.
\end{proof}

\begin{corollary}\label{derived category bifunctor localization if exact one variable}
Let $\mathcal{J}$ be a full additive cogenerating subcategory of $\mathcal{A}$ and assume:
\begin{enumerate}
    \item[(a)] for any $X\in\mathcal{J}$, $F(X,-)$ is exact;
    \item[(b)] for any $X'\in\mathcal{A}'$, $\mathcal{J}$ is $F(-,X')$-injective.
\end{enumerate}
Then $F$ is right derivable and for $X\in K^+(\mathcal{A})$, $X'\in K^+(\mathcal{A}')$,
\[R^+F(X,X')\cong Q''\circ K^+F(Y,X')\]
for any $(X\to Y)\in\Qis$ with $Y\in K^+(\mathcal{J})$. In particular, for $X\in\mathcal{A}$ and $X'\in\mathcal{A}'$, $R^+F(X,X')$ is the derived functor of $F(-,X')$ calculated at $X$, that is, we have
\[R^+F(X,X')=(R^+F(-,X'))(X).\]
\end{corollary}
\begin{proof}
The first assertion follows from \cref{derived category bifunctor F-injective if separate} by setting $\mathcal{J}'=\mathcal{A}'$, and the second one follows from \cref{triangle cat localization bifunctor if exact one variable}.
\end{proof}

\begin{corollary}\label{derived category bounded RHom exists if}
Let $\mathcal{A}$ be an abelian category and assume that there are subcategories $\mathcal{P},\mathcal{J}$ in $\mathcal{A}$ such that $(\mathcal{P}^{\op},\mathcal{J})$ is injective with respect to the functor $\Hom_\mathcal{A}$. Then the functor $\Hom_\mathcal{A}$ admits a right derived functor $R^+\Hom_\mathcal{A}:D^-(\mathcal{A})^{\op}\times D^+(\mathcal{A})\to D^+(\Z)$. In particular, $D^b(\mathcal{A})$ is a $\mathscr{U}$-category.
\end{corollary}
\begin{proof}
The first assertion follows from \cref{derived category bifunctor F-injective if separate}, and the second one is a concequence of \cref{derived category R^0Hom is Hom in D(A)}, since $R^0\Hom$ takes its values in $\mathscr{U}$-sets.
\end{proof}

\begin{example}\label{derived category bounded RHom expression eg}
Assume that $\mathcal{A}$ has enough injectives. Then by \cref{derived category bifunctor localization if exact one variable}, the derived Hom functor
\[R^+\Hom_\mathcal{A}:D^-(\mathcal{A})^{\op}\times D^+(\mathcal{A})\to D^+(\Z)\]
exists and may be calculated as follows. Let $X\in D^-(\mathcal{A})$ and $Y\in D^+(\mathcal{A})$. Then there exists a quasi-isomorphism $Y\to I$ in $K^+(\mathcal{A})$, with the $I^i$'s being injective, and
\[R^+\Hom_\mathcal{A}(X,Y)\cong\Hom_\mathcal{A}^\bullet(X,I).\]
If $\mathcal{A}$ has enough projectives, then $R^+\Hom_\mathcal{A}$ also exists, and for a quasi-isomorphism $P\to X$ with $P^i$'s being projective, we have
\[R^+\Hom_\mathcal{A}(X,Y)\cong\Hom_\mathcal{A}^\bullet(P,Y).\]
These isomorphisms hold in $D^+(\Z)$, which means $R^+\Hom_\mathcal{A}(X,Y)\in D^+(\Z)$ is represented by the simple complex associated with the double complex $\Hom_\mathcal{A}^{\bullet,\bullet}(X,I)$ or $\Hom_\mathcal{A}^{\bullet,\bullet}(P,Y)$.
\end{example}
\begin{example}\label{derived category bounded Ltensor expression eg}
Let $A$ be a $k$-algebra, with $k$ being a ring. Since the category $\mathbf{Mod}(A)$ has enough projectives, the left derived functor of the functor $-\otimes_A-$ is well defined:
\[-\otimes_A^L-:D^-(A^{\op})\times D^-(A)\to D^-(k).\]
This functor may by calculated as follows:
\[N\otimes_A^LM\cong\Tot(N\otimes_AP)\cong\Tot(Q\otimes_AM)\cong\Tot(Q\otimes_AP)\]
where $P$ is a complex of projective $A$-modules quasi-isomorphic to $M$ and $Q$ is a complex of projective $A^{\op}$-modules quasi-isomorphic to $N$. A classical notation is $\Tor_n^A(N,M):=H_n(N\otimes_A^LM)$.
\end{example}

\begin{proposition}\label{derived category bounded derived adjointness}
Let $F:\mathcal{A}\to\mathcal{A}':G$ be an adjoint pair of additive functors. Assume that $\mathcal{A}$ has enough projectives and $\mathcal{A}'$ has enough injective objects, then there exists a canonical isomorphism in $D^+(\Z)$:
\[R\Hom_{\mathcal{A}'}(L^-F(X),Y)\stackrel{\sim}{\to} R\Hom_{\mathcal{A}'}(X,R^+G(Y)),\]
where $X\in D^-(\mathcal{A})$ and $Y\in D^+(\mathcal{A}')$. In particular, we have canonical isomorphisms
\[\Hom_{D(\mathcal{A}')}(L^-F(X),Y)\stackrel{\sim}{\to} \Hom_{D(\mathcal{A})}(X,R^+G(Y))\]
\end{proposition}
\begin{proof}
We take a projective resolution $P\to X$ and an injective resolution $Y\to I$. By \cref{derived category bounded RHom expression eg}, we have
\[R\Hom_{\mathcal{A}'}(L^-F(X),Y)\cong \Hom^\bullet(K^-F(P),I).\]
By the adjointness, in view of the definition of $K^-F$ and the $\Hom$ complex, the RHS is isomorphic to $\Hom^\bullet(P,K^+G(I))$, which is $R\Hom_\mathcal{A}(X,R^+G(Y))$. The last assertion follows from \cref{derived category R^0Hom is Hom in D(A)} by taking $H^0$.
\end{proof}
Note that the functors $L^-F$ and $R^+G$ are not adjoint functors, since they are functors between different pair of categories. This problem shall be resolved after we introduce the unbounded version of derived functors.

\section{Unbounded derived categories}
In this section we study the unbounded derived categories of Grothendieck categories. We prove the existence of enough homotopically injective objects in order to define unbounded right derived functors, and we prove that these triangulated categories satisfy the hypotheses of the Brown representability theorem. We also study unbounded derived functors in particular for pairs of adjoint functors. We start this study in the framework of abelian categories with translation, then we apply it to the case of the categories of unbounded complexes in abelian categories.

\subsection{Derived categories of abelian categories with translation}
Let $(\mathcal{A},T)$ be an abelian category with translation. Recall  that, denoting by $\mathcal{N}$ the triangulated subcategory of the homotopy category $K_c(\mathcal{A})$ consisting of objects $X$ quasi-isomorphic to $0$, the derived category $D_c(\mathcal{A})$ of $(\mathcal{A},T)$ is the localization $K_c(\mathcal{A})/\mathcal{N}$. Also recall that an object $X$ is quasi-isomorphic to $0$ if and only the sequence
\[\begin{tikzcd}
T^{-1}(X)\ar[r,"{T^{-1}(d_X)}"]&X\ar[r,"d_X"]&T(X)
\end{tikzcd}\]
is exact.\par
For $X\in\mathcal{A}_c$, the differential $d_X:X\to T(X)$ is a morphism in $\mathcal{A}_c$, so its cohomology $H(X)$ is regarded as an object of $\mathcal{A}_c$ and similarly for $\ker d_X$ and $\im d_X$. Note that their differentials vanish.

\begin{proposition}\label{abelian translation derive direct sum}
Assume that $\mathcal{A}$ admits direct sums indexed by a set $I$ and that such direct sums are exact. Then $\mathcal{A}_c$, $K_c(\mathcal{A})$ and $D_c(\mathcal{A})$ admit such direct sums and the two functors $\mathcal{A}_c\to K_c(\mathcal{A})$ and $K_c(\mathcal{A})\to D_c(\mathcal{A})$ commute with such direct sums.
\end{proposition}
\begin{proof}
The result concerning $\mathcal{A}_c$ and $K_c(\mathcal{A})$ is immediate, and that concerning $D_c(\mathcal{A})$ follows from \cref{triangle cat localization and direct sum}.
\end{proof}

For an object $X$ of $\mathcal{A}$, we denote by $M(X)$ the mapping cone of $\id_{T^{-1}(X)}$, regarding $T^{-1}(X)$ as an object of $\mathcal{A}_c$ with zero differential. Hence $M(X)$ is the object $X\oplus T^{-1}(X)$ of $\mathcal{A}_c$ with the differential
\[d_{M(X)}=\begin{pmatrix}
0&0\\
\id_X&0
\end{pmatrix}:X\oplus T^{-1}(X)\to T(X)\oplus X.\]
The functor $M:\mathcal{A}\to\mathcal{A}_c$ is easily seen to be exact, and $M$ is a left adjoint of the forgetful functor $\mathcal{A}_c\to\mathcal{A}$, as seen by the following lemma.

\begin{lemma}\label{abelian translation adjoint of forgetful}
For $Y\in\mathcal{A}$ and $X\in\mathcal{A}_c$, we have the isomorphism
\[\Hom_{\mathcal{A}_c}(M(Y),X)\stackrel{\sim}{\to} \Hom_\mathcal{A}(Y,X)\]
\end{lemma}
\begin{proof}
Any morphism $(u,v):M(Y)\to X$ in $\mathcal{A}_c$ satisfies $d_X\circ (u,v)=T(u,v)\circ d_{M(X)}$, which reads as $d_X\circ u=Tv$ and $d_X\circ v=0$. Therefore, it is completely determined by $u:Y\to X$.
\end{proof}

\begin{proposition}\label{abelian translation complex cat is Grothendieck}
Let $\mathcal{A}$ be a Grothendieck category. Then $\mathcal{A}_c$ is again a Grothendieck category.
\end{proposition}
\begin{proof}
The category $\mathcal{A}_c$ is abelian and admits small inductive limits, and small filtrant inductive limits in $\mathcal{A}_c$ are clearly exact. Moreover, if $G$ is a generator in $\mathcal{A}$, then $M(G)$ is a generator in $\mathcal{A}_c$ by \cref{abelian translation adjoint of forgetful}.
\end{proof}

\begin{definition}
Let $(\mathcal{A},T)$ be an abelian category with translation.
\begin{itemize}
    \item An object $I\in K_c(\mathcal{A})$ is called \textbf{homotopically injective} if $\Hom_{K_c(\mathcal{A})}(X,I)=0$ for any $X\in K_c(\mathcal{A})$ that is quasi-isomorphic to $0$.
    \item An object $P\in K_c(\mathcal{A})$ is called \textbf{homotopically projective} if $\Hom_{K_c(\mathcal{A})}(P,X)=0$ for any $X\in K_c(\mathcal{A})$ that is quasi-isomorphic to $0$.
\end{itemize}
\end{definition}

We shall denote by $K_{c,hi}(\mathcal{A})$ the full subcategory of $K_c(\mathcal{A})$ consisting of homotopically injective objects and by $\iota:K_{c,hi}(\mathcal{A})\to K_c(\mathcal{A})$ the inclusion functor. Similarly, we denote by $K_{c,hp}(\mathcal{A})$ the full subcategory of $K_c(\mathcal{A})$ consisting of homotopically projective objects. Note that $K_{c,hi}(\mathcal{A})$ is obviously a full triangulated subcategory of $K_c(\mathcal{A})$.

\begin{lemma}\label{abelian translation K-injective Hom K is D}
Let $(\mathcal{A},T)$ be an abelian category with translation. If $I\in K_c(\mathcal{A})$ is homotopically injective, then
\[\Hom_{K_c(\mathcal{A})}(X,I)\stackrel{\sim}{\to} \Hom_{D_c(\mathcal{A})}(X,I)\]
for all $X\in K_c(\mathcal{A})$.
\end{lemma}
\begin{proof}
Let $X\in K_c(\mathcal{A})$ and $X'\to X$ be a quasi-isomorphism. Then for $I\in K_{c,hi}(\mathcal{A})$, the morphism
\[\Hom_{K_c(\mathcal{A})}(X,I)\to\Hom_{K_c(\mathcal{A})}(X',I)\]
is an isomorphism since there is a distinguished triangle $X'\to X\to N\to T(X)$ with $N$ quasi-isomorphic to zero and
\[\Hom_{K_c(\mathcal{A})}(N,I)\cong \Hom_{K_c(\mathcal{A})}(T^{-1}(N),I)=0.\]
Therefore, for any $X\in K_c(\mathcal{A})$ and $I\in K_{c,hi}(\mathcal{A})$, we have
\begin{equation*}
\Hom_{D_c(\mathcal{A})}(X,I)\cong\rlim_{(X'\to X)\in\Qis}\Hom_{K_c(\mathcal{A})}(X',I)\cong\Hom_{K_c(\mathcal{A})}(X,I).\qedhere
\end{equation*}
\end{proof}

We now introduce the following notation:
\[QM=\{f\in\Mor(\mathcal{A}_c):\text{$f$ is both a quasi-isomorphism and a monomorphism}\}.\]
Recall that an object $I$ of $\mathcal{A}_c$ is called $QM$-injective if for any morphism $f:X\to Y$ in $QM$, the induced map
\[f^*:\Hom_{\mathcal{A}_c}(Y,I)\to\Hom_{\mathcal{A}_c}(X,I)\]
is surjective.

\begin{proposition}\label{abelian translation QM-injective iff}
Let $I\in\mathcal{A}_c$, then $I$ is $QM$-injective if and only if it satisfies the following conditions:
\begin{enumerate}
    \item[(a)] $I$ is homotopically injective,
    \item[(b)] $I$ is injective as an object of $\mathcal{A}$.
\end{enumerate}
\end{proposition}
\begin{proof}

\end{proof}

We shall now prove the following theorem, which asserts that any Grothendieck category has enough $QM$-injective objects.
\begin{theorem}\label{abelian translation Grothendieck enough QM-injective}
Let $(\mathcal{A},T)$ be an abelian category with translation where $\mathcal{A}$ is a Grothendieck category. Then, for any $X\in\mathcal{A}_c$, there exists $u:X\to I$ such that $u\in QM$ and $I$ is $QM$-injective.
\end{theorem}

\begin{corollary}\label{abelian translation Grothendieck enough K-injective}
Let $(\mathcal{A},T)$ be an abelian category with translation where $\mathcal{A}$ is a Grothendieck category. Then for any $X\in\mathcal{A}_c$, there exists a quasi-isomorphism $X\to I$ such that $I$ is homotopically injective.
\end{corollary}

\begin{corollary}\label{abelian translation Grothendieck derived cat prop}
Let $(\mathcal{A},T)$ be an abelian category with translation where $\mathcal{A}$ is a Grothendieck category. Then
\begin{enumerate}
    \item[(a)] the localization functor $Q:K_c(\mathcal{A})\to D_c(\mathcal{A})$ induces an equivalence $Q\circ\iota:K_{c,hi}(\mathcal{A})\stackrel{\sim}{\to} D_c(\mathcal{A})$;
    \item[(b)] the category $D_c(\mathcal{A})$ is a $\mathscr{U}$-category;
    \item[(c)] the functor $Q:K_c(\mathcal{A})\to D_c(\mathcal{A})$ admits a right adjoint $R:D_c(\mathcal{A})\to K_c(\mathcal{A})$ so that $Q\circ R\cong\id$, and $R$ is the composition of $\iota:K_{c,hi}(\mathcal{A})\to K_c(\mathcal{A})$ and a quasi-inverse of $Q\circ\iota$;
    \item[(d)] for any triangulated category $D$, any triangulated functor $F:K_c(\mathcal{A})\to\mathcal{D}$ admits a right localization $RF:D_c(\mathcal{A})\to\mathcal{D}$, and $RF\cong F\circ R$.
\end{enumerate}
\end{corollary}
\begin{proof}
The functor $Q:K_{c,hi}(\mathcal{A})$ is fully faithful by \cref{abelian translation K-injective Hom K is D} and essentially surjective by \cref{abelian translation Grothendieck enough K-injective}, whence assertion (a), and (b), (c) follow immediately: in fact, for $X\in K_c(\mathcal{A})$ and $Y\in D_c(\mathcal{A})$, we have $(Q\circ\iota)^{-1}(Y)\in K_{c,hi}(\mathcal{A})$, so by \cref{abelian translation K-injective Hom K is D},
\begin{align*}
\Hom_{K_c(\mathcal{A})}(X,\iota\circ(Q\circ\iota)^{-1}(Y))&\cong \Hom_{K_c(\mathcal{A})}(X,(Q\circ\iota)^{-1}(Y))\\
&\cong \Hom_{D_c(\mathcal{A})}(Q(X),(Q\circ\iota)\circ(Q\circ\iota)^{-1}(Y))\\
&\cong\Hom_{D_c(\mathcal{A})}(Q(X),Y).
\end{align*}
Finally, (d) follows from \cref{category localization of functor exist if} and (c).
\end{proof}

\subsection{The Brown representability theorem}
We shall show that the hypotheses of the Brown representability theorem are satisfied for $D_c(\mathcal{A})$ when $\mathcal{A}$ is a Grothendieck abelian category with translation. Note that $D_c(\mathcal{A})$ admits small direct sums and the localization functor $Q:K_c(\mathcal{A})\to D_c(\mathcal{A})$ commutes with such direct sums by \cref{abelian translation derive direct sum}.
\begin{theorem}\label{abelian translation Grothendieck derived Brown representability}
Let $(\mathcal{A},T)$ be an abelian category with translation where $\mathcal{A}$ is a Grothendieck category. Then the triangulated category $D_c(\mathcal{A})$ admits small direct sums and a system of $t$-generators.
\end{theorem}
Applying (\cite{kashiwara_SAC} corollary 10.5.2 and corollary 10.5.3), we then obtain the following corollaries.
\begin{corollary}\label{abelian translation Grothendieck derived presheaf representable if}
Let $(\mathcal{A},T)$ be an abelian category with translation where $\mathcal{A}$ is a Grothendieck category. Let $G:D^c(\mathcal{A})^{\op}\to\mathbf{Mod}(\Z)$ be a cohomological functor which commutes with small products, then $G$ is representable.
\end{corollary}
\begin{corollary}\label{abelian translation Grothendieck derived triangulated functor adjoint if}
Let $(\mathcal{A},T)$ be an abelian category with translation where $\mathcal{A}$ is a Grothendieck category. Let $\mathcal{D}$ be a triangulated category and $F:D_c(\mathcal{A})\to\mathcal{D}$ be a triangulated functor. Assume that $F$ commutes with small direct sums, then $F$ admits a right adjoint.
\end{corollary}

We shall prove a slightly more general statement than \cref{abelian translation Grothendieck derived Brown representability}. Let $\mathcal{I}$ be a full subcategory of $\mathcal{A}$ closed by subobjects, quotients and extensions in $\mathcal{A}$, and also by small direct sums. Let us denote by $D_{c,\mathcal{I}}(\mathcal{A})$ the full subcategory of $D_c(\mathcal{A})$ consisting of objects $X\in D_c(\mathcal{A})$ such that $H(X)\in\mathcal{I}$. Then $D_{c,\mathcal{I}}(\mathcal{A})$ is a full triangulated subcategory of $D_c(\mathcal{A})$ closed by small direct sums.

\begin{proposition}\label{abelian translation Grothendieck t-generator if}
The triangulated category $D_{c,\mathcal{I}}(\mathcal{A})$ admits a system of $t$-generators.
\end{proposition}


\subsection{Unbounded derived category}
From now on and until the end of this section, we consider abelian categories $\mathcal{A}$, $\mathcal{A}'$, etc. We shall apply the results in the preceding paragraphs to the abelian category with translation given by shifting. Then we have $\mathcal{A}_c\cong\Ch(\mathcal{A})$, $K_c(\mathcal{A})\cong K(\mathcal{A})$ and $D_c(\mathcal{A})\cong D(\mathcal{A})$. Assume that $\mathcal{A}$ admits direct sums indexed by a set $I$ and that such direct sums are exact. Then, clearly, $\mathcal{A}_c$ has the same properties, so it follows from \cref{abelian translation derive direct sum} that $\Ch(\mathcal{A})$, $K(\mathcal{A})$ and $D(\mathcal{A})$ also admit such direct sums and the two functors $\Ch(\mathcal{A})\to K(\mathcal{A})$ and $K(\mathcal{A})\to D(\mathcal{A})$ commute with such direct sums.\par
We shall write $K_{hi}(\mathcal{A})$ for $K_{c,hi}(\mathcal{A})$, so $K_{hi}(\mathcal{A})$ is the full subcategory of $K(\mathcal{A})$ consisting of homotopically injective objects. Let us denote by $\iota:K_{hi}(\mathcal{A})\to K(\mathcal{A})$ the inclusion functor. Similarly we denote by $K_{hp}(\mathcal{A})$ the full subcategory of $K(\mathcal{A})$ consisting of homotopically projective objects. Recall that $I\in K(\mathcal{A})$ is homotopically injective if and only if $\Hom_{K(\mathcal{A})}(X,I)=0$ for all $X\in K(\mathcal{A})$ that is quasi-isomorphism to $0$. Note that an object $I\in K^+(\mathcal{A})$ whose components are all injective is homotopically injective in view of \cref{abelian cat injective qis zero is null-homotopy}.\par
Let $\mathcal{A}$ be a Grothendieck abelian category. Then $\Ch(\mathcal{A})$ is also a Grothendieck category, so applying \cref{abelian translation Grothendieck enough K-injective} and \cref{abelian translation Grothendieck derived Brown representability}, we get the following theorem.
\begin{theorem}\label{abelian Grothendieck unbounded derived category prop}
Let $\mathcal{A}$ be a Grothendieck category.
\begin{enumerate}
    \item[(\rmnum{1})] if $I\in K(\mathcal{A})$ is homotopically injective, then we have an isomorphism
    \[\Hom_{K(\mathcal{A})}(X,I)\stackrel{\sim}{\to} \Hom_{D(\mathcal{A})}(X,I)\]
    for any $X\in K(\mathcal{A})$;
    \item[(\rmnum{2})] for any $X\in\Ch(\mathcal{A})$, there exists a quasi-isomorphism $X\to I$ such that $I$ is homotopically injective;
    \item[(\rmnum{3})] the localization functor $Q:K(\mathcal{A})\to D(\mathcal{A})$ induces an equivalence $K_{hi}(\mathcal{A})\stackrel{\sim}{\to} D(\mathcal{A})$;
    \item[(\rmnum{4})] the category $D(\mathcal{A})$ is a $\mathscr{U}$-category;
    \item[(\rmnum{5})] the functor $Q:K(\mathcal{A})\to D(\mathcal{A})$ admits a right adjoint $R:D(\mathcal{A})\to K(\mathcal{A})$ such that $Q\circ R\cong\id$ and $R$ is the composition of $\iota:K_{hi}(\mathcal{A})\to K(\mathcal{A})$ and a quasi-inverse of $Q\circ\iota$;
    \item[(\rmnum{6})] for any triangulated category $\mathcal{D}$, any triangulated functor $F:K(\mathcal{A})\to\mathcal{D}$ admits a right localization $RF:D(\mathcal{A})\to\mathcal{D}$ and $RF\cong F\circ R$;
    \item[(\rmnum{7})] the triangulated category $D(\mathcal{A})$ admits small direct sums and a system of $t$-generators;
    \item[(\rmnum{8})] any cohomological functor $G:D(\mathcal{A})^{\op}\to\mathbf{Mod}(\Z)$ is representable as soon as $G$ commutes with small products;
    \item[(\rmnum{9})] for any triangulated category $\mathcal{D}$, any triangulated functor $F:D(\mathcal{A})\to\mathcal{D}$ admits a right adjoint as soon as $F$ commutes with small direct sum.
\end{enumerate}
\end{theorem}

\begin{corollary}\label{abelian Grothendieck unbounded derived category RHom}
Let $k$ be a commutative ring and let $\mathcal{A}$ be a Grothendieck
k-abelian category. Then $(K(\mathcal{A})^{\op},K_{hi}(\mathcal{A}))$ is $\Hom_{\mathcal{A}}$-injective and the functor $\Hom_\mathcal{A} $admits a right derived functor $R\Hom_\mathcal{A}:D(\mathcal{A})^{\op}\times D(\mathcal{A})\to D(k)$. Moreover, we have
\[H^n(R\Hom_\mathcal{A}(X,Y))\cong \Hom_{D(\mathcal{A})}(X,Y)\]
for $X,Y\in D(\mathcal{A})$.
\end{corollary}
\begin{proof}
The functor $\Hom_\mathcal{A}$ induces a functor $\Hom_\mathcal{A}^\bullet:K(\mathcal{A})^{\op}\times K(\mathcal{A})\to K(k)$ and by (\cite{kashiwara_SAC}, proposition 11.7.3) we have
\[H^n(\Hom_\mathcal{A}^\bullet(X,Y))=\Hom_{K(\mathcal{A})}(X,Y[n]).\]
Let $I\in K_{hi}(\mathcal{A})$; if $X\in K(\mathcal{A})$ is quasi-isomorphic to zero, then $\Hom_{K(\mathcal{A})}(X,I)=0$. Moreover, if $I\in K_{hi}(\mathcal{A})$ is quasi-isomorphic to zero, then $I$ is isomorphic to zero. Therefore $(K(\mathcal{A})^{\op},K_{hi}(\mathcal{A}))$ is $\Hom_{\mathcal{A}}$-injective and we can apply \cref{derived category bifunctor derive exists if}. The last assertion follows from \cref{derived category R^0Hom is Hom in D(A)}.
\end{proof}

\begin{remark}
Let $\mathcal{I}$ be a full subcategory of a Grothendieck category $\mathcal{A}$ and assume that $\mathcal{I}$ is closed by subobjects, quotients and extensions in $\mathcal{A}$, and also by small direct sums. Then by \cref{abelian translation Grothendieck t-generator if}, the triangulated category $D_\mathcal{I}(\mathcal{A})$ admits small direct sums and a system of $t$-generators. Hence $D_\mathcal{I}(\mathcal{A})\to D(\mathcal{A})$ has a right adjoint.
\end{remark}

\subsection{Left derived functors}
We now give a criterion for the existence of the left derived functor $LG:D(\mathcal{A})\to D(\mathcal{A}')$ of an additive functor $G:\mathcal{A}\to\mathcal{A}'$ of abelian categories, assuming that $G$ admits a right adjoint. For this, we shall assume throughout this paragraph that $\mathcal{A}$ admits small direct sums and small direct sums are exact in $\mathcal{A}$. Hence, by \cref{abelian translation derive direct sum}, $\Ch(\mathcal{A})$, $K(\mathcal{A})$ and $D(\mathcal{A})$ admit small direct sums. Note that Grothendieck categories satisfy these conditions.

