\section{The snake lemma, again}
\begin{lemma}\label{pull bak lem}
Let
\[\begin{tikzcd}
A\times_{C}B\ar[d,swap,"\psi'"]\ar[r,"\varphi'"]&B\ar[d,"\psi"]\\
A\ar[r,swap,"\varphi"]&C
\end{tikzcd}\]
be a fibered diagram in an abelian category, and assume $\varphi$ is an epimorphism. Then $\varphi'$ is also an epimorphism.
\end{lemma}
\begin{proof}
First, observe that if $\varphi:A\to C$ is an epimorphism, so is the map $A\oplus B\to C$ considered in Example\ref{fibered diagram}. Since epimorphisms are cokernels in an abelian
category and cokernels are cokernels of their kernels (Lemma~\ref{ker is ker of coker}), we see that $A\oplus B\to C$ is the cokernel of the natural morphism
\[A\times_{C}B\to A\oplus B\]
(by our construction of $A\times_{C}B$) To prove that $\varphi'$ is an epimorphism, it suffices to show that if $\zeta:B\to Z$ is a morphism for which $\zeta\circ\varphi'=0$, then $\zeta=0$.\par
For this, consider the morphism
\[A\oplus B\stackrel{(0,\zeta)}{\longrightarrow}Z\]
obtained by using the fact that $A\oplus B$ is a coproduct of $A$ and $B$. The composition
\[A\times_{C}B\to A\oplus B\to Z\]
agrees with $\zeta\circ\varphi'$, so it is the zero-morphism. By the universal property of cokernels, we have the factorization
\[\begin{tikzcd}
A\times_{C}B\ar[r]&A\oplus B\ar[d]\ar[r,"{(0,\zeta)}"]&Z\\
&C\ar[ru,swap,"\zeta'"]&
\end{tikzcd}\]
for a unique morphism $\zeta'$. By the commutativity of
\[\begin{tikzcd}
A\ar[rd,swap,"\varphi"]\ar[r]&A\oplus B\ar[r,"{(0,\zeta)}"]&Z\\
&C\ar[ru,swap,"\zeta'"]
\end{tikzcd}\]
we see that the composition $\zeta'\circ\varphi:A\to C\to Z$ is the zero-morphism. Since $\varphi$ is an epimorphism, it follows that $\zeta'=0$. This implies that $(0,\zeta):A\oplus B\to Z$ is the zero-morphism, and we are done: $\zeta=0$ as promised.
\end{proof}
Here is a quicker proof: if $\varphi$ is an epimorphism, then $A\oplus B\to C$ is an epimorphism, so the diagram is a push-out as well as a pull-back (Example~\ref{fibered diagram}). By Exercise~\ref{pull back ker}, $\coker\varphi'=\coker\varphi=0$; therefore $\varphi'$ is an epimorphism.\par
\begin{lemma}\label{exact iff}
Let \begin{tikzcd}X'\ar[r,"f"]&X\ar[r,"g"]&X''\end{tikzcd} be a complex $($i.e., $g\circ f=0$$)$. Then the conditions below are equivalent:
\begin{itemize}
\item[$(\rmnum{1})$] the complex \begin{tikzcd}X'\ar[r,"f"]&X\ar[r,"g"]&X''\end{tikzcd} is exact.
\item[$(\rmnum{2})$] the induced morphism $X'\to\ker g$ is an epimorphism.
\item[$(\rmnum{3})$] for any morphism $h:S\to X$ such that $g\circ h=0$, there exist an epimorphism $f':S'\twoheadrightarrow S$ and a commutative diagram
\[\begin{tikzcd}
S'\ar[d]\ar[r,twoheadrightarrow,"f'"]&S\ar[d,"h"]\ar[rd,"0"]&\\
X'\ar[r,"f"]&X\ar[r,"g"]&X''
\end{tikzcd}\]
\end{itemize}
\end{lemma}
\begin{proof}
$(\rmnum{1})\iff(\rmnum{2})$: the exactness is saying $\ker g=\im f$. If $X'\to\ker g$ is epic, by Exercise~\ref{epi mono decop}, $\ker g=\im f$ as needed. Conversely, if $\ker g=\in f$, by Lemma~\ref{im decomp}, $X'\to\ker g$ is epic.\par
$(\rmnum{1})\Rightarrow(\rmnum{3})$: It is enough to choose $X'\times_{\ker g}S$ as $S'$. Since $X'\to\ker g$ is an epimorphism, $S'\to S$ is an epimorphism by Lemma~\ref{pull bak lem}.\par
$(\rmnum{3})\Rightarrow(\rmnum{2})$: Choose $S=\ker g$. Then the diagram becomes
\[\begin{tikzcd}
S'\ar[d]\ar[r,twoheadrightarrow,"f'"]&\ker g\ar[d]\ar[rd,"0"]&\\
X'\ar[ru,dashed]\ar[r,"f"]&X\ar[r,"g"]&X''
\end{tikzcd}\]
since $g\circ f=0$, by the universal property of $\ker g$, there is a unique morphism $X'\to\ker g$. It follows that the composition $S'\to X'\to\ker g$ is an epimorphism. Hence $X'\to\ker g$ is an epimorphism.
\end{proof}
Going back to the snake lemma, start from a commutative diagram linking two exact sequences in an abelian category:
\[\begin{tikzcd}
&A_0\ar[r,"f_0"]\ar[d,"\alpha"]&B_0\ar[r,"g_0"]\ar[d,"\beta"]&C_0\ar[r]\ar[d,"\gamma"]&0\\
0\ar[r]&A_1\ar[r,"f_1"]&B_1\ar[r,"g_1"]&C_1
\end{tikzcd}\]
The main task is to construct a connecting morphism $\delta:\ker\gamma\to\coker\alpha$. The problem is that the universal properties of kernel and cokernel would seem to guarantee the existence of morphisms to $\ker$ and from $\coker$. How on to get a morphism from $\ker$ to $\coker$?\par
The answer is to view $\ker$ as a cokernel and $\coker$ as a kernel. This can be achieved by constructing suitable pull-back, resp., push-out, diagrams:
\[\begin{tikzcd}
B_0\times_{C_0}(\ker\gamma)\ar[d]\ar[r,"g'_0"]&\ker\gamma\ar[d]\\
B_0\ar[r,"g_0"]&C_0
\end{tikzcd}\quad\begin{tikzcd}
A_1\ar[d]\ar[r,"f_1"]&B_1\ar[d]\\
\coker\alpha\ar[r,"f'_1"]&(\coker\alpha)\amalg_{A_1}B_1
\end{tikzcd}\]
By Lemma~\ref{pull bak lem}, $g'_0$ is an epimorphism since $g_0$ is an epimorphism, and $\ker g_0$, $\ker g_0'$ have matching sources by Exercise~\ref{pull back ker} (which is $A_0$ by the exactness). Analogous (dual) statements hold for the second diagram. Putting everything into one commutative diagram,
\[\begin{tikzcd}
&A_0\ar[r,"\sigma"]\ar[d,equal]&B_0\times_{C_0}(\ker\gamma)\ar[ddd,bend right,"\epsilon"]\ar[d]\ar[r]&\ker\gamma\ar[d]\ar[r]&0\\
&A_0\ar[d,swap,"\alpha"]\ar[r]&B_0\ar[d]\ar[r]&C_0\ar[d,"\gamma"]\ar[r]&0\\
0\ar[r]&A_1\ar[d]\ar[r]&B_1\ar[d]\ar[r]&C_1\ar[d,equal]&\\
0\ar[r]&\coker\alpha\ar[r]&(\coker\alpha)\amalg_{A_1}B_1\ar[r,"\tau"]&C_1&
\end{tikzcd}\]
has exact rows (not columns!). We get a morphism
\[\epsilon:B_0\times_{C_0}(\ker\gamma)\to(\coker\alpha)\amalg_{A_1}B_1\]
Note that 
\[\epsilon\circ\sigma=0,\quad \tau\circ\epsilon=0\]
by the commutativity of the diagram: indeed, the compositions
\[A_0\to\coker\alpha,\quad \ker\gamma\to C_1\]
are both zero. Since $\ker\gamma$ plays the role of cokernel in the top row, $\epsilon\circ\sigma=0$ implies that $\epsilon$ must factor through $\ker\gamma$, giving a morphism
\[\epsilon':\ker\gamma\to (\coker\alpha)\amalg_{A_1}B_1\]
Since $\coker\alpha$ plays the role of kernel in the bottom row and $\tau\circ\epsilon':\ker\gamma\to\tau$ is the zero-morphism (since $g'_0$ is epic), $\epsilon'$ must factor through $\coker\lambda$, finally yielding
\[\delta:\ker\gamma\to\coker\alpha\]
Writing elements out, the reader can verify that this morphism $\delta$ agrees with the
connecting morphism $\delta$ defined in Lemma~\ref{snake lemma}.
\begin{lemma}
With $\delta$ as defined above, we have an exact sequence for the induced morphisms
\[\begin{tikzcd}[row sep=tiny, column sep=tiny]
0\ar[r]&\ker f_0\ar[r]&\ker\alpha\ar[r]&\ker\beta\ar[r]&\ker\gamma\ar[r,"\delta"]&\coker\alpha\ar[r]&\coker\beta\ar[r]&\coker\gamma\ar[r]&\coker g_1\ar[r]&0
\end{tikzcd}\]
\end{lemma}
\begin{proof}
First, we establish the induced morphisms at two ends. Consider the commutative diagram:
\[\begin{tikzcd}
&\ker\alpha\ar[d]&\\
\ker f_0\ar[ru,dashed]\ar[r]&A_0\ar[r,"f_0"]\ar[d,"\alpha"]&B_0\ar[d,"\beta"]\\
0\ar[r]&A_1\ar[r,"f_1"]&B_1
\end{tikzcd}\]
The composition $\ker f_0\to A_0\to B_0\to B_1$ is zero, therefore so is the composition $\ker f_0\to A_0\stackrel{f_0}{\to}A_1\to B_1$. Since $A_1\to B_1$ is monic, $\ker f_0\to A_0\stackrel{\alpha}{\to}A_1$ is zero also, and by the universal property of $\ker\alpha$, there is a unique morphism $i:\ker f_0\to\ker\alpha$. Similarly, we can obtain a morphism $j:\coker\gamma\to\coker g_1$.\par
For the induced morphism between kernels and cokernels, they just come from the commutative diagram and universal property. For example, comsider the diagram
\[\begin{tikzcd}
\ker\alpha\ar[r,dashed]\ar[d]&\ker\beta\ar[d]\\
A_0\ar[d,"\alpha"]\ar[r]&B_0\ar[d,"\beta"]\\
A_1\ar[r]&B_1
\end{tikzcd}\]
The composition $\ker\alpha\to B_1$ is zero again, and use the universal property of $\ker\beta$, we get $f^*_0:\ker\alpha\to\ker\beta$.\par
Now we may write our sequence as
\[\begin{tikzcd}[row sep=tiny, column sep=tiny]
0\ar[r]&\ker f_0\ar[r,"i"]&\ker\alpha\ar[r,"f^*_0"]&\ker\beta\ar[r,"g_0^*"]&\ker\gamma\ar[r,"\delta"]&\coker\alpha\ar[r,"f_1^*"]&\coker\beta\ar[r,"g^*_1"]&\coker\gamma\ar[r,"j"]&\coker g_1\ar[r]&0
\end{tikzcd}\]
The exactness needs some work. 
\begin{itemize}
\item The exactness of $i$ is easy: since  $\ker\alpha\circ i=\ker f_0$, and both $\ker\alpha,\ker f_0$ are monomorphisms, hence so is $i$. Similarly, $j$ is epic.
\item To show the exactness at $\ker\alpha$, it suffices to prove $\ker f_0$ is the kernel of $f_0^*$. So let $\varphi:Z\to\ker\alpha$ be a morphism such that $f_0^*\circ\varphi=0$:
\[\begin{tikzcd}
&Z\ar[d,"\varphi"]\ar[ddl,dashed]&\\
&\ker\alpha\ar[r,"f_0^*"]\ar[d]&\ker\beta\ar[d]\\
\ker f_0\ar[r]&A_0\ar[r,"f_0"]&B_0
\end{tikzcd}\]
The composition $Z\to B_0$ is therefore zero, hence by the universal property of $\ker f_0$, $\varphi$ factors through $\ker f_0$. This means $\ker f_0$ is exactly the kernel of $f_0^*$. So the exactness follows. The exactness at $\coker\gamma$ can be obtained by the same strategy.
\item To prove the exactness at $\ker\beta$, assume $\varphi:Z\to\ker\beta$ such that $g_0^*\circ\varphi=0$. The composition $Z\to\ker\beta\to B_0\to C_0$ is zero, by the commutative rectangle. Since the first row is exact, by Lemma~\ref{exact iff}, there is an epimorphism $h:Z'\twoheadrightarrow Z$ and a commutative diagram:
\[\begin{tikzcd}
Z'\ar[d]\ar[r,twoheadrightarrow,"h"]&Z\ar[d]\\
A_0\ar[r]&B_0
\end{tikzcd}\]
By considering the following diagram:
\[\begin{tikzcd}
&Z'\ar[d,dashed]\ar[dd,bend right=50]\ar[r,twoheadrightarrow]&Z\ar[d,"\varphi"]&\\
&\ker\alpha\ar[d]\ar[r,"f_0^*"]&\ker\beta\ar[d]\ar[r,"g_0^*"]&\ker\gamma\\
&A_0\ar[d,"\alpha"]\ar[r,"f_0"]&B_0\ar[d,"\beta"]\\
0\ar[r]&A_1\ar[r,"f_1"]&B_1
\end{tikzcd}\]
We find the composition $Z'\to A_0\stackrel{\alpha}{\to}A_1\to B_1$ is zero, and therefore, the composition $Z'\to A_0\stackrel{\alpha}{\to}A_1$ is also zero. From the univeral property of $\ker\alpha$, we get the unique morphism $\mu:Z\to\im f_0$. The composition $g_0^*\circ f_0^*=0$ is easy to verify, so the claim follows again from Lemma~\ref{exact iff}.
\item The exactness at $\ker\gamma$: Let $\psi:S\to\ker\gamma$ be a morphism such that $\delta\circ\psi=0$. Since $B_0\times_{C_0}(\ker\gamma)$ is an epimorphism, we can find an epimorphism $S'\to S$ and a commutative diagram (agian Lemma~\ref{exact iff})
\[\begin{tikzcd}
&&S'\ar[r]\ar[d]&S\ar[d]&\\
&&B_0\times_{C_0}(\ker\gamma)\ar[d]\ar[r,"g_0"]&\ker\gamma\ar[d]\ar[r]&0\\
&A_0\ar[d,"\alpha"]\ar[r,"f_0"]&B_0\ar[d,"\beta"]\ar[r,"g_0"]&C_0\ar[d,"\gamma"]\ar[r]&0\\
0\ar[r]&A_1\ar[r,"f_1"]&B_1\ar[r,"g_1"]&C_1\\
0\ar[r]&\coker\alpha\ar[r]&(\coker\alpha)\amalg_{A_1}B_1
\end{tikzcd}\]
The composition $B_0\times_{C_0}(\ker\gamma)\to B_0\to B_1\to C_1$ is zero, and the second row is exact, so $B_0\times_{C_0}(\ker\gamma)\to B_0\to B_1$ facctors through $\ker g_1$, that is, $A_1$. So we get the morphism $B_0\times_{C_0}(\ker\gamma)\to A_1$. Since the composition $S'\to B_0\times_{C_0}(\ker\gamma)\to A_1\to\coker\alpha=0$, and the sequence $A_0\to A_1\to\coker\alpha$ is exact, by Lemma~\ref{exact iff} there is an epimorphism $S''\twoheadrightarrow S'$ such that the following diagram commutes:
\[\begin{tikzcd}
&S''\ar[dd,dashed,"k"]\ar[rdd,dashed,swap,"\lambda"]\ar[r,twoheadrightarrow]&S'\ar[r,twoheadrightarrow]\ar[d]&S\ar[d]&\\
&&B_0\times_{C_0}(\ker\gamma)\ar[ldd,dashed]\ar[d]\ar[r,"g'_0"]&\ker\gamma\ar[d]\ar[r]&0\\
&A_0\ar[d,"\alpha"]\ar[r,"f_0"]&B_0\ar[d,"\beta"]\ar[r,"g_0"]&C_0\ar[d,"\gamma"]\ar[r]&0\\
0\ar[r]&A_1\ar[d]\ar[r,"f_1"]&B_1\ar[d]\ar[r,"g_1"]&C_1\\
0\ar[r]&\coker\alpha\ar[r]&(\coker\alpha)\amalg_{A_1}B_1
\end{tikzcd}\]
Now define $\lambda$ to be the composition $S''\to S'\to B_0$. We see
\[\beta\circ f_0\circ k=\beta\circ\lambda\]
so $f_0\circ k-\lambda$ factors through $\ker\beta$. Concluding we get the diagram
\[\begin{tikzcd}
S''\ar[d]\ar[dd,bend right=50,swap,"f_0\circ k-\lambda"]\ar[r,twoheadrightarrow]&S\ar[d]\\
\ker\beta\ar[d]\ar[r]&\ker\gamma\ar[d]\\
B_0\ar[r]\ar[r,"g_0"]&C_0
\end{tikzcd}\]
To apply Lemma~\ref{exact iff}, we still need to show $\delta\circ g_0^*=0$. Since we have $\ker\beta\to B_0$ and $\ker\beta\to\ker\gamma$, by the universal property of pull back, we get a unique morphsim $\ker\beta\to B_0\times_{C_0}(\ker\gamma)$. Compostion with $g_0'$ gives a morphsim $\ker\beta\to\ker\gamma$, which should be $g_0^*$ since the latter is unique. Now consider the diagram
\[\begin{tikzcd}
&\ker\beta\ar[rr,bend left,"g_0^*"]\ar[r,dashed]&B_0\times_{C_0}(\ker\gamma)\ar[ldd,dashed]\ar[d]\ar[r,"g'_0"]&\ker\gamma\ar[d]\ar[r]&0\\
&A_0\ar[d,"\alpha"]\ar[r,"f_0"]&B_0\ar[d,"\beta"]\ar[r,"g_0"]&C_0\ar[d,"\gamma"]\ar[r]&0\\
0\ar[r]&A_1\ar[d]\ar[r,"f_1"]&B_1\ar[d]\ar[r,"g_1"]&C_1\\
0\ar[r]&\coker\alpha\ar[r]&(\coker\alpha)\amalg_{A_1}B_1
\end{tikzcd}\]
$\ker\beta\stackrel{\ker\beta}{\to}B_0\stackrel{\beta}{\to}B_1\to(\coker\alpha)\amalg_{A_1}B_1$ is zero clearly, therefore so is the composition $\ker\beta\to B_0\times_{C_0}(\ker\gamma)\to A_1\to\coker\alpha\to(\coker\alpha)\amalg_{A_1}B_1$. Since $f_1$ is monic, by the dual of Lemma~\ref{pull bak lem} $\coker\alpha\to(\coker\alpha)\amalg_{A_1}B_1$ is also monic. Then the morphism $\ker\beta\to B_0\times_{C_0}(\ker\gamma)\to A_1\to\coker\alpha$ vanishes, which means $\delta\circ g_0^*$ is zero, as needed.
\item The exactness for the right half of the sequence follows by reversing the arrows. (this is the dual argument)
\end{itemize}
\end{proof}

\subsection{Working with elements in a small abelian category}
The category $\mathsf{Set}^*$ has a zero-object $0$ (the singleton $\{0\}$); hence we can talk of zero-morphisms in $\mathsf{Set}^*$. Further, the category $\mathsf{Set}^*$ has enough structure to make sense of the notion of an exact sequence of pointed sets: we can stipulate that
\[\begin{tikzcd}
\cdots\ar[r]&S\ar[r,"f"]&T\ar[r,"g"]&U\ar[r]&\cdots
\end{tikzcd}\]
is \textit{exact} at $T$ if the (set-theoretic!) image of $f$ equals the kernel of $g$ (that is, the preimage $g^{-1}(0)$). This notion should be taken with a grain of salt: while it is the case that $S\to T\to 0$ is exact at $T$ if and only if $S\to T$ is surjective, it is not true that $0\to T\to U$ is exact at $T$ in $\mathsf{Set}^*$ if and only if $T\to U$ is injective (the only requirement posed by the exactness of $0\to T\to U$ is that the only element of $T$ mapping to the distinguished element in $U$ is the distinguished element of $T$; this is weaker than injectivity). The category $\mathsf{Set}^*$ is, after all, not additive.\\
\vspace{5mm}
Now we want to assoicate each abelian category $\mathcal{A}$ with a set category. We will (often silently) assume that our category $\mathcal{A}$ is small, to steer clear of any possible set-theoretic subtlety. It is convenient to consider a companion category $\mathcal{A}_{\twoheadleftarrow}$, with the same objects as $\mathcal{A}$ but in which
\[\Hom_{\mathcal{A}_{\twoheadleftarrow}}(Z,W):=\{\text{epimorphisms }W\twoheadrightarrow Z\text{ in }\mathcal{A}\}\]
This is legal, because the composition of two epimorphisms in $\mathcal{A}$ is an epimorphism; but note that $\mathcal{A}_{\twoheadleftarrow}$ is no longer additive (the sum of two epimorphisms need not be an epimorphism, so we lose the algebraic structure on $\Hom$). Also note the reversing of arrows; this is in order that the functor that we are about to define be covariant.\par
We will use $\mathcal{A}_{\twoheadleftarrow}$ as index category for a limit. For a fixed object $A$ of $\mathcal{A}$, we define a covariant functor
\[\mathscr{H}_A:\mathcal{A}_{\twoheadleftarrow}\to\mathsf{Set}^*\]
by restricting the functor of points $h_A$ to $\mathcal{A}_{\twoheadleftarrow}$ and preserving the information of the zero-morphism. That is, we set
\[\mathscr{H}_A(Z)=\Hom_{\mathcal{A}}(Z,A)\]
for all objects $Z$ of $\mathcal{A}_{\twoheadleftarrow}$ (that is, of $\mathcal{A}$), viewing $\Hom_{\mathcal{A}}(Z,A)$ as a pointed set by distinguishing the zero-morphism, and define $\mathscr{H}_A$ on morphisms by composition: for a morphism $Z\to W$ in $\mathcal{A}_{\twoheadleftarrow}$ (that is, an epimorphism $\alpha:W\twoheadrightarrow Z$);
\[\begin{tikzcd}
W\ar[rd,swap,"w"]\ar[rr,twoheadrightarrow,"\alpha"]&&Z\ar[ld,"z"]\\
&A&
\end{tikzcd}\]
define
\[\mathscr{H}_A(Z)\to\mathscr{H}_A(W)\]
by sending $z$ to $w=z\circ\alpha$. Of course the zero-morphism is mapped to the zeromorphism, so this is indeed a morphism in $\mathsf{Set}^*$.
\begin{definition}
Let $A$ be an object of a small abelian category $\mathcal{A}$. The pointed set $\hat{A}$ of elements of $A$ is the colimit of $\mathscr{H}_A$:
\[\hat{A}:=\rlim\mathscr{H}_A\]
\end{definition}
So an element of $A$ consists of the choice of a morphism $z:Z\to A$ in $\mathcal{A}$, stipulating that two morphisms $z_1:Z_1\to A$, $z_2:Z_2\to A$ determine the same element if
there are epimorphisms $w_1:W\twoheadrightarrow Z_1$, $w_2:W\twoheadrightarrow Z_2$ such that the diagram
\[\begin{tikzcd}
&Z_1\ar[rd,"z_1"]&\\
W\ar[ru,twoheadrightarrow,"w_1"]\ar[rd,twoheadrightarrow,"w_2"]&&A\\
&Z_2\ar[ru,"z_2"]
\end{tikzcd}\]
commutes. And this equivalent to define $z_1\simeq z_2\iff\exists w:Z_1\twoheadrightarrow Z_2$. \par
Now we want to make the assignment $A\to\hat{A}$ into a covariant functor $A\to\mathsf{Set}^*$, and there is only one reasonable way to do so: for $\varphi:A\to B$ and $z:Z\to A$, define $\hat{\varphi}(z)$ by composing morphisms, i.e.,
\[\begin{tikzcd}
A\ar[rr,"\varphi"]&&B\\
&Z\ar[lu,"z"]\ar[ru,swap,"\hat{\varphi}(z):=\varphi\circ z"]&
\end{tikzcd}\]
This prescription is (manifestly) compatible with the equivalence relation, so it defines a set-function $\hat{\varphi}:\hat{A}\to\hat{B}$. The image of the zero-morphism is zero, so $\hat{\varphi}$ is a morphism in $\mathsf{Set}^*$, as needed. The covariance property of this assignment is evident.\par
In the following subsection, the symbol $\sim$ will stand for the equivalence relation defined above; thus, two morphisms $z_1:Z_1\to A$, $z_1:Z_2\to A$ determine the same element of $\hat{A}$ if and only if $z_1\sim z_2$.\par
One useful preliminary observation is that the only morphism representing the
distinguished element is the zero-morphism and that we can detect whether a morphism is $0$ by working in $\mathsf{Set}^*$:
\begin{lemma}\label{zero iff}
$z\sim 0\iff z=0$. Further, a morphism $\varphi:A\to B$ in $\mathcal{A}$ is $0$ if and
only if $\hat{\varphi}(z)=0$ for all $z\in\hat{A}$.
\end{lemma}
\begin{proof}
According to the definition given above, $z:Z\to A$ is equivalent to $0$ if and only if there is an epimorphism $W\twoheadrightarrow Z$ making the following diagram commute:
\[\begin{tikzcd}
&Z\ar[rd,"z"]&\\
W\ar[ru,twoheadrightarrow]\ar[rd,twoheadrightarrow]&&A\\
&0\ar[ru]
\end{tikzcd}\]
Since $W\twoheadrightarrow Z$ is an epimorphism, this diagram commutes if and only if $z=0$, as claimed.\par
For the second statement, consider the situation
\[\begin{tikzcd}
A\ar[r,"\varphi"]&B\\
Z\ar[u,"z"]\ar[ru,swap,"\hat{\varphi}(z)"]
\end{tikzcd}\]
If $\hat{\varphi}=0$, then so is $\hat{\varphi}(z)=\varphi\circ z$. Conversely, if $\hat{\varphi}(z)=0$ for all $z$, then taking $z:Z=A\to A$ to be the identity, we get $\hat{\varphi}(z)=\varphi=0$.
\end{proof}
This tells us in particular that we can verify the commutativity of a diagram in $\mathcal{A}$ by working in $\mathsf{Set}^*$: because we can check whether two morphisms $\varphi,\psi\in\Hom_{\mathcal{A}}(A,B)$ are equal by verifying that their difference $\varphi-\psi$ equals $0$, which we can do at the level of pointed sets (hence, by chasing elements!) by Lemma~\ref{zero iff}.
\begin{lemma}\label{mono inje}
Let $\varphi:A\to B$ be a morphism in $\mathcal{A}$. Then
\begin{itemize}
\item $\varphi$ is a monomorphism if and only if $\hat{\varphi}$ is injective.
\item $\varphi$ is an epimorphism if and only if $\hat{\varphi}$ is surjective.
\end{itemize}
\end{lemma}
\begin{proof}
Assume $\varphi$ is an epimorphism, and let $z:Z\to B$ represent an arbitrary element
of $\hat{B}$. Consider the fiber product:
\[\begin{tikzcd}
A\ar[r,twoheadrightarrow,"\varphi"]&B\\
A\times_{B}Z\ar[u,"z'"]\ar[r,"\varphi'"]&Z\ar[u,swap,"z"]
\end{tikzcd}\]
By Lemma~\ref{pull bak lem}, $\varphi'$ is an epimorphism. It follows that $\varphi\circ z'\sim z$, that is, $\hat{\varphi}(z')=z$. This shows that $\hat{\varphi}$ is surjective.\par
Conversely, assume $\hat{\varphi}:\hat{A}\to\hat{B}$ is surjective. In particular, there is an element of $\hat{A}$ mapping to $id_B$. This element is represented by a morphism $z:Z\to A$, and the condition $\hat{\varphi}(z)=id_B$ means that there are epimorphisms $w_1,w_2$, as in the commutative diagram:
\[\begin{tikzcd}
A\ar[rr,"\varphi"]&&B\\
Z\ar[u,swap,"z"]&W\ar[l,"w_1"]\ar[r,"w_2"]&B\ar[u,equal,swap,"id_B"]
\end{tikzcd}\]
\end{proof}
Next, let’s verify that the notions of kernel and image in an abelian category $\mathcal{A}$ match precisely their simple-minded counterparts in $\mathsf{Set}^*$.
\begin{lemma}
With notation as above, let $\varphi:A\to B$ be a morphism in a small abelian category $\mathcal{A}$, and let $\hat{\varphi}:\hat{A}\to\hat{B}$ be the corresponding function of pointed sets. Let $\ker\varphi:K\to A$, resp., $\im\varphi:I\to B$, be the kernel and the image of $\varphi$, respectively. Then
\begin{itemize}
\item $\widehat{\ker\varphi}$ identifies $\hat{K}$ with the subset $\hat{\varphi}^{-1}(0)$ of $\hat{A}$.
\item $\widehat{\im\varphi}$ identifies $\hat{I}$ with the image of $\hat{\varphi}$ in $\hat{B}$.
\end{itemize}
\end{lemma}
\begin{proof}
For the image, recall that we have a decomposition of $\varphi$,
\[\varphi:A\twoheadrightarrow I\stackrel{\im\varphi}{\rightarrowtail}B\]
and that $\im\varphi:I\to B$ is a kernel of $\coker\varphi$ (this is the definition of image; the fact that $A\to I$ is an epimorphism is part of the content of Theorem~\ref{image}).\par
To simplify the notation, let $j=\im\varphi$. Then (by Lemma~\ref{mono inje}) $\hat{j}$ is an injective function $\hat{I}\to\hat{B}$, mapping a representative $z:Z\to I$ to $j\circ z:Z\to B$; we have to verify that $\hat{j}(\hat{I})=\im\hat{\varphi}$. To verify that $\hat{j}(z)$ is in the image of $\hat{\varphi}$, consider the fiber product
\[\begin{tikzcd}
A\ar[r,twoheadrightarrow]\ar[rr,bend left,"\varphi"]&I\ar[r,rightarrowtail,"j"]&B\\
A\times_{I}Z\ar[r,twoheadrightarrow]\ar[u,"z'"]&Z\ar[u,"z"]\ar[ru,swap,"\hat{j}(z)"]
\end{tikzcd}\]
The bottom map is an epimorphism by virtue of Lemma~\ref{pull bak lem}, since the top map
in the fiber square is an epimorphism. This shows that $\varphi\circ z'\sim\hat{j}(z)$; that is, $\hat{j}(z)=\hat{\varphi}(z')$ is in the image of $\hat{\varphi}$, as claimed. This verifies that the image of $\hat{j}$ is contained in the image of $\hat{\varphi}$:
\[\hat{I}\to\im\hat{\varphi}\sub\hat{B}\]
To show that every element in the image of $\hat{\varphi}$ is obtained in this fashion, let
$z:Z\to B$ be in the image of $\hat{\varphi}$. That is, there is a $z':Z'\to A$ and epimorphisms $w_1, w_2$ making the following diagram commute:
\[\begin{tikzcd}
A\ar[rr,"\varphi"]&&B\ar[r,"\coker\varphi"]&\coker\varphi\\
Z'\ar[u,"z'"]&W\ar[l,"w_1",twoheadrightarrow]\ar[r,"w_2",twoheadrightarrow]&Z\ar[u,"z"]
\end{tikzcd}\]
Note that $(\coker\varphi)\circ z\circ w_2=(\coker\varphi\circ\varphi)\circ z'\circ w_1=0$; since $w_2$ is an epimorphism,
it follows that $\coker\varphi\circ z=0$. By the universal property of kernels, this says that $z$ factors uniquely through $\ker\coker\varphi=\im\varphi=j$:
\[\begin{tikzcd}
A\ar[r,twoheadrightarrow]&I\ar[r,rightarrowtail,"j"]&B\\
&&Z\ar[u,"z"]\ar[lu,"\exists!"]
\end{tikzcd}\]
This gives an element of $\hat{I}$ mapping to $z$, as needed.
\end{proof}
\begin{proposition}\label{exact in set iff}
Let $\mathcal{A}$ be a small abelian category. Then a sequence
\[\begin{tikzcd}
A\ar[r,"\varphi"]&B\ar[r,"\psi"]&C
\end{tikzcd}\]
in $\mathcal{A}$ is exact if and only if the corresponding sequence
\[\begin{tikzcd}
\hat{A}\ar[r,"\hat{\varphi}"]&\hat{B}\ar[r,"\hat{\psi}"]&\hat{C}
\end{tikzcd}\]
is exact in $\mathsf{Set}^*$.
\end{proposition}
\begin{remark}
Proposition~\ref{exact in set iff} does not contradict our previous observation, because not all function in $\mathsf{set}^*$ is a image of a morphism in $\mathcal{A}$. Althrough the exactness of $0\to A\stackrel{\varphi}{\to}B$ in $\mathsf{Set}^*$ is not equivalent to injection of $\varphi$ in general, but if $\varphi=\hat{\psi}$ for some $\psi\in\mathcal{A}$, then this is ture.
\subsection{What is missing?}
Pretty as it is, the construction presented in the previous subsection has several shortcomings:
\begin{itemize}
\item The structure of pointed set is less rigid than, say, that of an abelian group; since $\mathsf{Ab}$ is itself an abelian category, it would be more natural to land in $\mathsf{Ab}$ than $\mathsf{Set}^*$, if possible.
\item While we can check that a diagram commutes in $\mathcal{A}$ by performing some computation in $\mathsf{Set}^*$, we have stopped short of claiming that the functor $A\to\hat{A}$ is faithful, that is, that it induces injective functions $\Hom_{\mathcal{A}}(A,B)\to\Hom_{\mathsf{Set}^*}(\hat{A},\hat{B})$ for all $A$, $B$. In fact, \textbf{the functor is not faithful}.
\item The functor $A\to\mathsf{Set}^*$ is not full; that is, it is not surjective on $\Hom$-sets. In other words, there will be many functions of pointed sets that are not induced by morphisms in $\mathcal{A}$. Thus, we cannot use elements to construct morphisms in $\mathcal{A}$: the arrow-theoretic work performed in $\S 2.2$ in order to construct the connecting morphism remains necessary (even if the properties of this morphism can then be verified by using elements).
\end{itemize}
\end{remark}
It is in fact possible to construct fully faithful, exact functors from any given (small) abelian category to a category of modules.
\begin{theorem}[\textbf{Freyd-Mitchell theorem}]
Let $\mathcal{A}$ be a small abelian category. Then there is a fully faithful, exact functor $\mathcal{A}\to R$-$\mathsf{Mod}$ for a suitable ring $R$.
\end{theorem}
Caveat: In this statement $R$ is not necessarily commutative; $R$-$\mathsf{Mod}$ denotes
the category of left-$R$-modules.\par
This is impressive: it says that doing homological algebra in the category of left-modules over a ring is essentially as general as doing it in an arbitrary (small) abelian category, at least for what concerns constructions that do not require the category to have special objects.
\section{Exercise}
\begin{exercise}
Prove the \textbf{four-lemma} in every abelian category. In fact, show that it is only necessary to prove one of the two forms of the lemma, for the other then follows automatically.
\end{exercise}
\begin{proof}
if
\[\begin{tikzcd}
A_1\ar[r]\ar[d,"\alpha"]&B_1\ar[d,"\beta"]\ar[r]&C_1\ar[d,"\gamma"]\ar[r]&D_1\ar[d,"\delta"]\\
A_0\ar[r]&B_0\ar[r]&C_0\ar[r]&D_0
\end{tikzcd}\]
is a commutative diagram of an abealian category $\mathcal{A}$ with exact rows, $\alpha$ is an epimorphism, and $\beta$, $\delta$ are monomorphisms, then we need to show $\gamma$ is an monomorphism.\par
So assume $\varphi:Z\to C_1$ is a morphism such that $\gamma\circ\varphi=0$. Then $Z\to C_1\to D_1=0$ since $\delta$ is monic. Recall that the row is exact, by Lemma~\ref{exact iff}, there is $X\twoheadrightarrow Z$ such that the relating diagram commutes. Then $X\to B_1\to B_0\to C_0$ vanished due to $\gamma\circ\varphi=0$, again from the exactness of the second row and the lemma, there is $X'\twoheadrightarrow X$ such that the diagram commutes. 
\[\begin{tikzcd}
X'\ar[dd,bend right]\ar[r,twoheadrightarrow]&X\ar[d]\ar[r,twoheadrightarrow]&Z\ar[d,"\varphi"]\\
A_1\ar[r]\ar[d,"\alpha"]&B_1\ar[d,"\beta"]\ar[r]&C_1\ar[d,"\gamma"]\ar[r]&D_1\ar[d,"\delta"]\\
A_0\ar[r]&B_0\ar[r]&C_0\ar[r]&D_0
\end{tikzcd}\]
The condition $\alpha$ is epic implies $A_1\to A_0\to 0$ is exact. Since $X'\to A_0\to 0$ is also zero, there is a morphism $X''\twoheadrightarrow X'$, with $X''\to A_1$. Now the key point is that our diagram may \textbf{not be commutative}, we should check it:
\[\begin{tikzcd}
&X'\ar[dd,bend right=35]\ar[r,twoheadrightarrow,"f"]&X\ar[d,"\varphi'"]\ar[r,twoheadrightarrow]&Z\ar[d,"\varphi"]\\
X''\ar[r,"\mu"]\ar[ru,twoheadrightarrow,"\nu"]&A_1\ar[r,"f_1"]\ar[d,"\alpha"]&B_1\ar[d,"\beta"]\ar[r]&C_1\ar[d,"\gamma"]\ar[r]&D_1\ar[d,"\delta"]\\
&A_0\ar[r,"f_0"]&B_0\ar[r]&C_0\ar[r]&D_0
\end{tikzcd}\]
Let us denote the morphism $X'\to A_0$ by $\epsilon$, since it is not convenient to draw it on the diagram. In fact the only problem occurs at whether $f_1\circ\mu$ equals with $\varphi'\circ f\circ \nu$. Note that if we compose them with $\beta$ respectively, we do get the equality:
\[\beta\circ\varphi'\circ f\circ\mu=f_0\circ\epsilon\circ mu=f_0\circ\alpha\circ\mu=\beta\circ f_1\circ\mu\]
where we use the commutative of the original diagram, and our new morphisms about $X''$. Here the monicness of $\beta$ gives us $f_1\circ\mu=\varphi'\circ f\circ \nu$. So our diagram is indeed commutative.
Now consider the composition $X'\to\twoheadrightarrow X'\twoheadrightarrow X\twoheadrightarrow Z\to C_1$: The first row gives us $A_1\to B_1\to C_1=0$, so it's also zero. With $X''\to X'\to X'\to Z$ epic, our conclusion $\varphi=0$ follows.\par
The dual argument is also proved.
\end{proof}
\begin{remark}
The step where we check the commutativity of the diagram, is indespensible. Without it we can not establish $\varphi=0$. In fact, there is a conterexample:
\[\begin{tikzcd}
\Z\ar[dd,bend right]\ar[r,twoheadrightarrow]&\Z\ar[d]\ar[r,twoheadrightarrow]&Z\ar[d,"\varphi"]\\
0\ar[r]\ar[d,"0"]&\Z\ar[d,"0"]\ar[r]&\Z\ar[d,"0"]\ar[r]&0\ar[d,"0"]\\
0\ar[r]&\Z\ar[r]&\Z\ar[r]&0
\end{tikzcd}\]
\end{remark}
\begin{exercise}
Prove the short five-lemma in any abelian category: consider a commutative diagram
\[\begin{tikzcd}
0\ar[r]&M_1\ar[d,"\alpha"]\ar[r]&N_1\ar[d,"\beta"]\ar[r]&L_1\ar[d,"\gamma"]\ar[r]&0\\
0\ar[r]&M_0\ar[r]&N_0\ar[r]&L_0\ar[r]&0\\
\end{tikzcd}\]
with exact rows in any abelian category, and assume that $\alpha$ and $\gamma$ are isomorphisms; prove that $\beta$ is an isomorphism, by an explicit arrow-theoretic chase of the diagram.
\end{exercise}
\begin{proof}
Well, the claim follows immediately form snake lemma, or five lemma. But we provide a direct arrow-theoritic argument here.\par
Assume $\varphi:Z\to N_1$ satisfies $\beta\circ\varphi=0$. Similarly, we get $X\twoheadrightarrow Z$. Note that $X\to M_1\to M_0\to N_0$ is zero, and $M_1\to M_0$, $M_0\to N_0$ are both monic, so $X\to M_0$ is zero. This gives $\varphi=0$, since $X\to Z$ is epic. By the dual argument, we can show $\beta$ is epic.
\end{proof}