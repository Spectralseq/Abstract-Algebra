\chapter{Localizations}
\section{Prime ideals and maximal ideals}
\subsection{Prime ideals}
\begin{definition}
An ideal $\p$ of a ring $A$ is called \textbf{prime} if the ring $A/\p$ is an integral domain.
\end{definition}
A maximal ideal $\m$ of $A$ is prime since $A/\m$ is a field; then it follows from Krull's theorem that every proper ideal of $A$ is contained in at least one prime ideal. In particular, for prime ideals to exist in a ring $A$, it is necessary and sufficient that $A$ be not reduced to $0$.\par
Let $\rho:A\to B$ be a ring homomorphism and $\q$ an ideal of $B$. Set $\p=\rho^{-1}(\q)$. Then the homomorphism $\bar{\rho}:A/\p\to B/\q$ derived from $\rho$ by taking quotients is injective. Suppose that $\q$ is prime; as the ring $B/\q$ is an integral domain, so is $A/\p$, being isomorphic to a subring of $B/\q$. Consequently the ideal $\p=\rho^{-1}(\q)$ is prime. In particular, let $A$ be a subring of $B$. For every ideal $\q$ of $B$, $\q\cap A$ is a prime ideal $A$.\par
If $\rho$ is surjective, then $\bar{\rho}$ is an isomorphism, so the conditions "$\p$ is prime" and "$\q$ is prime" are then equivalent. Hence, if $\p$ and $\a$ are ideals of $A$ such that $\a\sub\p$, a necessary and sufficient condition for $\p$ to be prime is that $\p/\a$ be prime in $A/\a$.
\begin{proposition}\label{prime ideal containing product of ideal prop}
Let $A$ be a ring, $\a_1,\dots,\a_n$ ideals of $A$ and $\p$ a prime ideal of $A$. If $\p$ contains the product $\prod_{i=1}^{n}\a_i$, it contains at least one of the $\a_i$.
\end{proposition}
\begin{proof}
Suppose in fact that $\p$ contains none of the $\a_i$. For $1\leq i\leq n$l there exists then an element $s_i\in\a_i-\p$. Then $s=\prod_{i=1}^{n}s_i$ is contained in $\a_1\cdots\a_n$ and is not contained in $\p$, which is absurd.
\end{proof}
\begin{corollary}\label{prime ideal containing m^n}
Let $\m$ be a maximal ideal of $A$. Then for every integer $n>0$, the only prime ideal containing $\m^n$ is $\m$.
\end{corollary}
\begin{proposition}\label{prime ideal contained in union}
Let $A$ be a ring, $\a$ a non-empty set of $A$ which is closed under addition and multiplication and $\p_1,\dots,\p_n$ ideals of $A$. Suppose that $\a$ is contained in the union of the $\p_i$ and that at most two of the $\p_i$ are not prime. Then $\a$ is contained in one of the $\p_i$.
\end{proposition}
\begin{proof}
We argue by induction on $n$. The proposition is trivial if $n=1$. Suppose that $n\geq 2$; if there exists an index $j$ such that $\a\cap\p_j\sub\bigcup_{i\neq j}\p_i$, then the set $\a$, which is the union of the $\a\cap\p_i$ where $1\leq i\leq n$, is contained in $\bigcup_{i\neq j}\p_i$, and hence in one of the $\p_i$, by the induction hypothesis. Suppose then that such an index does not exist.\par
For every $1\leq j\leq n$ let $y_j$ be an element of $\a\cap\p_j$ not belonging to any $\p_i$ for $i\neq j$. Let $k$ be chosen such that $\p_k$ is prime if $n>2$ and chosen arbitrarily if $n=2$. Let $z=y_k+\prod_{j\neq k}y_j\in\a$. If $i\neq k$ then $\prod_{j\neq k}y_j$ belongs to $\p_i$, but $y_k\notin \p_i$, whence $z\notin\p_i$. On the other hand, $\prod_{j\neq k}y_j$ does not belong to $\p_k$, as none of the factors $y_j$ belongs to it and $\p_k$ is prime if $n>2$; as $y_k\in\p_k$, $z$ does not belong to $\p_k$, and the proposition is established.
\end{proof}
\begin{proposition}\label{prime ideal contain intersection}
Let $A$ be a ring. Let $\a_1,\dots,\a_n$ be ideals and let $\p$ be a prime ideal containing $\bigcap_{i=1}^{n}\a_i$. Then $\p\sups\a_i$ for some $i$. If $\p=\bigcap_{i=1}^{n}\a_i$, then $\p=\a_i$ for some $i$.
\end{proposition}
\begin{proof}
Suppose that $\p\nsupseteq\a_i$ for all $i$. Then for each $1\leq i\leq n$ there exist $x_i\in\a_i$ such that $x_i\notin\p$, and therefore $\prod x_i\in\prod\a_i\sub\bigcap_i\a_i$; but $\prod x_i\notin\p$ (since $\p$ is prime). Hence $\p\nsupseteq\bigcap_i\a_i$, contradiction. Finally, if $\p=\bigcap_{i=1}^{n}\a_i$, then $\p\sub\a_i$ for all $i$ and hence $\p=\a_i$ for some $i$.
\end{proof}
\subsection{Nilradical and Jacobson radical}
An element $x\in A$ is \textbf{nilpotent} if $x^n=0$ for some $n>0$.
\begin{proposition}
The set $\n$ of all nilpotent elements in a ring $A$ is an ideal, and $A/\n$ has no nilpotent element except $0$. This is called the \textbf{nilradical} of $A$.
\end{proposition}
\begin{proof}
If $x\in\n$, clearly $ax\in\n$ for all $a\in A$. Let $x,y\in\n$: say $x^m=0$, $y^n=0$. By the binomial theorem (which is valid in any commutative ring), $(x+y)^{m+n-1}$ is a sum of integer multiples of products $x^ry^s$. Since $r+s=m+n-1$, we can not have $r<m, s<n$ simutanously. Hence $(x+y)^{m+n-1}=0$, $x+y\in\n$.
\end{proof}
The following proposition gives an alternative definition of the nilradical.
\begin{proposition}\label{nilradical crit}
The nilradical of $A$ is the intersection of all the prime ideals of $A$.
\end{proposition}
\begin{proof}
We divide the proof into two parts. We first prove that $\n$ is contained in every prime ideal of $A$. Suppose $a$ is an element of the nilradical of $A$, so $a^n=0$. And let $\p$ be a prime ideal of $A$. If $a$ is not contained in $\p$, then $a+\p$ is nonzero in $A/\p$. But 
\[(a+\p)^n=a^n+\p=0+\p\]
since $\p$ is prime, $A/\p$ is an integral domain, which is a contradiction.\par
Now consider the family $\mathscr{F}$ of ideals of $A$ that intersect the set $\{r^n\}_{n\in\N}$ only at $0$. Clearly $\mathscr{F}$ is nonempty and ordered by inclusion. So by Zorn's lemma there is a maximal element $\p$ in $\mathscr{F}$.\par
Now we prove that $\p$ is prime: Suppose $a,b\notin\p$ but $ab\in\p$. Then the ideals $\p+(a)$ and $\p+(b)$ properly contain $\p$. Since $\p$ is maximal, these two ideals are not in $\mathscr{F}$. So there are $m,n>0$ such that
\[r^m\in\p+(a),\quad r^n\in\p+(b)\] 
Now we find $r^{m+n}\in\p+(ab)=\p$, which is a contradiction. This shows at least one of $a,b$ belongs to $\p$, hence $\p$ is prime.\par
Since we find a prime ideal $\p$ such that $r\notin\p$, we conclude that $r$ is not in the intersection of all prime ideals. The contrapositive gives another inclusion, which finishes the proof.
\end{proof}
\begin{proposition}\label{ruduced ring}
Let $A$ be a commutative ring, and let $\n$ be its nilradical. Then $A/\n$ contains no nonzero nilpotent elements. (Such a ring is said to be \textbf{reduced}.)
\end{proposition}
\begin{definition}
Define the \textbf{Jacobson radical} $\r$ of $A$ is defined to be the intersection of all the maximal ideals of $A$.
\end{definition}
\begin{proposition}\label{Jacobson radical crit}
An element $r$ of $A$ is in the Jacobson radical $\r$ if and only if $1+rs$ is invertible for every $s\in A$.
\end{proposition}
\begin{proof}
Consider the ideal $\a:=(1+rs)$, if $1+rs$ is not invertible, this is a proper ideal, hence is contained in some maximal ideal $\m$. Since $r$ is in the Jacobson radical, it is contained in $\m$, and then we find $1\in\m$. But $\m$ is proper, this is not possible.\par
Assume $1+rs$ is invertible for every $s\in A$. Consider a maximal ideal $\m$: If $r\notin\m$, then $\m+rA$ is an ideal properly contains $\m$, then $A=\m+rA$ since $\m$ is maximal. In particular, $1=m+rs$ for some $m\in\m$ and $s\in A$. Then $m=1-rs$ is invertible by assumption, which contradicts that $\m$ is a maximal ideal.
\end{proof}
\subsection{Coprime ideals}
Two ideals $\a,\b$ are said to be \textbf{coprime} (or \textbf{comaximal}) if $\a+\b=(1)$. Clearly two ideals $\a,\b$ are coprime if and only if there exist $x\in\a$ and $y\in\b$ such that $x+y=1$.
\begin{proposition}
Let $\a$ and $\b$ be two relatively prime ideals of a ring $A$. Let $\a'$ and $\b'$ be two ideals of $A$ such that every element of $\a$ (resp. $\b$) has a power in $\a'$ (resp. $\b'$). Then $\a'$ and $\b'$ are relatively prime.
\end{proposition}
\begin{proof}
Under the given hypothesis, every prime ideal which contains $\a'$ contains $\a$ and every prime ideal which contains $\b'$ contains $\b$. If a prime ideal contains $\a'$ and $\b'$, then it contains $\a$ and $\b$, which is absurd, since $\a$ and $\b$ are relatively prime; hence $\a'$ and $\b'$ are relatively prime.
\end{proof}
\begin{proposition}
Let $\a$, $\b_1,\dots,\b_n$ be ideals of a ring $A$. If $\a$ is relatively prime to each of the $\b_i$, it is relatively prime to $\prod_{i=1}^{n}\b_i$.
\end{proposition}
\begin{proof}
Let $\p$ be a prime ideal of $A$. If $\p$ contains $\a$ and $\prod_{i=1}^{n}\b_i$ then it contains one of the $\b_i$, which is absurd since $\a$ and $\b_i$ are relatively prime.
\end{proof}
\begin{proposition}\label{Chinese remainder thm}
Let $A$ be a ring and $\a_1,\dots,\a_n$ ideals of $A$. Define a homomorphism
\[\phi:A\to\prod_{i=1}^{n}(A/\a_i),\quad\phi(x)=(x+\a_1,\dots,x+\a_n).\]
\begin{itemize}
\item[(a)] If $\a_i,\a_j$ are coprime whenever $i\neq j$, then $\prod_i\a_i=\bigcap_i\b_i$.
\item[(b)] $\phi$ is surjective if and only if $\a_i,\b_j$ are coprime whenever $i\neq j$.
\item[(c)] $\phi$ is injective if and only if $\bigcap_i\a_i=(0)$.
\end{itemize}
\end{proposition}
\begin{proof}
First we prove (a). The case $n=2$ is dealt with above. Suppose $n>2$ and the result true for $\a_1,\dots,\a_{n-1}$, and let $\b=\a_1\cap\cdots\a_{n-1}$. Since $\a_i+\a_j=(1)$ we have equations $x_i+y_i=1$ ($x_i\in\a_i, y_i\in\a_n$) and therefore 
\[\prod_{i=1}^{n-1}x_i=\prod_{i=1}^{n-1}(1-y_i)\equiv 1\mod\a_n.\]
This means $\a_n+\b=(1)$. And so
\[\prod_{i=1}^{n}\a_i=\b\a_n=\b\cap\a_n=\bigcap_{i=1}^{n}\a_i.\]

Now we turn to part (b). First assume that $\phi$ is surjective. Let us show for example that $\a_1,\a_2$ are coprime. There exists $x\in A$ such that $\phi(x)=(1,0,\dots,0)$; hence 
\[x\equiv 1\mod\a_1\And x\equiv 0\mod\a_2,\]
so that $1=(1-x)+x\in\a_1+\a_2$. The other case is done similarly.\par
Now suppose that $\a_i$ and $\a_j$ are coprime. It is enough to show, for example, that there is an element $x\in A$ such that $\phi(x)=(1,0,\dots,0)$. Since $\a_1+\b_i=(1)$ for $i>1$, we have equations
\[u_i+v_i=1,\quad u_i\in\a_1,v_i\in\a_i,i=2,\dots,n.\]
Take $x=\prod_{i=2}^{n}v_i$. then
\[x=\prod_{i=2}^{n}v_i=\prod_{i=2}^{n}(1-u_i)=1\mod\a_1.\]
Hence $\phi(x)=(1,0,\dots,0)$. The other case is done similarly.
\end{proof}
\begin{corollary}[\textbf{Chinese remainder theorem}]
Let $\a_1,\dots,\a_n$ be ideals of $A$ such that $\a_i+\b_j=(1)$ for all $i\neq j$. Then the natural homomorphism
\[\phi:A\to\prod_{i=1}^{n}(A/\a_i)\]
is surjective and induces an isomorphism $\tilde{\phi}:A/\prod_{i=1}^{n}\a_i\to\prod_{i=1}^{n}(A/\a_i)$.
\end{corollary}
\subsection{Local rings and semilocal rings}
A ring $A$ with exactly one maximal ideal $\m$ is called a \textbf{local ring}. The field $k=A/\m$ is called the residue field of $A$.
\begin{proposition}\label{local ring iff}
Let $A\neq 0$ be a ring. The following are equivalent:
\begin{itemize}
\item[(a)] $A$ has a proper ideal $\m$ and every element of $A-\m$ is a unit.
\item[(b)] $A$ is not the zero ring and for every $x\in A$ either $x$ or $1-x$ is invertible or both.
\end{itemize}
\end{proposition}
\begin{proof}
Let $\m\neq(1)$ be an ideal. If every element of $A-\m$ is a unit then $\m$ is clearly maximal and every ideal of $A$ is contained in $\m$. Thus $A$ is local. The converse is clear.\par
If $A$ is local and $x$ is not a unit. Then $x\in\m$, and $1-x$ must be a unit since otherwise $1=1-x+x\in\m$. Conversely, the condition in (b) holds, and $\m_1$ and $\m_2$ are distince maximal ideal of $A$. Then by the Chinese remained theorem we can choose $x\in A$ such that $x\equiv 0$ mod $\m_1$ and $x\equiv 1$ mod $\m_2$. Then $x$ is not invertible and neither is $1-x$ which is a contradiction.
\end{proof}
\begin{lemma}
Let $\varphi:A\to B$ be a ring map. Assume $A$ and $B$ are local rings. The following are equivalent:
\begin{itemize}
\item[(a)] $\varphi(\m_A)\sub\m_B$.
\item[(b)] $\varphi^{-1}(\m_B)=\m_A$.
\item[(c)] For any $x\in A$, if $\varphi(x)$ is invertible in $B$, then $x$ is invertible in $A$.
\end{itemize}
If $\varphi$ satisfies these conditions, we say $\varphi$ is a \textbf{local homomorphism}.
\end{lemma}
\begin{proof}
For $(a)\Rightarrow(b)$, the condition $\varphi(\m_A)\sub\m_B$ clearly implies $\m_A\sub\varphi^{-1}(\varphi(A))\sub\varphi^{-1}(\m_B)$, hence $\m_A=\varphi^{-1}(\m_B)$.\par
For $(b)\Rightarrow(c)$, assume that $\varphi^{-1}(\m_B)=\m_A$. If $\varphi(x)$ is invertible, then $\varphi(x)\notin\m_B$, so $x\notin\m_A$. This implies that $x$ is invertible. The implication $(c)\Rightarrow(a)$ is clear.
\end{proof}
Now we consider semilocal rings. They are characterized by the following proposition.
\begin{proposition}\label{semilocal ring char}
Let $A$ be a ring. The following properties are equivalent:
\begin{itemize}
\item[(a)] The set of maximal ideals of $A$ is finite.
\item[(b)] The quotient of $A$ by its Jacobson radical is the direct product of a finite number of fields.
\end{itemize}
\end{proposition}
\begin{proof}
Suppose that the quotient of $A$ by its Jacobson radical $\r(A)$ is a direct product of a finite number of fields. Then $A/\r(A)$ possesses only a finite number of ideals and a fortiori only a finite number of maximal ideals. As every maximal ideal contains $\r(A)$, the maximal ideals of $A$ are the inverse images of the maximal ideals of $A/\r(A))$ under the canonical homomorphism $A\to A/\r(A)$; hence they are finite in number.\par
Conversely, suppose that $A$ has a finite number of distinct maximal ideals $\m_1,\dots,\m_n$. The $A/\m_i$ are fields and it follows from \cref{Chinese remainder thm} that the canonical map $A\to\prod_{i=1}^{n}A/\m_i$ is surjective; as its kernel $\bigcap_{i=1}^{n}\m_i$ is the Jacobson radical $A$. Thus $A/\r(A)$ is isomorphic to $\prod_{i=1}^{n}A/\m_i$.
\end{proof}
Every local ring is semilocal. Every quotient of a semilocal ring is semilocal. Every finite product of semi-local rings is semi-local. Another example, generalizing the construction of the local rings $A_\p$, is provided by the following proposition:
\begin{proposition}\label{localization of n prime ideals prop}
Let $A$ be a ring and $\p_1,\dots,\p_n$ prime ideals of $A$. We write $S=\bigcap_{i=1}^{n}(A-\p_i)=A-\bigcup_{i=1}^{n}\p_i$.
\begin{itemize}
\item[(a)] The ring $S^{-1}A$ is semilocal; if $\q_1,\dots,\q_r$ are the distinct maximal elements (with respect to inclusion) of the set of $\p_i$, the maximal ideals of $S^{-1}A$ are the $S^{-1}\q_j$ and these ideals are distinct.
\item[(b)] The ring $A_{\p_i}$ is canonically isomorphic to $(S^{-1}A)_{S^{-1}\p_i}$.
\item[(c)] If $A$ is an integral domain, then $S^{-1}A=\bigcap_{i=1}^{n}A_{\p_i}$ in the field of fractions of $A$. 
\end{itemize}
\end{proposition}
\begin{proof}
The ideals of $A$ not meeting $S$ are the ideals contained in the union of the $\p_i$ and hence in at least one of the $\p_i$ (by \cref{prime ideal contained in union}); the $\q_i$ are therefore the maximal elements of the set of ideals not meeting $S$; consequently, the $S^{-1}\q_i$ are the maximal ideals of $S^{-1}A$ by \cref{localization and ideals}. Also, (b) follows from \cref{localization at comparable prime}.\par
Suppose that $A$ is an integral domain. If $\p_i\sub\p_j$ then $A_{\p_i}\sups A_{\p_j}$; to prove (c), we may therefore suppose that no two $\p_i$ are comparable. Then it follows from (a) and \cref{integral domain inter of localization} that $S^{-1}A=\bigcap_{i=1}^{n}(S^{-1}A)_{S^{-1}\p_i}$; whence (c) in view of (b).
\end{proof}
\begin{corollary}\label{local ring if intersection of localization}
Let $A$ be an integral domain and $\p_1,\dots,\p_n$ prime ideals of $A$, no two of which are comparable with respect to inclusion. If $A=\bigcap_{i=1}^{n}A_{\p_i}$ in the field of fractions of $A$, then the maximal ideals of $A$ are $\p_1,\dots,\p_n$.
\end{corollary}
\begin{proof}
Setting $S=\bigcap_{i=1}^{n}(A-\p_i)$. Then by \cref{localization of n prime ideals prop} $S^{-1}A=A$; hence the elements of $S$ are invertible in $A$ and $S^{-1}\p_i=\p_i$ for all $i$. Our assertion then follows by virtue of \cref{localization of n prime ideals prop}(a).
\end{proof}
\subsection{Operations on ideals}
If $\a,\b$ are ideals in a ring $A$, their \textbf{ideal quotient} is
\[(\a:\b)=\{x\in A:x\b\sub \a\}\]
which is an ideal. In particular, $(0:\b)$ is called the \textbf{annihilator} of $\b$ and is also denoted by $\Ann(\b)$: it is the set of all $x\in A$ such that $x\b=0$. In this notation the set of all zero-divisors in $A$ is
\[D=\bigcup_{x\neq 0}\Ann(x)\]
If $\b$ is a principal ideal $(x)$, we shall write $(\a:x)$ in place of $(\a:(x))$.
\begin{example}
If $A=\Z$, $\a=(m)$, $\b=(n)$, where $m=\prod_p p^{\mu_p},\quad n=\prod_pp^{\nu_p}$, then $(\a:\b)=(q)$ where $q=\prod_pp^{\gamma_p}$ and
\[\gamma_p=\max(\mu_p-\nu_p,0)=\mu_p-\min(\mu_p,\nu_p)\]
Hence $q=m/(m,n)$.
\end{example}
\begin{proposition}\label{ideal quotient prop}
For ideals $\a$, $\b$, $\c$ of $A$, we have the following properties.
\begin{itemize}
\item[(a)] $\a\sub(\a:\b)$.
\item[(b)] $(\a:\b)\b\sub \a$.
\item[(c)] $((\a:\b):\c)=(\a:\b\c)=((\a:\c):\b)$.
\item[(d)] $(\bigcap_i\a_i:\b)=\bigcap_i(\a_i:\b)$.
\item[(e)] $(\a:\sum_i\b_i)=\bigcap_i(\a:\b_i)$.
\end{itemize}
\end{proposition}
\begin{proof}
Part (a), (b) and (e) follow from the defnition. As for (c), we have
\[((\a:\b):\c)=\{x\in A:x\c\sub(\a:\b)\}=\{x\in A:x\c\b\sub\a\}=(\a:\c\b),\]
and the second equality holds since $\b\c=\c\b$.\par
Now we prove (d). If $x\b\sub\bigcap_i\a_i$, then it is contained in $(\a_i,\b)$ for all $i$, hence $(\bigcap_i\a_i:\b)\sub\bigcap_i(\a_i:\b)$. Conversely, if $x\in(\a_i,\b)$ for all $i$, then $X\b\sub \a_i$, hence in $\bigcap_i\a_i$.
\end{proof}
Now if $\a$ is any ideal of $A$, the \textbf{radical} of $\a$ is defined to be
\[\sqrt{\a}=\{x\in A:\text{$x^n\in\a$ for some $n>0$}\}.\]
If $\phi:A\to A/\a$ is the standard homomorphism, then $\sqrt{\a}=\phi^{-1}(\n(A/\a))$ and hence $\sqrt{\a}$ is an ideal. Moreover, according to \cref{nilradical crit}, we see the radical of an idea $\a$ is the intersection of the prime ideals which contain $\a$:
\[\sqrt{\a}=\bigcap_{\p\sups \a}\p.\]
\begin{example}
If $A=\Z$, $\a=(m)$, let $p_i\,(1\leq i\leq r)$ be the distinct prime divisors of $m$. Then $\sqrt{\a}=(p_1\cdots p_r)=\bigcap_i(p_i)$.
\end{example}
\begin{proposition}\label{radical ideal prop}
Let $\a, \b$ be ideals of $A$. Prove the following
\begin{itemize}
\item[(a)] $\a\sub\sqrt{\a}$.
\item[(b)] $\sqrt{\sqrt{\a}}=\sqrt{\a}$.
\item[(c)] $\sqrt{\a\b}=\sqrt{\a\cap \b}=\sqrt{\a}\cap\sqrt{\b}$.
\item[(d)] $\sqrt{\a}=(1)$ if and only if $\a=(1)$.
\item[(e)] $\sqrt{\a+\b}=\sqrt{\sqrt{\a}+\sqrt{\b}}$.
\item[(f)] If $\p$ is prime, $\sqrt{\p^n}=\p$ for all $n>0$.
\end{itemize}
\end{proposition}
\begin{proof}
Part (a), (d) are clear, and (b) holds since 
\[x^n\in\sqrt{\a}\Rightarrow (x^n)^m\in \a\Rightarrow x^{nm}\in \a.\]
For (c), since $\a\b\sub \a\cap \b$, we have $\sqrt{\a\b}\sub\sqrt{\a\cap \b}$. Conversely, if $x\in\sqrt{\a\cap \b}$, then $x^n\in \a\cap \b$ for some $n>0$. This means $x^n\in \a$ and $x^n\in \b$, hence $x^{2n}=x^n\cdot x^n\sub \a\b$, which means $x\in\sqrt{\a\b}$.\par
The argument above also gives $x\in\sqrt{\a}$ and $x\in\sqrt{\b}$, hence $\sqrt{\a\cap \b}\sub\sqrt{\a}\cap\sqrt{\b}$. The other direction is also easy to verify.\par
Now we prove (e). Since $\a+\b\sub\sqrt{\a}+\sqrt{\b}$, we have $\sqrt{\a+\b}\sub\sqrt{\sqrt{\a}+\sqrt{\b}}$. Now assume that $x^n\in\sqrt{\a}+\sqrt{\b}$, then $x^n=m+n$ for $m^a\in \a$ and $n^b\in \b$. Then
\[(x^n)^{a+b}=(m+n)^{a+b}=\sum_{i=0}^{a+b}\binom{a+b}{i}m^in^{a+b-i}\]
It is clear that the terms of the sum either belongs to $\a$ or $\b$. Hence this sum belongs to $\a+\b$. So $x\in\sqrt{\a+\b}$.\par
For (f), since $\p^n\sub\p$, we have $\sqrt{\p^n}\sub\sqrt{\p}$. But since $\p$ is prime, we have $\sqrt{\p}=\p$, and therefore $\sqrt{\p^n}\sub\p$. Conversely, if $x\in\p$, then $x^n\in\p^n$, and therefore $x\in\sqrt{\p^n}$. These together prove (f).
\end{proof}
\begin{proposition}\label{radical coprime iff ideal coprime}
Let $\a$, $\b$ be ideals in a ring $A$ such that $\sqrt{\a}$, $\sqrt{\b}$ are
coprime. Then $\a$, $\b$ are coprime.
\end{proposition}
\begin{proof}
we have
\[\sqrt{\a+\b}=\sqrt{\sqrt{\a}+\sqrt{\b}}=\sqrt{(1)}=(1)\]
hence $\a+\b=(1)$, by \cref{radical ideal prop}.
\end{proof}
Let $\rho:A\to B$ be a ring homomorphism. If $\a$ is an ideal in $A$, the set $\rho(\a)$ is not necessarily an ideal in $B$ (e.g., let $\rho$ be the embedding of $\Z$ in $\Q$, the field of rationals, and take $\a$ to be any non-zero ideal in $\Z$.). We define the \textbf{extension} $\a^e$ of $\a$ to be the ideal generated by $\rho(\a)$ in $B$: explicitly,
\[\a^e=\big\{\sum y_i\rho(x_i): x_i\in \a,y_i\in B\big\}\]
If $\b$ is an ideal of $B$, then $\rho^{-1}(\b)$ is always an ideal of $A$, called the \textbf{contraction} $\b^c$ of $\b$. If $\b$ is prime, $\b^c$ is prime. If $\a$ is prime, $\a^e$ need not to be prime: For example, if $\rho:\Z\to\Q$ is the inclusion and $\p$ is any nonzero prime ideal of $\Z$, then $\p^e=\Q$ is not prime.\par
We can factorize $\rho$ as follows:
\[\begin{tikzcd}
A\ar[r,"\tilde{\rho}"]&\rho(B)\ar[r,"j"]&B
\end{tikzcd}\]
where $\tilde{\rho}$ is surjective and $j$ is injective. For $\rho$ the situation is very simple: there is a one-to-one correspondence between ideals of $\rho(A)$ and ideals of $A$ which contain $\ker f$, and prime ideals correspond to prime ideals. For $j$, on the other hand, the general situation is very complicated. The classical example is from algebraic number theory.
\begin{example}
Consider $\Z\to\Z[i]$, where $i=\sqrt{-1}$. A prime ideal $(p)$ of $\Z$ may or may not stay prime when extended to $\Z[i]$. In fact $\Z[i]$ is a principal ideal domain (because it has a Euclidean algorithm) and the situation is as follows:
\begin{itemize}
\item[(a)] $(2)^e=((1+i)^2)$, the square of a prime ideal in $\Z[i]$.
\item[(b)] If $p\equiv 1$ mod $4$ then $(p)^e$ is the product of two distinct prime ideals.
\item[(c)] If $p\equiv 3$ mod $4$ then $(p)^e$ is prime in $\Z[i]$.
\end{itemize}
In fact the behavior of prime ideals under extensions of this sort is one of the central problems of algebraic number theory.
\end{example}
\begin{proposition}\label{ext contract prop}
Let $\rho:A\to B$, $\a$ and $\b$ be as before. Then
\begin{itemize}
\item[(a)] $\a\sub \a^{ec}$, $\b\sups \b^{ce}$.
\item[(b)] $\b^c=\b^{cec}$, $\a^e=\a^{ece}$.
\item[(c)] If $C$ is the set of contracted ideals in $A$ and if $E$ is the set of extended ideals in $B$, then
\[C=\{\a\mid \a^{ec}=\a\},\quad E=\{\b\mid \b^{ce}=\b\},\]
and $\a\mapsto \a^e$ is a bijective map of $C$ onto $E$, whose inverse is $\b\mapsto \b^c$.
\end{itemize}
\end{proposition}
\begin{proof}
The first part is clear, and (b) follows from (a): 
\[\b^c\sub (\b^{c})^{ec}=\b^{cec}\And \b\sups \b^{ce}\Rightarrow \b^c\sups \b^{cec}\]
Hence $\b^c=\b^{cec}$. Also, part (c) is a concequence of part (b).
\end{proof}
\begin{proposition}\label{ideal extension contract prop}
If $\a_1,\a_2$ are ideals of $A$ and if $\b_1,\b_2$ are ideals of $S$, then we have
\begin{itemize}
\item[(a)] $(\a_1+\a_2)^e=\a_1^e+\a_2^e$ and $(\b_1+\b_2)^c\sups \b_1^c+\b_2^c$.
\item[(b)] $(\a_1\cap \a_2)^e\sub \a_1^e\cap \a_2^e$ and $(\b_1\cap \b_2)^c=\b_1^c\cap \b_2^c$.
\item[(c)] $(\a_1\a_2)^e=\a_1^e\a_2^e$ and $(\b_1\b_2)^c\sups \b_1^c\b_2^c$.
\item[(d)] $(\a_1:\a_2)^e\sub (\a_1^e:\a_2^e)$ and $(\b_1:\b_2)^c\sub(\b_1^c:\b_2^c)$.
\item[(e)] $(\sqrt{\a})^e\sub\sqrt{\a^e}$ and $(\sqrt{\b})^c=\sqrt{\b^c}$.   
\end{itemize}
Therefore, the set of ideals $E$ is closed under sum and product, and $C$ is closed under intersection and radical.
\end{proposition}
\begin{proof}
Note that the extension and contraction preserve inclusions. Since $f$ is a ring homomorphism, part (a), (b), and (c) are easy to see. We first show part (d). If $x\in(\a_1:\a_2)^e$, then $x=\sum y_if(x_i)$ for $x_i\in(\a_1:\a_2)$. Pick one term $y_if(x_i)$, since $x_i\a_2\sub \a_1$, we have $f(x_i)f(\a_2)\sub f(\a_1)$. So we have $y_if(x_i)\in(\a_1^e:\a_2^e)$ and then $x\in(\a_1^e:\a_2^e)$. This gives $(\a_1:\a_2)^e\sub (\a_1^e:\a_2^e)$. Similarly, if $x\in (\b_1:\b_2)^c$, then $f(x)\b_2\sub \b_1$. So $f(x\b_2^c)\sub f(x)\b_2\sub \b_1$. This shows $x\b_2^c\sub \b_1^c$, hence $(\b_1:\b_2)^c\sub(\b_1^c:\b_2^c)$.\par
We turn to part (e). If $x=\sum y_if(x_i)$ for $x_i^{n_i}\in \a$. Then 
\[(y_if(x_i))^{n_i}=y_i^nf(x_i^{n_i})\in \a^e.\]
Since $x$ is a finite sum, we get $x^N\in \a^e$ for some $N$. Hence $x\in\sqrt{\a^e}$, which shows $(\sqrt{\a})^e\sub\sqrt{\a^e}$. Finally, if $f(x)\in\sqrt{\b}$, then $(f(x))^n=f(x^n)\in \b$, so $x\in\sqrt{\b^c}$. So $(\sqrt{\b})^c\sub\sqrt{\b^c}$. Conversely, if $x^n\in \b^c$, then $f(x^n)=(f(x))^n\in \b$. Hence $f(x)\in\sqrt{\b}$, $x\in(\sqrt{\b})^c$.
\end{proof}
\subsection{Exercise}
\begin{exercise}\label{nilpotent + unit}
Let $x$ be a nilpotent element of a ring $A$. Show that $1+x$ is a unit of $A$. Deduce that the sum of a nilpotent element and a unit is a unit.
\end{exercise}
\begin{proof}
Assume $x^n=1$ for $n>0$, then consider the polynomials
\[1+x^n,\quad 1-x^n\]
Either of them has a root $-1$: $1+x^n$ for $n$ odd, $1-x^n$ for $n$ even. Then we can insert $x$ to get a decomposition of $1$ since $x^n=0$. Hence $1+x$ is a unit.\par
For $u+x$ with $u$ a unit, since there is $t\in A$ such that $ur=1$, we have $r(u+x)=ur+xr=1+xr$. Since $(xr)^n=0$, $1+xr$ is a unit. Then $u+x$ is also a unit.
\end{proof}
\begin{exercise}\label{nilpotent in polynomial ring}
Let $A$ be a ring and let $A[X]$ be the ring of polynomials in an indeterminate $X$, with coefficients in $A$. Let $f=a_0+a_1X+\cdots+a_nX^n\in A[X]$. Prove that
\begin{itemize}
\item[(a)] $f$ is a unit in $A[X]$ if and only if $a_0$ is a unit in $A$ and $a_1,\dots,a_n$ are nilpotent.
\item[(b)] $f$ is nilpotent if and only if $a_0,a_1,\dots,a_n$ are nilpotent.
\item[(c)] $f$ is a zero-divisor if and only if there exists $a\neq 0$ in $A$ such that $af=0$.
\item[(d)] $f$ is said to be primitive if $(a_0,a_1,\dots,a_n)=(1)$. Prove that if $f,g\in A[X]$, then $fg$ is primitive if and only if $f$ and $g$ are primitive.
\end{itemize}
These results can be generized to a polynomial ring $A[X_1,\dots,X_r]$ in several indeterminates.
\end{exercise}
\begin{proof}
Let $f$ be a unit in $A[X]$. Then for any prime ideal $\p$ of $A$, the image $\bar{f}$ in $(A/\p)[X]$ is also a unit. Since $A/\p$ is an integral domain, this implies $\deg(\bar{f})=0$, hence $a_1,\dots,a_n\in\p$ and $a_0\notin\p$. Since this holds for any prime ideal of $A$, it follows that $a_0$ is a unit and $a_1,\dots,a_n\in\bigcap\p=\n(A)$. That is, $a_1,\dots,a_n$ are nilpotents. Conversely, if $a_0$ is a unit, then $f$ has an inverse $g$ in $A[[X]]$. If in addition $a_1,\dots,a_n$ are nilpotent, this inverse $g$ is in $A[X]$, hence $f$ is a unit in $A[X]$.\par
Similarly, if $f$ is nilpotent then it image in $(A/\p)[X]$ is nilpotent for any prime ideal $\p$, hence is zero. This implies $a_0,\dots,a_n\in \n(A)$, so they are nilpotent. The converse can be proved similarly.\par
Let $g(X)=b_0+b_1X+\cdots+b_mX^m$ be such that $fg=0$ and $g\neq 0$. If $f=0$ then the claim follows, so we may assume that $f$ is nonzero and in particular that $a_n\neq 0$. Then $a_nb_m=0$ follows immediately. Note that $f\cdot(a_ng)$ is also zero, so by replacing $g$ with $a_ng$ we elimate the degree of $g$ by $1$. This argument always holds except $\deg g=0$, so there is $a\in A,a\neq 0$ such that $af=0$. The converse is trivial.\par
Note that a polynomial is primitive just if no maximal ideal contains all its coefficients. Let $\m\sub A$ be maximal. Since $A/\m$ is a field, $A[X]/\m[X]=(A/\m)[X]$ is an integral domain. Thus
\[f,g\notin\m[X]\iff\widebar{f},\widebar{g}\neq 0\text{ in }(A/\m)[X]\iff \widebar{fg}\neq 0\text{ in }(A/\m)[X]\iff fg\notin\m[X]\]
Therefore no maximal ideal contains all the coefficients of $fg$ just if the same holds for $f$ and $g$.
\end{proof}
\begin{exercise}
In the ring $A[X]$, the Jacobson radical is equal to the nilradical.
\end{exercise}
\begin{proof}
Since maximal ideal is prime, $\n(A)\sub \r(A)$ holds for all ring $A$. Now assume $f(x)=a_0+a_1X+\cdots+a_nX^n$ is in $J(A[X])$, then $1+f(X)g(X)$ is a unit for any $g(X)\in A[X]$. In particular, $1+f(X)$ is a unit, then by \cref{nilpotent in polynomial ring}, $a_i$ are nilpotent for $i\geq 1$. Let $g(X)=1+X$, we obtain $a_0+a_1$ is nilpotent (because it is the coefficient of $X$ in $1+fg$), hence $a_0$ is nilpotent. Agian by \cref{nilpotent in polynomial ring}, $f(X)$ is nilpotent.
\end{proof}
\begin{exercise}\label{formal series prop}
Let $A$ be a ring and let $A[[X]]$ be the ring of formal power series with coefficients in $A$. Show that
\begin{itemize}
\item[(a)] $f$ is a unit in $A[[X]]$ if and only if $a_0$ is a unit in $A$.
\item[(b)] If $f$ is nilpotent, then $a_n$ is nilpotent for all $n\geq 0$. Is the converse true?
\item[(c)] $f$ belongs to the Jacobson radical of $A[[X]]$ if and only if $a_0$ belongs to the Jacobson radical of $A$.
\item[(d)] The contraction of a maximal ideal $\m$ of $A[[X]]$ is a maximal ideal of $A$, and $\m$ is generated by $\m^c$ and $X$.
\item[(e)] Every prime ideal of $A$ is the contraction of a prime ideal of $A[[X]]$.
\end{itemize}
\end{exercise}
\begin{proof}
If $f$ is a unit, then $a_0$ is a unit. If $a_0$ is a unit, consider the expansion
\[\frac{1}{1+a_0^{-1}X}\sum_{n=0}^{\infty}(-a_0^{-1}X)^n\]
Plug into $X=f-a_0$, we get an inverse of $1+a_0^{-1}(f-a_0)=a_0^{-1}f$, hence $a_0^{-1}f$ is a unit, so is $f$.\par
If $f$ is nilpotent, then $a_0$ is nilpotent, and $f-a_0$ is nilpotent. By repeting this we can show $a_n$ are all nilpotent.\par 
To see that the converse fails, let $Y_n$ be a collection of indeterminates over $\Z$, indexed by the natural numbers. Let
\[A=\Z[Y_1,Y_2,\cdots]/(Y_1,Y_2^2,Y_3^3,\cdots)\]
Let 
\[f(X)=\sum_nY_nX^n\in A[[X]]\]
Then $a_n=Y_n$ is nilpotent for all $n$, but there is no $N$ such that $f^N=0$ (this would establish some non-trivial relation amongst the $Y_n$ for $n>N$ other than $Y_n^n=0$).\par
$f\in J(A[[X]])$ if and only if $1+fg$ is a unit for all $g\in A[[X]]$. Since for $g=\sum b_nX^n$, the contant term of $1+fg$ is $1+a_0b_0$, the condition is equivalent to that $1+a_0b_0$ is a unit for all $b_0$, and hence to $a_0\in \r(A)$.\par
Since $x$ has constant term $0\in \r(A)$ in $A[[X]]$, by (c) above we get $X\in J(A[[X]])$, and hence $(X)\sub J(A[[X]])$. As $\m-(X)=\m^c$, we get $\m=\m^c+(X)$. Now $A/\m^c\cong A[[X]]/\m$ is a field, so $\m^c$ is maximal.\par
To see $I=I^{ec}$, we only need to verify $I^e$ is prime when $I$ is prime. Let $f,g\in A[[X]]$ with contant term $a_0,b_0$. If $f,g\notin\p^e$, then $a_0,b_0\notin\p$. Since $fg$ has contant term $a_0b_0$, which is not in $\p$, $fg\notin\p^e$. So $\p^e$ is prime.
\end{proof}
\begin{exercise}
Let $A$ be a ring, $\n(A)$ its nilradical. Show that the following are equivalent:
\begin{itemize}
\item[(a)] $A$ has exactly one prime ideal.
\item[(b)] Every element of $A$ is either a unit or nilpotent.
\item[(c)] $A/\n(A)$ is a field.
\end{itemize}
\end{exercise}
\begin{proof}
Assume $A$ has exactly one prime ideal $\p$. Suppose $x\in A$ is not a unit, then $x\in\m$ for some maximal ideal $\m$. Since $\m$ is also prime, we must have $\m=\p$. Hence $x\in\p$. Since we also have $\n(A)=\p$, $x$ is nilpotent.\par 
Since $\n(A)$ contains all nilpotent elements, $A-\n(A)$ contains only units. Then every elements in $A/\n(A)$ is a unit, so $A/\n(A)$ is a field.\par 
If $A/\n(A)$ is a field, then $A-\n(A)$ contains only units. Let $\p$ be a prime ideal in $A$, then $\p\in\\n(A)$ since it contains no units, and then $\n(A)=\p$ since $\n(A)\sub\p$. So $A$ contains only one prime ideal $\n(A)$. 
\end{proof}
\begin{proof}
Let $x\in A$ be idempotent, and let $\m$ be the unique maximal ideal. Then in $A/\m$ we see
\[\widebar{x}(1-\widebar{x})=0\]
so $\widebar{x}=0$ or $\widebar{x}=1$, which means $x\in\m$ or $x\in 1+\m$. Since $\m=\r(A)$, if $x\in\m$ then $1-x$ is a unit, and if $x\in 1+\m$ then $x$ is a unit. Since $x(1-x)=0$, in the former case $x=0$, and in the latter case $x=1$.
\end{proof}
\begin{exercise}
In a ring $A$, let $\Sigma$ be the set of all ideals in which every element is a zero-divisor. Show that the set $\Sigma$ has maximal elements and that every maximal element of $\Sigma$ is a prime ideal. Hence the set of zero-divisors in $A$ is a union of prime ideals.
\end{exercise}
\begin{proof}
Clearly $(0)\in\Sigma$, so $\Sigma$ is non-empty. Let $(I_i)_{i\in I}$ be a chain in $\Sigma$. The union a of the chain is still an ideal, and consists only of zero-divisors. Thus $\Sigma$, so by Zorn's Lemma $\Sigma$ has maximal elements.\par
Let $I$ be a maximal element of $\Sigma$, and suppose that $xy\in I$ with $y\notin I$. Then $I\subsetneq(I,a)$ so $y$ is not a zero-divisor by the maximality of $I$. But $xy$ is a zero-divisor, so there exists some $z\in A, z\neq 0$ such that $zxy=0$. As $y$ is not a zero-divisor we must have that $xz=0$, so that $x$ is a zero-divisor and thus is contained in $I$.
\end{proof}
\section{Rings and modules of fractions}
\subsection{Rings of fractions}
\begin{definition}
A subset $S$ of a commutative ring $A$ is a \textbf{multiplicative subset} (or \textbf{multiplicatively closed}) if 
\begin{itemize}
\item $1\in S$.
\item if $s,t\in S$ then $st\in S$.
\end{itemize}
In other words, $S$ is a subsemigroup of the multiplicative semigroup of $A$.
\end{definition}
\begin{example}[\textbf{Example of multiplicative subsets}]
\mbox{}
\begin{itemize}
\item[(a)] For every $a\in A$, the set of $a^n$, where $n\in\N$, is a multiplicative subset of $A$.
\item[(b)] Let $A$ be a ring and $\a$ an ideal of $A$, then $1+\a$ is multiplicative closed.
\item[(c)] Let $\p$ be an ideal of $A$. For $A-\p$ to be a multiplicative subset of $A$, it is necessary and sufficient that $\p$ be prime.
\item[(d)] The set of elements of $A$ which are not divisors of zero is a multiplicative subset of $A$.
\item[(e)] If $S$ and $T$ are multiplicative subsets of $A$, the set $ST$ of products $st$, where $s\in S$ and $t\in T$, is a multiplicative subset.
\item[(f)] Let $\mathcal{S}$ be a directed set (with respect to inclusion) of multiplicative subsets of $A$. Then $T=\bigcup_{S\in\mathcal{S}}S$ is a multiplicative subset of $A$, as any two elements of $T$ belong to some subset $S\in\mathcal{S}$, hence their product belongs to $T$.  
\end{itemize}
\end{example}
\begin{proposition}\label{miltiplicative subset avioding prime}
Let $A$ be a ring, $S$ a multiplicative subset, and $\mathfrak{a}$ an ideal with $\mathfrak{a}\cap S=\emp$. Set $\mathscr{S}:=\{\text{ideals }\mathfrak{b}\mid \mathfrak{b}\sups \mathfrak{a}\text{ and }\mathfrak{b}\cap S=\emp\}$. Then $\mathscr{S}$ has a maximal element $\p$, and every such $\p$ is prime.
\end{proposition}
\begin{proof}
By Zorn's lemma, similar to \cref{nilradical crit}.
\end{proof}
For every subset $S$ of a ring $A$, there exist multiplicative subsets of $A$ containing $S$, for example $A$ itself. The intersection of all these subsets is the smallest multiplicative subset of $A$ containing $S$; it is said to be \textbf{generated} by $S$. It follows immediately that it is the set consisting of all the finite products of elements of $S$.
\begin{proposition}\label{localization def}
Let $A$ be a ring and $S$ a subset of $A$. There exists a ring $\tilde{A}$ and a homomorphism $i:A\to \tilde{A}$ with the following properties:
\begin{itemize}
\item[(a)] the elements of $i(S)$ are invertible in $\tilde{A}$;
\item[(b)] for every homomorphism $f$ of $A$ to a ring $B$ such that the elements of $f(S)$ are invertible in $B$, there exists a unique homomorphism $\tilde{f}:\tilde{A}\to B$ such that $f=\tilde{f}\circ i$.
\end{itemize}
\end{proposition}
\begin{proof}
Define a relation on the set of pairs $A\times S$ as follows:
\begin{align}\label{localization def-1}
(a,s)\sim(a',s')\iff (\exists t\in S)\ t(s'a-sa')=0
\end{align}
Now we denote by $a/s$ the equivalence class of $(a,s)$, and define the operations $+,\cdot$ on such fractions as 
\[\dfrac{a}{s}+\dfrac{a'}{s'}=\dfrac{as'+a's}{ss'},\quad \dfrac{a}{s}\cdot\dfrac{a'}{s'}=\dfrac{aa'}{ss'}.\]
The set $S^{-1}A$ of fractions, endowed with the operations $+,\cdot$, is a commutative ring and the function $a\mapsto a/1$ defines a ring homomorphism $i_A^S:A\to S^{-1}A$. It is now clear that $i(s)$ is invertible for any $s\in S$. If $\rho:A\to B$ is a homomorphism such that $f(S)$ are invertible, then the map $\tilde{\rho}:S^{-1}A\to B$ defined by $\tilde{f}(a/s)=f(a)f(s)^{-1}$ is well-defined and clearly satisfies the requirement.
\end{proof}
Let $A$ be a ring, $S$ a subset of $A$ and $\widebar{S}$ the multiplicative subset generated by $S$. The ring of fractions of $A$ defined by $\widebar{S}$ and denoted by $S^{-1}A$ is the quotient set of $A\times\widebar{S}$ under the equivalence relation $(\ref{localization def-1})$ with the ring structure dfined by
\[a/s+b/t=(ta+bs)/st,\quad (a/s)(b/t)=ab/st.\]
for $a,b\in A$ and $s,t\in\widebar{S}$. The canonical map $i_A^S$ of $A$ to $S^{-1}A$ is the homomorphism $a\mapsto a/1$, which makes $S^{-1}A$ into an $A$-algebra.
\begin{corollary}
If $\rho:A\to B$ is a ring homomorphism such that
\begin{itemize}
\item[(a)] $\rho(S)$ is invertible in $B$.
\item[(b)] $\rho(a)=0$ if and only if $as=0$ for some $s\in S$.
\item[(c)] Every element of $B$ is of the form $\rho(a)\rho(s)^{-1}$. 
\end{itemize}
Then there is a unique isomorphism $h:S^{-1}A\to B$ such that $\rho=h\circ i$.
\end{corollary}
\begin{proof}
We only have to show that $h:S^{-1}A\to B$, defined by
\[h(a/s)=\rho(a)\rho(s)^{-1}\]
is an isomorphism. By (c), his surjective. To show $h$ is injective, look at the kernel of $h$: if $h(a/s)=0$, then $\rho(a)=0$, hence by (b) we have $as=0$ for some $s\in S$, hence $(a,s)=(0,1)$, i.e., $a/s=0$ in $S^{-1}A$.
\end{proof}
\begin{example}[\textbf{Example of localizations}]
\mbox{}
\begin{itemize}
\item[(a)] Let $A$ be a commutative ring, and let $\p$ be a prime ideal of $A$. Then the set $S=A-\p$ is multiplicatively closed. The localizations $S^{-1}A$, $S^{-1}M$ are then denoted $A_{\p}$, $M_{\p}$. The process of passing from $A$ to $A_\p$ is called \textbf{localization at $\p$}.
\item[(b)] Let $f\in A$ and let $S=\{f^n\}$, we write $A_f$ for $S^{-1}A$ in this case.
\item[(c)] Let $A=\Z$, $\p=(p)$ where $p$ a prime number. Then $\Z_\p$ is set of all rational numbers $m/n$ where $n$ is prime to $p$; if $f\in\Z$ and $f\neq 0$, then $\Z_f$ is the set of all rational numbers whose denominator is a power of $f$.
\item[(d)] Let $A=k[X_1,\dots,X_n]$, where $k$ is a field and the $X_i$ are independent indeterminates, $\p$ a prime ideal in $A$. Then $A_\p$ is the ring of all rational functions $f/g$, where $g\notin\p$. If $V$ is the variety defined by the ideal $\p$, that is to say the set of all $x=(x_1,\dots,x_n)\in k^n$ such that $f(x)=0$ whenever $f\in\p$, then (provided $k$ is infinite) $A_\p$ can be identified with the ring of all rational functions on $k^n$ which are defined at almost all points of $V$; it is the local ring of $k^n$ \textbf{along the variety} $V$. This is the prototype of the local rings which arise in algebraic geometry.
\end{itemize}
\end{example}
\begin{proposition}\label{localization induced map}
Let $A$, $B$ be two rings, $S$ a subset of $A$, $T$ a subset of $B$ and $\rho:A\to B$ a homomorphism such that $\rho(S)\sub T$. There exists a unique homomorphism $\tilde{\rho}$ from $S^{-1}$ to $T^{-1}B$ such that $\tilde{\rho}(a/1)=\rho(a)/1$ for all $a\in A$.\par
Suppose further that $T$ is contained in the multiplicative subset of $B$ generated by $\rho(S)$. Then, if $\rho$ is surjective (resp. injective) so is $\tilde{\rho}$.
\end{proposition}
\begin{proof}
The first assertion amounts to saying that there exists a unique homomorphism $\tilde{\rho}:S^{-1}A\to T^{-1}B$ giving a commutative diagram:
\[\begin{tikzcd}
A\ar[r,"\rho"]\ar[d,swap,"i_A^S"]&B\ar[d,"i_B^T"]\\
S^{-1}A\ar[r,"\tilde{\rho}"]&T^{-1}B
\end{tikzcd}\]
Now the relation $\rho(S)\sub T$ implies that $i_B^T(\rho(S))$ is invertible in $T^{-1}B$ for all $s\in S$ and it is sufficient to apply \cref{localization def} to $i_B^T\circ\rho$. It follows easily from the definition that, for all $a\in A$ and $s\in\widebar{S}$ (multiplicative subset of $A$ generated by $S$),
\[\tilde{\rho}(a/s)=\rho(a)/\rho(s)\]
Suppose that $T$ is contained in the multiplicative subset generated by $\rho(S)$, which is precisely $\rho(\widebar{S})$. Then it follows from the formula above that, if $\rho$ is surjective, so is $\tilde{\rho}$. Suppose now that $\rho$ is injective. Let $a/s$ be an element of the kernel of $\tilde{\rho}$. As the multiplicative subset generated by $T$ is $\rho(\widebar{S})$, there is an element $s_1\in S$ such that $\rho(s_1)\rho(a)=0$, whence $\rho(s_1a)=0$ and consequently $s_1a=0$ since $\rho$ is injective; then $a/s=0$, which proves that $\tilde{\rho}$ is injective.
\end{proof}
\begin{corollary}\label{localization into invertible ring injective}
Let $A$ be a ring, $S$ a subset of $A$ and $\rho$ an injective homomorphism from $A$ to a ring $B$ such that the elements of $\rho(S)$ are invertible in $B$. The unique homomorphism $\tilde{\rho}:S^{-1}A\to B$ such that $\rho=\tilde{\rho}\circ i_A^S$ is then injective.
\end{corollary}
\begin{corollary}\label{localization inclusion of multiplicative sets}
Let $A$ be a ring and $S,T$ two subsets of $A$ such that $S\sub T$. Then there exists a unique homomorphism $i_A^{T,S}:S^{-1}A\to T^{-1}A$ such that $i_A^T=i_A^{T,S}\circ i_A^S$.
\end{corollary}
\begin{proof}
For all $a\in A$, $i_A^{T,S}$ then maps the element $a/s$ in $S^{-1}A$ to the element $a/s$ in $T^{-1}A$.
\end{proof}
\begin{corollary}
Let $A$, $B$, $C$ be three rings, $S$ (resp. $T$, $U$) a multiplicative subset of $A$ (resp. $B$, $C$), $\rho:A\to B$, $\nu:B\to C$ two homomorphism and $\eta=\nu\circ\rho$ the composite homomorphism. Suppose that $\rho(S)\sub T$ and $\nu(T)\sub U$. Let $\tilde{\rho}:S^{-1}A\to T^{-1}B$, $\tilde{\nu}:T^{-1}B\to U^{-1}C$, and $\tilde{\eta}:S^{-1}A\to U^{-1}C$ be the induced homomorphisms. Then $\tilde{\eta}=\tilde{\nu}\circ \tilde{\rho}$.
\end{corollary}
In particular, if $S$, $T$, $U$ are three multiplicative subsets of $A$ such that $S\sub T\sub U$, then $i_A^{U,S}=i_A^{U,T}\circ i_A^{T,S}$.
\begin{corollary}
Let $S$ be a subset of a ring $A$, $B$ a subring of $S^{-1}A$ containing the set $T:=i_A^S(A)$. Let $j:B\to S^{-1}A$ be the canonical injection, then the unique homomorphism $\eta:T^{-1}B\to S^{-1}A$ such that $\eta\circ i_A^S=j$ is an isomorphism.
\end{corollary}
\begin{proof}
The map $\eta$ is injective by \cref{localization into invertible ring injective}. The ring $\eta(T^{-1}B)$ contains $T$ and the inverse of the elements of $T$. Hence it is equal to $S^{-1}A$.
\end{proof}
\subsection{Modules of fractions}
The canonical homomorphism $i_A^S:A\to S^{-1}A$ allows us to consider every $S^{-1}A$-module as an $A$-module.
\begin{proposition}\label{localization module def}
Let $A$ be a ring, $S$ a subset of $A$, $M$ an A-module and $\tilde{M}$ the $A$-module $M\otimes_AS^{-1}A$. If $i_M^S$ is the canonical $A$-homomorphism $x\mapsto x\otimes 1$ of $M$ to $\tilde{M}$, then:
\begin{itemize}
\item[(a)] For all $s\in S$, the homothety $z\mapsto sz$ of $M'$ is bijective.
\item[(b)] For every $A$-module $N$ such that, for all $s\in S$, the homothety $z\mapsto sz$ of $N$ is bijective, and every homomorphism $\phi:M\to N$, there exists a unique homomorphism $\tilde{\phi}:\tilde{M}\to N$ such that $\phi=\tilde{\phi}\circ i_M^S$.
\end{itemize}
\end{proposition}
\begin{proof}
For every $A$-module $N$ and all $a\in A$, denote by $h_a$ the homothety $y\mapsto ay$ in $N$. Then $a\mapsto h_a$ is a ring homomorphism from $A$ to $\End_A(N)$ To say that $h_a$ is bijective means that $h_a$ is an invertible element of $\End_A(N)$. Suppose that, for all $s\in S$, $h_s$ is invertible in $\End_A(N)$. The elements $h_a$ where $a\in A$ and the inverses of the elements $h_s$ where $s\in S$ then generate in $\End_A(N)$ a commutative subring $B$ and the homomorphism $a\mapsto h_a$ from $A$ to $B$ is such that the images of the elements of $S$ are invertible. Then it follows that therc exists a unique homomorphism $\tilde{h}$ of $S^{-1}A$ to $B$ such that
\[\tilde{h}(a/s)=h_ah_s^{-1}.\]
We know that such a homomorphism defines on $N$ an $S^{-1}A$-module structure such that $(a/s)\cdot y=h_s^{-1}(a\cdot y)$. The $A$-module structure derived from this $S^{-1}A$-module structure by means of the homomorphism $i_A^S$ is precisely the structure given initially.\par
Conversely, if $N$ is an $S^{-1}A$-module and it is considered as an $A$-module by means of $i_A^S$, the homotheties $y\mapsto sy$, for $s\in S$, are bijective, for $y\mapsto(1/s)y$ is its inverse; and the $S^{-1}A$-module structure on $N$ derived from its $A$-module structure by the process described above is the $S^{-1}A$-module structure given initially. Thus there is a canonical one-to-one correspondence between $S^{-1}A$-modules and $A$-modules in which the homotheties induced by the elements of $S$ are bijective. Moreover, if $N$, $N'$ are two $A$-modules with this property every $A$-module homomorphism $\phi:N\to N'$ is also a homomorphism of the $S^{-1}A$-module structures of $N$ and $N'$, as, for all $y\in N$ and all $s\in S$, we may write $\phi(y)=\phi(s\cdot((1/s)y))=s\cdot\phi((1/s)y)$, whence $\phi((1/s)y)=(1/s)\phi(y)$; the converse is obvious.\par
This being so, the statement is just the characterization of the module obtained from $M$ by extending the scalars to $S^{-1}A$, taking account of the above interpretation.
\end{proof}
Let $A$ be a ring, $S$ a subset of $A$, $\widebar{S}$ the multiplicative subset of $A$ generated by $S$ and $M$ an $A$-module. Then the $S^{-1}A$-module $M\otimes_AS^{-1}A$ is called the \textbf{module of fractions of $M$ defined by $\bm{S}$} and denoted by $S^{-1}M$.  
\begin{proposition}
Let $A$ be a ring. Let $S\sub A$ be a multiplicative subset. Let $M$ be an $A$-module. Then
\[S^{-1}M=\rlim_{s\in S}M_s\]
where the partial ordering on $S$ is given by $s\geq s'$ if and only if $s=as'$ for some $a\in A$ in which case the map $M_{s'}\to M_s$ is given by $m/s^e=ma^e/(s')^e$.
\end{proposition}
\begin{proposition}
Let $S$ be a multiplicative subset of $A$ and $M$ an $A$-module. For $m/s=0$ where $m\in M$ and $s\in S$, it is necessary and saflicient that there exist $s'\in S$ such that $s'm=0$.
\end{proposition}
\begin{proof}
If $s'\in S$ is such that $s'm=0$, clearly $m/s=(s'm)/(ss')=0$. Conversely, suppose that $m/s=0$. As $1/s$ is invertible in $S^{-1}A$, $m/1=0$. For every sub-$A$-module $P$ of $S^{-1}A$ containing $1$, we denote by $\beta(P,m)$ the image of $(m,1)$ under the canonical map of $M\times P$ to $M\otimes_A P$. Then $(MS^{-1}A,m)=0$, so there exists a finitely generated submodule $P$ of $S^{-1}A$ containing $1$ and such that $\beta(P,m)=0$. For all $t\in S$ we denote by $A_t$ the set of $a/t$, where $a\in A$. As $P$ is finitely generated, there exists $t\in S$ such that $P\sub A_t$, whence $\beta(A_t,m)=0$. The map $a\mapsto a/t$ from $A$ to $A_t$ is surjective; let $B$ be its kernel. Then we have a surjective map $h:M\otimes_AA\to M\otimes_AA_t$ whose kernel is $B\otimes_AM$. By definition $\beta(A_t,m)=h(tm)$ and consequently $tm$ can be expressed in the form $\sum_ib_im_i$, where $b_i\in B$ and $m_i\in M$. Since $b_i/t=0$ for all $i$, there exists $t'\in S$ such that $t'tm=0$, whence the claim.
\end{proof}
\begin{corollary}\label{localization module isomorphism}
The module $S^{-1}M$ can be identified with the set $M\times S$ modulo the equivalence relation
\[(m,s)\sim(m',s')\iff \text{$\exists t\in S$ such that $t(ms'-m's)=0$}\]
and the $S^{-1}A$-module structure defined by $(a/s)(m,t)=(am,st)$.
\end{corollary}
\begin{corollary}\label{localization finite module is zero iff}
Let $M$ be a finitely generated $A$-module. For $S^{-1}M=0$, it is necessary and suficient that there exists $s\in S$ such that $sM=0$.
\end{corollary}
\begin{proof}
Without any conditions on $M$, clearly the relation $sM=0$ for some $s\in S$ implies $S^{-1}M=0$. Conversely, suppose that $S^{-1}M=0$ and let $x_1,\dots,x_n$ be a system of generators of $M$. The $m_i/1$ generate the $S^{-1}A$-module $S^{-1}M$, hence to say that $S^{-1}M=0$ amounts to saying that $m_i/1=0$ for all $i$. By \cref{localization module isomorphism} there exist $s_i\in S$ such that $s_im_i=0$ and, taking $s=s_1\cdots s_n$, we see $sm_i=0$ for all $i$ and hence $sM=0$.
\end{proof}
\begin{corollary}\label{localization and aM=M iff}
Let $M$ be a finitely generated $A$-module. For an ideal $\a$ of $A$ to be such that $\a M=M$, it is necessary and sufiicient that there exist $a\in\a$ such that $(1+a)M=0$.
\end{corollary}
\begin{proof}
Clearly the relation $(1+a)M=0$ implies $M=aM$. To prove the converse, we use the following result: For every ideal $\a$ of $A$, the set $S=1+\a$ is a multiplieative subset of $A$ and the set $S^{-1}\a$ is an ideal contained in the Jacobson radical of $S^{-1}A$. The first assertion is obvious, as well as the fact that $S^{-1}\a$ is an ideal of $S^{-1}A$. On the other hand, $(1/1)+(a/s)=(s+a)/s$ and $s+\a\in S$ for all $s\in S$ and $a\in\a$ by definition of $S$. Hence $(1/1)+(a/s)$ is invertible in $S^{-1}A$ for all $a/s\in S^{-1}\a$, which completes the proof of this result. This being done, if we set $N=S^{-1}M$, clearly $N$ is a finitely generated $S^{-1}A$-module. If $\a M=M$, then $S^{-1}\a N=N$ and it follows that $N=0$ by Nakayama's Lemma. The corollary then follows from \cref{localization finite module is zero iff}.
\end{proof}
\begin{remark}\label{localization M_f isomorphic to inductive limit}
Let $M$ be an $A$-module, $f$ be an element of $A$. Consider a sequence $(M_n)$ of $A$-modules, where $M_n=M$ and for any integers $m\leq n$, let $\varphi_{nm}:M_m\to M_n$ be the multiplication by $f^{n-m}$. Then it is immediate that $(M_n,\varphi_{nm})$ form an inductive system of $A$-modules, so let $N=\rlim M_n$ and $\varphi_n:N\to M_n$ be the canonical homomorphisms. We now define a functorial isomorphism from $N$ to $M_f$: for each $n$, let $\theta_n:z\mapsto z/f^n$ be an $A$-homomorphism from $M_n=M$ to $M_f$, and since we clearly have $\theta_n\circ\varphi_{nm}=\theta_m$, let $\theta:N\to M_f$ be the induced homomorphism. Since any element of $M_f$ is of the form $z/f^n$ for some $n$, so $\theta$ is surjective. On the other hand, if $\theta(\varphi_n(z))=0$, which means $z/f^n=0$, then there exists an integer $k>0$ such that $f^kz=0$, so $\varphi_{n+k,n}(z)=0$, which means $\varphi_n(z)=0$. We can then identify $M_f$ with $\rlim M_n$ by the isomorphism $\theta$.
\end{remark}
\begin{proposition}\label{localization module induced map}
Let $A$, $B$ be two rings, $S$ a multiplicative subset of $A$, $T$ a multiplicative subset of $B$ and $f$ a homomorphism from $A$ to $B$ such that $f(S)\sub T$. Let $M$ be an $A$-module, $N$ a $B$-module and $\phi:M\to N$ an $A$-linear homomorphism. Then there exists a unique $S^{-1}A$-linear map $\tilde{\phi}:S^{-1}M\to T^{-1}N$ such that $\tilde{\phi}(m/1)=\phi(m)/1$ for all $m\in M$.
\end{proposition}
\begin{proof}
Thc map $i_M^T\circ\phi$ from $M$ to $T^{-1}N$ is $A$-linear. Moreover, if $s\in S$, then $f(s)\in T$, hence the homothety induced by $s$ on $T^{-1}N$ is bijective. The existence and uniqueness of $\tilde{\phi}$ then follow from \cref{localization module def}. Then, for $m\in M$ and $s\in S$,
\[\tilde{\phi}(m/s)=\phi(m)/f(s)\]
With the same notation, let $C$ be a third ring, $U$ a multiplicative subset of $C$, $g$ a homomorphism from $B$ to $C$ such that $g(T)\sub U$, $P$ a $C$-module, $\psi$ a $B$-linear map from $N$ to $P$ and $\tilde{\phi}$ the $T^{-1}B$-linear map from $B$ to $U^{-1}P$ associated with $\psi$. Then
\[\widetilde{(\psi\circ\phi)}=\tilde{\psi}\circ\tilde{\phi}\]
where the left hand side is the $A$-linear map $S^{-1}M\to U^{-1}P$ associated with $\psi\circ\phi$. Similarly, if $\phi'$ is a second $A$-linear map from $M$ to $N$, then
\[\widetilde{(\phi+\phi')}=\tilde{\phi}'+\tilde{\phi}\]
the left-hand side being the $A$-linear map $S^{-1}M\to T^{-1}N$ associated with $\phi+\phi'$.
\end{proof}
If, in \cref{localization module induced map}, we take $B=A$, $T=S$ and $\phi=\id_A$, it is easily seen that $\tilde{\phi}$ is just the map $\phi\otimes 1:M\otimes_AS^{-1}A\to N\otimes_AS^{-1}A$. We shall henceforth denote it by $S^{-1}\phi$. If $S$ is the complement of a prime ideal $\p$ of $A$, we write $\phi_\p$ instead of $S^{-1}\phi$.
\begin{proposition}\label{localization module and ring on same set}
Let $f$ be a homomorphism from a ring $A$ to a ring $B$ and $S$ a multiplicative subset of $A$. Then there exists a unique map $j:f(S)^{-1}B\to S^{-1}B$ (where $B$ is considered as an $A$-module by means of $f$) such that $j(b/f(s))=b/s$ for all $b\in B$, $s\in S$. If $\tilde{\phi}:S^{-1}A\to f(S)^{-1}B$ is the ring homomorphism associated with $f$, then $j\circ\tilde{f}=S^{-1}f$. The map $j$ is an isomorphism of the $S^{-1}A$-module structure on $f(S)^{-1}B$ defined by $\tilde{f}$ onto that on $S^{-1}B$ and also of the $B$-module structure on $f(S)^{-1}B$ onto that on $S^{-1}B$ (resulting from the definition $S^{-1}B=S^{-1}A\otimes_AB$).
\end{proposition}
\begin{proof}
If $b,b'\in B$ and $s,s'\in S$, the conditions $b/s=b'/s'$ and $b/f(s)=b'/f(s')$ are equivalent, as follows from \cref{localization module isomorphism}, which establishes the existence of $j$ and shows that $j$ is bijective. The uniqueness of $j$ is obvious. Clearly $j$ is an additive group isomorphism. If $a\in A$, $b\in B$, $s,t\in S$ then
\[(a/s)\cdot(b/f(t))=\tilde{f}(a/s)(b/f(t))=f(a)/f(s)(b/f(t))=(f(a)b)/f(st)\]
from which its follows that $j$ is $S^{-1}A$-linear. Clearly $j\circ\tilde{f}=S^{-1}f$. Finally, if $b.b'\in B$, $s\in S$, then
\[j(b\cdot(b'/f(s))=j(bb'/f(s))=bb'/s=b\cdot(b'/s),\]
which proves the last assertion.
\end{proof}
The map $j$ of \cref{localization module and ring on same set} is called the canonical isomorphism of $f(S)^{-1}B$ onto $S^{-1}B$. These two sets are in general identified by means of $f$. Then $\tilde{f}=S^{-1}f$ and $i_B^S=i_B^{f(S)}$.
\subsection{Change of multiplicative subset}
Let $A$ be a ring, $S$ a multiplicative subset of $A$ and $M$ an $A$-module. If $T$ is a multiplicative subset of $A$ containing $S$, it follows from \cref{localization module induced map} that there exists a unique $S^{-1}A$-linear map $i_M^{T,S}:S^{-1}M\to T^{-1}M$ such that $i_M^T=i_M^{T,S}\circ i_M^T$. The map $i_M^S$ maps the element $m/s$ of $S^{-1}M$ to the element $m/s$ of $T^{-1}M$. It is easily verified that $i_M^{T,S}=i_A^{T,S}\otimes 1_M$. If $U$ is a third multiplicative subset of $A$ such that $T\sub U$, then $i_M^{U,S}=i_M^{U,T}\circ i_M^{T,S}$. Moreover, if $\phi:M\to N$ is an $A$-module homomorphism, the diagram
\[\begin{tikzcd}
S^{-1}M\ar[r,"s^{-1}\phi"]\ar[d,"i_M^{T,S}"]&S^{-1}N\ar[d,"i_N^{T,S}"]\\
T^{-1}M\ar[r,"T^{-1}\phi"]&T^{-1}N
\end{tikzcd}\]
is commutative.
\begin{proposition}\label{localization iteration multiplicative subset isomorphism}
Let $A$ be a ring and $S$, $T$ two multiplicative subsets of $A$.
\begin{itemize}
\item[(a)] There exists a unique isomorphism $i_A$ from the ring $(ST)^{-1}A$ onto the ring $(S^{-1}T)^{-1}(S^{-1}A)$ such that the diagram
\[\begin{tikzcd}
A\ar[r,"i_A^S"]\ar[d,swap,"i_A^{ST}"]&S^{-1}A\ar[d,"i_{S^{-1}A}^{S^{-1}T}"]\\
(ST)^{-1}A\ar[r,"i"]&(S^{-1}T)^{-1}(S^{-1}A)
\end{tikzcd}\] 
\item[(b)] Let $M$ be an $A$-module. There exists an $(ST)^{-1}A$-isomorphism $i_M$ from the $(ST)^{-1}A$-module $(ST)^{-1}M$ onto the $(S^{-1}T)^{-1}(S^{-1}A)$-module $(S^{-1}T)^{-1}(S^{-1}M)$ such that the diagram
\[\begin{tikzcd}
M\ar[r,"i_M^S"]\ar[d,swap,"i_M^{ST}"]&S^{-1}M\ar[d,"i_{S^{-1}M}^{S^{-1}T}"]\\
(ST)^{-1}M\ar[r,"i_M"]&(S^{-1}T)^{-1}(S^{-1}M)
\end{tikzcd}\] 
\end{itemize}
\end{proposition}
\begin{proof}
It suffices to prove (a). We use the definition of $(ST)^{-1}A$ as the solution of a universal map problem. Let $B$ be a ring and $f$ a homomorphism from $A$ to $B$ such that $f(ST)$ consists of invertible elements. As $f(S)$ consequently consists of invertible elements, there exists a unique homomorphism $f_1:S^{-1}A\to B$ such that $f=f_1\circ i_A^S$. For all $t\in T$, $f_1(i_A^S(t))=f(t)$ is invertible in $B$ by hypothesis, hence $f_1(S^{-1}T)$ consists of invertible elements; then there exists a unique homomorphism $f_2$ from $(S^{-1}T)^{-1}(S^{-1}A)$ to $B$ such that $f_1=f_2\circ i_{S^{-1}A}^{S^{-1}T}$, whence $f=f_2\circ i$, where $i=i_{S^{-1}A}^{S^{-1}T}\circ i_A^{S}$.\par
Moreover, if $f_2':(S^{-1}T)^{-1}(S^{-1}A)\to B$ is a second homomorphism such that $f_2'\circ i=f$, then $(f_2'\circ i_{S^{-1}A}^{S^{-1}T})\circ i_A^{S}=(f_2\circ i_{S^{-1}A}^{S^{-1}T})\circ i_A^{S}$, whence $f_2'\circ i_{S^{-1}A}^{S^{-1}T}=f_2\circ i_{S^{-1}A}^{S^{-1}T}$ and consequently $f_2'=f_2$. As the images under $i$ of the elements of $ST$ are invertible, the ordered pair $((S^{-1}T)^{-1}(S^{-1}A),i)$ is a solution of the universal map problem (relative to $A$ and $ST$). This shows the existence and uniqueness of $i_A$.
\end{proof}
\begin{corollary}\label{localization of product of multiplicative sets}
If $S$ and $T$ are two multiplicative subsets of $A$ with $S\sub T$, then writing $S^{-1}T$ for the image of $T$ in $S^{-1}A$, we have $(S^{-1}T)^{-1}(S^{-1}A)=T^{-1}A$.
\end{corollary}
\begin{corollary}\label{localization at comparable prime}
If $S\sub A$ is a multiplicative set and $\p$ is a prime ideal of $A$ disjoint from $S$ then $(S^{-1}A)_{S^{-1}\p}=A_\p$. In particular if $\p_1\sub\p_2$ are prime ideals of $A$, then $(A_{\p_2})_{\p_1}=A_{\p_1}$.
\end{corollary}
\begin{proposition}\label{localization ring isomorphism iff saturation}
Let $S,T$ be multiplicatively closed subsets of $A$, such that $S\sub T$. Then the following statements are equivalent:
\begin{itemize}
\item[(a)] The homomorphism $i_A^{T,S}$ is bijective.
\item[(b)] For every $A$-module $M$, the homomorphism $i_M^{T,S}:S^{-1}M\to T^{-1}M$ is bijective.
\item[(c)] For each $t\in T$, $t/1$ is a unit in $S^{-1}A$.
\item[(d)] For each $t\in T$ there exists $x\in A$ such that $xt\in S$.
\item[(e)] Every prime ideal which meets $T$ also meets $S$.
\end{itemize}
\end{proposition}
\begin{proof}
It has been seen above that $i_M^{T,S}=i_A^{T,S}\otimes 1_M$, which immediately proves the equivalence of $(a)$ and $(b)$. Also, the equivalence of (c) and (d) is easy to see. By \cref{localization iteration multiplicative subset isomorphism}, since $S\sub T$, $T^{-1}A$ is identified with $(S^{-1}T)^{-1}(S^{-1}A)$ and (a) is equivalent to saying that the elements of $S^{-1}T$ are invertible in $S^{-1}A$, so (a) is equivalent to (c). Now by definition, it is clear that (d) implies (e).\par
So assume (e). To show the injectivity of $i_A^{T,S}$, let $i_A^{T,S}(x/s)=0$ in $T^{-1}A$, so that $xt=0$ for some $t\in T$. Now assume that $x/s\neq 0$, which is to say $\Ann(x)\cap S=\emp$. Then by \cref{miltiplicative subset avioding prime}, there is a prime ideal $\p\sups\Ann(x)$ such that $\p\cap S=\emp$. By assumption, we also have $\p\cap T=\emp$, which is a contradiction. This proves $x/s=0$, so $i_A^{T,S}$ is injective.\par
For the surjectivity, let $t\in T$. We claim that there exists $x\in A$ such that $xt\in S$. Otherwise, $(t)\cap S=\emp$, and by \cref{miltiplicative subset avioding prime}, there is a prime ideal $\p\sups(t)$ such that $\p\cap S=\emp$. Then we have $\p\cap T=\emp$, which is a contradiction.
\end{proof}
By \cref{localization ring isomorphism iff saturation}, we see that amongst the multiplicative subsets $T$ of $A$ containing $S$ and satisfying the equivalent conditions of \cref{localization ring isomorphism iff saturation}, there exists a greatest, consisting of all the elements of $A$ which divide an element of $S$. This set can be characterized by another property, which we now define.\par
A multiplicative subset $S$ of a ring $A$ is called \textbf{saturated} if the relation $xy\in S$ implies $x\in S$ and $y\in S$. If this holds, then for any $x\in A\setminus S$ and any $s\in S$ we have $ax\notin S$, which shows $x/1$ is not a unit in $S^{-1}A$, hence there is a maximal ideal $\m$ contains $x/1$. Now consider the contraction $\m^c$ in $A$. It contains $x$, and is prime since $\m$ is prime. Hence we see $x$ is contained in a prime ideal $\p$ disjoint from $A$, thus $A\setminus S$ is a union of prime ideals of $A$. In fact this condition is also sufficient: if $A\setminus S=\bigcup\p_i$ for prime ideals $\p_i$, then assume $xy\in S$. If $x\in A\setminus S$, then $x\in\p_i$ for some $\p_i$. Since $\p_i$ is an ideal, $xy\in\p_i$ for any $y\in A$, which means $xy\in A\setminus S$, a contradiction. So $x,y\in S$.
\begin{proposition}\label{localization multiplicative saturation def}
Let $S$ be a multiplicative subset of $A$ and define
\[\widebar{S}=\{x\in A:\text{there exists $y\in S$ such that $xy\in S$}\}.\]
Then $S$ is the smallest saturated multiplicative subset containing $S$, called the \textbf{saturation} of $S$, and $A\setminus\widebar{S}$ is the union of the prime ideals of $A$ not meeting $S$.
\end{proposition}
\begin{proof}
It is easy to see $\widebar{S}$ is a miltiplicative subset and saturated. In particular, $A\setminus \widebar{S}=\bigcup_i\p_i$, where $\p_i$ are prime ideals. Now if $\p$ is a prime ideal such that $\p\cap S=\emp$, then for any $x\in\p$, if there exists $y\in S$ such that $xy\in S$, then $xy\in\p\cap S$, which is a contradiction. This shows $\p\sub A\setminus S$, which finishes the proof.
\end{proof}
\begin{corollary}
Let $S,T$ be multiplicatively closed subsets of $A$, such that $S\sub T$. Then $i_M^{T,S}$ is bijective if and only if $T$ is contained in the saturation of $S$.
\end{corollary}
\begin{proposition}\label{localization subring of localization char}
Let $A$ be a ring, $S$ be a multiplicative subset of $A$. If $B$ is a subring of $S^{-1}A$ containing $i_A^S(A)$, then we have
\[S^{-1}A=S^{-1}B=T^{-1}B\quad\text{where}\quad T=\{b\in B:\text{$b$ is a unit of $S^{-1}A$}\}.\]
\end{proposition}
\begin{proof}
By definition of $T$ we have $S\sub T$ and $T^{-1}B$ can be identified with a subring of $S^{-1}A$. We can write
\[\begin{tikzcd}
A\ar[d]\ar[r,"i_A^S"]&S^{-1}A\ar[d,"i"]\\
B\ar[r,"i_B^T"]\ar[d,hook]&T^{-1}B\ar[ld,"j"]\\
S^{-1}A&
\end{tikzcd}\]
where $i$ and $j$ the natural maps. Also, by definition we have $i\circ j=1_{S^{-1}A}$. Also, by the definition of $T$ and \cref{localization ring isomorphism iff saturation} we see $S^{-1}B=T^{-1}B$. Now, for any $b/t\in T^{-1}B$, we have $t^{-1}=a/s$ where $a\in A$, $s\in S$. Since $b\in B\sub S^{-1}A$, we then see
\[b/t=i(b)\cdot i(a/s)=i(b\cdot a/s)\]
whence $i$ is surjective. Thus $i$ and $j$ are mutually inverse, giving an isomorphism $S^{-1}A\cong T^{-1}B$.
\end{proof}
\begin{proposition}\label{localization module saturation correspondence}
Let $M$ be an $A$-module. For every submodule $N'$ of the $S^{-1}A$-module $S^{-1}M$, let $\phi(N')$ be the inverse image of $N'$ under $i_M^S$. Then:
\begin{itemize}
\item[(a)] $S^{-1}\phi(N')=N'$.
\item[(b)] For every submodule $N$ of $M$, the submodule $\phi(S^{-1}N)$ of $M$ consists of those $m\in M$ for which there exists $s\in S$ such that $sm\in N$.
\item[(c)] $\phi$ is an isomorphism of the set of sub-$S^{-1}A$-modules of $S^{-1}M$ onto the set of submodules $Q$ of $M$ which satisfy the following condition:
\begin{enumerate}
\item[(MS)] If $sm\in Q$, where $s\in S$, $m\in M$, then $m\in Q$.
\end{enumerate}
\end{itemize}
\end{proposition}
\begin{proof}
Obviously $S^{-1}\phi(N')\sub N'$. Conversely, if $n'=m/s\in N'$, then $m/1\in N'$, hence in $m\in\phi(N')$ and conscquently $n'\in S^{-1}(\phi(N'))$, whence (a). For an element $m\in M$ to be such that $m\in\phi(S^{-1}N)$, it is necessary and sufficient that $m/1\in S^{-1}N$, that is, there exist $s\in S$ and $n\in N$ such that $m/1=n/s$, which means there exists $s'\in S$ such that $s'sm=s'n\in N$, whence (b). Finally, the relation $sm\in\phi(N')$ is equivalent by definition to $sm/1\in N'$ and as $s/1$ is invertible in $S^{-1}A$, this implies $m/1\in N'$, or $m\in\phi(N')$, hence $\phi(N')$ satisfies condition (MS). On the other hand, it follows from (b) that, if $N$ satisfies (MS), then $\phi(S^{-1}N)=N$, which completes the proof of (c).
\end{proof}
The submodule $\phi(S^{-1}N)$ is called the \textbf{saturation} of $N$ in $M$ with respect to $S$, and the submodules satifying condition (MS) (and hence equal to their saturations) are said to be \textbf{saturated} with respect to $S$. The submodule $\phi(S^{-1}N)$ is the kernel of the composite homomorphism
\[\begin{tikzcd}
M\ar[r,"\pi"]&M/N\ar[r,"i_{M/N}^S"]&S^{-1}M/S^{-1}N
\end{tikzcd}\]
where $\pi$ is the canonical homomorphism, as follows from the commutativity of the diagram
\[\begin{tikzcd}
M\ar[r,"\pi"]\ar[d,"i_M^S"]&M/N\ar[d,"i_{M/N}^S"]\\
S^{-1}M\ar[r]&S^{-1}M/S^{-1}N
\end{tikzcd}\]
If $S$ is the complement in $A$ of a prime ideal $\p$, $\phi(S^{-1}N)$ is also called the saturation of $N$ in $M$ with respect to $\p$.
\begin{corollary}
Let $N_1$, $N_2$ be two submodules of an $A$-modale $M$. For $S^{-1}N_1\sub S^{-1}N_2$, it is necessary and sufficient that the saturation of $N_1$ with respect to $S$ be contained in that of $N_2$.
\end{corollary}
\begin{corollary}\label{Noe module localization is Noe}
If $M$ is a Noetherian (resp. Artinian) $A$-module, then $S^{-1}M$ is a Noetherian (resp. Artinian) $S^{-1}A$-module. In particular, if the ring $A$ is Noetherian (resp. Artinian), so is the ring $S^{-1}A$.
\end{corollary}
\subsection{Properties of the localization operation}
\begin{proposition}
Let $A$ be a ring and $S$ a multiplicative subset of $A$. Then $S^{-1}A$ is $A$-flat.
\end{proposition}
\begin{proof}
If $\phi:M\to N$ is an injective homomorphism of $A$-modules, it is necessary to establish that $S^{-1}\phi:S^{-1}M\to S^{-1}N$ is injective. Now, if $m/s$ is such that $\phi(m)/s=0$, this implies the existence of a $t\in S$ such that $t\phi(m)=0$ or $\phi(tm)=0$. As $\phi$ is injective, it follows that $tm=0$, whence $m/s=0$ in $S^{-1}M$.
\end{proof}
\begin{corollary}\label{localization of sum and intersection}
Formation of fractions commutes with formation of finite sums, finite intersections and quotients. Precisely, if $N, P$ are submodules of an $A$-module $M$, then
\begin{itemize}
\item[(a)] $S^{-1}(N+P)=S^{-1}N+S^{-1}P$.
\item[(b)] $S^{-1}(N\cap P)=S^{-1}N\cap S^{-1}P$.
\item[(c)] The $S^{-1}A$-modules $S^{-1}(M/N)$ and $(S^{-1}M)/(S^{-1}N)$ are isomorphic.
\end{itemize}
\end{corollary}
\begin{proof}
Part (a) follows readily from the definitions and (b) is easy to verify:
if $y/s=z/t$ ($y\in N$, $z\in P$, $s,t\in S$) then $u(ty-sz)=0$ for some $u\in S$, hence $w=uty=usz\in N\cap P$ and therefore $y/s=w/stu\in S^{-1}(N\cap P)$. Consequently $S^{-1}(N\cap P)\sub S^{-1}(N)\cap S^{-1}P$, and the reverse inclusion is obvious.\par
Now to prove (c), we apply $S^{-1}$ to the exact sequence 
\[\begin{tikzcd}
0\ar[r]&N\ar[r]&M\ar[r]&M/N\ar[r]&0
\end{tikzcd}\]
Since localization is exact, we get the claim.
\end{proof}
\begin{proposition}\label{localization and Ann}
Let $M$ be a finitely generated $A$-module, $S$ a multiplicatively closed subset of $A$. Then $S^{-1}\Ann(M)=\Ann(S^{-1}M)$.
\end{proposition}
\begin{proof}
If this is true for two $A$-modules, $M$, $N$, it is true for $M+N$:
\begin{align*}
S^{-1}(\Ann(M+N))&=S^{-1}(\Ann(M)\cap\Ann(N))=\Ann(S^{-1}M)\cap\Ann(S^{-1}N)\\
&=\Ann(S^{-1}M+S^{-1}N)=\Ann(S^{-1}(M+N))
\end{align*}
where we use \cref{localization of sum and intersection}. Hence it is enough to prove for $M$ generated by a single element: then $M=A/I$ as $A$-module, where $\a=\Ann(M)$. We have
\[S^{-1}(M)=S^{-1}(A/\a)\cong(S^{-1}A)/(S^{-1}\a)\]
by \cref{localization of sum and intersection}, which proves the claim.
\end{proof}
\begin{corollary}\label{localization and ideal quotient}
If $N,P$ are submodules of an $A$-module $M$ and if $P$ is finitely generated, then $S^{-1}(N:P)=(S^{-1}N:S^{-1}P)$.
\end{corollary}
\begin{proof}
We have $(N:P)=\Ann((N+P)/P)$, so the claim follows by \cref{localization and Ann}.
\end{proof}
\begin{proposition}\label{localization and direct limit}
Let $I$ be a directed set, $(S_i)_{i\in I}$ an inereasing family of multiplicative subsets of a ring $A$, and $S=\bigcup_iS_i$. We write $\rho_{\beta\alpha}=i_A^{S_\beta,S_\alpha}$. Then $(S_\alpha^{-1}A,\rho_{\beta,\alpha})$ is a direct system of rings and, if for $\alpha\in I$, $\rho_\alpha$ is the canonical map of $S_\alpha^{-1}A$ to $\rlim S_\alpha^{-1}A$, there exists a unique isomorphism $i:\rlim S_\alpha^{-1}A\to S^{-1}A$ such that $i\circ\rho_\alpha=i_A^{S,S_\alpha}$ for all $\alpha\in I$.
\end{proposition}
\[\begin{tikzcd}
S_\alpha^{-1}A\ar[rrrd,bend right=10pt,"\rho_\alpha"]\ar[r,"\rho_{\beta\alpha}"]&S_\beta^{-1}A\ar[r,"\rho_{\gamma\beta}"]\ar[rrd,bend right=10pt,"\rho_\beta"]&S_\gamma^{-1}A\ar[rd,"\rho_\gamma"]\ar[r]&\cdots\ar[d]&\\
&&&\rlim S_\alpha^{-1}A\ar[r,"i"]&S^{-1}A
\end{tikzcd}\]
\begin{proof}
For $\alpha\preceq\beta\preceq\gamma$, $\rho_{\gamma\alpha}=\rho_{\gamma\beta}\circ\rho_{\beta\alpha}$, hence $(S_\alpha^{-1}A,\rho_{\beta\alpha})$ is a direct system of rings, and we write $B=\rlim S_\alpha^{-1}A$. As
\[i_A^{S,S_\alpha}=i_A^{S,S_\beta}\circ\rho_{\beta\alpha}\]
for $\alpha\preceq\beta$, $(i_A^{S,S_\alpha})$ is a direct system of homomorphisms and $i=\rlim i_A^{S,S_\alpha}$ is the unique map such that $i\circ\rho_\alpha=i_A^{S,S_\alpha}$ for all $\alpha\in I$.\par
The homomorphisms $\rho_\alpha\circ i_A^{S_\alpha}$ are all equal since $\rho_{\alpha\beta}\circ i_A^{S_\alpha}=i_A^{S_\beta}$ for $\alpha\preceq\beta$, so let $\phi:A\to B$ be their common value. Clearly elements of $\phi(S)$ are invertible in $B$, which shows there exists a homomorphism $\tilde{\phi}:S^{-1}A\to B$ such that $\tilde{\phi}\circ i_A^S=\phi$. Then
\[i\circ\tilde{\phi}\circ i_A^S=i\circ\phi=i\circ\rho_\alpha\circ i_A^{S_\alpha}=i_A^{S,S_\alpha}\circ i_A^{S_\alpha}=i_A^{S}\]
for all $\alpha\in I$ and consequently $i\circ\tilde{\phi}$ is the identity map of $S^{-1}A$. On the other hand, for all $\alpha\in I$,
\[\tilde{\phi}\circ i\circ\rho_\alpha\circ i_A^{S_\alpha}=\tilde{\phi}\circ i_A^{S,S_\alpha}\circ i_A^{S_\alpha}=\tilde{\phi}\circ i_A^{S}=\phi=\rho_\alpha\circ i_A^{S_\alpha}\]
whence $\tilde{\phi}\circ i\circ\rho_\alpha=\rho_\alpha$ for all $\alpha\in I$; it follows that $\tilde{\phi}\circ i$ is the identity automorphism of $B$ and consequently $i$ is an isomorphism.
\end{proof}
\begin{corollary}
Under the hypotheses of \cref{localization and direct limit}, let $M$ be an $A$-module. We write $\tau_{\beta\alpha}=i_M^{S_\beta,S_\alpha}$ for $\alpha\preceq\beta$. Then $(S_\alpha^{-1}M,\tau_{\beta,\alpha})$ is a direct system of rings and, if for $\alpha\in I$, $\tau_\alpha$ is the canonical map of $S_\alpha^{-1}M$ to $\rlim S_\alpha^{-1}M$, there exists a unique isomorphism $i:\rlim S_\alpha^{-1}M\to S^{-1}M$ such that $i\circ\tau_\alpha=i_M^{S,S_\alpha}$ for all $\alpha\in I$. 
\end{corollary}
\begin{proof}
The corollary follows immediately from the definitions $S^{-1}_\alpha M=M\otimes_AS_\alpha^{-1}A$ and $S^{-1}M=M\otimes_AS^{-1}A$ and the fact that taking direct limits commutes with tensor products.
\end{proof}
\subsection{Ideals under localization}
\begin{proposition}\label{localization and ideals}
Let $S$ be a multiplicative subset of a commutative ring $A$, and consider the localization operation. Let $\a$ be an ideal in $A$, and $\b$ an ideal in $S^{-1}A$.
\begin{itemize}
\item[(a)] $(\b^c)^e=\b$, hence every ideal in $S^{-1}A$ is an extended ideal.
\item[(b)] $(\a^e)^c=\bigcup_{s\in S}(\a:s)$. Hence $S^{-1}\a$ is proper if and only if $\a\cap S\neq\emp$ and $\a$ is a contracted ideal if and only if no element of $S$ is a zero-divisor in $A/\a$.
\item[(c)] The assignment $\p\to S^{-1}\p$ gives an inclusion-preserving bijection between the set of prime ideals of $A$ disjoint from $S$ and the set of ideals of $S^{-1}A$.
\item[(d)] The operation $S^{-1}$ commutes with formation of finite sums, products, intersections and radicals.
\end{itemize}
\end{proposition}
\begin{proof}
Recall that we always have $\b\sub (\b^c)^e$. Conversely, if $a/s\in\b$ then $a/1\in\b$, so $a\in\b^c$. It follows that $a/s\in S^{-1}(\b^c)=(\b^c)^e$, so $\b=(\b^c)^e$. For (b), by defninition we have
\begin{align*}
(\a^e)^c&=\{a\in A:a/1\in\a^e\}=\{a\in A:\text{$(sa-r)u=0$ for some $r\in I$ and $u,s\in S$}\}\\
&=\{a\in A:(\exists s\in S)\ sx\in\a\}=\bigcup_{s\in S}(\a:s).
\end{align*}
From this, we see $\a^e=(\bigcup_{s\in S}(\a:s))^e$, and therefore $1\in \a^e$ if and only if $\a\cap S\neq\emp$. From the description of $\a^{ec}$, we also find that $\a^{ec}=\a$ if and only if $\bigcup_{s\in S}(\a:s)\sub\a$, if and only if no element of $S$ is a zero-divisor in $A/\a$. This proves (b).\par
If $\p$ is prime in $A$ such that $\p\cap S=\emp$, then we see that $(\p:s)=\p$ for all $s\in S$, whence $(\p^e)^c=\p$. Thus $(\p)^e$ is a bijection on prime ideals of $A$. If $\p$ is prime in $S^{-1}A$, then $\p^c$ is prime in $A$. Further, by part (b) we have $\p^c\cap S=\emp$.\par
If $\p$ is prime in $A$ such that $\p\cap S=\emp$, then $S^{-1}\p$ is a proper ideal, as we see in (b). Now we prove the extension $\p^e=S^{-1}\p$ is prime. Let $x/s,y/t\notin S^{-1}\p$ but $xy/st\in S^{-1}\p$. Then by definition for some $p\in\p,u\in S$ we have
\[u(xy-pst)=0.\]
That is, $uxy=ustp\in\p$. Since $\p\cap S=\emp$, we must have $xy\in\p$. But since $x/s, y/t\notin S^{-1}\p$, $x\notin\p$ and $y\notin\p$. This is a contradiction. Thus the extension and contraction corresponds the prime ideals in $S^{-1}A$ and prime ideals in $A$ disjoint from $S$.\par
Finally, we deal with (e). For sums and products, this follows from \cref{ideal extension contract prop}; for intersections, from Corallary~\ref{localization of sum and intersection}. As to radicals, we have $S^{-1}\sqrt{\a}\sub\sqrt{S^{-1}\a}$ from \cref{ideal extension contract prop} since $S^{-1}\a=\a^e$. For the converse, let $x/s\in\sqrt{S^{-1}\a}$, then $x^n/s^n\in S^{-1}\a$ for some $n>0$. This means $ux^n\in\a$ for some $u\in S$. Then $u^nx^n$ also belongs to $\a$, so $ux\in\sqrt{\a}$. Moreover, since $x/s=(ux)/(us)$, we get $x/s\in S^{-1}\sqrt{\a}$.
\end{proof}
\begin{corollary}
There is an inclusion-preserving bijection between the prime ideals of $A_{\p}$ and the prime ideals of $A$ contained in $\p$. Hence $A_{\p}$ is a local ring.
\end{corollary}
\begin{proof}
If $a,b\notin \p$, then $ab\notin\p$. So $A-\p$ is a multiplicative set. The claim now comes from \cref{localization and ideals}.\par
The elements $a/s$ with $a\in\p$ form an ideal $\m$ in $A_\p$. If $b/t\notin\m$, then $b\notin\p$, so $b\in A-\p$ and is a unit in $A_\p$. By \cref{local ring iff}, $A_\p$ is a local ring.\par 
Or we can use the correspondence to prove $S^{-1}\p$ is the unique maximal ideal.
\end{proof}
\begin{corollary}\label{localization and nilradical}
If $\n(A)$ is the nilradical of $A$, the nilradical of $S^{-1}A$ is $S^{-1}\n(A)$.
\end{corollary}
\begin{remark}
Thus the passage from $A$ to $A_\p$ cuts out all prime ideals except those contained in $\p$. In the other direction, the passage from $A$ to $A/\p$ cuts out all prime ideals except those containing $\p$. Hence if $\p,\q$ are prime ideals such that $\p\sups\q$, then by localizing with respect to $\p$ and taking the quotient mod $\q$ $($in either order: these two operations commute, by Corallary~\ref{localization of sum and intersection}$)$, we restrict our attention to those prime ideals which lie between $\p$ and $\q$. In particular, if $\p=\q$ we end up with a field, called the \textbf{residue field at $\p$}, which can be obtained either as the field of fractions of the integral domain $A/\p$ or as the residue field of the local ring $A_\p$.
\end{remark}
\begin{proposition}\label{localization and quotient ring}
Let $A$ be a ring and $S$ a multiplicative subset of $A$. For every ideal $\b$ of $S^{-1}A$, let $\a=(i_A^S)^{-1}(\b)$ be such that $S^{-1}\a=\b$.
\begin{itemize} 
\item[(a)] Let $\pi$ be the canonical homomorphism $A\to A/\a$. Then the homomorphism from $S^{-1}A$ to $\pi(S)^{-1}(A/\a)$ canonically associated with $\pi$ is surjective and its kernel is $\b$, which defines, by taking quotients, a canonical isomorphism of $(S^{-1}A)/\b$ onto $\pi(S)^{-1}(A/\a)$. Moreover, the canonical homomorphism from $A/\a$ to $\pi(S)^{-1}(A/\a)$ is injective.
\item[(b)] If $\q$ is a prime ideal of $S^{-1}A$ and $\p=(i_A^S)^{-1}(\q)$, then there exists an isomorphism of the ring of fractions $A_\p$ onto the ring $(S^{-1}A)_\q$, which maps $a/b$ to $(a/1)/(b/1)$, where $a\in A$, $b\in A-\p$.
\end{itemize}
\end{proposition}
\begin{proof}
The ring $\pi(S)^{-1}(A/\a)$ can be identified with $S^{-1}(A/\a)$ by means of the canonical isomorphism between these two modules. The exact sequence $0\to\a\to A\to A/\a\to 0$ then induces an exact sequence
\[\begin{tikzcd}
0\ar[r]&S^{-1}\a\ar[r]&S^{-1}A\ar[r]&S^{-1}(A/\a)\ar[r]&0
\end{tikzcd}\]
whose existence proves the first assertion of (a), taking account of the fact that $\b=S^{-1}\a$. Since $\a$ is saturated with respect $S$, the conditions $a\in A$, $s\in S$, $as\in\a$ imply $a\in\a$; the homothety of ratio $s$ on $A/\a$ is then injective, which proves the second assertion of (i).\par
Suppose that $\q$ is prime and such that $\p$ is also prime. The set $T=A-\p$ is a multiplicative subset of $A$ which contains $S$, whence $ST=T$. Then it follows from \cref{localization of product of multiplicative sets} that there exists a unique isomorphism $i$ of $T^{-1}A=A_\p$ onto $(S^{-1}T)^{-1}(S^{-1}A)$ such that
\[i(a/b)=(a/1)/(b/1)\]
where $a\in A$ and $b\in T$. On the other hand $S^{-1}T$ obviously does not meet $\q$. Conversely, let $a/s\in S^{-1}A$; since $1/s$ is invertible in $S^{-1}A$, the condition $a/s\notin\q$ is equivalent to $i_A^S(a)\notin\q$ and hence to $a\notin\p$. It follows that $S^{-1}A-\q=S^{-1}T$ and hence $(S^{-1}T)^{-1}(S^{-1}A)=(S^{-1}A)_\q$.
\end{proof}
\begin{proposition}\label{prime ideal is contraction of a prime iff}
Let $\rho:A\to B$ be a ring homomorphism and let $\p$ be a prime ideal of $A$. Then $\p$ is the contraction of a prime ideal of $S$ if and only if $\p^{ec}=\p$.
\end{proposition}
\begin{proof}
One direction is trivial. Conversely, suppose that $\p^{ec}=\p$ and consider the multiplicative subset $S=\rho(A-\p)$ of $B$. The hypothesis shows that $S\cap\p^e=\emp$, so there exists a prime ideal $\q$ of $B$ disjoint with $S$ such that $\p^e\sub\q$. Then $\q^c$ contains $\p$ and is disjoint with $A-\p$, whence $\q^c=\p$.
\end{proof}
\begin{corollary}\label{prime ideal is contraction iff exist faithfully flat module}
Let $\rho:A\to B$ be a ring homomorphism.
\begin{itemize}
\item[(a)] Suppose that there exists a $B$-module $E$ such that $\rho^*(E)$ is a faithfully flat $A$-module. Then, for every prime ideal $\p$ of $A$, there exists a prime ideal $\mathfrak{P}$ of $B$ such that $\mathfrak{P}^c=\p$.
\item[(b)] Conversely, suppose that $B$ is a flat $A$-module. If for every prime ideal $\p$ of $A$, there exists an ideal $\mathfrak{P}$ of $B$ such that $\mathfrak{P}^c=\p$, then $B$ is a faithfully flat $A$-module.
\end{itemize}
\end{corollary}
\begin{proof}
The hypothesis in (a) implies that, for every ideal $\a$ of $A$, we have $\a^{ec}=\a$ (\cref{module res faithfully flat prop}), and it is sufficient to apply \cref{prime ideal is contraction of a prime iff}. Now assume the conditions in (b). It is sufficient to show that, for every maximal ideal $\m$ of $A$, there exists a maximal ideal $\mathfrak{M}$ of $B$ such that $\mathfrak{M}^c=\m$ (\cref{ring faithfully flat iff}). Now there exists by hypothesis an ideal $\mathfrak{Q}$ of $B$ such that $\mathfrak{Q}^c=\m$. As $\mathfrak{Q}\neq B$, there exists a maximal ideal $\mathfrak{M}$ of $B$ containing $\mathfrak{Q}$ and consequently $\mathfrak{M}^c=\m$ since $\mathfrak{M}^c$ cannot contain $1$.
\end{proof}
\begin{corollary}\label{flat map localization is surjective on spec}
Let $\rho:A\to B$ be a ring homomorphism such that $B$ is a flat $A$-module. Let $\mathfrak{P}$ be a prime ideal of $B$ and let $\p=\mathfrak{P}^c$. Then $^{a}\!\rho_{\p}:\Spec(B_{\mathfrak{P}})\to\Spec(A_\p)$ is surjective.
\end{corollary}
\begin{proof}
The map $\rho_\p:A_\p\to B_{\mathfrak{P}}$ is a local homomorphism since we have
\[(\p A_\p)^e=\Big(\frac{\p}{A-\p}\Big)^e=\frac{\p^e}{f(A-\p)}=\frac{\mathfrak{P}^{ce}}{f(A-\p)}\sub\mathfrak{P}B_\mathfrak{P}\neq B_\mathfrak{P},\]
so $B_{\mathfrak{P}}$ is a faithfully flat $A_\p$-module by \cref{ring faithfully flat iff}, and the claim follows from \cref{prime ideal is contraction iff exist faithfully flat module}.
\end{proof}
\begin{proposition}\label{ring extension minimal prime is contracted of minimal}
Let $B$ be a ring and $A$ a subring of $B$. For every minimal prime ideal $\p$ of $A$ there exists a minimal prime ideal $\mathfrak{P}$ of $B$ such that $\mathfrak{P}\cap A=\p$.
\end{proposition}
\begin{proof}
Let $\p$ be a minimal prime of $A$ and $S=A-\p$. Then the ring $A_\p$ is identified with a subring of $S^{-1}B$ and it has a single prime ideal $\p A_\p$ since $\p$ is minimal. As $S^{-1}B$ is not reduced to $0$ (since it contains $A_\p$), it has at least one prime ideal $\mathcal{P}$ and therefore $\mathcal{P}\cap A_\p=\p A_\p$. If $\mathfrak{P}'=(i_B^S)^{-1}(\mathcal{P})$, we have
\[i_A^S(\mathcal{P}\cap A)\sub\mathcal{P}\cap A_\p=\p A_\p.\]
hence $\mathfrak{P}'\cap A\sub\p$, and as $\p$ is minimal, $\mathfrak{P}'\cap A=\p$. If $\mathfrak{P}$ is a minimal prime ideal of $B$ contained in $\mathfrak{P}'$, then we also have $\mathfrak{P}\cap A=\p$, whence the claim.
\end{proof}
\subsection{Localization of tensor products and Hom sets}
\begin{proposition}\label{localization of tensor product}
Let $A$ be a ring and $S$ a multiplicative subset of $A$.
\begin{itemize}
\item[(a)] If $M$ and $N$ are two $A$-modules, the there is a canonical isomorphism
\[S^{-1}M\otimes_{S^{-1}A}S^{-1}N\cong S^{-1}(M\otimes_AN).\] 
\item[(b)] If $M$ and $N$ are two $S^{-1}A$-modules, the canonical homomorphism $M\otimes_AN\to M\otimes_{S^{-1}A}N$ derived from the $A$-bilinear map $(x,y)\mapsto x\otimes y$ is bijective.
\end{itemize}
\end{proposition}
\begin{proof}
Assertion (a) is an immediate consequence of the definition $S^{-1}M=M\otimes_AS^{-1}A$ and the associativity of tensor products. To prove (b), we note first that in $M$ and $N$, considered as $A$-modules, the homotheties induced by the elements $s\in S$ are bijective, hence $M=S^{-1}M$ and $N=S^{-1}N$ and similarly $S^{-1}(M\otimes_AN)=M\otimes_AN$. Then (b) is then a special case of (a).
\end{proof}
\begin{corollary}
Let $M$ be an $A$-module and $\a$ an ideal of $A$. The sub-$S^{-1}A$-modules
\[(S^{-1}\a)(S^{-1}M),\quad\a(S^{-1}M),\quad(S^{-1}\a)i_M^S(M),\quad S^{-1}(\a M)\]
of $S^{-1}M$ are identical. In particular, if $\a$ and $\b$ are two ideals of $A$, then
\[(S^{-1}\a)(S^{-1}\b)=\a(S^{-1}\b)=(S^{-1}\a)\b=S^{-1}(\a\b).\]
\end{corollary}
\begin{proposition}\label{localization and Hom set if finite presented}
Let $A$ be a ring and $S$ a multiplicative subset of $A$.
\begin{itemize}
\item[(a)] If $M$ and $N$ are two $A$-modules and $M$ is finitely generated (resp. finitely presented), the canonical homomorphism
\[S^{-1}\Hom_A(M,N)\to\Hom_{S^{-1}A}(S^{-1}M,S^{-1}N)\]
is injective (resp. bijectioe). 
\item[(b)] If $M$ and $N$ are two $S^{-1}A$-modules, the canonical bijection
\[\Hom_{S^{-1}A}(M,N)\to\Hom_A(M,N)\]
is bijective. 
\end{itemize}
\end{proposition}
\begin{proof}
As $S^{-1}A$ is a flat $A$-module, (a) is a particular case of \cref{module flat algebra extension of Hom}. On the other hand, we note that every $A$-homomorphism of $S^{-1}A$-modules is necessarily  $S^{-1}A$-linear, whence (b).
\end{proof}
\begin{proposition}\label{localization and extension of scalar}
Let $A$, $B$ be two rings, $\rho:A\to B$ a homomorphism, $S$ a multiplicative subset of $A$, and $\tilde{\rho}:S^{-1}A\to S^{-1}B$ the homomorphism corresponding to $\rho$.
\begin{itemize}
\item[(a)] For every $B$-module $M$ there exists a unique $S^{-1}A$-isomorphism
\[i:S^{-1}\rho_*(M)\to\tilde{\rho}_*(S^{-1}M)\] 
\item[(b)] For every $A$-module $M$, there exists a unique isomorphism
\[j:(S^{-1}M)\otimes_{S^{-1}A}(S^{-1}B)\to S^{-1}(M\otimes_AB).\] 
\end{itemize}
\end{proposition}
\begin{proof}
If we consider $S^{-1}M$ as an $A$-module by means of the composite homomorphism $i_M^S\circ \rho$, the homotheties induced by the elements of $S$ are bijective, hence there exists a unique homomorphism $i$. It is claer that $i$ is surjective; moreover, if $m\in M$, $s\in S$ and $m/\rho(s)=0$, then there exists $t\in S$ such that $\rho(t)m=0$, and so $tm=0$ in $\rho_*(M)$, hence $m/s=0$ in $S^{-1}\rho_*(M)$.\par
For (b), as $(S^{-1}M)\otimes_{S^{-1}A}(S^{-1}B)=(M\otimes_AS^{-1}A)\otimes_{S^{-1}A}S^{-1}B$, we see
\[S^{-1}(M\otimes_AB)=(M\otimes_AB)\otimes_B(S^{-1}B)\]
the existence of $j$ follows from the associativity of tensor products.
\end{proof}
\section{Local rings and passing from local to global}
\subsection{Local rings}
\begin{proposition}\label{local ring iff non-invertible is ideal}
Let $A$ be a ring and $I$ the set of non-invertible elements of $A$. The set $I$ is the union of the ideals of $A$ which are distinct from $A$. Moreover, the following conditions are equivalent:
\begin{itemize}
\item[(\rmnum{1})] $I$ is an ideal.
\item[(\rmnum{2})] The set of ideals of $A$ distinct from $A$ has a greatest element.
\item[(\rmnum{3})] $A$ has a unique maximal ideal.
\end{itemize}
\end{proposition}
\begin{proof}
The relation $x\in I$ is equivalent to $1\notin xA$ and hence $xA\neq A$. If $\a$ is an ideal of $A$ distinct from $A$ and $x\in\a$, then $xA\sub\a$, hence $xA\neq A$ and $x\in I$. Hence every ideal of $A$ distinct from $A$ is contained in $I$ and every element $x\in I$ belongs to a principal ideal $xA\neq A$. This proves the first assertion, which immediately implies the equivalence of (\rmnum{1}), (\rmnum{2}) and (\rmnum{3}).
\end{proof}
A ring $A$ is called a \textbf{local ring} if it satisfies the equivalent conditions of \cref{local ring iff non-invertible is ideal}. The quotient of $A$ by its Jacobson radical (which is then the unique maximal ideal of $A$) is called the \textbf{residue field} of $A$. Let $A$, $B$ be two local rings and $\m$, $\n$ their respective maximal ideals. A homomorphism $\rho:A\to B$ is called local if $\rho(\m)\sub\n$. Note that this amounts to saying that $\rho^{-1}(\n)=\n$, for $\rho^{-1}(\n)$ is then an ideal containing $\m$ and not containing $1$ and hence equal to $\m$. Taking quotients, we then derive canonically from $\rho$ an injective homomorphism $A/\m\to B/\n$ from the residue field of $A$ to that of $B$.
\begin{example}[\textbf{Example of local rings}]
\mbox{}
\begin{itemize}
\item[(a)] A field is a local ring. A ring which is reduced to $0$ is not a local ring.
\item[(b)] Let $A$ be a local ring and $\kappa$ its residue field. The ring of formal power series $B=A[[X_1,\dots,X_n]]$ is a local ring, for the non-invertible elements of $B$ are the formal power series whose constant terms are not invertible in $A$. The canonical injection of $A$ into $B$ is a local homomorphism and the corresponding injection of residue fields is an isomorphism.
\item[(c)] Let $\a$ be an ideal of a ring $A$ which is only contained in a single maximal ideal $\m$. Then $A/\a$ is a local ring with maximal ideal $\m/\a$ and residue field canonically isomorphic to $A/\m$. This applies in particular to the case $\a=\m^n$, where $\m$ is any maximal ideal of $A$ (\cref{prime ideal containing m^n}). If $A$ itself is a local ring with maximal ideal $\m$, then for every proper ideal $\a$ of $A$, $A/\a$ is a local ring, the canonical homomorphism $A\to A/\a$ a local homomorphism and the corresponding homomorphism of residue fields an isomorphism.
\item[(d)] Let $X$ be a topological space, $x_0$ a point of $X$ and $A$ the ring of germs at the point $x_0$ of real-valued functions continuous in a neighbourhood of $x_0$. Clearly, for the germ at $x_0$ of a continuous function $f$ to be invertible in $A$, it is necessary and sufficient that $f(x_0)\neq 0$, since this implies that $f(x)\neq 0$ in a neighbourhood of $x_0$. The ring $A$ is therefore a local ring whose maximal ideal $\m$ is the set of germs of functions which are zero at $x_0$. Taking quotients, the map $f\mapsto f(x_0)$ of $A$ to $\R$ gives an isomorphism of the residue field $A/\m$ onto $\R$.  
\end{itemize}
\end{example}
\begin{proposition}\label{localization at prime is local ring}
Let $A$ be a ring and $\p$ a prime ideal of $A$. The ring $A_\p$ is local, and its maximal ideal is the ideal $\p A_\p=\p_\p$, generated by the canonical image of $\p$ in $A_\p$. Its residue field is canonically isomorphic to the field of fractions of $A/\p$.
\end{proposition}
\begin{proof}
Let $S=A-\p$ and $i_A^S:A\to A_\p$ be the canonical homomorphism; the hypothesis that $\p$ is prime implies that $\p$ is saturated with respect to $S$, hence $(i_A^S)^{-1}(\p A_\p)=\p$ and, as the ideals of $A$ not meeting $S$ are those contained in $\p$, the first two assertions are special cases of \cref{localization and ideals}. Moreover, if $\pi$ is the canonical homomorphism $A\to A/\p$, $\pi(S)$ is the set of nonzero elements of the integral domain $A/\p$ and hence the last assertion is a special case of \cref{localization and quotient ring}.
\end{proof}
Let $A$ be a ring and $\p$ a prime ideal of $A$. The ring $A_\p$ is called the local ring of $A$ at $\p$, or the local ring of $\p$, when there is no ambiguity. If $A$ is a local ring and $\m$ its maximal ideal, the elements of $A-\m$ are invertible and hence $A_\m$ is canonically identified with $A$.
\begin{example}
\mbox{}
\begin{itemize}
\item[(a)] Let $p$ be a prime number. The local ring $\Z_\p$ is the set of rational numbers $a/b$, where $a,b$ are rational integers with $b$ prime to $p$. The residue field of $\Z_\p$ is isomorphic to the prime field $\F_p=\Z/p\Z$.
\item[(b)] Let $V$ be an affine algebraic variety, $A$ the ring of regular functions on $V$, $W$ an irreducible subvariety of $V$ and $\p$ the (necessarily prime) ideal of $A$ consisting of the functions which are zero at every point of $W$. The ring $A_\p$ is called the local ring of $W$ on $V$.  
\end{itemize}
\end{example}
\begin{proposition}
Let $A$, $B$ be two rings, $\rho:A\to B$ a homomorphism, $\q$ a prime ideal of $B$ and $\p=\q^c$. Then there is a canonical homomorphism $\rho_\q:A_\p\to B_\q$ which is local.
\end{proposition}
\begin{proof}
As $\rho(A-\p)\sub B-\q$, a canonical homomorphism $\rho_\q:A_\p\to B_\q$ is derived from $\rho$, and is immediate that $\rho_\q(\p A_\p)\sub\q B_\q$, hence $\rho_\q$ is local.
\end{proof}
\subsection{Modules over a local ring}
\begin{proposition}\label{local ring module M/mM=0 iff M=0}
Let $A$ be a ring, $\m$ an ideal of $A$ contained in the Jacobson radical of $A$ and $M$ an $A$-module. Suppose that one of the following conditions holds:
\begin{itemize}
\item[(a)] $M$ is finitely generated;
\item[(b)] $\m$ is nilpotent.
\end{itemize}
Then the relation $(A/\m)\otimes_AM=0$ implies $M=0$.
\end{proposition}
\begin{proof}
The assertion with respect to hypothesis (a) is precisely Nakayama's Lemma. On the other hand, the relation $(A/\m)\otimes_AM=0$ is equivalent to $M=\m M$ and hence implies $M=\m^nM$ for every integer $n>0$; whence the assertion with respect to hypothesis (b).
\end{proof}
\begin{corollary}\label{local ring module surjective iff tensor with (A/m)}
Let $A$ be a ring, $\m$ an ideal of $A$ contained in the Jacobson radical of $A$, $M$ and $N$ two $A$-modules and $\phi:M\to N$ an $A$-linear map. If $N$ is finitely generated or $\m$ is nilpotent and
\[1\otimes\phi:(A/\m)\otimes_AM\to(A/\m)\otimes_AN\]
is surjective, then $\phi$ is surjective.
\end{corollary}
\begin{proof}
Since $(A/\m)\otimes_A(N/\phi(M))$ is canonically isomorphic to $((A/\m)\otimes_AM)/\im(1\otimes\phi)$, the hypothesis implies $(A/\m)\otimes_A(N/\phi(M))=0$, hence $N/\phi(M)=0$ by \cref{local ring module M/mM=0 iff M=0}.
\end{proof}
\begin{corollary}\label{local ring module generating set of M/mM}
Let $A$ be a ring, $\m$ an ideal of $A$ contained in the Jacobson radical of $A$, $M$ an $A$-module and $(x_i)_{i\in I}$ a family of elements of $M$. If $M$ is finitely generated or $\m$ is nilpotent and the elements $1\otimes x_i$ generate the $(A/\m)$-module $M/\m M$, then the $x_i$ generate $M$.
\end{corollary}
\begin{proposition}\label{local ring module M/mM is free then M free}
Let $A$ be a ring, $\m$ an ideal of $A$ contained in the Jacobson radical of $A$ and $M$ an $A$-module. Suppose that one of the following conditions holds:
\begin{itemize}
\item[(a)] $M$ is finitely presented;
\item[(b)] $\m$ is nilpotent.
\end{itemize}
Then, if $M/\m M$ is a free $(A/\m)$-module and the canonical homomorphism $\m\otimes_AM\to M$ is injective, then $M$ is a free $A$-module. More precisely, if $(x_i)_{i\in I}$ is a family of elements of $M$ such that $(1\otimes x_i)$ is a basis of the $(A/\m)$-module $M/\m M$, then $(x_i)$ is a basis of $M$.
\end{proposition}
\begin{proof}
We already know that the $x_i$ generate $M$ (\cref{local ring module generating set of M/mM}). We shall see that they are linearly independent over $A$. To this end, let us consider the free $A$-module $L=A^I$. Let $(e_i)$ be its canonical basis and $\phi:A^I\to M$ the $A$-linear map such that $\phi(e_i)=x_i$ for all $i\in I$. If $R$ is the kernel of $\phi$, we shall prove that that $\m R=R$. Let $j$ be the canonical injection $R\to L$; then there is a commutative diagram
\[\begin{tikzcd}
&\m\otimes R\ar[d,"\alpha"]\ar[r,"1\otimes j"]&\m\otimes L\ar[r,"1\otimes\phi"]\ar[d,"\beta"]&\m\otimes M\ar[d,"\gamma"]\ar[r]&0\\
0\ar[r]&R\ar[r,"j"]&L\ar[r,"\phi"]&M\ar[r]&0
\end{tikzcd}\]
in which the two rows are exact. By hypothesis we have $\ker\gamma=0$, hence by snake lemma, there is an exact sequence
\[\begin{tikzcd}
0\ar[r]&\coker\alpha\ar[r]&\coker\beta\ar[r]&\coker\gamma\ar[r]&0
\end{tikzcd}\]
Now $\coker\beta=(A/\m)\otimes L$ and $\coker\gamma=(A/\m)\otimes M$. Since $(1\otimes e_i)$ is a basis for $(A/\m)\otimes L$ and its image in $(A/\m)\otimes M$ is a basis for $(A/\m)\otimes M$, we see $\coker\alpha=0$, whence $\m R=R$.\par
Under hypothesis (a), $(A/\m)\otimes_AM$ is a finitely generated module, hence $I$ is necessarily finite and $R$ is a finitely generated $A$-module. Then by \cref{local ring module M/mM=0 iff M=0} we have $R=0$. Similarly, under hypothesis (b) we also have $R=0$, so the proof is completed.
\end{proof}
\begin{corollary}\label{local ring module basis of direct factor iff}
Let $A$ be a local ring and $\m$ its maximal ideal. For a family $(y_i)$ of elements of $M$ to be a basis of a direct factor of $M$, it is necessary and sufficient that the family $(1\otimes y_i)$ be free in $M/\m M$.
\end{corollary}
\begin{proof}
If this condition holds, it can be assumed that $(x_i)$ is a subfamily of a family $(x_i)$ of elements of $M$ such that $(1\otimes x_i)$ is a basis of $M/\m M$ and \cref{local ring module M/mM is free then M free} then proves that $(x_i)$ is a basis of $M$.
\end{proof}
\begin{corollary}\label{local ring fp module free iff flat iff projective}
Let $A$ be a local ring, $\m$ its maximal ideal and $M$ an $A$-module. Suppose that one of the following conditions holds:
\begin{itemize}
\item[(a)] $M$ is finitely presented;
\item[(b)] $\m$ is nilpotent.
\end{itemize}
Then the following properties are equivalent:
\begin{itemize}
\item[(\rmnum{1})] $M$ is free;
\item[(\rmnum{2})] $M$ is projective;
\item[(\rmnum{3})] $M$ is flat;
\item[(\rmnum{4})] the canonical homomorphism $\m\otimes_AM\to M$ is injective;
\item[(\rmnum{5})] $\Tor_A^1(A/\m,M)=0$.    
\end{itemize}
\end{corollary}
\begin{proof}
The implications (\rmnum{1})$\Rightarrow$(\rmnum{2})$\Rightarrow$(\rmnum{3})$\Rightarrow$(\rmnum{4}) are immediate, and we know that (\rmnum{4}) is equivalent to (\rmnum{5}). As $A/\m$ is a field, $(A/\m)\otimes_AM$ is a free $(A/\m)$-module and \cref{local ring module M/mM is free then M free} shows that (\rmnum{4}) implies (\rmnum{1}).
\end{proof}
\begin{proposition}\label{local ring finite free module split injection iff}
Let $A$ be a local ring and $\m$ its maximal radical. Let $M$ and $N$ be two finitely generated free $A$-modules and $\phi:M\to N$ a homomorphism. The following properties are equivalent:
\begin{itemize}
\item[(\rmnum{1})] $\phi$ is an isomorphism of $M$ onto a direct factor of $N$;
\item[(\rmnum{2})] $1\otimes\phi:(A/\m)\otimes_AM\to(A/\m)\otimes_AN$ is injective;
\item[(\rmnum{3})] $\phi$ is injective and $\coker\phi$ is a free $A$-module;
\item[(\rmnum{4})] the transpose homomorphism $\phi^t:N^*\to M^*$ is surjective.   
\end{itemize}
\end{proposition}
\begin{proof}
We know that, if $N/\phi(M)$ is free, then $\phi(M)$ is a direct factor of $N$, hence (\rmnum{3}) implies (\rmnum{1}). Conversely, (\rmnum{1}) implies that $\coker\phi$, isomorphic to a complement of $\phi(M)$ in $N$, is a finitely generated projective $A$-module and a fortiori finitely presented, hence this module is free by \cref{local ring fp module free iff flat iff projective} and thus (\rmnum{1}) implies (\rmnum{3}).\par
On the other hand, (\rmnum{1}) obviously implies (\rmnum{2}). For simplicity we write $\widetilde{M}=(A/\m)\otimes_AM$ and $\widetilde{N}=(A/\m)\otimes_AN$. As $M$ and $N$ are finitely generated, the duals $\widetilde{M}^*$ and $\widetilde{N}^*$ of the $(A/\m)$-modules $\widetilde{M}$ and $\widetilde{N}$ are canonically identified with $M^*\otimes_A(A/\m)$ and $N^*\otimes_A(A/\m)$ and $(1\otimes\phi)^t$ with $1\otimes\phi^t$. As $\widetilde{M}$ and $\widetilde{N}$ are vector spaces over the field $A/\m$, the hypothesis that $1\otimes\phi$ is injective implies that $(1\otimes\phi)^t$ is surjective. \cref{local ring module surjective iff tensor with (A/m)} then shows that $\phi^t$ is surjective and we have thus proved that (\rmnum{2}) implies (\rmnum{4}).\par
Finally we show that (\rmnum{4}) implies (\rmnum{1}). Suppose that $\phi^t$ is surjective. As $M^*$ is free, there exists a homomorphism $\psi^t:M^*\to N^*$ such that $\phi^t\circ\phi^t=\id$. As $M$ and $N$ are free and finitely generated, there exists a homomorphism $\psi:N\to M$  such that the dual of $\psi$ is $\psi^t$, whence 
\[\id_M^t=\id_{M^*}=\phi^t\circ\psi^t=(\psi\circ\phi)^t\]
whence $\psi\circ\phi=\id$. This proves that $\phi$ is an isomorphism of $M$ onto a submodule which is a direct factor of $N$.
\end{proof}
\begin{corollary}\label{local ring finite free module same rank iff}
Under the hypotheses of \cref{local ring finite free module split injection iff} the following properties are equivalent:
\begin{itemize}
\item[(\rmnum{1})] $\phi$ is an isomorphism of $M$ onto $N$;
\item[(\rmnum{2})] $M$ and $N$ have the same rank;
\item[(\rmnum{3})] $1\otimes\phi:(A/\m)\otimes_AM\to (A/\m)\otimes_AN$ is bijective.
\end{itemize}
\end{corollary}
\begin{proof}
Clearly (\rmnum{1}) implies (\rmnum{2}); (\rmnum{2}) implies that $1\otimes\phi$ is surjective; moreover the hypothesis that $M$ and $N$ have the same rank implies that so do the vector spaces $(A/\m)\otimes_A M$ and $(A/\m)\otimes_AN$ over $A/\m$, hence $1\otimes\phi$ is bijective and (\rmnum{2}) implies (\rmnum{3}). Finally, condition (\rmnum{3}) implies, by \cref{local ring finite free module split injection iff}, that $N$ is the direct sum of $\phi(M)$ and a free submodule $P$ and $\phi$ is an isomorphism of $M$ onto $\phi(M)$. If $P\neq 0$, then $(A/\m)\otimes_AP\neq 0$ and $1\otimes\phi$ would not be sutjcctive; hence (\rmnum{3}) implies (\rmnum{1}).
\end{proof}
\begin{proposition}\label{local ring reduced finite module free iff}
Let $A$ be a reduced local ring, $\m$ its maximal ideal, $(\p_i)_{i\in I}$ the family of minimal prime ideals of $A$, $K_i$ the field of fractions of $A/\p_i$ and $M$ a finitely generated $A$-module. For $M$ to be free it is necessary and sufficient that
\begin{align}\label{local ring reduced finite module free iff-1}
[(A/\m)\otimes_A:(A/\m)]=[K_i\otimes_AM:K]
\end{align}
for all $i\in I$.
\end{proposition}
\begin{proof}
If $M$ is free, clearly the two sides of $(\ref{local ring reduced finite module free iff-1})$ are equal to the rank of $M$ for all $i\in I$. Suppose now that the condition is satisfied and denote by $n$ the common value of the two sides of $(\ref{local ring reduced finite module free iff-1})$. By \cref{local ring module generating set of M/mM} $M$ has a system of $n$ generators $x_1,\dots,x_n$. Suppose first that $A$ is an integral domain, in which case $\p_i=0$ for all $i\in I$. The elements $(1\otimes x_i)$ generate the vector space $K\otimes M$ over the field of fractions $K$ of $A$. But as by hypothesis this space is of dimension $n$ over $K$, the elements $1\otimes x_i$ are linearly independent over $K$. It follows that the $x_i$ are linearly independent over $A$ and hence form a basis of $M$.\par
Passing to the general case, there exists a surjective homomorphism $\eta$ from $L=A^n$ onto $M$. Consider the commutative diagram
\[\begin{tikzcd}
L\ar[r,"\eta"]\ar[d,"\pi_L"]&M\ar[d,"\pi_M"]\\
\prod_{i}((A/\p_i)\otimes L)\ar[r,"\tilde{\eta}"]&\prod_i((A/\p_i)\otimes M)
\end{tikzcd}\]
where $\phi_L$ (resp. $\pi_M$) is the map $x\mapsto(\phi_i(x))$ (resp. $y\mapsto(\psi_i(y))$), with $\phi_i:L\to(A/\p_i)\otimes L$ (resp. $\psi_i:M\to(A/\p_i)\otimes M$) being the canonical map, and $\tilde{\eta}$ is the product of the $1_{A/\p_i}\otimes\eta$. Then as
\[((A/\p_i)/(\m/\p_i))\otimes_{A/\p_i}((A/\p_i)\otimes_AM)=(A/m)\otimes_AM\]
and $A/\p_i$ is a local integral domain, it follows from the first part of the argument that each of the $M$, $1_{A/\p_i}\otimes\eta$ is an isomorphism; then so is $\tilde{\eta}$. On the other hand, as $A$ is reduced, $\bigcap_i\p_i=(0)$ whence $\bigcap_i\p_iL=(0)$ since $L$ is free. As this shows that $\pi_L$ is injective, it follows that $\tilde{\eta}\circ\pi_L$ is injective, hence $\eta$ is injective and, as $\eta$ is sutjective by definition, this shows that $M$ is free.
\end{proof}
\subsection{Passing from local to global}
\begin{proposition}\label{localization module bijective map if aM=0}
Let $A$ be a ring, $\m$ a maximal ideal of $A$ and $M$ an $A$-module. If there exists an ideal $\a$ of $A$ such that $\m$ is the only maximal ideal of $A$ containing $\a$ and $\a M=0$, then the canonical homomorphism $M\to M_\m$ is bijective.
\end{proposition}
\begin{proof}
By hypothesis $A/\a$ is then a local ring with maximal ideal $\m/\a$; $M$ can be considered as an $(A/\a)$-module. For all $s\in A-\m$ the canonical image of $s$ in $A/\a$ is invertible, hence the homothety $x\mapsto sx$ of $M$ is bijective from the definition of $M_\m$ as the solution of a universal problem, whence the proposition.
\end{proof}
In particular, if there exists $k>0$ such that $\m^kM=0$, the homomorphism $M\mapsto M_\m$ is bijective.
\begin{proposition}\label{localization module surjective map of m^kM_m}
Let $A$ be a ring, $\m$ a maximal ideal of $A$, $M$ an $A$-module and $k>0$ an integer. The canonical homomorphism $M\to M_\m/\m^kM_\m$ is surjectice, has kernel $\m^kM$ and defines an isomorphism of $M/\m^kM$ onto $M_\m/\m^kM_\m$.
\end{proposition}
\begin{proof}
It follows from \cref{localization module bijective map if aM=0} the homomorphism $M/\m^kM\to(M/\m^kM)_\m$ is bijective. On the other hand $(M/\m^kM)_\m$ is canonically identified with $M_\m/(\m^kM)_\m$ and hence $(\m^kM)_\m=\m^kM_\m$, whence there is an isomorphism of $M/\m^kM$ onto $M_\m/\m^kM_\m$ which maps the class of an element $x\in M$ to the class of $x/1$.
\end{proof}
\begin{corollary}
Let $A$ be a ring, $\m_1,\dots,\m_n$ distinct maximal ideals of $A$, $M$ an $A$-module and $k_1,\dots,k_n$ positive integers. The canonical homomorphism from $M$ to $\bigoplus_{i=1}^{n}M_{\m_i}/\m_i^{k_i}M_{\m_i}$ is surjective and its kernel is $(\bigcap_{i=1}^{n}\m_i^{k_i})M$.
\end{corollary}
\begin{proposition}\label{localization module zero iff}
Let $A$ be a commutative ring, and let $M$ be an $A$-module. The following are equivalent
\begin{itemize}
\item[(a)] $M=0$.
\item[(b)] $M_\p=0$ for every prime ideal $\p$.
\item[(c)] $M_\m=0$ for every maximal ideal $\m$.
\end{itemize}
\end{proposition}
\begin{proof}
Clearly (a)$\Rightarrow$(b)$\Rightarrow$(c). So we only need to show $(c)\Rightarrow(a)$. Assume $M_\m=0$ for every maximal ideal $\m$ but $M\neq 0$. Let $x$ be a non-zero element of $M$, consider the ideal $\Ann(m)$: It is proper hence contained in a maxiaml ideal $\m$. Consider the module $M_\m$. We must have $x/1=0$ since $M_\m=0$, hence $x$ is killed by some element in $A-\m$. But this is impossible since $\Ann(x)\sub\m$.
\end{proof}
\begin{proposition}\label{localization homomorphism inj surj iff}
Let $\phi:M\to N$ be an $A$-module homomorphism. Then the following are equivalent:
\begin{itemize}
\item[(a)] $\phi$ is injective (resp. surjective).
\item[(b)] $\phi_\p:M_\p\to N_\p$ is injective (resp. surjective) for each prime ideal $\p$ of $A$.
\item[(c)] $\phi_\m:M_\m\to N_\m$ is injective (resp. surjective) for each maximal ideal $\m$ of $A$.
\end{itemize}
\end{proposition}
\begin{proof}
Since localization is exact, we see (a)$\Rightarrow$(b)$\Rightarrow$(c). Now assume (c) and let $M'=\ker\phi$, then the sequence $0\to M'\to M\to N$ is exact, hence $0\to M'_\m\to M_\m\to N_\m$ is exact and therefore $M'_\m\cong\ker\phi_\m=0$ by assumption. Hence $M'=0$ by \cref{localization module zero iff}, so $\phi$ is injective. For the other part of the proposition, just reverse all the arrows.
\end{proof}
\begin{corollary}\label{localization is faithfully exact}
The localization functor is faithfully exact in the following sense: let $A$ be a commutative ring, and let
\begin{equation}\label{local faith exact-1}
\begin{tikzcd}
0\ar[r]&A\ar[r]&B\ar[r]&C\ar[r]&0
\end{tikzcd}
\end{equation}
be a sequence of $A$-modules. Then (\ref{local faith exact-1}) is exact if and only if the induced sequence of $A_{\p}$-modules
\[\begin{tikzcd}
0\ar[r]&A_{\p}\ar[r]&B_{\p}\ar[r]&C_{\p}\ar[r]&0
\end{tikzcd}\]
is exact for every prime ideal $\p$ of $A$, if and only if it is exact for every maximal
ideal $\m$.
\end{corollary}
\begin{corollary}\label{localization submodule iff submodule of localization}
Let $M$ be an $A$-module, $N$ a submodule of $M$ and $x$ an element of $M$. For $x\in N$, it is necessary and sufficient that, for any maximal ideal $\m$, the canonical image of $x$ in $M_\m$ belong to $N_\m$.
\end{corollary}
\begin{corollary}
Let $M$ be an $A$-module and, for all maximal ideal $\m$, let $\phi_\m:M\to M_\m$ be the canonical map. The homomorphism $x\mapsto(\phi_\m(x))$ of $M$ to $\bigoplus_\m M_\m$ is injective.
\end{corollary}
\begin{corollary}\label{localization torsion free module is intersection of localization}
Let $A$ be an integral domain, $K$ its field of fractions and $M$ a torsion free $A$-module such that $M$ is canonically identified with a sub-$A$-module of $K\otimes_AM$. Then, for any maximal ideal $\m$, $M_\m$ is canonically identified with a sub-$A$-module of $K\otimes_AM$ and $M=\bigcap_{\m}M_\m$.
\end{corollary}
\begin{proof}
As $M$ is identified with a submodule of $K\otimes_AM$, $M_\m$ is identified with a sub-$A_\m$-module of $(K\otimes_AM)_\m=K_\m\otimes_AM=K\otimes_AM$, so $M_\m$ is torsion-free. Moreover, the commutativity of the diagram
\[\begin{tikzcd}
M\ar[r]\ar[d]&K\otimes_AM\ar[d]\\
M_\m\ar[r]&(K\otimes_AM)
\end{tikzcd}\]
proves that the canonical map $M\mapsto M_\m$ is injective. The corollary then follows from \cref{localization submodule iff submodule of localization} applied to the $A$-module $K\otimes_A M$ and its submodule $M$.
\end{proof}
\begin{proposition}\label{module flat localization is flat}
Let $S$ be a multiplicative subset of a ring $A$ and $M$ be an $A$-module. If $M$ is flat (resp. faithfully/flat), $S^{-1}M$ is a flat (resp. faithfully flat) $S^{-1}A$-module and a flat $A$-module.
\end{proposition}
\begin{proof}
As $S^{-1}M=M\otimes_AS^{-1}A$, the first assertion follows from \cref{module flat extension is flat} (resp. \cref{module faithfully flat extension}); moreover, $S^{-1}A$ is a flat $A$-module, hence if $M$ is a flat $A$-module, so is $S^{-1}M$ by virtue of \cref{module flat restriction is flat}.
\end{proof}
\begin{proposition}\label{module flat base change localization}
Let $A$ be a ring, $B$ an $A$-algebra and $T$ a multiplicative subset of $B$. If $N$ is a $B$-module which is flat as an $A$-module, $T^{-1}N$ is a flat $A$-module.
\end{proposition}
\begin{proof}
We have $T^{-1}N=T^{-1}B\otimes_BN$, so the proposition follows from \cref{module flat tensor product is flat}.
\end{proof}
\begin{proposition}\label{module flat over base change iff}
Let $\rho:A\to B$ be a homomorphism and $N$ be a $B$-module. The following properties are equivalent:
\begin{itemize}
\item[(a)] $N$ is a flat $A$-module;
\item[(b)] for every maximal ideal $\mathfrak{N}$ of $B$, $N_\mathfrak{N}$ is a flat $A$-module;
\item[(c)] for every maximal ideal $\mathfrak{N}$ of $B$, if $\m=\mathfrak{N}^c$, $N_\mathfrak{N}$ is a flat $A_\m$-module.
\end{itemize}
\end{proposition}
\begin{proof}
For all $a\notin\m$, the homothety of $N_\mathfrak{N}$ induced by $a$ is bijective, hence $N_\mathfrak{N}$ is canonically identified with $(N_\mathfrak{N})_\m$ and the equivalence of (b) and (c) then follows. The fact that (a) implies (b) is a special case of \cref{module flat base change localization}. It remains to prove that (b) implies (a), that is, that, if (b) holds, for every injective $A$-module homomorphism $u:M\to M'$, the homomorphism $v=1\otimes u:N\otimes_AM\to N\otimes_AM'$ is injective. Now, $v$ is also a $B$-module homomorphism and, for it to be injective, it is necessary and sufficient that $v_\mathfrak{N}:(N\otimes_AM)_{\mathfrak{N}}\to(N\otimes_AM')_\mathfrak{N}$ be so for every maximal ideal $\mathfrak{N}$ of $B$ (\cref{localization homomorphism inj surj iff}). As we have
\[(N\otimes_AM)_\mathfrak{N}=B_\mathfrak{N}\otimes_B(N\otimes_AM)=N_\mathfrak{N}\otimes_AM\]
the homomorphism $v_\mathfrak{N}$ is identified with $1\otimes u:N_\mathfrak{N}\otimes_AM\to N_\mathfrak{N}\otimes_AM'$, which is injective since $N_\mathfrak{N}$ is a flat $A$-module by hypothesis.
\end{proof}
\begin{proposition}\label{module flat iff localization at maximal}
For an $A$-module $M$ to be flat (resp. faithfully flat), it is necessary and sufficient that, for every maximal ideal $\m$ of $A$, $M_\m$ be a flat (resp. faithfully flat) $A_\m$-module.
\end{proposition}
\begin{proof}
By \cref{localization is faithfully exact}, the sequence
\[\begin{tikzcd}
0\ar[r]&A\otimes M\ar[r]&B\otimes M\ar[r]&C\otimes M\ar[r]&0
\end{tikzcd}\]
is exact if and only if
\[\begin{tikzcd}
0\ar[r]&(A\otimes M)_{\p}\ar[r]&(B\otimes M)_{\p}\ar[r]&(C\otimes M)_{\p}\ar[r]&0
\end{tikzcd}\]
is exact for any prime ideal $\p$, iff for any maximal ideal $\m$. From \cref{localization of tensor product}, we have $(A\otimes_A M)_{\p}\simeq A_\p\otimes_{A_\p}M_{\p}$, so the claim follows.
\end{proof}
\begin{proposition}\label{integral domain inter of localization}
Let $A$ be an integral domain with field of fractions $K$. We consider any ring of fractions of $A$ as a subring of $K$. Then in this sense we have
\[A=\bigcap_{\p\in\Spec(A)}A_\p=\bigcap_{\m\in\Max(A)}A_\m\]
\end{proposition}
\begin{proof}
For $x\in K$ the set $I=\{a\in A\mid ax\in A\}$ is an ideal of $A$. Now it is easy to see that $x\in A_\p$ if and only if $I\cap(A-\p)\neq\emp$, which is equivalenct to that $I\nsubseteq\p$, so if $x\in A_\m$ for every maximal ideal $\m$ then $I$ is not contained in any maximal ideal of $A$, whence $1\in I$, that is, $x\in A$.
\end{proof}
\begin{corollary}\label{localization at a point is intersection}
Let $A$ be an integral domain and $f\in A$, then $A_f=\bigcap_{f\notin\p}A_\p$.
\end{corollary}
\begin{proof}
If $A$ is an integral domain, then $A_f$ is also a domain. Now $\Spec(A_f)=\{\p\in\Spec(A)\mid f\notin\p\}$, so by \cref{integral domain inter of localization} we get the claim.
\end{proof}
\section{The spectrum of a ring}
\subsection{The space \texorpdfstring{$\Spec(A)$}{SpecA}}
Let $A$ be a ring. The \textbf{spectrum of $\bm{A}$} is the set of prime ideals of $A$. It is usually denoted $\Spec(A)$. To avoid confusion, we will sometimes write $[\p]$ for the point of $\Spec(A)$ corresponding to the prime $\p$ of $A$. Of course, the zero ideal $(0)$ is a prime if $A$ is a domain.
\begin{example}
Here are several examples.
\begin{itemize}
\item[(a)] Prime ideals in $\Z$ is the form $(p)$ with $p$ prime, and every set $\{(p)\}=V(p)$ is closed. Hence $\Spec(\Z)$ is a line, with one closed point for each prime number in $\Z$, and one non-closed point corresponding to the ideal $(0)$.
\item[(b)] As $\R$ is a field, $\Spec(\R)$ consists of a single point, corresponding to the ideal $(0)$.
\item[(c)] The maximal ideals of the ring $\C[X]$ are all of the form $(x-a)$ for $a\in\C$. The only non-maximal prime ideal of $\C[X]$ is $(0)$. Thus $\Spec(\C[X])$ is the complex plane together with a single non-closed point, corresponding to $(0)$.
\item[(d)] There are two types of maximal ideals in $\R[X]$, those generated by a polynomial of degree one, and those generated by a polynomial of degree two. The degree one maximal ideals are all of the form $(x-a)$ for $a\in\R$. The maximal ideals of degree two correspond to complex conjugate pairs of elements of $\C-\R$. There is only one non-maximal prime ideal, $(0)$. Therefore $\Spec(\R[X])$ is the upper-half plane, together with a single non-closed point.
\end{itemize}
\end{example}
Each element $f\in A$ defines a function, which we also write as $f$, on the space $\Spec(A)$: if $x=\p\in X=\Spec(A)$, we denote by $\kappa(x)$ or $\kappa(\p)$ the quotient field of the integral domain $A/\p$, called the \textbf{residue field} of $X$ at $x$, and we define $f(x)\in\kappa(x)$ to be the image of $f$ via the canonical maps
\[A\to A/\p\to\kappa(\p).\]
The functions induced by elements of $A$ are called the \textbf{regular functions} on $\Spec(A)$.
\begin{example}
Consider the ring of polynomials $\C[X]$, and let $p(x)$ be a polynomial. If $\alpha\in\C$ is a number, then $(x-\alpha)$ is a prime of $\C[X]$, and in fact is maximal. So $\kappa(x-\alpha)$ is just the field $\C$, and the value of $p(x)$ at the point $(x-\alpha)\in\Spec(\C[X])$ is the number $p(\alpha)$.\par
More generally, if $A$ is the coordinate ring of an affine variety $V$ over an algebraically closed field $K$ and $\m$ is the maximal ideal corresponding to a point $x\in V$ in the usual sense, then $\kappa(x)=K$ and $f(x)$ is the value of $f$ at $x$ in the usual sense.\par
In general, the function $f$ has values in fields that vary from point to point. Moreover, $f$ is not necessarily determined by the values of this function. For example, if $K$ is a field, the ring $A=K[X]/(x^2)$ has only one prime ideal, which is $(x)$; and thus the element $x\in A$, albeit nonzero, induces a function whose value is $0$ at every point of $\Spec(A)$.
\end{example}
By using regular functions, we make $\Spec(A)$ into a topological space. For each subset $S\sub A$, let
\[V(S)=\{x\in\Spec(A):\text{$f(x)=0$ for all $f\in S$}\}=\{\p\in\Spec(A):\p\sups S\}.\]
The impulse behind this definition is to make each $f\in A$ behave as much like a continuous function as possible. Of course the fields $\kappa(x)$ have no topology, and since they vary with $x$ the usual notion of continuity makes no sense. But at least they all contain an element called zero, so one can speak of the locus of points in $\Spec(A)$ on which $f$ is zero; and if $f$ is to be like a continuous function, this locus should be closed. Since intersections of closed sets must be closed, we are led immediately to the definition above: $V(S)$ is just the intersection of the loci where the elements of $S$ vanish.
\begin{proposition}\label{Spec of ring closed subsets prop}
Let $A$ be a ring.
\begin{itemize}
\item[(a)] If $\a$ is the ideal generated by $E$, then $V(E)=V(\a)=V(\sqrt{\a})$.
\item[(b)] Let $\a,\b$ be ideals of $A$, then $V(\a)\sub V(\b)$ iff $\b\sub\sqrt{\a}$.
\item[(c)] If $(E_i)_{i\in I}$ is any family of subsets of $A$, then
\[V\Big(\bigcup_{i\in I}E_i\Big)=\bigcap_{i\in I}V(E_i).\]
\item[(d)] For any ideals $\a,\b$ of $A$, we have $V(\a\cap\b)=V(\a\b)=V(\a)\cup V(\b)$.
\end{itemize}
In particular, the sets $V(E)$ satisfy the axioms for closed sets in a topological space. The resulting topology is called the \textbf{Zariski topology} on $\Spec(A)$.
\end{proposition}
\begin{proof}
We only prove (d). By \cref{radical ideal prop}, we have
\[V(\a\b)=V(\sqrt{\a\b})=V(\sqrt{\a\cap\b})=V(\a\cap\b).\]
Now we only need to prove $V(\a\cap\b)=V(\a)\cup V(\b)$. Since $\a\cap\b$ is contained in $\a$ and $\b$, for either $\a\sub\p$ or $\b\sub\p$ we have $\a\cap\b\sub\p$, whence $V(\a)\cup V(\b)\sub V(\a\cap\b)$. Conversely, let $\p\in V(\a\cap\b)$ so that $\a\cap\b\sub\p$. By \cref{prime ideal contain intersection}, $\a\sub\p$ or $\a\sub\p$, so $\p\in V(\a)\cup V(\b)$.
\end{proof}
Let $X$ be the prime spectrum of a ring $A$; for all $f\in A$, let us denote by $X_f$ the set of prime ideals of $A$ not containing $f$. Then $X_f=X-V(f)$ and $X_f$ is therefore an open set, called the \textbf{distinguished} (or \textbf{standard}) \textbf{open subsets} of $X$. By \cref{Spec of ring closed subsets prop}, every closed subset of $X$ is an intersection of closed sets of the form $V(f)$ and hence the $X_f$ form a base of the spectral topology on $X$.
\begin{proposition}\label{Spec of ring basic open subset prop}
Ler $A$ be a ring, $f,g\in A$ and $X=\Spec(A)$.
\begin{itemize}
\item[(a)] $D(f)=\emp$ iff $f$ is nilpotent and $D(f)=X$ iff $f$ is a unit.
\item[(b)] $D(f)=D(g)$ iff $\sqrt{(f)}=\sqrt{(g)}$.
\item[(c)] $D(f)\cap D(g)=D(fg)$.
\item[(d)] If $\a\sub A$ is an ideal and $\p$ is a prime of $A$ with $\p\notin V(\a)$, then there exists an $f\in A$ such that $\p\in D(f)$ and $D(f)\cap V(I)=\emp$.
\item[(e)] If $(f_i)_{i\in A}$ is a family of elements in $A$, then $\bigcup_iD(f_i)=X-V(\{f_i\})$.
\end{itemize}
\end{proposition}
\begin{proof}
Part (a), (b) and (c) are immediate from the definition of $D(f)$ and \cref{Spec of ring closed subsets prop}. Now if $\p\notin V(I)$, then there exists $f\in A$ such that $f\in\a-\p$. Then $f\notin\p$ so that $\p\in D(f)$. Also, if $\q\in D(f)$ then $f\notin\q$ and thus $\a$ is not contained in $\q$, which means $D(f)\cap V(\a)=\emp$. This proves (d).\par
Finally, for (e) we note that
\[\bigcup_{i\in I}D(f_i)=\bigcup_{i\in I}(X-V(f_i))=X-\bigcap_{i\in I}V(f_i)=X-V\Big(\bigcup_{i\in I}\{f_i\}\Big).\]
This completes the proof.
\end{proof}
For every subset $Y$ of $X$, let us denote by $I(Y)$ the intersection of the prime ideals of $A$ which belong to $Y$. Clearly $I(Y)$ is an ideal of $A$ and the map $Y\mapsto I(Y)$ is decreasing with respect to inclusion in $X$ and $A$, and we have
\[I\Big(\bigcup_{i\in I}Y_i\Big)=\bigcap_{i\in I}I(Y_i).\]
\begin{proposition}\label{Spec of ring I map prop}
Let $A$ be a ring, $\a$ an ideal of $A$ and $Y$ a subset of $X=\Spec(A)$.
\begin{itemize}
\item[(a)] $V(\a)$ is closed in $X$ and $I(Y)$ is a radical ideal of $A$.
\item[(b)] $I(V(\a))$ is the radical of $\a$ and $V(I(Y))$ is the closure of $Y$ in $X$.
\item[(c)] If $E$ is any subset of $A$ and $Y$ is any subset of $X$, then
\[V(E)=V(I(V(E))),\quad I(Y)=I(V(I(Y))).\]
\item[(d)] The maps $I$ and $V$ define decreasing bijections, one of which is the inverse of the other, between the set of closed subsets of $X$ and the set of radical ideals of $A$.
\end{itemize}
\end{proposition}
\begin{proof}
Assertion (a) and the first assertion of (b) follow from the definitions. If a closed set $V(E)$ (for some $E\sub A$) contains $Y$, then $E\sub\p$ for every prime ideal $\p\in Y$, whence $E\sub I(Y)$ and consequently $V(E)\sups V(I(Y))$. As $Y\sub V(I(Y))$, $V(I(Y))$ is then the smallest closed set of $X$ containing $Y$, which completes the proof of (b). Finally, it follows from (b) that, if $\a$ is a prime ideal equal to its radical, then $I(V(\a))=\a$ and that, if $Y$ is closed in $X$, then $V(I(Y))=Y$. This proves (d).
\end{proof}
\begin{corollary}\label{Spec of ring I of union of closed subsets}
For every family $(Y_i)_{i\in I}$ of closed subsets of $X$, $I(\bigcap_{i\in I}Y_i)$ is the radical of the sum of the ideals $I(Y_i)$.
\end{corollary}
\begin{proof}
It follows from \cref{Spec of ring I map prop} that $I(\bigcap_{i\in I}Y_i)$ is the smallest ideal which is equal to its radical and contains all the $I(Y_i)$; this ideal then contains $\sum_{i\in I}I(Y_i)$ and therefore also the radical of it, whence the corollary.
\end{proof}
\begin{corollary}\label{Spec of ring is Noe if A is Noe}
The space $X=\Spec(A)$ is Noetherian if $A$ is a Noetherian ring.
\end{corollary}
\begin{proof}
If $A$ is Noetherian, every ascending chain of prime ideals is stable, whence every descending chain of irreducible closed sets is stable. In other words, $\Spec(A)$ is Noetherian. 
\end{proof}
\begin{remark}
Note that a ring $A$ can be non-Noetherian even though its spectrum is Noetherian. For example, consider the ring
\[A=k[x_1,x_2,\dots]/(x_1,x_2^2,x_3^3,\dots)\]
where $k$ is a field. Then $A$ is not Noetherian since the ideal
\[\m=(x_1,x_2,x_3,\dots)/(x_1,x_2^2,x_3^3,\dots)\]
is not finitely generated. However, this is the unique prime ideal of $A$ (in fact maximal), so $\Spec(A)$ is a singleton and hence Noetherian.
\end{remark}
\begin{corollary}\label{Spec of ring subset irreducible iff ideal prime}
Let $A$ be a ring. For a subset $Y$ of $X=\Spec(A)$ to be irreducible, it is necessary and suficient that the ideal $I(Y)$ be prime.
\end{corollary}
\begin{proof}
Writing $\p=I(Y)$, we note that, for an element $f\in A$, the relation $f\in\p$ is equivalent to $Y\sub V(f)$. Suppose that $Y$ is irreducible and let $f,g$ be elements of $A$ such that $fg\in\p$. Then
\[Y\sub V(fg)=V(f)\cup V(g).\]
as $Y$ is irreducible and $V(f)$ and $V(g)$ are closed, $Y\sub V(f)$ or $Y\sub V(g)$, hence $f\in\p$ or $g\in\p$, which proves that $\p$ is prime.\par
Suppose now that $\p$ is prime; then $Y=V(\p)$ by \cref{Spec of ring I map prop} and, as $\p$ is prime, $\p=I(\{\p\})$, whence $\widebar{Y}=V(I(\{\p\}))=\widebar{\{\p\}}$. As a set consisting of a single point is irreducible, $Y$ is irreducible.
\end{proof}
\begin{corollary}\label{Spec of ring irreducible iff reduced ring integral}
For a ring $A$ to be such that $X=\Spec(A)$ is irreducible, it is necessary and sufiicient that the quotient of $A$ by its nilradical $\n$ be an integral domain.
\end{corollary}
\begin{proof}
By \cref{Spec of ring I map prop}, $I(X)$ is the radical of the ideal $(0)$, that is $\n$.
\end{proof}
\begin{remark}
Note that a ring $A$ can be non-Noetherian even through $\Spec(A)$ is Noetherian. For example, the ring $A=k[x_1,x_2,\dots]/(x_1^2,x_2^2,\dots)$ has a unique maximal ideal $\m=(x_1,x_2,\dots)$, whence $\Spec(A)$ is a singleton and is Noetherian. But $\m$ is not finitely generated, so $A$ is not Noetherian.
\end{remark}
\begin{corollary}\label{Spec of ring is Kolmogoroff}
The space $X=\Spec(A)$ is a Kolmogoroff space.
\end{corollary}
\begin{proof}
If $\p$ and $\q$ are two different points of $X$, we have, either $\p\nsubseteq\q$ or $\q\nsubseteq\p$, hence one of the points $\p$, $\q$ does not belong to the closure of the other.
\end{proof}
\begin{proposition}\label{Spec of ring quasi-compact prop}
Let $A$ be a ring. The topology on $X=\Spec(A)$ has the following properties:
\begin{itemize}
\item[(a)] $X$ is quasi-compact.
\item[(b)] $X$ has a basis for the topology consisting of quasi-compact opens.
\item[(c)] The intersection of any two quasi-compact opens is quasi-compact.
\end{itemize}
\end{proposition}
\begin{proof}
It suffices to prove that any covering of $\Spec(A)$ by standard opens can be refined by a finite covering. Thus suppose that $\Spec(A)=\bigcup_iD(f_i)$ for a set of elements $f_i$ of $A$. This means that $\bigcap_iV(f_i)=\emp$. According to \cref{Spec of ring closed subsets prop} this means that $V(\{f_i\}_{i\in I})=\emp$. Then the ideal generated by the $f_i$ is the unit ideal of $A$. This means that we can write 
\[1=\sum_{i=1}^{n}a_if_i.\]
It follows that $\Spec(A)=\bigcup_{i=1}^{n}D(f_i)$. Since $D(f)\cong\Spec(A_f)$, each standard open is quasi-compact, and so their intersection $D(f)\cap D(g)=D(fg)$ is also quasi-compact.
\end{proof}
\begin{proposition}\label{Spec of ring irreducible subset iff}
Let $A$ be a ring and $X=\Spec(A)$.
\begin{itemize}
\item[(a)] The irreducible closed subsets of $X$ are exactly the subsets $V(\p)$, with $\p\sub A$ a prime.
\item[(b)] The irreducible components of $X$ are exactly the subsets $V(\p)$, with $\p\sub A$ a minimal prime.
\end{itemize}
\end{proposition}
\begin{proof}
By \cref{Spec of ring I map prop} $V(\p)=\widebar{\{\p\}}$ is the closure of a singleton and hence irreducible. conversely, let $V(\a)\sub X$ with $\a$ a radical ideal. If $\a$ is not prime, then choose $a,b\in A$, $a,b\notin\a$ with $ab\in\a$. In this case $V(\a,a)\cup V(\a,b)=V(\a)$, but neither $V(\a,b)=V(\a)$ nor $V(\a,a)=V(\a)$ since $\sqrt{(\a,a)}\neq\sqrt{\a}=\a,\sqrt{(\a,b)}\neq\a$. Hence $V(\a)$ is not irreducible.
\end{proof}
\begin{corollary}\label{Spec of ring irreducible subset containing a point char}
Let $A$ be a ring, $X=\Spec(A)$, and $\p\in X$.
\begin{itemize}
\item[(a)] The set of irreducible closed subsets of $X$ containing $\p$ is in one-to-one correspondence with primes $\q\sups\p$.
\item[(b)] The set of irreducible component of $X$ containing $\p$ is in one-to-one correspondence with minimal primes $\q\sups\p$.
\end{itemize}
\end{corollary}
\begin{corollary}\label{Noe ring minimal prime is finite}
The set of minimal prime ideals of a Noetherian ring $A$ is finite.
\end{corollary}
\begin{proof}
The spectrum of a Noetherian is a Noetherian topological space, hence has finitely many irreducible components.
\end{proof}
\begin{proposition}\label{Spec of ring minimal prime and compact open}
Let $A$ be a ring and $\p$ be a minimal prime of $A$. Let $V\sub\Spec(A)$ be a quasi-compact open not containing the point $\p$. Then there exists an $f\in A$, $f\notin\p$ such that $D(f)\cap V=\emp$.
\end{proposition}
\begin{proof}
Since $V$ is quasi-compact we may write $V=\bigcup_{i=1}^{n}D(g_i)$. Since $\p\notin V$ we have $g_i\in\p$ for all $i$. Since $\p$ is minimal, we see $\n(A_\p)=\p A_\p$, so each $g_i$ is nilpotent in $A_\p$. Hence we can find an $f\in A$, $f\notin\p$ such that $fg_i^{n_i}=0$ for some $n_i>0$. Then $D(f)\cap D(g_i)=D(fg_i^{n_i})=\emp$, so this $D(f)$ works.
\end{proof}
\begin{proposition}\label{Spec of ring open dense char}
Let $A$ be a ring and $X=\Spec(A)$. Let $U$ be an open subset of $X$, then $U$ is dense in $X$ if and only if $U$ meets every irreducible components of $X$. In particular, for an element $f\in A$ to be such that $D(f)$ is dense in $X$, it is necessary and sufficient that $f$ does not belong to any minimal prime ideal of $A$. Moreover, every open dense subset of $X$ contains an open subset of the form $D(f)$ with $f$ not a zero divisor.
\end{proposition}
\begin{proof}
Let $\p_1,\dots,\p_n$ be the minimal prime ideals of $A$ and set $X_i=V(\p_i)$. If $\p_i\in U$ for all $i$, then $U\cap X_i\neq\emp$ so $U$ is dense (in that for this direction we do not need to assume that $X$ has finitely many irreducible components). Conversely, assume that $U$ is dense and let $Z=\bigcup_{X_i\cap U\neq\emp}X_i$. Then $Z$ is closed and contains $U$. But then $Z=X$ since $U$ is dense, so there is no $X_i$ such that $X_i\cap U=\emp$. In particular, since for an element $f\in A$, $D(f)\cap V(\p_i)\neq\emp$ if and only if $f\notin\p_i$, we see $(f)$ is dense if and only if $f\notin\p_i$ for all $i$.\par
On the other hand, if $U$ is an open dense subset of $X$, the complement of $U$ is of the form $V(\a)$, where $\a$ is an ideal which is not contained in any of the $\p_i$; it is therefore not contained in their union, and there then exists $f\in\a$ contained in $S$; hence $D(f)\sub U$.
\end{proof}
\begin{proposition}\label{Spec of ring Hausdorff iff}
Let $A$ be a ring. Let $X=\Spec(A)$ as a topological space. The following are equivalent
\begin{itemize}
\item[(a)] $X$ is Hausdorff.
\item[(b)] $X$ is totally disconnected.
\item[(c)] Every prime ideal of $A$ is maximal.
\item[(d)] Every quasi-compact open of $X$ is closed.
\item[(e)] Every standard open $D(f)\sub X$ is closed.
\end{itemize}
\end{proposition}
\begin{proof}
It is clear that (d) and (e) are equivalent as every quasi-compact open is a finite union of standard opens. The implication (c)$\Rightarrow$(d) follows from \cref{Spec of ring minimal prime and compact open}. Assume (d) holds. Let $\p,\p'$ be distinct primes of $A$. Choose an $f\in\p'$, $f\notin\p$. Then $\p'\notin D(f)$ and $\p\in D(f)$. By (d) the open $D(f)$ is also closed. Hence $\p$ and $\p'$ are in disjoint open neighbourhoods whose union is $X$. Thus $X$ is Hausdorff and totally disconnected. Thus (d) implies (a) and (b). Finally, if (b) holds, then the closure of $\{\p\}$ must be itself, so we see (c) holds. If $X$ is Hausdorff then every point is closed, so (a) implies (c). These together finish the proof.
\end{proof}
\subsection{Functoriality of the spactrum}
We will now show that $A\mapsto\Spec(A)$ defines a contravariant functor from the category of rings to the category of topological spaces. Let $\rho:A\to B$ be a homomorphism of rings. If $\q$ is a prime ideal of $B$, $\rho^{-1}(\q)$ is a prime ideal of $A$. Therefore we obtain a map
\[^{a}\!\rho=\Spec(\rho):\Spec(B)\to\Spec(A),\quad \q\mapsto\rho^{-1}(\q).\]
\begin{proposition}\label{Spec of ring induced map prop}
Let $\rho:A\to B$ be a ring homomorphism.
\begin{itemize}
\item[(a)] For every subset $S\sub A$, we have $(^{a}\!\rho)^{-1}(V(S))=V(\rho(S))$, so $^{a}\!\rho$ is continuous.
\item[(b)] If $\b$ is an ideal of $B$, then $\widebar{^{a}\!\rho(V(\b))}=V(\rho^{-1}(\b))$.
\end{itemize}
\end{proposition}
\begin{proof}
A prime ideal $\q$ of $B$ contains $\rho(S)$ if and only if $\rho^{-1}(\q)$ contains $S$, so (a) holds. For part (b), we can rewrite the left hand side as $VI(^{a}\!\rho(V(\b)))$. But
\[I(^{a}\!\rho(V(\b)))=\bigcap_{\q\in V(\b)}\rho^{-1}(\q)=\rho^{-1}\Big(\bigcap_{\q\in V(\b)}\q\Big)=\rho^{-1}(\sqrt{\b})=\sqrt{\rho^{-1}(\b)}.\]
and the claim follows by applying $V$.
\end{proof}
\begin{corollary}\label{Spec of ring map dominant iff ker nilpotent}
The map $^{a}\!\rho$ has dense image if and only if every element of $\ker(\rho)$ is nilpotent.
\end{corollary}
\begin{proof}
By letting $\b=0$ in \cref{Spec of ring induced map prop} we get $\widebar{^{a}\!\rho(Y)}=V(\ker\rho)$, so the claim follows from the fact that $V(\a)=\Spec(A)$ if and only if $\a\sub\n(A)$.
\end{proof}
\begin{proposition}\label{Spec of ring and quotient map}
Suppose that for any $g\in B$ there exists $f\in A$ such that $g=u\rho(f)$, where $u$ is invertible in $B$ (this is the case if $\rho$ is surjective). Then $^{a}\!\rho$ is a homeomorphism from $X$ onto $^{a}\!\rho(X)$. In particular, $\Spec(A)$ and $\Spec(A/\n(A))$ are naturally homeomorphic.
\end{proposition}
\begin{proof}
We show that for any subset $F\sub B$, there exist $E\sub A$ such that $V(F)=V(\rho(E))$; by virtue of the $T_0$-axiom and the formula of \cref{Spec of ring induced map prop}(a), this will first imply that $^{a}\!\rho(X)$ is injective, then, still by virtue of \cref{Spec of ring induced map prop}(a), that $^{a}\!\rho(X)$ is a homeomorphism. Now, it suffices for each $g\in F$ to take $f\in A$ such that $u\rho(f)=g$ with $u$ invertible in $B$; the set $E$ of these elements then satisfies the requirement.
\end{proof}
\begin{corollary}\label{Spec of ring induced map on quotient ring is restriction}
Let $\rho:A\to B$ be a homomorphism of rings. Let $\a$ be an ideal of $A$ and let $\b$ be its extension in $B$. Let $\bar{\rho}:A/\a\to B/\b$ be the induced homomorphism. If $\Spec(A/\a)$ is identified with its canonical image $V(\a)$ in $\Spec(A)$, and $\Spec(B/\b)$ with its image $V(\b)$ in $\Spec(B)$, then $^{a}\!\bar{\rho}$ is the restriction of $^{a}\!\rho$ to $V(\b)$.
\end{corollary}
\begin{proof}
Let $\q\in V(\b)$, then $\a\sub\a^{ec}=\b^c\sub\q^c$, so $^{a}\!\rho(V(\b))\sub V(\a)$. Let $\pi_A:A\to A/\a$, $\pi_B:B\to B/\b$ be the quotient maps. Then
\[\bar{\rho}\circ\pi_A(a)=\bar{\rho}(a+\a)=\rho(a)+\b=\pi_B\circ\rho(a).\] Hence $\bar{\rho}\circ\pi_A=\pi_B\circ\rho$, which implies $^{a}(\pi_A)\circ{^{a}\!\bar{\rho}}={^{a}\!\rho}\circ{^{a}(\pi_B)}$, so we have the followsing commutative diagram:
\[\begin{tikzcd}
\Spec(B/\b)\ar[d,swap,"^{a}(\pi_B)"]\ar[r,"^{a}\!\bar{\rho}"]&\Spec(A/\a)\ar[d,"^{a}(\pi_A)"]\\
V(\b)\ar[r,"^{a}\!\rho"]&V(\a)
\end{tikzcd}\]
which is exactly the claim.
\end{proof}
\begin{proposition}\label{Spec of ring and localization}
Let $S$ be a multiplicative subset of $A$ and let $i_A^S:A\to S^{-1}A$ be the canonical homomorphism. Then $^{a}(i_A^S)$ is a homeomorphism of $\Spec(S^{-1}A)$ onto the subspace of $\Spec(A)$ consisting of prime ideals $\p$ with $S\cap\p=\emp$.
\end{proposition}
\begin{proof}
We already know that $^{a}\!(i_A^S)$ is injective by \cref{localization and ideals}, and similar to \cref{Spec of ring and quotient map}, we see $^{a}\!(i_A^S)$ is closed, so the claim follows.
\end{proof}
Due to \cref{Spec of ring and localization}, we may therefore identify $\Spec(S^{-1}A)$ with a subspace of $\Spec(A)$ and write $S^{-1}X$ for $\Spec(S^{-1}A)$, if $X=\Spec(A)$.
\begin{corollary}\label{Spec of ring induced map on spec(A_f)}
Let $A$ be a ring and $f\in A$. Then the canonical map $i_A^{S_f}:A\to A_f$ induces a homeomorphism $\Spec(A_f)\cong D(f)\sub\Spec(A)$, with the inverse map given by $\p\to\p A_f$.
\end{corollary}
\begin{corollary}\label{Spec of ring induced map on spec(A_p)}
Let $A$ be a ring and $\p$ a prime ideal of $A$. Then the canonical image of $\Spec(A_\p)$ in $\Spec(A)$ is equal to the intersection of all the open neighborhoods of $\p$ in $\Spec(A)$.
\end{corollary}
\begin{proof}
We have that $\Spec(A_f)=\{\q\in\Spec(A):f\notin\q\}$ and $\Spec(A_\p)=\{\q\in\Spec(A):\q\sub\p\}$, therefore
\begin{align*}
\Spec(A_\p)=\bigcap_{f\in A-\p}\Spec(A_f)=\bigcap_{\p\in D(f)}D(f).
\end{align*}
That is, $\Spec(A_\p)$ is the intersection of all basic open neighborhoods of the point $\p$ in $\Spec(A)$. Since every open set is a union of principal open sets, we get the claim.
\end{proof}
\begin{proposition}\label{Spec of ring induced map on localization spec prop}
Let $\rho:A\to B$ be a homomorphism and $S$ a multiplicative subset of $A$. Then $^{a}(S^{-1}\rho):\Spec(S^{-1}B)\to\Spec(S^{-1}A)$ is the restriction of $^{a}\!\rho$ on $\Spec(S^{-1}B)$ and
\[\Spec(S^{-1}B)=(^{a}\!\rho)^{-1}(\Spec(S^{-1}A)).\]
\end{proposition}
\begin{proof}
Since $S^{-1}B=\rho(S)^{-1}B$, we have the following commutative diagram:
\[\begin{tikzcd}
A\ar[d,swap,"i_A^S"]\ar[r,"\rho"]&B\ar[d,"i_B^S"]\\
S^{-1}A\ar[r,"S^{-1}\rho"]&S^{-1}B
\end{tikzcd}\]
whence the first claim. Now, by definition, for $\q\in\Spec(B)$ we see $\q\cap\rho(S)=\emp$ if and only if $\q^c\cap S=\emp$, so the last claim follows.
\end{proof}
\begin{corollary}\label{Spec of ring fiber of a prime char}
Let $\rho:A\to B$ be a homomorphism and $\p$ be a prime ideal of $A$. Then the continuous map $^{a}\nu:\Spec(B\otimes_A\kappa(\p))\to\Spec(B)$, induced by the homomorphism $\nu:B\to B\otimes_A\kappa(\p)$, induces a homeomorphism from $\Spec(B\otimes_A\kappa(\p))$ onto the fiber of $\p$ under the map $^{a}\!\rho$.
\end{corollary}
\begin{proof}
The homomorphism $\nu$ is the composition of the quotient homomorphism $B\to\p^e$ and of the canonical homomorphism $B/\p^e$ in its ring of fractions $(B/\p^e)_\p$. According to \cref{Spec of ring and quotient map} and \cref{Spec of ring induced map on spec(A_p)}, $\nu^*$ thus induces a homeomorphism of $\Spec(B\otimes_A\kappa(\p))$ on the subspace of $\Spec(B)$ consists of the prime ideals $\mathfrak{P}$ of $B$ which contain $\p^e$ and are disjoint of $\rho(A-\p)$. That is to say, which are lying above $\p$.
\end{proof}
\begin{example}
Let $A$ be an integral domain with just one non-zero prime ideal $\p$. and let $K$ be the field of fractions of $A$. Let $B=(A/\p)\times K$. Define $\rho:A\to B$ by $\rho(x)=(\widebar{x},x)$, where $x$ is the image of $\widebar{x}$ in $A/\p$. Then $^{a}\!\rho$ is bijective but not a homeomorphism.\par
In fact, the idals in $A/\p$ are $(0),A/\p$, that in $K$ is $(0),K$. So by the claim below the ideals in $(A/\p)\times K$ are
\[(0)\times(0),(A/\p)\times(0),(0)\times K, (A/\p)\times K.\]
Note that $(0)\times(0)$ is not prime: $(a,0)\cdot(0,b)=0$ but neither of them is zero. So 
\[\Spec(B)=\{(A/\p)\times(0),(0)\times K\}\]
Since $\Spec(A)=\{0,\p\}$, $^{a}\!\rho$ is bijective.\par 
Now $\Spec(B)$ is a two point Hausdorff space, since the prime ideals in $B$ are maximal. But $\Spec(A)$ is non-Hausdorff because it contains $0$ and 
\[\widebar{\{0\}}=\{\p\sub A:\p\sups(0)\}=\{0,\p\}\]
Therefore $^{a}\!\rho$ cannot be a homeomorphism.
\end{example}
\begin{example}\label{PID spec example}
Let $A$ be a principal ideal domain. In this case, the maximal ideals are of the form $(p)$ for a prime element $p$ of $A$, and all prime ideals are maximal or the zero ideal. Therefore the closed points of $\Spec(A)$ correspond to equivalence classes of prime elements. Let $\eta\in\Spec(A)$ be the point corresponding to the zero ideal. Then the closure of $\{\eta\}$ is $\Spec(A)$.\par
As $A$ is a principal ideal domain, every closed subset of $\Spec(A)$ is of the form $V(f)$ for some $f\in A$. Assume $f\neq0$ and let $f=p_1^{e_1}\cdots p_r^{e_r}$ with pairwise non-equivalent prime elements $p_i$ and integers $e_i$. Then $V(f)$ consists of those closed points which correspond to the prime divisors of $f$, that is, $V(f)=\{(p_1),\dots,(p_r)\}$. Therefore the proper closed subsets are the finite sets consisting of closed points.\par
If $A$ is a local principal ideal domain, but not a field (i.e., $A$ is a discrete valuation ring), $\Spec(A)$ consists only of two points $\eta$ and $x$, where $\p_x$ is the maximal ideal and $\p_η=\{0\}$. The only nontrivial open subset of $\Spec(A)$ is then $\{\eta\}$.
\end{example}
\begin{example}\label{spec R[X] PID}
Let $A=R[X]$, where $R$ is a PID. We assume that $R$ is not a field and let $K$ be its field of fractions. Let $X=\Spec(R[X])$. Then the prime elements of $R[X]$ are either of the form $p$, where $p$ is a prime element of $R$, or of the form $f$, where $f\in R[X]$ is a primitive polynomial which is irreducible in $K[X]$\par
If $p\in R$ is a prime element, then the closure $V(p)$ is homeomorphic to $\Spec((R/p)[X])$. Since $(R/p)[X]$ is a PID but not be a field, we see that $(p)$ is not a maximal ideal, and the prime ideals in $V(p)$ different from $(p)$ are the maximal ideals generated by $p$ and $f$ where $f\in R[X]$ is a polynomial such that its image in $(R/p)[X]$ is irreducible.\par
The situation is more complicated for prime ideals of the form $(f)$, where $f$ is a primitive irreducible polynomial. If the leading coefficient of $f$ is a unit in $R$, it is possible to divide in $R[X]$ by $f$ with unique remainder, and therefore $R[X]/(f)$ is finitely generated as $R$-module. This implies that $(f)$ is not a maximal ideal by \cref{integral ring extension field iff}, as otherwise $R$ would be a field.\par
For other primitive irreducible polynomials $f$, $(f)$ might be a maximal ideal, namely if $R$ contains only finitely many prime elements (up to equivalence): If $0\neq a\in R$ is an element which is divisible by all prime elements of $R$ we have, with $f=ax-1$,
\[R[X]/(f)\cong R[a^{-1}]=K\]
which shows that $(f)$ is a maximal ideal.
\end{example}
\subsection{The support of a module}
Let $A$ be a ring and $M$ an $A$-module. The set of prime ideals $\p$ of $A$ such that $M_\p\neq 0$ is called the \textbf{support} of $M$ and is denoted by $\supp(M)$. As every maximal ideal of $A$ is prime, it follows immediately from \cref{localization module zero iff} that for $A$-module $M$ to be equal to $0$, it is necessary and suffieient that $\supp(M)=\emp$.
\begin{example}
Let $\a$ be an ideal of $A$, then we have $\supp(A/\a)=V(\a)$. In fact, if $\p$ is a prime of $A$ such that $\a\nsubseteq\p$, then $(A/\a)_\p=0$; if on the other hand $\a\sub\p$, then $\a A_\p$ is contained in the maximal ideal $\p A_\p$ of $A_\p$ and $(A/\a)_\p$ is isomorphic to $A_\p/\a A_\p$ and hence is non-zero; whence our assertion.
\end{example}
\begin{proposition}\label{supp of module and exact sequence}
Let $A$ be a ring and $M$ an $A$-module.
\begin{itemize}
\item[(a)] If $N$ is a submodule of $M$, then
\[\supp(M)=\supp(N)\cup\supp(M/N).\] 
\item[(b)] If $M$ is the sum of a family $(M_i)_{i\in I}$ of submodules, then
\[\supp(M)=\bigcup_{I\in I}\supp(M_i).\] 
\end{itemize}
\end{proposition}
\begin{proof}
From the exact sequence $0\to N\to M\to N/N\to 0$, we derive, for every prime ideal $\p$ of $A$, the exact sequence
\[\begin{tikzcd}
0\ar[r]&N_\p\ar[r]&M_\p\ar[r]&(M/N)_\p\ar[r]&0
\end{tikzcd}\]
For $M_\p$ to be reduced to $0$, it is necessary and sufficient that $N_\p$ and $(M/N)_\p$ be so. In other words, the relation $\p\notin\supp(M)$ is equivalent tO $\p\notin\supp(N)$ and $\p\notin\supp(M/N)$, which proves (a). Part (b) can be proved similarly.
\end{proof}
\begin{corollary}\label{supp of module into annihilator of generator}
Let $A$ be a ring, $M$ an $A$-module, $(m_i)_{i\in I}$ a system of generators of $M$ and $\a_i$ the annihilator of $m_i$. Then $\supp(M)=\bigcup_{i\in I}V(\a_i)$.
\end{corollary}
\begin{proof}
We have $\supp(M)=\bigcup_{i\in I}\supp(Am_i)$ by \cref{supp of module and exact sequence}. On the other hand, $Am_i$ is isomorphic to the $A$-module $A/\a_i$ and we have seen that $\supp(A/\a_i)=V(\a_i)$, hence the claim.
\end{proof}
\begin{proposition}\label{supp of finite module is V(Ann)}
Let $A$ be a ring, $M$ an $A$-module and $\a$ its annihilator. If $M$ is finitely generated, then $\supp(M)=V(\a)$ and $\supp(M)$ is therefore closed in $\Spec(A)$.
\end{proposition}
\begin{proof}
Let $m_1,\dots,m_n$ be a system of generators of $M$ and let $\a_i$ be the annihilator of $m_i$ for each $i$. Then $\a=\bigcap_{i=1}^{n}\a_i$, hence $V(\a)=\bigcup_{i=1}^{n}V(\a_i)$ and the proposition follows from the \cref{supp of module into annihilator of generator}.
\end{proof}
\begin{corollary}\label{supp of module intersection iff homothety is nilpotent}
Let $A$ be a ring, $M$ a finitely generated $A$-module and $a$ an element of $A$. For $a$ to belong to every prime ideal of the support of $M$, it is necessary and sufficient that the homothety of $M$ with ratio $a$ be nilpotent.
\end{corollary}
\begin{proof}
It follows from \cref{supp of finite module is V(Ann)} that the intersection of the prime ideals belonging to $\supp(M)$ is the radical of the annihilator $\a$ of $M$. To say that $a$ belongs to this radical is equ1valent to say that there exist a power $a^n$ such that $a^nM=0$.
\end{proof}
\begin{corollary}\label{supp of module contained in V(a) iff}
Let $A$ be a Noetherian ring, $M$ a finitely generated $A$-module and $\a$ an ideal of $A$. For $\supp(M)\sub V(\a)$, it is necessary and sufficient that there exist an integer $n$ such that $\a^nM=0$.
\end{corollary}
\begin{proof}
If $\b$ is the annihilator of $M$, the relation $\supp(M)\sub V(\a)$ is equivalent to $V(\b)\sub V(\a)$ by \cref{supp of finite module is V(Ann)} and hence to $\a\sub\sqrt{\b}$. Since $A$ is Noetherian, this condition is also equivalent to the existence of an integer $n>0$ such that $\a^n\sub\b$.
\end{proof}
\begin{lemma}\label{tensor of module finite over local ring nonzero iff}
Let $A$ be a local ring and $M$ and $N$ two finitely generated $A$-modules. If $M\neq 0$ and $N\neq 0$, then $M\otimes N\neq 0$.
\end{lemma}
\begin{proof}
Let $\kappa$ be the residue field of $A$. By virtue of \cref{local ring module M/mM=0 iff M=0}, $\kappa\otimes_AM\neq 0$, and $\kappa\otimes_AN\neq 0$, then we deduce that
\[(\kappa\otimes_AM)\otimes_\kappa(\kappa\otimes_AN)\neq 0\]
But, since the tensor product is associative, this tensor product is isomorphic to
\[M\otimes_A(\kappa\otimes_\kappa\kappa)\otimes_AN=M\otimes_A\kappa\otimes_AN\]
and therefore to $\kappa\otimes_A(M\otimes_AN)$, whence the lemma.
\end{proof}
\begin{proposition}\label{supp of module finite tensor}
Let $M$, $N$ be two finitely generated modules over a ring $A$, then
\[\supp(M\otimes_AN)=\supp(M)\cap\supp(N).\]
\end{proposition}
\begin{proof}
We need to prove that, if $\p$ is a prime ideal of $A$, the relations "$(M\otimes_AN)_\p\neq 0$" and "$M_\p=0$ and $N_\p\neq 0$" are equivalent. As the $A_\p$-modules $M_\p\otimes_{A_\p}N_\p$ and $(M\otimes_AN)_\p$ are isomorphic, our assertion follows from \cref{tensor of module finite over local ring nonzero iff}.
\end{proof}
\begin{corollary}\label{supp of module quotient by ideal product}
Let $M$ be a finitely generated $A$-module and $\n$ its annihilator. For every ideal $\a$ of $A$, $\supp(M/\a M)=V(\a)\cap V(\n)=V(\a+\n)$.
\end{corollary}
\begin{proof}
We have $M/\a M=M\otimes_A(A/\a)$ and $A/\a$ is finitely generated, so the claim follows from \cref{supp of module finite tensor}.
\end{proof}
\begin{lemma}\label{module extended nonzero if nonzero}
Let $A$, $B$ be two local rings, $\rho:A\to B$ a local homomorphism and $M$ a finitely generated $A$-module. If $M\neq 0$, then $\rho_*(M)\neq 0$.
\end{lemma}
\begin{proof}
Let $\m$ be the maximal ideal of $A$ and $\kappa=A/\m$ the residue field. The hypothesis implies that $B\otimes_A\kappa=B/\m B\neq 0$ and $M\otimes_A\kappa=M/\m M\neq 0$ by \cref{local ring module M/mM=0 iff M=0}. Since the tensor product is associative, we have
\[(M\otimes_AB)\otimes_A\kappa=M\otimes_A(B\otimes_A\kappa)=M\otimes_A(\kappa\otimes_\kappa(B\otimes_A\kappa))=(M\otimes_A\kappa)\otimes_\kappa(B\otimes_A\kappa)\]
therefore $M\otimes_AB\neq 0$.
\end{proof}
\begin{proposition}\label{supp of module extension ring}
Let $A$, $B$ be two rings, $\phi:A\to B$ a homomorphism and $\phi^*:\Spec(B)\to\Spec(A)$ the induced map. Then for every $A$-module $M$,
\[\supp(M\otimes_AB)\sub\phi^{*-1}(\supp(M)).\]
If also $M$ is finitely generated, then the equality holds.
\end{proposition}
\begin{proof}
Let $\q$ be a prime ideal of $B$ and $\p=\phi^{-1}(\q)$. Suppose that $\q$ belongs to $\supp(M\otimes_AB)$, then
\[(M\otimes_AB)\otimes_BB_\q=M\otimes_AB_\q=(M\otimes_AA_\p)\otimes_AB_\q.\]
since the homomorphism $A\to B\to B_\q$ factors into $A\to A_\p\to B_\q$. The hypothesis $M\otimes_AB\otimes_BB_\q\neq 0$ implies therefore $M\otimes_AA_\p\neq 0$, whence the first assertion. As the homomorphism $\phi_\q:A_\p\to B_\q$ is local, the second assertion follows from \cref{module extended nonzero if nonzero}.
\end{proof}
\begin{proposition}\label{supp of module and nonzero homomorphism to A/p}
Let $A$ be a ring and $M$ a finitely generated $A$-module. For every prime ideal $\p\in\supp(M)$, there exists a non-zero $A$-homomorphism $\eta:M\to A/\p$.
\end{proposition}
\begin{proof}
Let $\p\in\supp(M)$. As $M$ is finitely generated and $M\neq 0$,
\[M_\p/\p M_\p=M_\p\otimes_A(A_\p/\p A_\p)\neq 0\]
Let $\kappa=A_\p/\p A_\p$ be the field of fractions of the integral domain $A/\p$. Since $M_\p/\p M_\p$ is a vector space over $\kappa$ which is not reduced to $0$, there exists a non-zero linear form $u:M_\p/\p M_\p\to \kappa$. If $x_1,\dots,x_n$ is a system of generators of $M$, and $\bar{x}_i$ is the image of $x_i$ in the $(A/\p)$-module $M_\p/\p M_\p$. Let $\alpha\in A/\p$ be such that $\alpha u(\bar{x}_i)\in A/\p$ for $1\leq i\leq n$, then $v=\alpha u$ is a non-zero $(A/\p)$-linear map from $M_\p/\p M_\p$ to $A/\p$. The composition
\[\begin{tikzcd}
M\ar[r]&M_\p\ar[r]&M_\p/\p M_\p\ar[r,"v"]&A/\p
\end{tikzcd}\] 
is therefore the required homomorphism.
\end{proof}
\subsection{Clopen subsets of the spectrum}
\begin{lemma}\label{Spec of ring idempotent prop}
Let $A$ be a ring. Let $e\in A$ be an idempotent. In this case
\[\Spec(A)=D(e)\amalg D(1-e).\]
\end{lemma}
\begin{proof}
Note that an idempotent $e$ of a domain is either $1$ or $0$. Hence we see that
\begin{align*}
D(e)&=\{\p\in\Spec(A):e\neq 0\text{ in }A/\p\}=\{\p\in\Spec(A):e=1\text{ in }A/\p\}.
\end{align*}
Similarly we have
\begin{align*}
D(1-e)&=\{\p\in\Spec(A):e\neq 1\text{ in }A/\p\}=\{\p\in\Spec(A):e=0\text{ in }A/\p\}.
\end{align*}
Since the image of $e$ is either $1$ or $0$ we deduce that $D(e)$ and $D(1-e)$ cover all of $\Spec(A)$.
\end{proof}
\begin{lemma}\label{Spec of ring product of rings}
Let $A_1$ and $A_2$ be rings. Let $A=A_1\times A_2$. The maps $A\to A_1,(x,y)\mapsto x$ and $A\to A_2,(x,y)\mapsto y$ induce continuous maps $\Spec(A_1)\to\Spec(A)$ and $\Spec(A_2)\to\Spec(A)$. The induced map
\[\Spec(A_1)\amalg\Spec(A_2)\to\Spec(A)\]
is a homeomorphism. In other words, the spectrum of $A=A_1\times A_2$ is the disjoint union of the spectrum of $A_1$ and the spectrum of $A_2$.
\end{lemma}
\begin{proof}
Write $1=e_1+e_2$ with $e_1=(1,0)$ and $e_2=(0,1)$. Note that $e_1$ and $e_2=1-e_1$ are idempotents. Thus $\Spec(A)=D(e_1)\amalg D(e_2)$ by the previous lemma. Now consider the localization $A_{e_1}$: $\{0\}\times A_2$ is mapped to zero in it, so $A_1=A_{e_1}$. Similarly we have $A_{e_2}=A_2$. Thus by \cref{Spec of ring induced map on spec(A_f)} $D(e_1)=\Spec(A_1),D(e_2)=\Spec(A_2)$.
\end{proof}
We already know that an idempotent element gives an open and closed subset of $\Spec(A)$. We now prove a converse.
\begin{proposition}\label{Spec of ring clopen subsets char}
Let $A$ be a ring. For each $U\sub\Spec(A)$ which is open and closed there exists a unique idempotent $e\in A$ such that $U=D(e)$. This induces a one-to-one correspondence between open and closed subsets $U\sub\Spec(A)$ and idempotents $e\in A$.
\end{proposition}
\begin{proof}
Let $U\sub\Spec(A)$ be open and closed, and $V$ be its complement. We can write $U$ and $V$ as unions of standard opens such that \[U=\bigcup_{i=1}^{n}D(f_i)\And W=\bigcup_{j=1}^{m}D(g_j).\]
Since $\Spec(A)=U\amalg V$, we observe that the collection $\{f_i,g_j\}$ must generate the unit ideal in $A$ by \cref{Spec of ring basic open subset prop}. So the following sequence is exact
\[\begin{tikzcd}[column sep=25pt]
0\ar[r]&A\ar[r,"\alpha"]&\bigoplus_{i=1}^{n}A_{f_i}\oplus\bigoplus_{j=1}^{m}A_{g_j}\ar[r,"\beta"]&\bigoplus_{i_1,i_2}A_{f_{i_1}f_{i_2}}\oplus\bigoplus_{i,j}A_{f_ig_j}\oplus\bigoplus_{j_1,j_2}A_{g_{j_1}g_{j_2}}
\end{tikzcd}\]
However, notice that for any pair $i,j$, $D(f_ig_j)=D(f_i)\cap D(g_j)=\emp$ since $D(f_i)\sub U$ and $D(g_j)\sub V$. Therefore by \cref{Spec of ring induced map on spec(A_f)} we see $\Spec(A_{f_ig_j})=D(f_ig_j)=\emp$, which implies $A_{f_ig_j}$ is the zero ring for each pair $i,j$. Consider the element $(1,\dots,1,0,\dots,0)\in\bigoplus_{i=1}^{n}A_{f_i}\oplus\bigoplus_{j=1}^{m}A_{g_j}$ whose coordinates are $1$ in each $A_{f_i}$ and $0$ in each $A_{g_j}$. It is sent to $0$ under the map
\[\beta:\bigoplus_{i=1}^{n}A_{f_i}\oplus\bigoplus_{j=1}^{m}A_{g_j}\to\bigoplus_{i_1,i_2}A_{f_{i_1}f_{i_2}}\oplus\bigoplus_{j_1.j_2}A_{g_{j_1}g_{j_2}}\]
so by the exactness of the sequence, there must be some element of $A$ whose image under $\alpha$ is $(1,\dots,1,0,\dots,0)$, which we denote by $e$. We see that $\alpha(e^2)=\alpha^2(e)=(1,\dots,1,0,\dots,0)=\alpha(e)$. Since $\alpha$ is injective, $e=e^2$ in $A$ and so $e$ is an idempotent of $A$. We claim that $U=D(e)$. Notice that for any $j$, the map $A\mapsto A_{g_j}$ maps $e$ to $0$. Therefore there must be some positive integer $k_j$ such that $g_j^{k_j}e=0$ in $A$. Multiplying by $e$ as necessary, we see that $(g_je)^{k_j}=0$, so $g_je$ is nilpotent in $A$. By \cref{Spec of ring basic open subset prop} $D(e)\cap D(g_j)=D(eg_j)=\emp$. So since $V=\bigcup_{j=1}^{m}D(g_j)$, we get $D(e)\cap V=\emp$; and then $D(e)\sub U$. Furthermore, for any $i$, the map $A\mapsto A_{f_i}$ maps $e$ to $1$, so there must be some $l_i$ such that $f^{l_i}(e-1)=0$ in $A$. Hence $f_i^{l_i}e=f_i^{l_i}$. Suppose $\p\in\Spec(A)$ contains $e$, then $\p$ contains $f^{l_i}_ie=f_i^{l_i}$, and since $\p$ is prime, $f_i\in\p$. So $V(e)\sub V(f_i)$, implying that $D(f_i)\sub D(e)$. Therefore $U=\bigcup_{i=1}^{n}D(f_i)\sub D(e)$, and $U=D(e)$. Therefore any open and closed subset of $\Spec(A)$ is the standard open of an idempotent as desired.
\end{proof}
\begin{corollary}\label{Spec of ring connected iff}
Let $A$ be a nonzero ring. Then $\Spec(A)$ is connected if and only if $A$ has no nontrivial idempotents.
\end{corollary}
\begin{proposition}\label{Spec of ring idempotent ideal prop}
Let $A$ be a ring and $\a$ be a finitely generated ideal such that $\a=\a^2$. Then $V(\a)$ is open and closed in $\Spec(A)$, and $A/\a\cong A_e$ for some idempotent $e\in A$.
\end{proposition}
\begin{proof}
By Nakayama's Lemma there exists an element $f=1+a$, $a\in\a$ in $A$ such that $f\a=0$. It follows that $V(\a)=D(f)$ by a simple argument. Also,
$0=fa=a+a^2$, and hence 
\[f^2=(1+a)^2=1+2a+a^2=1+a=f,\]
so $f$ is an idempotent. Consider the canonical map $A\to A_f$. It is surjective since $x/f^n=x/f=xf/f^2=x/1$. Any element of $\a$ is in the kernel since $f\a=0$, and $x/1=0$ in $A_f$ implies $f^nx=fx=(1+a)x=0$, which means $x\in\a$.
\end{proof}
\subsection{Glueing properties of localization}
In this part we show that given an open covering $\Spec(A)=\bigcup_{i=1}^{n}D(f_i)$ by standard opens, and given an element $h_i\in A_{f_i}$ for each $i$ such that $h_i=h_j$ as elements of $A_{f_if_j}$, then there exists a unique $h\in A$ such that the image of $h$ in $A_{f_i}$ is $h_i$. This result can be viewed as a sheaf property for $\Spec(A)$, and will play an important rule when we consider schemes.
\begin{proposition}\label{localization glue exact seq}
Let $A$ be a ring, and let $f_1,\dots,f_n\in A$ generate the unit ideal of $A$. Then the following sequence is exact:
\begin{equation}\label{localization glue exact seq-1}
\begin{tikzcd}
0\ar[r]&A\ar[r,"\alpha"]&\bigoplus_iA_{f_i}\ar[r,"\beta"]&\bigoplus_{ij}A_{f_if_j}
\end{tikzcd}
\end{equation}
where the maps $\alpha:A\to\bigoplus_iA_{f_i}$ and $\beta:\bigoplus_iA_{f_i}\to\bigoplus_{i,j}A_{f_if_j}$ are defined as
\[\alpha(x)=(x/1,\dots,x/1),\quad \beta(x_1/f_1^{r_1},\dots,x_n/f_n^{r_n})=x_i/f_i^{r_i}-x_j/f_j^{r_j}.\]
\end{proposition}
\begin{proof}
First let $x\in A$ such that $\alpha(x)=0$. This means $x/1=0$ in $A_{f_i}$ and so there exists $n_i$ for each $i$ such that
\[f_i^{n_i}x=0.\]
Since $f_1,\dots,f_n$ generate $A$, we can find $a_i\in A$ such that $1=\sum_{i=1}^{n}a_if_i$. Then for all $N\geq\sum_{i}n_i$, we have
\[1^N=\Big(\sum_{i=1}^{n}a_if_i\Big)=\sum\binom{N}{u_1,\dots,u_n}a_i^{u_1}\cdots a_n^{u_n}f_1^{u_1}\cdots f_n^{u_n}.\]
where each term has a factor of at least $f_i^{n_i}$ for some $i$. Therefore,
\[x=1^N\cdot x=\sum\binom{N}{u_1,\dots,u_n}a_i^{u_1}\cdots a_n^{u_n}f_1^{u_1}\cdots f_n^{u_n}x=0.\]
Thus, if $\alpha(x)=0$, then $x=0$ and so $\alpha$ is injective.\par
Now we check that the image of $\alpha$ equals the kernel of $\beta$. First, note that for $x\in A$ we have $\beta(\alpha(x))=0$, therefore the image of $\alpha$ is in the kernel of $\beta$. Now assume we have $x_1,\dots,x_n\in A$ such that 
\[\beta\Big(\frac{x_1}{f_1^{r_1}},\dots,\frac{x_n}{f_n^{r_n}}\Big)=0.\]
Then, for all pairs $i,j$, there exists an $n^{ij}$ such that
\[f_i^{n_{ij}}f_j^{n_{ij}}(x_if_j^{r_j}-x_jf_i^{r_i})=0.\]
Choosing $N$ so $N\geq n_{ij}$ for all $i,j$, we see that
\[f_i^{N}f_j^{N}(x_if_j^{r_j}-x_jf_i^{r_i})=0.\]
Define elements $\tilde{x}_i$ and $\tilde{f}_i$ of $A$ as follows:
\[\tilde{f}_i=f_i^{N+r_i},\quad \tilde{x}_i=f_i^Nx_i\]
so that $x_i/f_i^{r_i}=\tilde{x}_i/\tilde{f}_i$. Then we can use this to rewrite the above equation to get the following equality, for all $i,j$,
\[\tilde{f}_j\tilde{x}_i=\tilde{f}_i\tilde{x}_j.\]
Since $f_1,\dots,f_n$ generate $A$, we clearly have that $\tilde{f}_1,\dots,\tilde{f}_n$ also generate $A$. Therefore,
there exist $a_1,\dots,a_n$ in $A$ so that
\[1=\sum_{i=1}^{n}a_i\tilde{f}_i.\]
Therefore, we finally conclude that for all $i$,
\[\frac{x_i}{f_i^{r_i}}=\frac{\tilde{x}_i}{\tilde{f}_i}=\sum_{j=1}^{n}\frac{a_j\tilde{f}_j\tilde{x}_i}{\tilde{f}_i}=\sum_{j=1}^{n}\frac{a_j\tilde{f}_i\tilde{x}_j}{\tilde{f}_i}=\frac{\sum_{j=1}^{n}a_j\tilde{x}_j}{1}.\]
which means
\[\alpha\Big(\sum_{j=1}^{n}a_j\tilde{x}_j\Big)=\Big(\frac{x_1}{f_1^{r_1}},\dots,\frac{x_n}{f_n^{r_n}}\Big).\]
as required. 
\end{proof}
\begin{corollary}\label{localization glue exact seq module}
Let $A$ be a ring. Let $f_1,\dots,f_n$ be elements of $A$ generating the unit ideal. Let $M$ be an $A$-module. Then the following sequence is exact:
\begin{equation}\label{localization glue exact seq module-1}
\begin{tikzcd}
0\ar[r]&M\ar[r,"\alpha"]&\bigoplus_iM_{f_i}\ar[r,"\beta"]&\bigoplus_{i,j}M_{f_if_j}
\end{tikzcd}
\end{equation}
\end{corollary}
\begin{proof}
This can be proved as \cref{localization glue exact seq}, using the definition of $M_f$.
\end{proof}
\begin{proposition}\label{localization direct sum map injection iff}
Let $A$ be a ring. Let $f_1,\dots,f_n\in A$ and $M$ be an $A$-module. Then $M\to\bigoplus M_{f_i}$ is injective if and only if the homomorphism
\[\psi:M\to\bigoplus_{i=1}^{n}M,\quad m\mapsto(f_1m,\dots,f_nm)\]
is injective.
\end{proposition}
\begin{proof}
The map $M\to\bigoplus_{i=1}^{n}M_{f_i}$ is injective if and only if for all $m\in M$ and $e_1,\dots,e_n\geq1$ such that $f_i^{e_i}m=0$, $1\leq i\leq n$ we have $m=0$. This clearly implies that $\psi$ is injective. Conversely, suppose $\psi$ is injective and $m\in M$ and $e_1,\dots,e_n\geq1$ are such that $f_i^{e_i}m=0$. Let $N\geq e_i$ for all $i$, then $f_i^Nm=0$ for all $i$, which is $\psi^N(m)=0$. Since $\psi$ is injective, this implies $m=0$.
\end{proof}
\begin{proposition}\label{Spec finite standard open cov glue module property}
Let $(f_i)_{i\in I}$ be a finite family of elements of a ring $A$ generating the unit ideal of $A$. Then the ring $B=\prod_{i\in I}A_{f_i}$ is a faifully flat $A$-module.
\end{proposition}
\begin{proof}
We know each of the $A_{f_i}$ is a flat $A$-module, hence so is $B$. On the other hand, if $\p$ is a prime ideal of $A$, there exists an index $i$ such that $f_i\notin\p$ and $\p_f=\p A_f$ is therefore a prime ideal of $A_f$. Then $\p B\sub\p A_{f_i}\times \prod_{j\neq i}A_{f_j}\neq B$ since $\p A_{f_i}\neq A_{f_i}$. This suffices to imply that $B$ is a faithfully flat $A$-module.
\end{proof}
\begin{proposition}\label{Spec finite standard open cov glue property}
Let $(f_i)_{i\in I}$ be a finite family of elements of a ring $A$ generating the unit ideal of $A$.
\begin{itemize}
\item[(a)] For an $A$-module $M$ to be zero, it is necessary and sufficient that, for every index $i$, the $A_{f_i}$-module $M_{f_i}$ is zero.
\item[(b)] For an $A$-module $M$ to be finitely generated (resp. finitely presented), it is necessary and sufficient that, for every index $i$, the $A_{f_i}$-module $M_{f_i}$ be finitely generated (resp. finitely presented).
\item[(c)] Let $\phi:M\to N$ be a homomorphism of $A$-modules. Then $\phi$ is injective (resp. surjective) if and only if for every index $i$, the homomorphism $\phi_{f_i}:M_{f_i}\to N_{f_i}$ is injective (resp. surjective). 
\end{itemize}
\end{proposition}
\begin{proof}
If $M=0$ then clearly $M_{f_i}=0$ for all $i$. Conversely, if each $M_{f_i}$ is zero, then the $A$-module $P=\bigoplus_{i\in I}M_{f_i}$ is zero. But $P=M\otimes_A\prod_{i\in I}A_{f_i}$, so the claim in (a) follows from \cref{Spec finite standard open cov glue module property}. Also, (c) follows immediately from (a).\par
Now if $M$ is finitely generated (resp. finitely presented) then clearly every $M_{f_i}$ is finitely generated (resp. finitely presented). Conversely, if all the $M_{f_i}$ are finitely generated (resp. finitely presented), then $P$ is a finitely generated (resp. finitely presented) $B$-module (for we can obviously suppose that for each $i$ there is an exact sequence $A_f^m\to A_f^n\to M_{f_i}\to 0$, where $m$ and $n$ are independent of $i$). Again by \cref{Spec finite standard open cov glue module property}, we see $M$ is finitely generated (resp. finitely presented).
\end{proof}
\begin{corollary}\label{Spec finite standard open cov glue Noe property}
Let $(f_i)_{i\in I}$ be a finite family of elements of a ring $A$ generating the unit ideal of $A$. Then $A$ is Noetherian if and only if for every index $i\in I$, then ring $A_{f_i}$ is Noetherian.
\end{corollary}
\begin{proof}
This follows from \cref{Spec finite standard open cov glue property} by applying to ideals of $A$.
\end{proof}
\begin{corollary}\label{Spec finite standard open cov glue algebra property}
Let $(f_i)_{i\in I}$ be a finite family of elements of a ring $A$ generating the unit ideal of $A$. For an $A$-algebra $B$ to be of finite type (resp. finitely presented) over $A$, it is necessary and sufficient that, for every index $i$, the $A_{f_i}$-algebra $B_{f_i}$ be of finite type (resp. finitely presented) over $A_{f_i}$ .
\end{corollary}
\begin{proof}
One direction is clear. Conversely, assume that each $B_{f_i}$ is of finite type (finite presented) over $A_{f_i}$. Since $A_{f_i}$ is finitely presented, we then see each $B_{f_i}$ is of finite type (finite presented) over $A$. For each $i\in I$ take a finite generating set $S_i$ of $B_{f_i}$. Without loss of generality, we may assume that the elements of $S_i$ are in the image of the localization map $B\to B_{f_i}$, so we take a finite set $T_i$ of preimages of the elements of $T_i$ in $B$. Let $T$ be the union of these sets. This is still a finite set. Consider the algebra homomorphism $A[X_t]_{t\in T}\to B$ induced by $T$. Since it is an algebra homomorphism, the image $B'$ is an $A$-submodule of the $A$-module $B$, so we can consider the quotient module $B/B'$. Now it is clear that \cref{Spec finite standard open cov glue property} implies both of the claims (note that $\{f_i\}$ also generates the unit ideal in $A[X_t]$).
\end{proof}
\begin{corollary}\label{Spec finite standard open cov on domain glue property}
Let $\rho:A\to B$ be a ring map and $\{g_i\}_{i\in I}$ elements of $B$ generating the unit ideal of $B$. Then for $B$ to be of finite type (finite presented) over $A$, it is necessary and sufficient that for every index $i\in I$, the algebra $B_{g_i}$ is of finite type (finite presented) over $A$.
\end{corollary}
\begin{proof}
Choose $h_i$ in $B$ such that $1=\sum_ig_ih_i$. We first prove that, if $B_{g_i}$ is of finite type, then $B$ is of finite type. For this, for each $i\in I$ choose a finite subset $S_i$ which generate $B_{g_i}$ as an $A$-algebra and let $T_i$ be its preimage in $B$ (as in the proof of \cref{Spec finite standard open cov glue algebra property}). Consider the $A$-subalgebra $B'\sub B$ generated by $\{g_i\}$, $\{h_i\}$ and $T_i$ for all $i\in I$. Since localization is exact, we see that $B'_{g_i}\to B_{g_i}$ is injective. On the other hand, it is surjective by our choice of $T_i$. The elements $\{g_i\}$ generate the unit ideal in $B'$ as $h_i\in B'$, so $B'\to B$ viewed as an $B'$-module homomorphism is an isomorphism by \cref{Spec finite standard open cov glue property}.\par
Now we consider the case for finite presentation. Assume that each $B_{g_i}$ is finitely presented. We already know that $B$ is of finite type, so we may identify $B$ with $A[x_1,\dots,x_n]/\a$ for some ideal $\a$. Moreover, we may choose elements $\tilde{g}_i,\tilde{h}_i\in A[x_1,\dots,x_n]$ whose image in $B$ is $g_i$ and $h_i$. Then we see 
\[B_{g_i}=A[x_1,\dots,x_n,y_i]/(\a_i+(1-y_i\tilde{g}_i)),\]
where $\a_i$ is the ideal of $A[x_1,\dots,x_n,y_i]$ generated by $\a$. By \cref{module ft exact sequence kernel finite}, we may choose a finite list of elements $f_{ij}\in\a$ such that the image of $f_{ij}$ in $\a_i+(1-y_i\tilde{g}_i)$ generate the ideal $\a_i+(1-y_i\tilde{g}_i)$. If we set
\[B'=A[x_1,\dots,x_n]/(\sum_i\tilde{g}_i\tilde{h}_i-1,f_{ij}),\]
there is then a surjective $A$-algebra map $B'\to B$. The classes of the elements $\tilde{g}_i$ in $B'$ generate the unit ideal and by construction the maps $B'_{g_i}\to B_{g_i}$ are injective, so we conclude that $B'\to B$ is injective, hence an isomorphism.
\end{proof}
\begin{proposition}\label{localization of quotient ring finite presentation}
Let $\a$ be an ideal of $A$, $f\in A$ such that $A_f/\a_f$ is an $A$-algebra of finite presentation. Then $\a_f$ is finitely generated in $A_f$.
\end{proposition}
\begin{proof}
By hypothesis, there exists a polynomial algebra $B=A[T_1,\dots,T_n]$ and a surjective $A$-homomorphism $\rho:B\to A_f/\a_f$, whose kernel $\b$ is finitely generated. Let $\pi:A_f\to A_f/\a_f$ be the canonical homomorphism; for each $1\leq i\leq n$, there exists $t_i\in A$ such that $\pi(t_i/s^k)=\rho(T_i)$ (we can choose an integer $k$ for all index $i$). Consider then the $A_f$-algebra $C=A_f[T_1,\dots,T_n]$ and the homomorphism $\varphi:C\to A_f$ such that $\varphi(T_i)=t_i/s^k$. The homomorphism $\varphi$ is clearly surjective, and so is the composition $\rho'=\pi\circ\varphi:C\to A_f/\a_f$. 
\[\begin{tikzcd}
&B=A[T_1,\dots,T_n]\ar[d,"\rho"]\\
A_f\ar[r,"\pi"]&A_f/\a_f\\
&C=A_f[T_1,\dots,T_n]\ar[lu,"\varphi"]\ar[u,swap,"\rho'"]
\end{tikzcd}\]

On ther other hand, any polynomial in $C$ is of the form $P/s^m$, where $P\in B$, and we have $\rho'(P/s^m)=(1/s^m)\rho(P)$. As $1/s$ is invertible in $A_f$, the relation $\rho'(P/s^m)=0$ in $A_f/\a_f$ is equivalent to $\rho(P)=0$, and therefore the kernel $\b'$ of $\rho'$ is generated by the image of $\b$ in $C$, and a fortiori is finitely generated in $C$. As $\b'=\varphi^{-1}(\pi^{-1}(0))=\varphi^{-1}(\a_f)$ and $\varphi$ is surjective, we see $\a_f=\varphi(\b')$, so $\a_f$ is finitely generated in $A_f$.
\end{proof}
\begin{remark}\label{algebra finite presentation surjection from ft algebra ker finitely generated}
If $B$ is an $A$-algebra of finite presentation, $B'$ is an $A$-algebra of finite type, and $\rho:B'\to B$ is a surjective homomorphism, then the kernel $\b$ of $\rho$ is finitely generated in $B'$. In fact, there exists a surjective homomorphism $\psi:C\to B'$, where $C$ is a polynomial algebra over $A$; as $\rho\circ\psi$ is surjective, $\ker(\rho\circ\psi)$ is a finitely generated ideal in $C$ by \cref{localization of quotient ring finite presentation}. As we have $\b=\psi(\ker(\rho\circ\psi))$, $\b$ is then finitely generated in $B'$.\par
Conversely, if $B'$ is an $A$-algebra of finite presentation, $\rho:B'\to B$ is a surjective homomorphism, and if $\b=\ker\rho$ is finitely generated in $B'$, then $B$ is an $A$-algebra of finite presentation. In fact, we have a surjective homomorphism $\psi:C\to B'$ where $C$ is a polynomial algebra and $\ker\psi$ is finitely generated in $C$. As $\ker(\rho\circ\psi)$ is generated by $\ker\psi$ and a system of elements whose image in $B'$ generate $\b$, it is finitely generated and therefore $B$ is an $A$-algebra of finite presentation.
\end{remark}
\section{Finitely generated projective modules}
\subsection{Local characterization}
\begin{proposition}\label{localization map lift to principal open}
Let $A$ be a ring, $\phi:M\to N$ an $A$-module homomorphism and $\p$ a prime ideal of $A$.
\begin{itemize}
\item[(a)] Suppose that $\phi_\p:M_\p\to N_\p$ is suriective and that $N$ is finitely generated. Then there exists $f\in A-\p$ such that $\phi_f:M_f\to N_f$ is surjective.
\item[(b)] Suppose that $\phi_\p:M_\p\to N_\p$ is bijectioe, that $M$ is finitely generated and that $N$ is finitely presented. Then there exists $f\in A-\p$ such that $\phi_f:M_f\to N_f$ is bijective.
\end{itemize}
\end{proposition}
\begin{proof}
Let $P$ and $Q$ be the kernel and cokernel of $\phi$. If $f\in A$, the kernel and cokernel of $\phi_f$ (resp. $\phi_\p$) are $P_f$ and $Q_f$ (resp. $P_\p$ and $Q_\p$). In the case (a), $\phi_\p$ is surjective so $Q_\p=0$; as $N$ is finitely generated, so is $Q$ and therefore there exists $f\in A-\p$ such that $fQ=0$, whence $Q_f=0$.\par
Under the hypotheses of (b), the sequence $0\to P_f\to M_f\to N_f\to 0$ is exact, hence $P_f$ is finitely generated. Now, $(P_f)_{\p P_f}=P_\p=0$, hence there exists $g/f^d\in A_f-\p A_f$ (whence $g\in A-\p$) such that $(g/f^d)P_f=0$. Then as $1/f$ is invertible, we have $(g/1)P_f=0$, whence $P_{fg}=(P_f)_{g/1}=0$. Since $fg\in A-\p$ and $Q_{fg}=0$, we get the claim.
\end{proof}
\begin{corollary}\label{localization module free at p then free at f}
If $N$ is finitely presented and $N_\p$ is a free $A_\p$-module of rank $p$, there exists $f\in A-\p$ such that $N_f$ is a free $A_f$-module of rank $p$.
\end{corollary}
\begin{proof}
There exist by hypothesis elements $x_1,\dots,x_p$ such that the $x_i/1$ form a basis of the free $A_\p$-modulc $N_\p$. Consider the homomorphism $\phi:A^p\to N$ such that $\phi(e_i)=x_i$, $(e_i)$ being the canonical basis of $A^p$. As $\phi_\p$ is bijective by hypothesis, there exists $f\in A-\p$ such that $\phi_f$ is bijective, by virtue of \cref{localization map lift to principal open}.
\end{proof}
\begin{remark}
In the language of algebraic geometry, Corallary~\ref{localization module free at p then free at f} shows that if $\mathscr{F}$ is a coherent sheaf over a scheme $X$ such that $\mathscr{F}|_x$ is free of rank $n$ for some point $x\in X$, then there exist a neighborhood $U$ of $x$ such that $\mathscr{F}|_U$ is free of rank $n$.
\end{remark}
\begin{corollary}\label{module finite local free rank upper semicontinuous}
Let $N$ be a finitely generated $A$-module and consider the function $\phi$ on $X=\Spec(A)$ defined by
\[\phi(\p)=\dim_{\kappa(\p)}(N_\p\otimes_{A_\p}\kappa(\p)).\]
Then $\phi$ is upper semi-continuous, i.e., for any $n\in\Z$, the set $\{\p\in X:\phi(\p)<n\}$ is open.
\end{corollary}
\begin{proof}
Note that $\phi(\p)=[N_\p/\p N_\p:\kappa(\p)]$. By Nakayama's Lemma, this number is equal to the minimal number of generators of $N_\p$ as an $A_\p$ module. Let $m_1,\dots,m_r$ be such a minimal generating set with $r<n$. Write $m_i=x_i/f$ for $x_i\in N$ and $f\in A-\p$, then we get a map $\phi:A_f^r\to N_f$ defined by $\phi(e_i/1)=x_i/f$, such that $\phi_\p$ is surjective. By \cref{localization map lift to principal open} there exist some $g\in A-\p$ such that $\phi_g:A_{fg}^r\to N_{fg}$ is surjective. Then for any $\q\in D(fg)$, by localizing we see $\phi_\q:A_\q^r\to N_\q$ is also surjective, whence $\phi(\q)\leq r<n$. Therefore the set $\{\p\in X:\phi(\p)<n\}$ is open.
\end{proof}
\begin{theorem}\label{module finite projective iff}
Let $A$ be a ring and $P$ an $A$-module. Then the following properties are equivalent:
\begin{itemize}
\item[(\rmnum{1})] $P$ is a finitely generated projective module.
\item[(\rmnum{2})] $P$ is a finitely presented module and, for every maximal ideal $\m$ of $A$, $P_\m$ is a free $A_\m$-module.
\item[(\rmnum{3})] $P$ is a finitely generated module, for all $\p\in\Spec(A)$, the $A_\p$-module $P_\p$ is free and, if we denote its rank by $\rank_\p$, the function $\p\mapsto\rank_\p$ is locally constant in the topological space $\Spec(A)$.
\item[(\rmnum{4})] There exists a finite family $(f_i)_{i\in I}$ of elements of $A$, generating the ideal $A$, such that for all $i\in I$, the $A_{f_i}$-module $P_{f_i}$ is free of finite rank.
\item[(\rmnum{5})] For every maximal ideal $\m$ of $A$, there exists $f\in A-\m$ such that $P_\m$ is free of finite rank.
\end{itemize}
\end{theorem}
\begin{proof}
By \cref{module fp prop}, we know that a finitely generated projective module is finitely presented. If $P$ is a projective $A$-module, $P_\m=P\otimes_AA_\m$ is a projective $A_\m$-module. Finally, as $A_\m$ is a local ring, every finitely presented projective $A_\m$-module is free (\cref{local ring fp module free iff flat iff projective}). This shows (\rmnum{1}) implies (\rmnum{2}). Also, (\rmnum{2}) implies (\rmnum{5}) by \cref{localization module free at p then free at f}.\par
Now assume (\rmnum{5}), and let $E$ be the set of $f\in A$ such that $P_f$ is a finitely generated free $A_f$-module. The hypothesis implies that $E$ is contained in no maximal ideal of $A$, hence $E$ generates the ideal $A$ and there therefore exist a finite family $(f_i)_{i\in I}$ of elements of $E$ that generates $A$, whence (\rmnum{4}).\par
Assume (\rmnum{4}) and consider the ring $B=\prod_{i\in I}A_{f_i}$ and the $B$-module $M=\bigoplus_{i\in I}P_{f_i}=P\otimes_AB$. For every index $i$, there exists a free $A_{f_i}$-module $L_i$ such that $P_i$ is a direct factor of $L_i$ and it may be assumed that all the $L_i$ have the same rank; then $L=\bigoplus_{i\in I}L_i$ is a free $B$-module of which $M$ is a direct factor, in other words $M$ is a finitely generated projective $B$-module. As $B$ is a faithfully flat $A$-module, we conclude that $P$ is a finitely generated projective $A$-module by \cref{module extension to faithfully flat ring finiteness iff}.\par
Finally, we show that (\rmnum{4}) implies (\rmnum{3}) and (\rmnum{3}) implies (\rmnum{5}). If (\rmnum{4}) holds, then it follows from \cref{Spec finite standard open cov glue property} that $P$ is finitely generated. On the other hand, for every prime ideal $\p$ of $A$, there exists an index $i$ such that $f_i\notin\p$; then $P_{\p}=(P_{f_i})_{\p_{f_i}}$ and hence by hypothesis $P_\p$ is free and of the same rank as $P_{f_i}$, which proves (\rmnum{3}).\par
Assume (\rmnum{3}), let $\m$ be a maximal ideal of $A$; write $\rank_\m=n$ and let $x_1,\dots,x_n$ be a basis of $P_\m$. We can assume that the $x_i$ are canonical images of elements $p_i\in P$ to within multiplication by an invertible element of $A_\m$. Let $(e_i)$ be the canonical basis of $A^n$ and $\phi:A^n\to P$ the homomorphism such that $\phi(e_i)=p_i$. As $P$ is finitely generated, it follows from \cref{localization map lift to principal open} that there exists $f\in A-\m$ such that $\phi_f$ is surjective. We conclude that $\phi_{fg}$ is also surjective for all $g\in A-\m$ and by hypothesis there exists $g\in A-\m$ such that $\rank_{\p}=n$ for $\p\in D(g)$. Then, replacing $f$ by $fg$, we may assume that $\rank_\p=n$ for all $\p\in D(f)$. Then $\phi_\p:A_\p^n\to P_\p$ is a surjective homomorphism and $P_\p$ and $A_\p^n$ are both free $A_\p$-modules of the same rank; hence by \cref{local ring finite free module same rank iff} $\phi_\p$ is bijective for all $\p\in D(f)$. Let $\p'$ be a prime ideal of $A_f$ and let $\p$ be its inverse image in $A$ under the canonical map; if $(A_f^n)_{\p'}$ and $(P_f)_{\p'}$ are identified with $A_\p^n$ and $P_\p$ under the canonical isomorphisms, $(\phi_f)_{\p'}$ is then identified with $\phi_\p$ and is consequently bijective. We conclude that $\phi$ is bijective, which establishes (\rmnum{5}).
\end{proof}
\begin{corollary}
Suppose that the equivalent properties of the statement of \cref{module finite projective iff} hold. Let $n$ be a positive integer such that, for every family $(x_1,\dots,x_n)$ of elements of $P$, there exists a family $(a_1,\dots,a_n)$ of elements of $A$, which are not all divisors of zero and for which $\sum_{i=1}^{n}a_ix_i=0$. Then, for all $\p\in\Spec(A)$, $\rank_\p\leq n$.
\end{corollary}
\begin{proof}
Let $\p$ be a prime ideal of $A$, set $\rank_\p=r$ and let $(y_1,\dots,y_r)$ be a basis of the free $A_\p$-module $P_\p$. There exist elements $(x_1,\dots,x_r)$ of $P$ and $s\in A-\p$ such that $y_i=x_i/s$ for all $i$. If $r>n$, then there exist a family $(a_1,\dots,a_r)$ of elements of $A$ such that $\sum_{i=1}^{r}a_ix_i=0$, which are not all divisors of zero (for example take $(a_1,\dots,a_n)$ as in the hypothesis and let $a_i=0$ for $i>n$). But then we have $\sum_{i=1}^{r}(a_i/1)y_i=0$ and therefore $a_i=0$ in $A_\p$, which implies $a_i$ is a divisor of zero, contradiction.
\end{proof}
\begin{corollary}\label{module fp flat is projective}
Every finitely presented flat module is projective.
\end{corollary}
\begin{proof}
If $P$ is a finitely presented flat $A$-module and $\m$ a maximal ideal of $A$, the $A_\m$-module $P_\m$ is flat and finitely presented and hence free (\cref{local ring fp module free iff flat iff projective}). Condition (\rmnum{2}) of \cref{module finite projective iff} therefore holds.
\end{proof}
\subsection{The rank function}
\begin{definition}
Let $P$ be a finitely generated projective $A$-module. For every prime ideal $\p$ of $A$, the rank of the free $A_\p$-module $P$, is called the \textbf{rank} of $P$ at $\p$ and is denoted by $\rank_\p(P)$.
\end{definition}
By \cref{module finite projective iff} the integer-valued function $\p\mapsto\rank_\p(P)$ is locally constant on $X=\Spec(A)$. It is therefore constant if $X$ is connected and in particular if the ring $A$ is an integral domain.
\begin{definition}
Let $n$ be a positive integer. A projective $A$-module $P$ is said to be \textbf{of rank $\bm{n}$} if it is finitely generated and $\rank_\p(P)=n$ for every prime ideal $\p$ of $A$.
\end{definition}
Clearly every finitely generated flee $A$-module $L$ is of rank $n$ in the sense of the above definition, $n$ being equal to the dimension (or rank) of $L$. A projective module of rank $0$ is zero. If $A$ is not reduced to $0$ and a projective $A$-module $P$ is of rank $n$, the integer $n$ is determined uniquely; it is then denoted by $\rank(P)$.
\begin{theorem}\label{projective module rank n iff}
Let $P$ be an $A$-module and $n$ a psotive integer. The following properties are equivalent:
\begin{itemize}
\item[(\rmnum{1})] $P$ is projective of rank $n$.
\item[(\rmnum{2})] $P$ is finitely generated and, for every maximal ideal $\m$ of $A$, the $A_\m$-module $P_\m$ is free of rank $n$.
\item[(\rmnum{3})] $P$ is finitely generated and, for every prime ideal $\p$ of $A$, the $A_\p$-module $P_\p$ is free of rank $n$.
\item[(\rmnum{4})] For every maximal ideal $\m$ of $A$, there exists $f\in A-\m$ such that the $A_\m$-module $P_\m$ is free of rank $n$.  
\end{itemize}
\end{theorem}
\begin{proof}
By definition and \cref{module finite projective iff}, (\rmnum{1}) and (\rmnum{3}) are equivalent; (\rmnum{2}) implies (\rmnum{3}), as, for every prime ideal $\p$ of $A$, there exists a maximal ideal $\m$ containing $\p$ and, writing $\p'=\p_\m$, $P_\p$ is isomorphic to $(P_\m)_{\p'}$. If $P_\m$ is free of rank $n$, so then is $P_\p$. Property (\rmnum{3}) implies (\rmnum{1}) by virtue of \cref{module finite projective iff} and the fact that, iff $f\in A-\m$ and $\m'=\m_f$, $P_\m$ is isomorphic to $(P_f)_{\m_f}$ and therefore the ranks of $P_f$ and $P_\m$ are equal. Finally, this last argument and \cref{module finite projective iff} show that (\rmnum{4}) implies (\rmnum{2}).
\end{proof}
\begin{corollary}
If $A$ is an integral domain and $P$ is a projective $A$-module, then $\rank_\p(P)=\rank(P)$ for all prime ideal $\p$ of $A$.
\end{corollary}
\begin{proof}
It is sufficient to apply \cref{projective module rank n iff}(\rmnum{3}) with $\p=(0)$.
\end{proof}
Let $E$ and $F$ be two finitely generated projective $A$-modules. We know that $E\oplus F$, $E\otimes_AF$, $\Hom_A(E,F)$ and the dual $E^*$ of $E$ are projective and finitely generated, so is the exterior power $\bigwedge^pE$ for every integer $p>0$. Also, it follows immediately from definition and the exactness of localization that, for every prime ideal $\p$ of $A$,
\begin{gather*}
\rank_\p(E\oplus F)=\rank_\p(E)+\rank_\p(F),\quad\rank_\p(E\times F)=\rank_\p(E)\rank_\p(F),\\
\rank_\p(\Hom_A(E,F))=\rank_\p(E)\rank_\p(F),\quad\rank_\p(E^*)=\rank_\p(E),\\
\rank_\p(\bigwedge\nolimits^pE)=\binom{\rank_\p(E)}{p}.
\end{gather*}
If the ranks of $E$ and $F$ are defined, so are those of $E\oplus F$, $E\otimes_AF$, $\Hom_A(E,F)$, $E^*$ and $\bigwedge^pE$ and the above equations also hold with the index $\p$ omittcd. Moreover:
\begin{corollary}
For a finitely generated firqjective $A$-module $P$ to be of rank $n$, it is necessary and sufficient that $\bigwedge^nP$ be of rank $1$.
\end{corollary}
\begin{proposition}\label{projective module rank of extension scalar}
Let $B$ be a $A$-algebra and $P$ a projective $A$-module of rank $n$. The $B$-module $B\otimes_AP$ is then projective of rank $n$.
\end{proposition}
\begin{proof}
We know that $B\otimes_AP$ is projective and finitely generated. If $\q$ is a prime ideal of $B$ and $\p$ its inverse image in $A$, then
\[(B\otimes_AP)_\q=(P\otimes_AB)\otimes_BB_\q=P\otimes_AB_\q=(P\otimes_AA_\p)\otimes_AB_\q\]
and, as, by hypothesis, $P\otimes_AA_\p$ is a free $A_\p$-module of rank $n$, $(B\otimes_AP)_\q$ is a free $B_\q$-module of rank $n$.
\end{proof}
\begin{proposition}\label{semilocal ring projective finite module rank defined then free}
Let $A$ be a semi-local ring and $P$ a finitely generated projective $A$-module. If the rank of $P$ is defined, $P$ is a free $A$-module.
\end{proposition}
\begin{proof}
Suppose first that $A=\prod_{i=1}^{n}K_i$ where $K_i$ are fields. Then the $K_i$ are then identified with the minimal ideals of $A$ and, for all $i$, the sum $\p_i=\sum_{j\neq i}K_j$ is a maximal ideal of $A$, and the $\p_i$'s are the only prime ideals of $A$. Every finitely generated $A$-module $P$ is therefore the direct sum of its isotypical components $P_i$, $P_i$ being isomorphic to a direct sum of a finite number $r$, of $A$-modules isomorphic to $K_i$ (A, \Rmnum{8}, $\S$5, no.1, Proposition 1 and no.3, Proposition 11); the ring $A_{\p_i}$ is identified with $K_i$ and annihilates the $P_j$ of index $j\neq i$, hence $n=\rank_{\p_i}(P)$; if all the $r_i$ are equal to the same number $r$, $P$ is isomorphic to $A^r$, whence the proposition in this case. In the general case, let $\r$ be the Jacobson radical of $A$ and $B=A/\r$; as $B$ is a product of fields, the projective $B$-module $P\otimes_AB$ is free by the remarks preceding \cref{projective module rank of extension scalar}. Also $P$ is a flat $A$-module and the proposition then follows from \cref{local ring module M/mM is free then M free}.
\end{proof}
\subsection{Projective modules of rank \boldmath\texorpdfstring{$1$}{1}}
\begin{theorem}\label{projective module rank 1 iff}
Let $A$ be a ring and $M$ a finitely generated $A$-module.
\begin{itemize}
\item[(a)] If there exists an $A$-module $N$ such that $M\otimes_AN$ is isomorphic to $A$, the module $M$ is projective of rank $1$.
\item[(b)] Conversely, if $M$ is projective of rank $1$ and $M^*$ is the dual of $M$, the canonical homomorphism $v:M^*\otimes_AM\to A$ corresponding to the canonical bilinear form $(x^*,x)\mapsto\langle x^*,x\rangle$ on $M^*\times M$ is bijective.
\end{itemize}
\end{theorem}
\begin{proof}
First assume that $M\otimes_AN\cong A$ for some $A$-module $N$. It is required to prove that, for every maximal ideal $\m$ of $A$, the $A_\m$-module $M_\m$ is free of rank $1$. We are free to replace $A$ by $A_\m$ and hence may assume that $A$ is a local ring. Let $k=A/\m$. The isomorphism $\eta:M\otimes_AN\to A$ defines an isomorphism
\[\eta\otimes 1_k:(M/\m M)\otimes_k(N/\m N)\to k\]
as the rank over $k$ of $(M/\m M)\otimes_k(N/\m N)$ is the product of the ranks of $M/\m M$ and $N/\m N$, these latter are necessarily equal to $1$, in other words $M/\m M$ is monogenous. It follows that $M$ is monogenous by \cref{local ring module generating set of M/mM}. On the other hand, the annihilator of $M$ also annihilates $M\otimes_AN$ and hence is zero, which proves that $M$ is isomorphic to $A$.\par
For (b), assume that $M$ is projective of rank $1$. It is sufficient to prove that, for every maximal ideal $\m$ of $A$, $v_\m$ is an isomorphism. As $M$ is finitely presented, $(M^*)_\m$ is canonically identified with the dual $(M_\m)^*$ (\cref{localization and Hom set if finite presented}) and, as $M_\m$ is free of rank $1$ like its dual $(M_\m)^*$, clearly the canonical homomorphism $v_\m:(M_\m)^*\otimes_{A_\m}(M_\m)\to A_\m$ is bijcctive, which completes the proof.
\end{proof}
\begin{remark}
If $M$ is projective of rank $1$ and $N$ is such that $M\otimes_AN$ is isomorphic to $A$, then $N$ is isomorphic to $M^*$, for there are isomorphisms
\[N\to N\otimes_AA\to N\otimes M\otimes M^*\to A\otimes M^*\to M^*.\]
Now let $A^n\to M$ be a surjective homomorphism. Then by applying the dual functor $\Hom(-,A)$ we get an injection $N\cong M^*\to (A^n)^*\cong A^n$. If we further assume that $A$ is Noetherian, then $N$ is a finitely generated $A$-module.\par
In the language of algebraic geometry, \cref{projective module rank 1 iff} and the argument above show that a coherent sheaf $\mathscr{F}$ on a Noetherian scheme $X$ is invertible (locall of rank $1$) if and only if there exist a coherent sheaf $\mathscr{G}$ such that $\mathscr{F}\otimes_{\mathscr{O}_X}\mathscr{G}\cong\mathscr{O}_X$. This justifies the terminology invertible: it means that $\mathscr{F}$ is an invertible element of the monoid of coherent sheaves under the operation $\otimes$.
\end{remark}
\begin{proposition}
Let $M$ and $N$ be projective $A$-modules of rank $1$. Then $M\otimes_AN$, $\Hom_A(M,N)$ and the dual $M^*$ of $M$ are projective of rank $1$.
\end{proposition}
Let us now note that every finitely generated $A$-module is isomorphic to a quotient module of $L=A^n$. We may therefore speak of the set $F(A)$ of classes of finitely generated $A$-modules with respect to isomorphism. We denote by $P(A)$ the subset of $F(A)$ consisting of the classes of projective $A$-modules of rank $1$ and by $\cl(M)$ the image in $P(A)$ of a projective $A$-module $M$ of rank $1$. It is immediate that, for two projective $A$-modules $M$, $N$ of rank $1$, $\cl(M\otimes_AN)$ depends only on $\cl(M)$ and $\cl(N)$. As definition we set
\begin{align}\label{projective module cl operation def}
\cl(M)+\cl(N)=\cl(M\otimes_AN)
\end{align}
and an internal law of composition is thus defined on $P(A)$.
\begin{proposition}
The set $P(A)$ of classes of projective $A$-modules of rank $1$, with the law of composition $(\ref{projective module cl operation def})$, is a commutative group. If $M$ is a projective $A$-module of rank $1$ and $M^*$ is its dual, then
\[\cl(M^*)=-\cl(M),\quad \cl(A)=0.\]
\end{proposition}
\begin{proof}
The associativity and commutativity of the tensor product show that the law of composition (6) is associative and commutative. The isomorphism between $A\otimes_AM$ and $M$ prove that $\cl(A)$ is the identity element under this law and, by virtue of \cref{projective module rank 1 iff}, $\cl(M)+\cl(M^*)=\cl(A)$, whence the proposition.
\end{proof}
Let $B$ be an $A$-algebra and $M$ a projective $A$-module of rank $1$. Then $B\otimes_AM$ is a projective $B$-module of rank $1$ by \cref{projective module rank of extension scalar}. Then there exists a map called canonical $\phi:P(A)\to P(B)$ such that
\[\phi(\cl(M))=\cl(B\otimes_AM).\]
The equation $(M\otimes_AB)\otimes_B(N\otimes_AB)=(M\otimes_AN)\otimes_AB$ for two $A$-modules $M$, $N$ proves that the map $\phi$ is a group homomorphism.
\begin{remark}
Condition (\rmnum{5}) of \cref{module finite projective iff} (equivalent to the fact that $P$ is projective and finitely generated) may also be expressed by saying that the sheaf of modules $P$ over $X=\Spec(A)$ associated with $P$ is locally free and of finite type and may therefore be interpreted as the sheaf of sections of a vector bundle over $X$. Conversely, every vector bundle over $X$ arises from a finitely generated projective module, which is determined to within a unique isomorphism. The projective modules of rank $n$ thus correspond to the vector bundles all of whose fibres have dimension $n$. In particular, the vector bundles of rank $1$ correspond to the projective modules of rank $1$. If we denote by $\mathscr{O}_X$ the structure sheaf of $X$ and by $\mathscr{O}_X^\times$ the sheaf of units of $\mathscr{O}_X$ (whose sections over an open set $U$ of $X$ are the invertible elements of the ring of sections of $\mathscr{O}_X$ over $U$), it follows that the group $P(A)$ is isomorphic to the first cohomology group $H^1(X,\mathscr{O}_X^\times)$.
\end{remark}
\subsection{Non-degenerate submodules and invertible submodules}
In this part and the two following, $A$ denotes a ring, $S$ a multiplicative subset of $A$ consisting of elements which are not divisors of zero in $A$, and $B$ the ring $S^{-1}A$. The ring $A$ is then canonically identified with a subring of $B$. The elements of $S$ are therefore invertible in $B$. One of the most important special cases for applications is that where $A$ is an integral domain and $S$ is the set of nonzero elements of $A$; in this case $B$ is the field of fractions of $A$.\par
Let $M$ be a sub-$A$-module of $B$. Then $M$ is called \textbf{non-degenerate} if $B\cdot M=B$. Note that if $B$ is a field, this condition simply means that $M$ is not reduced to $0$.
\begin{proposition}\label{nondegenerate submodule iff}
Let $M$ be a sub-$A$-module of $B$. The following conditions are equivalent:
\begin{itemize}
\item[(\rmnum{1})] $M$ is non-degenerate;
\item[(\rmnum{2})] $M$ meets $S$;
\item[(\rmnum{3})] If $i_M:M\to B$ is the canonical injection, then $S^{-1}i_M:S^{-1}M\to B$ is bijective.
\end{itemize}
\end{proposition}
\begin{proof}
Part (\rmnum{1}) implies (\rmnum{2}), for if $B\cdot M=B$, there exists $a\in A$, $s\in S$ and $x\in M$ such that $(a/s)x=1$, hence $ax=s$ belongs to $S\cap M$. To see that (\rmnum{2}) implies (\rmnum{3}), note that $S^{-1}i$ is already injective, since localization is a exact functor. Moreover, if $x\in M\cap S$, the image under $S^{-1}i$ of $x/x\in S^{-1}M$ in $B$ is equal to $1$ and $S^{-1}i$ is therefore surjective. Finally, (\rmnum{3}) clearly implies (\rmnum{1}).
\end{proof}
\begin{corollary}\label{nondegenerate submodule intersection product sum}
If $M$ and $N$ are two non-degenerate sub-$A$-modules of $B$, the $A$-modules $M+N$, $M\cdot N$ and $M\cap N$ are non-degenerate.
\end{corollary}
\begin{proof}
The assertion is trivial for $M+N$. On the other hand if $s\in S\cap M$ and
$t\in S\cap N$, then $st\in S\cap (M\cdot N)$ and $st\in S\cap(M\cap N)$.
\end{proof}
Given two sub-$A$-modules $M$ and $N$ of $B$, let us denote by $(N:M)$ the sub-$A$-module of $B$ consisting of those $b\in B$ such that $bM\sub N$. If every $b\in(N:M)$ is mapped to the homomorphism $h_b:x\mapsto bx$ of $M$ to $N$, a canonical homomorphism $b\mapsto h_b$, is obtained from $(N:M)$ to $\Hom_A(M,N)$.
\begin{proposition}\label{nondegenerate submodule (N:M) to Hom bijective}
Let $M$, $N$ be two sub-$A$-modules of $B$. If $M$ is non-degenerate, the canonical homomorphism from $(N:M)$ to $\Hom_A(M,N)$ is bijective.
\end{proposition}
\begin{proof}
Let $s\in S\cap M$. If $b\in(N:M)$ is such that $bx=0$ for all $x\in M$, then $bs=0$ whence $b=0$ since $s$ is not a divisor of $0$ in $B$. On the other hand, let $f\in\Hom_A(M,N)$ and set $b=f(s)/s$; for all $x\in M$, there exists $t\in S$ such that $tx\in A$. Then
\[f(x)=f(stx)/(st)=(txf(s))/(st)=bx\]
whence $b\in(N:M)$ and $f=h_b$, which proves the proposition.
\end{proof}
In particular, $(A:M)$ is canonically identified with the dual $M^*$ of $M$, the canonical bilinear form on $M^*\times M$ being identified with the restriction to $(A:M)\times M$ of the multiplication $B\times B\to B$.
\begin{definition}
A sub-$A$-module $M$ of $B$ is called \textbf{invertible} if there exists a sub-$A$-module $N$ such that $M\cdot N=A$.
\end{definition}
\begin{example}
If $b$ is invertible element of $B$, the $A$-module $Ab$ is invertible, as is seen by taking $N=Ab^{-1}$.
\end{example}
\begin{proposition}\label{invertible submodule prop}
Let $M$ be an invertible sub-$A$-module of $B$. Then:
\begin{itemize}
\item[(a)] There exists $s\in S$ such that $As\sub M\sub As^{-1}$ (and in particular $M$ is nondegenerate). 
\item[(b)] $(A:M)$ is the only sub-$A$-module $N$ of $B$ such that $M\cdot N=A$.
\item[(c)] $M$ is finitely generated and $(A:M)=A$ if and only if $M=A$. 
\end{itemize}
\end{proposition}
\begin{proof}
If there exists a sub-$A$-module $N$ such that $M\cdot N=A$, then
\[B\cdot M=B\cdot(B\cdot M)\sups B\cdot(M\cdot N)=B\cdot A=B\]
hence $M$ is non-degenerate. Similarly $N$ is non-degenerate. If $t_1\in S\cap M$ and $t_2\in S\cap N$, the element $s=t_1t_2$ belongs to $S\cap M\cap N$, whence $Ms\sub M\cdot N=A$ and therefore $As\sub M\sub As^{-1}$.\par
On the other hand, obviously $N\sub(A:M)$, whence
\[A=M\cdot N\sub M\cdot(A:M)\sub A\]
whence $A=M\cdot(A:M)$. Multiplying the two sides by $N$, we deduce then $N=(A:M)$, this prove (b). Now since $M\cdot N=A$, then there is an equation
\[\sum_ix_iy_i,\quad x_i\in M,y_i\in N,\]
so $M$ is generated by $(x_i)$ since $x=\sum_i(xy_i)x_i$ for each $x\in M$. Moreover, if $(A:M)=A$ then since $M$ is invertible we have $M\cdot A=A$, which implies $M=A$.
\end{proof}
\begin{theorem}\label{nondegenerate submodule invertible iff}
Let $M$ be a non-degenerate sub-$A$-module of $B$. The following properties are equivalent:
\begin{itemize}
\item[(\rmnum{1})] $M$ is invertible.
\item[(\rmnum{2})] $M$ is projective.
\item[(\rmnum{3})] $M$ is projective of rank $1$.
\item[(\rmnum{4})] $M$ is a finitely generated $A$-module and, for every maximal ideal $\m$ of $A$, the $A_\m$-module $M_\m$ is principal.
\end{itemize}
\end{theorem}
\begin{proof}
Let us show first the equivalence of properties (\rmnum{1}), (\rmnum{2}) and (\rmnum{3}). If (\rmnum{1}) holds and $N$ is sub-$A$-module of $B$ such that $M\cdot N=A$, then there is a relation
\[\sum_im_in_i=1\]
For all $x\in M$, set $v_i(x)=n_ix$, the $v_i$ are linear forms on $M$ and by $x=\sum_iv_i(x)m_i$ for all $x\in M$. This proves that $M$ is projective by(A, \Rmnum{2}, $\S$2, no.6, Proposition 12) and generated by the $x_i$; hence $M$ is a finitely generated projective module.\par
Let $\m$ be a maximal ideal of $A$. We show that the integer $r=\rank_\m(M)$ is equal to $1$. Let $S_\m$ be the image of $S$ in $A_\m$. As the elements of $S$ are not divisors of $0$ in $A$, those of $S_\m$ are not divisors of $0$ in $A_\m$, since $A_\m$ is a flat $A$-module. Then $(S_\m)^{-1}A_\m\neq 0$ and, as $M_\m$ is a free $A_\m$-module of rank $r$, $(S_\m)^{-1}M_\m$ is a free $(S_\m)^{-1}A_\m$-module of rank $r$. Also note that $(S_\m)^{-1}A_\m$ (resp. $(S_\m)^{-1}M_\m$) is canonically identified with $(S^{-1}A)_\m$ (resp. $(S^{-1}M)_\m$). Now $S^{-1}M=B$ by \cref{nondegenerate submodule iff}(\rmnum{3}) and hence $(S^{-1}M)_\m$ is a free $(S^{-1}A)_\m$ of rank $1$, which proves that $r=1$ and shows the implication $(\rmnum{1})\Rightarrow(\rmnum{3})$.\par
The implication $(\rmnum{3})\Rightarrow(\rmnum{2})$ is trivial. Let us show that $(\rmnum{2})\Rightarrow(\rmnum{1})$. There exists by hypothesis a family (not necessarily finite) $(f_i)_{i\in I}$ of linear forms on $M$ and a family $(m_i)_{i\in I}$ of elements of $M$ such that, for all $x\in M$, the family $(f_i(x))$ has finite support and $x=\sum_im_if_i(x)$ (A, \Rmnum{2}, $\S$2, no.6, Proposition 12). Since $M$ is non-degenerate, $f_i(x)=n_ix$ for some $n_i\in(A:N)$ by virtue of \cref{nondegenerate submodule (N:M) to Hom bijective}. Taking $x$ as an element of $M\cap S$, it is seen that of necessity $n_i=0$ except for a finite number of indices and $\sum_im_in_i=1$. This obviously implies $M\cdot(A:M)=A$, whence (\rmnum{1}).\par
By definition we have (\rmnum{3})$\Rightarrow$(\rmnum{4}), so let us show the converse. Since $M$ is non-degenerate, its annihilator is zero by \cref{nondegenerate submodule iff}, then so is the annihilator of $M_\m$. As $M_\m$ is assumed to be a monogenous $A_\m$-module, it is therefore free of rank $1$ and it then follows from \cref{projective module rank n iff} that $M$ is projective of rank $1$.
\end{proof}
\begin{corollary}\label{invertible module is flat and finite presented}
Every invertible sub-$A$-module of $B$ is flat and finitely presented.
\end{corollary}
\begin{proof}
This follows from \cref{nondegenerate submodule invertible iff}(\rmnum{3}).
\end{proof}
\begin{proposition}\label{invertible module tensor ideal quotient prop}
Let $M$, $N$ be two sub-$A$-modules of $B$. Suppose that $M$ is invertible.
\begin{itemize}
\item[(a)] The canonical homomorphism $M\otimes_AN\to M\cdot N$ is bijective.
\item[(b)] $(N:M)=N\cdot(A:M)$ and $N=(N:M)\cdot M$.
\end{itemize}
\end{proposition}
\begin{proof}
Let $i_N$ be the canonical injection from $N$ to $B$. Since $M$ is a flat $A$-module by \cref{invertible module is flat and finite presented}, $1\otimes i_N:M\otimes_AN\to M\otimes_AB$ is injective. But, as $B=S^{-1}A$, the $B$-module $M\otimes_AB$ is equal to $S^{-1}M$ and hence is identified with $B$ since $M$ is non-degenerate. If this identification is made, the image of $1\otimes i_N$ is $M\cdot N$, whence (a).\par
Now write $M^{-1}:=(A:M)$, then we have
\[M^{-1}\cdot N\sub(N:M)=M^{-1}\cdot M\cdot (N:M)\sub M^{-1}\cdot N\]
which proves $M^{-1}\cdot N=(N:M)$, whence $N=M\cdot(N:M)$.
\end{proof}
\subsection{The class group of invertible submodules}
Under multiplication, the sub-$A$-modules of $B$ form a commutative monoid, with $A$ as identity element. Then the invertible modules are the invertible elements and therefore form a commutative group $\mathfrak{I}$. We have seen that the inverse of $M\in\mathfrak{I}$ is $M^{-1}:=(A:M)$.\par
Let $A^\times$ (resp. $B^\times$) be the multiplicative group of invertible elements of $A$ (resp. $B$) and let $i_A^S$ denote the canonical injection $A\to B$. For all $b\in B^\times$, $\theta(b)=bA$ is an invertible sub-$A$-module. The map $\theta:B^\times\to\mathfrak{I}$ is a homomorphism whose kernel is $i_A^S(A^\times)$. Its cokernel will be denoted by $\mathfrak{C}$ or $\mathfrak{C}(A)$. The group $\mathfrak{C}$ is called the \textbf{group of classes of invertible sub-$\bm{A}$-modules} of $B$. The
following exact sequence has been constructed
\[\begin{tikzcd}
1\ar[r]&A^\times\ar[r,"i_A^S"]&B^\times\ar[r,"\theta"]&\mathfrak{I}\ar[r,"\pi"]&\mathfrak{C}\ar[r]&1
\end{tikzcd}\]
where $1$ denotcs the group consisting only of the identity element and $\pi$ is the canonical map $\mathfrak{I}\to\mathfrak{C}$.\par
As every invertible sub-$A$-module $M$ of $B$ is projective of rank $1$, the element $\cl(M)\in P(A)$ is defined.
\begin{proposition}\label{invertible module exact sequence of projective module}
The map $\cl:\mathfrak{I}\to P(A)$ defines, by taking quotients, an isomorphism from $\mathfrak{C}$ onto the kernel of the canonical homomorphism
\[\phi:P(A)\to P(B).\]
In other words, there is an exact sequence
\[\begin{tikzcd}
1\ar[r]&A^\times\ar[r,"i_A^S"]&B^\times\ar[r,"\theta"]&\mathfrak{I}\ar[r,"\cl"]&P(A)\ar[r,"\phi"]&P(B)
\end{tikzcd}\]
\end{proposition}
\begin{proof}
It follows from \cref{invertible module tensor ideal quotient prop} and the definition of addition in $P(A)$ that $\cl(M\cdot N)=\cl(M)+\cl(N)$ for $M,N$ in $\mathfrak{I}$, which shows that $\cl$ is a homomorphism. If $M\in\mathfrak{I}$ is isomorphic to $A$, there exists $b\in B$ such that $M=Ab$ and, as $M$ is invertible, there exists $b'\in B$ such that $bb'=1$, in othe rwords $b\in B^\times$. The converse is immediate. Hence the kernel of $\cl$ is contained in $\theta(B^\times)$.\par
Let us now determine the image of $\cl$. If $M\in\mathfrak{I}$, then $M\otimes_AB=S^{-1}M=B$, whence $\cl(M)\in\ker\phi$. Conversely, let $P$ be a projective $A$-module of rank $1$ such that $P(B)=P\otimes_AB$ is $B$-isomorphic to $B$. As $P$ is a flat $A$-module, the injection $i:A\to B$ defines an injection $i\otimes i_B:P\to P\otimes_AB=B$ and $P$ is thus identified with a sub-$A$-module of $B$. By virtue of \cref{nondegenerate submodule iff}(\rmnum{3}) $P$ is non-degenerate and \cref{nondegenerate submodule invertible iff} shows that $P$ is invertible. The kernel of $\phi$ is therefore equal to the image of $\cl$.
\end{proof}
\begin{corollary}
For two invertible sub-$A$-modules of $B$ to have the same image in $\mathfrak{C}$, it is necessary and sufficient that they be isomorphic.
\end{corollary}
\begin{corollary}
If the ring $B$ is semi-local, the group $\mathfrak{C}$ of classes of invertible sub-$A$-modules of $B$ is canonically identified with the group $P(A)$ of classes of projective $A$-modules of rank $1$.
\end{corollary}
\begin{proof}
In this case $P(B)=0$ by \cref{semilocal ring projective finite module rank defined then free}.
\end{proof}
\begin{example}
Let $A$ be an integral domain and $S$ is the set of nonzero elements of $A$, so that $B$ is the field of fractions of $A$. The invertible sub-$A$-modules of $B$ are also called in this case \textbf{invertible fractional ideals}. Those which are monogenous free $A$-modules are just the fractional principal ideals.
\end{example}
\subsection{Exercise}
\begin{exercise}
let $A$ be a commutative ring. The following are equivalent:
\begin{itemize}
\item[(a)] For every prime ideal $\p$, the localization $A_\p$ is an integral domain.
\item[(b)] For every maximal ideal $\m$, the localization $A_\m$ is an integral domain.
\item[(c)] For $x,y\in A$ such that $xy=0$ we have $\Ann(x)+\Ann(y)=A$. 
\end{itemize}
\end{exercise}
\begin{proof}
It is clear that (a) implies (b); for $(b)\Rightarrow(c)$, let $xy=0$, assume $\Ann(x)+\Ann(y)\sub\m$ for some maximal ideal $\m$, then $xy/1=0$ in $A_\m$. Then, since $A_\m$ is an integral domain, we have $x/1=0$ or $y/1=0$. Assume $x/1=0$, then $xs=0$ for some $s\in A-\m$. But $\Ann(x)\sub\m$, this is an contradiction. Similar for $y/1=0$.\par
We show that $(c)\Rightarrow(a)$. First we have the following observation: If $\p$ is a prime ideal, then for any $x\in A$ such that $x/1\neq 0\in A_\p$ then $\Ann(x)\sub\p$. In fact, assume $tx=0$ for $t\in A$. Then $t\notin A-\p$, hence $t\in\p$. Now, suppose that $x/s\cdot y/t=0\in A_\p$. Then $xyz=0$ for some $z\in A-\p$. It will suffice to prove that $x/1=0$ or $y/1=0$. Suppose otherwise, namely that $x/1\neq0$ and $y/1\neq 0$. Then, $xz/1\neq0$ since $z/1$ is invertible. By the observation, we know that $\Ann(xz)\sub\p$ and $\Ann(y)\sub\p$, thus $\Ann(x)+\Ann(y)\sub\p$, which is a contradiction.
\end{proof}
\begin{exercise}
Let $A$ be a ring. Suppose that, for each prime ideal $\p$, the local ring $A_\p$ has no nilpotent element $\neq 0$. Show that $A$ has no nilpotent element $\neq 0$. If each $A_\p$ is an integral domain, is $A$ necessarily an integral domain?
\end{exercise}
\begin{proof}
By corollary~\ref{localization and nilradical}, we have $\n(A_\p)=\n(A)_\p$ so $\n(A)_\p=0$ for any prime ideal $\p$. By \cref{localization module zero iff}, $\n(A)=0$.\par
The answer for next question is no. Suppose that $A=\prod_{i=1}^{n}k_i$, where $k_i=k$ an algebraic closed field. This is not integral domain, for example $(1,\dots,0)\cdot(0,\dots,1)=0$. Then $\Spec(A)=\{\p_i:=0\times\prod_{i\neq j}k_j\}$. Consider the localization $A_{\p_i}$: 
\[A-\p_i=\{(x_1,\dots,x_n)\in A:x_i\neq 0\}\]
We define a map $\varphi_i:A_{\p_i}\to k_i$ by
\[\frac{x}{s}\mapsto\frac{x_i}{s_i}\]
This is surjective of course, since we can choose $s_i=1$. To show the injectivity, let $x_i/s_i=y_i/t_i$, then since $k_i$ is a field, we have
\[x_it_i=y_is_i.\]
Let $e_i=(0,\dots,1,\dots,0)$, then $e_i\in A-\p_i$, and
\[e_i(xt-ys)=x_it_i-y_is_i=0\]
so $x/s=y/t$. Hence $\varphi_j$ is an isomorphism. In particular, $A_{\p_i}$ is an integral domain.
\end{proof}
\begin{exercise}\label{maximal multiplicative}
Let $A$ be a ring and let $\Sigma$ be the set of all multiplicatively closed subsets $S$ of $A$ such that $0\notin S$. Show that $\Sigma$ has maximal elements, and that $S\in\Sigma$ is maximal if and only if $A-S$ is a minimal prime ideal of $A$.\par
Moreover, by replace $0$ with any ideal $I$, we can show $S\in\Sigma$ is maximal if and only if $A-\p$ is a minimal prime ideal among those containing $I$.
\end{exercise}
\begin{proof}
By Zorn's lemma, $\Sigma$ has a maximal element. For any set $S\in\Sigma$, it is clear that $A-S$ has the prime property. When $S$ is maximal in $\Sigma$, we claim that $A-S$ is an ideal. Let $a\in A$, consider the smallest multiplicatively closed subsets containing $a$ and $S$: When $a\in S$, it is $S$, and when $a\notin S$, it is
\[S_a:=\{sa^n:s\in S,n\in\N\}\]
Now assume $a,b\in A-s$, then $S_a$ and $S_b$ both contains $S$ properly. By the maximality, $0\in S_a, S_b$, so there are $s_a,s_b\in S, m,n\in\N$ such that
\[s_aa^n=s_bb^m=0\]
Now assume $a+b\in S$, then $s_as_b(a+b)\in S$. Consider the power
\[(s_as_b(a+b))^{m+n}=(s_as_b)^{m+n}(a+b)^{m+n}=0\]
since every term of the sum is zero. This is a contradiction since $(s_as_b(a+b))^{m+n}\in S$. So $a+b\in A-S$.\par
Let $r\in A$ and $a\notin S$. If $ra\in S$, then $rs_aa\in S$, and $(rs_aa)^n=0\in S$, contradiction. Hence $A-S$ is a prime ideal. Since for any prime ideal $\p$, $A-\p$ is multiplicatively closed and does not contain $(0)$, it follows that $A-S$ is minimal.\par
Now assume $\p$ is a minimal prime ideal, then $A-\p\in\Sigma$. By Zorn's lemma, there is a maximal elements $T\in\Sigma$ such that $A-\p\sub T$. Then $A-T$ is prime and contained in $\p$, by assumption we have $T=A-\p$. So $A-\p$ is maximal.
\end{proof}
\begin{exercise}
Let $A$ be an integral domain and $M$ an $A$-module. An element $x\in M$ is a torsion element of $M$ if $\Ann(x)\neq 0$, that is if $x$ is killed by some non-zero element of $A$.\par
Show that the torsion elements of $M$ form a submodule of $M$. This submodule is called the \textbf{torsion submodule} of $M$ and is denoted by $T(M)$. If $T(M)=0$, the module $M$ is said to be torsion-free. Show that
\begin{itemize}
\item[(a)] If $M$ is any $A$-module, then $M/T(M)$ is torsion-free.
\item[(b)] If $f:M\to N$ is a module homomorphism, then $f(T(M))\sub T(N)$.
\item[(c)] The functor $T$ is left-exact.
\item[(d)] If $M$ is any $A$-module, then $T(M)$ is the kernel of the map $x\mapsto 1\otimes x$ of $M$ into $K\otimes_AM$, where $K$ is the field of fractions of $A$.
\end{itemize}
\end{exercise}
\begin{proof}
Let
\[\begin{tikzcd}
0\ar[r]&L\ar[r,"f"]&M\ar[r,"g"]&N
\end{tikzcd}\]
be an exact sequence. Consider the sequence
\[\begin{tikzcd}
0\ar[r]&T(L)\ar[r,"T(f)"]&T(M)\ar[r,"T(g)"]&T(N)
\end{tikzcd}\]
where $T(f)=f|_{T(M)}$. Clearly $T(g)\circ T(f)=0$. Since we have $\ker T(f)\sub \ker f$, we see $T(f)$ is injective.\par
Let $m\in\ker T(g)$, then $m$ is also in $\ker g$, so there is $\ell\in L$ such that $f(\ell)=m$. $m$ is torsion, so there is $r\in A$ such that $rm=0$. Then $f(r\ell)=rf(\ell)=rm=0$, and $r\ell=0$ since $f$ is injective. This implies $\ell\in T(L)$, so $T(f)(\ell)=m$.\par
Let $S=A-\{0\}$, then we have
\[(K\otimes_AM=S^{-1}A\otimes_AM=A\otimes_{S^{-1}A} S^{-1}M\cong S^{-1}M).\]
Now $1\otimes m$ is $m/1$ in $S^{-1}M$, so $1\otimes m=0$ if and only if there is $a\in A$ such that $am=0$.
\end{proof}
\begin{exercise}
Let $S$ be a multiplicatively closed subset of an integral domain $A$. Show that $T(S^{-1}M)=S^{-1}T(M)$. Deduce that the following are
equivalent:
\begin{itemize}
\item[(a)] $M$ is torsion-free.
\item[(b)] $M_\p$ is torsion-free for all prime ideals $\p$.
\item[(c)] $M_\m$ is torsion-free for all maximal ideals $\m$.
\end{itemize}
\end{exercise}
\begin{proof}
Let $m/s\in T(S^{-1}M)$, then there is $y/t\in S^{-1}A$ such that $ym/st=0$. Hence $\exists u\in S$ such that
\[uym=0\]
This implies $m\in T(m)$, so $m/s\in S^{-1}T(M)$.\par
Let $m/s\in S^{-1}T(M)$. Then there is $r\in A: rm=0$. Then $r/1\cdot m/s=0$ so $m/s\in T(S^{-1}M)$.\par
$M$ is torsion free if and only if $T(M)=0$. Use \cref{localization module zero iff}.
\end{proof}
\chapter{Graded rings and filtrations}
\section{Graduation on rings and modules}
\subsection{Graded ring and graded module}
\begin{definition}
Given an abelian group $G$ and a set $\Delta$, a \textbf{graduation} of type $\Delta$ on $G$ is a family $(G_\lambda)_{\lambda\in\Delta}$, of which $G$ is the direct sum. The set $G$ with the structure defined by its group law and the graduation is called a \textbf{graded group} of type $\Delta$.
\end{definition}
The set $\Delta$ is called the \textbf{set ef degrees} of $G$. An element $x\in G$ is called \textbf{homogeneous} if it belongs to one of the $G_\lambda$, \textbf{homogeneous of degree $\bm{\lambda}$} if $x\in G_\lambda$. The element $0$ is therefore homogeneous of all degrees; but if $x\neq 0$ is homogeneous, it belongs to only one of the $G_\lambda$; the index $\lambda$ such that $x\in G_\lambda$ is then called the degree of $x$ (or sometimes the weight of $x$) and is sometimes denoted by $\deg(x)$. Every $x\in G$ may be written uniquely as a sum $\sum_\lambda x_\lambda$ of homogeneous elements with $x_\lambda\in G_\lambda$; $y_\lambda$ is called the \textbf{homogeneous component} of degree $\lambda$, or simply the component of degree $\lambda$ of $x$.
\begin{example}
Given any commutative monoid $\Delta$ with identity element $0$ and abelian group $G$, a graduation $(G_\lambda)_{\lambda\in\Delta}$ is defined on $G$ by taking $G_0=G$ and $G_\lambda=\{0\}$ for $\lambda\neq 0$; this graduation is called \textbf{trivial}.
\end{example}
\begin{example}
Let $\Delta$ and $\Delta'$ be two sets and $\rho$ a map of $\Delta$ into $\Delta'$. Let $(G_\lambda)_{\lambda\in\Delta}$ be a graduation of type $\Delta$ on a commutative group $G$; for $\mu\in\Delta'$, let $G_\mu$ be the sum of the $G_\lambda$'s such that $\rho(\lambda)=\mu$; clearly $(G_\mu)_{\mu\in\Delta'}$ is a graduation of type $\Delta'$ on $G$, said to be \textbf{derived} from $(G_\lambda)$ by means of the map $\rho$.\par
When $\Delta$ is a commutative group written additively and $\rho$ the map $\lambda\mapsto-\lambda$ of $\Delta$ onto itself, $(G_\mu)$ is called the \textbf{opposite graduation} of $(G_\lambda)$.
\end{example}
\begin{example}
If $\Delta=\Delta_1\times\Delta_2$ is a product of two sets, a graduation of type $\Delta$ is called a \textbf{bigraduation} of types $\Delta_1$, $\Delta_2$. For all $\lambda\in\Delta_1$, let $G_\lambda=\bigoplus_{\mu\in\Delta_2}G_{\lambda\mu}$; for all $\mu\in\Delta_2$, let $G_\mu=\bigoplus_{\lambda\in\Delta_1}G_{\lambda\mu}$; clearly $(G_\lambda)_{\lambda\in\Delta_1}$ is a graduation of type $\Delta_1$ and $(G_\mu)_{\mu\in\Delta_2}$ is a graduation of type $\Delta_2$ on $G$; these graduations are called the \textbf{partial graduations} derived from the bigraduation $(G_{\lambda\mu})$. Note that $G_{\lambda\mu}=G_\lambda\cap G_\mu$. Conversely, if $(G_\lambda)_{\lambda\in\Delta_1}$ and $(G_\mu)_{\mu\in\Delta_2}$ are two graduations on $G$ such that $G$ is the direct sum of the $G_{\lambda\mu}=G_\lambda\cap G_\mu$, these subgroups form a bigraduation of types $\Delta_1$, $\Delta_2$ on $G$, of which $(G_\lambda)_{\lambda\in\Delta_1}$ and $(G_\mu)_{\mu\in\Delta_2}$ are the partial
graduations.
\end{example}
\begin{example}
Let $\Delta_0$ be a commutative monoid written additively, with identity
element denoted by $0$; let $I$ be any set and $\Delta_0^{\oplus I}$ denote the submonoid of the product set $\Delta_0^I$ consisting of the families $(\lambda_i)_{i\in I}$ of finite support. Let $\rho:\Delta\to\Delta_0$ be the surjective (codiagonal) homomorphism of $\Delta$ into $\Delta_0$ defined by $\rho((\lambda_i))=\sum_i\lambda_i$. From every graduation of type $\Delta$ a graduation of type $\Delta_0$ is derived by means of $\rho$; it is called the \textbf{total graduation} associated with the given "multigraduation" of type $\Delta$.
\end{example}
Often the set $\Delta$ may also endow some algebraic structure and one want to have operations between different components of the graduation $(G_\lambda)$. In the following, we may always assume that $\Delta$ is a commutative monoid, and consider graduations of type $\Delta$ on rings and modules.
\begin{definition}
Let $\Delta$ be a commutative moniod. Given a ring $A$ and a graduation $(A_\lambda)$ of type $\Delta$ on the additwe group $A$, this graduation is said to be \textbf{compatible} with the ring structure on $A$ if
\[A_\lambda A_\mu\sub A_{\mu+\lambda}\quad\text{ for all $\lambda,\mu\in\Delta$}.\]
The ring $A$ with this graduation is called a \textbf{graded ring} of type $\Delta$. 
\end{definition}
\begin{proposition}\label{graded ring subring if}
If the elements of $\Delta$ are cancellable and $(A_\lambda)$ is a graduation of type $\Delta$ compatible with the structure of a ring $A$, then $A_0$ is a subring of $A$.
\end{proposition}
\begin{proof}
As $A_0A_0\sub A_0$, by definition, it suffices to prove that $1\in A_0$. Let $1=\sum_{\lambda\in\Delta}e_\lambda$ be the decomposition of $1$ into its homogeneous components. If $x\in A_\mu$ then $x=x\cdot 1=\sum_\lambda xe_\lambda$; comparing the components of degree $\mu$, (since every element in $\Delta$ is cancellable) we get $x=xe_0$. Since this relation is true for every homogeneous element of $A$, it is true for all $x\in A$; in particular $1=1\cdot e_0=e_0\in A_0$.
\end{proof}
\begin{definition}
Let $A$ be a graded ring of type $\Delta$, $(A_\lambda)$ its graduation and $M$ a left $A$-module; a graduation $(M_\lambda)$ of type $\Delta$ on the addition group $M$ is compatible with the $A$-module structure on $M$ if
\[A_\lambda M_\mu\sub M_{\lambda+\mu}\quad\text{ for all $\lambda,\mu\in\Delta$}.\]
The module $M$ with this graduation is then called a \textbf{graded module} of type $\Delta$ over the graded ring $A$.
\end{definition}
When the elements of $\Delta$ are cancellable, it follows from \cref{graded ring subring if} that the $M_\lambda$ are $A_0$-modules. Clearly if $A$ is a graded ring of type $\Delta$, the $A$ is itself a left graded $A$-module of type $\Delta$.\par
Let $A$ be a graded ring and $M$ be a graded $A$-module. An element $x\in M$ is \textbf{homogeneous} if $x\in M_\lambda$ for some $\lambda$, and $\lambda$ is then called the \textbf{degree} of $x$. A general element $x\in M$ can be written uniquely in the form $x=\sum_{\lambda}x_\lambda$ with $x_i\in M_\lambda$; $x_\lambda$ is called the \textbf{homogeneous term of $\bm{x}$ of degree $\bm{\lambda}$}.
\begin{example}
On any ring $A$ the trivial graduation of type $\Delta$ is compatible with the ring structure. If $A$ is graded by the trivial graduation, for a graduation $(M_\lambda)$ of type $\Delta$ on an $A$-module $M$ to be compatible with the $A$-module structure, it is necessary and sufficient that the $M_\lambda$ be submodules of $M$.
\end{example}
\begin{example}
Let $A$ be a graded ring of type $\Delta$, $M$ a graded $A$-module of type $\Delta$ and $\rho$ a homomorphism of $\Delta$ into a commutative monoid $\Delta'$ whose identity element is denoted by $0$. Then $A$ is a graded ring of type $\Delta'$ and $M$ a graded module of type $\Delta'$ for the graduations of type $\Delta'$ derived from $\rho$ and the graduations of type $\Delta$ on $A$ and $M$: this follows immediately from the relation $\rho(\lambda+\mu)=\rho(\lambda)+\rho(\mu)$.\par
In particular, if $\Delta=A_1\times A_2$ is a product of two commutative monoids, the projections $\pi_1$ and $\pi_2$ are homomorphisms and the corresponding graduations are just the parlial graduations derived from the graduations of type $\Delta$; these partial graduations are thus compatible with the ring structure of $A$ and the module structure of $M$. Similarly, if $\Delta=\Delta_0^{\oplus I}$ (where $\Delta_0$ is a commutative monoid with identity element denoted by $0$), the total graduation of type $\Delta_0$ derived from the graduation of type $\Delta$ on $A$ (resp. $M$) by means of the codiagonal homomorphism is compatible with the ring structure on $A$ (resp. with the module structure on $M$).
\end{example}
\begin{example}
Let $A$ be a graded ring of type $\Delta$, $M$ a graded $A$-module of type $\Delta$ and $A_0$ an element of $\Delta$; for $\lambda\in\Delta$, let $M'_\lambda=M_{\lambda+\lambda_0}$, and let $M'$ be the submodule $M'=\bigoplus_{\lambda\in\Delta}M'_\lambda$ is an $A$-module and the $M'_\lambda$ form on $M'$ a graduation of type $A$ compatible with the $A$-module structure of $M'$; the graded $A$-module $M'$ of type $\Delta$ thus defined is said to be obtained by shifting by $\lambda_0$ the graduation of $M$ and it is denoted by $M(\lambda_0)$. When $\Delta$ is a group, the underlying $A$-module of the graded $A$-module $M'$ is identified with $M$ since in this case $\lambda\mapsto\lambda+\lambda_0$ is an isomorphism of $\Delta$ onto itself.
\end{example}
\begin{example}
Let $A$ be a ring. The polynomial ring $A[X]$ in one indeterminate is graded of type $\N$ by the subgroups $AX^n$. Similarly, the polynomial ring $A[X_1,\dots,X_n]$ is graded of type $\N$ by the subgroups of homogeneous polynomials.
\end{example}
The graduations most often used are of type $\Z$ or of type $\Z^n$; when we speak of graded (resp. bigraded, trigraded, etc.) modules and rings without mentioning the type, it is understood that we mean graduations of type $\Z$ (resp. $\Z^2$, $\Z^3$ etc.); a graded ring (resp. module) of type $\N$ is also called a \textbf{graded ring (resp. module) with positive degrees}.\par
Now we introduce the morphisms between graded rings and modules.
\begin{definition}
Let $A$, $B$ be two graded rings with the same type $\Delta$. and $(A_\lambda)$, $(B_\lambda)$ their respective graduations. A ring homomorphism $\rho:A\to B$ is called \textbf{graded} if $\rho(A_\lambda)\sub B_\lambda$ for all $\lambda\in\Delta$.\par
Let $M$, $N$ be two graded modules of type $\Delta$ over a graded ring $A$ of type $\Delta$. Let $\phi:M\to N$ be an $A$-homomorphism and $\delta$ an element of $\Delta$; $\phi$ is called \textbf{graded of degree $\bm{\delta}$} if $\phi(M_\lambda)\sub N_{\lambda+\delta}$ for all $\lambda\in\Delta$.
\end{definition}
An $A$-homomorphism $\phi:M\to N$ is called \textbf{graded} if there exists $\delta\in\Delta$ such that $\phi$ is graded of degree $\delta$. If $\phi\neq 0$ and every element of $\Delta$ is cancellable, then the degree $\delta$ of $\phi$ is uniquely determined.\par
If $\rho:A\to B$ and $\nu:B\to C$ are two graded homomorphisms of graded rings of type $\Delta$, so is $\nu\circ\rho:A\to B$; for a map $\rho:A\to B$ to be a graded ring isomorphism, it is necessary and sufficient that $\rho$ be bijective and that $\rho$ and the inverse map $\rho^{-1}$ be graded homomorphisms. It also suffices for this that $f$ be a bijective graded homomorphism. Thus it is seen that graded homomorphisms can be taken as the morphisms of the category of graded ring structure of type $\Delta$.\par
Similarly, if $\phi:M\to N$ and $\psi:N\to P$ are two graded homomorphisms of graded $A$-modules of type $\Delta$ of respective degrees $\delta$ and $\gamma$, $\psi\circ\phi:M\to P$ is a graded homomorphism of degree $\delta+\gamma$. If $\delta$ admits an inverse $-\delta$ in $\Delta$ and $\phi:M\to N$ is a bijective graded homomorphism of degree $\delta$, the inverse map $\phi^{-1}:N\to M$ is a bijective graded homomorphism of degree $-\delta$. It follows as above that the graded homomorphisms of degree $0$ can be taken as the morphisms of the category of graded $A$-module of type $\Delta$.
\begin{example}
If $M$ is a graded $A$-module and $M(\lambda_0)$ is a graded $A$-module obtained by shifting, the $\Z$-linear map of $M(\lambda_0)$ into $M$ which coincides with the canonical injection on each $M_{\lambda+\lambda_0}$ is a graded homomorphism of degree $\lambda_0$ (which is bijective when $\Delta$ is a group).
\end{example}
\begin{example}
If $a$ is a homogeneous element of degree $\delta$ belonging to the centre of $A$, the homothety $x\mapsto ax$ of any graded $A$-module $M$ is a graded homomorphism of degree $\delta$.
\end{example}
\begin{remark}
A graded $A$-module $M$ is called a graded free $A$-module if there exists a basis $(x_i)_{i\in I}$ of $M$ consisting of homogeneous elements. Suppose it is and $\Delta$ is a commutative group; let $\lambda_i$ be the degree of $x_i$ and consider for each $i$ the shifted $A$-module $A(-\lambda_i)$; if $e_i$ denotes the element $1$ of $A$ considered as an element of degree $\lambda_i$ in $A(-\lambda_i)$, the $A$-linear map $\phi:\bigoplus_{i\in I}A(-\lambda_i)\to M$ such that $\phi(e_i)=x_i$ for all $i$ is a graded $A$-module isomorphism.
\end{remark}
\begin{example}
A direct system $(A_\alpha,f_{\beta\alpha})$ of graded rings of type $\Delta$ is a direct system of rings such that each $A_\alpha$ is graded of type $\Delta$ and each $f_{\beta\alpha}$ is a homomorphism of graded ring. If $(A_\alpha^\lambda)$ is the graduation of $A_\alpha$ and we write
\[A=\rlim A_\alpha,\quad A^\lambda=\rlim A_\alpha^\lambda,\]
it follows that $(A^\lambda)$ is a graduation of $A$ and this graduation is compatible with the ring structure on $A$. The graded ring $A$ is called the \textbf{direct limit} of the direct system of graded rings $(A_\alpha,f_{\beta\alpha})$. If $f_\alpha:A_\alpha\to A$ is the canonical map, then $f_\alpha$ is a homomorphism of graded rings.\par
Similarly, a direct system $(M_\alpha,\phi_{\beta\alpha})$ of graded $A$-modules of type $\Delta$ is a direct system of $A$-modules such that each $M_\alpha$ is graded of type $\Delta$ and each $\phi_{\beta\alpha}$ is a homomorphism of graded $A$-modules of degree $0$. If $(M_\alpha^\lambda)$ is the graduation of $M_\alpha$ and 
\[M=\rlim M_\alpha,\quad M^\lambda=\rlim M_\alpha^\lambda,\]
then $(M^\lambda)$ is a graduation of $M$ and this graduation is compatible with the module structure on $M$. The graded module $M$ is called the \textbf{direct limit} of the direct system of graded $A$-modules $(M_\alpha,\phi_{\beta\alpha})$. If $\phi_\alpha:M_\alpha\to M$ is the canonical map, then $\phi_\alpha$ is a homomorphism of graded modules of degree $0$.
\end{example}
\begin{proposition}\label{graded submodule iff}
Let $A$ be a graded ring of type $\Delta$, $M$ a graded $A$-module of type $\Delta$, $(M_\lambda)$ its graduation and $N$ a sub-$A$-module of $M$. The following properties are equivalent:
\begin{itemize}
\item[(\rmnum{1})] $N=\bigoplus_{\lambda}(N\cap M_\lambda)$.
\item[(\rmnum{2})] The homogeneous components of every element of $N$ belongs to $N$
\item[(\rmnum{3})] $N$ can be generated by homogeneous elements.
\end{itemize}
\end{proposition}
\begin{proof}
Every element of $N$ can be written uniquely as a sum of elements of the $M_\lambda$, and hence it is immediate that (\rmnum{1}) and (\rmnum{2}) are equivalent and that (\rmnum{1}) implies (\rmnum{3}). We show that (\rmnum{3}) implies (\rmnum{2}). Then let $(x_i)_{i\in I}$ be a family of homogeneous generators of $N$ and let $\delta_i$ be the degree of $x_i$. Every element of $N$ can be written as $\sum_i a_ix_i$ with $a_i\in A$; if $a_{i\mu}$ is the component of $a_i$ of degree $\mu$, the conclusion follows from the relation
\[\sum_{i\in I}\Big(\sum_{\mu\in\Delta}a_{i\mu}x_i)=\sum_{\mu\in\Delta}\Big(\sum_{\mu+\delta_i=\lambda}a_{i\mu}x_i\Big).\]
This completes the proof.
\end{proof}
When a submodule $N$ of $M$ has the equivalent properties stated in \cref{graded submodule iff}, clearly the $N\cap M_\lambda$ form a graduation compatible with the $A$-module structure of $N$, called the graduatuin imluced by that on $M$; $N$ with this graduation is called a graded submodule of $M$.
\begin{corollary}
If $N$ is a graded submodule of $M$ and $(x_i)$ is a generating system of $N$, then the homogeneous componmts of the $(x_i)$ form a genetating system of $N$.
\end{corollary}
\begin{corollary}
If $N$ is a finitely generated submodule of $M$, then $N$ admits a finite generating system consists of homogeneous elements.
\end{corollary}
A graded submodule of $A$ is called a \textbf{graded ideal} of the graded ring $A$. For every subring $B$ of $A$, if we set $B_\lambda=B\cap A_\lambda$ then $B_\lambda B_{\mu}\sub B_{\lambda+\mu}$, thus the graduation induced on $B$ by that on $A$ is compatible with the ring structure on $B$; $B$ is then called a \textbf{graded subring} of $A$.\par
If $N$ is a graded submodule of a graded $A$-module $M$ and $(M_\lambda)_{\lambda\in\Delta}$ is the graduation of $M$, the submodules $(M_\lambda+N)/N$ of $M/N$ form a gnoduation compatible with the structure of this quotient module. For, if $N_\lambda=M_\lambda\cap N$, then $(M_\lambda+N)/N$ is identified with $M_\lambda/N_\lambda$ and it follows from \cref{graded submodule iff} that $M/N$ is their direct sum. Moreover,
\[A_\lambda(M_\mu+N)\sub A_\lambda M_\mu+N\sub M_{\lambda+\mu}+N\]
which establishes our assertion. The graduation $((M_\lambda+N)/N)_{\lambda\in\Delta}$ is called the \textbf{quotient graduation} of that on $M$ by $N$ and the quotient module $M/N$ with this graduation is called the \textbf{graded quotient module} of $M$ by the graded submodule $N$; the canonical homomorphism $\pi:M\to M/N$ is a graded homomorphism of degree $0$ for this graduation.\par
If $\a$ is a graded ideal of $A$, the quotient graduation on $A/\a$ is compatible with the ring structure on $A/\a$. The ring $A/\a$ with this graduation is called the \textbf{quotient graded ring} of $A$ by $\a$. The canonical homomorphism $\pi:A\to A/\a$ is a homomorphism of graded rings for this graduation.
\begin{proposition}\label{graded module homomorphism prop}
Let $A$ be a graded ring of type $\Delta$, $M$ and $N$ two graded $A$-modules of type $\Delta$ and $\phi:M\to N$ a graded $A$-homomorphism of degree $\delta$. Then:
\begin{itemize}
\item[(a)] $\im\phi$ is a graded submodule of $N$.
\item[(b)] If $\delta$ is a regular element of $\Delta$, then $\ker\phi$ is a graded submodule of $M$.
\item[(c)] If $\delta=0$, the bijection $M/\ker\phi\to\im\phi$ canonically associated with $\phi$ is an isomorphism of graded modules.
\end{itemize}
\end{proposition}
\begin{proof}
Assertion (a) follows immediately from the definitions and \cref{graded submodule iff}(\rmnum{3}). If $x$ is an element of $M$ such that $\phi(x)=0$ and $x=\sum x_\lambda$ is its decomposition into homogeneous components (where $x_\lambda$ is of degree $\lambda$), then
\[\sum_\lambda\phi(x_\lambda)=\phi(x)=0\]
and $\phi(x_\lambda)$ is of degree $\lambda+\delta$; if $\delta$ is regular the relation $\lambda+\delta=\mu+\delta$ implies $\lambda=\mu$, hence the $\phi(x_\lambda)$ are the homogeneous components of $\phi(x)$ and necessarily $\phi(x_\lambda)=0$ for all $\lambda\in\Delta$, which proves (b). The bijection $\pi:M/\ker\phi\to\im\phi$ canonically associated with $\phi$ is then a graded homomorphism of degree $\delta$, as follows from the definition of the quotient graduation; whence (c) when $\delta=0$.
\end{proof}
\begin{corollary}
Let $A$, $B$ be two graded rings of type $\Delta$ and $\rho:A\to B$ a graded homomorphism of graded rings. Then $\im\rho$ is a graded subring of $B$, $\ker\rho$ a graded ideal of type $\Delta$ and the bijection $A/\ker\rho\to\im\rho$ canonically associated with $\rho$ is an isomorphism of graded rings.
\end{corollary}
\begin{proof}
It suffices to apply \cref{graded module homomorphism prop} to $\rho$ considered as a homomorphism of degree $0$ of graded $\Z$-modules.
\end{proof}
\begin{proposition}\label{graded mudule sum intersect ann prop}
Let $A$ be a graded ring of type $\Delta$ and $M$ a graded $A$-module of type $\Delta$.
\begin{itemize}
\item[(a)] Every sum and every intersection of graded submodules of $M$ is a graded submodule.
\item[(b)] If $x$ is a homogeneous element of $M$ of degree $\mu$ which is cancellable in $\Delta$, then the annihilator of $x$ is a graded ideal of $A$.
\item[(c)] If the elements of $\Delta$ are cancellable, the annihilator ef a graded submodule of $M$ is a graded ideal of $A$.  
\end{itemize}
\end{proposition}
\begin{proof}
If $(N_i)$ is a family of graded submodules of $M$, property (\rmnum{3}) of \cref{graded submodule iff} shows that the sum of the $N_i$ is generated by homogeneous elements and property (\rmnum{2}) of \cref{graded submodule iff} proves that the homogeneous components of every element of $\bigcap_iN_i$ belongs to $\bigcap_iN_i$; whence (a).\par
To prove (b), it suffices to note that $\Ann(x)$ is the kernel of the homomorphism $a\mapsto ax$ of the $A$-module $A$ into $M$ and that this homomorphism is graded of degree $\mu$; the conclusion follows from \cref{graded module homomorphism prop}(b). Finally (c) is a consequence of (a) and (b) for the annihilator of a graded submodule $N$ of $M$ is the intersection of the annihilators of the homogeneous elements of $N$, by virtue of \cref{graded submodule iff}.
\end{proof}
\begin{example}
Let $M$ be a graded $A$-module and $E$ a submodule of $M$; it follows from \cref{graded mudule sum intersect ann prop}(a) that there exists a largest graded submodule $N_1$ of $M$ contained in $E$ and a smallest graded submodule $N_2$ of $M$ containing $E$; $N_1$ is the set of $x\in E$ all of whose homogeneous components belong to $E$ and $N_2$ is the submodule of $M$ generated by the homogeneous components of a generating system of $E$.
\end{example}
Let $\Delta$ be a commutative monoid, $A$ a graded ring of type $\Delta$, $(A_\lambda)_{\lambda\in\Delta}$ its graduation and $E$ an $A$-algebra. A graduation $(E_\lambda)_{\lambda\in\Delta}$ of type $\Delta$ on the additive group $E$ is said to be \textbf{compatible} with the $A$-algebra structure on $E$ if it is compatible both with the $A$-module and with the ring structure on $E$, in other words, if, for all $\lambda,\mu$ in $\Delta$,
\begin{align}\label{graded algebra condition}
A_\lambda E_\mu\sub E_{\lambda+\mu},\quad E_\lambda E_\mu\sub E_{\lambda+\mu}.
\end{align}
The $A$-algebra $E$, with this graduntion, is then called a \textbf{graded algebra} of type $\Delta$ over the graded ring $A$.\par
When the graduation on $A$ is trivial (that is, $A_0=A$, $A_\lambda=\{0\}$ for $\lambda\neq 0)$, condition (\ref{graded algebra condition}) means that the $E_\lambda$ are sub-$A$-modules of $E$. This leads to the definition of the notion of graded algebra of type $\Delta$ over a non-graded commutative ring $A$: $A$ is given the trivial graduation of type $\Delta$ and the above definition is applied.\par
When we consider graded $A$-algebras $E$ with a unit element $1$, it will always be understood that $1$ is of degree $0$. It follows that if an invertible element $x\in E$ is homogeneous and of degree $p$, its inverse $x^{-1}$ is homogeneous and of degree $-p$: it suffice to decompose $x^{-1}$ as a sum of homogeneous elements in the relation $x^{-1}x=xx^{-1}=1$.\par
Let $E$ and $F$ be two graded algebras of type $\Delta$ over a graded ring $A$ of type $\Delta$. An $A$-algebra homomorphism $u:E\to F$ is called a \textbf{graded algebra homomorphism} is $u(E_\lambda)\sub F_\lambda$ for all $\lambda\in\Delta$ (where $(E_\lambda)$ and $(F_\lambda)$ denote the respective graduations of $E$ and $F$); where $E$ and $F$ are associative and unital and $u$ is unital, this condition means that $u$ is a graded ring homomorphism.\par
\begin{remark}
Let $E$ be a graded $A$-algebra of type $\N$. Then $E$ often is identified with a graded $A$-algebra of type $\Z$ by writing $E_n=\{0\}$ for $n<0$.
\end{remark}
\begin{remark}
The definition can also be interpreted by saying that $E$ is a graded $A$-module and that $A$-linear map
\[m:E\otimes_AE\to E\]
defining the multiplication on $E$, is homogeneous of degree $0$ when $E\otimes_AE$ is given its graduation of type $\Delta$.\par
To define a graded $A$-algebra structure of type $\Delta$ on the graded ring $A$, with $E$ as underlying graded $A$-module, therefore amounts to defining for each ordered pair $(\lambda,\mu)$ of elements of $\Delta$ a $\Z$-bilinear map
\[m_{\lambda\mu}:E_\lambda\times E_\mu\to E_{\lambda+\mu}\]
such that for every triple of indices $(\lambda,\mu,\nu)$ and for $\alpha\in A_\lambda$, $x\in E_\mu$, $y\in E_\mu$, we have $\alpha m_{\mu\nu}(x,y)=m_{\lambda+\mu,\nu}(\alpha x,y)=m_{\mu,\lambda+\nu}(x,\alpha y)$.
\end{remark}
\begin{example}
\mbox{}
\begin{itemize}
\item[(a)] Let $B$ be a graded ring of type $\Delta$; if $B$ is given its canonical $\Z$-algebra structure, $B$ is a graded $A$-algebra ($\Z$ being given the trivial graduation).
\item[(b)] Let $M$ be a magma and $\phi:M\to\Delta$ a homomorphism. For all $\lambda\in\Delta$, we write $M_\lambda=\phi^{-1}(\lambda)$; then $M_\lambda M_\nu\sub M_{\lambda+\mu}$. Let $A$ be a graded commutative ring of type $\Delta$ and $(A_\lambda)_{\lambda\in\Delta}$ its graduation; we shall define a graded $A$-algebra structure on the algebra $E=A^{\oplus M}$ of the magma $M$. To this end, let $E_\lambda$ denote the additive subgroup of $E$ generated by the elements of the form $\alpha m$ such that $\alpha\in A_\mu$, $m\in M_\nu$, and $\mu+\nu=\lambda$. As the $M_\lambda$ are pairwise disjoint, $E$ is the direct sum of the $A_\mu M_\nu$ and hence is the direct sum of $E_\lambda$ and it is immediate that $E_\lambda$ satisfy the condition (\ref{graded algebra condition}). Therefore it defines on $E$ the desired graded $A$-algebra structure. If $M$ admits an identity element $e$, it may also be supposed that $\phi(e)=0$. A particular case is the one where the graduation of the ring $A$ is trivial; then $E_\lambda$ is the sub-$A$-module of $E$ generated by $M_\lambda$. More particularly, if we take $M=\N^{\oplus I}$, $\Delta=\N$ and $\phi$ the map such that $\phi((n_i))=\sum_in_i$, the ring $A$ having the trivial graduation, a graduation is thus obtained on the polynomial algebra $A[X_i]_{i\in I}$, for which the degree of a homogeneous polynomial is the total degree.\par
We now take $M$ to be the free monoid $\Delta(X)$ of a set $X$ and $\phi$ the homomorphism $\Delta(X)\to\N$ which associates with each word its length. Thus a graded $A$-algebra structure is obtained on the free associated algebra of the set $X$. 
\end{itemize}
\end{example}
Let $E$ be a graded algebra of type $\Delta$ over a graded ring $A$ of type $\Delta$. If $F$ is a sub-$A$-algebra of $E$ which is a graded sub-$A$-module, then the graduation $(F_\lambda)$ on $F$ is compatible with its $A$-algebra structure, since $F_\lambda=F\cap E_\lambda$; in this case $F$ is called a graded subalgebra of $E$ and the canonical injection $F\to E$ is a graded algebra homomorphism.\par
Similarly, if $\a$ is a left (resp. right) ideal of $E$ which is a graded sub-$A$-module, then $E_\lambda\a_\mu\sub\a_{\lambda+\mu}$ (resp. $\a_\lambda E_\mu\sub\a_{\lambda+\mu}$), since $\a_\lambda=\a\cap E_\lambda$; then $\a$ is called a graded ideal of the algebra $E$. If $\b$ is a graded two-sided ideal of $E$ the quotient graduation on the module $E/\b$ is compatible with the algebra structure on $E/\b$ and the canonical homomorphism $E\to E/\b$ is a graded algebra homomorphism.\par
If $u:E\to F$ is a graded algebra homomorphism, then $\im(u)$ is a graded sub-algebra of $F$ and $\ker u$ is a graded two-sided ideal of $E$ and the bijection $E/\ker u\to\im u$ canonically associated with $u$ is a graded algebra isomorphism.
\begin{proposition}\label{graded ring ideal generated by homogenous}
Let $A$ be a graded commutative ring of type $\Delta$, $E$ a graded $A$-algebra of type $\Delta$ and $S$ a set of homogeneous elements of $E$. Then the sub-$A$-algebra (resp. left ideal, right ideal, two-sided ideal) generated by $S$ is a graded subalgebra (resp. graded ideal).
\end{proposition}
\begin{proof}
The subalgebra of $E$ generated by $S$ is the sub-$A$-module generated by the finite products of elements of $S$, which are homogeneous. Simialrly, the left (resp. right) ideal generated by $S$ is the sub-$A$-module generated by the elements of the form $u_1(u_2(\cdots(u_ns))\cdots)$ (resp. $(\cdots((su_n)u_{n-1})\cdots)u_2)u_1$ where $s\in S$ and the $u_j\in E$ are homogeneous and these products are homogeneous, whence in the case the conclusion by \cref{graded submodule iff}. Finally, the two-sided ideal generated by $S$ is the union of the sequence $(\mathfrak{J}_n)_{n\geq 1}$, where $\mathfrak{J}_1$ is the left ideal generated by $S$ and $\mathfrak{J}_{2n}$ (resp. $\mathfrak{J}_{2n+1}$) is the right (resp. left) ideal generated by $\mathfrak{J}_{2n-1}$ (resp. $\mathfrak{J}_{2n})$, which completes the proof.
\end{proof}
Let $(A_\alpha,\phi_{\beta\alpha})$ is a directed direct system of graded commutative rings of type $\Delta$ and for each $\alpha$ let $E_\alpha$ be a graded $A_\alpha$-algebra of type $\Delta$; for $\alpha\leq\beta$ let $f_{\beta\alpha}:E_\alpha\to E_\beta$ be an $A_\alpha$-homomorphism of graded algebras and suppose that $f_{\gamma\alpha}=f_{\gamma\beta}\circ f_{\beta\alpha}$ for $\alpha\leq\beta\leq\gamma$; then we shall write $(E_\alpha,f_{\beta\alpha})$ a directed direct system of graded algebras of type $\Delta$ over the directed direct system $(A_\alpha,\phi_{\beta\alpha})$ of graded commutative rings of type $\Delta$. Then we know that $E=\rlim E_\alpha$ has canonically a graded module structure of type $\Delta$ over the graded ring $A=\rlim A_\alpha$ and a multiplication such that $E^\lambda E^\mu\sub E^{\lambda+\mu}$ (where $(E^\lambda)$ denotes the graduation on $E$); then this multiplication and the graded $A$-module structure on $E$ define on $E$ a graded $A$-algebra structure of type $\Delta$. The set $E$ with this structure is called a direct limit of the direct system $(E_\alpha,f_{\beta\alpha})$ of graded algebras. Moreover, if $F$ is a graded $A$-algebra of type $\Delta$ and $(u_\alpha)$ a direct system of $A_\alpha$-homomorphism $u_\alpha:E_\alpha\to F$, $u=\rlim u_\alpha$ is an $A$-homomorphism of graded algebras.
\subsection{Tensor products and Hom sets}
Let $\Delta$ be a commutative monoid with its identity element denoted by $0$, $A$ a graded ring of type $\Delta$ and $M$, $N$ graded $A$-modules of type $\Delta$. Let $(A_\lambda)$ (resp. $(M_\lambda)$, $(N_\lambda)$) be the graduation of $A$ (resp. $M$, $N$); the tensor product $M\otimes_\Z N$ of the $\Z$-modules $M$ and $N$ is the direct sum of the $M_\lambda\otimes_\Z N_\mu$ and hence the latter form a bigraduation of types $\Delta$, $\Delta$ on this $\Z$-module. Consider on $M\otimes_\Z N$ the total graduation of type $\Delta$ associated with this bigraduation; it consists of the sub-$\Z$-modules $P_\lambda=\sum_{\mu+\nu=\lambda}(M_\mu\otimes_\Z N_\nu)$. It is known that the $\Z$-module $M\otimes_A N$ is the quotient of $M\otimes_\Z N$ by the sub-$\Z$-module $Q$ generated by the elements $(ax)\otimes y-x\otimes(ay)$, where $x\in M$, $y\in N$ and $a\in A$; if, for all $\lambda\in\Delta$, $x_\lambda$, $y_\lambda$, $a_\lambda$ are the homogeneous components of degree $\lambda$ of $x$, $y$, $a$ respectively, clearly $(ax)\otimes y-x\otimes(ay)$ is the sum of the homogeneous elements $(a_\nu x_\mu)\otimes y_\lambda-x_\mu\otimes(a_\nu y_\lambda)$, in other words $Q$ is a graded sub-$\Z$-module of $M\otimes_\Z N$ and the quotient
\[M\otimes_AN=(M\otimes_\Z N)/Q\]
therefore has canonically a graded $\Z$-module structure of type $\Delta$. Moreover, the graduation which we have just defined on $M\otimes_AN$ is compatible with its $A$-module structure. For if $x\in M_\lambda$, $y\in N_\mu$, $a\in A_\nu$, the element $a(x\otimes y)$ belong to $(M\otimes_\Z N)_{\lambda+\mu+\nu}$ and hence its image in $M\otimes_A N$ belongs to $(M\otimes_A N)_{\lambda+\mu+\nu}$, which establishes our assertion. When we speak of $M\otimes_AN$ as a graded $A$-module, we always mean with the structure thus defined, unless otherwise mentioned. Note that $(M\otimes_AN)$ can be defined as the additive group of $M\otimes_AN$ generated by the $x_\mu\otimes y_\nu$, where $x_\mu\in M_\mu$, $y_\nu\in N_\nu$ and $\mu+\nu=\lambda$.\par
Let $M'$ (resp. $N'$) be another graded $A$-module and $\phi:M\to M'$, $\psi:N\to N'$ graded homomorphisms of respective degrees $\alpha$ and $\beta$. Then it follows immediately from the above remark that $\phi\otimes\psi$ is a graded $A$-module homomorphism of degree $\alpha+\beta$.\par
Similarly, a graduation (compatible with the $A$-module structure) is similarly defined on the tensor product of any finite number of graded $A$-modules; it is moreover immediate that the associativity isomorphisms such as $(M\otimes N)\otimes P\to M\otimes(N\otimes P)$ are isomorphisms of graded modules.
\begin{proposition}
Let $M$ and $N$ be graded $A$-modules of type $\Delta$, $P$ a graded $A$-module of type $\Delta$ and let $f$ be a $A$-bilinear map of $M\times N$ to $P$ such that
\[f(x_\lambda,y_\mu)\in P_{\lambda+\mu}\quad\text{for $x_\lambda\in M_\lambda$, $y_\nu\in N_\mu$, $\lambda,\mu\in\Delta$}.\]
Then $f(x,y)=\tilde{f}(x\otimes y)$ where $\tilde{f}:M\otimes_AN\to P$ is a graded $A$-module homomorphism of degree $0$.
\end{proposition}
\begin{proof}
It is clear that $\tilde{f}$ is $A$-linear, and the given condition implies that it is a homomorphism of graded modules.
\end{proof}
Let $B$ be another graded ring of type $\Delta$ and $\rho:A\to B$ a homomorphism of graded rings; then $B$ is a graded $A$-module of type $\Delta$ via the map $\rho$. If $E$ is a graded $A$-module of type $\Delta$ and $B\otimes_AE$ is given the graded $A$-module structure of type $\Delta$ defined above, the canonical $B$-module structure is compatible with the graduation of
\[\rho_!(E)=B\otimes_AE.\]
The graded $B$-module thus obtained is said to be obtained by extending the ring of scalars to $B$ by means of $f$ and when we speak of $\rho_!(E)$ as a graded $B$-module, we always mean this structure, unless otherwise mentioned.\par
Now we define graduations on the Hom sets. For simplicity we suppose that $\Delta$ is a group. Let $A$ be a graded ring of type $\Delta$ and $M$, $N$ two graded $A$-modules of type $\Delta$. Let $H_\lambda$ denote the additive group of graded homomorphisms of degree $\lambda$ of $M$ into $N$; in the additive group $\Hom_A(M,N)$ of all homomorphisms of $M$ into $N$ (with the un-graded $A$-module structures) the sum (for $\lambda\in\Delta$) of the $H_\lambda$ is direct. For, if there is a relation $\sum_\lambda\phi_\lambda=0$ with $\phi_\lambda\in H_\lambda$ for all $\lambda$, it follows that $\sum_\lambda\phi_\lambda(x_\mu)=0$ for all $\mu$ and all $x_\mu\in M_\mu$. As the elements of $\Delta$ are cancellable, $\phi_\lambda(x_\mu)$ is the homogeneous component of $\sum_\lambda\phi_\lambda(x_\mu)$ of degree $\lambda+\mu$; hence $\phi_\lambda(x_\mu)=0$ for every ordered pair $(\mu,\lambda)$ and every $x_\mu\in M_\mu$, which implies $\phi_\lambda=0$ for all $\lambda\in\Delta$. We shall denote (in this paragraph) by $\Hom\Hom\gr_A(M,N)$ the additive subgroup of $\Hom_A(M,N)$ the sum of the $H_\lambda$ and we shall call it the additive group of graded $A$-module homomorphisms of $M$ into $N$. For the canonical $A$-module structure on $\Hom_A(M,N)$, $\Hom\gr_A(M,N)$ is a submodule and the graduation $(H_\lambda)$ is compatible with the $A$-module structure: for if $a_\nu\in A_\nu$, $x_\mu\in N_\nu$ and $\phi_\lambda\in H_\lambda$, then by definition $(a_\nu\phi_\lambda)(x_\mu)=a_\nu\phi_\lambda(x_\mu)\in N_{\lambda+\mu+\nu}$ and hence $a_\nu\phi_\lambda\in H_{\lambda+\nu}$.\par
Let $M'$ and $N'$ be two graded $A$-modules of type $\Delta$. and $\phi:M'\to M$, $\psi:N\to N'$ graded homomorphisms of respective degrees $\alpha$ and $\beta$. Then it is immediate that $\Hom(\phi,\psi):\eta\mapsto\psi\circ\eta\circ\phi$ maps $\Hom\gr_A(M,N)$ into $\Hom\gr_A(M',N')$ and that its restriction to $\Hom\gr_A(M,N)$ is a graded homomorphism into $\Hom\gr_A(M',N')$ of degree $\alpha+\beta$. In particular, $\Hom\gr_A(M,M)$ is a graded subring of $\End_A(M)$, which is denoted by $\Hom\gr_A(M)$.\par
If $M$ and $N$ are graded $A$-modules, the set $\Hom\gr_A(M,N)$ is in general distinct from $\Hom_A(M,N)$. However there is a special case where these tow sets equal.
\begin{proposition}
Let $A$ be a graded ring of type $\Delta$ and $M$ a finitely generated graded $A$-module of type $\Delta$. Then for any graded $A$-module $N$ of type $\Delta$, we have $\Hom\gr_A(M,N)=\Hom_A(M,N)$.
\end{proposition}
\begin{proof}
Assume that $M$ is generated by homogeneous elements $x_1\dots,x_n$; let $d_i$ be the degree of $x_i$. Let $\phi\in\Hom_A(M,N)$ and for all $\lambda\in\Delta$ let $z_{i,\lambda}$ denote the homogeneous component of $\phi(x_i)$ of degree $\lambda+d_i$. We show that there exists a homomorphism $\phi_\lambda:M\to N$ such that $\phi_\lambda(x_i)=z_{i,\lambda}$. for all $i$. For this, it suffices to prove that if $\sum_ia_ix_i=0$ with $a_i\in A$ for $1\leq i\leq n$, then $\sum_ia_iz_{i,\lambda}=0$ for all $\lambda\in\Delta$, for then the formula $\sum_ia_ix_i\mapsto\sum_ia_iz_{i,\lambda}$ defines a map and satisfies the condition. To this end, it can be assumed that each $a_i$ is homogeneous of degree $\delta_i$ such that $d_i+\delta_i=\mu$ for all $i$; then $\sum_ia_i\phi(x_i)=0$; taking the homogeneous component of degree $\lambda+\mu$ on the left-hand sider, we obtain $\sum_ia_iz_{i,\lambda}=0$, whence the existence of the homomorphism $\phi_\lambda$; clearly $\phi_\lambda$ is graded of degree $\lambda$. Finally, $\phi_\lambda=0$ except for a finite number of values of $\lambda$, and $\phi=\sum_\lambda\phi_\lambda$. by definition, which proves our assertion.
\end{proof}
In particular, $\Hom\gr_A(A,M)=\Hom_A(A,M)$ for every graded $A$-module $M$; moreover $\Hom_A(A,M)$ has a graded $A$-module structure, and it is immediate that with this structure the canonical map of $M$ into $\Hom_A(A,M)$ is a graded $A$-module isomorphism.\par
Similarly, $\Hom\gr_A(M,A)$ has a graded $A$-module structure; it is called the \textbf{graded dual} of the graded $A$-module $M$ and is denoted by $M^{*\gr}$, or simply $M^*$ when no confusion results. If $\phi:M\to N$ is a graded homomorphism of degree $\delta$, it follows from the above that the restriction to $N^{*\gr}$ of $\phi^*$ is a graded homomorphism of the graded dual $N^{*\gr}$ into the graded dual $M^{*\gr}$, of degree $\delta$, called the graded dual of $\phi$.\par
Let $M$, $N$, $P$, $Q$ be graded $A$-modules of type $\Delta$. Then there are canonical graded homomorphisms of degree $0$:
\[\Hom\gr_A(M,\Hom\gr_A(N,P))\to\Hom\gr_A(M\otimes_AN,P)\]
\[\Hom\gr_A(M,N)\otimes_AP\to\Hom\gr_A(M,N\otimes_AP)\]
\[\Hom\gr_A(M,P)\otimes\Hom\gr_A(N,Q)\to\Hom\gr_A(M\otimes_AN,P\otimes_AQ)\]
(the tensor products being given the graduations defined above) obtained by restricting the canonical homomorphisms for $\Hom$ sets; for, if $\phi:M\to\Hom\gr_A(N,P)$ is graded of degree $\delta$, then, for all $x\in M_\lambda$, $\phi(x)$ is a graded homomorphism $N\to P$ of degree $\delta+\lambda$ and hence, for $y\in N_\mu$, $\phi(x)(y)\in P_{\delta+\lambda+\mu}$; if $\psi:M\otimes_A N\to P$ corresponds canonically to $\phi$, it is then seen that $\psi$ is a graded homomorphisms of degree $\delta$, whence our assertion concerning the first map; moreover it is seen that this homomorphism is bijective. The argument is similar for the other maps.
\subsection{Graduation by an ordered group}
An order structure (denoted by $\leq$) on a commutative group $\Delta$ written additively is said to be compatible with the group structure if, for all $\lambda,\mu,\rho\in\Delta$, the relation $\lambda\leq\mu$ implies $\lambda+\rho\leq\mu+\rho$. The group $\Delta$ with this order structure is then called an ordered group.\par
Let $\Delta$ be an ordered commutative group, $A$ a graded ring of type $\Delta$ and $(A_\lambda)$ its graduation. We say $A$ is a \textbf{graded ring with positive degrees} if $A_\lambda=\{0\}$ for $\lambda<0$. In this case, it follows from definition that $A_+=\bigoplus_{\lambda>0}A_\lambda$ is a graded ideal of $A$.
\begin{proposition}\label{graded ring positive degree multiplication by positive scalar}
Let $\Delta$ be an ordered group, $A$ a graded ring of type $\Delta$ with positive degrees, $(A_\lambda)$ its gradualion, $M$ a graded $A$-module of type $\Delta$ and $(M_\lambda)$ its graduation. Suppose that there exists $\lambda_0$ such that $M_{\lambda_0}\neq\{0\}$ and $M_{\lambda}=\{0\}$ for $\lambda<\lambda_0$. Then $A_+M\neq M$.
\end{proposition}
\begin{proof}
Let $x$ be a non-zero element of $M_{\lambda_0}$; suppose that $x\in A_+M$. Then $x=\sum_ia_ix_i$ where the $a_i$ are nonzero homogeneous elements of $A_+$ and the $x_i$ nonzero homogeneous elements of $M$ with $\deg(x)=\deg(a_i)+\deg(x_i)$ for all $i$. But, as $\deg(a_i)>0$, $\lambda_0=\deg(a_i)+\deg(x_i)>\deg(x_i)$, which contradicts the hypothesis.
\end{proof}
\begin{corollary}\label{graded ring A_+M=M iff M=0}
Let $\Delta$ be an ordered group, $A$ a graded ring of type $\Delta$ with positive degrees. If $M$ is a finitely generated graded $A$-module such that $A_+M=M$, then $M=\{0\}$.
\end{corollary}
\begin{proof}
Suppose $M\neq\{0\}$. Let $\lambda_0$ be a minimal element of the set of degrees of a finite generating system of $M$ consisting of nonzero homogeneous elements; then the hypotheses of \cref{graded ring positive degree multiplication by positive scalar} would be fulfilled, which implies a contradiction.
\end{proof}
\begin{corollary}\label{graded ring N+A_+M=M iff M=N}
Let $\Delta$ be an ordered group, $A$ a graded ring of type $\Delta$ with positive degrees. If $M$ is a finitely generated graded $A$-module and $N$ is a graded submodule of $M$ such that $N+A_+M=M$, then $N=M$.
\end{corollary}
\begin{proof}
The quotient module $M/N$ is a finitely generated graded $A$-module and the hypothesis implies that $A_+(M/N)=M/N$, henee $M/N=0$.
\end{proof}
\begin{corollary}\label{graded ring surjective iff tensor map surjective}
Let $\Delta$ be an ordered group, $A$ a graded ring of type $\Delta$ with positive degrees. Let $\phi:M\to N$ be a graded homomorphism of graded $A$-modules, where $N$ is assumed to be finitely generated. If the homomorphism
\[\phi\otimes\id:M\otimes_A(A/A_+)\to N\otimes_A(A/A_+)\]
is surjective, then $\phi$ is surjective.
\end{corollary}
\begin{proof}
Note that $\phi(M)$ is a graded submodule of $N$ and we have 
\[(M/\phi(M))\otimes_A(A/A_+)\cong(N\otimes_A(A/A_+))/\im(\phi\otimes\id)\]
as $(A/A_+)$-modules. The hypothesis therefore implies $(N/\phi(M))\otimes_A(A/A_+)=\{0\}$ and henee $N=\phi(M)$ by \cref{graded ring A_+M=M iff M=0}.
\end{proof}
\begin{remark}
It follows from the proof of \cref{graded ring A_+M=M iff M=0} that Corollaries~\ref{graded ring A_+M=M iff M=0} and ~\ref{graded ring N+A_+M=M iff M=N} (resp. \cref{graded ring surjective iff tensor map surjective}) are still valid when, instead of assuming that $M$ (resp. $N$) is finitely generated, the following hypothesis is made: there exists a subset $\Delta^+$ of $\Delta$ satisfying the following conditions:
\begin{itemize}
\item[(a)] $M_\lambda=\{0\}$ for all $\lambda\notin\Delta^+$.
\item[(b)] every non-empty subset of $\Delta^+$ has a least element.
\end{itemize}
This will be the case if $\Delta=\Z$ and $M$ (resp. $N$) is a graded module with positive degrees.
\end{remark}
\begin{proposition}\label{graded ring module M/A_+M free imply M free}
Suppose that $\Delta=\Z$. With the hypothesis of \cref{graded ring positive degree multiplication by positive scalar}, consider the graded $A_0$-module $N=M/A_+M$ and suppose the following conditions hold:
\begin{itemize}
\item[(a)] each of the $N_\lambda$ considered as an $A_0$-module admits a basis $(y_{i\lambda})_{i\in I_\lambda}$.
\item[(b)] the canonical homomorphism $A_+\otimes_AM\to M$ is injective.
\end{itemize}
Then $M$ is a graded free $A$-module and, to be precise, if $x_{i\lambda}$ is an element of $M_\lambda$ whose image in $N_\lambda$ is $y_{i\lambda}$, the family $(x_{i\lambda})_{(i,\lambda)\in I}$ (where $I$ is the disjoint union of the $I_\lambda$) is a basis of $M$.
\end{proposition}
\begin{proof}
We know that there is a graded free $A$-module $L$ (of graduation $(L_\lambda)$) and a surjective homomorphism $p:L\to M$ of degree $0$ such that $p(e_{i\lambda})=x_{i\lambda}$ for all $(i,\lambda)\in I$, where $(e_{i\lambda})_{(i,\lambda)\in I}$ is a basis of $L$ consisting of homogeneous elements $e_{i\lambda}\in L_\lambda$. It follows from the above Remark that $p$ is surjective. Consider the graded $A$-module $R=\ker p$ and note that $R_\lambda=\{0\}$ for $\lambda<\lambda_0$ by definition; we need to prove that $R=\{0\}$ and by \cref{graded ring positive degree multiplication by positive scalar} it suffices to show that $A_+R=R$. Consider the following commutative diagram:
\[\begin{tikzcd}
0\ar[r]&A_+\otimes R\ar[d,"\phi"]\ar[r]&A_+\otimes L\ar[d,"\psi"]\ar[r,"1\otimes p"]&A_+\otimes M\ar[d,"\eta"]\ar[r]&0\\
0\ar[r]&R\ar[r]&L\ar[r,"p"]&M\ar[r]&0
\end{tikzcd}\]
where the vertical maps are induced from the canonical injection $A_+\to A$. We need to show that $\phi$ is surjective. Note that, as $L$ is free (hence flat), $\psi$ is injective. Also, $\eta$ is injective by hypothesis. Then by the snake lemma, we have an exact sequence
\[\begin{tikzcd}
R/A_+R\ar[r]&L/A_+L\ar[r,"\bar{p}"]&M/A_+M\ar[r]&0
\end{tikzcd}\]
By hypothesis we see $\bar{p}$ is a bijection, hence the claim follows.
\end{proof}
\subsection{Graded rings of type \boldmath\texorpdfstring{$\Z$}{Z}}
In this paragraph, all the graduations considered are assumed to be of type $\Z$. If $A$ (resp. $M$) is a graded ring (resp. graded module), $A_i$ (resp. $M_i$) will denote the set of homogeneous elements of degree $i$ in $A$ (resp. $M$). Recall that if $A_i=\{0\}$ (resp. $M_i=\{0\}$) for $i<0$, $A$ (resp. $M$) will, to abbreviate, be called a graded ring (resp. module) with positive degrees.
\begin{proposition}\label{graded ring generating set iff}
Let $A=\bigoplus_{i\geq 0}A_i$ be a graded ring with positive degrees and $(x_\lambda)$ a family of homogeneous elements of $A_+$. The following conditions are equivalent:
\begin{itemize}
\item[(\rmnum{1})] The ideal of $A$ generated by the family $(x_\lambda)$ is equal to $A_+$.
\item[(\rmnum{2})] The family $(x_\lambda)$ is a system of generators of the $A_0$-algebra $A$.
\item[(\rmnum{3})] For all $i\geq 0$, the $A_0$-module $A_i$ is generated by the elements of the form $\prod_\lambda x_\lambda^{n_\lambda}$ which are of degree $i$ in $A$.
\end{itemize}
\end{proposition}
\begin{proof}
Clearly conditions (\rmnum{2}) and (\rmnum{3}) are equivalent. If they hold, every element of $A_+$ is of the form $f(x_\lambda)$ where $f$ is a polynomial of $A_0[X_\lambda]$ with no constant term; then $A_+=\sum_\lambda Ax_\lambda$, which proves that (\rmnum{3}) implies (\rmnum{1}). Conversely, suppose that condition (\rmnum{1}) holds. Let $A'=A_0[x_\lambda]$ be the sub-$A_0$-algebra of $A$ generated by the family $(x_\lambda)$ and let us show that $A'=A$. For this, it is sufficicnt to show that $A_i\sub B$ for all $i\geq 0$. We proceed by induction on $i$, the property being obvious for $i=0$. Then let $y\in A_i$ with $i\geq 1$. Sincey $y\in A_+$, which is the ideal generated by $(x_\lambda)$, there exists a family $(a_\lambda)$ of elements of $A$ of finite support such that $y=\sum a_\lambda x_\lambda$ and we may assume that each of the $a_\lambda$ is homogeneous of degree $i-\deg(x_\lambda)$ (by replacing it if need be by its homogeneous component of that degree); as $\deg(x_\lambda)>0$, the induction hypothesis shows that $a_\lambda\in A'$ for all $\alpha$, whence $y\in A'$ and $A_i\sub A'$, which proves that (\rmnum{1}) implies (\rmnum{2}).
\end{proof}
\begin{proof}
Part (a) follows from \cref{graded ring generating set iff}. As for (b), the condition in (b) is sufficient by Hilbert's basis theorem. If $A$ is Noetherian, $A_0=A/A_+$ is Noetherian and $A_+$ is finitely generated as an ideal. Thus $A$ is an $A_0$-algebra of finite type by (a).
\end{proof}
\begin{corollary}\label{graded ring finitely generated criterion}
Let $A=\bigoplus_{i\geq 0}A_i$ be a graded ring with positive degrees.
\begin{itemize}
\item[(a)] The ideal $A_+$ is finitely generated if and only if $A$ is an $A_0$-algebra of finite type.
\item[(b)] The ring $A$ is Noetherian if and only if $A_0$ is Noetherian and $A$ is an $A_0$-algebra of finite type.
\item[(c)] Suppose that the conditions in (a) hold and let $M=\bigoplus_{i\in\Z}M_i$ be a finitely generated graded $A$-module. Then, for all $i\in\Z$, $M_i$ is a finitely generated $A_0$-module and there exists $i_0$ such that $M_i=\{0\}$ for $i<i_0$.
\end{itemize}
\end{corollary}
\begin{proof}
Part (a) follows from \cref{graded ring generating set iff}. As for (b), the condition in (b) is sufficient by Hilbert's basis theorem. If $A$ is Noetherian, $A_0=A/A_+$ is Noetherian and $A_+$ is finitely generated as an ideal. Thus $A$ is an $A_0$-algebra of finite type by (a).\par
We prove (c). We may suppose that $A$ is generated (as an $A_0$-algebra) by homogeneous elements $a_1,\dots,a_r$ of degree $i\geq 1$ and $M$ is generated (as an $A$-module) by homogeneous elements $x_1,\dots,x_n$; let $h_i=\deg(a_i)$ and $k_j=\deg(x_j)$. Clearly $M_n$ consists of the linear combinations with coefficients in $A_0$ of the elements $a_1^{\alpha_1}\cdots a_r^{\alpha_r}x_j$ such that the $\alpha_i$ are positive integers satisfying the relation $k_j+\sum_{i=1}^{r}h_i\alpha_i=n$; for each $n$ there is only a finite number of families $(\alpha_i)_{i=1}^{r}$ satisfying these conditions, since $h_i\geq 0$ for all $i$; we conclude that $M_n$ is a finitely generated $A_0$-module and moreover clearly $M_n=\{0\}$ when $n<\inf(k_i)$.
\end{proof}
Let $A=\bigoplus_{i\in\Z}A_i$ be a graded ring and $M=\bigoplus_{i\in\Z}M_i$ a graded $A$-module; for each ordered pair $(d,r)$ with $d\geq 1$ and $0\leq r\leq d-1$, set
\[A^{(d)}=\bigoplus_{i\in\Z}A_{id},\quad M^{(d,r)}=\bigoplus_{i\in\Z}M_{id+r}.\]
Clearly $A^{(d)}$ is a graded subring of $A$ and $M^{(d,r)}$ a graded $A^{(d)}$-module; moreover, if $N$ is a graded submodule of $M$, $N^{(d,r)}$ is a graded sub-$A^{(d)}$-module of $M^{(d,r)}$. We shall write $M^{(d)}$ instead of $M^{(d,0)}$; for each $d\geq 1$, $M$ is the direct sum of the $A^{(d)}$-modules $M^{(d,r)}$ for $0\leq r\leq d-1$.
\begin{proposition}\label{graded ring finiteness of alter ring}
Let $A$ be a graded ring with positive degrees and $M=\bigoplus_{i\in\Z}M_i$ be a graded $A$-module. Suppose that $A$ is an $A_0$-algebra of finite type and $M$ a finitely generated $A$-module. Then 
\begin{enumerate}
\item[(a)] For every ordered pair $(d,r)$ of integers such that $d\geq 1$ and $0\leq r\leq d-1$, $M^{(d,r)}$ is a finitely generated $A^{(d)}$-module.
\item[(b)] $A_+^{(d)}$ is a finitely generated $A^{(d)}$-module and therefore $A^{(d)}$ is an $A_0$-algebra of finite type.
\end{enumerate}
\end{proposition}
\begin{proof}
Let us show that $A$ is a finitely generated $A^{(d)}$-module. Let $a_1,\dots,a_s$ be a system of generators of the $A_0$-algebra $A$ consisting of homogeneous elements. The elements of $A$ (finite in number) of the form $a_1^{\alpha_1}\cdots a_s^{\alpha_s}$ such that $0\leq a_i\leq d$ for $1\leq i\leq s$ constitute a system of generators of the $A^{(d)}$-module $A$; for every system of integers $n_i\geq 0$, $1\leq i\leq s$, there are positive integers $q_i$ and $r_i$ such that $n_i=q_id+r_i$, where $0\leq r_i<d$ for $1\leq i\leq s$; then
\[a_1^{n_1}\cdots a_s^{n_s}=(a_1^{q_1}\cdots a_s^{q_s})(a_1^{r_1}\cdots a_s^{r_s}).\]
which proves our assertion. Then, if $M$ is a finitely generated $A$-module, it is also a finitely generated $A^{(d)}$-module; as $M$ is the direct sum of the $M^{(d,r)}$ for $0\leq r\leq d-1$, each of the $M^{(d,r)}$ is a finitely generated $A^{(d)}$-module, which proves (a).\par
Now by \cref{graded ring finitely generated criterion}, $A_+$ is a finitely generated $A$-module, so (a) implies $A_+^{(d)}$ is a finitely-generated $A_0$-module, which again by \cref{graded ring finitely generated criterion} implies $A^{(d)}$ is a finite $A_0$-algebra.
\end{proof}
\begin{proposition}\label{graded ring finite module M_n+d=A_dM_n}
Let $A=\bigoplus_{i\geq 0}A_i$ be a graded ring with positive degrees which is a finitely generated $A_0$-algebra and $M=\bigoplus_{i\in\Z}M_i$ a graded $A$-module. Let $(x_\lambda)_{\lambda\in I}$ be a system of homogeneous generators of $M$ such that $\sup_\lambda\deg(x_\lambda)<+\infty$. Then there exists $n_0>0$ and $d>0$ such that, for any $n\geq n_0$ we have $M_{n+d}=A_dM_n$.
\end{proposition}
\begin{proof}
Let $A$ be generated as an $A_0$-algebra by homogeneous elements $a_1,\dots,a_s$, and set $d_i=\deg(a_i)$. Let $d$ be a common multiple of all the $d_i$ and set $b_i=a_i^{d/d_i}$ for $1\leq i\leq s$, so that $\deg(b_i)=d$ for all $i$. Consider the set $Z$ of elements of the form $a_1^{\alpha_1}\cdots a_s^{\alpha_s}x_\lambda$ with $0\leq \alpha_i\leq d/d_i$ for $1\leq i\leq s$ and $\lambda\in I$. Then by the hypothesis on $(x_\lambda)$, we can choose $n_0>0$ be larger than the degree of every element of $Z$. If $n\geq n_0$ then any element of $M_{n+d}$ can be written as an $A_0$-linear combination of elements $a_1^{n_1}\cdots a_s^{n_s}x_j$ as above. If $n_i=(d/d_i)\beta_i+\gamma_i$ with $\gamma_i<d/d_i$, then we can write
\[a_1^{n_1}\cdots a_s^{n_s}x_j=b_1^{\beta_1}\cdots b_s^{\beta_s}(a_1^{\gamma_1}\cdots a_s^{\gamma_s}x_j)=b_1^{\beta_1}\cdots b_s^{\beta_s}z\]
where every $b_i$ is homogenous of degree $d$ and $z\in Z$ is homogeneous of some degree $t$. Note that $n+d>n_0\geq t$, so the last expression makes sense. Now it is clear that $b_1^{\beta_1}\cdots b_s^{\beta_s}z$ belongs to $A_dM_n$, so $M_{n+d}=A_dM_n$.
\end{proof}
\begin{corollary}\label{graded ring bounded module M_n+d=A_dM_n}
Let $A$ be a graded ring such that $A=A_0[A_1]$, $M$ a graded $A$-module and $(x_\lambda)_{\lambda\in I}$ a system of homogeneous generators of $M$ such that $\deg(x_\lambda)\leq n_0$ for all $\lambda\in I$. Then for all $n\geq n_0$ and all $d>0$, $M_{n+d}=A_dM_n$.
\end{corollary}
\begin{proof}
In this case the integer $d$ in \cref{graded ring finite module M_n+d=A_dM_n} can be chosen to be $1$, and the set $Z$ has elements of degree smaller than $n_0$. Thus we have $M_{n+d}=A_1M_n$ for $n\geq n_0$, and by induction $M_{n+d}=A_dM_n$ for $n\geq n_0$ and $d>0$.
\end{proof}
\begin{corollary}\label{graded ring bounded algebra S^(d)=S_0[S_d]}
Let $A$ be a graded ring such that $A=A_0[A_1]$ and let $S=\bigoplus_{i\geq 0}S_i$ be a graded $A$-algebra with positive degrees which is a finitely generated $A$-module. Then there exists an integer $n_0>0$ such that:
\begin{enumerate}
\item[(a)] For $n\geq n_0$ and $d>0$, $S_{n+d}=S_dS_n$.
\item[(b)] For $d\geq n_0$, $S^{(d)}=S_0[S_d]$.
\end{enumerate}
\end{corollary}
\begin{proof}
By \cref{graded ring bounded module M_n+d=A_dM_n} there exists an integer $n_0\geq 0$ such that, for $n\geq n_0$ and $d>0$, $S_{n+d}=A_dS_n$, whence a fortiori $S_{n+d}=S_dS_n$, which establishes (a). Then, for $d\geq n_0$ and $i>0$, $S_{id}=(S_d)^i$ as follows by induction on $i$ applying (a); this establishes (b).
\end{proof}
\begin{corollary}
Let $A=\bigoplus_{i\geq 0}A_i$ be a graded ring with positive degrees which is a finitely generated $A_0$-algebra. There exists an integer $d>0$ such that $A^{(md)}=A_0[A_{md}]$ for all $m\geq 1$.
\end{corollary}
\begin{proof}
In \cref{graded ring finite module M_n+d=A_dM_n} we can assume that $d\geq n_0$ (since the proof works with $d$ any common multiple of the $d_i$). And therefore with $A=M$ we have $A_{d+d}=(A_d)^2$ and inductively $A_{id}=(A_d)^i$ for all $i>0$. This shows the claim.
\end{proof}
\begin{proposition}\label{graded ring integral iff}
Let $\Delta$ be an ordered group and $A$ a graded ring of type $\Delta$. If the product in $A$ of two homogeneous nonzero elements is nonzero, then the ring $A$ is integral.
\end{proposition}
\begin{proof}
Let $x=\sum_\lambda x_\lambda$ and $y=\sum_\lambda y_\lambda$ be two non-zero elements of $A$, with $x_\lambda$, $y_\lambda$ being homogeneous of degree $\lambda$ for all $\lambda\in\Delta$. Let $\alpha$ (resp. $\beta$) be the greatest of the elements $\lambda\in\Delta$ such that $x_\lambda\neq 0$ (resp. $y_\lambda\neq 0$); it is immediate that if $\lambda\neq\alpha$ or $\mu\neq\beta$, either $x_\lambda y_\mu=0$ or $\deg(x_\lambda y_\mu)<\lambda+\alpha$; the homogeneous component of $xy$ of degree $\alpha+\beta$ is therefore $x_\lambda y_\beta$, which is non-zero by hypothesis; whence $xy\neq 0$.
\end{proof}
\begin{proposition}\label{graded prime ideal iff}
Let $\p=\bigoplus_{i\geq 0}\p_i$ be a graded ideal of $A$; for $\p$ to be prime, it is necessary and sufficient that for homogeneous elements $x,y\in A$, $xy\notin\p$ if and only if $x\notin\p$ and $y\notin\p$.
\end{proposition}
\begin{proof}
The condition is obviously necessary. Conversely, if it is fulfilled, then in the graded ring $A/\p=\bigoplus_{i\geq 0}A_i/\p_i$ the product of two homogeneous nonzero elements is nonzero and hence $A/\p$ is an integral domain.
\end{proof}
Let $\a$ be an arbitrary ideal of $A$. We can associate to it a homogeneous ideal $\a^h=\bigoplus_{i\geq 0}(\a\cap A_i)$. It follows from the definition that $\a^h\sub\a$ and $\a$ is homogeneous if and only if $\a=\a^h$.
\begin{proposition}\label{graded ring associated prime ideal}
Let $A$ be a graded ring with positive degrees and $\p\sub A$ be a prime ideal. Let $\p^h$ be the homogeneous ideal generated by the homogeneous elements of $\p$. Then $\p^h$ is a prime ideal of $A$.
\end{proposition}
\begin{proof}
For any homogeneous element $f,g\in A$ such that $fg\in\p^h\sub\p$, we have $f\in\p$ or $g\in\p$. Then by the definition of $\p^h$, this implies $f\in\p^h$ or $g\in\p^h$. Thus $\p^h$ is prime by \cref{graded prime ideal iff}. 
\end{proof}
\begin{corollary}\label{graded ring minimal prime are graded}
Let $A=\bigoplus_{i\geq 0}A_i$ be a graded ring with positive degrees.
\begin{itemize}
\item[(a)] Any minimal prime of $A$ is a homogeneous ideal.
\item[(b)] Given a homogeneous ideal $\a\sub A$ any minimal prime over $\a$ is homogeneous.
\end{itemize}
\end{corollary}
\begin{proof}
The first claim is a direct application of \cref{graded ring associated prime ideal}, since the ideal $\p^h$ is contained in $\p$. The second follows by applying the result on $A/\a$.
\end{proof}
Let $A$ be a graded ring with positive degrees. Two graded ideals $\a=\bigoplus_{i\geq 0}\a_i$ and $\b=\bigoplus_{i\geq 0}\b_i$ of $A$ are said to be \textbf{equivalent} if there exists an integer $n_0$ such that $\a_n=\b_n$ for $n\geq n_0$ (clearly it is an equivalence relation). A graded ideal is called \textbf{essential} if it is not equivalent to $A_+$. As we will see, this notation plays an important role when we study graded prime ideals.
\begin{proposition}\label{graded ring contained in prime ideal iff}
Let $A$ be a graded ring with positive degrees, $\a$ a graded ideal, and $\p$ a graded prime ideal of $A$ with $\p\nsupseteq A_+$. Then $\a\sub\p$ if and only if there exists an integer $n_0\geq 0$ such that $\a_n\sub\p_n$ for all $n\geq n_0$.
\end{proposition}
\begin{proof}
One implication is clear. Conversely, assume that $a_n\sub\p_n$ for $n\geq n_0$. We now show that $a_{n_0-1}\sub\p_{n_0-1}$. Since $\p\nsupseteq A_+$, there exist $e>0$ and $a\in A_e$ with $a\notin\p$. If $b\in\a_{n_0-1}$ then $ab\in\a_{n_0-1+e}\sub\p_{n_0-1+e}$, so $b\in\p$ by primeness. This proves the claim.
\end{proof}
\begin{corollary}\label{graded ring prime ideal equal iff equivalent}
Let $A$ be a graded ring with positive degrees and $\p$, $\q$ be graded prime ideals and $\p$ with $\p\nsupseteq A_+$ and $\q\nsupseteq A_+$. Then $\p=\q$ if and only if they are equivalent.
\end{corollary}
\begin{proposition}\label{graded ring coincide partly with prime ideal iff}
Let $\a$ be a graded ideal of $A$ and $n_0>0$ an integer. For there to exist a graded prime ideal $\p$ such that $\p_n=\a_n$ for $n\geq n_0$, it is necessary and sufficient that, for any homogeneous elements $x$, $y$ of degrees $\geq n_0$, the relation $xy\in\a$ implies $x\in\a$ or $y\in\a$. If there exists $n\geq n_0$ such that $\a_n\neq A_n$, then the prime ideal satisfying the above condition is unique.
\end{proposition}
\begin{proof}
The condition of the statement is obviously necessary. If $\a_n=A_n$ for all $n\geq n_0$, clearly every prime ideal containing $A_+$ is a solution to the problem; there may therefore be several prime ideals which solve the problem; however, any two of these ideals are obviously equivalent.\par
Now suppose that there exists a homogeneous element $a\in A_d$ with $d\geq n_0$ not belonging to $\a_d$. Now we define
\[\p=\{x\in A:ax\in\a\}.\]
Clearly $\p$ is an ideal of $A$; as the homogeneous components of $ax$ are the products by $a$ of those of $x$ and $a$ is a graded ideal, $\p$ is a graded ideal; moreover, $1\notin\p$ and hence $\p\neq A$. To prove that $\p$ is prime, let $x\in A_m$ and $y\in A_n$ satisfy $x\notin\p$ and $y\notin\p$. Then $ax\notin\a_{m+d}$ and $ay\notin\a_{n+d}$, hence $a^2xy\notin\a_{m+n+2d}$ by hypothesis. Since $\a$ is an ideal, we then get $axy\notin\a_{m+n+d}$, so $xy\in\p$. Now if $n\geq n_0$ and $a\in A$, the conditions $x\in\a_n$ and $ax\in\a_{n+d}$ are equivalent by hypothesis and hence $\p\cap A_n=\a_n$, which completes the proof of the existence of the graded prime ideal $\p$ which solves the problem. The uniqueness part now follows from \cref{graded ring prime ideal equal iff equivalent}.
\end{proof}
\begin{proposition}\label{graded ring essential prime of A^(d)}
Let $A=\bigoplus_{i\geq 0}A_i$ be a graded ring with positive degrees and $d>0$ an integer.
\begin{itemize}
\item[(a)] For every essential graded prime ideal $\p$ of $A$, $\p\cap A^{(d)}$ is an essential graded prime ideal of $A^{(d)}$.
\item[(b)] Conversely, for every essential graded prime ideal $\p'$ of $A^{(d)}$, there exists a unique (necessarily essential) graded prime ideal $\p$ of $A$ such that $\p\cap A^{(d)}=\p'$.
\end{itemize}
\end{proposition}
\begin{proof}
If $a\in A_i$ does not belong to $\p_i$, then $a^{d}$ does not belong to $\p_{id}$, and hence $\p\cap A^{(d)}$ is essential. This proves (a). Now let $\p'$ be an essential graded prime ideal of $A^{(d)}$. If $\p$ is a graded prime ideal of $A$ such that $\p\cap A^{(d)}=\p'$, then for all $n\geq 0$, the set $\p\cap A_n$ must be equal to the set $\a_n$ of $x\in A_n$ such that $x^d\in\p'$. Let us show that $\a=\bigoplus_{n\in\N}\a_n$ is a graded prime ideal, which will proves (b). As $\p'$ is prime, $\a_n=\p_n$ when $n$ is a multiple of $d$. Now if $x\in\a_n$, $y\in\a_n$, then $(x-y)^{2d}$ is the sum of terms each of which is a product of $x^d$ or $y^d$ by a homogeneous element of degree $nd$ and hence $(x-y)^{2d}\in\p'$ and, since $\p'$ is prime, $(x-y)^d\in\p'$ and therefore $\a_n$ is a subgroup of $A$. As $\p'$ is an ideal of $A^{(d)}$, $\a$ is a graded ideal of $A$; finally, the relation $(xy)^d\in\p'$ implies $x^d\in\p'$ or $y^d\in\p'$, which completes the proof.
\end{proof}
Let $A$ be a graded ring with positive degrees and $\p$ an essential graded prime ideal of $A$. The set $S$ of homogeneous elements of $A$ not belonging to $\p$ is multiplicative and the ring of fractions $S^{-1}A$ is therefore graded canonically (note that there will in general be homogeneous nonzero elements of negative degree in this graduation). We shall denote by $A_\p$ the subring of $S^{-1}A$ consisting of the homogeneous elements of degree $0$, in other words the set of fractions $x/s$, where $x$ and $s$ are homogeneous of the same degree in $A$ and $s\notin\p$. Similarly, for every graded $A$-module $M$, $S^{-1}M$ is graded canonically (loc. cit.) and we shall denote by $M_\p$ the subgroup of homogeneous elements of degree $0$, which is obviously an $A_\p$-module.
\begin{proposition}\label{graded ring prime of A^(d) bijective}
Let $\p$ be a graded prime ideal of $A$, $d>0$ an integer, and $\p'$ be the graded prime ideal $\p\cap A^{(d)}$ of $A^{(d)}$. For every graded $A$-module $M$, the homomorphism $(M^{(d)})_{\p'}\to M_\p$ induced from the canonical injection $M^{(d)}\to M$ is bijective.
\end{proposition}
\begin{proof}
If $S$ is the set of homogeneous elements of $A$ not belonging to $\p$ and $S^{(d)}=S\cap A^{(d)}$, the canonical homomorphism $\phi:(S^{(d)})^{-1}M^{(d)}\to S^{-1}M$ is a homogeneous homomorphism of degree $0$ and it is injective, for, if $x\in M_{nd}$ satisfies $sx=0$ for $s\in A_m$, $s\notin\p$, then also $s^dx=0$ and $s^d\in A_{md}$, $s^d\notin\p'$. It remains to show that the image under $\phi$ of $(M^{(d)})_{\p'}$ is the whole of $M_\p$; but if $x\in M_n$, $s\in A_n$ and $s\notin\p$, then also $x/s=(xs^{d-1})/s^d$ where $xs^{d-1}\in A_{nd}$, $s^d\in A_{md}$ and $s^d\notin\p'$, whence our assertion.
\end{proof}
\begin{proposition}\label{graded ring prime avoidence}
Suppose $A$ is a graded ring with positive degrees, $\p_1,\dots,\p_r$ homogeneous prime ideals and $\a$ a homogeneous ideal. Assume that $\a\cap A_+\nsubseteq\p_i$ for all $i$, then there exists a homogeneous element $x\in\a\cap A_+$ of positive degree such that $x\notin\p_i$ for $1\leq i\leq r$.
\end{proposition}
\begin{proof}
We may assume that $\a\sub A_+$ and there are no inclusions among the $\p_i$. The result is true for $r=1$. Suppose the result holds for $r-1$. Pick $x\in\a$ homogeneous of positive degree such that $x\notin\p_i$ for all $i=1,\dots,r-1$. If $x\notin\p_r$ we are done, so assume that $x\in\p_r$. Since $\a\p_1\cdots\p_{r-1}$ is a homogeneous ideal that is not contained in $\p_r$, there is a homogeneous element $y\in\a\p_1\cdots\p_{r-1}$ of positive degree such that $y\notin\p_r$. Then $x^{\deg(y)}+y^{\deg(x)}$ works.
\end{proof}
\section{Filtration and topologies}
\subsection{Filtrated rings and modules}
\begin{definition}
An increasing (resp. decreasing) sequence $(G_n)_{n\in\Z}$ of subgroups of a commutative group $G$ is called an \textbf{increasing (resp. decreasing) filtration} on $G$. A commutative group with a filtration is called a filtered group.
\end{definition}
If $(G_n)_{n\in\Z}$ is an increasing (resp. decreasing) filtration on a commutative group $G$ and we write $G'_n=G_{-n}$, clearly $(G'_n)_{n\in\Z}$ is a decreasing (resp. increasing) filtration on $G$. We may therefore restrict our study to decreasing filtrations and hence forth when we speak of a filtration, we shall mean a decreasing filtration, unless otherwise stated.\par
Given a decreasing filtration $(G_n)_{n\in\Z}$ on a commutative group $G$, clearly $\bigcap_{n\in\Z}G_n$ and $\bigcup_{n\in\Z}G_n$ are subgroups of $G$; the filtration is called \textbf{separated} if $\bigcap_{n\in\Z}G_n$ is reduced to the identity element and \textbf{exhaustive} if $\bigcup_{n\in\Z}G_n=G$.
\begin{definition}
Given a ring $A$, a filtration $(A_n)_{n\in\Z}$ over the additive group $A$ is called compatible with the ring structure on $A$ if
\begin{itemize}
\item[(a)] $A_mA_n\sub A_{m+n}$ for all $m,n\in\Z$.
\item[(b)] $1\in A_0$.
\end{itemize}
The ring $A$ with this filtration is then called a \textbf{filtered ring}.
\end{definition}
Conditions (a) and (b) show that $A_0$ is a subring of $A$ and the $A_n$'s are $A_0$-modules. The set $B=\bigcup_{n\in\Z}A_n$ is a subring of $A$ and the set $\n=\bigcap_{n\in\Z}A_n$ is an ideal of $B$; for if $x\in\n$ and $a\in A_p$, then for all $k\in\Z$ we have $x\in A_{k-p}$, whence $ax\in A_k$ by (a); therefore $ax\in\n$. An important particular case is that $A_0=A$; then $A_n=A$ for all $n\leq 0$ and all the $A_n$ are ideals of $A$.
\begin{definition}
Let $A$ be a filtered ring, $(A_n)_{n\in\Z}$ its filtration and $M$ an $A$-module. A filtration $(M_n)_{n\in\Z}$ on $M$ is called compatible with its module structure over the filtered ring $A$ if
\begin{align}\label{filtered module-1}
A_mM_n\sub M_{m+n}\quad\text{for all $m,n\in\Z$}.
\end{align}
The $A$-module $M$ with this filtration is called a \textbf{filtered module}.
\end{definition}
If $M$ is a filtered module with filtration $(M_n)_{n\in\Z}$, then the $M_n$ are all $A_0$-modules; if $B=\bigcup_{n\in\Z}A_n$, clearly $\bigcup_{n\in\Z}M_n$ is a $B$-module and so is $\bigcap_{n\in\Z}M_n$ by the same argument as above for $\n$. If $A_0=A$, all the $M_n$'s are submodules of $M$.
\begin{example}
On a ring $A$ the sets $A_n$ such that $A_n=0$ for $n>0$, $A_n=A$ for $n\leq 0$ form what is called a \textbf{trivial filtration} associated with the trivial graduation on $A$; on an $A$-module $M$, every filtration $(M_n)$ consisting of sub-$A$-modules is then compatible with the module structure on $M$ over the filtered ring $A$. Then it is possible to say that every filtered commutative group $G$ is a filtered $\Z$-module, if $\Z$ is given the trivial filtration. 
\end{example}
\begin{example}
Let $A$ be a graded ring of type $\Z$; for all $i\in\Z$, let $A^n$ be the subgroup of homogeneous elements of degree $n$ in $A$ and set $A_n=\bigoplus_{i\geq n}A^i$. Then it is immediate that $(A_n)$ is an exhaustive and separated decreasing filtration which is compatible with the ring structure on $A$; this filtration is said to be associated with the graduation $(A^n)$ and the filtered ring $A$ is said to be associated with the given graded ring $A$.\par
Now let $M$ be a graded module of type $\Z$ over the graded ring $A$ and for all $i\in\Z$ let $M^n$ be the subgroup of homogeneous elements of degree $n$ of $M$. If $M_n=\bigoplus_{i\geq n}M^i$, then $(M_n)$ is an exhaustive and separated decreasing filtration which is compatible with the module structure on $M$ over the filtered ring $A$; this filtration is said to be \textbf{associated} with the graduation $(M^n)$ and the filtered module $M$ is said to be \textbf{associated} with the given graded module $M$.
\end{example}
\begin{example}
Let $A$ be a filtered ring, $(A_n)_{n\in\Z}$ its filtration and $M$ an $A$-module. Let us write $M_n=A_nM$; it follows that
\[A_mM_n=A_mA_nM\sub A_{m+n}M=M_{m+n}\]
and that $M_0=M$; therefore $(M_n)$ is an exhaustive filtration which is compatible with the $A$-module structure on $M$. This filtration is said to be \textbf{derived} from the given filtration $(A_n)$ on $A$; note that it is not necessarily separated, even if $(A_n)$ is separated.
\end{example}
\begin{example}[\textbf{The $\a$-adic filtration}]
Let $A$ be a ring and $\a$ an ideal of $A$. Let us write $A_n=\a^n$ for $n>0$ and $A_n=A$ for $n\leq 0$. It is immediate that $(A_n)$ is an exhaustive filtration on $A$, called the \textbf{$\a$-adic filtration}. Let $M$ be an $A$-module; the filtration $(M_n)$ derived from the $\a$-adic filtration on $A$ is called the $\a$-adic filtration on $M$. If $B$ is an $A$-algebra, then $\b=\a^e$ is an ideal of $B$ and for every $B$-module $N$ and $n\in\Z$ we have $\b^nN=\a^nN$, therefore the $\b$-adic filtration on $N$ coincides with the $\a$-adic filtration (if $N$ is considered as an $A$-module).
\end{example}
Let $G$ be a filtered group and $(G_n)_{n\in\Z}$ its filtration; clearly, for every subgroup $H$ of $G$, $(H\cap G_n)_{n\in\Z}$ is a filtration said to be \textbf{induced} by that on $G$. Similarly, if $H$ is a normal subgroup of $G$, the family $((H+G_n)/H)_{n\in\Z}$ is a filtration on the group $G/H$, called the \textbf{quotient} under $H$ of the filtration on $G$.\par
If $G$ and $G'$ are filtered groups and $(G_n)$, $(G'_n)$ are their filtrations. Then $(G_n\times G_n')_{n\in\Z}$ is a filtration on $G\times G'$ called the \textbf{product} of the filtrations on $G$ and $G'$, which is exhaustive (resp. separated) if $(G_n)$ and $(G_n')$ are.\par
Now let $A$ be a filtered ring and $(A_n)$ its filtration; on every subring $B$ of $A$, clearly the filtration induced by that on $A$ is compatible with the ring structure on $B$. If $\a$ is an ideal of $A$, the quotient filtration on $A/\a$ of that on $A$ is compatible with the structure of this ring, for
\[(A_m+\a)(A_n+\a)\sub A_mA_n+\a\sub A_{m+n}+\a.\]
If $A'$ is another filtered ring, the product filtration on $A\times A'$ is compatible with the structure of this ring.\par
Simialrly, let $M$ be a filtered $A$-module and $(M_n)$ its filtration; on every submodule $N$ of $M$, the filtration induced by that on $M$ is compatible with the $A$-module structure on $N$ and, on the quotient module $M/N$, the quotient filtration of that on $M$ is compatible with the $A$-module structure, as
\[A_m(N+M_n)\sub N+A_mM_n\sub N+M_{m+n}.\]
Note that if the filtration on $M$ is derived from that on $A$, so is the quotient filtration on $M/N$ because we have $(N+A_nM)/N=A_n(M/N)$. However, this is not true in general for the filtration induced on $N$.\par
Let $A$ be a filtered ring, $M$ a filtered $A$-module and $(M_n)$ the filtration of $M$. For all $x\in M$ let $v(x)$ denote the least upper bound in $\widebar{\R}$ of the set on integers $n\in\Z$ such that $x\in M_n$. Then the following equivalences hold:
\[\begin{cases}
v(x)=+\infty&x\in\bigcap_{n\in\Z}M_n\\
v(x)=p&x\in M_p\setminus M_{p+1}\\
v(x)=-\infty&x\notin\bigcup_{n\in\Z}M_n
\end{cases}\]
Thé map $v:M\to\widebar{\R}$ is called the \textbf{order function} of the filtered module $M$. If $v$ is known then so are the $M_n$, for $M_n$ is the set of $x\in M$ such that $v(x)\geq n$; the fact that the $M_n$ are additive subgroups of $M$ implies the relation
\begin{align}\label{filtration order function inequality-1}
v(x+y)\geq\min\{v(x),v(y)\}
\end{align}
The above definition applies in particular to the filtered $A$-module $A$. If $v_A$ is the order function of $A$ and $v_M$ the order function on $M$, it follows from (\ref{filtered module-1}) that for $a\in A$, and $x\in M$,
\[v_M(ax)\geq v_A(a)+v_M(x)\]
whenever the right-hand side is defined. In particular, for $a\in A$ and $b\in A$, we have $v_A(ab)\geq v_A(a)+v_A(b)$.
\subsection{The associated graded module}
Let $G$ be a commutative group (written additively) and $(G_n)$ a filtration on $G$. Let us write 
\[\gr_n(G)=G_n/G_{n+1},\quad \gr(G)=\bigoplus_{n\in\Z}\gr_n(G).\]
The commutative group $\gr(G)$ is then a graded group of type $\Z$, called the graded group associated with the filtered group $G$, the homogeneous elements of degree $n$ of $\gr(G)$ being those of $\gr_n(G)$.\par
Now let $A$ be a filtered ring, $(A_n)$ its filtration, $M$ a filtered $A$-module and $(M_n)$ its filtration. For all $p\in\Z$, $q\in\Z$, we can define a map
\begin{align}\label{filtration induce graduation-1}
\gr_p(A)\times\gr_q(M)\to\gr_{p+q}(M),\quad (\bar{a},\bar{x})\mapsto\widebar{ax}.
\end{align}
This is well-defined in view of the equality $ax-by=(a-b)x+b(x-y)$ and the relations $A_{p+1}M_q\sub M_{p+q+1}$, $A_pM_{q+1}\sub M_{p+q+1}$. It is immediate that the map defined is $\Z$-bilinear; by linearity, we derive a $\Z$-bilinear map
\[\gr(A)\times\gr(M)\to\gr(M).\]
If this definition is first applied to the case $M=A$, the map (\ref{filtration induce graduation-1}) is an internal law of composition on $\gr(A)$, which it is immediately verified is associative and has an identity element which is the canonical image in $\gr_0(A)$ of the unit element of $A$; it therefore defines on $\gr(A)$ a ring structure and the graduation $(\gr_n(A))_{n\in\Z}$ is by definition compatible with this structure. The graded ring $\gr(A)$ (of type $\Z$) thus defined is called the \textbf{graded ring associated with the filtered ring} $A$. The map (\ref{filtration induce graduation-1}) is on the other hand a $\gr(A)$-module external law on $\gr(M)$, the module axioms being trivially satisfied, and the graduation $(\gr_n(M))_{n\in\Z}$ on $\gr(M)$ is obviously compatible with this module structure. The graded $\gr(A)$-module $\gr(M)$ (of type $\Z$) thus defined is called the graded module associated with the filtered $A$-module $M$.
\begin{example}
Let $A$ be a ring and $t$ an element of $A$ which is not a divisor of $0$. Let us give $A$ the $(t)$-adic filtration. Then the associated graded ring $\gr(A)$ is canonically isomorphic to the polynomial ring $(A/(t))[X]$. For $\gr_n(A)=0$ for $n<0$ and by definition the ring $\gr_0(A)$ is the ring $A/(t)$. We now note that by virtue of the hypothesis on $t$ the relation $at^n\equiv 0$ mod $(t^{n+1})$ is equivalent to $a\equiv 0$ mod $t$; if $\tau$ is the canonical image of $t$ in $\gr_1(A)$, every element of $\gr_n(A)$ may then be written uniquely in the form $\alpha\tau^n$, where $\alpha\in\gr_0(A)$; whence our assertion.
\end{example}
\begin{example}
Let $K$ be a ring and $A$ the ring of formal power series
\[A=K\llbracket X_1,\dots,X_r\rrbracket.\]
Let $\m$ be the ideal of $A$ whose elements are the formal power series with no constant term. Let us give $A$ the $\m$-adic filtration; if $M_1,\dots,M_s$ are the distinct monomials in $X_1,\dots,X_r$ of total degree $n-1$, clearly every formal power series $u$ of total order $\deg(u)\geq n$ may be written as $\sum_{k=1}^{s}u_kM_k$, where the $u_k$ belong to $\m$; it is seen that $\m^n$ is the set of formal power series $u$ such that $\deg(u)\geq n$, which shows that $\deg$ is the order function for the $\m$-adic filtration. Then clearly, for every formal power series $u\in\m^n$, there exists a unique homogeneous polynomial of degree $n$ in the $X_i$'s which is congruent to $u$ mod $\m^{n+1}$, namely the sum of terms of degree $n$ of $u$; we conclude that $\gr(A)$ is canonically isomorphic to the polynomial ring $K[X_1,\dots,X_r]$.
\end{example}
\begin{example}\label{filtration of I-adic isomorphic to symmetric algebra}
More generally, let $A$ be a ring and $\a$ an ideal of $A$ and $A$ be given the $\a$-adic filtration. Then the identity map of the $A/\a$-module $\a/\a^2$ onto itself can be extended uniquely to a homomorphism $\phi$ from the symmetric algebra $\bm{S}(\a/\a^2)$ to the $A/\a$-algebra $\gr(A)$; it follows from the definition of $\gr(A)$ that $\phi$ is a surjeelive homomorphism of graded algebras; for $n\geq 1$, every element of $\gr_n(A)$ is a sum of classes mod $\a^{n+1}$ of elements of the form $y=x_1\cdots x_n$, where $x_i\in\a$ for $1\leq i\leq n$; if $\xi_i$ is the class of $x_i$ mod $\a^2$, clearly the class of $y$ mod $\a^{n+1}$ is the element $\phi(\xi_1)\cdots\phi(\xi_n)$, whence our assertion. In particular, every system of generators of the $A/\a$-module $\a/\a^2$ is a system of generators of the $A/\a$-algebra $\gr(A)$.\par
If now $M$ is an $A$-module and $M$ is given the $\a$-adic filtration, it is seen similarly that the graded $\gr(A)$-module $\gr(M)$ is generated by $\gr_0(M)=M/\a M$. To be precise, the restriction $\Gamma$ to $\gr(A)\times\gr_0(M)$ of the external law on the $\gr(A)$-module $\gr(M)$ is a $\Z$-bilinear map of $\gr(A)\times\gr_0(M)$ to $\gr(M)$. Moreover, it is immediately verified that, for $\alpha\in\gr(A)$, $\beta\in\gr_0(A)$, $\xi\in\gr_0(M)$, we have $\Gamma(\alpha\beta,\xi)=\Gamma(\alpha,\beta\xi)$ and hence $\Gamma$ defines a canonical surjective $\gr_0(A)$-linear map
\begin{align}\label{filtration gr(M) generated by gr_0(M)}
\gamma_M:\gr(A)\otimes_{\gr_0(A)}\gr_0(M)\to\gr(M)
\end{align}
\end{example}
\begin{example}
Let $A$ be a graded ring of type $\Z$ and $M$ a graded $A$-module of type $\Z$; let $A^i$, (resp. $M^i$) be the subgroup of homogeneous elements of degree $i$ $A$ (resp. $M$). Let $A$ and $M$ be given the filtrations associated with their graduations. Then it is immediate that the $\Z$-linear map $A\to\gr(A)$ which maps an element of $A^n$ to its canonical image in
\[\gr_n(A)=\bigoplus_{i\geq n}A^i/\bigoplus_{i\geq n+1}A^i\]
is a graded ring isomorphism. A canonical graded $A$-module isomorphism $E\to\gr(E)$ is defined similarly.
\end{example}
\begin{proposition}\label{filtration gr(A) integral prop}
Let $A$ be a filtered ring, $(A_n)_{n\in\Z}$ its filtration and $v$ its order function. Suppose that $\gr(A)$ is an integral domain. Then, for every elements $a,b$ of the ring $B=\bigcup_nA_n$, we have $v(ab)=v(a)+v(b)$.
\end{proposition}
\begin{proof}
As $\n=\bigcap_{n\in\Z}A_n$ is an ideal of the ring $B$, the formula holds if $v(a)$ or $v(b)$ is equal to $+\infty$. If not, $v(a)=r$ and $v(b)=s$ are integers. The classes $\bar{a}$ of $a$ mod $A_{r+1}$ and $\bar{b}$ of $b$ mod $A^{s+1}$ are nonzero by definition, whence by hypothesis $\bar{a}\bar{b}\neq 0$ in $\gr(A)$ and therefore $ab\notin A_{r+s+1}$; as $ab\in A_{r+s}$, we see $v(ab)=r+s$.
\end{proof}
\begin{corollary}\label{filtration gr(A) integral then B/n integral}
Let $A$ be a filtered ring and $(A_n)_{n\in\Z}$ its filtration. Define $B=\bigcup_{n\in\Z}A_n$ and $\n=\bigcap_{n\in\Z}A_n$. If the ring $\gr(A)$ is integral, so is $B/\n$.
\end{corollary}
\begin{proof}
If $a$ and $b$ are elements of $B$ not belonging to $\n$, then $v(a)\neq+\infty$ and $v(b)\neq+\infty$, whence $v(ab)\neq+\infty$ and therefore $ab\notin\n$.
\end{proof}
\subsection{Homomorphism compatible with filtrations}
Let $G$, $G'$ be two commutative groups (written additively), $(G_n)$ a filtration on $G$ and $(G_n')$ a filtration on $G'$. A homomorphism $\phi:G\to G'$ is called \textbf{compatible} with the filtrations on $G$ and $G'$ if $\phi(G_n)\sub G'_n$ for all $n\in\Z$. If this holds, then by taking quotients we get a homomorphism $\gr_n(\phi):G_n/G_{n+1}\to G_n'/G_{n+1}'$ there is therefore a unique additive group homomorphism $\gr(\phi):\gr(G)\to\gr(G')$ such that, for all $n\in\Z$, $\gr(\phi)$ coincides with $\gr_n(\phi)$ on $\gr_n(G)$. The map $\gr(\phi)$ is called the \textbf{graded group homomorphism associated with $\bm{\phi}$}. If $G''$ is a third filtered group and $\psi:G'\to G''$ is a homomorphism which is compatible with the filtrations, then $\psi\circ\phi$ is a homomorphism which is compatible with the filtrations and we have
\[\gr(\psi\circ\phi)=\gr(\psi)\circ\gr(\phi)\]
\begin{proposition}\label{filtration functor gr exact with induced filtration}
Let $G$ be a filtered group and $H$ a subgroup of $G$; let $H$ be given the induced filtration and $G/H$ the quotient filtration. If $\iota:H\to G$ is the canonical injection and $\pi:G\to G/H$ the canonical surjection, then $\iota$ and $\pi$ are compatible with the filtrations and the sequence
\[\begin{tikzcd}
0\ar[r]&\gr(H)\ar[r,"\gr(\iota)"]&\gr(G)\ar[r,"\gr(\pi)"]&\gr(G/H)\ar[r]&0
\end{tikzcd}\]
is exact.
\end{proposition}
\begin{proof}
The first assertion is obvious: if $(G_n)$ is the filtration on $G$, then
\[(H\cap G_n)\cap G_{n+1}=H\cap G_{n+1}.\]
and hence $\gr(\iota)$ is injective; moreover the canonical map $G_n\to(H+G_n)/H$ is surjective, hence so is $\gr(\pi)$ and $\gr(\pi)\circ\gr(\iota)=\gr(\pi\circ\iota)=0$. Finally, let $\xi\in\gr_n(G)$ belong to the kernel of $\gr(\pi)$; then there exists $x\in\xi$ such that $x\in H+G_{n+1}$; but as $G_{n+1}\sub G_n$,
\[G_n\cap(H+G_{n+1})=(H\cap G_n)+G_{n+1}\]
and hence $x=y+z$ where $y\in H\cap G_n$ and $z\in G_{n+1}$; this proves that $\xi$ is the class mod $G_{n+1}$ of $\iota(y)$, in other words it belongs to the image of $\gr(H)$ under $\gr(\iota)$.
\end{proof}
\begin{example}
Note that the functor $\gr(\cdot)$ is not exact. Let $A=A'=k[X]$ where $k$ is a field and give $A$ the trivial filtration while $A'$ the $(X)$-adic filtration. Then the identity map $\eta$ from $A$ to $A'$ is compatible with the filtrations and we get a sequence
\[\begin{tikzcd}
0\ar[r]&\gr(A)\ar[r,"\gr(\eta)"]&\gr(A')\ar[r]&0
\end{tikzcd}\]
It is easy to see the map $\gr(\eta)$ is neither injective nor surjective. The point is that the filtration on $A$ differs from the filtration induced by $A'$ via the map $\eta$.
\end{example}
If now $A$ and $B$ are two filtered rings and $\rho:A\to B$ a ring homomorphism which is compatible with the filtrations, it is immediately verified that the graded group homomorphism $\gr(\rho):\gr(A)\to\gr(B)$ is also a ring homomorphism. In particular, if $A'$ is a subring of $A$ with the induced filtration, $\gr(A')$ is canonically identified with a graded subring of $\gr(A)$ (\cref{filtration functor gr exact with induced filtration}). If $\a$ is an ideal of $A$ and $A/\a$ is given the quotient filtration, $\gr(A/\a)$ is canonically identified with the quotient graded ring $\gr(A)/gr(\a)$ (\cref{filtration functor gr exact with induced filtration}).\par
Finally, let $A$ be a filtered ring, $M$, $N$ two filtered $A$-modules and $\phi:M\to N$ an homomorphism compatible with the filtrations. Then it is immediate that $\gr(\phi):\gr(M)\to\gr(N)$ is a $\gr(A)$-linear map and hence a graded homomorphism of degree $0$ of graded $\gr(A)$-modules. Moreover, if $\psi:M\to N$ is another $A$-homomorphism compatible with the filtrations, so is $\phi+\psi$ and we gave $\gr(\phi+\psi)=\gr(\phi)+\gr(\psi)$.
\section{Filtrations and completions}
Let $G$ be a commutative group filtered by a family $(G_n)_{n\in\Z}$ of subgroups of $G$. There exists a unique topology on $G$ which is compatible with the group structure and for which the $G_n$ constitute a fundamental system of neighbourhoods of the identity element $0$ of $G$ (\cref{topological group given by filter of subgroups}); it is called the topology on $G$ defined by the filtration $(G_n)$. When we use topological notions concerning a filtered group, we shall mean, unless otherwise stated, with the topology defined by the filtration. Note that the $G_n$, being subgroups of $G$, are both open and closed (\cref{topological group subgroup prop}). As $G$ is commutative, we deduce that $G$ admits a Hausdorff completion group $\widehat{G}$ (\cref{topological group completion if}). For evcry subset $E$ of $G$, the closure of $E$ in $G$ is equal to
\[\widebar{E}=\bigcap_{n\in\Z}(E+G_n)=\bigcap_{n\in\Z}(G_n+E).\]
In particular $\bigcap_{n\in\Z}G_n$ is the closure of $\{0\}$; thus it is seen that for the topology on $G$ to be Hausdorff it is necessary and suflicient that the filtration $(G_n)$ be separated. On the other hand, for the topology on $G$ to be discrete, it is necessary and sufficient that there exist $n\in\Z$ such that $G_n=\{0\}$; in this case the filtration $(G_n)$ is called \textbf{discrete}.\par
Not let $G'$ be another filtered group and $\phi:G\to G'$ a homomorphism compatible with the filtrations; the definition of the topologies on $G$ and $G'$ shows immediately that $\phi$ is continuous. If $H$ is a subgroup of $G$, the topology induced on $H$ by that on $G$ (resp. the quotient topology with respect to $H$ of that on $G$) is the topology on $H$ (resp. $G/H$) defined by the filtration induced by that on $G$ (resp. quotient topology of that on $G$). The product topology of those on $G$ and $G'$ is the topology defined by the product of filtrations on $G$ and $G'$.
\begin{proposition}\label{filtration ring and module topology compatible}
Let $A$ be a filtered ring, $(A_n)$ its filtration and $B=\bigcup_{n\in\Z}A_n$. If $M$ is a filtered $B$-module, $(M_n)$ its filtration and $N=\bigcup_{n\in\Z}M_n$ of $M$, then the map $(a,x)\mapsto ax$ from $B\times N$ to $N$ is continuous.
\end{proposition}
\begin{proof}
Let $a_0\in B$, $x_0\in N$; there exists by hypothesis integers $r,s$ such that $a_0\in A_r$ and $x_0\in M_s$. The equality
\[ax-a_0x_0=(a-a_0)x_0+a_0(x-x_0)+(a-a_0)(x-x_0)\]
shows that if $a-a_0\in A_i$ and $x-x_0\in M_j$, then $ax-a_0x_0$ belongs to $M_{i+s}+M_{j+r}+M_{i+j}$. Then, given an integer $n$, $ax-a_0x_0\in M_n$ provided $i\geq n-s$, $j\geq n-r$ and $i+j\geq n$; that is so long as $i$ and $j$ are sufficiently large.
\end{proof}
\begin{corollary}
The ring $B$ is a topological ring and the $B$-module $N$ is a topological $B$-module.
\end{corollary}
It is seen in particular that a filtered ring $A$ whose filtration is exhaustive is a topological ring; if this is so every filtered $A$-module whose filtration is exhaustive is a topological $A$-module.
\begin{example}
Let $A$ be a ring and $\a$ an ideal of $A$; the topology defined on $A$ by the $\a$-adic filtration is called the $\a$-adic topology; as the $\a$-adic filtration is exhaustive, $A$ is a topological ring with this topology. Similarly, for every $A$-module $M$, the topology defined by the $\a$-adic filtration is called the $\a$-adic topology on $M$; $M$ is a topological $A$-module under this topology.
\end{example}
\begin{proposition}\label{filtration closure of prime ideal is prime}
Let $A$ be a ring filtered by an exhaustive filtration $(A_n)$ and $\p$ an ideal of $A$. Suppose that the ideal $\gr(\p)$ of the ring $\gr(A)$ is prime. Then the closure of $\p$ in $A$ is a prime ideal.
\end{proposition}
\begin{proof}
We know that $\gr(A/\p)$ is isomorphic to $\gr(A)/\gr(\p)$ (\cref{filtration functor gr exact with induced filtration}) and hence an integral domain; we conclude that $A/\bigcap_{n\in\Z}(\p+A_n)$ is an integral domain by \cref{filtration gr(A) integral then B/n integral}. Thus the closure $\widebar{\p}=\bigcap_{n\in\Z}(\p+A_n)$  is a prime ideal.
\end{proof}
\subsection{Complete filtered groups}
\begin{proposition}\label{filtration group complete iff}
Let $G$ be a filtered group with filtration $(G_n)$. The following conditions are equivalent:
\begin{itemize}
\item[(\rmnum{1})] $G$ is a complete topological group.
\item[(\rmnum{2})] The associated Hausdorff group $G'=G/(\bigcap_nG_n)$ is complete.
\item[(\rmnum{3})] Every Cauchy sequence in $G$ is convergent.
\item[(\rmnum{4})] Every family $(x_\lambda)_{\lambda\in I}$ of elements of $G'$ which converges to $0$ with respect to the filter $\mathfrak{F}$ complements of finite subsets of $I$ is summable in $G'$. 
\end{itemize}
\end{proposition}
\begin{proof}
For a filter on $G$ to be a Cauchy filter (resp. a convergent filter), it is necessary and sufficient that its image under the canonical map $G\to G'$ be a Cauchy (resp. convergent) filter (\cref{filter Cauchy on initial topology iff}); whence first of all the equivalence of (\rmnum{1}) and (\rmnum{2}); on the other hand, as $G'$ is first countable, the equivalence of (\rmnum{1}) and (\rmnum{3}) follows.\par
Suppose that $G'$ is complete and let $(x_\lambda)_{\lambda\in I}$ be a family of elements of $G'$ which converge to $0$ with respect to $G'$. For every neighbourhood $V'$ of $0$ in $G'$ which is a subgroup of $G'$, there exists a finite subset $J$ of $I$ such that $x_\lambda\in V'$ whenever $\lambda\in I\setminus J$; then $\sum_{\lambda\in K}x_\lambda\in V'$ for every finite subset $K$ of $I$ not meeting $J$, which shows that the family $(x_\lambda)_{\lambda\in I}$ is summable. Conversely, suppose that condition (\rmnum{4}) holds and let $(x_n)$ be a Cauchy sequence on $G'$,the family $(x_{n+1}-x_n)$ is then summable and in particular the series with general term $x_{n+1}-x_n$ is convergent and hence the sequence $(x_n)$ is convergent.
\end{proof}
Let $G$ be a filtered group whose filtration $(G_n)$; the quotient groups $G/G_n$ are discrete and hence complete, since the $G_n$'s are open in $G$. For each $n\in\Z$, we have a canonical map $\pi_n:G\to G/G_n$, and for $m<n$ there is a canonical map $\tau_{mn}:G/G_n\to G/G_m$ such that $(G/G_n,\pi_{mn})$ is an inverse system of discrete groups which is compatible with the maps $(\pi_n)$. By the universal property of inductive limits, we then get a map $\pi:G\to\llim G/G_n$. If $\iota:G\to\widehat{G}$ is the canonical map of $G$ to its Hausdorff completion $\widehat{G}$, then as the $G/G_n$ are complete, there exists a unique topological group isomorphism $\psi:\widehat{G}\to\llim G/G_n$ such that $\pi=\psi\circ\iota$ (\cref{topological group inverse limit of nbhd quotient and completion}), we shall call it the \textbf{canonical isomorphism} of $\widehat{G}$ onto $\llim G/G_n$.
\begin{figure}[htbp]
\centering
\begin{tikzcd}
G/G_n\ar[rr,"\tau_{mn}"]&&G/G_m\\
&\llim G/G_n\ar[lu,"\tau_n"]\ar[ru,swap,"\tau_m"]&\\
&G\ar[luu,bend left=20pt,"\pi_n"]\ar[ruu,bend right=20pt,swap,"\pi_m"]\ar[u,"\pi"]&
\end{tikzcd}\quad\quad\begin{tikzcd}
\widehat{G}\ar[r,"\psi"]&\llim G/G_n\\
G\ar[u,"\iota"]\ar[ru,swap,"\pi"]
\end{tikzcd}
\end{figure}

With the result above, we can apply the prperties of inverse limit to completion. First let us observe that the inverse system $\{G/G_n\}$ has the special property that the transition maps $\tau_{mn}:G/G_n\to G/G_m$ are always surjective. Any inverse system with this property we shall call a \textbf{surjective system}. For such a system, we have the following result:
\begin{proposition}\label{inverse limit derived exact sequence}
If $0\to \{A_n\}\to\{B_n\}\to\{C_n\}\to 0$ is an exact sequence of inverse systems then we have an exact sequence
\[\begin{tikzcd}[column sep=small]
0\ar[r]&\llim A_n\ar[r]&\llim B_n\ar[r]&\llim C_n\ar[r]&\llim\nolimits^1A_n\ar[r]&\llim\nolimits^1B_n\ar[r]&\llim\nolimits^1C_n\ar[r]&0
\end{tikzcd}\]
Moreover, $\llim\nolimits^1A_n=0$ if $\{A_n\}$ is a surjective system.
\end{proposition}
\begin{proof}
For an inverse system $\{A_n\}$ let $A=\prod_{n}A_n$ and define $d_A:A\to A$ by
\[d_A((a_n))=(a_n-\tau_{n,n+1}(a_{n+1}))\]
so that $\ker d_A=\llim A_n$ and we set $\llim\nolimits^1A_n:=\coker d_A$. The exact sequence of inverse systems then defines a commutative diagram of exact sequences
\[\begin{tikzcd}
0\ar[r]&A\ar[r]\ar[d,"d_A"]&B\ar[r]\ar[d,"d_B"]&C\ar[r]\ar[d,"d_C"]&0\\
0\ar[r]&A\ar[r]&B\ar[r]&C\ar[r]&0
\end{tikzcd}\]
and hence by snake lemma an exact sequence
\[\begin{tikzcd}[column sep=small]
0\ar[r]&\ker d_A\ar[r]&\ker d_B\ar[r]&\ker d_C\ar[r]&\coker d_A\ar[r]&\coker d_B\ar[r]&\coker d_C\ar[r]&0
\end{tikzcd}\]
To complete the proof we have only to prove that $d_A$ is surjective if $\{A_n\}$ is a surjective system. But this is clear because to show $d_A$ surjective we have only to solve inductively the equations $x_{n}-\tau_{n,n+1}(x_{n+1})=a_n$ for $x_n\in A_n$, given $a_n\in A_n$.
\end{proof}
\begin{corollary}\label{filtration completion is exact if induced} 
Let $G$ be a filtered group and $H$ a subgroup of $G$; let $H$ be given the induced filtration and $G/H$ the quotient filtration. If $\iota:H\to G$ is the canonical injection and $\pi:G\to G/H$ the canonical surjection, then the sequence
\[\begin{tikzcd}
0\ar[r]&\widehat{H}\ar[r]&\widehat{G}\ar[r]&\widehat{G/H}\ar[r]&0
\end{tikzcd}\]
is exact.
\end{corollary}
\begin{proof}
We have an exact sequence of the correspond inverse systems so \cref{inverse limit derived exact sequence} gives the claim immediately.
\end{proof}
\begin{corollary}\label{filtration completion of filtration subgroup quotient prop} 
Let $G$ be a filtered group with filtration $(G_n)$. Then $\widehat{G}_n$ is a subgroup of $\widehat{G}$ and $\widehat{G}/\widehat{G}_n\cong G/G_n$. Therefore, we have $\gr(G)\cong\gr(\widehat{G})$.
\end{corollary}
\begin{proof}
Just apply \cref{filtration completion is exact if induced} with $H=G_n$, in which case $G/G_n$ has the discrete topology so that $\widehat{G/G_n}=G/G_n$.
\end{proof}
\begin{example}
Let $G$ be a complete filtered group. Every closed subgroup of $G$ with the induced filtration is complete. Every quotient group of $G$ with the quotient filtration is complete.
\end{example}
\begin{example}
Let $A$ be a ring and $\a$ an ideal. The completion $\widehat{A}$ of $A$ under the $\a$-adic topology is called the $\a$-adic completion of $A$. The canonical map $\iota:A\to\widehat{A}$ is a continuous ring homomorphism, whose kernel is $\bigcap_n\a^n$.\par
Let $M$ be an $A$-module, the completion $\widehat{M}$ of $M$ under the $\a$-adic topology is a topological $\widehat{A}$-module. If $\phi:M\to N$ is any $A$-module homomorphism, then $\phi(\a^nM)=\a^n\phi(M)\sub \a^nN$, and therefore $\phi$ is continuous (with respect to the $\a$-topologies on $M$ and $N$) and so defines $\widehat{\phi}:\widehat{M}\to\widehat{N}$.
\end{example}
\begin{example}
Let $A$ be a filtered ring whose filtration we denote by $(A_n)_{n\in\Z}$; let $B=A\llbracket X_1,\dots,X_r\rrbracket$. For all $\alpha=(\alpha_1,\dots,\alpha_s)\in\N^s$, we write $|\alpha|=\sum_{i=1}^{s}e_i$ and $X^\alpha=X_1^{\alpha_1}\cdots X_s^{\alpha_s}$ so that every element $P\in B$ can be written uniquely $P=\sum_{\alpha\in\N^s}c_{\alpha,P}X^\alpha$, where $c_{\alpha,P}\in A$. For all $n\in\Z$, define
\[B_n=\{P=\sum_\alpha c_{\alpha,P}X^\alpha:\text{$c_{\alpha,P}\in A_{n-|\alpha|}$ for all $\alpha\in\N^s$}\}.\]
Clearly $B_n$ is an additive subgroup of $B$; on the other hand, if $P\in B_n$ and $Q\in B$, then, for all $\alpha\in\N^s$, $c_{\gamma,PQ}=\sum_{\alpha+\beta=\gamma}c_{\alpha,P}c_{\beta,Q}$; as the relation $\alpha+\beta=\gamma$ implies $|\alpha|\leq|\gamma|$, we have $PQ\in B_n$, so $B_n$ is an ideal of $B$. Moreover, if $Q\in B_m$, then for $\alpha+\beta=\gamma$,
\[c_{\alpha,P}c_{\beta,Q}\in A_{n-|\alpha|}A_{n-|\beta|}\sub A_{m+n-|\gamma|},\]
which proves that $(B_n)_{n\in\Z}$ is a filtration compatible with the ring structure on $B$ (for obviously $1\in B_0$). When in future we speak of $B$ as a filtered ring, we shall mean, unless otherwise stated, with the filtration $(B_n)$. Clearly $\bigcap_{n\in\Z}B_n$ is the set of formal power series all of whose coefficients belong to $\bigcap_{n\in\Z}A_n$; then, if $A$ is Hausdorff, so is $B$. If $A_0=A$, then $B_0=B$.
\end{example}
\begin{proposition}\label{filtration power series ring complete if}
With the above notation, suppose that $A_0=A$ and let $h$ denote the map $P\mapsto(c_{\alpha,P})_{\alpha\in\N^s}$. Then $h$ is an isomorphism of the additive topological group $B$ onto the additive topological group $A^{\N^s}$. The polynomial ring $A[X_1,\dots,X_s]$ is dense in $B$; if $A$ is complete, so is $B$.
\end{proposition}
\begin{proof}
Clearly $h$ is bijective; $V_n=h(B_n)$ is the set of $(c_\alpha)_{\alpha\in\N^s}$ such that $c_{\alpha}\in A_{n-|\alpha|}$ for all $\alpha\in\N^s$ such that $|\alpha|\leq n$; as these elements $\alpha$ are finite in number, $V_n$ is a neighbourhood of $0$ in $A^{\N^s}$. Conversely, if $V$ is a neighbourhood of $0$ in $A^{\N^s}$, there is a finite subset $E$ of $\N^s$ and an integer $\nu$ such that the conditions $c_\alpha\in A_\nu$ for all $\alpha\in E$ imply $(c_\alpha)\in V$; if then $n$ is the greatest of the integers $\nu+|\alpha|$ for $\alpha\in E$, then $h(A_n)\sub V$, which proves the first assertion. Moreover, with $n$ and $E$ defined as above, $h(P-\sum_{\alpha\in E}c_{\alpha,P}X^\alpha)\in V$ for all $P\in B$, which shows that $A[X_1,\dots,X_s]$ is dense in $B$. The last assertion follows from the first and the fact that a product of complete spaces is complete.
\end{proof}
Let $\a$ be an ideal of $A$ and suppose that $(A_n)$ is the $\a$-adic filtration; then, if $\b$ is the ideal of $B$ generated by $\a$ and $X_1,\dots,X_s$, the filtration $(B_n)$ is the $\b$-adic filtration. For clearly, for all $n\geq 0$, $\b^n$ is generated by the elements $cX^\alpha$ such that $c\in\b^{n-|\alpha|}$ for all $\alpha\in\N^s$ such that $|\alpha|\leq n$, whence $\b^n\sub B_n$. Conversely, for all $P\in B_n$, $P=P_1+P_2$, where
\[P_1=\sum_{|\alpha|<n}c_{\alpha,P}X^\alpha,\quad P_2=\sum_{|\alpha|\geq n}c_{\alpha,P}X^\alpha\]
Clearly it is possible to write $P_2=\sum_{|\alpha|=n}Q_\alpha X^\alpha$, where the $Q_\alpha$ are elements of $B$, whence $P_2\in\b^n$; on the other hand, since $P\in B_n$, we have $c_{\alpha,P}X^\alpha\in\b^n$ for all $|\alpha|\leq n$, whence $P_1\in\b^n$.
\begin{corollary}\label{filtration on power series complete}
Let $A$ be a ring and $B=A\llbracket X_1,\dots,X_r\rrbracket$ the ring of formal power series in $s$ indeterminates over $A$ and $\n$ the ideal of $B$ consisting of the formal power series with no constant term. The ring $B$ is Hausdorff and complete with the $\n$-adic topology and the polynomial ring $A[X_1,\dots,X_r]$ is everywhere dense in $B$.
\end{corollary}
\begin{proof}
It is sufficient to apply what has just been said to the case $\m=\{0\}$, so that the $\m$-adic topology on $A$ is trivial and $A$ is thus complete.
\end{proof}
\subsection{Completion of filtered rings and modules}
Let $A$ be a filtered ring, $M$ a filtered $A$-module and $(A_n)$ and $(M_n)$ the respective filtrations of $A$ and $M$; we shall assume that these filtrations are exhaustive so that for the corresponding topologies $A$ is a topological ring and $M$ a topological $A$-module (\cref{filtration ring and module topology compatible}). Then we have defined $\widehat{A}$ as a topological ring and $\widehat{M}$ as a topological $\widehat{A}$-module. If $\iota:A\to\widehat{A}$ is the canonical homomorphism, then $\iota(A_m)\iota(A_n)\sub\iota(A_{m+n})$, whence by the continuity of multiplication in $\widehat{A}$,
\[\widehat{A}_n\widehat{A}_m\sub\widehat{A}_{m+n}\]
where $\widehat{A}_n$ is the closure of $\iota(A_n)$ in $A$. It can be similarly shown that $\widehat{A}_n\widehat{M}_m\sub\widehat{M}_{m+n}$, in other words,
\begin{proposition}
Let $A$ be a filtered ring and $M$ a filtered $A$-module, the respective filtrations $(A_n)$, $(E_n)$ of $A$ and $M$ being exhaustive. Then $(\widehat{A}_n)$ is a filtration compatible with the ring structure on $\widehat{A}$ and $(\widehat{M}_n)$ is a filtration compatible with the module structure on $\widehat{A}$ over the filtered ring $\widehat{A}$. Moreover, these filtrations are exhaustive and define respectively the topologies on $\widehat{A}$ and $\widehat{M}$. Finally, the canonical maps $\gr(A)\to\gr(\widehat{A})$ and $\gr(M)\to\gr(\widehat{M})$ of graded $\Z$-modules are respectively a graded ring isomorphism and a graded $\gr(A)$ -module isomorphism.
\end{proposition}
In what follows, for every uniform space $X$, $\iota_X$ will denote the canonical map from $X$ to its Hausdorff completion $\widehat{X}$ and $X_0=\iota_X(X)$ the uniform subspace of $X$, which is the Hausdorff space associated with $X$. Recall that the topology on $X$ is the inverse image under $\iota_X$ of that on $X_0$, and for every uniformly continuous map $f:X\to Y$, $\hat{f}$ denotes the uniformly continuous map from $\widehat{X}$ to $\widehat{Y}$ such that $\hat{f}\circ\iota_X=\iota_Y\circ f$. If $X$ is a uniform subspace of $Y$ and $i$ is the canonical injection, then $\widehat{X}$ is identified with a uniform subspace of $\widehat{Y}$ and $\hat{i}$ is the canonical injection of $\widehat{X}$ into $\widehat{Y}$ (\cref{uniform space completion of initial topology}).
\begin{lemma}\label{topological group exact sequence of strict morphism completion}
Let $0\to X\to Y\to Z\to 0$ be an exact sequence of strict morphism of topological groups. Suppose that $X$, $Y$, $Z$ admit Hausdorff completion groups and that the identity elements of $X$, $Y$, $Z$ admit countable fundamental systems of neighbourhoods. Then
\[\begin{tikzcd}
0\ar[r]&\widehat{X}\ar[r]&\widehat{Y}\ar[r]&\widehat{Z}\ar[r]&0
\end{tikzcd}\]
is an exact sequence of strict morphisms.
\end{lemma}
\begin{proof}
Let $f:X\to Y$ and $g:Y\to Z$ be the given maps. We have already seen that the completion of a strict injection is strict injective, so it remains to show the following: if $Y=X/N$ (where $N$ is a subgroup of $X$) and $f:X\to Y$ is the canonical map, then $\hat{f}$ is a surjective strict morphism with kernel $\widehat{N}$.\par
Let $X_0=\iota_X(X)$ and $Y_0=\iota_Y(Y)$; let $f_0:X_0\to Y_0$ be the map which coincides with $\hat{f}$ on $X_0$; as $\iota_X$ (resp. $\iota_Y$) is a surjective strict morphism of $X$ onto $X_0$ (resp. $Y$ onto $Y_0$), $f_0$ is a surjective strict morphism by \cref{topological group stric morphism hereditary on quotient}. Now $X_0$ and $Y_0$ are metrizable (GT, \Rmnum{9}, $\S$3, no.1 Proposition 1), so it follows from (GT, \Rmnum{9}, $\S$3, no.1 Corollary to Proposition 4 and Lemma 1) that $\hat{f}_0=\hat{f}$ is a surjective strict morphism and has kernel the closure $\widehat{N}_0'$ in $\widehat{X}$ of the kernel $N_0'$ of $f_0$. Then it will be sufficient for us to prove that $\widehat{N}_0'=\widehat{N}$. Now $N_0'$ obviously contains $N_0=\iota_X(N)$, so $\widehat{N}_0'\sups \widehat{N}$; it will be sufficient to show that $N_0'$ is contained in the closure $\widebar{N}_0$ of $N_0$ in $X_0$. Now,
\[U=\iota_X^{-1}(X_0\setminus\widebar{N}_0)=X\setminus\iota_X^{-1}(\widebar{N}_0)\]
is an open set in $X$ which does not meet $N$; as $f$ is a surjective strict morphism, $V=f(U)$ is an open set in $Y$ not containing the identity element $0$ of $Y$ and hence not meeting the closure of $\{0\}$; then $\iota_Y(V)$ does not contain the identity element of $Y_0$. But $\iota_Y(V)=f_0(X_0\setminus N_0)$ and hence $N_0'\sub\widebar{N}_0$, which complete the proof.
\end{proof}
\begin{proposition}\label{filtration on completion is product with completion ring}
Let $A$ be a filtered ring, $(A_n)$ its filtration, $M$ an $A$-module and $(M_n)$ the filtration on $M$ derived from that on $A$. Suppose that the filtration $(A_n)$ is exhaustive and the module $M$ is finitely generated. If $\iota_M:M\to\widehat{M}$ is the canonical map, then, for all $n\in\Z$,
\[\widehat{M}_n=\widehat{A}_n\widehat{M}=\widehat{A}_n\iota_M(M),\quad \widehat{M}=\widehat{A}\widehat{M}=\widehat{A}\iota_M(M).\]
In particular $\widehat{M}$ is a finitely generated $\widehat{A}$-module.
\end{proposition}
\begin{proof}
By hypothesis there exists a surjective homomorphism $\phi:L\to M$, where $L=A^I$, $I$ being a finite set; let $L$ be given the product filtration, consisting of the $L_n=A_n^I$; then $\widehat{L}=\widehat{A}^I$ and $\widehat{L}_n=\widehat{A}_n^I$. Let $\iota_L:L\to\widehat{L}$ be the canonical map and $(e_i)_{i\in I}$ the canonical basis of $L$; for an element $\sum_ia_i\iota_L(e_i)$ of $\widehat{L}_n$ to belong to $\widehat{L}_n$, it is necessary and sufficient that $a_i\in\widehat{A}_n$ for all $i$, thus we see $\widehat{L}_n=\widehat{A}_n\widehat{L}=\widehat{A}_n\iota_L(L)$.\par
Now note that by definition $\phi(L_n)=M_n$ and hence $\phi$ is a strict morphism of $L$ onto $M$. \cref{topological group exact sequence of strict morphism completion} then shows that $\hat{\phi}:\widehat{L}\to\widehat{M}$ is a surjective strict morphism. As $L_n$ is an open subgroup of $\widehat{L}$, $\hat{\phi}(\widehat{L}_n)$ is an open (and therefore closed) subgroup of $M$; but by what we have just proved,
\[\hat{\phi}(\widehat{L}_n)=\hat{\phi}(\widehat{A}_n\iota_L(L))=\widehat{A}_n\iota_M(M).\]
As $\iota_M(M_n)\sub\widehat{A}_n\iota_M(M)$, this then implies
\[\widehat{M}_n\sub\widehat{A}_n\iota_M(M)\sub\widehat{A}_n\widehat{M}\sub\widehat{M}_n\]
and therefore $\widehat{M}_n=\widehat{A}_n\widehat{M}=\widehat{A}_n\iota_M(M)$. Take the union of all $n\in\Z$, we obtain the second formula.
\end{proof}
\begin{corollary}\label{filtration complete ring finite module is complete}
Under the conditions of \cref{filtration on completion is product with completion ring}, if $A$ is complete, so is $M$.
\end{corollary}
\begin{proof}
As the canonical map $A\to\widehat{A}$ is then surjective, $M=\iota_M(M)$ by \cref{filtration group complete iff} and the conclusion follows from \cref{filtration group complete iff}.
\end{proof}
\begin{corollary}\label{filtration I-adic completion of finite ideal prop}
Let $A$ be a ring, $\a$ a finitely generated ideal of $A$ and $\widehat{A}$ the Hausdorff ompletion of $A$ with respect to the $\a$-adic topology. Then $\widehat{\a^n}=(\hat{\a})^n=\widehat{A}\a^n$ for every integer $n>0$ and the topology on $\widehat{A}$ is the $\hat{\a}$-adic topology.
\end{corollary}
\begin{proof}
Let us write $A_n=\a^n$, which is a finitely generated ideal of $A$. The formula $\a^pA_n=\a^{n+p}$ shows that the topology induced on $A_n$ by the $\a$-adic topology coincides with the $\a$-adic topology on the $A$-module $A_n$. By \cref{filtration on completion is product with completion ring} applied to $M=A_n$, $\widehat{A}_n=\widehat{A}A_n$, in other words $\widehat{\a^n}=\widehat{A}\a^n$. In particular $\hat{\a}=\widehat{A}\a$, whence $(\hat{\a})^n=\widehat{A}\a^n$.
\end{proof}
\begin{example}
Let $A$ be a graded ring of type $\N$ and let $(A^n)_{n\in\N}$ be its graduation; let it be given the associated filtration which is separated and exhaustive. The additive group $A$ is canonically identified with the subspace $\bigoplus_{n\in\N}A^n$ in $B=\prod_{n\in\N}A^n$. If $B$ is given the topology the product of the discrete topologies, the topology induced on $A$ is the topology defined by the filtration on $A$; also $B$ is a complete topological group and $A$ is dense in $B$. The additive topological group $B$ is then identified with the completion $\widehat{A}$ of the Hausdorff additive group $A$ and it has a unique ring structure which makes it the completion of the topological ring $A$. To define multiplication in this ring, note that, if we write $A_n=\bigoplus_{i\geq n}A^i$, the closure in $B$ of the ideal $A_n$ is the set $B_n$ of $x=(x_n)$ such that $x_n=0$ for $i<n$. Let $x=(x_n)$, $y=(y_n)$ be two elements of $B$ and $z=(z_n)$ their product. Then, for all $n>0$, $x\equiv x^{(n)}$ mod $B_{n+1}$ and $y\equiv y^{(n)}$ mod $B_{n+1}$. But $x^{(n)}$ and $y^{(n)}$ belong to $A$ and it is therefore seen that, for all $n\in\N$,
\[z_n=\sum_{i+j=n}x_iy_j.\]
In particular, we again obtain that, if $A$ is a ring, the completion of $A[X_1,\dots,X_r]$ with the filtration associated with its usual graduation (by total degree) is canonically identified with the ring of formal power series $A\llbracket X_1,\dots,X_r\rrbracket$. 
\end{example}
\begin{example}
Let $\alpha$ be a non-zero non-invertible element of a principal ideal domain; the $(\alpha)$-adic topology on $A$ is also called the $\alpha$-adic topology; it is Hausdorff for the intersection of the ideals $(\alpha^n)$ reduces to $0$. Note that the completion of $A$ with respect to this topology is not necessarily an integral domain. The associated graded ring $\gr(A)=\gr(\widehat{A})$ is canonically isomorphic to $(A/\alpha)[X]$. If $A=\Z$, the completion of $\Z$ with respect to the $\alpha$-adic topology ($n>1$) is denoted by $\Z_\alpha$, and its elements are called $\alpha$-adic integers.
\end{example}
\begin{example}
Let $k$ be a field, $A=k[(X_n)_{n\in\N}]$, and $\m=(X_1,X_2,\dots)$. We will think of an element $f\in\widehat{A}$ as a (possibly) infinite sum
\[f(X)=\sum_Ia_IX^I\]
(using multi-index notation) such that for each $d\geq 0$ there are only finitely many nonzero $a_I$ for $|I|=d$. The maximal ideal $\hat{\m}$ in $\widehat{A}$ is the collection of $f$ with zero constant term. In particular, the element
\[f=X_1+X_2^2+X_3^2+\cdots\]
is in $\hat{\m}$ but not in $\m\widehat{A}$. This shows the inclusion $\m\widehat{A}\sub\hat{\m}$ is proper.
\end{example}
\begin{proposition}\label{filtration completion of a family of ideals}
Let $A$ be a ring and $(\a_\lambda)_{\lambda\in I}$ a family of ideals of $A$, distinct from $A$, such that $\a_\lambda$ and $\a_\mu$ are relatively prime for $\alpha,\mu\in I$. For every family $s=(s_\lambda)_{\lambda\in I}$ of positive integers of finite support, set $\a_s=\bigcap_{\lambda\in I}\a_\lambda^{s_\lambda}$, and let $\mathcal{T}$ be the topology compatible with the ring structure on $A$ such that the $\a_s$'s form a fundamental system of neighbourhoods of $0$. Let $\widehat{A}$ be the Hausdorff completion of $A$ with respect to this topology. On the other hand, for all $\lambda\in I$, let $A_\lambda$ be the ring $A$ with the $\a_\lambda$-adic topology and let $\widehat{A}_\lambda$ be its Hausdorff completion. If $\phi:A\to\prod_{\lambda\in I}A_\lambda$ denotes the diagonal homomorphism, then $\phi$ is continuous and the corresponding homomorphism
\[\hat{\phi}:\widehat{A}\to\prod_{\lambda\in I}\widehat{A}_\lambda\]
is a topological ring isomorphism.
\end{proposition}
\begin{proof}
Note that $\mathcal{T}$ is the least upper bound of all $\a_\lambda$-adic topologies. The first assertion follows from \cref{topo ring given by filter of ideal}. Let us set $B=\prod_{\lambda\in I}A_\lambda$; as the topology on $A$ is finer than each of the $\a_\lambda$-adic topologies, the maps $\pi_\lambda\circ\phi$ are continuous and hence $\phi$ is continuous. Also, $\phi(\a_s)$ is the intersection of the diagonal $\Delta$ of $B$ and the open set $\bigcap_{\lambda\in I}\pi_\lambda^{-1}(\a_\lambda^{s_\lambda})$, it follows that $\phi$ is a strict morphism from the additive group $A$ to $B$ with image $A$. Now $A$ is dense in $B$. For let $b=(a_\lambda)$ be an element of $B$; every neighbourhood of $b$ in $B$ contains a set of the form $b+V$, where $V=\bigcap_{\lambda\in I}\pi_\lambda^{-1}(\a_\lambda^{s_\lambda})$ for a family $s=(s_\lambda)$ of positive integers with finite support. As the $\a_\lambda$'s are relatively prime in pairs, there exists $x\in A$ such that $x\equiv a_\lambda$ mod $\a_\lambda^{s_\lambda}$ for all $\lambda\in I$ (by CRT, note that $(s_\lambda)$ has finite support) and hence $(b+V)\cap\Delta\neq\emp$. The Hausdorff completion of the group $B/\Delta$ is then $\{0\}$; applying \cref{topological group exact sequence of strict morphism completion} to the exact sequences $0\to A\to B\to B/\Delta\to 0$, we see $\hat{\phi}$ is an isomorphism of $\widehat{A}$ onto $\widehat{B}$. 
\end{proof}
\begin{corollary}\label{filtration PID completion of all pi-adic topology}
Let $A$ be a PID and $P$ a representative system of elements of $A$ such that the ideals $\pi A$ with $\pi\in P$ run though all the maximal ideals of $A$. The topology on $A$ with respect to which the nonzero ideals of $A$ form a fundamental system of neighbourhoods of $0$, which is compatible with the ring structure on $A$, is Hausdorff and the completion of $A$ with this topology is canonically isomorphic to the product of the completions of $A$ with respect to the $\pi$-adic topologies, where $\pi$ runs through $P$.
\end{corollary}
\begin{proof}
The principal ideals $\pi A$ where $\pi\in P$ are maximal and distinct and hence relatively prime, we have already seen that the $\pi$-adic topologies are Hausdorff and hence so is the topology defined in the statement of \cref{filtration completion of a family of ideals}, which is finer than each of the $\pi$-adic topologies.
\end{proof}
If \cref{filtration PID completion of all pi-adic topology} is applied when $A=\Z$, we denote by $\widehat{Z}$ the completion of $\Z$ with respect to the topology for which all the nonzero ideals of $\Z$ form a fundamental system of neighbourhoods of $0$, the ring isomorphic to the product $\prod_{p\in P}\Z_p$ of the rings of $p$-adic integers ($P$ being the set of prime numbers), and is usually denoted by $\widehat{\Z}$.
\begin{proposition}\label{filtration of maximal-adic ideals completion isomorphism}
Let $A$ be a ring, $(\m_i)_{1\leq i\leq r}$ a finite family of distinct maximal ideals of $A$ and set
\[\t=\prod_{i=1}^{r}\m_i=\bigcap_{i=1}^n\m_i,\quad S=\bigcap_{i=1}^{r}(A-\m_i).\]
Endow $A$ with the $\t$-adic topology, $S^{-1}A$ the $S^{-1}\t$-adic topology and each of the local ring $A_{\m_i}$ the $(\m_iA_{\m_i})$-adic topology. Let $\phi:A\to S^{-1}A$ and $\psi:S^{-1}A\to\prod_{i=1}^{r}A_{\m_i}$ be the canonical homomorphisms. Then $\phi$ and $\psi$ are continuous and the corresponding homomorphisms $\hat{\phi}$ and $\hat{\psi}$ are topological ring isomorphism.
\end{proposition}
\begin{proof}
Since $\m_i\cap S=\emp$ for each $i\in I$, the ideal $\m_i'=\m_iB$ of $B$ is maximal and we have $S^{-1}\t=\bigcap_{i=1}^{r}\m_i'$. Also, $B_{\m_i'}=A_{\m_i}$ up to a canonical isomorphism. As $\phi^{-1}(\t B)=\t$ and $\psi_i^{-1}(\m_iA_{\m_i})\sups\t B$, $\phi$ and $\psi$ are continuous. Then it is sufficient to prove that, if
\[\omega=\psi\circ\phi:A\to\prod_{i=1}^{r}A_{\m_i}\]
then $\hat{\omega}$ is an isomorphism of $\widehat{A}$ onto $\prod_{i=1}^{r}\widehat{A}_{\m_i}$, for this result applied to $B$ and the $\m_i'$ will show that $\hat{\psi}$ is an isomorphism and therefore also $\hat{\phi}$. Note that every product of powers of the $\m_i$ contains a power of $\t$ and hence the $\t$-adic topology is the least upper bound of the $\m_i$-adic topologies; moreover, if $A_i$ denotes the ring $A$ with the $\m_i$-adic topology and $\gamma:A\to\prod_{i=1}^{r}A_i$ is the diagonal map, then $\hat{\gamma}:\widehat{A}\to\prod_{i=1}^{r}\widehat{A}_i$ is an isomorphism (\cref{filtration completion of a family of ideals}). Then it all amounts to proving that, if $v_i:A_i\to A_{\m_i}$ is the canonical map, then $\hat{v}_i:\widehat{A}_i\to\widehat{A}_{\m_i}$ is an isomorphism. Now, for all $n$, the map
\[v_{i,n}:A/\m_i^n\to A_{\m_i}/\m_i^nA_{\m_i}\]
derived from $v_i$ by taking quotients is an isomorphism; our assertion follows from the fact that $\widehat{A}_i$ (resp. $\widehat{A}_{\m_i}$) is the inverse limit of the discrete rings $A/\m_i^n$ (resp. $A_{\m_i}/\m_iA_{\m_i})$.
\end{proof}
As a particular example, we want to consider the $\m$-adic completion of a maximal ideal $\m$ of $A$. For this, we need the following useful lemma.
\begin{lemma}\label{filtration complete Hausdorff ring invertibility}
Let $A$ be a complete Hausdorff topological ring, in which there exists a fundamental system of neighbourhoods of $0$ consisting of additive subgroups of $A$.
\begin{itemize}
\item[(a)] For all $x\in A$ such that $\lim_nx^n=0$, the element $1-x$ is invertible in $A$ and its inverse is equal to $\sum_nx^n$.
\item[(b)] Let $\a$ be an ideal of $A$ such that $\lim_nx^n=0$ for any $x\in \a$. For an element $y$ of $A$ to be invertible, it is necessary and suflicient that its class mod $\a$ be invertible in $A/\a$. In particular, $\a$ is contained in the Jacobson radical of $A$.
\end{itemize}
\end{lemma}
\begin{proof}
Note that
\[(1-x)(1+x+x^2+\cdots+x^{n})=1-x^{n+1}\]
so for (a) it all amounts to proving that the series with general term $x^n$ is convergent in $A$ when $\lim_nx^n=0$; now, by hypothesis, for every symmetric neighbourhood $V$ of $0$ in $A$, there exists an integer $p>0$ such that $x^n\in V$ for all $n\geq p$. We conclude that
\[x^p+\cdots+x^{q}\in V\]
for all $q\geq p$ and our assertion then follows from Cauchy's criterion, since $A$ is complete.\par
Suppose that there exists $y\in A$ such that $yz\equiv 1$ mod $\a$. The hypothesis on $\a$ implies, by (a), that $yz$ is invertible in $A$ and hence $y$ is invertible in $A$. In particular, every $x\in \a$ is such that $1-x$ is invertible in $A$ and, as $\a$ is an ideal of $A$, it is contained in the Jacobson radical of $A$.
\end{proof}
\begin{proposition}\label{filtration m-adic completion is local}
Let $A$ be a ring and $\m$ a maximal ideal of $A$. The Hausdorff completion $A$ of $A$ with respect to the $\m$-adic topology is a local ring whose maximal ideal is $\widehat{\m}$.
\end{proposition}
\begin{proof}
If $\n=\bigcap_{n\geq 1}\m^n$, then $\widehat{A}$ is the completion of the Hausdorff ring $A/\n$ associated with $A$ and, as $\m/\n$ is maximal in $A/\n$, we may assume that $A$ is Hausdorff with respect to the $\m$-adic topology. As $A/\m$ and $\widehat{A}/\widehat{\m}$ are isomorphic rings, $\widehat{\m}$ is maximal in $\widehat{A}$. As the topology on $A$ is defined by the filtration $(\m^n)$, the proposition will be a consequence of \cref{filtration complete Hausdorff ring invertibility}: apply it to the topological ring $\widehat{A}$ and the ideal $\widehat{\m}$, as, for all $x\in\widehat{\m}$, $x^n\in(\widehat{\m})^n\sub\widehat{\m^n}$ and the sequence $(x^n)$ therefore tends to $0$.
\end{proof}
\begin{example}
If we take $A=\Z$, every maximal ideal of $\Z$ is of the form $p\Z$ where $p$ is prime. The ring of $p$-adic numbers $\Z_p$ is then a local ring of which $\p\Z_p$ is the maximal ideal and whose residue field is isomorphic to $\Z/p\Z=\F_p$, and $\Z_{(p)}$ with the $p\Z_{(p)}$-adic topology is identified with a topological subring of $\Z_p$ containing $\Z$.
\end{example}
\begin{corollary}\label{filtration completion of semilocal ring with Jacobson radical}
Let $A$ be a semi-local ring, $\m_1,\dots,\m_r$ its distinct maximal ideals and
\[\t=\bigcap_{i=1}^{r}\m_i=\prod_{i=1}^{r}\m_i\]
its Jacobson radical. Then the Hausdorff completion $\widehat{A}$ of $A$ with respect to the $\t$-adic topology is a semi-local ring, canonically isomorphic to the product $\prod_{i=1}^{r}\widehat{A}_{\m_i}$, where $\widehat{A}_{\m_i}$ is the Hausdorff completion ring of the local ring $A_{\m_i}$ with respect to the $(\m_iA_{\m_i})$-adic topology.
\end{corollary}
\subsection{Lifting properties of the associated graduation}
\begin{theorem}\label{filtration gr(phi) injective surjective iff}
Let $X$, $Y$ be two filtered groups with filtrations $(X_n)$, $(Y_n)$; let $\phi:X\to Y$ be a homomorphism compatible with the filtrations.
\begin{itemize}
\item[(a)] Suppose that the filtration $(X_n)$ is exhaustive. For $\gr(\phi)$ to be injective, it is necessary and sufiicient that $\phi^{-1}(Y_n)=X_n$ for all $n\in\Z$.
\item[(b)] Suppose that one of the following hypotheses holds:
\begin{itemize}
\item[($\alpha$)] $X$ is complete and $Y$ Hausdorff.
\item[($\beta$)] $Y$ is discrete.
\end{itemize}
Then, for $\gr(\phi)$ to be surjective, it is necessary and suficient that $Y_n=\phi(X_n)$ for all $n\in\Z$.
\end{itemize}
\end{theorem}
\begin{proof}
To say that the map $\gr_n(\phi)$ is injective means that
\[X_n\cap\phi^{-1}(Y_{n+1})\sub X_{n+1}.\]
This is obviously the case if $\phi^{-1}(Y_{n+1})=X_{n+1}$. Conversely, if this holds for all $n$, we deduce by induction on $k$ that
\[X_{m}\cap\phi^{-1}(Y_{n+1})\sub X_{n+1}\quad\text{for all $m,n\in\Z$ with $m<n+1$}.\]
As the filtration $(X_n)$ is exhaustive, we see $\bigcup_{m<n}X_m=X$, hence $\phi^{-1}(Y_{n+1})\sub X_{n+1}$ for all $n\in\Z$ and therefore $\phi^{-1}(Y_{n+1})=X_{n+1}$, which completes the proof of (a).\par
To say that the map $\gr_n(\phi)$ is surjective means that
\[Y_n=\phi(X_n)+Y_{n+1}.\]
This is obviously the case if $Y_n=\phi(X_n)$. Conversely, suppose that $Y_n=\phi(X_n)+Y_{n+1}$ for all $n\in\Z$. Let $n$ be an integer and $y$ an element of $Y_n$; we now define a sequence $(x_k)$ of elements of $X_n$ such that for all $k\geq 0$,
\[x_{k+1}\equiv x_k\mod X_{n+k}\And \phi(x_k)\equiv y\mod Y_{n+k}.\]
First we take $x_0$ equal to the identity element of $X$, which certainly gives $\phi(x_0)\equiv y$ mod $Y_n$. Suppose that $x_k\in X_{n}$ has been constructed such that $\phi(x_k)\equiv y$ mod $Y_{n+k}$; then $y-\phi(x_k)\in Y_{n+k}=\phi(X_{n+k})+Y_{n+k+1}$ so there exists $t\in X_{n+k}$ such that $y-\phi(x_k)\equiv\phi(t)$ mod $Y_{n+k+1}$ and hence $\phi(x_k+t)\equiv y$ mod $Y_{n+k+1}$; it then suffices to take $x_{k+1}=x_k+t$ to carry out the induction.\par
This being done, if $Y$ is discrete, there exists $k\geq 0$ such that $Y_{n+k}=\{0\}$, whence $\phi(x_k)=y$ and hence in this case it has been proved that $\phi(X_n)=Y_n$ for all $n$. Suppose now that $X$ is complete and $Y$ Hausdorff. As $x_k-x_j\in X_{n+k}$ for $j\geq k$, $(x_k)$ is a Cauchy sequence in $X$; as $X_n$ is closed in $X$ and hence complete, this sequence has at least one limit $x$ in $X_n$. By virtue of the continuity of $\phi$, $\phi(x)$ is the unique limit of the sequence $(\phi(x_k))$ in $Y$, $Y$ being Hausdorff. But the relations $\phi(x_k)\equiv y$ mod $Y_{n+k+1}$ show that $y$ is also a limit of this sequence, whence $\phi(x)=y$ and it has also been proved that $\phi(X_n)=Y_n$.
\end{proof}
\begin{corollary}\label{filtration gr(phi) injective imply phi injective if}
Suppose that $X$ is Hausdorff and its filtration exhaustive. Then, if $\gr(\phi)$ is injective, $\phi$ is injective.
\end{corollary}
\begin{proof}
Since $\phi^{-1}(Y_n)=X_n$, we have $\ker\phi\sub\bigcap_n\phi^{-1}(Y_n)=\bigcap_nX_n$, whence the corollary.
\end{proof}
\begin{corollary}\label{filtration gr(phi) surjective imply phi surjective if}
Suppose that one of the following hypotheses holds:
\begin{itemize}
\item[($\alpha$)] $X$ is complete, $Y$ is Hausdorff and its filtration is exhaustive;
\item[($\beta$)] $Y$ is discrete and its filtration is exhaustive. 
\end{itemize}
Then, if $\gr(\phi)$ is surjective, $\phi$ is surjective.
\end{corollary}
\begin{proof}
In this case $Y=\bigcup_nY_n=\bigcup_n\phi(X_n)\sub\phi(X)$ by \cref{filtration gr(phi) injective surjective iff}.
\end{proof}
\begin{corollary}\label{filtration gr(phi) bijective imply phi bijective if exhaustive and domain complete}
Suppose that $X$ and $Y$ are Hausdorff, the filtrations of $X$ and $Y$ exhaustive and $X$ complete. Then, if $\gr(\phi)$ is bijective, $\phi$ is bijective.
\end{corollary}
\begin{proposition}\label{filtration generating set of gr(M) lifting prop}
Let $A$ be a filtered ring, $M$ a filtered $A$-module and $(A_n)$ and $(M_n)$ the respective filtrations on $A$ and $M$. Suppose that $A$ is complete and the filtration $(M_n)$ is exhaustive and separated. Let $(x_i)_{i\in I}$ be a finite family of elements of $M$ and for $i\in I$ let $n_i$ be an integer such that $x_i\in M_{n_i}$. Finally let $\xi_i$ be the class of $x_i$ in $\gr_{n_i}(M)$. Then, if $(\xi_i)$ is a system of generators of the $\gr(A)$-module $\gr(M)$, then $(x_i)$ is a system of generators of the $A$-module $M$.
\end{proposition}
\begin{proof}
In the $A$-module $L=A^I$ let $L_n$ denote the set $(a_i)$ such that $a_i\in A_{n-n_i}$ for all $i\in I$; if $p$ and $q$ are the least and greatest of the $n_i$, then $A_{n-p}^I\sub L_n\sub A_{n-q}^I$, so the topology defined on $L$ by the definition $(L_n)$ is the same as the product topology; hence $L$ is a complete filtered $A$-module. As $L$ is free, there exists an $A$-linear map $\phi:L\to M$ such that $\phi((a_i))=\sum_ia_ix_i$ and it is obviously compatible with the filtrations; we must prove that $\phi$ is surjective and for this it is sufficient, by virtue of \cref{filtration gr(phi) surjective imply phi surjective if}, to show that $\gr(\phi)$ is surjective or also that, for all $x\in M_n$, there exist a family $(a_i)$ such that $a_i\in A_{n-n_i}$ for all $i\in I$ and $x\equiv\sum_ia_ix_i$ mod $M_{n+1}$. Let $\xi$ be the class of $x$ in $\gr_n(M)$; since the $\xi_i$ generate the $\gr(A)$-module $\gr(M)$, there exist $\alpha_i\in\gr(A)$ such that $\xi=\sum_ia_i\xi_i$, and we may assume that $\alpha_i\in\gr_{n-n_i}(A)$ by replacing if need be $\alpha_i$ by its homogeneous component of degree $n-n_i$. Then $\alpha_i$ the image of an element $a_i\in A_{n-n_i}$ and the family $(a_i)$ has the required property.
\end{proof}
\begin{corollary}\label{filtration finiteness of gr(M) imply that of M}
Let $A$ be a complete filtered ring and $M$ a filtered $A$-module whose filtration is exhaustive and separated. If $\gr(M)$ is a finitely generated (resp. Noetherian) $\gr(A)$-module, then $M$ is a finitely generated (resp. Noetherian) $A$-module.
\end{corollary}
\begin{proof}
If $\gr(M)$ is finitely generated, there is a finite system of homogeneous generators and \cref{filtration generating set of gr(M) lifting prop} shows that $M$ is finitely generated. Suppose now that $\gr(M)$ is Noetherian and let $N$ be a submodule of $M$; the filtration induced on $N$ by that on $M$ is exhaustive and separated and $\gr(N)$ is identified with a sub-$\gr(A)$-module of $\gr(M)$ by \cref{filtration functor gr exact with induced filtration} and hence is finitely generated by hypothesis; we conclude that $N$ is a finitely generated $A$-module and hence $M$ is Noetherian.
\end{proof}
\begin{corollary}\label{filtration complete ring gr(A) Noe imply A Noe}
Let $A$ be a complete Hausdorff filtered ring whose filtration is exhaustive. If $\gr(A)$ is Noetherian, so is $A$.
\end{corollary}
\begin{corollary}\label{filtration complete ring M=M_1+N then M=N}
Let $A$ be a complete filtered ring, $(A_n)$ its filtration, $M$ a Hausdorff filtered $A$-module, $(M_n)$ its filtration and $N$ a finitely generated submodule of $M$; suppose that $A_0=A$ and $M_0=M$.
\begin{itemize}
\item[(a)] If for all $n\in\Z$ we have $M_n=M_{n+1}+A_nN$, then $M=N$.
\item[(b)] If the filtration on $M$ is derived from that on $A$, the relation $M=M_1+N$ implies $M=N$.
\end{itemize}
\end{corollary}
\begin{proof}
Let $\xi_1,\dots,\xi_s$ be the classes mod $M_1$ of a finite system of generators of $N$. It follows from the given hypothesis that for all $n\geq 0$ every element of $\gr_n(M)$ can be expressed in the form $\sum_{i=1}^{s}\alpha_i\xi_i$, where $\alpha_i\in\gr(A)$; the $\xi_i$ therefore generate the $\gr(A)$-module $\gr(M)$, which proves (a) by virtue of \cref{filtration generating set of gr(M) lifting prop}. If the filtration on $M$ is derived from that on $A$, the relation $M=M_1+N$ implies
\[M_n=A_nM=A_nM_1+A_nN=A_nA_1M+A_nN\sub A_{n+1}M+A_nN=M_{n+1}+A_nN\sub M_n\]
and assertion (b) then follows from this.
\end{proof}
\begin{proposition}\label{filtered discrete ring generator of A and gr(A)}
Let $A$ be a filtered ring and $(A_n)$ its filtration. Suppose that there exist a subring $C$ of $A_0$ such that $C\cap A_1=\{0\}$ and $C$ is identified with a subring of $\gr_0(A)$. Let $(x_1,\dots,x_r)$ be a family of element of $A$ with $x_i\in A_{n_i}$ and let $\xi_i$ be the class of $x_i$ in $\gr_{n_i}(A)$.
\begin{itemize}
\item[(a)] If the family $(\xi_i)$ of elements of $\gr(A)$ is algebraically independent over $C$, then the family $(x_i)$ is algebraically independent over $C$.
\item[(b)] If the filtration on $A$ is exhaustive and discrete and $(\xi_i)$ is a system of generators of the $C$-algebra $\gr(A)$, then $(x_i)$ is a system of generators of the $C$-algebra $A$. 
\end{itemize}
\end{proposition}
\begin{proof}
Let $B$ be the polynomial algebra $C[X_1,\dots,X_r]$ over $C$ and give $B$ the graduation $(B_n)$ of type $\Z$ where $B_n$ is the set of $C$-linear combinations of the monomials $X_1^{s_1},\dots,X_r^{s_r}$ such that $\sum_in_is_i=n$. Let $\phi$ be the homomorphism $f\mapsto f(x_1,\dots,x_r)$ from the $C$-algebra $B$ to the $C$-algebra $A$. By definition, $\phi(B_n)\sub A_n$ for all $n\in\Z$ and hence $\phi$ is compatible with the filtrations. The hypothesis of (a) means that $\gr(\phi)$ is injective. As the filtration on $B$ is exhaustive and separated, \cref{filtration gr(phi) injective imply phi injective if} may be applied and $\phi$ is injective, which proves the conclusion of (a). Similarly, the hypothesis (b) on the $(\xi_i)$ means that $\gr(\phi)$ is surjective. As $A$ is discrete and its filtration is exhaustive, \cref{filtration gr(phi) injective surjective iff} may be applied and $\phi$ is surjective, which proves the conclusion of (b).
\end{proof}
\begin{proposition}\label{filtered complete ring generator of A and gr(A)}
Let $A$ be a complete Hausdorff filtered ring and $(A_n)$ its filtration. Suppose that there exist a subring $C$ of $A_0$ such that $C\cap A_1=\{0\}$ and $C$ is identified with a subring of $\gr_0(A)$. Let $(x_1,\dots,x_r)$ be a family of element of $A$ with $x_i\in A_{n_i}$ and let $\xi_i$ be the class of $x_i$ in $\gr_{n_i}(A)$.
\begin{itemize}
\item[(a)] There exists a unique $C$-homomorphism $\rho$ from $B=C\llbracket X_1,\dots,X_r\rrbracket$ to $A$ such that $\rho(X_i)=x_i$.
\item[(b)] If the family $(x_i)$ is algebraically independent over $C$, the homomorphism $\rho$ is injective.
\item[(c)] If the filtration on $A$ is exhaustive and $(\xi_i)$ is a system of generators of the $C$-algebra $\gr(A)$, then $\rho$ is surjective.
\end{itemize}
\end{proposition}
\begin{proof}
As $n_i\geq 1$ for all $i$, we have $\sum_in_is_i\geq\sum_is_i$ for every monomial $X_1^{s_1},\dots,X_r^{s_r}$ and on the other hand $\sum_in_is_i\leq N$ if $N$ is the greatest of the $n_i$. If $B_n$ denotes the set of formal power series whose non-zero terms $a_sX^s$ satisfy $\sum_in_is_i\leq n$, it follows from \cref{filtration on power series complete} that $B$ is Hausdorff and complete with the exhaustive filtration $(B_n)$ and that $B'=C[X_1,\dots,X_r]$ is dense in $B$. Moreover the homomorphism $\phi$ defined in the proof of \cref{filtered discrete ring generator of A and gr(A)} is continuum on $B'$ and can be extended uniquely to a continuous homomorphism $\rho:B\to A$, since $A$ is Hausdorff and complete, which proves (a). Also, $\gr(B)=\gr(B')$ and $\gr(v)=\gr(\phi)$, so (b) and (c) follow respectively from \cref{filtration gr(phi) injective imply phi injective if} and \cref{filtration gr(phi) injective surjective iff} in view Of the hypotheses on $A$.
\end{proof}
Let $A$ be a local ring, $\m$ its maximal ideal and $M$ an $A$-module; let $A$ and $M$ be given the $\m$-adic filtrations and let $\gr(A)$ and $\gr(M)$ be the graded ring and the graded $\gr(A)$-module associated with $A$ and $M$. We have seen (\cref{filtration of I-adic isomorphic to symmetric algebra}) that the canonical map (\ref{filtration gr(M) generated by gr_0(M)}) is always surjective; we are going to consider the following property of $M$:
\begin{enumerate}[leftmargin=40pt]
\item[(GR)] The canonical map $\gamma_M:\gr(A)\otimes_{\gr_0(A)}\gr_0(M)\to\gr(M)$ is bijective.
\end{enumerate}
\begin{proposition}\label{filtration local ring module (GR) propoty prop}
Let $A$ be a local ring, $\m$ its maximal ideal, $M$, $N$ two $A$-modules and $\phi:N\to M$ an $A$-homomorphism. Suppose that $M$ and $N$ are given the $\m$-adic filtrations and
\begin{itemize}
\item[(a)] $M$ satisfies property (GR);
\item[(b)] $\gr_0(\phi):\gr_0(N)\to\gr_0(M)$ is injective.
\end{itemize}
Then $\gr(\phi):\gr(N)\to\gr(M)$ is injective and $N$ and $P=\coker\phi$ satisfy property (GR).
\end{proposition}
\begin{proof}
It is immediately verified that the diagram
\[\begin{tikzcd}
\gr(A)\otimes_{\gr_0(A)}\gr_0(N)\ar[d,"\gamma_N"]\ar[r,"1\otimes\gr_0(\phi)"]&\gr(A)\otimes_{\gr_0(A)}\gr_0(M)\ar[d,"\gamma_M"]\\
\gr(N)\ar[r,"\gr(\phi)"]&\gr(M)
\end{tikzcd}\]
is commutative. As $\gr_0(A)=A/\m$ is a field, the hypothesis implies that $1\otimes\gr_0(\phi)$ is injective; as by hypothesis $\gamma_M$ is injective, so is $\gamma_M\circ(1\otimes\gr_0(\phi))$. This implies first that $\gamma_N$ is injective and hence bijective and therefore that $\gr(\phi)$ is injective. The formula $\phi^{-1}(\m^nM)=\m^nN$ is then a consequence of \cref{filtration gr(phi) injective surjective iff}(a).\par
Also, let us write $N'=\phi(N)$ and let $\iota:N'\to M$ be the canonical injection. If $\pi:M\to P=M/N'$ is the canonical homomorphism, then in the commutative diagram
\[\begin{tikzcd}[column sep=15pt]
0\ar[r]&\gr(A)\otimes_{\gr_0(A)}\gr_0(N')\ar[d,"\gamma_{N'}"]\ar[r]&\gr(A)\otimes_{\gr_0(A)}\gr_0(M)\ar[d,"\gamma_M"]\ar[r]&\gr(A)\otimes_{\gr_0(A)}\gr(P)\ar[d,"\gamma_P"]\ar[r]&0\\
0\ar[r]&\gr(N)\ar[r,"\gr(\iota)"]&\gr(M)\ar[r,"\gr(\pi)"]&\gr(P)\ar[r]&0
\end{tikzcd}\]
the lower row is exact by \cref{filtration functor gr exact with induced filtration} and so is the upper row by virtue of \cref{filtration functor gr exact with induced filtration} and the fact that $\gr_0(A)$ is a field. The first part of the argument applied to $\iota$ shows that $\gamma_{N'}$ is bijective; as $\gamma_M$ is also bijective by hypothesis, we conclude that $\gamma_P$ is bijective by the snake lemma.
\end{proof}
\begin{corollary}
Under the hypotheses of \cref{filtration local ring module (GR) propoty prop}, if we assume also that $N$ is Hausdorff with the $\m$-adic filtration, then $\phi$ is injective.
\end{corollary}
\begin{proposition}\label{filtration module free iff M/mM is free and (GR)}
Let $A$ be a ring, $\m$ an ideal of $A$ contained in the Jacobson radical of $A$ and $M$ an $A$-module. Let $A$ and $M$ be given the $\m$-adic filtrations. Suppose that one of the following conditions holds:
\begin{itemize}
\item[(a)] $M$ is a finitely generated $A$-module and $A$ is Hausdorff.
\item[(b)] $\m$ is nilpotent.
\end{itemize}
Then for $M$ to be a free $A$-module, it is necessary and suffcient that $M/\m M$ be a free $(A/\m)$-module and that $M$ satisfies property (GR).
\end{proposition}
\begin{proof}
If $M$ is a free $A$-module and $(x_\lambda)$ a basis of $M$, $\m^nM$ is the direct sum of the submodules $\m^nx_\lambda$ of $M$ for all $n\geq 0$; then $\m^nM/\m^{n+1}M$ is identified with the direct sum of the $\m^nx_\lambda/\m^{n+1}x_\lambda$. We deduce first (for $n=0$) that the classes $1\otimes x_\lambda$ of the $x_\lambda$ in $M/\m M=(A/\m)\otimes_AM$ form a basis of the $(A/\m)$-module $M/\m E$, since the canonical map
\[(\m^n/\m^{n+1})\otimes_A(M/\m M)\to\m^n M/\m^{n+1}M\]
is bijective for all $n\geq 0$; hence $\gamma_M$ is bijective. Note that this part of the proof uses neither condition (a) nor condition (b).\par
Suppose conversely that the conditions of the statement hold and let $(x_\lambda)_{\lambda\in I}$ be a family of elements of $M$ whose classes mod $\m M$ form a basis of the $(A/\m)$-module $M/\m M$; let $L$ be the free $A$-module $A^{\oplus I}$ and $(e_\lambda)$ a basis of $L$; let $\phi:L\to M$ the $A$-linear map such that $\phi(e_\lambda)=x_\lambda$ for all $\lambda\in I$. Now, each of the hypotheses (a) and (b) implies that $A$ is Hausdorff and hence so is $L$ with the $\m$-adic filtration, since $\m^nL=(\m^n)^{\oplus I}$. Also, since $\gr(L)$ is identified with $\gr(A)\otimes_{A/\m}(L/\m L)$ from the first part of the proof, it is possible to write $\gr(\phi)$ as
\[
\begin{tikzcd}
\gr(L)\ar[r,"\cong"]&\gr(A)\otimes_{A/\m}(L/\m L)\ar[r,"\eta"]&\gr(A)\otimes_{A/\m}(M/\m M)\ar[r,"\gamma_M"]&\gr(M)
\end{tikzcd}
\]
where $\eta$ is the bijection map the class $1\otimes\bar{e}_\lambda$ onto $1\otimes\bar{x}_\lambda$. The hypothesis then implies that $\gr(\phi)$ is bijective and the conclusion follows from \cref{filtration gr(phi) injective imply phi injective if}, \cref{module generating set in M/mM} and \cref{filtration gr(phi) surjective imply phi surjective if}.
\end{proof}
We have already seen that if $A$ is a ring and $\a$ is an ideal of $A$ such that $A$ is Hausdorff and complete with the $\a$-adic topology, then the topological ring $A$ is cannically identified with the inverse limit of the discrete rings $A_i=A/\a^{i+1}$ ($i\in\N$) with respect to the canonical homomorphisms 
\[\pi_{ij}:A/\a^{j+1}\to A/\a^{i+1}\]
for $i\leq j$. Note that $\pi_{ij}$ is surjective and if $\n_{ij}$ is its kernel, then
\[\n_{ij}=\a^{i+1}/\a^{j+1}=(\a/\a^{j+1})^{i+1}=(\n_{0,j})^{i+1};\]
in particular $(\n_{0,j})^{j+1}=0$. Conversely, we have the following result:
\begin{proposition}\label{ring inverse limit complete and finiteness prop}
Let $(A_i,\pi_{ij})$ be an inverse system of discrete rings indexed by $\N$ and let $(M_i,u_{ij})$ be an inverse system of modules over the inverse system of rings $(A_i,\pi_{ij})$. Let $\n_j$ be the kernel of $\pi_{0,j}:A_j\to A_0$ and set $A=\llim A_i$, $M=\llim M_i$. Suppose that
\begin{itemize}
\item[(a)] for $i\leq j$, $\pi_{ij}$ and $u_{ij}$ are surjective;
\item[(b)] for $i\leq j$, the kernel of $\pi_{ij}$ and $u_{ij}$ are $\n_j^{i+1}$ and $\n_j^{i+1}M_j$, respectively.
\end{itemize}
Then the following assertions hold:
\begin{itemize}
\item[(\rmnum{1})] $A$ is a complete Hausdorff topological ring, $M$ is a complete topological $A$-module and the canonical homomorphisms $\pi_i:A\to A_i$, $u_i:M\to M_i$ are surjective.
\item[(\rmnum{2})] If $M_0$ is a finitely generated $A_0$-module, then $M$ is a finitely generated $A$-module. In fact, every finite subset $E$ of $M$ such that $u_0(E)$ generates $M_0$ is a system of generators of $M$.
\end{itemize}
\end{proposition}
\begin{proof}
The first assertion follows from \cref{uniform space inverse limit of complete is complete} and \cref{set inverse limit countable cofinal surjective}. For each $i$, let $\m_{i+1}=\ker\pi_i$ and $N_{i+1}=\ker u_i$; then by hypothesis (b),
\[\m_{i+1}=\llim_k(\ker\pi_{i,i+k})=\llim_k\n_{i+1}^{i+k},\quad N_{i+1}=\llim_k(\ker u_{i,i+k})=\llim_k\n_{i+k}^{i+1}M_{i+k},\]
and as $\pi_{i+k}$ and $u_{i+k}$ are surjective, we also have
\begin{align}\label{ring inverse limit complete and ft prop-1}
\pi_{i+k}(\m_{i+1})=\n_{i+k}^{i+1},\quad u_{i+k}(N_{i+1})=\n_{i+k}^{i+1}M_{i+k}.
\end{align}
We first prove that $\m_iN_j\sub N_{i+j}$, which amounts to saying that $u_{i+j-1}(\m_iN_j)=0$. For this, note that $\n_k^{k+1}=0$ as it is the kernel of $\pi_{kk}=\id$, so
\[u_{i+j-1}(\m_iN_j)=\pi_{i+j-1}(\m_i)u_{i+j-1}(N_j)=\n_{i+j-1}^i(\n_{i+j-1}^jM_{i+j-1})=0.\]
Similarly, we see that $\m_i\m_j\sub\m_{i+j}$. If we set $\m_i=A$ and $N_i=M$ for $i\leq 0$, then $(\m_i)_{i\in\Z}$ is a filtration of $A$ and $(N_i)_{i\in\Z}$ is a filtration of $M$ compatible with that of $A$, and the topology on $A$ and $M$ are obviously those defined by these filtrations. Now let $\a$ be an ideal of $A$ such that $\pi_1(\a)=\n_1$ and $M'$ be the submodule of $M$ generated by a subset $E$ of $M$ such that $u_0(E)$ generates $M_0$. We are going to prove that for each $i\geq 0$,
\begin{align}\label{ring inverse limit complete and ft prop-2}
N_i=\a^iM'+N_{i+1}.
\end{align}
Let us write $\a_i=\pi_i(\a)$ and $M'_i=u_i(M')$. From the definition of $N_{i+1}$, it then suffices to show that
\[u_i(N_i)=\a_i^iM'_i.\]
This is true if $i=0$, since $N_0=M$ and $M'_0=M_0$ by hypothesis. If $i\geq 1$, then $u_i(N_i)=\n_i^iM_i$ by (\ref{ring inverse limit complete and ft prop-1}). As $\pi_{1,i}$ is surjective and $\pi_{0,i}=\pi_{0,1}\circ\pi_{1,i}$, the transition homomorphism $\pi_{1,i}$ maps the kernel $\n_i$ of $\pi_{0,i}$ onto the kernel $\n_1$ of $\pi_{0,1}$ and $\n_i=\pi_{1,i}^{-1}(\n_1)$, so
\[\pi_{1,i}(\a_i)=\pi_1(\a)=\n_1=\pi_{1,i}(\n_i).\]
As the kernel of $\pi_{1,i}$ is $\n_i^2$, we then conclude that $\a_i\sub\n_i\sub\a_i+\n_i^2$, whence $\n_i=\a_i+\n_i^2$. On the other hand, we have $u_{0,i}(M'_i)=u_0(M')=M_0=u_{0,i}(M_i)$, and as $\ker u_{0,i}=\n_iM_i$ by hypothesis, this implies $M_i=M'_i+\n_iM_i$, so
\begin{align}\label{ring inverse limit complete and ft prop-3}
\n_i^iM_i=(\a_i+\n_i^2)^i(M'_i+\n_iM_i).
\end{align}
We note that $\a_i^k\n_i^{i+1-k}\sub\n_i^{i+1}=0$ for $0\leq k\leq i$, so it follows from (\ref{ring inverse limit complete and ft prop-1}) and (\ref{ring inverse limit complete and ft prop-3}) that
\[u_i(N_i)=\n_i^iM_i=\a_i^iM'_i,\]
which proves (\ref{ring inverse limit complete and ft prop-2}). Now since $\m_1=\pi_1^{-1}(\n_1)$, we have $\a\sub\m_1$ and then $\a^i\sub\m_1^i\sub\m_i$, whence $N_i\sub\m_iM'+N_{i+1}$ in view of (\ref{ring inverse limit complete and ft prop-2}). On the other hand, it is clear that $\m_iM\sub N_i$, so $N_i=\m_iM'+N_{i+1}$ for $i\geq 0$. It then follows from \cref{filtration complete ring M=M_1+N then M=N} that $M'=M$, which completes the proof.
\end{proof}
\begin{corollary}\label{ring inverse limit topology is adic if finite ideal}
Under the hypotheses of \cref{ring inverse limit complete and finiteness prop}, suppose that $M_0$ is a finitely generated $A_0$-module and that the ideal $\n_1$ of $A_1$ is finitely generated. Let $\m_{i+1}$ (resp. $N_i$) be the kernel of the canonical homomorphism $\pi_i:A\to A_i$ (resp. $u_i:M\to M_i$), and set $\m=\m_1$. Then
\begin{itemize}
\item[(a)] For each $i>0$, we have $\m_i=\m^i$ and $N_i=\m^iM$. In other words, the topology of $A$ and $M$ are the $\m$-adic topology.
\item[(b)] $\m/\m^2$ is a finitely generated $A$-module.
\end{itemize}
\end{corollary}
\begin{proof}
We preserve the notation of the proof of \cref{ring inverse limit complete and finiteness prop}. The hypotheses here allow us to assume that the ideal $\a$ is \textit{finitely generated}. Let $i,j$ be positive integers, then by (\ref{ring inverse limit complete and ft prop-2}) we have
\[N_{i+j}=\a^j(\a^iM)+N_{i+j+1}\sub\m_j(\a^iM)+N_{i+j+1}.\]
Conversely, $\m_j(\a^iM)\sub\m_j\m_iM\sub\m_{i+j}N\sub N_{i+j}$, so
\[N_{i+j}=\m_i(\a^iM)+N_{i+j+1}.\]
As $\a$ and $M$ are finitely generated $A$-modules, so is $\a^iM$. Applying \cref{filtration complete ring M=M_1+N then M=N} to the module $N_i$ with the filtration $(N_{ij})_{j\in\Z}$ defined by $N_{ij}=N_i$ if $j<0$ and $N_{ij}=N_{i+j}$ if $j\geq 0$, we then obtain $N_i=\a^iM$, whence $N_i\sub\m^iM$. But we also have $\m^iM\sub\m_iM\sub N_i$, so $N_i=\m^iM$. In particular, applying this to the case where $M_i=A_i$ and $u_{ij}=\pi_{ij}$, we obtain that $\m_i=\m^i$. Moreover, $\m=\a+\m^2$ by (\ref{ring inverse limit complete and ft prop-2}), which proves the last assertion of the corollary.
\end{proof}
\begin{corollary}\label{ring inverse limit Noe iff A_0 Noe}
Under the hypotheses of \cref{ring inverse limit topology is adic if finite ideal}, for $A$ to be Noetherian, it is necessary and sufficient that $A_0$ is Noetherian.
\end{corollary}
\begin{proof}
This condition is necessary since $A_0$ is isomorphic to a quotient of $A$, and is sufficient in view of \cref{filtration complete ring gr(A) Noe imply A Noe}.
\end{proof}
\section{The adic topology on Noetherian rings}
Let $A$ be a filtered ring, $M$ a filtered $A$-module and $(A_n)$ and $(M_n)$ be the corresponding filtrations. Since we only consider the $\a$-adic filtration, we may assume that all filtrations are exhaustive and consist of submodules. The \textbf{Rees ring} of $A$ is then defined by
\[\Rees(A)=\bigoplus_{n\in\N}A_nX^n\sub A[X]\]
and the \textbf{Rees mudule} of $M$ is defined by
\[\Rees(M)=\bigoplus_{n\in\N}M_n\otimes_AAX^n\sub M\otimes_AA[X].\]
Note that $\Rees(M)$ is a graded $\Rees(A)$-module of type $\N$, since we have
\[A_mX^m(M_n\otimes_AAX^n)\sub(A_mM_n\otimes_AAX^{m+n})\sub M_{m+n}\otimes_AAX^{m+n}.\]
The most important case is that $\a$ is an ideal of $A$ and $A$, $M$ are given the $\a$-adic filtration. In this case we write $\Rees_\a(A)$ for the Rees ring $\bigoplus_{n\in\N}\a^nX^n$ and $\Rees_\a(M)$ for $\bigoplus_{n\in\N}\a^nM\otimes_AAX^n$.
\subsection{Good filtrations}
Let $A$ be a ring, $\a$ an ideal of $A$, $M$ an $A$-module and $(M_n)$ a filtration on the additive group $M$ consisting of submodules of $M$. The filtrations $(M_n)$ is called \textbf{$\a$-good} if:
\begin{itemize}
\item[(a)] $\a M_n\sub M_{n+1}$ for all $n\in\Z$.
\item[(b)] There exists an integer $n_0$ such that $\a M_n=M_{n+1}$ for all $n\geq n_0$.
\end{itemize}
If $(M_n)$ is $\a$-good, then by induction on we see $\a^pM_n=M_{n+p}$ for all $n\geq n_0$ and $p>0$. Note that condition (a) means that the filtration $(M_n)$ is compatible with the $A$-module structure on $M$ if $A$ is given the $\a$-adic filtration. Clearly, on every $A$-module $M$, the $\a$-adic filtration is $\a$-good.
\begin{theorem}\label{filtration I-good iff Res(M) is finite}
Let $A$ be a ring, $\a$ an ideal of $A$, and give $A$ the $\a$-adic filtration. Let $M$ be an $A$-module and $(M_n)$ a filtration of $M$ consisting of finitely generated submodules. Suppose that $\a M_n\sub M_{n+1}$ for all $n$. Then the two following conditions are equivalent:
\begin{itemize}
\item[(a)] The filtration $(M_n)$ is $\a$-good.
\item[(b)] $\Rees_\a(M)$ is a finitely generated $\Rees_\a(A)$-module.
\end{itemize}
\end{theorem}
\begin{proof}
Suppose that $\a M_{n}=M_{n+1}$ for $n\geq n_0\geq 0$. For $i\leq n_0$, let $(e_{ij})_{1\leq j\leq r_i}$ be a finite system of generators of the $A$-module $M_i$. As the $A$-module $M_n\otimes_AAX^n$ is generated by the elements $e_{nj}\otimes X^n$ for $0\leq n\leq n_0$ and is equal to
\[\a^{n-n_0}M_{n_0}\otimes_AAX^n\]
for $n>n_0$, the $\Rees(A)$-module $\Rees(M)$ is generated by the elements $e_{nj}\otimes X^n$ for $0\leq n\leq n_0$ and $1\leq j\leq r_n$; then it is certainly finitely generated.\par
Conversely, if $\Rees(M)$ is a finitely generated $\Rees(A)$-module, it is generated by a finite family of elements of the form $e_k\otimes X^{n_k}$, where $e_k\in M_{n_k}$. Let $n_0$ be the greatest of the integers $n_k$. Then for $n\geq n_0$ and $x\in M_n$,
\[x\otimes X^n=\sum_k t_k(e_k\otimes X^{n_k})\]
where $t_k\in\Rees(A)$; replacing if need be to by its homogeneous component of degree $n-n_k$, we may assume that $t_k=a_kX^{n-n_k}$ where $a_k\in \a^{n-n_k}$. As the unique element $X^n$ forms a basis of the $A$-module $AX^n$, the equation $x\otimes X^n=(\sum_ka_ke_k)\otimes X^n$ implies $x=\sum_ka_ke_k$. Then $M_n\sub \a^{n-n_0}M_{n_0}$; since the opposite inclusion is obvious by hypothesis, we see $M_n=\a^{n-n_0}M_{n_0}$, whence $\a M_n=M_{n+1}$ for $n\geq n_0$.
\end{proof}
\begin{lemma}\label{filtration Noe imply Res(A) Noe}
Let $A$ be a Noetherian ring and $\a$ an ideal of $A$. Then the subring $\Rees_\a(A)$ of $A[X]$ is Noetherian.
\end{lemma}
\begin{proof}
Note that $\Rees_\a(A)$ is an $A$-algebra generated by $\a X$; as $A$ is Noetherian, $\a X$ is a finitely generated $A$-module, so the claim follows from Hilbert basis theorem.
\end{proof}
\begin{proposition}\label{filtration I-good submodule and quotient}
Let $A$ be a filtered ring, $\a$ an ideal of $A$, and $M$ is an $A$-module with a $\a$-good filtration. If $N$ is a submodule of $M$ then the quotient filtration on $M/N$ is $\a$-good. If further $A$ is Noetherian, then the submodule filtration on $N$ is $\a$-good. 
\end{proposition}
\begin{proof}
If $N$ and $M/N$ are endowed with the submodule (resp. quotient) filtrations respectively, then $\Rees_\a(N)$ is a $\Rees_\a(A)$-submodule of $\Rees_\a(M)$ and we have $\Rees_\a(M/N)\cong\Rees_\a(M)/\Rees_\a(N)$ as $\Rees_\a(A)$-modules. Since the filtration on $M$ is $\a$-good, $\Rees_\a(M)$ is a finitely generated $\Rees_\a(A)$-module and so $\Rees_\a(M/N)$ is finitely generated, which implies the filtration on $M/N$ is $\a$-good. If further $A$ is Noetherian then $\Rees_\a(A)$ is Noetherian, and so $\Rees_\a(N)$ is finitely generated. It then follows that the filtration on $N$ is $\a$-good.
\end{proof}
\begin{corollary}[\textbf{Artin-Rees Lemma}]\label{filtration Artin-Ress lemma}
Let $A$ be a Noetherian ring, $\a$ an ideal of $A$, $M$ a finitely generated $A$-module and $N$ a submodule of $M$. The filtration induced on $N$ by the $\a$-adic filtration on $M$ is $\a$-good. In other words, there exists an integer $n_0$ such that for $n\geq n_0$ we have
\[\a(\a^n M\cap N)=\a^{n+1}M\cap N.\]
\end{corollary}
\begin{corollary}
Let $A$ be a Noetherian ring and $\a$, $\b$ two ideals of $A$. Then there exists an integer $n>0$ such that $\a^n\cap \b\sub \a \b$.
\end{corollary}
\begin{proof}
By the Artin-Rees Lemma, there exist $n>0$ such that $\a^{n+1}\cap\b=\a(\a^n\cap\b)\sub\a\b$.
\end{proof}
\begin{corollary}
Let $A$ be a Noetherian ring, $\a$ an ideal of $A$ and $x$ an element of $A$ which is not a divisor of $0$. There exists an integer $n_0>0$ such that, for all $n\geq n_0$, we have $(\a^n:x)\sub \a^{n-n_0}$.
\end{corollary}
\begin{proof}
\cref{filtration Artin-Ress lemma} applied to $M=A$, $N=Ax$ shows that there exists $n_0$ such that, for all $n\geq n_0$, $\a^n\cap Ax=\a^{n-n_0}(\a^{n_0}\cap Ax)$. Then, if $xy\in \a^n$,
\[xy\in \a^n\cap Ax\sub \a^{n-n_0}x\]
and, as $x$ is not a divisor of $0$, we deduce that $y\in \a^{n-n_0}$.
\end{proof}
\begin{corollary}
Let $A$ be a Noetherian ring, $\a$ an ideal of $A$, $M$ a finitely generated $A$-module and $(M_n)$ and $(M_n')$ two filtrations consisting of submodules of $M$. Suppose that the filtrations $(M_n)$ and $(M_n')$ are compatible with the $A$-module structure on $M$ when $A$ is given the $\a$-adic filtration. If the filtration $(M_n)$ is $\a$-good and $M_n'\sub M_n$ for all $n\in\Z$, then the filtration $(M_n')$ is $\a$-good.
\end{corollary}
\begin{proof}
This is a special case of \cref{filtration I-good submodule and quotient}.
\end{proof}
\begin{lemma}\label{scalar extension Hom finite lemma}
Let $A$, $B$ be two Noetherian rings, $\rho:A\to B$ a ring homomorphism, $M$ a finitely generated $A$-module and $N$ a finitely generated $B$-module. Then $\Hom_A(M,N)$ is a finitely generated $B$-module.
\end{lemma}
\begin{proof}
There exists by hypothesis a surjective $A$-homomorphism $\eta:A^n\to M$; the map $\phi\mapsto\phi\circ\eta$ of $\Hom_A(M,N)$ to $\Hom_A(A^n,N)$ is therefore injective and, as $B$ is Noetherian, it is sufiicient to prove that $\Hom_A(A^n,N)$ is a finitely generated $B$-module; which 1s immediate since it is isomorphic to $N^n$.
\end{proof}
\begin{proposition}
Let $A$ be a Noetherian ring, $\a$ an ideal of $A$ and $M$, $N$ two finitely generated $A$-modules. If $(N_n)$ is an $\a$-good filtration on $N$, the submodules $\Hom_A(E,N_n)$ form an $\a$-good filtration on the $A$-module $\Hom_A(M,N)$.
\end{proposition}
\begin{proof}
As $\a^kN_n\sub N_{n+k}$ for $n\in\Z$ and $k\geq 0$, it is also true that
\[\a^k\Hom_A(M,N_n)\sub\Hom_A(M,N_{n+k})\]
the family $(\Hom_A(M,N_n))$ is then a filtration on $\Hom_A(M,N)$ compatible with its module structure over the ring $A$ filtered by the $\a$-adic filtration. Since $M$ is finitely generated, there exists an integer $r>0$ and a surjective $A$-homomorphism $\eta:A^r\to M$ which defines an injective $A$-homomorphism
\[\eta^*:\Hom_A(M,N)\to\Hom_A(A^r,N)\]
clearly $\eta^*$ is compatible with the filtrations $(\Hom_A(M,N))$ and $(\Hom_A(A^r,N))$. As the $A$-modules $\Hom_A(M,N)$ and $\Hom_A(A^r,N)$ are finitely generated (\cref{scalar extension Hom finite lemma}), it is sufficient by virtue of \cref{filtration I-good submodule and quotient} to show that the filtration $(\Hom_A(A^r,N))$ is $\a$-good; but this is immediate by virtue of the existence of the canonical isomorphism $\Hom_A(A^r,N)\to N^r$, and the fact that the relation $\a N=N_{n+1}$ implies $\a(N_n^r)=(\a N_n)^r=N_{n+1}^r$.
\end{proof}
\begin{proposition}\label{filtration I-adic complete Noe ring good filtration iff}
Let $A$ be a Noetherian ring and $\a$ an ideal of $A$ such that $A$ is Hausdorff and complete with respect to the $\a$-adic topology. Let $M$ be a filtered $A$-module over the filtered ring $A$, the filtration $(M_n)$ of $A$ being such that $M_0=M$ and $M$ is Hausdorff with respect to the topology defined by $(M_n)$. Then the following conditions are equivalent:
\begin{itemize}
\item[(\rmnum{1})] $M$ is a finitely generated $A$-module and $(M_n)$ is an $\a$-good filtration.
\item[(\rmnum{2})] $\gr(M)$ is a finitely generated $\gr(A)$-module.
\item[(\rmnum{3})] $\gr_n(M)$ is a finitely generated $A$-module for all $n$ and there exists $n_0$ such that for $n\geq n_0$ the canonical homomorphism $\gr_1(A)\otimes_{A}\gr_n(M)\to\gr_{n+1}(M)$ is surjective.
\end{itemize}
\end{proposition}
\begin{proof}
It follows immediately from the definitions that (\rmnum{1}) implies (\rmnum{3}). Also, (\rmnum{2}) is equivalent to (\rmnum{3}): the fact that (\rmnum{2}) implies (\rmnum{3}) is a consequence of \cref{graded ring bounded module M_n+d=A_dM_n}; conversely, if (\rmnum{3}) holds, clearly $\gr(M)$ is generated as a $\gr(A)$-module by the sum of the $\gr_p(M)$ for $p\leq n_0$ and hence by hypothesis admits a finite system of generators. It remains to prove that (\rmnum{3}) implies (\rmnum{1}).\par
Assume (\rmnum{3}), as the $\gr_n(M)$ are finitely generated and $M_0=M$, clearly first, by induction on $n$, $M/M_n$ is a finitely generated $A$-module for all $n$; it will therefore be sufficient to prove that, for $n>n_0$, $M_n$ is a finitely generated $A$-module and that $\a M_n=M_{n+1}$. Now, consider the $A$-module $M_{n+1}$ with the exhaustive and separated filtration $(M_{n+k})_{k\geq 1}$. Now since $\gr_1(A)=\a/\a^2$, hypothesis (\rmnum{3}) implies that the image of $\a M_n$ in $\gr_{n+1}(M)$ is equal to $\gr_{n+1}(M)$ and generates the graded $\gr(A)$-module $\gr(M_{n+1})$. As $\gr_{n+1}(M)$ is by hypothesis a finitely generated $A$-module, it follows from \cref{filtration generating set of gr(M) lifting prop} that $\a M_n=M_{n+1}$ and that $M_{n+1}$ is a finitely generated $A$-module.
\end{proof}
\begin{remark}
Note that in the proof of \cref{filtration I-adic complete Noe ring good filtration iff}, only the implication (\rmnum{3})$\Rightarrow$(\rmnum{1}) uses the hypothesis $A$ is $\a$-adic complete and Hausdorff. In particular, we see if $A$ is Noetherian and the filtration $(M_n)$ is $\a$-good then $\gr(M)$ is a finitely generated $\gr(A)$-module.
\end{remark}
\subsection{The adic topology on Noetherian rings}
\begin{proposition}\label{filtration I-good induce I-adic topo}
Let $A$ be a Noetherian ring, $\a$ an ideal of $A$ and $M$ a finitely generated $A$-module. All the $\a$-good filtrations on $M$ define the same topology on $M$, namely the $\a$-adic topology.
\end{proposition}
\begin{proof}
Let $(M_n)$ be an $\a$-good filtration on $M$. As this filtration is exhaustive, every element of $M$ belongs to one of the $M_n$ and, as $M$ is finitely generated (hence Noetherian) and the $M_n$ are $A$-modules, there exists an integer $n_1$ such that $M_{n_1}=M$. On the other hand let $n_0$ be such that $\a M_n=M_{n+1}$ for $n\geq n_0$; then, for $n>n_0-n_1$,
\[\a^nM\sub M_{n+n_1}=\a^{n+n_1-n_0}M_{n_0}\sub \a^{n+n_1-n_0}M,\]
which proves the proposition.
\end{proof}
\begin{theorem}[\textbf{Krull}]\label{filtration I-topo induce on submodule}
Let $A$ be a Noetherian ring, $\a$ an ideal of $A$, $M$ a finitely generated $A$-module and $N$ a submodule of $M$. Then the $\a$-adic topology on $N$ is induced by the $\a$-adic topology on $M$.
\end{theorem}
\begin{proof}
It follows from Artin-Rees Lemma that the filtration induced on $N$ by the $\a$-adic filtration on $M$ is $\a$-good and the conclusion then follows from \cref{filtration I-good induce I-adic topo}.
\end{proof}
\begin{corollary}
Let $A$ be a Noetherian ring, $\a$ an ideal of $A$, $M$ an $A$-module and $N$ a finitely generated $A$-module. Then every homomorphism $\phi:M\to N$ is a strict morphism for the $\a$-adic topologies.
\end{corollary}
\begin{proof}
As $\phi(\a^nM)=\a^n\phi(M)$, $\phi$ is a strict morphism of $M$ onto $\phi(M)$ for the $\a$-adic topologies on these two modules and the $\a$-adic topology on $\phi(M)$ is induced by the $\a$-adic topology on $N$ by \cref{filtration I-topo induce on submodule}, hence the claim.
\end{proof}
\begin{corollary}\label{filtration Noe ring I-adic exact seq}
Let $0\to M_1\to M_2\to M_3\to 0$ be an exact sequence of finitely-generated modules over a Noetherian ring $A$. Let $\a$ be an ideal of $A$, then the sequence of $\a$-adic completions
\[\begin{tikzcd}
0\ar[r]&\widehat{M}_1\ar[r]&\widehat{M}_2\ar[r]&\widehat{M}_3\ar[r]&0
\end{tikzcd}\]
is exact.
\end{corollary}
\begin{proposition}\label{Noe ring finite module closure of 0 char}
Let $A$ be a Noetherian ring, $\a$ an ideal of $A$ and $M$ a finitely generated $A$-module. The closure $\bigcap_{n}\a^nM$ of $\{0\}$ in $M$ with respect to the $\a$-adic topology is the set of $x\in M$ for which there exists an element $a\in \a$ such that $(1-a)x=0$.
\end{proposition}
\begin{proof}
If $x=ax$ mod $\a$ where $a\in \a$, then $x=a^nx$ for every integer $n>0$ and hence $x\in\bigcap_{n}\a^nM$. Conversely, if $x\in\bigcap_n\a^nM$, $Ax$ is contained in the intersection of the neighbourhoods of $0$ in $M$; it then follows from \cref{filtration I-topo induce on submodule} that the $\a$-adic topology on $Ax$, which is induced by that on $M$, is the trivial topology; as $\a x$ is by definition a neighbourhood of $0$ with this topology, $\a x=A x$ and hence there exists $a\in \a$ such that $x=ax$.
\end{proof}
\begin{corollary}[\textbf{Krull's Intersection Theorem}]\label{Noe ring Krull intersection thm}
Let $A$ be a Noetherian ring and $\a$ an ideal of $A$. Then the ideal $\bigcap_n\a^n$ is the set of elements $x\in A$ for which there exists $a\in\a$ such that $(1-a)x=0$. In particular, for $\bigcap_n\a^n=\{0\}$, it is necessary and sufficient that no element of $1+\a$ is a divisor of $0$ in $A$.
\end{corollary}
\begin{remark}
The hypothesis that $A$ is Noetherian is essential in this corollary. For example, let $A$ be the ring of infinitely differentiable maps from $\R$ to itself and let $\m$ be the (maximal) ideal of $A$ consisting of the functions $f$ such that $f(0)=0$. It is immediate that $\bigcap_{n}\m^n$ is the set of functions $f$ such that $f^{(n)}(0)=0$ for all $n\geq 0$ and there exist such functions with $f(x)\neq 0$ for all $x\neq 0$, for example the function $f$ defined by $f(x)=e^{-1/x^{2}}$ for $x\neq 0$ and $f(0)=0$. Also, note that for such an $f$, the elment $1+f$ is even invertible in $A$.
\end{remark}

\subsection{The adic completion of a Noetherian ring}
Let $A$ be a ring, $\a$ an ideal of $A$ and $M$ an $A$-module; let $\widehat{A}$ and $\widehat{M}$ denote the respective Hausdorff completions of $A$ and $M$ with respect to the $\a$-adic topology and $\iota_M:M\to\widehat{M}$ the canonical map. The $A$-bilinear map $(a,x)\mapsto a\iota_M(x)$ of $\widehat{A}\times M$ to $\widehat{M}$ defines a canonical $A$-linear map
\[\alpha_M:\widehat{A}\otimes_AM\to\widehat{M}.\]
Let $\phi:M\to N$ be an $A$-module homomorphism and let $\hat{\phi}:\widehat{M}\to\widehat{N}$ be the map obtained by passing to the Hausdorff completions; for $a\in\widehat{A}$ and $x\in M$,
\[\alpha_N(a\otimes\phi(x))=a\iota_M(\phi(x))=a\hat{\phi}(\iota_M(x))=\hat{\phi}(\alpha_M(a\otimes x)).\]
In other words, the following diagram is commutative:
\[\begin{tikzcd}
\widehat{A}\otimes_AM\ar[d,"\alpha_M"]\ar[r,"1\otimes\phi"]&\widehat{A}\otimes N\ar[d,"\alpha_N"]\\
\widehat{M}\ar[r,"\hat{\phi}"]&\widehat{N}
\end{tikzcd}\]
Finally, recall that by \cref{filtration on completion is product with completion ring}, if $M$ is finitely generated then the homomorphism $\alpha_M$ is surjective.
\begin{theorem}\label{filtration Noe I-adic completion is tensor}
Let $A$ be a Noetherian ring, $\a$ an ideal of $A$ and $M$, $N$, $L$ be finitely generated $A$-modules. Then
\begin{itemize}
\item[(a)] If $0\to M\to N\to L\to 0$ is an exact sequence of $A$-modules, then the sequence
\[\begin{tikzcd}
0\ar[r]&\widehat{M}\ar[r]&\widehat{N}\ar[r]&\widehat{L}\ar[r]&0
\end{tikzcd}\]
obtained by passing to the Hausdorff completions (with respect to the $\a$-adic topologies) is exact.
\item[(b)] The canonical $\widehat{A}$-linear map $\alpha_M:\widehat{A}\otimes M\to\widehat{M}$ is bijective.
\item[(c)] The $A$-module $\widehat{A}$ is flat.  
\end{itemize}
\end{theorem}
\begin{proof}
We have seen that $M$ and $L$ carry the submodule filtration and quotient filtration, respectively, so claim (a) follows from \cref{topological group exact sequence of strict morphism completion}. Assertion (b) is obvious when $M=A$ and the case where $M$ is a free finitely generated $A$-module can be immediately reduced to that. In the general case, $M$ admits a finite presentation
\[\begin{tikzcd}
0\ar[r]&N\ar[r]&A^n\ar[r]&M\ar[r]&0
\end{tikzcd}\]
so we derive a commutative diagram
\[\begin{tikzcd}
&\widehat{A}\otimes_A N\ar[r]\ar[d,"\alpha_{N}"]&\widehat{A}\otimes_A A^n\ar[r]\ar[d,"\alpha_{A^n}"]&\widehat{A}\otimes_AM\ar[r]\ar[d,"\alpha_{M}"]&0\\
0\ar[r]&\widehat{N}\ar[r]&\widehat{A}^n\ar[r]&\widehat{M}\ar[r]&0
\end{tikzcd}\]
The first row is exact and so is the second by (a). Since $A$ is Noetherian we know that $N$ is also finitely generated, so $\alpha_M$ and $\alpha_N$ are both surjective. Since $\alpha_{A^n}$ is bijective, we then deduce that $\alpha_M$ is injective by the snake lemma.\par
Then it follows from (a) and (b) that, if $\a$ is an ideal of $A$ (necessarily finitely generated), the canonical map $\widehat{A}\otimes_A\a\to\widehat{A}$ is injective, being the composition of $\hat{\a}\to\widehat{A}$ and $\alpha_{\a}$, which proves that $A$ is a flat $A$-module.
\end{proof}
Under the conditions of Theorem, $\widehat{A}\otimes_AM$ is often identified with $M$ by means of the canonical map $\alpha_M$. If $\phi:M\to N$ is a homomorphism of finitely generated $A$-modules, $\hat{\phi}:\widehat{M}\to\widehat{N}$ is then identified with $1\otimes\phi$ by virtue of the commutativity of diagram.
\begin{corollary}\label{filtration completion not zero divisor}
Let $A$ be a Noetherian ring, $\a$ an ideal of $A$ and $\widehat{A}$ the Hausdorff completion of $A$ with respect to the $\a$-adic topology. If an element $a\in A$ is not a divisor of zero in $A$, its canonical image $\hat{a}$ in $A$ is not a divisor of zero in $\widehat{A}$.
\end{corollary}
\begin{proof}
This follows from the fact that $\widehat{A}$ is a flat $A$-module.
\end{proof}
\begin{corollary}\label{filtration I-adic completion Noe ring compatible with operation}
Let $A$ be a Noetherian ring, $\a$ an ideal of $A$, $M$ a finitely generated $A$-module and $N$ and $P$ two submodules of $M$. With the $\a$-adic topologies and let $\iota_M$ be the canonical map from $M$ to $\widehat{M}$, then
\[\widehat{N}=\widehat{A}\iota_M(N),\quad \widehat{(N+P)}=\widehat{N}+\widehat{P},\quad \widehat{N\cap P}=\widehat{N}\cap\widehat{P},\quad \widehat{(N:P)}=(\widehat{N}:\widehat{P}).\]
Moreover, if $\a$ and $\b$ are two ideals of $A$, then $\widehat{\a\b}=\hat{\a}\hat{\b}$.
\end{corollary}
\begin{proof}
By \cref{filtration Noe I-adic completion is tensor}, $\widehat{M}$, $\widehat{N}$, $\widehat{P}$ are canonically identified with $\widehat{A}\otimes_AM$, $\widehat{A}\otimes_AN$, and $\widehat{A}\otimes_AP$, which establishes the formulas. Finally, as $\hat{\a}=\widehat{A}\a$, $\hat{\b}=\widehat{A}\b$, we see that
\[\widehat{\a\b}=\widehat{A}\a\b=\widehat{A}\a\widehat{A}\b=\hat{\a}\hat{\b},\]
and this finishes the proof.
\end{proof}
\begin{corollary}
Let $A$ be a Noetherian ring, $\a$ an ideal of $A$ and $\widehat{A}$ the Hausdorff completion of $A$ with respect to the $\a$-adic topology. Then the topology on $\widehat{A}$ is the $\hat{\a}$-adic topology.
\end{corollary}
\begin{proof}
By \cref{filtration I-adic completion Noe ring compatible with operation} we have $\widehat{(\a^n)}=(\hat{\a})^n$. Since $\widehat{(\a^n)}$ is a fundamental system of neighbourhoods in $\widehat{A}$ by \cref{uniformity by dense subset}, the claim follows.
\end{proof}
\begin{corollary}
If $A$ is a Noetherian ring, the ring of normal power series $A\llbracket X_1,\dots,X_n\rrbracket$ is a flat $A$-module.
\end{corollary}
\begin{proof}
The ring $A\llbracket X_1,\dots,X_n\rrbracket$ is the completion of $B=A[X_1,\dots,X_n]$ with respect to the $\m$-adic topology, where $\m$ is the set of polynomials with no constant terms. As $B$ is Noetherian, $A\llbracket X_1,\dots,X_n\rrbracket$ is a flat $B$-module by \cref{filtration Noe I-adic completion is tensor} and, as $B$ is free $A$-module, we deduce that $A\llbracket X_1,\dots,X_n\rrbracket$ is a flat $A$-module by \cref{module flat restriction is flat}.
\end{proof}
\begin{proposition}\label{filtration I-adic completion maximal ideal}
Let $A$ be a Noetherian ring, $\a$ an ideal of $A$, $\widehat{A}$ the Hausdorff completion of $A$ with respect to the $\a$-adic topology and $\iota$ the canonical map from $A$ to $\widehat{A}$. Then
\begin{itemize}
\item[(a)] The map $\m\mapsto\hat{\m}$ is a bijection of the set of maximal ideals of $A$ containing $\a$ onto the set of maximal ideals of $\widehat{A}$ and $\n\mapsto\iota^{-1}(\n)$ is the inverse bijection.
\item[(b)] Let $\m$ be a maximal ideal of $A$ containing $\a$. Then the homomorphism $\iota_\m:A_\m\to\widehat{A}_{\hat{\m}}$ derived from $\iota$ is an embedding with dense image if $A_\m$ is given the $\m A_\m$-adic topology and $\widehat{A}_{\hat{\m}}$ the $\hat{\m}\widehat{A}_{\hat{\m}}$-adic topology.
\end{itemize}
\end{proposition}
\begin{proof}
The assertion of (a) follows immediately from the fact that the canonical homomorphism $A/\a\to\widehat{A}/\hat{\a}$ derived from $\iota$ is bijective and the fact that every maximal ideal of $\widehat{A}$ contains $\hat{\a}$, since $\hat{\a}$ is contained in the Jacobson radical of $\widehat{A}$ (\cref{filtration complete Hausdorff ring invertibility}).\par
Finally let us prove (b). As $\m=\iota^{-1}(\hat{\m})$, $\iota(A-\m)\sub\widehat{A}-\hat{\m}$ and $\iota$ certainly defines a homomorphism $\iota_{\m}:A_\m\to\widehat{A}_{\hat{\m}}$. Let us show that $\iota_\m$ is injective; let $a\in A$ and $s\in A-\m$ be such that
\[\iota_\m(a/s)=\iota(a)/\iota(s)=0.\]
then there exists $t\in\widehat{A}-\hat{\m}$ such that $t\iota(a)=0$ and the annihilator of $\iota(a)$ in $A$ is therefore not contained in $\hat{\m}$. Now, if $\b$ is the annihilator of $a$ in $A$, then the annihilator of $\iota(a)$ in $A$ is $\hat{\b}$ (\cref{filtration completion not zero divisor}); hence $\b\nsubseteq\m$, which shows that $a/s=0$.\par
Moreover, there is a commutative diagram
\[\begin{tikzcd}
A/\m^k\ar[r]\ar[d,"\tau"]&A_\m/(\m A_\m)^k\ar[d,"\tau_\m"]\\
\widehat{A}/\hat{\m}^k\ar[r]&\widehat{A}_{\hat{\m}}/(\hat{\m}\widehat{A}_{\hat{\m}})^k
\end{tikzcd}\]
where $\tau$ and $\tau_\m$ are derived from $\iota$ and $\iota_\m$ respectively and the horizontal arrows are the canonical isomorphisms. As $\m^k$ is an open ideal of $A$ (since it contains $\a^k$), $\tau$ is bijective and hence so is $\tau_\m$. This shows first that $(\hat{\m}\widehat{A}_{\hat{\m}})^k=\iota_\m((\m A_\m)^k)$ and hence the topology on $A_\m$ is induced by that on $\widehat{A}_{\hat{\m}}$. Moreover, $\widehat{A}_{\hat{\m}}=A_\m+(\hat{\m}\widehat{A}_{\hat{\m}})^k$ for all $k>0$ and hence $A_\m$ is dense in $\widehat{A}_{\hat{\m}}$.
\end{proof}
\begin{corollary}\label{Noe semilocal ring completion is Noe semilocal}
Let $A$ be a Noetherian local (resp. semi-local) ring and $\r$ its Jacobson radical. Then $\widehat{A}$ is a Noetherian local (resp. semi-local) ring whose Jacobson radical is $\hat{\r}$.
\end{corollary}
\begin{proof}
As $\widehat{A}/\hat{\m}$ is isomorphic to $A/\m$, it is a Noetherian ring and $\hat{\m}=\iota(\m)$ is a finitely generated $A$-module and therefore $\widehat{A}$ is Noetherian. The rest part follows from \cref{filtration I-adic completion maximal ideal} and the third formula in \cref{filtration I-adic completion Noe ring compatible with operation}.
\end{proof}
\begin{corollary}\label{Noe ring completion commute with localization for maximal}
Let $A$ be a Noetherian ring and $\m$ an maximal ideal of $A$. Then the $\m$-adic completion $\widehat{A}$ of $A$ is a Noetherian local ring and is isomomorphic to the $\m A_\m$-adic completion $\widehat{A_\m}$ of $A_\m$.
\end{corollary}
\begin{proof}
Taking $\a$ to be the maximal ideal $\m$ in \cref{filtration I-adic completion maximal ideal}(a), we see the $\m$-adic completion $\widehat{A}$ is Noetherian local with maximal ideal $\hat{\m}$. Then by \cref{filtration I-adic completion maximal ideal}(b), since $\widehat{A}_{\hat{\m}}$ is identified with $\widehat{A}$, we have an dense embedding $\iota_\m:A_\m\to\widehat{A}$, where $A_\m$ is given the $\m A_\m$-adic topology and $\widehat{A}$ the $\hat{\m}$-adic topology. Since $\widehat{A}$ is complete, this proves $\widehat{A}\cong\widehat{A_\m}$.  
\end{proof}
\subsection{Zariski rings}
For a topological ring $A$, if the given topology on $A$ is the $\a$-adic topology for an ideal $\a$ of $A$, then $\a$ is called a \textbf{defining ideal} of the topology on $A$.
\begin{remark}\label{Noe ring defining ideal prop}
Let $A$ be a Noetherian ring and $\a$ be an ideal of $A$. Note that if $\b$ is a defining ideal of the $\a$-adic topology on $A$, then there exists an integer $n>0$ such that $\b^n\sub\a$ and hence $\b\sub\sqrt{\a}$. Conversely, since $A$ is Noetherian, there exists an integer $k>0$ such that $(\sqrt{\a})^k\sub\a$ (\cref{Noe ideal contain power of radical}) and hence $\sqrt{\a}$ is the largest defining ideal of the $\a$-adic topology.
\end{remark}
\begin{proposition}\label{Zariski ring def}
Let $A$ be a Noetherian ring and $\a$ an ideal of $A$. The following properties are equivalent:
\begin{itemize}
\item[(\rmnum{1})] $\a$ is contained in the Jacobson radical of $A$.
\item[(\rmnum{2})] Every finitely generated $A$-module is Hausdorff with the $\a$-adic topology.
\item[(\rmnum{3})] For every finitely generated $A$-module $M$, every submodule of $M$ is closed with respect to the $\a$-adic topology on $M$.
\item[(\rmnum{4})] Every maximal ideal of $A$ is closed with respect to the $\a$-adic topology.
\item[(\rmnum{5})] The Hausdorff completion $\widehat{A}$ is a faithfully flat $A$-module.
\end{itemize}
\end{proposition}
\begin{proof}
Let us show that (\rmnum{1}) implies (\rmnum{2}). Suppose that $\a$ is contained in the Jacobson radical of $A$ and let $M$ be a finitely generated $A$-module. If $x\in M$ and $a\in \a$ are such that $(1-a)x=0$, then $x=0$, for $1-a$ is invertible in $A$. Then $M$ is Hausdorff with the $\a$-adic topology.\par
Next we show that (\rmnum{2}) implies (\rmnum{3}). Suppose (\rmnum{2}) holds. Let $M$ be a finitely generated $A$-module and $N$ a submodule of $M$. Then $M/N$ is Hausdorff with the $\a$-adic topology, which is the quotient topology of the $\a$-adic topology on $M$; thus $N$ is closed in $M$.\par
Clearly (\rmnum{3}) implies (\rmnum{4}). Now assume (\rmnum{4}), then for every maximal ideal $\m$ of $A$, the $A$-module $A/\m$ is Hausdorff with the $\a$-adic topology. This implies $\a(A/\m)\neq A/\m$, unless the $\a$-adic topology on $A/\m$ were the trivial topology and $A/\m$ were reduced to $0$, which is absurd since $A/\m$ is a field. The canonical image of $\a$ in $A/\m$ is therefore an ideal of $A/\m$ distinct from $A/\m$ and hence reduced to $0$; then $\a\sub\m$, which proves that $\a$ is contained in the Jacobson radical of $A$.\par
Finally, for every finitely generated $A$-module $M$, the canonical map $M\mapsto M\otimes_A\widehat{A}$ is identified with the canonical map $M\mapsto\widehat{M}$ from $M$ to its Hausdorff completion with respect to the $\a$-adic topology (by \cref{filtration Noe I-adic completion is tensor}) and the kernel of this map is then the closure of $\{0\}$ in $M$ with respect to this topology. As we already know that $\widehat{A}$ is a flat $A$-module, the equivalence of (\rmnum{5}) and (\rmnum{2}) follows from the characterization of faithfully flat modules (\cref{module faithfully flat iff}). This finishes the proof.
\end{proof}
A topological ring $A$ is called a \textbf{Zariski ring} if it is Noetherian and there exists a defining ideal $\a$ for the topology on $A$ satisfiyng the equivalent conditions of \cref{Zariski ring def}. A Zariski ring $A$ is necessarily Hausdorff (\cref{Zariski ring def}) and every defining ideal of its topology is contained in the Jacobson radical of $A$.
\begin{example}[\textbf{Examples of Zariski rings}]\label{Zariski ring eg}
\mbox{}
\begin{itemize}
\item[(a)] Let $A$ be a Noetherian ring and $\a$ an ideal of $A$. If $A$ is Hausdorff and complete with the $\a$-adic topology, then $A$ is a Zariski ring with this topology. In particular, $\widehat{A}$ is a Zariski ring.
\item[(b)] Every quotient ring $A/\b$ of a Zariski ring $A$ is a Zariski ring, for it is Noetherian and, if $\a$ is a defining ideal of $A$, then $\a(A/\b)=(\a+\b)/\b$ is contained in the Jacobson radical of $A/\b$.
\item[(c)] Let $A$ be a Noetherian semi-local ring and $\t$ its Jacobson radical. Then $A$ with the $\t$-adic topology is a Zariski ring. This will always be the topology in question (unless otherwise stated) when we consider a Noetherian semi-local ring as a topological ring. 
\end{itemize}
\end{example}
If $A$ is a Zariski ring and $M$ is a finitely generated $A$-module, then since $M$ is Hausdorff, we may (by virtue of \cref{Zariski ring def}) identify $M$ with a subset of $\widehat{M}$ by means of the canonical map $\iota_M$. With this identification, by \cref{Zariski ring def} we have
\begin{corollary}\label{Zariski ring completion of submodule is intersection}
Let $A$ be a Zariski ring, $M$ a finitely generated $A$-module and $N$ a submodule of $M$. Then $N=\widehat{N}\cap M$.
\end{corollary}
\begin{corollary}\label{Zariski ring finite module completion free}
Let $A$ be a Zariski ring and $M$ a finitely generated $A$-madule. If $\widehat{M}$ is a free $\widehat{A}$-module, then $M$ is a free $A$-madule.
\end{corollary}
\begin{proof}
Let $\a$ be a defining ideal of $A$, which is therefore contained in the Jacobson radical of $A$. We apply the criterion of \cref{local ring module M/mM is free then M free}: the canonical map $\iota_M:M\to\widehat{M}$ defines a bijection $i_M:M/\a M\to\widehat{M}/\widehat{\a M}$. Similarly the canonical map $\iota_M:A\to\widehat{A}$ defines a bijection $i_A:A/\a\to \widehat{A}/\hat{\a}$, which is a ring isomorphism. Then $\widehat{M}/\widehat{\a M}$ is given an $(\widehat{A}/\hat{\a})$-module structure and hence (by means of $i_A$) an $(A/\a)$-module structure. It is immediate that $i_M$ is $(A/\a)$-linear, so that it is an $(A/\a)$-module isomorphism. As $\widehat{M}/\widehat{\a M}$ is a free $(\widehat{A}/\hat{\a})$-module, $M/\a M$ is a free $(A/\a)$-module.\par
On the other hand, let $\eta:\a\otimes_AM\to M$ be the canonical homomorphism; as $(\a\otimes_AM)\otimes_A\widehat{A}$ is canonically identified with $\hat{\a}\otimes_A\widehat{M}$ and $M\otimes_A\widehat{A}$ with $\widehat{M}$, the hypothesis that $M$ is a free $A$-module implies that the homomorphism $\eta\otimes 1:\hat{\a}\otimes_A\widehat{M}\to\widehat{M}$ is injective. As $\widehat{A}$ is a faithfully flat $A$-module, we conclude that $\eta$ is injective and the conditions for applying the above mentioned criterion are indeed fulfilled.
\end{proof}
\begin{corollary}\label{Zariski ring completion principal then prncipal}
Let $A$ be a Zariski ring such that $A$ is an integral domain and let $\a$ be an ideal of $A$. If the ideal $\hat{\a}$ of $A$ is principal, then $\a$ is principal.
\end{corollary}
\begin{corollary}\label{Zariski ring completion fraction field prop}
Let $A$ be a Zariski ring such that $\widehat{A}$ is an integral domain, $L$ the field of fractions of $\widehat{A}$ and $K\sub L$ the field of fractions of $A$. Then $\widehat{A}\cap K=A$.
\end{corollary}
\begin{proof}
Clearly $A\sub\widehat{A}\cap K$; on the other hand, if $x\in\widehat{A}\cap K$, then $\widehat{A}x\sub\widehat{A}$ and hence, as $\widehat{A}x=\widehat{A}\otimes_AAx$, we have
\[\widehat{A}\otimes_A((Ax+A)/A)=(\widehat{A}\otimes_AAx)/(\widehat{A}\otimes_AA)=0.\]
As $\widehat{A}$ is a faithfully flat $A$-module, we deduce that $Ax\sub A$, whence $x\in A$.
\end{proof}
\begin{corollary}\label{filtration ring homomorphism prop iff completion prop}
Let $A$ be a Noetherian ring, $M$, $N$ be finitely generated $A$-modules and $\phi:M\to N$ a homomorphism. For every maximal ideal $\m$ of $A$, let $A(\m)$ (resp. $M(\m)$, $N(\m)$) denote the Hausdorff completion of $A$ (resp. $M$, $N$) with respect to the $\m$-adic topology and $\phi(\m):M(\m)\to N(\m)$ the corresponding homomorphism to $\phi$. For $\phi$ to be injective (resp. surjective, bijective, zero), it is necessary and sufiicient that $\phi(\m)$ be so for every maximal ideal $\a$ of $A$.
\end{corollary}
\begin{proof}
We know that for $\phi$ to be injective (resp. surjective, bijective, zero), it is necessary and sufficient that $\phi_\m:M_\m\to N_\m$ be so for every maximal ideal $\m$ of $A$. We now note that $A_\m$ is a Noetherian local ring and hence a Zariski ring and there is a canonical $A$-algebra isomorphism $(A_\m)(\m A_\m)\to A(\m)$ (\cref{filtration completion of semilocal ring with Jacobson radical}). On the other hand, there is a commutative diagram
\[\begin{tikzcd}
M_\m\otimes_{A_\m}A(\m)\ar[d,"\phi_\m\otimes 1"]\ar[r]&M\otimes_AA(\m)\ar[d,"\phi\otimes 1"]\ar[r]&M(\m)\ar[d,"\phi(\m)"]\\
N_\m\otimes_{A_\m}A(\m)\ar[r]&N\otimes_AA(\m)\ar[r]&N(\m)
\end{tikzcd}\]
where the horizontal arrows on the left arise from the associativity of the tensor product and the isomorphisms $M_\m\to M\otimes_AA_\m$, $N_\m\to N\otimes_AA_\m$. As $M$ and $N$ are finitely generated A-modules, it follows from \cref{filtration Noe I-adic completion is tensor} that the rows of this diagram consist of isomorphisms; thus we are reduced to proving that $\phi_\m$ being injective (resp. surjective, bijective, zero) is equivalent to $\phi_\m\otimes 1$ being so. But this follows from the fact that $(A_\m)(\m A_\m)$ (and hence also $A(\m)$) is a faithfully flat $A_\m$-module.
\end{proof}
\begin{proposition}\label{Zariski ring extension of scalar}
Let $A$, $B$ be two rings and $\rho:A\to B$ be a ring homomorphism. Suppose that $A$ is Noetherian and $B$ is a finitely generated $A$-module (with the structure defined by $f$). Let $\a$ be an ideal of $A$, then
\begin{itemize}
\item[(a)] For the $\a^e$-adic topology on $B$ to be Hausdorff, it is necessary and suficient that the elements of $1+\rho(\a)$ be not divisors of $0$ in $B$.
\item[(b)] If $A$ with the $\a$-adic topology is a Zariski ring, then $B$ with the $\a^e$-adic topology is a Zariski ring.
\item[(c)] If $\rho$ is injective (thus identifying $A$ with a subring of $B$), the $\a^e$-adic topology on $B$ induces on $A$ the $\a$-adic topology. 
\end{itemize}
\end{proposition}
\begin{proof}
Recall that the $\a^e$-adic filtration on $B$ coincides with the $\a$-adic filtration on the $A$-module $B$. Assertion (a) is thus a special case of \cref{Zariski ring def} and assertion (c) a special case of Artin-Rees Lemma. Finally let us show (b). Suppose that $A$ is a Zariski ring with the $\a$-adic topology and let $N$ be a finitely generated $B$-module; it is also a finitely generated $A$-module and the $\a$-adic and $\a^e$-adic filtrations on $N$ coincide; then $N$ is Hausdorff with the $\a^e$-adic topology. Finally the $A$-module $B$ is Noetherian and hence the ring $B$ is Noetherian, which completes the proof that $B$ is a Zariski ring.
\end{proof}
\begin{proposition}\label{Zariski ring hat f bijective}
Let $A$, $B$ be two Zariski rings, $\widehat{A}$, $\widehat{B}$ their completions, $\rho:A\to B$ a continuous ring homomorphism and $\hat{\rho}:\widehat{A}\to\widehat{B}$ the homomorphism obtained from $f$ by passing to the completions. If $\hat{\rho}$ is bijective, then the $A$-module $B$ is faithfully flat.
\end{proposition}
\begin{proof}
As $A$ and $B$ are Hausdorff, the hypothesis that $\hat{\rho}$ is bijective implies first that $\rho$ is injective. Identifying (algebraically) $A$ with $\rho(A)$ by means of $\rho$ and $\widehat{A}$ with $\widehat{B}$ by means of $\hat{\rho}$, we then obtain the inclusions $A\sub B\sub\widehat{A}=\widehat{B}$. Then we know that $\widehat{A}$ is a faithfully flat $A$-module and a faithfully flat $B$-module, hence $B$ is a faithfully flat $A$-module.
\end{proof}
\begin{proposition}\label{filtration Noe local ring contained in completion}
Let $A$ be a Noetherian local ring, $\m$ its maximal ideal, $A$ its $\m$-adic completion and $B$ a ring such that $A\sub B\sub\widehat{A}$. Suppose that $B$ is a Noetherian local ring whose maximal ideal $\n$ satisfies the relation $\n=\m B$. Then for $k\geq 0$,
\[\n^k=\m^kB=\hat{\m}^k\cap B,\]
the $\n$-adic topology on $B$ is induced by the $\hat{\m}$-adic topology on $\widehat{A}$, $B$ is a faithfully flat $A$-module and there is an isomorphism of $\widehat{A}$ onto the $\n$-adic completion $\widehat{B}$ of $B$, which extends the canonical injection $A\to B$.
\end{proposition}
\begin{proof}
It is sufficient to verify the relation $\n^k=\hat{\m}^k\cap B$, for, as $B$ is dense in $A$ and the $\n$-adic topology is induced by the $\hat{\m}$-adic topology, the last assertion will follow from \cref{uniform space completion of subspace} and the last but one from \cref{Zariski ring hat f bijective}. The injections $A\to B\to\widehat{A}=\widehat{B}$ induce injective homomorphisms
\[\begin{tikzcd}
A/(\hat{\m}\cap A)\ar[r]&B/(\hat{\m}\cap B)\ar[r]&\widehat{A}/\hat{\m}
\end{tikzcd}\]
We know that $\hat{\m}\cap A=\m$ and that $A/\m\cong\widehat{A}/\hat{\m}$, hence $B/(\hat{\m}\cap B)\cong\widehat{A}/\hat{\m}$, which shows that $B/(\hat{\m}\cap B)$ is a field, hence that $\hat{\m}\cap B$ is a maximal ideal of $B$ and therefore $\hat{\m}\cap B=\n$. As $A/\m\cong\widehat{A}/\hat{\m}$ we have $\widehat{A}=A+\hat{\m}$, thus $B=A+\n=A+\m B$. By induction on $k$ we deduce that
\[B=A+\m^kB=A+\n^k\]
for all $k>1$. As $\n^k\sub\hat{\m}^k+B$, it is sufficient to show that $\hat{\m}^k\cap B\sub\n^k$. If $b\in\hat{\m}^k\cap B$, we may write $b=a+z$ where $a\in A$, $z\in\n^k$; whence
\[a=b-z\in\hat{\m}^k\cap A=\m^k\sub\n^k\]
and $b\in\n^k$.
\end{proof}
\begin{example}
An important case where this applies is the following: $B$ is the ring of integral series in $n$ variables over a complete valued field $K$, which converge in the neighbourhood of $0$, $A$ is the local ring $K[X_1,\dots,X_n]_\m$ where $\m$ is the maximal ideal consisting of the polynomials with no constant term and $\widehat{A}$ is the ring of formal power series $K\llbracket X_1,\dots,X_n\rrbracket$.
\end{example}
\begin{proposition}\label{Zariski ring and localization}
Let $A$ be a Noetherian ring, $\a$ an ideal of $A$, $S$ the multiplicative subset $1+\a$ of $A$ and $M$ a finitely generated $A$-module.
\begin{itemize}
\item[(a)] $S^{-1}A$ is a Zariski ring with the $(S^{-1}\a)$-adic topology.
\item[(b)] The canonical map $i:M\to S^{-1}M$ is continuous if $M$ is given the $\a$-adic topology and $S^{-1}M$ the $(S^{-1}\a)$-adic topology and $\hat{i}:\widehat{M}\to\widehat{S^{-1}M}$ is an isomorphism. 
\end{itemize}
\end{proposition}
\begin{proof}
Every element of $1+S^{-1}\a$ is of the form
\[1+(a/(1+b))=(1+a+b)/(1+b)\]
where $a,b\in \a$; it is therefore invertible in $S^{-1}A$, which proves that $S^{-1}\a$ is contained in the Jacobson radical of $S^{-1}A$. As $S^{-1}A$ is Noetherian, it is then a Zariski ring with the $(S^{-1}\a)$-adic topology, which proves (a). Let us show (b). For all $n>0$, we have
\[i^{-1}((S^{-1}\a)^nM)=i^{-1}(S^{-1}\a^nM)=\a^nM\]
since by \cref{localization and ideals}, if $x\in i^{-1}(S^{-1}\a^nM)$ then $(1-a)x=y$, where $a\in \a$, $y\in \a^nM$, whence
\[x=(1+a+a^2+\cdots+a^{n-1})y+a^nx\in \a^nM.\]
This proves that $i$ is continuous and a strict morphism (\cref{localization and ideals}). Moreover, the kernel of $i$, which is the set of $x\in M$ for which there exists some $s\in S$ such that $sx=0$, is identical with the kernel of the canonical map $\iota:M\to\widehat{M}$. Then there exists a topological isomorphism $\iota_0:i(M)\to\iota(M)$ such that the following diagram is commutative:
\[\begin{tikzcd}[row sep=small]
&\iota(M)\ar[dd,"\iota_0"]\ar[r,hook]&\widehat{M}\ar[dd,"\hat{i}"]\\
M\ar[ru,"\iota"]\ar[rd,swap,"i"]&&\\
&i(M)\ar[r,hook]&\widehat{S^{-1}M}
\end{tikzcd}\]
Therefore the problem reduces to verify that $i(M)$ is dense in $S^{-1}M$. Now every element of $S^{-1}M$ may be written as $x/(1-a)$, where $a\in \a$, and it is immediately verified that
\[x/(1-a)\equiv (1+a+\cdots+a^{n-1})x\mod S^{-1}\a^nM
\]
which completes the proof.
\end{proof}
\section{Lifting in complete rings}
\subsection{Strongly relatively prime polynomials}
Let $A$ be a ring. Two elements $x$, $y$ of $A$ are called \textbf{strongly relatively prime} if the principal ideals $(x)$ and $(y)$ are relatively prime, in other words if $Ax+Ay=A$; it amounts to the same to say that there exist two elements $a$, $b$ of $A$ such that $ax+by=1$.
\begin{lemma}[\textbf{Euclid Lemma}]\label{polynomial ring strongly relative prime Euclid lemma}
Let $x$, $y$ be two strongly relatively prime elements of $A$; if $z\in A$ is such that $x$ divides $yz$, then $x$ divides $z$.
\end{lemma}
\begin{proof}
If $1=ax+by$, then $z=x(az)+(yz)b$, so $x$ divides $z$.
\end{proof}
If $x$ and $y$ are strongly relatively prime in $A$, then $(xy)=(x)\cap(y)$. If $A$ is an integral domain, two strongly relatively prime elements then have an lcm equal to their product and are therefore relatively prime in the sense that $\gcd(x,y)=1$. Conversely, if $A$ is a principal ideal domain, two relatively prime elements are also strongly relatively prime, as follows from Bezout's identity.\par
For polynomial rings there is the following result:
\begin{proposition}\label{polynomial ring strongly relative prime prop}
Let $A$ be a ring and $F$, $G$ two strongly relatively prime polynomials in $A[X]$. Suppose that $F$ is monic and of degree $s$. Then every polynomial $T$ in $A[X]$ may be written uniquely in the form
\begin{align}\label{polynomial ring strongly relative prime prop-1}
T=PF+QG
\end{align}
where $P,Q\in A[X]$ and $\deg(Q)<s$. If further $\deg(T)\leq n$ and $\deg(G)\leq n-s$, then $\deg(P)\leq n-s$.
\end{proposition}
\begin{proof}
As $F$ is monic, $FP\neq 0$ for every nonzero polynomial $P$ of $A[X]$ and in this case $\deg(FP)=s+\deg(P)$. Let $T$ be any polynomial in $A[X]$. As the ideal generated by $F$ and $G$ is the whole of $A[X]$, there exist polynomials $P_1$ and $Q_1$ such that
\[T=P_1F+Q_1G.\]
As $F$ is monic of degree $s$, Euclidean division shows that there exist two polynomials $Q'$ and $Q''$ such that $Q_1=Q''F+Q'$ where $\deg(Q')<s$. Then we deduce that
\[T=P_1F+Q_1G=P_1F+Q''FG+Q'G=PF+Q'G\]
where $P=P_1+Q''G$. To show the uniqueness of formula (\ref{polynomial ring strongly relative prime prop-1}), it is sufficient to prove that the relations
\[0=PF+QG,\quad\deg(Q)<s.\]
imply $P=Q=0$. Now, if this holds, $F$ divides $QG$ and, as $F$ and $G$ are strongly relatively prime, $F$ divides $Q$ by \cref{polynomial ring strongly relative prime Euclid lemma}, which is impossible unless $Q=0$. This then implies $PF=0$, and so $P=0$ by the remark at the begining.\par
Finally, suppose that $\deg(T)\leq n$ and $\deg(G)\leq n-s$. With the polynomial $T$ in the form (\ref{polynomial ring strongly relative prime prop-1}),
\[\deg(QG)\leq\deg(Q)+\deg(G)<s+\deg(G)\leq n\]
and therefore
\[s+\deg(P)=\deg(PF)=\deg(T-QG)\leq n\]
whence $\deg(P)\leq n-s$.
\end{proof}
\begin{example}
For a polynomial $P\in A[X]$ to be strongly relatively prime to $X-a$ (where $a\in A$), it is necessary and sufiicient that $P(a)$ be invertible in $A$. For if $P$ and $X-a$ are strongly relatively prime, it follows from \cref{polynomial ring strongly relative prime prop} that there exist $c\in A$ and a polynomial $Q\in A[X]$ such that $1=(X-a)Q+cP$, whence $cP(a)=1$ and $P(a)$ is invertible. Conversely, by Euclidean division
\[P=(X-a)G+P(a)\]
and, if $P(a)=b^{-1}$ where $b\in A$, we deduce that $1=bP-b(X-a)G$, which shows that $P$ and $X-a$ are strongly relatively prime.
\end{example}
Let $A$ and $B$ be two rings and $\rho:A\to B$ a ring homomorphism. If $P=\sum_{n=0}^{\infty}a_nX^n$ is a formal power series in $A\llbracket X\rrbracket$, let $f(P)$ denote the formal power series $\sum_{n=0}^{\infty}\rho(a_n)X^n$ in $B\llbracket X\rrbracket$. If $P$ is a polynomial, so is $\rho(P)$ and, if further $P$ is monic, then $\rho(P)$ is monic of the same degree as $P$. Finally, $P\mapsto f(P)$ is clearly a homomorphism of $A\llbracket X\rrbracket$ to $B\llbracket X\rrbracket$ which extends $f$ and maps $X$ to $X$.
\begin{proposition}\label{polynomial ring strongly relative prime under homomorphism}
Let $A$ and $B$ be two rings, $\rho:A\to B$ a homomorphism and $P$, $Q$ two polynomials in $A[X]$. If $P$ and $Q$ are strongly relatively prime in $A[X]$, then $\rho(P)$ and $\rho(Q)$ are strongly relatively prime in $B[X]$. The converse is true if $\rho$ is surjective, if its kernel is contained in the Jacobson radical of $A$ and if $P$ is monic.
\end{proposition}
\begin{proof}
Suppose that $P$ and $Q$ are strongly relatively prime; then there exist polynomials $U$, $V$ in $A[X]$ such that $PU+QV=1$; we deduce that
\[\rho(P)\rho(U)+\rho(Q)\rho(V)=1\]
whence the first assertion. To show the second, let $\a$ denote the kernel of $\rho$. Let $M=A[X]$ and $N$ be the ideal of $M$ generated by $P$ and $Q$. As $\rho$ is surjective and $\rho(P)$ is monic, \cref{polynomial ring strongly relative prime prop} shows that for every polynomial $T\in A[X]$ there exist two polynomials $U$, $V$ in $A[X]$ such that
\[\rho(T)=\rho(P)\rho(U)+\rho(Q)\rho(V)\]
whence the relation $M=N+\a M$. Now, $M/N$ is a finitely generated $A$-module, for every polynomial is congruent mod $P$ to a polynomial of degree strictly smaller than $\deg(P)$, $P$ being monic. As $\a(M/N)=M/N$ and $\a$ is contained in the Jacobson radical of $A$, Nakayama's Lemma shows that $M/N=0$, which means that $P$ and $Q$ are strongly
relatively prime.
\end{proof}
\subsection{Restricted power series and Hensel's lemma}
\begin{definition}
A commutative topological ring $A$ is said to be linearly topologized (and its topology is said to be \textbf{linear}) if there exists a fundamental system $\mathcal{B}$ of neighbourhoods of $0$ consisting of ideals of $A$.
\end{definition}
Note that in such a ring, the ideals $\a\in\mathcal{B}$ are open and closed. For all $\a\in\mathcal{B}$, the quotient topological ring $A/\a$ is then discrete. For $\a,\b\in\mathcal{B}$ and $\b\sub \a$, let
\[\pi_{\a\b}:A/\b\to A/\a\]
be the canonical map. We know that $(A/\a,\pi_{\a\b})$ is an inverse system of discrete rings (relative to the indexing set $\mathcal{B}$ which is ordered by inclusion and directed), whose inverse limit is a complete Hausdorff linearly topologized ring $\widetilde{A}$. Further, a strict morphism $\iota:A\to\widetilde{A}$ is defined, whose kernel is the closure of $\{0\}$ in $A$ and whose image is dense in $\widetilde{A}$, so that $\widetilde{A}$ is canonically identified with the Hausdorff completion of $A$.
\begin{definition}
Given a commutative topological ring $A$, a formal power series
\[P=\sum_{\alpha\in\N^p}c_\alpha X^\alpha\]
in the ring $A\llbracket X_1,\dots,X_p\rrbracket$ is called \textbf{restricted} if, for every neighbourhood $V$ of $0$ in $A$, there is only a finite number of coefiicients $c_\alpha$ not belonging to $V$ (in other words, the family $(c_\alpha)$ tends to $0$ in $A$ with respect to the filter of complements of finite subsets of $\N^p$.
\end{definition}
If $A$ is linearly topologized, the restricted formal power series in $A\llbracket X_1,\dots,X_p\rrbracket$ form a subring of $A\llbracket X_1,\dots,Xp\rrbracket$, denoted by $A\{X_1,\dots,X_p\}$, for if $P=\sum a_\alpha X^\alpha$ and $Q=\sum b_\alpha X^\alpha$ are two restricted formal power series and $\a$ a neighbourhood of $0$ in $A$ which is an ideal of $A$, there exists an integer $m$ such that $a_\alpha\in \a$ and $b_\alpha\in \a$ for every $\alpha\in\N^p$ such that $|\alpha|\geq m$. Now, if
\[PQ=\sum c_\alpha X^\alpha\quad\text{where}\quad c_\alpha=\sum_{\beta+\gamma=\alpha}a_\beta b_\gamma.\]
We conclude that if $|\alpha|\geq 2m$, then $|\beta|\geq m$ or $|\gamma|\geq m$ and hence, since $\a$ is an ideal, $c_\alpha\in \a$ so long as $|\alpha|\geq 2m$, which establishes our assertion. Moreover, every derivative $\partial(PQ)/\partial X_i$ ($1\leq i\leq p$) of a restricted formal power series is restricted, as follows immediately from the definition and the fact that the neighbourhoods $\a\in\mathcal{B}$ are additive subgroups of $A$.\par
Let us always assume that $A$ is linearly topologized and let $\mathcal{B}$ be a fundamental system of neighbourhoods of $0$ in $A$ consisting of ideals of $A$; for all $\a\in\mathcal{B}$, let $\pi_\a:A\to A/\a$ be the canonical homomorphism. By definition, for every restricted formal power series $T\in A\{X_1,\dots,X_n\}$,
\[\pi_\a(T)\in(A/\a)[X_1,\dots,X_p]\]
Clearly $((A/\a)[X_1,\dots,X_p],\pi_{\a\b})$ is an inverse system of rings and $(\pi_\a)$ is an inverse system of homomorphisms $A\{X_1,\dots,X_p\}\to(A/\a)[X_1,\dots,X_p]$; as every polynomial is a restricted formal power series, $\pi_\a$ is is surjective; its kernel $N_\a$ is the ideal of $A\{X_1,\dots,X_p\}$ consisting of the restricted formal power series all of whose coefficients belong to $\a$; we shall give $A\{X_1,\dots,X_p\}$ the (linear) topology for which the $N_\a$ (for $\a\in\mathcal{B}$) form a fundamental system of neighbourhoods of $0$ (a topology which obviously depends only on that on $A$). Then it follows from \cref{topological group inverse limit of nbhd quotient and completion} that
\[\pi=\llim\pi_\a:A\{X_1,\dots,X_p\}\to\llim\nolimits_\a(A/\a)[X_1,\dots,X_p]\]
is a strict morphism whose kernel is the closure of $\{0\}$ in $A\{X_1,\dots,X_p\}$ and whose image is dense in $B=\llim\nolimits_\a(A/\a)[X_1,\dots,X_p]$.
\begin{proposition}\label{filtration restricted power series ring is limit}
If the linearly topologized ring $A$ is Hausdorff and complete, the canonical homomorphism $\pi$ is a topological ring isomorphism.
\end{proposition}
\begin{proof}
For all $\alpha\in\N^p$ and all $\a\in\mathcal{B}$, let $\phi_\alpha^\a$ be the map $(A/\a)[X_1,\dots,X_p]\to A/\a$ which maps every polynomial to the coefficient of $X^\alpha$ in this polynomial; clearly the $(\phi_\alpha^\a)_\a$ form an inverse system of $(A/\a)$-module homomorphisms (relative to the ordered set $\mathcal{B}$) and, as $A$ is canonically identified with $\llim\nolimits_\a(A/\a)$ by hypothesis $\phi_\alpha=\llim\nolimits_\a\phi^\a_\alpha$ is a continuous $A$-homomorphisms from $B$ to $A$. For every element $S=(S_\a)_{\a\in\mathcal{B}}$ of $B$, we consider the formal power series $T=\sum_\alpha\phi_\alpha(S)X^\alpha$. For each $\a\in\mathcal{B}$, since $S_\a$ is a polynomial, there exist $n>0$ such that $\phi_\alpha^\a(S_\a)=0$ for $|\alpha|\geq n$, which implies $\phi_\alpha^\b(S_\b)\in \a/\b$ for all $\b\in\mathcal{B}$, $\b\sub \a$. Taking limits for these $\b$'s, we then see $\phi_\alpha(S)\in \a$ for $|\alpha|\geq n$, which proves $T$ is restricted. Also, it is immediate that $\pi(T)=S$. As $A$ is Hausdorff, the intersection of the $N_\a$ reduces to $0$ and hence $\pi$ is bijective, which completes the proof, since $\pi$ is a strict morphism.
\end{proof}
\begin{proposition}
Let $A$, $B$ be two linearly topologized rings, $B$ being Hausdorff and complete, and $\rho:A\to B$ a continuous homomorphism. For every family $(b_i)_{1\leq i\leq p}$ of elements of $B$, there exists a unique continuous homomorphism
\[\tilde{\rho}:A\{X_1,\dots,X_p\}\to B\]
such that $\tilde{\rho}(a)=f(a)$ for all $a\in A$ and $\tilde{\rho}(X_i)=b_i$ for $1\leq i\leq p$.
\end{proposition}
\begin{proof}
There exists a unique homomorphism $v:A[X_1,\dots,X_p]\to B$ such that $v(a)=\rho(a)$ for $a\in A$ and $v(X_i)=b_i$, for $1\leq i\leq p$. Moreover, if $\b$ is a neighbourhood of $0$ in $B$ which is an ideal, $\rho^{-1}(\a)$ is an ideal of $A$ which is a neighbourhood of $0$ and, for every polynomial $P\in N_\a$, clearly $v(P)\in \b$ and hence $v$ is continuous. As $A[X_1,\dots,X_p]$ is dense in $A\{X_1,\dots,X_p\}$, the existence and uniqueness of $\tilde{\rho}$ follow from \cref{topological group complete Hausdorff into extension} and the principle of extension of identities.
\end{proof}

\begin{example}
In the special case where $A=B$ and $f$ is the identity map we shall write $P(b_1,\dots,b_p)$ for the value of $\tilde{f}(P)$ for every restricted formal power series $P\in A\{X_1,\dots,X_p\}$.
\end{example}

Now we turn to the most important application of restricted power series in this part. In a topological ring $A$, an element $x$ is called \textbf{topologically nilpotent} if $0$ is a limit of the sequence $(x^n)$. If $A$ is a linearly topologized ring, to say that $x$ is topologically nilpotent means that for every open ideal $\a$ of $A$ the canonical image of $x$ in $A/\a$ is a nilpotent element of that ring. If $\r_\a$ is the nilradical of $A/\a$, clearly $(\r_\a)$ is an inverse system of subsets and the set $\t$ of topological nilpotent elements of $A$ is the inverse image of $\llim\nolimits_\a \r_\a$ under the canonical homomorphism $A\to\llim A/\a$; it is therefore a closed ideal of $A$. If also $A$ is Hausdorff and complete, this ideal is contained in the Jacobson radical of $A$ and, for an element $x\in A$ to be invertible, it is necessary and sufficient that its class mod $\t$ be invertible in $A/\t$ (\cref{filtration complete Hausdorff ring invertibility}). Note that if $A$ is a ring and $\a$ an ideal of $A$, the elements of $\a$ are topologically nilpotent with respect to the $\a$-adic topology.

\begin{theorem}[\textbf{Hensel's Lemma}]
Let $A$ be a complete Hausdorff linearly topologized ring. Let $\m$ be a closed ideal of $A$ whose elements are topologically nilpotent. Let $A/\m$ be the quotient topological ring and $\pi:A\to A/\m$ the canonical map. Let $T$ be a restricted formal power series in $A\{X\}$, $\bar{P}$ a monic polynomial in $(A/\m)[X]$ and $\bar{Q}$ a restricted formal power series in $(A/\m)\{X\}$. Suppose that $\pi(T)=\bar{P}\bar{Q}$ and that $\bar{P}$ and $\bar{Q}$ are strongly relatively prime in $(A/\m)\{X\}$. Then there exists a unique ordered pair $(P,Q)$ consisting of a monic polynomial $P\in A[X]$ and a restricted formal power series $Q\in A\{X\}$ such that
\begin{align}\label{filtration Hensel's lemma-1}
T=PQ,\quad \pi(P)=\bar{P},\quad \pi(Q)=\bar{Q}.
\end{align}
Moreover, $P$ and $Q$ are strongly relatively prime in $A\{X\}$ and, if $T$ is a polynomial, so is $Q$.
\end{theorem}
\begin{proof}
The proof is divided into several steps. In the first three we assume that $A$ is discrete, in which case $T$ and $Q$ are \textit{polynomials}. First assume that $\m^2=0$. Let $P_1$, $Q_1$ be two polynomials of $A[X]$ such that $P_1$ is monic and $\pi(P_1)=\bar{P}$, $\pi(Q_1)=\bar{Q}$. \cref{polynomial ring strongly relative prime under homomorphism} shows that $P_1$ and $Q_1$ are strongly relatively prime; hence (\cref{polynomial ring strongly relative prime prop}) there exists a unique ordered pair of polynomials $(U,V)$ of $A[X]$ such that
\[T-P_1Q_1=P_1U+Q_1V,\quad \deg(V)<\deg(P_1)=\deg(\bar{P}).\]
Applying $\pi$ on this equation, we then see
\[\bar{P}\pi(U)+\bar{Q}\pi(V)=\pi(T)-\bar{P}\bar{Q}=0.\]
As $\bar{P}$ is monic, $\bar{P}$ and $\bar{Q}$ strongly relatively prime and $\deg(\pi(V))<\deg(\bar{P})$, the uniqueness part of \cref{polynomial ring strongly relative prime prop} implies that $\pi(U)=\pi(V)=0$, in other words the coefficients of $P_1$ and $Q_1$ belong to $\m$ and the relation $\m^2=0$ gives
\[T=P_1Q_1+P_1U+Q_1V=P_1Q_1+P_1U+Q_1V+UV=(P_1+V)(Q_1+U)\]
so the polynomials $P=P_1+V$ and $Q=Q_1+U$ are the solutions to the problem.\par
Now suppose that $\m$ is nilpotent and let $n$ be the smallest integer such that $\m^n=0$ and let us argue by induction on $n>2$, the theorem having been shown for $n=2$. Let
\[A'=A/\m^{n-1},\quad \m'=\m/\m^{n-1}\]
As $(\m')^{n-1}=0$, by induction hypothesis there exists a unique ordered pair $(P',Q')$ of polynomials in $A'[X]$ such that $P'$ is monic and
\[T'=P'Q',\quad\pi'(P')=\bar{P},\quad \pi'(Q')=\bar{Q},\]
where $\pi':A'\to A'/\m'=A/\m$ and $\theta:A\to A'$ denote the canonical homomorphisms, and $T'=\theta(T)$. On the other hand, as $(\m')^2=0$, there exists a unique ordered pair $(P,Q)$ of polynomials in $A[X]$ such that $P$ is monic and
\[T=PQ,\quad\theta(P)=P',\quad\theta(Q)=Q'.\]
As $\pi=\pi'\circ\theta$, this shows the existence and uniqueness of $P$ and $Q$ satisfying (\ref{filtration Hensel's lemma-1}). Moreover $P'$ and $Q'$ are strongly relatively prime by the induction hypothesis and hence so are $P$ and $Q$.\par
Next we turn to the case where $A$ is discrete. In this case $\m$ is no longer necessarily nilpotent, but it is always a nilideal (that is, every element in $\m$ is nilpotent), since $\m$ is topological nilpotent and $\{0\}$ is open in $A$. Let $P_0$, $Q_0$ be two polynomials of $A[X]$ such that $(P_0)=\bar{P}$, $\pi(Q_0)=\bar{Q}$ and $P_0$ is monic. Let us consider the ideal $\n$ of $A$ generated by the coefficients of $T-P_0Q_0$; it is finitely generated and contained in $\m$, hence it is nilpotent and by definition, if $\psi:A\to A/\n$ is the canonical map, then $\psi(T)=\psi(P_0)\psi(Q_0)$. Moreover, $\psi(P_0)$ and $\psi(Q_0)$ are strongly relatively prime, as follows from the hypothesis on $P$ and $Q$ and \cref{polynomial ring strongly relative prime under homomorphism} applied to the canonical homomorphism $A/\n\to A/\m$. By virtue of the nilpotent case, there therefore exists an ordered pair $(P,Q)$ of polynomials in $A[X]$ such that $P$ is monic and relations (\ref{filtration Hensel's lemma-1}) hold. The fact that $P$ and $Q$ are strongly relatively prime implies also here that $P$ and $Q$ are strongly relatively prime in $A[X]$ by virtue of \cref{polynomial ring strongly relative prime under homomorphism}, for $\m$ is contained in the Jacobson radical of $A$. Suppose finally that $P_1$, $Q_1$ are two polynomials in $A[X]$ satisfying (\ref{filtration Hensel's lemma-1}) and such that $P_1$ is monic and let $\n_1$ be the finitely generated ideal of $A$ generated by the coefficients of $P-P_1$ and the coefficients of $Q-Q_1$; as $\n_1$ is contained in $\m$, it is nilpotent and, if $\psi_1:A\to A/\n_1$ is the canonical map, then $\psi_1(P)=\psi_1(P_1)$ and $\psi_1(Q)=\psi_1(Q_1)$. The uniqueness property for nilpotent case therefore implies $P=P_1$ and $Q=Q_1$.\par
Finally, we deal with the general case. Let $\mathcal{B}$ be a fundamental system of neighbourhoods of $0$ in $A$ consisting of ideals of $A$. For all $\a\in\mathcal{B}$, let $\pi_\a$ be the canonical map $A\to A/\a$, $\phi_\a$ the canonical map
\[A/\a\to(A/\a)/((\m+\a)/\a)=A/(\m+\a),\]
and $\psi_\a$ the canonical map $A/\m\to A/(\m+\a)$ and write $T_\a=\pi_\a(T)$, $\bar{P}_\a=\psi_\a(\bar{P})$, $\bar{Q}_\a=\psi_\a(\bar{Q})$. As each ring $A/\a$ is discrete, the preceeding argument can be applied to it and we see that there exists a unique ordered pair $(P_\a,Q_\a)$ of polynomials in $(A/\a)[X]$ such that $P_\a$ is monic and $T_\a=P_\a Q_\a$, $\phi_\a(P_\a)=\bar{P}_\a$, $\phi_\a(Q_\a)=\bar{Q}_\a$. The uniqueness of this ordered pair implies that, if $\a,\b\in\mathcal{B}$, $\b\sub \a$ and $\pi_{\a\b}:A/\b\to A/\a$ is the canonical map, then $P_\a=\pi_{\a\b}(P_{\b})$ and $Q_\a=\pi_{\a\b}(Q_{\b})$. Then it follows from the canonical identification of $A\{X\}$ with $\llim\nolimits_\a(A/\a)[X]$ that there exists $P\in A\{X\}$ and $Q\in A\{X\}$ such that $T=PQ$ and $\pi_\a(P)=P_\a$,$\pi_\a(Q)=Q_\a$ for all $\a\in\mathcal{B}$. Moreover
\[\psi_\a(\bar{P}-\pi(P))=\psi_\a(\bar{Q}-\pi(Q))=0,\]
for all $\a\in\mathcal{B}$, which means that for all $\a\in\mathcal{B}$ the coefficients of $\bar{P}-\pi(P)$ and $\bar{Q}-\pi(Q)$ all belong to $(\m+\a)/\m$. But, as $\m$ is closed in $A$, $\bigcap_\a(\m+\a)=\m$, whence $\bar{P}=\pi(P)$, $\bar{Q}=\pi(Q)$ and $P$ and $Q$ certainly satisfy (\ref{filtration Hensel's lemma-1}). Moreover, as the $P_\a$ are monic and of the same degree, the restricted formal power series $P$ is a monic polynomial. If $(P',Q')$ were another ordered pair satisfying (\ref{filtration Hensel's lemma-1}) and such that $P'$ is a monic polynomial, we would deduce that 
\[T_\a=\pi_\a(P')\pi_\a(Q'),\quad \phi_\a(\pi_\a(P'))=\bar{P}_\a,\quad \phi_\a(\pi_\a(Q'))=\bar{Q}_\a.\]
By the uniqueness in the discrete case we have $\pi_\a(P')=P_\a$ and $\pi_\a(Q')=Q_\a$ for all $\a\in\mathcal{B}$, which implies that $P=P'$ and $Q=Q'$. Let us show finally that $P$ and $Q$ are strongly relatively prime; by virtue of the discrete case and \cref{polynomial ring strongly relative prime prop}, for all $\a\in\mathcal{B}$, there exists a unique ordered pair $(U_\a,V_\a)$ of polynomials in $(A/\a)[X]$ such that
\[1=P_\a U_\a+Q_\a V_\a,\quad \deg(V_\a)<\deg(P_\a)=\deg(\bar{P}).\]
The uniqueness of this ordered pair shows immediately that, for $\a,\b\in\mathcal{B}$ with $\b\sub \a$, $U_\a=\pi_{\a\b}(U_\b)$ and $V_\a=\pi_{\a\b}(V_\b)$. Taking account of \cref{filtration restricted power series ring is limit}, we conclude that there exist two restricted formal power series $U$, $V$ of $A\{X\}$ such that $U_\a=\pi_\a(U)$, $V_\a=\pi_\a(V)$ and $1=PU+QV$.\par
It remains to verify that, if $T$ is a polynomial, so is $Q$. Now, the $Q_\a$ are polynomials by construction and, as $P_\a$ is monic, the relation $T_\a=P_\a Q_\a$ implies
\[\deg(Q_\a)\leq\deg(T_\a)\leq\deg(T)\]
for all $\a\in\mathcal{B}$, whence immediately the required result by definition of $Q$.
\end{proof}

\subsection{System of equations in complete rings}
Let $A$ be a ring; we shall say that a system
\[\bm{f}=(f_1,\dots,f_p)\in(A\llbracket X_1,\dots,X_q\rrbracket)^p\]
of formal power series in the $X_1,\dots,X_q$ with coefficients in $A$, is \textbf{without constant term} if this is true of all the $f_j$. For every systems of formal power series
\[\bm{f}=(f_1,\dots,f_p)\in(A\llbracket X_1,\dots,X_q\rrbracket)^p,\quad \bm{g}=(g_1,\dots,g_q)\in(A\llbracket X_1,\dots,X_r\rrbracket)^q\]
such that $\bm{g}$ is without constant term, we shall denote by $\bm{f}\circ\bm{g}$ (or $\bm{f}(\bm{g})$) the system of formal power series $f_j(g_1,\dots,g_q)$ in $(A\llbracket X_1,\dots,X_r\rrbracket)^p$. If
\[\bm{h}=(h_1,\dots,h_r)\in(A\llbracket X_1,\dots,X_s\rrbracket)^r\]
is a third system without constant term, then we have
\[(\bm{f}\circ\bm{g})\circ\bm{h}=\bm{f}\circ(\bm{g}\circ\bm{h}).\]
For every system $\bm{f}$, we shall denote by $M_{\bm{f}}$ or $M_{\bm{f}}(X)$ the Jacobian matrix $(\partial f_i/\partial X_j)$, where $i$ is the index of the rows and $j$ that of the columns. For two systems $\bm{f}$ and $\bm{g}$, where $\bm{g}$ is without constant term, we have
\begin{align}\label{Jacobi of composition}
M_{\bm{f}\circ\bm{g}}=M_{\bm{f}}(\bm{g})\cdot M_{\bm{g}}.
\end{align}
where $M_{\bm{f}}(\bm{g})$ is the matrix whose elements are obtained by substituting $g_j$ for $X_j$ in each series element of $M_{\bm{f}}$. We shall denote by $M_{\bm{f}}(0)$ the matrix of constant terms of the elements of $M_{\bm{f}}$. Then we deduce from (\ref{Jacobi of composition}) that
\[M_{\bm{f}\circ\bm{g}}(0)=M_{\bm{f}}(0)\cdot M_{\bm{g}}(0).\]
Given an integer $n>0$, we shall write
\[\mathbf{1}_n=\bm{X}=(X_1,\dots,X_n)\in(A\llbracket X_1,\dots,X_n\rrbracket)^n\]
which will be considered as a matrix with a single column.\par
For every system $\bm{f}=(f_1,\dots,f_n)\in (A\llbracket X_1,\dots,X_n\rrbracket)^n$, $M_{\bm{f}}$ is a square matrix of order $n$; we shall denote by $J_{\bm{f}}$ or $J_{\bm{f}}(X)$ its determinant and by $J_{\bm{f}}(0)$ the constant term of $J_{\bm{f}}$, equal to $\det M_{\bm{f}}(0)$. If $\bm{g}=(g_1,\dots,g_n)$ is a system without constant term in $(A\llbracket X_1,\dots,X_n\rrbracket)^n$, then
\[J_{\bm{f}\circ\bm{g}}=J_{\bm{f}}(\bm{g})J_{\bm{g}},\quad J_{\bm{f}\circ\bm{g}}(0)=J_{\bm{f}}(0)J_{\bm{g}}(0).\]
\begin{proposition}
Let $A$ be a ring and $\bm{f}=(f_1,\dots,f_n)$ a system without constant term of $n$ series in $A\llbracket X_1,\dots,X_n\rrbracket$. Suppose that $J_{\bm{f}}(0)$ is invertible in $A$. Then there exists a system without constant term $\bm{g}=(g_1,\dots,g_n)$ of $n$ series in $A\llbracket X_1,\dots,X_n\rrbracket$ such that $\bm{f}\circ \bm{g}=\mathbf{1}_n$. This system is unique and $\bm{g}\circ\bm{f}=\mathbf{1}_n$.
\end{proposition}
\begin{proof}
The existence and uniqueness of $\bm{g}$ are clear. It then follows that $J_{\bm{f}}(0)J_{\bm{g}}(0)=1$ and hence $J_{\bm{g}}(0)$ is also invertible. We conclude that there exists a system $\bm{h}=(h_1,\dots,h_n)$ of $n$ series without constant term in $A\llbracket X_1,\dots,X_n\rrbracket$ such that $\bm{g}\circ\bm{h}=\mathbf{1}_n$; from this relation it then follows that
\[\bm{h}=\mathbf{1}_n\circ\bm{h}=(\bm{f}\circ\bm{g})\circ\bm{h}=\bm{f}\circ(\bm{g}\circ\bm{h})=\bm{f}\circ\mathbf{1}_n=\bm{f}\]
so the second clam follows.
\end{proof}
To abbreviate, we shall say in what follows that a ring satisfies \textbf{Hensel's conditions} if it is linearly topologized, Hausdorff and complete; given an ideal $\m$ in such a ring, $\m$ (or the ordered pair $(A,\m)$) will be said to satisfy Hensel's conditions if $\m$ is closed in $A$ and its elements are topologically nilpotent. The ideal $\t$ of $A$ consisting of all the topologically nilpotent elements satisfies Hensel's conditions. In particular, if $A$ is a ring and $\m$ an ideal of $A$ and $A$ is Hausdorff and complete with respect to the $\m$-adic topology, the ordered pair $(A,\m)$ satisfies Hensel's conditions.
\begin{proposition}\label{power series to Hensel ring comtinuous map}
Let $A$ be a commutative ring, $B$ a ring satisfying Hensel's conditions and $\phi:A\to B$ a homomorphism. For every family $\bm{x}=(x_1,\dots,x_n)$ of topologically nilpotent elements of $B$, there exists a unique homomorphism $\tilde{\phi}$ from $A\llbracket X_1,\dots,X_n\rrbracket$ to $B$ such that $\tilde{\phi}(a)=\phi(a)$ for all $a\in A$ and $\tilde{\phi}(X_i)=x_i$ for $1\leq i\leq n$. Moreover, if $\m$ denotes the ideal of series without constant term in $A\llbracket X_1,\dots,X_n\rrbracket$, then $\tilde{\phi}$ is continuous for the $\m$-adic topology.
\end{proposition}
\begin{proof}
Let $\a$ be the finitely generated ideal generated in $B$ by the $x_i$'s; for every open ideal $\b$ of $B$, the images of the $x_i$ in $B/\b$ are nilpotent, hence the ideal $(\a+\b)/\b$ is nilpotent in $B/\b$ and there exists an integer $k$ such that, for $\alpha\in\N^n$ with $|\alpha|\geq k$ we have $\bm{x}^\alpha\in \b$. As every element of $\m^k$ is a finite sum of formal power series of the form $X^\alpha g(X)$, where $|\alpha|\geq k$, it is seen that, if $\tilde{\phi}$ solves the problem, then $\tilde{\phi}(\m^k)\sub \b$, which proves the continuity of $\tilde{\phi}$. There obviously exists a unique homomorphism
\[v:A[X_1,\dots,X_n]\to B\]
such that $v(a)=\phi(a)$ for $a\in A$ and $v(X_i)=x_i$, for $1\leq i\leq n$ and the above argument shows that $v$ is continuous with respect to the topoloogy induced on $A[X_1,\dots,X_n]$ by the $\m$-adic topology. As $A[X_1,\dots,X_n]$ is dense in $A\llbracket X_1,\dots,X_n\rrbracket$ with the $\m$-adic topology and $B$ is Hausdorff and complete, this completes the proof of the existence and uniqueness of $\tilde{\phi}$.
\end{proof}
If $B=A$ and $\phi$ is the identity map, we shall write $f(x_1,\dots,x_n)$ or $f(\bm{x})$ for the element $\tilde{\phi}(f)$ for every formal power seriesfe $A\llbracket X_1,\dots,X_n\rrbracket$. For every system $\bm{f}=(f_1,\dots,f_r)$ of formal power series of $A\llbracket X_1,\dots,X_n\rrbracket$, let $\bm{f}(\bm{x})$ denote the element $(f_1(\bm{x}),\dots,f_r(\bm{x}))$ of $A^r$, then it is said to be obtained by substituting the $x_i$ for the $X_i$ in $\bm{f}$. If $n\leq m$ and $F$ is a formal power series of $A\llbracket X_1,\dots,X_m\rrbracket$, it is possible to consider $F$ as a formal power series in $X_{n+1},\dots,X_m$ with coefficients in $A\llbracket X_1,\dots,X_n\rrbracket$; let
\[F(x_1,\dots,x_n,X_{n+1},\dots,X_m)\]
denote the formal power series in $A\llbracket X_{n+1},\dots,X_m\rrbracket$ obtained by substituting the $x_i$ for the $X_i$ in the coefficients of $F$, for $1\leq i\leq n$.
\begin{corollary}\label{power series composition and substitution}
Let $A$ be a ring satisfying Hensel's condition and $\bm{x}=(x_1,\dots,x_n)$ a family of topologically nilpotent elements of $A$. Let $\bm{g}=(g_1,\dots,g_q)$ be a system without constant term of series in $A\llbracket X_1,\dots,X_n\rrbracket$ and $\bm{f}=(f_1,\dots,f_p)$ a system of formal power series in $A\llbracket X_1,\dots,X_q\rrbracket$. Then $\bm{g}(\bm{x})=(g_1(\bm{x}),\dots,g_q(\bm{x}))$ is a family of topologically nilpotent elements of $A$ and
\begin{align}\label{power series composition and substitution-1}
(\bm{f}\circ\bm{g})(\bm{x})=\bm{f}(\bm{g}(\bm{x})).
\end{align}
\end{corollary}
\begin{proof}
The fact that the $g_i(\bm{x})$ are topologically nilpotent follows immediately from \cref{power series to Hensel ring comtinuous map} and the fact that in $A$ the ideal of topologically nilpotent elements is closed. Relation (\ref{power series composition and substitution-1}) is obvious when the $f$, $g$ are polynomials; on the other hand, if $\m$ and $\m'$ are the ideals of series without constant term in $A\llbracket X_1,\dots,X_q\rrbracket$ and $A\llbracket X_1,\dots,X_n\rrbracket$, respectively, clearly the relation $f\in\m^k$ implies $f(g_1,\dots,g_q)\in(\m')^k$. The two sides of (\ref{power series composition and substitution-1}) are therefore continuous functions of $\bm{f}$ to $(A\llbracket X_1,\dots,X_q\rrbracket)^p$ if $A\llbracket X_1,\dots,X_q\rrbracket$ is given the $\m$-adic topology, by virtue of the above remark and \cref{power series to Hensel ring comtinuous map}, whence the claim.
\end{proof}
In what follows, for a ring $A$ and an ideal $\m$ of $A$ we shall denote by $\m^{\times n}$ the product set $\prod_{i=1}^{n}\m_i$ in $A^n$, where $\m_i=\m$ for all $i$, to avoid ambiguity.
\begin{proposition}\label{power series substitution is bijective if}
Let $A$ be a ring and $\m$ an ideal of $A$ such that the ordered pair $(A,\m)$ satisfies Hensel's conditions. Let $\bm{f}=(f_1,\dots,f_n)$ be a system without constant term of series in $A\llbracket X_1,\dots,X_n\rrbracket$ such that $J_{\bm{f}}(0)$ is invertible in $A$. Then, for all $\bm{x}\in\m^{\times n}$, $\bm{f}(\bm{x})\in\m^{\times n}$ and $\bm{x}\mapsto\bm{f}(\bm{x})$ is a bijection of $\m^{\times n}$ onto itself, the inverse bijection being $\bm{x}\mapsto\bm{g}(\bm{x})$, where $\bm{g}$ is given by relation $\bm{f}\circ\bm{g}=\mathbf{1}_n$.
\end{proposition}
\begin{proof}
The fact that $\bm{f}(\bm{x})\in\m^{\times n}$ is obvious when the $f_i$ are polynomials and follows in the general case from \cref{power series to Hensel ring comtinuous map} and the fact that $\m$ is closed in $A$. The other assertions of the proposition are then immediate consequences of (\ref{power series composition and substitution-1}).
\end{proof}
\begin{corollary}
With the assumption of \cref{power series substitution is bijective if}, let $\a$ be a closed ideal of $A$ contained in $\m$. Then the relation $\bm{x}\equiv\bm{y}$ mod $\a^{\times n}$ is equivalent to $\bm{f}(\bm{x})\equiv\bm{f}(\bm{y})$ mod $\a^{\times n}$.
\end{corollary}
\begin{proof}
For every formal power series $f\in A\llbracket X_1,\dots,X_n\rrbracket$,
\[f(X_1,\dots,X_n)-f(Y_1,\dots,Y_n)=\sum_{i=1}^{n}(X_i-Y_i)h_i(X_1,\dots,X_n,Y_1,\dots,Y_n)\]
where the $h_i$ belong to $A\llbracket X_1,\dots,X_n,Y_1,\dots,Y_n\rrbracket$. It follows immediately that the relation $\bm{x}\equiv\bm{y}$ mod $\a^{\times n}$ implies $\bm{f}(\bm{x})\equiv \bm{f}(\bm{y})$ mod $\a^{\times n}$. The converse is obtained by replacing $\bm{f}$ by its "inverse" $\bm{g}$.
\end{proof}
\begin{theorem}\label{Hensel ring power series expansion by Jacobian}
Let $A$ be a ring and $\m$ an ideal of $A$ such that the ordered pair $(A,\m)$ satisfies Hensel's conditions. Let $\bm{f}=(f_1,\dots,f_n)$ be a system of $n$ elements of $A\{X_1,\dots,X_n\}$, let $\bm{a}\in A^n$ and write $e=J_{\bm{f}}(\bm{a})$. There exists a system $\bm{g}=(g_1,\dots,g_n)$ of restricted formal power series without constant term in $A\{X_1,\dots,X_n\}$ such that
\begin{itemize}
\item[(a)] $M_{\bm{g}}(0)=\a_n$.
\item[(b)] For all $\bm{x}\in A^n$,
\begin{align}\label{Hensel ring power series expansion by Jacobian-1}
\bm{f}(\bm{a}+e\bm{x})=\bm{f}(\bm{a})+M_{\bm{f}}(\bm{a})\cdot e\bm{g}(\bm{x}).
\end{align}
\item[(c)] Let $\bm{h}=(h_1,\dots,h_n)$ be the system offormal power series without constant term such that $\bm{g}\circ\bm{h}=\a_n$. For all $\bm{y}\in\m^{\times n}$,
\begin{align}\label{Hensel ring power series expansion by Jacobian-2}
\bm{f}(\bm{a}+e\bm{h}(\bm{y}))=\bm{f}(\bm{a})+M_{\bm{f}}(\bm{a})\cdot e\bm{y}.
\end{align}
\end{itemize}
\end{theorem}
\begin{proof}
For every formal power series $f\in A\llbracket X_1,\dots,X_n\rrbracket$,
\begin{align}\label{Hensel ring power series expansion by Jacobian-3}
f(\bm{X}+\bm{Y})=f(\bm{X})+M_f(\bm{X})\cdot\bm{Y}+\sum_{1\leq i\leq j\leq n}G_{ij}(\bm{X},\bm{Y})Y_iY_j
\end{align}
where the $G_{ij}$ are well determined formal power series in $A\llbracket X_1,\dots,X_n,Y_1,\dots,Y_n\rrbracket$. If $f$ is restricted, so are the elements of $M_f$, and the $G_{ij}$ for these formal power series are polynomials if $f$ is a polynomial and it follows from their uniqueness that for every open ideal $\a$ of $A$, denoting by $\pi_\a:A\to A/\a$ the canonical map, the image of $G_{ij}$ under $\pi_\a$ is the coefficient of $Y_iY_j$ in $\pi_\a(F)$ where $F$ is the formal power series $f(\bm{X}+\bm{Y})$ in $A\llbracket X_1,\dots,X_n,Y_1,\dots,Y_n\rrbracket$; whence our assertion.\par
This being so, writing formula (\ref{Hensel ring power series expansion by Jacobian-3}) for each series $f_i$, we obtain for all $x\in A^n$,
\begin{align}\label{Hensel ring power series expansion by Jacobian-4}
\bm{f}(\bm{a}+e\bm{x})=\bm{f}(\bm{a})+M_{\bm{f}}(\bm{a})\cdot e\bm{x}+e^2\bm{r}(\bm{x})
\end{align}
where $\bm{r}=(r_1,\dots,r_n)$ is a system of restricted formal power series each of which is of total order $\geq 2$. Now there exists a square matrix $N\in\mathcal{M}_n(A)$ such that $M_{\bm{f}}(\bm{a})N=e\a_n$, whence using this in (\ref{Hensel ring power series expansion by Jacobian-4}),
\[\bm{f}(\bm{a}+e\bm{x})=\bm{f}(\bm{a})+M_{\bm{f}}(\bm{a})\cdot e\bm{x}+M_{\bm{f}}(\bm{a})N\cdot e\bm{r}(\bm{x})\]
Writing $\bm{g}=\mathbf{1}_n+N\cdot\bm{r}$, we see that $\bm{g}$ satisfies conditions (a) and (b); then it is sufficient to replace $\bm{x}$ by $\bm{h}(\bm{y})$ to obtain (c).
\end{proof}
\begin{corollary}\label{Hensel ring root of restricted power series}
Let $A$ be a ring and $\m$ an ideal of $A$ such that the ordered pair $(A,\m)$ satisfies Hensel's conditions. Let $f\in A\{X\}$, $a\in A$ and write $e=f'(a)$. If $f(a)\equiv 0$ mod $e^2\m$, then there exists $b\in A$ such that $f(b)=0$ and $b\equiv a$ mod $e\m$. If $b'$ is another element such that $f(b')=0$ and $b'\equiv a$ mod $e\m$, then $e(b-b')=0$. In particular, $b$ is unique if $e$ is not a divisor of zero in $A$.
\end{corollary}
\begin{proof}
Let $f(a)=e^2c$ where $c\in\m$; formula (\ref{Hensel ring power series expansion by Jacobian-2}) for $n=1$ gives
\[f(a+eh(y))=e^2(c+y)\]
and it is therefore sufficient to take $y=-c$, whence $b=a+eh(-y)$. Moreover if $b=a+ex$, $b'=a+ex'$ are such that $x,x'\in\m$ and $f(b)=f(b')=0$, then we deduce from
(\ref{Hensel ring power series expansion by Jacobian-1}) that $e^2(g(x)-g(x'))=0$. As $g(X)-g(Y)=(X-Y)p(X,Y)$, where $p$ is restricted and $p(0,0)=1$, we see $g(x)-g(x')=(x-x')v$ where $v\in A$. Note that since $\m$ is closed, $p$ is restricted and $1$ is the constant term of $p$, we have $p(x,x')-1\in\m$, whence $v\in 1+\m$ and so $v$ is invertible. This proves the relation $e(b-b')=0$.
\end{proof}
\begin{example}
Let $p$ be a prime number $\neq 2$ and $n$ an integer whose class mod $p$ is a nonzero square in the prime field $\F_p$. If $\Z_p$ is the ring of $p$-adic integers, the application of \cref{Hensel ring root of restricted power series} to the polynomial $X^2-n$ shows that $n$ is a square in $\Z_p$; for example $7$ is a square in $\Z_3$.
\end{example}
\begin{example}
Let $A=K\llbracket Y\rrbracket$ be the ring of formal power series in one indeterminate with coefficients in a field $K$; with the $(Y)$-adic topology, the ring $A$ is Hausdorff and complete and the map $f(Y)\mapsto f(0)$ defines by passing to the quotient ring an isomorphism of $\kappa_A$ onto the field $K$. By \cref{Hensel ring root of restricted power series}, if $F(Y,X)$ is a polynomial in $X$ with coefficients in $A$ and $a$ is a simple root of $F(0,X)$ in $K$, there exists a unique formal power series $f(Y)$ such that $f(0)=a$ and $F(Y,f(Y))=0$.
\end{example}

\begin{corollary}
Let $A$ be a ring and $\m$ an ideal of $A$ such that the ordered pair $(A,\m)$ satisfies Hensel's conditions. Let $r$, $n$ be integers such that $0\leq r<n$ and $\bm{f}=(f_{r+1},\dots,f_n)$ is a system of $n-r$ elements of $A\{X_1,\dots,X_n\}$. Let $J_{\bm{f}}^{(n-r)}$ denote the minor of $M_{\bm{f}}(X)$ consisting of the columns of index $j$ such that $r+1\leq j\leq n$. Let $\bm{a}\in A^n$ be such that $J_{\bm{f}}(\bm{a})$ is invertible in $A$ and $\bm{f}(\bm{a})\equiv 0$ mod $\m^{\times(n-r)}$. Then there exists a unique $\bm{x}=(x_1,\dots,x_n)\in A^n$ such that $x_j=a_j$ for $1\leq j\leq r$ and $x\equiv a$ mod $\m^{\times n}$ and $\bm{f}(\bm{x})=0$.
\end{corollary}
\begin{proof}
Substituting $a_j$ for $X_j$ for $1\leq j\leq r$ in the $f_i$, we see immediately that we may restrict our attention to the case where $r=0$ to prove the corollary. \cref{power series substitution is bijective if} shows then that $\bm{f}$ defines a bijection on $\bm{a}+\m^{\times n}$ onto $\bm{f}(\bm{a})+\m^{\times n}=\m^{\times n}$; the corollary follows from the fact that $0\in\m^{\times n}$.
\end{proof}

\section{Complements on complete rings}
\subsection{Admissible rings}
Let $A$ be a linearly topologized ring and $\mathfrak{I}$ be an open ideal of $A$. We say that $\mathfrak{I}$ is a \textbf{nilideal} if for any open neighborhood $V$ of $0$ in $A$, there exists an integer $n>0$ such that $\mathfrak{I}^n\sub V$ (which means, by abusing language, that $(\mathfrak{I}^n)$ tends to $0$). The linearly topologized $A$ is called \textbf{preadmissible} if there exists a nilideal of $A$, and $A$ is \textbf{admissible} if it is preadmissible and is complete and separated.\par
It is clear that if $\mathfrak{I}$ is a nilideal and $\mathfrak{K}$ is an open ideal of $A$, then $\mathfrak{I}\cap\mathfrak{K}$ is also a nilideal. The nilideals of a preadmissible ring $A$ then form a fundamental system of open ideals (but note that the power $\mathfrak{I}^n$ of a nilideal if not necessarily open).
\begin{lemma}\label{topo ring nilpotent element iff}
Let $A$ be a linearly topologized ring.
\begin{itemize}
\item[(a)] For an element $x\in A$ to be topologically nilpotent, it is necessary and sufficient that for any open ideal $\mathfrak{I}$ of $A$, the canonical image of $x$ in $A/\mathfrak{I}$ is nilpotent. The set $\mathfrak{T}$ of topological nilpotent elements of $A$ is therefore an ideal.
\item[(b)] Suppose that $A$ is preadmissible and let $\mathfrak{I}$ be a nilideal of $A$. For an element $x\in A$ to be topologically nilpotent, it is necessary and sufficient that the canonical image of $x$ in $A/\mathfrak{I}$ is nilpotent. The ideal $\mathfrak{T}$ is then the inverse image of the nilradical of $A/\mathfrak{I}$ in $A$, and hence open.
\end{itemize}
\end{lemma}
\begin{proof}
By definition, $x$ is topologically nilpotent if and only if $x^n$ is contained in $\mathfrak{I}$ for sufficiently large $n$, whence the assertion of (a). As for assertion (b), it suffices to note that for any neighborhoof $V$ of $0$ in $A$, there exists $n>0$ such that $\mathfrak{I}^n\sub V$. If $x\in A$ is such that $x^m\in\mathfrak{I}$, then $x^{mq}\in V$ for $q\geq n$, so $x$ is topologically nilpotent.
\end{proof}
\begin{proposition}\label{admissible ring nilpotent iff contained in nilideal}
Let $A$ be a preadmissible ring and $\mathfrak{I}$ be a nilideal of $A$.
\begin{itemize}
\item[(a)] For an ideal $\mathfrak{J}$ of $A$ to be contained in a nilideal, it is necessary and sufficient that there exists an integer $n>0$ such that $\mathfrak{J}^n\sub\mathfrak{I}$.
\item[(b)] For an element $x\in A$ to be contained in a nilideal, it is necessary and sufficient that it is topologically nilpotent.
\end{itemize}
\end{proposition}
\begin{proof}
If $\mathfrak{J}^n\sub\mathfrak{I}$, then $\mathfrak{I}+\mathfrak{J}$ is a nilideal, because it is open and $(\mathfrak{I}+\mathfrak{J})^n\sub\mathfrak{I}$; this proves assertion (a). For (b), the condition is evidently necessary, and it is sufficient because if this is satisfied, then there exists an integer $n>0$ such that $x^n\in\mathfrak{I}$, so $\mathfrak{J}=\mathfrak{I}+Ax$ is a nilideal (it is open and $\mathfrak{J}^n\sub\mathfrak{I}$).
\end{proof}
\begin{corollary}\label{admissible ring open prime contain any nilideal}
In a preadmissible ring $A$, an open prime ideal contains any nilideal.
\end{corollary}
\begin{proof}
If $x\in\mathfrak{I}$ is an element of a nilideal and $\p$ is an open prime ideal, then $x$ is topologically nilpotent by \cref{admissible ring nilpotent iff contained in nilideal}(b), so $x^n\in\p$ for some integer $n>0$, and hence $x\in\p$.
\end{proof}
\begin{corollary}\label{admissible ring largest nilideal iff}
Let $A$ be a preadmissible ring. Then the following properties for an ideal $\mathfrak{I}_0$ of $A$ are equivalent:
\begin{itemize}
\item[(\rmnum{1})] $\mathfrak{I}_0$ is the largest nilideal of $A$;
\item[(\rmnum{2})] $\mathfrak{I}_0$ is a maximal nilideal of $A$;
\item[(\rmnum{3})] $\mathfrak{I}_0$ is a nilideal such that the ring $A/\mathfrak{I}_0$ is reduced.
\end{itemize}
For there exists an ideal $\mathfrak{I}_0$ satisfying these properties, it is necessary and sufficient that there exists a nilideal $\mathfrak{I}$ such that the nilradical $A/\mathfrak{I}$ is nilpotent. In this case $\mathfrak{I}_0$ is then equal to the ideal $\mathfrak{T}$ of topologically nilpotent elements of $A$, and we denote by $A_{\red}$ the reduced quotient ring $A/\mathfrak{T}$.
\end{corollary}
\begin{proof}
It is clear that (\rmnum{1}) imples (\rmnum{2}), and (\rmnum{3}) implies (\rmnum{1}) in view of \cref{admissible ring nilpotent iff contained in nilideal}(b) and \cref{topo ring nilpotent element iff}(b). On the other hand, if $\mathfrak{I}_0$ is maximal among nilideals, then for any nilideal $\mathfrak{I}$, there exists an integer $n>0$ such that $\mathfrak{I}_0^n\sub\mathfrak{I}$, so $(\mathfrak{I}_0+\mathfrak{I})$ is a nilideal of $A$. By the maximality, we then conclude that $\mathfrak{I}_0+\mathfrak{I}=\mathfrak{I}_0$, so $\mathfrak{I}\sub\mathfrak{I}_0$ and $\mathfrak{I}_0$ is the largest nilideal. The last assertion follows from \cref{admissible ring nilpotent iff contained in nilideal}(a) and \cref{topo ring nilpotent element iff}(b).
\end{proof}
\begin{corollary}\label{Now admissible ring largest nilideal exsit}
A Noetherian preadmissible ring admits a largest nilideal.
\end{corollary}
\begin{proof}
If $A$ is Noetherian preadmissible, then $A/\mathfrak{I}$ is also Noetherian, so its nilradical is nilpotent.
\end{proof}
We note that any defining ideal $\mathfrak{I}$ of $A$ is necessarily a nilideal, but the converse is not true. In view of this, we say that a ring $A$ is \textbf{preadic} if there exists a nilideal $\mathfrak{I}$ of $A$ such that $\mathfrak{I}^n$ form a fundamental system of open neighborhoods of $0$ in $A$ (which means each $\mathfrak{I}^n$ is open, or that $\mathfrak{I}$ is a defining ideal of $A$). The ring $A$ is \textbf{adic} if it is preadic and separated and complete. If $\mathfrak{J}$ is a nilideal of a preadic ring $A$ with defining ideal $\mathfrak{I}$, then it is easy to see that each $\mathfrak{J}^n$ is open in $A$, hence also a defining ideal of $A$. In other words, if $A$ is a preadic ring, then any nilideal defines the topology of $A$, hence is a defining ideal of $A$.
\begin{proposition}\label{admissible ring nilideal contained in Jacobson radical}
Let $A$ be a admissible ring and $\mathfrak{I}$ be a nilideal of $A$. Then $\mathfrak{I}$ is contained in the Jacobson radical of $A$. In particular, if $A$ is adic, then it is a Zariski ring.
\end{proposition}
\begin{proof}
In fact, since $A$ is separated and complete, the conclusion follows from \cref{filtration complete Hausdorff ring invertibility}(b) as any element of $\mathfrak{I}$ is topologically nilpotent.
\end{proof}
We shall now provide a characterization for admissible rings via projective limits of discrete rings. For this, let us recall that if $A$ is a linearly topologized ring and $(\mathfrak{I}_\lambda)$ is a fundamental system of open ideals of $A$, then the canonical homomorphisms $\varphi_\lambda:A\to A/\mathfrak{I}_\lambda$ form a projective system of discrete rings, hence define a continuous homomorphism $\varphi:A\to\llim A/\mathfrak{I}_\lambda$ (the latter can be identified with the completion of $A$). This homomorphism is injective if and only if $A$ is separated (and hence identified $A$ with a dense subring of $\llim A/\mathfrak{I}_\lambda$), and is an isomorphism if $A$ is separated and complete.
\begin{proposition}\label{admissible rings iff projective limit of discrete}
For a linearly topologized ring be to admissible, it is necessary and sufficient that it is isomorphic to a projective limit $A=\llim A_\lambda$, where $(A_\lambda,u_{\lambda\mu})$ is a projective system of discrete rings and the index set $I$ has a smallest element $0$ and satisfies the following conditions:
\begin{itemize}
\item[(a)] the homomorphisms $u_\lambda:A\to A_\lambda$ are surjective;
\item[(b)] the kernel $\mathfrak{I}_\lambda$ of $u_{0,\lambda}:A_\lambda\to A_0$ is nilpotent.
\end{itemize}
If these are satisfied, the kernel $\mathfrak{I}$ of $u_0:A\to A_0$ is equal to $\llim\mathfrak{I}_\lambda$.
\end{proposition}
\begin{proof}
This condition is necessary in view of the above remarks, since we can choose a fundamental system $(\mathfrak{I}_\lambda)$ of neighbrohoods of $0$ formed by nilideals contained in a fixed nilideal $\mathfrak{I}_0$, and note that $\mathfrak{I}_0^n\sub\mathfrak{I}_\lambda$ for sufficiently large $n$. The converse of this follows from the definition of projective limits, since $\mathfrak{I}$ is then a nilideal of $A$, and the last assertion is immediate.
\end{proof}
Let $A$ be an admissible ring and $\mathfrak{I}$ be an ideal of $A$ contained in a nilideal (which means $(\mathfrak{I}^n)$ tends to $0$). We can then consider the $\mathfrak{I}$-adic topology on $A$, and the hypothesis on $A$ implies that $\bigcap_n\mathfrak{I}^n=0$, so the $\mathfrak{I}$-adic topology on $A$ is separated.

\begin{proposition}\label{admissible ring I-adic complete separated if I^n nilpotent closed}
If $A$ is an admissible ring and $\mathfrak{I}$ be an ideal of $A$ contained in a nilideal such that the $\mathfrak{I}^n$ are closed in $A$, then $A$ is separated and complete for the $\mathfrak{I}$-adic topology.
\end{proposition}
\begin{proof}
This follows from \cref{topological group abelian two topo complete subset}, since the $\mathfrak{I}$-adic topology is finer than the original topology on $A$.
\end{proof}

\begin{corollary}\label{admissible ring nilideal closed Noe iff A/I Noe}
Let $A$ be an admissible ring, $\mathfrak{I}$ be a nilideal of $A$ such that the $\mathfrak{I}^n$ are closed in $A$. For $A$ to be Noetherian, it is necessary and sufficient that $A/\mathfrak{I}$ is Noetherian and that $\mathfrak{I}/\mathfrak{I}^2$ is a finitely generated $(A/\mathfrak{I})$-module.
\end{corollary}
\begin{proof}
These conditions are clearly necessary. Conversely, if they are satisfied, then since $\gr(A)$ is generated by $\gr_1(A)=\mathfrak{I}/\mathfrak{I}^2$ over $\gr_0(A)=A/\mathfrak{I}$, it is Noetherian by Hilbert basis theorem, and the conclusion follows from \cref{filtration complete ring gr(A) Noe imply A Noe}.
\end{proof}

If $A$ is an adic ring and $\mathfrak{I}$ is a defining ideal of $A$, then $A$ is the projective limit of the discrete rings $A_i=A/\mathfrak{I}^{i+1}$, which is identified with the $\mathfrak{I}$-adic completion of $A$. Conversely, we can in fact use this to characterize adic rings such that there exists a defining ideal $\mathfrak{I}$ such that $\mathfrak{I}/\mathfrak{I}^2$ is finitely generated, as the following proposition shows:
\begin{proposition}\label{adic ring given by I-adic completion}
Let $A$ be a ring, $\mathfrak{I}$ be an ideal of $A$ such that $\mathfrak{I}/\mathfrak{I}^2$ is a finitely generated $A/\mathfrak{I}$-module.
\begin{itemize}
\item[(a)] The ring $\widehat{A}=\llim(A/\mathfrak{I}^{n+1})$ is an adic ring. If $\widehat{\mathfrak{I}}$ is the closure of the canonical image of $\mathfrak{I}$ in $A$, then $\widehat{\mathfrak{I}}$ is a finitely generated nilideal of $A$, and $\widehat{\mathfrak{I}}^n$ is the closure of the canonical image of $\mathfrak{I}^n$ in $A$. Moreover, $\widehat{A}/\widehat{\mathfrak{I}}^n$ is isomorphic to $A/\mathfrak{I}^n$ and $\widehat{\mathfrak{I}}/\widehat{\mathfrak{I}}^2$ to $\mathfrak{I}/\mathfrak{I}^2$, and $\widehat{\mathfrak{I}}$ is isomorphic to $\llim(\mathfrak{I}/\mathfrak{I}^{n+1})$.
\item[(b)] Let $M$ be an $A$-module such that $M/\mathfrak{I}M$ is a finitely generated $A/\mathfrak{I}$-module. Then $\widehat{M}=\llim(M/\mathfrak{I}^{n+1}M)$ is a finitely generated $A$-module, $\widehat{\mathfrak{I}}^n\widehat{M}$ is the closure of the canonical image of $\mathfrak{I}^nM$, and $\widehat{M}/\widehat{\mathfrak{I}}^n\widehat{M}$ is isomorphic to $M/\mathfrak{I}^nM$.
\end{itemize}
Moreover, for $A$ to be Noetherian, it is necessary and sufficient that $A_0$ is Noetherian.
\end{proposition}
\begin{proof}
It is clear that the conditions of \cref{ring inverse limit complete and finiteness prop} and \cref{ring inverse limit topology is adic if finite ideal} are satisfied, and the last assertion follows from \cref{ring inverse limit Noe iff A_0 Noe}.
\end{proof}

\subsection{Complete localizations}
Let $A$ be a linearly topologized ring, $(\mathfrak{I}_\lambda)$ be a fundamental system of neighborhoods of $0$ formed by ideals of $A$, and $S$ be a multiplicative subset of $A$. Let $u_\lambda:A\to A_\lambda=A/\mathfrak{I}_\lambda$ be the canonical homomorphism, and for $\mathfrak{I}_\mu\sub\mathfrak{I}_\lambda$, let $u_{\lambda\mu}:A_\mu\to A_\lambda$ be the canonical homomorphism. Put $S_\lambda=u_\lambda(S)$, so that $u_{\lambda\mu}(S_\mu)=S_\lambda$. The homomorphisms $u_{\lambda\mu}$ then induce canonical surjections $S_\mu^{-1}A_\mu\to S_\lambda^{-1}A_\lambda$, which form a projective system and we denote by $A\{S^{-1}\}$ its projective limit. It is clear that this does not depend on the choice of the system $(\mathfrak{I}_\lambda)$.
\begin{proposition}\label{topo ring complete localization char}
The ring $A\{S^{-1}\}$ is homeomorphic to the completion of the localization $S^{-1}A$ for the topology induced by the closed ideals $S^{-1}\mathfrak{I}_\lambda$.
\end{proposition}
\begin{proof}
In fact, if $v_\lambda:S^{-1}A\to S_\lambda^{-1}A_\lambda$ is the canonical homomorphism induced by $u_\lambda$, the kernel of $v_\lambda$ is $S^{-1}\mathfrak{I}_\lambda$ and $v_\lambda$ is surjective, and the completion of $S^{-1}A$ is identified with the limit $A\{S^{-1}\}$.
\end{proof}
\begin{corollary}\label{topo ring complete localization image in completion}
If $\widehat{S}$ is the image of $S$ in the completion $\widehat{A}$ of $A$, then $A\{S^{-1}\}$ is canonically identified with $\widehat{A}\{\widehat{S}^{-1}\}$.
\end{corollary}
\begin{remark}
Note that even if $A$ is separated and complete, the localization $S^{-1}A$ many be neither separated nor complete. For example, let $S$ be the subset $f^n$ for $n\geq 0$, where $f$ is a topologically nilpotent element in $A$ but not nilpotent. Then $S^{-1}A$ is nonzero, but for any $\lambda$ there exists an integer $n$ such that $f^n\in\mathfrak{I}_\lambda$, so $1=f^n/f^n\in S^{-1}\mathfrak{I}_\lambda$ and $S^{-1}\mathfrak{I}_\lambda=S^{-1}A$.
\end{remark}
\begin{corollary}\label{topo ring complete localization nonzero if}
If $0$ is not contained in the closure of $S$, then the ring $A\{S^{-1}\}$ is nonzero.
\end{corollary}
\begin{proof}
In fact, $0$ is not contained in the closure of $\{1\}$ in the ring $S^{-1}A$, since otherwise $1\in S^{-1}\mathfrak{I}_\lambda$ for any ideal $\mathfrak{I}_\lambda$, whence $\mathfrak{I}_\lambda\cap S\neq\emp$, a contradiction.
\end{proof}
The ring $A\{S^{-1}\}$ is called the \textbf{complete localization} of the ring $A$. It is clear that the inverse image of $S^{-1}\mathfrak{I}_\lambda$ in $A$ contains $\mathfrak{I}_\lambda$, so the canonical homomorphism $A\to S^{-1}A$ is continuous. By composing with the canonical homomorphism $A\to A\{S^{-1}\}$, we then obtain a canonical continuous homomorphism $A\to A\{S^{-1}\}$, which is the projective limit of the homomorphisms $A\to S_\lambda^{-1}A_\lambda$. The ring $A\{S^{-1}\}$ satisfies a similar universal property as the localization rings:
\begin{proposition}\label{topo ring complete localization universal prop}
Any continuous homomorphism $u$ from $A$ into a separated and complete linearly topologized ring $B$ such that $u(S)$ is invertible in $B$ factors uniquely through $A\{S^{-1}\}$:
\[\begin{tikzcd}
A\ar[r]&A\{S^{-1}\}\ar[r,"\tilde{u}"]&B
\end{tikzcd}\]
where $\tilde{u}$ is continuous.
\end{proposition}
\begin{proof}
In fact, $u$ factors through $S^{-1}A$:
\[\begin{tikzcd}
A\ar[r]&S^{-1}A\ar[r,"\tilde{v}"]&B
\end{tikzcd}\]
Now since for any open ideal $\mathfrak{K}$ of $B$, $u^{-1}(\mathfrak{K})$ contains an ideal $\mathfrak{I}_\lambda$, we see that $\tilde{v}^{-1}(\mathfrak{K})$ contains $S^{-1}\mathfrak{I}_\lambda$, so $\tilde{v}$ is continuous. Since $B$ is complete and separated, the homomorphism $\tilde{v}$ then factors into the following
\[\begin{tikzcd}
S^{-1}A\ar[r]&A\{S^{-1}\}\ar[r,"\tilde{u}"]&B
\end{tikzcd}\]
where $\tilde{u}$ is continuous; this proves the proposition.
\end{proof}
Let $B$ be a second linearly topologized ring, $T$ be a multiplicative subset of $B$, and $\varphi:A\to B$ be a continuous homomorphism such that $\varphi(S)\sub T$. Then by \cref{topo ring complete localization universal prop}, the continuous homomorphism
\[\begin{tikzcd}
A\ar[r,"\varphi"]&B\ar[r]&B\{T^{-1}\}
\end{tikzcd}\]
then factors into 
\[\begin{tikzcd}
A\ar[r]&A\{S^{-1}\}\ar[r,"\tilde{\varphi}"]&B\{T^{-1}\}
\end{tikzcd}\]
where $\tilde{\varphi}$ is continuous. In particular, if $A=B$ and $\varphi$ is the identity, we see that for any multiplicative subsets $S\sub T$ we have a continuous homomorphism $\rho^{T,S}:A\{S^{-1}\}\to A\{T^{-1}\}$, obtained by passing to completions from the canonical homomorphism $S^{-1}A\to T^{-1}A$. If $U$ is a thrid multiplicative subset of $A$ such that $S\sub T\sub U$, we have $\rho^{U,S}=\rho^{U,T}\circ\rho^{T,S}$.\par
Let $S_1$ and $S_2$ be multiplicative subset of $A$, and $\tilde{S}_2$ be the canonical image of $S_2$ in $A\{S_1^{-1}\}$. We then have a canonical isomorphism $A\{(S_1S_2)^{-1}\}\cong A\{S_1^{-1}\}\{\tilde{S}_2^{-1}\}$, which comes from the canonical isomorphism $(S_1S_2)^{-1}A\cong \bar{S}_2^{-1}(S_1^{-1}A)$, where $\bar{S}_2$ is the canonical image of $S_2$ in $S_1^{-1}A$.
Let $\a$ be an open ideal of $A$. We may suppose that $\mathfrak{I}_\lambda\sub\a$ for any $\lambda$, and therefore $S^{-1}\mathfrak{I}_\lambda\sub S^{-1}\a$ in the ring $S^{-1}A$, which means $S^{-1}\a$ is an open ideal of $S^{-1}A$. We denote by $\a\{S^{-1}\}$ the completion of $S^{-1}\a$, which is identified with $\llim(S^{-1}\a/S^{-1}\mathfrak{I}_\lambda)$ and is an open ideal of $A\{S^{-1}\}$. Moreover, the discrete ring $A\{S^{-1}\}/\a\{S^{-1}\}$ is canonically isomorphic to $S^{-1}A/S^{-1}\a=S^{-1}(A/\a)$.
\begin{proposition}\label{topo ring complete localization open prime char}
Let $A$ be a linearly topologized ring and $S$ be a multiplicative subset.
\begin{itemize}
\item[(a)] Any open ideal of $A\{S^{-1}\}$ is of the form $\a\{S^{-1}\}$, where $\a$ is an open ideal of $A$.
\item[(b)] The map $\p\mapsto\p\{S^{-1}\}$ is an increasing bijection from the set of open prime ideals $\p$ in $A$ such that $\p\cap S=\emp$ to the set of open prime ideals of $A\{S^{-1}\}$. Moreover, the fraction field of $A\{S^{-1}\}/\p\{S^{-1}\}$ is canonically isomorphic to $A/\p$.
\end{itemize}
\end{proposition}
\begin{proof}
If $\a'$ is an open ideal of $A\{S^{-1}\}$, then $\a'$ contains an ideal of the form $\mathfrak{I}_\lambda\{S^{-1}\}$, hence is the inverse image of an ideal of $S^{-1}A/S^{-1}\mathfrak{I}_\lambda$, which is necessarily of the form $S^{-1}\a$, where $\a\sups\mathfrak{I}_\lambda$. The second assertion follows from \cref{localization and ideals} and the fact that taking completion does not change residue fields.
\end{proof}
\begin{proposition}\label{topo ring complete localization admissible adic prop}
Let $A$ be a linearly topologized ring and $S$ be a multiplicative subset of $A$.
\begin{itemize}
\item[(a)] If $A$ is admissible, so is $A'=A\{S^{-1}\}$, and for any nilideal $\mathfrak{I}$ of $A$, $\mathfrak{I}'=\mathfrak{I}\{S^{-1}\}$ is a nilideal of $A\{S^{-1}\}$.
\item[(b)] Suppose that $A$ is adic and $\mathfrak{I}$ is a definig ideal of $A$ such that $\mathfrak{I}/\mathfrak{I}^2$ is finitely generated over $A/\mathfrak{I}$. Then $A'$ is an $\mathfrak{I}'$-adic ring and $\mathfrak{I}'/\mathfrak{I}'^2$ is finitely generated over $A'/\mathfrak{I}'$.
\end{itemize}
\end{proposition}
\begin{proof}
If $\mathfrak{I}$ is a nilideal of $A$, then it is clear that $S^{-1}\mathfrak{I}$ is nilideal of $S^{-1}A$, because $(S^{-1}\mathfrak{I})^n=S^{-1}\mathfrak{I}^n$ for each $n$. Now let $\widebar{A}$ be the separated ring associated with $S^{-1}A$, and $\widebar{\mathfrak{I}}$ be the image of $S^{-1}\mathfrak{I}$ in $\widebar{A}$. Then the image of $S^{-1}\mathfrak{I}^n$ is $\widebar{\mathfrak{I}}^n$, so $\widebar{\mathfrak{I}}^n$ tends to $0$ in $\widebar{A}$. As $\mathfrak{I}'$ is the closure of $\widebar{\mathfrak{I}}$ in $A'$, $\mathfrak{I}'^n$ is contained in the closure of $\widebar{\mathfrak{I}}^n$, hence tends to $0$ in $A'$.\par
For (b), put $A_i=A/\mathfrak{I}^{i+1}$, and let $u_{ij}:A_j\to A_i$ be the canonical homomorphism for $i\leq j$. Let $S_i$ be the image of $S$ in $A_i$, put $A'_i=S_i^{-1}A_i$, and let $u'_{ij}:A_j'\to A_i'$ be the induced homomorphism. We show that the projective system $(A_i',u'_{ij})$ satisfies the conditions of \cref{ring inverse limit complete and finiteness prop}. It is clear that each $u'_{ij}$ is surjective, and by the flatness of localization, the kernel of $u'_{ij}$ is equal to $S_j^{-1}(\mathfrak{I}^{i+1}/\mathfrak{I}^{j+1})=\mathfrak{I}'^{i+1}_j$, where $\mathfrak{I}'_j=S_j^{-1}(\mathfrak{I}/\mathfrak{I}^{i+1})$. Finally, $\mathfrak{I}'_1/\mathfrak{I}'^2_1=S_1^{-1}(\mathfrak{I}/\mathfrak{I}^2)$ is finitely generated over $A_1'$ since $\mathfrak{I}/\mathfrak{I}^2$ is finitely generated over $A/\mathfrak{I}$, and $A_0'=S_0^{-1}(A/\mathfrak{I})$ is Noetherian if $A$ is Noetherian.
\end{proof}
\begin{corollary}\label{topo ring complete localization adic ring ideal power prop}
Under the hypothesis of \cref{topo ring complete localization admissible adic prop}(b), we have $(\mathfrak{I}\{S^{-1}\})^n=\mathfrak{I}^n\{S^{-1}\}$.
\end{corollary}
\begin{proof}
This follows from \cref{ring inverse limit complete and finiteness prop}, in view of the proof of \cref{topo ring complete localization admissible adic prop}.
\end{proof}
\begin{proposition}\label{Noe adic ring complete localization flat}
Let $A$ be an adic Noetherian ring and $S$ be a multiplicative subset of $A$. Then $A\{S^{-1}\}$ is a flat $A$-module.
\end{proposition}
\begin{proof}
If $\mathfrak{I}$ is a defining ideal of $A$, $A\{S^{-1}\}$ is the completion of the Noetherian ring $S^{-1}A$ for the $S^{-1}\mathfrak{I}$-adic topology, so $A\{S^{-1}\}$ is a flat $S^{-1}A$-module (\cref{filtration Noe I-adic completion is tensor}). As $S^{-1}A$ is flat over $A$, we conclude the proposition.
\end{proof}
\begin{corollary}\label{Noe adic ring complete successive localization flat}
Let $A$ be an adic Noetherian ring and $T\sub S$ be multiplicative subsets of $A$. Then $A\{S^{-1}\}$ is a flat $A\{T^{-1}\}$-module.
\end{corollary}
\begin{proof}
We have remarked that $A\{S^{-1}\}$ is canonically identified with $A\{T^{-1}\}\{S_0^{-1}\}$, where $S_0$ is the canonical image of $S$ in $A\{T^{-1}\}$, and $A\{T^{-1}\}$ is Noetherian by \cref{topo ring complete localization admissible adic prop}.
\end{proof}
For any element $f\in A$, we denote by $A_{\{f\}}$ the complete localization $A\{S_f^{-1}\}$, where $S_f=\{f^n\}$. For any open ideal $\a$ of $A$, we denote by $\a_{\{f\}}$ the ideal $\a\{S_f^{-1}\}$. If $g$ is another element of $A$, we then have a continuous homomorphism $A_{\{f\}}\to A_{\{fg\}}$. As $f$ runs through a multiplicative subset $S$ of $A$, the rings $A_{\{f\}}$ then form an inductive system, and we set $A_{\{S\}}=\rlim_{f\in S}A_{\{f\}}$. For any $f\in S$, we have a canonical homomorphism $A_{\{f\}}\to A\{S^{-1}\}$, and they form an inductive system which induces a canoncal homomorphism $A_{\{S\}}\to A\{S^{-1}\}$.
\begin{proposition}\label{Noe ring complete localization flat over}
If $A$ is a Noetherian ring, $A\{S^{-1}\}$ is a flat module over $A_{\{S\}}$.
\end{proposition}
\begin{proof}
By \cref{Noe adic ring complete successive localization flat} the ring $A\{S^{-1}\}$ is flat over each $A_{\{f\}}$ for $f\in S$, so the conclusion follows from \cref{module flat direct limit}.
\end{proof}
\begin{proposition}\label{admissible ring complete localization at prime residue prop}
Let $\p$ be an open prime ideal of an admissible ring $A$, and let $S=A-\p$. Then the ring $A\{S^{-1}\}$ and $A_{\{S\}}$ are local, the canonical homomorphism $A_{\{S\}}\to A\{S^{-1}\}$ is local, and the residue fields of $A_{\{S\}}$ and $A\{S^{-1}\}$ are isomorphic to $\kappa(\p)$.
\end{proposition}
\begin{proof}
Let $\mathfrak{I}\sub\p$ be a nilideal of $A$. We then have $S^{-1}\mathfrak{I}\sub S^{-1}\p=\p A_\p$, so $A_\p/S^{-1}\mathfrak{I}$ is a local ring, and we conclude from \cref{topo ring complete localization admissible adic prop}(a) that $A\{S^{-1}\}$ is a local ring. Put $\m=\rlim_{f\in S}\p_{\{f\}}$, which is an ideal of $A_{\{S\}}$, we then see that any element of $A_{\{S\}}-\m$ is invertible, so $A_{\{S\}}$ is a local ring with maximal ideal $\m$. In fact, any such element is the image of an element $z\in A_{\{f\}}=\p_{\{f\}}$ in $A_{\{S\}}$, for an element $f\in S$. Its canonical image $z_0$ in $A_{\{f\}}/\mathfrak{I}_{\{f\}}=S_f^{-1}(A/\mathfrak{I})$ is then not contained in $S_f^{-1}(\p/\mathfrak{I})$, which means $z_0=\bar{x}/\bar{f}^k$, where $x\notin\p$ and $\bar{x},\bar{f}$ are the classes of $x,f$ mod $\mathfrak{I}$. As $x\in S$, we have $g=xf\in S$, so $y_0=x^{k+1}/g^k$, the canonical image of $x/f^k\in S_f^{-1}A$ in the ring $S_g^{-1}A$, then admits an inverse $x^{k-1}f^{2k}/g^k$. This implies a fortiori that the image of $y_0$ in $S_g^{-1}A/S_g^{-1}\mathfrak{I}$ is invertible, so (\cref{topo ring complete localization adic ring ideal power prop}) the canonical image $y$ of $z$ in $A_{\{g\}}$ is invertible, and so is its image in $A_{\{S\}}$. Now, the image of $\p_{\{f\}}$ in $A\{S^{-1}\}$ is contained in the maximal ideal $\p\{S^{-1}\}$ of this ring, so the image of $\m$ in $A\{S^{-1}\}$ is contained in $\p\{S^{-1}\}$, which means the homomorphism $A_{\{S\}}\to A\{S^{-1}\}$ is local. Finally, as any element $A\{S^{-1}\}/\p\{S^{-1}\}$ is the image of an element of $S_f^{-1}A$ for some $f\in S$, the homomorphism $A_{\{S\}}\to A\{S^{-1}\}/\p\{S^{-1}\}$ is surjective, and hence induces an isomorphism on residue fields. This residue field is isomorphic to $\kappa(\p)$ since taking completion does not change residue fields. 
\end{proof}
\begin{corollary}\label{Noe ring complete localization at prime faithfully flat over limit}
Under the hypotheses of \cref{admissible ring complete localization at prime residue prop}, if we suppose that $A$ is an adic Noetherian ring, the local rings $A\{S^{-1}\}$ and $A_{\{S\}}$ are Noetherian, and $A\{S^{-1}\}$ is a faithfully flat $A_{\{S\}}$-module.
\end{corollary}
\begin{proof}
By \cref{topo ring complete localization admissible adic prop}(b) and \cref{Noe adic ring complete successive localization flat}, the ring $A\{S^{-1}\}$ is Noetherian and flat over $A_{\{S\}}$. As the homomorphism $A_{\{S\}}\to A\{S^{-1}\}$ is local, we then conclude that $A\{S^{-1}\}$ is faithfully flat over $A_{\{S\}}$, and hence Noetherian (\cref{ring faithfully flat Noe Artin descent}).
\end{proof}
\subsection{Complete tensor products}
Let $A$ be a linearly topologized ring and $M,N$ be two linearly topologized $A$-modules. Let $\mathfrak{I}$, $V$, $W$ be neighborhoods of $0$ in $A$, $M$, $N$, respectively, which are $A$-modules and such that $\mathfrak{I}\cdot M\sub V$, $\mathfrak{I}\cdot N\sub W$. Then the quotient modules $M/V$ and $N/W$ can be considered as $(A/\mathfrak{I})$-modules. If $\mathfrak{I}$, $V$, $W$ runs through the systems of open neighborhoods satisfying the preceding conditions, it is immediate that the modules $(M/V)\otimes_{A/\mathfrak{I}}(N/W)$ form a projective system of modules over the projective system $A/\mathfrak{I}$, so by passing to limit we obtain a module over the completion $\widehat{A}$ of $A$, which is called the complete tensor product of $M$ and $N$, and denoted by $M\hat{\otimes}_AN$. Note that since $M/V$ is canonically identified with $\widehat{M}/\widehat{V}$, where $\widehat{V}$ is the closure of $V$ in $\widehat{M}$, the complete tensor product is canonically identified with the completion of $\widehat{M}\otimes_{\widehat{A}}\widehat{N}$, hence with $\widehat{M}\hat{\otimes}_{\widehat{A}}\widehat{N}$.\par
We also note that the tensor products $(M/V)\otimes_A(N/W)$ and $(M/V)\otimes_{A/\mathfrak{I}}(N/W)$ are both canonically idnetified with the quotient $(M\otimes_AN)/(\im(V\otimes_AN)+\im(M\otimes_AW))$, so the complete tensor product $M\hat{\otimes}_AN$ is the completion of the $A$-module $M\otimes_AN$ endowed with the topology induced by the submodules $\im(V\otimes_AN)+\im(M\otimes_AW)$. For simplicity, we say that this topology is the \textbf{tensor product} of the given topologies on $M$ and $N$.\par
Let $u:M\to M'$ and $v:N\to N'$ be continuous homomorphisms of linearly topologized $A$-modules. It is immediate that $u\otimes v$ is continuous for the tensor product topologies on $M\otimes_AN$ and $M'\otimes_AN'$, respectively, so by passing to completion, we deduce a continuous homomorphism $M'\hat{\otimes}_AN'\to M\hat{\otimes}_AN$, which we denote by $u\hat{\otimes}v$. In this way, the construction $(M,N)\mapsto M\hat{\otimes}_AN$ is then a bifunctor on the category of linearly topologized $A$-modules. Moreover, in the same way we can define the complete tensor product for finitely many linearly topologized $A$-modules, and it is clear that this construction satisfies the commutativity and associativity.\par
Let $B$ and $C$ be linearly topologized $A$-algebras. Then the tensor product of the topologies on $B$ and $C$ is a linear topology on $B\otimes_AC$, so $B\hat{\otimes}_AC$ is endowed with a topological $\widehat{A}$-algebra structure. This algebra is called the complete tensor product of the algebras $B$ and $C$.\par
The homomorphisms $b\mapsto b\otimes 1$ and $c\mapsto 1\otimes c$ of $B$ and $C$ into $B\otimes_AC$ are continuous for the tensor product topology, so by composing with the canonical homomorphism from $B\otimes_AC$ to its completion, we obtain canonical homomophisms $\rho:B\to B\hat{\otimes}_AC$ and $\sigma:C\to B\hat{\otimes}_AC$. The algebra $B\hat{\otimes}_AC$ and the homomorphisms $\rho$ and $\sigma$ then satisfies the following universal property:
\begin{proposition}\label{topo ring complete tensor universal prop}
For any complete and separated linearly topologized $A$-algebra $D$ and any couple of continuous $A$-homomorphisms $u:B\to D$, $v:C\to D$, there exists a continuous $A$-homomorphism $w:B\hat{\otimes}_AC\to D$ such that the following diagram is commutative
\[\begin{tikzcd}
A\ar[r]\ar[d]&B\ar[d,swap,"\rho"]\ar[rdd,bend left=20pt,swap,"u"]&\\
C\ar[r,"\sigma"]\ar[rrd,bend right=20pt,"v"]&B\hat{\otimes}_AC\ar[rd,"w"{anchor=south}]&\\
&&D
\end{tikzcd}\]
\end{proposition}
\begin{proof}
In fact, there exists a unique homomorphism $w_0:B\otimes_AC\to D$ such that $u(b)=w_0(b\otimes 1)$ and $v(c)=w_0(1\otimes c)$, and it suffices to prove that $w_0$ is continuous, since it then induces a continuous homomorphism $w$ by passing to completion. Now if $\mathfrak{M}$ is an open ideal of $D$, there exists by hypothesis open ideals $\mathfrak{K}\sub B$, $\mathfrak{R}\sub C$ such that $u(\mathfrak{K})\sub\mathfrak{M}$, $v(\mathfrak{R})\sub\mathfrak{M}$, so the image of $\im(\mathfrak{K}\otimes C)+\im(B\otimes\mathfrak{R})$ under $w_0$ is contained in $\mathfrak{M}$, whence the assertion.
\end{proof}
\begin{proposition}\label{topo ring complete tensor nilideal}
If $B$ and $C$ are two preadmissible $A$-algebras, $B\hat{\otimes}_AC$ is admissible, and if $\mathfrak{K}$ (resp. $\mathfrak{R}$) is a nilideal of $B$ (resp. $C$), then the closure of $\mathfrak{L}=\im(\mathfrak{K}\otimes C)+\im(B\otimes\mathfrak{R})$ in $B\hat{\otimes}_AC$ is a nilideal.
\end{proposition}
\begin{proof}
It suffices to show that $\mathfrak{L}^n$ tends to $0$ in the tensor product topology, and this follows immediately from the inclusion $\mathfrak{L}^{2n}\sub\im(\mathfrak{K}^n\otimes C)+\im(B\otimes\mathfrak{R}^n)$.
\end{proof}
\begin{proposition}\label{topo ring complete tensor base change transitive}
Let $B,C$ be linearly topologized $A$-algebras, $M$ be a linearly topologized $B$-module, whose topology is finer than the topology induced by $B$. Then the canonical isomorphism $M\otimes_B(B\otimes_AC)\cong M\otimes_AC$ is a topological isomorphism for the tensor product topologies. In particular, the comple tensor product $M\hat{\otimes}_B(B\hat{\otimes}_AC)$ is topologically isomorphic to $M\hat{\otimes}_AC$.
\end{proposition}
\begin{proof}
An open neighborhood of $0$ in the tensor product $M\otimes_B(B\otimes_AC)$ contains a neighborhood of the form
\[W=\im(V\otimes_B(B\otimes_AC))+\im(M\otimes_B(\im(\mathfrak{K}\otimes_AC))+\im(B\otimes_A\mathfrak{R}))\]
where $V$ is an open submodule of $M$, $\mathfrak{K}$ (resp. $\mathfrak{R}$) is an open ideal of $B$ (resp. $C$). This submodule can also be written as 
\[\im(V\otimes_AC)+\im(M\otimes_A\mathfrak{R})+\im(\mathfrak{K}M\otimes_AC)\]
and hence contains $W'=\im(V\otimes_AC)+\im(M\otimes_A\mathfrak{R})$, which is an open neighborhood of $0$ in $M\otimes_AC$. Conversely, if $\mathfrak{K}$ is taken so that $\mathfrak{K}M\sub V$, which is possible by hypothesis, then $W=W'$ and the first assertion follows. The second assertion follows from the equality
\begin{equation*}
M\hat{\otimes}_B(B\hat{\otimes}_AC)=\widehat{M}\hat{\otimes}_{\widehat{B}}\widehat{(B\otimes_AC)}=\widehat{(M\otimes_B(B\otimes_AC))}=\widehat{M\otimes_AC}=M\hat{\otimes}_AC.\qedhere
\end{equation*}
\end{proof}
\begin{proposition}\label{Now adic ring finite module complete tensor with adic prop}
Let $A$ be a preadic ring, $\mathfrak{I}$ be a defining ideal of $A$, and $M$ be a finitely generated $A$-module. Then for any adic and Noetherian topological $A$-algebra $B$, $B\otimes_AM$ is identified with the complete tensor product $B\hat{\otimes}_AM$.
\end{proposition}
\begin{proof}
If $\mathfrak{K}$ is a defining ideal of $B$, then there exists by hypothesis an integer $m$ such that $\mathfrak{I}^mB\sub\mathfrak{K}$, so
\[\im(B\otimes_A\mathfrak{I}^{nm}M)=\im(\mathfrak{I}^{mn}B\otimes_AM)\sub\im(\mathfrak{K}^{n}B\otimes_AM)=\mathfrak{K}^n(B\otimes_AM).\]
We then conclude that over $B\otimes_AM$, the tensor product topology is equivalent to the $\mathfrak{K}$-adic topology. As $B\otimes_AM$ is a finitely generated $B$-module, it is equal to the $\mathfrak{K}$-adic completion (\cref{filtration Noe I-adic completion is tensor}), since $B$ is a Zariski ring (\cref{Zariski ring def}).
\end{proof}
\begin{proposition}\label{topo ring complete tensor with complete ring prop}
Let $A$ be a topological ring, $B,C$ be topological Noetherian local $A$-algebras with maximal ideals $\m,\n$, respectively, endowed with the adic topologies. Suppose that $C$ is complete and that the residue field $B/\m$ is a finitely generated $A$-module. Let $E=B\hat{\otimes}_AC$ be the complete tensor product.
\begin{itemize}
\item[(a)] $E$ is a complete Noetherian semi-local ring. 
\item[(b)] The ideal $\m E$ is contained in the radical of $E$, and for any $k>0$, $E/\m^kE$ is isomorphic to $(B/\m^k)\otimes_AC$.
\item[(c)] If $C$ is a flat $A$-module, then $E$ is a flat $B$-module.
\end{itemize}
\end{proposition}
\begin{proof}

\end{proof}
\begin{corollary}\label{module complete tensor with complete ring isomorphism}
Under the hypotheses of \cref{topo ring complete tensor with complete ring prop}, let $M$ be a finitely generated $B$-module, endowed with the $\m$-adic topology. Then $M\hat{\otimes}_AC$ is isomorphic to $M\otimes_BE$.
\end{corollary}
\begin{proof}
In fact, as $\m E\sub\r E$, the tensor product topology on $M\otimes_BE$ is the $\r$-adic topology, and as $E$ is complete for this topology, so is $M\otimes_BE$. It then suffices to apply \cref{topo ring complete tensor base change transitive}.
\end{proof}
\begin{corollary}\label{Noe local algebra over field complete tensor with complete prop}
Let $k$ be a field, $A,B$ be Noetherian local rings containing $k$, with maximal ideals $\m,\n$, respectively. Suppose that $B$ is complete and that the residue field $A//\m$ is a finite extension of $k$. Let $C$ be the complete tensor product $A\hat{\otimes}_kB$.
\begin{itemize}
\item[(a)] $C$ is a complete Noetherian semi-local ring.
\item[(b)] $C$ is a flat $A$-module and a flat $B$-module.
\item[(c)] $\m C$ is contained in the Jacobson radical of $C$ and $C/\m C$ is isomorphic to $(A/\m)\otimes_kB$.
\item[(d)] The residue fields of $C$ are finite extensions of that of $B$.
\end{itemize}
Moreover, if $A'$ is a finite local $A$-algebra containing $A$, the canonical homomorphism $A\hat{\otimes}_kB\to A'\hat{\otimes}_kB$ is injective.
\end{corollary}
\begin{proof}

\end{proof}
\subsection{Flatness of graded modules}
Let $A$ be a ring and $\mathfrak{I}$ an ideal of $A$. An $A$-module $M$ is called \textbf{ideally Hausdorff with respect to $\mathfrak{I}$} (or simply ideally Hausdorff if there is no ambiguity) if, for every finitely generated ideal $\a$ of $A$, the $A$-module $\a\otimes_AM$ is Hausdorff with the $\mathfrak{I}$-adic topology.
\begin{example}[\textbf{Examples of ideally Hausdorff modules}]
\mbox{}
\begin{itemize}
\item[(a)] If $A$ is Noetherian and $\mathfrak{I}$ is contained in the Jacobson radical of $A$ (in other words if $A$ is a Zariski ring with the $\mathfrak{I}$-adic topology), every finitely generated $A$-module is ideally Hausdorff by \cref{Zariski ring def}.
\item[(b)] Every direct sum of ideally Hausdorff modules is an ideally Hausdorff module, by virtue of the observation
\[\mathfrak{I}^n(\a\otimes_A\bigoplus_{i\in I}M_i)=\mathfrak{I}^n\bigoplus_{i\in I}(\a\otimes_AM_i)=\bigoplus_{i\in I}\mathfrak{I}^n(\a\otimes_AM_i).\]
\item[(c)] If an $A$-module $M$ is flat and Hausdorff with the $\mathfrak{I}$-adic topology, it is ideally Hausdorff, for $\a\otimes_AM$ is then identified with a submodule of $M$ and the $\mathfrak{I}$-adic topology on $\a\otimes_AM$ is finer than the topology induced on $\a\otimes_AM$ by the $\mathfrak{I}$-adic topology on $M$, which is Hausdorff by hypothesis. 
\end{itemize}
\end{example}
Let $A$ be a ring, $\mathfrak{I}$ an ideal of $A$, $M$ an $A$-module and $\gr(A)$ and $\gr(M)$ the graded ring and graded $\gr(A)$-module associated respectively with the ring $A$ and with the module $M$ with the $\mathfrak{I}$-adic filtrations. We have seen that for every nonngegative integer $n$ there is a surjective $\Z$-module homomorphism (see (\ref{filtration gr(M) generated by gr_0(M)}))
\[\gamma_n:(\mathfrak{I}^n/\mathfrak{I}^{n+1})\otimes_{A/\mathfrak{I}}(M/\mathfrak{I}M)\to \mathfrak{I}^nM/\mathfrak{I}^{n+1}M\]
and a graded homomorphism of degree $0$ of graded $\gr(A)$-modules
\[\gamma_M:\gr(A)\otimes_{\gr_0(A)}\gr_0(M)\to\gr(M)\]
whose restriction to $\gr_n(A)\otimes\gr_0(M)$ is $\gamma_n$ for all $n$ and which is therefore surjective.
\begin{proposition}\label{filtration module CR condition and product with power}
Let $A$ be a ring, $\mathfrak{I}$ an ideal of $A$ and $M$ an $A$-module. The following conditions are equivalent:
\begin{itemize}
\item[(\rmnum{1})] For all $n>0$, $\Tor_1^A(A/\mathfrak{I}^n,M)=0$.
\item[(\rmnum{2})] For all $n>0$, the canonical homomorphism $\theta_n:\mathfrak{I}^n\otimes_AM\to \mathfrak{I}^nM$ is bijective.
\end{itemize}
Moreover these conditions imply:
\begin{itemize}
\item[(\rmnum{3})] The canonical homomorphism $\gamma_M:\gr(A)\otimes_{\gr_0(A)}\gr_0(M)\to\gr(M)$ bijective. 
\end{itemize}
Conversely, if $\mathfrak{I}$ is nilpotent, then (\rmnum{3}) implies (\rmnum{1}) and (\rmnum{2}).
\end{proposition}
\begin{proof}
The equivalence of (\rmnum{1}) and (\rmnum{2}) follows from the exact sequence
\[\begin{tikzcd}
\Tor_1^A(A,M)\ar[r]&\Tor_1^A(A/\mathfrak{I}^n,M)\ar[r]&\mathfrak{I}^n\otimes_AM\ar[r]&A\otimes_AM
\end{tikzcd}\]
Consider next the diagram
\begin{equation}\label{filtration module CR condition and product with power-1}
\begin{tikzcd}[row sep=20pt, column sep=20pt]
&\mathfrak{I}^{n+1}\otimes_AM\ar[r]\ar[d,"\theta_{n+1}"]&\mathfrak{I}^n\otimes_AM\ar[r]\ar[d,"\theta_n"]&(\mathfrak{I}^n/\mathfrak{I}^{n+1})\otimes_A(M/\mathfrak{I}M)\ar[d,"\gamma_n"]\ar[r]&0\\
0\ar[r]&\mathfrak{I}^{n+1}M\ar[r]&\mathfrak{I}^nM\ar[r]&\gr_n(M)\ar[r]&0
\end{tikzcd}
\end{equation}
where we note that $(\mathfrak{I}^n/\mathfrak{I}^{n+1})\otimes_A(M/\mathfrak{I}M)$ is canonically identified with $(\mathfrak{I}^{n}/\mathfrak{I}^{n+1})\otimes_{A/\mathfrak{I}}(M/\mathfrak{I}M)$. This diagram is commutative by definition of $\gamma_n$, and its rows are exact. If (\rmnum{2}) holds, $\theta_n$ and $\theta_{n+1}$ are bijective and so therefore is $\gamma_n$, by definition of cokernel, hence (\rmnum{2}) implies (\rmnum{3}).\par
Conversely, assuming that $\mathfrak{I}$ is nilpotent, let us show that (\rmnum{3}) implies (\rmnum{2}). We shall argue by descending induction on $n$, since $\mathfrak{I}^n\otimes_AM=0$ for $n$ sufficiently large. Suppose then that in diagram (\ref{filtration module CR condition and product with power-1}), $\gamma_n$ and $\theta_{n+1}$ are bijective. Then so is $\theta_n$ by the snake lemma, and the induction proves $\theta_n$ is bijective for all $n$.
\end{proof}
\begin{theorem}\label{filtration module and flatness}
Let $A$ be a ring, $\mathfrak{I}$ an ideal of $A$ and $M$ an $A$-module. Consider the following properties:
\begin{itemize}
\item[(\rmnum{1})] $M$ is a flat $A$-module.
\item[(\rmnum{2})] $\Tor_1(N,M)=0$ for every $A$-module $N$ annihilated by $\mathfrak{I}$.
\item[(\rmnum{3})] $M/\mathfrak{I}M$ is a flat $A/\mathfrak{I}$-module and the canonical map $\mathfrak{I}\otimes_AM\to \mathfrak{I}M$ is bijective.
\item[(\rmnum{4})] $M/\mathfrak{I}M$ is a flat $A/\mathfrak{I}$-module and the canonical map $\gamma_M:\gr(A)\otimes_{\gr_0(A)}\gr_0(M)\to\gr(M)$ is bijective.
\item[(\rmnum{5})] For all $n>0$, $M/\mathfrak{I}^nM$ is a flat $(A/\mathfrak{I}^n)$-module. 
\end{itemize}
Then (\rmnum{1})$\Rightarrow$(\rmnum{2})$\Leftrightarrow$(\rmnum{3})$\Rightarrow$(\rmnum{4})$\Leftrightarrow$(\rmnum{5}). If further $\mathfrak{I}$ is nilpotent or if $A$ is Noetherian and $M$ is ideally Hausdorff, then these properties are equivalent.
\end{theorem}
\begin{proof}
The implication (\rmnum{1})$\Rightarrow$(\rmnum{2}) is immediate. Also, condition (\rmnum{2}) is equivalent to the following:
\begin{itemize}
\item[(\rmnum{2}')] $\Tor_1^A(N,M)=0$ for every $A$-module $N$ annihilated by a power of $\mathfrak{I}$.
\end{itemize}
In fact, if (\rmnum{2}) holds, then then in particular $\Tor_1^A(\mathfrak{I}^n/\mathfrak{I}^{n+1},M)=0$ for all $n$. From the exact sequence
\[\begin{tikzcd}
0\to \mathfrak{I}^{n+1}N\ar[r]&\mathfrak{I}^nN\ar[r]&\mathfrak{I}^nN/\mathfrak{I}^{n+1}N\ar[r]&0
\end{tikzcd}\]
wc derive the exact sequence
\[\begin{tikzcd}
\Tor_1^A(\mathfrak{I}^{n+1}N,M)\ar[r]&\Tor_1^A(\mathfrak{I}^{n}N,M)\ar[r]&\Tor_1^A(\mathfrak{I}^nN/\mathfrak{I}^{n+1}N,M)
\end{tikzcd}\]
and, as there exists an integer $m$ such that $\mathfrak{I}^mN=0$, we deduce by descending induction on $n$ that $\Tor_1^A(\mathfrak{I}^nN,M)=0$ for all $n<m$ and in particular for $n=0$. It follows from this that if $\mathfrak{I}$ is nilpotent, (\rmnum{2}) implies (\rmnum{1}), for (\rmnum{2}') then means that $\Tor_1^A(N,M)=0$ for every $A$-module $N$ and hence that $M$ is flat. Note that the equivalence (\rmnum{2})$\Leftrightarrow$(\rmnum{3}) is a special case of \cref{Tor functor zero extension and restriction} applied to $R=A$, $S=A/\mathfrak{I}$, $F=M$, $E=N$, taking account of the fact that an $(A/\mathfrak{I})$-module $N$ is the same as an $A$-module $N$ annihilated by $\mathfrak{I}$.\par
Now if (\rmnum{2}) holds, so does (\rmnum{2}') and \cref{filtration module CR condition and product with power} shows that $\gamma_M$ is an isomorphism. On the other hand, we already know that (\rmnum{2}) implies (\rmnum{3}) and hence $M/\mathfrak{I}M$ is a flat $(A/\mathfrak{I})$-module, which completes the proof that (\rmnum{2}) implies (\rmnum{4}). Also, \cref{filtration module CR condition and product with power} shows that, if $\mathfrak{I}$ is nilpotent, (\rmnum{4}) implies (\rmnum{3}). Taking account of the equivalence of (\rmnum{2}) and (\rmnum{3}), we have therefore proved in this case that (\rmnum{1}), (\rmnum{2}), (\rmnum{3}) and (\rmnum{4}) are equivalent.\par
We prove the equivalence of (\rmnum{4}) and (\rmnum{5}). For all $n>0$, $M/\mathfrak{I}^nM$ has a canonical $(A/\mathfrak{I}^n)$-module structure. If it is filtered by the $(\mathfrak{I}/\mathfrak{I}^n)$-adic filtration, it is immediate that
\[\gr_m(M/\mathfrak{I}^nM)=\begin{cases}
\gr_m(M)&m<n\\
0&m\geq n.
\end{cases}
\]
For all $k>0$, let $A_k=A/\mathfrak{I}^k$, $\mathfrak{I}_k=\mathfrak{I}/\mathfrak{I}^k$, and $M_k=M/\mathfrak{I}^kM$. Let (\rmnum{4}$)_k$ (resp. (\rmnum{5}$)_k$) denote the assertion derived from (\rmnum{4}) (resp. (\rmnum{5})) by replacing $A$, $\mathfrak{I}$, $M$ by $A_k$, $\mathfrak{I}_k$, $M_k$. It follows from what has just been said that (\rmnum{4}) is equivalent to "for all $k>0$, (\rmnum{4}$)_k$" and obviously (\rmnum{5}) is equivalent to "for all $k>0$, (\rmnum{5}$)_k$". Then it will suffice to establish the equivalence (\rmnum{4}$)_k$ and (\rmnum{5}$)_k$ for all $k$ or also to show that (\rmnum{4})$\Leftrightarrow$(\rmnum{5}) when $\mathfrak{I}$ is nilpotent. Now we have seen that in that case (\rmnum{4}) is equivalent to (\rmnum{1}), and (\rmnum{1}) is clearly equivalent to (\rmnum{5}). We have therefore shown the equivalence (\rmnum{4})$\Leftrightarrow$(\rmnum{5}) in all cases and also that of all the properties of the theorem in the case where $\mathfrak{I}$ is nilpotent.\par
Finaly we show the implication (\rmnum{5})$\Rightarrow$(\rmnum{1}) when $A$ is Noetherian and $M$ ideally Hausdorff. It is sufficient to prove that for every ideal $\a$ of $A$ the canonical map $j:\a\otimes_AM\to M$ is injective. As $\a\otimes_AM$ is Hausdorff with the $\mathfrak{I}$-adic topology, it suffices to verify that $\ker j\sub \mathfrak{I}^n(\a\otimes_AM)$ for every integer $n>0$. By \cref{filtration I-topo induce on submodule}, there exists an integer $k$ such that $\a_k:=\a\cap \mathfrak{I}^k\sub \mathfrak{I}^n\a$. Now, denoting by $\iota:\a_k\to\a$, $\pi:\a\to\a/\a_k$ and $\psi:\a/\a_k\to A/\mathfrak{I}^k$ the canonical maps, there is a commutative diagram
\[\begin{tikzcd}[column sep=20pt,row sep=15pt]
\a_k\otimes_AM\ar[r,"\iota\otimes 1_M"]&\a\otimes_AM\ar[d]\ar[r,"\pi\otimes 1_M"]&(\a/\a_k)\otimes_AM\ar[d,"\psi\otimes 1_M"]\ar[r]&0\\
&M\ar[r]&(A/\mathfrak{I}^k)\otimes_AM
\end{tikzcd}\]
in which the first row is exact. The map $\psi\otimes 1_M$ is injective since it can also be written into
\[\psi\otimes 1_{M/\mathfrak{I}^kM}:(\a/\a_k)\otimes_{A/\mathfrak{I}^k}(M/\mathfrak{I}^kM)\to M/\mathfrak{I}^kM\]
and, as $\psi$ is injective and by (\rmnum{5}) $M/\mathfrak{I}^kM$ is a flat $(A/\mathfrak{I}^k)$-module. By a simple diagram chasing we see $\ker j\sub\im(\iota\otimes 1_M)$, which shows the claim and completes the proof.
\end{proof}
\begin{proposition}\label{Noe ring radical extension finite module is ideally Hausdorff}
Let $A$ be a ring, $\mathfrak{I}$ an ideal of $A$ and $B$ a Noetherian $A$-algebra such that $\mathfrak{I}$ is contained in the Jacobson radical of $B$. Then every finitely generated $B$-module $M$ is an ideally Hausdorff $A$-module with respect to $\mathfrak{I}$.
\end{proposition}
\begin{proof}
Wc shall see more generally that for every finitely generated $A$-module $N$, $N\otimes_AM$ is Hausdorff with the $\mathfrak{I}$-adic topology. Let $\r$ be the Jacobson radical of $B$. As $\mathfrak{I}B$ is contained in $\r$ and the $B$-module $N\otimes_AM$ is canonically identified with $N_{(B)}\otimes_BM$ by virtue of the associativity of the tensor product, the $\mathfrak{I}$-adic topology on $N\otimes_AM$ is therefore identified with a finer topology than the $\r$-adic topology on $N_{(B)}\otimes_BM$. But this latter topology is Hausdorff since $N_{(B)}\otimes_BM$ is a finitely generated $B$-module, whence the conclusion.
\end{proof}
\begin{proposition}\label{Noe ring ideally Hausdorff module flat iff}
Let $A$ be a ring, $B$ an $A$-algebra, $\mathfrak{I}$ an ideal of $A$ and $M$ a $B$-module. Suppose that $B$ is a Noetherian ring and a flat $A$-module and that $M$ is ideally Hausdorff with respect to $\mathfrak{I}B$. The following conditions are equivalent:
\begin{itemize}
\item[(\rmnum{1})] $M$ is a flat $B$-module.
\item[(\rmnum{2})] $M$ is a flat $A$-module and $M/\mathfrak{I}M$ is a flat $(B/\mathfrak{I}B)$-module.
\end{itemize}
If further the canonical homomorphism $A/\mathfrak{I}\to B/\mathfrak{I}B$ is bijective, then these are also equivalent to:
\begin{itemize}
\item[(\rmnum{3})] $M$ is a flat $A$-module.
\end{itemize}
\end{proposition}
\begin{proof}
Condition (\rmnum{1}) implies (\rmnum{2}) by \cref{module flat extension is flat} and \cref{module flat restriction is flat} and the fact that $M/\mathfrak{I}M$ is isomorphic to $M\otimes_BB/\mathfrak{I}B$. Suppose condition (\rmnum{2}) holds. To show that $M$ is a flat $B$-module, we shall apply \cref{filtration module and flatness} with $A$ replaced by $B$ and $\mathfrak{I}$ by $\mathfrak{I}B$. It will therefore be sufficient to show that the canonical map $\phi:\mathfrak{I}B\otimes_BM\to \mathfrak{I}M$ is injective. Let $\phi_1$ be the canonical map $\mathfrak{I}\otimes_AB\to \mathfrak{I}B$ and $\phi_2$ the canonical isomorphism $\mathfrak{I}\otimes_AM\to(\mathfrak{I}\otimes_AB)\otimes_BM$. Then
\[\begin{tikzcd}
\mathfrak{I}\otimes_AM\ar[r,"\phi_2"]&(\mathfrak{I}\otimes_AB)\otimes_BM\ar[r,"\phi_1\otimes 1_M"]&\mathfrak{I}B\otimes_BM\ar[r,"\phi"]&\mathfrak{I}M
\end{tikzcd}\]
Now $\psi:\mathfrak{I}\otimes_AM\to \mathfrak{I}M$ is an isomorphism since $M$ is a flat $A$-module, whilst $\phi_1$ is an isomorphism because $B$ is flat over $A$. Therefore $\phi$ is then an isomorphism.\par
Let $\rho:A/\mathfrak{I}\to B/\mathfrak{I}B$ be the canonical homomorphism. The $(A/\mathfrak{I})$-module structure on $M/\mathfrak{I}M$ derived by means of $\rho$ is isomorphic to that on $M\otimes_A(A/\mathfrak{I})$. Then it follows that, if $M$ is a flat $A$-module, $M/\mathfrak{I}M$ is a flat $(A/\mathfrak{I})$-module and hence also a flat $(B/\mathfrak{I}B)$-module if $\rho$ is an isomorphism. We have thus proved that (\rmnum{3})$\Rightarrow$(\rmnum{2}) in that case.
\end{proof}
\begin{corollary}\label{Noe ring ideally Hausdorff module flat iff completion}
Let $A$ be a Noetherian ring, $\mathfrak{I}$ an ideal of $A$, $\widehat{A}$ the Hausdorff completion of $A$ with respect to the $\mathfrak{I}$-adic topology and $M$ an ideally Hausdorff $\widehat{A}$-module with respect to $\widehat{\mathfrak{I}}$. For $M$ ta be a flat $A$-module, it is necessary and sufficient that $M$ be a flat $\widehat{A}$-module.
\end{corollary}
\begin{proof}
We know in fact that $A$ is a Noetherian ring and a flat $A$-module, that $\widehat{\mathfrak{I}}=\mathfrak{I}\widehat{A}$ and that the canonical homomorphism $A/\mathfrak{I}\to\widehat{A}/\widehat{\mathfrak{I}}$ is bijective. \cref{Noe ring ideally Hausdorff module flat iff} can therefore be applied.
\end{proof}
\begin{proposition}\label{Noe ring module flat iff completion is completion flat}
Let $A$ and $B$ be two Noetherian rings, $\rho:A\to B$ a ring homomorphism, $\mathfrak{I}$ an ideal of $A$ and $\mathfrak{K}$ an ideal of $B$ containing $\mathfrak{I}B$ and contained in the Jacobson radical of $B$. Let $\widehat{A}$ be the Hausdorff completion of $A$ with respect to the $\mathfrak{I}$-adic topology and $B$ the Hausdorff completion of $B$ with respect to the $\mathfrak{K}$-adic topology. Let $M$ be a finitely generated $B$-module and $\widehat{M}$ its Hausdorff completion with respect to the $\mathfrak{K}$-adic topology. Then the following properties are equivalent:
\begin{itemize}
\item[(\rmnum{1})] $M$ is a flat $A$-module.
\item[(\rmnum{2})] $\widehat{M}$ is a flat $A$-module.
\item[(\rmnum{3})] $\widehat{M}$ is a flat $\widehat{A}$-module.
\end{itemize}
\end{proposition}
\begin{proof}
As $B$ with the $\mathfrak{K}$-adic topology is a Zariski ring, $\widehat{B}$ is a faithfully flat $B$-module and $\widehat{M}$ is canonically isomorphic to $M\otimes_B\widehat{B}$. It is immediately verified that this canonical isomorphism is an isomorphism of the $A$-module structure on $\widehat{M}$ onto the $A$-module structure on $M\otimes_B\widehat{B}$ derived from that on $M$. Applying \cref{module tensor faithfully flat iff}, we see that for $M$ to be a flat $A$-module, it is necessary and sufficient that $\widehat{M}$ be a flat $A$-module. Moreover, $\widehat{M}$ is a finitely generated $B$-module and $\mathfrak{I}\widehat{B}$ is contained in $\widehat{\mathfrak{K}}=\mathfrak{K}\widehat{B}$ and hence in the Jacobson radical of $\widehat{B}$ (\cref{filtration I-adic completion maximal ideal}). Therefore $\widehat{M}$ is an ideally Hausdorff $\widehat{A}$-module with respect to $\widehat{\mathfrak{I}}$ by \cref{Noe ring radical extension finite module is ideally Hausdorff}. Conditions (\rmnum{2}) and (\rmnum{3}) are therefore equivalent by the Corollary to \cref{Noe ring ideally Hausdorff module flat iff}.
\end{proof}
\section{Exercise}
\begin{exercise}\label{p-adic right exact}
Show that $p$-adic completion is not a right-exact functor on the category of all $\Z$-modules.
\end{exercise}
\begin{proof}
Let $A=(\Z/p\Z)^{\oplus\N}$ and $B=\bigoplus_{n}\Z/p^n\Z$. Consider the short exact sequence 
\[\begin{tikzcd}
0\ar[r]&A\ar[r,hook,"\iota"]&B\ar[r,"\pi"]&B/A\ar[r]&0
\end{tikzcd}\]
If we give them the topology induced by $B$, namely $\{\iota^{-1}(p^nB)\}$ and $\{\pi(p^nB)\}$, then by \cref{filtration completion is exact if induced}  the induced conplition sequence is exact. Note that on $B/\alpha(A)$, the induced topology is just the same as the $p$-adic topology. So if we instead use the $p$-adic topology, the sequence is not exact.
\end{proof}
\begin{remark}
Though it does preserve surjectivity, because if $\rho:B\to C$ is a surjection, we have 
\[\rho(p^nB)=p^n\rho(B)=p^nC,\] 
so we have a surjective map  of surjective inverse systems, and \cref{filtration completion is exact if induced}  tells us the map $\widehat{\rho}:\widehat{B}\to\widehat{C}$ is surjective.\par 
The $p$-adic completion is not left-exact, by the same example; the essential reason is that given a general homomorphism $\alpha:A\to B$, we needn't have $\alpha(p^nA)=\alpha(A)\cap p^nB$, so the $p$-adic topology on $A$ is not that induced from $B$ and the hypotheses of \cref{filtration completion is exact if induced}  are not met.
\end{remark}
\begin{exercise}
In \cref{p-adic right exact}, let $A_n=\iota^{-1}(p^nB)$, and consider the exact sequence
\[\begin{tikzcd}
0\ar[r]&A_n\ar[r]&A\ar[r]&A/A_n\ar[r]&0
\end{tikzcd}\]
Show that $\llim$ is not right exact, and compute $\llim\nolimits^1A_n$.
\end{exercise}
\begin{proof}
From \cref{p-adic right exact} we already have
\[A_n=\bigoplus_{j>n}G_j,\quad A/A_n=\bigoplus_{j=1}^{n}G_j\]
where $G_i=\Z/p\Z$ for all $i$. The maps 
\[A_{n+1}\hookrightarrow A_n,\quad A\stackrel{\id}{\to} A,\quad A/A_{n+1}\rightarrowtail A/A_n\]
give rise to an inverse system, hence we can form the inverse limits, and by \cref{inverse limit derived exact sequence} we get an exact sequence
\[\begin{tikzcd}[column sep=small]
0\ar[r]&\llim A_n\ar[r]&\llim A\ar[r]&\llim A/A_n\ar[r]&\llim\nolimits^1A_n\ar[r]&\llim\nolimits^1A\ar[r]&\llim\nolimits^1A/A_n\ar[r]&0
\end{tikzcd}\]
Since $\{A\}$ and $\{A/A_n\}$ are surjective inverse systems, $\llim\nolimits^1A=\llim\nolimits^1A/A_n=0$. Also, the inverse limits can be calculate to be
\[\llim A_n=0,\quad \llim A=A=\bigoplus_{j=1}^{n}G_j,\quad \llim A/A_n=\prod_{i=1}^{\infty}G_j\]
Hence we get a short exact sequence
\[\begin{tikzcd}
0\ar[r]&\bigoplus_{j=1}^{n}G_j\ar[r,"\psi"]&\prod_{j=1}^{\infty}G_j\ar[r]&\llim\nolimits^1A_n\ar[r]&0
\end{tikzcd}\]
Then the map $\psi$ is simply the inclusion, and we see $\llim\nolimits^1A_n=\prod_{j=1}^{\infty}G_j/\bigoplus_{j=1}^{n}G_j$.
\end{proof}
\begin{exercise}
Let $A$ be a Noetherian ring and let $\a,\b$ be ideals in $A$. If $M$ is any $A$-module, let $M^\a$, $M^\b$ denote its $\a$-adic and $\b$-adic completions respectively. If $M$ is finitely generated, prove that $(M^\a)^\b\cong M^{\a+\b}$.
\end{exercise}
\begin{proof}
Take the $\a$-adic complition of $\begin{tikzcd}[column sep=15pt]0\ar[r]&\b^mM\ar[r]&M\ar[r]&M/\b^mM\ar[r]&0\end{tikzcd}$, we get
\[\begin{tikzcd}
0\ar[r]&(\b^mM)^\a\ar[r]&M^\a\ar[r]&(M/\b^mM)^\a\ar[r]&0
\end{tikzcd}\]
By \cref{filtration Noe I-adic completion is tensor} we have
\begin{align*}
(\b^mM)^\a&\cong A^\a\otimes_A \b^mM=A^\a\otimes_A \b^m\otimes_AM=A^\a\otimes_A M\otimes_A\b^m\\
&\cong \b^m\otimes M^\a=\b^mM^\a
\end{align*}
so from the sequence we get $(M/\b^mM)^\a\cong M^\a/\b^mM^\a$. Now we have
\[(M^\a)^\b=\llim_{m}M^\a/\b^mM^\a=\llim_{m}(M/\b^mM)^\a=\llim_{m}\llim_{n}\frac{M/\b^mM}{\a^n(M/\b^mM)}.\]
Since we have
\[\a^n(M/\b^mM)\cong\frac{\a^nM+\b^mM}{\b^mM}.\]
we get
\[(M^\a)^\b=\llim_{m}\llim_{n}M/(\a^nM+\b^mM)\cong \llim_{n}M/(\a^nM+\b^nM)\]
Now note that
\[(\a+\b)^{2n}\sub \a^n+\b^n\sub (\a+\b)^n\]
so the topology induced by $\a^nM+\b^nM$ is the same as that by $(\a+\b)^n$, hence we get the desired result.
\end{proof}
\begin{exercise}
Let $A$ be the local ring of the origin in $\C^n$ (i.e., the ring of all rational functions $f/g\in\C(z_1,\dots,z_n)$ with $g(0)\neq0)$, let $B$ be the ring of power series in $z_1,\dots,z_n$ which converge in some neighborhood of the origin, and let $C$ be the ring of formal power series in $z_1,\dots,z_n$, so that $A\sub B\sub C$. Show that $B$ is a local ring and that its completion for the maximal ideal topology is $C$. Assuming that $B$ is Noetherian, prove that $B$ is $A$-flat.
\end{exercise}
\begin{proof}
The surjective ring homomorphism 
\[\phi:B\to\C,\quad f\mapsto f(0)\] 
shows the ideal $\m=(z_1,\dots,z_n)$ of power series with zero constant term is maximal. To show $B$ is local, note that if $f\notin\m$, then $f(0)\neq 0$ and $f$ is a unit in $C$, and thus a unit in $B$. So $\m$ is maximal.\par
We first compute the complition, note that
\[\C[z_1,\dots,z_n]\sub A\sub B\sub C\]
and that $C$ is the complition of $\C[z_1,\dots,z_n]$ by the $(z_1,\dots,z_n)$-adic topology. Thus we have
\[C=\widehat{\C[z_1,\dots,z_n]}\sub\widehat{A}\sub\widehat{B}\sub\widehat{C}=C\]
So $\widehat{A}=\widehat{B}=C$. Since we know $C$ is a faithfully flat $B$-algebra and $A$-algebra, we conclude $B$ is a flat $A$-algebra.
\end{proof}
\chapter{Associated prime ideals and primary decomposition}
\section{Associated prime ideals of a module}
\subsection{Associated prime ideals}
Let $M$ be a module over a ring $A$. A prime ideal $\p$ is said to be \textbf{associated with $\bm{M}$} if there exists $x\in M$ such that $\p$ is equal to the annihilator of $x$. The set of prime ideals associated with $M$ is denoted by $\Ass_A(M)$, or simply $\Ass(M)$.
\begin{example}
Let $\a$ be an ideal in the polynomial ring $A=\C[X_1,\dots,X_n]$, $V$ the corresponding affine algebraic variety and $V_1,\dots,V_p$ the irreducible components of $V$. If $M$ is taken to be the ring $A/\a$ of functions which are regular on $V$, the set of prime ideals associated with $M$ consists of the ideals of $V_1,\dots,V_p$ and in general other prime ideals each of which contains one of the ideals of the $V_i$.
\end{example}
As the annihilator of $0$ is $A$, an element $x\in M$ whose annihilator is a prime ideal is necessarily nonzero. To say that a prime idea $\p$ is associated with $M$ amounts to saying that $M$ contains a submodule isomorphic to $A/\p$ (namely $Ax$, for all $x\in M$ whose annihilator is $\p$). If an $A$-module $M$ is the union of a family $(M_i)_{i\in I}$ of submodules, then clearly
\begin{align}\label{associated prime union of modules}
\Ass(M)=\bigcup_{i\in I}\Ass(M_i).
\end{align}
\begin{proposition}\label{associated prime of A/p}
For every prime ideal $\p$ of a ring $A$ and every nonzero submodule $M$ of $A/\p$, $\Ass(M)=\{\p\}$.
\end{proposition}
\begin{proof}
As the ring $A/\p$ is an integral domain, the annihilator of an nonzero element of $A/\p$ is $\p$.
\end{proof}
\begin{proposition}\label{associated prime maximal element of Ann}
Let $M$ be a module over a ring $A$. Every maximal element of the set of ideals $\Ann(x)$ of $A$, where $x$ runs through the set of nonzero elements of $M$, belongs to $\Ass(M)$.
\end{proposition}
\begin{proof}
Let $\p=\Ann(x)$ (where $x\in M$ is nonzero) be such a maximal element; it is sufficient to show that $\p$ is prime. As $x\neq 0$, $\p$ is proper. Let $a,b$ be elements of $A$ such that $ab\in\p$ but $a\notin\p$. Then $ax\neq 0$ and $\p\sub\Ann(ax)$. As $\p$ is maximal, $\Ann(ax)=\p$, whence $b\in\p$, so that $\p$ is prime.
\end{proof}
\begin{corollary}\label{associated prime empty iff}
Let $M$ be a module over a Noetherian ring $A$. Then the condition $M\neq\{0\}$ is equivalent to $\Ass(M)\neq\emp$.
\end{corollary}
\begin{proof}
If $M=\{0\}$, clearly $\Ass(M)$ is empty (without any hypothesis on $A$). If $M\neq\{0\}$, the set of ideals of the form $\Ann(x)$, where $x\in M$ and $x\neq 0$, is non-empty and consists of ideals of $A$. As $A$ is Noetherian, this set has a maximal element; then it suffices to apply \cref{associated prime maximal element of Ann}.
\end{proof}
\begin{corollary}\label{associated prime and homothety injective}
Let $A$ be a Noetherian ring, $M$ an $A$-module and $a\in A$. For the homothety on $M$ with ratio $a$ to be injective, it is necessary and sufficient that $a$ belong to no prime ideal associated with $M$.
\end{corollary}
\begin{proof}
If $a$ belongs to a prime ideal $\p\in\Ass(M)$, then $\p=\Ann(x)$ where $x\in M$ and $x\neq 0$, whence $ax=0$ and the homothety with ratio $a$ is not injective. Conversely, if $ax=0$ for some $x\in M$ such that $x\neq 0$, then $Ax\neq\{0\}$, whence $\Ass(Ax)\neq\emp$. Let $\p\in\Ass(Ax)$, then obviously $\p\in\Ass(M)$ and $\p=\Ann(bx)$ for some $b\in A$; then $a\in\p$, since $abx=0$.
\end{proof}
\begin{corollary}\label{Noe ring divisor of zero union of associated prime}
The set of divisors of zero in a Noetherian ring $A$ is the union of the ideals in $\Ass(A)$.
\end{corollary}
\begin{proposition}\label{associated prime and exact sequence}
If we have a short exact sequence of $A$-modules
\[\begin{tikzcd}
0\ar[r]&L\ar[r]&M\ar[r]&N\ar[r]&0
\end{tikzcd}\]
then $\Ass(L)\sub\Ass(M)\sub\Ass(L)\cup\Ass(N)$.
\end{proposition}
\begin{proof}
The inclusion $\Ass(L)\sub\Ass(M)$ is obvious. Let $\p\in\Ass(M)$, $E$ be a submodule of $M$ isomorphic to $A/\p$ and $F=E\cap L$. If $F=\{0\}$, $E$ is isomorphic to a submodule of $N$, whence $\p\in\Ass(N)$. If $F\neq\{0\}$, the annihilator of every nonzero element of $F$ is $\p$ (\cref{associated prime maximal element of Ann}) and hence $\p\in\Ass(F)\sub\Ass(L)$.
\end{proof}
\begin{corollary}\label{associated prime of direct sum}
If $M=\bigoplus_{i\in I}M_i$ then $\Ass(M)=\bigcup_{i\in I}\Ass(M_i)$.
\end{corollary}
\begin{proof}
It may be reduced to the case where $I$ is finite by means of (\ref{associated prime union of modules}), then to the case where $|I|=2$ by induction. Then let $I=\{i,j\}$, where $i\neq j$. As $M/M_i$ is isomorphic to $M_j$, we have $\Ass(M)\sub\Ass(M_i)\cup\Ass(M_j)$. Moreover, $\Ass(M_i)$ and $\Ass(M_j)$ are contained in $\Ass(M)$ by \cref{associated prime and exact sequence}, whence the result.
\end{proof}
\begin{corollary}\label{associated prime of union of zero intersection module}
Let $M$ be an $A$-module and $(Q_i)_{i\in I}$ a finite family of submodules of $M$. If $\bigcap_{i\in I}Q_i=\{0\}$ then
\[\Ass(M)\sub\bigcup_{i\in I}\Ass(M/Q_i).\] 
\end{corollary}
\begin{proof}
The canonical map $M\to\bigoplus_i(M/Q_i)$ is injective; then it suffices to apply \cref{associated prime and exact sequence} and \cref{associated prime of direct sum}.
\end{proof}
\begin{proposition}\label{associated prime submodule with given subset}
Let $M$ be an $A$-module and $\Phi$ a subset of $\Ass(M)$. Then there exists a submodule $N$ of $M$ such that $\Ass(N)=\Ass(M)-\Phi$ and $Ass(M/N)=\Phi$.
\end{proposition}
\begin{proof}
Let $\mathcal{M}$ be the set of submodules $P$ of $M$ such that $\Ass(P)\sub\Ass(M)-\Phi$. Formula (\ref{associated prime union of modules}) shows that the set $\mathcal{M}$, ordered by inclusion, is inductive. Moreover, $\{0\}\in\mathcal{M}$ and hence $\mathcal{M}\neq\emp$. Let $N$ be a maximal element of $\mathcal{M}$. Then $\Ass(N)\sub\Ass(M)-\Phi$. We shall see that $\Ass(M/N)\sub\Phi$, which, by \cref{associated prime and exact sequence}, will complete the proof. Let $\p\in\Ass(M/N)$; then $M/N$ contains a submodule $F/N$ isomorphic to $A/\p$. By \cref{associated prime and exact sequence} and \cref{associated prime of A/p}, $\Ass(F)\sub\Ass(N)\cup\{\p\}$. Since $N$ is maximal in $\mathcal{M}$, $F\notin\mathcal{M}$ and hence $\p\in\Phi$.
\end{proof}
\begin{proposition}\label{associated prime of localization}
Let $A$ be a ring, $S$ a multiplicative subset of $A$, $\Phi$ the set of ideals of $A$ which do not meet $S$ and $M$ an $A$-module. Then:
\begin{itemize}
\item[(a)] The map $\p\mapsto S^{-1}\p$ is a bijection of $\Ass_A(M)\cap\Phi$ onto a subset of $\Ass_{S^{-1}A}(S^{-1}M)$.
\item[(b)] If $\p\in\Phi$ is a finitely generated ideal and $S^{-1}\p\in\Ass_{S^{-1}A}(S^{-1}M)$, then $\p\in\Ass_A(M)$.
\end{itemize}
\end{proposition}
\begin{proof}
Recall that the map $\p\mapsto S^{-1}\p$ is a bijection of $\Phi$ onto the set of prime ideals of $S^{-1}A$. If $\p\in\Ass_A(M)\cap\Phi$, $\p$ is the annihilator of a monogenous submodule $N$ of $M$; then $S{^-1}\p$ is the annihilator of the monogenous submodule $S^{-1}\N$ of $S^{-1}M$ (\cref{localization and Ann}) and hence $S^{-1}\p\in\Ass_{S^{-1}A}(S^{-1}M)$. Conversely, suppose that $\p\in\Phi$ is finitely generated and such that $S^{-1}\p$ is associated with $S^{-1}M$, then there exists $x\in M$ and $t\in S$ such that $S^{-1}\p$ is the annihilator of $x/t$. Let $a_1,\dots,a_n$ be a system of generators of $\p$; then $(a_i)(x/t)=0$ and hence there exists $s_i\in S$ such that $s_ia_ix=0$. Let us write $s=s_1\cdots s_n$, then for all $a\in\p$ we have $sax=0$, whence $\p\sub\Ann(sx)$. On the other hand, if $b\in A$ satisfies $bsx=0$, then $b/1\in S^{-1}\p$ by definition, whence $b\in\p$. Then $\p=\Ann(sx)$ and $\p\in\Ass_A(M)$.
\end{proof}
\begin{corollary}
If the ring $A$ is Noetherian, the map $\p\mapsto S^{-1}\p$ is a bijection of $\Ass_A(M)\cap\Phi$ onto $\Ass_{S^{-1}A}(S^{-1}M)$.
\end{corollary}
\begin{proposition}\label{associated prime and kernel of localization}
Let $A$ be a Noetherian ring, $M$ an $A$-module, $S$ a multiplicative subset of $A$ and $\Phi$ the set of elements of $\Ass_A(M)$ which do not meet $S$. Then the kernel $N$ of the canonical map $M\mapsto S^{-1}M$ is the unique submodule of $M$ which satisfies the relations
\[\Ass(N)=\Ass(M)-\Phi,\quad \Ass(M/N)=\Phi.\]
\end{proposition}
\begin{proof}
By \cref{associated prime submodule with given subset}, there exists a submodule $N'$ of $M$ which satisfies the relations $\Ass(N')=\Ass(M)-\Phi$ and $\Ass(M/N')=\Phi$. We need to prove $N'=N$. Consider the commutative diagram
\[\begin{tikzcd}
M\ar[r,"\pi"]\ar[d,swap,"i_M^S"]&M/N'\ar[d,"i_{M/N'}^S"]\\
S^{-1}M\ar[r,"S^{-1}\pi"]&S^{-1}(M/N')
\end{tikzcd}\]
where $\pi$, $i_M^S$, and $i_{M/N'}^S$ are the canonical homomorphisms. We shall show that $S^{-1}\pi$ and $i_{M/N'}^S$ are injective, which will prove that $i_M^S$ and $\pi$ have the same kernel and hence $N=N'$.\par
As $\Ass(N')\cap \Phi=\emp$, every element of $\Ass(N')$ meets $S$. Then by \cref{associated prime of localization} we have $\Ass_{S^{-1}A}(S^{-1}N')=\emp$, whence $S^{-1}N'=\{0\}$ by \cref{associated prime empty iff}. This proves that $S^{-1}\pi$ is injective. On the other hand, if $x$ belongs to the kernel $K$ of $i_{M/N'}^S$, then $\Ann(x)\cap S\neq\emp$, so $\Ass(K)\cap\Phi=\emp$. But $\Ass(K)\sub\Ass(M/N')=\Phi$, which means $\Ass(K)=\emp$ and $K=\{0\}$ by \cref{associated prime empty iff}. This shows $i_{M/N'}^S$ is injective.
\end{proof}
Let $M$ be a module over a ring $A$. Recall that the set of prime ideals $\p$ of $A$ such that $M_\p\neq 0$ is called the \textit{support} of $M$ and is denoted by $\supp(M)$.
\begin{proposition}\label{associated prime and supp}
Let $A$ be a ring and $M$ an $A$-module.
\begin{itemize}
\item[(a)] Every prime ideal $\p$ of $A$ containing an element of $\Ass(M)$ belongs to $\supp(M)$.
\item[(b)] Conversely, if $A$ is Noetherian, every ideal $\p\in\supp(M)$ contains an element of $\Ass(M)$.
\end{itemize}
\end{proposition}
\begin{proof}
If $\p$ contains an element $\q$ of $\Ass(M)$, then $\q\cap(A-\p)=\emp$ and hence, if we write $S=A-\p$, $S^{-1}\p$ is a prime ideal associated with $S^{-1}M=M_\p$, and a fortiori $M_\p\neq\emp$, hence $\p\in\supp(M)$. Conversely, if $A$ is Noetherian, so is $A_\p$. If $M_\p\neq\emp$, then $\Ass_{A_\p}(M_\p)\neq\emp$ and hence there exists $\q\in\Ass_A(M)$ such that $\q\cap(A-\p)=\emp$ by \cref{associated prime of localization}.
\end{proof}
\begin{corollary}\label{associated prime ideal contained in supp}
If $M$ is a module over a Noetherian ring, then $\Ass(M)\sub\supp(M)$ and these two sets have the same minimal elements.
\end{corollary}
\begin{corollary}
The nilradical of a Noetherian ring $A$ is the intersection of the ideals $\p\in\Ass(A)$.
\end{corollary}
\subsection{Associated prime ideals of Noetherian rings}
\begin{proposition}\label{associated prime of Noe chain of submodule}
Let $A$ be a Noetherian ring and $M$ a finitely generated $A$-module. Then there exists a chain $(M_i)_{0\leq i\leq n}$ of submodules of $M$ such that $M_{i}/M_{i+1}$ is isomorphic to $A/\p_i$ where $\p_i$ is a prime ideal of $A$.
\end{proposition}
\begin{proof}
Let $\mathcal{M}$ be the set of submodules of $M$ which have a chain with the property of the statement. As $\mathcal{M}$ is non-empty (for $\{0\}$ belongs to $\mathcal{M}$) and $M$ is Noetherian, $\mathcal{M}$ has a maximal element $N$. If $M\sub N$, then $M/N\neq\emp$ and hence $\Ass(M/N)\neq\emp$ by \cref{associated prime empty iff}. The module $M/N$ therefore contains a submodule $N'/N$ isomorphic to an $A$-module of the form $A/\p$, where $\p$ is prime; then by definition $N'\in\mathcal{M}$, which contradicts the maximal character of $N$. Then necessarily $N=M$.
\end{proof}
\begin{theorem}\label{associated prime of Noe prime ideals in composition}
Let $M$ be a finitely generated module over a Noetherian ring $A$ and $(M_i)_{0\leq i\leq n}$ a chain of submodules of $M$ such that $M_{i}/M_{i+1}$ is isomorphic to $A/\p_i$ where $\p_i$ is a prime ideal of $A$. Then
\begin{align}\label{associated prime of Noe prime ideals in composition-1}
\Ass(M)\sub\{\p_0,\dots,\p_{n-1}\}\sub\supp(M).
\end{align}
The minimal elements of these three sets are the same and coincide with the minimal elements of the set of prime ideals containing $\Ann(M)$.
\end{theorem}
\begin{proof}
The inclusion $\Ass(M)\sub\{\p_0,\dots,\p_{n-1}\}$ follows immediately from \cref{associated prime and exact sequence}. For each $i$, we have
\[\p_i\in\supp(A/\p_i)=\supp(M_i/M_{i+1})\]
so $\p_i\in\supp(M_i)\sub\supp(M)$ (\cref{supp of module and exact sequence}), which shows $\{\p_0,\dots,\p_{n-1}\}\sub\supp(M)$. \cref{associated prime and supp} shows that $\Ass(M)$ and $\supp(M)$ have the same minimal elements and (\ref{associated prime of Noe prime ideals in composition-1}) shows that these are just the minimal elements of $\{\p_0,\dots,\p_{n-1}\}$. The last assertion then follows from \cref{supp of finite module is V(Ann)}.
\end{proof}
\begin{corollary}\label{associated prime ideal finite if Noe finite}
If $M$ is a finitely generated module over a Noetherian ring $A$, $\Ass(M)$ is finite.
\end{corollary}
\begin{proposition}
Let $A$ be a Noetherian ring, $\a$ an ideal of $A$ and $M$ a finitely generated $A$-module. The following conditions are equivalent:
\begin{itemize}
\item[(\rmnum{1})] there exists a nonzero element $x\in M$ such that $\a x=0$;
\item[(\rmnum{2})] for all $a\in\a$, there exists a nonzero element $x\in M$ such that $ax=0$;
\item[(\rmnum{3})] there exists $\p\in\Ass(M)$ such that $\a\sub\p$. 
\end{itemize}
\end{proposition}
\begin{proof}
Clearly (\rmnum{1}) implies (\rmnum{2}). By virtue of \cref{associated prime and homothety injective}, condition (\rmnum{2}) means that the ideal $\a$ is contained in the union of the prime ideals associated with $M$ and hence in one of them since $\Ass(M)$ is finite (\cref{prime ideal contained in union}), thus (\rmnum{2}) implies (\rmnum{3}). Finally, if there exists $\p\in\Ass(M)$ such that $\a\sub\p$, then $\p$ is the annihilator of a nonzero element $x\in M$ such that $\a x=0$, thus (\rmnum{3}) implies (\rmnum{1}).
\end{proof}
\begin{proposition}\label{associated prime intersection and a^nM=0}
Let $A$ be a Noetherian ring, $\a$ an ideal of $A$ and $M$ a finitely generated $A$-module. For there to exist an integer $n>0$ such that $\a^nM=0$, it is necessary and sufficient that $\a$ be contained in the intersection of the prime ideals associated with $M$.
\end{proposition}
\begin{proof}
This intersection is also that of the minimal elements of $\supp(M)$ and to say that $\a$ is contained in this intersection is equivalent to saying that $V(\a)\sups\supp(M)$. The conclusion then follows from \cref{supp of module contained in V(a) iff}.
\end{proof}
Given an $A$-module $M$, an endomorphism $\phi$ of $M$ is called \textbf{locally nilpotent} if, for all $x\in M$, there exists an integer $n(x)>0$ such that $\phi(x)^{n(x)}=0$. It is clear that, if $M$ is finitely generated, every locally nilpotent endomorphism is nilpotent.
\begin{corollary}\label{associated prime intersection and locally nilpotent of homothety}
Let $A$ be a Noetherian ring, $M$ an $A$-module and $a$ an element of $A$. For the homomorphism $h_a:x\mapsto ax$ of $M$ to be locally nilpotent, it is necessary and suficient that $a$ belong to every ideal of $\Ass(M)$.
\end{corollary}
\begin{proof}
The condition of the statement is equivalent to saying that for all $x\in M$ there exists $n(x)>0$ such that $(Aa)^{n(x)}(Ax)=0$. By \cref{associated prime intersection and a^nM=0} this means also that $a$ belongs to all the prime ideals associated with the submodule $Ax$ of $M$. The corollary then follows from the fact that $\Ass(M)$ is the union of the $\Ass(Ax)$ where $x$ runs through $M$.
\end{proof}
\begin{proposition}\label{associated prime for Hom set}
Let $A$ be a Noetherian ring, $M$ a finitely generated $A$-module and $N$ an $A$-module. Then
\[\Ass(\Hom_A(M,N))=\Ass(N)\cap\supp(M).\]
\end{proposition}
\begin{proof}
By hypothesis, $M$ is isomorphic to an $A$-module of the form $A^n/R$, hence $\Hom_A(M,N)$ is isomorphic to a submodule of $\Hom_A(A^n,N)$ and the latter is isomorphic to $N^n$. Now, $\Ass(N^n)=\Ass(N)$ by \cref{associated prime of direct sum}, thus $\Ass(\Hom_A(M,N))\sub\Ass(N)$. On the other hand, for every prime ideal $\p$ of $A$, $\Hom_{A_\p}(M_\p,N_\p)$ is isomorphic to $(\Hom_A(M,N))_\p$ by \cref{localization and Hom set if finite presented}
\[\supp(\Hom_A(M,N))\sub\supp(M).\]
Then we conclude from \cref{associated prime of Noe prime ideals in composition} that $\Ass(\Hom_A(M,N))\sub\supp(M)$.\par
Conversely, let $\p$ be a prime ideal of $A$ belonging to $\Ass(N)\cap\supp(M)$. By definition, $N$ contains a submodule isomorphic to $A/\p$. On the other hand, since $M$ is finitely generated and $M_\p\neq 0$, we know that there exists a nonzero homomorphism $\eta:M\to A/\p$ (\cref{supp of module and nonzero homomorphism to A/p}). As there exists an injective homomorphism $\iota:A/\p\to N$, we have $\iota\circ\eta\in\Hom(M,N)$ and it is nonzero. On the other hand, since the annihilator of every nonzero element of $A/\p$ is $\p$, we see $\Ann(\iota\circ\eta)=\p$, whence $\p\in\Ass(\Hom_A(M,N))$.
\end{proof}
\section{Primary decomposition}
\subsection{Primary submodules}
\begin{proposition}\label{primary submodule def}
Let $A$ be a Noetherian ring and $M$ an $A$-module. Then the following conditions are equivalent:
\begin{itemize}
\item[(\rmnum{1})] $\Ass(M)$ is reduced to a single element.
\item[(\rmnum{2})] $M\neq 0$ and every homothety of $M$ is either injective or locally nilpotent.
\end{itemize}
If these conditions are fulfilled and $\p$ is the set of $a\in A$ such that the homothety of ratio $a$ is locally nilpotent, then $\Ass(M)=\{\p\}$.
\end{proposition}
\begin{proof}
This follows from \cref{associated prime intersection and locally nilpotent of homothety} and \cref{associated prime and homothety injective}.
\end{proof}
Let $A$ be a Noetherian ring, $M$ an $A$-module and $N$ a submodule of $N$. If the module $M/N$ satisfies the conditions of \cref{primary submodule def}, $N$ is called \textbf{$\p$-primary} with respect to $M$ (or in $M$).\par
In particular consider the case $M=A$, where the submodules of $M$ are then the ideals of $A$ and hence an ideal $\q$ of $A$ is called primary if $\Ass(A/\q)$ has a single element or, what amounts to the same, \textit{if $A\neq\q$ and every divisor of zero in the ring $A/\q$ is nilpotent}. If $\q$ is $\p$-primary, then since $\supp(A/\q)=V(\Ann(A/\q))=V(\q)$, by \cref{associated prime of Noe prime ideals in composition} $\p$ is the radical of the ideal $\q$.
\begin{example}[\textbf{Example of primary submodules}]\label{primary submodule eg}
\mbox{}
\begin{itemize}
\item[(a)] If $\p$ is a prime ideal of $A$, then $\p$ is $\p$-primary.
\item[(b)] Let $\q$ be an ideal of $A$ such that there exists a single prime ideal $\m$ (necessarily maximal) containing $\q$. Then, if $M$ is an $A$-module such that $\q M\neq M$, $\q M$ is $\m$-primary with respect to $M$, for every element of $\Ass(M/\q M)$ contains $\q$, hence is equal to $\m$ and $\Ass(M/\q M)\neq\emp$. In particular we see $\q$ is an $\m$-primary ideal in $A$. 
\item[(c)] Let $\m$ be a maximal ideal of a Noetherian ring $A$. Then the $\m$-primary ideals are the ideals $\q$ of $A$ for which there exists an integer $n>0$ such that $\m^n\sub\q\sub\m$, for if $\m^n\sub\q\sub\m$, then $\m$ is the only prime ideal containing $\q$ and the conclusion follows from (b). Conversely, if $\q$ is $\m$-primary, then $\m$ is the radical of $\q$ and there therefore exists $n>1$ such that $\m^n\sub\q$ by \cref{Noe ideal contain power of radical}. In particular, if $A$ is a PID then the primary ideals are $(0)$ and the ideals of the form $(p^n)$, where $p$ is a prime element and $n>0$.
\end{itemize}
\end{example}
\begin{remark}
The powers of any prime ideal are not necessarily primary ideals. For an example, let $K$ be a field and $A$ the quotient ring of $K[X,Y,Z]/(Z^2-XY)$. Let $x$, $y$, $z$ be the canonical images of $X$, $Y$, $Z$ in $A$. Then the ideal $\p=(x,z)$ is prime, since $A/\p\cong K[X]$. Note that $\p^2=(x^2,z^2,xz)=(x^2,xy,xz)$ is not primary, since $x\notin\p^2$, $y\notin\p$ but $xy\in\p^2$. Also, we have the following decomposition
\[\p^2=(x)\cap(x,y,z^2)\]
On the other hand, there exist primary ideals which are not powers of prime ideals. For example, consider the ring $A=\Z[X]$ and the maximal ideal $\m=(2,X)$. The ideal $\q=(4,X)$ is $\m$-primary since $\sqrt{\q}=\m$, but it is not a power of $\m$.
\end{remark}
\begin{proposition}\label{primary submodule intersection is primary}
Let $M$ be a module over a Noetherian ring $A$, $\p$ a prime ideal of $A$ and $(Q_i)_{i\in I}$ a non-empty finite family of submodules of $M$ which are $\p$-primary with respect to $M$. Then $\bigcap_{i\in I}Q_i$ is $\p$-primary with respect to $M$.
\end{proposition}
\begin{proof}
The quotient $M/(\bigcap_iQ_i)$ is isomorphic to a submodule of the direct sum $\bigoplus_i(M_i/Q_i)$. Now $\Ass(\bigoplus_i(M_i/Q_i))=\bigcup_i\Ass(M_i/Q_i)=\{\p\}$, whence $\Ass(M/(\bigcap_iQ_i))=\{\p\}$ and $\bigcap_iQ_i$ is $\p$-primary.
\end{proof}
\begin{proposition}\label{primary module and localization}
Let $A$ be a Noetherian ring, $S$ a multiplicative subset of $A$, $\p$ a prime ideal of $A$, $M$ an $A$-module, $N$ a submodule of $M$ and $i_M^S$ the canonical map of $M$ to $S^{-1}M$.
\begin{itemize}
\item[(a)] Suppose that $\p\cap S\neq\emp$. If $N$ is $\p$-primary with respect to $M$, then $S^{-1}N=S^{-1}M$.
\item[(b)] Suppose that $\p\cap S=\emp$. Then for $N$ to be $\p$-primary with respect to $M$, it is necessary and sufficient that $N$ be of the form $(i_M^S)^{-1}(N')$, where $N'$ is a sub-$S^{-1}A$-module of $S^{-1}M$ which is $(S^{-1}\p)$-primary with respect to $S^{-1}M$. In this case we have $N'=S^{-1}N$.
\end{itemize}
\end{proposition}
\begin{proof}
If $\p\cap S\neq\emp$ and $N$ is $\p$-primary with respect to $M$, then
\[\Ass_{S^{-1}A}(S^{-1}(M/N))=\emp\]
and hence $S^{-1}(M/N)=0$, whence $S^{-1}N=S^{-1}M$.\par
Suppose that $\p\cap S=\emp$. If $N$ is $\p$-primary with respect to $M$, then $\Ass_{S^{-1}A}(S^{-1}(M/N))=\{S^{-1}\p\}$ and hence the submodule $N'=S^{-1}N$ of $S^{-1}M$ is $(S^{-1}\p)$-primary. Moreover, if $s\in S$ and $m\in M$ are such that $sm\in N$, then $m\in N$, for the homothety with ratio $s$ in $M/N$ is injective, whence $N=(i_M^S)^{-1}(N')$.\par
Conversely, let $N'$ be a submodule of $S^{-1}M$ which is $(S^{-1}\p)$-primary with respect to $S^{-1}M$. Let us write $N=(i_M^S)^{-1}(N')$, then $N'=S^{-1}N$ and
\[\Ass_{S^{-1}A}(S^{-1}(M/N))=\Ass_{S^{-1}A}((S^{-1}M)/N')=\{S^{-1}\p\}.\]
Since $N$ is saturated, the canonical map $M/N\to S^{-1}(M/N)$ is injective, whence no prime ideal of $A$ associated with $M/N$ meets $S$ (\cref{associated prime and kernel of localization}). It then follows that $\Ass(M/N)=\{\p\}$, so that $N$ is $\p$-primary with respect to $M$.
\end{proof}
\subsection{Primary decompositions}
Let $A$ be a Noetherian ring, $M$ an $A$-module and $N$ a submodule of $M$. A finite family $(Q_i)_{i\in I}$ of submodule of $M$ which are primary with respect to $M$ and such that $N=\bigcap_{i}Q_i$ is called a \textbf{primary decomposition} of $N$ in $M$.
\begin{example}
Let us take $A=\Z$, $M=\Z$, $N=(n)$ for some integer $n>0$. If $n=p_1^{\alpha_1}\cdots p_k^{\alpha_k}$ is the decomposition of $n$ into prime factors, then
\[(n)=(p_1^{\alpha_1})\cap\cdots\cap(p_k^{\alpha_k})\]
is a primary decomposition of $(n)$ in $\Z$.
\end{example}
By an abuse of language, the relation $N=\bigcap_{i\in I}Q_i$ is called a primary decomposition of $N$ in $M$. It amounts to the same to say that $\{0\}=\bigcap_{i\in I}(Q_i/N)$ is a primary decomposition of $\{0\}$ in $M/N$. If $(Q_i)_{i\in I}$ is a primary decomposition of $N$ in $M$, the canonical map from $M/N$ to $\bigcap_{i\in I}(M/Q_i)$ is injective. Conversely let $N$ be a submodule of $M$ and $\phi$ an injective homomorphism from $M/N$ to a finite direct sum $P=\bigoplus_iP_i$, where each set $\Ass(P_i)$ is reduced to a single element $\p_i$. Let $\phi_i$ be the homomorphism $M/N\to P_i$ obtained by taking the composition off with the projection $P\to P_i$, and let $Q_i/N$ be the kernel of $\phi_i$; then the $Q_i$ distinct from $M$ are primary with respect to $M$ and $N=\bigcap_{i\in I}Q_i$. Moreover, $\Ass(M/N)\sub\bigcup_{i\in I}\{\p_i\}$ by virtue of \cref{associated prime and exact sequence}.
\begin{theorem}\label{primary decomposition module exist}
Let $M$ be a finitely generated module over a Noetherian ring and let $N$ be a submodule of $M$. Then there exists a primary decomposition of $N$ of the form
\begin{align}
N=\bigcap_{\p\in\Ass(M/N)}Q(\p)
\end{align}
where for each $\p\in\Ass(M/N)$, $Q(\p)$ is $\p$-primary with respect to $M$.
\end{theorem}
\begin{proof}
We may replace $M$ by $M/N$ and therefore suppose that $N=0$. By \cref{associated prime ideal finite if Noe finite}, $\Ass(M)$ is finite. By \cref{associated prime submodule with given subset}, there exists, for each $\p\in\Ass(M)$, a submodule $Q(\p)$ of $M$ such that $\Ass(M/Q(\p))=\{\p\}$ and $\Ass(Q(\p))=\Ass(M)-\{\p\}$. Let us write $P=\bigcap_{\p\in\Ass(M)}Q(\p)$. For all $\p\in\Ass(M)$, $\Ass(P)\sub\Ass(Q(\p))$ and hence $\Ass(P)=\{\p\}$, which implies $P=0$ and therefore proves the theorem.
\end{proof}
Let $M$ be a module over a Noetherian ring and $N$ a submodule of $M$. A primary decomposition $N=\bigcap_{i\in I}Q_i$ of $N$ in $M$ is called \textbf{reduced} if the following conditions are fulfilled:
\begin{itemize}
\item[(a)] there exists no index $i\in I$ such that $\bigcap_{j\neq i}Q_j\sub Q_i$;
\item[(b)] if $\Ass(M/Q_i)=\{\p_i\}$, the $\p_i$'s are distinct.
\end{itemize}
From every primary decomposition $N=\bigcap_{i\in I}Q_i$ of $N$ in $M$ a reduced primary decomposition of $M$ in $N$ can be deduced as follows: let $J$ be a minimal element of the set of subsets $I'$ of $I$ such that $N=\bigcap_{i\in I'}Q_i$. Clearly $(Q_i)_{i\in J}$ satisfies condition (a). Then let $\Phi$ be the set of $\p_i$ for $i\in J$. For all $\p\in\Phi$, let $H(\p)$ be the set of $i\in J$ such that $\p_i=\p$ and let $Q(\p)=\bigcap_{i\in H(\p)}Q_i$. It follows from \cref{primary submodule intersection is primary} that $Q(\p)$ is $\p$-primary with respect to $M$; further $N=\bigcap_{\p\in\Phi}Q(\p)$ and the family $(Q(\p))_{\p\in\Phi}$ is therefore a reduced primary decomposition of $N$ in $M$. 
\begin{proposition}\label{primary decomposition reduced iff}
Let $M$ be a module over a Noetherian ring, $N$ a submodule of $M$, $N=\bigcap_{i\in I}Q_i$ a primary decomposition of $N$ in $M$ and, for all $i\in I$, let $\{\p_i\}=\Ass(M/Q_i)$. For this decomposition to be reduced, it is necessary and sufficient that the $\p_i$ be distinct and belong to $\Ass(M/N)$. In this case, we have
\begin{align}\label{primary decomposition reduced iff-1}
\Ass(M/N)=\bigcup_{i\in I}\p_i,\quad \Ass(Q_i/N)=\bigcup_{j\neq i}\{\p_i\}\quad\text{for all $i\in I$}.
\end{align}
\end{proposition}
\begin{proof}
If the condition of the statement is fulfilled, $N=\bigcap_{j\neq i}Q_j$ cannot hold, for we would deduce that $\Ass(M/N)\sub\bigcup_{j\neq i}\{\p_j\}$ by \cref{associated prime of union of zero intersection module}, contrary to the hypothesis. The primary decomposition $(Q_i)_{i\in I}$ of $N$ is then certainly reduced.\par
Conversely, assume that $(Q_i)_{i\in I}$ is reduced. Note that $\Ass(M/N)\sub\bigcup_{i\in I}\{\p_i\}$ always holds by \cref{associated prime of union of zero intersection module}. On the other hand, for all $i\in I$, let us write $P_i=\bigcap_{j\neq i}Q_j$. Then $P_i\cap Q_i=N$ and $P_i\neq N$ if $(Q_i)_{i\in I}$ is reduced, hence $P_i/N$ is non-zero and is isomorphic to the submodule $(P_i+Q_i)/Q_i$ of $M/Q_i$, whence $\{\p_i\}=\Ass(P_i/N)$ by \cref{associated prime and exact sequence}. As $P_i/N\sub M/N$, we get $\p_i\in\Ass(M/N)$, which completes the proof of the necessity of the condition in the statement and the first formula of (\ref{primary decomposition reduced iff-1}). Finally, for each $i\in I$, as $N=\bigcap_{j\neq i}(Q_i\cap Q_j)$, by \cref{associated prime of union of zero intersection module} we have
\[\Ass(Q_i/N)\sub\bigcup_{j\neq i}\Ass(Q_i/(Q_i\cap Q_j))=\bigcup_{j\neq i}\Ass((Q_i+Q_j)/Q_j)\sub\bigcup_{j\neq i}\{\p_j\}.\]
Taking account of the first formula of (\ref{primary decomposition reduced iff-1}) and \cref{associated prime and exact sequence}, the second formula of (\ref{primary decomposition reduced iff-1}) follows easily.
\end{proof}
\begin{corollary}\label{primary decomposition card iff reduced}
Let $A$ be a Noetherian ring, $M$ an $A$-module, $N$ a submodule of $M$ and $(Q_i)_{i\in I}$ a primary decomposition of $N$ in $M$. Then $|I|\geq|\Ass(M/N)|$, and the equality holds if and only if $(Q_i)_{i\in I}$ is reduced.
\end{corollary}
\begin{proof}
It follows from the remarks preceding \cref{primary decomposition reduced iff} that there exists a reduced primary decomposition $(R_i)_{i\in J}$ of $N$ in $M$ such that $\mathrm{Card}(J)\sub\mathrm{Card}(I)$. The first assertion then follows from the second and the latter is a consequence of \cref{primary decomposition reduced iff}.
\end{proof}
\begin{proposition}\label{primary decomposition minimal part and saturation}
Let $A$ be a Noetherian ring, $M$ an $A$-module and $N$ a submodule of $M$. Let $N=\bigcap_{i\in I}Q_i$ be a reduced primary decomposition of $N$ where $Q_i$ is $\p_i$-primary. If $\p_i$ is a minimal element of $\Ass(M/N)$, then $Q_i$ is equal to the saturation on $N$ with respect to $\p_i$.
\end{proposition}
\begin{proof}
We can obviously restrict our attention to the case where $N=0$, replacing if need be $M$ by $M/N$. If $\p_i$ is minimal in $\Ass(M)$, then the set of elements of $\Ass(M)$ which do not meet $A-\p_i$ reduces to $\p_i$. The proposition then follows from the second formula of (\ref{primary decomposition reduced iff-1}) above and \cref{associated prime and kernel of localization}, since the kernel of the canonical map $M\mapsto M_{\p_i}$ equals to the saturation of $0$ with respect to $\p_i$.
\end{proof}
The minimal elements of $\Ass(M/N)$ are called the \textbf{minimal} or \textbf{isolated prime ideals} associated with $M/N$, and the prime ideals $\p\in\Ass(M/N)$ which are not minimal are called the \textbf{embedded prime ideals} associated with $M/N$. If $M/N$ is finitely generated, then by \cref{associated prime ideal contained in supp} we know the isolated ideals of $M/N$ are just the minimal prime ideals containing $\supp(M/N)=V(\Ann(M/N))$, so for $\p\in\Ass(M/N)$ to be embedded, it is necessary and sufficient that $V(\p)$ be not an irreducible component of $\supp(M/N)$.\par
If $(Q(\p))$ and $(R(\p))$ are two reduced primary decompositions of $N$ in $M$, then by \cref{primary decomposition minimal part and saturation} we see $Q(\p)=R(\p)$ if $\p$ is an isolated prime associated with $M/N$, being the saturation of $N$ with respect to $\p$. However, it may happen that $Q(\p)\neq R(\p)$ for $\p$ an embedded prime, as the following example shows.
\begin{example}
Consider the ring $A=K[X,Y]$ where $K$ is a field. Then for every positive integer $n$, a primary decomposition of the ideal $\a=(X^2,XY)$ is
\[(X^2,XY)=(X)\cap (X^2,XY,Y^n)\]
where $(X)$ is prime and $(X^2,XY,Y^n)$ is $(X,Y)$-primary. Note that $(X)$ is isolated and $(X,Y)$ is embedded.
\end{example}
Given a submodule $N$ of a module $M$ over a Noetherian ring $A$, to simplify we shall denote by $D_I(M/N)$, in this part, the set of reduced primary decompositions of $N$ in $M$ whose indexing set is $I$.
\begin{proposition}\label{primary decomposition and localization}
Let $A$ be a Noetherian ring, $M$ an $A$-module, $N$ a submodule of $M$ and $I=\Ass(M/N)$. Let $S$ be a multiplicative subset of $A$ and $I_S$ the subset $I$ consisting of the indices $i$ such that $S\cap\p_i=\emp$. Let $\phi(N)$ be the saturation of $N$ with respect to $S$ in $M$. Then:
\begin{itemize}
\item[(a)] If $(Q_i)_{i\in I}$ is an element of $D_I(M/N)$, the family $(Q_i)_{i\in I_S}$ is an element of $D_{I_S}(M/\phi(N))$ and the family $(S^{-1}Q_i)_{i\in I_S}$ is an element of $D_{I_S}(S^{-1}M/S^{-1}N)$.
\item[(b)] The map $(Q_i)_{i\in I_S}\mapsto (S^{-1}Q_i)_{i\in I_S}$ is a bijection from $D_{I_S}(M/\phi(N))$ to $D_{I_S}(S^{-1}M/S^{-1}N)$.
\item[(c)] If $(Q_i)_{i\in I_S}$ is an element of $D_{I_S}(M/\phi(N))$ and $(R_i)_{i\in I}$ is an element of $D_I(M/N)$, then the family $(T_i)_{i\in I}$ defined by $T_i=Q_i$ when $i\in I_S$ and $T_i=R_i$ for $i\in I-I_S$ is an element of $D_I(M/N)$.
\end{itemize}
\end{proposition}
\begin{proof}
We know that for $i\in I_S$, $S^{-1}Q_i$ is $S^{-1}\p_i$-primary and that for $i\in I-I_S$, $S^{-1}Q_i=S^{-1}M$. As $S^{-1}N=\bigcap_{i\in I}S^{-1}Q_i$, we see $S^{-1}N=\bigcap_{i\in I_S}S^{-1}Q_i$. The ideals $S^{-1}\p_i$ for $i\in I_S$ are distinct and their set is $\Ass(S^{-1}M/S^{-1}N)$ by \cref{associated prime of localization}. Then by \cref{primary decomposition reduced iff} $(S^{-1}Q_i)_{i\in I_S}$ is a reduced primary decomposition of $S^{-1}N$. Moreover, $Q_i$ is saturated for $i\in I_S$, hence $\phi(N)=(i_M^S)^{-1}(S^{-1}N)=\bigcap_{i\in I_S}Q_i$, and $(Q_i)_{i\in I_S}$ is obviously a reduced primary decomposition of $\phi(N)$ in $M$.\par
As $S^{-1}\phi(N)=S^{-1}N$, we may replace $N$ by $\phi(N)$ and suppose that $I=I_S$. Let $(P_i)_{i\in I}$ be a reduced primary decomposition Of $S^{-1}N$ in $S^{-1}M$ and let us write $Q_i=(i_M^S)^{-1}(P_i)$. It follows from \cref{primary module and localization} that $Q_i$ is $\p_i$-primary and $(Q_i)_{i\in I}$ then a reduced primary decomposition of $N$ of $M$ by virtue of Corollaries~\ref{primary decomposition card iff reduced}. Finally, since $I=I_S$, by \cref{primary module and localization} we see every $\p_i$-primary submodule of $M$ is saturated, so the two maps
\[D_I(M/N)\to D_I(S^{-1}M/S^{-1}N)\And D_I(S^{-1}M/S^{-1}N)\to D_I(M/N)\] are inverse of each other, which proves (b).\par
Finally, we prove (c). From (a) we have $\phi(N)=\bigcap_{i\in I_S}R_i$, whence
\[N=\phi(N)\cap\bigcap_{i\in I-I_S}R_i=\Big(\bigcap_{i\in I_S}Q_i\Big)\cap\Big(\bigcap_{i\in I-I_S}R_i\big)\]
and it follows immediately from \cref{primary decomposition card iff reduced} that this primary decomposition is reduced.
\end{proof}
\begin{corollary}
The maps
\[D_I(M/N)\to D_{I_S}(M/\phi(N))\And D_I(M/N)\to D_{I_S}(S^{-1}M/S^{-1}N)\]
defined in \cref{primary decomposition and localization} are surjective.
\end{corollary}
\begin{proof}
\cref{primary decomposition and localization}(c) shows that the map $D_I(M/N)\to D_{I_S}(M/\phi(N))$ is suijective and \cref{primary decomposition and localization}(b) then shows that $D_I(M/N)\to D_{I_S}(M/N)\to D_{I_S}(S^{-1}M/S^{-1}N)$ is surjective.
\end{proof}
\subsection{Finite length modules}
If an $A$-module $M$ is of finite length, we shall denote this length by $\ell_A(M)$ or $\ell(M)$. Recall that every Artinian ring is Noetherian and that every finitely generated module over an Artinian ring is of finite length.
\begin{proposition}\label{associated prime maximal iff finite length}
Let $M$ be a finitely generated module over a Noetherian ring $A$. The following properties are equivalent:
\begin{itemize}
\item[(\rmnum{1})] $M$ is of finite length.
\item[(\rmnum{2})] Every ideal $\p\in\Ass(M)$ is a maximal ideal of $A$.
\item[(\rmnum{3})] Every ideal $\p\in\supp(M)$ is a maximal ideal of $A$.
\end{itemize}
\end{proposition}
\begin{proof}
Let $(M_i)$ be a chain of submodules of $M$ such that $M_i/M_{i+1}$ is isomorphic to $A/\p_i$ where $\p_i$ is prime. If $M$ is of finite length, so is each of the $A$-modules $A/\p_i$, which implies that each of the rings $A/\p_i$ is Artinian. But as $A/\p_i$ is an integral domain, it is therefore a field, in other words $\p_i$ is maximal; we conclude that (\rmnum{1}) implies (\rmnum{2}). Condition (\rmnum{2}) implies (\rmnum{3}) by \cref{associated prime and supp}. Finally, if all the ideals of $\supp(M)$ are maximal, so are the $\p_i$, hence the $A/\p_i$ are simple $A$-modules and $M$ is of finite length, which completes the proof.
\end{proof}
\begin{corollary}\label{associated prime of finite length is supp}
For every module of finite length $M$ over a Noetherian ring $A$, we have $\Ass(M)=\supp(M)$.
\end{corollary}
\begin{proof}
Every element of $\supp(M)$ is then minimal in $\supp(M)$ and the conclusion follows from \cref{associated prime ideal contained in supp}.
\end{proof}
\begin{corollary}\label{associated prime of finite length localization iff}
Let $M$ be a finitely generated module over a Noetherian ring $A$ and $\p$ a prime ideal of $A$. For $M_\p$ to be a non-zero $A_\p$-module of finite length, it is necessary and sufficient that $\p$ be a minimal element of $\Ass(M)$.
\end{corollary}
\begin{proof}
By \cref{associated prime of localization}, $\Ass_{A_\p}(M_\p)$ is the set of ideals $\q A_\p$, where $\q$ runs through the set of ideals of $\Ass(M)$ which are contained in $\p$. Thus by \cref{associated prime maximal iff finite length}, for $M_\p$ to be an $A_\p$-module of finite length, it is necessary and sufficient that no element of $\Ass(M)$ be strictly contained in $\p$. On the other hand, for $M_\p\neq 0$, it is necessary and sufficient by definition that $\p\in\supp(M)$, that is that $\p$ contain an element of $\Ass(M)$. This proves the corollary.
\end{proof}
\begin{remark}
Note that \cref{associated prime maximal iff finite length} is not equivalent to the condition that each prime ideal $\p\in\Ass(M)$ is \textit{maximal in $\Ass(M)$}. In fact, this condition signifies that each element of $\Ass(M)$ is minimal, so is a minimal prime ideal of $\supp(M)$ (in other words, $M$ has no embedded prime ideals). 
\end{remark}
\begin{remark}\label{Noe ring finite module length of M_p is number of p_i}
Let $M$ be a finitely generated module over a Noetherian ring $A$ and let $(M_i)$ be a chain of $M$ such that $M_i/M_{i+1}$ is isomorphic to $A/\p_i$ where $\p_i$ is a prime ideal of $A$. If $\p$ is a minimal element of $\Ass(M)$, the modules $(M_i)_{\p}$ then form a chain of $M_\p$ and $(M_i)_\p/(M_{i+1})_\p$ is isomorphic to $(A/\p_i)_\p$ and hence to $\{0\}$ if $\p_i\neq\p$, and to $(A/\p)_\p$ which is a field, if $\p_i=\p$.
\end{remark}
\begin{proposition}\label{primary decomposition of finite length module prop}
Let $M$ be a module of finite length over a Noetherian ring $A$.
\begin{itemize}
\item[(a)] There exists a unique primary decomposition $\{0\}=\bigcap_{\p\in\Ass(M)}Q(\p)$ of $\{0\}$ with respect to $M$ indexed by $\Ass(M)$, where $Q(\p)$ is $\p$-primary with respect to $M$.
\item[(b)] There exists an integer $n_0$ such that, for all $n\geq n_0$ and all $\p\in\Ass(M)$, $Q(\p)=\p^nM$.
\item[(c)] For all $\p\in\Ass(M)$, the canonical map of $M$ to $M_\p$ is surjective and its kernel is $Q(\p)$. Therefore $M_\p$ can be identified with $M/Q(\p)$.
\item[(d)] The canonical injection of $M$ into $\bigoplus_{\p\in\Ass(M)}(M/Q(\p))$ is bijective.  
\end{itemize}
\end{proposition}
\begin{proof}
Since each ideal of $M$ is maximal, part (a) follows from \cref{primary decomposition minimal part and saturation}. As $M/Q(\p)$ is finitely generated, by \cref{associated prime intersection and a^nM=0} there exists $n_0>0$ such that $\p^nM\sub Q(\p)$ for all $n\geq n_0$. But as $\p$ is a maximal ideal, $\p^nM$ is $\p$-primary with respect to $M$ and, as $\bigcap_{\p\in\Ass(M)}\p^nM=\{0\}$, it follows from (a) that necessarily $\p^nM=Q(\p)$ for all $\p\in\Ass(M)$, whence (b) follows. As the $\p^n$, for $\p\in\Ass(M)$, are relatively prime in pairs, the canonical map $M\mapsto\bigoplus_{\p\in\Ass(M)}(M/\p^nM)$ is suijective, whence (d).\par
Now $\Ass(Q(\p))=\Ass(M)-\{\p\}$ and $\Ass(M/Q(\p))=\{\p\}$. As the elements of Ass(M) are maximal ideals, $\p$ is the only element of $\Ass(M)$ which does not meet $A-\p$. Then $Q(\p)$ is the kernel of the canonical map $i_\p:M\to M_\p$ by \cref{associated prime submodule with given subset}. If $s\in A-\p$, the homothety of $M/Q(\p)$ with ratio $s$ isinjective by virtue of the relation $\Ass(M/Q(\p))=\{\p\}$. Since $M/Q(\p)$ is Artinian, this homothety is bijective. Therefore, if $m\in M$ and $s\in A-\p$, then there exists $n\in M$ such that $\bar{m}=s\bar{n}$ in $M/Q(\p)$, whence $m-ns\in Q(\p)$. By the characterization of $M_\p$, we can then find $t\in A-\p$ such that $t(m-ns)=0$, that is, $m/s=n/1$ in $M_\p$. This proves the canonical map $M\to M_\p$ is surjective and finishes the proof.
\end{proof}
\begin{corollary}\label{module of finite length equals sum of localization length}
If $M$ is a module of finite length over a Noetherian ring $A$, then
\[\ell_A(M)=\sum_{\p\in\Ass(M)}\ell_{A_\p}(M_\p).\]
\end{corollary}
\begin{proof}
This will follow from \cref{primary decomposition of finite length module prop}(d) if we prove that
\[\ell_A(M/Q(\p))=\ell_{A_\p}(M_\p).\]
Now, since $\Ass(M/Q(\p))=\{\p\}$, for all $s\in A-\p$ the homothety with ratio $s$ on $M/Q(\p)$ is injective; the homothety with ratio $s$ on every submodule $N$ of $M/Q(\p)$ is therefore injective and, as $N$ is Artinian, it is bijective. We conclude that the sub-$A$-modules of $M/Q(\p)$ are the images under the bijection $M_\p\to M/Q(\p)$ of the sub-$A_\p$-modules of $M_\p$, whence our assertion.
\end{proof}
\begin{proposition}\label{associated prime maximal of A iff Artin}
Let $A$ be a Noetherian ring. The following conditions are equivalent:
\begin{itemize}
\item[(\rmnum{1})] $A$ is Artinian.
\item[(\rmnum{2})] All the prime ideals of $A$ are maximal ideals.
\item[(\rmnum{3})] All the prime ideals of $\Ass(A)$ are maximal ideals.
\end{itemize}
If these conditions are fulfilled, $A$ has only a finite number prime ideals, which are all maximal and associated with the $A$-module $A$. Further, $A$ is a semi-local ring and its Jacobson radical is nilpotent.
\end{proposition}
\begin{proof}
To say that $A$ is Artinian is equivalent to saying that $A$ is an $A$-module of finite length, hence (\rmnum{1}) and (\rmnum{2}) are equivalent by \cref{associated prime maximal iff finite length}. We already know that (\rmnum{1}) and (\rmnum{2}) are equivalent by \cref{Noe local is Artin iff power of maximal ideal}.\par
Suppose they hold. As every prime ideal of $A$ belongs to $\supp(A)$ and every element of $\supp(A)$ contains an element of $\Ass(A)$, it follows from (\rmnum{3}) that $\Ass(A)$ is the set ofall prime ideals of $A$. Then $A$ has only a finite number of prime ideals, all of them maximal and associated with the $A$-module $A$. Finally, we know that the Jacobson radical of an Artinian ring is nilpotent.
\end{proof}
\begin{corollary}
Every Artinian ring $A$ is isomorphic to the direct composition of a finite family of Artinian local rings.
\end{corollary}
\begin{proof}
It follows from \cref{associated prime maximal of A iff Artin} and \cref{primary decomposition of finite length module prop} (c) and (d) that, if $(\m_i)$ is the family of maximal ideals of $A$, the canonical map $A\to\prod_iA_{\m_i}$ is bijective.
\end{proof}
\begin{corollary}\label{Zariski ring quotient Artinian iff semilocal}
Let $A$ be a Noetherian ring and $\a$ an ideal of $A$. The following conditions are equivalent:
\begin{itemize}
\item[(\rmnum{1})] $A$ is a semi-local ring and $\a$ is a defining ideal of $A$.
\item[(\rmnum{2})] $A$ is a Zariski ring with the $\a$-adic topology and $A/\a$ is Artinian.
\end{itemize}
\end{corollary}
\begin{proof}
If (a) holds then by \cref{Zariski ring eg} we know $A$ is a Zariski ring with the $\r$-adic topology, where $\r$ is the Jacobson radical of $A$. Moreover, as by hypothesis $\a$ contains a power of $\r$, every prime ideal of $A$ which contains $\a$ also contains $\r$ and is therefore maximal, since $\r$ is a finite intersection of maximal ideals. This shows that $A/\a$ is Artinian.\par
Conversely, if (b) holds, then $\a\sub\r$, so every maximal ideal $\p$ of $A$ containing $\r$ must contain $\a$. As $A/\a$ is Artinian, the ideals $\p/\a$ are finite in number (\cref{associated prime maximal of A iff Artin}) and hence $A$ has only a finite number of maximal ideals, which implies that it is semi-local.
\end{proof}
\begin{corollary}\label{Noe semilocal ring finite extension is Noe semilocal}
Let $\rho:A\to B$ be a ring homomorphism. Suppose that $A$ is semi-local and Noetherian and that $B$ is a finitely generated $A$-module. Then the ring $B$ is semi-local and Noetherian. If $\a$ is a defining ideal of $A$, then $\b=\a^e$ is a defining ideal of $B$.
\end{corollary}
\begin{proof}
We know that $B$ is a Zariski ring with the $\b$-adic topology (\cref{Zariski ring extension of scalar}). As $A/\a$ is Artinian by \cref{Zariski ring quotient Artinian iff semilocal} and $B/\b$ is a finitely generated $(A/\a)$-module, it is an Artinian ring, hence $B$ is semilocal and $\b$ is a defining ideal of $B$ by \cref{Zariski ring quotient Artinian iff semilocal}.
\end{proof}
\begin{corollary}
Let $A$ be a complete semi-local Noetherian ring, $\a$ a defining ideal of $A$, $M$ a finitely generated $A$-module and $(N_n)$ a decreasing sequence of submodules of $M$ such that $\bigcap_nN_n=0$. Then, for all $p>0$, there exists $n>0$ such that $N_n\sub\m^pM$.
\end{corollary}
\begin{proposition}\label{Noe ring minimal prime and Ass localization prop}
Let $A$ be a Noetherian ring and $\p_1,\dots,\p_n$ the prime ideals associated with the $A$-module $A$.
\begin{itemize}
\item[(a)] The set $S=\bigcap_{i=1}^{n}(A-\p_i)$ is the set of elements which are not divisors of $0$ in $A$.
\item[(b)] If all the $\p_i$ are minimal elements of $\Ass(A)$, then the total ring of fractions $S^{-1}A$ of $A$ is Artinian.
\item[(c)] If the ring $A$ is reduced, then all the $\p_i$ are minimal elements of $\Ass(A)$ (and therefore are the minimal prime ideals of $A$) and each of the $A_{\p_i}$ is a field. For each index $i$, the canonical homomorphism $S^{-1}A\to A_{\p_i}$ is surjective and its kernel is $S^{-1}\p_i$. Finally, the canonical homomorphism $S^{-1}A\to\prod_{i=1}^{n}(S^{-1}A/S^{-1}\p_i)\cong\prod_{i=1}^{n}A_{\p_i}$ is bijective.
\end{itemize}
\end{proposition}
\begin{proof}
The fact that $S$ is the set of elements which are not divisors of $0$ in $A$ has already been seen in \cref{associated prime maximal element of Ann}. The prime ideals of $S^{-1}A$ are of the form $S^{-1}\p$, where $\p$ is a prime ideal of $A$ contained in $\bigcup_i\p_i$, and hence is contained in one of the $\p_i$. If $\p_i$ is a minimal element of $\Ass(A)$, it is then a minimal element of $\Spec(A)=\supp(A)$. If each of the $\p_i$ is a minimal element of $\Ass(A)$, we then see that the prime ideals of $S^{-1}A$ are the $S^{-1}\p_i$ and they are therefore all maximal, which proves that $S^{-1}A$ is Artinian.\par
Suppose finally that the ring $A$ is reduced. Then $\bigcap_i\p_i=\{0\}$, and we deduce that $\{0\}=\bigcap_i\p_i$ is a reduced primary decomposition of the ideal $\{0\}$ by \cref{primary decomposition card iff reduced}. In particular, none of the $\p_i$ can contain a $\p_i$ of index $j\neq i$ and therefore the $\p_i$ are all minimal elements of $\Ass(A)$. The ring $S^{-1}A$ is then Artinian by (b). The $S^{-1}\p$ are prime ideals associated with the $S^{-1}A$-module $S^{-1}A$ and $\{0\}=S^{-1}(\bigcap_{i=1}^{n}\p_i)=\bigcap_{i=1}^{n}S^{-1}\p_i$. As the $S^{-1}\p_i$ are distinct, $(S^{-1}\p_i)$ is a reduced primary decomposition of $\{0\}$ in $S^{-1}A$ by \cref{primary decomposition card iff reduced}. \cref{primary decomposition of finite length module prop} then shows that the canonical homomorphism $\phi_i:S^{-1}A\to(S^{-1}A)_{\p_i}$ is surjective and has kernel $S^{-1}\p_i$, and the canonical homomorphism
\[S^{-1}A\to\prod_{i=1}^{n}(S^{-1}A/S^{-1}\p_i)\]
is bijective. We know moreover that the canonical homomorphism $S^{-1}A\to A_{\p_i}$ is given by
\[\begin{tikzcd}
S^{-1}A\ar[r,"\phi_i"]&(S^{-1}A)_{\p_i}\ar[r,"\cong"]&(S^{-1}A)_{S^{-1}\p_i}\ar[r,"\cong"]&A_{\p_i}
\end{tikzcd}\]
Finally, it follows from \cref{primary decomposition of finite length module prop} that $(S^{-1}A)_{S^{-1}\p_i}$ is isomorphic to $S^{-1}A/S^{-1}\p_i$, and hence is a field since $S^{-1}\p_i$ is a maximal ideal.
\end{proof}
\subsection{Extension of scalars}
In this part, $A$ and $B$ will denote two rings and we shall consider a ring homomorphism $\rho:A\to B$ which makes $B$ into an $A$-algebra. Recall that, for every $B$-module $P$, $\rho^*(P)$ is the $A$-module whose structure is defined by $a\cdot y=\rho(a)y$ for all $a\in A$, $y\in P$.
\begin{lemma}\label{associated prime extension scalar contraction prop}
Let $A$ be a Noetherian ring, $\p$ a prime ideal of $A$, $M$ an $A$-module whose annihilator contains a power of $\p$ and such that $\Ass_A(M)=\{\p\}$. Let $P$ be a $B$-module such that $\rho^*(P)$ is a flat $A$-module. The condition $\mathfrak{P}\in\Ass_B(M\otimes_AP)$ then implies $\mathfrak{P}^c=\p$.
\end{lemma}
\begin{proof}
If $n$ is such that $\p^nM=0$, then $\p^nB\sub\Ann(M\otimes_AP)$, whence $\p^nB\sub\mathfrak{P}$, which implics $\p^n\sub\mathfrak{P}^c$ and therefore $\p\sub\mathfrak{P}$ sincen $\mathfrak{P}^c$ is prime. Moreover, if $a\in A-\p$, the homothety $h_a$ with ratio $a$ on $M$ is injective. As $h_a\otimes 1_P$ is the homothety $h_{\rho(a)}$ with ratio $\rho(a)$ on $M\otimes_AP$ and $\rho^*(P)$ is flat, we see $h_{\rho(a)}$ is injective. This proves that $\rho(a)\notin\mathfrak{P}$, whence $\mathfrak{P}^c=\p$.
\end{proof}
\begin{proposition}\label{associated prime extension scalar M otimes N}
Let $\rho:A\to B$ be a ring homomorphism, $M$ an $A$-module and $P$ a $B$-module such that $\rho^*(P)$ is a flat $A$-module. Then
\begin{align}\label{associated prime extension scalar M otimes N-1}
\Ass_B(M\otimes_AP)\sups\bigcup_{\p\in\Ass_A(M)}\Ass_B(P/\p P)
\end{align}
and the equality holds if $A$ is Noetherian.
\end{proposition}
\begin{proof}
Let $\p\in\Ass_A(M)$. By definition there exists an injection $A/\p\to M$, and by tensoring with $P$ we get an injection $P/\p P\to M\otimes_AP$, whence $\Ass_B(P/\p P)\sub\Ass_B(M\otimes_AP)$ and (\ref{associated prime extension scalar M otimes N-1}) follows. Suppose now that $A$ is Noetherian and let us prove the opposite inclusion.\par
Suppose first that $M$ is a finitely generated $A$-module and that $\Ass_A(M)$ is reduced to a single element $\p$. By \cref{associated prime of Noe prime ideals in composition} there exists a chain $(M_i)$ of $M$ such that $M_i/M_{i+1}$ is isomorphic to $A/\p_i$ where $\p_i$ is a prime ideal of $A$ containing $\p$. As $P$ is a flat $A$-module, $(M_i\otimes_AP)$ is then a chain of $M\otimes_AP$ such that
\[(M_i\otimes_AP)/(M_{i+1}\otimes_AP)=(A/\p_i)\otimes_AP=P/\p_iP.\]
By \cref{associated prime and exact sequence}, we get
\[\Ass_B(M\otimes_AP)\sub\bigcup_{i=0}^{n-1}\Ass_B(P/\p_iP).\]
We know that $M$ is annihilated by a power of $\p$ by \cref{associated prime intersection and a^nM=0}, so \cref{associated prime extension scalar contraction prop} shows that $\mathfrak{P}^c=\p$ for all $\mathfrak{P}\in\Ass_B(M\otimes_AP)$. Also, since $P/\p_iP$ is annihilated by $\p_i$, by \cref{associated prime extension scalar contraction prop} we have $(\mathfrak{P}')^c=\p_i$ for all $\mathfrak{P}'\in\Ass_B(P/\p_iP)$. This then shows $\Ass_B(M\otimes_AP)\cap\Ass_B(P/\p_iP)=\emp$ if $\p_i\neq\p$, which proves the claim in this case.\par
Next we drop the assumption $\Ass(M)=\{\p\}$ and assume only that $M$ is finitely generated. Let $\Ass_A(M)=\{\p_1,\dots,\p_n\}$ and let $\{0\}=\bigcap_iQ_i$ be a corresponding reduced primary decomposition of $\{0\}$. Then $M$ is isomorphic to a submodule of the direct sum of the submodules $M_i=M/Q_i$ and, as $f^*(P)$ is a flat $A$-module, $M\otimes_AP$ is isomorphic to a submodule of the direct sum of the submodules $M_i\otimes_AP$. We deduce that
\[\Ass_B(M\otimes_AP)\sub\bigcup_{i=1}^{n}\Ass_B(M_i\otimes_AP).\]
But $M_i$ is a finitely generated $A$-module such that $\Ass_A(M_i)$ is reduced to a single element $\p_i$. By what we have proved, $\Ass_B(M_i\otimes_AP)=\Ass_B(P/\p_iP)$, whence the claim in this case.\par
Now we turn to the general case. The $B$-module $M\otimes_AP$ is the union of the submodules $N\otimes_AP$, where $N$ runs through the set of finitely generated submodules of the $A$-module $M$. If $\mathfrak{P}$ belongs to $\Ass_B(M\otimes_AP)$, then there exists a finitely generated submodule $N$ of $M$ such that $\mathfrak{P}\in\Ass_B(N\otimes_AP)$. By the preceeding argument, there exists $\p\in\Ass_A(N)$ such that $\mathfrak{P}\in\Ass_B(P/\p P)$. As $\Ass_A(N)\sub\Ass_A(M)$, this completes the proof.
\end{proof}
\begin{corollary}\label{associated prime of M otimes N contraction char prop}
In the situation of \cref{associated prime extension scalar M otimes N}, if $A$ is Noetherian and $\mathfrak{P}\in\Ass_B(M\otimes_AP)$, then $\mathfrak{P}^c\in\Ass_A(M)$ and $\mathfrak{P}^c$ is the only prime ideal $\p$ of $A$ such that $\mathfrak{P}\in\Ass_B(P/\p P)$.
\end{corollary}
\begin{proof}
By \cref{associated prime extension scalar M otimes N} we see $\mathfrak{P}\in\Ass_B(P/\p P)=\Ass_B((A/\p)\otimes_AP)$ where $\p\in\Ass_A(M)$. Then \cref{associated prime extension scalar contraction prop} says $\mathfrak{P}^c=\p$.
\end{proof}
\begin{corollary}\label{primary module tensor primary iff}
Suppose that $A$ and $B$ are Noetherian and that $B$ is a flat $A$-module. Let $\p$ be a prime ideal of $A$, $Q$ a $\p$-primary submodule of an $A$-module $M$ and $\mathfrak{P}$ a prime ideal of $B$. For $Q\otimes_AB$ to be a $\mathfrak{P}$-primary submodule of $M\otimes_AB$, it is necessary and sufficient that $\p B$ be a $\mathfrak{P}$-primary ideal of $B$.
\end{corollary}
\begin{proof}
Let us apply \cref{associated prime extension scalar M otimes N} to the $A$-module $M/Q$ and the $B$-module $B$. Then $\Ass_A(M/Q)=\{\p\}$ and $(M/Q)\otimes_AB$ is isomorphic to $(M\otimes_AB)/(Q\otimes_AB)$ and hence $\Ass_B((E\otimes_AB)/(Q\otimes_AB))=\Ass_B(B/\p B)$. To say that $Q\otimes_AB$ is $\mathfrak{P}$-primary in $M\otimes_AB$ therefore means that $\Ass_B(B/\p B)$ is reduced to $\{\mathfrak{P}\}$, whence the corollary.
\end{proof}
\begin{proposition}\label{primary decomposition extension scalar by tensoring}
Suppose that $A$ and $B$ are Noetherian and that $B$ is a flat $A$-module. Let $M$ be an A-module and $N$ a submodule of $M$ such that, for every prime ideal $\p\in\Ass_A(M/N)$, $\p B$ is a prime ideal of $B$ or equal to $B$. Let $N=\bigcap_{\p\in\Ass_A(M/N)}Q(\p)$ be a reduced primary decomposition of $N$ in $M$, $Q(\p)$ being $\p$-primary for all $\p\in\Ass(M/N)$.
\begin{itemize}
\item[(a)] If $\p\in\Ass(M/N)$ and $\p B=B$, then $Q(\p)\otimes_AB=M\otimes_AB$.
\item[(b)] If $\p\in\Ass(M/N)$ and $\p B$ is prime, then $Q(\p)\otimes_AB$ is $\p B$-primary in $M\otimes_AB$.
\item[(c)] If $\Phi$ is the set of $\p\in\Ass(M/N)$ such that $\p B$ is prime, then
\[N\otimes_AB=\bigcap_{\p\in\Phi}(Q(\p)\otimes_AB)\]
and this relation is a reduced primary decomposition of $N\otimes_AB$ in $M\otimes_AB$. 
\end{itemize}
\end{proposition}
\begin{proof}
If $\p B=B$, \cref{associated prime extension scalar M otimes N} applied to $M/Q(\p)$ and $B$ shows that
\[\Ass_B((M/Q(\p))\otimes_AB)=\emp\]
and, as $B$ is Noetherian and is a flat $A$-module, we conclude that $Q(\p)\otimes_AB=M\otimes_AB$. Assertion (b) follows from \cref{primary module tensor primary iff}, taking $\mathfrak{P}=\p B$. Finally the relation $N\otimes_AB=\bigcap_{\p\in\Phi}(Q(\p)\otimes_AB)$ follows from the fact that $B$ is a flat $A$-module. Also, for each $\p\in\Phi$ we have $\Ass_B((M/Q(\p))\otimes B)=\{\p B\}$, whence $\p=(\p B)^c$ by \cref{associated prime extension scalar contraction prop}, and $\p B\neq\q B$ for two distinct ideals $\p,\q$ of $\Phi$. On the other hand, by \cref{associated prime extension scalar M otimes N},
\[\Ass((M\otimes_AB)/(N\otimes_AB))=\Phi\]
we conclude from \cref{primary decomposition card iff reduced} that
\[N\otimes_AB=\bigcap_{\p\in\Phi}(Q(\p)\otimes_AB)\]
is a reduced primary decomposition.
\end{proof}
\begin{corollary}
In the situation of \cref{primary decomposition extension scalar by tensoring}, suppose that $\p B$ is prime for all $\p\in\Ass_A(M/N)$. Then, if $\p_1,\dots,\p_n$ are the minimal elements of $\Ass_A(M/N)$, the $\p_i B$ are minimal elements of $\Ass_A((M\otimes_AB)/(N\otimes_AB))$.
\end{corollary}
\begin{proof}
It is easy to see $\p B$ is a minimal prime if $\p$ is, and by \cref{primary decomposition extension scalar by tensoring} we see $\p_i B\neq\p_j B$ when $i\neq j$.
\end{proof}
\begin{example}
\mbox{}
\begin{itemize}
\item[(a)] Let us take $B=S^{-1}A$, where $S$ is a multiplicative subset of $A$. If $A$ is Noetherian, the hypotheses of \cref{primary decomposition extension scalar by tensoring} are satisfied and we recover a part of \cref{associated prime of localization}.
\item[(b)] Let $A$ be a Noetherian ring, $\a$ an ideal of $A$ and $B$ the Hausdorff completion of $A$ with respect to the $\a$-adic topology. Then $B$ is a flat $A$-module and \cref{associated prime extension scalar M otimes N} may be applied with $P=B$. But in general the hypotheses of \cref{primary decomposition extension scalar by tensoring} are not satisfied for the prime ideals of $A$.
\item[(c)] Let $A$ be a Noetherian ring and $B$ the polynomial algebra $A[X_1,\dots,X_n]$. Then $B$ is Noetherian and is a free $A$-module and therefore flat. Also, if $\p$ is a prime ideal of $A$, $B/\p B$ is isomorphic to $(A/\p)[X_1,\dots,X_n]$, which is an integral domain, and hence $\p B$ is prime. The hypotheses of \cref{primary decomposition extension scalar by tensoring} are therefore satisfied for every $A$-module $M$ and every submodule $N$ of $M$.
\end{itemize}
\end{example}
\subsection{Primary decompositions of graded modules}
\begin{proposition}\label{graded ring ass is homogeneous}
Let $\Delta$ be a torsion free commutative group, $A$ a graded ring of type $\Delta$ and $M$ a graded $A$-module of type $\Delta$. Every prime ideal associated with $M$ is graded and is the annihilator of a homogeneous element of $M$.
\end{proposition}
\begin{proof}
We know that $\Delta$ can be given a total order structure compatible with its group structure. Let $\p$ be a prime ideal associated with $M$, the annihilator of an element $x\in M$, and let $(x_i)_{i\in\Delta}$ be the family of homogeneous components of $x$; let $i(1)<i(2)<\cdots<i(r)$ be the values of $i$ for which $x_i\neq 0$. Consider an element $a\in\p$ and let $(a_i)_{i\in\Delta}$ be the family of its homogeneous components; we shall prove that $a_i\in\p$ for all $i\in\Delta$, which will show that $\p$ is a homogeneous ideal.\par
We argue by induction on the number of indices $i$ such that $a_i\neq 0$. Our assertion is obvious if this number is $0$; if not, let $m$ be the gratest of the indices $i$ for which $a_i\neq 0$; if we prove that $a_m\in\p$, the induction hypothesis applied to $a-a_m$ will give the conclusion. Now, $ax=0$; for all $j\in\Delta$, using the fact that the homogeneous component of degree $m+j$ of $ax$ is $0$, we obtain $\sum_{i\in\Delta}a_{m-i}x_{j+i}=0$; we conclude that $a_mx_j$ is a linear combination of the $x_i$ of indices $i>j$. In particular, $a_mx_{i(r)}=0$, whence, by descending induction on $k<r$, $a_m^{r-k+1}x_{i(k)}=0$. Then $a_m^rx=0$, whence $a_m^r\in\p$ and, as $\p$ is prime, $a_m\in\p$.\par
We now show that $\p$ is the annihilator of a homogeneous element of $M$. Let us write $\b_n=\Ann(x_{i(n)})$ for $1\leq n\leq r$. For every homogeneous element $b$ of $\p$ and all $n$ the homogeneous component of $bx$ of degree $i(n)+\deg(b)$ is $bx_{i(n)}$, hence $bx_{i(n)}=0$ and therefore $b\in\b_n$; as $\p$ is generated by its homogeneous elements, $\p\sub\b_n$. On the other hand, clearly $\bigcap_{n=1}^{r}\b_n=\p$; as $\p$ is prime, there exists an $n$ such that $\b_n\sub\p$ (\cref{prime ideal contained in union}), whence $b_n=\p=\Ann(x_{i(n)})$, which completes the proof.
\end{proof}
\begin{corollary}\label{graded module ass injection of A/p}
For every (necessarily homogeneous) prime ideal $\p$ associated with a graded $A$-module $M$, there exists an index $k\in\Delta$ such that the shifted graded $A$-module $(A/\p)(k)$ is isomorphic to a graded submodule of $M$.
\end{corollary}
\begin{proof}
With the notation of the proof of \cref{graded ring ass is homogeneous}, consider the homomorphism obtained, by taking quotients, from the homomorphism $a\mapsto ax_{i(n)}$ of $A$ to $M$; the latter is a graded homomorphism of degree $i(n)$ and hence it gives on taking quotients a graded bijective homomorphism of degree $i(n)$ of $A/\p$ onto a graded submodule of $M$.
\end{proof}
\begin{proposition}\label{graded module composition series of ass}
Let $\Delta$ be a torsion-free commutative group, $A$ a graded Noetherian ring of type $\Delta$ and $M$ a graded finitely generated $A$-module of type $\Delta$. There exists a composition series $(M_i)_{0\leq i\leq n}$ consisting of graded submodules of $M$ such that the graded module $M_i/M_{i+1}$ is isomorphic to a shifted graded module $(A/\p_i)(k_i)$, where $\p_i$ is a homogeneous prime ideal of $A$ and $k_i\in\Delta$.
\end{proposition}
\begin{proof}
It is sufficient to retrace the argument of \cref{associated prime of Noe prime ideals in composition} taking on this occasion $\mathcal{M}$ to be the set of graded submodules of $M$ with a composition series with the properties of the statement; we conclude using \cref{graded module ass injection of A/p}.
\end{proof}
\begin{proposition}\label{graded module almost nilpotent primary}
Let $\Delta$ be a torsion-free commutative group, $A$ a graded Noetherian ring of type $\Delta$, $\p$ a homogeneous ideal of $A$ and $M$ a graded $A$-module of type $\Delta$ not reduced to $0$. Suppose that for every homogeneous element $a$ of $\p$ the homothety of ratio $a$ on $M$ is almost nilpotent and that for every homogeneous element $b$ of $A-\p$ the homothety of ratio $b$ on $M$ is infective. Then $\p$ is prime and the submodule $\{0\}$ of $M$ is $\p$-primary.
\end{proposition}
\begin{proof}
It suffices to show that $\Ass(M)=\{p\}$ (\cref{primary submodule def}). Let $\q$ be a prime ideal associated with $M$; it is a homogeneous ideal and it is the annihilator of a homogeneous element $x\neq 0$ of $M$ by \cref{graded ring ass is homogeneous}. For every homogcneous element $a$ of $\q$, $ax=0$ and hence the homothety of ratio $a$ on $M$ is not injective, whence $a\in\p$. Conversely, let $b$ be a homogeneous element of $\p$; there exists an integer $n>0$ such that $b^nx=0$, whence $b^n\in\Ann(x)=\q$ and, as $\q$ is prime, $b\in\q$. As $\p$ and $\q$ are generated by their respective homogeneous element, $\p=\q$, which proves that $\Ass(M)\sub\{p\}$. As $M\neq\{0\}$, $\Ass(M)\neq\emp$ (\cref{associated prime empty iff}), whence $\Ass(M)=\{\p\}$.
\end{proof}
\begin{proposition}\label{graded module primary submodule homogeneousing}
Let $\Delta$ be a torsion-free commutative group, $A$ a graded Noetherian ring of type $\Delta$ and $M$ a graded $A$-module of type $\Delta$. Let $\p$ be a prime ideal of $A$ and $N$ a submodule of $M$ which is $\p$-primary with respect to $M$.
\begin{itemize}
\item[(a)] The homogeneous ideal $\p^h$ of $A$ is prime.
\item[(b)] The graded submodule $N^h$ of $N$ is $\p^h$-primary with respect to $M$.
\end{itemize}
\end{proposition}
\begin{proof}
We know that the homogeneous elements of $\p^h$ (resp. $N^h$) are just the homogeneous elements of $\p$ (resp. $N$). Let $a$ be a homogeneous element of $\p$; if $x$ is a homogeneous element of $M$, there exists an integer $n>0$ such that $a^nx\in N$; as $a^nx$ is homogeneous, $a^nx\in N^h$; as every $y\in M$ is the direct sum of a finite number of homogeneous elements, we conclude that there exists an integer $m>0$ such that $a^my\in N^h$, so that the homothety with ratio $a$ in $M/N^h$ is almost nilpotent.\par
Consider now a homogeneous element $b$ of $A-\p^h$; then $b\notin\p$ since $b$ is homogeneous. Let $x$ be an element of $M$ such that $bx\in N^h$ and let $(x_i)_{i\in\Delta}$ be the family of homogeneous components of $x$. As $N^h$ is graded, $bx_i\in N^h$ for all $i$, hence $bx_i\in N$ and, as $b\notin\p$, we conclude that $x_i\in N$; as $x_i$ is homogeneous, $x_i\in N^h$, whence $x\in N^h$ and the homothety with ratio $b$ on $M/N^h$ is injective. The proposition then follows from \cref{graded module almost nilpotent primary} applied to $\p^h$ and $M/N^h$.
\end{proof}
\begin{proposition}\label{graded module primary decomposition homogeneousing}
Let $\Delta$ be a torsion-free commutative group, $A$ a graded Noetherian ring of type $\Delta$, $M$ a graded $A$-module of type $M$, $N$ a graded submodule of $M$ and $N=\bigcap_{i\in I}Q_i$ a primary decomposition of $N$ in $M$.
\begin{itemize}
\item[(a)] Let $Q_i^h$ be the graded submodule of $Q_i$, then the $Q_i^h$ are primary and $N=\bigcap_{i\in I}Q_i^h$.
\item[(b)] If the primary decomposition $N=\bigcap_{i\in I}Q_i$ is reduced, so is the primary decomposition $N=\bigcap_{i\in I}Q_i^h$, and for all $i\in I$ the prime ideals corresponding to $Q_i$ and $Q_i^h$ are equal.
\item[(c)] If $Q_i$ corresponds to a prime ideal $\p_i$ which is a minimal element $\Ass(M/N)$, $Q_i$ is a graded submodule of $M$.
\end{itemize}
\end{proposition}
\begin{proof}
We have seen (\cref{graded module primary submodule homogeneousing}) that the $Q_i^h$ are primary with respect to $M$ and $N\sub Q_i^h\sub Q_i$, which proves (a). \cref{graded module primary submodule homogeneousing} also shows that the prime ideal $\p)i^h$ corresponding to $Q_i^h$ is the largest homogeneous ideal contained in the prime ideal $\p_i$ corresponding to $Q_i$. If the decomposition $N=\bigcap_{i\in I}Q_i$ is reduced, $\p_i\in\Ass(M/N)$ for all $i$ (\cref{primary decomposition reduced iff}), hence $\p_i$ is a graded ideal (\cref{graded ring ass is homogeneous}) and therefore $\p_i=\p_i^h$; then $\Ass(M/N) =\bigcup_{i\in I}\{\p_i\}$ (\cref{primary decomposition reduced iff}), which proves that the decomposition $N=\bigcap_{i\in I}Q_i^h$ is reduced (\cref{primary decomposition reduced iff}). Finally, if $\p_i$ is a minimal element of $\Ass(M/N)$, then $\p_i^h=\p_i$ since $\p_i$ is homogeneous (\cref{graded ring ass is homogeneous}), whence $Q_i^h=Q_i$, by virtue of \cref{primary decomposition minimal part and saturation}.
\end{proof}
\chapter{Integral dependence}
\section{Integral elements}
\begin{proposition}\label{integral element def}
Let $A$ be a ring, $R$ an algebra over $A$ and $x$ an element of $R$. The following properties are equivalent:
\begin{itemize}
\item[(\rmnum{1})] $x$ is a root of a monic polynomial in the polynomial ring $A[X]$.
\item[(\rmnum{2})] The subalgebra $A[x]$ of $R$ is a finitely generated $A$-module.
\item[(\rmnum{3})] There exists a faithful $A[x]$-module which is a finitely generated $A$-module.
\end{itemize}
\end{proposition}
\begin{proof}
The implications (\rmnum{1})$\Rightarrow$(\rmnum{2})$\Rightarrow$(\rmnum{3}) are clear. Now assume that there exists a faithful $A[x]$-module $M$ which is finitely generated as an $A$-module. Let $\phi$ to be multiplication by $x$ and $I=A$ (we have $xM\sub M$ since $M$ is an $A[x]$-module). Since $M$ is faithful, we have $x^n+a_{n-1}x^{n-1}+\cdots+a_0=0$ for suitable $a_i\in A$ in view of \cref{finite module homomorphism phi(M) sub IM}, whence (\rmnum{1}) holds.
\end{proof}
An element $x\in R$ is called integral over $A$ if it satisfies the equivalent properties of \cref{integral element def}. A relation of the form $P(x)=0$, where $P$ is a monic polynomial in $A[X]$ is also called an equation of integral dependence with coefficients in $A$.
\begin{example}
\mbox{}
\begin{itemize}
\item[(a)] Let $K$ be a field and $R$ a $K$-algebra; to say that an element $x\in R$ is integral over $K$ is equivalent to saying that $x$ is a root of a non-constant polynomial in the ring $K[X]$. Generalizing the terminology introduced when $R$ is an extension of $K$, the elements $x\in R$ which are integral over $K$ are also called the \textbf{algebraic elements} of $R$ over $K$.
\item[(b)] The elements of $\Q(i)$ which are integral over $\Z$ are the elements of the form $a+ib$ where $a\in\Z$ and $b\in\Z$ ("Gaussian integers"); the elements of $\Q(\sqrt{5})$ which are integral over $\Z$ are the elements of the form $(a+b\sqrt{5})/2$ where $a$ and $b$ belong to $\Z$.
\end{itemize}
\end{example}
\begin{proposition}\label{integral iff A[x] contained in finite subalgebra}
Let $A$ be a ring, $R$ an algebra over $A$ and $x$ an element of $R$. For $x$ to be integral over $A$, it is necessary and sufficient that $A[x]$ be contained in a subalgebra $R'$ of $R$ which is a finitely generated $A$-module.
\end{proposition}
\begin{proof}
The condition is obviously necessary by virtue of \cref{integral element def}. It is also sufficient by virtue of \cref{integral element def}, for $R'$ is a faithful $A[x]$-module (since it contains the unit element of $R$).
\end{proof}
\begin{corollary}\label{integral over Noe ring iff finite submodule}
Let $A$ be a Noetherian ring, $R$ an $A$-algebra and $x$ an element of $R$. For $x$ to be integral over $A$, it is necessary and sufficient that there exist a finitely generated submodule of $R$ containing $A[x]$.
\end{corollary}
\begin{proof}
The condition is necessary by virtue of \cref{integral element def}. It is sufficient for if $A[x]$ is a sub-$A$-module of a finitely generated $A$-module, it is itself a finitely generated $A$-module.
\end{proof}
Let $A$ be a ring. An $A$-algebra $R$ is called \textbf{integral over $\bm{A}$} if every element of $R$ is integral over $A$. Recall that $R$ is called finite over $A$ if $R$ is a finitely generated $A$-\textit{module}, and finite type if a finitely generated $A$-\textit{algebra}. It follows from \cref{integral iff A[x] contained in finite subalgebra} that every finite $A$-algebra is integral.
\begin{example}
If $M$ is a finitely generated $A$-module, the algebra $\End_A(M)$ of endomorphisms of $M$ is integral over $A$ by virtue of \cref{integral element def}. In particular, for every integer $n$, the matrix algebra $\mathcal{M}_n(A)=\End_A(A^n)$ is integral (and even finite) over $A$.
\end{example}
\begin{proposition}\label{integral element under homomorphism}
Let $A$, $B$ be two rings, $R$ an $A$-algebra, $S$ an $B$-algebra, and $\rho:A\to B$ and $\tau:R\to S$ two ring homomorphisms such that the diagram
\[\begin{tikzcd}
A\ar[r,"\rho"]\ar[d]&B\ar[d]\\
R\ar[r,"\tau"]&S
\end{tikzcd}\]
is commutative. If an element $x\in R$ is integral over $A$, then $\tau(x)$ is integral over $B$.
\end{proposition}
\begin{proof}
If $P(x)=0$ where $P\in A[X]$, then $\tau(x)$ is a zero of $\rho(P)$, hence $\tau(x)$ is integral over $B$.
\end{proof}
\begin{corollary}\label{integral element over bigger ring}
Let $A$ be a ring, $B$ an $A$-algebra and $C$ a $B$-algebra. Then every element $x\in C$ which is integral over $A$ is integral over $B$.
\end{corollary}
\begin{corollary}\label{integral element and conjugate}
Let $K$ be a field, $L$ an extension of $K$ and $x$, $y$ two elements of $L$ which are conjugate over $K$. If $A$ is a subring of $K$ and $x$ is integral over $A$, then $y$ is also integral over $A$.
\end{corollary}
\begin{proof}
There exists a $K$-isomorphism $\sigma$ of $K(x)$ onto $K(y)$ such that $\sigma(x)=y$ and the elements of $A$ are invariant under $\sigma$.
\end{proof}
\begin{proposition}\label{integral of product ring iff}
Let $(R_i)_{1\leq i\leq n}$ be a finite family of $A$-algebras and let $R=\prod_{i=1}^{n}R_i$ be their product. For an element $x=(x_i)$ of $R$ to be integral over $A$, it is necessary and sufficient that each of the $x_i$ be integral over $A$. For $R$ to be integral over $A$, it is necessary and sufficient that each of the $R_i$ be integral over $A$.
\end{proposition}
\begin{proof}
It is obviously sufficient to prove the first assertion. The condition is necessary by \cref{integral element under homomorphism}. Conversely, if each of the $x_i$ is integral over $A$, the subalgebra $A[x_i]$ of $R_i$ is a finitely generated $A$-module and hence so is the subalgebra $\prod_{i=1}^{n}A[x_i]$ of $R$. As $A[x]$ is contained in this subalgebra, $x$ is integral over $A$.
\end{proof}
\begin{proposition}\label{integral element A[x] is integral}
Let $A$ be a ring, $B$ an $A$-algebra and let $x_1,\dots,x_n$ be integral over $A$. Then the ring $A[x_1,\dots,x_n]$ is a finitely generated $A$-module.
\end{proposition}
\begin{proof}
Proof by induction. For $n=1$, this is a part of \cref{integral element def}. Assume $n>1$, let $A_{n-1}=A[x_1,\dots,x_{n-1}]$; then by the induction hypothesis $A_{n-1}$ is a finitely generated $A$-module. Moreover, $A_n=A_{n-1}[x_n]$ is a finitely generated $A_{n-1}$-module by the case $n=1$, since $x_n$ is integral over $A_{n-1}$. Hence $A_n$ is finitely generated as an $A$-module.
\end{proof}
\begin{corollary}
Let $A$ be a ring, $R$ an $A$-algebra and $E$ a set of elements of $R$ which are integral over $A$. Then the sub-$A$-algebra of $R$ generated by $E$ is integral over $A$.
\end{corollary}
\begin{proof}
Let $B$ be the $A$-algebra generated by $E$ in $R$. Then every element of $B$ belongs to a sub-$A$-algebra of $B$ generated by a finite subset of $E$, whence is integral over $A$ by \cref{integral element A[x] is integral}.
\end{proof}
\begin{corollary}\label{integral closure is a ring}
The set $\widebar{A}$ of elements of $B$ which are integral over $A$ is a subring of $B$ containing $A$.
\end{corollary}
\begin{proof}
If $x,y\in\widebar{A}$ then $A[x,y]$ is a finitely generated $A$-module by \cref{integral element A[x] is integral}. Hence $x\pm y$ and $xy$ are integral over $A$, by (c) of \cref{integral element def}.
\end{proof}
\begin{proposition}\label{integral and Ind functor}
Let $A$ be a ring and $B$ and $R$ two $A$-algebras. If $R$ is integral over $A$, then $R\otimes_AB$ is integral over $B$.
\end{proposition}
\begin{proof}
Consider any element $y=\sum_i\otimes b_i$ of $R\otimes_AB$, where the x, belong to $R$ and the $b_i$ to $B$. As $x_i\otimes a_i=(x_i\otimes 1)a_i$ and the $x_i\otimes 1$ are integral over $B$ by \cref{integral element under homomorphism}, so is $x$.
\end{proof}
\begin{corollary}\label{integral ring generating ring is integral}
Let $R$ be a ring and $A$, $B$, $C$ subrings of $R$ such that $A\sub B$. If $B$ is integral over $A$, then $C[B]$ is integral over $C[A]$.
\end{corollary}
\begin{proof}
The rng $B\otimes_AC[A]$ is integral over $C[A]$ by \cref{integral and Ind functor} and hence so is the canonical image $C[B]$ of $B\otimes_AC[A]$ in $R$ (considered as an $A$-algebra) by \cref{integral element under homomorphism}.
\end{proof}
\begin{proposition}\label{integral and quotient}
Let $A$ be a ring and $B$ an $A$-algebra. If $B$ integral over $A$, $\b$ is an ideal of $B$ and $\a=\b^c$, then $B/\b$ is integral over $A/\a$.
\end{proposition}
\begin{proof}
The follows by reducing an integral equation of an element of $B$ mod $\b$.
\end{proof}
\begin{proposition}\label{integral transitive}
Let $A$ be a ring, $B$ an $A$-algebra and $C$ a $B$-algebra. If $B$ is integral over $A$ and $C$ is integral over $B$, then $C$ is integral over $A$.
\end{proposition}
\begin{proof}
It is sufficient to verify that every $x\in C$ is integral over $A$. Then we have an equation
\[x^n+b_{n-1}x^{n-1}+\cdots+b_0,\quad b_i\in B.\]
The ring $B'=A[b_1,\dots,b_n]$ is a finitely generated $A$-module by \cref{integral element A[x] is integral}, and $B'[X]$ is a finitely generated $B'$-module (since $x$ is integral over $B'$). Hence $B'[X]$ is a finitely generated $A$-module and therefore $x$ is integral over $A$ by (c) of \cref{integral element def}.
\end{proof}
\begin{corollary}\label{integral tensor of two integral algebra}
Let $A$ be a ring and $R$, $S$ two $A$-algebras integral over $A$. Then $R\otimes_AS$ is integral over $A$.
\end{corollary}
\begin{proof}
The ring $R\otimes_AS$ is integral over $S$ by \cref{integral and Ind functor} and hence the conclusion follows from \cref{integral transitive}.
\end{proof}
\subsection{The integral closure of a ring}
Let $A$ be a ring and $R$ a $A$-algebra. The sub-$A$-algebra $\widebar{A}$ of $R$ consisting of the elements of $R$ integral over $A$ is called the integral closure of $A$ in $R$. If $\widebar{A}$ is equal to the canonical image of $A$ in $R$, $A$ is called integrally closed in $R$.
\begin{remark}
If $A$ is a field, the integral closure $\widebar{A}$ of $A$ in $R$ consists of the elements of $R$ which are algebraic over $A$. Generalizing the terminology used for field extensions, $\widebar{A}$ is then also called the algebraic closure of the field $A$ in the algebra $R$ and $A$ is called algebraically closed in $R$ if $\widebar{A}=A$.
\end{remark}
If $A$ is an integral domain, the integral closure of $A$ in its field of fractions is called the \textbf{integral closure of $\bm{A}$}. An integral domain is called \textbf{integrally closed} if it is equal to its integral closure.
\begin{proposition}\label{integral closure is integrally closed}
Let $A$ be a ring and $R$ an $A$-algebra. The integral closure $\widebar{A}$ of $A$ in $R$ is a subring integrally closed in $R$.
\end{proposition}
\begin{proof}
The integral closure of $\widebar{A}$ in $R$ is integral over $A$. It is therefore equal to $\widebar{A}$.
\end{proof}
\begin{corollary}\label{integral closure of domain is domain}
The integral closure of an integral domain $A$ is an integrally closed domain.
\end{corollary}
\begin{proof}
Let $K$ be the field of fractions of $A$ and $B$ the integral closure of $A$. Clearly $K$ is the field of fractions of $B$ and it is sufficient to apply \cref{integral closure is integrally closed} to $R=K$.
\end{proof}
\begin{proposition}\label{integrally closed intersection is integrally closed}
Let $R$ be a ring, $(B_i)_{i\in I}$ a family of subrings of $R$ and for each $i\in I$ let $A_i$ be a subring of $B_i$. If each $A_i$ is integrally closed in $B_i$, then $A=\bigcap_iA_i$ is integrally closed in $B=\bigcap_iB_i$.
\end{proposition}
\begin{proof}
This follows from \cref{integral element over bigger ring}.
\end{proof}
\begin{corollary}
Every intersection of a non-empty family of integrally closed subdomains of an integral domain $A$ is an integrally closed domain.
\end{corollary}
\begin{proof}
It is sufficient to apply \cref{integrally closed intersection is integrally closed}: taking $R$ and the $B_i$ equal to the field of fractions $K$ of $A$ and noting that a subring of $K$ integrally closed in $K$ is a fortiori an integrally closed domain since its field of fractions is contained in $K$.
\end{proof}
\begin{proposition}\label{integral closure of product of ring}
Let $A$ be a ring, $(R_i)_{1\leq i\leq n}$ a finite family of $A$-algebras and $\widebar{A}_i$ the integral closure of $A$ in $R_i$. Then the integral closure of $A$ in $R=\prod_{i=1}^{n}R_i$ is $\prod_{i=1}^{n}\widebar{A}_i$.
\end{proposition}
\begin{proof}
This is an immediate consequence of \cref{integral of product ring iff}.
\end{proof}
\begin{corollary}\label{Noe reduced integral closure in Q(A)}
Let $A$ be a reduced Noetherian ring, $\p_1,\dots,\p_n$ its distinct minimal prime ideals, $K_i$ the field of fractions of the integral domain $A/\p_i$ and $\widebar{A}_i$ the integral closure of $A$ in $K_i$. Then the canonical isomorphism of the total ring of fractions $Q(A)$ of $A$ onto $\prod_{i=1}^{n}K_i$ maps the integral closure of $A$ in $Q(A)$ onto the product ring $\prod_{i=1}^{n}\widebar{A}_i$.
\end{corollary}
\begin{proof}
This follows from \cref{Noe ring minimal prime and Ass localization prop} and \cref{integral closure of product of ring}.
\end{proof}
\begin{corollary}
For a reduced Noetherian ring to be integrally closed in its total ring of fractions, it is necessary and sufficient that it be a direct product of integrally closed (Noetherian) domains.
\end{corollary}
\subsection{Examples of integrally closed domains}
\begin{proposition}
A UFD is integrally closed.
\end{proposition}
\begin{proof}
Let $A$ be a UFD and $K$ be its fraction field. Then an element $x=a/b\in K$ (where $a,b$ are coprime) is integral over $A$ if and only if $x\in A$. In fact, if there is an equation
\[a^n/b^n+c_{n-1}a^{n-1}/b^{n-1}+\cdots+c_1a/b+c_0=0\]
then multiplying $b^n$ on both sides we see $b\mid a$, which is a contradiction.
\end{proof}
\begin{example}
Let $k$ be a field and $X$ an indeterminate over $k$; set $A=k[X^2,X^3]\sub B=k[X]$. Then $A$ and $B$ both have the same field of fractions $K=k(X)$. Since $B$ is a UFD, it is integrally closed; but $X$ is integral over $A$, so that $B$ is the integral closure of $A$ in $K$.\par 
Note that in this example $A\cong k[X,Y]/(Y^2-X^3)$. Thus $A$ is the
coordinate ring of the plane curve $Y^2=X^3$, which has a singularity at the origin. The fact that $A$ is not integrally closed is related to the existence of this singularity.
\end{example}
\begin{proposition}\label{integral polynomial product coefficient integral then}
Let $A$ be a ring, $R$ an $A$-algebra and $P$ and $Q$ monic polynomials in $R[X]$. If the cofficients of $PQ$ are integral over $A$, then the coefficients of $P$ and $Q$ are integral over $A$.
\end{proposition}
\begin{proof}
There exists a ring $R'$ containing $R$ and families of elements $(a_i)$, $(b_j)$ of $R'$ such that in $R'[X]$ we have
\[P(X)=\prod_{i=1}^{n}(X-a_i),\quad Q(X)=\prod_{j=1}^{m}(X-b_j)\]
The coefficients of $PQ$ are integral over $A$ and so belong to the integral closure $A'$ of $A$ in $R'$. Thus the elements $a_i$ and $b_j$ are integral over $A'$ and hence belong to $A'$. It the follows that the coefficients of $P$ and $Q$ are integral over $A$.
\end{proof}
Let $A$ be an integral domain, $K$ its field of fractions and $L$ a $K$-algebra. Given an element $x\in L$ algebraic over $K$, the polynomials $P\in K[X]$ such that $P(x)=0$ form a nonzero ideal $a$ of $K[X]$, necessarily principal. There exists a unique monic polynomial which generates $\a$. Generalizing the tenninology, this monic polynomial will be called the \textbf{minimal polynomial} of $x$ over $K$.
\begin{corollary}\label{integral element iff minimal polynomial coefficient}
Let $A$ be an integral domain, $K$ its field of fractions and $x$ an element of a $K$-algebra $L$. Then $x$ is integral over $A$ if and only if the coefficients of the minimal polynomial $P$ of $x$ over $K$ are integral over $A$.
\end{corollary}
\begin{proof}
If all coefficients of the minimal polynomial of $x$ are integral over $A$, then by \cref{integral transitive} we see $x$ is integral over $A$. Conversely, assume that $x$ is integral over $A$. Then there exists by hypothesis a monic polynomial $Q\in A[X]$ such that $Q(x)=0$. As $P$ divides $Q$ in $K[X]$, it follows from \cref{integral polynomial product coefficient integral then} that the coefficients of $P$ are integral over $A$.
\end{proof}
\begin{example}
Let $A$ be a UFD in which $2$ is a unit and $a\in A$. Then $A[\sqrt{a}]$ is an integrally closed domain if and only if $a$ is squre-free. In fact, let $\alpha$ be a square root of $a$. Let $K$ be the field of fractions of $A$; then $A$ is integrally closed in $K$, so that if $\alpha\in K$ we have $\alpha\in A$ and $A[\alpha]=A$, and the assertion is trivial. If $\alpha\notin K$ then the field of fractions of $A[\alpha]$ is $K(\alpha)$, and every element $\xi\in K(\alpha)$ can be written in a unique way as $\xi=x+y\alpha$ with $x,y\in K$. The minimal polynomial of $\xi$ over $K$ is 
\[X^2-2xX+(x^2-y^2a)\]
Since $\alpha$ is integral over $A$, $\xi\in K(\alpha)$ is integral over $A[\alpha]$ if and only if it is integral over $A$, and by \cref{integral element iff minimal polynomial coefficient}, if and only if $2x\in A$ and $x^2-y^2a\in A$.\par
If so, then by assumption, $2x\in A$ implies $x\in A$. Hence $y^2a\in A$. From this, if some prime $p$ of $A$ divides the denominator of $y$, we must have $p^2\mid a$, which is impossible if $a$ is square-free. Thus $y\in A$ and $A[\alpha]$ is integrally closed in this case.\par
Note that if $a$ is not squre-free then we can choose $y$ such that $y^2a\in A$ and $y\notin A$. Thus, in this case $A[\alpha]$ is not integrally closed.
\end{example}
\begin{example}
If $A$ is an integrally closed domain and $\p$ is a prime ideal in $A$, then the residue ring $A/\p$ is not integrally closed in general. In fact, any finite integral domain $k[x_1,\dots,x_n]$ over a field $k$ is of the form $A/\p$, where $A$ is the polynomial ring $k[X_1,\dots,X_n]$. But finite integral domains are not in general integrally closed. In the case $n=2$ the simplest example is the one in which $\p$ is the principal ideal $(X_1^2-X_2^3)$. In that case, $x_1/x_2$ does not belong to the ring $k[x_1,x_2]$, but $x_1/x_2$ is integral over that ring since $(x_1/x_2)^2= x_2$.
\end{example}
Let $A$ be a ring and $R$ an $A$-algebra. The homomorphism $A\to R$ defining the $A$-algebra structure on $R$ can be extended uniquely to a homomorphism $A[X]\to R[X]$ of polynomial rings over $A$ and $R$, leaving $X$ invariant and hence $R[X]$ is given a canonical $A[X]$-algebra structure.
\begin{proposition}\label{integral polynomial iff coefficient is integral}
Let $A$ be a ring, $R$ an $A$-algebra and $P$ a polynomial in $R[X_1,\dots,X_n]$. For $P$ to be integral over $A[X_1,\dots,X_n]$, it is necessary and sufficient that the coefficients of $P$ be integral over $A$.
\end{proposition}
\begin{proof}
By considering the polynomials of $R[X_1,\dots,X_n]$ as polynomials in $X_n$ with coefficients in $R[X_1,\dots,X_{n-1}]$, we see immediately‘that it is reduced to proving the proposition for $n=1$. Then let $P$ be a polynomial in $R[X]$. It follows immediately from \cref {integral and Ind functor} that, if the coefficients of $P$ are in the integral closure $B$ of $A$ in $R$, the element $P$, which belongs to $B[X]=B\otimes_AA[X]$, is integral over $A[X]$. Conversely, suppose that $P$ is integral over $A[X]$ and let
\[Q(Y)=Y^m+F_{m-1}Y^{m-1}+\cdots+F_0\]
be a monic polynomial with coefficients $F_i\in A[X]$ with $P$ as a root. Let $r$ be an integer strictly greater than all the degrees of the polynomials $P$ and $F_i$ and let us write $P_1(X)=P(X)-X^r$. Then $P_1$ is a root of the polynomial
\[Q_1(Y)=Q(Y+X^r)=Y^m+G_{m-1}Y^{m-1}+\cdots+G_0\]
with coefficients in $A[X]$. We may therefore write
\[G_0=-P_1(P_1^{m-1}+G_{m-1}P_1^{m-2}+\cdots+G_{1}).\]
Now the choice of $r$ implies that $-P_1$ is a manic polynomial of $R[X]$ and so is $G_0(X)=Q(X^r)$, the degrees of the polynomials $F_i(X)X^{r(m-i)}$ being all smaller than $rm$ for $i\geq 1$. We conclude first of all that the polynomial
\[P_1^{m-1}+G_{m-1}P_1^{m-2}+\cdots+G_{1}\]
of $R[X]$ is also monic. Moreover, as the coefficients of $G_0$ belong to $A$, \cref{integral polynomial product coefficient integral then} shows that $P_1$ has coefficients integral over $A$ and the coefficients of $P$ are therefore certainly integral over $A$.
\end{proof}
\begin{corollary}\label{integral closure of a polynomial ring in algebra}
Let $A$ be a ring, $R$ an $A$-algebra and $\widebar{A}$ the integral closure of $A$ in $R$. Then the integral closure of $A[X_1,\dots,X_n]$ in $R[X_1,\dots,X_n]$ is equal to $\widebar{A}[X_1,\dots,X_n]$.
\end{corollary}
\begin{corollary}\label{integral closure of a polynomial ring}
Let $A$ be an integral domain and $\widebar{A}$ its integral closure. Then the integral closure of the polynomial ring $A[X_1,\dots,X_n]$ is $\widebar{A}[X_1,\dots,X_n]$.
\end{corollary}
\begin{proof}
Arguing by induction on $n$, the problem is immediately reduced to the case $n=1$. Let $K$ be the field of fractions of $A$, which is also that of $\widebar{A}$. If an element $P$ of the field of fractions $K(X)$ of $A[X]$ is integral over $A[X]$, it belongs to the polynomial ring $K[X]$, for the latter is a principal ideal domain and hence integrally closed. The corollary then follows from \cref{integral closure of a polynomial ring in algebra} applied to $R=K$.
\end{proof}
\begin{corollary}\label{integral closed polynomial ring iff coefficient}
Let $A$ be an integral domain. For the polynomial ring $A[X_1,\dots,X_n]$ to be integrally closed, it is necessary and sufiicient that $A$ be integrally closed.
\end{corollary}
\subsection{Completely integrally closed domains}
\begin{proposition}\label{almost integral def}
Let $A$ be a ring, $K$ the filed of fraction of $A$ and $x$ an element of $K$. The following properties are equivalent:
\begin{itemize}
\item[(\rmnum{1})] There exists nonzero $a\in A$ such that $ax^n\in A$.
\item[(\rmnum{2})] All the powers $x^n$ (with $n\geq 0$) are contained in a finitely generated sub-$A$-module of $K$.
\item[(\rmnum{3})] The subalgebra $A[X]$ of $K$ is a fractional ideal of $K$.
\end{itemize}
\end{proposition}
\begin{proof}
It is clear that (\rmnum{1}) and (\rmnum{3}) are equivalent and (\rmnum{3}) implies (\rmnum{2}). Also, if all powers $x^n$ is contained in a finitely generated sub-$A$-module $M$ of $K$, then it can be easily seen that there exists $s\in A$ such that $sM\sub A$, whence (\rmnum{1}) holds.
\end{proof}
An element $x$ of $K$ is called almost integral over $A$ if it satisfies the conditions of \cref{almost integral def}. An integral domain $A$ is called \textbf{completely integrally closed} if every almost integral element of $K$ belongs to $A$. Clearly a completely integrally closed domain is integrally closed. Conversely, \cref{integral over Noe ring iff finite submodule} shows that an integrally closed Noetherian domain is completely integrally closed. If $(A_i)$ is a family of completely integrally closed domains with the same field of fractions $K$, then $A=\bigcap_iA_i$ is completely integrally closed. For if $x\in K$ is such that for somc non-zero $d$ in $A$, $dA[x]$ belongs to $A$, the hypothesis implies that $x\in A_i$ for all $i$ and hence $x\in A$.
\begin{proposition}
Let $A$ be a completely integrally closed domain. Then every polynomial ring $A[X_1,\dots,X_n]$ (resp. every ring of formal power series $A\llbracket X_1,\dots,X_n\rrbracket$) is completely integrally closed.
\end{proposition}
\begin{proof}
By induction on $n$, it is sufficient to prove that $A[X]$ (resp. $A\llbracket X\rrbracket$) is completely integrally closed. Then let $P$ be an element of the field of fractions of $A[X]$ (resp. $A\llbracket X\rrbracket$) and suppose that there exists a non-zero, element $Q\in A[X]$ (resp. $Q\in A\llbracket X\rrbracket$) such that, $QP^m\in A[X]$ (resp. $QP^m\in A\llbracket X\rrbracket$) for every integer $m\geq 0$. If $K$ is the field of fractions of $A$, then $A[X]$ (resp. $A\llbracket X\rrbracket$) is a subring of $K[X]$ (resp. $K\llbracket X\rrbracket$) and $K[X]$ (resp. $K\llbracket X\rrbracket$) is a principal ideal domain and hence integrally closed and Noetherian and therefore completely integrally closed, therefore we have already seen that $P\in K[X]$ (resp. $P\in K\llbracket X\rrbracket$). Write
\[P=\sum_{k=0}^{\infty}a_kX^k,\quad Q=\sum_{k=0}^{\infty}b_kX^k\]
where $a_k\in K$ and $b_k\in A$, and we argue by reductio ad absurdum by supposing that the $a_k$ do not all belong to $A$. Then there is a least index $i$ such that $a_i\notin A$. If we write $P_1=\sum_{k=0}^{i-1}a_kX^k\in A[X]$, it follows immediately from the hypothesis that also $Q(P-P_1)^m\in A[X]$ (resp. $Q(P-P_1)^m\in A\llbracket X\rrbracket$) for all $m\geq 0$. Let $j$ be the least integer such that $b_j\neq 0$. Then in $Q(P-P_1)^m$ the term of least degree with a nonzero coefficient is $b_ja_i^mX^{j+mi}$ and hence $b_ja_i^m\in A$ for all $m\geq 0$. But as $A$ is completely integrally closed this implies $a_i\in A$, contrary to the hypothesis. 
\end{proof}
\begin{proposition}\label{filtration gr(A) completely integrally closed then}
Let $A$ be a filtered ring whose filtration is exhaustive and such that every principal ideal of $A$ is closed under the topology defined by the filtration. If the associated graded ring $\gr(A)$ is a completely integrally closed domain, then $A$ is a completely integrally closed domain.
\end{proposition}
\begin{proof}
Let $(A_n)_{n\in\Z}$ be the filtration defined on $A$. As $\bigcap_nA_n$ is the closure of the ideal $(0)$, the hypothesis implies first that the filtration $(A_n)$ is separated and, as $\gr(A)$ is an integral domain, so then is $A$ (\cref{filtration gr(A) integral then B/n integral}). Let $x=a/b$ be an element of the field of fractions $K$ of $A$ for which there exists a nonzero element $d$ of $A$ such that $dx^n\in A$ for all $n$. We must prove that $a\in Ab$ and, as by hypothesis the ideal $Ab$ is closed, it is sufficient to show that, for all $n\in\Z$, $a\in A_b+A_n$. As the filtration of $A$ is exhaustive, there exists an integer $p\in\Z$ such that $a\in Ab+A_p$. It will therefore suffice to prove that the relation $a\in Ab+A_m$ implies $a\in Ab+A_{m+1}$ for any $m\in\N$.\par
Suppose then $a=by+z$ where $y\in A$ and $z\in A_{m}$. Since by hypothesis $dx^n\in A$ for all $n$, we have $d(x-y)^n\in A$ for all $n$. In other words, $dz^n=b^nt_n$ where $t_n\in A$ for all $n$. We may assume that $z\neq 0$. Let $v$ denote the order function on $A$ and let us write $v(z)=n_0$. If $\bar{x}$ denote the canonical image of an element $x\in A$ in $\gr(A)$, then since $b\neq 0$, we deduce that $\bar{d}(\bar{z}/\bar{b})^n\in\gr(A)$ for all $n\in\N$. Since $\gr(A)$ is completely integrally closed, there exists an element $\bar{s}\in\gr(A)$ such that $\bar{z}=\bar{b}\bar{s}$. Decomposing $\bar{s}$ into a sum of homogeneous elements, it is further seen (since $\bar{z}$ and $\bar{b}$ are homogeneous) that $\bar{s}$ may be assumed to be homogeneous and that is the image of an element $s\in A$. Then $v(bs)=v(z)=n_0$ and $z\equiv bs$ mod $A_{n_0+1}$. As we have $n_0\geq m$, a fortiori $z\equiv bs$ mod $A_{m+1}$, hence $a\equiv b(y+s)$ mod $A_{m+1}$ and the claim follows.
\end{proof}
\subsection{The integral closure of a ring of fractions}
\begin{proposition}\label{integral closure and localization}
Let $A$ be a ring, $R$ an $A$-algebra, $\widebar{A}$ the integral closure of $A$ in $R$ and $S$ a multiplicative subset of $A$. Then $S^{-1}\widebar{A}$ is the integral closure of $S^{-1}A$ in $S^{-1}B$.
\end{proposition}
\begin{proof}
Let $b/s\in S^{-1}\widebar{A}$, where $b$ is integral over $A$, say
\[b^n+a_{n-1}b^{n-1}+\cdots+a_0=0,\quad a_i\in A.\]
Then the equation of $b$ gives an equation
\[(b/s)^n+(a_{n-1}/s)(b/s)^{n-1}+\cdots+a_0/s^n=0\]
Hence $S^{-1}\widebar{A}$ is integral over $S^{-1}A$. Conversely, if $b/s\in S^{-1}B$ is integral over $S^{-1}A$, then we have an equation of the form
\[(b/s)^n+(a_{n-1}/s_{n-1})(b/s)^{n-1}+\cdots+a_0/s_0=0\]
where $a_i\in A$, $s_i\in S$. Let $t=s_1\cdots s_n$, and multiply this equation by $(st)^n$ throughout. Then it becomes an equation of integral dependence for $bt$ over $A$. Hence $bt\in\widebar{A}$ and therefore $b/s=bt/st\in S^{-1}\widebar{A}$.
\end{proof}
\begin{corollary}
Let $A$ be an integral domain, $\widebar{A}$ its integral closure and $S$ a multiplicative subset such that $0\notin S$. Then the integral closure of $S^{-1}A$ is $S^{-1}\widebar{A}$.
\end{corollary}
\begin{proof}
The field of fractions $K$ of $A$ is also the field of fractions of $S^{-1}A$ since $0\notin S$, so \cref{integral closure and localization} is then applied to $K$.
\end{proof}
\begin{corollary}\label{integral closure in algebra of fraction field char}
Let $A$ be an integral domain, $K$ its field of fractions, $R$ an algebra over $K$ and $B$ the integral closure of $A$ in $R$. The elements of $R$ which are algebraic over $K$ are the elements of the form $a^{-1}b$ where $b\in B$ and $a\in A$. If $L$ is the algebraic closure of $K$ in $R$, then there exists a basis of $L$ over $K$ contained in $B$.
\end{corollary}
\begin{proof}
The first assertion follows from \cref{integral closure and localization} applied in the case $S=A-\{0\}$. If $(x_i)_{i\in I}$ is a basis of $L$ over $K$, then there exists for all $i\in I$ an element $a_i$ of $A$ such that $a_ix_i\in B$. Then $(a_ix_i)_{i\in I}$ is also a basis of $L$ over $K$.
\end{proof}
\begin{proposition}\label{integral closed is local property}
Let $A$ be an integral domain. Then the following are equivalent:
\begin{itemize}
\item[(\rmnum{1})] $A$ is integrally closed.
\item[(\rmnum{2})] $A_\p$ is integrally closed for each prime ideal $\p$.
\item[(\rmnum{3})] $A_\m$ is integrally closed for each maximal ideal $\m$.
\end{itemize}
\end{proposition}
\begin{proof}
It follows from \cref{integral closure and localization} that the condition (\rmnum{3}) is necessary. The condition is sufficient, for $A=\bigcap_{\m}A_\m$ by \cref{integral domain inter of localization} and it is sufficient to apply the \cref{integrally closed intersection is integrally closed}.
\end{proof}
\begin{corollary}\label{integral domain conductor localization prop}
Let $A$ be an integral domain, $K$ its field of fractions and $S$ a multiplicative subset of $A$ such that $0\notin S$.
\begin{itemize}
\item[(a)] Let $B$ be a subring of $K$ which is integral over $A$ and let $\f$ be the annihilator of the $A$-module $B/A$ (called the \textbf{conductor} of $A$ in $B$). Then $S^{-1}\f$ is contained in the conductor of $S^{-1}A$ in $S^{-1}B$ and is equal to it if $B$ is a finitely generated $A$-module.
\item[(b)] Let $\widebar{A}$ be the integral closure of $A$. For $S^{-1}A$ to be integrally closed, it is sufficient that the conductor $\f$ of $A$ in $\widebar{A}$ meet $S$. This condition is also necessary if $\widebar{A}$ is a finitely generated $A$-module.
\end{itemize}
\end{corollary}
\begin{proof}
As $\f B\sub A$, $(S^{-1}\f)(S^{-1}B)\sub S^{-1}A$ and hence $S^{-1}\f$ is contained in $\Ann(S^{-1}B/S^{-1}A)$. If $B$ is a finitely generated A-module, the equation $S^{-1}\f=\Ann(S^{-1}B/S^{-1}A)$ is a special case of \cref{localization and Ann}, $S^{-1}B/S^{-1}A$ being canonically identified with $S^{-1}(B/A)$. This proves (a).\par
By \cref{integral closure and localization}, $S^{-1}\widebar{A}$ is the integral closure of $S^{-1}A$. As the relations $\f\cap S\neq\emp$ and $S^{-1}\f=S^{-1}A$ are equivalent, (b) is an immediate consequence of (a).
\end{proof}
\begin{corollary}\label{integral closure localization and conductor}
Let $A$ be an integral domain, $\widebar{A}$ its integral closure and $\f$ the conductor of $A$ in $\widebar{A}$. Suppose that $\widebar{A}$ is a finitely generated $A$-module. Then the prime ideals $\p$ of $A$ such that $A_\p$ is not integrally closed are those which contain $\f$.
\end{corollary}
\subsection{Norms and traces of integral elements}
\begin{proposition}\label{integral squre matrix iff}
Let $A$ be a ring, $B$ be an $A$-algebra, and $T$ be a square matrix of order $n$ over $B$. The following properties are equivalent:
\begin{itemize}
\item[(\rmnum{1})] $T$ is integral over $A$;
\item[(\rmnum{2})] There exists a finitely generated sub-$A$-module $M$ of $B^n$ such that $T(M)\sub M$ and $M$ is a system of generators of the $B$-module $B^n$.
\item[(\rmnum{3})] The coefficients of the characteristic polynomial of $T$ are integral over $A$. 
\end{itemize}
\end{proposition}
\begin{proof}
If $\chi(X)$ is the characteristic polynomial of $X$, the Cayley-Hamilton Theorem shows that $\chi(T)=0$ and, as $\chi(X)$ is a monic polynomial, (\rmnum{3}) implies (\rmnum{1}) by the transitivity of integrality. Suppose in the second place that (\rmnum{1}) holds. If $e_1,\dots,e_n$ is the canonical basis of $B^n$, the sub-$A$-module $M$ of $B$ generated by the $T^ke_i$ is a finitely generated $A$-module, since the $A$-algebra $A[T]$ is a finitely generated $A$-module. As $M$ contains the $e_i$, it is seen that (\rmnum{1}) implies (\rmnum{2}). The converse is a consequence of \cref{integral element def}.\par
Finally let us prove that (\rmnum{1}) implies (\rmnum{3}). As $T$ is integral over $A$ and a fortiori over the polynomial ring $A[X]$, the polynomial $X-T$ is integral over $A[X]$ by \cref{integral polynomial iff coefficient is integral}. By \cref{integral polynomial iff coefficient is integral}, the problem is seen to reduce (by replacing $T$ by $X-T$ and $A$ by $A[X]$) to proving that, if $T$ is integral over $A$, then $\det(T)$ is an element of $B$ which is integral over $A$. Now, we have seen above that the endomorphism $\phi$ of $B^n$ defined by the matrix $T$ preserves a finitely generated sub-$A$-module $M$ containing the $e_i$. Then $\bigwedge^nM$ is a finitely generated sub-$A$-module in $\bigwedge^nB^n$ containing $e_1\wedge\cdots\wedge e_n$ and which is stable under $\wedge^n\phi$, in other words under the homothety of ratio $\det(T)$. As $e_1\wedge\cdots\wedge e_n$ generates $\bigwedge^nB^n$, \cref{finite module homomorphism phi(M) sub IM} then proves that $\det(T)$ is integral over $A$.
\end{proof}
\begin{corollary}\label{integral element in field extension char polynomial integral}
Let $A$ be an integral domain, $K$ its field of fractions and $L$ a finite dimensional $K$-algebra. If $x\in L$ is integral over $A$, then the coefieients of the characteristic polynomial of $x$ over $K$ are integral over $A$.
\end{corollary}
\begin{proof}
Let $h_x$ be the homothety of ratio $x$ on $L$. Then by definition $h_x$ is integral over $A$, and the characteristic polynomial of $h_x$ is that of $x$ over $K$.
\end{proof}
\begin{corollary}\label{integral element in field extension norm and trace integral}
Let $A$ be an integral domain, $K$ its field of fractions and $L$ a finite dimensional $K$-algebra. If $x\in L$ is integral over $A$, then $N_{L/K}(x)$ and $\tr_{L/K}(x)$ are integral over $A$.
\end{corollary}
\begin{proposition}\label{field separable ext trace is nondegenerate}
If $L/K$ is a finite separable extension, then $(x,y)=\tr_{L/K}(xy)$ is a nondegenerate bilinear form on $L$. 
\end{proposition}
\begin{proof}
Let $\theta$ be a primitive element for $L$, so that $L=K(\theta)$. Then the bilinear form $(x,y)\mapsto\tr_{L/K}(xy)$ has matrix $G=(\tr_{L/K}(\theta^{i-1}\theta^{j-1}))$ under the basis $1,\theta,\dots,\theta^{n-1}$, . It is nondegenerate because, if $\theta_1=\theta,\dots,\theta_n$ are the $K$-conjugates of $\theta$, then
\[\det(G)=\prod_{i<j}(\theta_i-\theta_j)\neq 0.\]
This proves the claim.
\end{proof}
\begin{theorem}\label{integral closure in finite separable contained in finite}
Let $A$ be an integrally closed domain, $K$ its quotient field, $L$ a finite-dimensional separable $K$-algebra, and $B$ the integral closure of $A$ in $L$. There exists a basis $\{x_1,\dots,x_n\}$ of $L$ over $K$ such that $B$ is contained in the $A$-module $\sum_iAx_i$.
\end{theorem}
The proposition will follow from the following more precise lemma:
\begin{lemma}\label{integral closure in finite separable dual basis relation}
Under the hypotheses of \cref{integral closure in finite separable contained in finite}, let $(w_1,\dots,w_n)$ be a basis of $L$ over $K$ contained in $B$ (\cref{integral closure in algebra of fraction field char}); then there is a unique dual basis $(w_1^*,\dots,w_n^*)$ of $L$ over $K$ with respect to the trace form $\tr_{L/K}$; if $d=D_{L/K}(w_1,\dots,w_n)$ is the discriminant of the basis $(w_1,\dots,w_n)$, then $d\neq 0$ and
\begin{align}\label{integral closure in finite separable dual basis relation-1}
\sum_{i=1}^{n}Aw_i\sub B\sub\sum_{i=1}^{n}Aw_i^*\sub d^{-1}\Big(\sum_{i=1}^{n}Aw_i\Big).
\end{align}
In particular, if $d$ is invertible in $A$, $B$ is a free $A$-module with basis $(w_1,\dots,w_n)$.
\end{lemma}
\begin{proof}
As $L$ is a separable $K$-algebra, $d\neq 0$ by (A, \Rmnum{9}, $\S$2, proposition 5) and the $K$-bilinear form $\tr_{L/K}$ is non-degenerate. This shows the existence and uniqueness of the dual basis $(w_i^*)$. This being so, the first inclusion of (\ref{integral closure in finite separable dual basis relation-1}) is obvious. Let $x$ be an element of $B$; let us write $x=\sum_{i=1}^{n}\xi_iw_i^*$ where $\xi_i\in K$; for all $i$, we have $\xi_i=\tr_{L/K}(xw_i)$, so $\xi_i$ is integral over $A$ (\cref{integral element in field extension norm and trace integral}) and, as $A$ is integrally closed, $\xi_i\in A$ for all $i$; this shows the second inclusion in (\ref{integral closure in finite separable dual basis relation-1}). Finally, let us write $w_j^*=\sum_{i=1}^{n}\alpha_{ji}w_i$ where $\alpha_{ji}\in K$; then $\sum_{i=1}^{n}\alpha_{ji}\tr_{L/K}(w_iw_k)=\delta_{jk}$ for all $j$ and $k$; Cramer's formulae show that the $\alpha_{ji}$ belong to $d^{-1}A$, whence the third inclusion (\ref{integral closure in finite separable dual basis relation-1}). The last assertion follows immediate from (\ref{integral closure in finite separable dual basis relation-1}), which in this case gives $B=\sum_{i=1}^{n}Aw_i$.
\end{proof}
\begin{proof}
We first notice that $L=S^{-1}B$, where $S=A-\{0\}$; that is, given any element $x$ of $L$ there exists a non-zero element $s$ of $A$ such that $sx\in B$: in fact, if $p(X)=X^n+a_{n-1}X^{n-1}+\cdots+a_0$ is the minimal polynomial of $x$ over $K$, and if we take a common denominator $s$ in $A$ such that $sc_i=a_i\in A$, then we have 
\[(sx)^n+a_{n-1}(sx)^{n-1}+\cdots+s^{n-1}a_0=0\]
and $sx$ is integral over $A$. It follows from this observation that there exists a basis $\{u_1,\dots,u_n\}$ of $L$ over $K$ such that $u\in B$ for every $i$. Let $\{v_1,\dots,v_n\}$ be the dual basis of $\{u_1,\dots,u_n\}$ under the bilinear form $\tr(xy)$. If an element $x=\sum_ja_jx_j$ is in $B$, we have $xu_i\in B$ for every $i$, whence $\tr(xu_i)\in A$ ($A$ is integrally closed). Since $\tr(xu_i)=\sum_ia_i\tr(u_iv_j)=a_i$, we then have $B\sub\sum_jAv_j$.
\end{proof}
\begin{corollary}\label{integral closure in finite separable finite if Noe}
The assumptions being the same as in \cref{integral closure in finite separable contained in finite}, let us furthermore assume that the ring $A$ is Noetherian. Then $B$ is a finite $A$-module and is a Noetherian ring.
\end{corollary}
\begin{proof}
In fact, $B$ is a submodule of the finite $A$-module $\sum_i Ax_i$, and is therefore a finite $A$-module. Thus $B$ satisfies the a.c.c. as an $A$-module, and a fortiori satisfies the a.c.c. as an $B$-module. That is, $B$ is Noetherian.
\end{proof}
\begin{corollary}\label{integral closure in finite separable free if PID}
The assumptions being the same as in \cref{integral closure in finite separable contained in finite}, let us furthermore assume that $A$ is a PID. Then there exists a basis $\{y_i\}$ of $L$ over $K$ such that $B=\sum_i Ay_i$.
\end{corollary}
\begin{proof}
It was just shown that $B$ is contained in an $A$-module generated by $n$ elements $x$. Hence $B$ has a basis consisting of $n$ elements $\{y_i\}$, Since $L=S^{-1}B$, the set $\{y_i\}$ is necessarily also a basis of $L$ over $K$.
\end{proof}
\cref{integral closure in finite separable free if PID} is of particular importance for the case in which $A$ is either the ring $\Z$ of rational integers, $L$ being then an algebraic number field, or a polynomial ring $k[X]$ in one variable over a field $k$, $L$ being then a field of algebraic functions of one variable. In the first case, the elements of $L$ which are integral over $\Z$ are called the \textbf{algebraic integers} of the number field $L$; in the second case, the elements of $L$ which are integral over $k[X]$ are called the \textbf{integral functions} of the function field $L$ (with respect to the element $X$). \cref{integral closure in finite separable free if PID} shows that these algebraic integers (or integral functions) are the linear combinations, with ordinary integral coefficients (or with coefficients in $k[X]$), of $n=[L:K]$ linearly independent algebraic integers $y_i$. Such a basis $\{y_i\}$ of $L$ over the rational field (or over the rational function field $k(X)$) is called an \textbf{integral basis} of $L$.
\begin{example}
Let $X$ be an indeterminate over a field $k$ of characteristic $\neq 2$, and $p$ an irreducible polynomial over $k$. Let $A=k[X]$ and $K=k(X)$, then the function field $L=k(X,Y)$ admits $\{1,Y\}$ as an integral basis (with respect to $A$). In fact, in the first place $1$ and $Y$ are integral over $A$. Furthermore, let $f$ be an element of $L$ which is integral over $A$. We write $f=a(X)+b(X)Y$ with $a,b\in k(X)$. Then the trace $2a(X)$ and the norm $a(X)^2-b(X)^2p(X)$ of $f$ over $k(X)$ belong to $k[X]$, whence both $a(X)$ and $b(X)$ are polynomials, since otherwise $p(X)$ would be divisible by the square of the denominator of $b(X)$. Consequently, the integral closure of $A=k[X]$ in $L=k(X,Y)$ is the ring $k[X,Y]$.
\end{example}
\begin{proposition}\label{integrally closed algebra separable extension}
Let $k$ be a field, $L$ a separable extension of $k$ and $A$ an integrally closed $k$-algebra. If the ring $L\otimes_kA$ is an integral domain, it is integrally closed.
\end{proposition}
\begin{proof}
Let $K$ be the field of fractions of $A$; as $k$ is a field, $L\otimes_kA$ is canonically identified with a sub-$k$-algebra of $L\otimes_kK$ and $L$ and $A$ with sub-k-algebras of $L\otimes_kA$. Moreover, since an element $s\neq 0$ of $A$ is not a divisor of $0$ in $A$, $1\otimes s$ is not a divisor of zero in $L\otimes_kA$ since $L$ is flat over $k$; identifying $s$ with $1\otimes s$, it is therefore seen that, if $S=A\setminus\{0\}$, $L\otimes_kK$ is identified with $S^{-1}(L\otimes_kK)$; as $L\otimes_kA$ is assumed to be an integral domain, $L\otimes_kK$ is thus identified with a subring of the field of fractions $\Omega$ of $L\otimes_kA$.\par
First suppose that $L$ is a \textit{finite} extension of $k$; then $L\otimes_kK$ is an algebra of finite rank over $K$ and by hypothesis has no divisor of $0$; hence it is a field (it is sandwiched by $K$ and $\Omega$) and it is in this case the field of fraction $\Omega$ of $L\otimes A$. Let $(w_1,\dots,w_n)$ be a basis of $L$ over $k$, which is therefore also a basis of $L\otimes_kK$ over $K$. There exists a basis $(w_1^*,\dots,w_n^*)$ of $L$ such that $\tr_{L/k}(w_iw_j^*)=\delta_{ij}$ (\cref{integral closure in finite separable dual basis relation}); every $z\in L\otimes_kK$ may be written uniquely $z=\sum_{i=1}^{n}a_iw_i$ where $a_i\in K$; then
\[\tr_{(L\otimes K)/K}(zw_j^*)=\sum_{i=1}^{n}a_i\tr_{(L\otimes K)/K}(w_iw_j^*)\]
and as in $L$ the traces $\tr_{(L\otimes K)/K}$ and $\tr_{L/K}$ coincide, finally $\tr_{(L\otimes K)/K}(zw_j^*)=a_j$ for each $j$. Note on the other hand that the elements of $L$ are integral over $k$ and hence also over $A$; therefore $L\otimes_kA$ is integral over $A$ (\cref{integral tensor of two integral algebra}). This being so, suppose that $z\in L\otimes_kK$ is integral over $L\otimes_kA$; then $z$ is integral over $A$, hence so is $zw_j^*$ and therefore also $a_j=\tr_{L/K}(zw_j^*)$ for all $j$ (\cref{integral element in field extension norm and trace integral}). As $A$ is integrally closed, $a_j\in A$ for all $j$ and hence $z\in L\otimes_kA$, which proves the proposition in this case.\par
Suppose now that $L$ is a finitely generated separable extension of $k$; then there exists a separating transcendence basis $(x_1,\dots,x_d)$ of $L$ over $k$; as $L$ and $K$ are algebraically disjoint over $k$ in the field $\Omega$, the $x_i$ are algebraically independent over $K$; hence $A[x_1,\dots,x_d]$ is integrally closed (\cref{integral closure of a polynomial ring}). Let $T$ be the set of nonzero elements of the ring $R=k[x_1,\dots,x_d]\sub L$, so that the field $k_1=k(x_1,\dots,x_d)\sub L$ is equal to $T^{-1}R$; then
\[k_1\otimes_kA=(T^{-1}R)\otimes_kA=T^{-1}R\otimes_R(R\otimes_kA)=T^{-1}(R\otimes_kA)=T^{-1}A[x_1,\dots,x_d]\]
by the associativity of the tensor product, hence this domain is integrally closed (\cref{integral closure and localization}). But $L\otimes_kA$ is identified with $L\otimes_{k_1}(k_1\otimes_kA)$ and by definition $L$ is a finite separable extension of $k_1$; it follows therefore from the preceding argument that $L\otimes_kA$ is integrally closed.\par
In the general case, if $z$ is an element of $\Omega$ which is integral over $L\otimes_kA$, it satisfies a relation ofthe form $z^n+b_{n-1}z^{n-1}+\cdots+b_0=0$, where the $b_i$ belong to $L\otimes_kA$; then there exists a finitely generated sub-extension $L'$ of $L$ over $k$ such that the $b_i$ belong to $L'\otimes_kA$ for all $i$ and $z$ to $L'\otimes_kK$. Then it follows from the previous argument that $z\in L'\otimes_kA$ and hence $L\otimes_kA$ is integrally closed.
\end{proof}
\begin{remark}
Let $V$ be an affine irreducible algebraic variety over a field $k$ and $A$ the ring of functions regular on $V$ defined over $k$; if $A$ is integrally closed, $V$ is called normal over $k$; \cref{integrally closed algebra separable extension} shows that, if $V$ is normal over $k$, it remains normal over every separable extension $L$ of $k$.
\end{remark}
\begin{corollary}\label{integrally closed algebra tensor product}
Let $k$ be a field and $A$ and $B$ two integrally closed $k$-algebras. Suppose that the ring $A\otimes_kB$ is an integral domain and that the fields of fractions $K$ and $L$ of $A$ and $B$ respectively are separable over $k$. Then the ring $A\otimes_kB$ is an integrally closed domain.
\end{corollary}
\begin{proof}
As $A$ and $B$ are identified with subalgebras of $A\otimes_kB$, $K$ and $L$ are identified with subfields of the field of fractions $\Omega$ of $A\otimes_kB$ which are linearly disjoint over $k$ (A, \Rmnum{5}, $\S$2, no.3, proposition 5). It then follows from \cref{integrally closed algebra separable extension} that $A\otimes_kL$ and $K\otimes_kB$ are integrally closed domains; as their intersection is $A\otimes_kB$ (\cref{module quotient flat tensor intersection}), $A\otimes_kB$ is an integrally closed domain (\cref{integrally closed intersection is integrally closed}).
\end{proof}
\subsection{Integers over a graded ring}
All the graduations considered in this part are of type $\Z$; if $A$ is a graded ring and $i\in\Z$, $A_i$ denotes the set of homogeneous elements of degree $i$ of the ring $A$. Let $A$ be a graded ring and $B$ a graded $A$-algebra. Let $x$ be a homogeneous element of $B$ which is integral over $A$; then there is a relation
\[x^n+a_{n-1}x^{n-1}+\cdots+a_0=0\quad a_i\in A.\]
Let $m=\deg(x)$ and let $a_i'$ be the homogeneous componenet of degree $m(n-i)$ of $a_i$ for each $i$; then obviously
\begin{align}\label{graded ring integral equation homogeneous}
x^n+a_{n-1}'x^{n-1}+\cdots+a_0'=0
\end{align}
in other words $x$ satisfies an equation of integral dependence with homogeneous coefficients.\par
Let $A[X,X^{-1}]$ denote the ring of fractions $S^{-1}A[X]$ of the polynomial ring $A[X]$ in one indeterminate, $S$ being the multiplicative subset of $A[X]$ consisting of the powers $X^n$ of $X$; as $X$ is not a divisor of $0$ in $A[X]$ it is immediate that the $X^i$'s form a basis over $A$ of the $A$-module $A[X,X^{-1}]$. For every element $a\in A$ with homogeneous components $(a_i)_{i\in\Z}$, we write
\[j_A(a)=\sum_{i\in\Z}a_iX^i\in A[X,X^{-1}]\]
it is immediate that $j_A:A\to A[X,X^{-1}]$ is an injective ring homomorphism.
\begin{proposition}\label{graded ring integral closure in algebra is graded}
Let $A=\bigoplus_{i\in\Z}A_i$ be a graded ring and $B$ a graded $A$-algebra. Then the set $R$ of elements of $B$ integral over $A$ is a graded subalgebra of $B$. Moreover, if $A_i=0$ for $i<0$ and $R$ is a reduced ring, then $R_i=0$ for $i<0$.
\end{proposition}
\begin{proof}
The diagram
\[\begin{tikzcd}
A\ar[r,"\rho"]\ar[d,"j_A"]&B\ar[d,"j_B"]\\
A[X,X^{-1}]\ar[r,"\tilde{\rho}"]&B[X,X^{-1}]
\end{tikzcd}\]
(where $\rho$ is the homomorphism defining the $A$-algebra structure on $B$ and $\tilde{\rho}$ the homomorphism canonically derived from it) is commutative, as is immediately verified from the definition. Let $x$ be an element of $B$ integral over $A$; then $j_B(x)$ is integral over $A[X,X^{-1}]$ (\cref{integral element under homomorphism}) and it therefore follows from \cref{integral closure and localization} that there exists an integer $m>0$ such that $X^mj_B(x)$ is an element of $B[X]$ integral over $A[X]$. We then deduce from \cref{integral polynomial iff coefficient is integral} that the coefficients of the polynomial $X^mj_B(x)$ are integral over $A$; as these coefficients are by definition the homogeneous components of $x$, it is seen that these are integral over $A$, which proves that $R$ is a graded subalgebra of $B$.\par
Suppose now that $x\in R_i$ where $i<0$; the remark at the beginning of this part shows that $x$ satisfies an equation of the form (\ref{graded ring integral equation homogeneous}) where $a_k'\in A_{ki}$ for $1\leq k\leq n$. If $A_j=0$ for $j<0$, then $x^n=0$ and if $B$ is a reduced ring we conclude that $x=0$ and hence $R_i=0$ for all $i<0$ in this case.
\end{proof}
Recall that, if $A=\bigoplus_{i\in\Z}A_i$ is a graded ring and $S$ is a multiplicative subset of $A$ consisting of homogeneous elements, a graded ring structure is defined on $S^{-1}A$ by taking the set $(S^{-1}A)_i$ of homogeneous elements of degree $i$ to be the set of elements of the form $a/s$, where $a\in A$ and $s\in S$ are homogeneous and such that $\deg(a)-\deg(s)=i$.
\begin{lemma}\label{graded ring localization of homogeneous element is polynomial}
Let $A=\bigoplus_{i\in\Z}A_i$ be a graded integral domain and $S$ the set of nonzero homogeneous elements of $A$.
\begin{itemize}
\item[(a)] Every nonzero homogeneous element of $S^{-1}A$ is invertible, the ring $K=(S^{-1}A)_0$ is a field and the set of $i\in\Z$ such that $(S^{-1}A)_i\neq 0$ is a subgroup $q\Z$ of $\Z$ (where $q\neq 0$).
\item[(b)] Suppose that $q\geq 1$ and let $t$ be a non-zero element of $(S^{-1}A)_q$. Then the $K$-homomorphism $f$ of the polynomial ring $K[X]$ to $S^{-1}A$ which maps $X$ to $t$ extends to an isomorphism of $K[X,X^{-1}]$ onto $S^{-1}A$ and $S^{-1}A$ is integrally closed.
\end{itemize}
\end{lemma}
\begin{proof}
The assertions in (a) follow immediately from the definitions and the hypothesis that $A$ is an integral domain, for if $a/s$ and $b/t$ are two nonzero homogeneous elements of $S^{-1}A$ of degrees $i$ and $j$, $ab/st$ is a nonzero homogeneous element and of degree $i+j$. To show (b), we note that since $t$ is invertible in $S^{-1}A$ the homomorphism $f$ extends in a unique way to a homomorphism $\tilde{f}:K[X,X^{-1}]\to S^{-1}A$ and necessarily $\tilde{f}(X^{-1})=t^{-1}$. On the other hand, by definition of $q$, every nonzero homogeneous element of $S^{-1}A$ is of degree $qn$ ($n\in\Z$) and hence can be written uniquely in the form $\lambda t^n$ where $\lambda\in K$ (since $S^{-1}A$ is an integral domain); hence $\tilde{f}$ is bijective. Finally, we know that $K[X]$ is integrally closed and hence so is $K[X,X^{-1}]$ (\cref{integral closure and localization}), which completes the proof of the Lemma.
\end{proof}
\begin{proposition}\label{graded ring integral closure is graded}
Let $A=\bigoplus_{i\in\Z}A_i$ be a graded integral domain and $S$ the set of nonzero homogeneous elements of $A$. The integral closure $B$ of $A$ is then a graded subring of $S^{-1}A$. If further $A_i=0$ for $i<0$, then $B_i=0$ for $i<0$.
\end{proposition}
\begin{proof}
If $A=A_0$, the proposition is trivial. Otherwise we may apply \cref{graded ring localization of homogeneous element is polynomial}; the ring $S^{-1}A$ is an integrally closed domain and therefore $B\sub S^{-1}A$; as $S^{-1}A$ is graded, so is $B$ by \cref{graded ring integral closure in algebra is graded}; the latter assertion also follows from \cref{graded ring integral closure in algebra is graded}.
\end{proof}
\begin{corollary}\label{graded ring integrally closed iff homogeneous}
With the hypotheses and notation of \cref{graded ring integral closure is graded}, if every homogeneous element of $S^{-1}A$ which is integral over $A$ belongs to $A$, then $A$ is integrally closed.
\end{corollary}
\begin{proof}
Then $B_i\sub A$ for all $i\in\Z$, hence $B=A$.
\end{proof}
\begin{corollary}\label{graded ring integrally closed then A_0}
If $A=\bigoplus_{i\in\Z}A_i$ is an integrally closed graded domain, the domain $A_0$ is integrally closed.
\end{corollary}
\begin{proof}
The field of fractions $K_0$ of $A_0$ is identified (in the notation of \cref{graded ring integral closure is graded}) with a subring of the ring of homogeneous elements of degree $0$ of $S^{-1}A$; every element of $K_0$ integral over $A_0$ (and a fortiori over $A$) belongs therefore by hypothesis to $A_0$.
\end{proof}
\begin{corollary}
Let $A=\bigoplus_{i\in\Z}A_i$ be an integrally closed graded domain. Then, for every integer $d>0$, the ring $A^{(d)}$ is an integrally closed domain.
\end{corollary}
\begin{proof}
Let $U$ be the set of nonzero homogeneous elements of $A^{(d)}$ and let $x$ be a homogencous element of $U^{-1}A^{(d)}$ integral over $A^{(d)}$ and hence over $A$; as $x\in S^{-1}A$, $x$ belongs to $A$ by hypothesis; as its degree is divisible by $d$, it belongs to $A^{(d)}$ and it then follows from \cref{graded ring integrally closed iff homogeneous} that $A^{(d)}$ is integrally closed.
\end{proof}
\subsection{The lift of prime ideals}
Let $A$, $B$ be two rings and $\rho:A\to B$ a ring homomorphism. An ideal $\b$ of $B$ is said to \textbf{lie over} an ideal $\a$ of $A$ if $\a=\rho^{-1}(\b)$.\par
To say that a prime ideal $\mathfrak{P}$ of $B$ lies over an ideal $\p$ of $A$ therefore means that $\p$ is the image of $\mathfrak{P}$ under the continuous map $f^*:\Spec(B)\to\Spec(A)$ associated with $f$. Note that for there to exist an ideal of $B$ lying over the ideal $(0)$ of $A$, it is necessary and sufficient that $\rho:A\to B$ be injective.\par
Let $\a$ be an ideal of $A$. By taking quotients the homomorphism $\rho$ gives a homomorphism $\bar{\rho}:A/\a\to B/\b$. To say that $\b$ is an ideal of $B$ lying over $\a$ is equivalent to saying that $\a B\sub\b$ and that $\b/\a A$ is an ideal of $B/\a B$ lying over $(0)$.
\begin{lemma}\label{integral lift of prime lemma}
Let $\rho:A\to B$ be a ring homomorphism, $S$ a multiplicative subset of $A$, $i_A^S$ and $i_B^S$ the canonical homomorphisms, so that there is a commutative diagram
\[\begin{tikzcd}
A\ar[r,"\rho"]\ar[d,"i_A^S"]&B\ar[d,"i_B^S"]\\
S^{-1}A\ar[r,"S^{-1}\rho"]&S^{-1}B
\end{tikzcd}\]
Let $\p$ be a prime ideal of $A$ such that $\p\cap S=\emp$. Then $\b\mapsto S^{-1}\b$ is a bijective map of the set of ideals of $B$ lying over $\p$ and saturated with $S$ onto the set of ideals of $S^{-1}B$ lying over $S^{-1}\p$. In particular, $\mathfrak{P}\mapsto S^{-1}\mathfrak{P}$ is a bijection on the set of prime ideals of $B$ lying over $\p$ onto the set of prime ideals of $S^{-1}B$ lying over $S^{-1}\p$.
\end{lemma}
\begin{proof}
This follows from \cref{localization and ideals} and \cref{localization module saturation correspondence}.
\end{proof}
\begin{proposition}\label{integral ring extension field iff}
Let $A\sub B$ be integral domains, $B$ integral over $A$. Then $B$ is a field if and only if $A$ is a field.
\end{proposition}
\begin{proof}
IfA is a field, then, for all nonzero $y$ in $B$, $A[y]$ is by hypothesis  a finitely generated $A$-module. As $A[y]$ is an integral domain, it is a field and a fortiori $y$ is invertible in $B$ and hence $B$ is a field. Conversely, suppose that $B$ is a field and let $x$ be nonzero in $A$. As $x^{-1}\in B$ and $B$ is integral over $A$, there is an equation of integral dependence
\[x^{-n}+a_{n-1}z^{-(n-1)}+\cdots+a_0=0\]
where the $a_i\in A$; now this relation shows that
\[-z^{-1}=a_{n-1}+a_{n-2}z+\cdots+a_0z^{n-1}\in A,\]
hence $A$ is a field.
\end{proof}
\begin{corollary}\label{integral ring maximal ideal iff contraction is}
Let $\rho:A\to B$ be a ring homomorphism such that $B$ is integral over $A$, $\mathfrak{P}$ a prime ideal of $B$ and $\p=f^{-1}(\mathfrak{P})$. For $\p$ to be maximal, it is necessary and sufficient that $\mathfrak{P}$ is maximal.
\end{corollary}
\begin{proof}
Let $\bar{\rho}:A/\p\to B/\mathfrak{P}$ be the homomorphism derived from $f$ by taking quotients. Then $A/\p$ and $B/\mathfrak{P}$ are integral domains and $B/\mathfrak{P}$ is integral over $A/\p$. To say that $\p$ (resp. $\mathfrak{P}$) is maximal means that $A/\p$ (resp. $B/\mathfrak{P}$) is a field. The corollary then follows from \cref{integral ring extension field iff}.
\end{proof}
\begin{corollary}\label{integral ring extension prime lying over same contraction}
Let $\rho:A\to B$ a ring homomorphism such that $B$ is integral over $A$, $\p$ a prime ideal of $A$ and $\mathfrak{P}_1$ and $\mathfrak{P}_2$ two ideals of $B$ lying over $\p$ such that $\mathfrak{P}_1\sub\mathfrak{P}_2$. If $\mathfrak{P}_1$ and $\mathfrak{P}_2$ are prime, then $\mathfrak{P}_1=\mathfrak{P}_2$.
\end{corollary}
\begin{proof}
Let us write $S=A-\p$. Then $S^{-1}B$ is integral over $S^{-1}A$ by \cref{integral closure and localization}, $S^{-1}\p$ is a maximal ideal of $S^{-1}A$, $S^{-1}\mathfrak{P}_1$ and $S^{-1}\mathfrak{P}_2$ are ideals of $S^{-}B$ lying over $S^{-1}\p$ and $S^{-1}\mathfrak{P}_1\sub S^{-1}\mathfrak{P}_2$. As $S^{-1}\mathfrak{P}_1$ and $S^{-1}\mathfrak{P}_2$ are prime, they are maximal by \cref{integral ring maximal ideal iff contraction is} and hence $S^{-1}\mathfrak{P}_1=S^{-1}\mathfrak{P}_2$. Therefore $\mathfrak{P}_1=\mathfrak{P}_2$.
\end{proof}
\begin{corollary}\label{integral ring extension restriction is injective}
Let $B$ be an integral domain, $A$ a subring of $B$ such that $B$ is integral over $A$ and $f$ a homomorphism from $B$ to a ring $C$. If the restriction of $f$ to $A$ is injective, then $f$ is injective.
\end{corollary}
\begin{proof}
If $\b$ is the kernel of $f$, the hypothesis means that $\b\cap A=(0)$. As $B$ is an integral domain, \cref{integral ring extension prime lying over same contraction} may be applied taking $\p$ and $\mathfrak{P}_1$ to be the ideal $(0)$ of $A$ and the ideal $(0)$ of $B$ respectively, whence $\b=(0)$.
\end{proof}
\begin{corollary}\label{integral finite maximal ideal lying over prop}
Let $A\sub B$ be rings, $B$ integral over $A$ and $\m$ a maximal ideal of $A$. Suppose that there are only finitely many distinct maximal ideals $\mathfrak{M}_1,\dots,\mathfrak{M}_s$ in $B$ lying over $\m$. Let $\mathfrak{Q}_i$ be the saturation of $\m B$ with respect to $\mathfrak{M}_i$. Then:
\begin{itemize}
\item[(a)] In the ring $B/\mathfrak{Q}_i$, the zero divisors are the elements of $\mathfrak{M}_i/\mathfrak{Q}_i$, and they are nilpotent.
\item[(b)] $\m B=\bigcap_{i=1}^{s}\mathfrak{Q}_i=\prod_{i=1}^{s}\mathfrak{Q}_i$.
\item[(c)] The canonical homomorphism $B/\m B\to\prod_{i=1}^{s}B/\mathfrak{Q}_i$ is an isomorphism.
\end{itemize} 
\end{corollary}
\begin{proof}
For a prime ideal of $B$ to contain $\m B$, it is necessary and sufficient that its intersection with $A$ contains $\m$, and hence that it lies over $\m$, since $\m$ is maximal in $A$. Therefore $\mathfrak{M}_i$ are therefore the only prime ideals of $B$ containing $\m B$ and therefore $\sqrt{\m B}=\bigcap_{i=1}^{s}\mathfrak{M}_i$. By definition of $\mathfrak{Q}_i$, the image of an element of $B-\mathfrak{M}_i$ in $B/\mathfrak{Q}_i$ is not a divisor of $0$. On the other hand, as the $\mathfrak{M}_i$ are distinct maximal ideals, for every index $i$ there exists an element $a_i$ belonging to $\bigcap_{j\neq i}\mathfrak{M}_j$ and not to $\mathfrak{M}_i$. Then for all $x\in\mathfrak{M}_i$ we have $a_ix\in\sqrt{\m B}$, hence the image of $a_ix$ in $B/\mathfrak{Q}_i$ is nilpotent, and, as that of $a_i$ is not a divisor of $0$, we conclude that the image of $x$ is nilpotent. In other words $\mathfrak{M}_i$ is the radical of $\mathfrak{Q}_i$ and proves (a).\par
It follows that $\mathfrak{Q}_i$ are relatively coprime in pair, and (c) will be a concequence of (b). To establish (b), we note that in the ring $B/\m B$ the $\mathfrak{M}_j/\m B$ are the only maximal ideals and $\mathfrak{Q}_j/\m B$ is the saturation of $(0)$ with respect to $\mathfrak{M}_j/\m B$. We may therefore restrict our attention to the case $\m B=(0)$. The assertion of (b) then follows from \cref{localization homomorphism inj surj iff}, by the definition of a saturation of $(0)$.
\end{proof}
\begin{theorem}\label{integral ring lying over prime exist}
Let $\rho:A\to B$ be a ring homomorphism such that $B$ is integral over $A$ and let $\p$ be a prime ideal of $A$ containing $\ker\rho$. Then there exists a prime ideal $\mathfrak{P}$ of $B$ such that $\mathfrak{P}\cap A=\p$.
\end{theorem}
\begin{proof}
By consider the image $\rho(A)$ in $B$, we may assume that $\rho$ is injective. Suppose first that $A$ is a local ring and $\p$ the maximal ideal of $A$. Then, for every maximal ideal $\mathfrak{M}$ of $B$, $\mathfrak{M}^c$ is a maximal ideal of $A$ and hence equal to $\p$, which proves the theorem in this case (since $B$ contains $A$ by hypothesis and is therefore not reduced to $0$). In the general case, let us write $S=A-\p$, then $S^{-1}A$ is a local ring whose maximal ideal is $S^{-1}\p$, $S^{-1}f:S^{-1}A \to S^{-1}A$ is injective and $S^{-1}B$ is integral over $S^{-1}A$. Then there exists a prime ideal $\mathfrak{P}'$ of $S^{-1}B$ lying over $S^{-1}\p$ and we know that $\mathfrak{P}'=S^{-1}\mathfrak{P}$ where $\mathfrak{P}$ is a prime ideal of $B$ lying over $\p$ by \cref{integral lift of prime lemma}.
\end{proof}
\begin{corollary}\label{integral ring lying over extend prime}
Let $\rho:A\to B$ be a ring homomorphism such that $B$ is integral over $A$, $\a$ and $\p$ two ideals of $A$ such that $\a\sub\mathfrak{P}$ and $\b$ an ideal of $B$ lying over $\a$. Suppose that $\p$ is prime, then there exists a prime ideal $\mathfrak{P}$ of $B$ lying over $\p$ and containing $\b$.
\end{corollary}
\begin{proof}
If $\bar{\rho}:A/\a\to B/\b$ is the homomorphism derived from $f$ by taking quotients, then $\bar{\rho}$ is injective by hypothesis and $B/\b$ is integral over $A/\a$, so the claim follows from \cref{integral ring lying over prime exist}.
\end{proof}
\begin{corollary}\label{integral maximal spec correspondence}
Let $A$ be a ring and $B$ a ring containing $A$ and integral over $A$. Then there exists one-to-one correspondence between maximal ideals of $A$ and maximal ideals of $B$.
\end{corollary}
\begin{proof}
This follows from \cref{integral ring maximal ideal iff contraction is} and \cref{integral ring lying over prime exist}.
\end{proof}
\begin{corollary}
Let $A$ be a ring and $B$ a ring containing $A$ and integral over $A$. If $\mathfrak{R}$ is the Jacobson radical of $B$, then $\mathfrak{R}\cap A$ is the Jacobson radical of $A$.
\end{corollary}
\begin{corollary}[\textbf{Going-up theorem}]\label{integral ring extension going up prop}
Let $A$ be a ring and $B$ a ring containing $A$ and integral over $A$. Let $\p_1\subset\cdots\subset\p_n$ be a chain of prime ideals of $A$ and $\mathfrak{P}_1\subset\cdots\subset\mathfrak{P}_m$ a chain of prime ideals of $B$ such that $\mathfrak{P}_i\cap A=\p_i$ for $1\leq i\leq m$ and $m<n$. Then the chain $\mathfrak{P}_1\subset\cdots\subset\mathfrak{P}_m$ can be extended to a chain $\mathfrak{P}_1\subset\cdots\subset\mathfrak{P}_n$ such that $\mathfrak{P}_i\cap A=\p_i$ for $1\leq i\leq n$.
\end{corollary}
\begin{proof}
By induction we reduce immediately to the case $m=1$, $n=2$, and this follows from \cref{integral ring lying over extend prime}.
\end{proof}
\begin{corollary}\label{integral ring closed map on spec}
Let $\rho:A\to B$ be a ring homomorphism such that $B$ is integral over $A$. Then the associated continuous map $\rho^*:\Spec(B)\to\Spec(A)$ is closed.
\end{corollary}
\begin{proof}
For every ideal $\b$ of $B$, $B/\b$ is integral over $B$, hence also over $A$ and $\Spec(B/\b)$ is identified with the closed subspace $V(\b)$ of $\Spec(B)$. To show that $\rho^*$ is closed, we see then (replacing $B$ by $B/\b$) that it is sufficient to prove that the image of $\Spec(B)$ under $\rho^*$ is a closed subset of $\Spec(A)$. Now it follows from \cref{integral ring lying over prime exist} that this image is just the set of prime ideals of $A$ containing the ideal $\ker\rho$ and this set is closed by definition of the topology on $\Spec(A)$.
\end{proof}
\begin{corollary}\label{integral extension homomorphism to algebraically closed}
Let $A$ be a ring, $B$ a ring containing $A$ and integral over $A$ and $\phi$ a homomorphism from $A$ to an algebraically closed field $\Omega$. Then $\phi$ can be extended to a homomorphism from $B$ to $\Omega$.
\end{corollary}
\begin{proof}
Let $\p$ be the kernel of $\phi$, which is a prime ideal since $\phi(A)\sub\Omega$ is an integral domain. Let $\mathfrak{P}$ be a prime ideal of $B$ lying over $\p$. Then $A/\p$ is canonically identified with a subring of $B/\mathfrak{P}$ and $B/\mathfrak{P}$ is integral over $A/\p$. The homomorphism $\phi$ defines, by taking the quotient, an isomorphism of $A/\p$ onto the subring $\phi(A)$ of $\Omega$, which can be extended to an isomorphism $\psi$ of the field of fractions $K$ of $A/\p$ onto a subfield of $\Omega$. As the field of fractions $L$ of $B/\mathfrak{P}$ is algebraic over $K$, $\psi$ can be extended to an isomorphism $\psi'$ of $L$ onto a subfield of $\Omega$. If $\pi:B\to B/\mathfrak{P}$ is thc canonical homomorphism, $\psi'\circ\pi$ is a homomorphism from $B$ to $\Omega$ extending $\phi$.
\end{proof}
\begin{proposition}\label{integral ring extension localization at prime and lying over}
Let $\rho:A\to B$ be a ring homomorphism such that $B$ is integral over $A$, $\p$ a prime ideal of $A$, and $S=A-\p$. If $(\mathfrak{P}_i)_{i\in I}$ is the family of all the prime ideals of $B$ lying over $\p$ and $T=\bigcap_{i\in I}(B-\mathfrak{P}_i)$, then $S^{-1}B=T^{-1}B$.
\end{proposition}
\begin{proof}
By definition $\rho(S)\sub T$ and, as $\rho(S)^{-1}B=S^{-1}B$, it suffices to prove, by virtue of \cref{localization ring isomorphism iff saturation}, that, if a prime ideal $\mathfrak{Q}$ of $B$ does not meet $\rho(S)$, it does not meet $T$ either. Now, suppose that $\mathfrak{Q}\cap\rho(S)=\emp$ and let $\q=\rho^{-1}(\mathfrak{Q})$. Then $\q\cap S=\emp$, in other words $\q\sub\p$. As $\mathfrak{Q}$ lies over $\q$ by definition, it follows from \cref{integral ring lying over extend prime} that there is an index $i$ such that $\mathfrak{Q}_i\sub\p_i$ and hence $\mathfrak{Q}\cap T=\emp$, which completes the proof.
\end{proof}
\begin{corollary}\label{integral closure of localization at prime and lying over}
Let $A$ be an integral domain, $\widebar{A}$ its integral closure, and $\p$ a prime ideal of $A$. If $(\mathfrak{P}_i)_{i\in I}$ is the family of all prime ideals of $\widebar{A}$ lying over $\p$ and $T=\bigcap_{i\in I}(\widebar{A}-\mathfrak{P}_i)$, then $T^{-1}\widebar{A}$ is the integral closure of $A_\p$.
\end{corollary}
\begin{proof}
By \cref{integral closure and localization}, the integral closure of $A_\p$ is $\widebar{A}_\p$, which equals to $T^{-1}\widebar{A}$ by \cref{integral ring extension localization at prime and lying over}.
\end{proof}
\begin{proposition}\label{integral finite ring lying over prime finite}
Let $\rho:A\to B$ be a ring homomorphism sueh that $B$ is a finitely generated $A$-module. Then, for every prime ideal $\p$ of $A$, the set of prime ideals of $B$ lying over $\p$ is finite.
\end{proposition}
\begin{proof}
By taking localization and quotient, we may reduce the problem to $\p=(0)$ and $A$ is a field. The ring $B$ is then an $A$-algebra of finite rank and therefore Artinian and we know that in such an algebra there is only a finite number of prime ideals.
\end{proof}
\begin{proposition}\label{integral closed ring extension lying contract prime}
Let $A$ be an integrally closed domain and $B$ a ring containing $A$ and integral over $A$. Suppose that $B$ is a torsion-free $A$-module. Let $\p\sub\q$ be two prime ideals of $A$ and $\mathfrak{Q}$ a prime ideal of $A$ lying over $\q$. Then there exists a prime ideal $\mathfrak{P}$ of $B$ lying over $\p$ and such that $\mathfrak{P}\sub\mathfrak{Q}$.
\end{proposition}
\begin{proof}
Suppose first that $B$ is an integral domain. Let $K$, $L$ be the fields of fractions of $A$ and $B$ respectively. Let $\Omega$ be the algebraic closure of $L$ and $\widebar{A}$ the integral closure of $A$ in $\Omega$, so that $A\sub B\sub\widebar{A}$. Let $\mathcal{P}$ be a prime ideal of $\widebar{A}$ lying over $\p$, then by \cref{integral ring lying over extend prime} there is a prime ideal $\mathcal{Q}$ of $\widebar{A}$ lying over $\q$ and such that $\mathcal{P}\sub\mathcal{Q}$. Finally let $\mathcal{Q}_1$ be a prime ideal of $\widebar{A}$ lying over $\mathfrak{Q}$. By \cref{integrally closed normal extension transitive action}, there exists a $K$-automorphism $\sigma$ of $\Omega$ such that $\sigma(\mathcal{Q})=\mathcal{Q}_1$. Then $\sigma(\mathcal{P})$ is a prime ideal of $\widebar{A}$ lying over $\p$ such that $\sigma(\mathcal{P})\sub\mathcal{Q}_1$ and hence $\mathfrak{P}=B\cap\sigma(\mathcal{P})$ is a prime ideal of $B$ lying over $\p$ and contained in $\mathfrak{Q}$.
\[\begin{tikzcd}[row sep=8pt,column sep=8pt]
\sigma(\mathcal{P})\ar[rr,hookrightarrow]&&\mathcal{Q}_1&\\
&\mathcal{P}\ar[lu,swap,"\sigma"]\ar[rr,hookrightarrow]&&\mathcal{Q}\ar[lu,swap,"\sigma"]\\
\mathfrak{P}\ar[uu,no head]\ar[rr,hookrightarrow]&&\mathfrak{Q}\ar[uu,no head]&\\
&{}&&{}\\
\p\ar[uu,no head]\ar[rr,hookrightarrow]&&\q\ar[uu,no head]&\\
&\p\ar[lu,equal]\ar[uuuu,no head,crossing over]\ar[rr,hookrightarrow]&&\q\ar[lu,equal]\ar[uuuu,no head]
\end{tikzcd}\]

We pass to the general case. As $A$ is an integral domain and $\mathfrak{Q}$ is prime, the subsets $A\setminus\{0\}$ and $B\setminus\mathfrak{Q}$ of $B$ are multiplicative; then their product $S=(A\setminus\{0\})(B\setminus\mathfrak{Q})$ is a multiplicative subset of $B$ which does not contain $0$ since the non-zero elements of $A$ are not divisors of $0$ in $B$. Then there exists a prime ideal $\mathfrak{M}$ of $B$ disjoint from $S$, in other words such that $\mathfrak{M}\sub\mathfrak{Q}$ and $\mathfrak{M}\cap A=0$. Let $\pi$ be the canonical homomorphism $B\to B/\mathfrak{M}$. Then the restriction of $\pi$ to $A$ is injective and hence $\pi(A)$ is integrally closed. Since $B/\mathfrak{M}$ is an integral domain, the first part of the proof proves that there exists a prime ideal $\mathfrak{N}$ of $B/\mathfrak{M}$ such that $\mathfrak{N}\cap\pi(A)=\pi(\p)$ and $\mathfrak{N}\sub\pi(\mathfrak{Q})$. The ideal $\mathfrak{P}=\pi^{-1}(\mathfrak{N})$ is a prime ideal of $B$ and $\mathfrak{P}\sub\mathfrak{Q}$, since $\mathfrak{Q}$ contains the kernel of $\pi$. As $\pi$ is injective on $A$, we see $\mathfrak{P}\cap A=\p$.
\end{proof}
\begin{corollary}[\textbf{Going-down theorem}]\label{integrally closed ring extension going down prop}
Let $A$ be an integrally closed domain and $B$ a ring containing $A$ and integral over $A$. Suppose that $B$ is a torsion-free $A$-module. Let $\p_1\sups\cdots\sups\p_n$ be a chain of prime ideals of $A$ and $\mathfrak{P}_1\sups\cdots\sups\mathfrak{P}_m$ a chain of prime ideals of $B$ such that $\mathfrak{P}_i\cap A=\p_i$ for $1\leq i\leq m$ and $m<n$. Then the chain $\mathfrak{P}_1\sups\cdots\sups\mathfrak{P}_m$ can be extended to a chain $\mathfrak{P}_1\sups\cdots\sups\mathfrak{P}_n$ such that $\mathfrak{P}_i\cap A=\p_i$ for $1\leq i\leq n$.
\end{corollary}
\begin{corollary}\label{integral prime lying over is minimal elment}
let $\p$ be a prime ideal of $A$. The prime ideals of $B$ lying over $\p$ are the minimal elements of the set $\mathcal{E}$ of prime ideals of $B$ containing $\p B$.
\end{corollary}
\begin{proof}
A prime ideal of $B$ lying over $\p$ is minimal in $\mathcal{E}$ by virtue of \cref{integral ring extension prime lying over same contraction}. Conversely, let $\mathfrak{Q}$ be a minimal element of $\mathcal{E}$. As $\mathfrak{Q}\cap A\sups\p$, the going down theorem shows that there exists a prime ideal $\mathfrak{P}$ of $B$ lying over $\p$ such that $\mathfrak{P}\sub\mathfrak{Q}$. As $\mathfrak{P}$ contains $\p B$, the hypothesis made on $\mathfrak{Q}$ implies that $\mathfrak{P}=\mathfrak{Q}$ and hence $\mathfrak{Q}$ lies over $\p$.
\end{proof}
\begin{proposition}\label{flat going down}
Let $\rho:A\to B$ be a flat homomorphism of rings. Then $\rho$ has the going-down property.
\end{proposition}
\begin{proof}
Let $\p_1\sub\p$ be prime ideals in $A$, $\mathfrak{P}$ a prime ideal in $B$ such that $\mathfrak{P}\cap A=\p$. Consider the localization $A_{\p}$ and $B_{\mathfrak{P}}$. From \cref{flat map localization is surjective on spec} we know the map $\rho^*:\Spec(B_{\mathfrak{P}})\to\Spec(A_{\p})$ is surjective, hence there is a prime ideal $\mathfrak{P}_1'$ lying over $\p_1$. Contracting $\mathfrak{P}_1'$ to $B$ gives the desired prime ideal.
\end{proof}
\begin{theorem}
Let $A\sub B$ be integral domains such that $A$ is integrally closed and $B$ is integral over $A$. Let $\phi:\Spec(B)\to\Spec(A)$ be the canonical map. For $b\in B$, let
\[f(X)=X^n+a_{n-1}X^{n-1}+\cdots+a_0\] 
be a monic polynomial with coefficients in $A$ having $b$ as a root and of minimal degree. Then
\[\phi(D(b))=\bigcup_{i=1}^{n}D(a_i).\]
In particular, $\phi$ is an open map.
\end{theorem}
\begin{proof}
By hypothesis, $f$ is the minimal polynomial of $b$ over the field of fraction of $A$. If we set $C=A[b]$ then $C\cong A[X]/(f(X))$ is a free $A$-module with basis $1,b,b^2,\dots,b^{n-1}$ and is hence faithfully flat over $A$. Suppose that $\mathfrak{P}\in D(b)$, so that $\mathfrak{P}\in\Spec(B)$ with $b\notin\mathfrak{P}$, and set $\p=\mathfrak{P}\cap A$. Then $\p\in\bigcup_iD(a_i)$, since otherwise $a_i\in\p$ for all $i$, and so $b^n\in\mathfrak{P}$, hence $b\in\mathfrak{P}$, which is a contradiction. So we have $\phi(D(b))\sub\bigcup_{i=1}^{n}D(a_i)$.\par
Conversely, let $\p\in\bigcup_iD(a_i)$ and let $\mathcal{P}$ be a prime of $C$ lying over $\p$ and $\mathfrak{P}$ a prime of $B$ lying over $\mathcal{P}$. Suppose that $b\in\sqrt{\p C}$, then for sufficiently large $m$ we have 
\[b^m=\sum_{i=1}^{n}p_ib^i,\quad p_i\in\p.\]
By the assumption on $f(X)$ we must have $m\geq n$. Then $X^m-\sum_{i=1}^{n}p_iX^i$ is divisible by $f(X)$ in $A[X]$, which implies that $X^m$ is divisible by $\widebar{f}(X)=X^n+\sum_{i=1}^{n-1}\widebar{a}_iX^i$ in $(A/\p)[X]$; since at least one of the $\widebar{a}_i$ is nonzero, this is a contradiction.\par 
Thus $b\notin\sqrt{\p C}$ and there exists some $\mathcal{P}_1\in\Spec(C)$ with $b\notin\mathcal{P}_1$ and $\p C\subset\mathcal{P}_1$. By \cref{integral prime lying over is minimal elment} we have $\mathcal{P}\sub\mathcal{P}_1$, so $\mathfrak{P}\in D(b)$ since otherwise $b\in\mathfrak{P}\cap C=\mathcal{P}\sub\mathcal{P}_1$, which is a contradiction. This proves $\phi(D(b))=\bigcup_{i=1}^{n}D(a_i)$. Since any open set of $\Spec(B)$ is a union of open sets of the form $D(a)$, it follows that $\phi$ is open.
\end{proof}
\begin{proposition}\label{Spec of ring map going up and going down}
A ring homomorphism $\rho:A\to B$ is said to have the \textbf{going-up property} (resp. the \textbf{going-down property}) if the conclusion of the going-up theorem (resp. the going-down theorem) holds for $B$ and its subring $\rho(A)$. Let $\rho^*:\Spec(B)\to\Spec(A)$ be the map associated with $\rho$.
\begin{itemize}
\item[(\rmnum{1})] Consider the following three statements:
\begin{itemize}
\item[(a)] $\rho^*$ is a closed map.
\item[(b)] $\rho$ has the going-up property.
\item[(c)] Let $\mathfrak{P}$ be any prime ideal of $B$ and let $\p=\mathfrak{P}^c$. Then $\rho^*:\Spec(B/\mathfrak{P})\to\Spec(A/\p)$ is surjective.
\end{itemize}
Then we have (a)$\Leftrightarrow$(b)$\Leftrightarrow$(c).
\item[(\rmnum{2})] Consider the following three statements:
\begin{itemize}
\item[(a')] $\rho^*$ is an open map.
\item[(b')] $\rho$ has the going-down property.
\item[(c')] Let $\mathfrak{P}$ be any prime ideal of $B$ and let $\p=\mathfrak{P}^c$. Then $\rho^*:\Spec(B_{\mathfrak{P}})\to\Spec(A_\p)$ is surjective.
\end{itemize}
Then we have (a')$\Rightarrow$(b')$\Leftrightarrow$(c').
\end{itemize}
\end{proposition}
\begin{proof}
The equivalences (b)$\Leftrightarrow$(c) and (b')$\Leftrightarrow$(c') are clear. For (a)$\Rightarrow$(b), assume that $\rho^*$ is closed, we claim that $\rho^*(V(\mathfrak{P}))=V(\p)$ if $\p=\mathfrak{P}^c$. To see this, we note that since $\rho^*$ is closed, $\rho^*(V(\mathfrak{P}))$ is closed and contains $\p$, so $V(\p)=\overline{\{\p\}}\sub\rho^*(V(\mathfrak{P}))$. But $\mathfrak{P}'\sups\mathfrak{P}$ implies $(\mathfrak{P}')^c\sups\mathfrak{P}^c=\p$, hence we also have $\rho^*(V(\mathfrak{P}))\sub V(\p_1)$, and therefore $\rho^*(V(\mathfrak{P}))=V(\p)$. In particular, the map $\rho^*$ is surjective on $V(\p)$, so if $\p\sub\p'$ then $\p'\in V(\p)$ and there is a $\mathfrak{P}'\in V(\mathfrak{P})$ such that $(\mathfrak{P}')^c=\p'$.\par
For (a')$\Rightarrow$(c'), we first prove the following \textit{downward property}: If $\p'\sub\p$ and $\p\in U$ for open set $U$, then $\p'\in U$. In fact, write $U=X\setminus C$ where $C$ is closed, then $\p'\in C$ would implies $V(\p')=\widebar{\{\p'\}}\sub C$ since $C$ is closed. But $\p\in V(\p')$ is not in $C$, contrdiction. By this property, if $\rho^*(U)$ is open, then for any $\p'\sups\p$ in $A$ and $\mathfrak{P}$ in $B$ such that $\mathfrak{P}^c=\p$. Then let $U$ be a neighborhood of $\mathfrak{P}$ in $B$, so $\rho^*(B)$ is a neighbrhood of $\p_1$ in $A$. Now since $\p'\sub\p$, we conclude $\p'\in f(U)$, so $\p'$ has a preimage $\mathfrak{P}'$. This shows the map $\rho^*:\Spec(B_{\mathfrak{P}})\to\Spec(A_\p)$ is surjective.\par
Now assume the going up property, we show that $\rho^*$ is closed. To this end, let $V(\b)$ be a closed subset of $\Spec(B)$. Since $V(\b)$ is identified with $\Spec(B/\b)$ and the induced map $A/\a\to B/\b$ satisfies the going up property (where $\a=\b^c$), we only need to show the image of $\Spec(B)$ under $\rho^*$ is closed. For this, let $T$ be the image of $\rho^*$, and let $\p\in\Spec(A)$ be in the closure of $T$. This means for every $f\in A-\p$ we have $D(f)\cap T\neq\emp$. Note that $D(f)\cap T$ is the image of $\Spec(B_f)$ in $\Spec(A)$. Hence we conclude that $B_f\neq 0$. Since $B_\p$ is the directed limit of the rings $B_f$ with $f\in A-\p$, we conclude that $B_\p\neq 0$ by \cref{ring direct limit zero iff one zero}. If $\mathcal{P}'$ is a prime ideal of $B_\p$, then $\p'=\mathcal{P}'\cap A$ is a prime ideal of $A$ such that $\p'\sub\p$, and it is in the image of $\rho^*$ since $\mathfrak{P}'\cap A=\p'$, where $\mathfrak{P}'=(i_B^\p)^{-1}(\mathcal{P}')$. As we assumed that $\rho$ satisfies the going up property, there exist a prime ideal $\mathfrak{P}$ of $B$ lying over $\p$, whence $\p\in T$ and $T$ is closed.
\end{proof}
\section{Group action on algebras}
\subsection{Group acting on an algebra}
Given a ring $R$, a $R$-algebra $A$ and a group $G$ acting on $A$, we shall say the action of $G$ is \textbf{$\bm{R}$-linear} or $G$ is a $R$-action if for each $\sigma\in G$, the map $x\mapsto\sigma(x)$ is an endomorphism of the $R$-algebra $A$. We shall denote by $A^G$ the set of elements of $A$ which are invariant under $G$. Clearly it is a sub-$R$-algebra of $A$. We shall say that the \textbf{action is locally finite} if every orbit of $G$ in $A$ is finite.
\begin{proposition}\label{algebra action integral over fix point}
Let $A$ be a $R$-algebra and $G$ a locally finite $R$-action on $A$. Then $A$ is integral over the subalgebra $A^G$.
\end{proposition}
\begin{proof}
For all $x\in A$, let $x_1,\dots,x_n$ be the distinct elements of the orbit of $x$ under $G$. For all $\sigma\in G$, there exists a permutation $\pi_\sigma\in\mathfrak{S}_n$ such that $\sigma(x_i)=x_{\pi_\sigma(i)}$ for $1\leq i\leq n$. Therefore the elementary symmetric functions of the $x_i$ are elements of $A$ which are invariant under $G$, in other words elements of $A^G$. As $x$ is a root of the monic polynomial $\prod_i(X-x_i)$ and the coefficients of this polynomial belong to $A^G$, $x$ is integral over $A^G$.
\end{proof}
\begin{theorem}\label{algebra action invariant subring finite over Noe}
Let $A$ be a finitely generated $R$-algebra and $G$ a locally finite $R$-action on $A$. Then $A$ is a finitely generated $A^G$-module. If further $R$ is Noetherian, $A^G$ is a finitely generated $R$-algebra.
\end{theorem}
\begin{proof}
Let $a_1,\dots,a_n$ generate the $R$-algebra $A$. Then we see $A=A^G[a_1,\dots,a_n]$ and the $a_i$ are integral over $A^G$ by \cref{algebra action integral over fix point}, the first assertion follows. The second is a consequence of the following lemma:
\begin{equation}\label{Artin-Tate lemma}
\parbox{\dimexpr\linewidth-6em}
{\strut
Let $A$ be a Noetherian ring, $B$ a finitely generated $A$-algebra and $C$ a sub-$A$-algebra of $B$ such that $B$ is integral over $C$. Then $C$ is a finitely generated $A$-algebra.
\strut}
\end{equation}

To prove the lemma, let $x_1,\dots,x_n$ be a finite system of generators of the $A$-algebra $B$. For all $i$, there exists by hypothesis a monic polynomial $P_i\in C[X]$ such that $P_i(x_i)=0$. Let $C'$ be the sub-$A$-algebra of $C$ generated by the coefficients of the $P_i$. Clearly the $x_i$ are integral over $C'$ and $B=C'[x_1,\dots,x_n]$, hence $B$ is a finitely generated $C'$-module. On the other hand $C'$ is a Noetherian ring, hence $C$ is a finitely generated $C'$-module, which proves that $C$ is a finitely generated $A$-algebra.
\end{proof}
Let $S$ be a multiplicative subset of a ring $A$ and $G$ a group acting on $A$ and for which $S$ is stable. Then, for all $\sigma\in G$, there exists a unique endomorphism $z\mapsto\sigma(z)$ of the ring $S^{-1}A$ such that $\sigma(a/1)=\sigma(a)/1$ for all $a\in A$, which is given by $\sigma(a/s)=\sigma(a)/\sigma(s)$. If $\tau$ is another element of $G$, then clearly $\sigma(\tau(z))=\sigma\tau(z)$ for all $z\in S^{-1}A$, whence the group $G$ operates on the ring $S^{-1}A$.
\begin{proposition}\label{algebra action and localization}
Let $A$ be a $R$-algebra, $G$ a locally finite $R$-action on $A$, $S$ a multiplicative subset of $A$ stable under $G$ and $S^G$ the set $S\cap A^G$. Then the canonical map $i_A^{S^G,S}:(S^G)^{-1}A\to S^{-1}A$ is an isomorphism which maps $(S^G)^{-1}A^G$ to $(S^{-1}A)^G$.
\end{proposition}
\begin{proof}
For all $s\in S$, let $s_1=s,s_2,\dots,s_n$ be the distinct elements of the orbit of $s$ under $G$. As $\prod_{i=1}^{n}s_i\in S^G$, the first assertion follows from \cref{localization ring isomorphism iff saturation}. Identifying canonically $(S^G)^{-1}A$ with $S^{-1}A$, clearly every element of $(S^G)^{-1}A^G$ is invariant under $G$. Conversely, let $a/s$ be an element of $(S^G)^{-1}A$ which is invariant under $G$. If $a_1,\dots,a_m$ are the distinct elements of the orbit of $a$ under $G$, then $a_j/s=a/s$ for $1\leq j\leq m$ and therefore there exists $t\in S^G$ such that $t(a_j-a)=0$ for $1\leq j\leq m$. In other words, $ta$ is invariant under $G$ and, as $a/s=(ta)/(ts)$, certainly $a/s\in (S^G)^{-1}A^G$.
\end{proof}
\begin{corollary}\label{algebra action invariant and fraction field}
Let $A$ be an integral domain, $K$ its field of fractions and $G$ a locally finite $R$-action on $A$. Then $G$ acts on $K$ and $K^G$ is the field of fractions of $A^G$.
\end{corollary}
\subsection{Decomposition group and inertial group}
Let $B$ be a ring and $G$ a group acting on $B$. Given a prime ideal $\mathfrak{P}$ of $B$ the subgroup of elements $\sigma\in G$ such that $\sigma(\mathfrak{P})=\mathfrak{P}$ is called the \textbf{decomposition group} of $\mathfrak{P}$ (with respect to $G$) and is denoted by $G^Z(\mathfrak{P})$. The ring of elements of $B$ invariant under $G^Z(\mathfrak{P})$ is called the \textbf{decomposition ring} of $\mathfrak{P}$ (with respect to $G$) and is denoted by $A^Z(\mathfrak{P})$.\par
We often write $G^Z$ and $A^Z$ instead of $G^Z(\mathfrak{P})$ and $A^Z(\mathfrak{P})$ respectively, when there is no ambiguity. For all $\sigma\in G^Z(\mathfrak{P})$, we also denote by $z\mapsto\sigma(z)$ the endomorphism of the ring $B/\mathfrak{P}$ derived from the endomorphism $x\mapsto\sigma(x)$ of $B$ by taking quotients; clearly the group $G^Z(\mathfrak{P})$ operates in this way on the ring $B/\mathfrak{P}$. The subgroup of $G^Z(\mathfrak{P})$ consisting of those $\sigma$ such that the endomorphism $z\mapsto\sigma(z)$ of $B/\mathfrak{P}$ is the identity is called the \textbf{inertia group} of $\mathfrak{P}$ (with respect to $G$) and denoted by $G^T(\mathfrak{P})$. Similarly, we write $A^T(\mathfrak{P})$ for the subring of $B$ invariant under $G^T(\mathfrak{P})$.\par
If $A$ is the subring of $B$ consisting of the invariants of $G$ (i.e., $A=B^G$), clearly we have
\[A\sub A^Z(\mathfrak{P})\sub A^T(\mathfrak{P})\sub B.\]
It follows from definition that, for $\rho\in G$,
\begin{align}\label{algebra decomposition group conjugation relation}
G^Z(\rho(\mathfrak{P}))=\rho G^Z(\mathfrak{P})\rho^{-1},\quad G^T(\rho(\mathfrak{P}))=\rho G^T(\mathfrak{P})\rho^{-1}.
\end{align}
We may also write $\mathfrak{P}^Z:=\mathfrak{P}\cap A^Z$ and $\mathfrak{P}^T=\mathfrak{P}\cap A^T$.\par
If, for all $\sigma\in G^Z(\mathfrak{P})$, $\bar{\sigma}$ denotes the automorphism $z\mapsto\widebar{\sigma(z)}$ of $B/\mathfrak{P}$, then the map $\sigma\mapsto\bar{\sigma}$ is a homomorphism (called canonical) of $G^Z(\mathfrak{P})$ to the group of automorphisms of $B/\mathfrak{P}$ leaving invariant the elements of $A^Z/(\mathfrak{P}^Z)$, and by definition $G^T(\mathfrak{P})$ is the kernel of this canonical homomorphism, so it is a normal subgroup of $G^Z(\mathfrak{P})$. If $\kappa(\mathfrak{P})$ is the field of fractions of $B/\mathfrak{P}$, every automorphism of $B/\mathfrak{P}$ can be extended uniquely to an automorphism of $\kappa(\mathfrak{P})$, so that $\sigma\mapsto\bar{\sigma}$ can be considered as a homomorphism from $G^Z(\mathfrak{P})$ to the group of automorphisms of $\kappa(\mathfrak{P})$. Note that, since $G^T(\mathfrak{P})$ is normal in $G^Z(\mathfrak{P})$, the ring $A^T(\mathfrak{P})$ is stable under $G$, in view of (\ref{algebra decomposition group conjugation relation}).
\begin{proposition}\label{algebra decomposition inertial and localization}
Let $B$ be a ring, $G$ a group acting on $B$, $A$ the ring of invariants of $G$, $\mathfrak{P}$ a prime ideal of $B$ and $S$ a multiplicative subset of $A$ not meeting $\mathfrak{P}$. Then 
\[G^Z(S^{-1}\mathfrak{P})=G^Z(\mathfrak{P}),\quad G^T(S^{-1}\mathfrak{P})=G^T(\mathfrak{P})\]
and, if the action of $G$ is locally finite, then
\[S^{-1}A^Z(\mathfrak{P})=A^Z(S^{-1}\mathfrak{P}),\quad S^{-1}A^T(\mathfrak{P})=A^T(S^{-1}\mathfrak{P}).\]
\end{proposition}
\begin{proof}
As the elements of $S$ are invariant under $G$, clearly, if $\sigma(\mathfrak{P})=\mathfrak{P}$, also $\sigma(S^{-1}\mathfrak{P})=S^{-1}\mathfrak{P}$. Conversely, suppose that $\sigma\in G$ is such that $\sigma(S^{-1}\mathfrak{P})=S^{-1}\mathfrak{P}$, then, if $x\in\mathfrak{P}$, we have $\sigma(x/1)\in S^{-1}\mathfrak{P}$ and there therefore exists $s\in S$ such that $s\sigma(x)\in\mathfrak{P}$, whence $\sigma(x)\in\mathfrak{P}$ since $\mathfrak{P}$ is prime and $s\notin\mathfrak{P}$. This proves that $\sigma(\mathfrak{P})\sub\mathfrak{P}$ and it can be similarly shown that $\sigma^{-1}(\mathfrak{P})\sub\mathfrak{P}$, hence $\sigma(\mathfrak{P})=\mathfrak{P}$ and $\sigma\in G^Z(\mathfrak{P})$. If $\sigma\in G^T(\mathfrak{P})$, then $\sigma(x)-x\in\mathfrak{P}$ for all $x\in B$, hence also, for all $s\in S$,
\[\sigma(x/s)-(x/s)=(\sigma(x)-x)/s\in S^{-1}\mathfrak{P}\]
and therefore $\sigma\in G^T(S^{-1}\mathfrak{P})$. Conversely, suppose that $\sigma\in G^T(S^{-1}\mathfrak{P})$. Then, for all $x\in B$, $\sigma(x/1)-(x/1)\in S^{-1}\mathfrak{P}$ and therefore there exists $s\in S$ such that $s(\sigma(x)-x)\in\mathfrak{P}$, whence as above $\sigma(x)-x\in\mathfrak{P}$, which proves that $\sigma\in G^T(\mathfrak{P})$. The last two assertions follow from \cref{algebra action and localization}.
\end{proof}
\begin{proposition}\label{algebra finite group action prop}
Let $B$ be a ring, $G$ a finite group acting on $B$ and $A$ the ring of invariants of $G$ so that $B$ is integral over $A$.
\begin{itemize}
\item[(a)] Given two prime ideals $\mathfrak{P}$, $\mathfrak{Q}$ of $B$ lying over the same prime ideal $\p$ of $A$, there exists $\sigma\in G$ such that $\sigma(\mathfrak{P})=\mathfrak{Q}$. In other words, $G$ operates transitively on the set of prime ideals of $B$ lying over $\p$.
\item[(b)] Let $\mathfrak{P}$ be a prime ideal of $B$ and $\p=\mathfrak{P}\cap A$. Then $\kappa(\mathfrak{P})$ is a normal extension of $\kappa(\p)$ and the canonical homomorphism $\sigma\mapsto\bar{\sigma}$ defines by taking the quotient an isomorphism of $G^Z(\mathfrak{P})/G^T(\mathfrak{P})$ onto $\Gal(\kappa(\mathfrak{P})/\kappa(\p))$. In other words, we get the following exact sequence
\[\begin{tikzcd}
1\ar[r]&G^T(\mathfrak{P})\ar[r]&G^Z(\mathfrak{P})\ar[r]&\Gal(\kappa(\mathfrak{P})/\kappa(\p))\ar[r]&1
\end{tikzcd}\] 
\end{itemize}
\end{proposition}
\begin{proof}
If $x\in\mathfrak{Q}$, then $\prod_{\sigma\in G}\sigma(x)\in\mathfrak{Q}\cap A=\p\sub\mathfrak{P}$. Hence there exists $\sigma\in G$ such that $\sigma(x)\in\mathfrak{P}$, that is $x\in\sigma^{-1}(\mathfrak{P})$. We then conclude that $\mathfrak{Q}\sub\bigcup_{\sigma\in H}\sigma(\mathfrak{P})$, and hence ($G$ is finite) $\mathfrak{Q}\sub\sigma(\mathfrak{P})$ for some $\sigma\in G$. Since $\mathfrak{Q}$ and $\sigma(\mathfrak{P})$ both lie over $\p$, we see $\mathfrak{Q}=\sigma(\mathfrak{P})$.\par
To see that $\kappa(\mathfrak{P})$ is a normal extension of $\kappa(\p)$, it suffices to prove that every element $\bar{x}\in B/\mathfrak{P}$ is a root of a polynomial $P$ in $\kappa(\p)[X]$ all of whose roots are in $B/\mathfrak{P}$. Now, let $x\in B$ be a representative of the class $\bar{x}$. Then the polynomial $P(X)=\prod_{\sigma\in G}(X-\sigma(x))$ has all its coefficients in $A$. Let $\bar{P}(X)$ be the polynomial in $(A/\p)[X]$ which is the image of $P(X)$, then we see $\bar{P}(X)=\prod_{\sigma\in G}(X-\widebar{\sigma(x)})$ and therefore solves the problem.\par
Clearly, for all $\sigma\in G^Z$, $\bar{\sigma}$ is a $\kappa(\p)$-automorphism of $\kappa(\mathfrak{P})$. It remains to verify that $\sigma\mapsto\bar{\sigma}$ maps $G^Z$ onto the group of all $\kappa(\p)$-automorphisms of $\kappa(\mathfrak{P})$. Since $\kappa(\p)$ and $\kappa(\mathfrak{P})$ are not changed under localization and by \cref{algebra decomposition inertial and localization} neither $G^Z$ nor its operation on $\kappa(\mathfrak{P})$ is changed, we may therefore restrict our attention to the case where $\p$ is maximal, in which case we know that so is $\mathfrak{P}$ and every element of $\kappa(\mathfrak{P})$ is therefore of the form $\bar{x}$ for some $x$ in $B$. It has been seen above that such an element is a root of a polynomial in $\kappa(\p)[X]$ of degree smaller than $|G|$. As every finite separable extension of $\kappa(\p)$ admits a primitive element, it is seen that every finite separable extension of $\kappa(\p)$ contained in $\kappa(\mathfrak{P})$ is of degree smaller than $|G|$, whence the separable closure $\kappa(\p)^{\text{sep}}$ in $\kappa(\mathfrak{P})$ is of degree smaller than $|G|$. Let $y\in B$ be an element such that $\bar{y}$ is a primitive element of $\kappa(\p)^{\text{sep}}$. The ideals $\sigma(\mathfrak{P})$ for $\sigma\in G-G^Z$ are maximal and distinct from $\mathfrak{P}$ by definition. By Chinese remainder therorem there then exists $x\in B$ such that $x\equiv y$ mod $\mathfrak{P}$ and $x\in\sigma(\mathfrak{P})$ for all $\sigma\in G-G^Z$. Now let $\theta$ be a $\kappa(\p)$-automorphism of $\kappa(\mathfrak{P})$ and let $P(X)=\prod_{\sigma\in G}(X-\widebar{\sigma(x)})$. As $\bar{x}$ is a root of $P$ and $P\in\kappa(\p)[X]$, $\theta(\bar{x})$ is also a root of $P$ in $\kappa(\mathfrak{P})$ and hence there exists $\tau\in G$ such that
\[\theta(\bar{x})=\tau(\bar{x}).\]
Since $\sigma(\bar{x})=0$ for all $\sigma\in G-G^Z$ and $\theta(\bar{x})\neq 0$, we conclude that necessarily $\tau\in G^Z$. As $\theta$ and $\tau$ have the same value for the primitive element $\bar{x}=\bar{y}$ of $\kappa(\p)^{\text{sep}}$, they coincide on $\kappa(\p)^{\text{sep}}$ and, as $\kappa(\mathfrak{P})$ is an inseparable extension of $\kappa(\p)^{\text{sep}}$, they coincide on $\kappa(\mathfrak{P})$.
\end{proof}
\begin{corollary}\label{algebra finite group action homomorphism same restriction}
With the hypotheses and notation of \cref{algebra finite group action prop}, let $f_1$ and $f_2$ be two homomorphism of $B$ to a field $\Omega$ with the same restriction to $A$. Then there exists a $\sigma\in G$ such that $f_2=f_1\circ\sigma$.
\end{corollary}
\begin{proof}
Let $\mathfrak{P}_i$ be the kernel of $f_i$, which is a prime ideal of $B$. By hypothesis $\mathfrak{P}_1\cap A=\mathfrak{P}_2\cap A$ and this intersection is a prime ideal $\p$ of $A$. There therefore exists $\tau\in G$ such that $\tau(\mathfrak{P}_2)=\mathfrak{P}_1$. Replacing $f_1$ by the homomorphism $f_1\circ\tau$, we may then assume that $\mathfrak{P}_1=\mathfrak{P}_2$ (an ideal which we shall denote by $\mathfrak{P}$). By taking the quotient and localization we then derive from $f_1$ and $f_2$ two injective homomorphisms $\tilde{f}_1$, $\tilde{f}_2$ from $\kappa(\mathfrak{P})$ to $\Omega$. As $\kappa(\mathfrak{P})$ is a normal extension of $\kappa(\p)$, we see $\tilde{f}_2\circ\tilde{f}_1^{-1}$ is a $\kappa(\p)$-automorphism of $\kappa(\mathfrak{P})$ (\cref{field ext is a spliting iff}) and it follows from \cref{algebra finite group action prop} that $\tilde{f}_2\circ\tilde{f}_1^{-1}=\bar{\sigma}$ for some $\sigma\in G$. In particular, for all $x\in B$ the elements $f_2(x)$ and $f_1(\sigma(x))$ are equal.
\end{proof}
\begin{proposition}\label{algebra finite subgroup fixed ring prop}
Let $B$ be a ring, $G$ a finite group acting on $B$ and $H$ a subgroup of $G$. Let $A^G$ and $A^H$ be the rings of invariants of $G$ and $H$, respectively, so that $A^G\sub A^H$. For a prime ideal $\mathfrak{P}$ of $B$, set $\p^G=\mathfrak{P}\cap A^G$ and $\p^H=\mathfrak{P}\cap A^H$.
\begin{itemize}
\item[(a)] For $H$ to be contained in the decomposition group $G^Z(\mathfrak{P})$, it is necessary and sufficient that $\mathfrak{P}$ be the only prime ideal of $A$ lying over $\p^H$.
\item[(b)] If $H$ contains $G^Z(\mathfrak{P})$, then 
\begin{itemize}
\item[($\alpha$)] The rings $A^G/\p^G$ and $A^H/\p^H$ have the same field of fractions;
\item[($\beta$)] The maximal ideal of the local ring $(A^H)_{\p^H}$ is generated by the image of $\p^G$.
\end{itemize}
\item[(c)] Suppose further that $B$ is an integral domain and that $\bigcap_{n=1}^{\infty}(\p^G)^nB_{\mathfrak{P}}=0$, then the conditions of (b) imply that $G^Z(\mathfrak{P})$ preserves the elements of $A^H$.
\end{itemize}
\end{proposition}
\begin{proof}
It follows from \cref{algebra finite group action prop}(a) that the prime ideals of $B$ lying over $\p^H$ are the ideals of the form $\sigma(\mathfrak{P})$ where $\sigma\in H$; whence immediately (a). We now prove (b); by taking localization, we may assume that $\p^G$ is maximal. To establish ($\alpha$) it will be sufficient to prove that
\begin{align}\label{algebra finite subgroup fixed ring prop-1}
A^H=A^G+\p^H,
\end{align}
for this will show that the fields $(A^G/\p^G)$ and $(A^H)/(\p^H)$ are canonically isomorphic. By \cref{algebra finite group action prop} there is only a finite number of primes of $B$ lying over $\p^G$, therefore there is only a finite number of prime ideals of $A^H$ lying over $\p^G$; let $\n_1,\dots,\n_r$ denote those of these ideals that are different from $\p^H$. Let $x$ be an element of $A^H$, as the ideals $\p^H$ and $\n_i$ are maximal, there exists $y\in A^H$ such that $y\equiv x$ mod $\p^H$ and $y\in\n_i$ for $1\leq j\leq r$. Let $y_1=y,y_2,\dots,y_s$ be distinct elements of the orbit of $y$ under $G$. Clearly
\[z=y_1+y_2+\cdots+y_s\in A^G\]
and to establish (\ref{algebra finite subgroup fixed ring prop-1}) it will be sufficient to show that $y_i\in\mathfrak{P}$ for $i\geq 2$, for then we shall deduce that $z-y\in\mathfrak{P}\cap A^H=\p^H$, whence $x\in A^G+\p^H$ since $x\equiv y$ mod $\p^H$. Now let $i\geq 2$ and $\sigma\in G$ be such that $\sigma(y)=y_i$. Then $\sigma^{-1}(\mathfrak{P})$ does not lie over $\p^H$, for otherwise there would exist $\tau\in H$ such that $\sigma^{-1}(\mathfrak{P})=\tau(\mathfrak{P})$, whence $\sigma\tau\in G^Z\sub H$ (\cref{algebra finite group action prop}(a)). But as $y\in A^H$ and $\sigma(\tau(y))=\sigma(y)\neq y$, this is a contradiction. We conclude that $\sigma^{-1}(\mathfrak{P})$ lies over one of the ideals $\n_j$ and as $y\in\n_j$ by construction, we see $y\in\sigma^{-1}(\mathfrak{P})$ or $y_i=\sigma(y)\in\mathfrak{P}$.\par
To prove ($\beta$) we show that $\p^H$ is contained in the saturation $\q$ of the ideal $(\p^G)A^H$ with respect to $\p^H$. As $\p^H$ is contained in none of the $\n_i$, by \cref{prime ideal contained in union} it suffices to prove that
\begin{align}\label{algebra finite subgroup fixed ring prop-2}
\p^H\sub\q\cup\n_1\cup\cdots\cup\n_r.
\end{align}
For this, we consider an element $x\in\p^H$ belonging to none of the $\n_i$. Let $x_1=x,x_2,\dots,x_m$ be the distinct elements of the orbit of $x$ under $G$. Write $u=x_1x_2\cdots x_m$ and $v=x_2\cdots x_m$, then $u\in A^G$. On the other hand, if $\tau\in H$ then $\tau(x)=x$ and hence necessarily $\tau(x_i)\neq x$ for $i\geq 2$, which shows that $\tau(v)=v$ and hence $v\in A^H$. It can be shown as in the proof of ($\alpha$) that, if $\sigma\in G$ is such that $\sigma(x)=x_i$ where $i\geq 2$, then $\sigma^{-1}(\mathfrak{P})$ lies over one of the $\n_j$ and, as $x\notin\n_j$, we have $x\notin\sigma^{-1}(\mathfrak{P})$ whence $x_i\notin\mathfrak{P}$. We conclude that $v\notin\mathfrak{P}$ and therefore $v\notin\p^H$. On the other hand, clearly $u\in\mathfrak{P}\cap A^G=\p^G$ and the relation $u=xv$ then shows that $x$ is in the saturation of $\p^G(A^H)$ with respect to $\p^H$ and hence establishes (\ref{algebra finite subgroup fixed ring prop-2}).\par
Suppose that $B$ is an integral domain, that $\bigcap_{n=0}^{\infty}(\p^G)^nB_\mathfrak{P}=0$ and that conditions ($\alpha$) and ($\beta$) of (b) hold. With the same notation as in (b), by localization we may assume also that the ideal $\p^G$ is maximal. The hypotheses ($\alpha$) and ($\beta$) then imply that for all $n>0$ we have (by induction)
\[(A^H)_{\p^H}=A^G+(\p^G)^n(A^H)_{\p^H}.\]
Then let $\sigma$ be an element of $G^Z$ and $x$ be an element of $A^H$. For all $n>0$, there exists $a_n\in A^G$ such that $x-a_n\in(\p^G)^n(A^H)_{\p^H}\sub(\p^G)^nB_{\mathfrak{P}}$. As $\sigma(a_n)=a_n$ and $\sigma(\mathfrak{P})=\mathfrak{P}$, we deduce that $\sigma(x)-x\in(\p^G)^nB_{\mathfrak{P}}$. Since this relation holds for all $n$, we conclude from the hypothesis that $\sigma(x)=x$.
\end{proof}
\begin{corollary}
Under the hypotheses of \cref{algebra finite subgroup fixed ring prop} the rings $(A^G)/(\p^G)$ and $A^Z/(\mathfrak{P}^Z)$ have the same field of fractions and the maximal ideal of the local ring $(A^Z)_{\mathfrak{P}^Z}$ is generated by $\p^G$.
\end{corollary}
\begin{proof}
This follows from \cref{algebra finite subgroup fixed ring prop} by taking $H=G^Z$.
\end{proof}
\begin{corollary}\label{algebra finite group decomposition field char}
Let $B$ be an integral domain, $G$ a finite group acting on $B$, $A$ the ring of invariants of $G$ and $\mathfrak{P}$ a prime ideal of $B$. Let $K$, $K^Z$ and $L$ be the fields of fractions of $A$, $A^Z$ and $L$ respectively. Then $B$ is a Galois extension of $K$ with Galois group $G$ and the subfields $E$ of $L$ containing $K$ and such that $\mathfrak{P}$ is the only prime ideal of $B$ lying over the ideal $\mathfrak{P}\cap E$ of $B\cap E$ are just those which contain $K^Z$.
\end{corollary}
\begin{proof}
By \cref{algebra action and localization} applied to $S=A-\{0\}$, we see $K$ is the field of invariants of $G$ in $L$ and similarly $K^Z$ is the field of invariants of $G^Z$. Therefore, $L$ is a Galois extension of $K$ with Galois group $G$. By Galois correspondence, we see the condition "$E$ contains $K^Z$" is equivalent to "$H$ is contained in $G^Z$", where $H$ is the fixed group of $E$. As $E$ is then the field of invariants of $H$ in $L$, $B\cap E$ is the ring of invariants of $H$ in $B$. The second assertion then follows from \cref{algebra finite subgroup fixed ring prop}.
\end{proof}
\begin{proposition}\label{algebra finite group inertial field char}
Let $B$ be an integral domain, $G$ a finite group acting on $B$, $A$ the ring of invariants of $G$, $\p$ a prime ideal of $A$ and $\mathfrak{P}$ a prime of $B$ lying over $A$. Then
\begin{itemize}
\item[(a)] The residue field of $\mathfrak{P}^T$ is the separable closure of $\kappa(\p)$ in $\kappa(\mathfrak{P})$.
\item[(b)] The extension $K^T/K$ is Galois and $\Gal(K^T/K)$ is isomorphic to $\Gal(\kappa(\mathfrak{P}^T)/\kappa(\p))$.
\item[(c)] There exists a correspondence between the set of all intermediate extensions of $K^T/E/K$ and the set of all separable extensions $\kappa(\mathfrak{P}^T)/\kappa/\kappa(\p)$. The extension $\kappa/\kappa(\p)$ is a Galois extension if and only if $E/K$ is Galois. Moreover, $\kappa$ is the residue field of $\mathfrak{P}\cap E$.
\end{itemize}
\end{proposition}
\begin{proof}
By localization this may be reduced to the case where $\p$ is a maximal ideal of $A$, which implies that $\mathfrak{P}$, $\mathfrak{P}^Z$ and $\mathfrak{P}^T$ are maximal in $B$, $A^Z$ and $A^T$ respectively. Since $G^T$ is a normal subgroup of $G$, we know $K^T/K$ is Galois with Galois group $G/G^T$. Since $K^T/K$ is normal, by \cref{algebra finite group action prop} we see $\kappa(\mathfrak{P}^T)$ is a normal extension of $\kappa(\p)$, and \cref{algebra finite group action prop} shows that $\Gal(K^T/K)$ is isomorphic to $\Gal(\kappa(\mathfrak{P}^T)/\kappa(\p))$, hence the correspondence between the sets of intermediate fields $K\sub E\sub K^T$ and $\kappa(\p)\sub \kappa\sub\kappa(\mathfrak{P}^T)$ follows from Galois correspondence as well as the statement concerning normality. To see the last statement of (c), observe that $K^T$ is also the inertial field of $\mathfrak{P}\cap E$ for any intermediate extension $K\sub E\sub K^T$, therefore if $\kappa(E)$ is the residue field of $\mathfrak{P}\cap E$, then $\Gal(K^T/E)\cong\Gal(\kappa(\mathfrak{P}^T)/\kappa(E))$, by what we have just seen.\par
For all $x\in B$, the polynomial $P(X)=\prod_{\sigma\in G^T}(X-\sigma(x))$ has its coefficients in the inertia ring $A^T$ and, by definition of $G^T$, all its roots in $B$ are congruent mod $\mathfrak{P}$. Let $\pi:B\to B/\mathfrak{P}$ be the canonical homomorphism; the polynomial $\pi(P)(X)$ over $A^T/\mathfrak{P}^T$ therefore has all its roots in $B/\mathfrak{P}$ equal to $\pi(x)$, which shows that $\kappa(\mathfrak{P})$ is purely inseparable over $\kappa(\mathfrak{P}^T)$, so $\kappa(\p)^{\mathrm{sep}}\sub\kappa(\mathfrak{P}^T)$.\par
We know that $\kappa(\p)^{\mathrm{sep}}/\kappa(\p)$ is a Galois extension and it follows from \cref{algebra finite group action prop} that its Galois group is isomorphic to $G'=G^Z/G^T$. As $\kappa(\mathfrak{P}^T)$ is a purely inseparable extension of $\kappa(\p)^{\mathrm{sep}}$, $\kappa(\mathfrak{P}^T)$ is a normal extension of $\kappa(\p)$ and the separable degree of $\kappa(\mathfrak{P}^T)$ over $\kappa(\p)$ is 
\[[\kappa(\mathfrak{P}^T):\kappa(\p)]_s=[\kappa(\p)^{\mathrm{sep}}:\kappa(\p)]=[G^Z:G^T]=:q.\]
It remains to see that $\kappa(\mathfrak{P}^T)$ is a separable extension of $\kappa(\p)$. We have seen above that $G'$ is identified with an automorphism group of $A^T$ and that $A^Z$ is the ring of invariants of $G'$. If $x\in A^T$, the polynomial $Q(X)=\prod_{\sigma'\in G'}(X-\sigma'(x))$ therefore has its coefficients in $A^Z$. The polynomial $\pi(Q)$ over $A^Z/\p^Z$ whose coefficients are the images of those of $Q$ under the homomorphism $\pi$ is of degree $q$ and has a root $\pi(x)\in A^T/\p^T$. As $A^Z/\p^Z=\kappa(\p)$ by \cref{algebra finite subgroup fixed ring prop}(b), we see that every element of $\kappa(\mathfrak{P}^T)$ is of degree $\leq q$ over $\kappa(\p)$.\par
This being so, let $\kappa$ be the field of invariants of the Galois group $\Gal(\kappa(\mathfrak{P}^T)/\kappa(\p))$; then $[\kappa(\mathfrak{P}^T):\kappa]=q$ by \cref{field ext normal inseparable closure is fixed field}. Let $\xi$ be a primitive element of $\kappa(\mathfrak{P}^T)$ over $\kappa$; as it is of degree $q$ over $\kappa$ and of degree $\leq q$ over $\kappa(\p)$, it is of degree $q$ over $\kappa(\p)$ and its minimal polynomial over $\kappa$ has coefficients in $\kappa(\p)$; this shows that $\xi$ is separable over $\kappa(\p)$. On the other hand, for all $\zeta\in\kappa$, there exists a power $p^f$ of the characteristic $p$ such that $\zeta^{p^f}\in\kappa(\p)$. We conclude that $\kappa(\p)(\xi-\zeta)$, which contains
\[(\xi-\zeta)^{p^f}=\xi^{p^f}-\zeta^{p^f}\]
contains $\xi^{p^f}$ and consequently $\kappa(\p)(\xi^{p^f})$. But as $\xi$ is separable over $\kappa(\p)$, $\kappa(\p)(\xi)=\kappa(\p)(\xi^{p^f})$ by \cref{field ext simple separable iff}, whence $\kappa(\p)(\xi)\sub\kappa(\p)(\xi-\zeta)$. As $\xi$ is of degree $q$ over $\kappa(\p)$ and $\xi-\zeta$ of degree $\leq q$, it follows that $\kappa(\p)(\xi)=\kappa(\p)(\xi-\zeta)$, whence $\zeta\in\kappa(\p)(\xi)$. This shows that $\zeta$ is separable over $\kappa(\p)$, hence $\kappa=\kappa(\p)$ and $\kappa(\mathfrak{P}^T)$ is separable over $\kappa(\p)$. 
\end{proof}
\begin{corollary}\label{algebra finite group residue field Galois if order of inertial}
If the order of the inertia group $G^T(\mathfrak{P})$ is relatively prime to the characteristic $p$ of $\kappa(\p)$, then $\kappa(\mathfrak{P})/\kappa(\p)$ is a Galois extension.
\end{corollary}
\begin{proof}
With the notation of the proof of \cref{algebra finite group inertial field char}, the polynomial $\pi(P)$ has coefficients in $\kappa(\mathfrak{P}^T)=\kappa(\p)^{\mathrm{sep}}$; and all its roots equal to the image $\bar{x}$ of $x$ in $B/\mathfrak{P}$. We immediately deduce that $\pi(P)$ is a power of a minimal polynomial of $\bar{x}$ over $\kappa(\p)^{\mathrm{sep}}$. But the latter has degree equal to a power of $p$ and hence, as the degree of $\pi(P)$ is equal to the order of $G^T$, the hypothesis implies that $\pi(P)$ has degree zero, in other words $\bar{x}\in\kappa(\p)^{\mathrm{sep}}$. Thus $\kappa(\mathfrak{P}^T)=\kappa(\p)^{\mathrm{sep}}$ and $\kappa(\mathfrak{P})/\kappa(\p)$ is a Galois extension.
\end{proof}
\subsection{Applications for integrally closed domains}
\begin{lemma}\label{integrally closed domain lift of prime to inseparable extension}
Let $A$ be an integrally closed domain, $K$ its field of fractions, $p$ the characteristic of $K$, $L$ a radicial extension of $K$ and $B$ a subring of $L$ containing $A$ and integral over $A$. For every prime ideal $\p$ of $A$, there exists a unique prime ideal $\mathfrak{P}$ of $B$ lying over $\p$ and $\mathfrak{P}$ is the set of $x\in B$ such that there exists an integer $n>0$ for which $x^{p^n}\in\p$.
\end{lemma}
\begin{proof}
The existence of $\mathfrak{P}$ follows from \cref{integral ring lying over prime exist}. If $x\in\mathfrak{P}$, there exists $n>0$ such that $x^{p^n}\in K$, whence $x^{p^n}\in A$ since $A$ is integrally closed, hence $x^{p^n}\in A\cap\mathfrak{P}=\p$. Conversely, if $x\in B$ is such that $x^{p^n}\in\p\sub\mathfrak{P}$ then $x\in\p$ since $\mathfrak{P}$ is prime.
\end{proof}
\begin{proposition}\label{integrally closed normal extension transitive action}
Let $A$ be an integrally closed domain, $K$ its field of fractions, $L$ a normal extension of $K$ and $B$ the integral closure of $A$ in $L$. Then
\begin{itemize}
\item[(a)] For every prime ideal $\p$ of $A$, the group $G$ of $K$-automorphisms of $L$ acts transitively on the set of prime ideals of $B$ lying over $\p$.
\item[(b)] For every prime ideal $\mathfrak{P}$ of $B$ and $\p=\mathfrak{P}\cap A$, the residue field $\kappa(\mathfrak{P})$ is a nromal extension of $\kappa(\p)$ and the canonical homomorphism $\sigma\mapsto\widebar{\sigma}$ defines by taking the quotient an isomorphism of $G^Z(\mathfrak{P})/G^T(\mathfrak{P})$ onto $\Gal(\kappa(\mathfrak{P})/\kappa(\p))$.
\end{itemize}
\end{proposition}
\begin{proof}
Suppose first that $L$ is a finite Galois extension of $K$. Then $A=B\cap K$ since $A$ is integrally closed and $A$ is therefore the ring of invariants of $G$ in $B$. As $G$ is finite, the proposition follows in this case from \cref{algebra finite group action prop}.\par
Suppose secondly that $L$ is any Galois extension of $K$. Then $L$ is the union of a right directed family $(K_\alpha)_{\alpha\in I}$ of finite Galois extensions of $K$. To show (a), consider two prime ideals $\mathfrak{P}$, $\mathfrak{Q}$ of $B$ lying over $\p$. For all $\alpha\in I$, $\mathfrak{P}\cap K_\alpha$ and $\mathfrak{Q}\cap K_\alpha$ are two prime ideals of $B\cap K_\alpha$ lying over $\p$. Since $B\cap K_\alpha$ is the integral closure of $A$ in $K_\alpha$ and the restrictions to $K_\alpha$ of the elements of $G$ form the group of $K$-automorphisms of $K_\alpha$, it follows from the finite case that there exists $\sigma\in G$ such that $\sigma(\mathfrak{P}\cap K_\alpha)=\mathfrak{Q}\cap K_\alpha$. Let $\mathcal{E}_\alpha$ be the set of $\sigma\in G$ which have the above property. Let $\sigma\in G-\mathcal{E}_\alpha$, then for all $\tau\in G$ leaving invariant the elements of $K_\alpha$, we have
\[(\sigma\tau)(\mathfrak{P}\cap K_\alpha)=\sigma(\mathfrak{P}\cap K_\alpha)\neq\mathfrak{Q}\cap K_\alpha\]
and hence $\sigma\tau\in G-\mathcal{E}_\alpha$. It follows that $\mathcal{E}_\alpha$ is closed in the topological Galois group $G$ and clearly the family $\mathcal{E}_\alpha$ is directed. As $G$ is compact and the $\mathcal{E}_\alpha$ are non-empty, the intersection $\mathcal{E}$ of the family $(\mathcal{E}_\alpha)$ is non-empty and $\sigma(\mathfrak{P})=\mathfrak{Q}$ for all $\sigma\in\mathcal{E}$, whence (a).\par
To show (b), note that $\kappa(\mathfrak{P})$ is the union of the right directed family $(\kappa_\alpha)_{\alpha\in I}$ where $\kappa_\alpha$ is the residue field of $\mathfrak{P}\cap K_\alpha$. As each $\kappa_\alpha$ is a normal extension of $\kappa(\p)$ by \cref{algebra finite group action prop}, so is $\kappa(\mathfrak{P})$. On the other hand, let $\theta$ be a $\kappa(\p)$-automorphism of $\kappa(\mathfrak{P})$. By virtue of \cref{algebra finite group action prop} applied to $B\cap K_\alpha$, there exists for all $\alpha$ a non-empty set $\mathcal{F}_\alpha$ of elements $\sigma\in G$ such that $\sigma(\mathfrak{P}\cap K_\alpha)=\mathfrak{P}\cap K_\alpha$ and $\theta(\bar{x})=\widebar{\sigma(x)}$ for all $x\in B\cap K_\alpha$. As above it is seen that $\mathcal{F}_\alpha$ is closed in $G$ and, as $(\mathcal{F}_\alpha)$ is a left directed set, its intersection $\mathcal{F}$ is non-empty. Clearly for $\sigma\in G$ we have $\sigma\in G^Z$ and $\bar{\sigma}=\theta$, which completes the proof of (b) in this case.\par
Finally we deal with the general case. The field of invariants $K_1$ of $G$ is a purely inseparable extension of $K$. There therefore exists a single prime ideal $\p_1$ of $A_1=B\cap K_1$ lying over $\p$ (\cref{integrally closed domain lift of prime to inseparable extension}). If $\mathfrak{P}$ and $\mathfrak{Q}$ are two prime ideals of $B$ lying over $\p$, then they lie over $\p_1$. As $L$ is a Galois extension of $K_1$ and $B\cap K_1$ is integrally closed by \cref{integrally closed intersection is integrally closed}, it follows from the Galois case that there exists a $\sigma\in G$ such that $\sigma(\mathfrak{P})=\mathfrak{Q}$, whence (a). On the other hand, clearly the residue field of $\p_1$ is a purely inseparable extension of $\kappa(\p)$. As $\kappa(\mathfrak{P})$ is a normal extension of $\kappa_1$ by the Galois case, we then see $\kappa(\mathfrak{P})$ is a normal extension of $\kappa(\p)$ (since every $\kappa(\p)$-isomorphism of $\kappa(\mathfrak{P})$ is a $\kappa_1$-isomorphism). This last remark shows also, taking account of the Galois case, that every $\kappa(\p)$-automorphism of $\kappa(\mathfrak{P})$ is of the form $\bar{\sigma}$ where $\sigma\in G^Z(\mathfrak{P})$, which completes the proof of (b).
\end{proof}
\begin{corollary}\label{integrally closed normal extension homomorphism same restriction}
Let $A$ be an integrally closed domain, $K$ its field of fractions, $L$ a normal extension of $K$ and $B$ the integral closure of $A$ in $L$. Let $\theta_1$ and $\theta_2$ be two homomorphisms of $B$ to a field $\Omega$ with the same restriction on $A$. Then there exists a $K$-automorphism $\sigma$ of $L$ such that $\theta_1=\theta_2\circ\sigma$.
\end{corollary}
\begin{proof}
This can be proved as \cref{algebra finite group action homomorphism same restriction}.
\end{proof}
\begin{proposition}\label{integrally closed finite algebra extension finite lying over}
Let $A$ be an integrally closed domain, $K$ its field of fractions, $L$ a finite algebraic extension of $K$ and $B$ a subring of $L$ containing $A$ and integral over $A$.
\begin{itemize}
\item[(a)] For every prime ideal $\p$ of $A$ the set of prime ideals of $B$ lying over $\p$ is at most $[L:K]_s$.
\item[(b)] If $\mathfrak{P}$ is a prime ideal of $B$ lying over $\p$, every element of $\kappa(\mathfrak{P})$ is of degree smaller than $[L:K]$ over $\kappa(\p)$.
\end{itemize}
\end{proposition}
\begin{proof}
We may first restrict our attention to the case where $L$ is a separable extension of $K$, for in general $L$ is a purely inseparable extension of the separable closure $K_0$ of $K$ contained in $L$, and $[L:K]_s=[K_0:K]$ by definition and, if $A_0=B\cap K_0$, the prime ideals of $A_0$ and $B$ are in one-to-one correspondence (\cref{integrally closed domain lift of prime to inseparable extension}).\par
Suppose therefore that $L$ is separable over $K$ and let $N$ be the Galois extension of $K$ generated by $L$ in an algebraic closure of $K$, $G$ its Galois group, $\widebar{A}$ the integral closure of $A$ in $N$ and $\mathcal{P}$ a prime ideal of $\widebar{A}$ lying over $\p$. Let $H$ be the Galois group of $N$ over $L$ and $G^Z$ the decomposition group of $\mathcal{P}$. The prime ideals of $\widebar{A}$ lying over $\p$ are the $\sigma(\mathcal{P})$ where $\sigma\in G$ (\cref{algebra finite group action prop}) and the relation $\sigma(\mathcal{P})=\sigma'(\mathcal{P})$ means that $\sigma'=\sigma\tau$ where $\tau\in G^Z$. On the other hand, in order that $\sigma(\mathcal{P})\cap L=\sigma'(\mathcal{P})\cap L$, it is necessary and sufficient that $\sigma(\mathcal{P})=\theta\sigma(\mathcal{P})$, where $\theta\in H$ (\cref{algebra finite group action prop}), whence finally $\sigma'=\theta\sigma\tau$ where $\theta\in H$ and $\tau\in G^Z$. The number of prime ideals of $B$ lying over $\p$ is therefore equal to the cardinality of the double coset $H\backslash G/G^Z$, and this number is smaller than $[G:H]=[L:K]$ be Galois theory.\par
The coefficients of the minimal polynomial (over $K$) of any element $x\in B$ belong to $A$ (\cref{integral element iff minimal polynomial coefficient}). Applying the canonical homomorphism $B\to B/\mathfrak{P}$ to the coefficients of this polynomial, an equation of integral dependence with coefficients in $A/\p$ and of degree $\leq[L:K]$ is obtained for the class mod $\mathfrak{P}$ of $x$, whence the conclusion.
\end{proof}
\begin{proposition}\label{integrally closed Galois extension decomposition}
Let $A$ be an integrally closed domain, $K$ its field of fractions, $L$ a Galois extension of $K$, $B$ the integral closure of $A$ in $L$, $\mathfrak{P}$ a prime ideal of $B$, $\p=A\cap\mathfrak{P}$. For a subfield $E$ of $L$ containing $K$, set $A(E)=B\cap E$ and $\p(E)=\mathfrak{P}\cap E$.
\begin{itemize}
\item[(a)] If $E$ is contained in $K^Z$, then $\p$ and $\p(E)$ have the same residue field and the maximal ideal of $(A(E))_{\p(E)}$ is generated by $\p$.
\item[(b)] Conversely, if these two conditions in (a) hold and $\bigcap_{n=0}^{\infty}\p^nB_{\mathfrak{P}}=0$, then $E$ is contained in $K^Z$.
\end{itemize}
\end{proposition}
\begin{proof}
Let $G$ be the Galois group of $L$ over $K$ and $H$ the Galois group of $L$ over $E$. Then to say that $E\sub K^Z$ means that $G^Z\sub H$ and the assertions are therefore special cases of \cref{algebra finite subgroup fixed ring prop} when $[L:K]$ is finite. In the general case the argument is as in the proof of \cref{integrally closed normal extension transitive action}.
\end{proof}
\begin{proposition}\label{integrally closed Galois extension inertial}
Let $A$ be an integrally closed domain, $K$ its field of fractions, $L$ a Galois extension of $K$, $B$ the integral closure of $A$ in $L$, $\mathfrak{P}$ a prime ideal of $B$, $\p=A\cap\mathfrak{P}$.
\begin{itemize}
\item[(a)] The residue field $\kappa(\mathfrak{P}^T)$ is the separable closure of $\kappa(\p)$ in $\kappa(\mathfrak{P})$.
\item[(b)] The extension $K^T/K$ is Galois and $\Gal(K^T/K)$ is isomorphic to $\Gal(\kappa(\mathfrak{P}^T)/\kappa(\p))$.
\item[(c)] There exists an inclusion-preserving bijective correspondence between the set of all intermediate extensions $K\sub E\sub K^T$ and the set of all separable extensions $\kappa(\p)\sub\kappa\sub\kappa(\mathfrak{P}^T)$. The extension $\kappa/\kappa(\p)$ is a Galois extension if and only if $E/K$ is Galois. Moreover, in this case $\kappa$ is the residue field of $\mathfrak{P}\cap E$.
\end{itemize}
\end{proposition}
\begin{proof}
The ring $A$ is the ring of invariants in $B$ of the Galois group of $L$ over $K$. If $L$ is of finite degree over $K$, the corollary then follows from \cref{algebra finite subgroup fixed ring prop} and \cref{algebra finite group inertial field char}. Consider now the general case, $L$ therefore being the union of a right directed set $(K_\alpha)$ of finite Galois extensions of $K$.\par
Suppose now that $x\in A^T$. There exists $\alpha$ such that $x\in A^T(\mathfrak{P}\cap K_\alpha)$ and \cref{algebra finite group inertial field char} shows that the class $\bar{x}$ of $x$ mod $\mathfrak{P}^T\cap K_\alpha$ is algebraic and separable over $\kappa(\p)$. A fortiori the class mod $\mathfrak{P}^T$ of $x$ is separable over $\kappa(\p)$. To complete the proof of the corollary, it is sufficient to show that $\kappa(\mathfrak{P})$ is a purely inseparable extension of $\kappa(\mathfrak{P}^T)$. Now, $\kappa(\mathfrak{P})$ is the union of the right directed family of residue fields $\kappa_\alpha$ of $\mathfrak{P}\cap K_\alpha$. It follows therefore from \cref{algebra finite group inertial field char} that, if an element of $\kappa(\mathfrak{P})$ belongs to $\kappa_\alpha$, it is purely inseparable over the residue field of $\mathfrak{P}^T\cap K_\alpha$, and a fortiori over $\kappa(\mathfrak{P}^T)$.
\end{proof}
Let $A$ be an integrally closed domain, $K$ its field of fractions, $L$ a normal extension of $K$ and $B$ the integral closure of $A$ in $L$. The field of invariants $K^Z(\mathfrak{P})$ (resp. $K^T(\mathfrak{P})$) of the group $G^Z(\mathfrak{P})$ (resp. $G^T(\mathfrak{P})$) in the field $L$ is called the \textbf{decomposition field} (resp. inertia field) of $\mathfrak{P}$ with respect to $K$. We also write $K^z$ (resp. $K^T$) in place of $K^Z(\mathfrak{P})$ (resp. $K^T(\mathfrak{P})$). It follows from \cref{algebra action and localization} that $K^Z$ (resp. $K^T$) is the field of fractions of the ring $A^Z$ (resp. $A^T$), and by \cref{integrally closed intersection is integrally closed} $A^Z$ (resp. $A^T$) is the integral closure of $A$ in $K^Z$ (resp. $K^T$).
\begin{proposition}
Let $A$ be an integrally closed domain, $K$ its field of fractions, $L$ a Galois extension of $K$, $B$ the integral closure of $A$ in $L$, $\mathfrak{P}$ a prime ideal of $B$. For a subfield $E$ of $L$ containing $K$, set $A(E)=B\cap E$ and $\p(E)=\mathfrak{P}\cap E$. Then the decomposition field (resp. inertia field) of $\mathfrak{P}$ with respect to $E$ is $EK^Z$ (resp. $EK^T$). If further $E$ is a Galois extension of $K$, the decomposition field of $\p(E)$ with respect to $K$ is $E\cap K^Z$.
\end{proposition}
\begin{proof}
If $H$ is the Galois group of $L$ over $E$, clearly the decomposition group (resp. inertia group) of $\mathfrak{P}$ with respect to $E$ is $H\cap G^Z$ (resp. $H\cap G^T$) and the first assertion follows from Galois theory if $L$ is a finite Galois extension of $K$. In the general case it follows from the fact that $A^Z$ (resp. $A^T$) is the union of the $A^Z(\mathfrak{P}\cap K_\alpha)$ (resp. $A^T(\mathfrak{P}\cap K_\alpha)$): every element $x\in L$ belongs to some $K_\alpha$ and if it is invariant under $G^Z(\mathfrak{P})\cap H$ (resp. $G^T(\mathfrak{P})\cap H$) it is also invariant under $G^Z(\mathfrak{P}\cap K_\beta)\cap H$ (resp. $G^T(\mathfrak{P}\cap K_\beta)\cap H$) for some suitable $\beta$; hence it belongs by the beginning of the argument to $EK^Z(\mathfrak{P}\cap K_\alpha)\sub EK^Z$ (resp. $EK^T(\mathfrak{P}\cap K_\alpha)\sub EK^T$).\par
Suppose now that $E$ is a Galois extension of $K$. The restriction to $E$ of every $\sigma\in G^Z$ then preserves $\p(E)$ and hence belongs to the decomposition group of $\p(E)$ with respect to $K$. Conversely, let $\tau$ be an automorphism of $E$ belonging to this group and let $\sigma$ be an extension of $\tau$ to a $K$-automorphism of $L$. We write $\mathfrak{Q}=\sigma(\mathfrak{P})$. As $\mathfrak{P}$ and $\mathfrak{Q}$ both lie over $\p(E)$, there exists an automorphism $\rho\in H$ such that $\mathfrak{Q}=\rho(\mathfrak{P})$, whence $\rho^{-1}\sigma\in G^Z$ and $\tau$ is the restriction of $\rho^{-1}\sigma$ to $E$. In other words, the decomposition group of $\p(E)$ with respect to $K$ is identical with the group of restrictions to $E$ of the automorphisms $\sigma\in G^Z$ which proves the second assertion.
\end{proof}
\section{Finite generated algebras over a field}
In this part, $k$ will be a field. Recall that, if $A$ is a ring, by a finite generated $A$-algebra we mean an $A$-algebra $B$ that is isomorphic to a quotient of some polynomial algebra $A[X_1,\dots,X_n]$.
\begin{theorem}[\textbf{Normalization Lemma}]\label{algebra finite over field normalization}
Let $A$ be a finitely generated $k$-algebra and let $\a_1\sub\a_2\subset\cdots\subset\a_p$ be an increasing finite sequence of peroper ideals of $A$. Then there exists a sequence $(x_i)_{1\leq i\leq n}$ of elements of $A$ which are algebraically independent over $k$ and such that:
\begin{itemize}
\item[(a)] $A$ is integral over the ring $B=k[x_1,\dots,x_n]$.
\item[(b)] There exists an increasing sequence $(h_j)_{1\leq j\leq n}$ of integers such that for all $j$ the ideal $\a_j\cap B$ of $B$ is generated $x_1,\dots,x_{h_j}$.
\end{itemize}
\end{theorem}
\begin{proof}
It is sufficient to prove the theorem when $A$ is a polynomial algebra $k[Y_1,\dots,Y_m]$. In fact, assume the theorem for this case. In the general case, $A$ is isomorphic to a quotient of such an algebra $\tilde{A}$ by an ideal $\tilde{\a}_0$. Let $\tilde{\a}_j$ denote the inverse image of $\a_j$ in $\tilde{A}$ and let $(\tilde{x}_i)_{1\leq i\leq r}$ be elements of $\tilde{A}$ satisfying the conditions of the statement for the ring $\tilde{A}$ and the increasing sequence of ideals of $\tilde{\a}_0\subset\cdots\subset\tilde{\a}_p$. Then the images $x_i$ of the $\tilde{x}_i$ in $A$ for $i>h_0$ satisfy the desired conditions. In fact, condition (b) is obvious and condition (a) follows from \cref{integral element under homomorphism}. Finally, if the $x_i$ for $i>h_0$ were not algebraically independent over $k$, there would be a non-zero polynomial $Q\in k[X_{h_0+1},\dots,X_r]$ such that $Q(x_{h_0+1},\dots,x_r)\in\tilde{\a}_0\cap\tilde{B}$, where $\tilde{B}=k[\tilde{x}_1,\dots,\tilde{x}_r]$. But by hypothesis, the ideal $\a_0\cap\tilde{B}$ is generated by $x_1,\dots,x_{h_0}$, a contradiction which proves our assertion.\par
We shall therefore suppose in the rest of the proof that $A=k[Y_1,\dots,Y_m]$ and we shall argue by induction on $p$. First assume $p=1$, and we may further assume that $\a_i$ is a principal ideal generated by an element $x_1\notin k$. Then $x_1=P(Y_1,\dots,Y_m)$, where $P$ is a non-constant polynomial. We shall see that, for a suitable choice of integers $r_i>0$, the ring $A$ is integral over $B=k[x_1,x_2,\dots,x_m]$, where
\begin{equation}\label{algebra finite over field normalization-1}
x_i=Y_i-Y_1^{r_1}\for 2\leq i\leq m.
\end{equation}
For this it is sufficient to choose the $r_i$ such that $Y_1$ is integral over $B$ (\cref{integral element A[x] is integral}). Now, there is the relation
\begin{align}\label{algebra finite over field normalization-2}
P(Y_1,x_2+Y_1^{r_2},\dots,x_m+Y_1^{r_m})-x_1=0
\end{align}
Write $P=\sum_\alpha a_\alpha Y^\alpha$, where $\alpha\in\N^m$ and $a_\alpha\in k$ are nonzero elements of $k$. Then (\ref{algebra finite over field normalization-2}) becomes
\begin{align}\label{algebra finite over field normalization-3}
\sum_\alpha a_\alpha Y_1^{\alpha_1}(x_2+Y_1^{r_2})^{\alpha_2}\cdots(x_m+Y_1^{r_m})^{\alpha_m}-x_1=0.
\end{align}
Let us write $f(\alpha)=\alpha_1+r_2\alpha_2+\cdots+r_m\alpha_m$ and suppose that the $r_i$ are chosen so that the $f(\alpha)$ are distinct (it suffices for example to take $r_i=h^i$, where $h$ is an integer strictly greater than all the $\alpha_i$. Then there will be a unique system $\alpha=(\alpha_1,\dots,\alpha_m)$ such that $f(\alpha)$ is maximal and relation (\ref{algebra finite over field normalization-3}) may be written
\begin{align}\label{algebra finite over field normalization-4}
a_\alpha Y_1^{f(\alpha)}+\sum_{j<f(\alpha)}Q_j(x_1,\dots,x_m)Y_1^j=0.
\end{align}
where the $Q_i$ are polynomials in $k[Y_1,\dots,Y_m]$. As $a_\alpha\neq 0$ is invertible in $k$, (\ref{algebra finite over field normalization-4}) is certainly an equation of integral dependence with coefficients in $B$, whence our assertion.\par
The field of fractions $k(Y_1,\dots,Y_m)$ of $A$ is therefore algebraic over the field of fractions $k(x_1,\dots,x_m)$ of $B$, which proves that the $x_i$ are algebraically independent. Moreover, every element $z\in\a_1\cap B$ may be written $z=x_1\tilde{z}$, where $\tilde{z}\in A\cap k(x_1,\dots,x_m)$. But $A\cap k(x_1,\dots,x_m)=k[x_1,\dots,x_m]=B$ since $B$ is integrally closed by \cref{integral closure of a polynomial ring}, whence $\a_1\cap B=Bx_1$.\par
We now do not assume that $\a_1$ is principal and prove the theorem for $p=1$ by induction on $m$. We may obviously suppose that $\a_1\neq 0$. Let $x_1$ be a non-zero element of $\a_1$. By the argument above there exist $t_2,\dots,t_m$ such that $x_1,t_2\dots,t_m$ are algebraically independent over $k$, $A$ is integral over $C=k[x_1,t_2,\dots,t_m]$ and $x_1A\cap C=x_1C$. By the induction hypothesis there exist algebraically independent elements $x_2,\dots,x_m$ of $k[t_2,\dots,t_m]$ and an integer $h$ such that $k[t_2,\dots,t_m]$ is integral over $\tilde{B}=k[x_2,\dots,x_m]$, and the ideal $\a_1\cap \tilde{B}$ is generated by $x_2,\dots,x_h$. Then $C$ is integral over $B=k[x_1,x_2,\dots,x_m]$ and hence so is $A$, therefore $x_1,\dots,x_m$ are algebraically independent over $k$. Finally, as $x_1\in\a_1$ and $B=\tilde{B}[x_1]$, $\a_1\cap B=Bx_1+(\a_1\cap \tilde{B})$ and, as $\a_1\cap \tilde{B}$ is generated (in $\tilde{B}$) by $x_2,\dots,x_h$, we see $\a_1\cap B$ is generated (in $B$) by $x_1,x_2,\dots,x_h$.\par
Now assume the we have proved the theorem for $p-1$ idaels and let $t_1,\dots,t_m$ be elements of $A$ satisfying the conditions of the theorem for the increasing sequence of ideals $\a_1\subset\cdots\subset\a_{p-1}$ and let us write $r=h_{p-1}$. Apply the preceding proof for $\a_p$, there exist algebraically independent elements $x_{r+1},\dots,x_m$ of $C=k[t_{r+1},\dots,t_m]$ and an integer $s$ such that $C$ is integral over $\tilde{B}=k[x_{r+1},\dots,x_m]$ and the ideal $\a_p\cap \tilde{B}$ is generated by $x_{r+1},\dots,x_s$. Now write $x_i=t_i$ for $1\leq i\leq r$ and $h_p=s$, then $A$ is integral over $C[t_1,\dots,t_r]$ and hence also over $B=k[x_1,\dots,x_m]=\tilde{B}[t_1,\dots,t_r]$ since $C$ is integral over $\tilde{B}$ (\cref{integral ring generating ring is integral}). In particular, we see the $x_i$ are algebraically independent over $k$. On the other hand, for $j\leq p-1$, the ideal
\[\a_j\cap k[x_1,\dots,x_r,t_{r+1},\dots,t_m]\]
is by hypothesis the set of polynomials in $x_1,\dots,x_r,t_{r+1},\dots,t_m$ all of whose monomials contain one of the elements $x_1,\dots,x_{h_j}$. As $x_{r+1},\dots,x_m$ are polynomials in $t_{r+1},\dots,t_m$ with coeflficients in $k$, it is seen immediately that a polynomial in $x_1,\dots,x_m$ (with coefficients in $k$) can belong to $\a_j$ only if all its monomials contain one of the elements $x_1,\dots,x_{h_j}$. Finally, as $x_1,\dots,x_r$ belong to $\a_{p-1}$ and hence also to $\a_p$, $\a_p\cap \tilde{B}[x_1,\dots,x_r]$ consists of the polynomials in $x_1,\dots,x_r$ with coefficients in $\tilde{B}$ whose constant term belongs to $\a_p\cap \tilde{B}$. This ideal is therefore generated by $x_1,\dots,x_r,x_{r+1},\dots,x_s$. Therefore we see $(x_i)_{1\leq i\leq m}$ satisfies the required conditions, which proves the theorem.
\end{proof}
\begin{corollary}\label{integral domain finite algebra normalization}
Let $A$ be an integral domain and $B$ a finitely generated $A$-algebra containing $A$ as a subring. Then there exist a nonzero element $s$ of $A$ and a subalgebra $C$ of $B$ isomorphic to a polynomial algebra $A[Y_1,\dots,Y_n]$ such that $B_s$ is integral over $C_s$.
\end{corollary}
\begin{proof}
Wewrite $S=A-\{0\}$ and let $k=S^{-1}A$ the field of fractions of $A$. Clearly $S^{-1}B$ is a finitely generated $k$-algebra and, as it contains $k$ by hypothesis, it is not reduced to $0$. By \cref{algebra finite over field normalization} applied to $\a_1=(0)$, there exists therefore a finite sequence $(x_i)_{1\leq i\leq n}$ of elements of $S^{-1}B$ which are algebraically independent over $k$ and such that $S^{-1}B$ is integral over $k[x_1,\dots,x_n]$. Let $(z_j)_{1\leq j\leq m}$ be a system of generators of the $A$-algebra $B$. Then in $S^{-1}B$ each of the $z_j/1$ satisfies an equation of integral dependence
\[(z_j/1)^{q_j}+\sum_{k<q_j}P_{kj}(x_1,\dots,x_n)(z_j/1)^k=0.\]
where the $P_{kj}$ are polynomials in the $x_i$ with coefficients in $k$. There exists an element $s\neq 0$ of $A$ such that we may write $x_i=y_i/s$ where $y_i\in B$ and all the coefficients of the $P_{kj}$ are of the form $c/s$ where $c\in A$. Finally, replacing if need be, $s$ by a product of elements of $S$, we may assume that in $B$,
\begin{equation}\label{integral domain finite algebra normalization-1}
sz_j^{q_j}+\sum_{k<q_j}Q_{kj}z_j^h=0.
\end{equation}
where the $Q_{kj}$ are polynomials in $y_1,\dots,y_n$ with coeflicients in $A$. If we write $\tilde{z}_j=sz_j$, it is seen from (\ref{integral domain finite algebra normalization-1}) by multiplying $s^{q_j-1}$ that $\tilde{z}_j$ is integral over $\tilde{B}=A[y_1,\dots,y_n]$. We show that the $y_i$ are algebraically independent over $A$. If there is a relation of the form $\sum_\alpha a_\alpha y_1^{\alpha_1}\cdots y_n^{\alpha_n}=0$ where $a_\alpha\in A$ for all $\alpha$, we deduce that $\sum_\alpha b_\alpha x_1^{\alpha_1}\cdots x_n^{\alpha_n}=0$ in $S^{-1}B$, where $b_\alpha=a_\alpha s^{|\alpha|}$. By hypothesis therefore $b_\alpha=0$ for all $\alpha$, whence $a_\alpha=0$ for all $\alpha$. Moreover, by \cref{integral closure and localization} in the ring $B_s$ each of the $\tilde{z}_j/1$ is integral over $\tilde{B}_s$ and, as $z_j/1=(\tilde{z}_j/1)(1/s)$ in $B_s$, it is seen that the $z_j/1$ are integral over $\tilde{B}_s$, which completes the proof.
\end{proof}
\begin{corollary}\label{integral domain subfield of fraction finite algebra}
Let $L$ be a field, $A$ a subring of $L$ and $K$ the field of fractions of $A$. If $L$ is a finitely generated $A$-algebra, then $[L:K]$ is finite and there exists a nonzero element $a$ in $A$ such that $K=A_a$.
\end{corollary}
\begin{proof}
It follows from \cref{integral domain finite algebra normalization} that there exist elements $x_1,\dots,x_n$ of $L$ and an element $a\neq 0$ of $A$ such that $x_1,\dots,x_n$ are algebraically independent over $A$ (and therefore over $K$) and that $L$ is integral over the subring $A[x_1,\dots,x_n]_a$. Then it follows from \cref{integral ring extension field iff} that $A[x_1,\dots,x_n]_a$ is a field. But the only invertible elements of a polynomial ring $C[Y_1,\dots,Y_n]$ over an integral domain $C$ are the invertible elements of $C$. Applying this remark to $C=A_a$, it is seen that necessarily $n=0$ and that $A_a$ is a field equal to $K$ by definition of the latter. As $L$ is integral over $K$ and is a finitely generated $K$-algebra, the degree $[L:K]$ is finite.
\end{proof}
\begin{corollary}\label{algebra finite homomorphism to algebraically closed prop}
Let $A$ be an integral domain, $B$ a finitely generated $A$-algebra and $b$ a nonzero element of $B$ such that $z^nb\neq 0$ for all $z\neq 0$ in $A$ and every integer $n>0$. Let $\rho:A\to B$ be the canonical homomorphism. Then there exists a nonzero $a$ in $A$ such that, for every homomorphism $\phi$ from $A$ to an algebraically closed field $\Omega$ such that $\phi(a)\neq 0$, there exists a homomorphism $\tilde{\phi}$ from $B$ to $\Omega$ such that $\tilde{\phi}(b)\neq 0$ and $\phi=\tilde{\phi}\circ f$.
\end{corollary}
\begin{proof}
The hypothesis on $b$ implies that, if $i$ is the canonical homomorphism $x\mapsto x/1$ of $B$ to $B_b$, the homomorphism $i\circ f$ of $A$ to $B_b$ is injective. By \cref{integral domain finite algebra normalization} there therefore exist an element $a\neq 0$ of $A$ and a subring $\tilde{B}$ of $B_b$ such that $(B_{b})_{a}$ is integral over $\tilde{B}_a$ and $\tilde{B}$ is isomorphic to a polynomial algebra $A[Y_1,\dots,Y_n]$. Let $\phi$ be a homomorphism from $A$ to an algebraically closed field $\Omega$ such that $\phi(a)\neq 0$. There exists a homomorphism from $A[Y_1,\dots,Y_n]$ to $\Omega$ extending $\phi$ and hence there exists a homomorphism from $\tilde{B}$ to $\Omega$ extending $\phi$, which we still denote by $\phi$. As $\phi(a)\neq 0$ in $\Omega$, there exists a homomorphism $\psi$ from $\tilde{B}_a$ to $\Omega$ such that
\[\psi(x/a^n)=\phi(x)(\phi(a))^{-n}\]
for all $x\in\tilde{B}$ and all $n>0$. Finally, as $(B_b)_a$ is integral over $\tilde{B}_a$, there exists a homomorphism $\tilde{\psi}$ from $(B_b)_a$ to $\Omega$ extending $\psi$ (\cref{integral extension homomorphism to algebraically closed}). If $j:x\mapsto x/1$ is the canonical homomorphism from $B$ to $(B_b)_a$, then $\tilde{\phi}=\tilde{\psi}\circ j$ solves the problem for $j(b)$ is invertible in $(B_b)_a$ and hence $\tilde{\psi}(j(b))\neq 0$ in $\Omega$.
\end{proof}
\begin{theorem}\label{algebra finite integral closure in finite algebraic extension is finite}
Let $A$ be a finitely generated integral $k$-algebra, $K$ its field of fractions, $L$ a finite algebraic extension of $K$, and $\widebar{A}$ the integral closure of $A$ in $L$. Then $\widebar{A}$ is a finitely generated $A$-module and a finitely generated $k$-algebra.
\end{theorem}
\begin{proof}
By \cref{algebra finite over field normalization} there exists a subalgebra $C$ of $A$ isomorphic to a polynomial algebra $k[X_1,\dots,X_n]$ and such that $A$ is integral over $C$. Then $\widebar{A}$ is obviously the integral closure of $C$ in $L$ (\cref{integral transitive}). We may therefore confine our attention to the case where $A=k[X_1,\dots,X_n]$.\par
Let $N$ be the normal extension of $L$ (in an algebraic closure of $L$) generated by $L$, which is a finite algebraic extension of $K$. It will suffice to prove that the integral closure of $A$ in $N$ is a finitely generated $A$-module, for $\widebar{A}$ is a sub-$A$-module of that ring and $A$ is a Noetherian ring. We may therefore confine our attention to the case where $L$ is a finite normal extension of $K$. Then we know that $L$ is a (finite) Galois extension of a (finite) purely inseparable extension $K'$ of $K$. If $A'$ is the integral closure of $A$ in $K'$, $\widebar{A}$ is the integral closure of $A'$ in $L$ and it will suffice to prove that $A'$ is a finitely generated $A$-module and $\widebar{A}$ is a finitely generated $A'$-module. Now, if it has been proved that $A'$ is a finitely generated $A$-module, it is a Noetherian domain, integrally closed by definition. The fact that $\widebar{A}$ is a finitely generated $A'$-module will follow from \cref{integral closure in finite separable finite if Noe}.\par
We see that we may confine our attention to the case where $A=k[X_1,\dots,X_n]$ and where $L$ is a finite purely inseparable extension of $K=k(X_1,\dots,X_n)$. Then $L$ is generated by a finite family of elements $(y_i)_{1\leq i\leq m}$ and there exists a power $q$ of the characteristic of $k$ such that $y_i^q\in k(X_1,\dots,X_n)$ for all $i$. Let $(c_i)_{1\leq i\leq r}$ be the coefficients of the numerators and denominators of the rational functions in $X_1,\dots,X_n$ that equal to $y_i^q$. Then $L$ is contained in the extension $L'=k'(X_1^{1/q},\dots,X_n^{1/q})$, where $k'=k(c_1^{1/q},\dots,c_r^{1/q})$ (we are in an algebraic closure of $L$) and $\widebar{A}$ is contained in the algebraic closure $A'$ of $A$ in $L'$. Now, $k'$ is algebraic over $k$ and hence $C'=k'[X_1,\dots,X_n]$ is integral over $A$. As $k'[X_1^{1/q},\dots,X_n^{1/q}]$ is integrally closed by \cref{integral closed polynomial ring iff coefficient}, it is seen that this ring is the integral closure of $C'$ in $L'$ and hence also that of $A$, in other words, $A'=k'[X_1^{1/q},\dots,X_n^{1/q}]$. Now clearly $A'$ is a finitely generated $C'$-module and, as $k'$ is a finite extension of $k$, $C'$ is a finitely generated $A$-module and hence $A'$ is a finitely generated $A$-module. Since $A$ is Noetherian and $\widebar{A}\sub A'$, we see $\widebar{A}$ is a finitely generated $A$-module.
\end{proof}
\subsection{The Nullstellensatz}
\begin{proposition}\label{algebra finite homomorphism to algebraic closure prop}
Let $A$ be a finitely generated algebra over a field $k$ and $\Omega$ the algebraic closure of $k$.
\begin{itemize}
\item[(a)] If $A\neq\{0\}$, there exists a nonzero $k$-homomorphism from $A$ to $\Omega$.
\item[(b)] Let $\phi_1,\phi_2$ be two $k$-homomorphisms from $A$ to $\Omega$. For $\phi_1$ and $\phi_2$ to have the same kernel, it is necessary and sufficient that there exist a $k$-automorphism $\sigma$ of $\Omega$ such that $\phi_2=\sigma\circ\phi_1$.
\item[(c)] Let $\a$ be an ideal of $A$. For $\a$ to be maximal, it is necessary and suffieient that it be the kernel of a $k$-homomorphism from $A$ to $\Omega$.
\item[(d)] For an element $x$ of $A$ to be such that $\phi(x)=0$ for every $k$-homomorphism from $A$ to $L$, it is necessary and sufficient that $x$ be nilpotent.  
\end{itemize}
\end{proposition}
\begin{proof}
Assertion (a) follows \cref{algebra finite homomorphism to algebraically closed prop} applied replacing $A$ by $k$, $B$ by $A$, $b$ by the unit element of $B$ and $\phi$ by the canonical injection of $k$ into $\Omega$.\par
If $\phi$ is a $k$-homomorphism from $A$ to $\Omega$, $\phi(A)$ is a subring of $\Omega$ containing $k$. As $\Omega$ is an algebraic extension of $k$, $\phi(A)$ is a field and, if $\a$ is the kernel of $\phi$, $A/\a$ is isomorphic to $\phi(A)$, therefore is a field, which proves that $\a$ is maximal. Conversely, if $\a$ is a maximal ideal of $A$ it follows from (a) that there exists a $k$-homomorphism from $A/\a$ to $\Omega$ and hence a $k$-homomorphism of $A$ to $\Omega$ whose kernel $\b$ contains $\a$. But as $\a$ is maximal, $\b=\a$, this proves (c).\par
We now prove (b). If $\sigma$ is a $k$-automorphism of $\Omega$ such that $\phi_2=\sigma\circ\phi_1$, clearly $\phi_1$ and $\phi_2$ have the same kernel. Conversely, suppose that $\phi_1$ and $\phi_2$ have the same kernel. Then there exists a $k$-isomorphism $\sigma_0$ of the field $\phi_1(A)$ onto the field $\phi_2(A)$ such that $\phi_2=\sigma\circ\phi_1$. But by \cref{field embedding ext algebraic case}, $\sigma_0$ extends to a $k$-automorphism $\sigma$ of $\Omega$ and hence $\phi_2=\sigma\circ\phi_1$.\par
Finally, if $x\in A$ is such that $x^n=0$, for every $k$-homomorphism $\phi$ from $A$ to $\Omega$, $(\phi(x))^n=\phi(x^n)=0$ and hence $\phi(x)=0$ since $\Omega$ is a field. Conversely, suppose that $x\in A$ is not nilpotent. Then $A_x$ is a finitely generated $A$-algebra (and therefore a finitely generated $k$-algebra) not reduced to $0$ and hence there exists a $k$-homomorphism $\psi$ from $A_x$ to $\Omega$ by (a). If $j:A\to A_x$ is the canonical homomorphism, $\phi=\psi\circ j$ a $k$-homomorphism from $A$ to $\Omega$ and $\phi(x)\psi(1/x)=\psi(x/1)\psi(1/x)=\psi(1)=1$, whence $\phi(x)\neq 0$.
\end{proof}
\begin{lemma}\label{algebra finite generating relation prop}
Let $A$ be a finitely generated algebra over a field $k$, $(a_i)_{1\leq i\leq n}$ a system of generators of this algebra and $\r$ the ideal of algebraic relations between the $a_i$ with eoeflieients in $k$. For every extension field $L$ of $k$, the map $\phi\mapsto(\phi(a_i))$ is a bijection of the set of $k$-homomorphisms from $A$ to $L$ onto the set of zeros of $\r$ in $L^n$.
\end{lemma}
\begin{proof}
There exists a unique $k$-algebra homomorphism $\eta$ of $k[X_1,\dots,X_n]$ onto $A$ such that $\eta(X_i)=a_i$ for $1\leq i\leq n$ and by definition $\r$ is the kernel of $\eta$. The map $\phi\mapsto\phi\circ\eta$ is a bijection of the set of $k$-homomorphisms from $A$ to $\Omega$ onto the set of $k$-homomorphisms from $k[X_1,\dots,X_n]$ to $\Omega$ which are zero on $\r$. For every polynomial $P\in k[X_1,\dots,X_n]$ and every element $\bm{x}=(x_1,\dots,x_n)\in L^n$ we write $\eta_{\bm{x}}(P)=P(\bm{x})$. Then the map $\bm{x}\mapsto \eta_{\bm{x}}$ is a bijection of $L^n$ onto the set of $k$-homomorphisms from $k[X_1,\dots,X_n]$ to $L$ (such a homomorphism being determined by its values at the $X_i$). To say that $\eta_{\bm{x}}$ is zero on $\r$ means that $\bm{x}$ is a zero of $\r$ in $L^n$, whence the lemma.
\end{proof}
\begin{proposition}[\textbf{Hilbert's Nullstellensatz}]
Let $k$ be a field and $\Omega$ an algebraic closure of $k$.
\begin{itemize}
\item[(a)] Every proper ideal $\r$ of $k[X_1,\dots,X_n]$ admits at least one zero in $\Omega^n$.
\item[(b)] Let $\bm{x},\bm{y}$ be two elements of $\Omega^n$. For the set of polynomials of $k[X_1,\dots,X_n]$ zero at $\bm{x}$ to be identical with the set of polynomials of $k[X_1,\dots,X_n]$ zero at $\bm{y}$, it is necessary and sufficient that there exists a $k$-automorphism $\sigma$ such that $\bm{y}=\sigma(\bm{x})$.
\item[(c)] For an ideal $\a$ of $k[X_1,\dots,X_n]$ to be maximal, it is necessary and sufficient that there exist an $\bm{x}$ in $\Omega^n$ such that $\a$ is the set of polynomials of $k[X_1,\dots,X_n]$ zero at $\bm{x}$.
\item[(d)] For a polynomial $P$ of $k[X_1,\dots,X_n]$ to be zero on the set of zeros in $\Omega^n$ of an ideal $\r$ of $k[X_1,\dots,X_n]$, it is necessary and sufficient that there exist an integer $n>0$ such that $P^n\in\r$.
\end{itemize}
\end{proposition}
\begin{proof}
Apply \cref{algebra finite homomorphism to algebraic closure prop} to the algebra $A=k[X_1,\dots,X_n]/\r$ and use \cref{algebra finite generating relation prop}, we obtain part (a) and (d). Also, part (b) and (c) follows by the same argument applied to $A=k[X_1,\dots,X_n]$.
\end{proof}
\subsection{Jacobson rings}
\begin{proposition}\label{Jacobson ring def}
Let $A$ be a ring. Then the following conditions are equivalent.
\begin{itemize}
\item[(\rmnum{1})] Every prime ideal of $A$ is the intersection of a family of maximal ideals.
\item[(\rmnum{2})] For every ideal $\a$ of $A$, the Jacobson radical of $A/\a$ be equal to its nilradical.
\item[(\rmnum{3})] For every ideal $\a$ of $A$, the radical of $\a$ is the intersection of the maximal ideals of $A$ containing $\a$.
\item[(\rmnum{4})] Every prime ideal in $A$ which is not maximal is equal to the intersection of the prime ideals which contain it strictly.
\end{itemize}
\end{proposition}
\begin{proof}
The Jacobson radical (resp. nilradical) of $A/\a$ is the intersection of the maximal (resp. prime) ideals of $A/\a$. Thus (\rmnum{2}) means for every ideal $\a$ of $A$ the intersection of the prime ideals containing $\a$ is equal to the intersection of the maximal ideals containing $\a$. This condition obviously holds for every ideal $\a$ of $A$ if (\rmnum{1}) holds, and it implies (\rmnum{1}) by taking $\a$ to be prime ideals of $A$. This proves the equivalence of (\rmnum{1}) and (\rmnum{2}). Also, it is clear that (\rmnum{3}) is equivalent to (\rmnum{2})\par
Now assume (\rmnum{1}) and let $\p$ be a prime ideal of $A$ that is not maximal. Then by (\rmnum{1}), since $\p$ is non maximal, we have
\[\p=\bigcap_{\m\supsetneq\p,\m\in\Max(A)}\m\sups\bigcap_{\q\supsetneq\p,\q\in\Spec(A)}\q\sups\p\]
whence (\rmnum{4}) holds. Conversely, suppose there is a prime ideal which is not an intersection of maximal ideals. Passing to the quotient ring, we may then assume that $A$ is an integral domain whose Jacobson radical $\r$ is not zero. Let $a$ be a non-zero element of $\r$. Then $A_a\neq 0$ since $a$ is not nilpotent, hence has a maximal ideal, whose contraction in $A$ is a prime ideal $\p$ such that $a\notin\p$. Since $a\in\r$ we see $\p$ is not maximal. Moreover, by our choice of $\p$, every prime ideal of $A$ that strictly contains $\p$ has nonempty with the multiplicative subset $S=\{a^n:n>0\}$ and therefore contains $a$. Hence $a$ is contained in the intersection of prime ideals containing $\p$ strictly, which therefore is not equal to $\p$. This shows the equivalence of (\rmnum{1}) and (\rmnum{4}) and completes the proof.
\end{proof}
A ring $A$ is called a \textbf{Jacobson ring} if it satisfies the equivalent conditions of \cref{Jacobson ring def}.
\begin{example}[\textbf{Examples of Jacobson rings}]
\mbox{}
\begin{itemize}
\item[(a)] Every field is a Jacobson ring.
\item[(b)] The ring $\Z$ is a Jacobson ring, the unique prime ideal which is not maximal $(0)$ being the intersection of the maximal ideals $(p)$ of $\Z$, where $p$ runs through the set of prime numbers.
\item[(c)] Let $A$ be a Jacobson ring and let $\a$ be an ideal of $A$. Then $A/\a$ is a Jacobson ring, for the ideals of $A/\a$ are of the form $\b/\a$, where $\b$ is an ideal of $A$ containing $\a$ and $\b/\a$ is prime (resp. maximal) if and only if $\b$ is.
\end{itemize}
\end{example}
\begin{proposition}\label{Jacobson PID iff}
Let $A$ be a PID and $(p_i)_{i\in I}$ a representative system of irreducible elements of $A$. For $A$ to be a Jacobson ring, it is necessary and sufficient that $I$ be infinite.
\end{proposition}
\begin{proof}
The maximal ideals of $A$ are the $(p_i)$. If $I$ is finite, their intersection is the ideal $(x)$ where $x=\prod_ip_i$ and hence different from $(0)$. On the other hand, if $I$ is infinite, the intersection of the
$(p_i)$ is $(0)$, since every nonzero element of $A$ can be divisible by only a finite number of irreducible elements. The proposition then follows from the fact that $(0)$ is the only prime ideal which is not maximal in $A$.
\end{proof}
\begin{proposition}\label{Jacobson ring integral algebra is Jacobson}
Let $A$ be a ring and $B$ an $A$-algebra integral over $A$. If $A$ is a Jacobson ring, so is $B$.
\end{proposition}
\begin{proof}
Replacing $A$ by its canonical image in $B$, we may assume that $A\sub B$. Let $\mathfrak{P}$ be a prime ideal of $B$ and let $\p=A\cap\mathfrak{P}$. There exists by hypothesis a family $(\m_i)_{i\in I}$ of maximal ideals of $A$ whose intersection is equal to $\p$. For all $i\in I$ there exists a maximal ideal $\mathfrak{M}_i$ of $B$ lying over $\m_i$ and containing $\mathfrak{P}$ by the going up theorem. If we write $\mathfrak{Q}=\bigcap_i\mathfrak{M}_i$, then $\mathfrak{Q}\cap A=\bigcap_i\m_i=\p$ and $\mathfrak{Q}\sups\mathfrak{P}$, whence $\mathfrak{Q}=\mathfrak{P}$.
\end{proof}
\begin{proposition}\label{Jacobson ring finite algebra is Jacobson}
Let $A$ be a Jacobson ring, $B$ a finitely generated $A$-algebra and $\rho:A\to B$ the canonical homomorphism. Then $B$ is a Jacobson ring and for every maximal ideal $\mathfrak{M}$ of $B$, $\m=\mathfrak{M}^c$ is a maximal ideal of $A$ and $B/\mathfrak{M}$ is a finite algebraic extension of $A/\m$.
\end{proposition}
\begin{proof}
Let $\mathfrak{P}$ be a prime ideal of $B$ and $\p=\mathfrak{P}^c$. Let $v$ be a nonzero element of $B/\mathfrak{P}$. As $B/\mathfrak{P}$ is a finitely generated integral $(A/\p)$-algebra and the canonical homomorphism $\eta:A/\p\to B/\mathfrak{P}$ is injective, there exists a nonzero element $u$ of $A/\p$ such that, for every homomorphism $\phi$ from $A/\p$ to an algebraically closed field $\Omega$ such that $\phi(u)\neq 0$, there exists a homomorphism $\psi$ from $B/\mathfrak{P}$ to $\Omega$ such that $\psi(v)\neq 0$ and for which $\phi=\psi\circ\eta$ (\cref{algebra finite homomorphism to algebraically closed prop}). Since $A$ is a Jacobson ring, there exists a maximal ideal $\m$ of $A$ containing $\p$ and such that $u\notin\m/\p$. We take $\Omega$ to be an algebraic closure of $A/\m$ and $\phi$ to be the canonical homomorphism $A/\p\to\Omega$. Let $\psi:B/\mathfrak{P}\to\Omega$ be a homomorphism such that $\phi=\psi\circ\eta$ and $\psi(v)\neq 0$. Then
\[A/\m\sub\phi(B/\mathfrak{P})\sub\Omega\]
hence $\psi(B/\mathfrak{P})$ is a subfield of $\Omega$ and the kernel of $\psi$ is therefore a maximal ideal of $B/\mathfrak{P}$ not containing $v$. Thus it is seen that the intersection of the maximal ideals of $B/\mathfrak{P}$ is reduced to $0$, which proves that $B$ is a Jacobson ring. Moreover, if $\mathfrak{P}$ is maximal, $\psi$ is necessarily injective and hence $\p=\m$ is maximal. Finally $B/\mathfrak{P}$ is then a finitely generated algebra over the field $A/\m$ and hence is a finite extension of $A/\m$.
\end{proof}
\begin{corollary}\label{Jacobson ring finite algebra field then field}
Let $A$ be a Jacobson ring. If there exists a finitely generated $A$-algebra $B$ containing $A$ and which is a field, then $A$ is a field and $B$ is an algebraic extension of $A$.
\end{corollary}
\begin{proof}
It suffices to apply \cref{Jacobson ring finite algebra is Jacobson} with $\mathfrak{M}=(0)$.
\end{proof}
\begin{corollary}
Every finitely generated algebra $A$ over $\Z$ is a Jacobson ring. For a prime ideal $\p$ of $A$ to be maximal, it is necessary and sufficient that the ring $A/\p$ be finite.
\end{corollary}
\begin{proof}
If the integral domain $A/\p$ is finite, it is a field, as, for every $u\neq 0$ in $A/\p$, the map $v\mapsto uv$ of $A/\p$ to itself is injective and hence bijective since $A/\p$ is finite. Conversely, for every maximal ideal $\m$ of $A$, the inverse image of $\m$ in $\Z$ is a maximal ideal $(p)$ and $A/\m$ is finite over the prime field $\Z/(p)=\F_p$ by \cref{Jacobson ring finite algebra is Jacobson}.
\end{proof}
\begin{remark}
Let $A$ be a Jacobson ring. Then the ring $A\llbracket X\rrbracket$ is not Jacobson: in fact, let $\p$ be a prime ideal of $A$, then $\p A\llbracket X\rrbracket$ is not an intersection of maximal ideals of $A\llbracket X\rrbracket$ since any such ideal contains $X\cdot A\llbracket X\rrbracket$.
\end{remark}
\begin{theorem}\label{Jacobson ring iff finite algebra field is finite}
Let $A$ be a ring. Then the following are equivalent.
\begin{itemize}
\item[(\rmnum{1})] $A$ is a Jacobson ring.
\item[(\rmnum{2})] Every finitely generated $A$-algebra $B$ that is a field is finite over $A$.
\end{itemize}
\end{theorem}
\begin{proof}
First assume (\rmnum{2}). Let $\p$ be a prime ideal of $A$ which is not maximal, and let $B=A/\p$. Let $v$ be a non-zero element of $B$. Then $B_v=B[v^{-1}]$ is a finitely generated $A$-algebra. If it is a field then by condition (\rmnum{2}) it is finite over $A$, hence integral over $B$ and therefore $B$ is a field by \cref{integral ring extension field iff}. Since $\p$ is not maximal, we then see $B_v$ is not a field and therefore has a non-zero maximal ideal, whose contraction in $B$ is a non-zero prime ideal $\q$ such that $v\notin\q$. By \cref{Jacobson ring def}(\rmnum{4}), this shows $A$ is a Jacobson ring.\par
Conversely, let $A$ be a Jacobson ring and $B$ be a finitely generated $A$-algebra that is a field. Since a quotient ring of a Jacobson ring is Jacobson, we can assume $B$ contains $A$ as a subring. Let $s\in A$ be such that for every homomorphism $\phi$ from $A$ to an algebraically closed field $\Omega$ such that $\phi(s)\neq 0$, there exists a homomorphism $\tilde{\phi}$ from $B$ to $\Omega$ extending $\phi$ (\cref{algebra finite homomorphism to algebraically closed prop}). Since $A$ is then a integral domain and Jacobson, there exists a maximal ideal $\m$ of $A$ not containing $s$. We take $\Omega$ to be the algebraic closure of $A/\m$ and $\phi$ the canonical homomorphism $A\to A/\m\to\Omega$. Then $\phi$ extends to a homomorphism $\psi$ of $B$ into $\Omega$. Since $B$ is a field, $\psi$ is injective, and so $B$ is algebraic (thus finite algebraic) over $A/\m$, whence finite over $A$.
\end{proof}
Since a field is Jacobson, we get immediately the following statement.
\begin{corollary}[\textbf{Zariski's Lemma}]\label{field ft algebra Zariski lemma}
Let $k$ ba a field and $K$ a finitely generated $k$-algebra. If $K$ is a field then it is a finite field extension of $k$.
\end{corollary}
\section{Exercise}
\begin{exercise}
Let $A$ be a Noetherian ring and $\p_1,\dots,\p_n$ be all the minimal prime ideals of $A$. Suppose that $A_{\p}$ is an integral domain for all $\p\in\Spec(A)$. Then
\begin{itemize}
\item[(a)] $\Ass(A)=\{\p_1,\dots,\p_n\}$.
\item[(b)] $\n(A)=\bigcap_{i=1}^{n}\p_i=(0)$.
\item[(c)] $\p_i+\bigcap_{j\neq i}\p_j=A$ for all $i$.
\end{itemize}
Deduce that $A\cong A/\p_1\times\cdots\times A/\p_n$.
\end{exercise}
\begin{proof}
Recall $\n(A_\p)=(\n(A))_\p$ from \cref{localization and nilradical}, but $(\n(A))_\p=(0)$ for all $\p\in\Spec(A)$, so by \cref{localization module zero iff} $\n(A)=\bigcap_{i=1}^{n}\p_i=(0)$. By \cref{associated prime of union of zero intersection module},we then have
\[\Ass(A)\sub\bigcup_{i=1}^{n}\Ass(A/\p_i)=\{\p_1,\dots,\p_n\}.\]
Since $\supp(A)=\Spec(A)$ and $\p_1,\dots,\p_n$ are the minimal element of $\Spec(A)$, we then get $\Ass(A)=\{\p_1,\dots,\p_n\}$.\par
We first prove that $\p_i+\p_j=A$ for any $i\neq j$: Assume the converse, then $\p_i+\p_j\sub\m$ for some maximal ideal $\m$. Then, in the localization $A_\m$ there are two distinct minimal ideals $\p_i A_{\m}$ and $\p_j A_\m$. Thus
\[\n(A_\m)=\p_i A_{\m}\cap \p_j A_{\m}\]
Since $A_\m$ is an integral domain, and $\p_i\cap\p_j\neq 0$, this is a contradiction.\par
Now since $\p_i+\p_j=A$, for $j\neq i$ we have
\[x_j+y_j=1,\quad x_j\in\q_i,y_j\in\q_j\]
Now 
\[\prod_{j\neq i}(1-x_i)=\prod_{j\neq i}y_j\in\bigcap_{j\neq i}\q_j\]
and expand the term $\prod_{j\neq i}(1-x_i)$ we get $1+x$ for $x\in \q_i$. Hence $\q_i+\bigcap_{j\neq i}\p_j=A$.\par
Finally, $\p_i$ and $\p_j$ are coprime, by the Chinese remainder theorem we have the claimed isomorphism.
\end{proof}
\chapter{Valuation rings}
\section{Valuation rings of fields}
Let $A$ be an integral domain, $K$ its field of fractions. $A$ is a \textbf{valuation ring} if, for each $x\in K$ and $x\neq 0$, either $x\in A$ or $x^{-1}\in A$ (or both). The case $A=K$ is the trivial valuation ring.
\begin{proposition}\label{valuation iff}
Let $A$ be an integral domain, $K$ its field of fractions. Then $A$ is a valuation ring of $K$ if and only if idels of $A$ is totally ordered by inclusion.
\end{proposition}
\begin{proof}
Assume that $A$ is a valuation ring of $K$ and let $I,J$ be two ideals such that there is $b\in J$, $b\notin I$. Now for $a\in I$, either $a/b\in A$ or $b/a\in A$. If $b/a\in A$, then $b=(b/a)\cdot a\in I$, contradiction. So $a/b\in A$, and $a=b(a/b)\in J$. Hence $I\sub J$. Conversely, assume that ideals of $A$ are totally ordered. Let $a,b\in A$. Then $(a)\sub(b)$ or $(b)\sub(a)$. Hence it is clear $a/b\in A$ or $b/a\in A$. This shows $A$ is a valuation ring.
\end{proof}
\begin{corollary}
If $A$ is a valuation ring and $\p$ is a prime ideal of $A$, then $A/\p$ and $A_\p$ are valuation rings of their fields of fractions.
\end{corollary}
\begin{proposition}\label{valuation ring prop}
If $A$ is a valuation ring of $K$, then
\begin{itemize}
\item[(a)] $A$ is a local ring and if we write $\m$ for the maximal ideal, then
\[K-A=\{x\in K^\times:x^{-1}\in\m\}\]
Thus $A$ is determined by $K$ and $\m$.
\item[(b)] $A$ is integrally closed.
\end{itemize}
\end{proposition}
\begin{proof}
By \cref{valuation iff}, $A$ has a unique maximal ideal, so is a local ring. Moreover, if $x\notin A$, then $x^{-1}\in A$ and $x^{-1}$ is not a unit in $A$, so $x^{-1}\in\m$. conversely, if $x^{-1}\in\m$, then $x\notin A$ since $x^{-1}$ is not a unit. Thus
\[K-A=\{x\in K^\times:x^{-1}\in\m\}\] 
Now given $K$ and $\m$, the set $K-A$ is then determined, so $A=K-(K-A)$ is uniquely determined.\par
Let $x\in K$ be integral over $A$. Then we have, say,
\[x^n+a_{n-1}x^{n-1}+\cdots+a_0=0\]
with the $b_i\in A$. If $x\in A$ there is nothing to prove. If not, then $x^{-1}\in A$, hence $x=-(a_{n-1}+a_{n-2}x^{-1}+\cdots+a_0x^{-n})\in A$.
\end{proof}
\begin{proposition}\label{valuation ring finite generated ideal is principal}
Let $A$ be an integral domain, then the followings are equivalent:
\begin{itemize}
\item[(\rmnum{1})] $A$ is a valuation ring.
\item[(\rmnum{2})] $A$ is a local domain and any finitely generated ideal is principal.
\end{itemize}
\end{proposition}
\begin{proof}
Let $A$ be a valuation ring. By \cref{valuation ring prop} $A$ is a local ring. Let $I=(x_1,\dots,x_n)$, since by \cref{valuation iff} the set $\{(x_1),\dots,(x_n)\}$ is totally ordered. Let's assume that $(x_1)$ is the biggest element in it, then $(x_i)\sub(x_1)$ for all $i$, that is, $x_i=a_ix_1$. This implies $I$ is generated by $x_1$.\par
Conversely, assume (b), and let $\m$ be the unique maximal ideal of $A$. Write $\m$ as the unqiue maximal ideal of $A$, so $J(A)=\m$. Let $a,b\in A$, and write $(a,b)=(h)$. Then we have
\[a=uh,\ b=vh,\quad h=xa+yb,\quad u,v,x,y\in A.\]
Then $h=xuh+yvh$ are since $A$ is a domain we get $1=xu+yv$. If $u$ is not a unit, then $u\in\m$ and hence $1-xu=yv$ is a unit. This implies $v$ is a unit, and vice versa. Hence either $a/b\in A$ or $b/a\in A$, so $A$ is a valuation ring.
\end{proof}
\begin{proposition}\label{valuation ring finitely generated module is free}
Let $A$ be a valuation ring. Every finitely generated torsion-free $A$-madule is free. Every torsion-free $A$-module is flat.
\end{proposition}
\begin{proof}
Let $M$ be a finitely generated torsion-free A-module and let $x_1,\dots,x_n$ be generators of $E$ which are minimal in number; we show that they are linearly independent. If $\sum_{i=1}^{n}a_ix_i=0$ is a non-trivial relation between the $x_i$, one of the $a_i$'s, say $a_1$, divides all the others since the set of principal ideals of $A$ is totally ordered by inclusion; then $a_1\neq 0$ since the relation is non-trivial. As $M$ is torsion-free, we can divide by $a_1$, which amounts to assuming that $a_1=1$. But then $x_1$ is a linear combination of $x_2,\dots,x_n$, contrary to the minimal character of $n$. Hence $M$ is free. The second claim follows since every finitely generated ideal is principal.
\end{proof}
Quite generally, we write $\m_A$ for the maximal ideal of a local ring $A$. If $A$ and $B$ are local rings with $B\sub A$ and $\m_B\cap A=\m_A$, we say that $B$ \textbf{dominates} $A$ and write $B\geq A$. If $B\geq A$ and $B\neq A$, we write $B>A$.
\begin{proposition}\label{local ring dominate iff}
Let $A$ and $B$ be local rings such that $A\sub B$. Then the following conditions are equivalent:
\begin{itemize}
\item[(\rmnum{1})] $\m_A\sub\m_B$;
\item[(\rmnum{2})] $B$ dominates $A$;
\item[(\rmnum{3})] the ideal generated by $\m_A$ in $B$ is proper.
\end{itemize}
\end{proposition}
\begin{proof}
If $\m_A\sub\m_B$, then $\m_B\cap A$ is a proper prime ideal containing the maximal ideal of $A$, so $\m_A=\m_B\cap A$ and $B$ dominates $A$. If $B$ dominates $A$, then the ideal $\m_AB$ is contained in $\m_B$ and so is proper, this shows (\rmnum{2})$\Rightarrow$(\rmnum{3}). Finally, if (\rmnum{3}) holds, $\m_AB$ is contained in the unique maximal ideal of $\m_B$ of $B$, hence (\rmnum{1}). 
\end{proof}
\begin{theorem}\label{valuation ring dominate prop}
Let $K$ be a field, $A\sub K$ a subring, and $\p$ a prime ideal of $A$. Then there exists a valuation ring $B$ of $K$ satisfying
\[A\sub B,\quad \m_B\cap A=\p.\]
Moreover, valuation rings are maximal with respect to the domination relation.
\end{theorem}
\begin{proof}
Replacing $A$ by $A_\p$ we can assume that $A$ is a local ring with $\p=\m_A$. Now write $\mathscr{F}$ for the set of all subrings of $K$ containing $A$ and to which $\p$ extended to a nontrivial ideal. Now $A\in\mathscr{F}$, and if $\mathscr{L}\sub\mathscr{F}$ is a subset totally ordered by inclusion then the union of all the elements of $\mathscr{L}$ is again an element of $\mathscr{F}$, so that, by Zorn's lemma $\mathscr{F}$ has a maximal element $B$. Since $\p^e\neq B$, there is a maximal ideal $\m$ of $B$ containing $\p^e$. Then $B\sub B_\m\in\mathscr{F}$, so that $B=B_\m$, and $B$ is local. Also $\p\sub\m\cap A$ and $\p$ is a maximal ideal of $A$, so that $\m\cap A=\p$. Thus it only remains to prove that $B$ is a valuation ring of $K$.\par 
Let $x\in K$. If $x\notin B$ then since $B[x]\notin\mathscr{F}$ and $B[x]\sups A$, we must have $1\in\p B[x]$, and there exists a relation of the form
\[1=a_0+a_1x+\cdots+a_nx^n,\quad a_i\in \p B\]
Since $a_0\in\p B\sub\m=J(B)$, $1-a_0$ is unit of $B$ and we can modify this to get a relation
\begin{align}\label{valuation-1}
1=b_1x+\cdots+b_nx^n,\quad b_i\in\m
\end{align}
Among all such relations, choose one for which $n$ is as small as possible. If we also have $x^{-1}\notin B$, then by the same argument, there is a relation
\begin{align}\label{valuation-2}
1=c_1x^{-1}+\cdots+c_mx^{-m},\quad c_i\in\m
\end{align}
for which $m$ is as small as possible. If $n\geq m$ we can multiply (\ref{valuation-2}) by $b_nx^n$ and substract from (\ref{valuation-1}) to obtain a relation of the form (\ref{valuation-1}) but with a strictly smaller degree $n$, which is a contradiction; if $n<m$ then we get the same contradiction on interchanging the roles of $x$ and $x^{-1}$. Thus if $x\notin B$ we must have $x^{-1}\in B$.\par
Finally, if $A\sub B$ are valuation rings of $K$, then by \cref{valuation ring superring char} we have $\m_B\sub\m_A$, and $\m_A=\m_B$ iff $A=B$. It follows that $B$ dominates $A$ iff $A=B$.
\end{proof}
\begin{theorem}\label{integral closure is intersection of valuation ring}
Let $K$ be a field, $A\sub K$ a subring, and let $\widebar{A}$ be the integral closure of $A$ in $K$. Then $\widebar{A}$ is the intersection of all the valuation rings of $K$ containing $A$. If $A$ is local, $\widebar{A}$ is the intersection of the valuation rings of K which dominate A.
\end{theorem}
\begin{proof}
Every element $x\in\widebar{A}$ is integral over $A$, so over any valuation ring $B\sups A$. But since $B$ is integer closed, we must have $x\in B$, thus $\widebar{A}\sub B$ for every valuation ring containing $A$. Conversely, let $x\notin\widebar{A}$. Then $x\notin A[x^{-1}]$, so $x^{-1}$ is not a unit of $A[x^{-1}]$. It follows that $x^{-1}$ is contained in a maximal ideal $\mathfrak{M}$ of $A[x^{-1}]$, and by \cref{valuation ring dominate prop} there exists a valuation ring $B$ of $K$ such that $B\sups A[x^{-1}]$ and $\m_B\cap A[x^{-1}]=\mathfrak{M}$. Now since $x^{-1}\in\mathfrak{M}\sub\m_B$, we have $x\notin B$. Further $\mathfrak{M}\cap A$ is a maximal ideal of $A$, hence, if $A$ is local, $\mathfrak{M}\cap A=\m_A$ and $B$ dominates $A$.
\end{proof}
\begin{corollary}\label{integrally closed iff intersection of valuation ring}
For an integral domain to be integrally closed, it is necessary and sufficient that it be the intersection of a family of valuation rings of its field of fractions.
\end{corollary}
\begin{corollary}\label{valaution ring integral closure is intersection}
Let $K$ be a field, $L$ an extension of $K$ and $A$ a valuation ring of $K$. The integral closure of $A$ in $L$ is the intersection of the valuation rings $B$ of $L$ such that $B\cap K=A$.
\end{corollary}
\begin{proof}
If $B$ is a valuation ring of $L$, then $B\cap K$ is a valuation ring of $K$ and $B$ dominates $B\cap K$. For $B$ to dominate $A$, it is necessary and sufficient that $B\cap K$ dominate $A$ and therefore be equal to it.
\end{proof}
\section{Ordered groups and valuations}
Let $\Gamma$ be a commutative group. A order $\leq$ on $\Gamma$ is said to be compatible with the group structure of $\Gamma$ if for $\alpha,\beta,\gamma,\delta\in\Gamma$,
\[\alpha\geq\beta,\gamma\geq\delta\Rightarrow\alpha+\gamma\geq\beta+\delta.\]
In this case, the group $\Gamma$ is called an \textbf{ordered group}, and a \textbf{totally ordered group} if its order is a total order. In this section we only consider totally ordered groups and if $\Gamma$ is an ordered group, we usually make an ordered set $\Gamma\cup\{\infty\}$ by adding to $\Gamma$ an element $\infty$ bigger than all the elements of $\Gamma$, and fix the conventions $\infty+\alpha=\infty$ for $\alpha\in\Gamma$ and $\infty+\alpha=\infty$.
\begin{example}[\textbf{Examples of totally ordered groups}]
\mbox{}
\begin{itemize}
\item[(a)] The additive group of real numbers $\R$, or any subgroup of this.
\item[(b)] The direct product $\Z^n$ of $n$ copies of $\Z$, with lexicographical order.
\item[(c)] The group $\Q$ of rational integers.
\end{itemize}
\end{example}
\begin{theorem}\label{totally ordered group iff torsion free}
For a commutative group $\Gamma$ to be such that there exists on $\Gamma$ a total ordering compatible with the group struclure of $\Gamma$, it is necessary aad sufficient that $\Gamma$ be torsion free.
\end{theorem}
\begin{proof}
If there exists such an order structure on $\Gamma$ and if $\alpha>0$, then $\alpha+\mu>0$ for all $\mu>0$ and in particular, by induction we see $n\alpha\geq 0$ which proves that $\Gamma$ is torsion free. Conversely, if $\Gamma$ is torsion-free, it is a sub-$\Z$-module of a vector $\Q$-space, which may be assumed of the form $\Q^{\oplus I}$. If $I$ is given a well ordering and $\Q^{\oplus I}$ its usual ordering, the set $\Q^{\oplus I}$ with the lexicographical ordering is totally ordered. It is immediate that this ordering is compatible with the additive group structure of $\Q^{\oplus I}$.
\end{proof}
Let $K$ be a field and $\Gamma$ a totally ordered group. A map $v:K\to\Gamma\cup\{\infty\}$ is called an \textbf{additive valuation} or just a \textbf{valuation} of $K$ if it satisfies the conditions
\begin{itemize}
\item[(a)] $v(xy)=v(x)+v(y)$.
\item[(b)] $v(x+y)\geq\min\{v(x),v(y)\}$.
\item[(c)] $v(x)=\infty$ if and only if $x=0$.
\end{itemize}
If we write $K^\times$ for the multiplicative group of $K$ then $v$ defines a homomorphism $K^\times\to\Gamma$ whose image is a subgroup of $\Gamma$, called the \textbf{value group} of $v$. The valuation $v$ defined by $ν(x)=0$ for any $x\in K^\times$ is called the \textbf{trivial valuation}.
\begin{proposition}\label{valuation addition prop}
Let $v$ be a valuation on a field $K$. For any elements $x_1,\dots,x_n\in K$, we have
\[v(\sum_{i=1}^{n}x_i)\geq\min\{v(x_1),\dots,v(x_n)\}.\]
Moreover, if there exists a single index $k$ such that $v(x_k)=\min_iv(x_i)$, then the equality holds. In particular, if $v(x)\neq v(y)$, then $v(x+y)=\min\{v(x),v(y)\}$.
\end{proposition}
\begin{proof}
The first inequality follows by induction. Now if there exists a single index $k$ such that $v(x_k)=\min_iv(x_i)$, then, writing $y=\sum_{i\neq k}x_i$ and $z=\sum_{i=1}^{n}x_i$, we have $v(y)>v(x_k)$ and $v(z)\geq v(x_k)$. If $v(z)>v(x_k)$, then the relation $x_k=z-y$ implies $v(x_k)\geq\min\{v(y),v(z)\}>v(x_k)$, which is absurd. Thus $v(z)=v(x_k)$ and the claim is proved. 
\end{proof}
Valuations are connected to valuation rings in the following way. If $v$ is a valuation on $K$, we set
\[A_v=\{x\in K\mid v(x)\geq 0\},\quad \m_v=\{x\in K\mid v(x)>0\},\quad\kappa_v=A_v/\m_v.\]
Then by the properties of $v$ we see that $A_v$ is a valuation ring and $\m_v$ is a maximal ideal. We call $A_v$ the \textbf{valuation ring} of $v$, $\m_v$ the \textbf{valuation ideal} of $v$, and $\kappa_v$ the \textbf{residue field} of $v$. Conversely, if $A$ is a valuation ring of $K$, then the set $\Gamma=\{xA:x\in K^\times\}$ is a group with operation defined by $xA\cdot yA=xyA$. Moreover, if we define an order $\leq$ on $\Gamma$ by
\[xA\leq yA\iff yA\sub xA\iff y/x\in A\]
then $\Gamma$ becomes a totally ordered group with $\leq$ (since $A$ is a valuation ring), and we obtain an additive valuation of $K$ with value group $\Gamma$ by setting $v(x)=xA$ and $v(0)=\infty$. It is easy to see the valuation ring of $v$ is $A$.\par
The additive valuation corresponding to a valuation ring $A$ is not quite unique, so we introduce the following notation. Two valuations $v$ and $w$ on a field $K$ with value groups $\Gamma_v$ and $\Gamma_w$ are called equivalent if and only if there exists an order-isomorphism $\varphi:\Gamma_v\to\Gamma_w$ such that $w=\varphi\circ v$.
\begin{theorem}\label{valuation correspond to valuation ring}
Let $K$ be a field. Then valuation rings of $K$ are in one-to-one correspondence to equivalent class of valuations on $K$.
\end{theorem}
\begin{proof}
One direction is clear: If two valuations are equivalent, then they determine the same valuation ring. For the converse, assume that $v$ and $w$ are valuations on $K$ with value groups $\Gamma_v$ and $\Gamma_w$ such that $A_v=A_w$. We define the homomorphism $\varphi:\Gamma_v\to\Gamma_w$ by
\[\varphi(v(x))=w(x)\for x\in K^\times.\]
Since $U(A_v)=U(A_w)$, it is easy to see $\varphi$ is well defined and injective. Clearly $\varphi$ satisfies the condition $w=\varphi\circ v$ and is surjective. Now, for $x,y\in K^\times$ we have
\[\varphi(v(x)+v(y))=\varphi(v(xy))=w(xy)=w(x)+w(y),\]
and
\[v(x)>0\iff x\in A_v=A_w\iff w(x)=\varphi(v(x))>0.\]
Thus $\varphi$ is an order-isomorphism from $\Gamma_v$ to $\Gamma_w$ and $v$ and $w$ are equivalent.
\end{proof}
In view of this proposition, we think of valuation rings and additive valuations as being two aspects of the same thing, and we may assume that all valuations are \textit{surjective}. In this part, we use this point of view to establish some results for valuation rings.\par
The first non-trivial examples of valuations are the \textbf{$\bm{p}$-adic valuation} $v_p$ on $\Q$ where $p$ is a prime number, and similarly the $p$-adic valuation $v_p$ on the rational function field $k(X)$ where $p$ is any irreducible polynomial from $k[X]$, $k$ being an arbitrary field. In the second case, $v_p$ restricted to $k$ is trivial.\par
There is one more interesting valuation on $k[X]$, trivial on $k$. This is the degree valuation $v_\infty$ defined for non-zero polynomials $f,g\in k[X]$ by
\[v_\infty(f/g)=\deg g-\deg f.\]
One easily checks the axioms of a valuation. Moreover, $f/g$ is a unit in the valuation ring $A_{v_\infty}$ if and only if $\deg f=\deg g$. Thus if
\[f=\sum_{i=0}^{n}a_iX^i\]
with $a_i\in k$ and $a_n\neq 0$, then we see that $f(X)/X^n$ is a unit that maps to $a_n$ in the residue class field. Hence it follows that the residue class field is exactly $k$.
\begin{proposition}
Every non-trivial valuation on $\Q$ is a $p$-adic valuation for some rational prime $p$. Every non-trivial valuation on $k(X)$, trivial on $k$, is either the degree valuation $v_\infty$ or a $p$-adic valuation for some irreducible polynomial $p\in k[X]$.
\end{proposition}
\begin{proof}
Let $K$ be either $\Q$ or $k(X)$ and $v$ some non-trivial valuation on $K$. Then the valuation ring $A$ is different from $K$. In the second case $v$ is assumed to be trivial on $k$. This means that $k\sub A$.\par
First consider the case $K=\Q$. Since $1\in A$, we have $\Z\sub A$. As $A\neq\Q$, at least one prime $p$ must lie in $\m_A$. If $q$ is a prime different from $p$, we have $ap+bq=1$ for some $a,b\in\Z$, whence $q\notin\m_A$. Hence all primes $q\neq p$ are units in $A$. Using the factorization of integers we therefore see that for $a,b\in\Z$, relatively prime, we have $a/b\in A$ iff $p\nmid b$, whence $A=\Z_{(p)}$, i.e., $v$ is equivalent to the $p$-adic valuation $v_p$.\par
Let $K=k(X)$ and $\Z\sub A\neq K$. If $X\in A$, then $k[X]\sub O$ and we can argue as in the case of $\Q$ replacing $\Z$ by $k[X]$. If $X\notin A$, then $X^{-1}\in\m_A$. In this case $v(X)<0$ and $v(X^m)<v(X^n)$ whenever $0<n<m$. Since $v(a)=0$ for every $a\in k^\times$, we get
\[v(a_nX^n+\cdots+a_1X+a_0)=v(a_nX^n)=nv(X)\]
in the case $a_n\neq 0$. Hence the value group of $v$ is $v(X)\Z$. By sending $v(X)$ to $-1$, we therefore get an order-preserving isomorphism with $\Z$, showing that $v$ is equivalent to the degree valuation.
\end{proof}
\begin{definition}
A subset $\Delta$ of a totally ordered group $\Gamma$ is called a \textbf{segment} if $\Delta$ is non-empty and if for any element $\alpha$ of $\Gamma$ which belongs to $\Delta$, all the elements $\beta$ of $\Gamma$ which lie between $\alpha$ and $-\alpha$ also belong to $\Delta$. A subgroup of $\Gamma$ is called an \textbf{isolated subgroup} if it is a segment of $\Gamma$.
\end{definition}
Isolated subgroups occurs naturally in the theory of totally ordered groups due to the following proposition.
\begin{proposition}\label{isolated group iff kernel}
A subgroup $\Delta$ of $\Gamma$ is isolated if and only if it is the kernel of an order homomorphism from $\Gamma$ to a totally ordered group.
\end{proposition}
\begin{proof}
Let $\phi:\Gamma\to\Gamma'$ be an order-homomorphism between to tally ordered groups and $\Delta=\ker\phi$. we show that $\Delta$ is isolated. In fact, if $\alpha\in\Delta$ and $\beta$ lie between $\alpha$ and $-\alpha$, then since $\phi(\alpha)=\varphi(-\alpha)$, we must have $0\leq\phi(\beta)\leq 0$, whence $\beta\in\Delta$.\par
Conversely, let $\Delta$ be a isolated subgroup of $\Gamma$ and $\Gamma/\Delta$ be the quotient group. We define an order on $\Gamma/\Delta$ by
\[\alpha+\Delta\leq\beta+\Delta\iff\alpha\leq\beta.\]
This definition makes sense since for $\alpha>0$ and $\alpha\notin\Delta$ we have $\alpha>\beta$ for all $y\in\Delta$. It is easy to see $\Gamma/\Delta$ becoms a tally ordered group and the quotient map $\pi:\Gamma\to \Gamma/\Delta$ is an order homomorphism with kernel $\Delta$.
\end{proof}
Note that the set of segements of a tally ordered group $\Gamma$ is totally ordered by inclusion. Therefore, we defined the \textbf{rank} of $\Gamma$, denoted by $\rank(\Gamma)$, to be the maximal length of ascending chains in this set (just like the definition of the Krull dimension of a ring). The \textbf{rank of a valuation} is defined to be the rank of the corresponding value group.\par
Using the valuation map, there is a connection between idelas of a valuation ring and isolated subgroups of its value group, which we now explain.
\begin{theorem}\label{isolated subgroup and ideal of valuation ring}
Let $v:K^\times\to\Gamma$ be a valuation and $A$ be its valuation ring. For any subset $S\sub A$, we define
\[\Delta_S=\{\alpha\in\Gamma:\text{$\alpha\notin v(S)$ and $\alpha\notin -v(S)$}\}.\]
\begin{itemize}
\item[(a)] If $I$ is an ideal of $A$ then $\Delta_I$ is a segement in $\Gamma$, and is an isolated subgroup if and only if $I$ is prime.
\item[(b)] The map $I\mapsto\Delta_I$ is an order-reversing bijection from the set of all ideals of $A$ onto the set of all segments of $\Gamma$.
\end{itemize}
\end{theorem}
\begin{proof}
Let $I$ be an ideal of $A$. To show $\Delta_I$ is a segement, we only need to verify that, if $\alpha\notin\Delta_I$ and $\beta>\alpha>0$, then $\beta\notin\Delta_I$. Now by definition, if $\alpha\notin\Delta_I$ and $\alpha>0$, then $\alpha\in v(I)$. Since $\beta>\alpha$, we have $\beta-\alpha\in v(A)$, and hence $\beta=\alpha+\beta-\alpha\in v(A)v(I)\sub v(I)$. Furthermore, from the definition of $\Delta_I$, it is easy to see $\Delta_I$ is a subgroup if and only if $I$ is prime. For (b), we construct an inverse of the map $I\mapsto\Delta_I$. Let $\Delta$ be a segment in $\Gamma$, and define
\[I_\Delta=\{a\in A:v(a)>\alpha\text{ for all $\alpha\in\Delta$}\}.\]
It is easy to see $I_\Delta$ is an ideal of $A$, and we have $\Delta_{I_\Delta}=\Delta$, $I_{\Delta_I}=I$, so the claim is proved.
\end{proof}
\begin{corollary}\label{valuation ring dimension is rank}
The rank of a valuation equals the Krull dimension of its valuation ring.
\end{corollary}
\subsection{Valuations of rank one}
As an important example, we consider the condition under which a valuation has rank $1$. By \cref{valuation ring dimension is rank}, we know this happens if and only if $\dim A_v=1$. Here we provides another characterization.
\begin{proposition}\label{valuation rank 1 char}
Let $v$ be a nontrivial valuation, $\Gamma$ be the value group of $v$, and $A$ the valuation ring. Then the following are equivalent.
\begin{itemize}
\item[(\rmnum{1})] $\rank(v)=1$, or equivalently $\dim A=1$.
\item[(\rmnum{2})] $\Gamma$ is order isomorphic to a subgroup of $\R$.
\item[(\rmnum{3})] $\Gamma$ is \textbf{Archimedean}, that is, for any $\alpha,\beta\in \Gamma$ with $\alpha>0$, there exist $n\in\N$ such that $nx>y$.
\end{itemize} 
\end{proposition}
\begin{proof}
We will show that (\rmnum{3})$\Rightarrow$(\rmnum{2})$\Rightarrow$(\rmnum{1})$\Rightarrow$(\rmnum{3}). First, assume that $\Gamma$ is Archimedean, and we may suppose that $\Gamma\neq\{0\}$. Fix some $0<\alpha\in \Gamma$. For any $\beta\in \Gamma$, there is a well-defined largest integer $n$ such that $n\alpha\leq\beta$ (if $\beta\geq 0$ this is clear by assumption; if $\beta<0$, let $m$ be the smallest integer such that $-\beta\leq m\alpha$, and set $n=-m$), which we will denote by $n_0$. Now set $\beta_1=\beta-n_0\alpha$ and let $n_1$ be the largest integer $n$ such that $n\alpha<10\beta_1$; we have $0\leq n_1<10$. Set $\beta_2=10\beta_1-n_1\alpha$ and let $n_2$ be the largest integer $n$ such that $n\alpha<10\beta_2$. Continuing in the same way, we find a sequence of integers $\{n_i\}$ and set $\phi(\beta)$ to be the real number given by the decimal expression $n_0.n_1n_2\dots$. Then it can easily be checked that $\phi:\Gamma\to\R$ is order-preserving and injective. Now let $\beta,\beta'\in \Gamma$. For $r\in\N$ arbitrary, let $n$ and $n'$ be the interger determined by the inequalities
\[n\alpha\leq 10^r\beta<(n+1)\alpha,\quad n'\alpha\leq 10^r\beta'\leq(n'+1)\alpha.\]
Then we see $(n+n')\alpha\leq 10^r(\beta+\beta')<(n+n'+2)\alpha$. If we use $\phi_r(\beta)$ to denote the number obtained by taking the first $r$ decimal places of $\phi(\beta)$, then $n=\phi_r(\beta)$, $n'=\phi_r(\beta')$, and hence
\begin{align*}
|\phi(\beta+\beta')-\phi(\beta)-\phi(\beta')|&=|\phi(\beta+\beta')-\phi_r(\beta+\beta')|+|\phi_r(\beta+\beta')-\phi_r(\beta)-\phi_r(\beta')|\\
&\,+|\phi_r(\beta)-\phi(\beta)|+|\phi(\beta')-\phi_r(\beta')|\leq 4\cdot 10^{-r}.
\end{align*}
Since $r$ is arbitrary, we see $\phi$ is a homomorphism, hence $\Gamma$ is order isomorphic to a subgroup of $\R$.\par
Now let $\Gamma$ be a nonzero subgroup of $\R$. Since $\Gamma\neq 0$, $A$ is not a field. Suppose that $\p$ is a prime ideal of $A$ distinct from $\m_A$. Let $a\in\m_A-\p$ and set $v(a)=\alpha$. Suppose that $0\neq b\in\p$, and set $\beta=v(b)$; then $\beta\in\Gamma$ and $\alpha>0$, so that $n\alpha>y$ for some sufficiently large natural number $n$. This means $a^n/b\in A$, so that $a^n\in bA\sub\p$; then since $\p$ is prime we have $a\in\p$, which is a contradiction. Thus $\p=(0)$ and the only prime ideals in $A$ are $(0)$ and $\m_A$. That is, $\dim A=1$.\par
Finally, assume that $\dim A=1$. If $0\neq b\in\m_A$ then $\m_A$ is the unique prime ideal containing $b$, and hence $\sqrt{bA}=\m_A$. Thus for any $a\in\m_R$ there exists a natural number $n$ such that $a^n\in bA$. From this one sees easily that $\Gamma$ satisfies the Archimedean axiom.
\end{proof}
\begin{proposition}\label{valuation rank 1 iff completely integrally closed}
Let $K$ be a field, $v$ a nontrivial valuation on $K$ and $A$ the ring of $v$. For $A$ to be completely integrally closed, it is necessary and sufficient that $v$ be of rank $1$.
\end{proposition}
\begin{proof}
Suppose $v$ is of rank $1$. Let $x\in K$ be such that the $A[x]$ are all contained in a finitely generated sub-$A$-module of $K$. There exists $d\in A-\{0\}$ such that $dx^n\in A$ for all $n\geq 0$. Then $v(d)+nv(x)\geq 0$, that is $n(-v(x))\leq v(d)$ for all $n\geq 0$, whence $-v(x)\leq 0$ (since $\Gamma_v$ is Archimedean) and $x\in A$. Thus $A$ is completely integrally closed.\par
Suppose now that $v$ is not of rank $1$. Then there exist $y\in\m_v$ and $t\in A$ such that $nv(y)<v(t)$ for all $n\geq 0$. Then $ty^{-n}\in A$ for all $n\geq 0$, but $y^{-1}\notin A$. Hence $A$ is not completely integrally closed.
\end{proof}
\begin{corollary}\label{valuation rank 1 intersection ring is completely integrally closed}
Let $K$ be a field, $(v_i)_{i\in I}$ a family of valuations of rank $1$ on $K$ and $A$ the intersection of the rings of the $v_i$. Then $A$ is a completely integrally closed domain.
\end{corollary}
\begin{proof}
An intersection of completely integrally closed rings is completely integrally closed, hence the claim.
\end{proof}
\subsection{Discrete valuation rings}
A valuation ring whose value group is isomorphic to $\Z$ is called a \textbf{discrete valuation ring} (DVR). Discrete refers to the fact that the value group is a discrete subgroup of $\R$.
\begin{example}
The two standard examples are:
\begin{itemize}
\item[(a)] $K=\Q$. Take a fixed prime $p$, then any non zero $x\in\Q$ can be written uniquely in the form $p^ay$, where $a\in\Z$ and both numerator and denominator of $y$ are prime to $p$. Define $v_p(x)$ to be $a$. The valuation ring of $v_p$ is the local ring $\Z_{(p)}$.
\item[(b)] $K=k(X)$, where $k$ is a field and $X$ an indeterminate. Take a fixed irreducible polynomial $f\in k[X]$ and define $v_f$ just as in (a). The valuation ring of $v_f$ is then the local ring $k[X]_{(f)}$.
\item[(c)] Let $k$ be a field, and let $k((X))$ be the field of formal Laurent series in one variable over $k$. For every non-zero series
\[f(X)=\sum_{n\geq n_0}a_nX^n,\quad a_{n_0}\neq 0\]
one defines the order $v(f)$ of $f$ to be the integer $n_0$. One obtains thereby a discrete valuation of $k((X))$, whose valuation ring is $k\llbracket X\rrbracket$, the set of formal series; its residue field is $k$.
\end{itemize}
\end{example}
\begin{proposition}\label{DVR for valuation ring iff}
Let $A$ be a valuation ring. Then the following conditions are equivalent.
\begin{itemize}
\item[(a)] $A$ is a DVR.
\item[(b)] $A$ is a PID.
\item[(c)] $A$ is Noetherian.
\item[(d)] The value group $\Gamma$ of $A$ is discrete under order topology and has rank $1$.
\end{itemize}
\end{proposition}
\begin{proof}
Let $K$ be the field of fractions of $A$ and $\m$ its maximal ideal. Since $A$ is a valuation ring, we know finitely generated idels of $A$ are principal, so $A$ is a PID if and only if it is Noetherian. Now, we first show that if $A_v$ is Noetherian then $v$ must has rank $1$. For suppose that $\rank(v)>1$, there must exists a nontrivial isolated subgroup $\Delta$. Fix a positive element $x\in\Delta$, then we get a chain of elements
\[x<2x<\cdots<nx<\cdots.\]
Since $\Delta$ is proper, we can find $y\in G$ which does not belong to $\Delta$. Then since $\Delta$ is isolated, $y$ is bigger than any elements in $\Delta$, and in particular bigger than any $nx$, $n\in\N$. Equivalently, we get a decreasing sequence of positive elements in $G$:
\[y-x>y-2x>\cdots>y-nx>\cdots>0\]
which gives an infinite strictly descending segements in $G$, hence an infinite ascending sequence of ideals in $A_v$ (by \cref{isolated subgroup and ideal of valuation ring}). Hence $R_v$ is not Noetherian.\par
Now since $v$ has rank $1$, $G$ can be identified as a subgroup of $\R$. Moreover, the maximal ideal $\m$ is principal, which means $G$ has a smallest positive element. Then we see $G$ is discrete and hence isomorphic to $\Z$. The converse is obvious, since the group $\Z$ has no infinite descending sequence of positive elements.\par
Finally, it is clear that $\Z$ is discrete and has rank $1$. Conversely, if the value group $G$ of $A$ has rank $1$ then it can be embedded into $\R$, and is isomorphic to $\Z$ if it is discrete. This completes the proof.
\end{proof}
If $A$ is a DVR with maximal ideal $\m$ then an element $t\in R$ such that
$\m=(t)$ is called a \textbf{uniformising element} (or \textbf{prime element}) of $A$.
\begin{example}
A valuation ring $B$ whose maximal ideal $\m_B$ is principal does not have to be a DVR. In fact, the maximal ideal of a valuation ring is principal if and only if its value group $G$ has a smallest positive element, if and only if $G$ is discrete under order topology.\par
To obtain a counter-example, let $K$ be a field, and $A$ a DVR of $K$. If $\bar{B}$ is a DVR for the residue field $k=A/\m_A$, then their composite $B$ will be a valuation ring of $K$ with rank bigger than $2$, hence not a DVR. However, let $s\in B$ be a preimage of a uniformising element $\bar{s}$ of $\bar{B}$. Then $s\notin\m_A$ and we have
\[\m_A\sub\m_B\sub B\sub A,\quad \m_B/\m_A=\bar{s}\bar{B},\]
so $\m_B=\m_A+sB$. On the other hand, since $s\in A-\m_A$, we have $s^{-1}\in A$ and thus $s^{-1}\m_A\sub A\sub B$. This implies $\m_A\sub sB$, so $\m_B=sB$.
\end{example}
The previous theorem gives a characterisation of DVRs among valuation rings; now we consider characterisations among all rings.
\begin{proposition}\label{integral domain DVR iff}
Let $A$ be a ring with fraction field $K$. Then the following conditions are equivalent:
\begin{itemize}
\item[(a)] $A$ is a DVR.
\item[(b)] $A$ is a local PID, and not a field.
\item[(c)] $A$ is a Noetherian local ring with positive dimension and the maximal ideal $\m_A$ is principal.
\item[(d)] $A$ is a one-dimensional normal Noetherian local ring.
\end{itemize}
\end{proposition}
\begin{proof}
If $A$ is a DVR, conditions (b), (c), (d) hold by \cref{DVR for valuation ring iff}, \cref{valuation ring prop} and \cref{integral closure and localization}. Conversely, we now show that either one of these three conditions implies $A$ is a DVR.\par
If $A$ is a local PID and not a field, then $A$ is Noetherian and the maximal ideal $\m$ is nonzero and principal, so (b) implies (c). Now assume (c), and let $\m=(t)$ be the maximal ideal of $A$. Then $t$ can not by nilpotent since otherwise $\m$ is the unique prime ideal of $A$, whence $\dim A=0$. By the Krull intersection theorem we have $\bigcap_{n=1}^{\infty}(t^n)=(0)$, so that for $0\neq x\in A$ there is a well-determined $n$ such that $x\in(t^n)$ but $x\notin(t^{n+1})$. If $x=t^nu$, then since $u\notin(t)$ it must be a unit. Similarly, for $0\neq y\in A$ we have $y=t^mv$, with $v$ a unit. Therefore $yz=t^{m+n}uv\neq 0$, and so $A$ is an integral domain. Finally, any element $\alpha$ of $K$ can be written $\alpha=x^\nu u$, with $u$ a unit of $A$ and $\nu\in\Z$, and it is easy to see that setting $v(t)=\nu$ defines an additive valuation of $K$ whose valuation ring is $A$.\par
Finally, assume that $A$ is a one-dimensional noraml Noetherian local ring. Let $x$ be a nonzero element of $\m$. By hypothesis, $\m$ is the only nonzero prime ideal, so $\sqrt{(x)}=\m$. Then by \cref{Noe ideal contain power of radical} $(x)$ contains a power of $\m$, and we may assume that $(x)\neq\m$. Then there exists $n>1$ such that $\m^n\sub(x)$ but $\m^{n-1}\nsubseteq(x)$. If $y\in\m^{n-1}-(x)$ and $\beta=x/y\in K$, then we have $\beta^{-1}=y/x\notin A$ since $y\notin(x)$. Since $A$ is integrally closed, $\beta^{-1}$ is not integral over $A$. But then $\beta^{-1}\m\nsubseteq\m$, in view of \cref{finite module homomorphism phi(M) sub IM}. Now use $y\in\m^{n-1}$ we have
\[\beta^{-1}\m=(y/x)\m\sub(1/x)\m^n\sub(1/x)(x)=A.\]
Thus $\beta^{-1}\m$ is an ideal of $A$, and it must be $A$ since $\beta^{-1}\m\nsubseteq\m$. Hence $\m$ is the principal ideal $(\beta)$, and by the preceeding argument we can show that $A$ is a DVR.
\end{proof}
To conclude this part, we introduce the concept of rational rank of a totally ordered group $\Gamma$, denoted by $\rank_\Q(\Gamma)$, which is defined to be the maximal number of rationally independent elements of $\Gamma$. In other words,
\[\rank_\Q(\Gamma):=\dim_\Q(\Q\otimes_\Z \Gamma).\]
Note that the rational rank of a group $\Gamma$ is zero if and only if $\Gamma$ is a torsion group. If $\Gamma$ is a value group of a valuation, $\Gamma$ is totally ordered, then its rational rank is zero if and only if $\Gamma=\{0\}$, i.e. if and only if the
valuation is the trivial valuation.
\begin{proposition}\label{ordered group rank and rational rank}
Let $\Gamma$ be an abelian group and $\Delta$ a subgroup of $\Gamma$. Then
\begin{itemize}
\item[(a)] $\rank_\Q(\Gamma)=\rank_\Q(\Gamma/\Delta)+\rank_\Q(\Delta)$.
\item[(b)] If $\Gamma$ is ordered then $\rank(\Gamma)=\rank(\Gamma/\Delta)+\rank(\Delta)$.
\item[(c)] If $\Gamma$ is ordered then $\rank(\Gamma)\leq\rank_\Q(\Gamma)$.
\end{itemize}
\end{proposition}
\begin{proof}
Since $\Q$ is $\Z$-flat (it is torsion free), we see that
\[(\Q\otimes_\Z \Gamma)/(Q\otimes_\Z\Delta)=\Q\otimes_\Z(\Gamma/\Delta)\]
whence the first equality follows. Now it is easy to see the isolated groups in $\Gamma/\Delta$ are in one-to-one correspondence to isolated subgroups in $\Gamma$ containing $\Delta$, so the second equality follows. Finally, assume that $\Gamma$ is ordered and let $\Delta_0\leq\Delta_1\leq\cdots\leq\Delta_{s-1}$ be a stirct chain of isolated subgroups of $\Gamma$. For each $i$, choose an element $\alpha_i\in\Delta_i\setminus\Delta_{i-1}$ (with $\Delta_s=\Gamma$), then we see $\alpha_1,\dots,\alpha_r$ are rationally independent: let $\sum_{i=1}^{t}r_i\alpha_i=0$ with $r_t\neq 0$ and $t\leq s$, then $r_t\alpha_t\in\Delta_{t-1}$, and since $\Delta_{r-1}$ is a segment and $r_t\neq 0$, we find $\alpha_t\in\Gamma_{t-1}$, contradiction. Thus $\rank(\Gamma)\leq\rank_\Q(\Gamma)$.
\end{proof}
\section{Compare valuation rings of a field}
As an immediate observation, if $\p$ is a prime ideal of $A$, then the localized ring $A_\p$ is a local ring containing $A$. Therefore, $A_\p$ is also a valuation of $K$. We have seen in \cref{isolated subgroup and ideal of valuation ring} that such a prime ideal $\p$ corresponds to a proper isolated subgroup $\Delta$ of $G$. Now we determine the value group of $A_\p$.
\begin{proposition}\label{value group of localization}
The value group of $A_\p$ is isomorphic to the quotient group $\Gamma/\Delta$, where $\Delta$ is the isolated subgroup of $\Gamma$ corresponds to $\p$. In this way the valuation $v_\p$ associated to $A_\p$ is given by the composition of $v:K^\times\to\Gamma$ and $\pi:\Gamma\to \Gamma/\Delta$.
\end{proposition}
\begin{proof}
We define $\tilde{v}=\pi\circ v:K^\times\to \Gamma/\Delta$, which is easily seen to be a valuation of $K$. If we embedd $A_\p$ in $K$, then we have $v(A_\p)=\Delta\cup \Gamma_+$. On the other hand, we observe that
\[A_{\tilde{v}}=\{x\in K:\pi(v(x))\geq 0\}=\{x\in K:\text{$v(x)>\Delta$ or $v(x)\in\Delta$}\}=A_\p,\]
so the claim follows from \cref{valuation correspond to valuation ring}.
\end{proof}
With \cref{value group of localization}, we can now classifies all valuation rings of $K$ containing $A$: they are in one-to-one correspondence to prime ideals of $A$.
\begin{proposition}\label{valuation ring superring char}
Let $A$ be a valuation ring and $\m_A$ be the maximal ideal of $A$.
\begin{itemize}
\item[(a)] Any localization $A_\p$ of $A$ at a prime ideal $\p$ of $A$ is a valuation ring of $K$ containing $A$, and $\p$ is the maximal ideal of $A_\p$.
\item[(b)] The map $\p\mapsto A_\p$ is a bijection from the set of prime ideals of $A$ onto the set of rings $B$ with $A\sub B\sub K$, which is orderreversing for the relation of inclusion. The inverse map is defined by $B\mapsto\m_B$.
\end{itemize}
\end{proposition}
\begin{proof}
We have already seen that $A_\p$ is a valuation ring containing $A$. Now, from the definition of the valuation ideal and \cref{value group of localization}, we have
\[\m_{A_\p}=\{x\in K:v_\p(x)>0\}=\{x\in K:\pi(v(x))>0\}=\{x\in K:v(x)>\Delta\}=\p.\]
Therefore $\p$ is the maximal ideal of $A_\p$.\par
Now let $B$ be a ring such that $A\sub B\sub K$. Note that if $x\in\m_B$ then $x^{-1}\notin B\sups A$, and hence $x\in\m_A$. This implies $\m_B\sub\m_A\sub A\sub B$. Also, note that by (a), $\m_B$ is the maximal ideal of both $A_{\m_B}$ and $B$. Since a valuation ring is uniquely determined by its maximal ideal, we get $B=A_{\m_B}$, hence the claim.
\end{proof}
Since $A_\p$ is a valuation ring and $\p$ is its maximal ideal, we can consider its residue field $\kappa_\p=A_\p/\p$. Then since $A$ is a valuation ring, it is easy to see $\bar{A}=A/\p$ is a valuation ring of $\kappa_\p$. Moreover, we can determine the value group of $\bar{A}$.
\begin{proposition}\label{valuation of residue field value group}
Let $\p$ be a prime ideal of $A$ with corresponding isolated subgroup $\Delta$. Then $\bar{A}$ is a valuation ring of $\kappa_v$ with value group $\Delta$.
\end{proposition}
\begin{proof}
Let $\pi:A_\p\to A_\p/\p$ be the quotient map. Then since $A/\p$ is local with maximal ideal $\m_A/\p$, we see $U(A/\p)=\pi(U(A))$, hence there is an isomorphism $\psi$ given by
\[\psi:U(A_\p)/U(A)\to U(A_\p/\p)/U(A/\p),\quad x+U(A)\mapsto\pi(x)+U(A/\p).\]
Also, we note that $U(A_\p)/U(A)\cong\Delta$, with the isomorphism given by
\[\psi:U(A_\p)/U(A)\to\Delta,\quad x+U(A)\mapsto v(x).\]
With these observations, we now define a valuation on $\kappa_v^\times=U(A_\p/\p)$ by
\[\bar{v}:U(A_\p/\p)\to\Delta,\quad \bar{v}(\pi(x))=\psi(\phi^{-1}(\pi(x)+U(A/\p)))=v(x).\]
With this, it is easy to check that
\[A_{\bar{v}}=\{\pi(x):\bar{v}(\pi(x))\geq 0\}=\{\pi(x):v(x)\geq 0\}=A/\p.\]
Thus the claim follows.
\end{proof}
It turns out that all valuation rings contained in a given valuation ring arise in this way: we have the following proposition.
\begin{proposition}\label{valuation ring contained char}
Let $A$ be a valuation ring of $K$ and $\pi:A\to A/\m_A$ be the canonical map.
\begin{itemize}
\item[(a)] Let $B\sub A$ be a valuation ring of $K$, then the image $\bar{B}$ under $\pi_A$ is a valuation ring of $\kappa_v$.
\item[(b)] Let $\bar{B}$ be a valuation ring of $\kappa_v$, then its preimage $B\sub A$ under $\pi_A$ is a valuation ring of $K$. 
\end{itemize}
\end{proposition}
\begin{proof}
Let $B\sub A$ be a valuation ring of $K$ and $\pi(x)\in\kappa_v-\bar{B}$. Then $x\notin B$, and hence $x^{-1}\in B$. Since $\pi(x^{-1})=\pi(x)^{-1}$, we see $\pi(x)^{-1}\in\bar{B}$, hence $\bar{B}$ is a valuation ring of $\kappa_v$.\par
Conversely, let $\bar{B}$ be a valuation ring of $K$ and $B$ be its preimage under $\pi$. Let $x\in K-B$, then $\pi(x)\in\kappa_v-\bar{B}$ and hence $\pi(x^{-1})\in\bar{B}$. Then it follows that $x^{-1}\in B$ so $B$ is a valuation of $K$.
\end{proof}
The formulation of \cref{valuation ring contained char} can be defined on valuations as follows. Let $v:K^\times\to\Gamma$ be a valuation of $K$ with residue field $\kappa_v$, and let $\bar{v}:\kappa_v\to\widebar{\Gamma}$ be a valuation of $\kappa_v$. Then we can define a new valuation $w$ of $K$ by letting its valuation ring to be
\[A_{w}=\{a\in A:\bar{v}(\bar{a})\geq 0\}.\]
The valuation $w$ is called the composite of $v$ and $\bar{v}$, and denoted by $w=v\circ\bar{v}$.\par
Let $v$ and $w$ be two valuations of $K$. We say $w$ dominates $v$, or $w\geq v$, if $A_w\sub A_v$ or equivalently $\m_{w}\sups\m_v$. Then \cref{valuation ring contained char} just says all valuations $w\geq v$ are composite of $v$. By \cref{value group of localization} and \cref{valuation of residue field value group}, the following result is immediate.
\begin{proposition}\label{valuation composite rank}
If $w$ is the composite valuation $v\circ\bar{v}$ we have the equalities
\[\rank(w)=\rank(v)+\rank(\bar{v}),\quad\rank_\Q(w)=\rank_\Q(v)+\rank_\Q(\bar{v}).\]
\end{proposition}
In fact, by \cref{value group of localization} and \cref{valuation of residue field value group}, if we have the valuations $v$ of $K$ and $\bar{v}$ of $\kappa_v$, then the composite valuation $w=v\circ\bar{v}$ defines an extension of the value group $\Gamma$ of $v$ by the value group $\widebar{\Gamma}$ of $\bar{v}$, i.e. an exact sequence of totally ordered groups:
\[\begin{tikzcd}
0\ar[r]&\widebar{\Gamma}\ar[r]&\widetilde{\Gamma}\ar[r]&\Gamma\ar[r]&0
\end{tikzcd}\]
If this exact sequence splits, the value group $\widetilde{\Gamma}$ is isomorphic to the group $\Gamma\times\widebar{\Gamma}$ with the lexicographic order. If the valuation $v$ is a discrete valuation of rank one, i.e. for $\Gamma=\Z$, the exact sequence always splits (this follows from the observation that $\Z$ can be embedded into any nonzero totally ordered group) and we can describe the composite valuation $w=v\circ\bar{v}$ in the following way. The maximal ideal of the valuation ring $A_v$ associated to $v$ is generated by an element $t$ and we can associate to any non zero element $x$ in $K$ the nonzero element $\pi(x t^{-v(x)})$ in the residue field, where $\pi:A_v\to\kappa_v$ is the canonical map. The composite valuation $w$ is then defined by $w(x)=(v(x),\bar{v}(\pi(x t^{-v(x)}))$.
\section{Dependent valuations and induced topology}
With the notation of composite, we say two valuations $v$, $w$ of $K$ are \textbf{dependent} if they are the composite of a common nontrivial valuation. If these do not hold, we say $v$ and $w$ are \textbf{independent}. Equivalently, by \cref{valuation ring contained char} $v$ and $w$ are independent if and only if the valuation rings $A_v$ and $A_w$ are not contained in a proper valuation ring of $K$, if and only if $AB$, the smallest ring containing $A$ and $B$, is a proper subring of $K$.\par
we see from \cref{valuation ring superring char} that the dependence relation is an equivalence relation. Indeed, if $A$ and $B$ are dependent and also $B$ and $C$ are dependent, then $AB$ and $BC$ are both overrings of $B$. Thus there is some inclusion, say $AB\sub BC$. But then $A$ and $B$ are both contained in $BC$. Hence they are dependent.
We shall now study the topology induced by a valuation on a field $K$. Let $v:K^\times\to\Gamma$ be a valuation of $K$. For each $\gamma\in\Gamma$ and each $x\in K$ we define the set
\[U_\gamma(x)=\{y\in K:v(y-x)>\gamma\}.\]
These sets form a basis of open neighborhoods of $x$, due to the following observations:
\begin{itemize}
\item $x\in U_\gamma(x)$,
\item If $y\in U_\gamma(x)\cap U_{\gamma'}(x')$ and $\delta>\max\{\gamma,\gamma'\}$, then for $z\in U_\delta(y)$, we have
\[v(z-x)=v(z-y+y-x)>\gamma,\quad v(z-x')=v(z-y+y-x')>\gamma'\]
whence $U_\delta(y)\sub U_\gamma(x)\cap U_{\gamma'}(x')$.
\end{itemize}
Therefore, the sets $\{U_\gamma(x):x\in K,\gamma\in\Gamma\}$ is a basis for a topology $\mathcal{T}_v$ on $K$, called the \textbf{topology induced by the valuation $v$}. Some properties of the topology $\mathcal{T}_v$ are gathered below.
\begin{proposition}\label{valuation induced topology prop}
Let $v$ be a valuation on $K$ and $\mathcal{T}_v$ the topology induced by $v$.
\begin{itemize}
\item[(a)] The topology $\mathcal{T}_v$ is Hausdorff and totally disconnected, and it is discrete if and only if $v$ is trivial.
\item[(b)] The field operations are continuous with respect to $\mathcal{T}_v$.
\item[(c)] The valuation $v:K^\times\to\Gamma$ is continuous if $\Gamma$ is given the discrete topology.
\item[(d)] The quotient topology induced on the residue field $\kappa_v$ is discrete. 
\end{itemize}
\end{proposition}
\begin{proof}
Let $A$ be the valuation ring of $v$. Let $x,y\in K$ and $x\neq y$. If $\gamma=v(x-y)$, then $\gamma\neq\infty$ and we see $U_\gamma(x)\cap U_\delta(y)=\emp$ if $\gamma,\delta>v(x-y)$. Therefore $\mathcal{T}_v$ is Hausdorff. Also, it is easy to see $\mathcal{T}_v$ is discrete if and only if $v$ is trivial. We next claim that the set $U_\gamma^c(x)=\{y\in K:v(y-x)\leq\gamma\}$ is also open in $K$. To prove this, we first observe that, if $v(z-y)>v(y-x)$ then we have $(z-y)/(y-x)\in\m_A$, whence $1+(z-y)/(y-x)=(z-x)/(y-x)$ is a unit in $A$. That is, we have
\[v(z-y)>v(y-x)\Rightarrow v(z-x)=v(y-x).\]
From this, it follows that $U_{v(y-x)}\sub U_\gamma(x)^c$ if $y\in U_\gamma(x)^c$, so $U_\gamma(x)^c$ is open. Since $U_\gamma(x)$ is a basis for $\mathcal{T}_v$, it follows that $\mathcal{T}_v$ is totally disconnected.\par
Since $v(x+y)\geq\min\{v(x),v(y)\}$, we have $U_\gamma(x)+U_\gamma(y)\sub U_\gamma(x+y)$. Moreover, from
\[ab-xy=(a-x)(b-y)+(a-x)y+(b-y)x\]
it follows that $U_\gamma(x)U_\gamma(y)\sub U_\delta(xy)$, where $\delta=\min\{2\gamma,\gamma+v(x),\gamma+v(y)\}$. These shows addition and multiplication are continuous under $\mathcal{T}_v$.\par
Let $x_0\in K^\times$, if $x\in K^\times$ satisfies $v(x-x_0)>\max\{\gamma+2v(x_0),v(x_0)\}$, then we have $v(x)=v(x_0)$ and
\begin{equation}\label{valuation field inverse map continuous}
\begin{aligned}
v(x^{-1}-x_0^{-1})&=v(x^{-1}(x_0-x)x_0^{-1})=v(x_0-x)-v(x)-v(x_0)\\
&>\gamma+2v(x_0)-2v(x_0)=\gamma
\end{aligned}
\end{equation}
hence the inverse map is also continuous. These proves (b). Also, the single condition $v(x-x_0)>v(x_0)$ implies $v(x)=v(x_0)$, hence the map $v:K^\times\to G$ is continuous if $G$ is given the discrete topology. Finally, note that $\m_v=U_0(0)$ is clopen in $A$ and so the quotient topology is discrete. 
\end{proof}
\begin{theorem}\label{valuation dependent iff topology same}
Two nontrivial valuations $v$ and $w$ of $K$ are dependent if and only if they induce the same topology on $K$.
\end{theorem}
\begin{proof}
Since two dependent valuation rings have a common non-trivial coarsening, to show that they induce the same topology it is enough to consider the particular case $A_v\sub A_w$.\par
Let $v:K^\times\to\Gamma$ be the valuation. Since $A_v\sub A_w$, by \cref{valuation ring superring char} there exists a isolated subgroup $\Delta$ of $\Gamma$ such that $w$ is given by $K^\times\to\Gamma\to\Gamma/\Delta$. Since $w$ is nontrivial, $A_w\neq K$ and $\Delta\neq\Gamma$. Write
\[U_\gamma(0)=\{x\in K:v(x)>\gamma\},\quad U_{\gamma+\Delta}(0)=\{x\in K:w(x)>\gamma+\Delta\},\]
then we see $U_{\gamma+\Delta}(0)\sub U_\gamma(0)$. On the other hand, if $v(x)>2\gamma$ with $0<\gamma\notin\Delta$ then $2\gamma>\gamma+\Delta$, hence $U_{2\gamma}(0)\sub U_{\gamma+\Delta}(0)$, whence $v$ and $w$ induces the same topology on $K$.\par
Conversely, let $\m_v$ and $\m_w$ be the valuation ideals of $v$ and $w$. If $v$ and $w$ induce the same topology on $K$, then $w$ is an open neighbourhood of $0$ in $\mathcal{T}_v$, so there exists $a\in K^\times$ such that $a\m_v\sub\m_w$. As $\m_w$ is the maximal ideal of the valuation ring $A_w$, the set $K-\m_w$ is multiplicatively closed. Thus we can form the ring
\[A=\{x/y\in K:x\in A_v,y\in A_v-\m_w\}=\{x/y\in K:v(x)\geq 0,v(y)\geq 0,w(y)\leq 0\}.\]
Since $A_v\sub A$, $A$ is also a valuation ring. Moreover, $A$ contins $A_w$ since if $x\in A_w-A_v$ then $w(x)\geq 0$ and we have $x=1/x^{-1}\in A$. Finally, $A\neq K$: let $z\in A_v\setminus\{0\}$, then $\frac{1}{az}\notin A$ since otherwise $1/(az)=x/y$ with $x\in A_v$ and $y\in A_v-\m_w$, and we have $y=azx\in a\m_v\sub\m_w$, a contradiction. Hence $v$ and $w$ are dependent.
\end{proof}
\cref{valuation dependent iff topology same} is another way to see that the dependence relation among the valuation rings of a field $K$ is an equivalence relation, as we already saw above. Let $A$ be a non-trivial valuation ring of $K$ and take the dependence class $[A]$ of all non-trivial valuation rings of $K$ dependent on $A$. Clearly $[A]$ is an upwardly directed set with respect to the partial order of inclusion. 
\begin{proposition}\label{valuation ring dependence class prop}
Let $A$ be any non-trivial valuation ring of $K$. Then we have the following case distinction:
\begin{itemize}
\item[(a)] $[A]$ has a maximal valuation ring $B_0$ which is a maximal non-trivial overring of $A$; moreover $B_0$ has dimension $1$ and its maximal ideal is the intersection of all non-zero prime ideals of $A$, or
\item[(b)] there is no maximal non-trivial overring of $A$. Then the maximal ideals $\m_B$ of valuation rings $B\in[A]$ form a neighborhood system of $0$ for the topology induced by $A$. In this case the set of all non-zero prime ideals of $A$ is also a neighborhood system of $0$.
\end{itemize}
\end{proposition}
\begin{proof}
If $[A]$ has a maximal element $B_0$, then it is clear that $B_0$ contains $A$ and is the maximal overring of $A$. Thus the first part follows from \cref{valuation ring superring char}. Now assume that $[A]$ has no maximal element. Suppose we are given a positive $\delta\in\Gamma$. We seek a valuation ring $B\sups A$ such that $B\neq K$ and whose maximal ideal $\m_B$ satisfies $\m_B\sub U_\delta(0)$. Let $\Delta$ be the convex hull of the subgroup generated by $\delta$ in $\Gamma$, i.e.,
\[\Delta=\{\gamma\in\Gamma:\gamma,-\gamma<n\delta\text{ for some $n\in\N$}\}.\]
Then $\Delta$ defines a valuation ring $B\sups A$ with $\m_B\sub U_\delta(0)$. It remains to show that $B\neq K$.\par
If $B=K$, then $\Delta=\Gamma$ and thus $\delta$ is an element of $\Gamma$ such that $\Gamma$ is the convex hull of $\delta\Z$. Let $\Delta^*$ be the largest isolated subgroup of $\Gamma$ not containing $\delta$. Then $\Gamma^*=\Gamma/\Delta^*$ is archimedean ordered and thus the overring $A^*$ of $A$ corresponding to $\Delta^*$ is maximal according to \cref{valuation ring superring char}. This contradicts our assumption.
\end{proof}
We are now in a position to prove an approximation theorem for independent valuations. For this we need some preparations.
\begin{lemma}\label{valuation ring intersection lemma}
Let $v_1,\dots,v_n$ be valuations on the field $K$ and $x\in K^\times$. Then there exists a polynomial $f(X)$ of the form
\begin{align}\label{valuation ring intersection lemma-1}
f(X)=1+n_1X+\cdots+n_{k-1}X^{k-1}+X^k,\quad n_i\in\Z,k\geq 2
\end{align}
such that $f(x)\neq 0$ and $z=f(x)^{-1}$ satisfies
\begin{alignat*}{2}
&v_i(z)=0\quad&\text{ if }v_i(x)\geq 0,\\
&v_i(z)+v_i(x)>0\quad&\text{ if }v_i(x)<0.
\end{alignat*}
\end{lemma}
\begin{proof}
Let $I$ be the set of indices $i$ such that $v_i(x)\geq 0$. For all $i\in I$, let $x_i$ denote the canonical image of $x$ in $\kappa_{A_i}$. For all $i\in I$ we construct a polynomial $f_i$ as follows: if there exists a polynomial $g(X)$ of the form (\ref{valuation ring intersection lemma-1}) such that $g(\bar{x}_i)=0$ in $\kappa_{A_i}$, we take $f_i$ to be such a polynomial; otherwise we take $f_i=1$. Then we write $f(X)=1+X^2\prod_{i\in I}f_i(X)$. It is obviously a polynomial of the form (\ref{valuation ring intersection lemma-1}). If $i\in I$, then $f(x)\in A_i$ and also $f(\bar{x}_i)\neq 0$ by construction; hence $f(x)\notin\m_i$, $v_i(f(x))=0$ and $v_i(z)=0$. If $i\notin I$, then $v_i(x)<0$, whence $v_i(f(x))=kv_i(x)$ (by \cref{valuation addition prop}) and
\[v_i(x)+v(z)=(1-k)v_i(x)>0,\]
so the lemma is proved.
\end{proof}
\begin{proposition}\label{valuation ring intersection localization prop}
Let $A_1,\dots,A_n$ be valuation rings of a field $K$, $B=\bigcap_{i=1}^{n}A_i$ and $\p_i=\m_{A_i}\cap B$. Then $A_i=B_{\p_i}$ and the fraction field of $B$ is $K$.
\end{proposition}
\begin{proof}
Clearly $B_{\p_i}\sub A_i$ since $B\sub A_i$ and $B-\p_i\sub A-\m_i$ is a unit in $A_i$. Now let $x$ be a nonzero element in $A_i$. We apply the lemma to $x$ and valuations $v_i$ associated with the $A_i$. Then $v_i(z)\geq 0$ and $v_i(zx)\geq 0$ for all $i$, hence $z,zx\in B$. As $v_i(x)\geq 0$ and $v_i(z)=0$, we see $z\notin\p_i$. Hence $x=xz/z\in B_{\p_i}$. The field of fractions of $B$ then contains $A_i$ and hence is $K$.
\end{proof}
\begin{proposition}\label{valuation ring intersection maximal ideal char}
With the hypotheses of \cref{valuation ring intersection localization prop}, suppose further that $A_i\nsubseteq A_j$ for $i\neq j$. Then the $\p_i$'s are distinct maximal ideals of $B$ and every maximal ideal of $B$ is equal to one of the $\p_i$.
\end{proposition}
\begin{proof}
If $\p_i\sub\p_j$ for $i\neq j$ then $A_i=B_{\p_i}\sup B_{\p_j}=A_j$, which is a contradiction. Then the claim follows from \cref{local ring if intersection of localization}. 
\end{proof}
\begin{corollary}\label{valuation ring incomparable Chinese remainder thm}
Suppose that $A_1,\dots,A_n$ are incomparable valuation rings. For every family of element $a_1\in A_1,\dots,a_n\in A_n$, there exists $x\in B$ such that $x\equiv a_i$ mod $\m_i$ for each $i$.
\end{corollary}
\begin{proof}
Since the $\p_i$ are maximal ideals of $B$, $A/\m_A=B_{\p_i}/\p_i B_{\p_i}=B/\p_i$ and it may therefore be assumed that $a_i\in B$ for all $i$. The corollary then follows from the fact that the canonical map from $B$ to $\prod_{i=1}^{n}(B/\p_i)$ is surjective.
\end{proof}
\begin{corollary}\label{valuation ring incomparable separate prop}
Suppose that $A_1,\dots,A_n$ are incomparable valuation rings. There exist element $x_1,\dots,x_n$ of $K$ such that $v_i(x_i)=0$ and $v_j(x_i)>0$ for $j\neq i$.
\end{corollary}
\begin{proof}
For each index $i$ apply \cref{valuation ring incomparable Chinese remainder thm} to the family $(a_i)$ such that $a_i=1$ and $a_j=0$ for $j\neq i$.
\end{proof}
\begin{corollary}\label{valuation ring containing intersection of valuation ring}
Every valuation ring of $K$ containing $B$ contains one of the $A_i$.
\end{corollary}
\begin{proof}
We may confine our attention to the case where $A_i$'s are incomparable. Let $A$ be a valuation ring of $K$ containing $B$. We write $\p=\m_A\cap B$. There exists a maximal ideal $\p_i$ of $B$ containing $\p$, whence $A_i=B_{\p_i}\sub B_\p\sub A$.
\end{proof}
Now for independent valuations, we have the following approximation theorem.
\begin{theorem}[\textbf{Approximation Theorem}]\label{valuation approximation thm}
Suppose $A_1,\dots,A_n$ are pairwise independent valuation rings of $K$. For every $i$, let $v_i:K^\times\to\Gamma_i$ be the valuation corresponding to $A_i$. Then for any $a_1,\dots,a_n\in K$ and $\gamma_1\in\Gamma_1,\dots,\gamma_n\in\Gamma_n$, there exists an $x\in K$ with
\[v_i(x-a_i)>\gamma_i\quad\text{ for all $i\in\{1,\dots,n\}$}.\]
\end{theorem}
\begin{proof}
Let $A_i$ be the valuation ring of $v_i$ and set $B=\bigcap_{i=1}^{n}A_i$, $\p_i=\m_{A_i}\cap B$. By \cref{valuation ring intersection localization prop} we have $A_i=B_{\p_i}$ and $K$ is the fraction field of $B$, so the $a_i$ may be written as $a_i=b_i/s$ with $b_i\in B$ and $s\in B\setminus\{0\}$. If we write $x=y/s$ and $\gamma_i'=\gamma_i+v(s)$, then 
\[v_i(x-a_i)=v_i(y-b_i)-v(s)>\gamma_i\iff v_i(y-b_i)>\gamma_i'.\]
This shows that we may assume that $a_i\in B$ for all $i$; we may also assume that $\gamma_i>0$ for all $i$. Define ideals of $A_i$ and $B$ by
\[I_i=\{z\in K:v_i(z)\geq\gamma_i\}\sub\m_i,\quad J_i=I_i\cap B\sub\p_i.\]
For $x\in B$, $v_i(x-a_i)\geq\gamma_i$ is equivalent to $x\equiv a_i$ mod $I_i$. We therefore need to show that the canonical homomorphism $B\to\prod_{i=1}^{n}(B/J_i)$ is surjective, that is, $J_i$'s are pairwise coprime. As the maximal ideals of $B$ are the $\p_i$, it will suffice for this to show that $J_i\nsubseteq\p_j$ for $i\neq j$.\par
Suppose that there exists $i,j$ such that $J_i\sub\p_j$ and $i\neq j$. We shall see shortly that the radical of $J_i$ is a prime ideal $\p$ of $B$, so that $\p\sub\p_j$. Since $\gamma_i>0$ we also have $J_i\sub\p_i$, so $A_j=B_{\p_j}\sub B_\p$ and similarly $A_i\sub B_\p$. Now, as $I_i\neq (0)$ and $I_i=B_{\p_i}J_i$ (\cref{localization and ideals}), $J_i\neq(0)$ whence $\p\neq(0)$ and $B_\p\neq K$. This contradicts the hypothesis that $A_i$ and $A_j$ are independent.\par
It remains to show that $\p=\sqrt{J_i}$ is prime for each $i$. Suppose that $xy\in\p$, so that $x^ny^n\in J_i$ for some $i$. Then, if for example $v(x)\geq v(y)$, we have
\[v(x^{2n})=2nv(x)\geq nv(x)+nv(y)=v(x^ny^n)\geq\gamma_i\] 
whence $x^{2n}\in J_i$ and $x\in\p$. This finishes the proof.
\end{proof}
\begin{corollary}\label{valuation approximation coro}
For every family of elements $\gamma_1\in\Gamma_1,\dots,\gamma_n\in\Gamma_n$, there exists $x\in K$ such that $v_i(x)=\gamma_i$.
\end{corollary}
\begin{proof}
We may assume that $A_i\neq K$ for all $i$. Then, there exists for all $i$ an $a_i\in K$ such that $v_i(a_i)=\gamma_i$, and an $\alpha_i\in\Gamma_i$ such that $\gamma_i<\alpha_i$. We apply \cref{valuation approximation thm} to these elements $a_i$ there exists a $x\in K$ such that $v_i(x-a_i)>\gamma_i=v_i(a_i)$, whence $v_i(x)=v_i(a_i)=\gamma_i$.
\end{proof}
Here we introduce the concept of absolute values over a field, which has a deep connection with valuations of rank $1$. Let $K$ be a field. An \textbf{absolute value} on $K$ is a map $|\cdot|:K\to\R$ satisfying the following axioms for all $x,y\in K$,
\begin{itemize}
\item[(a)] $|x|>0$ for all $x\neq 0$ and $|0|=0$.
\item[(b)] $|xy|=|x||y|$.
\item[(c)] $|x+y|\leq|x|+|y|$. 
\end{itemize}
The absolute value sending all $x\neq 0$ to $1$ is called the \textbf{trivial absolute value} on $K$. Note that by (b) we have $|1|=1$, $|-x|=|x|$, and $|x^{-1}|=|x|^{-1}$, thus $|\cdot|$ is a group homomorphism from $K^\times$ to $\R^\times$.
\begin{proposition}\label{absolute value ultrametric inequality iff}
The set $\{|n\cdot 1|:n\in\Z\}$ is bounded if and only if $|\cdot|$ satisfies the \textbf{ultrametric inequality}
\[|x+y|\leq\max\{|x|,|y|\}\]
for all $x,y\in K$.
\end{proposition}
\begin{proof}
If $|\cdot|$ satisfies the inequality, then by induction, the set $\{|n\cdot 1|:n\in\Z\}$ is bounded by $1$. Conversely, let $|n\cdot 1|\leq C$. Then
\[|x+y|^n=|(x+y)^n|\leq\sum_{\nu}\Big|\binom{n}{\nu}x^\nu y^{n-\nu}\Big|\leq(n+1)C\max\{|x|,|y|\}^n.\]
Taking $n$-th roots and letting $n$ go to infinity proves the assertion of the proposition.
\end{proof}
If an absolute value satisfies the ultrametric inequality, it is called \textbf{non-archimedean}; otherwise it is called \textbf{archimedean}. Clearly, if $\char K\neq 0$, $K$ cannot carry any archimedean absolute value. A typical example of an archimedean absolute value is the ordinary absolute value on $\R$, which is given by $|x|_0=x$ if $x\geq 0$ and $-x$ if $x<0$. Note that if $v:K^\times\to\R$ is a rank $1$ valuation on $K$ then the absolute value defined by
\[|x|=e^{-v(x)},\quad x\in K\]
is non-archimedean. Conversely, if $|\cdot|$ is a non-archimedean absolute value then $v(x):=-\ln|x|$ is a valuation on $K$. Thus we see non-archimedean absolute values corresponds to rank $1$ valuations.\par
An absolute value $|\cdot|$ on $K$ defines a metric by taking $|x-y|$ as distance, for $x,y\in K$. In particular, $|\cdot|$ induces a topology on $K$. If two absolute values induce the same topology on $K$, they are called dependent (otherwise independent).
\begin{proposition}\label{absolute value independent iff}
Let $|\cdot|_1$ and $|\cdot|_2$ be two non-trivial absolute values on $K$.
They are dependent if and only if for all $x\in K$,
\[|x|_1<1\Rightarrow|x|_2<1.\]
If they are dependent, then there exists a real number $\lambda>0$ such that $|x|_1=|x|_2^\lambda$ for all $x\in K$.
\end{proposition}
\begin{proof}
For $|\cdot|_1$ and $|\cdot|_2$ non-trivial and dependent absolute values on $K$ there exists $\eps>0$ such that $\{x\in K:|x|_1<\eps\}\sub\{x\in K:|x|_2<1\}$. If $|x|_1<1$, there is $m\geq 1$ such that $(|x|_1)^m=|x^m|<\eps$. Hence $(|x|_2)^m=|x^m|_2<1$ and consequently $|x|_2<1$, as required.\par
Conversely, by the non-triviality of $|\cdot|_1$, there exists $z\in K$ with $|z|_2>1$. Thus $|z^{-1}|_1<1$ and so $|z^{-1}|_1<1$, by assumption. Hence $|z|_2>1$, too. Now we claim that, for every $x\in K$, $x\neq 0$, we have
\[\frac{\log|x|_1}{\log|x|_2}=\frac{\log|z|_1}{\log|z|_2}.\]
In fact, for $m,n\in\Z$, $n>0$, such that
\[\frac{m}{n}>\frac{\log|x|_1}{\log|z|_1},\]
it follows that $(|z|_1)^m>(|x|_1)^n$. Consequently, $|x^nz^{-m}|_1<1$ and then, by assumption, $|x^nz^{-m}|_2<1$. Walking back the steps of the last argument one gets
\[\frac{m}{n}>\frac{\log|x|_2}{\log|z|_2}.\]
It then follows that
\[\frac{\log|x|_1}{\log|z|_1}\geq\frac{\log|x|_2}{\log|z|_2}.\]
Similarly one proves the reverse inequality. So the claim follows.\par
Now set $\lambda=\log|z|_1/\log|z|_2$, it follows that $|x|_1=|x|_2^\lambda$, for every $x\in K$, as required. Finally, the last equation implies that $|\cdot|_1$ and $|\cdot|_2$ are dependent.
\end{proof}
Now we prove an approximation theorem for absolute values.
\begin{theorem}
Let $K$ be a field and $|\cdot|_1,\dots,|\cdot|_n$ be non-trivial pairwise-independent absolute values on $K$. Moreover let $x_1,\dots,x_n\in K$, and $0<\eps\in\R$. Then there exists $x\in K$ such that $|x-x_i|<\eps$ for all $i$.
\end{theorem}
\begin{proof}
We shall first prove that for every $1\leq i\leq n$ there exists $a_i\in K$ such that $|a_i|_i>1$ and $|a_i|_j<1$, for all $j\neq i$. We may fix $i=1$ without loss of generality, and write $a=a_1$. We proceed by induction on $n$. For $n=2$, \cref{absolute value independent iff} implies the existence of $b,c\in K$ such that
\[|b|_1<1,|b|_2\geq 1,\quad |c|_1\geq 1,|c|_2<1.\]
Thus $a=b^{-1}c$ has the desired properties.\par
Assume next that there is $y\in K$ such that $|y|_1>1$ and $|y|_j<1$ for all $j=2,\dots,n-1$. Applying the first case to $|\cdot|_1$ and $|\cdot|_n$, one has $|z|_1>1$ and $|z|_n<1$, for some $z\in K$. Therefore, if $|y|_n\leq 1$, then $|zy^\nu|_1>1$ and $|zy^\nu|_n<1$ for every integer $\nu\geq 1$. On the other hand, for a sufficiently large integer $\nu\geq 1$, $|zy^\nu|_j<1$ for every $j=2,\dots,n-1$. For such a $\nu$, $a=zy^\nu$ satisfies the requirements.\par
Consider now the case $|y|_n>1$, and form the sequence
\[w_\nu=\frac{y^\nu}{1+y^\nu}\]
The usual properties of sequences of ordinary real numbers imply that $\lim_\nu|w_\nu|_j=0$ for $j=2,\dots,n-1$ and $\lim_\nu|w_\nu-1|_j=0$ for $j=1$ and $n$. Consequently, $\lim_\nu|zw_\nu|_j=0$ for $j=2,\dots,n-1$ and $\lim_\nu|zw_\nu|_j=|z|_j$ for $j=1$ and $n$. Hence, for sufficiently large $\nu$, $a=zw_\nu$ has the required properties.\par
Now we prove that for any real number $\eps>0$ and every $i$ such that $1\leq i\leq n$, there exists $c_i\in K$ such that $|c_i-1|_i<\eps$ and $|c_i|_j<\eps$, for all $j\neq 1$. As before, it is enough to consider the case $i=1$. Let $a\in K$ satisfy $|a|_1>1$ and $|a|_j<1$ for $j>1$. Then the sequence $|a^\nu/(1+a^\nu)|_j$ converges to $1$ for $j=1$, and converges to $0$ if $j>1$. Thus for sufficiently large $\nu$,
\[c_1=\frac{a^\nu}{1+a^\nu}\]
has the required property.\par
According to the argument above, there exist elements $c_1,\dots,c_n$ in $K$ such that $c_i$ is close to $1$ at $|\cdot|_i$, and for every $j\neq i$, $c_i$ is close to $0$ at $|\cdot|_j$. The element $x=c_1x_1+\cdots+c_nx_n$ is then arbitrarily close to $x_i$ at $|\cdot|_i$, for every $i=1,\dots,n$, and therefore satisfies the requirements of the theorem.
\end{proof}
We now prove some result about topological vector spaces over a valued field, which will be used later.
\begin{proposition}\label{valued field TVS dim=1 is isomorphic}
Let $(K,v)$ be a valued field with valued group $\Gamma$. Let $X$ be a topological vector space over $K$ which is Hausdorff and of dimension $1$. Suppose that $v$ is not trivial. For all $x_0\in X$, the map $a\mapsto ax_0$ of $K$ onto $X$ is a topological isomorphism.
\end{proposition}
\begin{proof}
This map is a continuous algebraic isomorphism. It is sufficient to show that it is bicontinuous. Let $\alpha\in\Gamma$. We need to show that there exists a neighbourhood $V$ of $0$ in $X$ such that the relation $ax_0\in V$ implies $v(a)>\alpha$. Let $a_0\in K^\times$ be such that $v(a_0)=\alpha$. As $X$ is Hausdorff, there exists a neighbourhood $W$ of $0$ in $X$ such that $a_0x_0\notin W$. Since $v$ is not improper, there exist a neighbourhood $W'$ of $0$ in $X$ and an element $\beta$ of $\Gamma$ such that the relations $y\in W'$ and $v(a)\geq\beta$ imply $ay\in W'$. Let $a_1\in K^\times$ be such that $v(a_1)=-\beta$. The relations $ax_0\in a_1^{-1}W'$ and $v(a)\leq\alpha$ imply $a_1ax_0\in W'$ and
\[v(a_0a^{-1}a_1^{-1})=v(a_0)-v(a)-v(a_1)=\alpha+\beta-v(a)\geq\beta\]
hence $a_0x_0=(a_0a^{-1}a_1^{-1})a_1ax_0\in W$, which is a contradiction. In other words, the relation $ax_0\in a_1^{-1}W'$ implies $v(a)>\alpha$, so we are done.
\end{proof}
\begin{proposition}\label{valuation TVS hyperplane is topologically complemented}
Suppose that $v$ is not trivial. If $M$ is a closed maximal subspace of a topological vector space $X$ over $K$ then any algebraic complement $N$ of $M$ is a topological complement.
\end{proposition}
\begin{proof}
Let $M$ be a closed maximal subspace of a topological vector space $X$. Since $M$ is maximal, $\dim X/M=1$; hence if $N$ is an algebraic complement of $M$, $\dim N=1$. Therefore there must be some $x\neq 0$ such that $N=Kx$. Consequently $X=M\oplus Kx$ and each vector $y$ has a unique representation in the form $y=m+tx$, $m\in M$, $t\in K$. To show that $Kx$ is a topological complement of $M$, we use the criterion of \cref{topological group complemented iff}: We show that the projection $P$ on $Kx$ along $M$, $tx+m\mapsto tx$, is continuous.\par
To this end note that $N(P)=M$. As $M$ is closed and $x\notin M$, there exists a neighborhood $U$ of $0$ in $X$ such that $(x+U)\cap M=\emp$ and $xU\sub U$ whenever $v(x)\geq-\gamma_0$, for some fixed $\gamma_0\in\Gamma$. Now if $m+tx\in U$ then we must have $v(t)>\gamma_0$: otherwise we have $v(t)\leq\gamma_0$ and so $v(t^{-1})\geq-\gamma_0$, thus $t^{-1}(m+tx)=m/t+x\in U$, which contradicts $(x+U)\cap M=\emp$. Hence if $\gamma>\gamma_0$ and $m+tx\in U_\gamma(0)U$, then $t>\gamma+\gamma_0$.\par
To establish the continuity of $P$ at $0$, suppose that the net $m_\alpha+t_\alpha x\to 0$. As such, for any $\gamma>0$, $m_\alpha+t_\alpha x\in U_\gamma(0)U$ eventually. Therefore $t_\alpha>\gamma+\gamma_0$ eventually. In other words, $t_\alpha\to 0$ in $K$, which implies that $t_\alpha x=P(m_\alpha+t_\alpha x)\to 0$, and proves the continuity of $P$.
\end{proof}
\begin{proposition}\label{valued field TVS dim=n is isomorphic}
Suppose that $v$ is nontrivial and $K$ is complete. Let $X$ be a topological vector space over $K$, which is Hausdorff and of finite dimension $n$. For every basis $(e_i)_{i=1}^{n}$ of $X$ over $K$, the map $(a_i)\mapsto\sum_{i=1}^{n}a_ie_i$ from $K^n$ onto $X$ is a topological vector space isomorphism.
\end{proposition}
\begin{proof}
This follows from induction, using \cref{valued field TVS dim=1 is isomorphic} and \cref{valuation TVS hyperplane is topologically complemented}.
\end{proof}
\begin{corollary}\label{valued field TVS finite dim subspace closed}
Suppose that $v$ is nontrivial and $K$ is complete. Let $X$ be a Hausdorff topological vector space over $K$ and $F$ a finite-dimensional vector subspace of $X$. Then $F$ is closed.
\end{corollary}
\section{Completion of valued fields}
Since a non-trivial absolute value $v$ makes $K$ into a topological field, we may consider the completion of $K$ with respect to the additive uniform structure. The next theorem will show that every field $K$ with a non-trivial valuation can be densely embedded into a field complete with respect to a valuation extending the given one on $K$.
\begin{theorem}\label{valuation completion prop}
Let $K$ be a field, $v$ a valuation on $K$ and $\Gamma$ the value group of $v$ with the discrete topology.
\begin{itemize}
\item[(a)] The completion ring $\widehat{K}$ of $K$ (with respect to $\mathcal{T}_v$) is a topological field.
\item[(b)] The map $v:K^\times\to\Gamma$ can be extended uniquely to a continuous map $\hat{v}:\widehat{K}^\times\to\Gamma$. The map $\hat{v}$ is a valuation on $\widehat{K}$ with value group $\Gamma$.
\item[(c)] The topology on $\widehat{K}$ is the topology defined by the valuation $\hat{v}$.
\item[(d)] If $U_\gamma(x), V_\gamma(x)$ and $\widehat{U}_\gamma(x),\widehat{V}_\gamma(x)$ are the basic neighborhood of $x$ in $(K,v)$ and $(\widehat{K},\hat{v})$, then $\widehat{U}_\gamma(x)$ is the closure of $U_\gamma(x)$ in $\widehat{K}$ and $\widehat{V}_\gamma(x)$ is the closure of $V_\gamma(x)$ in $\widehat{K}$.
\item[(e)] The valuation ring of $\hat{v}$ is the completion $\widehat{A}$ of the valuation ring $A$ of $v$; the maximal ideal $\widehat{\m}$ is the completion of the maximal ideal $\m$ of $v$.
\item[(f)] $\widehat{A}=A+\widehat{\m}$ and the residue field of $\hat{v}$ is canonically identified with that of $v$. 
\end{itemize}
\end{theorem}
\begin{proof}
By \cref{topological division ring completion iff}, to prove (a), it suffices to show the following: let $\mathfrak{U}$ be a Cauchy filter (with respect to the additive uniform structure) on $K^\times$ for which $0$ is not a cluster point; then the image of $\mathfrak{U}$ under the bijection $x\mapsto x^{-1}$ is a Cauchy filter (with respect to the additive uniform structure). For since $0$ is not a clustcr point of $\mathfrak{U}$, there exists $V\in\mathfrak{U}$ and $\gamma\in\Gamma$ such that $\gamma$ is an upper bound of $v(M)$. Let $\alpha\in\Gamma$. If $V'$ is an element of $\mathfrak{U}$ contained in $M$ and such that $v(x-y)>\max\{\alpha+2\gamma,\gamma\}$ for $x,y\in V'$, then $v(x^{-1}-y^{-1})>\gamma$ for $x,y\in V$ (see (\ref{valuation field inverse map continuous})). Whence (a) follows.\par
By \cref{valuation induced topology prop}, $v:K^\times\to\Gamma$ is a continuous homomorphism from $K^\times$ to $\Gamma$ and hence can be extended uniquely to a continuous homomorphism $\hat{v}$ from $\widehat{K}^\times$ to $\Gamma$. The relation $\hat{v}(x+y)\geq\min\{\hat{v}(x),\hat{v}(y)\}$ holds in $K^\times$ and hence also holds in $\widehat{K}^\times$ by continuity. Thus $\hat{v}$ (extended by $\widehat{K}$ by $\hat{v}(0)=\infty$) is a valuation on $\widehat{K}$ and (b) is proved.\par
Wc now show (d). Let $\gamma\in\Gamma$ and $x\in\widebar{U}_\alpha(0)$. For $y$ in $U_\alpha(0)$ sufficiently close to $x$, $v(y)=\hat{v}(y)=\hat{v}(x)$ and hence $\hat{v}(x)>\gamma$. Conversely, let $x\in\widehat{K}^\times$ be such that $\hat{v}(x)>\alpha$; for $y$ in $K^\times$ sufficiently close to $x$, $v(y)=\hat{v}(y)=\hat{v}(x)$ and therefore $y\in U_\gamma(0)$, whence $x\in\widebar{U}_\gamma(0)$. Thus $\widebar{U}_\gamma(0)$ is the set of $x\in\widehat{K}$ such that $\hat{v}(x)>\gamma$. The general case for $U_\gamma(x)$ now follows from homogeneity, and the result for $V_\gamma(x)$ can be proved similarly.\par
Taking account of \cref{topological group base of identity in completion is closure}, assertion (c) is a consequence of (d). Assertion (e) is a special case of (d). Finally let $x\in\widehat{A}$; since $\widehat{A}$ is the closure of $A$ in $\widehat{K}$, there exists $y\in A$ such that $\hat{v}(x-y)>0$; then $z=x-y\in\widehat{\m}$ and hence $x=y+z\in A+\widehat{\m}$; thus $\widehat{A}=A+\widehat{\m}$ which shows (f).
\end{proof}
\subsection{Archimedean complete fields}
Let $K$ be a field complete with respect to an archimedean absolute value $|\cdot|$. Since the set $\{|n\cdot 1|:n\in\Z\}$ is not bounded, $\char K=0$. Thus $K$ contains the field $\Q$ of rationals. We shall first show that $|\cdot|$ restricted to $\Q$ is dependent on the usual absolute value of $\Q$. Thus the complete field $K$ contains the completion of $\Q$ with respect to the ordinary absolute value, i.e., $K$ contains $\R$ as a closed subfield. We shall then show that $K$ must be equal to $\R$ or to $\C$. Consequently, every field $K$ admitting an archimedean absolute value may be considered as a subfield of $\C$ or even $\R$ with the absolute value dependent on the induced one from $\C$ (or from $\R$).
\begin{proposition}\label{absolute value archimedean iff usual}
Every archimedean absolute value on $\Q$ is dependent on the usual one.
\end{proposition}
\begin{proof}
Let $|\cdot|$ be an archimedean absolute value on $\Q$. Denote by $|\cdot|_0$ the usual absolute value on $\Q$. Next, for integers $m,n\geq 2$ and $t\geq 1$, expand $m^t$ in powers of $n$:
\[m^t=c_0+c_1n+\cdots+c_sn^s,\quad 0\leq c_0,\dots,c_s<n,c_s\neq 0.\]
Since each $c_i$ is integer, we have $|c_i|\leq c_i<n$. It follows that
\[|m|^t\leq\sum_{i=0}^{s}|c_i||n|^i\leq n\sum_{i=0}^{s}|n|^i\leq n(s+1)\max\{1,|n|^s\}.\]
As $n^s\leq m^t$, $s\leq t(\log m)/\log n$. Thus
\[|m|^t\leq n\Big(\frac{t\log m}{\log n}+1\Big)\max\{1,|n|\}^{t(\log m/\log n)}\]
or equivalently,
\[|m|\leq n^{1/t}\Big(\frac{t\log m}{\log n}+1\Big)^{1/t}\max\{1,|n|\}^{\log m/\log n}.\]
Letting $t$ go to infinity and taking limits, one gets
\[|m|\leq\max\{1,|n|\}^{\log m/\log n}.\]
Now, if $|n|<1$ for some $n\in\N$, the above inequality implies $|m|<1$ for every integer $m\geq 2$, contradicting the archimedeanness of $|\cdot|$. Therefore, $|n|>1$ for all integers $n\geq 2$, and thus
\[|m|\leq|n|^{\log m/\log n}.\]
Interchanging the roles of $m$ and $n$ in the above inequality gives the reverse inequality. Hence
\[|m|=|m|^{\log m/\log n}\]
Therefore, if $m>n\geq 2$, then $\log m/\log n>1$ and so $|m|>|n|$. Since $|-m|=|m|$ for all $m\in\Z$, it follows that $|m|_0>|n|_0$ implies $|m|>|n|$, for non-zero $m,n\in\Z$. Consequently, if $m/n\in\Q$ satisfies $|m/n|_0<1$, then $|m/n|<1$. By \cref{absolute value independent iff}, $|\cdot|_0$ and $|\cdot|$ are dependent.
\end{proof}
In case $E$ is a field extension of $K$, every absolute value $|\cdot|'$ of $E$ that restricts to $|\cdot|$ on $K$ is a norm of $E$ compatible with $|\cdot|$. If $E$ has finite degree over $K$ and $K$ is complete with respect to $|\cdot|$, the next proposition will imply that up to equivalence of norms (and hence up to dependence as absolute values), $E$ admits only one absolute value $|\cdot|'$ restricting to $|\cdot|$ on $K$. Moreover, $E$ is complete with respect to $|\cdot|'$.
\begin{proposition}\label{absolute value complete field vector space norm unique}
Let $K$ be a field complete with respect to a non-trivial absolute value $|\cdot|$. Then every two norms (compatible with $|\cdot|$) of a finite dimensional $K$-vector space $E$ are equivalent.
\end{proposition}
\begin{proof}
We shall prove that any such norm on $E$ is equivalent to the max-norm of $E$. This will be done by induction on the dimension $n$ of the $K$-vector space $E$. For $n=1$ the statement is obvious. Assume the proposition is proved for $n-1$, $n\geq 2$. One inequality is very simple to prove. Fix a basis $\omega_1,\dots,\omega_n$ of $E$ over $K$, and for $\xi=x_1\omega_1+\cdots+x_n\omega_n$, denote $\|\xi\|_{\max}=\max_i|x_i|$. Then
\[\|\xi\|\leq\sum_{i=1}^{n}|x_i|\|\omega_i\|\leq C\|\xi\|_{\max}\quad\text{where $C=n\max_i\|\omega_i\|$}.\]
We must now prove that there exists a number $C'>0$ such that for all $\xi\in E$,
\[\|\xi\|_{\max}\leq C'\|\xi\|.\]
Suppose no such number exists. Then, for every positive integer $m$ there exists $\xi\in E$ such that $\|\xi\|_{\max}>m\|\xi\|$. Let $j$ be such that $|x_j|=\max_{1\leq i\leq n}|x_i|$. Letting $\xi_m=x_j^{-1}\xi$ we get $\|\xi\|_{\max}=1$ and thus $\|\xi_m\|<1/m$. For every $m\geq 1$, one of the components of $\xi_m$ equals $1$. Thus there must be an infinite subset $T$ of $\N$ and a fixed number $j$ such that the $j$-th component of $\xi_m$ equals $1$ for all $m\in T$. We fix this number $j$ from now until the end of the proof.\par
Consider the subspace $E_1$ of $E$ consisting of all vectors whose $j$-th coordinate is equal to $0$, equipped with the norm induced by $\|\cdot\|$. By induction, the restrictions of $\|\cdot\|$ and max-norm $\|\cdot\|_{\max}$ to $E_1$ are equivalent. In particular, a sequence of elements of $E_1$ converges to $\zeta\in E_1$ with respect to $\|\cdot\|_{\max}$ if and only if it converges to $\zeta$ with respect to $\|\cdot\|$.\par
For each $m\in T$ we can write $\xi_m=\omega_j+\zeta_m$, for some $\zeta_m\in E_1$. Now, for every $\eps>0$, take $N\in\N$ such that $2/N<\eps$. If $m,n\geq N$, $m,n\in T$, then
\[\|\zeta_m-\zeta_n\|=\|\zeta_m+\omega_j-\omega_j-\zeta_n\|=\|\xi_m-\xi_n\|\leq\|\xi_m\|+\|\xi_n\|<\frac{1}{m}+\frac{1}{n}\leq\frac{2}{N}<\eps.\]
Consequently, $(\zeta_m)$ is a Cauchy sequence with respect to the restriction of $\|\cdot\|$ to $E_1$. From the induction hypothesis, it follows that this sequence is also a Cauchy sequence with respect to the max-norm. Since $E_1$ is complete with respect to the max-norm, this sequence converges to some $\zeta\in E_1$ (with respect to the max-norm).\par
The choice of $T$ implies $\|\omega_j+\zeta_m\|<1/m$ for each $m\in T$. Since the restrictions of $\|\cdot\|$ and the max-norm to $E_1$ are equivalent,
\[\|\zeta-\zeta\|\leq C\|\zeta_m-\zeta\|_{\max}\]
for some number $C>0$. Therefore, 
\[\|\omega_j+\zeta\|\leq\|\omega_j+\zeta_m\|+\|\zeta-\zeta_m\|\leq\frac{1}{m}+C\|\zeta-\zeta_m\|_{\max}.\]
Letting $m\in T$ go to infinity, the right-hand side of the preceding inequality tends to $0$. Hence $\omega_j+\zeta=0$. But, this cannot occur, because $\zeta\in E_1$ has the $j$-th coordinate equal to $0$ and $\omega_1,\dots,\omega_n$ is a basis of $E$ over $K$. This contradiction finishes the proof of the proposition.
\end{proof}
\begin{theorem}\label{absolute value extending R char}
Let $K$ be a field containing $\R$ and having an absolute value that induces the ordinary one on $\R$. Then $K=\R$ or $K=\C$.
\end{theorem}
In particular, the only fields complete with respect to an archimedean absolute value $|\cdot|$ are (up to isomorphism) $\R$ and $\C$ with $|\cdot|$ dependent on the ordinary absolute value.\par
This theorem is a consequence of the following proposition:
\begin{proposition}[\textbf{Gelfand-Mazur}]\label{Gelfand-Mazur thm}
Let $A$ be a commutative normed algebra with identity and assume that $A$ contains an element $j$ such that $j^2=-1$, and let $\C=\R+j\R\sub A$ (identify $\R\cdot 1$ with $\R$). Then for every nonzero element $x_0\in A$, there exists an element $c\in \C$ such that $x_0-c$ is not invertible in $A$.
\end{proposition}
\begin{proof}
Suppose that $x_0-z$ is invertible for all $z\in\C$. The map $f:\C\to A$ defined by $f(z)=(x_0-z)^{-1}$, is then well defined. Moreover, we shall see that taking inverses is a continuous operation on the group of units of $A$, from which it will follow that $f$ is continuous.
In order to show that $x\mapsto x^{-1}$ is continuous on the group of units of $A$, note that for any units $a$ and $x$ in $A$,
\[\|x^{-1}-a^{-1}\|=\|(a-x)a^{-1}x^{-1}\|\leq\|x-a\|\|a^{-1}\|\|x^{-1}\|.\]
Thus it remains to show that $\|x^{-1}\|$ is bounded as $x$ varies through units near $a$. Let $\|a^{-1}\|\|x-a\|\leq 1/2$, and set $w=a^{-1}(x-a)$. Then clearly $\|w\|\leq 1/2$. Hence we get
\[\Big\|\frac{1}{1+w}\Big\|=\Big\|1-\frac{w}{1+w}\Big\|\leq 1+\frac{\|w\|}{1+\|w\|}\leq 1+\frac{1}{2}\Big\|\frac{1}{1+w}\|.\]
This implies $\|(1+w)^{-1}\|\leq 1/2$. Thus finally we get
\[\|x^{-1}\|=\|a^{-1}(1+w)^{-1}\|\leq 2\|a^{-1}\|.\]
Back to the proof of the proposition, observe that for $0\neq z\in\C$,
\[f(z)=\frac{1}{x_0-z}=\frac{1}{z}\Big(\frac{1}{x_0z^{-1}-1}\Big).\]
Since $z^{-1}$ and $x_0z^{-1}$ approach $0$ when $z$ goes to infinity in $\C$, it follows that $f(z)\to 0$ when $z\to\infty$. On the other hand, the map $z\mapsto\|f(z)\|$ is continuous, being the composition of two continuous maps. Consequently, $\|f(z)\|\to 0$ when $z\to\infty$. Hence this map may be considered as a real valued continuous map on the one-point compactification $\widehat{\C}=\C\cup\{\infty\}$ of $\C$. Hence $\|f\|$ has a maximum $M\in\R$, $M>0$. Let $D$ be the set of elements $z\in\C$ such that $\|f(z)\|=M$. Then $D$ is a non-empty, bounded, and closed subset of $\C$. We shall prove that it is also open, a contradiction.\par
For $z_0\in D$, let us write $y=x_0-z_0$ and let $t$ be a nonzero complex number to be determined later. Consider the polynomial $h(X)=X^n-t^n=\prod_{j=1}^{n}(X-\zeta^jt)$, whence
\[\frac{h'(X)}{h(X)}=\frac{nX^{n-1}}{X^n-t^n}=\sum_{j=1}^{n-1}\frac{1}{X-\zeta_n^jt}+\frac{1}{X-t}.\]
Dividing by $n$ and replacing every occurrence of $X$ by $y$, taking account of the definitions of $f$ and $y$, we obtain
\begin{align}\label{Gelfand-Mazur thm-1}
f(z_0+t)+\sum_{j=1}^{n-1}f(z_0+\zeta_n^jt)-nf(z_0)=\frac{ny^{n-1}}{y^n-t^n}-\frac{n}{y}=\frac{n}{y}\cdot\frac{t^ny^{-n}}{1-t^ny^{-n}}.
\end{align}
If we choose $t$ such that $\|ty^{-1}\|<1$, then the last expression in (\ref{Gelfand-Mazur thm-1}) tends to $0$ as $n$ tends to $\infty$; hence
\begin{align}\label{Gelfand-Mazur thm-2}
\|f(z_0+t)\|=\lim_{n\to\infty}\|nf(z_0)-\sum_{j=1}^{n-1}f(z_0+\omega_n^jt)\|.
\end{align}
Now, $\|f(z_0)\|=M$ and $\|f(z_0+\omega_n^jt)\|\leq M$ by definition of $M$, whence
\[\|nf(z_0)-\sum_{j=1}^{n-1}f(z_0+\omega_n^jt)\|\geq n\|f(z_0)\|-\sum_{j=1}^{n-1}\|f(z_0+\omega_n^jt)\|\geq nM-(n-1)M=M.\]
Therefore by (\ref{Gelfand-Mazur thm-2}), letting $n$ tend to $\infty$, $\|f(z_0+t)\|\geq M$ and by definition of $M$ this implies $\|f(z_0+t)\|=M$; in other words $z_0+t\in D$. This proves that the set $D$ is open in $\C$; as it is also closed and non-empty and $\C$ is connected, $D=\C$ and $f$ is therefore constant on $\C$; as this function tends to $0$ at the point at infinity, $\|f(z)\|=0$ in $\C$ and in particular $\|f(0)\|=\|x^{-1}\|=0$, which is absurd.
\end{proof}
\begin{proof}[Proof of \cref{absolute value extending R char}]
We apply \cref{Gelfand-Mazur thm} to the $\R$-algebra $K$, and let the given absolute value serve as norm. If $K$ contains $\C$, then $K=\C$, because every element of $K$ is invertible. If $K$ does not contain $\C$, let $L=K(j)$, where $j^2=-1$. Define a norm on $L$ by putting $\|x+yj\|=\|x\|+\|y\|$ for $x,y\in K$. This clearly makes $L$ a normed $\R$-algebra. Moreover, by standard calculations one proves that $\|1\|=1$ and $\|zw\|\leq\|z\|\|w\|$. Now, applying \cref{Gelfand-Mazur thm} to $A=L$, we obtain $L=\C$ as before. Thus $K$ must be $\R$.
\end{proof}
\subsection{Non-archimedean complete fields}
As we have already noticed, non-archimedean absolute values corresponds to rank $1$ valuations. Therefore, for non-archimedean absolute values, we will use this additive notation and refer non-archimedean absolute values to valuations of rank $1$.\par
Now we prove an important property of a field $K$ complete with respect to a (non-trivial) valuation $v$. This theorem is widely known as Hensel's Lemma.
\begin{theorem}[\textbf{Hensel's Lemma}]\label{Hensel lemma}
Let $K$ be a field complete with respect to a non-archimedean absolute value $v$. Let $f\in A_v[X]$ be a polynomial, and let $a_0\in A_v$ be such that $v(f(a_0))>2v(f'(a_0))$. Then there exists some $a\in A_v$ with $f(a)=0$ and $v(a_0-a)>v(f'(a_0))$.
\end{theorem}
\begin{proof}
The natural way to reach the conclusions of the theorem is to construct
a suitable Cauchy sequence which must converge to a root $a$ of $f$ (recall that every polynomial $f$ is a continuous map).\par
We use the Nowton approximation method: define a sequence $(a_n)$ by
\[a_{n+1}=a_n-\frac{f(a_0)}{f'(a_0)}.\]
We will show that $(a_n)$ has a limit $a$ and this element $a$ satisfies the requirements. In fact, set $r=v(f(a_0))-2v(f'(a_0))>0$, we will prove by induction that
\begin{itemize}
\item[(a)] $v(a_n)\geq 0$.
\item[(b)] $v(a_{n+1}-a_n)=v(f(a_n)/f'(a_n))\geq v(f(a_n)/f'(a_n)^2)\geq 2^nr$.
\item[(c)] $v(a_n-a_0)\geq v(f(a_0)/f'(a_0))$.
\item[(d)] $v(f'(a_n))=v(f'(a_0))$.
\end{itemize}
By hypothesis, (a), (b), and (c) holds for $n=0$. Now assume these conditions for $n$, then we can see (a) holds for $n+1$. Since $a_n$ and $f(a_n)/f'(a_n)$ are in $A_v$, by Talor expansion we have
\begin{equation}\label{Hensel lemma-1}
\begin{aligned}
v(f(a_{n+1}))&=v\Big(f\Big(a_n-\frac{f(a_n)}{f'(a_n)})\Big)=v\Big(f(a_n)-f'(a_n)\cdot\frac{f(a_n)}{f'(a_n)}+M\Big(\frac{f(a_n)}{f'(a_n)}\Big)^2\Big)\\
&=v\Big(M\Big(\frac{f(a_n)}{f'(a_n)}\Big)^2\Big)\geq v\Big(\Big(\frac{f(a_n)}{f'(a_n)}\Big)^2\Big)
\end{aligned}
\end{equation}
where $M\in A_v$. Similarly,
\[\frac{f'(a_{n+1})}{f'(a_n)}=\frac{1}{f'(a_n)}\Big(f'(a_i)+M'\frac{f(a_n)}{f'(a_n)}\Big)=1+M'\frac{f(a_n)}{f'(a_n)^2}\]
for some $M'\in A_v$, whence
\begin{align}\label{Hensel lemma-2}
v\Big(\frac{f'(a_{n+1})}{f'(a_n)}\Big)=0
\end{align}
and $v(f'(a_{n+1}))=v(f'(a_n))\neq\infty$. Now by (\ref{Hensel lemma-1}) and (\ref{Hensel lemma-2}),
\begin{equation}\label{Hensel lemma-3}
\begin{aligned}
v\Big(\frac{f(a_{n+1})}{f'(a_{n+1})^2}\Big)&\geq v\Big(\Big(\frac{f(a_n)}{f'(a_n)}\Big)^2\Big)-v(f'(a_{n+1})^2)=v\Big(\Big(\frac{f(a_n)}{f'(a_n)}\Big)^2\Big)-v(f'(a_{n})^2)\\
&=2v\Big(\frac{f(a_n)}{f'(a_n)^2}\Big)\geq 2^{n+1}r.
\end{aligned}
\end{equation}
This proves condition (b) for $n+1$. Moreover, by (\ref{Hensel lemma-3}),
\[v\Big(\frac{f(a_{n+1})}{f'(a_{n+1})^2}\Big)\geq v\Big(\frac{f(a_n)}{f'(a_n)^2}\Big).\]
From (\ref{Hensel lemma-2}), this implies
\[v\Big(\frac{f(a_{n+1})}{f'(a_{n+1})}\Big)\geq v\Big(\frac{f(a_n)}{f'(a_n)}\Big)\geq\cdots\geq v\Big(\frac{f(a_0)}{f'(a_0)}\Big)\]
which is to say
\[v(a_{n+1}-a_n)\geq v(a_{n}-a_{n-1})\geq\cdots\geq v(a_1-a_0),\]
and therefore
\[v(a_n-a_0)=v(\sum_{i=1}^{n}(a_n-a_{n-1}))=v(a_1-a_0)=v\Big(\frac{f(a_0)}{f'(a_0)}\Big).\]
This proves (c), and completes the induction process.\par
From (a) and (b) we see $(a_n)$ is a Cauchy sequence in $K$, so admits a limit $a\in K$. Also, since the valuation is continuous, we have $v(a)\geq 0$, whence $a\in A_v$. Taking limit in the recursion formula, we see $f(a)=0$. Also, by (c), we have $v(a-a_0)\geq v(f(a_0)/f'(a_0))>v(f'(a_0))$.
\end{proof}
\begin{corollary}
Let $K,v$ be as in \cref{Hensel lemma}. If $f\in A_v[X]$ has a simple zero $a_0$ in the residue field $\kappa_v$, i.e., $\bar{f}(\bar{a}_0)=0$ and $\bar{f}'(a_0)\neq 0$, then $f$ has a zero $a\in A_v$ such that $\bar{a}=\bar{a}_0$.
\end{corollary}
The proof of \cref{Hensel lemma} clearly shows that under the assumption of \cref{Hensel lemma}, the sequence $(f(a_n))$ converges to $0$. At this point, essential use is made of the fact that $v(K^\times)$ is a subgroup of the additive reals. Thus even without the completeness of $K$, we could still notice:
\begin{proposition}
Let $v$ be a non-archimedean absolute value on $K$. Then for every $f\in A_v[X]$, if $\bar{f}$ has a simple zero in $\kappa_v$, then $f(K)$ approximates $0$.
\end{proposition}
Now consider the completion $(\widehat{K},\hat{v})$ of the field with respect to a non-archimedean absolute value $v$, but now using the additive notation for absolute values. The density of $K$ in $\widehat{K}$ has the following important consequence.
\begin{theorem}\label{absolute value completion is immediate extension}
Denote by $\widehat{A}_v$, $\kappa_{\hat{v}}$ and $A_v$, $\kappa_v$ the valuation ring and the residue field of $\hat{v}$ and $v$, respectively. Then the residue fields $\kappa_v$ and $\kappa_{\hat{v}}$, as well as the groups $\Gamma_v$ and $\Gamma_{\hat{v}}$, are canonically isomorphic.
\end{theorem}
\begin{proof}
It follows from the constructions that $A_{\hat{v}}\cap K=A_v$ and $\m_{\hat{v}}\cap K=\m$, where $\m_{\hat{v}}$ and $\m_v$ are the respective maximal ideals. Thus the map that sends the residue class of $a\in A_v$ to the residue class $\bar{a}\in A_{\hat{v}}/\m_{\hat{v}}$ is well defined; and it is clearly a ring homomorphism. It remains to be seen that it is surjective. For every $x\in A_{\hat{v}}$, the set $x+\m_{\hat{v}}$ is an open neighbourhood of $x$. It consists of all elements $z$ such that $\hat{v}(z-x)>0$, or $|z-x|<1$ in terms of the absolute value. Thus the set $(x+\m_v)\cap K$ is non-empty, by the density property. Hence the residue class of $y\in(x+\m_{\hat{v}})\cap K$ is sent by the map above to $\bar{x}$, as required.\par
Similarly, the map $\Gamma_v\to\Gamma_{\hat{v}}$ sending $v(x)$ to $\hat{v}(x)$ for every $x\in K^\times$ is an order-preserving group monomorphism. In order to show surjectivity, let $x\in\widehat{K}^\times$ be given. By the density of $K$ in $\widehat{K}$ there exists $z\in K$ with $\hat{v}(z-x)>v(x)$, or $|z-x|_{\widehat{K}}<|x|$ in terms of the absolute value. But then $\hat{v}(z)=v(x)$.
\end{proof}
\begin{example}
Now consider the $p$-adic valuation on $\Q$; the completion of $(\Q,v_p)$ is denoted by $\Q_p$, and is called the field of \textbf{$p$-adic numbers}. The valuation ring of $\Q_p$, denoted by $\Z_p$, is the ring of $p$-adic integers; it is the topological closure of $\Z$ in $\Q_p$, as we shall see below. According to our previous discussion of this example, \cref{absolute value completion is immediate extension} implies that the residue class field of $\Z_p$ is $\F_p$. Observe also that $p$ is a local parameter for $v_p$ in both fields, $\Q$ and $\Q_p$.
\end{example}
\begin{example}
If we fix $X$ as the irreducible polynomial $p$, then the completion of $(k(X),v_X)$ is the field $k((X))$ of formal Laurent series over $k$, and the valuation ring is $k\llbracket X\rrbracket$, the ring of formal power series.
\end{example}
The last two examples above are special cases of the following more general result:
\begin{proposition}\label{DVR completion element representation}
Let $v$ be a discrete absolute value on the field $K$, with uniformizer $\pi$. Then every element $x\in K^\times$ can be written uniquely as a convergent series
\[x=\sum_{i=\nu}^{\infty}r_i\pi^i\]
where $\nu=v(x)$, $r_\nu\neq 0$, and the coefficients $r_i$ are taken from a set $R\sub A_v$ of representatives of the residue classes in the field $\kappa_v$ (i.e., the canonical map $A_v\to\kappa_v$ induces a bijection of $R$ onto $\kappa_v$). In particular, the valuation ring of $\widehat{K}$ is a complete discrete valuation ring with uniformizer $\pi$, and we have an isomorphism of topological rings $\widehat{A}\cong\llim A/\pi^nA$.
\end{proposition}
\begin{proof}
We proceed by induction. As observed above, $u=x\pi^{-\nu}$ is a unit in $A_v$. Choose $r_\nu\in R$ such that $\bar{r}_\nu=\bar{u}$. Then clearly $v(x\pi^{-\nu}-r_\nu)>0$ or, equivalently,
\[v(x-r_\nu\pi^\nu)>v(\pi^\nu)=\nu.\]
Let $x_1=x-r_\nu\pi^\nu$ and $\nu_1=v(x_1)>\nu$. Then by the same argument we get $r_{\nu_1}\in R$ such that
\[v(x-(r_\nu\pi^\nu+r_{\nu_1}\pi^{\mu_1}))=v(x_1-r_{\mu_1}\pi^{\mu_1})>\nu_1.\]
Repeating this argument and adding "zero coefficients" (i.e. a representative for zero in $\kappa_v$) if necessary, we obtain the existence of the "series"
\[r_\nu\pi^{\nu}+r_{\nu+1}\pi^{\nu+1}+\cdots\]
At the same time we see that it converges to $x$.\par
The uniqueness of the coefficients is clear. Indeed, otherwise $0$ would have a representation
\[0=(r_\nu-r_\nu')\pi^{\nu}+(r_{\nu+1}-r_{\nu+1}')\pi^{\nu+1}+\cdots\]
with $r_\nu\neq r_{\nu}'\in R$, and hence $r_\nu-r_\nu'\neq 0$ in $\kappa_v$. But $v(0)=\infty$, a contradiction.\par
Let $\widehat{A}$ be the valuation ring of $\hat{v}$. Then $\widehat{A}$ is complete (it is closed in $\widehat{K}$) and contains $A$. For each $n\geq 1$ we define a ring homomorphism $\phi_n:\widehat{A}\to A/\pi^nA$ as follows: for each $x=\sum_{i=0}^{\infty}r_i\pi^i$ let $\phi_n(x)$ be the $n$-th truncation $\sum_{i=0}^{n-1}r_i\pi^i$. We thus obtain an infinite sequence of surjective maps $\phi_n:\widehat{A}\to A/\pi^nA$ that are compatible in that for all $n\geq m>0$ and all $x\in\widehat{A}$ the image of $\phi_n(x)$ in $A/\pi^mA$ is $\phi_m(x)$. This defines a surjective ring homomorphism $\phi:\widehat{A}\to\llim A/\pi^nA$. Now note that $\ker\phi=\bigcap_{n}\pi^n\widehat{A}=\{0\}$ so $\phi$ is injective and therefore an isomorphism.\par
To show that $\phi$ is also a homeomorphism, it suffices to note that if $x+\pi^mA$ is a coset of $\pi^mA$ in $A$ and $U$ is the corresponding open set in $\llim A/\pi^nA$, then $\phi^{-1}(U)$ is the closure of $x+\pi^mA$ in $\widehat{A}$, which is the coset $x+\pi^m\widehat{A}$, an open subset in $\widehat{A}$ (as explained in the discussion above, every open set in the inverse limit corresponds to a finite union of cosets $x+\pi^mA$ for some m). Conversely $\phi$ maps open sets $x+\pi^m\widehat{A}$ to open sets in $\llim A/\pi^nA$.
\end{proof}
\begin{example}
Returning once more to our typical examples, any $p$-adic number $z\in\Q_p^\times$ has a unique representation in the form
\[z=\sum_{i=\nu}^{\infty}a_ip^i\]
where $\nu=v_p(z)$, $0\leq a_i<p$ for every $i$, and $a_\nu\neq 0$. The set of representatives chosen here is then $R=\{0,\dots,p-1\}$. If $z\in\Z_p$, i.e., if $v(z)\geq 0$, then $z=\sum_{i=0}^{\infty}a_ip^i$. This shows, in particular, that $\Z$ is dense in $\Z_p$. The reader should be aware of the fact that addition of two "series" of the form $\sum_{i=\nu}a_ip^i$ is not coefficient-wise, as the set $R$ is not closed under addition. As a simple example observe that (choosing $p=7$)
\[5p^i+4p^i=p^{i+1}+2p^i\]
\end{example}
\begin{example}
For the $X$-adic valuation of $k(X)$, we can take the elements of $k$ itself as representatives of the residue field. In this case every $z\in k(X))^\times$ has a unique representation in the form
\[z=\sum_{i=\nu}^{\infty}a_iX^i\]
where $v_X(z)=\nu\in\Z$ and $a_i\in k$ for every $i$. This time, addition of two such series is coefficient-wise. These series are called formal Laurent series. They form a field $k((X))$, the field of formal Laurent series. The canonical discrete absolute value on $k((X))$ is just given by
\[v\Big(\sum_{i=\nu}^{\infty}a_iX^i\Big)=\nu\quad\text{if $\nu\neq 0$}.\]
Clearly its valuation ring consists of the ring $k\llbracket X\rrbracket$ of formal power series, i.e., series of the type $\sum_{i=0}^{\infty}a_iX^i$.
\end{example}
\section{Extensions of valuation rings}
In this part we consider valuation rings in different fields. In particular, we will see how different valuation rings relate in a field extension, and we will prove a fundamental inequality in this situation.
\subsection{Ramification index and inertial degree}
Let $K\sub L$ be a field extension and $w$ a valuation of $L$. Then it is easy to see $v=w|_{K}$ is a valuation of $K$ and for maximal ideals we have
\[\m_v=\m_w\cap K.\] 
That is, $A_v$ is dominated by $A_w$. Conversely, if $v$ is a valuation of $K$, then by \cref{valuation ring dominate prop} we can find a valuation ring of $L$ dominating $A_v$. If $w$ is the valuation of this ring, then the valuation $v$ is given by the restriction of $w$ on $K$ and for value groups we have $\Gamma_v\sub\Gamma_w$. Also, since $A_w$ dominates $A_v$, we also have an extension of residue fields : $\kappa_v\sub\kappa_w$. Now we make the following definition.
\begin{definition}
The index $[\Gamma_w:\Gamma_v]$ is called the \textbf{ramification index} of $w$ over $v$ and denoted by $e(w/v)$. If this index equals $1$ then $w$ is called \textbf{unramified} over $v$. 
\end{definition}
\begin{definition}
The index $[\kappa_w:\kappa_v]$ is called the \textbf{inertial degree} of $w$ over $v$ and denoted by $f(w/v)$.
\end{definition}
In particular, if $e(w/v)=1$ and $f(w/v)=1$, the extension $w/v$ is called \textbf{immediate}. We follow the usual convention that both $e$ and $f$ can be either finite or infinite, without distinguishing between different infinite cardinalities. With this convention, we have the following transitivity due to that of group index and field degree.
\begin{proposition}\label{valuation ring extension degree transitivity}
Let $K\sub L\sub L'$ be field extension, $w'$ an extension of $L'$ and $w$, $v$ the restriction of $w'$ to $L'$ and $L$. Then
\[e(w'/v)=e(w'/w)e(w/v),\quad f(w'/v)=f(w'/w)f(w/v).\]
\end{proposition}
Now, before we prove anything about the quantities $e$ and $f$, we first provide some examples.
\begin{example}
Let $(K,v)$ be a valued field with value group $\Gamma_v$ and $\Gamma$ a totally ordered group with $\Gamma_v\sub\Gamma$. Consider some $t$ transcendental over $K$ and any $\gamma\in\Gamma$. Define a valuation $w$ on $K(t)$ with values in $\Gamma$ via
\[w\Big(\sum_{i=0}^{n}a_it^i\Big)=\min\{v(a_i)+i\gamma:0\leq i\leq n\}\]
for $f=\sum_{i=0}^{n}a_it^i\in K[t]$ and extend $w$ to all of $K(t)$. Then $w$ is a well-defined valuation extending $v$ with value group $\Gamma_w=\Gamma_v+\gamma\Z$.
\end{example}
The following inequality is fundamental for extension of valuations.
\begin{proposition}\label{valuation extension independent product set}
Let $K$ be a field, $L$ a extension of $K$ of degree $n$, $w$ a valuation on $L$ and $v$ its restriction to $K$. Choose $x_1,\dots,x_e\in L^\times$ and $y_1,\dots,y_f\in A_w$ such that
\begin{itemize}
\item[(a)] the values $w(x_1),\dots,v(x_e)$ are representatives of the distinct cosets of $\Gamma_w/\Gamma_v$.
\item[(b)] the residues $\bar{y}_1,\dots,\bar{y}_f\in\kappa_w$ are linearly independent over $\kappa_v$;
\end{itemize}
Then for all $a_{ij}\in K$, we have
\[w\Big(\sum_{ij}a_{ij}x_iy_j\Big)=\min_{ij}\{w(a_{ij}x_iy_j)\}\]
In particular, the products $x_iy_j$ are linearly independent over $K$.
\end{proposition}
\begin{proof}
Let $a_{ij}\in K$, not all zero, and $I\in\{1,\dots,e\}$ and $J\in\{1,\dots,f\}$ such that
\[w(a_{IJ}x_{I})=\min_{ij}\{w(a_{ij}x_i)\}.\]
and observe first that $w(a_{IJ}x_{I})<w(a_{ij}x_i)$ for all $i\neq I$ since otherwise
\[w(x_{I})-w(x_i)=w(a_{ij})-w(a_{IJ})=v(a_{ij}/a_{IJ})\in\Gamma_v,\]
which contradicts our assumption on $x$.\par
Next, write $z=\sum_{ij}a_{ij}x_iy_j$. Since $w(y_j)\geq 0$ for each $j$, for the sake of obtaining a contradiction, we assume that $w(z)>w(a_{IJ}x_I)$. Then $z(a_{IJ}x_I)^{-1}\in\m_w$. Also, according to the previous paragraph we have $a_{ij}x_i(a_{IJ}x_I)^{-1}\in\m_w$ for all $i\neq I$. Dividing everything by $a_{IJ}x_I$ one gets
\[\sum_j\frac{a_{Ij}}{a_{IJ}}y_j=\frac{z}{a_{IJ}x_I}-\sum_{j=1}^{f}\sum_{i\neq I}\frac{a_{ij}x_i}{a_{IJ}x_I}y_j\in\m_w\]
which gives a relation $\sum_j\bar{a}_{Ij}(\bar{a}_{IJ})^{-1}\bar{y}_j=0$, this contradicts the hypothesis made on $y_j$.
\end{proof}
\begin{corollary}\label{valuation extension inequality of index}
Let $K$ be a field, $L$ a finite extension of $K$ of degree $n$, $w$ a valuation on $L$ and $v$ its restriction to $K$. Then the $e(w/v)f(w/v)\leq n$. In particular, $e(w/v)$ and $f(w/v)$ are finite.
\end{corollary}
\begin{theorem}\label{valuation algebraic extension prop}
Let $K$ be a field, $L$ an algebraic extension of $K$, $w$ a valuation on $L$, $v$ its restriction to $K$. Then $\Gamma_w/\Gamma_v$ is torsion and $\kappa_w/\kappa_v$ is algebraic.
\end{theorem}
\begin{proof}
Pick $x\in L$ such that $w(x)=\gamma\in\Gamma_w$. If $L'=K(x)$ and $v'$ is the restriction of $w$ on $L'$ then by \cref{valuation extension inequality of index}, since $L'/K$ is finite degree, the group $\Gamma_{v'}/\Gamma_v$ is finite, whence torsion. It then follows that $\Gamma_w/\Gamma_v$ is torsion.\par
Similarly, for $x\in A_w$ we take $L'$ and $v'$ as above. It follows from \cref{valuation extension inequality of index} that the residue class field $\kappa_{v'}$ is a finite extension of $\kappa_v$. Therefore $\bar{x}\in\kappa_{v'}\sub\kappa_{w}$ is algebraic over $\kappa_v$.
\end{proof}
\begin{corollary}\label{valuation extension rank equal}
Let $K$ be a field, $L$ an algebraic extension of $K$, $w$ a valuation on $L$, $v$ its restriction to $K$. Then the map $\Delta\mapsto\Delta\cap\Gamma_v$ is a one-to-one correspondence between isolated subgroups of $\Gamma_w$ and $\Gamma_v$. In particular, $\rank(w)=\rank(v)$.
\end{corollary}
\begin{proof}
It is clear that $\Delta\cap\Gamma_v$ is an isolated subgroup, if $\Delta$ is. Now let $\Delta\sub\Gamma_v$ be an isolated subgroup and $\Delta'$ denote the set of $\alpha\in\Gamma_w$ such that there exists $\beta\in\Delta$ satisfying $-\beta\leq\alpha\leq\beta$; it is immediately verified that $\Delta'$ is an isolated subgroup of $\Gamma_v$ with $\Delta'\cap\Delta=\Delta$ (since $\Delta$ is isolated); hence the map $\Delta\mapsto\Delta\cap\Gamma_v$ is surjective. Finally, let $\Delta_1$ and $\Delta_2$ be two isolated subgroups of $\Gamma_w$ such that $\Delta:=\Delta_1\cap\Gamma_v=\Delta_2\cap\Gamma_v$ and assume, for example, $\Delta_1\sub\Delta_2$; then $\Delta_2/\Gamma_1$ is a totally ordered group and is isomorphic to a quotient group of $\Delta_2/\Delta$  which itselfis identified with a subgroup of $\Gamma_w/\Gamma_v$. Hence $\Delta_2/\Delta_1$ is a torsion group and therefore reduces to $0$.
\end{proof}
\begin{corollary}\label{valuation extension discrete iff}
Suppose that $L$ is a finite extension of $K$. For $w$ to be discrete, it is necessary and sufficient that $v$ be discrete.
\end{corollary}
\begin{proof}
If $w$ is discrete, then $\Gamma_v$ is isomorphic to a non-zero subgroup of $\Z$ and hence to $\Z$. Conversely, if $v$ is discrete, $\Gamma_v$ is isomorphic to $\Z$ and $\Gamma_w/\Gamma_v$ is a finite group. Hence $\Gamma_w$ is a finitely generated commutative group of rank $1$ and torsion-free. Consequently it is isomorphic to $\Z$.
\end{proof}
\subsection{Prolongations of a valuation}
Given a valuation ring $A\sub K$, and an extension $L/K$, any valuation ring $B$ of $L$ dominating $A$ is called a \textbf{prolongation} of $A$ to $L$. Equivalently, if $v$ is the valuation corresponding to $K$, then prolongations of $A$ corresponds to extensions of $v$ to $L$. In this part, we consider the connection bewteen these prolongations.\par
The following lemma will be used throughout this part, and it states that two distinct prolongations of $A$ to an algebraic extension $L/K$ are incomparable.
\begin{lemma}\label{valuation ring extension comparable iff equal}
Suppose $L/K$ is an algebraic extension of fields, $A$ is a valuation ring of $K$, and $B_1$, $B_2$ are two prolongations of $A$ to $L$. If $B_1\sub B_2$, then $B_1=B_2$.
\end{lemma}
\begin{proof}
The valuation ring $B_1$ maps to a valuation ring $\widebar{B}_1=B_1/\m_{B_2}$ of the residue field $\kappa_{B_2}=B_2/\m_{B_2}$. Since $B_1$ is an extension of $A$, it follows that $\widebar{B}_1$ is also an extension of $\kappa_v$. According to \cref{valuation algebraic extension prop}, $\kappa_{B_2}$ is an algebraic extension of $\kappa_v$, so it follows that $\widebar{B}_1$ is also a field. But $\widebar{B}_1$ is a valuation ring of $\kappa_{B_2}$, so it follows that $\widebar{B}_1=\kappa_{B_2}$ whence $B_1=B_2$. 
\end{proof}
There may exist infinitely many valuation rings of $L$ lying over $A$, of course. But sometimes, their number has a natural bound. This is the content of the next result.
\begin{theorem}\label{valuation ring extension number bound}
Let $L$ be algebraic over $K$ and $[L:K]_s<\infty$. Let $A$ be a valuation ring of $K$. Then the number $n$ of all prolongations of $A$ to $L$ is finite, and $n\leq[L:K]_s$.
\end{theorem}
\begin{proof}
Let $B_1,\dots,B_n$ be the distinct prolongations of $A$ to $L$, with maximal ideals $\m_1,\dots,\m_n$, respectively. Also, by \cref{valuation ring extension comparable iff equal} these prolongations are pairwise incomparable. Therefore, by \cref{valuation ring incomparable Chinese remainder thm} there exist $x_1,\dots,x_n\in L$ such that for all $i,j\in\{1,\dots,n\}$,
\[x_i-1\in\m_i,\quad x_i\in\m_j\text{ for $i\neq j$}.\]
If $K$ has characteristic $p>0$, pick $k$ large enough to guarantee that $x_i^{p^k}\in K^s$, the separable closure of $K$ in $L$. Then we claim that these $n$ elements $x_i^{p^k}$ are lineraly independent over $K$. Consequently, $n\leq[K^s:K]=[L:K]_s$, and the result follows.\par
We shall prove the claim by contradiction. For $a_1,\dots,a_n\in K$, not all zero, such that $\sum_{i=1}^{n}a_ix_i^{p^k}=0$, pick $j$ such that
\[v(a_j)=\min\{v(a_1),\dots,v(a_n)\}.\]
Then $a_j\neq 0$ and we have
\[x_j^{p^k}=-\sum_{i\neq j}a_ix_i^{p^k}\in\m_j.\]
Since this would imply $x_j\in\m_j$ and hence $1\in\m_j$, we get the desired contradiction.
\end{proof}
\begin{corollary}\label{valuation ring unique extension if purely inseparable}
If the extension $L/K$ is purely inseparable, then every valuation ring of $K$ has a unique prolongation to $L$.
\end{corollary}
With \cref{valuation ring extension number bound} in hand, we now classifies all prolongations of a given valuation ring to $L$: they corresponds to maximal ideals of the integral closure of $A$ in $L$.
\begin{theorem}\label{valuation ring extension corresponds to maximal ideal}
Let $L$ be an algebraic extension of a field $K$, $A$ be a valuation ring of $K$, and $R$ the integral closure of $A$ in $L$.
\begin{itemize}
\item[(a)] If $B$ is a prolongation of $A$ to $L$ then $\m=\m_B\cap R$ is a maximal ideal of $R$.
\item[(b)] If $\m$ is a maximal ideal of $R$ then $R_\m$ is a prolongation of $A$ to $L$.
\item[(c)] The map $B\mapsto\m_B\cap R$ gives a one-to-one correspondence between maximal ideals of $R$ and prolongations of $A$ to $L$, and its inverse is given by $\m\mapsto R_{\m}$. 
\end{itemize}
\end{theorem}
\begin{proof}
Note that any prolongation $B$ of $A$ to $L$ must contains $R$, by \cref{integral closure is intersection of valuation ring}. Since $\m_B$ is maximal in $B$, $\m$ is also maximal in $R$. Conversely, since a contraction of maximal ideal is maximal, we see $R_\m$ is a prolongation of $A$ for every maximal ideal $\m$ in $R$.\par
In remains to prove that, for a prolongation $B$ of $A$ to $L$ we have $B=R_{\m}$, where $\m=\m_B\cap R$. The inclusion $R_\m\sub B$ is clear. We first consider the case where the extension $L/K$ is finite. In this case, there are finitely many prolongations of $A$ to $L$ by \cref{valuation ring extension number bound}, say $B_1,\dots,B_n$ with maximal ideals $\m_1,\dots,\m_n$. Then by \cref{valuation ring intersection localization prop} we have $B_i=R_{\m_i}$, where $\m_i=\m_{B_i}\cap R$, whence the claim holds in this case.\par
Now consider the general case. Let $L'$ be a subfield of $L$ such that $L'/K$ is finite. Then we see $R'=R\cap L'$ is the integral closure of $A$ in $L'$ and $B'=B\cap L'$ is a prolongation of $A$ to $L'$ with maximal ideal $\m_{B'}=\m_B\cap L'$. By the argument above, we get $B'=R'_{\m'}$, where $\m'=\m\cap L'=\m_{B'}\cap R'$. Since the subfield $L'$ is arbitrary, it follows that $B=R_{\m}$.
\end{proof}
Next we shall consider the set of all prolongations of a fixed valuation ring of a field $K$ to normal extensions of $K$. It turn out that the automorphism group $\Aut(L/K)$ plays an important rule.
\begin{theorem}\label{valuation ring extension conjugation thm}
Suppose $L/K$ is a normal extension of fields, with $G=\Aut(L/K)$. Suppose $A$ is a valuation ring of $K$, and $B_1$ and $B_2$ are valuation rings in $L$ extending $A$. Then $B_1$ and $B_2$ are conjugate over $K$, i.e., there exists $\sigma\in G$ with $\sigma(B_1)=B_2$.
\end{theorem}
\begin{proof}
First, split the extension $L/K$ into the steps $K\sub K^s\sub L$. \cref{valuation ring unique extension if purely inseparable} implies that every extension of $A$ to $K^s$ has just one prolongation to $L$. Furthermore, $\Aut(K^s/K)$ and $G$ can be canonically
identified. Therefore, we see that it is enough to consider the case where $L$ is separable over $K$.\par
We may assume $[L:K]<\infty$, and the general case follows by considering the compactness of Galois groups. In this case let
\[H_1=\{\sigma\in G:\sigma(B_1)=B_1\},\quad H_2=\{\tau\in G:\tau(B_2)=B_2\}.\]
Then $H_1$ and $H_2$ are subgroups of $G$. Moreover, for every $\sigma\in H_1$, it follows that $\sigma(\m_1)=\m_1$, for the maximal ideal of $B_1$. Indeed, it is enough to observe that $\sigma(\m_1)$ must be the maximal ideal of $\sigma(B_1)$. Analogously for the maximal ideal $\m_2$ of $B_2$, it follows that $\tau(\m_2)=\m_2$ for all $\tau\in H_2$. Next write $G$ as disjoint unions of cosets of $H_1$ and $H_2$, respectively:
\[G=\bigcup_{i=1}^{n}H_1\sigma_i^{-1},\quad G=\bigcup_{k=1}^{m}H_2\tau_k^{-1}.\]
Suppose now, for the sake of contradiction, that $\sigma_i(B_1)\nsubseteq\tau_k(B_2)$ and $\tau_k(B_2)\nsubseteq B_1$ for all $i,k$. Since $\sigma_1^{-1},\dots,\sigma_n^{-1}$ is a complete set of representatives of cosets of $H_1$, for all $i\neq j$, $\sigma_i(B_1)\nsubseteq\sigma_j(B_1)$. Similarly, $\tau_k(H_2)\nsubseteq\tau_l(B_2)$ for every $k\neq l$. Now take
\[C=\bigcap_{i=1}^{n}\sigma_i(H_1)\cap\bigcap_{k=1}^{m}\tau_j(H_2).\]
According to \cref{valuation ring incomparable Chinese remainder thm}, there exists an $x\in C$ such that, for each $i,k$,
\[x-1\in\sigma_i(\m_1),\quad x\in\tau_k(\m_2)\]
As a consequence, for $\sigma\in G$, writing $\sigma=\rho\sigma_i^{-1}$ and $\rho\in H_1$, it follows that
\[\sigma(x-1)\in\rho\sigma_i^{-1}(\sigma_i(\m_1))=\rho(\m_1)=\m_1.\]
Analogously, $\sigma(x)\in\m_2$ for every $\sigma\in G$. Taking norms, it then follows
\[N_{L/K}(a)=\prod_{\sigma\in G}\sigma(a)\in(\m_1+1)\cap K=\m+1\]
and
\[N_{L/K}(a)=\prod_{\sigma\in G}\sigma(a)\in\m_2\cap K=\m.\]
This contradiction implies $\sigma_i(\m_1)\sub\tau_j(\m_2)$ or $\tau_j(\m_2)\sub\sigma_i(\m_1)$. Thus the claim holds in this case by \cref{valuation ring extension comparable iff equal}.
\end{proof}
Now, since we now all prolongations of $A$ are conjugate under $\Aut(L/K)$, we are ready to prove the following theorem.
\begin{theorem}\label{valuation normal extension prop}
Let $L$ be a normal extension of a field $K$, $A$ a valuation ring of $K$, and $B$ a prolongation of $A$ to $L$. Let $v$ and $w$ be valuations corresponding to $A$ and $B$.
\begin{itemize}
\item[(a)] For $\sigma\in\Aut(L/K)$, the map $w\circ\sigma$ is the unique valuation of $L$ that corresponds to $\sigma^{-1}(B)$. In particular, if $\sigma(B)=B$, then $w\circ\sigma=w$.
\item[(b)] $\kappa_B$ is a normal extension of $\kappa_v$.
\item[(c)] The map $x\mapsto\widebar{\sigma(x)}$ induces a $K$-isomorphism from $\sigma^{-1}(B)/\sigma^{-1}(\m_B)$ onto $\kappa_B$. In particular, if $\sigma(B)=B$ then $\bar{\sigma}\in\Aut(\kappa_B/\kappa_v)$.
\item[(d)] For every $\sigma\in\Aut(L/K)$ we have $e(\sigma^{-1}(B)/A)=e(B/A)$ and $f(\sigma^{-1}(B)/A)=f(B/A)$.
\end{itemize}
\end{theorem}
\begin{proof}
Part (a), (b), (d) are easily verified, so we concentrate on (b). Let $\bar{f}\in\kappa_v[X]$ be an irreducible polynomial with a root $\bar{x}\in\kappa_v$. Let $R$ be the integral closure of $A$ in $L$ and write $\m=\m_B\cap R$. By \cref{valuation ring extension corresponds to maximal ideal}, we have $B=R_{\m}$. Observe next that $R$ is fixed by $\Aut(L/K)$. Now let $x\in R$ be a preimage of $\bar{x}$, and let $g\in A[X]$ be the minimal polynomial of $x$ over $K$. Since $L/K$ is normal, $g$ splits completely as $g(X)=\prod_{i=1}^{n}(X-x_i)$ with $x_i\in R$. From $\bar{g}(\bar{x})=0$, it follows that $\bar{f}$ divides $\bar{g}$. But $\bar{g}=\prod_{i=1}^{n}(X-\bar{x}_i)$ in $\kappa_B$, hence $\bar{f}$ has all its roots in $\kappa_B$, proving (b).
\end{proof}
\subsection{Henselian fields}
In this part we introduce the basic concepts of a Henselian field. These will be used when we establish the structure theorem about extension of valuations.\par
We have seen that every rank-one valuation $v$ of a complete field $K$ has a unique prolongation to each algebraic extension of $K$. By \cref{valuation ring unique extension if purely inseparable}, also every valuation of a separably closed field $K$ has this property. Valuation rings, or valuations, with this property are very important. They are the so-called "Henselian" valuation rings. We shall see that they are the suitable substitute for rank-one valuation rings of complete fields. In particular, they are the valuation rings for which Hensel's Lemma holds.\par
A valued field $(K,v)$ is called \textbf{Henselian} if $v$ has a unique extension to every algebraic extension $L$ of $K$. We see from the definition that the property of being Henselian is hereditary. Let $\bar{K}$ be an algebraic closure and $K^s$ be the separable closure of $K$ in $\bar{K}$. We then have the following apparently easier characterization of Henselian valuations.
\begin{proposition}\label{valuation Henselian iff}
A valued field $(K,v)$ is Henselian if and only if it extends uniquely to $K^s$.
\end{proposition}
\begin{proof}
By definition, if $(K,v)$ is Henselian then $v$ extends uniquely to $K^s$. Conversely, take an algebraic extension $L$ of $K$. By \cref{valuation ring unique extension if purely inseparable}, every extension of $v$ to $L\cap K^s$ has an extension to $K^s$. By assumption, $v$ extends uniquely to $L\cap K^s$, so $v$ has a unique prolongation to $L$.
\end{proof}
The next theorem will highlight the importance of Henselian valuation rings. Comparing with completions we see that rank-one complete valuation rings are a particular case of Henselian valuation rings. Before stating the theorem we need a little preparation.\par
Let $v:K^\times\to\Gamma$ be a valuation of the field $K$. The \textbf{Gauss extension} $w$ of $v$ to the rational function field $K(X)$ is given by
\[w(\sum_{i=0}^{n}a_iX^i)=\min_i\{v(a_i)\}.\]
We say a polynomial is primitive if $w(f)=0$. The following properties of primitive polynomials hold, which can be seen as a generalization of the well known Gauss's lemma.
\begin{lemma}\label{valuation primitive polynomial prop}
Let $v:K^\times\to\Gamma$ be a valuation with valuations ring $A$.
\begin{itemize}
\item[(a)] If two polynomials $f$ and $g$ are primitive, then so is their product $fg$.
\item[(b)] Every $f\in K[X]$ admits a decomposition $f=a\tilde{f}$ with $a\in K$ and $\tilde{f}\in K[X]$ primitive.
\item[(c)] If $f\in A[X]$ decomposes as $f=\prod_{i=1}^{m}g_i$ with irreducible factors $g_1,\dots,g_m\in K[X]$, then there are $h_1,\dots,h_m\in A[X]$, irreducible in $K[X]$, such that $f=\prod_{i=1}^{m}h_i$.
\end{itemize}
\end{lemma}
\begin{proof}
Part (a) and (b) are easy to prove. To see (c), write $f=a\tilde{f}$ and $g_i=b_i\tilde{g}_i$, where $a,b_1,\dots,b_m\in K$ and $\tilde{f},\tilde{g}_1,\dots,\tilde{g}_m$ are primitive. Then
\[v(a)=w(f)=\sum_{i=1}^{n}w(g_i)=w(b_1\cdots b_m).\]
Since $f\in A[X]$, we have $v(a)\geq 0$, whence $b_1\cdots b_m\in A$. Define $h_1=b_1\cdots b_m\tilde{g}_1$ and $h_i=\tilde{g}_i$, we see the desired decomposition.
\end{proof}
\begin{theorem}\label{valuation Henselian iff hensel lemma}
Let $(K,v)$ be a valued field with valuation ring $A$ and $\m$ its maximal ideal. The following statements are equivalent:
\begin{itemize}
\item[(\rmnum{1})] $(K,v)$ is Henselian.
\item[(\rmnum{2})] For each irreducible polynomial $f\in A[X]$ with $\bar{f}\notin(\kappa_v)[X]$, there exists $g\in A[X]$ such that $\bar{g}$ is irreducible in $(\kappa_v)[X]$ and $\bar{f}=\bar{g}^s$, for some $s\geq 1$.
\item[(\rmnum{3})] Let $f,g,h\in A[X]$ satisfy $\bar{f}=\bar{g}\bar{h}$, with $\bar{g},\bar{h}$ relatively prime in $(\kappa_v)[X]$. Then there exist $g_1,h_1\in A[X]$ such that
\[f=g_1h_1,\quad\bar{g}_1=\bar{g},\bar{h}_1=\bar{h},\quad\deg g_1=\deg g,\deg h_1=\deg h.\]
\item[(\rmnum{4})] For each $f\in A[X]$ and $a\in A$ with $\bar{f}(\bar{a})=0$ and $\bar{f}'(\bar{a})\neq 0$, there exists an $\alpha\in A$ with $f(\alpha)=0$ and $\bar{\alpha}=\bar{a}$.
\item[(\rmnum{5})] For each $f\in A[X]$ and $a\in A$ with $v(f(a))>2v(f'(a))$, there exists an $\alpha\in A$ with $f(\alpha)=0$ and $v(a-\alpha)>v(f'(\alpha))$.
\item[(\rmnum{6})] Every polynomial $f(X)=X^n+a_{n-1}X^{n-1}+\cdots+a_1X+a_0\in A[X]$ with $a_{n-1}\in\notin\m$ and $a_{n-2},\dots,a_0\in\m$ has a zero in $K$.
\item[(\rmnum{7})] Every polynomial $f(X)=X^n+X^{n-1}+\cdots+a_1X+a_0\in A[X]$ with $a_{n-2},\dots,a_0\in\m$ has a zero in $K$.
\end{itemize}
\end{theorem}
\begin{proof}
Let $(K,v)$ be Henselian, denote by $w$ the unique extension of $v$ to $\widebar{K}$, the algebraic closure of $K$. Write $B$ the valuation ring of $w$ and $\m_B$ for the maximal ideal, and $\kappa_B$ for the residue class field of $B$. Since $w$ is the unique extension of $v$, for every $K$-automorphism $\sigma$ of $\widebar{K}$ we have $\sigma(B)=B$ and $\sigma(\m_B)=\m_B$. Let $f$ an be irreducible polynomial $f\in A[X]$ with $\bar{f}\notin(\kappa_v)[X]$. Write
\[f(X)=\prod_{i=1}^{n}(aX-x_j)\]
where $a,x_1,\dots,x_n\in K$ and $a$ is an $n$-th root of the leading coefficient of $f$. Then $a\in B$ since $w(a)=v(a)/n\geq 0$. Also, $(-1)^nx_1\dots x_n=f(0)\in A$.\par
The roots $x_1/a,\dots,x_n/a$ of $f$ are all $K$-conjugate, and $w$ is invariant under $\Aut(\widebar{K}/K)$. Consequently, there exists $\gamma\in\Gamma_w$ such that $w(x_j/a)=\gamma$ for all $j$. So, for $\delta=\gamma+w(a)$, we have that $w(x_j)=\delta$ for each $j$. Therefore, $x_1,\dots,x_n\in A$ implies $\delta\geq 0$, and thus $x_1,\dots,x_n\in\m_B$ or $x_1,\dots,x_n\in B-\m_B$.\par
In the first case, $\bar{f}=(\bar{a}X)^n$ and the claim (b) follows. In the second case, we obtain
\[\bar{f}=\prod_{i=1}^{n}(\bar{a}X-\bar{x}_i)\]
with $\bar{x}_i\neq 0$. Since $\bar{f}\notin(\kappa_v)[X]$, we also have $\bar{a}\neq 0$. For the sake of seeking a contradiction, let us assume that $\bar{f}=\bar{g}\bar{h}$ for some relatively prime polynomials $\bar{g}$ and $\bar{h}$ with $g,h\in A[X]$. Let $\bar{x_i/a}$ be a root of $\bar{g}$ and take some $x_j\neq x_i$ such that $\bar{x_j/a}$ is a root of $\bar{h}$. Hence $g(x_i/a)\in\m_A$ and $h(x_j/a)\in\m_A$. Take $\sigma\in\Aut(\widebar{K}/K)$ such that $\sigma(x_i/a)=x_j/a$. Then
\[g(x_j/a)=g(\sigma(x_i/a))=\sigma(g(x_i/a))\in\sigma(\m_B)=\m_B\]
which means $\bar{x_j/a}$ is also a root of $\bar{g}$, contradiction. This proves $(\rmnum{1})\Rightarrow(\rmnum{2})$.\par
Now assume (\rmnum{2}), and let $f=gh$ with $f,g,h\in A[X]$ and $g,h$ coprime. By \cref{valuation primitive polynomial prop}, let $f=\prod_{i=1}^{m}g_i$ be a factorization of $f$ with irreducible factors $g_1,\dots,g_m\in A[X]$. By condition (\rmnum{2}), for every $i$, either $g_i\in K$ or there exists $p_i\in A[X]$ such that $p_i$ is irreducible and $g_i=p_i^{t_i}$ for some $t_i\geq 1$. Clearly, we may assume that $p_i$ has no non-zero coefficient in $\m_A$. Renumbering the polynomials $g_1,\dots,g_m$, we may assume that
\[\bar{g}=\bar{a}\prod_{i=1}^{k}\bar{p}_i^{t_i},\quad \bar{b}=\bar{h}\prod_{i=k+1}^{\ell}\bar{p}_i^{t_i},\quad\bar{c}=\prod_{i=\ell+1}^{m}\bar{p}_i^{t_i}\]
for some $a,b,c\in A-\m$, since $\bar{g}$ and $\bar{h}$ are coprime. Define now
\[g_1=a\prod_{i=1}^{k}p_i^{t_i},\quad h_1=\Big(b\prod_{i=k+1}^{\ell}p_i^{t_i}\Big)\Big(c^{-1}\prod_{i=\ell+1}^{m}p_i^{t_i})\]
then the polynomials $g_1$ and $h_1$ satisfies the requirements in (\rmnum{3}), so we have shown $(\rmnum{2})\Rightarrow(\rmnum{3})$.\par
Next, we will deduce (\rmnum{4}) from (\rmnum{3}). If $f\in A[X]$ and $a\in A$ with $\bar{f}(\bar{a})=0$ and $\bar{f}'(\bar{a})\neq 0$, set $g(X)=X-a$ and $\bar{h}=\bar{f}/\bar{g}$, then $\bar{f}=\bar{g}\bar{h}$. Since $\bar{f}'(\bar{a})\neq 0$, $\bar{g}$ and $\bar{h}$ are coprime, so by condition (\rmnum{3}) there exist polynomials $g_1$ and $h_2$ such that
\[f=g_1h_1,\quad \bar{g}_1=\bar{g}=X-\bar{a},\bar{h}_1=\bar{h},\quad\deg g_1=\deg g=1,\deg h_1=\deg h.\]
It then follows that $g_1=u(X-b)$ for some $u\in A^\times$ and $b\in A$, with $\bar{u}=1$ and $\bar{b}=\bar{a}$. Clearly $f(b)=0$, so this proves (\rmnum{4}).\par
For each $f\in A[X]$ and $a\in A$ with $v(f(a))>2v(f'(a))$, we have $f'(a)\neq 0$. Now by a simple computation we know $f(a-X)=f(a)-f'(a)X+X^2g(X)$ for some $g\in A[X]$, and by a change of variable $X=f'(a)Y$ we can define
\[h(Y):=\frac{f(a-f'(a)Y)}{f'(a)^2}=\frac{f(a)}{f'(a)^2}-Y+Y^2g(f'(a)Y).\]
Since $v(f(a))>2v(f'(a))$ we have $h\in A[Y]$. Note that $f(a)/f'(a)^2\in\m_A$, so $\bar{h}$ has a simple root $\bar{0}$ in $\kappa_v$. If condition (\rmnum{4}) holds, this will give a root $b\in\m$ of $h$. Then by the construction of $h$, $\alpha=a-f'(a)b$ will a root of $f$. Moreover, since $v(b)>0$, we have $v(a-\alpha)>v(f'(a))$. This shows $(\rmnum{4})\Rightarrow(\rmnum{5})$.\par
The implication $(\rmnum{5})\Rightarrow(\rmnum{6})$ is immediate. In fact, let $f(X)=X^n+a_{n-1}X^{n-1}+\cdots+a_1X+a_0$ be as in (\rmnum{6}). Then
\[\bar{f}(X)=X^n+\bar{a}_{n-1}X^{n-1}=X^{n-1}(X+\bar{a}_{n-1}).\]
Hence $-\bar{a}_{n-1}$ is a nonzero simple zero of $\bar{f}$. In particular, $f(-a_{n-1})>0$ and $f'(-a_{n-1})\in A-\m$. This means
\[v(f(-a_{n-1}))>0=2v(f'(-a_{n-1}))\]
thus $f$ has a root in $A$, by (\rmnum{5}).\par
It is clear that $(\rmnum{6})\Rightarrow(\rmnum{7})$. Conversely, if (\rmnum{7}) holds and $f(X)=X^n+a_{n-1}X^{n-1}+\cdots+a_1X+a_0$ is a polynomial as in (\rmnum{6}), then the changing $X=a_{n-1}Y$ and dividing by $a_{n-1}^n$, we obtain
\[g(Y)=Y^{n}+Y^{n-1}+\frac{a_{n-2}}{a_{n-1}^2}Y^{n-2}+\cdots+\frac{a_1}{a_{n-1}^n}Y+\frac{a_0}{a_{n-1}^n}.\]
It is clear that a root of $g(Y)$ gives a root of $f$, so (\rmnum{6}) and (\rmnum{7}) are equivalent.\par
Finally, we prove $(\rmnum{7})\Rightarrow(\rmnum{1})$. Suppose $(K,v)$ were not henselian. Then there would be a finite Galois extension $L/K$ with Galois group $\Gal(L/K)$ in which $v$ has more than one extension. Let $B$ be a prolongation of $A$ to $L$ and set $H=\{\sigma\in\Gal(L/K):\sigma(B)=B\}$. As $B$ is not the only prolongation of $A$ to $L$, the group $H$ is a proper subgroup of $\Gal(L/K)$, and thus the fixed field $L':=L^H$ of $H$ is a proper extension of $K$. Let $B_1,\dots,B_n$ be all
conjugates of $B$ in $L$ with maximal ideals $\m_1,\dots,\m_n$ and consider the subring
\[R=\bigcap_{i=1}^{m}(B_i\cap L')\]
of $L'$. By \cref{valuation ring incomparable Chinese remainder thm}, we find $\alpha\in R$ such that $\alpha-1\in\m_1$ and $\alpha\in\m_i$ for $i=2,\dots,n$. Since $n>1$, $\alpha$ cannot lie in $K$, so its minimal polynomial
\[f=\min(\alpha,K)=X^n+a_{n-1}X^{n-1}+\cdots+a_1X+a_0\]
cannot have a zero in $K$. However, if $\alpha_1=\alpha,\alpha_2,\dots,\alpha_n$ are conjugates of $\alpha$ in $L$, we have
\begin{align}\label{valuation Henselian iff hensel lemma-1}
a_{n-i}=(-1)^ie_i(\alpha_1,\dots,\alpha_n),
\end{align}
where $e_i$ is the $i$-th symmetric function. For each $i\geq 2$, we have $\alpha_i\neq\alpha$, so $\alpha_i=\tau(\alpha)$ for some $\tau\in\Gal(L/K)\setminus H$ (note that $\alpha\in L'$). Hence $\tau^{-1}(B)=B_j$ for some $j\geq 2$. Since by the approximation condition we have $\alpha\in\m_j=\tau^{-1}(\m_1)$, we then get $\alpha_i=\tau(\alpha)\in\m_1$. By (\ref{valuation Henselian iff hensel lemma-1}), this implies $1+a_{n-1}\in\m_1$ and $a_{n-2},\dots,a_1,a_0\in\m_1$. Thus by (\rmnum{6}) the polynomial $f$ has a root in $K$, contradiction.
\end{proof}
\begin{corollary}\label{valuation composition henselian iff}
Let $(K,w)$ be a composition of valuations $(K,v)$ and $(\kappa_v,\bar{v})$. Then $(K,w)$ is Henselian if and only if both $(K,v)$ and $(\kappa_v,\bar{v})$ are Henselian.
\end{corollary}
\begin{proof}
Suppose $(K,w)$ is Henselian. Then $(K,v)$ is also Henselian, using $\m_v\sub\m_w\sub A_w\sub A_v$ and \cref{valuation Henselian iff hensel lemma}. To show that $(\kappa_v,\bar{v})$ is Henselian, let $\bar{f}=X^n+X^{n-1}+\cdots+\bar{a}_0$, with $a_i\in\m_{\bar{v}}$ we must show that $\bar{f}$ has a zero in $\kappa_v$ (again using \cref{valuation Henselian iff hensel lemma}). The polynomial
\[f=X^n+X^{n-1}+\cdots+a_0\in A_w[X]\]
has a zero $x\in A_w$ (yet again by \cref{valuation Henselian iff hensel lemma}), since $a_i\in\m_v\sub\m_w$); therefore $\bar{x}\in A_{\bar{v}}$ is a zero of $\bar{f}$.\par
Now we prove the converse. Let $f=X_n+a_{n-1}X^{n-1}\cdots+a_0\in A_w[X]$, and suppose that $\bar{f}$ has a simple zero in the residue field $\kappa_w$. As $\kappa_w$ is also the residue field of $\bar{v}$, and $\bar{v}$ is Henselian, this simple zero of $\bar{f}$ lifts to $\kappa_v$. As the field $(K,v)$ is also Henselian, the zero (which is again simple) can be lifted further to $K$.
\end{proof}
Let $(K,v)$ be a valued field. Assume $L/K$ is a Galois extension. We then say $(K,v)$ is \textbf{$\bm{L}$-Henselian} if $v$ has exactly one extension to $L$. By taking $L=K^s$, the separable closure of $K$, we return to our usual definition of Henselian field. Just as \cref{valuation Henselian iff hensel lemma}, we can also obtain many equivalent characterizations of $L$-Henselian fields. One need only go through the proof of \cref{valuation Henselian iff hensel lemma} and restrict consideration to those polynomials $f$ that split in $L$. The corresponding condition for the Galois extension $K^s/K$ is always satisfied, hence, of course, was not mentioned in \cref{valuation Henselian iff hensel lemma}.
\subsection{The fundamental inequality}
\begin{definition}
Let $K$ be a field, $v$ a valuation on $K$ and $L$ an extension of $K$. A family $(w_i)_{i\in I}$ of valuations on $L$ which extend $v$ and such that every valuation on $L$ extending $v$ is equivalent to a unique $w_i$ is called a \textbf{complete system of extensions of $\bm{v}$ to $\bm{L}$}.
\end{definition}
We have already proved that the completion is an immediate extension of a given valued field. Now we go deeper to this result and consider the case of a extension of valued fields.
\begin{proposition}\label{valuation extension fundamental inequality completion case}
Let $(K,v)$ be a valued field and $(\widehat{K},\hat{v})$ its completion of $K$. Let $L$ be a finite extension of $K$ of degree $n$.
\begin{itemize}
\item[(a)] Let $w$ be a valuation on $L$ extending $v$ and $(\widehat{L}_{w},\hat{w})$ denote the completion of $(L,w)$. Identity $\widehat{K}$ with the closure of $K$ in $\widehat{L}_{w}$, then
\[e(\hat{w}/\hat{v})=e(w/v),\quad f(\hat{w}/\hat{v})=f(w/v),\quad e(w/v)f(w/v)\leq[\widehat{L}_w:\widehat{K}]\leq n.\] 
\item[(b)] Every set of pairwise independent valuations on $L$ extending a nontrivial valuation $v$ is finite. Let $(w_i)_{1\leq i\leq s}$ denote pairwise independent valuations on $L$ extending $v$ such that every valuation on $L$ extending $v$ is dependent on one of the $w_i$; let $L_i$ be the field $L$ with the topology defined by $w_i$ and $\widehat{L}_i$ its completion. Then the canonical map
\[\phi:\widehat{K}\otimes_KL\to\prod_{i=1}^{s}\widehat{L}_i\]
(extending by continuity the diagonal map $L\to\prod_{i=1}^{s}L_i$) is surjective, its kernel is the Jacobson radical of $\widehat{K}\otimes_KL$. In particular, $\sum_{i=1}^{s}[\widehat{L}_i:\widehat{K}]\leq n$.
\end{itemize}
\end{proposition}
\begin{proof}
Let us first prove (a). Suppose that $v$ is nontrivial. As $v$ and $\hat{v}$ (resp. $w$ and $\hat{w}$) have the same order group and the same residue field, we see $e(\hat{w}/\hat{v})=e(w/v)$ and $e(\hat{w}/\hat{v})=e(w/v)$. In particular, $e(w/v)f(w/v)\leq[\widehat{L}_w:\widehat{K}]$ by \cref{valuation extension inequality of index}. Finally the vector sub-$\widehat{K}$-space of $\widehat{L}_w$ generated by $L$ is closed in $\widehat{L}_w$ (by \cref{valued field TVS finite dim subspace closed}) and also dense in $\widehat{L}_w$ so it equals to $\widehat{L}_w$. This shows $[\widehat{L}_w:\widehat{K}]\leq n$ and completes the proof of (a).\par
We now pass to (b). We may still assume that $v$ is not trivial. Let $(w_1,\dots,w_r)$ be any finite family of pairwise independent valuations on $L$ extending $v$. The image of $L$ in $\prod_{i=1}^{r}L_i$ under the diagonal map is dense by the approximation theorem, and $\prod_iL_i$ is dense in $\prod_{i=1}^{r}\widehat{L}_i$. Hence the canonical Image of $\widehat{K}\otimes_KL$ in $\prod_{i=1}^{r}\widehat{L}_i$ is dense. On the other hand this image is a vector sub-$\widehat{K}$-space of $\prod_{i=1}^{r}\widehat{L}_i$; as $\prod_{i=1}^{r}\widehat{L}_i$ is of finite dimension over $\widehat{K}$ by (a), every subspace of $\prod_{i=1}^{r}\widehat{L}_i$ is closed. Thus the image of $\widehat{K}\otimes_KL$ is closed and equal to $\prod_{i=1}^{r}\widehat{L}_i$. As the dimension of $K\otimes_KL$ over $K$ is $n$, we see $\sum_{i=1}^{s}[\widehat{L}_i:\widehat{K}]\leq n$. This shows in particular that the integer $r$ is bounded above by $n$ and shows the first assertion of (b).\par
We now take $(w_1,\dots,w_s)$ as in the statement. The fact that 
\[\phi:\widehat{K}\otimes_KL\to\prod_{i=1}^{s}\widehat{L}_i\]
is suriective have already been shown. It remains to verify that the kernel of $\phi$ is the Jacobsen radical $J$ of $K\otimes_KL$. As $\prod_{i=1}^{s}\widehat{L}_i$ is semi-simple, we see $J\sub\ker\phi$. On the other hand, for every maximal ideal $\m$ of $\widehat{K}\otimes_KL$, the quotient field $L(\m)=(\widehat{K}\otimes_KL)/\m$ is a composite extension of $\widehat{K}$ and $L$ over $K$ (\cref{composite extension char}). There exists a valuation $\mu$ on $L(\m)$ extending $\hat{v}$; the restriction $w$ of $\mu$ to $L$ then extends $v$. As $[L(\m):\widehat{K}]$ is finite, $L(\m)$ is complete with respect to $\mu$ (\cref{valued field TVS dim=n is isomorphic}). Now the closure of $L$ in $L(\m)$ is a field containing $\widehat{K}$ and $L$ and hence is equal to $L(\m)$. Consequently $L(\m)$ is identified with the completion $\widehat{L}_w$ and $\m$ is the kemel of the canonical map of $\widehat{K}\otimes_KL$ onto $\widehat{L}_w$. Now, by hypothesis, there exists an index $i$ such that $w$ and $w_i$ are dependent; whence $\widehat{L}_w=\widehat{L}_{i}$. Thus $\ker\phi\sub\m$, which proves that $\ker\phi\sub J$ and completes the proof.
\end{proof}
\begin{corollary}\label{valuation extension complete is dependent}
If $K$ is complete with respect to $v$ and $v$ is nontrivial, two valuations on $L$ extending $v$ are dependent.
\end{corollary}
\begin{proof}
In this case we have $\widehat{K}\otimes_KL=K\otimes_KL=L$, so the claim follows by \cref{valuation extension fundamental inequality completion case}(b).
\end{proof}
\begin{corollary}
If the extension $L/K$ or $\widehat{K}/K$ is separable then the map $\phi$ is an isomorphism.
\end{corollary}
\begin{proof}
In this case the Jacobsen radical of $\widehat{K}\otimes_KL$ is zero.
\end{proof}
\begin{theorem}[\textbf{The Fundamental Inequality}]\label{valuation extension fundamental inequality}
Let $(K,v)$ be a valued field and $L$ a finite extension of $K$ of degree $n$. Then every complete system $(w_i)_{i\in I}$ of extensions of $v$ to $L$ is finite and we have
\[\sum_{i\in I}e(w_i/v)f(w_i/v)\leq n.\]
\end{theorem}
\begin{proof}
Since the theorem is trivial if $v$ is trivial, we shall assume that $v$ is nontrivial. Let $(w_1,\dots,w_s)$ be any finite family of valuations on $L$ extending $v$, no two of which are equivalent. We shall show that $\sum_{i=1}^{s}e(w_i/v)f(w_i/v)\leq n$. This will prove the theorem.\par
We argue by induction on $s$ and suppose therefore that the inequality has been established for the case of valuations of number smaller than $s$. We distinguish two cases.\par
Suppose that there exist at least two independent valuations of $w_i$. Then there exists a partition $[1,s]=I_1\cup\cdots\cup I_t$ of $[1,s]$ such
that:
\begin{itemize}
\item[(\rmnum{1})] for $w_i$ and $w_j$ to be dependent, it is necessary and suflicient that $i$ and $j$ belong to the same $I_k$;
\item[(\rmnum{2})] $|I_k|<s$ for all $k$.
\end{itemize}
We choose in each $I_k$ an index $i_k$. Let $\widehat{L}_k$ denote the completion of $L$ with respect to $v_{i_k}$ and $n_k=[\widehat{L}_k:\widehat{K}]$. For all $i\in I_k$, $w_i$ defines on $L$ the same topology as $w_{i_k}$ and hence may be extended to a valuation $\hat{w}_i$ on $\widehat{L}_k$ whose restriction to $\widehat{K}$ is $\hat{v}$. Since no two of the $w_i$ for $i\in I_k$ are equivalent, the same is true of the $\hat{w}_i$. The induction hypothesis applied to the ordered pair $(\widehat{K},\widehat{L}_k)$ shows
\[\sum_{i\in I_k}e(w_i/v)f(w_i/v)\leq n_k.\]
As $\sum_kn_k\leq n$ by \cref{valuation extension fundamental inequality completion case}, we see the claim follows in this case.\par
We now pass to the case where any two of the $w_i$ are dependent. In this case there is a nontrivial valuation $w'$ of $L$ such that $w_i=w\circ\bar{w}_i$, where $\bar{w}_i$ is a valuation of the residue field $\kappa_{w'}$. Denote by $v'$ the restriction of $w'$ to $K$, then $v$ is also a composition of $v'$, say $v=v'\circ\bar{v}$, where $\bar{v}$ is a valuation of $\kappa_{v'}$. If we choose $w$ such that the valuation rings $A_{w_i}$ generates $A_{w'}$, then the valuation rings of $\bar{w}_i$ generates $\kappa_{w'}$, so that they are not all dependent. From the first part of the proof, we then have
\[\sum_{i=1}^{s}e(\bar{w}_i/\bar{v})f(\bar{w}_i/\bar{v})\leq[\kappa_{w'}:\kappa_{v'}]=f(w'/v')\]
and hence
\[\sum_{i=1}^{s}e(w'/v')e(\bar{w}_i/\bar{v})f(\bar{w}_i/\bar{v})\leq e(w'/v')f(w'/v')\leq n.\]
Now it suffices to prove that
\begin{align}\label{valuation extension fundamental inequality-1}
f(w_i/v)=f(\bar{w}_i/\bar{v}),\quad e(w'/v')e(\bar{w}_i/\bar{v})=e(w_i/v).
\end{align}
For this, we note that $v$ and $\bar{v}$ (resp. $w_i$ and $\bar{w}_i$) have the same residue field, so the first equality holds. For the second there is, by \cref{valuation of residue field value group} and \cref{value group of localization}, the following commutative diagram, where the rows are exact sequences and the vertical arrows represent canonical injections:
\[\begin{tikzcd}[column sep=12pt,row sep=12pt]
&0\ar[d]&0\ar[d]&0\ar[d]&\\
0\ar[r]&\Gamma_{\bar{v}}\ar[d]\ar[r]&\Gamma_v\ar[d]\ar[r]&\Gamma_{v'}\ar[d]\ar[r]&0\\
0\ar[r]&\Gamma_{\bar{w}_i}\ar[d]\ar[d]\ar[r]&\Gamma_{w_i}\ar[d]\ar[r]&\Gamma_{w'}\ar[d]\ar[r]&0\\
0\ar[r]&\Gamma_{\bar{w}_i}/\Gamma_{\bar{v}}\ar[d]\ar[r]&\Gamma_{w_i}/\Gamma_v\ar[r]\ar[d]&\Gamma_{w'}/\Gamma_{v'}\ar[r]\ar[d]&0\\
&0&0&0&
\end{tikzcd}\]
From this we see the second equality is valid, therefore the proof is completed.
\end{proof}
Now we preceed a step further to investigate when the equality in \cref{valuation extension fundamental inequality} holds. For this, we need the following definition.
\begin{definition}
Let $\Gamma$ be a totally ordered group. A subset of $\Gamma$ is called \textbf{major} if the relations $\alpha\in M$, $\beta\in\Gamma$ and $\beta\geq\alpha$ imply $\beta\in M$.
\end{definition}
\begin{example}
Let $\Delta$ be a isolated subgroup of a totally ordered group $\Gamma$. Then the set
\[M=\{\alpha\in\Gamma:\text{$\alpha\geq\gamma$ for all $\gamma\in\Delta$}\}\]
is a major subset of $\Gamma$.
\end{example}
Let $K$ be a field, $v$ a valuation on $K$, and $A$ the ring of $v$ and $\Gamma$ the order group of $v$. For every major subset $M\sub G$, let $\a(M)$ be the set of $x\in K$ such that $v(x)\in M\cup\{\infty\}$. Clarly $\a(M)$ is a sub-$A$-module of $K$.
\begin{proposition}\label{major subset correspond to fractional ideal}
The map $M\mapsto\a(M)$ is an increasing bijection of the set of major subsets of $\Gamma$ onto the set of sub-$A$-modules of $K$.
\end{proposition}
\begin{proof}
Let $\b$ be a sub-$A$-module of $K$. The set of $v(x)$ for $x\in\b\setminus\{0\}$ is a major subset $M(\b)$ of $\Gamma$. We now show the following equations are satisfied:
\begin{itemize}
\item[(a)] $M(\a(N))=N$ for every major subset $N$ of $\Gamma$;
\item[(b)] $\a(M(\b))=\b$ for every sub-$A$-module $\b$ of $K$.
\end{itemize}
Part (a) is easy, since, for all $\alpha\in N$, there exists $x\in K$ such that $v(x)=\alpha$. Then obviously $\b\sub\a(M(\b))$; conversely, let $x\in\a(M(\b))$ and suppose $x\neq 0$, then $v(x)\in M(\b)$ and therefore there exists $y\in\b$ such that $v(x)=v(y)$; whence $x=uy$ where $v(u)=0$, which proves that $x\in\b$ and completes the proof.
\end{proof}
\begin{corollary}
Let $\Gamma_+$ be the set of positive elements in $\Gamma$. The map $M\mapsto\a(M)$ is a bijection of the set of major subsets of $\Gamma_+$ onto the set of ideals of $A$.
\end{corollary}
\begin{proof}
As $A=\a(\Gamma_+)$, $\a(M)\sub A$ is equivalent to $M\sub\Gamma_+$.
\end{proof}
\begin{definition}
Let $\Gamma$ be a totally ordered group and $\Delta$ a subgroup of $\Gamma$ of finite index. The number of major subsets of $\Gamma$ consisting of strictly positive elements and containing all strict positive elements of $\Delta$ is called the \textbf{initial index} of $\Delta$ in $\Gamma$ and denoted by $\eps(\Gamma,\Delta)$.
\end{definition}
This initial index is a natural number by virtue of the following proposition:
\begin{proposition}\label{ordered group initial index prop}
Let $\Gamma$ be a totally ordered group and $\Delta$ a subgroup of $\Gamma$ of finite index. If $\Gamma$ has no least positive element, then $\eps(\Gamma,\Delta)=1$ for all $\Delta$. If there exists a least positive element of $\Gamma$ and $\Gamma_0$ denotes the subgroup it generates, then
\[\eps(\Gamma,\Delta)=[\Gamma_0:\Gamma_0\cap\Delta].\]
\end{proposition}
\begin{proof}
In the first case, let $\alpha$ be a strict positive element in $\Gamma$. The set of $\alpha\in\Gamma$ such that $0<\beta<\alpha$ is infinite and hence there exist two elements of this set which are distinct and congruent modulo $\Delta$; their difference is an element $\gamma$ of $\Gamma$ such that $0<\gamma<\alpha$. Hence every major subset which contains all the strictly positive elements of $\Delta$ contains $\alpha$ and hence all strict positive elements of $\Gamma$.\par
In the second case, let $\delta$ be the least strict positive element of $\Gamma$ and let $n$ be the least positive integer such that $mx\in\Delta$. Clearly $n=[\Gamma_0:\Gamma_0\cap\Delta]$. On the other hand, writing $M(\alpha)$ for the set of $\beta\in\Gamma$ such that $\beta>\alpha$, it is immediately seen that the admissible major sets are just $M(\delta)$, $M(2\delta),\dots,M(n\delta)$.
\end{proof}
\begin{corollary}\label{ordered group initial index for Z}
The initial index $\eps(\Gamma,\Delta)$ divides the index $[\Gamma:\Delta]$ and is equal to it if $\Gamma$ is isomorphic to $\Z$.
\end{corollary}
\begin{definition}
Let $K$ be a field, $L$ a finite extension of $K$, $w$ a valuation on $L$, $v$ its restriction to $K$ and $\Gamma_v$ and $\Gamma_w$ their order groups. The initial index of $\Gamma_v$ in $\Gamma_w$ is called the \textbf{initial ramification index} of $w$ with respect to $v$ (or with respect to $K$) and denoted by $\eps(w/v)$.
\end{definition}
From the above corollary, $\eps(w/v)$ divides $e(w/v)$ with equality in the case of a discrete valuation.
\begin{proposition}\label{valuation extension initial product inertial}
Let $K$ be a field, $L$ a finite extension of $K$, $w$ a valuation on $L$, $v$ its restriction to $K$. Then
\[\dim_{\kappa_v}(A_w/\m_vA_w)=\eps(w/v)f(w/v).\]
\end{proposition}
\begin{proof}
The ideals of $A_w$ containing $\m_v$ and distinct from $A_w$ correspond to the major subsets of $\Gamma_w$, consisting of strict positive elements and containing the elements strict positive of $\Gamma_v$. They are therefore equal in number to $\eps(w/v)$ and, as they form a totally ordered set under inclusion, this number is equal to the length of the quotient ring $A_w/\m_vA_w$. Now a module of length $1$ over $A_w$ is a $1$-dimensional vector space over $\kappa_w$ and hence a module of length $f(w/v)$ over $A$; hence, as $A_w/\m_vA_w$ is of length $\eps(w/v)$ over $A_w$, it is of length $\eps(w/v)f(w/v)$ over $A_v$, that is over $\kappa_v$.
\end{proof}
\begin{theorem}\label{valuation extension length of integral closure equality}
Let $K$ be a field, $v$ a valuation on $K$, $A$ its ring, $\m$ its maximal ideal, $L$ a finite extension of $K$ of degree $n$, $B$ the integral closure of $A$ in $L$ and $(w_i)_{1\leq i\leq s}$ a complete system of extensions of $v$ to $L$. Then
\[\dim_{\kappa_v}(B/\m B)=\sum_{i=1}^{s}\eps(w_i/v)f(w_i/v)\leq\sum_{i=1}^{s}e(w_i/v)f(w_i/v)\leq n.\]
\end{theorem}
\begin{proof}
Let $B_i$ be the ring of $w_i$; then $B_i=B_{\m_i}$, where $\m_i$ runs through the family of maximal ideals of $B$. Let $\q_i$ be the saturation of $\m B$ with respect to $\m_i$. By \cref{integral finite maximal ideal lying over prop}, the canonical homomorphism $B/\m B\to\prod_{i=1}^{s}B/\q_i$ is an isomorphism and $\m_i$ is the only maximal ideal containing $\q_i$. Hence $B/\q_i$ is canonically isomorphic to $(B/\q_i)_{\m_i}$. On the other hand, we note that
\[(B/\q_i)_{\m_i}=B_{\m_i}/\m B_{\m_i}=B_i/\m B_i,\]
therefore there is a canonical isomorphism $B/\m B\to\prod_{i=1}^{s}B_i/\m B_i$, whence the result in view of \cref{valuation extension initial product inertial}.
\end{proof}
\begin{theorem}\label{valuation extension fundamental equality iff}
With the hypotheses and notation of \cref{valuation extension length of integral closure equality}, the following conditions are equivalent:
\begin{itemize}
\item[(a)] $B$ is a finitely generated $A$-module;
\item[(b)] $B$ is a free $A$-module;
\item[(c)] $\dim_{\kappa_v}(B/\m B)=n$;
\item[(d)] the equality in \cref{valuation extension length of integral closure equality} holds and $\eps(w_i/v)=e(w_i/v)$ for all $i$.
\end{itemize}
\end{theorem}
\begin{proof}
The equivalence of (a) and (b) follows from \cref{valuation ring finitely generated module is free}. Also, the equivalence of (c) and (d) follows from \cref{valuation extension length of integral closure equality}. It remains to show that (c) and (b) are equivalent.\par
First, assume that $B$ is a free $A$-module. From the proof of \cref{integral closure in finite separable contained in finite}, we see for any $x\in L$ there exists $s\in A$ such that $sx$ is integral over $K$, hence in $B$. Since $K$ is the fraction field of $A$, any independent set of $B$ over $A$ is an independent subset of $L$ over $K$. These together show that any basis of $B$ over $A$ is also a basis of $L$ over $K$, hence $\rank_A(B)=n$. Next, we prove that $\dim_{\kappa_v}(B/\m B)$ also equals to $n$. Let $\{x_1,\dots,x_n\}$ be a basis of $B$ over $A$. If $\bar{x}_i$ denotes the image of $x_i$ in $B/\m B$ then $\bar{x}_1,\dots,\bar{x}_n$ span the vector space $B/\m B$ over $\kappa_v$. We assert that the $n$ vectors are independent, hence the claim. For this, it suffices to show that if we have a relation of the form $\sum_{i=1}^{n}a_ix_i\in\m B$, $a_i\in A$, then the $a_i$'s necessarily belong to $A$. But this follows at once from the linear independence of the $x_i$ over $A$, for we have, by assumption: $\sum_{i=1}^{n}a_ix_i=\sum_{i=1}^{n}b_ix_i$, where the $y_i$ are suitable elements of $\m$, and this relation implies $a_i=b_i$ for each $i$.\par
Conversely, assume that $\dim_{\kappa_v}(B/\m B)=n$. Let $x_1,\dots,x_n$ be elements of $B$ whose canonical images in $B/\m B$ form a basis of $B/\m B$ and let $M\sub B$ be the sub-$A$—module which they generate. As $M$ is torsion-free ($M\sub B$ and $B$ is an integral domain) and finitely generated, it is free by \cref{valuation ring finitely generated module is free}. We shall see that $B=M$. Let $y\in B$; we write $N=M+Ay$; this is also a free $A$-module. The canonical injections $M\to N\to B$ give canonical homomorphisms
\[M/\m M\to N/\m N\to B/\m B.\]
Now, by hypothesis, $M/\m M\to B/\m B$ is surjective and $B/\m B$ is $n$-dimensional, hence $M/\m M$ and $N/\m N$ are $n$-dimensional and $M/\m M\to N/\m N$ is surjective. As $N$ is finitely generated, $M\to N$ is then surjective by \cref{module generating set in M/mM}, whence $M=N$ and $B=M$. Hence $B$ is free.
\end{proof}
\begin{corollary}
With the same hypotheses and notation, suppose further that $v$ is discrete and $L$ separable. Then
\[\sum_{i=1}^{s}e(w_i/v)f(w_i/v)=n.\]
\end{corollary}
\begin{proof}
In this case $B$ is a free $A$-module of rank $n$ by \cref{integral closure in finite separable contained in finite}, since $A$ is a PID.
\end{proof}
\begin{corollary}\label{valuation of complete discrete field equality}
Let $(K,v)$ be a discrete valued complete field and $L$ a finite extension of $K$ of degree $n$. Then $v$ admits a unique (up to equivalence) extension $w$ to $L$, the ring $B$ of $w$ is a finitely generated free module over the ring $A$ of $v$ and $e(w/v)f(w/v)=n$.
\end{corollary}
\begin{proof}
All the extensions of $v$ to $L$ are dependent by \cref{valuation extension complete is dependent}. Since they are discrete by \cref{valuation extension discrete iff}, they are therefore equivalent. This shows the uniqueness of $w$. The integral closure of $A$ in $L$ is therefore $B$. As $v$ is discrete, the topology induced on $A$ by that on $K$ is the $\m$-adic topology (where $\m=\m_A$); the ring $A$ is complete, for it is closed in $K$. We conclude that, since $B/\m B$ is a finite-dimensional vector $(A/\m)$-space (\cref{valuation extension initial product inertial}), $B$ is a finitely generated $A$-module (\cref{filtration complete ring M=M_1+N then M=N}). It is therefore free and $e(v'/v)f(v'/v)=n$ by \cref{valuation extension fundamental equality iff}.
\end{proof}
\begin{corollary}\label{valuation extension of rank 1 formula for norm and trace}
Suppose that $v$ is of rank $1$ and that the equivalent conditions of \cref{valuation extension fundamental equality iff} hold. If $\widehat{L}_i$ is the completion of $L$ with respect to $w_i$, then the degree $n_i=[\widehat{L}_i:\widehat{K}]$ is equal to $e(w_i/v)f(w_i/v)$ for all $i$ and the canonical homomorphism
\[\phi:\widehat{K}\otimes_KL\to\prod_{i=1}^{s}\widehat{L}_i\]
is bijective. For all $x\in L$, the characteristic polynomial $\min_{L/K}(x)$ is equal to the product of the characteristic polynomials $\min_{\widehat{L}_i/\widehat{K}}(x)$. In particular, we have
\begin{align}\label{valuation extension of rank 1 formula for norm and trace-1}
\tr_{L/K}(x)=\sum_{i=1}^{s}\tr_{\widehat{L}_i/\widehat{K}}(x),\quad N_{L/K}(x)=\prod_{i=1}^{s}N_{\widehat{L}_i/\widehat{K}}(x),\quad v(N_{L/K}(x))=\sum_{i=1}^{s}n_iw_i(x).
\end{align}
\end{corollary}
\begin{proof}
As no two of the $w_i$ are equivalent and they are of rank $1$ by \cref{valuation extension rank equal}, they are independent and \cref{valuation extension fundamental inequality} therefore shows that $e(w_i/v)f(w_i/v)\leq n_i$ for all $i$ and $\sum_in_i\leq n$. The first assertion therefore follows from these inequalities and the relation $\sum_ie(w_i/v)f(w_i/v)=n$. Under the isomorphism $\phi$ the endomorphism $z\mapsto z(1\otimes x)$ of $\widehat{K}\otimes_KL$ (for $x\in L$) is transformed into the endomorphism of $\prod_{i=1}^{s}\widehat{L}_i$, leaving invariant each of the factors and reducing on each factor to multiplication by $x$ ($L_i$ being canonically imbedded in its completion $\widehat{L}_i$); whence the assertion relating to the characteristic polynomial of $x$ and the first two formulae of (\ref{valuation extension of rank 1 formula for norm and trace-1}). Finally, let $E$ be a finite quasi-Galois extension of $\widehat{K}$ containing $\widehat{L}_i$. As $\widehat{K}$ is complete and $v$ has rank $1$, there exists only one valuation (up to equivalence) $w$ on $E$ extending $\hat{v}$ (\cref{valuation extension complete is dependent}). Then for every $\widehat{K}$-automorphism $\sigma$ of $E$ we have $w(\sigma(x))=w(x)$ and therefore
\[\hat{v}(N_{\widehat{L}_i/\widehat{K}}(x))=n_iw_i(x)\]
which proves the formula.
\end{proof}
\section{Exercise}
\begin{exercise}
If $A$ is a valuation ring of dimension $\geq2$ then the formal power series ring $A\llbracket X\rrbracket$ is not integrally closed.
\end{exercise}
\begin{proof}
Let $0\subset\p_1\subset\p_2$ be a strict chain. Let $b\in\p_1$, $a\in\p_2-\p_1$, then $a^n/b\notin A$ for all $n$:  
\[a^n/b\in A\Rightarrow a^n=(a^n/b)\cdot b\in\p_1\Rightarrow a\in\p_1(\text{$\p_1$ is prime})\]
Now $A$ is a valuation ring so $ba^{-n}\in A$ for all $n$.\par 
Let $f=\sum_{n=0}^{\infty}u_nx^n$ be a solution of
\[f^2+af+x=0\]
We shall take $u_0=-a$, then from the equation
\[f^2+af=\sum_{n=0}^{\infty}(u_0u_n+\cdots+u_nu_0+au_n)x^n\]
we claim that $u_n\in a^{-2n+1}A$ for all $n\geq 1$. In fact, we can compute $u_1=a^{-1}$. And assume this holds for $u_1,\dots,u_{n-1}$, for $u_n$ we have an equation
\[u_0u_n+u_1u_{n-1}+\cdots+u_{n-1}u_1+u_nu_0+au_n=0\]
For $1\leq i\leq n-1$, since $u_i\in a^{-2i+1}A$ we have
\[u_iu_{n-i}=a^{-2i+1}r_1\cdot a^{-2(n-i)+1}r_2=a^{-2n+2}r_1r_2\in a^{-2n+2}A\]
Hence
\[u_n=-(a+2u_0)^{-1}(u_1u_{n-1}+\cdots+u_{n-1}u_1)\in a^{-1}\cdot a^{-2n+2}A=a^{-2n+1}A.\]
Now we only need to prove that $f$ is in the field of fractions of $A\llbracket X\rrbracket$. By our observation we have $bf\in A\llbracket X\rrbracket$, so $f\in b^{-1}A\llbracket X\rrbracket\sub\mathrm{Frac}(A\llbracket X\rrbracket)$. But since $a\in\p_2$, $a^{-1}\notin A$, which means $f\notin A\llbracket X\rrbracket$. So $A\llbracket X\rrbracket$ is not integrally closed.
\end{proof}
\begin{exercise}
If $v$ is an additive valuation of a field $K$ and $\alpha_1,\dots,\alpha_n\in K$ are such that $\alpha_1+\cdots+\alpha_n=0$ then there exist two indices $i,j$ such that $i\neq j$ and $v(\alpha_i)=v(\alpha_j)$.
\end{exercise}
\begin{proof}
If all $\alpha_i=0$ then $v(\alpha_i)=\infty$ for all $i$, and the claim holds. Hence we may assume that $a_n\neq 0$, and from $\alpha_1+\cdots+\alpha_n=0$ we get
\[1=-\Big(\frac{\alpha_1}{\alpha_n}+\cdots+\frac{\alpha_{n-1}}{\alpha_n}\Big).\]
Now, since $1\notin\m$, there exist an index $1\leq i\leq n-1$ such that $\alpha_i/\alpha_n$ is not in $\m$, hence is a unit. Then by our definition $v(\alpha_i/\alpha_n)=0$, so $v(\alpha_i)=v(\alpha_n)$.
\end{proof}
\begin{exercise}
Let $k$ be a field, $x$ and $y$ indeterminates, and supposeth $\alpha$ is a positive irrational number. Then the map $v:k[x,y]\to\R\cup\{\infty\}$ defined by 
\[\sum_{n,m}c_{n,m}x^ny^m\mapsto \min\{n+m\alpha\mid c_{n,m}\neq 0\}\] determines a valuation of $k(x,y)$ with value group $\Z+\alpha\Z$.
\end{exercise}
\begin{proof}
For $f=\sum_{n,m}a_{n,m}x^ny^m$, $g=\sum_{n,m}b_{n,m}x^ny^m$, we have
\[v(fg)=\min\{n_1+n_2+(m_1+m_2)\alpha\mid a_{n_1,m_1}\neq 0,b_{n_2,m_2}\neq 0\}\]
which is easily seen to be equal to $v(f)+v(g)$.\par
For $v(f+g)$, it is also easy to verify $v(f+g)\geq\min\{v(f),v(g)\}$. Also, $v(0)=\infty$. Thus by the previous exercise, $v$ extends to a valuation $k(x,y)\to\R\cup\{\infty\}$. It is clear that $\Gamma_v=\Z+\alpha\Z$.
\end{proof}
\begin{exercise}
Let $A$ be a DVR and $\m$ its maximal ideal; then the $\m$-adic completion $\widehat{A}$ of $A$ is again a DVR. 
\end{exercise}
\begin{proof}
Let $\m=(x)$, then every element in $A$ has a form $rx^n$ for $r$ a unit. Hence we can think the complition of $A$ as the power series ring $(A-\m)\llbracket t\rrbracket$. Thus, we can define a valuation for $f=\sum_{n=0}^{\infty}u_nt^n$:
\[v:\widehat{A}\to\Z,\quad v(f)=\{n\mid u_n\neq 0\}\]
Since $f=\sum_{n=0}^{\infty}u_nt^n$ is a unit if and only if $u_0$ is a unit, this is well defined, and extending it to the field of fractions of $\widehat{A}$ gives makes $\widehat{A}$ a valuation ring.
\end{proof}
\chapter{Krull domains and divisors}
\section{Krull domains}
\subsection{Fractional ideals}
\begin{definition}
Let $A$ be an integral domain and $K$ its field of fractions. A sub-$A$-module $\a$ of $K$ such that there exists a nonzero element $d\in A$ for which $d\a\sub A$ is called a \textbf{fractional ideal} of $A$ (or of $K$, by an abuse of language).
\end{definition}
Every finitely generated sub-$A$-module $\a$ of $K$ is a fractional ideal: for if $a_1,\dots,a_n$ is a system of generators of $\a$, we may write $a_i=x_i/y_i$, where $x_i,y_i\in A$ and $y_i\neq 0$. If $d=y_1\cdots y_n$ then clearly $d\a\sub A$. In particular the principal sub-$A$-modules of $K$ are fractional ideals. It is clear that every ideal of $A$ is a fractional ideal. To avoid confusion, these will also be called the \textbf{integral ideals} of $A$. We denote by $\mathfrak{F}(A)$ the set of non-zero fractional ideals of $A$. Given two elements $\a,\b$ of $\mathfrak{F}(A)$, we shall write $\a\preceq\b$ for the relation "every fractional principal ideal containing $\a$ also contains $\b$. Clearly this relation is a \textbf{preordering} on $\mathfrak{F}(A)$ (it is reflexive and transitive). Consider the associated equivalence relation "$\a\preceq\b$ and $\b\preceq\a$" and let $\mathfrak{D}(A)$ be the equivalent class of this relation. We shall say that the elements of $\mathfrak{D}(A)$ are the \textbf{divisors} of $A$ and, for every fractional ideal $\a\in \mathfrak{F}(A)$, we shall denote by $\div(\a)$ (or $\div_A(\a)$) the canonical image of $\a$ in $\mathfrak{D}(A)$ and we shall say that $\div(\a)$ is the \textbf{divisor of $\a$}. If $a=Ax$ is a fractional principal ideal, we write $\div(x)$ instead of $\div(Ax)$ and $\div(x)$ is called the \textbf{divisor of $\bm{x}$}. The elements of $\mathfrak{D}(A)$ of the form $\div(x)$ are called \textbf{principal divisors}. By taking the quotient, the preordering $\preceq$ on $\mathfrak{F}(A)$ defines on $\mathfrak{D}(A)$ an ordering which we shall denote by $\leq$.\par
For any $\a\in \mathfrak{F}(A)$ there exists by hypothesis some $d\in A$ such that $\a\sub Ad^{-1}$. The intersection $\tilde{\a}$ of the fractional principal ideals containing $\a$ is therefore an element of $\mathfrak{F}(A)$. Clearly the relation $\a\preceq\b$ is equivalent to the relation $\tilde{\a}\sups\tilde{\b}$, which is the case if $\a\sups\b$. For two elements $\a$, $\b$ of $\mathfrak{F}(A)$ to be equivalent modulo $R$, it is necessary and sufficient that $\tilde{\a}=\tilde{\b}$.
\begin{definition}
Every element $\a$ of $\mathfrak{F}(A)$ such that $\a=\tilde{\a}$ is called a \textbf{divisorial fractional ideal} of $A$.
\end{definition}
In other words a divisorial ideal is just a non-zero intersection of a non-empty family of fractional principal ideals. Every non-zero intersection of divisorial ideals is a divisorial ideal. If $\a$ is divisorial, so is $\a x$ for all $x\in K$, the map $\b\mapsto\b x$ being a bijection of the set of fractional principal ideals onto itself. By definition, for all $\a\in \mathfrak{F}(A)$, $\tilde{\a}$ is then the least divisorial ideal containing $\a$ and is equivalent to $\a$ modulo $R$. Moreover, if $\b$ is a divisorial ideal equivalent to $\a$ modulo $R$, then $\tilde{\a}=\tilde{\b}=\b$. Hence $\tilde{\a}$ is the unique divisorial ideal $\b$ such that $\div(\a)=\div(\b)$ (in other words, the restriction of the map $\a\mapsto\div(\a)$ to the set of divisorial ideals is bijective).\par
Let $\a$ and $\b$ be two fractional ideals of $K$. Recall that $(\b:\a)$ denotes the set of $x\in K$ such that $x\a\sub\b$. This is obviously an $A$-module. Note that $(\b:\a)$ is also a fractional ideal, for if $d$ is a non-zero element of $A$ such that $d\b\sub A$ and $d\a\sub A$ and $a$ is a non-zero element of $A\cap\a$, then $da(\b:\a)\sub A$. On the other hand, if $b\neq 0$ belongs to $\b$, then $bd\a\sub\b$, hence $bd\in(\b:\a)$ and $(\b:\a)\neq 0$. Note that the definition of $(\b:\a)$ can also be written as
\begin{align}\label{ideal quotient written as intersection}
(\b:\a)=\bigcap_{x\in\a,x\neq 0}\b x^{-1}.
\end{align}
\begin{proposition}\label{fractional ideal ideal quotient prop}
Let $\a,\b$ be fractional ideals of $A$.
\begin{itemize}
\item[(a)] If $\b$ is a divisorial ideal then $(\b:\a)$ is divisorial.
\item[(b)] In order that $\div(\a)=\div(\b)$, it is necessary and sufficient that $(A:\a)=(A:\b)$.
\item[(c)] We have $\tilde{\a}=(A:(A:\a))$.
\end{itemize}
\end{proposition}
\begin{proof}
Assertion (a) follows immediately from equation (\ref{ideal quotient written as intersection}) since, if $\b$ is divisorial, so is $\b x^{-1}$ for all $x\neq 0$. To show (b), let $P(\a)$ denote the set of fractional principal ideals containing $\a$. The relation $Ax\in P(\a)$ is equivalent to $x^{-1}\a\sub A$ and hence to $x^{-1}\in(A:\a)$. As the relation $\div(\a)=\div(\b)$ is by definition equivalent to $P(\a)=P(\b)$, it is also equivalent to $(A:\a)=(A:\b)$.\par
Finally, as $\a(A:a)\sub A$, we see $\a\sub(A:(A:\a))$. Replacing $\a$ by $(A:\a)$ in this formula, it is seen that $(A:\a)\sub(A:(A:(A:\a)))$. On the other hand, the relation $\a\sub(A:(A:\a))$ implies
\[(A:\a)\sups(A:(A:(A:\a)))\]
Therefore $(A:\a)=(A:(A:(A:\a)))$ and it follows from (b) that $\div(\a)=\div(A:(A:\a))$. As $(A:(A:\a))$ is divisorial by (a), certainly $\tilde{\a}=(A:(A:\a))$, which proves (c).
\end{proof}
\begin{proposition}[\textbf{Lattice Structure in $\mathfrak{D}(A)$}]\label{fractional ideal D(A) is a lattice}
\mbox{}
\begin{itemize}
\item[(a)] In $\mathfrak{D}(A)$ every non-empty set bounded above admits a least upper
bound. Moreprecisely, if $(\a_i)$ is a non-empty family of elements of $\mathfrak{F}(A)$ which is bounded above, then
\[\sup_i(\div(\a_i))=\div(\bigcap_i\tilde{\a}_i).\]
\item[(b)] In $\mathfrak{D}(A)$ every non-empty set bounded below admits a greatest lower bound. More precisely, if $(\a_i)$ is a non-empty family of elements of $\mathfrak{F}(A)$ which is bounded below, then
\[\inf_i(\div(\a_i))=\div(\sum_i\tilde{\a}_i).\] 
\item[(c)] $\mathfrak{D}(A)$ is a lattice.
\end{itemize}
\end{proposition}
\begin{proof}
Let $(\a_i)$ be a non-empty family of elements of $\mathfrak{F}(A)$ which is bounded above. To say that a divisorial ideal $\b$ bounds this family above amounts to saying that it is contained in all the $\tilde{\a}_i$. Hence $\bigcap_i\tilde{\a}_i\neq 0$ and $\sup_i(\div(\a_i))=\div(\bigcap_i\tilde{\a}_i)$.\par
Now let $(\a_i)$ be a non-empty family of elements of $\mathfrak{F}(A)$ which is bounded below. To say that a divisorial ideal $\b$ bounds this family below means that it contains all the $\tilde{\a}_i$, so we see $\inf_i(\div(\a_i))=\div(\sum_i\tilde{\a}_i)$.\par
Finally, it is sufficient to note that, if $\a$, $\b$ are in $\mathfrak{F}(A)$, then $\a\cap\b$ and $\a+\b$ are both nonzero fractional ideals, so the claim follows.
\end{proof}
\begin{corollary}\label{fractional ideal div map valuation inequality}
If $x,y,x+y\in K^\times$ then $\div(x+y)\geq\inf\{\div(x),\div(y)\}$.
\end{corollary}
\begin{proof}
We have $A(x+y)\sub Ax+Ay$ and hence $\div(x+y)\geq\div(Ax+Ay)$.
\end{proof}
\subsection{The monoid structure on \texorpdfstring{$\mathfrak{D}(A)$}{DA}}
\begin{proposition}\label{fractional ideal order product inequality}
Let $\a$, $\a'$, $\b$, $\b'$ be elements of $\mathfrak{F}(A)$. Then the relations $\a\succeq a'$ and $\b\succeq\b'$ imply $\a\b\succeq\a'\b'$.
\end{proposition}
\begin{proof}
We may restrict our attention to the case where $\b=\b'$, by transitivity. Then let $Ax$ be a fractional principal ideal containing $\a'\b$. For every non-zero element $y$ of $\b$, $Ax\sups\a'y$ and hence $Axy^{-1}\sups\a'$, whence $Axy^{-1}\sups\a$ and $Ax\sups\a y$. Varying $y$, it is seen that $Ax\sups\a\b$, whence $\a\b\succeq\a'\b$.
\end{proof}
It follows from \cref{fractional ideal order product inequality} that multiplication on $\mathfrak{F}(A)$ defines, by passing to the quotient, a law of composition on $\mathfrak{D}(A)$ which is obviously associative and commutative. It is written additively so that we may write:
\[\div(\a\b)=\div(\a)+\div(\b)\]
for $\a$, $\b$ in $\mathfrak{F}(A)$. Clearly $\div(1)$ is an identity element for this addition. This element is denoted by $0$. \cref{fractional ideal order product inequality} proves further that the order structure on $\mathfrak{D}(A)$ is compatible with this addition:
\begin{align*}
\inf(\div(\a)+\div(\b),\div(\a)+\div(\c))&=\inf(\div(\a\b),\div(\a\c))=\div(\a\b+\a\c)\\
&=\div(\a(\b+\c))=\div(\a)+\div(\b+\c)\\
&=\div(\a)+\inf(\div(\b),\div(\c)).
\end{align*}
For a fractional ideal $\a$ to be such that $\div(\a)\geq 0$ in $\mathfrak{D}(A)$, it is necessary and sufficient that $\a\sub A$ (in other words, that $\a$ be an integral ideal of $A$).\par
For two elements $x$, $y$ of $K^\times$, the relation $\div(x)=\div(y)$ is equivalent to $Ax=Ay$. Therefore the set of principal divisors of $A$ with the order relation and the monoid law induced by that on $\mathfrak{D}(A)$ is an ordered group canonically isomorphic to the multiplicative group of fractional principal ideals ordered by the opposite order relation to inclusion. The relation $S$ between two elements $P$, $Q$ of $\mathfrak{D}(A)$:
\[\text{there exists $x\in K^\times$ such that $P=Q+\div(x)$}\]
is therefore an equivalence relation since the relation $P=Q+\div(x)$ is equivalent to $Q=P+\div(x^{-1})$. If $P$ and $Q$ are congruent modulo $S$, they are called \textbf{equivalent divisors} of $A$. Clearly moreover the relation $S$ is compatible with the law of the monoid $\mathfrak{D}(A)$ and the latter therefore defines, by taking quotients, a monoid structure on $\mathfrak{D}(A)/S$. This monoid is called the \textbf{divisor class monoid} of $A$.
\begin{proposition}\label{fractional ideal divisorial divisor equivalent iff}
Let $\a$, $\b$ be two divisorial fractional ideals of $A$. Then $\div(\a)$ and $\div(\b)$ are equivalent divisors if and only if there exists $x\in K^\times$ such that $\b=x\a$.
\end{proposition}
\begin{proof}
If $\div(\b)=\div(\a)+\div(x)$ for some $x\in K^\times$, then $\div(\b)=\div(x\a)$ and, as $\b$ and $x\a$ are divisorial, $\b=x\a$, which proves the proposition.
\end{proof}
Let $\a$ be an invertible fractional ideal; then $\a=(A:(A:\a))$ by \cref{invertible submodule prop}(b) and hence $\a$ is divisorial by \cref{fractional ideal ideal quotient prop}. The group $\mathfrak{I}(A)$ of invertible fractional ideals is therefore identified with a subgroup of the monoid $\mathfrak{D}(A)$ and the canonical image of $\mathfrak{I}(A)$ in $\mathfrak{D}(A)/S$ with the group of classes of projective $A$-modules of rank $1$ (\cref{invertible module exact sequence of projective module}).
\begin{proposition}\label{fractional ideal divisor is group iff completely integrally closed}
Let $A$ be an integral domain. For the monoid $\mathfrak{D}(A)$ of divisors of $A$ to be a group, it it necessary and sufficient that $A$ be completely integrally closed.
\end{proposition}
\begin{proof}
Suppose that $\mathfrak{D}(A)$ is a group. Let $x\in K$ and suppose that $A[x]$ is contained in a finitely generated sub-$A$-module of $K$. Then we have seen that $\a=A[x]$ is an element of $\mathfrak{F}(A)$. Then $x\a\sub\a$ and hence $\div(x)+\div(\a)\geq\div\a$. Since $\mathfrak{D}(A)$ is an ordered group, we conclude that $\div(x)\geq 0$, whence $x\in A$. Thus $A$ is completely integrally closed.\par
Conversely, suppose that $A$ is completely integrally closed. Let $\a$ be a divisorial ideal. We shall show that $\div(\a(A:\a))=0$, which will prove that $\mathfrak{D}(A)$ is a group. As $\a(A:\a)\sub A$, it suffices to verify that every fractional principal ideal $Ax^{-1}$ which contains $\a(A:\a)$ also contains $A$. Now, for $y\in K^\times$, the relation $Ay\sups\a$ implies $y^{-1}\in(A:\a)$, whence $y^{-1}\a\sub\a(A:\a)\sub Ax^{-1}$ and hence $x\a\sub Ay$. As $\a$ is divisorial, we then deduce that $x\a\sub\a$, whence $x^{n}\a\sub\a$ for all $n\in\N$. Since $\a$ is divisorial, there exist elements $x_0$, $x_1$ of $K^\times$ such that $Ax_0\sub\a\sub Ax_1$; therefore $x^nx_0\in Ax_1$, whence $x^n\in Ax_1x_0^{-1}$. As $A$ is completely integrally closed, $x\in A$, that is $Ax^{-1}\sups A$, which completes the proof.
\end{proof}
\begin{remark}
Note that, if $A$ is completely integrally closed (and even Noetherian), a divisorial ideal of $A$ is not necessarily invertible, in other words, in general $\mathfrak{I}(A)\neq\mathfrak{D}(A)$.
\end{remark}
\begin{corollary}\label{fractional ideal div of (a:b) in complete integrally closed}
Let $A$ be a completely integrally closed domain and $\a$ a divisorial fractional ideal of $A$. Then, for every nonzero fractional ideal $\b$ of $A$, $\div(\a:\b)=\div(\a)-\div(\b)$.
\end{corollary}
\begin{proof}
By the formula (\ref{ideal quotient written as intersection}),
\[\div(\a:\b)=\div\Big(\bigcap_{y\in\b,y\neq 0}y^{-1}\a\Big)=\sup_{y\in\b,y\neq 0}\div(y^{-1}\a)\]
taking account of \cref{fractional ideal D(A) is a lattice} and the fact that the fractional ideals $y^{-1}\a$ are divisorial. But since $D(A)$ is an ordered group,
\[\sup_{y\in\b,y\neq 0}\div(y^{-1}\a)=\sup_{y\in\b,y\neq 0}(\div(\a)-\div(y))=\div(\a)-\inf_{y\in\b,y\neq 0}\div(y)=\div(\a)-\div(\b)\]
so the claim follows.
\end{proof}
\subsection{Krull domains}
\begin{definition}
An integral domain $A$ is called a \textbf{Krull domain} if there exists a family $(v_i)_{i\in I}$ of valuations on the field of fractions $K$ of $A$ with the following properties:
\begin{enumerate}[leftmargin=40pt]
\item[(K1)] the valuations $v_i$ are discrete;
\item[(K2)] the intersection of the rings of the $v_i$ is $A$;
\item[(K3)] for all $x\in K^\times$, the set of indices $i\in I$ such that $v_i(x)\neq 0$ is finite.
\end{enumerate}
Such a family $(v_i)_{i\in I}$ is called a \textbf{defining family of valuations}.
\end{definition}
\begin{example}[\textbf{Examples of Krull domains}]
\mbox{}
\begin{itemize}
\item[(a)] Every discrete valuation ring is a Krull domain.
\item[(b)] More generally, every PID $A$ is a Krull domain. For let $(p_i)_{i\in I}$ be a representative system of irreducible elements of $A$ and let $v_i$ be the valuation on the field of fractions of $A$ defined by $p_i$. It is immediately seen that the family $(v_i)_{i\in I}$ satisfies properties (K1), (K2) and (K3).
\end{itemize}
\end{example}
Let $A$ be a Krull domain and let $(v_i)_{i\in I}$ be a family of valuations on the field of fractions $K$ of $A$ satisfying (K1), (K2) and (K3). The $v_i$ may be assumed to be normed. For all $\a\in I(A)$, we shall write:
\[v_i(\a)=\sup_{\a\sub Ax}v_i(x).\]
Then $v_i(\a)\in\Z$, for, if $a$ is a non-zero element of $\a$, the relation $Ax\sups Aa$ implies that $v_i(x)\leq v_i(a)$ (by (K2)), which shows that the family $\{v_i(x):\a\sub Ax\}$ is bounded above. We establish the following properties.
\begin{proposition}\label{Krull domain valuation on divisorial ideal prop}
Let $A$ be a Krull domain, $(v_i)_{i\in I}$ a family of normed valuations on $K$ satisfying (K1), (K2) and (K3), and $v_i(\a)$ be defined above.
\begin{itemize}
\item[(a)] If $x\in K^\times$ then $v_i(Ax)=v_i(x)$.
\item[(b)] Let $\a$ and $\b$ be two divisorial fractional ideals of $A$. In order that $\a\sub\b$, it is necessary and sufiicient that $v_i(\a)\geq v_i(\b)$ for all $i\in I$.
\item[(c)] For all $\a\in I(A)$, the indices $i\in I$ such that $v_i(\a)\neq 0$ are finite in number.  
\end{itemize}
\end{proposition}
\begin{proof}
If $Ay\sups Ax$, then $v_i(y)\leq v_i(x)$ by (K2) and the minimum value of $v_i(y)$ is taken at $y=x$. This proves (a). For a divisorial ideal $\a$ be, the relation $y\in\a$ is equivalent to the relation "$\a\sub Ax$ implies $y\in Ax$". Now, by (K2), the relation $y\in Ax$ is equivalent to $v_i(y)\geq v_i(x)$ for all $i\in I$, whence we see $y\in\a$ if and only if $v_i(y)\geq v_i(\a)$ for all $i$, and (b) follows.\par
Finally, for $\a\in I(A)$ there exist $x$, $y$ in $K^*$ such that $Ax\sub\a\sub Ay$. By properties (a) and (b), $v_i(x)\geq v_i(\a)\geq v_i(y)$ for all $i\in I$. It then follows from (K3) that (c) holds.
\end{proof}
\begin{corollary}\label{Krull domain divisorial ideal isomorphic to Z^I}
If $A$ is a Krull domain and $(v_i)_{i\in I}$ is a defining family of normed valuations on $K$, the map $\a\mapsto(v_i(\a))_{i\in I}$ is a decreasing injective map of the set of divisorial integer ideals of $A$ (ordered by inclusion) to the set of positive elements of the ordered group the direct sum $\Z^{\oplus I}$. In particular every non-empty family of divisorial integral ideals of $A$ admits a maximal element.
\end{corollary}
The condition in \cref{Krull domain divisorial ideal isomorphic to Z^I} is also suffcient. In fact, we have the following characterization for Krull domains.
\begin{theorem}\label{Krull domain iff completely integrally closed}
Let $A$ be an integral domain. For $A$ to be a Krull domain, it is necessary and sufficient that the two following conditions be satisfied:
\begin{itemize}
\item[(a)] $A$ is completely integrally closed.
\item[(b)] every non-empty family of divisorial integral ideals of $A$ admits a maximal element (with respect to inclusion). 
\end{itemize}
Moreover, if $P(A)$ is the set of irreducible elements of $\mathfrak{D}(A)$, then $P(A)$ is a basis of the $\Z$-module $\mathfrak{D}(A)$ and the positive elements of $\mathfrak{D}(A)$ are the linear combinations of the elements of $P(A)$ with positive coefficients.
\end{theorem}
\begin{proof}
A Krull domain is completely integrally closed by \cref{valuation rank 1 intersection ring is completely integrally closed}, and it satisfies (b) by \cref{Krull domain divisorial ideal isomorphic to Z^I}. Conversely, let $A$ be an integral domain satisfying properties (a) and (b) of the statement. Since $A$ is completely integrally closed, $\mathfrak{D}(A)$ is an ordered group and a lattice. By condition (b) of the statement, every non-empty family of positive elements of $\mathfrak{D}(A)$ has a minimal element. Let $P(A)$ be the set of irreducible elements of $\mathfrak{D}(A)$. Then by (A, \Rmnum{6}, $\S$1, no.13, Theorem 2) $P(A)$ is a basis of the $\Z$-module $\mathfrak{D}(A)$ and the positive elements of $\mathfrak{D}(A)$ are the linear combinations with positive integer coefficients of the elements of $P(A)$. Thus, for $x\in K^\times$, integers $v_P(x)$ are defined (for $P\in P(A)$) by writing:
\begin{align}\label{Krull domain essential valuatin def}
\div(x)=\sum_{P\in P(A)}v_P(x)P.
\end{align}
From the relations $\div(xy)=\div(x)+\div(y)$ and $\div(x+y)\geq\inf\{\div(x),\div(y)\}$ for $x,y,x+y\in K^\times$, we deduce that the $v_P$ are discrete valuations on $K$. In order that $x\in A$, it is necessary and sufficient that $\div(x)\geq 0$, that is that $v_P(x)\geq 0$ for all $P\in P(A)$. Thus the $v_P$ satisfy conditions (K1) and (K2) and obviously also (K3).
\end{proof}
\begin{corollary}\label{Noe ring Krull iff integrally closed}
For a Noetherian ring to be a Krull domain, it is necessary and sufieient that it be an integrally closed domain.
\end{corollary}
\begin{proof}
An integrally closed Noetherian domain is completely integrally closed.
\end{proof}
Note that there are non-Noetherian Krull domains, for example the polynominal ring $K[X_n]_{n\in\N}$ over a field $K$ in an infinity of indeterminates.
\subsection{Essential valuations for a Krull domain}
Let $A$ be a Krull domain and $K$ its field of fractions. The valuations defined by formula (\ref{Krull domain essential valuatin def}) (for $x\in K^\times$) are called the \textbf{essential valuations} of $K$ (or $A$). We have remarked in the course of the proof of \cref{Krull domain iff completely integrally closed} that the valuations $(v_P)$ satisfy properties (K1), (K2) and (K3). Moreover, these discrete valuations $v_P$ are normed: for every irreducible divisor $P\in P(A)$, $P<2P$ and hence, if $\a$ and $\b$ are the divisorial ideals corresponding to $P$ and $2P$, then $\a\supsetneq\b$. For $x\in\a-\b$, $\div(x)\geq P$ and $\div(x)\ngeq 2P$, whence $v_P(x)=1$, which proves our assertion.
\begin{proposition}\label{Krull domain divisor char by essential valuation}
Let $A$ be a Krull domain, $K$ its field of fractions and $(v_P)_{P\in P(A)}$ the family of its essential valuations. Let $(n_P)_{P\in P(A)}$ be a family of integers with finite support. Then the set of $x\in K$ such that $v_P(x)\geq n_P$ for all $P\in P(A)$ is the divisorial ideal $\a$ of $A$ such that $\div(\a)=\sum_Pn_PP$.
\end{proposition}
\begin{proof}
Let $x\in K^\times$ and $\a$ the divisorial ideal such that $\div(\a)=\sum_Pn_PP$. In order that $x\in\a$, it is necessary and sufficient that $Ax\sub\a$, hence that $\div(x)\geq\div(\a)$ and hence, by (\ref{Krull domain essential valuatin def}), that $v_P(x)\geq n_P$ for all $P\in P(A)$.
\end{proof}
\begin{proposition}\label{Krull domain localization char}
Let $A$ be a Krull domain, $K$ its field of fractions, $(v_i)_{i\in I}$ a defining family of valuations on $K$ and $A_i$ the ring of $v_i$. Let $S$ be a multiplicative subset of $A$ not containing $0$, then
\[S^{-1}A=\bigcap_{S^{-1}A\sub A_i}A_i.\]
In particular $S^{-1}A$ is a Krull domain.
\end{proposition}
\begin{proof}
Let $J=\{i\in I:S^{-1}A\sub A_i\}$ and write $B=\bigcap_{i\in J}A_i$, then we see $i\in J$ if and only if $v_i$ is zero on $S$. Let $x\in B$ and $J(x)$ denote the finite set of indices $i\in I$ such that $v_i(x)<0$. For each $i\in J(x)$ we have $x\notin A_i$, hence $i\notin J$ and so there exists $s_i\in S$ such that $v_i(s_i)>0$. Let $n_i$ be a positive integer such that $v_i(s_i^{n_i}x)\geq 0$ and we write $s=\prod_{i\in J}s_i^{n_i}$. Then $v_i(s_ix)\geq 0$ for all $i\in I$ and hence $sx\in A$ and $x\in s^{-1}A$. Thus $B=S^{-1}A$.
\end{proof}
\begin{corollary}\label{Krull domain irreducible divisor residue field}
Let $P$ be an irreducible divisor of $A$ and $\p$ the corresponding divisorial ideal. Then $\p$ is a prime ideal of height $1$, the ring of $v_P$ is $A_\p$ and the residue field of $v_P$ is identified with the field of fractions of $A/\p$.
\end{corollary}
\begin{proof}
By \cref{Krull domain divisor char by essential valuation} we see $v_P$ is zero on $S=A-\p$ and positive on $\p$, hence $\p=\m_{v_P}\cap A$ and $\p$ is prime. On the other hand, since $P$ is irreducible, for every divisor $Q\neq P$, $Q\ngeq P$ and hence the divisorial ideal $\q$ corresponding to $Q$ is not contained in $\p$. This proves $\height\p=1$ and the corollary then follows from \cref{Krull domain localization char} and \cref{localization and quotient ring}
\end{proof}
\begin{corollary}\label{Krull domain essential valuation equivalent to defining}
Let $A$ be a Krull domain, $K$ its field of fractions and $(v_i)_{i\in I}$ a defining family of valuations. Then every essential valuation of $A$ is equivalent to one of the $v_i$.
\end{corollary}
\begin{proof}
Let $P$ be an irreducible divisor of $A$ and $\p$ the corresponding divisorial ideal. By \cref{Krull domain localization char} we have
\[A_\p=\bigcap_{A_\p\sub A_i}A_i=\bigcap_{\m_i\cap A\sub\p}A_i.\]
If $\m_i\cap A=\{0\}$ then $A_i=K$ which is a contradiction, so $\m_i\cap A\neq\{0\}$, and since $\p$ is of height $1$ we must have $\m_i\cap A=\p$. Since $\p$ in nonzero, we then see there are only finitely many $i\in I$ such that $A_\p\sub A_i$ by (K3). Now $A_\p$ is an intersection of finitely many valuation rings and we can apply \cref{valuation ring intersection maximal ideal char} \cref{valuation ring intersection localization prop} to claim that there is a unique $i\in I$ such that $A_\p=A_i$, so $v_P$ is equivalent to $v_i$.
\end{proof}
\begin{proposition}\label{Krull domain coefficient of P is inf}
Let $A$ be a Krull domain, $(v_P)_{P\in P(A)}$ the family of its essential valuations and $\a\in\mathfrak{F}(A)$. Then the coefficient of $P$ in $\div(\a)$ is $\inf_{x\in\a}(v_P(x))$. If $\p$ is the divisorial prime ideal corresponding to the irreducible divisor $P$, then $\a A_\p=\tilde{\a}A_\p$.
\end{proposition}
\begin{proof}
As $\a=\sum_{x\in\a}Ax$, \cref{fractional ideal D(A) is a lattice} shows that $\div(\a)=\inf_{x\in\a}\div(Ax)$, whence our first assertion. The second follows immediately, since $\div(\tilde{\a})=\div(\a)$ and $A_\p$ is the ring of the discrete valuation $v_P$.
\end{proof}
\begin{proposition}\label{Krull domain Noetherian primary decomposition}
Let $A$ be an integrally closed Noetherian domain.
\begin{itemize}
\item[(a)] Let $P$ be an irreducible divisor of $A$ and $\p$ the corresponding divisorial prime ideal. Then the $n$-th \textbf{symbolic power} $\p^{(n)}$ (that is, the saturation of $\p^n$ with respect to $S=A-\p$) is the set of $x\in A$ such that $v_P(x)\geq n$ and is a $\p$-primary ideal.
\item[(b)] Let $\a$ be a divisorial integral ideal and $\div(\a)=\sum_{i=1}^{r}n_iP_i$, with $\p_i$ the corresponding prime ideal of $P_i$. Then
\[\a=\bigcap_{i=1}^{r}\p_i^{(n_i)}\] 
is the unique reduced primary decomposition of $\a$ and the $\p_i$ are isolated primes of $\a$.
\end{itemize}
\end{proposition}
\begin{proof}
By \cref{Krull domain irreducible divisor residue field}, the relation $x\in \p^nA_\p=(\p A_\p)^n$ is equivalent to $v_P(x)\geq n$. On the other hand, as $A_\p$ is a discrete valuation ring, $(\p A_\p)^n$ is $(\p A_\p)$-primary and hence $\p^{(n)}$ is $\p$-primary; this shows (a). \cref{Krull domain divisor char by essential valuation} certainly shows that $\a=\bigcap_{i=1}^{r}\p_i^{(n_i)}$. As $\p_1,\dots,\p_r$ are primes of height $1$, this primary decomposition is reduced.  The uniqueness follows from \cref{primary decomposition minimal part and saturation}.
\end{proof}
\subsection{Approximation theorem}
As the essential valuations of a Krull domain are discrete and normed, no two of them are equivalent and hence they are independent. \cref{valuation approximation coro} to the approximation theorem may therefore be applied to them: given some $n_i\in\Z$ and some essential valuations $v_i$ finite in number and distinct, there exists $x\in K$ such that $v_i(x)=n_i$ for all $i$. But here there is a more precise result:
\begin{proposition}\label{Krull domain approximation thm}
Let $v_1,\dots,v_r$ be essential valuations of a Krull domain $A$ and $n_1,\dots,n_r$ integers. There exists an element $x$ of the field of fractions $K$ of $A$ such $v_i(x)=n_i$ for all $i$ and $v(x)\geq 0$ for every essential valuation $v$ of $A$ distinct from $v_1,\dots,v_r$.
\end{proposition}
\begin{proof}
Let $\p_1,\dots,\p_r$ be the divisorial ideals of $A$ corresponding to the valuations $v_1,\dots,v_r$. There exists $y\in K$ such that $v_i(y)=n_i$ for all $i$ by \cref{valuation approximation coro}. The essential valuations $w_1,\dots,w_s$ of $A$ distinct from the $v_i$ for which the integer $w_i(y)=-m_j<0$ are finite in number. Let $\q_1,\dots,\q_s$ be the corresponding ideals. There exists no inclusion relation between $\p_1,\dots,\p_r,\q_1,\dots,\q_s$ since these ideals correspond to irreducible divisors and these ideals are prime. Hence the integral ideal $\a=\q_1^{m_1}\cdots\q_s^{m_s}$ is contained in none of the $\p_i$ and is therefore not contained in their union. Therefore there exists $z\in\a$ such that $z\notin\p_i$ for all $i$. Then $v_i(z)=0$ for all $i$ and $w_j(z)\geq m_j$ for all $j$; hence the element $x=yz$ solves the problem.
\end{proof}
\begin{corollary}\label{Krull domain divisorial ideal inclusion}
Let $A$ be a Krull domain, $K$ its field of fractions and $\a$, $\b$ and $\c$ three divisorial fractional ideals of $A$ such that $\a\sub\b$. Then there exists $x\in K$ such that $\a=\b\cap x\c$.
\end{corollary}
\begin{proof}
Let $(v_i)_{i\in I}$ be the family of essential valuations of $A$ and let $(m_i)$ (resp. $(n_i)$, $(p_i)$) be the family of integers (zero except for a finite number of indices) such that $\a$ (resp. $\b$, $\c$) is the set of $x\in K$ for which $v_i(x_i)\geq m_i$ (resp. $n_i$, $p_i$) for all $i\in I$ (\cref{Krull domain divisor char by essential valuation}). The set $J$ of $i\in I$ such that $m_i>n_i$ is finite. As $p_i=m_i=0$ except for a finite number of indices, \cref{Krull domain approximation thm} shows that there exists $x\in K^\times$ such that $v_i(x^{-1})+m_i=p_i$ for $i\in J$ and
\[v_i(x^{-1})+m_i\geq p_i\]
for $i\in I-J$. Then, for all $i\in I$, $m_i=\sup(n_i,v_i(x)+p_i)$. Whence $\a=\b\cap x\c$.
\end{proof}
\begin{corollary}\label{Krull domain divisorial ideal iff intersection of two}
Let $A$ be a Krull domain. For a fractional ideal $\a$ of $A$ to be divisorial, it is necessary and sufficient that it be the intersection of two fractional principal ideals.
\end{corollary}
\begin{proof}
The sufficiency is obvious. The necessity follows from \cref{Krull domain divisorial ideal inclusion} by taking $\b$ and $\c$ to be principal and such that $\b\sups\a$.
\end{proof}
\subsection{Prime ideals of height \boldmath\texorpdfstring{$1$}{1}}
We have seen in \cref{Krull domain irreducible divisor residue field} that irreducible divisors corresponds to prime ideals of height $1$. Now we show that the converse also holds. 
\begin{theorem}\label{Krull domain irreducible divisor iff prime height 1}
Let $A$ be a Krull domain and $\p$ an integral ideal of $A$. For $\p$ to be the divisorial ideal corresponding to an irreducible divisor, it is necessary and suficient that $\p$ be a prime ideal of height $1$.
\end{theorem}
\begin{proof}
If $\p$ is the divisorial ideal corresponding to an irreducible divisor, we know that $\p$ is prime of height $1$. Conversely, let $\p$ be a nonzero prime ideal. As $A_\p\neq K$, by \cref{Krull domain localization char} $A_\p$ is the intersection of a non-empty family $(A_i)$ of essential valuation rings, each $A_i$ being of the form $A_{\q_i}$ (\cref{Krull domain irreducible divisor residue field}) and from $A_\p\sub A_{\q_i}$ we deduce that $\q_i\sub\p$. Thus, if $\p$ is of height $1$, then $\p=\q$, which shows that $\p$ is the divisorial ideal corresponding to an irreducible divisor.
\end{proof}
\begin{corollary}\label{Krull domain prime ideal div zero if height not 1}
In a Krull domain every non-zero prime ideal $\p$ contains a prime ideal of height $1$. If $\p$ is not of height $1$, then $\div(\p)=0$ and $(A:\p)=A$.
\end{corollary}
\begin{proof}
The first assertion is clear. If $\p$ is not of height $1$ and $\q$ is a prime ideal of height $1$ contained in $\p$, then $\q\sub\tilde{\p}$ and $\q\neq\tilde{\p}$. As $\div(\q)$ is irreducible, necessarily $\div(\p)=\div(\tilde{\p})=0$; hence $\div(A:\p)=0$ and, as $(A:\p)$ is divisorial, we get $(A:\p)=A$.
\end{proof}
\begin{corollary}\label{Krull domain valuation equivalent to essential if}
Let $A$ be a Krull domain, $K$ its field of fractions, $v$ a valuation on $K$ which is positive on $A$ and $\p$ the set of $x\in A$ such that $v(x)>0$. If the prime ideal $\p$ is of height $1$, $v$ is equivalent to an essential valuation of $A$.
\end{corollary}
\begin{proof}
Let $B$ be the ring of $v$ and $\m$ its ideal. Then $\m\cap A=\p$ and hence $A_\p\sub B$. Now $A_\p$ is a DVR by \cref{Krull domain irreducible divisor iff prime height 1} and $B\neq K$, thus $B=A_\p$.
\end{proof}
\begin{theorem}\label{Krull domain iff prime of height 1}
Let $A$ be an integral domain. For $A$ to be a Krull domain, it is necessary and suficient that the following properties are satisfied:
\begin{itemize}
\item[(a)] For all prime ideal $\p$ of $A$ of height $1$, $A_\p$ is a DVR.
\item[(b)] $A=\bigcap_{\height\p=1}A_\p$.
\item[(c)] For all $x\neq 0$ in $A$, there exists only a finite number of prime ideals $\p$ of height $1$ such that $x\in\p$. 
\end{itemize}
Moreover, in the admissible case, the valuations corresponding to the $A$ for prime ideals of height $1$ are the essential valuations of $A$.
\end{theorem}
\begin{proof}
The conditions are trivially sufficient. Their necessity follows immediately from \cref{Krull domain irreducible divisor iff prime height 1}, \cref{Krull domain irreducible divisor residue field} and the fact that the essential valuations of $A$ is a defining family.
\end{proof}
\begin{proposition}\label{Krull Noe domain divisorial iff prime ideal belong height 1}
Let $A$ be an integrally closed Noetherian domain and $\a$ an integral ideal of $A$. Then $\a$ is divisorial if and only if the prime ideals associated with $A/\a$ are of height $1$.
\end{proposition}
\begin{proof}
Recall that, if $\a=\bigcap_{i=1}^{n}\q_i$ is a reduced primary decomposition of $\a$ and $\p_i$ denotes the prime ideal corresponding to $\q_i$, the prime ideals associated with $A/\a$ are just the $\p_i$. The necessity then follows from \cref{Krull domain Noetherian primary decomposition}. Conversely, if, in the above notation, the $\p_i$ are of height $1$, then $A_{\p_i}$ is a DVR and each $\p_i$ is isolated, so $\q_i=\q_iA_{\p_i}\cap A$ (\cref{primary module and localization}). Denoting by $v_i$ the essential valuation corresponding to $\p_i$, there therefore exists an integer $n_i$ such that $\q_i$ is the set of $x\in A$ such that $v_i(x)\geq n_i$ ($n_i$ is just the integer such that $\q_i A_{\p_i}=(\p_i A_{\p_i})^{n_i}$ in $A_{\p_i}$). This shows the $\q_i$ are divisorial by \cref{Krull domain divisor char by essential valuation}, hence also is $\a$.
\end{proof}
\begin{proposition}\label{Krull local domain DVR iff}
Let $A$ be a local Krull domain (in particular an integrally closed local Noetherian domain) and $\m$ its maximal ideal. The following conditions are equivalent:
\begin{itemize}
\item[(\rmnum{1})] $A$ is a DVR;
\item[(\rmnum{2})] $\m$ is invertible;
\item[(\rmnum{3})] $(A:\m)\neq A$;
\item[(\rmnum{4})] $\m$ is divisorial;
\item[(\rmnum{5})] $\m$ is the only non-zero prime ideal of $A$.
\end{itemize}
\end{proposition}
\begin{proof}
As every non-zero ideal of a DVR is principal, it is invertible and hence (\rmnum{1}) implies (\rmnum{2}). If $\m$ is invertible, its inverse is $(A:\m)$ and hence $(A:\m)\neq A$, hence (\rmnum{2}) implies (\rmnum{3}). If $(A:\m)\neq A$, then $(A:(A:\m))\neq A$. Now $\m\sub(A:(A:\m))$, hence $\m=(A:(A:\m))$ since $\m$ is maximal, so that $\m$ is divisorial. Thus (\rmnum{3}) implies (\rmnum{4}). The fact that (\rmnum{4}) implies (\rmnum{5}) follows from \cref{Krull domain irreducible divisor iff prime height 1}, Finally, if $\m$ is the only non-zero prime ideal of $A$, it is of height $1$ and hence $A_\m$ is a DVR by \cref{Krull domain iff prime of height 1}. As $A$ is local, $A_\m=A$, which shows that (\rmnum{5}) implies (\rmnum{1}).
\end{proof}
\subsection{Permanence properties}
\begin{proposition}\label{Krull domain integral closure is Krull}
Let $A$ be a Krull domain, $K$ its field of fractions, $L$ a finite extension of $K$ and $R$ the integral closure of $A$ in $L$. Then $R$ is a Krull domain. The essential valuations of $R$ are the normed discrete valuations on $L$ which are equivalent to the extensions of the essential valuations of $A$.
\end{proposition}
\begin{proof}
Let $(v_i)_{i\in I}$ be a complete family of extensions to $L$ of the essential valuations of $A$. Since the degree $n=[L:K]$ is finite, the $v_i$ are discrete valuations on $L$ by \cref{valuation extension discrete iff}. Let $B_i$ be the ring of $v_i$. Then $R\sub\bigcap_{i\in I}B_i$ by \cref{integral closure is intersection of valuation ring}. Conversely, every element of $x\in\bigcap_iB_i$ is integral over each of the essential valuation rings of $A$ by \cref{valaution ring integral closure is intersection}, hence the coefficients of the minimal polynomial of $x$ over $K$ belong to $A$ (\cref{integral element iff minimal polynomial coefficient}), so that $x\in R$; thus $R=\bigcap_iB_i$. Now let $x$ be a non-zero element of $R$. It satisfies an equation of the form
\[x^n+a_{n-1}x^{n-1}+\cdots+a_1x+a_0=0\]
where $a_i\in A$ and $a_0\neq 0$. If $v_i(x)>0$, then $v_i(a_0)>0$. Now the essential valuations $v$ of $A$ such that $v(a_0)>0$ are finite in number and the valuations on $L$ extending a given valuation on $K$ are also finite in number, hence $v_i(x)=0$ except for a finite number of indices $i\in I$. Thus it has been proved that $R$ is a Krull domain.\par
It remains to show that the $v_i$ are equivalent to essential valuations of $R$, that is, that the prime ideal $\p_i$ consisting of the $x\in R$ such that $v_i(x)>0$ is of height $1$. If this were not so, there would exist a prime ideal $\q$ of $R$ such that $(0)\sub\q\sub\p_i$ distinct from $(0)$ and $\p_i$. Then $(0)\sub\q\cap A\sub\p_i\cap A$ and $\q\cap A$ would be distinct from $(0)$ and $\p_i\cap A$ by \cref{integral ring extension prime lying over same contraction}. The prime ideal $\p_i\cap A$ would therefore not be of height $1$, which contradicts the fact that it corresponds to an essential valuation of $A$.
\end{proof}
\begin{corollary}\label{Krull domain integral closure prime lying over iff valuation}
Let $\p$ (resp. $\mathfrak{P}$) be a prime ideal of $A$ (resp. $R$) of height $1$ and $v$ (resp. $w$) the essential valuation of $A$ (resp. $R$) corresponding to it. For $\mathfrak{P}$ to lie over $\p$, it is necessary and sufficient that the restriction of $w$ to $K$ be equivalent to $v$.
\end{corollary}
\begin{proof}
The valuation $w$ is equivalent to the extension of an essential valuation $v'$ of $A$. Let $\q=\mathfrak{P}\cap A$, which is a prime ideal of $A$ of height $1$. For the restriction of $w$ to $K$ to be equivalent to $v$, it is necessary and suflicient that $v'=v$ and hence that $\q=\p$.
\end{proof}
\begin{lemma}\label{Krull domain intersection is Krull if}
Let $A$ be an integral domain, $K$ its field of fraction and $L$ an extension of $K$. If $(A_i)_{i\in I}$ is a family of Krull domains contained in $L$ satisfying the following two conditions:
\begin{itemize}
\item[(a)] $A=\bigcap_iA_i$.
\item[(b)] Each nonzero element of $A$ is a nonunit in only finitely many $A_i$.
\end{itemize}
Then $A$ is a Krull doamin.
\end{lemma}
\begin{proof}
For each $i\in I$ let $(v_{ij})_{j\in J_i}$ be a defining family of valuations of $A_i$. Let $x\in A$ be nonzero, then $x$ is not a unit for only finitely many $A_i$'s. Also, for each $i$, since $A_{i}$ is a Krull domain, $v_{ij}(x)\neq 0$ for only finitely many $j\in J_{i}$, whence we see the family $(v_{ij})_{i\in I,j\in J_i}$ satisfies the condition (K3) and hence $A$ is a Krull domain. 
\end{proof}
\begin{proposition}\label{Krull domain polynomial ring is Krull}
Let $A$ be a Krull domain, then the ring $A[X_1,\dots,X_n]$ is a Krull domain.
\end{proposition}
\begin{proof}
Arguing by induction on $n$, it is sufficient to show that, if $X$ is an indeterminate, $A[X]$ is a Krull domain. Let $K$ be the field of fractions of $A$. The field of fractions of $A[X]$ is $K(X)$. The ring $K[X]$ is a PID and therefore a Krull ring. Moreover, for every prime ideal $\p$ of $A$ of height $1$, let $\bar{v}_\p$ be the extension of $v$ to $K(X)$ defined by
\[\bar{v}_\p\Big(\sum_ja_jX^j\Big)=\inf_jv_\p(a_j)\]
for $\sum_ja_jX^j\in K[X]$. Then the valuation ring of $\bar{v}_\p$ in $K(X)$ is $A[X]_{\p A[X]}$. We also note that $K\cap A[X]_{\p A[X]}=A_\p[X]$, in fact, for $\sum_ja_jX^j\in K[X]$, the relation $\bar{v}_\p(\sum_ja_jX^j)\geq 0$ is equivalent to $v_\p(a_j)\geq 0$ for all $j$, hence equivalent to $\sum_ja_jX^j\in A_\p[X]$. With this, we have
\[A[X]=K[X]\cap\bigcap_{\height\p=1}A[X]_{\p A[X]}\]
Since $A$ is Krull we see every nonzero element in $A[X]$ is a nonunit in only finitely many $A_\p[X]$, so the condition of \cref{Krull domain intersection is Krull if} is satisfied and we see $A[X]$ is a Krull domain.
\end{proof}
\begin{corollary}\label{Krull domain prime ideal height 1 of A[X]}
Let $A$ be a Krull domain, then the prime ideals of $A[X]$ of height $1$ are:
\begin{itemize}
\item[(a)] the prime ideals of the form $\p A[X]$, where $\p$ is a prime ideal of $A$ of height $1$;
\item[(b)] the prime ideals of the form $\m\cap A[X]$, where $\m$ is a (necessarily principal) prime ideal of $K[X]$.
\end{itemize}
\end{corollary}
\begin{proof}
This follows from the observation that that the prime ideals of height $1$ in $K[X]$ are exactly maximal ideals.
\end{proof}
\begin{proposition}\label{Krull domain power series ring is Krull}
Let $A$ be a Krull domain, then the ring $A\llbracket X_1,\dots,X_n\rrbracket$ is a Krull domain.
\end{proposition}
\begin{proof}
By induction we only need to show $A\llbracket X\rrbracket$ is Krull. Since $A$ is Krull we have $A=\bigcap_{\height\p=1}A_\p$, and therefore $A\llbracket X\rrbracket=\bigcap_{\height\p=1}A_\p\llbracket X\rrbracket$. Also, each $A_\p$ is a DVR and thus $A_\p\llbracket X\rrbracket$ is integrally closed and Noetherian, hence a Krull domain. However, as $X$ is a non-unit of all the rings $A_\p\llbracket X\rrbracket$, we can not apply \cref{Krull domain intersection is Krull if} directly. On the other hand, note that
\[A\llbracket X\rrbracket=K\llbracket X\rrbracket\cap\bigcap_{\height\p=1}(A_\p\llbracket X\rrbracket[X^{-1}]).\]
Now the hypothesis in \cref{Krull domain intersection is Krull if} is easily verified for the rings $A_\p\llbracket X\rrbracket[X^{-1}]$. Indeed, an element $f(X)=\sum_{i=r}^{\infty}a_iX^i$ (with $a_r\neq 0$) in $A\llbracket X\rrbracket$ is a nonunit in $A_\p\llbracket X\rrbracket[X^{-1}]$ if and only if $a_r$ is a nonunit of $A_\p$, and there finitely many such $\p$. Therefore $A\llbracket X\rrbracket$ is a Krull domain.
\end{proof}
\subsection{Divisor classes in a Krull domain}
Let $A$ be a Krull domain. Recall that the group $\mathfrak{D}(A)$ of divisors of $A$ is the free commutative group generated by the set $P(A)$ of its irreducible elements and that $P(A)$ is identified with the set of prime ideals of $A$ of height $1$. For $\p\in P(A)$ we shall denote by $v_\p$ the normed essential valuation corresponding to $\p$. Recall that the ring of $v_\p$ is $A_\p$. We shall denote by $\mathfrak{P}(A)$ the subgroup of $\mathfrak{D}(A)$ consisting of the principal divisors and by $\mathfrak{C}(A)=\mathfrak{D}(A)/\mathfrak{P}(A)$ the divisor class group of $A$.
\begin{proposition}\label{Krull domain divisor map i prop}
Let $A$ be a Krull domain and $B$ a Krull domain containing $A$. Suppose that the following condition holds
\begin{enumerate}[leftmargin=40pt]
\item[(PD)] For every prime ideal $\mathfrak{P}$ of $B$ of height $1$, the prime ideal $\mathfrak{P}\cap A$ is zero or of height $1$.
\end{enumerate}
For $\p\in P(A)$ the $\mathfrak{P}\in P(B)$ such that $\mathfrak{P}\cap A=\p$ are finite in number. We write
\[i(\p)=\sum_{\mathfrak{P}\in P(B),\mathfrak{P}\cap A=\p}e(\mathfrak{P}/\p)\mathfrak{P},\]
where $e(\mathfrak{P}/\p)$ denotes the ramification index of $v_{\mathfrak{P}}$ over $v_\p$. Then $i$ defines, by linearity, an increasing homomorphism of $\mathfrak{D}(A)$ to $\mathfrak{D}(B)$, which enjoys the following properties:
\begin{itemize}
\item[(a)] for every non-zero element $x$ of the field of fractions of $A$,
\[i(\div_A(x))=\div_B(x),\] 
\item[(b)] for all $D_1,D_2$ in $\mathfrak{D}(A)$,
\[i(\sup\{D_1,D_2\})=\sup\{i(D_1),i(D_2)\}.\] 
\end{itemize}
\end{proposition}
\begin{proof}
Let $\p\in P(A)$, consider a non-zero element $a$ of $\p$; the $\mathfrak{P}\in P(B)$ which contain $a$ are finite in number by \cref{Krull domain iff prime of height 1}. A fortiori the $\mathfrak{P}\in P(B)$ such that $\mathfrak{P}\cap A=\p$ are finite in number. We now show (a). By additivity, it may be assumed that $x\in A-\{0\}$. By definition, $\div_B(x)=\sum_{\mathfrak{P}\in P(B)}v_{\mathfrak{P}}(x)\mathfrak{P}$. For all $\mathfrak{P}\in P(B)$ such that $v_{\mathfrak{P}}(x)>0$, $\mathfrak{P}\cap A$ is non-zero (for $x$ is in it) and is therefore of height $1$ by (PD). Setting $\p=\mathfrak{P}\cap A$, by definition of the ramification index, $v_{\mathfrak{P}}(x)=e(\mathfrak{P}/\p)v_\p(x)$ (since $v_{\mathfrak{P}}$ and $v_\p$ are normed). As $\div_A(x)=\sum_{\p\in P(A)}v_\p(x)\p$, and $i(\q)=0$ for all $\q\in P(A)$ wich is not of the form $\mathfrak{Q}\cap A$ for some $\mathfrak{Q}\in P(B)$, we deduce (a).\par
To prove (b) we write
\[D_1=\sum_{\p\in P(A)}n_1(\p)\p,\quad D_2=\sum_{\p\in P(A)}n_2(\p)\p\]
the coefficient of $\p$ in $\sup\{D_1,D_2\}$ is $\sup\{n_1(\p),n_2(\p)\}$. Let $\mathfrak{P}$ be an element of $P(B)$. If $\mathfrak{P}\cap A=(0)$, the coefficients of $\mathfrak{P}$ in $i(D_1)$, $i(D_2)$, and hence also in $\sup\{i(D_1),i(D_2)\}$ are zero. Also the coefficient of $\mathfrak{P}$ in $i(\sup\{D_1,D_2\})$ is zero. If $\mathfrak{P}\cap A\neq(0)$, it is a prime ideal $\p$ of height $1$ (by (PD)); writing $e=e(\mathfrak{P}/\p)$, the coeflicients of $\mathfrak{P}$ in $i(D_1)$, $i(D_2)$ and $i(\sup\{D_1,D_2\})$ are respectively $en_1(\p)$, $en_2(\p)$ and $e\sup\{n_1(\p),n_2(\p)\}$. That of $\sup\{i(D_1),i(D_2)\}$ is
\[\sup\{en_1(\p),en_2(\p)\}=e\sup\{n_1(\p),n_2(\p)\}.\]
This proves (b).
\end{proof}
Under the hypotheses of \cref{Krull domain divisor map i prop}, it follows from (a) that $i$ defines, by taking quotients, a canonical homomorphism $\bar{i}$ of $\mathfrak{C}(A)$ to $\mathfrak{C}(B)$.
\begin{example}
Let $A$, $B$ be Krull domains and $B$ be integral over $A$. In this case, for the prime ideal $\mathfrak{P}$ of $B$ to be of height $1$, it is necessary and sufficient that $\p=\mathfrak{P}\cap A$ be of height $1$ (\cref{integral ring extension prime lying over same contraction}), so the condition (PD) is satisfied.
\end{example}
\begin{proposition}\label{Krull domain flat extension divisor prop}
Let $A$ and $B$ be Krull domains such that $B$ contains $A$ and is a flat $A$-module. Then:
\begin{itemize}
\item[(a)] the condition (PD) is satisfied;
\item[(b)] for every divisorial ideal $\a$ of $A$, $B\a$ is the divisorial ideal of $B$ which corresponds to the divisor $i(\div_A(\a))$.
\end{itemize}
\end{proposition}
\begin{proof}
To show (a), suppose that there exists a prime ideal $\mathfrak{P}$ of $B$ of height $1$ such that $\mathfrak{P}\cap A$ is neither $0$ nor of height $1$. Take an elemcnt $x\neq 0$ in $\mathfrak{P}\cap A$. The ideals $\p_i$ of $A$ of height $1$ which contain $x$ are finite in number and none contains $\mathfrak{P}\cap A$. There therefore exists an element $y$ of $\mathfrak{P}\cap A$ such that $y\notin\p_i$ for all $i$. Thus $\div_A(x)$ and $\div_A(y)$ are relatively prime elements of the ordered group $\mathfrak{D}(A)$, so that
\[\sup\{\div_A(x),\div_A(y)\}=\div_A(x)+\div_A(y)=\div_A(xy).\]
As $\sup\{\div_A(x),\div_A(y)\}=\div(Ax\cap Ay)$ and the ideals $Ax\cap Ay$ and $Axy$ are divisorial, we deduce that $Ax\cap Ay=Axy$. Since $B$ is a flat $A$-module, this implies $Bx\cap By=Bxy$ (\cref{module extension to faithfully flat ring submodule prop}), and therefore
\[\sup\{v_{\mathfrak{P}}(x),v_{\mathfrak{P}}(y)\}=v_{\mathfrak{P}}(Bx\cap By)=v_{\mathfrak{P}}(xy)=v_{\mathfrak{P}}(x)+v_{\mathfrak{P}}(y).\]
Since $v_{\mathfrak{P}}(x)$ and $v_{\mathfrak{P}}(y)$ are positive integers (which hold since $x$ and $y$ are in $\mathfrak{P}$), this is a contradiction. Thus (a) has been proved by reductio ad absurdum.\par
We now show (b). If $\a$ is a divisorial ideal of $A$, it is the intersection of two fractional principal ideals (\cref{Krull domain divisorial ideal iff intersection of two}), say
\[\a=d^{-1}(Aa\cap Ab)\]
where $a$, $b$, $d$ are in $A$ and nonzero. As $B$ is flat over $A$, $Ba=d^{-1}(Ba\cap Bb)$, which shows that $B\a$ is divisorial. This shows also that $\div_B(B\a)=\sup\{\div_B(a),\div_B(b)\}-\div_B(d)$, using \cref{Krull domain divisor map i prop} (a) and (b), it is seen that
\begin{align*}
\div_B(B\a)&=\sup\{i(\div_A(a)),i(\div_A(b))\}-i(\div_A(d))\\
&=i(\sup\{\div_A(a),\div_A(b)\})-i(\div_A(d))\\
&=i(\div_A(Aa\cap Ab))-i(\div_A(d))\\
&=i(\div_A(d^{-1}(Aa\cap Ab)))=i(\div_A(a))
\end{align*}
which completes the proof.
\end{proof}
\begin{corollary}\label{Krull local domain flat DVR extension is DVR}
Let $A$ be a local Krull domain and $B$ a DVR such that $B$ dominates $A$ and is a flat $A$-module. Then $A$ is a field or a DVR.
\end{corollary}
\begin{proof}
Let $\mathfrak{M}$ be the maximal ideal of $B$. By (PD), $\mathfrak{M}\cap A$ is zero or of height $1$. As it is, by hypothesis the maximal ideal of $A$, our assertion follows from \cref{Krull local domain DVR iff}.
\end{proof}
\begin{remark}
As the elements of $P(B)$ form a basis of $\mathfrak{D}(B)$ and two distinct ideals of $P(A)$ cannot be the traces on $A$ of the same ideal of $P(B)$, for the injectivity of $i$ it amounts to verifying that $i(\p)\neq 0$ for all $\p\in P(A)$. Now this is equivalent to say every $\p\in P(A)$ is the contraction of an element in $P(B)$, and in particular this holds if $B$ is a faithfully flat $A$-algebra or $B$ is a superring of $A$ and integral over $A$.
\end{remark}
In what follows, we propose to study the canonical homomorphism $i$ from $\mathfrak{C}(A)$ to $\mathfrak{C}(B)$ for certain ordered pairs of Krull domains $A,B$.
\begin{proposition}\label{Krull domain completion and divisor class}
Let $A$ be a Zariski ring such that its completion $\widehat{A}$ is a Krull domain. Then $A$ is a Krull domain and the canonical homomorphism $\bar{i}$ from $\mathfrak{C}(A)$ to $\mathfrak{C}(\widehat{A})$ (which is defined since $\widehat{A}$ is a flat $A$-module) is injective.
\end{proposition}
\begin{proof}
As $\widehat{A}$ is an integral domain and $A\sub\widehat{A}$, $A$ is an integral domain. Let $L$ be the field of fractions of $\widehat{A}$ and $K\sub L$ that of $A$. As $A=\widehat{A}\cap K$ (\cref{Zariski ring completion fraction field prop}), $A$ is a Krull domain. The fact that $\bar{i}:\mathfrak{C}(A)\to\mathfrak{C}(\widehat{A})$ is injective follows from \cref{Krull domain flat extension divisor prop} and the fact that, if $\b\widehat{A}$ is principal, $\b$ is principal (\cref{Zariski ring completion principal then prncipal})
\end{proof}
Now let $A$ be a Krull domain and $S$ a multiplicative subset of $A$ not containing $0$. The group $\mathfrak{D}(A)$ (resp. $\mathfrak{D}(S^{-1}A)$ is the free commutative group with basis the set of $\div(\p)$ (resp. $\div(S^{-1}\p)$), where $\p$ runs through the set of prime ideals of $A$ of height $1$ (resp. the set of prime ideals of $A$ of height $1$ such that $\p\cap S=\emp$) and, if $\p\cap S=\emp$, then $i(\div(\p))=\div(S^{-1}\p)$. Thus $\mathfrak{D}(S^{-1}A)$ is identified with the direct factor of $\mathfrak{D}(A)$ generated by the elements $\div(\p)$ such that $\p\cap S=\emp$ and admits as complement the free commutative subgroup of $\mathfrak{D}(A)$ with basis the set of $\div(\p)$ such that $\p\cap S\neq\emp$. We shall denote this complement by $\mathfrak{G}(S)$. As the map $i:\mathfrak{D}(A)\to\mathfrak{D}(S^{-1}A)$ is suijective, so is the induced map $\bar{i}:\mathfrak{C}(A)\to\mathfrak{C}(S^{-1}A)$ and
\begin{align}\label{Krull domain divisor group map to localization kernel}
\ker\bar{i}=(\mathfrak{G}(S)+\mathfrak{P}(A))/\mathfrak{P}(A)=\mathfrak{G}(S)/(\mathfrak{G}(S)\cap\mathfrak{P}(A)).
\end{align}
In fact, if an element of $\mathfrak{D}(S^{-1}A)$ is equal to $\div_{S^{-1}A}(x/s)$, where $x\in A$ and $s\in S$, it is then the image under $i$ of the principal divisor $\div_A(x/s)$ (\cref{Krull domain divisor map i prop}).
\begin{proposition}\label{Krull domain divisor ideal of localization}
Let $A$ be a Krull domain and $S$ a multiplicative subset of $A$ not containing $0$. Then the canonical homomorphism $\bar{i}$ from $\mathfrak{C}(A)$ to $\mathfrak{C}(S^{-1}A)$ is suijective. If further $S$ is generated by a family of elements $p_i$ such that the principal ideals $Ap_i$ are all prime, then $\bar{i}$ is bijective.
\end{proposition}
\begin{proof}
Suppose now that $S$ is generated by a family of elements $(p_i)_{i\in I}$ of $A$ such that the principal ideals $Ap_i$ are all prime. Then, if $\p$ is a prime ideal of $A$ of height $1$ such that $p\cap S\neq\emp$, $\p$ contains a product of powers of the $p_i$ and therefore one of the $p_i$, say $p_0$. As $Ap_0$ is non-zero and prime and $\p$ is of height $1$, it follows that $\p=Ap_0$. In the above notation, we then have $\mathfrak{G}(S)\sub\mathfrak{P}(A)$ and the kernel of $\bar{i}$ is zero.
\end{proof}
\begin{proposition}\label{Krull domain divisor of A[X]}
Let $R$ be a Krull domain and consider the polynomial ring $R[X]$. The canonical homomorphism of $\mathfrak{C}(R)$ to $\mathfrak{C}(R[X])$ is bijective.
\end{proposition}
\begin{proof}
Let $R$ be a Krull domain; take $A$ to be the polynomial ring $A=R[X]$ and $S$ to be the set $R-(0)$ of non-zero constant polynomials of $A$. The prime ideals $\p$ of $A$ of height $1$ such that $\p\cap S\neq\emp$ are those of the form $\p_0A$, where $\p_0$ is a prime ideal of $R$ of height $1$. Hence, in the notation introduced above, $\mathfrak{G}(S)$ is identified with $\mathfrak{D}(R)$ by identifying $\div_A(\p_0A)$ with $\div_R(\p_0)$. On the other hand $\mathfrak{G}(S)\cap\mathfrak{P}(A)$ is identified with $\mathfrak{P}(R)$: for if an ideal $\a_0$ of $R$ generates a principal ideal $f(X)A$ in $A=R[X]$, then $f(0)\in\a_0A$ since $\a_0A$ is a graded ideal of the ring $A$ (graded by the usual degree of polynomials) and hence $f(0)\in\a_0$. Further, for any $a\in\a_0$ we have $a=f(X)g(X)$ where $g(X)\in R$, whence $a=f(0)g(0)$. It follows that $\a_0$ is the principal ideal of $R$ generated by $f(0)$. Finally, denoting by $K$ the field of fractions of $R$, $S^{-1}A$ is identified with the polynomials ring $K[X]$, which is a principal ideal domain; hence $\mathfrak{C}(S^{-1}A)=(0)$. Thus $\mathfrak{C}(A)=\ker\bar{i}$ is identified with $\mathfrak{C}(R)$ and we have proved the proposition.
\end{proof}
\subsection{Dedekind domains}
Let $A$ be an integral domain. Clearly the following conditions are equivalent:
\begin{itemize}
\item[(\rmnum{1})] no two of the non-zero prime ideals of $A$ are comparable with respect to inclusion;
\item[(\rmnum{2})] the non-zero prime ideals of $A$ are maximal;
\item[(\rmnum{3})] the non-zero prime ideals of $A$ are of height $1$.
\end{itemize}
A Krull domain which satisfies the above equivalent conditions is called a \textbf{Dedekind domain}.
\begin{example}[\textbf{Examples of Dedekind domains}]
\mbox{}
\begin{itemize}
\item[(a)] Every principal ideal domain is a Dedekind domain.
\item[(b)] Let $K$ be a finite extension of $\Q$ and $A$ the integral closure of $\Z$ over $K$. The ring $A$ is a Krull domain by \cref{Krull domain integral closure is Krull}. Let $\p$ be a non-zero prime ideal of $A$. Then by the going down theorem $\p\cap Z$ is non-zero and hence is a maximal ideal of $\Z$. Hence $\p$ is a maximal ideal of $A$. Therefore, $A$ is a Dedekind domain. In general, $A$ is not a principal ideal domain.
\item[(c)] Let $V$ be an affine algebraic variety and $A$ the ring of functions regular on $V$. Suppose that $A$ is not a field (i.e. that $V$ is not reduced to a point). For $A$ to be a Dedekind domain, it is necessary and sufficient that $V$ be an irreducible curve with no singular point: for to say that $A$ is an integral domain amounts to saying that $V$ is irreducible; to say that every non-zero prime ideal of $A$ is maximal amounts to saying that $\dim A=1$, so $A$ is a curve; finally, as $A$ is Noetherian, to say that it is a Krull domain amounts to saying that it is integrally closed, that is, $V$ is a normal curve, or also that it has no singular point.
\item[(d)] A ring of fractions $S^{-1}A$ of a Dedekind domain $A$ is a Dedekind domain if $0\notin S$. For $S^{-1}A$ is a Krull domain and every nonzero prime ideal of $S^{-1}A$ is maximal. 
\end{itemize}
\end{example}
\begin{theorem}\label{Dedekind domain iff}
Let $A$ be an integral domain and $K$ its field of fractions. Then the following conditions are equivalent:
\begin{itemize}
\item[(\rmnum{1})] $A$ is a Dedekind domain;
\item[(\rmnum{2})] $A$ is a Krull domain and every nontrivial valuation on $K$ which is positive on $A$ is equivalent to an essential valuation of $A$;
\item[(\rmnum{3})] $A$ is a Krull domain and every nonzero fractional ideal $\a$ of $A$ is divisorial;
\item[(\rmnum{4})] every nonzero fractional ideal $\a$ of $A$ is invertible;
\item[(\rmnum{5})] $A$ is a Noetherian integrally closed domain and every non-zero prime ideal of $A$ is maximal;
\item[(\rmnum{6})] $A$ is Noetherian and, for every maximal ideal $\m$ of $A$, $A_\m$ is either a field or a DVR;
\item[(\rmnum{7})] $A$ is Noetherian and, for every maximal ideal $\m$ of $A$, $A_\m$ is a PID.
\end{itemize}
\end{theorem}
\begin{proof}
We show first the equivalence of (\rmnum{1}) and (\rmnum{2}). \cref{Krull domain valuation equivalent to essential if} shows immediately that (\rmnum{1}) implies (\rmnum{2}). Conversely, (\rmnum{2}) implies (\rmnum{1}), since, for every prime ideal $\p$ of $A$, there exists a valuation ring of $K$ which dominates $A_\p$, and (\rmnum{2}) this implies that $\p$ has height $1$, so $A$ is a Dedekind domain.\par
If $A$ is a Dedekind domain and $\a$ is a non-zero fractional ideal, then $\a A_\m=\tilde{\a}A_\m$ for every maximal ideal $\m$ (by \cref{Krull domain coefficient of P is inf}) and hence $\a=\tilde{\a}$; thus (\rmnum{1}) implies (\rmnum{3}).\par
We now show that (\rmnum{3}) implies (\rmnum{4}). If (\rmnum{3}) holds, the map $\a\mapsto\div(\a)$ is a bijection of $\mathfrak{F}(A)$ onto $\mathfrak{D}(A)$, as it is a homomorphism and $\mathfrak{D}(A)$ is a group, every element of $\mathfrak{F}(A)$ is invertible.\par
We show that (\rmnum{4}) implies (\rmnum{5}). If (\rmnum{4}) holds, every integral ideal of $A$ is finitely generated and hence $A$ is Noetherian. As $\mathfrak{F}(A)$ is a group, $\mathfrak{D}(A)$ is a group and $A$ is therefore completely integrally closed. Finally, if $\p$ is a non-zero prime ideal of $A$ and $\m$ is a maximal ideal of $A$ containing $\p$, the ring $A_\m$ is a PID by \cref{nondegenerate submodule invertible iff}. As $\p A_\m$ is prime and non-zero, necessarily $\p A_\m=\m A_\m$, whence $\p=\m$ and $\p$ is maximal.\par
We now show that (\rmnum{5}) implies (\rmnum{6}). If $\m$ is a maximal ideal of $A$ and (\rmnum{5}) holds, $A_\m$ is an integrally closed Noetherian domain and its maximal ideal $\m A_\m$ is, either $(0)$, or the only non-zero prime ideal of $A_\m$. Hence $A_\m$ is a field or a DVR by \cref{Krull local domain DVR iff}.\par
The fact (\rmnum{6}) implies (\rmnum{7}) is obvious. We show finally that (\rmnum{7}) implies (\rmnum{1}). As $A$ is the intersection of the $A_\m$ where $\m$ runs through the set of maximal ideals (\cref{integral domain inter of localization}), (\rmnum{7}) implies that $A$ is integrally closed and Noetherian and hence that $A$ is a Krull domain. On the other hand, it can be shown that every non-zero prime ideal of $A$ is maximal as in the proof that (\rmnum{4}) implies (\rmnum{5}).
\end{proof}
\begin{proposition}\label{Dedeking domain semilocal is PID}
A semi-local Dedekind domain is a PID.
\end{proposition}
\begin{proof}
Let $A$ be a semi-local Dedekind domain, $K$ its field of fractions, $\m_1,\dots,\m_n$ its maximal ideals and $v_1,\dots,v_n$ the corresponding essential valuations. These are the only essential valuations of $A$. Let $\a$ be a non-zero integral ideal of $A$. Since it is divisorial, there exists integers $q_1,\dots,q_r$ such that $\a$ is the set of $x\in K$ such that $v_i(x)\geq\q_i$ for all $i$ (\cref{Krull domain Noetherian primary decomposition}). Let $x_0$ be an element of $K$ such that $v_i(x_0)=q_i$ for all $i$ (\cref{Krull domain approximation thm}). Then $\a$ is the set of $x\in K$ such that $v_i(xx_0^{-1})\geq 0$ for all $i$ and thus $\a=Ax_0$.
\end{proof}
If $A$ is a Dedekind domain, it has been seen, in \cref{Dedekind domain iff}, that the group $\mathfrak{D}(A)$ of divisors of $A$ is identified with the group $\mathfrak{F}(A)$ of nonzero fractional ideals $\a$ (as $A$ is Noetherian, every non-zero fractional ideal is finitely generated). The divisor class group $\mathfrak{C}(A)$ of $A$ is then identified with the group of class of nonzero ideals of $A$.\par
Let $A$ be a Dedekind domain, $\mathfrak{F}(A)$ the ordered multiplicative group of nonzero fractional ideals of $A$ and $\mathfrak{D}(A)$ the group of divisors of $A$. The isomorphism $\a\mapsto\div(\a)$ of $\mathfrak{F}(A)$ onto $\mathfrak{D}(A)$ maps the non-zero prime ideals of $A$ to irreducible divisors and hence the multiplicative group $\mathfrak{F}(A)$ admits as basis the set of non-zero prime ideals of $A$. In other words, every non-zero fractional ideal $\a$ of $A$ admits a unique decomposition of the form:
\begin{align}\label{Dedekind domain decomposition of fractional ideal}
\a=\prod\p^{n_\p}
\end{align}
where the product extends to the non-zero prime ideals of $A$, and the exponents $n_\p$ equals $v_\p(\a)$, where $v_\p$ denotes the essential valuation corresponding to $\p$. Further $\a$ is integral if and only if the $n_\p$ are all positive. The relation (\ref{Dedekind domain decomposition of fractional ideal}) is called the \textbf{decomposition of $\a$ into prime factors}. For $\a,\b\in\mathfrak{F}(A)$, write
\[\a=\prod\p^{n_\p},\quad \b=\prod\p^{m_\p}\]
then we have
\[\a\b=\prod\p^{n_\p+m_\p},\quad(\a:\b)=\a\b^{-1}=\prod\p^{n_\p-m_\p},\]
\[\a+\b=\prod\p^{\inf\{n_\p,m_\p\}},\quad\a\cap\b=\prod\p^{\sup\{n_\p,m_\p\}}.\]
\begin{proposition}\label{Dedekind domain approximation thm}
Let $A$ be a Dedekind domain, $K$ its field of fractions and $\mathcal{P}$ the set of non-zero prime ideals of $A$. For $\p\in\mathcal{P}$ let $v_\p$ denote the corresponding essential valuation of $A$. Let $\p_1,\dots,\p_r$ be distinct elements of $\mathcal{P}$ and $n_1,\dots,n_r$ integers and $x_1,\dots,x_r$ elements of $K$. Then there exists $x\in K$ such that $v_{\p_i}(x-x_i)\geq n_i$ for all $i$ and $v_\p(x)\geq 0$ for all $\p\in\mathcal{P}$ distinct from the $\p_i$.
\end{proposition}
\begin{proof}
Replacing if need be the $n_i$ by greater integers, they may be assumed all to be positive. We examine first the case where the $x_i$ are in $A$. It obviously amounts to finding an $x\in A$ satisfying the congruences
\[x\equiv x_i\mod\p_i^{n_i}\]
and the existence of $x$ then follows from Chinese remainder theorem.\par
We pass now to the general case. We may write $x_i=y_i/s$, where $y_i,s$ are in $A$. Writing $x=y/s$, it amounts to finding a $y\in A$ such that, on the one hand, $v_{\p_i}(y-y_i)\geq n_i+v_{\p_i}(s)$ and, on the other, $v_\p(y)\geq v_\p(s)$ for all $p\in\mathcal{P}$ distinct from the $\p_i$. As $v_\p(s)=0$ except for a finite number of indices $\p$, it is thus reduced to the above case, whence the proposition.
\end{proof}
\begin{proposition}\label{Dedekind domain ideal 2-generated}
Let $A$ be a Dedekind domain and $\a$ a fractional ideal of $A$. Then $\a$ is generated by at most two elements.
\end{proposition}
\begin{proof}
By multiplying by an element of $A$, we can reduce to the case that $\a$ is an ideal of $A$. Write $\a=\prod\p_i^{n_i}$ as a finite product and choose $a\in\a$ with $aA=\prod\q_j^{m_j}\prod\p_i^{f_i}$ and the $\q_j$ different from $\p_i$. Then since $aA\sub\a$ we have $f_i\geq n_i$ for each $i$. Choose $b\in A$ such that $v_{\p_i}(b)=n_i$ and $v_{\q_j}(b)=0$, then we see
\[aA+bA=\prod\p_i^{\inf\{f_i,f_i+1\}}\q_j^{\inf\{0,m_j\}}=\prod\p_i^{n_i}=\a.\]
This proves the claim.
\end{proof}
\subsection{The Krull-Akizuki theorem}
\begin{lemma}\label{Noe domain dim 1 finite tosion module length finite}
Let $A$ be a one-dimensional Noetherian domain and $M$ a finitely generated torsion $A$-module. Then the length $\ell_A(M)$ of $M$ is finite.
\end{lemma}
\begin{proof}
As $M$ is a torsion module, every prime ideal associated with $M$ is nonzero and therefore maximal. The lemma then follows from \cref{associated prime maximal iff finite length}.
\end{proof}
\begin{lemma}\label{module length direct limit prop}
Let $A$ be a ring, $M$ an $A$-module and $(M_i)_{i\in I}$ a directed family of submoduler of $M$ with union $M$. Then $\ell_A(M)=\sup_i\ell_A(M_i)$.
\end{lemma}
\begin{proof}
We have $\ell_A(M_i)\leq\ell_A(M)$ for all $i$. The lemma is obvious if no integer exceeds the $\ell_A(M_i)$, both sides then being infinite. Otherwise, let $i_0$ be an index for which $\ell_A(M_{i_0})$ takes its greatest value. Then $M=M_{i_0}$ since the family $(M_i)$ is directed, whence our assertion in this case.
\end{proof}
\begin{lemma}\label{Noe domain dim 1 quotient module length inequality}
Let $A$ be a one-dimensional Noetherian domain and $M$ a torsion-free $A$-module of rank $r<\infty$. Then for every nonzero $a\in A$, we have
\begin{align}\label{Noe domain dim 1 quotient module length inequality-1}
\ell_A(M/aM)\leq r\cdot\ell_A(A/aA).
\end{align}
\end{lemma}
\begin{proof}
\cref{Noe domain dim 1 finite tosion module length finite} shows that $\ell_A(A/Aa)$ is finite. We show (\ref{Noe domain dim 1 quotient module length inequality-1}) first in the case where $M$ is finitely generated. As $M$ is torsion-free and of rank $r$, there exists a submodule $L$ of $M$ which is isomorphic to $A^r$ and such that $Q=M/L$ is a finitely generated torsion $A$-module and hence of finite length (\cref{Noe domain dim 1 finite tosion module length finite}). For every integer $n\geq 1$, we have an exact sequence
\[\begin{tikzcd}
0\ar[r]&L/(a^nM\cap L)\ar[r]&M/a^nM\ar[r]&Q/a^nQ\ar[r]&0
\end{tikzcd}\]
as $a^nL\sub a^nM\cap L$, this implies
\begin{equation}\label{Noe domain dim 1 quotient module length inequality-2}
\ell_A(M/a^nM)\leq\ell_A(L/a^nL)+\ell_A(Q/a^nQ)\leq\ell_A(L/a^nL)+\ell_A(Q).
\end{equation}
Now, since $M$ is torsion-free, multiplication by $a$ defines an isomorphism of $M/aM$ onto $aM/a^2M$ and similarly for $L$; whence, by induction on $n$, the formulae:
\begin{align}\label{Noe domain dim 1 quotient module length inequality-3}
\ell_A(M/a^nM)=n\cdot\ell_A(M/aM),\quad \ell_A(L/a^nL)=n\cdot\ell_A(L/aL)
\end{align}
Taking account of (\ref{Noe domain dim 1 quotient module length inequality-2}), we deduce that
\begin{align}\label{Noe domain dim 1 quotient module length inequality-4}
\ell_A(M/aM)\leq\ell_A(L/aL)+n^{-1}\ell_A(Q)
\end{align}
for all $n>0$. As $L$ is isomorphic to $A^r$, we see $\ell_A(L/aL)=r\ell_A(A/aA)$, so we get (\ref{Noe domain dim 1 quotient module length inequality-1}) by letting $n$ tend to infinity in (\ref{Noe domain dim 1 quotient module length inequality-4}).\par
We now pass to the general case. Let $(M_i)$ be the family of finitely generated submodules of $M$. Then the module $T=M/aM$ is the union of the submodules $T_i=M_i/(M_i\cap aM)$. Now, $T_i$ is isomorphic to a quotient of $M_i/aM_i$ and hence
\[\ell_A(T_i)\leq r\cdot\ell_A(A/aA)\]
by what we have just proved. Whence $\ell_A(T)\leq r\cdot\ell_A(A/aA)$ by \cref{module length direct limit prop}.
\end{proof}
\begin{proposition}[\textbf{Krull-Akizuki}]\label{Krull-Akizuki thm}
Let $A$ be a one-dimensional Noetherian domain, $K$ its field of fractions, $L$ a finite extension of $K$ and $B$ a subring of $L$ containing $A$. Then $B$ is a one-dimensional Noetherian domain. Moreover, for every nonzero ideal $\b$ of $B$, $B/\b$ is an $A$-module of finite length.
\end{proposition}
\begin{proof}
Let $\b$ be a non-zero ideal of $B$. We shall show that $B/\b$ is an $A$-module of finite length (hence, a fortiori, a $B$-module of finite length) and that $\b$ is a finitely generated $B$-module. A non-zero element $y$ of $\b$ satisfies an equation of the form:
\[a_ny^n+a_{n-1}y^n+\cdots+a_1y+a_0=0,\quad a_i\in A,a_0\neq 0.\]
This equation shows that $a_0\in By\sub\b$. Applying \cref{Noe domain dim 1 quotient module length inequality} to $M=B$, it is seen that $B/a_0B$ is an $A$-module of finite length, and so is $B/\b$ which is a quotient module of it. Further the $B$-module $\b$ contains, as a submodule, $a_0B$ which is finitely generated. As $\b/a_0B$ is of finite length (as a submodule of $B/a_0B$) and hence finitely generated, $\b$ is certainly a finitely generated $B$-module.\par
The above shows first that $B$ is Noetherian. On the other hand, if $\mathfrak{P}$ is a nonzero prime ideal of $B$, the ring $B/\mathfrak{P}$ is an integral domain and of finite length and hence is a field, so that $\mathfrak{P}$ is maximal.
\end{proof}
\begin{corollary}\label{Krull-Akizuki prime ideal lying over is finite}
Let $A$ be a one-dimensional Noetherian domain, $K$ its field of fractions, $L$ a finite extension of $K$ and $B$ a subring of $L$ containing $A$. Then for every prime ideal $\p$ of $A$, the set of prime ideals of $B$ lying over $\p$ is finite.
\end{corollary}
\begin{proof}
Suppose first that $\p=(0)$. Then the only prime ideal $\mathfrak{P}$ of $B$ such that $\mathfrak{P}\cap A=(0)$ is $(0)$: otherwise, writing $S=A-\{0\}$, then $S^{-1}\mathfrak{P}$ would be a non-zero prime ideal of $S^{-1}B$ and $S^{-1}B$ is just the field of fractions of $B$, for it is a subring of $L$ containing $K$, whence an absurd conclusion. If $\p$ is nonzero, it follows from \cref{Krull-Akizuki thm} that $B/\p B$ is a finite-dimensional vector space over the field $A/\p$, hence an Artinian ring and therefore has only a finite number of prime ideals, which proves that there is only a finite number of prime ideals of $B$ containing $\p$.
\end{proof}
\begin{corollary}\label{Krull-Akizuki integral closure is Dedekind}
Let $A$ be a one-dimensional Noetherian domain, $K$ its field of fractions, $L$ a finite extension of $K$ and $B$ a subring of $L$ containing $A$. Then the integral closure of $A$ in $L$ is a Dedekind domain. In particular, the integral closure of a Dedekind domain in a finite extension of its field of fractions is a Dedekind domain.
\end{corollary}
\begin{proof}
This integral closure is a one-dimensional integrally closed Noetherian domain, hence Dedekind.
\end{proof}
\begin{proposition}\label{Dedekind domain decomposition ideal in integral closure}
Let $A$ be a Dedekind domain, $K$ its field of fractions, $L$ a finite extension of $K$ and $B$ the integral closure of $A$ in $L$. Let $\p$ be a non-zero prime ideal of $A$, $v_\p$ the corresponding essential valuation of $K$ and
\[B\p=\prod_i\mathfrak{P}_i^{e_i}\]
the decomposition of the ideal $B\p$ as a product of prime ideals. Then:
\begin{itemize}
\item[(a)] the prime ideals of $B$ lying over $\p$ are the $\mathfrak{P}_i$ such that $e_i>0$.
\item[(b)] the valuations $v_i$ on $L$ corresponding to these ideals $\mathfrak{P}_i$ are, up to equivalence, the valuations on $L$ extending $v_\p$.
\item[(c)] $[B/\mathfrak{P}_i:A/\p]=f(v_i/v)$ and $e_i=e(v_i/v)$.   
\end{itemize}
\end{proposition}
\begin{proof}
To say that a prime ideal $\mathfrak{P}$ of $B$ lies over $\p$ amounts to saying that $\mathfrak{P}\sups\p$, hence that $\mathfrak{P}\sups B\p$ and that $\mathfrak{P}$ contains one of the $\mathfrak{P}_i$, such that $e_i>0$. Now part (b) follows from \cref{Krull domain valuation equivalent to essential if}. Finally, the residue field of $v$ is identified with $A/\p$ and that of $v_i$ with $B/\mathfrak{P}_i$, hence the first claim of (c). Let $t$ (resp. $t_i$) be a uniformizer for $v$ (resp. $v_i$). Then
\begin{align*}
tB_{\mathfrak{P}_i}&=tA_\p B_{\mathfrak{P}_i}=\p B_{\mathfrak{P}_i}=\Big(\prod_i\mathfrak{P}_j^{e_j}\Big)B_{\mathfrak{P}_i}=\prod_j(\p_jB_{\mathfrak{P}_i})^{e_j}=(\p_iB_{\mathfrak{P}_i})^{e_i}=t_i^{e_i}B_{\mathfrak{P}_i}
\end{align*}
since $\p_jB_{\mathfrak{P}_i}=B_{\mathfrak{P}_i}$ for $j\neq i$, whence the second claim of (c), since $e(v_i/v)=v_i(t)$.
\end{proof}
\subsection{Unique factorization domains}
A Krull domain all of whose divisorial ideals are principal is called a \textbf{factorial} or \textbf{unique factorization domain}. In other words, $A$ is factorial if and only if $\mathfrak{C}(A)$ is reduced to $0$.
\begin{example}
\mbox{}
\begin{itemize}
\item[(a)] Every PID is factorial (and, recall, is a Dedekind domain). Conversely, every factorial Dedekind domain is a PID by \cref{Dedekind domain iff}.
\item[(b)] In particular, if $K$ is a field, the rings $K[X]$ and $K\llbracket X\rrbracket$ are factorial domains.
\item[(c)] The local ring of a simple point of an algebraic variety is a factorial domain. The ring of germs of functions analytic at the origin of $\C^n$ is a factorial domain.
\end{itemize}
\end{example}
\begin{theorem}\label{factorial domain iff}
Let $A$ be an integral domain. Thefollowing conditions are equivalent
\begin{itemize}
\item[(\rmnum{1})] $A$ is factorial; 
\item[(\rmnum{2})] the ordered group of non-zero fractional principal ideal of $A$ is a direct sum of groups isomorphic to $\Z$ (ordered by the product order).
\item[(\rmnum{3})] Every non-empty family of integral principal ideals of $A$ has a maximal element and the intersection of two principal ideals of $A$ is a principal ideal; 
\item[(\rmnum{4})] Every non-empty family of integral principal ideals of $A$ has a maximal element and every irreducible element of $A$ is prime;
\item[(\rmnum{5})] $A$ is a Krull domain and every prime ideal of height $1$ is principal.
\end{itemize}
\end{theorem}
\begin{proof}
We shall denote by $K$ the field of fractions of $A$ and by $\mathscr{P}$ (or $\mathscr{P}(A)$) the ordered group of non-zero fractional principal ideals of $A$. We show that (\rmnum{1}) implies (\rmnum{2}). If $A$ is factorial, $\mathscr{P}$ is isomorphic to the group of divisors of $A$ and hence to a direct sum of groups $\Z$.\par
Note now that the relation "the intersection of two integral principal ideals of $A$ is a principal ideal" means that every ordered pair ofelements of $A$ admits a lcm, that is, $\mathscr{P}$ is a lattice-ordered group. The fact that (\rmnum{2}) implies (\rmnum{3}) (and even is equivalent to it) therefore follows from (A, \Rmnum{6}, $\S$1, no.13, Theorem 2). The fact that (\rmnum{3}) implies (\rmnum{4}) follows from (A, \Rmnum{6}, $\S$1, no.13, Proposition 14), and the fact that (\rmnum{4}) implies (\rmnum{2}) follows from (A, \Rmnum{6}, $\S$1, no.13, Theorem 2) applied to the group $\mathscr{P}$.\par
We show that (\rmnum{2}) implies (\rmnum{5}). If (\rmnum{2}) holds, there is an isomorphism of $\mathscr{P}$ onto $\Z^{\oplus I}$. Let $(v_i(x))_{i\in I}$ denote the element of $\Z^{\oplus I}$ corresponding to the ideal $Ax$. It is seen immediately that each $v_i$ is a discrete valuation on $K$, that $A$ is the intersection of the rings of the vI and that, for $x\neq 0$ in $A$, $v_i(x)\neq 0$ for a finite number of indices $i$. Hence $A$ is a Krull domain. On the other hand, let $\p$ be a prime ideal of $A$ of height $1$; it contains a non-zero element $a$ which is necessarily not invertible and hence also (by definition of a prime ideal) one of the irreducible elements $p$ of $A$. As $p$ is prime and non zero, $\p=p$, which proves that $\p$ is principal.\par
Finally we show that (\rmnum{5}) implies (\rmnum{1}). Let $\a$ be a divisorial ideal of $A$. There exist prime ideals $\p_i$ of $A$ of height $1$ such that $\div(\a)=\sum_in_i\div(\p_i)$ where $n_i\in\Z$. If (\rmnum{5}) holds, $\p_i$ is of the form $Ap_i$ whence $\div(\a)=\div(Ap_i^{n_i})$ and hence $\a=\prod_iAp_i^{n_i}$ since $\a$ is divisorial.
\end{proof}
\begin{proposition}\label{Krull domain divisorial is invertible iff}
Let $A$ be a Krull domain. If every divisorial ideal of $A$ is invertible, then, for every maximal ideal $\m$ of $A$, $A_\m$ is factorial. The converse is true if it is also assumed that every divisorial ideal of $A$ is finitely generated (in particular if $A$ is Noetherian).
\end{proposition}
\begin{proof}
Suppose that every divisorial ideal of $A$ is invertible. As $A_\m$ is a Krull domain, every divisorial ideal $\a$ of $A_\m$ is the intersection of two principal fractional ideals (\cref{Krull domain divisorial ideal iff intersection of two}), hence $\a=\b A_\m$, where $\b$ is a divisorial ideal of $A$ (\cref{localization and ideal quotient}). As $\b$ is invertible by hypothesis, we deduce from \cref{nondegenerate submodule invertible iff} that $\a$ is principal and hence $A_\m$ is a factorial domain.\par
Conversely, if all the $A_\m$ are factorial and $\a$ is a finitely generated divisorial ideal of $A$, then $\a A_\m$ is a divisorial ideal of $A_\m$, as follows from \cref{Krull domain divisorial ideal iff intersection of two} and \cref{localization of sum and intersection}. By hypothesis $\a A_\m$ is principal and hence it follows from \cref{nondegenerate submodule invertible iff} that $\a$ is invertible.
\end{proof}
Let $A$ be an integral domain, $K$ its field of fractions and $U(A)$ the multiplicative group of invertible elements of $A$. Recall that there is a canonical isomorphism of $K^\times/U$ onto the group $\mathscr{P}$ of non-zero fractional principal ideals of $A$. Condition (\rmnum{2}) of \cref{factorial domain iff} may then be translated as follows:
\begin{proposition}
Let $A$ be an integral domain. For $A$ to befactorial, it is necessary and sufficient that there exist a subset $P$ of $A$ such that every $a\in A$ may be written uniquely in the form $a=u\prod_{p\in P}p^{n_p}$, where $u\in U(A)$ and the $n_p$ are positive integers which are zero exceptfor a finite number of them.
\end{proposition}
If $P$ satisfies this condition, clearly all its elements are irreducible and every irreducible element of $A$ is associated with a unique element of $P$. Recall that $P$ is then called a representative system of irreducible elements of $A$.\par
Suppose always that $A$ is factorial. It has been seen that the group $\mathscr{P}$ is a lattice. In particular, every element of $K^\times$ may be written in an essentially unique way in the form of an irreducible fraction. Any two elements $a,b$ of $K^\times$ have a gcd and a lcm. If $a=u\prod p^{n_p}$ and $b=v\prod p^{m_p}$ are decompositions of $a$ and $b$ as products of irreducible elements, then, then
\[\gcd(a,b)=w\prod p^{\inf\{n_p,m_p\}},\quad\lcm(a,b)=w\prod p^{\sup\{n_p,m_p\}}.\]
For all $p\in P$, the map $a\mapsto n_p$ is a discrete valuation $v_p$ on $K$ whose ring is obviously $A_{Ap}$. It follows from \cref{factorial domain iff} that the $v_p$ are just the essential valuations of $A$ and that the ideals $Ap$ are just the prime ideals of $A$ of height $1$.
\begin{proposition}
Let $A$ be a Krull domain and $S$ a multiplicative subset of $A$ not containing $0$.
\begin{itemize}
\item[(a)] If $A$ is factorial, $S^{-1}A$ is factorial;
\item[(b)] If $S$ is generated by a family of elements $p_i$ such that the principal ideals $Ap_i$ are prime and $S^{-1}A$ isfactorial, then $A$ is factorial.
\end{itemize}
\end{proposition}
\begin{proof}
This follows from the definition of factorial domains and \cref{Krull domain divisor ideal of localization}.
\end{proof}
Let $A$ be a factorial domain, $K$ its field of fractions and $f$ a non-zero element of $K[X]$. The \textbf{content} of $f$ is defined to be the gcd of the coefficients of $f$ (up to a unit of $A$) and denote by $\mathrm{cont}(f)$. Let $v$ be a valuation on $K$ which is essential for $A$ and $\bar{v}$ its canonical extension to $K[X]$, then
\[\bar{v}(f)=v(\mathrm{cont}(f)).\]
Moreover, it is easy to see $\mathrm{cont}(f)$ is uniquely determined by this property.
\begin{lemma}[\textbf{Guass's Lemma}]\label{factorial domain Gauss lemma}
Let $f$, $g$ be non-zero elements of $K[X]$, then
\[\mathrm{cont}(fg)=\mathrm{cont}(f)\mathrm{cont}(g).\]
\end{lemma}
\begin{proof}
For every valuation $v$ on $K$ which is essential for $A$, let $\bar{v}$ denote its canonical extension to $K[X]$. Then
\[v(\mathrm{cont}(fg))=\bar{v}(fg)=\bar{v}(f)+\bar{v}(g)=v(\mathrm{cont}(f))+v(\mathrm{cont}(g))=v(\mathrm{cont}(fg))\]
which proves the claim.
\end{proof}
Note that for every $f\in K[X]$, $f/\mathrm{cont}(f)$ is a polynomial with content $1$ (such a polynomial is called \textbf{primitive}) and we can always write
\[f=\mathrm{cont}(f)\tilde{f}\]
where $\tilde{f}$ is a primitive polynomial.  
\begin{theorem}
Let $A$ be a factorial domain, $K$ its field of fractions, $(p_i)$ a representative system of irreducible elements of $A$ and $(P_\alpha)$ a representative system of primitive irreducible polynomials of $K[X]$. Then:
\begin{itemize}
\item[(a)] $A[X]$ is a factorial domain;
\item[(b)] the set $p_i$ and $P_i$ is a representative system of irreducible elements of $A[X]$. 
\end{itemize}
\end{theorem}
\begin{proof}
Let $f$ be a non-zero element of $A[X]$. In the ring $K[X]$, $f$ can be decomposed uniquely in the form:
\[f=a\prod_\alpha P_\alpha^{n_\alpha}\]
where $a\in K^\times$ and $n_\alpha\geq 0$. \cref{factorial domain Gauss lemma} proves that $a$ is the content of $f$. Hence $a\in A$. As $A$ is factorial, $a$ can be decomposed uniquely in the form:
\[a=u\prod_ip_i^{m_i}\]
whence the existence and uniqueness of the decomposition
\[f=u\prod_ip_i^{m_i}\prod_\alpha P_\alpha^{n_\alpha}.\]
Note that this theorem proves that every element of $A$ admits the same decomposition into irreducible elements in $A$ and $A[X]$. The gcd of a family of elements of $A$ is therefore the same in $A$ and in $A[X]$.
\end{proof}
\begin{corollary}
If $A$ is a factorial domain, the domain $A[X_1,\dots,X_n]$ is factorial. Also the domain $A[X_n]_{n\in\N}$ is factorial.
\end{corollary}
\begin{proposition}\label{factorial domain completion}
Let $A$ be a Zariski ring and $\widehat{A}$ its completion. If $\widehat{A}$ is a factorial domain, $A$ is a factorial domain.
\end{proposition}
\begin{proof}
This follows from definition of factorial domain and \cref{Krull domain completion and divisor class}.
\end{proof}
\begin{proposition}\label{DVR power seires is UFD}
Let $C$ be a ring which is either a field or a DVR. Then the domain of formal power series $C\llbracket X_1,\dots,X_n\rrbracket$ is factorial.
\end{proposition}
\begin{proof}
Let $\m$ be the maximal ideal of $C$ and $\pi$ a generator of $\m$ (if $C$ is a field, then $\pi=0$). Let $C$ be given the $\m$-adic topology, which is Hausdorff. As $C$ is a Noetherian local ring, $B=C\llbracket X_1,\dots,X_n\rrbracket$ is a Noetherian local ring and its completion is $\widehat{C}\llbracket X_1,\dots,X_n\rrbracket$. By (AC, \Rmnum{7}, $\S$3, Corollary to Proposition 4, no.6), it suffices to prove that $\widehat{C}\llbracket X_1,\dots,X_n\rrbracket$ is factorial. Now, if $C$ is a field, then $\widehat{C}=C$; if $C$ is a DVR, the same is true of $\widehat{C}$ (\cref{valuation completion prop}). We shall therefore assume in the remainder of the proof that $C$ is complete.\par
Arguing by induction starting with the trivial case n = 0, we shall assume that it has been proved that $A=C\llbracket X_1,\dots,X_{n-1}\rrbracket$ is factorial. We shall identify $B$ with $A\llbracket X_n\rrbracket$ and denote by $\mathfrak{M}$ the maximal ideal of $A$ (generated by $\pi,X_1,\dots,X_{n-1}$). We shall prove that every non-zero element $g$ of $B$ is, in an essentially unique way, a product of irreducible elements.\par
Let $K$ be the field $C/C\pi$. As $B/B\pi$ is identified with $K\llbracket X_1,\dots,X_n\rrbracket$, the ideal $B\pi$ is prime and $\pi$ is irreducible. If $\pi\neq 0$, $B_{B\pi}$ is therefore the ring of a normed discrete valuation $w$; every nonzero element $g$ of $B$ may therefore be written as $g=\pi^{w(g)}f$, where $f\in B$ and $f$ is not a multiple of $\pi$. It will therefore suffice to show that $f$ is an essentially unique product of extremal elements. Now the canonical image of $f$ in $K\llbracket X_1,\dots,X_n\rrbracket$ is not zero; (AC, \Rmnum{7}, $\S$3, Lemma 3, no.7) therefore shows that there exists an automorphism of $B$ which maps $f$ to an element $\tilde{f}$ such that the coefficients of $\tilde{f}(0,\dots,0,X_n)$ are not all in $C\pi$; this means that the coefficients of the series $\tilde{f}$, considered as a formal power series in $X_n$, are not all in $\mathfrak{M}$. It will suffice to prove our assertion for $\tilde{f}$.\par
In what follows, all the elements of $B$ will be considered as formal power series in $X_n$ with coefficients in $A$. By (AC, \Rmnum{7}, $\S$3, Proposition 6, no.8) (applicable since $C$ and therefore $A$ are separable and complete and the reduced series of $\tilde{f}$ is nonzero), $\tilde{f}$ is associated, in $B$, with a unique distinguished polynomial $F$. By (AC, \Rmnum{7}, $\S$3, Proposition 7, no.8), every series which divides $\tilde{f}$ (or, what amounts to the same, which divides $F$) is associated with a distinguished polynomial which divides $F$ and every decomposition of $\tilde{f}$ is, to within invertible factors, of the form $\tilde{f}=uF_1\cdots F_r$, where $u$ is invertible and the $F_i$ are irreducible distinguished polynomials (in $B$) such that $F=F_1\dots F_r$. By (AC, \Rmnum{7}, $\S$3, Corollary to Proposition 7, no.8), the $F_i$ are also irreducible in $A[X_n]$. Now, as $A$ is factorial by the induction hypothesis, so is $A[X_n]$; hence, since they are monic, the $F_i$ are uniquely determined by $F$ (up to a permutation). This shows the uniqueness of the decomposition $\tilde{f}=uF_1\cdots F_r$; its existence follows from the fact that $B$ is Noetherian, which completes the proof.
\end{proof}
\section{Modules over integrally closed Noetherian domains}
Throughout this section, $A$ will be a integral domain with field of fractions $K$. Then $P(A)$, $\mathfrak{D}(A)$ and $\mathfrak{C}(A)$ will respectively denote the set of prime ideals of $A$ of height $1$, the divisor group of $A$ and the divisor class group of $A$, these latter being written additively.\par
The general method of studying finite modules over an integrally closed Noetherian domain $A$ consists of "localizing" the modules with respect to all the prime ideals $\p\in P(A)$ of height $1$ in $A$. As $A_\p$ is then a DVR, the structure of finitely generated $A_\p$-modules is well known and therefore gives information about the structure of finitely generated $A$-modules. In the particular case where $A$ is a Dedekind domain, we can arrive at as complete a theory as when $A$ is a principal ideal domain.
\subsection{Lattices of a vector space}
Let $V$ be a finite-dimensional vector space over the field $K$. A \textbf{lattice of $\bm{V}$ with respect to $\bm{A}$} (or simply a lattice of $V$) is defined to be any sub-$A$-module $M$ of $V$ such that there exist two free sub-$A$-modules $L_1$, $L_2$ of $V$ such that $L_1\sub M\sub L_2$ and $\rank_A(L_1)=\dim_KV$.
\begin{example}
\mbox{}
\begin{itemize}
\item[(a)] If we take $V=K$, the lattices of $K$ are just the nonzero fractional ideals of $K$.
\item[(b)] If $\dim_KV=n$, every free sub-$A$-module $L$ of $V$ has a basis containing at most $n$ elements, every subset of $V$ which is free over $A$ being free over $K$. For $L$ to be a lattice of $V$, it is necessary and sufficient that $L$ have a basis of $n$ elements (in other words, that $\rank_A(L)=n$).
\item[(c)] If $A$ is a PID, every lattice $M$ of $V$ is a finitely generated $A$-module (since $A$ is Noetherian) which is torsion-free and hence a free $A$-module.
\end{itemize}
\end{example}
\begin{proposition}\label{lattice in vector space iff}
For a sub-$A$-module $M$ of $V$ to be a lattice of $V$, it is necessary and sufficient that $KM=V$ and that $M$ be contained in a finitely generated sub-$A$-module of $V$.
\end{proposition}
\begin{proof}
The conditions are obviously necessary, for a free sub-$A$-module of $V$ with the same rank as $V$ generates $V$. Conversely, if $KM=V$, $M$ contains a basis $a_1,\dots,a_n$ of $V$ over $K$ and hence it contains the free sub-$A$-modulc $L_1$ generated by the $a_i$. On the other hand, if $M\sub M_1$, where $M_1$ is a sub-$A$-module of $V$ generated by a finite number of elements $b_i$ and $e_1,\dots,e_n$ is a basis of $V$ over $K$, there exists a nonzero element $s$ of $A$ such that each of the $b_i$ is a linear combination of the $s^{-1}e_i$ with coefficients in $A$. If $L_2$ is the free sub-$A$-modules of $V$ generated by the $s^{-1}e_i$, then $M\sub L_2$.
\end{proof}
\begin{corollary}\label{lattice in vector space Noe iff}
Suppose that $A$ is Noetherian. For a sub-$A$-module $M$ of $V$ to be a lattice of $V$, it is necessary and safiicient that $KM=V$ and $M$ be finitely generated.
\end{corollary}
\begin{proposition}\label{lattice in vector space iff sandwich}
Let $M$ be a lattice of $V$ and $M_1$ a sub-$A$-modale of $V$. If there exist two elements $x,y\in K^\times$ such that $xM\sub M_1\sub yM$, then $M_1$ is a lattice of $V$. Conversely, if $M_1$ is a lattice of $V$, there exist two non-zero elements $a$, $b$ of $A$ such that $aM\sub M_1\sub b^{-1}M$.
\end{proposition}
\begin{proof}
If $L_1$, $L_2$ are two free lattices of $V$ such that $L_1\sub M\sub L_2$, the relations $xM\sub M_1\sub yM$ imply $xL_1\sub M_1\sub yL_2$ and $xL_1$ and $yL_2$ are free lattices. Conversely, if $M_1$ is a lattice and $e_1,\dots,e_n$ is a basis of $L_2$ over $A$, the relation $KM_1=V$ implies the existence of $x=a/s\in K^\times$ (where $a$ and $s$ are non-zero elements of $A$) such that $xe_i\in M_1$ for all $i$, whence $xM\sub xL_2\sub M_1$ and a fortiori $aM\sub M_1$. Exchanging the roles of $M$ and $M_1$ it can be similarly shown that there exists $b\neq 0$ in $A$ such that $bM_1\sub M$.
\end{proof}
\begin{proposition}[\textbf{Properties of Lattices}]\label{lattice in vector space operation}
\mbox{}
\begin{itemize}
\item[(a)] If $M_1$ and $M_2$ are lattices of $V$, so are $M_1\cap M_2$ and $M_1+M_2$.
\item[(b)] If $W$ is a vector subspace of $V$ and $M$ is a lattice of $V$, then $M\cap W$ is a lattice of $W$.
\item[(c)] Let $V$, $V_1,\dots,V_k$ be vector spaces of finite dimension over $K$ and let $f:V_1\times\cdots\times V_k\to V$ be a multilinear map whose image generates $V$. If $M_i$ is a lattice of $V_i$ for each $i$, then the sub-$A$-module of $V$ generated by $f(M_1\times\cdots\times M_k)$ is a lattice of $V$.
\item[(d)] Let $V$ and $W$ be two vector spaces of finite dimension over $K$, $M$ a lattice of $V$ and $N$ a lattice of $W$. Then the sub-$A$-module $(N:M)$ of $\Hom_K(V,W)$, consisting of the $K$-linear maps $f$ such that $f(M)\sub N$, is a lattice of $\Hom_K(V,W)$.  
\end{itemize}
\end{proposition}
\begin{proof}
For (a), by virtue of \cref{lattice in vector space iff sandwich}, there exist non-zero $a$ and $b$ in $A$ such that $aM_1\sub M_2\sub b^{-1}M_1$. We conclude that $M_1\cap M_2$ and $M_1+M_2$ lie betwcen $aM_1$ and $b^{-1}M_1$ and are therefore lattices by virtue of \cref{lattice in vector space iff sandwich}.\par
As for (b), let $S$ be a complement of $W$ in $V$, $L_W$ a free lattice of $W$ and $L_S$ a free lattice of $S$, so that $L=L_S+L_W$ is a free lattice of $V$. Then there exist $x$, $y$ in $K^\times$ such that $xL\sub M\sub yL$. We deduce that $xL_W\sub M\cap W\sub yL_W$, which shows that $M\cap W$ is a lattice of $W$.\par
Now we prove (c). As $KM_i=V_i$ clearly by linearity $f(M_1\times\cdots\times M_k)$ generates the vector $K$-space $V$. On the other hand, for all $i$, there exists a finitely generated sub-$A$-module $N_i$ of $V_i$ such that $M_i\sub N_i$. The sub-$A$-module $N$ of $V$ generated by $f(N_1\times\cdots\times N_k)$ is finitely generated and contains $M$ and hence $M$ is a lattice of $V$ (\cref{lattice in vector space iff}).\par
Finally, consider (d). Let $P$ (resp. $Q$) be a free lattice of $V$ (resp. $W$) containing $M$ (resp. contained in $N$). Obviously $(N:M)\sub(Q:P)$. Now it is immediate that $(Q:P)$ is isomorphic to $\Hom_A(P,Q)$, hence is a free $A$-module of rank $(\rank_AP)(\rank Q)$ and therefore a lattice of $\Hom_K(V,W)$. Similarly, if $P'$ (resp. $Q'$) is a free lattice of $V$ (resp. $W$) contained in $M$, (resp. containing $N$), then $(Q':P')\sub(N:M)$ and $(Q':P')$ is a lattice of $\Hom_K(V,W)$, whence the conclusion.
\end{proof}
In the notation of \cref{lattice in vector space operation}(d), the canonical map $(N:M)\to\Hom_A(M,N)$ which maps every $K$-linear map $f\in(N:M)$ to the $A$-linear map from $M$ to $N$ which has the same graph as $f|_M$, is bijective: for every $A$-linear map $g:M\to N$ can be imbedded in a $K$-linear map
\[g\otimes 1:M\otimes_AK\to N\otimes_AK\]
and by \cref{lattice in vector space iff} $M\otimes_AK$ and $N\otimes_AK$ are respectively identified with $V$ and $W$.\par
In particular, if we take $W=K$, $N=A$, $\Hom_K(V,W)$ is just the dual vector $K$-space $V^*$ of $V$ and $(A:M)$ is identified with the dual $A$-module $M^*$ of $M$. We shall henceforth make this identification and we shall say that $M^*$ is the dual lattice of $M$: it is therefore the set of $x^*\in V^*$ such that $\langle x^*,x\rangle\in A$ for all $x\in M$.
\begin{corollary}\label{lattice in vector space quotient by bilinear map}
Let $U$, $V$, $W$ be three vector spaces of finite dimension over $K$ and $f:U\times V\to W$ a left non-degenerate $K$-bilinear map. If $M$ is a lattice of $V$ and $N$ a lattice of $W$, then the set $(N:_{f}M)$ of $x\in U$ such that $f(x,y)\in N$ for all $y\in M$ is a lattice of $U$.
\end{corollary}
\begin{proof}
Let $\phi_f:U\to\Hom_K(V,W)$ be the $K$-linear map left associated with $f$ such that $\phi_f(x)$ is the linear mapy $y\mapsto f(x,y)$.Recall that to say that $f$ is left non-degenerate means that $\phi_f$ is injective. By \cref{lattice in vector space operation}(d) $(N:M)$ is a lattice of $\Hom_K(V,W)$, as $(N:_{f}M)=\phi_f^{-1}(N:M)$ and $\phi_f$ is injective, the corollary follows from \cref{lattice in vector space operation}(b).
\end{proof}
\begin{example}
Let $S$ be a (not necessarily associative) $K$-algebra of finite dimension with a unit element. Then the bilinear map $(x,y)\mapsto xy$ of $S\times S$ to $S$ is (left and right) non-degenerate. If $M$ and $N$ are lattices of $S$ with respect to $A$, so are $M\cdot N$ and the set of $x\in S$ such that $xM\sub N$. Note that there exists a sub-$A$-algebra of $S$ containing the unit element of $S$ which is a lattice of $S$: for consider a basis $e_1,\dots,e_n$ of $S$ such that $e_1$ is the unit element of $S$ and let $e_ie_j=\sum_kc_{ij}^ke_k$ be the multiplication table of $S$, so that $c_{1j}^k=\delta_{j}^{k}$ and $c_{i1}^k=\delta_{i}^k$. Let $s\in A$ be a non-zero element such that $\tilde{c}_{ij}^k:=s\cdot c_{ij}^{k}\in A$ for all triplets of indices $(i,j,k)$. If we write $w_i=s^{-1}e_i$ for $i\geq 2$, then
\[w_iw_j=s\tilde{c}_{ij}^1e_1+\sum_{k\geq 2}\tilde{c}_{ij}^kw_k\]
for $i,j\geq 2$. The lattice of $S$ with basis $e_1$ and the $w_2,\dots,w_n$ is a sub-$A$-algebra of $S$ with unit element $e_1$.
\end{example}
\begin{example}
Let $V$ be a finite-dimensional vector space over $K$ and $f$ a non-degenerate bilinear form on $V$. If $M$ is a lattice of $V$, it follows from the \cref{lattice in vector space quotient by bilinear map} that the set $M_f^*$ of $x\in V$ such that $f(x,y)\in A$ for all $y\in M$ is also a lattice of $V$. If $\phi_f:V\to V^*$ is the linear map left associated with $f$ (which is bijective), $\phi_f(M_f^*)$ is just the dual lattice $M^*$ of $M$.
\end{example}
\begin{proposition}\label{lattice in vector space extension of scalar}
Let $B$ be an integral domain, $A$ a subring of $B$ and $K$ and $L$ the respective field of fractions of $A$ and $B$. Let $V$ be a finite-dimensional vector space over $K$.
\begin{itemize}
\item[(a)] For every lattice $M$ of $V$ with respect to $A$, the image $BM$ of $M_{(B)}=M\otimes_AB$ in $V_{(L)}=V\otimes_KL$ is a lattice of $V_{(L)}$ with respect to $B$.
\item[(b)] Suppose further that $B$ is a flat $A$-module. Then the canonical map $M_{(B)}\to BM$ is bijective. If further $B$ is faithfully flat, the map which maps every lattice $M$ of $V$ with respect to $A$ to the lattice $BM$ of $V_{(L)}$ with respect to $B$ is injective.
\end{itemize}
\end{proposition}
\begin{proof}
As $KM=V$, clearly $L(BM)=V_{(L)}$. On the other hand $M$ is contained in a finitely generated sub-$A$-module $M_1$ of $V$ and hence $BM$ is contained in $BM_1$ which is a finitely generated $B$-module; whence assertion (a).\par
We see that $V_{(L)}=V\otimes_KL=V\otimes_AL$ and, as $L$ is a flat $B$-module, it is also a flat $A$-module. If $B$ is a flat $A$-module, then the canonical map $M\otimes_AB\to V\otimes_AB$ is injective. On the other hand, since $V$ is a free $K$-module and $K$ a flat $A$-module, $V$ is a flat $A$-module and hence the canonical map $V\otimes_AB\to V\otimes_AL$ is injective, which establishes the first assertion. To see also that the relation $BM_1=BM_2$ implies $M_1=M_2$ for two lattices $M_1$, $M_2$ of $V$ with respect to $A$ when $B$ is a faithfully flat $A$-module, note first that $BM_1\cap BM_2=B(M_1\cap M_2)$. We may therefore confine our attention to the case where $M_1\sub M_2$ and our assertion then follows from \cref{module faithfully flat iff} applied to the canonical injection $M_1\to M_2$.
\end{proof}
\begin{corollary}\label{lattice in vector space extension to completion}
Suppose that $A$ is a DVR. Let $\widehat{A}$ be its completion and let $\widehat{K}$ be the field of fractions of $\widehat{A}$. The map $\eta$ which maps every lattice $M$ of $V$ to the lattice $\widehat{A}M$ of $\widehat{V}=V\otimes_K\widehat{K}$ with respect to $A$ is bijective and its inverse maps every lattice $\widehat{M}$ of $\widehat{V}$ with respect to $\widehat{A}$ to its intersection $\widehat{M}\cap V$ ($V$ being canonically identified with a vector sub-$K$-space of $\widehat{V}$).
\end{corollary}
\begin{proof}
If $L$ is a free lattice of $V$, the lattices $aL$ (for $a\in A$ nonzero) form a fundamental system of neighbourhoods of $0$ for a topology $\mathcal{T}$ on $V$ (compatible with its $A$-module structure), which (when a basis of $L$ over $A$ is taken) is identified with the product topology on $K^n$. By virtue of \cref{lattice in vector space iff sandwich}, a fundamental system of neighbourhoods of $0$ for $\mathcal{T}$ also consists of all the lattices of $V$ with respect to $A$. Clearly $\widehat{V}$ is the completion of $V$ with respect to $\mathcal{T}$. Moreover, if $\m$ is the maximal ideal of $A$, the topology $\mathcal{T}$ induces on every lattice $M$ of $V$ with respect to $A$ the $\m$-adic topology since $M$ is a finitely generated $A$-module (\cref{filtration I-topo induce on submodule}) and $\widehat{A}M$ is the completion of $M$ with respect to this topology (\cref{filtration I-good submodule and quotient}). Moreover, as $M$ is open (and therefore closed) in $V$, $\widehat{A}M\cap V=M$, which proves again the fact that $\eta$ is injective (which follows directly from \cref{lattice in vector space extension of scalar}(b), since $\widehat{A}$ is a faithfully flat $A$-module).\par
Finally, if $\widehat{M}$ is a lattice of $\widehat{V}$ with respect to $A$ and $M=\widehat{M}\cap V$, then since $\widehat{A}L$ is a free lattice of $\widehat{V}$ with respect to $\widehat{A}$ and every element of $\widehat{A}$ is the product of an element of $A$ and an invertible element of $\widehat{A}$, there exists $a,b\in\widehat{A}$ such that $a\widehat{A}L\sub\widehat{M}\sub b^{-1}\widehat{A}L$, whence $aL\sub\widehat{M}\cap V\sub b^{-1}L$, so $\widehat{M}\cap V$ is a lattice of $V$ with respect to $A$. Moreovcr $\widehat{M}$ is open in $V$ and, as $V$ is dense in $\widehat{V}$, $\widehat{M}$ is the completion of $\widehat{M}\cap V=M$. This proves that $\eta$ is surjective, whence the corollary
\end{proof}
Let $S$ be a multiplicative subset of $A$ not containing $0$. Then we can apply \cref{lattice in vector space extension of scalar} to $B=S^{-1}A$ with $L=K$, $BM=S^{-1}M$. Hence $S^{-1}M$ is a lattice of $V$ with respect to $S^{-1}A$. Moreover:
\begin{proposition}\label{lattice in vector space localization and quotient}
Let $V$, $W$ be vector spaces of finite dimension over $K$, $M$ a lattice of $V$ and $N$ a lattice of $W$. If $M$ is finitely generated, then
\[S^{-1}(N:M)=(S^{-1}N:S^{-1}M)\]
in $\Hom_K(V,W)$.
\end{proposition}
\begin{proof}
Clearly the left hand side is contained in the right hand side. Conversely, let $f\in(S^{-1}N:S^{-1}M)$ and let $x_1,\dots,x_n$ be a system of generators of $M$. There exists $s\in S$ such that $f(x_i)\in s{^-1}N$ for all $i$ and hence $sf\in(N:M)$, which proves the proposition.
\end{proof}
\subsection{Duality and reflexive modules}
From now on the domain $A$ is assumed to be Noetherian and integrally closed and that $P(A)$ denotes the set of prime ideals of $A$ of height $1$. Every lattice with respect to $A$ is a finitely generated A-module by \cref{lattice in vector space Noe iff}.\par 
Let $V$ be a vector space of finite dimension over $K$, $V^*$ its dual and $V^{**}$ its bidual. We shall identify $V$ and $V^{**}$ by means of the canonical map $J_M$. Let $M$ be a lattice of $V$, recall that the dual $A$-module $M^*$ of $M$ is canonically identified with the dual lattice of $M$, which is the set of $x^*\in V^*$ such that $\langle x^*,x\rangle\in A$ for all $x\in M$. The bidual $A$-module $M^{**}$ of $M$ is therefore a lattice of $V$ which contains $M$. Moreover $M^{***}=M^*$, for the relation $M\sub M^{**}$ implies $(M^{**})^*\sub M^*$ and on the other hand $M^*\sub (M^*)^{**}$ by the above.\par
If $\p$ is a prime ideal, \cref{lattice in vector space localization and quotient} applied with $N=A$ gives the relation $(M^*)_\p=(M_\p)^*$, which justifies the notation $M^*_\p$ for both terms.
\begin{theorem}\label{lattice in vector space dual is intersection of localization}
If $M$ is a lattice of $V$, then $M^*=\bigcap_{\p\in P(A)}M^*_\p$.
\end{theorem}
\begin{proof}
Clearly $M^*$ is contained in each of the $M^*_\p$. Conversely, suppose that $x^*\in\bigcap_{\p\in P(A)}M^*_\p$. If $x\in M$, then $\langle x^*,x\rangle\in\bigcap_{\p\in P(A)}A_\p=A$, and thus $x^*\in M^*$.
\end{proof}
\begin{corollary}\label{lattice in vector space bidual is intersection of localization}
If $M$ is a lattice of $V$, then $M^{**}=\bigcap_{\p\in P(A)}M_\p$.
\end{corollary}
\begin{proof}
Theorcm~\ref{lattice in vector space dual is intersection of localization} applied to $M^*$ shows that $M^{**}=\bigcap_{\p\in P(A)}M_\p^{**}$. But as $A_\p$ is a principal ideal domain, $M_\p^*$ is a finitely generated free $A_\p$-module and hence $M_\p^{**}$ is canonically identified with $M_\p$, whence the corollary.
\end{proof}
For any lattice $M$ with respect to $A$, the canonical map $J_M:M\to M^{**}$ identifies an element $x\in M$ with itself, for $x$ is the unique element $y$ of $V=V^{**}$ such that $\langle x^*,x\rangle=\langle x^*,y\rangle$ for all $x^*\in M^*$, since $M^*$ generates $V^*$. We shall say that $M$ is \textbf{reflexive} if $M^{**}=M$. As we have above $M^*=(M^*)^{**}$, it is seen that the dual of any lattice $M$ is always reflxive.
\begin{remark}\label{module reflexive is lattice of vector space}
Let $M$ be a finitely generated $A$-module; it is immediate that the dual $M^*$ of $M$, identified with a sub-$A$-module of $\Hom_A(M,K)$, is a lattice of the vector $K$-space $\Hom_A(M,K)$; in particular, every finitely generated reflexive $A$-module is isomorphic to a lattice of a suitable vector $K$-space.
\end{remark}
\begin{theorem}\label{lattice in vector space reflexive iff}
If $M$ is a lattice of $V$, the following conditions are equivalent:
\begin{itemize}
\item[(\rmnum{1})] $M$ is reflexive;
\item[(\rmnum{2})] $M=\bigcap_{\p\in P(A)}M_\p$;
\item[(\rmnum{3})] $\Ass(V/M)\sub P(A)$.
\end{itemize}
\end{theorem}
\begin{proof}
The equivalence of (\rmnum{1}) and (\rmnum{2}) follows from the \cref{lattice in vector space bidual is intersection of localization}. If (\rmnum{2}) holds, then $\bigcap_{\p\in P(A)}(V/M_\p)=\{0\}$ in $V/M$, hence by \cref{associated prime of union of zero intersection module} we have
\[\Ass(V/M)\sub\bigcup_{p\in P(A)}\Ass(V/M_\p).\]
As $V/M_\p$ is an $A_\p$-module, an element of $A-\p$ cannot annihilate a nonzero element of $V/M_\p$, since the elements of $A-\p$ are invertible in $A_\p$. The elements of $\Ass(V/M_\p)$ are therefore contained in $\p$ and are nonempty, since $V/M_\p$ is a torsion $A_\p$-module. As $\p$ is of height $1$, necessarily $\Ass(V/M_\p)=\{p\}$ if $V/M_\p\neq\{0\}$ and $\Ass(V/M_\p)=\emp$ if $V/M_\p=\{0\}$; hence $\Ass(V/M)\sub P(A)$.\par
Finally, if condition (\rmnum{3}) holds, then
\[\Ass(M^{**}/M)\sub\Ass(V/M)\sub P(A)\]
On the other hand, if $\p\in P(A)$, then it has been seen in the proof of the \cref{lattice in vector space bidual is intersection of localization} that $_\p^{**}=M_\p$, whence $\p\notin\Ass(M^{**}/M)$ by \cref{associated prime and supp}. We conclude that $\Ass(M^{**}/M)=\emp$, whence $M^{**}=M$.
\end{proof}
\begin{corollary}\label{lattice in vector space reflexive module contain iff}
Let $M$, $N$ be two lattices of $V$ with respect to $A$ such that $N$ is reflexive. In order that $M\sub N$, it is necessary and sufficient that, for all $\p\in P(A)$, $M_\p\sub N_\p$.
\end{corollary}
\begin{proof}
The condition is obviously necessary and, if it is fulfilled, then
\[\bigcap_{\p\in P(A)}M_\p\sub\bigcap_{\p\in P(A)}N_\p=N\]
As $M\sub M^{**}=\bigcap_{\p\in P(A)}M_\p$, certainly $M\sub N$.
\end{proof}
\begin{example}[\textbf{Example of reflexive submodules}]
\mbox{}
\begin{itemize}
\item[(a)] Every free lattice is reflexive.
\item[(b)] Take $V=K$. For a fractional ideal $\a$ of $K$ to be a reflexive lattice, it is necessary and sufficient that it be a divisorial ideal by the observation $\a^*=(A:\a)$ and \cref{fractional ideal ideal quotient prop}.
\item[(c)] Let $M$ be a lattice with respect to $A$. If $S$ is a multiplicative subset of $A$ not containing $0$, \cref{lattice in vector space localization and quotient} shows that $S^{-1}(M^*)=(S^{-1}M)^*$. If $M$ is reflexive, $S^{-1}M$ is therefore a reflexive lattice with respect to $S^{-1}A$.
\item[(d)] If $M$ is a finitely generated $A$-module and $T$ its torsion submodule, the dual $M^*$ of $M$ is the same as the dual of $M/T$, since, for every linear form $f$ on $M$, the image $f(T)$ is a torsion submodule of $A$ and hence zero. As $M/T$ is isomorphic to a lattice of a vector space over $K$, it is seen that the dual of every finitely generated $A$-module is reflexive. 
\end{itemize}
\end{example}
\begin{proposition}\label{lattice in vector space reflexive module prop}
Let $V$ be a vector space of finite dimension over $K$.
\begin{itemize}
\item[(a)] If $M_1$ and $M_2$ are reflexive lattices of $V$, so is $M_1\cap M_2$.
\item[(b)] If $W$ is a vector subspace of $V$ and $M$ is a reflexive lattice of $V$, then $M\cap W$ is a reflexive lattice of $W$.
\item[(c)] Let $V$, $W$ be two vector spaces of finite dimension over $K$ and $M$ (resp. $N$) a lattice of $V$ (resp. $W$). If $N$ is reflexive, the lattice $(N:M)$ is reflexive.
\end{itemize}
\end{proposition}
\begin{proof}
For (a), note that $(M_1\cap M_2)_\p=(M_1)_\p\cap (M_2)_\p$ for all $\p\in P(A)$. If $M_1=\bigcap_{\p\in P(A)}(M_1)_\p$ and $M_2=\bigcap_{\p\in P(A)}(M_2)_\p$, then
\[M_1\cap M_2=\bigcap_{\p\in P(A)}(M_1\cap M_2)_\p\]
whence the conclusion by virtue of \cref{lattice in vector space reflexive iff}. Simialrly $(M\cap W)_\p=M_\p\cap W$, whence (b). Now for (c), as $M$ is finitely generated, it follows from \cref{lattice in vector space localization and quotient} that
$(N:M)_\p=(N_\p:M_\p)$. Moreover, the relation $N=\bigcap_{\p\in P(A)}N_\p$ implies
\[(N:M)=\bigcap_{\p\in P(A)}(N_\p:M_\p)\]
For if $f\in\bigcap_{\p\in P(A)}(N_\p:M_\p)$ and $x\in M$, then $f(x)\in N_\p$ for all $\p\in P(A)$, whence $f\in(N:M)$. This shows $(N:M)$ is reflexive.
\end{proof}
\begin{proposition}\label{lattice in vector space reflexive and exact sequence}
Let $0\to M\to N\to Q\to 0$ be an exact sequence of $A$-modules. Suppose that $N$ is finitely generated and torsion-free.
\begin{itemize}
\item[(a)] If $M$ is reflexive, then $\Ass(Q)\sub P(A)\cup\{(0)\}$ (in other words, every ideal associated with $Q$ is either zero or of height $1$).
\item[(b)] Conversely, if $N$ is reflexive and $\Ass(Q)\sub P(A)\cup\{(0)\}$, then $M$ is reflexive.  
\end{itemize}
\end{proposition}
\begin{proof}
As $A$ is Noetherian, $M$ is finitely generated. If we write $V=M_{(K)}$ and $W=N_{(K)}$, then $M$ (resp. $N$) is canonically identified with a lattice of $V$ (resp. $W$) (\cref{lattice in vector space iff}). Consider the two exact sequences:
\[0\to V/M\to W/M\to W/V\to 0,\quad 0\to Q\to W/M\to W/N\to 0.\]
from which we deduce that
\[\Ass(Q)\sub\Ass(W/M)\sub\Ass(V/M)\cup\Ass(W/V)\]
If $M$ is reflexive, then $\Ass(V/M)\sub P(A)$. On the other hand, clearly $\Ass(W/V)$ is, either empty, or reduced to $\{0\}$, whence (a). Similarly,
\[\Ass(V/M)\sub\Ass(W/M)\sub\Ass(Q)\cup\Ass(W/N)\]
The hypotheses $N$ is reflexive then imply that $\Ass(V/M)\sub P(A)\cup\{(0)\}$. But $V/M$ is a torsion $A$-module and hence $(0)\notin\Ass(V/M)$; \cref{lattice in vector space reflexive iff} then shows that $M$ is reflexive.
\end{proof}
\begin{proposition}\label{lattice in vector space reflexive and extension of scalar}
Let $A$ and $B$ be two rings, $\rho:A\to B$ a ring homomorphism and $M$ a finitely generated $A$-module. Suppose that $A$ is Noetherian and that $B$ is a flat $A$-module. Then, if $M$ is reflexive, so is the $B$-module $M_{(B)}=M\otimes_AB$.
\end{proposition}
\begin{proof}
We know that there exists a canonical isomorphism $\omega_M:(M^*)_{(B)}\to(M_{(B)})^*$, such that
\[\langle\omega_M(x^*\otimes 1),x\otimes 1\rangle=f(\langle x^*,x\rangle)\]
for $x\in M$, $x^*\in M^*$. As $M$ is a quotient of a finitely generated free $A$-module $L$, $M^*$ is isomorphic to a sub-$A$-module of the dual $L^*$ and $L^*$ is free and finitely generated; since $A$ is Noetherian, $M^*$ is therefore also a finitely generated $A$-module, whence an isomorphism $\omega_{M^*}:(M^{**})_{(B)}\to((M^*)_{(B)})^*$ such that
\[\langle\omega_{M^*}(x^{**}\otimes 1),x^*\otimes 1\rangle=f(\langle x^{**},x^{*}\rangle)\]
for $x^*\in M$ and $x^{**}\in M^{**}$. On the other hand, there is an isomorphism $\omega_M^t:(M_{(B)})^{**}\to ((M^*)_{(B)})^*$, whence by composition a canonical isomorphism
\[\phi=(\omega_M^t)^{-1}\circ\omega_{M^*}:(M^{**})_{(B)}\to (M_{(B)})^{**}\]
such that, in the above notation:
\begin{align}\label{lattice in vector space reflexive and extension of scalar-1}
\langle\phi(x^{**}\otimes 1),\omega_M(x^*\otimes 1)\rangle=f(\langle x^{**},x^{*}\rangle).
\end{align}
We consider now the canonical homomorphism $J_M:M\to M^{**}$ and show that the composite homomorphism:
\[\psi=(J_M\otimes 1)\circ\phi:M_{(B)}\to (M^{**})_{(B)}\to (M_{(B)})^{**}\]
is just the canonical homomorphism $J_{M_{(B)}}$. This follows immediately from (\ref{lattice in vector space reflexive and extension of scalar-1}) which gives the relations:
\[\langle \psi(x\otimes 1),\omega_M(x^*\otimes 1)\rangle=f(\langle J_M(x),x^*\rangle)=f(\langle x^*,x\rangle)=\langle \omega_M(x^*\otimes 1),x\otimes 1\rangle.\]
and from the fact that the elements $\omega_M(x^*\otimes 1)$ generate $(M_{(B)})^*$. This being so, the hypothesis that $M$ is reflexive means that $J_M$ is bijective, hence so is $J_M\otimes 1$ and therefore $\psi=J_{M_{(B)}}$ is bijective, which shows the proposition.
\end{proof}
\subsection{Local construction of reflexive modules}
We keep the notation and hypotheses of the last paragraph. We shall say that a property holds "for almost all $\p\in P(A)$" if the set of $\p\in P(A)$ for which it is not true is finite.
\begin{lemma}\label{Krull Noe localization of prime ideals intersection to 0}
Let $\p$ and $\q$ be two prime ideals of $A$ such that $(0)$ is the only prime ideal of $A$ contained in $\p\cap\q$. Then, for every sub-$A$-module $E$ of $V$, $(E_\p)_\q=KE$.
\end{lemma}
\begin{proof}
Let $S$ be the multiplicative subset $(A-\p)(A-\q)$ of $A$; by \cref{localization of product of multiplicative sets}, $(E_\p)_\q=S^{-1}E$. The prime ideals of $S^{-1}A$ correspond to the prime ideals $\q$ of $A$ such that $\p'\cap S=\emp$ and by hypothesis $(0)$ is the only prime ideal of $A$ not meeting $S$; hence $S^{-1}A=K$ and $S^{-1}E=KE$.
\end{proof}
\begin{theorem}\label{Krull Noe lattice equal at almost every prime}
Let $V$ be a vector space of finite dimension over $K$ and $M$ a lattice of $V$ with respect to $A$.
\begin{itemize}
\item[(a)] Let $N$ be a lattice of $V$ with respect to $A$. Then for every prime ideal $\p$ of $A$, $N_\p$ is a lattice of $V$ with respect to $A_\p$ and, for almost all $\p\in P(A)$, $N_\p=M_\p$.
\item[(b)] Conversely, suppose given for all $\p\in P$ a lattice $N(\p)$ of $V$ with respect to $A_\p$ such that $N(\p)=M_\p$ for almost all $\p\in P(A)$. Then $N=\bigcap_{\p\in P(A)}N(\p)$ is a reflexive lattice of $V$ with respect to $A$ and it is the only reflexive lattice of $V$ with respect to $A$ whose localization at $\p$ is $N(\p)$ for all $\p\in P(A)$.
\end{itemize}
\end{theorem}
\begin{proof}
The first assertion follows from \cref{lattice in vector space extension of scalar}. Moreover, there exist $x,y$ in $K^\times$ such that $xN\sub M\sub yN$. We know that, for almost all $\p\in P(A)$, $v_\p(x)=v_\p(y)=0$, which shows that $x$ and $y$ are invertible in $A$ and hence $M_\p=N_\p$.\par
Now we prove (b). We may replace $M$ by $x^{-1}M$ where $x\neq 0$ in $A$ and assume that $N(\p)\sub M_\p$ for all $\p\in P(A)$. Let $\p_1,\dots,\p_n$ be the elements of $P(A)$ such that $N(\p)=$ for $\p$ distinct from the $\p_i$, we write:
\[Q=M\cap N(\p_1)\cap\cdots\cap N(\p_n).\]
As each of the $N(\p_i)$ contain a free lattice with respect to $A_{\p_i}$, it contains a fortiori a lattice of $V$ with respect to $A$, hence $Q$ contains a lattice of $V$ with respect to $A$ (\cref{lattice in vector space operation}) and, as $Q$ is contained in $M$, $Q$ is a lattice with respect to $A$. If $\p\in P$ is distinct from the $\p_i$, \cref{Krull Noe localization of prime ideals intersection to 0} applied to $N(\p_i)$ gives
\[(N(\p_i))_\p=((N(\p_i))_{\p_i})_\p=KN(\p_i)=V\]
since the $\p_i$ and $\p$ are of height $1$. Then
\[Q_\p=M_\p\cap(N(\p_i))_\p\cap\cdots\cap(N(\p_n))_\p=M_\p=N(\p).\]
On the other hand, if $\p$ is equal to $\p_i$, then $(N(\p_i))_{\p_j}=V$ for $i\neq j$ by the argument as above and $(N(\p_i))_{\p_i}=N(\p_i)$, whence
\[Q_{\p_i}=M_{\p_i}\cap N(\p_i)=N(\p_i).\]
We have therefore proved that $Q_\p=N(\p)$ for all $\p\in P(A)$. Then $N=Q^{**}=\bigcap_{\p\in P(A)}Q_\p$ is reflexive and satisfies the relations $N_\p=Q_\p=N(\p)$ for all $\p\in P(A)$. The uniqueness property follows immediately from \cref{lattice in vector space reflexive iff}.
\end{proof}
\begin{proposition}\label{Krull Noe localization at prime ideals height 1 zero iff}
Let $M$ be a finitely generated $A$-module. The following conditions are equivalent:
\begin{itemize}
\item[(\rmnum{1})] $M_\p=0$ for every prime ideal $\p$ of height $\leq 1$.
\item[(\rmnum{2})] The annihilator $\a$ of $M$ is nonzero and $(A:\a)=A$. 
\end{itemize}
An $A$-module $M$ is called \textbf{pseudo-zero} if it is finitely generated and it satisfies the equivalent conditions above.
\end{proposition}
\begin{proof}
We know (by \cref{localization finite module is zero iff}) that the condition $M_\p=0$ is equivalent to $\a\nsubseteq\p$ and hence to $\a A_\p=A_\p$. On the other hand, for every integral ideal $\b\neq 0$ of $A$, the relation "$\b A_\p=A_\p$ for all $\p\in P(A)$" is equivalent to $\div(\b)=\div(A)=0$ in $\mathfrak{D}(A)$ (\cref{Krull domain divisor char by essential valuation}), or also to $\div(A:\b)=0$ and, as $(A:\b)$ is divisorial, this relation is also equivalent to $(A:\b)=A$. The proposition then follows by noting that to say that $\a\nsubseteq(0)$ means that $\a\neq (0)$.
\end{proof}
\begin{example}[\textbf{Example of pseudo-zero modules}]
\mbox{}
\begin{itemize}
\item[(a)] If $A$ is a Dedekind domain, every prime ideal of $A$ is of height $1$, so to say that $M$ is pseudo-zero means then that $\supp(M)=\emp$ and hence that $M=0$.
\item[(b)] Let $k$ be a field and $A=k[X,Y]$ the polynomial ring over $k$ in two indeterminates. If $\m$ is the maximal ideal $AX+AY$ of $A$, the $A$-module $A/\m$ is pseudo-zero, for its annihilator $\m$ is not of height $\leq 1$ since it contains the principal prime ideals $AX$ and $AY$ and is distinct from them; therefore $(A:\m)=A$.
\end{itemize}
\end{example}
Let $\phi:M\to N$ a homomorphism of $A$-modules. Then $\phi$ is called \textbf{pseudo-injective} (resp. \textbf{pseudo-surjective}, \textbf{pseudo-zero}) if $\ker\phi$ (resp. $\coker\phi$, $\im\phi$) is pseudo-zero. Also $\phi$ is called \textbf{pseudo-bifective} if it is both pseudo-injective and pseudo-surjective. A pseudo-bijective homomorphism is also called a \textbf{pseudo-isomorphism}.\par
Suppose that $M$ and $N$ are finitely generated. Then, for $\phi:M\to N$ to be pseudo-injective (resp. pseudo-surjective, pseudo-zero), it is necessary and sufficient that, for all $\p\in P(A)\cup\{(0)\}$, $\phi_\p:M_\p\to N_\p$ be injective (resp. surjective, zero) (this follows from the exactness of the $A$-module $A_\p$).
\begin{example}\label{Krull Noe module torsion free pseudo isomorphic to bidual}
Let $M$ be a torsion-free finitely generated $A$-module. Then $M$ is identified with a lattice of $V=M\otimes_AK$. We have seen that $(M_\p)_{\q}=V$ for $\p\neq\q$ in $P(A)$, therefore $M_\p=M_\p^{**}$ for all $\p\in P(A)$. Also, for $\p=0$, $M_\p$ and $M_\p^{**}$ are both equal to $V$. Therefore the canonical map $J_M:M\to M^{**}$ of $M$ to its bidual is a pseudo-isomorphism. 
\end{example}
\begin{example}
For all $\p\in P(A)$, the canonical map $\phi:A/\p^n\to A/\p^{(n)}=A/(A\cap\p^nA_\p)$ is a pseudo-isomorphism, as, for all $\q\in P(A)$ distinct from $\p$, $A_\q/\p^nA_\q=A_\q/\p^{(n)}A_\q=0$ and $A_\p/\p^nA_\p=A_\p/\p^{(n)}A_\p$.
\end{example}
\begin{theorem}\label{Krull Noe pseudo isomorphic to torsion product}
Let $E$ be a finitely generated $A$-module, $T$ the torsion submodule of $E$ and $M=E/T$. There exists a pseudo-isomorphism $\phi:E\to T\times M$.
\end{theorem}
For this theorem, we shall first prove two lemmas.
\begin{lemma}\label{Krull domain localization at finite primes is PID}
Let $\p_1,\dots,\p_k$ be a family of prime ideals of $A$ of height $1$ and let $S=\bigcap_{i=1}^{k}(A-\p_i)$. Then the ring $S^{-1}A$ is a principal ideal domain.
\end{lemma}
\begin{proof}
The ring $S^{-1}A$ is a semi-local ring whose maximal ideals are the $\m_i=S^{-1}\p_i$ for $1\leq i\leq k$, the local ring $(S^{-1}A)_{\m_i}$ being isomorphic to $A_{\p_i}$ and hence a discrete valuation ring. The ring $S^{-1}A$ is therefore a Dedekind domain and, as it is semi-local it is a principal ideal domain.
\end{proof}
\begin{lemma}\label{Krull Noe module pseudo isomorphism on torsion part}
There exists a homomorphism $\eta:E\to T$ whose restriction to $T$ is both a homothety and a pseudo-isomorphism.
\end{lemma}
\begin{proof}
Let $\a$ be the annihilator of $T$. As $T$ is a finitely generated torsion $A$-module, $\a\neq 0$. Let $\p_1,\dots,\p_k$ be the prime ideals of height $1$ containing $\a$ (which are finite in number). If this number is $0$, $T$ is pseudo-zero (\cref{Krull Noe localization at prime ideals height 1 zero iff}) and we may take $\eta=0$. Otherwise, let $S=\bigcap_{i=1}^{k}(A-\p_i)$; by \cref{Krull domain localization at finite primes is PID}, $S^{-1}A$ is a PID and hence $S^{-1}M$, which is a torsion-free finitely generated $S^{-1}A$-module, is free and, as $S^{-1}M=(S^{-1}E)/(S^{-1}T)$, $S^{-1}T$ is a direct factor of $S^{-1}E$. Now,
\[\Hom_{S^{-1}A}(S^{-1}E,S^{-1}T)=S^{-1}\Hom_A(E,T)\]
hence there exist $s_0\in S$ and $\eta_0\in\Hom_A(E,T)$ such that $s_0^{-1}\eta_0$ is a projector of $S^{-1}E$ onto $S^{-1}T$. If $\psi_0\in\Hom_A(T,T)$ denotes the restriction of $\eta_0$ to $T$, there therefore exists $s_1\in S$ such that $s_1\psi_0(x)=s_1s_0x$ for all $x\in T$. Writing $s=s_1s_0$, $\eta=s_1\eta_0$, $\psi=s_1\psi_0$, then $\psi$ is the homothety of ratio $s$ on $T$ and is the restriction of $\eta$ to $T$. It remains to verify that $\psi$ is a pseudo-isomorphism.\par
Now, if $\p=0$ or if $\p\in P(A)$ is distinct from the $\p_i$, then $T_\p=0$ and $\psi_\p:T_\p\to T_\p$ is an isomorphism. If on the contrary $\p$ is equal to one of the $\p_i$, then $s$ is invertible in $A_{\p_i}$ and $h_{\p_i}$ is the homothety of ratio $s$ on $T_{\p_i}$ is also an isomorphism, which completes the proof of \cref{Krull Noe module pseudo isomorphism on torsion part}.
\end{proof}
\begin{proof}[Proof of \cref{Krull Noe pseudo isomorphic to torsion product}]
Let $\eta:E\to T$ be a homomorphism satisfying the properties of \cref{Krull Noe module pseudo isomorphism on torsion part}. Let $\psi$ be the restriction of $\eta$ to $T$ and let $\pi$ be the canonical projection of $E$ onto $M$. We show that the homomorphism $\phi=(\eta,\pi):E\to T\times M$ solves the problem. There is the commutative diagram:
\[\begin{tikzcd}
0\ar[r]&T\ar[d,"\psi"]\ar[r]&E\ar[d,"\phi"]\ar[r]&M\ar[d,"1_M"]\ar[r]&0\\
0\ar[r]&T\ar[r]&T\times M\ar[r]&M\ar[r]&0
\end{tikzcd}\]
where the rows are exact. The snake diagram gives the exact sequence:
\[\begin{tikzcd}
0\ar[r]&\ker\psi\ar[r]&\ker\phi\ar[r]&0\ar[r]&\coker\psi\ar[r]&\coker\phi\ar[r]&0
\end{tikzcd}\]
and hence $\ker\phi$ is isomorphic to $\ker\psi$ and $\coker\phi$ to $\coker\psi$. As $\psi$ is a pseudo-isomorphism, so is $\phi$.
\end{proof}
We can say that "to within a pseudo-isomorphism" \cref{Krull Noe pseudo isomorphic to torsion product} reduces the study of finitely generated $A$-modules to that of torsion-free modules on the one hand and to that of torsion modules on the other. Moreover, we have seen above that a torsion-free module is pseudo-isomorphic to its bidual and hence to a reflexive module. As for torsion modules, there is the following result, which determines them to within a pseudo-isomorphism:
\begin{theorem}\label{Krull Noe torsion module pseudo isomorphic thm}
Let $T$ be a finitely generated torsion $A$-module. There exist two finite families $(n_i)_{i\in I}$ and $(\p_i)_{i\in I}$ where the $n_i$ are positive integers and the $\p_i$ are prime ideals of $A$ of height $1$ such that there exists a pseudo-isomorphism of $T$ to $\bigoplus_{i\in I}A/\p_i^{n_i}$. Moreover, the families $(n_i)_{i\in I}$ and $(\p_i)_{i\in I}$ with this property are unique to within a bijection of the indexing set and the $\p_i$ containing the annihilator of $T$.
\end{theorem}
\begin{proof}
Let $T'=\bigoplus_{i\in I}A/\p_i^{n_i}$. If $\phi:T\to T'$ is a pseudo-isomorphism and $\p\in P(A)$, then $\phi_\p:T_\p\to T_\p$ is an isomorphism. Now, $T'_\p$ is the direct sum of the $A_\p/\p^{n_i}A_\p$ the sum being over the indices $i$ such that $\p_i=\p$. The $\p^{n_i}A_\p$ are therefore the elementary divisors of the torsion $A_\p$-module $T_\p$ and are therefore unique.\par
We may confine our attention to the case where $T\neq 0$. Let $\a$ be the annihilator (non-zero and distinct from $A$) of $T$ and $\p_1,\dots,\p_k$ the prime ideals of $A$ of height $1$ containing $\a$ (which are finite in number and $S=\bigcap_{i=1}^{k}(A-\p_i)$. The semi-local ring $B=S^{-1}A$ is a principal ideal domain and has maximal ideals the $\m_i=S^{-1}\p_i$; as $S^{-1}T$ is a finitely generated torsion $B$-module, it is isomorphic to a finite direct sum $\bigoplus_{i=1}^{k}B/\m_i^{n_i}$. As $B/\m_i^{n_i}$ is isomorphic to $S^{-1}(A/\p_i^{n_i})$, we have obtained a torsion $A$-module $T'$ of the desired type and an isomorphism $\phi_0$ of $S^{-1}T$ onto $S^{-1}T'$. As $\Hom_{S^{-1}A}(S^{-1}T,S^{-1}T')$ is equal to $S^{-1}\Hom_A(T,T')$, there exist $s\in S$ and a homomorphism $\phi:T\to T'$ such that $\phi_0=s^{-1}f$. It remains to show that $f$ is a pseudo-isomorphism. Now if $\p=0$ or $\p\in P(A)$ is distinct from the $\p_i$, then $T_\p=T'_\p=0$. If on the contrary $\p$ is one of the $\p_i$, $s$ is invertible in $A_{\p_i}$ and, as $\phi_{\p_i}=s(\phi_0)$ and $(\phi_0)_{\p_i}$ is an isomorphism of $T_{\p_i}=(S^{-1}T)_{\m_i}$ onto $T'_{\p_i}=(S^{-1}T')_{\m_i}$, so is $\phi_{\p_i}$.
\end{proof}
Given an exact sequence of $A$-modules $0\to E\to F\to G\to 0$, if $E$ and $G$ are pseudo-zero, so is $F$, as follows from definition and the exactness of localization. In the language of categories, we may then say that, in the category $\mathcal{C}$ of $A$-modules, the sub-category $\mathcal{C}'$ of pseudo-zero modules is full and we may then define the quotient category $\mathcal{C}/\mathcal{C}'$. The objects in this category are also $A$-modules but the set of morphisms from $E$ to $F$ (for $E$, $F$ in $\mathcal{C}$) is the direct limit of the set of commutative groups $\Hom_A(E',F')$, where $E'$ (resp. $F'$) runs through the set of submodules of $E$ (resp. the set of quotient modules $F/F''$ of $F$) such that $E/E'$ (resp. $F''$) is pseudo-zero. Of course, for every ordered pair of $A$-modules $E$, $F$, there is a canonical homomorphism $\Hom_{\mathcal{C}}(E,F)\to\Hom_{\mathcal{C}/\mathcal{C}'}(E,F)$. To say that a homomorphism $\phi\in\Hom_A(E,F)$ is pseudo-zero (resp. pseudo-injective, pseudo-surjective, pseudo-bijective) means that its canonical image in $\Hom_{\mathcal{C}/\mathcal{C}'}(E,F)$ is zero (resp. a monomorphism, an epimorphism, an isomorphism).
\subsection{Divisors of finite generated torsion modules}
We keep to assume that $A$ is Noetherian and integrally closed. Recall that $\mathfrak{D}(A)$ denotes the divisor group of $A$, written additively. We know that $\mathfrak{D}(A)$ is the free $\Z$-module generated by the elements of $P(A)$.\par
Let $T$ be a finitely generated torsion $A$-module. For all $\p\in P(A)$, $T_\p$ is a finitely generated torsion $A_\p$-module and hence a module of finite length (\cref{associated prime of finite length localization iff}). We shall denote this length by $\ell_\p(T)$. Now $T_\p=0$ for all $\p$ not containing the annihilator of $T$ and hence for almost all $\p$, which justifies the following definition:
\begin{definition}
If $T$ is a finitely generated torsion $A$-module, the divisor:
\[\chi(T)=\sum_{\p\in P(A)}\ell_\p(T)\cdot\p\]
is called the \textbf{content} of $T$.
\end{definition}
\begin{proposition}\label{Krull Noe content of module prop}
Let $\chi$ be the content function defined above.
\begin{itemize}
\item[(a)] The function $\chi$ is additive.
\item[(b)] If $T_1$ and $T_2$ are pseudo-isomorphic, then $\chi(T_1)=\chi(T_2)$.
\item[(c)] In order that $\chi(T)=0$, it is necessary and sufficient that $T$ be pseudo-zero.
\end{itemize}
\end{proposition}
\begin{proof}
In view of the definition, it suffices to consider for each $\p\in P(A)$ the values of $\ell_\p$, for the torsion modules considered. The claimed properties then follows from that of length function.
\end{proof}
\begin{corollary}
If there is a long exact sequence
\[\begin{tikzcd}
0\ar[r]&T_n\ar[r]&T_{n-1}\ar[r]&\cdots\ar[r]&T_0\ar[r]&0
\end{tikzcd}\]
of finitely generated torsion $A$-modules, then $\sum_{i=0}^{n}(-1)^i\chi(T_i)=0$.
\end{corollary}
Recall that we may speak of the set $F(A)$ of classes of finitely generated $A$-madules with respect to the relation of isomorphism; for every finitely generated $A$-module $M$, let $\cl(M)$ denote the corresponding element of $F(A)$. We shall denote by $T(A)$ the subset of $F(A)$ consisting of the classes of finitely generated torsion $A$-modules. Clearly $\chi$ defines a map of $T(A)$ to $\mathfrak{D}(A)$, also denoted by $\chi$, such that $\chi(\cl(T))=\chi(T)$.
\begin{proposition}\label{Krull Noe uniqueness of the map chi}
Let $G$ be a commutative group and $\delta:T(A)\to G$ a function. For every finitely generated torsion $A$-module $T$, we also write, by an abuse of language, $\delta(T)=\delta(\cl(T))$. Suppose that the following conditions are satisfied:
\begin{itemize}
\item[(a)] The function $\delta$ is additive.
\item[(b)] If $T$ is pseudo-zero, then $\delta(T)=0$.
\end{itemize}
Then there exists a unique homomorphism $\theta:\mathfrak{D}(A)\to G$ such that $\delta=\theta\circ\chi$.
\end{proposition}
\begin{proof}
As $\chi(A/\p)=\p$ for all $\p$, necessarily $\theta(\p)=\delta(A/\p)$ for all $\p\in P(A)$, which proves the uniqueness of $\theta$, since the elements of $P(A)$ form a basis of $\mathfrak{D}(A)$. Conversely, let $\theta$ be the homomorphism from $\mathfrak{D}(A)$ to $G$ such that $\theta(p)=\delta(A/\p)$ for all $\p\in P(A)$ and let us show that it solves the problem. For this, we write
\[\psi(T)=\delta(T)-\theta(\chi(T))\]
for every finitely generated torsion $A$-module $T$. Clearly conditions (a) and (b) are also satisfied if $\delta$ is replaced by $\psi$. On the other hand, $\psi(A/\p)=0$ if $\p\in P(A)$. If $\p$ is a nonzero prime ideal and not in $P$, the annihilator of $A/\p$ is contained in no ideal of $P(A)$, hence $A/\p$ is pseudo-zero and therefore $\psi(A/\p)=0$. This being so, every finitely generated torsion $A$-module $T$ admits a chain og submodules whose factors are isomorphic to $A$-modules of the form $A/\p$, where $\p\in\supp(T)$ (\cref{associated prime of Noe prime ideals in composition}), and hence $\p\neq 0$ since $T$ is a torsion module. By induction on the length of this decomposition series, we deduce (in view of property (a) for $\psi$) that $\psi(T)=0$.
\end{proof}
\begin{remark}
We may consider the quotient category $\mathcal{A}/\mathcal{A}'$ of the category $\mathcal{A}$ of finitely generated torsion $A$-modules by the full sub-category $\mathcal{A}'$ of pseudo-zero finitely generated torsion $A$-modules. In the language of Abelian categories, \cref{Krull Noe uniqueness of the map chi} then expresses the fact that the Grothendieck group of the Abelian category $\mathcal{A}/\mathcal{A}'$ is canonically isomorphic to $\mathfrak{D}(A)$.
\end{remark}
\begin{proposition}\label{Krull Noe content of A/a}
If $\a$ is a nonzero ideal of $A$, then
\[\chi(A/\a)=\chi((A:\a)/A)=\div(\a).\]
\end{proposition}
\begin{proof}
Let $\p\in P(A)$. Then since $A_\p$ is a DVR we have $\a A_\p=\p^{n_p}A_\p$ where some $n_\p\neq 0$. As $(A/\a)_\p=A_\p/\a A_\p=(A/a)$, we then see $\ell_\p(A/\a)=n_\p$, whence $\chi(A/\a)=\sum_{\p\in P(A)}n_\p\p=\div(\a)$ by \cref{Krull domain coefficient of P is inf}. On the other hand, $(A:a)_\p=(A_\p:\a A_\p)=\p^{-n_\p}A_{\p}$, hence $\ell_\p((A:\a)/A)=n_\p$ and we conclude in the same way.
\end{proof}
\begin{corollary}
Let $M$ be a finitely generated tosion $A$-module. If $\bigoplus_{i\in I}A/\p_i^{n_i}$ is the canonical $A$-module to which $M$ is pseudo-isomorphic to, then
\[\chi(M)=\sum_{i\in I}n_i\p_i.\]
\end{corollary}
\begin{proof}
By \cref{Krull Noe content of A/a}, we see $\chi(A/\p_i^{n_i})=\div(\p_i^{n_i})=n_i\p_i$, whence the claim.
\end{proof}
Now let $V$ be a vector space of dimension $n$ over $K$ and $M$ a lattice of $V$ with respect to $A$. Let $W$ be the exterior power $\bigwedge^nV$, which is a one-dimensional vector space over $K$ and let $M_W$ denote the lattice of $W$ generated by the image of $M$ under the canonical map $V^n\to\bigwedge^nV$ (\cref{lattice in vector space operation}). If $e$ is a basis of $W$ over $K$, we may write $M_W=\a e$, where $\a$ is a nonzero fractional ideal of $A$.\par
Let $M'$ be another lattice of $V$ and let us write $M'_W=\a'e$, where $\a'$ is a nonzero fractional ideal of $A$. The divisor $\div(\a)-\div(\a')$ does not depend on the choice of basis $e$ of $W$, $\a$ and $\a'$ being multiplied by the same element of $K^\times$ when the basis is changed. We shall write $\chi(M,M')=\div(\a)-\div(\a')$ and say that this divisor is the \textbf{relative invariant} of $M'$ with respect to $M$. Clearly, if $M$, $M'$, $M''$ are three lattices of $V$, then:
\[\chi(M,M')+\chi(M',M'')+\chi(M,M'')=0,\quad \chi(M,M')+\chi(M',M)=0.\]
For all $\p\in P(A)$, it follows immediately from the definitions that $(M_W)_\p=(M_\p)_W$. Moreover, since $M_\p$ is then a free $A$-module since $A$ is a principal ideal domain, a basis of $M$ over $A$ is a basis of $V$ over $K$, hence $(M_\p)_W=\bigwedge^n(M_\p)$ and the fractional ideal $\a_\p$ equals $\a A_\p$. If we set $\a_\p=\p^{n_\p}A_\p$ and $\a'_\p=\p^{n'_\p}A_\p$, then
\[\chi(M,M')=\sum_{\p\in P(A)}(n_\p-n'_\p)\p,\]
which may also be written as:
\[\chi(M,M')=\sum_{\p\in P(A)}\chi(M_\p,M_\p').\]
identifying $\mathfrak{D}(A_\p)$ with the sub-$\Z$-module of $\mathfrak{D}(A)$ generated by $\p$.
\begin{proposition}\label{Krull Noe relative invariant and automorphism}
Let $M$ be a lattice of $V$ and $\phi$ a $K$-automorphism of $V$. Then:
\[\chi(M,\phi(M))=-\div(\det(\phi)).\]
\end{proposition}
\begin{proof}
For all $\p\in P(A)$, we have $\bigwedge^n\phi(M_\p)=\bigwedge^n\phi(M)_\p$. If $e_1,\dots,e_n$ is a basis of $M_\p$, then
\[\bigwedge\nolimits^n(M_\p)=A_\p\cdot e_1\wedge\cdots\wedge e_n,\quad \bigwedge\nolimits^n\phi(M_\p)=A_\p(\det(\phi))\cdot e_1\wedge\cdots\wedge e_n\]
whence the claim.
\end{proof}
\begin{proposition}\label{Krull Noe content of quotient module and relative invariant}
If $M$, $M'$ are two lattices of $V$ such that $M'\sub M$, then $M/M'$ is a finitely generated torsion $A$-module and $\chi(M,M')=-\chi(M/M')$.
\end{proposition}
\begin{proof}
Clearly $M/M'\sub V/M'$ is a finitely generated torsion module. On the other hand, for all $\p\in P(A)$, since $A_\p$ is a PID, we know that there exist bases $e_1,\dots,e_n$ of $M_\p$ and $e_1',\dots,e_n'$ of $M_\p'$ such that $e_i'=\pi^{\nu_i}e_i$ for all $i$ and integers v, $\pi$ being a uniformizer of $A_\p$. Therefore (in the notation introduced above) $n_i-n_i'=\sum_{i=1}^{n}\nu_i$. Also, $(M/M')_\p=M_\p/M_\p'$ is isomorphic to the torsion $A_\p$-module $\bigoplus_{i=1}^{n}A_\p/\p_i^{\nu_i}A_\p$, and hencc its length is $\sum_{i=1}^{n}\nu_i$, which proves the proposition.
\end{proof}
\begin{corollary}\label{Krull Noe free module homomorphism coker torsion iff}
Let $L_1$, $L_2$ be two free $A$-modules of the same rank $n$ and let $\phi:L_1\to L_2$ be a homomorphism. For $\coker\phi$ to be a torsion $A$-module, it is necessary and sufficient that $\det(\phi)\neq 0$ and in this case,
\[\chi(\coker\phi)=\div(\det(\phi)).\]
\end{corollary}
\begin{proof}
The modules $L_1$ and $L_2$ can be considered as lattices in $V_1=L_1\otimes_AK$ and $V_2=L_2\otimes_AK$ respectively, $\phi$ extending to a $K$-homomorphism $\phi_{(K)}$ from $V_1$ to $V_2$. Then $(\coker\phi)_{(K)}=\coker\phi_{(K)}$, and to say that $\coker\phi$ is a torsion $A$-module means that $\coker\phi_{(K)}=0$. Now, it amounts to the same to say that $\phi_{(K)}$ is surjective or that $\det(\phi_{(K)})=\det(\phi)\neq 0$, whence the first assertion. On the other hand, we may write $\phi(L_1)=\psi(L_2)$, where $\phi$ is an endomorphism of $L_2$ of determinant $\det(\phi)$. As $\coker\phi=L_2/\psi(L_2)$, the formula follows from \cref{Krull Noe relative invariant and automorphism}.
\end{proof}
\begin{example}
If $A=\Z$, the divisor group of $A$ is identified with the multiplicative group $\Q^\times_+$ of rational numbers $>0$. For every finite commutative group $T$, $\chi(T)$ is the order of $T$; the above corollary shows that the order of the group $\coker\phi$ is equal to the absolute value of $\det(\phi)$.
\end{example}
\subsection{Divisor classes of finite generated modules}
Recall that $\mathfrak{C}(A)$ denotes the divisor class group of $A$, the quotient of $\mathfrak{D}(A)$ by the subgroup of principal divisors. For every divisor $D\in\mathfrak{D}(A)$, we shall denote by $c(D)$ its class in $\mathfrak{C}(A)$.
\begin{proposition}\label{Krull Noe finite module divisor class def}
Let $M$ be a finitely generated $A$-module. There exists a free submodule $L$ of $M$ such that $M/L$ is a torsion module and the element $c(\chi(M/L))$ of $\mathfrak{C}$ does not depend on the free submodule $L$. The element $c(M):=-c(\chi(M/L))$ will be called the \textbf{divisor class attached to $\bm{M}$}.
\end{proposition}
\begin{proof}
We write $S=A-\{0\}$ and let $V=S^{-1}M=M\otimes_AK$. If $n$ is the dimension of $V$ over $K$, there exist $n$ elements $e_1,\dots,e_n$ of $M$ whose canonical images in $V$ form a basis of $V$. These elements are obviously linearly independent in $M$ and hence generate a free submodule $L$ of $M$ such that $S^{-1}(M/L)=S^{-1}M/S^{-1}L=0$, so that $M/L$ is a torsion module.\par
Now let $L_1$ be another free submodule of $M$ of rank $n$. Since $S^{-1}L=S^{-1}L_1$, there exists $s\in S$ such that $sL_1\sub L$. We may therefore limit ourselves to proving that, if $L_1\sub L_2$ are two free submodules of $M$ of rank $n$, then
\[c(\chi(M/L_1))=c(\chi(M/L_2)).\]
Now, $\chi(M/L_1)=\chi(M/L_2)+\chi(L_2/L_1)$ and it follows from \cref{Krull Noe free module homomorphism coker torsion iff} that $\chi(L_2/L_1)$ is a principal divisor and therefore the claim follows.
\end{proof}
\begin{proposition}[\textbf{Properties of the divisor class of modules}]\label{Krull Noe module divisor class prop}
\mbox{}
\begin{itemize}
\item[(a)] The function $c$ is additive.
\item[(b)] If $M_1$ and $M_2$ are pseudo-isomorphic, then $c(M_1)=c(M_2)$. 
\item[(c)] If $T$ is a torsion $A$-module, then $c(T)=-c(\chi(T))$.
\item[(d)] If $\a$ is a nonzero fractional ideal of $A$, then $c(\a)=-c(\div(\a))$.
\item[(e)] If $L$ is a free $A$-module then $c(L)=0$. 
\end{itemize}
\end{proposition}
\begin{proof}
To prove (a), consider a exact sequence
\[\begin{tikzcd}
0\ar[r]&M_1\ar[r,"\phi"]&M_2\ar[r,"\psi"]&M_3\ar[r]& 0
\end{tikzcd}\]
Then there are free sub-module $L_1$ (resp. $L_3$) of $M_1$ (resp. $M_3$) such that $M_1/L_1$ (resp. $M_3/L_3$) is a torsion module. Since $L_3$ is free and $\psi$ is surjective, there exists in $\psi^{-1}(L_3)$ a free complement $L_{23}$ of $\ker\psi$ which is isomorphic to $L_3$. But $\ker\psi=\im\phi$ contains $\phi(L_1)=L_{12}$ which is free since $\phi$ is injective. The sum $L_2=L_{12}+L_{23}$ is direct and $L_2$ is therefore a free submodule of $M_2$. There is moreover the commutative diagram:
\[\begin{tikzcd}
0\ar[r]&L_1\ar[r]\ar[d]&L_2\ar[r]\ar[d]&L_3\ar[r]\ar[d]&0\\
0\ar[r]&M_1\ar[r,"\phi"]&M_2\ar[r,"\psi"]&M_3\ar[r]& 0
\end{tikzcd}\]
where the rows are exact and the vertical arrows are injections. We therefore obtain from the snake diagram the exact sequence:
\[\begin{tikzcd}
0\ar[r]&M_1/L_1\ar[r]&M_2/L_2\ar[r]&M_3/L_3
\end{tikzcd}\]
As $M_1/L_1$ and $M_3/L_3$ are torsion modules, this exact sequence shows first that so is $M_2/L_2$ and then
\[\chi(M_2/L_2)=\chi(M_1/L_1)+\chi(M_3/L_3)\]
which proves (a).\par
Assertions (c) and (e) are obvious from the definition. We prove (b). Therefore let $\phi:M_1\to M_2$ be a pseudo-isomorphism and let $L_1$ be a free submodule of $M_1$ such that $M_1/L_1$ is a torsion module. We set $L_2=\phi(L_1)$; as $\ker\phi$ is pseudo-zero, it is a torsion module, hence $\ker\phi\cap L_1=0$ and therefore $L_2$ is free. Let $\bar{\phi}:M_1/L_1\to M_2/L_2$ be the homomorphism derived from $\phi$ by taking quotients; $\ker\phi$ is isomorphic to $\ker\bar{\phi}$ and $\coker\phi$ to $\coker\bar{\phi}$ and hence $\bar{\phi}$ is a pseudo-isomorphism. Moreover $\coker\bar{\phi}=M_2/\bar{\phi}(M_1)$ is a torsion module and so is $\phi(M_1)/L_2=\phi(M_1/L_1)$, hence $M_2/L_2$ is a torsion module and it follows that $\chi(M_1/L_1)=\chi(M_2/L_2)$.\par
Finally it remains to prove (d). Let $x\in K^\times$ be such that $\a\sub xA$. By considering the exact sequence $0\to\a\to xA\to xA/\a\to 0$, we obtain
\[c(\a)=c(xA)-c(xA/\a)=-c(xA/\a)\]
by (a) and (e). But $xA/\a$ is isomorphic to $A/x^{-1}\a$, whence, by virtue of (c),
\[c(xA/\a)=-c(\chi(A/x^{-1}\a))=-c(\div(x^{-1}\a))=-c(\div(\a))\]
This completes the proof.
\end{proof}
\begin{corollary}\label{Krull Noe module divisor class of long exact seq zero}
If there is a long exact sequence
\[\begin{tikzcd}
0\ar[r]&M_n\ar[r]&M_{n-1}\ar[r]&\cdots\ar[r]&M_0\ar[r]&0
\end{tikzcd}\]
of finitely generated $A$-modules, then $\sum_{i=0}^{n}(-1)^ic(M_i)=0$.
\end{corollary}
\begin{proof}
We argue by induction on $n$, the case $n=2$ being \cref{Krull Noe module divisor class prop}(a). If $N_{n-1}=\coker(M_n\to M_{n-1})$, there are the two exact sequences:
\[\begin{tikzcd}
0\ar[r]&M_n\ar[r]&M_{n-1}\ar[r]&N_{n-1}\ar[r]&0
\end{tikzcd}\]
\[\begin{tikzcd}
0\ar[r]&N_{n-1}\ar[r]&M_{n-2}\ar[r]&\cdots\ar[r]&M_0\ar[r]&0
\end{tikzcd}\]
The first shows that $M_{n-1}$ is finitely generated and the induction hypothesis gives
\[(-1)^{n-1}c(N_{n-1})+\sum_{i=0}^{n-2}c(M_i)=0\]
and
\[c(N_{n-1})=c(M_{n-1})-c(M_n),\]
whence the corollary.
\end{proof}
\begin{corollary}\label{Krull Noe finite free resolution then principal}
If a nonzero divisorial fractional ideal $\a$ of $A$ admits a finite free resolution, it is principal.
\end{corollary}
\begin{proof}
In fact we apply \cref{Krull Noe module divisor class of long exact seq zero} to a finite free resolution of $\a$:
\[\begin{tikzcd}
0\ar[r]&L_n\ar[r]&L_{n-1}\ar[r]&\cdots\ar[r]&L_0\ar[r]&\a\ar[r]&0
\end{tikzcd}\]
By \cref{Krull Noe module divisor class prop}(e), we see $c(\a)=0$, so $\div(\a)$ is principal (\cref{Krull Noe module divisor class prop}(d)). As $\a$ is assumed to be divisorial, it is principal.
\end{proof}
\begin{corollary}\label{Krull Noe free resolution then UFD}
If every nonzero divisorial ideal of $A$ admits a finite free resolution, $A$ is factorial.
\end{corollary}
If $M$ is a finitely generated $A$-module, we shall denote its rank by $\rank(M)$ (recall that it is the dimension over $K$ of $M\otimes_AK)$. We know that $\rank(M)$ is a additive function. We write
\[\gamma(M)=(\rank(M),c(M))\in\Z\times\mathfrak{C}(A).\]
Then $\gamma$ is also additive and, if $M$ is pseudo-zero, $\gamma(M)=0$ (since $M$ is a torsion module). There exists a unique map from $F(A)$ to $\Z\times\mathfrak{C}(A)$, also denoted by $\gamma$, such that $\gamma(M)=\gamma(\cl(M))$ for every finitely generated $A$-module $M$. We shall see that the above properties essentially characterize $\gamma$:
\begin{proposition}\label{Krull Noe divisor class of module uniqueness}
Let $G$ be a commutative group and $\delta$ a map from the set $F(A)$ of classes of finitely generated $A$-modules to $G$. For every finitely generated $A$-module $M$ we also write, by an abuse of language, $\delta(M)=\delta(\cl(M))$. Suppose the following conditions are satisfied:
\begin{itemize}
\item[(a)] $\delta$ is additive.
\item[(b)] If $T$ is pseudo-zero, then $\delta(T)=0$.
\end{itemize}
Then there exists a unique homomorphism $\delta:\Z\times\mathfrak{C}(A)\to G$ such that $\delta=\theta\circ\gamma$.
\end{proposition}
\begin{proof}
By \cref{Krull Noe module divisor class prop}, every element of $\Z\times\mathfrak{C}(A)$ is of the form $(\rank(M),c(M))$ for some suitable finitely generated $A$-module $M$; whence the uniqueness of $\theta$. We apply \cref{Krull Noe uniqueness of the map chi} to the restriction of $-\delta$ to $T(A)$: then there exists a homomorphism $\theta_0:\mathfrak{D}(A)\to G$ such that
\[-\delta(T)=\theta_0(\chi(T)).\]
for every finitely generated torsion $A$-module $T$. Let $x$ be a non-zero element of $A$; applying property (a) to the exact sequence:
\[\begin{tikzcd}
0\ar[r]&A\ar[r,"h_x"]&A\ar[r]&A/xA\ar[r]&0
\end{tikzcd}\]
where $h_x$ is multiplication by $x$, we obtain $\delta(A/xA)=0$, whence $\theta_0(\div(x))=0$. Taking quotients, $\theta_0$ therefore defines a homomorphism $\theta_1:\mathfrak{C}(A)\to G$ and $\delta(T)=\theta_1(c(T))$ for every torsion $A$-module $T$. We show now that the homomorphism $\theta$ defined by $\theta(n,z)=n\delta(A)+\theta_0(z)$ solves the problem. For this, we write $\delta'(M)=\delta(M)-\theta(\gamma(M))$ for every finitely generated $A$-module $M$. Clearly condition (a) is still satisfied if $\delta$ is replaced by $\delta'$. Moreover, $\delta'(M)=0$ when $M$ is a torsion module or a free module. But as for every finitely generated $A$-module $M$, there exists a free sub-module $L$ of $M$ such that $M/L$ is a torsion module, property (a) shows that $\delta'(M)=0$ for every finitely generated $A$-module $M$.
\end{proof}
\begin{remark}
In the language of Abelian categories, \cref{Krull Noe divisor class of module uniqueness} shows that $\Z\times\mathfrak{C}(A)$ is canonically isomorphic to the Grothendieck group of the quotient category $\mathcal{F}/\mathcal{F}'$, where $\mathcal{F}$ is the category of finitely generated $A$-modules and $\mathcal{F}'$ the full sub-category of $\mathcal{F}$ consisting of the pseudo-zero modules.
\end{remark}
\subsection{Finite field extensions}
In this part $A$ and $B$ denote two integrally closed Naetherian domains such that $A\sub B$ and $B$ is a finitely generated $A$-module. Let $K$ and $L$ the fields of fractions of $A$ and $B$ respectively. We shall write  instead of $\div_A$, $\chi_A$, $c_A$, $\chi_A$ respectively where $A$-modules are concerned and use analogous notation for $B$-modules.\par
We know that $B$ is integral over $A$ by \cref{integral element def}, so for a prime ideal $\mathfrak{P}$ of $B$ to be of height $1$, it is necessary and sufficient that $\p=\mathfrak{P}\cap A$ be of height $1$. Moreover by \cref{Krull-Akizuki prime ideal lying over is finite}, for $\p\in P(A)$, there is only a finite number of prime ideals $\mathfrak{P}\in P(B)$ lying over $\p$. To abbreviate, we shall denote by $\mathfrak{P}|\p$ the relation $"\mathfrak{P}$ lies over $\p$". We shall then denote by $e_{\mathfrak{P}/\p}$ or $e(\mathfrak{P}/\p)$ the ramification index of the valuation $v_{\mathfrak{P}}$ over the valuation $v_\p$ and by $f_{\mathfrak{P}/\p}$ or $f(\mathfrak{P}/\p)$ the residue degree $f(v_{\mathfrak{P}}/v_\p)$. Recall that the discrete valuations $v_\p$ and $v_{\mathfrak{P}}$ are normed and that $f_{\mathfrak{P}/\p}$ is the degree of the field of fractions of $B/\mathfrak{P}$ over the field of fractions of $A/\p$. We set $n=\rank_A(B)$, where $B$ is considered as an $A$-module; hence by definition $n=[L:K]$ and, for all $\p\in P(A)$, $n$ is also the rank of the free $A_\p$-module $B_{\mathfrak{P}}$ for all $\mathfrak{P}|\p$. Then it follows from \cref{valuation extension fundamental equality iff} that for all $\p\in P(A)$:
\begin{align}\label{Krull Noe extension of field equality}
\sum_{\mathfrak{P}|\p}e_{\mathfrak{P}/\p}f_{\mathfrak{P}/\p}=n.
\end{align}
This being so, as $\mathfrak{D}(A)$ and $\mathfrak{D}(B)$ are free $\Z$-modules, we define an increasing homomorphism of ordered groups $N:\mathfrak{D}(B)\to\mathfrak{D}(A)$ (also denoted by $N_{B/A}$) by the condition:
\[N(\mathfrak{P})=f_{\mathfrak{P}/\p}\p\quad\text{for $\mathfrak{P}\in P(B)$ and $\mathfrak{P}|\p$}.\]
On the other hand we have defined an increasing homomorphism of ordered groups $i:\mathfrak{D}(A)\to\mathfrak{D}(B)$ (also denoted by $i_{B/A}$) by the condition:
\[i(\p)=\sum_{\mathfrak{P}|\p}e_{\mathfrak{P}/\p}\mathfrak{P}\]
for $\p\in P(A)$. Clearly for every family $(D_i)$ (resp. $(E_i)$) of divisors of $A$ (resp. $B$):
\begin{align}
&&i(\sup_iD_i)&=\sup_ii(D_i),& i(\inf_iD_i)&=\inf_ii(D_i),\label{Krull Noe extension of field map i prop}&&\\
&&N(\sup_iE_i)&=\sup_iN(E_i),& N(\inf_iE_i)&=\inf_iN(E_i).\label{Krull Noe extension of field map N prop}&&
\end{align}
Also, the formula (\ref{Krull Noe extension of field equality}) implies $N\circ i=n\cdot 1_{\mathfrak{D}(A)}$.\par
Recall that in \cref{Krull domain divisor map i prop} we have proved that $i(\div_A(a))=\div_B(a)$ for any $a\in A$. Now let $b\in B$ and consider $N(\div_B(b))$. We recall that the norm $N_{L/K}(b)$ of $b$ is in $A$ and, as $v_{\mathfrak{P}}$ and $v_\p$ are normed, we have
\[(v_{\mathfrak{P}})|_K=e_{\mathfrak{P/\p}}\cdot v_\p\quad\text{for $\mathfrak{P}|\p$}.\]
The formula in (\ref{valuation extension of rank 1 formula for norm and trace-1}) then becomes
\[v_\p(N_{L/K}(b))=\sum_{\mathfrak{P}|\p}f_{\mathfrak{P}/\p}(b)\]
which implies $N(\div_B(b))=\div_A(N_{L/K}(b))$. Therefore, by taking quotients, the homomorphisms $N$ and $i$ define homomorphisms which will also be denoted, by an abuse of language, by:
\[i:\mathfrak{C}(A)\to\mathfrak{C}(B),\quad N:\mathfrak{C}(B)\to\mathfrak{C}(A).\]
\begin{proposition}\label{Krull Noe field extension Res and N i prop}
Let $A\sub B$ be Noetherian integrally closed domains and $B$ a finitely generated $A$-module.
\begin{itemize}
\item[(a)] For $E$ to be a pseudo-zero $B$-module, it is necessary and sufficient that the $A$-module $\Res_A^BE$ be pseudo-zero.
\item[(b)] For $E$ to be finitely generated torsion $B$-module, it is necessary and sufieient that $\Res_A^BE$ is a finitely generated torsion $A$-module and in this case we have
\[\chi_A(\Res_A^BE)=N(\chi_B(E)).\]
\item[(c)] For $E$ to be finitely generated $B$-module, it is necessary and suflicient that $\Res_A^BE$ be a finitely generated $A$-module and in this case we have
\[c_A(\Res_A^BE)=N(c_B(E))+\rank_B(E)c_A(B),\quad \rank_A(\Res_A^BE)=n\cdot\rank_B(E).\]
\end{itemize}
\end{proposition}
\begin{proof}
As $B$ is a finitely generated $A$-module, for $E$ to be a finitely generated $B$-module, it is necessary and sufficient that $\Res_A^BE$ be a finitely generated $A$-module. Moreover, if $\b$ is the annihilator of $E$, $\b\cap A=\a$ is the annihilator of $\Res_A^BE$. As $B$ is integral over $A$, there is no ideal other than $0$ lying over the ideal $0$ of $A$ and hence it amounts to the same to say that $\a\neq 0$ or that $\b\neq 0$.\par
By virtue of this last remark, we may confine our attention to the case where $E$ is a torsion $B$-module. If $\b$ is contained in a prime ideal $\mathfrak{P}\in P(B)$, $\a$ is contained in $\mathfrak{P}\cap A=\p$, which is of height $1$. Conversely, if $\a$ is contained in a prime ideal $\p\in P(A)$, there exists a prime ideal $\mathfrak{P}$ of $B$ which contains $\b$ and lies over $\p$. Assertion (a) follows from these remarks and \cref{Krull Noe localization at prime ideals height 1 zero iff}.\par
For every finitely generated torsion $B$-module $E$, we write
\[\delta(E)=\chi_A(\Res_A^BE).\]
clearly $\delta$ satisfies that conditions \cref{Krull Noe uniqueness of the map chi} (taking account of part (a)). There therefore exists a homomorphism $\theta:\mathfrak{D}(B)\to\mathfrak{D}(A)$ such that $\delta(E)=\theta(\chi_B(E))$ for every finitely generated torsion $B$-module $E$. The homomorphism $\theta$ is determined by its value for every $B$-module of the form $B/\mathfrak{P}$ where $\mathfrak{P}\in P(B)$, since $\chi_B(B/\mathfrak{P})=\mathfrak{P}$. Now, for every prime ideal $\q\neq\p=\mathfrak{P}\cap A$ in $P(A)$, $\p\nsubseteq\q$ and hence $(B/\mathfrak{P})_\q=0$. On the other hand, $\mathfrak{P}B_{\mathfrak{P}}$ is a maximal ideal of $B_{\mathfrak{P}}$ and $(B/\mathfrak{P})_\p=B_{\mathfrak{P}}/\mathfrak{P}B_{\mathfrak{P}}$ is isomorphic to the field of fractions of $B/\mathfrak{P}$, that is to the residue field of $v_{\mathfrak{P}}$. Its length as an $A_\p$-module is therefore $f_{\mathfrak{P}/\p}$, which proves that $\theta=N$, by the definition of $\chi_A$.\par
If $T$ is the torsion submodule of $E$, then $\Res_A^BT$ is the torsion submodule of $\Res_A^BE$ and $\Res_A^B(E/T)=\Res_A^BE/\Res_A^BT$. To prove (c) we may therefore confine our attention to the case where $E$ is torsion-free. Then $E$ is identified with a sub-$B$-module of $E_{(L)}$ and contains a basis $e_1,\dots,e_m$ over $L$. If $b_1,\dots,b_s$ is a basis of $L$ over $K$ consisting ofelements of $B$, the $b_je_i$ form a basis of $E_{(L)}$ over $K$ consisting of elements of $E$, whence the second equality (c). On the other hand, let $F$ be a free sub-$B$-module of $E$ such that $E/F$ is a torsion $B$-module; as $\Res_A^BF$ is a direct sum of $\rank_B(E)$ $A$-modules isomorphic to $B$, by \cref{Krull Noe module divisor class prop},
\[c_A(\Res_A^BF)=\rank_B(E)\cdot c_A(B).\]
Moreover, by definiton of $N$ and part (b), we have
\[c_A(\Res_A^B(E/F))=-c_A(N(\chi_B(E/F)))=-N(c_B(\chi_B(E/F)))=N(c_B(E))\]
Then it suffices to apply \cref{Krull Noe module divisor class prop} to finish the proof.
\end{proof}
\begin{proposition}
Let $E$ be a finitely generated $B$-module. For $E$ to be reflexive, it is
necessary and suficient that $\Res_A^BE$ be a reflexive $A$-module.
\end{proposition}
\begin{proof}
We have remarked in the proof of \cref{Krull Noe field extension Res and N i prop} that for $E$ to be a torsion-free $B$-module, it is necessary and sufficient that $\Res_A^BE$ be a torsion-free $A$-module. We may therefore assume that $E$ is a lattice of $W=E\otimes_BL$ with respect to $B$. We shall use the following lemma:
\begin{lemma}
Let $W$ be a vector space of finite dimension over $L$ and let $E$ be a lattice of $W$ with respect to $B$. Then, for all $\p\in P(A)$, we have $(\Res_A^BE)_\p=\bigcap_{\mathfrak{P}|\p}E_{\mathfrak{P}}$.
\end{lemma}
If $S=A-\p$, the prime ideals of the ring $S^{-1}B$ are generated by the prime ideals of $B$ not meeting $S$, in other words the ideals $\mathfrak{P}_i$ lying over $\p$ and the ideal $(0)$. This shows that $S^{-1}B$ is a semi-local ring whose maximal ideals are the $\m_i=S^{-1}\mathfrak{P}_i$ for $1\leq i\leq m$. Moreover the local ring $(S^{-1}B)_{\m_i}$ in is isomorphic to $B_{\mathfrak{P}_i}$ and hence is a discrete valuation ring. The ring $S^{-1}B$ is therefore a Dedekind domain and, as it is semi-local, it is a principal ideal domain. This being so, $(\Res_A^BE)_\p$ is equal to $S^{-1}E$ considered as an $A_\p$-module; by the above, $S^{-1}E$ is a free lattice of $W$ with respect to $S^{-1}B$ (hence reflexive) and \cref{lattice in vector space reflexive iff} may therefore be applied to it, giving $S^{-1}E=\bigcap_i(S^{-1}E)_{\m_i}$. But $(S^{-1}E)_{\m_i}=E_{\mathfrak{P}_i}$, which proves the lemma.\par
Returning to the proof, by the above lemma we have
\[\bigcap_{\mathfrak{P}\in P(B)}E_{\mathfrak{P}}=\bigcap_{\p\in P(A)}(\Res_A^BE)_\p.\]
and the conclusion follows from \cref{lattice in vector space reflexive iff}.
\end{proof}
\begin{corollary}
The ring $B$ is a reflexive $A$-module.
\end{corollary}
\begin{proposition}\label{Krull Noe field extension Ind and N i prop}
Let $A\sub B$ be Noetherian integrally closed domains and $B$ a finitely generated $A$-module.
\begin{itemize}
\item[(a)] For a finitely generated $A$-module $M$ to be pseudo-Zero, it is necessary and sufficient that $M\otimes_AB$ be a pseudo-zero $B$-module.
\item[(b)] If $M$ is a finitely generated torsion $A$-module, then $M\otimes_AB$ is a finitely generated $B$-module and
\[\chi_B(M\otimes_AB)=i(\chi_A(M)).\] 
\item[(c)] If $M$ is a finitely generated $A$-module, $M\otimes_AB$ is a finitely generated $B$-module
\[c_B(M\otimes_AB)=i(c_A(M)),\quad \rank_B(M\otimes_AB)=\rank_A(M).\] 
\end{itemize}
\end{proposition}
\begin{proof}
Let $\mathfrak{P}$ be a prime ideal of $B$ and $\p=\mathfrak{P}\cap A$, then $(M\otimes_AB)_{\mathfrak{P}}=M\otimes_AB_{\mathfrak{P}}$, and on the other hand
\[M\otimes_AB_{\mathfrak{P}}=(M\otimes_AA_\p)\otimes_{A_\p}B_{\mathfrak{P}}=M_\p\otimes_{A_\p}B_{\mathfrak{P}}.\]
the relation $M_\p=0$ is therefore equivalent to $(M\otimes_AB)_{\mathfrak{P}}=0$. It suffices to apply this remark to the ideal $\mathfrak{P}=(0)$ and the ideals $\mathfrak{P}\in P(B)$ to prove (a), taking account of the definition. To prove (b), we shall use the following lemma:
\begin{lemma}
Let $M_1$, $M_2$ be two finitely generated $A$-modules, and $\phi:M_1\to M_2$ an injective homomorphism. Then the kernel of $\phi\otimes 1:M_1\otimes_AB\to M_2\otimes_AB$ is pseudo-zero.
\end{lemma}
Let $\p$ be a prime ideal of $A$ of height $\leq 1$. Then $(M_i\otimes_AB)_\p=(M_i)_\p\otimes_{A_\p}B_\p$ and $(\phi\otimes 1_B)_\p=\phi_\p\otimes 1_{B_\p}$ the hypothesisthat $\phi$ is injective implies that so is $\phi_\p$. On the other hand, in view of the choice of $\p$, $A_\p$ is a principal ideal domain and $B_\p$ a finitely generated torsion-free $A_\p$ -module and hence free; we conclude that $\phi_\p\otimes 1_{B_\p}$ is itself injective. If $I=\ker(\phi\otimes 1)$, then $I_\p=\ker((\phi\otimes 1)_\p)$, therefore $I_\p=0$, whence a fortiori $I_{\mathfrak{P}}=(I_\p)_{\mathfrak{P}}=0$ for $\mathfrak{P}|\p$, which proves the lemma.\par
We return now to the proof of (b). For every finitely generated torsion Amodule $M$, write $\delta(M)=\chi_B(M\otimes_AB)$. It follows from (a) that, if $M$ is pseudo-zero, then $\delta(M)=0$. On the other hand, consider an exact sequence of finitely generated torsion $A$-modules:
\[\begin{tikzcd}
0\ar[r]&M_1\ar[r]&M_2\ar[r]&M_3\ar[r]&0
\end{tikzcd}\]
It follows from the above lemma that there is an exact sequence of $B$-modules:
\[\begin{tikzcd}
0\ar[r]&I\ar[r]&M_1\otimes_AB\ar[r]&M_2\otimes_AB\ar[r]&M_3\otimes_AB\ar[r]&0
\end{tikzcd}\]
where $I$ is pseudo-zero. Using the additivity of $\chi$ we therefore see $\delta$ is additive. We therefore conclude from \cref{Krull Noe uniqueness of the map chi} that there exists a homomorphism $\delta:\mathfrak{D}(A)\to\mathfrak{D}(B)$ such that $\delta(M)=\theta(\chi_A(M))$ for every finitely generated torsion $A$-module $M$. To prove that $\theta=i$, it suffices to show that $\delta(A/\p)=i(\p)$ for all $\p\in P(A)$. Now $(A/\p)\otimes_AB=B/\p B$ and, for all $\mathfrak{P}\in P(B)$, $(B/\p B)_{\mathfrak{P}}=B_{\mathfrak{P}}/\p B_{\mathfrak{P}}$. The last module is $0$ if $\mathfrak{P}$ does not lie over $\p$. If on the contrary $\mathfrak{P}|\p$, then $B_{\mathfrak{P}}/\p B_{\mathfrak{P}}$ is a $B_{\mathfrak{P}}$-module of length $e(\mathfrak{P}/\p)$ by definition of the ramification index. Therefore $\chi_B(B/\p B)=\sum_{\mathfrak{P}|\p}e_{\mathfrak{P}/\p}\mathfrak{P}=i(\p)$, which proves (b).\par
The second formula in (c) is immediate, for
\[(M\otimes_AB)\otimes_BL=M\otimes_AL=M\otimes_AK\otimes_KL\]
and the rank of $(M\otimes_AK)\otimes_KL$ over $L$ is equal to the rank of $M\otimes_AK$ over $K$. To show the first, consider a free submodule $H$ of $M$ such that $Q=M/H$ is a torsion $A$-module. Applying the lemma as above, we obtain an exact sequence of $B$-modules:
\[\begin{tikzcd}
0\ar[r]&I\ar[r]&H\otimes_AB\ar[r]&M\otimes_AB\ar[r]&Q\otimes_AB\ar[r]&0
\end{tikzcd}\]
where $I$ is pseudo-zero. It therefore follows from \cref{Krull Noe module divisor class prop}(a), (b) and (e) that
\[c_B(M\otimes_AB)=c_B(Q\otimes_AB)-c_B(\chi_B(Q\otimes_AB))=-c_B(i(\chi_A(Q)))=-i(c_A(\chi_A(Q)))=i(c_A(M))\]
by virtue of (b), which completes the proof.
\end{proof}
\subsection{Modules over Dedekind domains}
We now assume that $A$ is a Dedekind domain. Then we know that the ideals $\p\in P(A)$ are maximal and that they are the only nonzero prime ideals of $A$. The group $\mathfrak{D}(A)$ is identified with the group $\mathfrak{F}(A)$ of fractional nonzero ideals of $A$.
\begin{proposition}
Let $A$ be a Dedekind domain. Every pseudo-zero $A$-module is zero. Every pseudo-infective (resp. pseudo-surjeetive, pseudo-bijective, pseudo-zero) $A$-module homomorphism is injective (resp. smjeetive, bijective, zero).
\end{proposition}
\begin{proof}
The first assertion has already been shown; the others follow from it immediately.
\end{proof}
\begin{proposition}\label{Dedekind domain reflexive iff}
Let $A$ be a Dedekind domain and $M$ a finitely generated $A$-module. The following properties are equivalent:
\begin{itemize}
\item[(a)] $M$ is torsion-free;
\item[(b)] $M$ is reflexive;
\item[(c)] $M$ is projective.
\end{itemize}
\end{proposition}
\begin{proof}
We already know (with no hypothesis on the integral domain $A$) that (\rmnum{2}) implies (\rmnum{1}) and that (\rmnum{3}) implies (\rmnum{2}). If $M$ is torsion-free, it is identified with a lattice of $V=M\otimes_AK$ with respect to $A$. $M_\p$ is therefore a free $A_\p$-module for every maximal ideal $\p\in P(A)$, since $A_\p$ is a principal ideal domain. The conclusion then follows from \cref{module finite projective iff}.
\end{proof}
\begin{corollary}
Let $M$ be a finitely generated $A$-module and let $T$ be its torsion submodule. Then $T$ is a direct factor of $M$.
\end{corollary}
\begin{proof}
As $M/T$ is torsion-free and finitely generated, it is projective by \cref{Dedekind domain reflexive iff} and the corollary therefore follows.
\end{proof}
\begin{proposition}\label{Dedekind torsion module calssfication}
Let $A$ be a Dedekind domain and $T$ a finitely generated torsion $A$-module. There exist two finite families $(n_i)_{i\in I}$ and $(\p_i)_{i\in I}$ where the $n_i$ are positive integers and the $\p_i$ are elements of $P(A)$, such that $T$ is isomorphic to the direct sum $\bigoplus_{i\in I}(A/\p_i^{n_i})$. Further, the families $(n_i)_{i\in I}$ and $(\p_i)_{i\in I}$ are unique to within a bijection of the indexing set.
\end{proposition}
\begin{proof}
This follows from \cref{Krull Noe torsion module pseudo isomorphic thm} since a pseudo-isomorphism is here an isomorphism.
\end{proof}
\begin{proposition}\label{Dedekind torsion-free module calssfication}
Let $A$ be a Dedekind domain and $M$ a finitely generated torsion-free $A$-module of rank $n\geq 1$. Then there exists a nonzero ideal $\a$ of $A$ such that $M$ is isomorphic to the direct sum of the modules $A^{n-1}$ and $\a$. Moreover, the class of the ideal $\a$ is determined uniquely by this condition.
\end{proposition}
\begin{proof}
(CA, Theorem 6 of no.9) shows that there exists a free submodule $L$ of $M$ such that $M/L$ is isomorphic to an ideal $\b$ of $A$. If $\b=0$, we take $\a=A$. Otherwise, $\b$ is of rank $1$, hence $L=A^{n-1}$ and $\b$ is a projective module by \cref{Dedekind domain reflexive iff}. $M$ is therefore isomorphic to the direct sum of $L$ and $\b$, which proves the first part of the proposition. Moreover, it follows from \cref{Krull Noe module divisor class prop} that $c(M)=c(\a)$ whence the uniqueness of the class of $\a$.
\end{proof}
\chapter{Dimension and Hilbert function}
\section{Dimension of rings}
\subsection{Dimension of rings and modules}
\begin{definition}
Let $A$ be a ring. Then \textbf{dimension} of $A$, denoted by $\dim(A)$, is defined to be the Krull dimension of the topological space $\Spec(A)$. For a prime ideal $\p$ of $A$, the dimension of $A$ at $\p$, denoted by $\dim_\p(A)$, is defined to be $\dim_\p(\Spec(A))$.
\end{definition}
Recall that the map $\p\mapsto V(\p)$ is a bijection from prime ideals of $A$ to irreducible closed subsets of $\Spec(A)$. The dimension of $A$ is then the supremum of the length of chains of prime ideals of $A$. Since we do not consider the zero ring, $\dim(A)$ will always be a positive integer.\par
Let $\p\in\Spec(A)$, the sets $D(f)$, where $f$ runs through $A$, form a basis of the topology of $\Spec(A)$, and $\p$ belongs to the open set $D(f)$ if and only if $f$ does not belong to $\p$. Therefore, $\dim_\p(A)$ is the infimum of the numbers $\dim(A_f)$, where $f$ runs through $A-\p$ (\cref{Spec of ring induced map on spec(A_f)}).
\begin{example}[\textbf{Examples of dimension of rings}]\label{dimension of ring example}
\mbox{}
\begin{itemize}
\item[(a)] For $\dim(A)=0$, it is necessary and sufficient that every prime ideal of $A$ is maixmal. Thus a zero-dimensional integral domain is a field, and a zero-dimensional Noetherian ring is Artinian. 
\item[(b)] A Dedekind domain is a Noetherian integrally closed domain of dimension one. More generally, a ring is a finite product of Dedekind's rings if and only if it is Noetherian, reduced, integrally closed in its total ring of fractions, and of dimension one.
\item[(c)] Let $k$ be a field and $A$ an $k$-algebra that is integral over $k$. Then since $\dim(k)=0$ we have $\dim(A)=0$.
\item[(d)] For a ring $A$ we have
\[\dim(A[X])\geq\dim(A)+1.\]
In fact, if $\p_0\subset\cdots\subset\p_n$ is a chain of prime ideals of $A$, then we have a chain $\mathfrak{P}_0\subset\cdots\subset\mathfrak{P}_n\subset\mathfrak{P}_{n+1}$ of prime ideals of $A[X]$, where $\mathfrak{P}_i=\p_iA[X]$ for $1\leq i\leq n$ and $\mathfrak{P}_{n+1}=\p_nA[X]+XA[X]$. By the same reasoning, we prove the inequality $\dim(A\llbracket X\rrbracket)\geq\dim(A)+1$. We deduces by induction the inequalities
\[\dim(A[X_1,\dots,X_n])\geq\dim(A)+n,\quad \dim(A\llbracket X_1,\dots,X_n\rrbracket)\geq\dim(A)+n.\]
Later we will see that the equality holds if $A$ is a Noetherian ring.
\end{itemize}
\end{example}
\begin{proposition}\label{dimension of ring prop}
Let $A$ be a ring.
\begin{itemize}
\item[(a)] For any ideal $\a$ of $A$, we have $\dim(A/\a)\leq\dim(A)$.
\item[(b)] Let $S$ be a multiplicative subset of $A$, then $\dim(S^{-1}A)\leq\dim(A)$.
\item[(c)] We have $\dim(A)=\sup_\p\dim(A/\p)=\sup_{\m}\dim(A_\m)$, where $\p$ runs through minimal prime ideals of $A$ and $\m$ runs through maximal ideals of $A$.
\item[(d)] Assume that $A$ has finitely many minimal prime ideals. Then for any prime ideal $\p$ of $A$,
\[\dim_\p(A)=\sup_{\q}\dim_{\p/\q}(A/\q)\]
where $\q$ runs through minimal prime ideals of $A$ contained in $\p$.
\item[(e)] Let $\a$ be an ideal of $A$ which is not contained in any minimal prime ideal of $A$. Then we have $\dim(A)\geq\dim(A/\a)+1$. In particular, if $A$ is an integral domain, then $\dim(A)\geq\dim(A/\a)+1$ for any nonzero ideal $\a$ of $A$.   
\end{itemize}
\end{proposition}
\begin{proof}
By \cref{Spec of ring and quotient map}, $\Spec(A/\a)$ is homeomorphic to the subspace $V(\a)$ of $\Spec(A)$, and we get $\dim(A/\a)\leq\dim(A)$ therefore. Similarly, part (b) follows from \cref{Spec of ring induced map on spec(A_f)}. Also, by \cref{Spec of ring irreducible subset iff}, the irreducible components of $\Spec(A)$ are homeomorphic to $\Spec(A/\p)$, where $\p$ are minimal prime ideals of $A$. This proves (c), and (d) follows from \cref{topo space Krull dimension and cover}.\par
Finally, let $\a$ be an ideal of $A$ which is not contained in any minimal prime ideal of $A$. We prove that, for any chain $\p_0\subset\cdots\subset\p_n$ of prime ideals of $A$ such that $\a\sub\p_0$ we have $\dim(A)\geq n+1$. By the hypothesis on $\a$ we see $\p_0$ is not minimal, so there exists a prime ideal $\p_{-1}$ of $A$ contained in $\p_0$, distinct from $\p_0$, and $\p_{-1}\subset\p_0\subset\cdots\subset\p_0$ is a chain of prime ideals of $A$, whence $\dim(A)\geq n+1$.
\end{proof}
\begin{corollary}\label{dimension of ring extension prop}
Let $\rho:A\to B$ be a ring homomorphism. Then $\dim(B)$ is the supremum of the numbers $\dim(B/\p^e)$, where $\p$ covers the set of minimal prime ideals of $A$.
\end{corollary}
\begin{proof}
In fact, for each minimal prime ideal $\mathfrak{P}$ of $B$, there exists a minimal prime ideal $\p$ of $A$ contained in $\mathfrak{P}^c$, and therefore
\[\dim(B/\mathfrak{P})\leq\dim(B/\p^e)\leq\dim(B)\]
whence the claim by \cref{dimension of ring prop}. 
\end{proof}
For an ideal $\a$ of $A$, we define the \textbf{height} (resp. \textbf{coheight}) $\height(\a)$ (resp. $\coht(\a)$) of $\a$ to be the codimension (resp. dimension) of $V(\a)$ in $\Spec(A)$. Also, the ring is called \textbf{catenary} if $\Spec(A)$ is catenary.
\begin{proposition}\label{height of ring prop}
Let $A$ be a ring.
\begin{itemize}
\item[(a)] The height of a prime ideal $\p$ is the supremum of the length of chains $\p_0\sub\cdots\sub\p_n$ of prime ideals of $A$ such that $\p_n=\p$.
\item[(b)] Let $\p$ be a prime ideal of $A$ and $\a$ an ideal of $A$. Then $\dim(A_\p/\a A_\p)=-\infty$ if $\a$ is not contained in $\p$, and $\dim(A_\p/\a A_\p)=\codim(V(\p),V(\a))$ if $\a$ is contained in $\p$. In particular, for a prime ideal $\p$ of $A$, we have $\dim(A_\p)=\height(\p)$.
\item[(c)] For any ideal $\a$ of $A$, we have $\height(\a)=\inf_{\p\sups\a}\height(\p)=\inf_{\p\sups\a}\dim(A_\p)$.
\end{itemize}
\end{proposition}
\begin{proof}
Assertion (a) follows from definition, and (b) follows from the fact that the map $\q\mapsto\q(A_\p/\a A_\p)$ is an increasing isomorphism of the set of prime ideals $\q$ of $A$ such that $\a\sub\q\sub\p$ on the set of prime ideals of the local ring $A_\p/\a A_\p$. Now let $\a$ be an ideal of $A$. Then the irreducible closed subset of $V(\a)$ are the sets $V(\p)$, where $\p$ is a prime ideal of $A$ containing $\a$. Part (c) then follows from the definition of codimension.
\end{proof}
\begin{corollary}\label{height of ring localization prime}
Let $\p$ be a prime ideal of $A$ and $S$ a multiplicative subset of $A$ not meeting $\p$. Then $\height(\p)=\height(S^{-1}\p)$. 
\end{corollary}
\begin{proof}
This follows from \cref{localization and ideals}.
\end{proof}
\begin{proposition}\label{dimension of ring codim inequality}
Let $A$ be a ring. Let $\a,\b$ be proper ideals of $A$ such that $\a\sub\b$. Then
\[\codim(V(\b)),V(\a))\leq\dim(A/\a)-\dim(A/\b)\]
In particular, for any ideal $\a$ of $A$, we have $\height(\a)+\dim(A/\a)\leq\dim(A)$.
\end{proposition}
\begin{proof}
This from \cref{topo space codim prop} and of the relations $\dim(A/\a)=\dim(V(\a))$, $\dim(A/\b)=\dim(V(\b))$.
\end{proof}
\begin{proposition}\label{ring catenary iff triple if tuple}
Let $A$ be a ring.
\begin{itemize}
\item[(a)] For $A$ to be catenary, it is necessary and sufficient that, for every triple $(\p,\q,\r)$ of prime ideals of $A$ such that $\r\sub\q\sub\p$, we have
\[\dim(A_\p/\q A_\p)+\dim(A_\q/\r A_\q)=\dim(A_\p/\r A_\p).\]
\item[(b)] If $A$ is an integral domain with finite dimension, then for $A$ to be catenary, it is necessary and sufficient that, for every pair $(\p,\q)$ of prime ideals of $A$ such that $\sub\q\sub\p$, we have
\[\height(\q)+\dim(A_\p/\q A_\p)=\height(\p).\]
\end{itemize}
\end{proposition}
\begin{proof}
This follows directly from \cref{topo space catenary iff} and \cref{topo space irreducible finite dim catenary iff}.
\end{proof}
\begin{proposition}\label{ring of finite dimension chain same length prop}
Let $A$ be a ring of finite dimension such that all maximal chains of prime ideals have the same length. Then $A$ is catenary, and for any prime ideal $\p$ of $A$ we have
\[\height(\p)=\dim(A)-\coht(\p).\]
Moreover, for any pair $(\p,\q)$ of prime ideals such that $\q\sub\p$, we have
\[\dim(A_\p/\q A_\p)=\coht(\q)-\coht(\p).\]
\end{proposition}
\begin{proof}
This is a reformulation of \cref{topo space maximal chain same length is catenary}.
\end{proof}
Now let $M$ be a finitely generated $A$-module. The \textbf{dimension} (resp. \textbf{codimension}) of $M$, denoted by $\dim_A(M)$ (resp. $\codim(M)$), is defined to be the dimension (resp. codimension) of the support of $M$. The support of the $A$-module $A$ is $\Spec(A)$, so the dimension of the $A$-module $A$ equals the dimension of the ring $A$. Also, since $M$ is finitely generated, we have
\[\supp(M)=V(\a)=\supp(A/\a),\]
where $\a$ is the annihilator of $M$. Consequently, the dimension of the $A$-mqdule $M$, the dimension of the $A$-module $A/\a$, the dimension of the ring $A/\a$ and the dimension of the $(A/\a)$-module $M$ are all the supremum of the set of lengths of the chains $\p_0\subset\cdots\subset\p_n$ of prime ideals of $A$ such that $\a\sub\p_0$. By \cref{dimension of ring prop}(c), the dimension of $M$ is also the supremum of the dimensions of the rings (or of the $A$-modules) $A/\p$, where $\p$ runs through the set of prime ideals of $A$, minimal among those which contain $\a$.
\begin{example}
If $A$ is Noetherian, then by \cref{associated prime maximal iff finite length}, we see $\dim_A(M)=0$ if and only $\supp(M)$ consists of maximal ideals, or equivalently $M$ is of finite length.
\end{example}
\begin{example}\label{dimension of module sup of dim(A/p)}
If $M$ is a finitely generated module over a Noetherian ring $A$, $\dim_A(M)$ is then the supremum of the numbers $\dim(A/\p)$, where $\p$ runs through the set $\Ass_A(M)$ of prime ideals of $A$ associated with $M$ (\cref{associated prime and supp}).
\end{example}
\begin{proposition}\label{dimension of module prop}
Let $A$ be a ring and $M$ a finitely generated $A$-module.
\begin{itemize}
\item[(a)] For each $\p\in\supp(M)$, we have $\dim_{A_\p}(M_\p)=\codim(V(\p),\supp(M))$.
\item[(b)] We have $\dim_A(M)=\sup_{\p}\dim_{A_\p}(M_\p)$, where $\p$ runs through prime ideals in $\supp(M)$ (resp. maximal ideal in $\supp(M)$).
\item[(c)] Let $0\to M'\to M\to M''\to 0$ be an exact sequence of finitely generated $A$-modules. Then $\dim_A(M)=\sup\{\dim_A(M'),\dim_A(M'')\}$.
\end{itemize}
\end{proposition}
\begin{proof}
Let $\a$ be the annihilator of $M$, then the annihilator of $M_\p$ is $\a A_\p$, whence $\dim_{A_\p}(M_\p)=\dim(A_\p/\a A_\p)$, and the claim follows from \cref{height of ring prop} and $\supp(M)=V(\a)$. Also, part (b) follows from (a), and (c) follows from \cref{supp of module and exact sequence}.
\end{proof}
Now let $A$ be a Noetherian ring. Let $Z(A)$ be the free $\Z$-module generated by the set of closed irreducible subsets of $\Spec(A)$. For any irreducible closed subset $Y$ of $\Spec(A)$, we denote by $[Y]$ the element correspondent of $Z(A)$. The elements of $Z(A)$ are sometimes called \textbf{cycles}. Let $M$ be a finitely generated $A$-module. For any prime ideal $\p$ of $A$ which is a minimal element of $\supp(M)$, we have $0<\ell_{A_\p}(M_\p)<+\infty$ by \cref{associated prime of finite length localization iff}. We set
\[z(M)=\sum_{\p}\ell_{A_\p}(M_\p)\cdot[V(\p)]\]
where $\p$ runs through the finite set of minimal element of $\supp(M)$.
\begin{example}
For any $\p\in\Spec(A)$, we have $z(A/\p)=[V(\p)]$. More generally, let $M$ be a finitely generated $A$-module, and let $(M_i)_{1\leq i\leq n}$ be a chain of $M$ such that for each $i$, the modulus $M_i/M_{i+1}$ is isomorphic to $A/\p_i$, where $\p_i$ is a prime ideal of $A$. Then we have $z(M)=\sum_{i\in J}\ell_i[V(\p_i)]$, where $J$ is the subset of $I$ consists of $i$ such that $\p_i$ be a minimal element of $\{\p_0,\dots,\p_{n-1}\}$, and $\ell_i$ is the number of indices $j$ such that $\p_j=\p_i$.
\end{example}
For any integer $d$, let $Z_{\leq d}$ (resp. $Z_d$, resp. $Z^{\geq d}$, resp. $Z^d$) be the $\Z$-submodule of $Z(A)$ generated by the elements $[V(\p)]$ where $\p$ is a prime ideal of $A$ such that $\coht(\p)\leq d$ (resp. $\coht(\p)=d$, resp. $\height(\p)\geq d$, resp. $\height(\p)=d$). An element of $Z_d$ (resp. $Z^d$) is called a \textbf{cycle} of dimension $d$ (resp. codimension $d$). We have
\[Z_{\leq d}=Z_{\leq d-1}\oplus Z_d,\quad Z^{\geq d}=Z^{\geq d-1}\oplus Z^d.\]
Let $\mathcal{C}$ be the set of classes of finitely generated $A$-modules, and for each integer $d$, let $C_{\leq d}$ (resp. $\mathcal{C}^{\geq d}$) be the subgroup generated by finitely generated $A$-modules of dimension $\leq d$ (resp. of codimension $\geq d$).
\begin{lemma}\label{cycle of module projection on Z}
Let $M$ be a finitely generated $A$-module.
\begin{itemize}
\item[(a)] For $M$ be in $\mathcal{C}_{\leq d}$, it is necessary and sufficient that $z(M)\in Z_{\leq d}$. In this case, the projection $z_d(M)$ of $z(M)$ on $Z_d$ parallel to $Z_{\leq d-1}$ is given by
\[z_d(M)=\sum_{\coht(\p)=d}\ell_{A_\p}(M_\p)\cdot[V(\p)].\] 
\item[(b)] For $M$ be in $\mathcal{C}^{\geq d}$, it is necessary and sufficient that $z(M)\in Z^{\geq d}$. In this case, the projection $z^d(M)$ of $z(M)$ on $Z^d$ parallel to $Z^{\geq d-1}$ is given by
\[z^d(M)=\sum_{\height(\p)=d}\ell_{A_\p}(M_\p)\cdot[V(\p)].\]
\end{itemize}
\end{lemma}
\begin{proof}
For $M$ to be dimension $\leq d$, it is necessary and sufficient that for any minimal prime $\p$ of $\supp(M)$, we have $\coht(\p)\leq d$, which means $z(M)\in Z_{\leq d}$. Suppose now $\dim(M)\leq d$, and let $\p\in\Spec(A)$ be such that $\coht(\p)=d$. Then, either $\p\notin\supp(M)$ and $M_\p=0$, or $\p\in\supp(M)$ and $\p$ is a minimal element of $\supp(M)$. The coefficient of $[V(\p)]$ in $z(M)$ is in both cases $\ell_{A_\p}(M_\p)$, whence (a). Part (b) is proved analogously, noting that a finitely generated module $M$ is in $\mathcal{C}^{\geq d}$ if and only if we have $M_\p=0$ for any prime ideal $\p$ of height $<d$.
\end{proof}
By \cref{supp of module and exact sequence}, the sets $\mathcal{C}_{\leq d}$ and $C^{\geq d}$ are hereditary, and we can consider the corresponding Grothendieck groups $K(\mathcal{C}_{\leq d})$ and $K(\mathcal{C}^{\geq d})$, which are subgroups of $K(\mathcal{C})$, the Grothendieck group of finitely generated $A$-modules. By \cref{cycle of module projection on Z}, the functions $z_d$ and $z^d$ are additive, whence extends to homomorphisms
\[\zeta_d:K(\mathcal{C}_{\leq d})\to Z_d,\quad \zeta^d:K(\mathcal{C}^{\geq d})\to Z^d\]
Moreover, since $\mathcal{C}_{\leq d-1}\sub\mathcal{C}_{\leq d}$ and $\mathcal{C}^{\geq d+1}\sub\mathcal{C}^{\geq d}$, we have canonical homomorphisms
\[i_d:K(\mathcal{C}_{\leq d-1})\to K(\mathcal{C}_{\leq d}),\quad i^d:K(\mathcal{C}^{\geq d+1})\to K(\mathcal{C}^{\geq d}).\]
\begin{proposition}\label{cycle of module exact sequence of K}
The following sequence of $\Z$-modules
\[\begin{tikzcd}
K(\mathcal{C}_{\leq d-1})\ar[r,"i_d"]&K(\mathcal{C}_{\leq d})\ar[r,"\zeta_d"]&Z_d\ar[r]&0
\end{tikzcd}\]
\[\begin{tikzcd}
K(\mathcal{C}^{\geq d+1})\ar[r,"i^d"]&K(\mathcal{C}^{\geq d})\ar[r,"\zeta^d"]&Z^d\ar[r]&0
\end{tikzcd}\]
are exact.
\end{proposition}
\begin{proof}
We have $\zeta_d\circ i_d=0$ by \cref{cycle of module projection on Z}. For every $\p\in\Spec(A)$ such that $\coht(\p)=d$, we see $\zeta_d([A/\p])=z_d(Z/\p)=[V(\p)]$, so the homomorphism $\zeta_d$ is surjective. Now by \cref{associated prime of Noe chain of submodule}, $K(\mathcal{C}_{\leq d})$ is generated by the elements $[A/\p]$, where $p\in\Spec(A)$ and $\coht(\p)\leq d$. Therefore, every element $\xi$ can be written into $\xi=i_d(\eta)+\sum_{i=1}^{k}n_i[A/\p_i]$, where $\eta\in K(\mathcal{C}_{\leq d-1})$, $n_i\in\Z$ and $\coht(\p_i)=d$ for each $i$. From this, we see $\zeta_d(\xi)=\sum_{i=1}^{k}n_i[V(\p_i)]$, and $\zeta_d(\xi)=0$ if and only if $\xi\in\im i_d$, which proves the claim. The proof for $i^d$ and $\zeta^d$ can be done similarly.
\end{proof}
\begin{example}
Let $A$ be a Noetherian integral domain. Then $Z^0=\Z\Spec(A)$ and $z^0(M)=\rank(M)\cdot\Spec(A)$. Therefore the modules in $\mathcal{C}^{\geq 1}$ are the torsion modules.
\end{example}
\begin{example}
Assume that $A$ is Noetherian and integrally closed. Then $Z^1$ is identified with the group $\mathfrak{D}(A)$ of divisors of $A$. The modules in $\mathcal{C}^{\geq 2}$ are therefore pseudo-zero modules (\cref{Krull Noe localization at prime ideals height 1 zero iff}). If $M$ is a finitely generated torsion module, then $z^1(M)\in Z^1\mathfrak{D}(A)$ is the content $\chi(M)$ of $M$. \cref{Krull Noe content of module prop} and \cref{Krull Noe uniqueness of the map chi} is then equivalent to the exactness of the sequence $\begin{tikzcd}[column sep=15pt]K(\mathcal{C}^{\geq 2})\ar[r]&K(\mathcal{C}^{\geq 1})\ar[r]&Z^1\ar[r]&0\end{tikzcd}$.
\end{example}
\begin{example}
The elements in $\mathcal{C}_{0}$ are finitely generated $A$-modules of dimension $0$, that is, of finite length. Let $\epsilon:Z_0\to\Z$ be the evaluation map, then $\ell_A(M)=\epsilon(z_0(M))$, in view of \cref{module of finite length equals sum of localization length}.
\end{example}
\begin{example}
Let $A$ be a integral domain of finite dimension. Let $d=\dim(A)$, then $\mathcal{C}_{\leq d}=\mathcal{C}$ and $Z_d=\Z\Spec(A)$. Also, we have $z_d(M)=\rank(M)\Spec(A)=z^0(M)$, whence the modules in $\mathcal{C}_{\leq d-1}$ are precisely the torsion $A$-modules.
\end{example}
\subsection{Dimension and going-down property}
Let $\rho:A\to B$ be a ring homomorphism. Recall that we say $\rho$ has the going-down property if it satisfies the going down theorem. 
\begin{proposition}\label{ring homomorphism going-down irreducible component under map}
Let $\rho:A\to B$ be a ring homomorphism with the going-down property. Let $F$ be a closed subset of $\Spec(A)$. If $Y$ is an irreducible component of the inverse image of $F$ under $\rho^*:\Spec(B)\to\Spec(A)$, then the closure of $\rho^*(Y)$ is an irreducible component of $F$.
\end{proposition}
\begin{proof}
Let $\a$ be an ideal of $A$ such that $F=V(\a)$. By \cref{Spec of ring induced map prop}, the inverse image by $\rho^*$ of $F$ is the subset $V(\a^e)$ of $\Spec(B)$. Let $Y=V(\mathfrak{P})$ be an irreducible component of $V(\a^e)$, then the closure of $\rho^*(Y)$ is the irreducible closed subset $V(\mathfrak{P}^c)$. So we have to prove that, if $\mathfrak{P}$ is a prime ideal minimal among the prime ideals of $B$ containing $\a^e$, then $\mathfrak{P}^c$ is minimal among the prime ideals of $A$ containing $\a$.\par
Suppose the contrary, then there would exist a prime ideal $\q$ of $A$ with $\a\sub\q\subset\mathfrak{P}^c$. According to the going-down property, there exist a prime ideal $\mathfrak{Q}$ of $B$ lying over $\q$ such that $\mathfrak{Q}\subset\mathfrak{P}$, whence $\a^e\sub\mathfrak{Q}\subset\mathfrak{P}$, contrary to the hypothesis made on $\mathfrak{P}$.
\end{proof}
\begin{proposition}\label{ring homomorphism going-down dim inequality}
Let $\rho:A\to B$ be a ring homomorphism with the going-down property. Then we have the inequality
\begin{align}\label{ring homomorphism going-down dim inequality-1}
\dim(B)\geq\dim(A)+\inf_{\m\in\Max(A)}\dim(B/\m B)\geq\dim(A)+\inf_{\p\in\Spec(A)}\dim(B\otimes_A\kappa(\p)).
\end{align}
\end{proposition}
\begin{proof}
We have $\dim(A)=\sup_\m\dim(A_\m)$, so it suffices to show the inequality
\begin{align}\label{ring homomorphism going-down dim inequality-2}
\dim(B)\geq\dim(A_\m)+\dim(B/\m B)
\end{align}
for each maximal ideal $\m$ of $A$. In other words, we have to prove the inequality $\dim(B)\geq n+r$ if $\p_0\subset\cdots\subset\p_n$ is a chain of prime ideals of $A$ contained in $\m$ and $\widebar{\mathfrak{Q}}_0\subset\cdots\subset\widebar{\mathfrak{Q}}_r$ is a chain of prime ideals of $B/\m B$. For $1\leq i\leq r$, there exists a prime ideal $\mathfrak{Q}_{n+i}$ of $B$ containing $\m B$ such that $\widebar{\mathfrak{Q}}_i=\mathfrak{Q}_{n+i}/\m B$, and $\mathfrak{Q}_{n}\subset\cdots\subset\mathfrak{Q}_{n+r}$ is a chain of prime ideals of $B$. Let $\mathfrak{Q}$ be a prime ideal lying over $\m$, then by the going down property, there exists prime ideals $\mathfrak{Q}_0\subset\cdots\subset\mathfrak{Q}_{n-1}\subset\mathfrak{Q}$ such that $\mathfrak{Q}_i$ is lying over $\p_i$ for $0\leq i\leq n-1$. Then we see $\mathfrak{Q}_0\subset\cdots\subset\mathfrak{Q}_{n-1}\subset\mathfrak{Q}_{n}\subset\cdots\subset\mathfrak{Q}_{n+r}$ is a chain of prime ideals of $B$, whence $\dim(B)\geq n+r$.
\end{proof}
\begin{corollary}\label{ring homomorphism going-down height inequality}
Let $\rho:A\to B$ be a ring homomorphism with the going-down property. Then for any ideal $\a$ of $A$ we have $\height(\a)\leq\height(\a^e)$ .
\end{corollary}
\begin{proof}
Let $\mathfrak{P}$ be a prime ideal of $B$ containing $\a^e$, and $\p=\mathfrak{P}^c$. By \cref{ring homomorphism condition PM prop}, the homomorphism $\rho_\p:A_\p\to B_{\mathfrak{P}}$ satisfies condition (PM), whence $\dim(A_\p)\leq\dim(B_{\mathfrak{P}})$ by \cref{ring homomorphism going-down dim inequality}. But $\height(\a)\leq\dim(A_\p)$ since $\p$ contains $\a$, so $\height(\a)\leq\dim(B_{\mathfrak{P}})$ for any prime ideal $\mathfrak{P}$ containing $\a^e$, whence $\height(\a)\leq\height(\a^e)$.
\end{proof}
\begin{remark}
Let $\rho:A\to B$ be a local homomorphism of Noetherian local rings satisfying the going down property. Then (\cref{Noe local ring homomorphism dim and secant prop}) the equality holds in (\ref{ring homomorphism going-down dim inequality-1}). On the other hand, the inequality in (\ref{ring homomorphism going-down dim inequality-1}) can be strict (c.f. \cref{ring homomorphism going down prop strict inequality}).
\end{remark}
\subsection{Dimension of finite type algebras}
\begin{proposition}\label{dimension ring extension plus fiber inequality}
Let $\rho:A\to B$ be a ring homomorphism. If $n=\sup_{\p}\dim(B\otimes_A\kappa(\p))$ with $\p$ taking over prime ideals of $A$, then we have the inequality
\[\dim(B)+1\leq(n+1)(\dim(A)+1).\]
In particular, if $\dim(A)$ is finite and $\dim(B\otimes_A\kappa(\p))$ is bounded from above, then $\dim(B)$ is finite.
\end{proposition}
\begin{proof}
We may suppose that $\dim(A)<+\infty$ and $n<+\infty$. Let $\mathfrak{P}_0\subset\cdots\subset\mathfrak{P}_m$ be a chain of prime ideals of $B$ and set $\p_i=\mathfrak{P}_i^c$. Then the sequence $\p_i$ is increasing, so the set of its values is of cardinal $\leq\dim(A)+1$. For each $\p\in\Spec(A)$, the set of $\mathfrak{P}_i$ such that $\mathfrak{P}_i=\p$ is a chain of the subset $\rho^{*-1}(\p)$ of $\Spec(B)$, therefore has cardinality less than $\dim(B\otimes_A\kappa(\p))+1$, and concequent to $(n+1)$. It follows that $m+1\leq(n+1)(\dim(A)+1)$, hence the proposition.
\end{proof}
\begin{corollary}\label{dimension of polynomial ring upper bound}
Let $A$ be a ring, then
\[\dim(A)+1\leq\dim(A[X])\leq 2\dim(A)+1.\]
\end{corollary}
\begin{proof}
One direction is established. For the other, note that for any prime ideal $\p\in\Spec(A)$, $A[X]\otimes_A\kappa(\p)$ is isomorphic to $\kappa(\p)[X]$, which is a PID hence of dimension $1$. Thus the other side comes from \cref{dimension ring extension plus fiber inequality}.
\end{proof}
\begin{corollary}\label{algebra finite dimension finite}
Let $A$ be a ring and $B$ a finitely generated $A$-algebra. If $\dim(A)<+\infty$, then we also have $\dim(B)<+\infty$.
\end{corollary}
\begin{proof}
It follows from \cref{dimension of polynomial ring upper bound} that the polynomial ring $A[X_1,\dots,X_n]$ has finite dimension for all $n$, whence $\dim(B)<+\infty$.
\end{proof}
\begin{lemma}\label{integral fiber map contraction of prime prop}
Let $\rho:A\to B$ be a ring homomorphism such that, for each prime ideal $\p$ of $A$, the algebra $B\otimes_A\kappa(\p)$ is integral over $\kappa(\p)$. If $\mathfrak{P}$ and $\mathfrak{Q}$ are prime ideals of $B$ such that $\mathfrak{P}\subset\mathfrak{Q}$, then $\mathfrak{P}^c\neq\mathfrak{Q}^c$. 
\end{lemma}
\begin{proof}
If $\mathfrak{P}^c=\mathfrak{Q}^c=\p$, then $\mathfrak{P}\subset\mathfrak{Q}$ is a chain in $B\otimes_A\kappa(\p)$, whence $\dim(B\otimes_A\kappa(\p))\geq 1$. But since $\kappa(\p)$ is a field and $B\otimes_A\kappa(\p)$ is integral over $\kappa(\p)$, we have $\dim(B\otimes_A\kappa(\p))=0$, a contradiction.
\end{proof}
\begin{theorem}\label{integral extension dimension of module ideal prop}
Let $\rho:A\to B$ be a ring homomorphism such that $B$ is integral over $A$, and $\rho^*:\Spec(B)\to\Spec(A)$ the induced map.
\begin{itemize}
\item[(a)] Let $M$ be a finitely generated $A$-module. Then $\dim_B(M\otimes_AB)\leq\dim_A(M)$. If $\rho^*$ is surjective, then $\dim_B(M\otimes_AB)=\dim_A(M)$.
\item[(b)] Let $\b$ be an ideal of $B$. Then $\height(\b)\leq\height(\b^c)$ and $\dim(B/\b)=\dim(A/\b^c)$. If $\rho^*$ is surjective, then $\height(\a^e)\leq\height(\a)$ for any ideal $\a$ of $A$.
\item[(c)] Suppose that $B$ is a finitely generated $A$-module and let $N$ be a finitely generated $B$-module. Then we have $\dim_B(N)=\dim_A(N)$. In particular, we have $\dim(B)=\dim_A(B)$.
\end{itemize}
\end{theorem}
\begin{proof}
By \cref{integral tensor of two integral algebra}, the algebra $B\otimes_A\kappa(\p)$ is integral over $\kappa(\p)$. Let $\mathfrak{P}_0\subset\cdots\subset\mathfrak{P}_m$ be a chain of prime ideals of $B$. Then by \cref{integral fiber map contraction of prime prop}, the ideals $\p_i=\mathfrak{P}_i^c$ are pairwise distinct, whence $\p_0\subset\cdots\subset\p_m$ is a chain of prime ideals of $A$, and $m\leq\dim(A)$. This proves $\dim(B)\leq\dim(A)$.\par
Suppose further that $\rho^*$ is surjective. Let $\p_0\subset\cdots\subset\p_n$ be a chain of prime ideals of $A$. Then there exists a prime ideal $\mathfrak{P}_0$ of $B$ lying over of $\p_0$. According to \cref{integral ring lying over extend prime}, we can construct, by induction on $n$, a chain $\mathfrak{P}_0\subset\cdots\subset\mathfrak{P}_n$ of prime ideals of $B$ such that $\mathfrak{P}_i$ is lying over $\p_i$ for each $i$. We therefore have $n\leq\dim(B)$ and consequently $\dim(A)=\dim(B)$.\par
This proves (a) in the case $M=A$. In the general case, let $\a$ be the annihilator of $M$, so that the support of $M$ is identified with $\Spec(A/\a)$, and we have $\dim_A(M)=\dim(A/\a)$. According to \cref{supp of module extension ring}, the support of $M\otimes_AB$ is the inverse image by $\rho^*$ of the support of $M$, therefore identifies with $\Spec(B/\a^e)$, and we have $\dim_B(M\otimes_AB)=\dim(B/\a^e)$. It remains to notice that the homomorphism $\bar{\rho}:A/\a\to B/\a^e$ deduced from $\rho$ makes $B/\a^e$ an integral $(A/\a)$-algebra, and that $\bar{\rho}^*$ is surjective when $\rho^*$ is.\par
Now we prove (b). By the definition of height, it suffices to prove that $\height(\b)\leq\dim(A_\p)$ for any prime ideal $\p$ of $A$ containing $\b^c$. Let $\p$ be such an ideal, according to \cref{integral ring lying over extend prime}, there exists a prime ideal $\mathfrak{P}$ of $B$ above $\p$ and containing $\b$, and we have $\height(\b)\leq\dim(B_{\mathfrak{P}})$. Now $B_{\mathfrak{P}}$ is identified with a ring of fractions of the integral $A_\p$-algebra $B\otimes_AA_\p$, from which we get
\[\dim(B_{\mathfrak{P}})\leq\dim(B\otimes_AA_\p)\leq\dim(A_\p)\]
by (a) and \cref{dimension of ring prop}. We have thus proved the inequality $\height(\b)\leq\height(\b^c)$. Moreover, the homomorphism of $A/\a$ in $B/\b$ deduced from $\rho$ is injective and makes $B/\b$ an integral $(A/\a)$-algebra. We therefore have $\dim(B/\b)=\dim(A/\a)$ according to (a). Assume now $\rho^*$ surjective and let $\a$ be an ideal of $A$ and $\p$ a prime ideal of $A$ containing $\a$. There exists by hypothesis a prime ideal $\mathfrak{P}$ of $B$ above $\p$. We have $\a^e\sub\mathfrak{P}$, hence $\height(\a^e)\leq\height(\mathfrak{P})\leq\height(\p)$ according to the above argument. Passing to the infimum, we obtain $\height(\a^e)\leq\height(\a)$. Now part (c) follows from (b) by letting $\b$ be the annihilator of $N$ and noting that the annihilator of $\rho^*(N)$ is then $\b^c$.
\end{proof}
\begin{theorem}\label{integrally closed integral extension height prop}
Let $A$ be an integrally closed ring, $B$ a ring containing $A$ and integral over $A$. Suppose that $B$ is a torsion-free $A$-module, then
\begin{itemize}
\item[(a)] For any ideal $\a$ of $A$, we have $\height(\a)=\height(\a^e)$. 
\item[(b)] For any ideal $\b$ of $B$, we have $\height(\b)=\height(\b^c)$. 
\end{itemize}
\end{theorem}
\begin{proof}
Let $\rho:A\to B$ be the canonical homomorphism. Let $\a$ be an ideal of $A$. If $\a=A$, then (a) is clear by \cref{integral extension dimension of module ideal prop}. Suppose $\a\neq A$, so that $\a^e\neq B$. Since $\rho$ is injective, $\rho^*$ is surjective by \cref{integral ring lying over prime exist}. Let $\mathfrak{P}$ be a prime ideal of $B$ containing $\a^e$ and $\p=\mathfrak{P}\cap A$. We have $\a\sub\p$, whence $\height(\a)\leq\height(\p)$. Let $\p_0\subset\cdots\subset\p_n$ be a chain of prime ideals of $A$ with $\p_n=\p$. According to the going down theorem (\cref{integrally closed ring extension going down prop}), we have a chain $\mathfrak{P}_0\subset\cdots\subset\mathfrak{P}_n$ of prime ideals of $B$ such that $\mathfrak{P}_n=\mathfrak{P}$ and $\mathfrak{P}_i$ is above $\p_i$. We then have $n\leq\height(\mathfrak{P})$, whence $\height(\a)\leq\height(\mathfrak{P})$. By taking infimum we get $\height(\a)\leq\height(\a^e)$. The inequality $\height(\a^e)\leq\height(\a)$ follows from \cref{integral extension dimension of module ideal prop}, whence the first equality.\par
Let $\b$ be an ideal of $B$ and $\a=\b^c$. We have $\a^e\sub\b$, hence $\height(\a)=\height(\a^e)\leq\height(\b)$. The inequality $\height(\b)\leq\height(\a)$ follows from \cref{integral extension dimension of module ideal prop}, hence the theorem.
\end{proof}
\subsection{Dimension of finite type algebras over a field}
As a special case of the materials discussed in the last paragraph, we now consider finite type algebras over a field $k$. As we shall see, the dimension theory for such algebras is extremely simple and beautiful as it connects with the transcendental degree of such algebras.
\begin{lemma}[\textbf{Noether's Normalization Theorem}]\label{algebra finite over field maximal chain normalization}
Let $A$ be a finitely generated $k$-algebra and $\p_0\subset\cdots\subset\p_m$ a maximal chain of prime ideals of $A$. Then there exists an integer $n\geq m$ and a sequence $(x_i)_{1\leq i\leq n}$ of elements of $A$ algebraically independent over $k$ such that
\begin{itemize}
\item[(a)] $A$ is integral over $B=k[x_1,\dots,x_n]$.
\item[(b)] The ideal $\p_j\cap B$ is generated by $x_1,\dots,x_{n-m+j}$. 
\end{itemize}
\end{lemma}
\begin{proof}
By \cref{algebra finite over field normalization}, there exists an integer $n\geq 0$, a sequence $(x_i)_{1\leq i\leq n}$ of elements of $A$ algebraically independent over $k$ and an increasing sequence $(h_j)_{1\leq j\leq m}$ such that $A$ is integral over $B=k[x_1,\dots,x_n]$ and $\q_j=\p_j\cap B$ is generated by $x_1,\dots,x_{h_j}$. By passing to the quotients, we deduce from the injective canonical homomorphism of $B$ into $A$ an injective homomorphism from $B/\q_j$ into $A/\p_j$ which makes $A/\p_j$ an integral $(B/\q_j)$-algebra. As the ring $B/\q_j$ is isomorphic to a polynomial algebra in $n-h_j$ indeterminate over $k$, it is integrally closed. According to \cref{integrally closed integral extension height prop}, we therefore have
\[1=\height(\p_{j+1}/\p_j)=\height(\q_{j+1}/\q_j)\geq h_{j+1}-h_j.\]
whence $h_{j+1}\leq h_j+1$. Since $h_j$ is increasing and the ideals $\q_j$ are pairwise distinct, we get $h_{j+1}=h_j+1$. Also note that $h_m=n$ since $\q_m$ is maximal ($\p_m$ is maximal), so $h_j=n-m+j$ and the claim follows.
\end{proof}
\begin{theorem}\label{algebra finite over field dimension prop}
Let $A$ be a finitely generated $k$-algebra.
\begin{itemize}
\item[(a)] For every minimal prime ideal $\p$ of $A$, the maximal chains of prime ideals of $A$ starting from $\p$ have length $\tr.\deg_k(\kappa(\p))$.
\item[(b)] The ring $A$ is catenary and its dimension is the supremum of the integers $\tr.\deg_k(\kappa(\p))$, where $\p$ runs through minimal prime ideals of $A$.
\item[(c)] Suppose that $A$ is an integral domain. Then the maximal chains of prime ideals of $A$ have the same length, and the dimension of $A$ equals to $\tr.\deg_k(K)$, where $K$ is the field of fraction of $A$.
\end{itemize}
\end{theorem}
\begin{proof}
Suppose that $A$ is an integral domain and consider a maximal chain $\p_0\subset\cdots\subset\p_m$ of prime ideals of $A$. We have $\p_0=0$. We then deduce from \cref{algebra finite over field maximal chain normalization} the existence of a injective homomorphism $\varphi:k[X_1,\dots,X_m]\to A$ of $k$-algebras which makes $A$ an integral $k[X_1,\dots,X_m]$-algebra. Consequently, the transcendence degree over $k$ of the field of fractions of $A$ is equal to $m$, hence (c). Assertion (a) follows from (c) applied to the ring $A/\p$ and assertion (b) is an immediate consequence of (a).
\end{proof}
\begin{corollary}\label{algebra finite over field dim=n iff}
Let $n$ be a positive integer, then $\dim(k[X_1,\dots,X_n])=n$. Moreover, for a finitely generated $k$-algebra $A$ to be of dimension $n$, it is necessary and sufficient that there exists a injective $k$-homomorphism $\varphi:k[X_1,\dots,X_n]\to A$ such that $A$ is finite over $k[X_1,\dots,X_n]$.
\end{corollary}
\begin{proof}
This follows from \cref{algebra finite over field dimension prop}, \cref{algebra finite over field normalization}, and \cref{integral extension dimension of module ideal prop}(a).
\end{proof}
\begin{corollary}\label{algebra finite over field integral domain height formula}
Let $A$ be a finitely generated $k$-algebra that is an integral domain. Then for any prime ideal $\p$ of $A$, we have
\[\height(\p)=\dim(A_\p)=\dim(A)-\dim(A/\p)=\dim(A)-\tr.\deg_k(\kappa(\p)).\]
\end{corollary}
\begin{proof}
This is a concequence of \cref{algebra finite over field dimension prop} and \cref{ring of finite dimension chain same length prop}.
\end{proof}
\begin{corollary}\label{algebra finite over field localization non zerodivisor dimension}
Let $A$ be a finitely generated $k$-algebra and let $f$ be an element of $A$ which does not belong to any minimal prime ideal of $A$. Then $\dim(A)=\dim(A_f)$. 
\end{corollary}
\begin{proof}
The map $\p\mapsto\p A_f$ is a bijection between the set of minimal prime ideals of $A$ and that of $A_f$. Since $A/\p$ and $A_f/\p A_f$ have the same field of fraction, it suffices to apply \cref{algebra finite over field dimension prop}.
\end{proof}
\begin{corollary}\label{algebra finite over field prime ideal maximal char}
Let $A$ be a finitely generated $k$-algebra and $\p$ a prime ideal of $A$.
\begin{itemize}
\item[(a)] For $\p$ to be maximal, it is necessary and sufficient that $\kappa(\p)$ is a finite extension of $k$.
\item[(b)] Let $f\in A-\p$. Then the ideal $\p$ is a maximal ideal of $A$ if and only if $\p A_f$ is a maximal ideal of $A_f$.
\end{itemize}
\end{corollary}
\begin{proof}
If $\p$ is a maixmal ideal of $A$, then $A/\p$ is a field, whence a finitely generated extension of $k$ of transcendence degree $0$. It is therefore a finite extension of $k$ (\cref{algebra finite over field dimension prop}). Conversely, if the field $\kappa(\p)$ is a finite extension of $k$, we have $\dim(A/\p)=0$, so $\p$ is maximal. The assertion (b) now follows from (a), taking into account that $A/\p$ and $A_f/\p A_f$ have the same field of fractions.
\end{proof}
\begin{corollary}\label{algebra finite algebra over field local dimension formula}
Let $A$ be a finitely generated $k$-algebra, $\p$ a prime ideal of $A$ and $(\p_i)_{i\in I}$ the family of minimal prime ideals of $A$. Then we have
\[\dim_\p(A)=\sup_{i\in I_\p}\dim(A/\p_i)=\dim(A_\p)+\dim(A/\p)=\dim(A_\p)+\tr.\deg_k(\kappa(\p))\]
where $I_\p$ is the subset consists of $i\in I$ such that $\p_i\sub\p$.
\end{corollary}
\begin{proof}
We have $\dim_\p(A)=\sup_{i\in I_\p}\dim_{\p/\p_i}(A/\p_i)$ by \cref{dimension of ring prop}, since $A$ is Noetherian. Also, from \cref{algebra finite over field localization non zerodivisor dimension} we conclude that
\[\dim_{\p/\p_i}(A/\p_i)=\inf_{f\notin\p/\p_i}\dim((A/\p_i)_f)=\dim(A/\p_i)\]
so the first equality follows. Now since $A/\p_i$ is an integral domain, \cref{algebra finite over field integral domain height formula} shows $\dim(A/\p_i)=\dim((A/\p_i)_\p)+\dim(A/\p)$. The other equalities then follows from the observation that $\dim(A_\p)=\sup_{i\in I_\p}\dim((A/\p_i)_\p)$ and \cref{algebra finite over field dimension prop}.
\end{proof}
\begin{corollary}\label{algebra finite over field dim(A/p)=n iff}
Let $A$ be a finitely generated $k$-algebra, $n$ a positive integer. Then the following conditions are equivalent:
\begin{itemize}
\item[(\rmnum{1})] For every $\p\in\Ass(A)$, we have $\dim(A/\p)=n$.
\item[(\rmnum{2})] Every prime ideal associated with $A$ is minimal and all irreducible components of $\Spec(A)$ are of dimension $n$.
\item[(\rmnum{3})] There exists an injective homomorphism $\varphi:k[X_1,\dots,X_n]\to A$ such that $A$ is a finitely generated torsion-free $k[X_1,\dots,X_n]$-module.
\end{itemize}
\end{corollary}
\begin{proof}
The equivalence of (\rmnum{1}) and (\rmnum{2}) is immediate. Now suppose that (\rmnum{2}) holds. Then $\dim(A)=n$ and by \cref{algebra finite over field dim=n iff} there is an injective $k$-homomorphism
\[\varphi:k[X_1,\dots,X_n]\to A\]
such that $A$ is finite over $k[X_1,\dots,X_n]$. For any prime ideal $\p\in\Ass(A)$, the ring $A/\p$ is then integral over $k[X_1,\dots,X_n]$, and we have so $n=\dim(A/\p)=\dim(k[X_1,\dots,X_n]/\varphi^{-1}(\p))$ according to \cref{integral extension dimension of module ideal prop}(a), which proves $\varphi^{-1}(\p)=0$. By \cref{Noe ring divisor of zero union of associated prime}, this shows the image by the injective homomorphism $\varphi$ of a nonzero element of $k[X_1,\dots,X_n]$ is not a divisor of $0$ in $A$, whence (c).\par
Conversely, assume that (\rmnum{3}) holds. For each prime ideal $\p\in\Ass(A)$, the homomorphism $\varphi:k[X_1,\dots,X_n]\to A/\p$ induced by $\varphi$ is injective by \cref{Noe ring divisor of zero union of associated prime}, whence $\dim(A/\p)=n$ by \cref{algebra finite over field dim=n iff}.
\end{proof}
\begin{corollary}\label{algebra finite over field dim(A/p)=n local dimension equal}
Under the condition of \cref{algebra finite over field dim(A/p)=n iff}, $\dim_\p(A)=\dim(A)$ for every prime ideal $\p$ of $A$.
\end{corollary}
\begin{proof}
This follows from \cref{algebra finite algebra over field local dimension formula} and \cref{algebra finite over field dim(A/p)=n iff}.
\end{proof}
\begin{proposition}\label{algebra finite over field tensor with field dimension}
Let $A$ and $B$ be finitely generated $k$-algebras and $\rho:A\to B$ a $k$-algebra homomorphism. Suppose that $A$ is an integral domain and $B$ is a torsion-free $A$-module, and let $K$ be the field of fraction of $A$. Then we have
\[\dim(B)=\dim(A)+\dim(B\otimes_AK).\]
\end{proposition}
\begin{proof}
Suppose that $B$ is an integral domain. The algebra $B\otimes_AK$ is then a ring of fractions of $B$ defined by a multiplicative part not containing $0$. It therefore has field of fractions the field of fractions $L$ of $B$. According to \cref{algebra finite over field dimension prop}, we have
\[\dim(B)=\tr.\deg_k(L),\quad\dim(A)=\tr.\deg_k(K),\quad\dim(B\otimes_AK)=\tr.\deg_K(L).\]
By the transitivity of transcendental degree, we see the claim is true in this case.\par
Let's consider to the general case. Every minimal prime ideal $\mathfrak{P}$ of $B$ is formed by divisors of zero in $B$, so is lying over the ideal $0$ of $A$. By \cref{localization and ideals} it follows that the map $\mathfrak{P}\mapsto\mathfrak{P}(B\otimes_AK)$ is a bijection of the set of minimal prime ideals of $B$ on the set of minimal prime ideals of $B\otimes_AK$. The proposition follows therefore from the first part of the proof and from \cref{dimension of ring prop}.
\end{proof}
\begin{corollary}\label{algebra finite over field subalgebra dimension smaller}
Let $\rho:A\to B$ be an injective homomorphism of finitely generated $k$-algebras. Then $\dim(A)\leq\dim(B)$.
\end{corollary}
\begin{proof}
In fact, let $\p$ be a minimal prime of $A$ such that $\dim(A)=\dim(A/\p)$. Then there exists a prime ideal $\mathfrak{P}$ of $A$ lying over $\p$ by \cref{ring extension minimal prime is contracted of minimal}. According to \cref{algebra finite over field tensor with field dimension} applied to $A/\p$ and $B/\mathfrak{P}$, we have $\dim(A)=\dim(A/\p)\leq\dim(B/\mathfrak{P})\leq\dim(B)$, whence the corollary.
\end{proof}
\begin{lemma}\label{algebra finite over field tensor of torsion-free}
Let $A$ and $B$ be $k$-algebras that are integral domains, $M$ a torsion-free $A$-module, $N$ a torsion-free $B$-module. Then the ring $A\otimes_kB$ is an integral domain, and $M\otimes_kN$ is a torsion-free $A\otimes_kB$ module.
\end{lemma}
\begin{proof}
Let $K$ (resp. $L$) be the field of fraction of $A$ (resp. $B$). Then there exists a set $I$ (resp. $J$) such that $M$ (resp. $N$) is isomorphic to a submodule of $K^{\oplus I}$ (resp. $L^{\oplus J}$). The $(A\otimes_kB)$-module $M\otimes_kN$ is then isomorphic to a submodule of $K^{\oplus I}\otimes_kL^{\oplus J}$, which is isomorphic to $(K\otimes_kL)^{\oplus(I\times J)}$. As $K\otimes_kL$ is a ring of fractions of the ring $A\otimes_kB$, it is a torsion-free $A\otimes_kB$-module, hence the lemma.
\end{proof}
\begin{proposition}\label{algebra finite over field dimension of tensor ring}
Let $k'$ be an extension field of $k$, $A$ a finitely generated $k$-algebra and $B$ a finitely generated $k'$-algebra.
\begin{itemize}
\item[(a)] The $k'$-algebra $A\otimes_kB$ is finitely generated and we have
\[\dim(A\otimes_kB)=\dim(A)+\dim(B).\] 
\item[(b)] Let $\r$ be a prime ideal of $A\otimes_kB$. If $\p$ (resp. $\q$) is the contraction of $\r$ to $A$ (resp. $B$), then
\[\dim_\r(A\otimes_kB)=\dim_\p(A)+\dim_\q(B).\] 
\end{itemize}
\end{proposition}
\begin{proof}
Put $n=\dim(A)$ and $m=\dim(B)$. There exists according to \cref{algebra finite over field dim=n iff} injective homomorphisms of algebras $\varphi:k[X_1,\dots,X_n]\to A$ and $\psi:k'[Y_1,\dots,Y_m]\to B$ making respectively of $A$ and $B$ finite algebras over $k[X_1,\dots,X_n]$ and $k'[Y_1,\dots,Y_m]$. The homomorphism $\varphi\otimes\psi$ is then injective and makes $A\otimes_kB$ a finite algebra over the $k'$-algebra $k[X_1,\dots,X_n]\otimes_kk'[Y_1,\dots,Y_m]$, which identifies with $k'[X_1,\dots,X_n,Y_1,\dots,Y_m]$ (to see that $\varphi\otimes\psi$ is injective, we may consider the following commutative diagram
\[\begin{tikzcd}
k[X]\otimes k'[Y]\ar[d,swap,"\varphi\otimes 1"]\ar[r,"1\otimes\psi"]&k[X]\otimes B\ar[d,"\varphi\otimes 1"]\\
A\otimes k'[Y]\ar[r,"1\otimes\psi"]&A\otimes B
\end{tikzcd}\]
where $\varphi\otimes\psi$ is the composition homomorphism, and note that all modules are flat since we are tensoring over $k$). We then have $\dim(A\otimes_kB)=n+m$ by \cref{algebra finite over field dim=n iff}, which proves (a).\par
Note that when $A$ and $B$ are integral domains, the $k'[X_1,\dots,X_n,Y_1,\dots,Y_m]$-module $A\otimes_kB$ is torsion free by \cref{algebra finite over field tensor of torsion-free} and therfore we have
\[\dim_\r(A\otimes_kB)=\dim(A\otimes_kB)=\dim(A)+\dim(B)\]
for any prime ideal $\r$ of $A\otimes_kB$ by \cref{algebra finite over field dim(A/p)=n local dimension equal}.\par
Now let us show (b). Let $\r_0$ be a minimal prime ideal of $A\otimes_kB$ contained in $\r$, and let $\p_0$ (resp. $\q_0$) be the contraction of $\r_0$ to $A$ (resp. $B$). The ring $(A\otimes_kB)/\r_0$ is isomorphic to a quotient of the ring $(A/\p_0)\otimes_k(B/\q_0)$. Thus we have, by (a),
\[\dim((A\otimes_kB)/\r_0)\leq\dim((A/\p_0)\otimes_k(B/\q_0))=\dim(A/\p_0)+\dim(B/\q_0).\]
According to \cref{algebra finite algebra over field local dimension formula}, this shows
\[\dim_\r(A\otimes_kB)\leq\dim_{\p}(A)+\dim_{\q}(B).\]
Conversely, let $\p_0$ (resp. $\q_0$) be a minimal prime ideal of $A$ (resp. $B$) contained in $\p$ (resp. $\q$). By the observation above, we have
\[\dim(A/\p_0)+\dim(B/\q_0)=\dim_{\widebar{\r}}((A/\p_0)\otimes_k(B/\q_0))\leq\dim_\r(A\otimes_kB)\]
where $\widebar{\r}$ is the image of $\r$ by the canonical surjection $A\otimes_kB\to(A/\p_0)\otimes_k(B/\q_0)$. Applying \cref{algebra finite algebra over field local dimension formula}, we deduce the inequality
\[\dim_\p(A)+\dim_\q(B)\leq\dim_\r(A\otimes_kB)\]
whence (b) follows.
\end{proof}
\begin{corollary}\label{algebra finite over field extension field prop}
Let $A$ be a finitely generated $k$-algebra, $k'$ an extension field of $k$, and $A'$ the $k'$-algebra $A\otimes_kk'$.
\begin{itemize}
\item[(a)] We have $\dim(A')=\dim(A)$.
\item[(b)] Let $\p'$ be a prime ideal of $A$ and $\p$ its contraction to $A$. Then $\dim_{\p'}(A')=\dim_\p(A)$.
\item[(c)] Let $\p'$ be a minimal prime ideal of $A$ and $\p$ be the contraction of $\p'$. Then $\p$ is minimal and $\dim(A'/\p')=\dim(A/\p)$.
\item[(d)] If $k'$ is a purely inseparable extension of $k$, then the canonical map $^{a}\!\rho:\Spec(A')\to\Spec(A)$ is a homeomorphism. 
\end{itemize}
\end{corollary}
\begin{proof}
Assertions (a) and (b) follow from \cref{algebra finite over field dimension of tensor ring} by setting $B=k'$. For (c), note that the homomorphism $\nu:k\to k'$ satisfies condition (PM), so the ideal $\p'$ is minimal by \cref{ring homomorphism condition PM prop}(a) and (c). By corollary~\ref{algebra finite algebra over field local dimension formula}, we then have
\[\dim(A'/\p')=\dim_{\p'}(A')=\dim_{\p}(A)=\dim(A/\p)\]
which completes the proof of (c).\par
Finally, assume that $k'/k$ is purely inseparable and let $\rho:A\to A'$ be the canonical homomorphism. Let $\p\in\Spec(A)$, recall that the fiber of $\p$ under $^{a}\!\rho$ is homeomorphic to $\Spec(\kappa(\p)\otimes_kk')$. Now the set $\n$ of nilpotent elements of $\kappa(\p)\otimes_kk'$ is a prime ideal (A, \Rmnum{5}, p.134, corollarie) and the quotient ring $(\kappa(\p)\otimes_kk')/\n$ is a field by (A, \Rmnum{5}, p.16, cor.1 et p.10, prop.1). Therefore $(^{a}\!\rho)^{-1}(\p)$ is reduced to a singleton. It follows that the map $^{a}\!\rho$ is a bijection of $\Spec(A')$ on $\Spec(A)$. Since we have observed that $\rho$ has the going up property (\cref{ring homomorphism condition PM prop}), it follows from \cref{Spec of ring map going up and going down} that $^{a}\!\rho$ is a closed map, whence a homeomorphism.
\end{proof}
\begin{theorem}[\textbf{Grothendieck's Generic Freeness Lemma}]\label{Noe domain generic freeness lemma}
Let $A$ be a Noetherian domain, $B$ be an $A$-algebra of finite type, and $M$ be a finitely generated $B$-module. Then there exists a nonzero element $f\in A$ such that $M_f$ is a free $A_f$-module.
\end{theorem}
\begin{proof}
Let $K$ be the fraction field of $A$; then $B\otimes_AK$ is a finite type algebra over $K$ and $M\otimes_AK$ is a finitely generated $(B\otimes_AK)$-module. We preceed by induction on the dimension $d=\dim(M\otimes_AK)$. If $d=-\infty$, we have $M\otimes_AK=0$, which means $M$ is annihilated by an nonzero element of $f\in A$, so $M_f=0$ and the theorem is trivially satisfied.\par
Now assume that $d\geq 0$. By \cref{associated prime of Noe chain of submodule}, there exists a filtration $(M_i)_{0\leq i\leq n}$ of the $B$-module $M$ such that $N_i=M_i/M_{i+1}\cong B/\mathfrak{P}_i$, where $\mathfrak{P}_i$ is a prime ideal of $B$. If the theorem is proved for each $N_i$, then there exists for each $i$ an element $f_i\in A$ such that $(N_i)_{f_i}$ is free over $A_{f_i}$. If we set $f=f_1\cdots f_{n-1}$, then $(N_i)_f$ is a free $A_f$-module for $0\leq i\leq n-1$, and since $(N_i)_f=(M_i)_f/(M_{i+1})_f$, the $A_f$-module $M_f$ then possesses a filtration by free modules, hence is free. By replacing $B$ with $B/\mathfrak{P}$ (where $\mathfrak{P}$ is a prime ideal of $B$), which is of finite type over $A$, we see that we are reduced to the case where $M=B$ and $B$ is an integral domain. By \cref{integral domain finite algebra normalization}, there then exists a nonzero element $g\in A$ and algebraically indepedent elements $x_1,\dots,x_m$ of $B$ such that $B_g$ is integral over $A_g[x_1,\dots,x_m]$. By replacing $A$ by $A_g$ and $B$ by $B_g$, we may assume that $B$ is integral over $C=A[x_1,\dots,x_m]$, hence a finitely generated torsion-free $C$-module. By \cref{algebra finite over field dim=n iff}, we also note that the dimension of $B\otimes_AK$ is equal to $m$, so we have $m=d$.\par
Now if $r$ is the rank of the torsion-free $C$-module $B$, there exists an exact sequence of $C$-modules
\[\begin{tikzcd}
0\ar[r]&C^r\ar[r]&B\ar[r]&M'\ar[r]&0
\end{tikzcd}\]
where $M'$ is a finitely generated torsion $C$-module. The support of $M'$ then does not contain the generic point of $\Spec(C)$, and therefore the support of $M'\otimes_AK$ does not contain the generic point of $\Spec(C\otimes_AK)$ (\cref{supp of module extension ring}). We then conclude that $\dim_C(M')<d$, so by the induction hypotheses, there exists a nonzero element $f\in A$ such that $M'_f$ is free over $A_f$. Since $C_f$ is also free over $A_f$ (a basis of $C_f$ is the image of the monomials in the $x_i$), we then conclude that $B_f$ is a free $A_f$-module.
\end{proof}
\subsection{Exercise}
\begin{exercise}
In this exercise, we develop a way to measure the complexity of a partially ordered set.
\begin{itemize}
\item[(a)] Prove that, for any partially ordered set $E$ we can associate an element of $\N\cup\{\pm\infty\}$, denoted by $\dev(E)$, and called the \textbf{deviation} of $E$, which satisfies the following conditions:
\begin{itemize}
\item[($\alpha$)] A trivial poset (one in which no two elements are comparable) has deviation $-\infty$.
\item[($\beta$)] A poset $E$ is said to have deviation at most $n$ (for a positive integer $n$) if for every descending chain of elements $(a_k)_{k\in\N}$, all but a finite number of the posets $(a_k,a_{k+1})$ have deviation less than $n$. The deviation of $E$ (if it exists) is the minimum value of $\alpha$ for which this is true.
\end{itemize} 
\item[(b)] Show that for $\dev(E)\leq 0$, it is necessary and sufficient that any decreasing sequence of $E$ is stationary. We have $\dev(\N)=0$, $\dev(\Z)=1$, $\dev(\Q)=+\infty$.
\item[(c)] Let $E$ and $F$ be partially ordered sets. Show that if there exists an increasing map from $E$ to $F$, then $\dev(E)\leq\dev(F)$. Show that $\dev(E\times F)=\sup\{\dev(E),\dev(F)\}$.
\end{itemize}
\end{exercise}
\begin{exercise}\label{Noe ring dev and dim equal}
Let $A$ be a ring (not necessarily commutative), $M$ a left $A$-module, we denote by $\dev(M)$ the deviation of the set of submodules of $M$, orederd by inclusion. We put $\dev(A)$ to be the deviation of the left $A$-module $A$.
\begin{itemize}
\item[(a)] If $N$ is a sub-$A$-module of $M$, then
\[\dev(M)=\sup\{\dev(N),\dev(M/N)\}.\] 
\item[(b)] Suppose that $A$ is commutative. If $\p$ and $\q$ are distinct prime ideals of $A$ and $\p\sub\q$, then $\dev(A/\p)>\dev(A/\q)$. In particular, $\dim(A)\leq\dev(A)$.
\item[(c)] Suppose that $A$ is commutative and Noetherian. Let $\mathscr{P}_A$ be the set of prime ideals of $A$ such that $\dim(A/\p)=\dim(A)$ and $S$ be complement of the union of prime ideals in $\mathscr{P}_A$. Let $M$ be a finitely generated $A$-module. If $S^{-1}M=0$ then
\[\dev(M)\leq\sup_{\p\notin\mathscr{P}_A}\dev(A/\p).\]
\item[(d)] Suppose that $A$ is commutative and Noetherian. Prove that in this case we have $\dim(A)=\dev(A)$ and $\dim(M)=\dev(M)$ for any finitely generated $A$-module $M$.
\end{itemize}
\end{exercise}
\begin{proof}
For a module $E$, we let $\mathcal{L}(E)$ denote the lattice of submodules of $E$. Then if $N$ is a submodule of $M$, there are ordered homomorphisms $\mathcal{L}(N)\to\mathcal{L}(M)$ and $\mathcal{L}(M/N)\to\mathcal{L}(M)$, whence
\[\dev(M)\geq\sup\{\dev(N),\dev(M/N)\}.\]
On the other hand, we have an ordered homomorphism
\[\mathcal{L}(M)\to\mathcal{L}(N)\times\mathcal{L}(M/N),\quad M'\mapsto(M'\cap N,M'+N/N).\]
This proves the equality in (a).\par
Suppose that $A$ is commutative and let $\p\sub\q$ be distinct prime ideals. By replacing $A$ with $A/\p$, we may assume that $\p=0$. For any nonzero element $x\in\q$, consider the sequence $(x^nA)$ in $A$. We have
\[\dev(A)>\dev(x^nA/x^{n+1}A)=\dev(A/xA)\geq\dev(A/\q)\]
so the claim in (b) follows. Now let $\p_0\subset\cdots\subset\p_n$ be a saturated chain in $A$. Then $\dev(A/\p_0)>\cdots>\dev(A/\p_n)$, which shows $\dev(A)\geq n=\dim(A)$.\par
Suppose that $A$ is Noetherian and let $M$ be a finitely generated $A$-module. Then there is a composition series $(M_i)_{0\leq i\leq n}$ of $M$ such that $M/M_{i+1}$ is isomomorphic to $A/\p_i$. If $S^{-1}M=0$, we have $\p_i\notin\mathscr{P}_A$ for all $i$. Since $\dev(M)=\sup_i\dev(M_i/M_{i+1})$, we conclude that
\[\dev(M)\leq\sup_{\p\notin\mathscr{P}_A}\dev(A/\p).\]
Finally, we show that $\dev(A)=\dim(A)$ by induction on $\dim(A)$. So assume that $\dev(B)=\dim(B)$ for every Noetherian commutative $B$ such that $\dim(B)<\dim(A)$. Then in the notation of (c), since $\dim(A/\p)<\dim(A)$, we have $\dev(A/\p)=\dim(A/\p)$ for $\p\notin\mathscr{P}_A$, whence
\begin{align}
\dev(M)\leq\sup_{\p\in\mathscr{P}_A}\dim(A/\p)<\dim(A)
\end{align}
if $M$ is finitely generated and $S^{-1}M=0$. Now let $(\a_k)$ be a sequence of decreasing ideals in $A$. Since $S^{-1}A$ is Artinian (it has dimension zero), there exists an integer $N$ such that
\[S^{-1}\a_n=S^{-1}\a_{n+1}=\cdots\]
for $n\geq N$. In other words, $S^{-1}(\a_n/\a_{n+1})=0$ for $n\geq N$, so by (\ref{Noe ring dev and dim equal}) we have $\dev(\a_n/\a_{n+1})<\dim(A)$ for $n\geq N$. From the definition of $\dev(A)$, this shows $\dev(A)\leq\dim(A)$, which gives $\dev(A)=\dim(A)$.\par
Now let $M$ be a finitely generated $A$-module and let $(M_i)_{0\leq i\leq n}$ be the composition series of $M$ such that $M_i/M_{i+1}\cong A/\p_i$, where $\p_i$ is a prime ideal of $M$. Then
\[\dev(M)=\sup_i\{\dev(M_i/M_{i+1})\}=\sup_i\{\dev(A/\p_i)\}=\sup_i\{\dim(A/\p_i)\}.\]
We also note that, since the minimal elements in $\{\p_0,\dots,\p_{n-1}\}$ are the embedded primes for $M$, which correspond to irreducible components of $\a=\Ann(M)$. Since $\dim(M)=\dim(V(\a))$ is the supremum of the dimension of these components, we conclude that $\dev(M)=\dim(M)$.
\end{proof}
\begin{exercise}
Let $A$ be a Noetherian commutative ring, $\a$ an ideal of $A$ contained in the Jacobson radical of $A$, and $\gr(A)$ the graded ring associated with the $\a$-adic filtration. Associate to each ideal of $A$ an ideal of $\gr(A)$, and deduce that $\dev(A)\leq\dev(\gr(A))$. Concequently, deduce that $\dim(A)\leq\dim(\gr(A))$.
\end{exercise}
\begin{exercise}\label{Noe ring if maximal localization Noe}
Let $A$ be a ring such that
\begin{itemize}
\item[(a)] For every maximal ideal $\m$ of $A$, $A_\m$ is Noetherian.
\item[(b)] Every nonzero element $x\in A$ is contained in a finite number of maximal ideals of $A$.
\end{itemize}
Show that $A$ is Noetherian.
\end{exercise}
\begin{proof}
Let $(\a_n)$ be an increasing sequence of ideals in $A$, and assume that $\a_1\neq 0$. Then by condition (b), the ideal $\a_1$ is contained in finitely many maximal ideals, say $\m_1,\dots,\m_r$. Since each ring $A_{\m_i}$ is Noetherian, there exists an integer $N$ such that
\[\a_nA_{\m_i}=\a_{n+1}A_{\m_i}=\cdots\]
for $n\geq N$ and $i=1,\dots,r$. But if $\m$ is a maximal ideal other than $\m_i$, then $\a_n\nsubseteq\m$ for all $n$, so 
\[\a_nA_{\m}=\a_{n+1}A_{\m}=\cdots=A_{\m}\]
trivially holds. By \cref{localization submodule iff submodule of localization}, we conclude that $\a_n=\a_{n+1}=\cdots$ for $n\geq N$, so $A$ is Noetherian.
\end{proof}
\begin{exercise}\label{Noe ring infinite dimension example}
Let $K$ be a field, $(n_i)_{i\geq 1}$ a strict increasing sequence of positive integers; for each $i\geq 0$, put $\p_i$ to be the ideal of the ring $R=K[(X_j)_{j\in\N}]$ generated by the elements $X_j$ where $n_1+\cdots+n_i\leq j<n_1+\cdots+n_{i+1}$. Let $S$ be the complement of the union of the $\p_i$'s, and consider the ring $A=S^{-1}R$. Prove that $S^{-1}R$ is Noetherian but has infinite dimension.
\end{exercise}
\begin{proof}
We first note that an nonzero element in $S^{-1}R$ can only be contained in finitely many $S^{-1}\p_i$ (since this is true in $R$ for $\p_i$). Since the ideals of $S^{-1}A$ correspond to ideals of $R$ contained in $\bigcup_i\p_i$, and any such ideal can only be contained in finitely many $\p_i$, by prime avoidence we conclude that the $S^{-1}\p_i$ are the only maximal ideals in $S^{-1}R$. Moreover, for each $S^{-1}\p_i$, $(S^{-1}R)_{S^{-1}\p_i}$ is isomomorphic to $R_{\p_i}$, which is Noetherian since it is just $K(x_J)[x_I]_{(x_I)}$ where $I$ is indices appears in the generators of $\p_i$ and $J$ its complement. By \cref{Noe ring if maximal localization Noe}, we conclude that $S^{-1}R$ is Noetherian. Now it is easy to see each maximal ideal $S^{-1}\p_i$ has height $\geq n_i$ in $S^{-1}R$, so $S^{-1}R$ has infinite dimension.
\end{proof}
\begin{exercise}
Let $A$ be the ring of germs at $0$ of smooth functions on $\R$. We propose to show that the ring $A$ is of infinite dimension. We denote by $C$ the algebra of smooth functions on $\R$, $F$ the ideal of functions zero on a neighborhood of $0$, so that $A=C/F$.\par
\begin{itemize}
\item[(a)] Let $(x_n)_{n\in\N}$ be a strict decreasing sequence of positive numbers tending to $0$, and let $\mathcal{U}$ be an ultrafilter on $\N$ finer than the filter of complements of finite subsets. Show that the set $J_1$ of $f\in C$ such that the set of $n\in\N$ such that $f(x_n)=0$ belongs to $\mathcal{U}$, is a prime ideal of $C$.
\item[(b)] We denote by $\tau_n(f)$ the function $x\mapsto f(x+x_n)$. Show that if $J$ is a prime ideal of $C$, the set $\varphi(J)$ of $f\in C$ such that the set of $n\in\N$ where $\tau_n(f)\in J$ belongs to $\mathcal{U}$ is a prime ideal of $C$.
\item[(c)] We define a sequence $I_n$ of ideals of $C$ by $I_0=J_1$ and $I_{n+1}=\varphi(I_n)$. Show that $(I_n)_{n\in\N}$ is a strict decreasing sequence of prime ideals of $C$ containing $F$.
\end{itemize}
\end{exercise}
\begin{proof}
Let $(x_n)$ and $\mathcal{U}$ be as in (a), and define $J_1$ by
\[J_1=\{f\in C:\{n\in\N:f(x_n)=0\}\in\mathcal{U}\}.\]
Now let $f,g\in C$ and assume that $fg\in J_1$; put
\[E_{f}=\{n\in\N:f(x_n)=0\},\quad E_{g}=\{n\in\N:g(x_n)=0\}.\]
Then by definition, the set $E_{fg}=\{n\in\N:f(x_n)g(x_n)=0\}$ belongs to $\mathcal{U}$. Assume that $E_f,E_g\notin\mathcal{U}$, then since $\mathcal{U}$ is a ultrafilter, we have $E_f^c,E_g^c\in\mathcal{U}$. But then
\[E_f^c\cap E_g^c=\{n\in\N:f(x_n)\neq 0,g(x_n)\neq 0\}=\{n\in\N:f(x_n)g(x_n)\neq 0\}=E_{fg}^c\in\mathcal{U}\]
which is a contradiction since $E_{fg}\in\mathcal{U}$. This proves that either of $E_f$ and $E_g$ is in $\mathcal{U}$, so $J_1$ is a prime ideal.\par
Now for a prime ideal $J$ of $C$, define $\varphi(J)$ by
\[\varphi(J)=\{f\in C:\{n\in\N:\tau_n(f)\in J\}\in\mathcal{U}\}.\]
Again, for an element $f\in C$ we put $E_f(J)=\{n\in\N:\tau_n(f)\in J\}$. By noting that, since $J$ is a prime ideal,
\[E_f(J)^c\cap E_g(J)^c=\{n\in\N:\tau_n(f)\notin J,\tau_n(g)\notin J\}=\{n\in\N:\tau_n(f)\tau_n(g)\notin J\}=E_{fg}(J)^c,\]
it is easy to show that $\varphi(J)$ is a prime ideal, just as in (a).\par
Define the sequence $(I_n)$ by $I_0=J_1$ and $I_{n+1}=\varphi(I_n)$. To prove the first assertion in (c), we first note that, since $\mathcal{U}$ is finer than the finite complement filter on $\N$, every element in $\mathcal{U}$ is infinite (otherwise its complement belongs to $\mathcal{U}$). Let $f\in F$, then there exists a neighborhood $U$ of $0$ such that $f|_U=0$. Then since $x_n\to 0$, there exists an integer $N$ such that $x_n\in U$ when $n\geq N$, whence $f(x_n)=0$ for $n\geq N$. Since $\mathcal{U}$ is finer than the finite complement filter on $\N$, this means $E_f=\{n\in\N:f(x_n)=0\}\in\mathcal{U}$, so $f\in J_1$. This shows $F\sub J_1=I_0$. Now let $J$ be a prime ideal of $C$ containing $F$, and consider $\varphi(J)$. Let $f\in F$, and let $U$ be a neighborhood of $0$ such that $f|_{U}\equiv 0$. Then since $x_n$ is strict decreasing and tend to $0$, there exists $N>0$ such that $(x_n+U)\cap U$ contains a neighborhood of $0$ for $n\geq N$, whence $\tau_n(f)\in F\sub J$ for $n\geq N$. This shows $f\in\varphi(J)$, so $F\sub\varphi(J)$. Thus we have shown that $I_n$ contains $F$ for each $n$.\par
It remains to prove that $(I_n)$ is strict decreasing. We prove a more general fact: let $J$ be a prime ideal of $C$ such that, if $f_n,f$ are elements in $C$ and $f_n\to f$ pointwisely and $f_n\in J$ for all $n$, then $f\in J$ (we say $J$ is \textit{pointwise closed} if this is true). Then $\varphi(J)\sub J$ and $\varphi(J)$ is pointwise closed. For this, let $f\in\varphi(J)$, then $E_f(J)$ is infinite, hence contains a sequence $(y_n)$ tends to $0$; we have $\tau_{y_n}(f)\in J$ for each $n$, and letting $n$ tends to infinite shows that $f\in J$, since $y_n\to 0$ by hypothesis and $J$ is pointwise closed; this shows $\varphi(J)\sub J$.
\end{proof}
\begin{exercise}\label{DVR polynomial ring not catenary and gr dim}
Let $R$ be a DVR and $\pi$ a uniformizer for $R$. In the ring $R[T]$, the ideal $\m_1=(\pi T-1)$ is maximal with height $1$, the ideal $\m_2=(\pi)+(T)$ is maximal with height $2$. The field $R[T]/\m_1$ and $R[T]/\m_2$ are isomomorphic to the field of fractions of $R$ and to the residual field of $R$ respectively. We have
\begin{align}\label{DVR polynomial ring not catenary and gr dim-1}
\dim(R[T])=2,\quad \dim(R[T]/\m_1)+\dim(R[T]_{\m_1})=1,\quad \dim(\gr_{\m_1}(R[T]))=1.
\end{align}
\end{exercise}
\begin{proof}
Let $K$ be the fraction field of $R$. Then it is easy to see $R[T]/\m_1\cong R[\pi^{-1}]=K$, so $\m_1$ is maximal. Moreover, since $\m_1$ is principal, it has height $1$ by \cref{Krull principal ideal theorem}. Also, the quotient $R[T]/\m_2$ is clearly isomomorphic to $\kappa_R$, so $\m_2$ is maximal, and has height $2$ by \cref{Krull principal ideal theorem}.\par
Clearly $\dim(R[T])=\dim(R)+1=2$ since $R$ is Noetherian. The ring $R[T]_{\m_1}$ has dimension equal to $\height(\m_1)$, which is $1$. These prove the first two equalities in (\ref{DVR polynomial ring not catenary and gr dim-1}). To see $\dim(\gr_{\m_1}(R[T]))=1$, note that $\gr_{\m_1}(R[T])$ is isomomorphic to a polynomial ring over $K$.
\end{proof}
\begin{exercise}
With the notations of the previous exercise, suppose that the residue field of $R$ and fraction field of $R$ are isomomorphic (this is the case for example when $R=k\llbracket X\rrbracket$, the field $k$ being the field of fractions of a ring of formal series with an infinite number of indeterminates with coefficients in a field). Let $\sigma$ be an isomorphism of $R[T]/\m_1$ to $R[T]/\m_2$. Let $C$ be the subring of $R[T]=E$ formed of the elements of $E$ whose classes modulo $\m_1$ and $\m_2$ are associated by $\sigma$: in other words, if $\pi_i:R[T]\to R[T]/\m_i$ is the natural map, then
\[C=\{x\in R[T]:\sigma(\pi_1(x))=\pi_2(x)\}.\]
Show that $C$ is noetherian and that $\m_1\m_2=\m_1\cap\m_2$ is a maximal ideal of $C$. Show that $E$ is the integral closure of $C$ (note that we have $E=C+(\pi T-1)C$ and that $(\pi T-1)^2+(\pi T-1)$ belongs to $C$).\par
Let $A$ be the local ring $C_{\m_1\m_2}$. Then $A$ is integral with dimension $2$, hence catenary; the integral closure $B=E\otimes_CA$ of $A$ is a semi-local Noetherian ring with exactly two maximal ideals $\n_1$ and $\n_2$, and we have $\dim(B_{\n_1})=1$, $\dim(B_{\n_2})=2$.
\end{exercise}
\begin{proof}

\end{proof}
\begin{exercise}
Retain the notations of the previous exercise, and identify $B$ with a quotient ring of the polynomial ring $A[U]$ by the prime ideal generated by $U^2+U-\pi T(\pi T-1)$. Let $\q_i$ be the ideal of $A[U]$ such that $\n_i=\q_i/\p$. Then $\height(\q_1)=3$, $\height(\p)=1$ and the chain $\p\sub\q_1$ is saturated. In particular, $A[U]$ and $A[U]_{\q_1}$ are not catenary.
\end{exercise}
\begin{exercise}
Let $A$ be a ring.
\begin{itemize}
\item[(a)] Assume that $A$ is an integral domain and let $f\in A[T]$, $f\neq 0$. Prove that there exists a maximal ideal $\m$ of $A[T]$ such that $f\notin\m$.
\item[(b)] Prove that $\dim(\Max(A[T]))\geq\dim(A)+1$, so $\dim(\Max(A[T]))=\dim(A[T])=\dim(A)+1$ when $A$ is Noetherian.
\item[(c)] Let $k$ be a field, $S$ the subset $1+(X)$ of $k[X,Y]$, and $A=S^{-1}k[X,Y]$. Show that we have $\dim(\Max(A))=1$ and $\dim(\Spec(A))=2$. Generalize this to several indeterminates.
\item[(d)] Show that for any couple of integers $0\leq n<m$, there exists a Noetherian ring $A$ such that $\dim(\Max(A))=n$ and $\dim(\Max(A[T]))=m$.
\item[(e)] Let $A$ be the Noetherian ring defined in \cref{Noe ring infinite dimension example}. Show that $\dim(\Max(A))=1$ and $\dim(\Max(A[T]))=\infty$.
\end{itemize}
\end{exercise}
\begin{proof}
Since the Jacobson radical of $A[T]$ is zero if $A$ is integral, we have (a). Now let $\p_0\subset\cdots\subset\p_n$ be a chain of prime ideals in $A$. Consider the chain
\[\p_0[T]\subset\cdots\subset\p_1[T]\subset\cdots\subset\p_n[T]\subset(\p_n,T)\]
of prime ideals of $A[T]$ and put
\[\begin{cases}
U_i=V(\p_i[T])\cap\Max(A[T]),&0\leq i\leq n\\
U_{n+1}=V(\p_n,T)\cap\Max(A[T]).
\end{cases}\]
It is easy to see the chain $(U_i)_{0\leq i\leq n+1}$ is decreasing and each $U_i$ is closed; the fact it is strict follows immediately from (a). It remains to show that each $U_i$ is irreducible. This shows the equality $\dim(\Max(A[T]))\geq\dim(A)+1$. Recall that $\dim(\Max(A))\leq\dim(A)$ for any ring and $\dim(A[T])=\dim(A)+1$ if $A$ is Noetherian. So if $A$ is Noetherian, we have
\[\dim(A)+1\leq\dim(\Max(A[T]))\leq\dim(A[T])=\dim(A)+1.\]

Let $A$ be the ring in (c). Then the ring $A=S^{-1}k[X,Y]$ is just the ring $S^{-1}k[Y][X]$, and it is easy to see the chain $0\sub(X)\sub(X,Y)$ corresponds to a chain of length $2$ in $A$, whence $\dim(A)=2$. For the space $\Max(A)$, note that a maximal ideal $(X-a,Y-b)$ is disjoint from $S$ if and only if $a=0$, which indicates $\dim(\Max(A))=1$ at least when $k$ is algebraically closed. In general, let $0\leq n\leq m$ and consider the ring $k[X_1,\dots,X_m]$. Let $S$ be the subset $1+(X_1,\dots,X_{m-n})$, and $A=S^{-1}k[X_1,\dots,X_m]$. Then $\dim(A)=m$ and $\dim(\Max(A))=n$. Note that in this case, since $A$ is Noetherian, we have $\dim(\Max(A[T]))=\dim(A)+1=m+1$, so (d) is proved.\par
Finally, consider the example of \cref{Noe ring infinite dimension example}. Since $A$ is Noetherian, we have
\[\dim(\Max(A[T]))=\dim(A[T])=\infty.\]
But the maximal ideals of $A$ are the $S^{-1}\p_i$'s, and if $S^{-1}\p$  is a prime ideal of $A$, it is contained in finitely many $S^{-1}\p_i$'s. However, note that each singleton in $\Max(A)$ is closed, so the set $V(S^{-1}\p)\cap\Max(A)$ is not irreducible unless it is a singleton or the whole space $\Max(A)$. This shows $\dim(\Max(A))=1$ (in fact, the topology on $\Max(A)$ is the cofinite topology, hence has Krull dimension $1$).
\end{proof}
\begin{exercise}\label{ring homomorphism going down prop strict inequality}
Let $R$ be a DVR with fraction field $K$ and $n$ an integer. Let $A$ be the subring of the polynomial ring $K[X_1,\dots,X_n]$ consists of polynomials whose constant terms are in $R$. Then the canonical homomorphism $\rho:R\to A$ satisfies the going down property and the canonical homomorphism $R/\m_R\to A/\m_RA$ is an isomorphism. On the other hand, we have $\dim(A)\geq n+1$ and $\dim(R)=1$. As long as $n\geq 1$, we have $\dim(A)>\dim(R)+\dim(A/\m_RA)$. Show that $A$ is not Noetherian. By using formal series, construct an similar example where $A$ is local and not Noetherian. 
\end{exercise}
\begin{proof}
It is clear that $A$ is a subring of $K[X_1,\dots,X_n]$ and $A/\m_RA$ is isomomorphic to $R/\m_R$; $A$ is flat over $R$ since it is torsion-free (\cref{valuation ring finitely generated module is free}). Since $\a^{ec}=\a$ for every ideal of $R$, by \cref{ring faithfully flat iff} we conclude that $A$ is faithfully flat over $R$, whence the going down property is satisfied (\cref{ring homomorphism condition PM prop}). Let $\m_R[X_1,\dots,X_n]$ be the ideal consists of polynomials with coefficients in $\m_R$. Then it is clear that $\m_R[X_1,\dots,X_n]$ is prime in $A$ and we have a chain
\[0\subset\m_R[X_1,\dots,X_n]\subset\m_R[X_1,\dots,X_n]+(X_1)\subset\cdots\subset\m_R[X_1,\dots,X_n]+(X_1,\dots,X_n)\]
in the ring $K[X_1,\dots,X_n]$. Contracting this to $A$, we get a chain of length $n+1$ in $A$, so $\dim(A)\geq n+1$. Since $A/\m_RA\cong R/\m_R$ is a field, this shows
\[\dim(A)\geq n+1>\dim(R)+\dim(A/\m_RA)\]
as long as $n\geq 1$.\par
To see $A$ is not Noetherian, we claim that the ideal $\p=(X_1,\dots,X_n)\cap A$ in $A$ is not finitely generated. In fact, if $f_1,\dots,f_n\in A$ generate $\p$, then $\sum_if_i(0)g_i(0)$ can take any element in $K$ as $g_i$ varies in $A$. But this is not possible since $f_i(0),g_i(0)\in R$.\par
Finally, replacing $K[X_1,\dots,X_n]$ with $K\llbracket X_1,\dots,X_n\rrbracket$, we see the resulting ring $A$ is then local with maximal ideal $\m_R\llbracket X_1,\dots,X_n\rrbracket+(X_1,\dots,X_n)$.
\end{proof}
\begin{exercise}
Let $R$ be a DVR with fraction field $K$ and residue field $k$. Put $A=K\times k$, and let $\rho:R\to A$ be the homomorphism deduced from the canonical homomorphisms $R\to K$ and $R\to k$. Then the homomorphism $\rho$ is injective, the induced map $\rho^*:\Spec(A)\to\Spec(R)$ is surjective, $A$ is a finitely generated $R$-algebra, and we have $\dim(A)<\dim(R)$.
\end{exercise}
\begin{proof}
Since $R\to K$ is injective, it is clear that $\rho$ is injective. Also, we have $\Spec(A)=\Spec(K)\amalg\Spec(k)$, which consists of two elements $(0)_K$ and $(0)_k$. The image of $(0)_K$ under $\rho$ is $(0)_R$, and that of $(0)_k$ is $\m_R$. Therefore $\rho^*$ is surjective. Since $K$ and $k$ are finitely generated $R$-algebra, so is $A$, and we have $\dim(A)=0<\dim(R)=1$.
\end{proof}
\begin{exercise}\label{polynomial ring chain of prime contraction prop}
Let $A$ be an integral domain and $\mathfrak{P}_0\subset\mathfrak{P}_1\subset\mathfrak{P}_2$ be a chain of prime ideals in $A[X]$. Show that $\mathfrak{P}_0,\mathfrak{P}_1,\mathfrak{P}_2$ can not have the same contraction in $A$. In particular, if $A$ is one-dimensional and $\mathfrak{P}_i$ are nonzero, then $\mathfrak{P}_0\cap A=\{0\}$, $\mathfrak{P}_2=(\mathfrak{P}_1\cap A)\cdot A[X]$, and $\mathfrak{P}_2$ is maximal.
\end{exercise}
\begin{proof}
Let $\mathfrak{P}_0\cap A=\mathfrak{P}_1\cap A=\mathfrak{P}_2\cap A=\p$. Then $\mathfrak{P}_i=\p A[X]+(X)$ for $i=0,1,2$, which contradicts that $\mathfrak{P}_i$ are distinct. Now assume that $\dim(A)=1$ and $\mathfrak{P}_i$ are nonzero. Recall that we have $2\leq \dim(A[X])\leq 3$ (\cref{dimension of polynomial ring upper bound}), and since $0$ is a prime ideal of $A[X]$, the chain $\mathfrak{P}_0\subset\mathfrak{P}_1\subset\mathfrak{P}_2$ shows that $\dim(A[X])=3$. Therefore $\mathfrak{P}_2$ is a maximal ideal in $A[X]$ and it contains $(X)$. Now consider the contracted chain
\[0\sub\mathfrak{P}_0^c\sub\mathfrak{P}_1^c\sub\mathfrak{P}_2^c.\]
Since $A$ has dimension $1$ and $\mathfrak{P}_2$ is maximal in $A[X]$, $\mathfrak{P}_2^c$ must be a maximal ideal in $A$. The inclusion $0\sub\mathfrak{P}_0^c$ can not be strict since otherwise we have $\mathfrak{P}_0^c=\mathfrak{P}_1^c=\mathfrak{P}_2^c$. This proves $\mathfrak{P}_0^c=\{0\}$. We also note that $\mathfrak{P}_1^c\neq\{0\}$ since otherwise $\mathfrak{P}_0=\mathfrak{P}_1=(X)$ (again by \cref{polynomial ring prime ideal contration prop}). Therefore $\mathfrak{P}_1^c=\mathfrak{P}_2^c$ is a maximal ideal of $A$, and since $\mathfrak{P}_1\neq\mathfrak{P}_2$, we have $X\notin\mathfrak{P}_2$ and $\mathfrak{P}_2=\mathfrak{P}_2^c\cdot A[X]$.
\end{proof}
\begin{exercise}\label{integral closed domain ring A[x] prop}
Let $A$ be an integrally closed domain with $K$ the fraction field. For any element $x\in K$, we denote by $A[x]$ the sub-$A$-algebra of $K$ generated by $x$. Let $x$ be nonzero in $K$ and $x^{-1}\notin A$; let $A[X]$ be the polynomial ring over $A$, and $\varphi:A[X]\to A[x]$ the canonical map $P\mapsto P(x)$.
\begin{itemize}
\item[(a)] The kernel of $\varphi$ is the ideal $\mathfrak{Q}$ generated by the polynomials of the form $aX+b$, where $a,b\in A$ and $ax+b=0$.
\item[(b)] Let $A$ be local with maximal ideal $\m_A$. Then the ideal $\m_AA[x]$ of $A[x]$ is prime, and is maximal if and only if $x\in A$.
\end{itemize}
\end{exercise}
\begin{proof}
Let $\varphi$ be the canonical map. Assume that $P(X)=\sum_{i=0}^{n}a_iX^i$ and $P(x)=0$. Then by multiplying $a_n^{n-1}$, we get
\[(a_nx)^n+a_{n-1}(a_nx)^{n-1}+\cdots+a_n^{n-2}a_1(a_nx)+a_n^{n-1}a_0=0,\]
so $-b:=a_nx$ is integral over $A$, hence belongs to $A$. Consider the polynomial $a_nX+b$: it has $x$ as a root and the same leading coefficient with $P$. Hence $a_nX+b\mid P(X)$ and assertion (a) is proved. We also note that, since $x^{-1}\notin A$, the element $b$ can not be invertible, hence $b\in\m_A$.\par
Now by (a), we have
\[A[x]/\m_AA[x]\cong A[X]/(aX+b,\m_A)=\kappa_A[X]/(\bar{a}X+\bar{b})=\kappa_A[X]/(\bar{a}X)\]
where $a,b$ are the generators of $\ker\varphi$, and we have remarked that $b\in\m_A$. Since these are irreducible elements, we conclude that $\m_AA[x]$ is prime, and maximal iff there exists $a\in A$ in these generators such that $\bar{a}\neq 0$. But this just means $a\notin\m_A$, hence invertible, and we then have $x=a^{-1}b\in A$.
\end{proof}
\begin{exercise}
Let $A$ be an integrally closed domain with fraction field $K$. Given a prime ideal $\p$ of $A$, an element $x$ of $K$ such that $x\notin A_\p$, and $x^{-1}\notin A_\p$, the ideal $\p A[x]$ of $A[x]$ is prime; we have $\p A[x]\cap A=\p$ and the canonical homomorphism $(A/\p)[X]\to A[x]/\p A[x]$ is an isomorphism.
\end{exercise}
\begin{proof}
Since $x\notin A_\p$, and $x^{-1}\notin A_\p$, we have $x\notin A$, so we can prove that $\p A[x]$ is prime. Since $A$ is integrally closed, it is not hard to se $\p A[x]\cap A=\p$. Since $x\notin A_\p$ and $x^{-1}\notin A_\p$, by Exericse~\ref{integral closed domain ring A[x] prop}, the kernel of the canonical map $\varphi:A[X]\to A[x]$ the ideal $\mathfrak{Q}$ generated by the polynomials of the form $aX+b$, where $a,b\in\p$ and $ax+b=0$. Since we have
\[A[x]/\p A[x]=A[X]/(\p,\mathfrak{Q})=(A/\p)[X]/\bar{\mathfrak{Q}}\]
and the image of $\mathfrak{Q}$ in $A/\p$ is zero, we see $(A/\p)[X]$ is isomorphic to $A[x]/\p A[x]$.
\end{proof}
\begin{exercise}\label{polynomial ring dimension preparation}
We say an integral domain $A$ is an \textbf{$\bm{F}$-ring} if $\dim(A)=1$ but $\dim(A[X])>2$.
\begin{itemize}
\item[(a)] Using the previous exercise, show that if $A$ is a one-dimensional integral domain then $\dim(A[X])=2$ if and only if the localization at any prime ideal of the integral closure of $A$ is a valuation ring.
\item[(b)] Show that if we have $\dim(A)=n$ and $\dim(A[X])>n+1$, there exists a prime ideal $\p$ of $A$ such that $\height(\p)=1$ and, either $\dim(A_\p[X])>2$, or $\dim((A/\p)[X])>\dim(A/\p)+1$.
\end{itemize}
\end{exercise}
\begin{proof}
Let $A$ be as in (a) and by taking integral closure and note that if $R$ is the integral closure of $A$, then $R[X]$ is the integral closure of $A[X]$, we may assume that $A$ is integrally closed. Assume that there exists $0\neq x\in K$ such that $x,x^{-1}\notin A_\p$. Then by \cref{integral closed domain ring A[x] prop} applied to $A_\p$, the ideal $\p A_\p[x]$ is not maximal in $A_\p[x]$, whence $A_\p[x]$ has dimension $\geq 2$. But since $A_\p[x]$ is a quotient of $A_\p[X]$, this implies $\dim(A[X])\geq\dim(A_\p[X])\geq 3$, so $\dim(A[X])=2$ iff there is not such $x$, or in other words $A_\p$ is a valuation ring.\par
Now let $A$ be $n$-dimensional and $\dim(A[X])\neq n+1$. Suppose that for some minimal prime ideal $\p$ of $A$, $\p A[X]$ is not minimal in $A[X]$; that is, there exists a prime ideal $\mathfrak{P}$ of $A[X]$ such that
\[0\subset\mathfrak{P}\subset\p A[X].\]
Then $0\subset\mathfrak{P}A_\p[X]\subset\p\cdot A_\p[X]$ is also a chain of prime ideals in $A_\p[X]$, as one easily verifies. Since $\p\cdot A_\p[X]$ is not maximal (for example $\p\cdot A_\p[X]+(X)$ contains it), this shows that $A_\p$ is an $F$-ring. We pass then to the case that $\p A[X]$ is minimal for every minimal prime ideal $\p$ of $A$. Let
\[0\subset\mathfrak{P}_1\subset\cdots\subset\mathfrak{P}_{n+2}\]
be a chain of prime ideals in $A[X]$ (recall that $\dim(A[X])\geq n+2$). If $\mathfrak{P}_1\cap A=\p$ is nonzero, then $A/\p$ is at most $(n-1)$-dimensional, and $A[X]/\p A[X]$ is a polynomial ring in one variable over $A/\p$ and is at least $(n+1)$-dimensional, which proves the claim. So we may suppose $\mathfrak{P}_1\cap A=0$. But then $\mathfrak{P}_2\cap A=\p_2\neq 0$ (by \cref{polynomial ring chain of prime contraction prop}); let $\p$ be a minimal prime ideal contained in $\p_2$-such exists since $A$ is finite dimensional; then $\p A[x]\sub\mathfrak{P}_2$, properly, since $\p A[X]$ is minimal but $\mathfrak{P}_2$ is not. Replacing $\mathfrak{P}_1$ by $\p A[X]$, we come back to a previous case, and the proof is complete. 
\end{proof}
\begin{exercise}\label{polynomial ring dimension counterexample}
Let $O$ be an integrally closed ring, with fraction field $k$. We put $d=\dim(O)$ and $t=\dim(O[X])$. Let $k'$ be a non-trivial extension of $k$ in which $k$ is algebraically closed and let $R$ be a discrete valuation ring (which is not a field) with residual field $k'$. We denote $K$ the field of fractions of $R$ and $A$ the set of elements of $R$ whose image in $k'$ belongs to $O$.
\begin{itemize}
\item[(a)] Show that the ring $A$ is integrally closed with fraction field $K$.
\item[(b)] Let $\p$ be the kernel of the surjective homomorphism $A\to O$. Show that every nonzero prime ideal of $A$ contains $\p$ and hence $\dim(A)=d+1$.
\item[(c)] Considering an element $x$ of $R$ whose image in $k'$ does not belong to $k$, show that the local ring $A_\p$ is not not a valuation ring, and that $\p A_\p[x]$ is not minimal among non-zero prime ideals of $A_\p[x]$. By using \cref{polynomial ring dimension preparation}, deduce that we have $\dim(A[x])\geq t+2$. Conclude that we have $\dim(A[x])=t+2$ by a direct argument.
\item[(d)] Finally, show that for any pair $(d,t)$ of integers with $d+1\leq t\leq 2d+1$, there exists an integrally closed domain $A$ of dimension $d$ such that $\dim(A[X])=t$.
\end{itemize}
\end{exercise}
\begin{proof}
Let $v$ be the valuation of $K$ associated with $R$. Note that any element in $K$ can be written as $a/b$ where $a,b\in R$. Moreover, we can assume that $a,b$ has positive valuations, whence are in $\m_R\sub A$. This shows the fraction field of $A$ is $K$. Now let $x\in K$ be integral over $A$, and let
\[x^n+a_{n-1}x^{n-1}+\cdots+a_0=0\]
be an equation of integral dependence. Dividing this equation by $x^s$ and supposing $x^{-1}$ to have residue $0$ in $k'$, we get the contradiction $1=0$, so $x^{-1}\notin\m_R$ and thus $x\in R$; we then have
\[\bar{x}^n+\bar{a}_{n-1}\bar{x}^{n-1}+\cdots+\bar{a}_0=0\]
where the bars indicate residues. Since each $\bar{a}_i$ is in $O$ and $k$ is algebraically closed in $k'$ we deduce that $\bar{x}\in K$; whence $\bar{x}\in O$, since $O$ is integrally closed. Hence $A$ is integrally closed.\par
Let $\p$ be the kernel of the canonical map $A\to O$. Then $\p$ is a prime ideal. From the definitions one obtains $A/\p\cong O$, whence $A$ is at least $(n+1)$-dimensional. If $\q$ is a nonzero prime ideal in $A$, then $\q$ contains $\p$: In fact, let $x\in\q$; since $x$ is $A\sub R$, we have $v(x)=e\geq 0$. Since any element in $\p$ has valuation $\geq 1$, its $(e+1)$-th power is divisible by $x$, whence $\p^{e+1}\sub\q$ and so $\p\sub\q$. From this it follows that $A$ is at most $(d+1)$-dimensional, so $\dim(A)=d+1$.\par
The quotient ring $A_\p$ is integrally closed and has one nonzero prime ideal. Moreover it is not a valuation ring if $k\neq k'$. In fact, let $x\in R$ be an element having residue in $k'$ but not in $k$ (as in (c)). Note that $x$ is in the quotient field of $A_\p$, which is $K$; but neither $x$ nor $x^{-1}$ has residue in $k$, so neither $x$ nor $x^{-1}$ is in $A_\p$. Thus $A_\p$ is not a valuation ring, and hence is an $F$-ring (\cref{polynomial ring dimension preparation}(a) applied to $A_\p$). It follows that $\p A[X]$ is not minimal among nonzero prime ideals in $A[X]$. Now
\[A[X]/\p A[X]\cong(A/\p)[X]\cong O[X]\]
so $A[X]$ is at least $(t+2)$-dimensional (recall that is integral).\par
Finally, let $0=\mathfrak{P}_0\subset\mathfrak{P}_1\subset\cdots\subset\mathfrak{P}_n$ be a chain of prime ideals in $A[X]$. Assume that the chain $0\subset\mathfrak{P}_1$ is saturated; then $\mathfrak{P}_1\cap A=0$, as otherwise
\[\p\subset\mathfrak{P}_1\cap A\And \p\cdot A[X]\subset\mathfrak{P}_1.\]
Similarly one concludes that if the chain $0\subset\mathfrak{P}_1\subset\mathfrak{P}_2$ is saturated, then
\[\mathfrak{P}_2\cap A=\p\And \mathfrak{P}_2=\p A[X]\]
(by \cref{polynomial ring chain of prime contraction prop}, $\mathfrak{P}_2\cap A$ can not be $0$). From this it follows at once that $A[X]$ is at most $(t+2)$-dimensional.\par
We say an integral domain $A$ is \textit{of type $(d,t)$} if $\dim(A)=d$ and $\dim(A[X])=t$. Now any field is of type $(0,1)$, and from the above construction we get an integrally closed domain of type $(1,3)$. Note that a Noetherian integrally closed domain is of type $(1,2)$. We now prove (d) by induction on $d$. So suppose by induction that for some $d$ and each $t$ (where $d+1\leq t\leq 2d+1$), we have an integrally closed ring of type $(d,t)$. Note that if $d+3\leq t\leq 2d+3$, then $d+1\leq t-2\leq 2d+1$ and from an integrally closed ring of type $(d,t-2)$ we get an integrally closed ring of type $(d,t)$. If on the other hand $t=d+2$, then a Noetherian integrally closed domain of dimension $d+1$ (for example the ring $k[X_1,\dots,X_{d+1}]$ where $k$ is a field) is of type $(d+1,d+2)$, so the claim is proved.
\end{proof}
\begin{exercise}
\mbox{}
\begin{itemize}
\item[(a)] Let $A$ be an integral domain, $K$ its fraction field and $n$ an integer. Let $t_1,\dots,t_n$ be elements of $K$ and $\varphi:A[X_1,\dots,X_n]\to A[t_1,\dots,t_n]$ the homomorphism such that $\varphi(X_i)=t_i$. Show that the height of the kernel of $\varphi$ equals to $n$.
\item[(b)] Deduce that for any $n$-tuple $(t_1,\dots,t_n)$ of elements of $K$, we have
\[\dim(A[t_1,\dots,t_n])\leq\dim(A[X_1,\dots,X_n])-n.\]
\end{itemize}
\end{exercise}
\begin{exercise}
Let $A$ be an integral domain with fraction field $K$. Let $0\subset\p_1\subset\cdots\subset\p_n$ be a chain of prime ideals of $A$. Show that there exists a valuation ring $R$ of $K$ and a chain of prime ideals $0\subset\q_1\subset\cdots\subset\q_n$ of $R$ such that $\q_i\cap A=\p_i$.
\end{exercise}
\begin{exercise}
Let $A$ be an integral domain and $K$ its fraction field. For any integer $k\geq 0$, the following properties are equivalent:
\begin{itemize}
\item[(\rmnum{1})] Any subring $B$ of $K$ containing $A$ has dimension $\leq k$.
\item[(\rmnum{2})] Any valuation ring $R$ of $K$ containing $A$ has dimension $\leq k$.
\item[(\rmnum{3})] For any $k$ elements $(t_1,\dots,t_k)$ of $K$, we have $\dim(A[t_1,\dots,t_k])\leq k$.
\item[(\rmnum{4})] We have $\dim(A[X_1,\dots,X_m])\leq k+m$ for any integer $m\geq 0$.
\end{itemize}
\end{exercise}
\section{Dimension of Noetherian rings}
\subsection{Dimension of a quotient ring}
\begin{proposition}\label{Noe integral domain minimal prime of element height 1}
Let $A$ be a Noetherian integral domain, $x$ a nonzero element of $A$ and $\p$ a minimal prime belonging to $x$. Then $\p$ has height $1$.
\end{proposition}
\begin{proof}
Let $\q\subset\p$ be a prime ideal distinct from $\p$. Then $x\notin\q$ by the hypothesis on $\p$. Since $A$ is integral, $A_\p$ is identified with a subring of $A_\q$. For any positive integer $n$, we denote by $\q_n$ the ideal $\q^nA_\q\cap A_\p$ of $A_\p$. The minimal property of $\p$ means that the local ring $A_\p/xA_\p$ is $0$-dimensional. It is therefore Artinian and by \cref{Noe local is Artin iff power of maximal ideal} there is a positive integer $n_0$ such that
\[\q_n+xA_\p=\q_{n+1}+xA_\p\for n\geq n_0.\]
Let's now fix an integer $n\geq n_0$. Given $y\in\q_n$, there exists $a\in A_\p$ such that $y-ax\in\q_{n+1}$. We then have $ax\in\q_n$, whence $a\in\q_n$ since $x\notin\q$, and finally we have $y\in\q_{n+1}+x\q_n$. So we have proved that
\[\q_{n}=\q_{n+1}+x\q_n.\]
As $x$ belongs to the maximal ideal of the Noetherian local ring $A_\p$, the Nakayama lemma shows that we have $\q_n=\q_{n+1}$. As $\q_nA_\q=(\q A_\q)^n$, we conclude that
\[(\q A_\q)^n=(\q A_\q)^{n+1}\for n\geq n_0.\]
Since $A_\q$ is a local Noetherian ring, we have $\bigcap_n(\q A_\q)^n=\{0\}$ by \cref{Noe ring Krull intersection thm}, whence $(\q A_\q)^{n_0}=0$ and the prime ideal $\q A_\q$ of $A_\q$ reduces to $0$. Then we must have $\q=0$ and therefore $\p$ has height $1$. 
\end{proof}
\begin{lemma}\label{Noe ring chain of prime adjusting for an element}
Let $A$ be a Noetherian ring, $\p_0\subset\cdots\subset\p_n$ a chain of prime ideals of $A$ and $x$ an element of $\p_n$. Then there exists a chain $\p_0'\subset\cdots\subset\p_n'$ of prime ideals of $A$ such that $\p_0'=\p_0$, $\p_n'=\p_n$, and $x\in\p_1'$.
\end{lemma}
\begin{proof}
We prove by induction. The case $n=1$ is trivial, so suppose $n\geq 2$ and that $x$ does not belong to $\p_{n-1}$ (otherwise we can apply to $\p_0\subset\cdots\subset\p_{n-1}$ the induction hypothesis). Let $\p_{n-1}'$ be a minimal element of the set of prime ideals of $A$ contained in $\p_n'=\p_n$ and containing $\p_{n-2}+Ax$. According to \cref{Noe integral domain minimal prime of element height 1}, the ideal $\p_{n-1}'/\p_{n-2}$ of the ring $A/\p_{n-2}$ is of height $1$, and since $\p_{n-2}\subset\p_{n-1}\subset\p_n$ is a chain of length $2$, so is $\p_{n-2}\subset\p_{n-1}'\subset\p_n'$. Since now $x\in\p_{n-1}'$, the induction hypothesis applied to the chain $\p_0\subset\cdots\subset\p_{n-1}'$ shows that there is a chain $\p_0'\subset\cdots\subset\p_{n-1}'$ with $x\in\p_1'$ and $\p_0'=\p_0$. The chain
\[\p_0'\subset\cdots\subset\p_{n-1}'\subset\p_n'\]
then satisfies the requirements. 
\end{proof}
\begin{proposition}\label{Noe ring dimension of quotient by ideal in Jacobson radical}
Let $A$ be a Noetherian ring and $\a$ an ideal contained in the Jacobson radical of $A$ and generated by $m$ elements. Then we have
\[\dim(A/\a)\leq\dim(A)\leq\dim(A/\a)+m.\]
\end{proposition}
\begin{proof}
The inequality $\dim(A/\a)\leq\dim(A)$ is clear. By induction it suffices to prove the inequality
\[\dim(A)\leq\dim(A/xA)+1\]
for any element $x$ contained in the Jacobson radical of $A$. That is, to show that $\dim(A/xA)\geq n-1$ for any chain $\p_0\subset\cdots\subset\p_n$ of prime ideals of $A$. Now since $x$ is in the Jacobson radical of $A$, we may assume that $x\in\p_n$. Then it suffices to construct a chain $\q_1\subset\cdots\subset\q_n$ of prime ideals of $A$, with $x\in\q_1$ and this follows from \cref{Noe ring chain of prime adjusting for an element}.
\end{proof}
\begin{corollary}\label{Noe ring dimension height prop}
\mbox{}
\begin{itemize}
\item[(a)] Every Noetherian semilocal ring is of finite dimension.
\item[(b)] Let $A$ be a Noetherian ring. Then every proper ideal of $A$ has finite height.
\item[(c)] Any decreasing sequence of prime ideals of a Noetherian ring $A$ is stationary.
\end{itemize}
\end{corollary}
\begin{proof}
Let $A$ be a Noetherian semilocal ring and $\r$ its Jacobson radical. The quotient ring $A/\r$ is then of dimension $0$ by \cref{prime ideal contain intersection}. If the ideal $\r$ is generated by $m$ elements, then we have $\dim(A)\leq m$ by \cref{Noe ring dimension of quotient by ideal in Jacobson radical}.\par
Let $A$ be Noetherian and $\a$ a proper ideal of $A$. Let $\m$ be a maximal ideal of $A$ containing $A$, then $0\leq\height(\a)\leq\dim(A_\m)$. Since $A_\m$ is a Noetherian local ring, we have $\dim(A_\m)<+\infty$ by (a), whence (b) follows.\par
Finally, any finite strictly decreasing sequence $(\p_i)_{1\leq i\leq n}$ of prime ideals of a Noetherian ring $A$ defines a chain $\p_n\subset\cdots\subset\p_0$, hence $n<\dim(A_{\p_0})<+\infty$ and (c) follows.
\end{proof}
\begin{corollary}\label{Noe local ring dim(A/xA) prop}
Let $A$ be a Noetherian local ring.
\begin{itemize}
\item[(a)] Let $x\in\m_A$. Then $\dim(A)-1\leq\dim(A/xA)\leq\dim(A)$, and $\dim(A/xA)=\dim(A)-1$ if and only if $x$ does not belong to any of the minimal prime ideals $\p$ of $A$ such that $\dim(A/\p)=\dim(A)$, and it suffices that $x$ is not a zero divisor in $A$.
\item[(b)] Let $\a$ be a proper ideal of $A$ such that $\dim(A/\a)<\dim(A)$. Then there exists $x\in\a$ such that $\dim(A/xA)=\dim(A)-1$.
\item[(c)] If $\dim(A)\geq 1$, then there exists $x\in\m_A$ such that $\dim(A/xA)=\dim(A)-1$.
\end{itemize}
\end{corollary}
\begin{proof}
By \cref{Noe ring dimension of quotient by ideal in Jacobson radical}, we see $\dim(A)-1\leq\dim(A/xA)\leq\dim(A)$. For $\dim(A/xA)=\dim(A)=n$, it is necessary and sufficient that there exists a chain $\p_0\subset\cdots\subset\p_n$ such that $x\in\p_0$, and this is equivalent to say there exists a prime ideal $\p_0$ containing $x$ such that $\dim(A/\p_0)=\dim(A)$. But such an ideal prime $\p_0$ is necessarily minimal, and any element of $\p_0$ is therefore a divisor of $0$ in $A$. This proves (a).\par
Let $\Phi$ be the set of minimal prime ideals of $A$, and $\Phi'$ the subset $\p\in\Phi$ such that $\dim(A)=\dim(A/\p)$. Since $A$ is Noetherian, $\Phi$ is finite. Let $\a$ be a proper ideal such that $\dim(A/\a)<\dim(A)$. Then for every $\p\in\Phi'$ we have $\dim(A/\a)<\dim(A/\p)$, whence $\a\nsubseteq\p$. Then by \cref{prime ideal contained in union}, there exists $x\in\a$ such that $x\notin\p$ for all $\p\in\Phi'$, and therefore $\dim(A/xA)=\dim(A)-1$ by (a), whence (b). Now (c) follows from (b) by taking $\a=\m_A$.
\end{proof}
\begin{example}
The claim in \cref{Noe integral domain minimal prime of element height 1} fails if $A$ is not Noetherian. For example, let $A$ be a valuation ring of dimension $d\geq 2$. Then the prime ideals of $A$ form a chain of length $d$:
\[0=\p_0\subset\p_1\subset\cdots\subset\p_d=\m_A.\]
Therefore, for each $i$, there exists $x_i\in\p_i\setminus\p_{i-1}$, and it is easy to see $\p_i$ is the only minimal prime belonging to $x_i$. 
\end{example}
\subsection{Dimension and secant sequences}
Let $A$ be a Noetherian, $M$ a finitely generated $A$-module, $S$ a subset of $A$ contained in the Jacobson radical of $A$. Let $\mathfrak{S}$ be the ideal generated by $S$ and $\a$ the annihilator of $M$. Then we have
\begin{align}\label{dimension and secant sequence-1}
\dim_A(M)=\dim(A/\a)\leq\dim(A).
\end{align}
Moreover, the support of the $A$-module $M/SM$ is equal to $V(\a+\mathfrak{S})$ by \cref{supp of module quotient by ideal product}, hence
\begin{align}\label{dimension and secant sequence-2}
\dim_A(M/SM)=\dim(A/(\a+I)).
\end{align}
By \cref{Noe ring dimension of quotient by ideal in Jacobson radical}, we then have the following inequality
\begin{align}\label{dimension and secant sequence-3}
\dim_A(M/SM)\leq\dim_A(M)\leq|S|+\dim_A(M/SM).
\end{align}
\begin{definition}
Let $A$ be a Noetherian local ring and $S$ a subset of $\m_A$. Then $S$ is said to be \textbf{secant for $\bm{M}$} if
\begin{align}\label{Noe local ring secant def}
\dim_A(M)=\dim_A(M/SM)+|S|.
\end{align}
A family $(x_i)_{i\in I}$ of element of $\m_A$ is said to be \textbf{secant for $\bm{M}$} if
\[\dim_A(M)=|I|+\dim(M/\sum x_iM)\]
and an element $x\in\m_A$ is called secant for $M$ if $\{x\}$ is secant for $M$.
\end{definition}
By definition, if $\a$ is the annihilator of $M$, then a subset $S$ of $\m_A$ is secant for $M$ if and only if it is secant for the $A$-module $A/\a$. Also note that, since $A$ is Noetherian and local, we have $\dim(A)<+\infty$, so if $S$ is $M$-secant then $S$ is finite.
\begin{proposition}\label{Noe local ring disjoint subset secant iff}
Let $A$ be a Noetherian local ring, $M$ a finitely generated $A$-module, and $S,S'$ two disjoint subsets of $\m_A$. Then the set $S\cup S'$ is secant for $M$ if and only if $S$ is secant for $M$ and $S'$ is secant for $M'=M/SM$.
\end{proposition}
\begin{proof}
For this, we first observe that
\[M/(SM+S'M)=(M/SM)/((SM+S'M)/SM)=M'/S'M'\]
and therefore
\begin{align*}
\dim_A(M/(SM+S'M))&=\dim_A(M'/S'M')\geq\dim_A(M')-|S'|\geq\dim_A(M)-|S|-|S'|.
\end{align*}
This proves the claim by the definition of secanteness.
\end{proof}
\begin{corollary}\label{Noe local ring sequence secant iff successive quotient}
Let $A$ be a Noetherian local ring and $x_1,\dots,x_r$ be distinct elements of $\m_A$. Define $M_1=M$ and $M_{i+1}=M_i/x_iM_i$. Then the following conditions are equivalent.
\begin{itemize}
\item[(\rmnum{1})] The sequence $(x_1,\dots,x_r)$ is $M$-secant.
\item[(\rmnum{2})] The element $x_i$ is $M_i$-secant for each $i$.
\item[(\rmnum{3})] The element $x_i$ is $M/(x_1M+\cdots+x_{i-1}M)$-secant for each $i$.
\end{itemize}
\end{corollary}
\begin{proof}
The equivalence of (\rmnum{1}) and (\rmnum{2}) follows by applying \cref{Noe local ring disjoint subset secant iff} on the subsets $\{x_1\},\dots,\{x_n\}$, and that of (\rmnum{1}) and (\rmnum{3}) follows similarly.
\end{proof}
\begin{proposition}\label{Noe local ring sequence secant iff power secant}
Let $A$ be a Noetherian local ring, $M$ a finitely generated $A$-module, $x_1,\dots,x_r$ elements of $\m_A$ and $n_1,\dots,n_r$ integers. Then the sequence $(x_1,\dots,x_r)$ is secant for $M$ if and only if the sequence $(x_1^{n_1},\dots,x_r^{n_r})$ is secant for $M$.
\end{proposition}
\begin{proof}
We may reduce the case $r=1$, so let $x\in\m_A$ and $n$ be a positive integer. If $\a$ is the annihilator of $M$, then by \cref{supp of module quotient by ideal product},
\[\supp(M/xM)=V(\a)\cap V(x),\quad \supp(M/x^nM)=V(\a)\cap V(x^n).\]
By \cref{Spec of ring closed subsets prop} we have $V(x^n)=V(x)$, whence the proposition.
\end{proof}
\begin{proposition}\label{Noe local ring secant iff minimal prime of supp}
Let $A$ be a Noetherian local ring and $M$ a finitely generated $A$-module. For an element $x$ in $\m_A$ to be secant for $M$, it is necessary and sufficient that it does not belong to any minimal elements $\p$ of $\supp(M)$ such that $\dim(A/\p)=\dim_A(M)$, and it suffices that the homothety $h_x$ with ratio $x$ on $M$ be injective.
\end{proposition}
\begin{proof}
Let $\a$ be the annihilator of $M$. Then $x$ is secant for $M$ means it is secant for $A/\a$, and if $h_x$ is injective on $M$, then $x$ is not a zero divisor of $A/\a$. The claim now follows from \cref{Noe local ring dim(A/xA) prop} applying to the ring $A/\a$.
\end{proof}
\begin{corollary}\label{Noe local ring complete secant is secant}
Any seuqnece of elements of $\m_A$ that is complete secant for $M$ is secant for $M$.
\end{corollary}
\begin{proof}
Let $(x_1,\dots,x_r)$ be a sequence of elements of $\m_A$ that is complete secant for $M$. Put $M_0=M$ and $M_i=M_{i-1}/x_iM_{i-1}$ for $1\leq i\leq r$. Then by \cref{Koszul complex completely secant annd regular relation}, the homothety of ratio $x_i$ is injective on $M_{i-1}$, so $\dim_A(M_i)=\dim_A(M_{i-1})-1$ by \cref{Noe local ring secant iff minimal prime of supp}. We then conclude that
\[\dim_A(M)=r+\dim_A(M/(x_1M+\cdots+x_rM)),\]
so the sequence $(x_1,\dots,x_r)$ is secant for $M$.
\end{proof}
\begin{remark}
A sequence $(x_1,\dots,x_r)$ is said to be \textbf{regular for $\bm{M}$} (or an \textbf{$\bm{M}$-regular sequence}) if for each $i$, the element $x_i$ is not a zerodivisor on $M/(x_1M+\cdots+x_{i-1}M)$. By \cref{Noe local ring sequence secant iff successive quotient} and \cref{Noe local ring secant iff minimal prime of supp}, it  follows that every regular sequence for $M$ is secant for $M$.
\end{remark}
\begin{theorem}\label{Noe local ring secant sequence prop}
Let $A$ be a Noetherian local ring, $M$ a finitely generated $A$-module, and $S$ a subset of $\m_A$.
\begin{itemize}
\item[(a)] If $M/SM$ has finite length, then $|S|\geq\dim_A(M)$.
\item[(b)] If $S$ is secant for $M$, then $|S|\leq\dim_A(M)$.
\item[(c)] Any secant subset for $M$ is contained in a maximal secant subset for $M$.
\item[(d)] The following conditions are equivalent:
\begin{itemize}
\item[(\rmnum{1})] $S$ is a maximal secant sequence for $M$;
\item[(\rmnum{2})] $S$ is a secant sequence for $M$ and $|S|=\dim_A(M)$;
\item[(\rmnum{3})] $M/SM$ has finite length and $|S|=\dim_A(M)$;
\item[(\rmnum{4})] $S$ is a secant sequence for $M$ and $M/SM$ has finite length.
\end{itemize}   
\end{itemize}
\end{theorem}
\begin{proof}
Since $S\sub\m_A$, by Nakayama lemma we have $M/SM\neq 0$, whence $\dim_A(M/SM)\geq 0$ with equality if and only if $M/SM$ is of finite length. Assertions (a) and (b) then follows from (\ref{dimension and secant sequence-3}) and (\ref{Noe local ring secant def}), as well as the equivalences of (\rmnum{2}), (\rmnum{3}), and (\rmnum{4}) in (d).\par
By (a), any secant subset for $M$ with cardinality $\dim_A(M)$ is maximal. It remains to prove that, if $S$ is secant for $M$ and if $|S|<\dim_A(M)$, then $S$ is not not maximum. Let $\a$ be the annihilator of $M$, and $B$ the local Noetherian ring $A/(\a+I)$, where $I$ is the ideal generated by $S$. According to \cref{Noe local ring dim(A/xA) prop}(c), there exists an element $x$ of $\m_A$ such that $\dim(B/xB)=\dim(B)-1$, whence $x\notin S$. According to \cref{Noe local ring disjoint subset secant iff}, the subst $S\cup\{x\}$ of $\m_A$ is secant for $A/\a$, hence for $M$. This proves assertion (c), and the equivalence of (\rmnum{1}) and (\rmnum{2}) in (d).
\end{proof}
\begin{corollary}\label{Noe local ring module dimension char}
Let $A$ be a Noetherian local ring, $M$ a finitely generated $A$-module. Then the dimension of $M$ is the smallest of the positive integers $d$ for which it $d$ there exists a sequence $(x_1,\dots,x_d)$ of elements of $\m_A$ such that the $A$-module $M/\sum_ix_iM$ is of finite length.
\end{corollary}
\begin{proof}
As $\emp$ is a secant subset for $M$, \cref{Noe local ring secant sequence prop}(c) shows the existence of a secant sequence for $M$, say $(x_1,\dots,x_d)$. But then $d=\dim_A(M)$ and the $A$-module $M/\sum_ix_iM$ is of finite length by property (\rmnum{3}) of \cref{Noe local ring secant sequence prop}(d). Conversely if $(x_1,\dots,x_{d'})$ is a sequence of elements of $\m_A$ such that the $A$-module of $M/\sum_ix_iM$ is of finite length, we have $d'\geq\dim_A(M)$ according to \cref{Noe local ring secant sequence prop}(a).
\end{proof}
If $\dim_A(M)=d$, then a sequence $(x_1,\dots,x_d)$ of elements of $\m_A$ such that $M/\sum_ix_iM$ is of finite length is called a \textbf{system of parameters} for $M$. Note that such a sequence must be secant for $M$, by (\rmnum{3}) of \cref{Noe local ring secant sequence prop}(d).\par
Let $A$ be a Noetherian local ring and $\m_A$ be its maximal ideal. Endow $A$ with its $\m_A$-adic topology, recall that an ideal $\a$ of $A$ is called a \textbf{defining ideal} of $A$ if the $\a$-adic topology coincides with its $\m_A$-adic topology.
\begin{proposition}\label{Noe local ring defining ideal iff}
Let $A$ be a Noetherian local ring and $\a$ an ideal of $A$. Then the following conditions are equivalent.
\begin{itemize}
\item[(\rmnum{1})] $\a$ is a defining ideal of $A$.
\item[(\rmnum{2})] $\a$ is $\m_A$-primary.
\item[(\rmnum{3})] There exist a positive integer $n$ such that $\m_A^n\sub\a\sub\m_A$.
\item[(\rmnum{4})] $\a$ is a proper ideal and $A/\a$ is an $A$-module of finite legnth.
\end{itemize}
\end{proposition}
\begin{proof}
By \cref{primary submodule eg}, we see (\rmnum{2}) and (\rmnum{3}) are equivalent, also (\rmnum{1}) is equivalent to (\rmnum{3}) by the definition of adic topologies and the fact that $\m_A$ is a prime ideal. Finally, the equivalence of (\rmnum{1}) and (\rmnum{4}) follows from \cref{associated prime maximal iff finite length} and $\supp(A/\a)=V(\a)$. 
\end{proof}
\begin{corollary}\label{Noe local ring dimension char defining ideal}
The dimension of a local Noetherian ring $A$ is the smallest of the positive integers $d$ for which there exists a defining ideal of $A$ generated by $d$ elements.
\end{corollary}
\begin{proof}
This follows from \cref{Noe local ring module dimension char} and \cref{Noe local ring defining ideal iff}.
\end{proof}
Let $A$ be a ring and $X=\Spec(A)$ be its spectrum. A closed subset of the form $V(x)$ with $x\in A$ of $X$ is called a \textbf{hypersurface} of $X$.
\begin{proposition}\label{Noe ring codim of intersection with hypersurface}
Let $A$ be a Noetherian ring, $Y$ be a closed subset of $X=\Spec(A)$, $H_1,\dots,H_m$ be hypersurfaces of $X$, and $\tilde{Y}=Y\cap H_1\cap\cdots\cap H_m$.
\begin{itemize}
\item[(a)] For any closed subset $V$ of $Y$ contained in $\tilde{Y}$, we have $\codim(V,\tilde{Y})\geq\codim(V,Y)-m$.
\item[(b)] For any irreducible componenet $Z$ subset of $\tilde{Y}$, we have $\codim(Z,Y)\leq m$.
\item[(c)] If $Z$ is an irreducible closed subset of $X$ contained in $Y$ such that $\codim(Z,Y)\leq m$, there are hypersurfaces $\tilde{H}_1,\dots,\tilde{H}_m$ such that $Z$ is an irreducible component of $Y\cap \tilde{H}_1\cdots\cap\tilde{H}_m$.
\end{itemize}
\end{proposition}
\begin{proof}
Let $\a$ be an ideal of $A$ and $x_1,\dots,x_m$ be elements of $A$ such that $Y=V(\a)$ and $H_i=V(x_i)$ for each $i$. Let $V(\p)$ be an irreducible closed subset contained in $Y$, where $\p$ is a prime ideal of $A$ containing $\a$. Suppose first that $Z$ is contained in $\tilde{Y}$ and denote by $\xi_i$ the image of $x_i$ in the Noetherian local ring $B=A_\p/\a A_\p$. According to \cref{height of ring prop}, we have
\[\codim(Z,Y)=\dim(B),\quad\codim(Z,\tilde{Y})=\dim(B/(\xi_1B+\cdots+\xi_mB))\]
so by \cref{Noe ring dimension of quotient by ideal in Jacobson radical} we have
\begin{align}\label{Noe ring codim of intersection with hypersurface-1}
\codim(Z,\tilde{Y})\geq\codim(Z,Y)-m.
\end{align}
If $Z$ is an irreducible component of $\tilde{Y}$, we have $\codim(Z,\tilde{Y})=0$, whence $\codim(Z,Y)\leq m$. This proves (b), and (a) follows by taking in both sides of (\ref{Noe ring codim of intersection with hypersurface-1}) the lower bound on the set of irreducible components $Z$ of $V$.\par
Conversely, suppose that we have $\codim(Z,Y)\leq m$, ie, $\dim(B)\leq m$. As any element of $A_\p$ is the product of an invertible element of $A_\p$ by the image of an element of $A$, \cref{Noe local ring dimension char defining ideal} demonstrates the existence of elements $\tilde{x}_1,\dots,\tilde{x}_m$ of $A$ whose images in $B$ generate a defining ideal of $B$. Set $\tilde{H}_i=V(\tilde{x}_i)$, it is then clear that $Z$ is an irreducible component of $Y\cap\tilde{H}_1\cap\cdots\cap\tilde{H}_m$.
\end{proof}
\begin{corollary}[\textbf{Krull's Hauptidealsatz}]\label{Krull principal ideal theorem}
Let $A$ be a Noetherian ring and $x_1,\dots,x_m$ be elements of $A$.
\begin{itemize}
\item[(a)] For any prime ideal $\p$ belonging to $Ax_1+\cdots+Ax_m$, we have $\height(\p)\leq m$.
\item[(b)] If $\p$ is a prime ideal of $A$ such that $\height(\p)\leq m$, then there exist elements $\tilde{x}_1,\dots,\tilde{x}_m$ of $A$ such that $\p$ is a minimal prime belonging to $A\tilde{x}_1+\cdots+A\tilde{x}_m$. 
\end{itemize}
\end{corollary}
\begin{proof}
This is a reformulation of \cref{Noe ring codim of intersection with hypersurface} in the case $Y=X$.
\end{proof}
\begin{corollary}\label{Noe ring codim of hypersurface}
Let $A$ be a Noetherian ring and $H$ be a hypersurface of $X=\Spec(A)$. Then the codimension of $H$ in $X$ is equal to $0$ or $1$. We have $\codim(H,X)=1$ if and only if $H$ does not contain any irreducible component of $X$. If so, all the irreducible components of $H$ are of codimension $1$ in $X$.
\end{corollary}
\begin{proof}
By \cref{Noe ring codim of intersection with hypersurface} we have $\codim(H,X)\leq 1$, and we have $\codim(H,X)=0$ if and only if $H$ contains an irreducible component of $X$. By definition,
\[\codim(H,X)=\inf_Z\codim(Z,X)\]
where $Z$ runs through the set of irreducible components of $H$. The claim then follows from these remarks.
\end{proof}
\begin{corollary}\label{Noe ring irreducible closed intersection with hypersurface}
Let $A$ be a Noetherian ring, $Y$ be an irreducible closed subset of $X=\Spec(A)$, and $H$ a hypersurface in $X$. Then only three cases are possible:
\begin{itemize}
\item[(a)] $Y\sub H$;
\item[(b)] $Y\cap H$ is nonempty and each of its irreducible components satisfies $\codim(Z,X)=1$;
\item[(c)] $Y\cap H$ is empty.
\end{itemize}
\end{corollary}
\begin{proof}
Suppose that $Y\cap H$ is nonempty and is not equal to $Y$. Then any irreducible component $Z$ of $Y\cap H$ satisfies $\codim(Z,Y)\leq 1$ by \cref{Noe ring codim of intersection with hypersurface}. Since $Z$ and $Y$ are both irreducible closed subsets, $\codim(Z,Y)=0$ iff $Y=Z$, which is impossible.
\end{proof}
\begin{corollary}\label{Noe UFD prime of height 1 prop}
If $A$ is a Noetherian UFD, then the prime ideals of height $1$ of $A$ are the principal ideals generated by the irreducible elements of $A$. If moreover $A$ is local, we have $\dim(A/\p)=\dim(A)-1$ for any prime ideal $\p$ of height $1$ of $A$. 
\end{corollary}
\begin{proof}
Let $x$ be an irreducible element of $A$. Then $Ax$ is a prime ideal because $x$ is prime, of height $1$ because $A$ is integral. Let $\p$ be a prime ideal of height $1$ of $A$. Then $V(\p)$ is an irreducible component of a hypersurface $V(x)$ for some $x$ (\cref{Noe ring codim of intersection with hypersurface}). Let $x=\prod_ip_i^{n_i}$ be a decomposition of $x$ into products of irreducible elements. Then the irreducible components of $V(x)$ is the $V(p_i)$, so $\p=Ap_i$ for some $i$. The last statement follows from \cref{Noe local ring dim(A/xA) prop}.
\end{proof}
\begin{proposition}\label{Noe ring sandwich prime union and intersection}
Let $A$ be a Noetherian ring and $\p\subset\q$ be an unsaturated chain of prime ideals of $A$. Then the set $E$ of prime ideals $\r$ of $A$ such that $\p\subset\r\subset\q$ is infinite and we have
\[\bigcap_{\r\in E}\r=\p,\quad\bigcup_{\r\in E}\r=\q.\]
\end{proposition}
\begin{proof}
By passing to the quotient ring, we may assume that $\p=\{0\}$. Now by \cref{Noe ring chain of prime adjusting for an element} we have $\bigcup_{\r\in E}\r=\q$, and it follows from \cref{prime ideal contained in union} that $E$ is infinite.\par
Now let $y$ be a nonzero element in $\bigcap_{\r\in E}\r$. Since the height of $\q$ is finite by \cref{Noe ring dimension height prop}, by localization at a prime ideal of height $2$, we may assume that $\height(\q)=2$. Then the ring $A/yA$ has dimension $1$, hence each prime ideal $\r\in E$ is minimal. But a Noetherian ring can have only finitely many minimal prime ideals, a contradiction. Therefore $\bigcap_{\r\in E}\r=\{0\}$, and the proof is finished.
\end{proof}
\begin{proposition}\label{Noe ring prime given height infinite}
Let $A$ be a Noetherian ring with dimension $\geq 2$, and $h$ an integer such that $0<h<\dim(A)$.
\begin{itemize}
\item[(a)] The ring $A$ has infinitely many prime ideals of $A$ with height $h$.
\item[(b)] If $A$ is of finite dimension, then there are infinitely many prime ideals $\p$ such that $\height(\p)=h$, $\coht(\p)=\dim(A)-h$. 
\end{itemize}
\end{proposition}
\begin{proof}
Since a Noetherian local ring has finite dimension, each prime ideal of $A$ has finite height. Since $h<\dim(A)$, there exist an integer $n>h$, a prime ideal $\p$ with $\height(\p)=n$, and a chain $\p_0\subset\cdots\subset\p_n=\p$ of prime ideals of $A$. Then $\height(\p_i)=i$ for each $i$, so $\height(\r)=h$ for every prime ideal $\r$ such that $\p_{h-1}\subset\r\subset\p_{h+1}$. By \cref{Noe ring sandwich prime union and intersection} the set $E$ of such prime ideals $\r$ is infinite, whence (a).\par
If $A$ is of finite dimension, then we may take $n=\dim(A)$ in the above proof. Then for each $\r\in E$, we have $\height(\r)=h$ and $\coht(\r)\leq n-h$. Since $\r\subset\p_{h+1}\subset\cdots\subset\p_n$ is a chain of length $n-h$, we then get $\coht(\r)=\dim(A)-h$.
\end{proof}
\begin{example}
There exist non-Noetherian integral domains of dimension $2$ with only one prime ideal of height $1$, for example the ring of a valuation of rank $2$.
\end{example}
\subsection{Extension of scalars}
\begin{proposition}\label{Noe local ring extension dim of tensor}
Let $\rho:A\to B$ be a local homomorphism of Noetherian local rings, $M$ a finitely generated $A$-module and $N$ a finitely generated $B$-module. If $\widebar{B}=B\otimes_A\kappa_A$ and $\widebar{N}=N\otimes_B\widebar{B}$, then 
\[\dim_B(M\otimes_AN)\leq\dim_A(M)+\dim_{\widebar{B}}(\widebar{N})\]
and the equality holds if $N$ is flat over $A$.
\end{proposition}
\begin{proof}
Let $S\sub\m_A$ be a system of parameters for the $A$-module $M$ and $T\sub\m_B$ a system of parameters for the $\widebar{B}$-module $\widebar{N}$. Let $E$ be the $B$-module $M\otimes_AN$. Since $\rho$ is local, we have $\rho(S)\sub\m_B$, and
\begin{align*}
E/(\rho(S)E+TE)&=E\otimes_B(B/(\rho(S)B+TB))=M\otimes_AN\otimes_B(B/TB)\otimes_A(A/SA)\\
&=M\otimes_A(A/SA)\otimes_AN\otimes_B(B/TB)=(M/SM)\otimes_A(N/TN)
\end{align*}
Moreover, by \cref{associated prime maximal iff finite length}, we have
\[\supp(M/SM)=\{\m_A\},\quad\supp(\widebar{N}/T\widebar{N})=\{\m_B\}\]
which implies, according to \cref{supp of module finite tensor} and \ref{supp of module extension ring}, that
\begin{align*}
\supp((M/SM)\otimes_A(N/TN))&=\supp(N/TN)\cap\rho^{*-1}(\{\m_A\})\\
&=\supp((N/TN)\otimes_A\kappa_A)=\supp(\widebar{N}/T\widebar{N})=\{\m_B\}.
\end{align*}
Therefore the $B$-module $E/(\rho(S)E+TE)$ has finite length, which implies by \cref{Noe local ring secant sequence prop} that
\[\dim_B(E)\leq|S|+|T|=\dim_A(B)+\dim_{\widebar{B}}(\widebar{N}).\]

Suppose now that $N$ is flat over $A$. Let $\a$ (resp. $\b$) be the annihilator of $M$ (resp. $N$). Then we have
\[\supp(E)=\supp(N)\cap\rho^{*-1}(\supp(N))=V(\b)\cap\rho^{*-1}(V(\a))=V(\b+\a B),\]
and therefore $\dim_B(M\otimes_AN)=\dim(B/(\b+\a B))$. On the other hand,
\[\dim_A(M)+\dim_{\widebar{B}}(\widebar{N})=\dim(A/\a)+\dim(B/(\b+\m_AB)).\]
Let $A'=A/\a$ and $B'=B/(\b+\a B)$ and let $\rho':A'\to B'$ be the local homomorphism deduced from $\rho$ by passing to quotients. Since the annihilator of $N$ is $\b$, $N$ is a finitely generated $B/\b$-module with support $\Spec(B/\b)$, and flat on $A$. By \cref{supp of module finite tensor} and \cref{local ring local homomorphism condition PM}, the local homomorphism $A\to B/\b$ deduced from $\rho$ therefore has the property (PM). By extension of the scalars, we deduce that $\rho'$ has the property (PM). According to \cref{ring homomorphism going-down dim inequality}, we have
\[\dim(B')\geq\dim(A')+\dim(B'/\m_{A'}B')\]
and as the ring $B'/\m_{A'}B'$ is isomorphic to $B/(\b+\m_AB)$, our assertion follows.
\end{proof}
\begin{corollary}\label{Noe local ring homomorphism dim and secant prop}
Let $\rho:A\to B$ be a local homomorphism of Noetherian local rings.
\begin{itemize}
\item[(a)] We have $\dim(B)\leq\dim(A)+\dim(B\otimes_A\kappa_A)$, and the equality holds if $B$ is flat over $A$.
\item[(b)] Suppose that $B$ is flat over $A$, then a subset $S$ of $\m_A$ is secant for $A$ if and only if $\rho(S)$ is secant for $B$.
\end{itemize}
\end{corollary}
\begin{proof}
The assertion in (a) follows from \cref{Noe local ring extension dim of tensor}, by setting $M=A$, $N=B$. For (b), if $B$ is flat over $A$, then
\[\dim(B)=\dim(A)+\dim(\widebar{B})\]
where $\widebar{B}=B\otimes_A\kappa_A$. Since $\rho$ is injective by \cref{ring faithfully flat iff}, we have $|\rho(S)|=|S|$. Finally $B'=B/\rho(S)B$ is a flat module on $A'=A/SA$, hence
\[\dim(B')=\dim(A')+\dim(\widebar{B})\]
since $B'/\m_{A'}B'$ is isomorphic to $\widebar{B}$. The claim in (b) then follows from these equalities.
\end{proof}
\begin{corollary}\label{Noe ring homomorphism dim inequality}
Let $\rho:A\to B$ be a ring homomorphism of Noetherian rings. Then
\[\dim(B)\leq\dim(A)+\sup_{\p\in\Spec(A)}\dim(B\otimes_A\kappa(\p)).\]
\end{corollary}
\begin{proof}
Let $\mathfrak{P}$ be a prime ideal of $B$ and $\p=\mathfrak{P}^c$. By \cref{Noe local ring homomorphism dim and secant prop}, we have 
\[\dim(B_{\mathfrak{P}})\leq\dim(A_\p)+\dim(B_{\mathfrak{P}}\otimes_A\kappa(\p))\leq\dim(A)+\dim(B\otimes_A\kappa(\p))\]
takign supremum on $\mathfrak{P}$, we get the claim.
\end{proof}
\begin{corollary}\label{Noe ring homomorphism going up down fiber inequality}
Let $\rho:A\to B$ be a ring homomorphism of Noetherian rings and $\p\subset\q$ be prime ideals of $A$. 
\begin{itemize}
\item[(a)] If the going-up property holds for $\rho$, then
\[\dim(B\otimes_A\kappa(\p))\leq\dim(B\otimes_A\kappa(\q)).\]
\item[(b)] If the going-down property holds for $\rho$, then
\[\dim(B\otimes_A\kappa(\p))\geq\dim(B\otimes_A\kappa(\q)).\]
\end{itemize}
\end{corollary}
\begin{proof}
First assume the going-up property for $\rho$. Let $\mathfrak{P}_0\subset\cdots\subset\mathfrak{P}_r$ be a chain of prime ideals of $B$ lying over $\p$ and $\p=\p_0\subset\cdots\subset\p_s=\q$ be a chain of prime ideals of $A$. Then by the going-up property, there is a chain $\mathfrak{P}_r\subset\cdots\subset\mathfrak{P}_{r+s}$ of prime ideals of $B$ such that $\mathfrak{P}_{r+i}^c=\p_i$. If $\mathfrak{Q}:=\mathfrak{P}_{r+s}$, then $\mathfrak{Q}$ is lying over $\q$ and $\height(\mathfrak{Q}/\p^e)\geq r+s$. Applying \cref{Noe ring homomorphism dim inequality} to the homomorphism $(A/\p)_\q\to (B/\p^e)_{\mathfrak{Q}}$, we get
\[r+s\leq\height(\mathfrak{Q}/\p^e)\leq\height(\q/\p)+\dim(B_{\mathfrak{Q}}/\q B_{\mathfrak{Q}})\leq s+\dim(B\otimes_A\kappa(\q))\]
whence the claim in (a).\par
Now we prove (b), so assume the going down property for $\rho$. We may assume that $\height(\q/\p)=1$, and it is enough to prove that, given a chain $\mathfrak{Q}_0\subset\cdots\subset\mathfrak{Q}_t$ of prime ideals of $B$ lying over $\q$ such that $\height(\mathfrak{Q}_i/\mathfrak{Q}_{i-1})=1$, we can construct a chain of prime ideals $\mathfrak{P}_0\subset\cdots\subset\mathfrak{P}_t$ of $B$ lying over $\p$ such that $\mathfrak{P}_i\subset\mathfrak{Q}_i$ and $\height(\mathfrak{P}_i/\mathfrak{P}_{i-1})=1$ for all $i$.\par
The existence of $\mathfrak{P}_0$ is guaranteed by the going-down property, and if $\mathfrak{P}_0,\dots,\mathfrak{P}_i$ is constructed, take $x\in\q\setminus\p$ and let $\mathfrak{T}_1,\dots,\mathfrak{T}_s$ be the minimal prime ideals belonging to $\mathfrak{P}_i+xB$. Then by \cref{Noe integral domain minimal prime of element height 1} we have $\height(\mathfrak{T}_j/\mathfrak{P}_i)=1$, while $\height(\mathfrak{Q}_{i+1}/\mathfrak{P}_i)\geq 2$, so there exists an element $y\in\mathfrak{Q}_{i+1}\setminus\bigcup_{j=1}^{s}\mathfrak{T}_j$. Let $\mathfrak{P}_{i+1}$ be a minimal prime ideal of $\mathfrak{P}_i+yB$, then $\mathfrak{P}_{i+1}\neq\mathfrak{T}_j$ for all $j$ and \cref{Noe integral domain minimal prime of element height 1} shows that $\height(\mathfrak{P}_{i+1}/\mathfrak{P}_i)=1$, hence $\rho(x)\notin\mathfrak{P}_{i+1}$. Since $x\in\q$ and $\height(\q/\p)=1$, we must have $\mathfrak{P}_{i+1}^c=\p$, which finishes the induction process. 
\end{proof}
\begin{corollary}\label{Noe ring dimension of polynomial ring}
Let $A$ be a Noetherian ring and $n$ a positive integer. Then
\[\dim(A[X_1,\dots,X_n])=\dim(A)+n.\]
\end{corollary}
\begin{proof}
Let $B=A[X_1,\dots,X_n]$, then for any prime ideal $\p$ of $A$, $B\otimes_A\kappa(\p)$ is identified with the polynomial ring on $\kappa(\p)$ with indeterminates $X_1,\dots,X_n$, whence of dimension $n$. By \cref{Noe ring homomorphism dim inequality} we then have $\dim(B)\leq\dim(A)+n$, and the reverse inequality is already established.
\end{proof}
\begin{corollary}\label{Noe ring ideal extension height inequality}
Let $\rho:A\to B$ be a homomorphism of Noetherian rings and $\rho^*:\Spec(B)\to\Spec(A)$ be the induced map.
\begin{itemize}
\item[(a)] If $\rho^*$ is surjective, then $\height(\a^e)\leq\height(\a)$ for any ideal $\a$ of $A$.
\item[(b)] If $B$ is a faithfully flat $A$-module, then $\height(\a^e)=\height(\a)$ for any ideal $\a$ of $A$.
\end{itemize}
\end{corollary}
\begin{proof}
If $B$ is faithfully flat, then $\rho^*$ is surjective by \cref{prime ideal is contraction iff exist faithfully flat module} and we have $\height(\a)\leq\height(\a^e)$ by \cref{ring homomorphism condition PM prop}. Therefore it suffices to prove assertion (a). Assume that $\rho^*$ is surjective and let $\p$ be a prime ideal of $A$ containing $\a$ such that $\height(\p)=\height(\a)$. Let $\mathfrak{P}$ be a prime ideal of $B$ lying over $\p$ whose image in $B\otimes\kappa(\p)$ is minimal. Then $\dim(B_{\mathfrak{P}}\otimes\kappa(\p))=0$ and \cref{Noe ring homomorphism dim inequality} implies $\dim(B_{\mathfrak{P}})\leq\dim(A_\p)$, whence
\[\height(\a^e)\leq\height(\mathfrak{P})=\dim(B_{\mathfrak{P}})\leq\dim(A_\p)=\height(\p)=\height(\a).\]
This proves the claim.
\end{proof}
\begin{proposition}\label{Noe ring module dimension of completion}
Let $A$ be a Noetherian ring, $\a$ an ideal of $A$, and $M$ a finitely generated $A$-module. Let $\widehat{A}$ (resp. $\widehat{M}$) be the Hausdorff completions of $A$ (resp. $M$) under the $\a$-adic topology.
\begin{itemize}
\item[(a)] Let $\p$ be a prime ideal of $A$ containing $\a$. Then $\hat{\p}$ is a prime ideal of $\widehat{A}$ and $\dim_{\widehat{A}_{\hat{\p}}}(\widehat{M}_{\hat{\p}})=\dim_{A_\p}(M_\p)$. 
\item[(b)] We have $\dim_{\widehat{A}}(\widehat{M})=\sup_\m\dim_{\widehat{A}_{\hat{\p}}}(\widehat{M}_{\hat{\p}})=\sup_\m\dim_{A_\m}(M_\m)$, where $\m$ runs through prime (resp. maximal) ideals of $A$ containing $\a$. In particular, $\dim_{\widehat{A}}(\widehat{M})\leq\dim_A(M)$.
\end{itemize} 
\end{proposition}
\begin{proof}
Since the $\a$-adic topology on $A/\p$ is trivial, by \cref{filtration Noe ring I-adic exact seq} $\widehat{A}/\hat{\p}$ is identified with $A/\p$, so $\hat{\p}$ is a prime ideal of $\widehat{A}$. Since $\widehat{A}$ is flat over $A$, $\widehat{A}_{\hat{\p}}$ is flat over $A_\p$. Moreover the canonical homomorphism of $A$ in $\widehat{A}$ induces an isomorphism of $A/\a$ on $\widehat{A}/\hat{\a}$, therefore also an isomorphism of $A_\p/\p A_\p$ on $\widehat{A}_{\hat{\p}}/\p\widehat{A}_{\hat{\p}}$. We conclude assertion (a) by applying \cref{Noe local ring extension dim of tensor} to the homomorphism $A_\p\to\widehat{A}_{\hat{\p}}$ and to the modules $M_\p$ and $\widehat{A}_{\hat{\p}}$, noting that $M_\p\otimes_{A_\p}\widehat{A}_{\hat{\p}}$ is isomorphic to $\widehat{M}_{\hat{\p}}$.\par
By \cref{filtration I-adic completion maximal ideal}, the map $\m\mapsto\hat{\m}$ is a bijection from the set of maximal ideals of $A$ containing $\a$ to the set of maximal ideals of $\widehat{A}$. Thus assertion (b) follows from \cref{dimension of module prop}.
\end{proof}
\begin{corollary}\label{Noe Zariski ring module dimension of completion}
Let $A$ be a Noetherian Zariski ring. Then for any finitely generated $A$-module $M$, we have $\dim_A(M)=\dim_{\widehat{A}}(\widehat{M})$.
\end{corollary}
\begin{proof}
The topology on $A$ is $\a$-adic, where $\a$ is an ideal contained in the Jacobson radical of $A$. Thus the claim follows from \cref{Noe ring module dimension of completion}.
\end{proof}
\begin{corollary}\label{Noe ring dimension of completion inequality}
Let $A$ be a Noetherian ring, $\a$ an ideal of $A$, and $\widehat{A}$ the Hausdorff completion with respect to the $\a$-adic topology. Then $\dim(\widehat{A})\leq\dim(A)$, and the equality holds if $A$ is local and $\a$ is proper.
\end{corollary}
\begin{corollary}\label{Noe ring dimension of power series ring}
Let $A$ be a Noetherian ring and $n$ a positive integer. Then
\[\dim(A\llbracket X_1,\dots,X_n\rrbracket)=\dim(A)+n.\]
\end{corollary}
\begin{proof}
The ring $A\llbracket X_1,\dots,X_n\rrbracket$ is the Hausdorff completion of $A[X_1,\dots,X_n]$ with respect to the $\m$-adic topology, where $\m$ is generated by $X_1,\dots,X_n$. We then have
\[\dim(A\llbracket X_1,\dots,X_n\rrbracket)\leq\dim(A[X_1,\dots,X_n])=\dim(A)+n\]
and the reverse inequality is already established.
\end{proof}
\begin{corollary}\label{Noe ring dimension of restricted power series ring}
Let $A$ be a Noetherian ring and $\a$ an ideal of $A$. Suppose that $A$ is Hausdorff and complete under the $\a$-adic topology. Then for any positive integer $n$, we have
\[\dim(A\{X_1,\dots,X_n\})=\dim(A)+n.\]
\end{corollary}
\begin{proof}
The ring $A\{X_1,\dots,X_n\}$ is the $\a$-adic completion of $A[X_1,\dots,X_n]$, so
\[\dim(A\{X_1,\dots,X_n\})\leq\dim(A)+n.\]
The reverse inequality can be established just like the power series ring.
\end{proof}
\begin{corollary}\label{Noe local ring contained in completion dim}
Let $A$ be a Noetherian local ring, identified with a subring of its completion $\widehat{A}$, and $B$ a subring of $\widehat{A}$ containing $A$. Suppose that $B$ is a Noetherian local ring and that we have $\m_AB=\m_B$. Then $\dim(A)=\dim(B)$. 
\end{corollary}
\begin{proof}
By \cref{filtration Noe local ring contained in completion} the canonical injection of $A$ into $B$ extends to an isomorphism of $\widehat{A}$ on the completion $\widehat{B}$ of $B$, whence $\dim(B)=\dim(A)$.
\end{proof}
\subsection{Exersise}
\begin{exercise}
Let $k$ be a field, and set 
\[A=k[x,y],\quad B=[x,y,x/y],\quad \p=(x,y)A,\quad \q=(y,x/y)B\]
check that
\[\p=\q\cap A,\quad \height\p=\height\q=2,\quad\dim B_\q/\p B_\q=1\]
so that
\[\height\q<\height\p+\dim B_\q/\p B_\q\]
Show also by a concrete example that the going-down property does not hold between $A$ and $B$.
\end{exercise}
\begin{proof}
Clealry we can see $(x,y)A\sub(y,x/y)B$ so that $\p\sub\q\cap A$. Also, $\p$ is a maximal ideal in $A$, hence $\p=\q\cap A$.\par
We have $(y)=(y,x/y)$ in $B$, so
\[(y,x/y)\supset(y,x)\supset(0)\]
is a maximal chain of $\q$. For $\p$, it is easy to verify height $\p=2$.\par
For $\dim B_\q/\p B_\q$, note that
\[(x,y)B\subset(y,x/y)B\]
Since $(x,y)$ is maximal in $A$, if there is an ideal $Q$ sandwiched between them, then $Q$ contains a polynomial in $x/y$, hence contains $x/y$ since $Q$ is prime. This implies $\dim B_\q/\p B_\q=1$.\par
Consider the following chain in $A$:
\[P_1=(y)A\subset(x,y)A=P_2\]
And choose a prime $Q_2=(x,y)B$ in $B$ such that $Q_2^c=P_2$. If $Q_1$ is a prime ideal in $B$ such that $Q_1^c=P_1$, then $(y)A\sub Q_1$, in particular $y\in Q_1$. Note that since $x/y\in B$, this implies $x=(x/y)\cdot y\in Q_1$. But this means $Q_2\sub Q_1$, hence there is no prime ideal contained in $Q_2$ satisfying the condition, therefore the going down property does not hold.
\end{proof}
\begin{exercise}
Let $A$ be a ring, $U$ an open subset of $\Spec(A)$.
\begin{itemize}
\item[(a)] Show that the map $\p\mapsto V(\p)\cap U$ induce a bijection of $U$ onto the set of irreducible closed subsets of $U$.
\item[(b)] Show that $\dim(U)$ is the supremum of the length of chains of prime ideals of $A$ contained in $U$.
\item[(c)] Let $\Phi$ be the set of minimal prime ideals of $A$ contained in $U$. Show that
\[\dim(U)=\sup_{\p\in\Phi}\dim(V(\p)\cap U)\]
\end{itemize}
\end{exercise}
\section{Hilbert-Samuel series}
Let $A$ be an Artinian ring and consider the power series ring $A\llbracket T\rrbracket$ with one indeterminate on $A$. In this section, we denote by $A((T))$ the fraction ring $A\llbracket T\rrbracket_T$ of $A\llbracket T\rrbracket$ at the element $T$, which is identified with the subring of the Laurent series ring $A\llbracket T,T^{-1}\rrbracket$ consisting of elements $\sum_{n\in\Z}a_nT^n$ that is \textbf{bounded below}, that is, there exist $n_0\in\N$ such that $a_n=0$ for all $n<n_0$. The ring $\Z((T))$ will be our main focus in this section.\par
For any $n,p\in\Z$, we extend the definition of the binomial coefficient $\binom{n}{p}$ by defining $\binom{n}{p}=0$ if $p<0$ or $p>n$.
\begin{lemma}\label{inverse of (1-T)^-r expression}
The element $(1-T)$ is invertible in $\Z((T))$, and for any $r>0$ we have
\[(1-T)^{-r}=\sum_{n\in\Z}\binom{n+r-1}{r-1}T^n=\sum_{n\in\N}\binom{n+r-1}{r-1}T^n.\]
\end{lemma}
\begin{proof}
The element $1-T$ is invertible with inverse $\sum_{n\in\N}T^n$, so
\[(1-T)^{-r}=\Big(\sum_{n\in\N}T^n\Big)^r=\sum_{n_1,\dots,n_r\in\N}T^{n_1+\cdots+n_r}=\sum_{n\in\N}\binom{n+r-1}{r-1}T^n\]
so the claim follows.
\end{proof}
Let $Q(T)\in\Z[T,T^{-1}]$ and $r$ be a positive integer. If $P(T)=(1-T)^{-r}Q(T)$, then it is easy to see $P(T)\in\Z((T))$. In fact, if
\[Q(T)=\sum_{n\in\Z}a_nT^n,\quad P=\sum_{n\in\Z}b_nT^n\]
then
\[b_n=\sum_{i\in\Z}a_i\binom{n-i+r-1}{r-1}=\sum_{i\leq n}a_i\binom{n-i+r-1}{r-1}\]
where the summation is finite since $Q(T)$ is bounded below. If $n_0$ is the supremum of the integers $i\in\Z$ such that $a_i\neq 0$, then for $n\geq n_0$, we can write
\[b_n=\sum_{i\in\Z}a_i\binom{n-i+r-1}{r-1}=\frac{1}{(r-1)!}\sum_{i\in\Z}a_i\prod_{j=1}^{r-1}(n-i+j).\]
If $c=Q(1)=\sum_ia_i$, then we get
\begin{align}\label{polynomial product with (1-T)^-r coefficients}
b_n=c\frac{n^{r-1}}{(r-1)!}+\rho_nn^{r-2}
\end{align}
where the rational number $\rho_n$ tends to a limit as $n$ increases indefinitely. Therefore we deduce the relationship
\begin{align}\label{polynomial in Z[T,T^-1] value of Q(1) as limit}
Q(1)=(r-1)!\lim_{n\to\infty}n^{1-r}b_n.
\end{align}
For two elements $F=\sum_na_nT^n$ and $G=\sum_nb_nT^n$ in $\Z((T))$, we denote $F\leq G$ by the relation $a_n\leq b_n$ for all $n$, which is an order relation compatible with the ring structure of $\Z((T))$. We have $(1-T)^{-r}\geq 1$, and if an element $Q$ is bigger than $0$, then the integer $Q(1)$ is positive.
\begin{lemma}\label{series in Z((T)) product with (1-T) polynomial iff}
Let $P$ be a nonzero element in $\Z((T))$ such that $(1-T)^{r}P\in\Z[T,T^{-1}]$ for some $r\in\Z$. Then there exists a unique $Q\in\Z[T,T^{-1}]$ such that $P=(1-T)^{-d}Q$ and $Q(1)\neq 0$, for some $d\in\Z$. If $P\geq 0$, then $Q(1)\geq 0$.
\end{lemma}
\begin{proof}
Since $(1-T)^rP\in\Z[T,T^{-1}]$, we can write $P=(1-T)^{-1}T^nR(T)$ with $R(T)\in\Z[T]$, and by Euclidean division we may assume that $R(1)\neq 0$, which proves the existence of $d$ and $Q$. On the other hand, if we have
\[(1-T)^rQ(T)=(1-T)^sR(T)\]
with $r>s$ and $Q,R\in\Z[T,T^{-1}]$, then $R(T)=(1-T)^{r-s}Q(T)$, whence $R(1)=0$. This proves the uniqueness part.\par
Finally, assume that $P\geq 0$. If $P=(1-T)^{-d}Q$ with $d<0$ then $P(1)=0$ and we must have $P=0$, which is a contradiction. Therefore $d\geq 0$, and if $d=0$ then $P(1)=Q(1)\geq 0$. If $d>0$, then $Q(1)$ is given by the formula (\ref{polynomial in Z[T,T^-1] value of Q(1) as limit}), so $Q(1)\geq 0$.
\end{proof}
\begin{lemma}\label{polynomial product by (1-T) order iff}
Let $P,Q$ be elements of $\Z[T,T^{-1}]$, $p,q\in\Z$ and $P(1)>0$. If
\[(1-T)^{-p}P\leq(1-T)^{-q}Q\]
then, either $p<q$, or $p=q$ and $P(1)\leq Q(1)$.
\end{lemma}
\begin{proof}
Suppose $p\geq q$, then we have $(1-T)^{-p}[(1-T)^{p-q}Q-P]\geq 0$. Since $R=(1-T)^{p-q}Q-P\in\Z[T,T^{-1}]$, by \cref{polynomial product by (1-T) order iff} we have $R(1)\geq 0$. If $p>q$, then $R(1)=-P(1)\geq 0$, which is a contradiction. If $p=q$, then $R(1)=Q(1)-P(1)$, whence $P(1)\leq Q(1)$.
\end{proof}
\subsection{Poincar\'e series of graded modules over polynomial rings}
Let $A_0$ be an Artinian ring, $I$ a finite set and consider the ring $A=A_0[(X_i)_{i\in I}]$. For each $i\in I$, let $d_i$ be a positive integer. Endow $A$ with the graded ring structure of type $\Z$ such that the elements of $A_0$ are homogeneous of degree $0$ and each $X_i$ is homogeneous of degree $d_i$. If $d_i=1$ for each $i$, we then get the usual graduation on the polynomial ring $A$.\par
Let $M$ be a finitely generated graded $A$-module such that all the homogeneous components of $M$ are $A_0$-module of finite length. The \textbf{Poincar\'e series} of $M$, denoted by $P_M$, is the elemnt in $\Z((T))$ defined by
\[P_M=\sum_{n\in\Z}\ell_{A_0}(M_n)\cdot T^n.\]
\begin{theorem}\label{graded module Poincare series Q_M is polynomial}
The element $Q_M=P_M\prod_{i\in I}(1-T^{d_i})$ is in $\Z[T,T^{-1}]$.
\end{theorem}
\begin{proof}
If $I=\emp$, then $A=A_0$, and the family $(\ell_{A_0}(M_n))_{n\in\Z}$ has finite support since $M$ is a finitely generated $A_0$-module, whence has finite length. The theorem is therefore proved in this case.\par
Now we prove by induction on the cardinality of $I$. Fix $j\in I$, $J=I\setminus\{j\}$, and denote by $A'$ the subring of $A$ generated by $A_0$ and the $X_i$ with $i\in J$. Consider the homothety $h_{X_j}$ with ratio $X_j$ with kernel $R$ and cokernel $S$. Then for each $n\in\Z$, we have an exact sequence
\[\begin{tikzcd}
0\ar[r]&R_{n-d_j}\ar[r]&M_{n-d_j}\ar[r,"h_{X_j}"]&M_n\ar[r]&S_n\ar[r]&0
\end{tikzcd}\]
whence $R_n$ and $S_n$ have finite legnth, and we have
\begin{align}\label{graded module Poincare series Q_M is in Z[T]-1}
\ell_{A_0}(M_n)-\ell_{A_0}(M_{n-d_j})=\ell_{A_0}(S_n)-\ell_{A_0}(R_{n-d_j}).
\end{align}
Since $M$ is a finitely generated module over the Noetherian $A$, it is Noetherian so the $A$-modules $R$ and $S$ are finitely generated. Since they are both annihilated by $X_j$, they are finitely generated as $A'$-modules. According to the induction hypothesis, the elements $Q_R$ and $Q_S$ are in $\Z[T,T^{-1}]$. Moreover, due to (\ref{graded module Poincare series Q_M is in Z[T]-1}), we have 
\begin{align}\label{graded module Poincare series Q_M is in Z[T]-2}(1-T^{d_j})P_M=P_M-T^{d_j}P_M=P_S-T^{d_j}P_R
\end{align}
which implies
\[Q_M=P_M\prod_{i\in I}(1-T^{d_i})=P_S\prod_{i\in J}(1-T^{d_i})-T^{d_j}P_R\prod_{i\in J}(1-T^{d_i}).\]
This shows $Q_M$ is in $\Z[T,T^{-1}]$, and our conclusion is then proved.
\end{proof}
\begin{example}
Let $M_0$ be an $A_0$-module and $M=A\otimes_{A_0}M_0$. Then with the notations above, we have $R=0$ and $S=A'\otimes_{A_0}M_0$, so $Q_M=Q_S$ by (\ref{graded module Poincare series Q_M is in Z[T]-2}). Since $Q_{M_0}=\ell_{A_0}(M_0)$, by induction $Q_{M}=\ell_{A_0}(M_0)$, whence
\[P_M=\ell_{A_0}(M_0)\prod_{i\in I}(1-T^{d_i})^{-1}.\]
\end{example}
\begin{corollary}\label{graded module legnth given by c_M}
Assume that $d_i=1$ for all $i\in I$ and set $r=|I|$, $c_M=Q_M(1)$.
\begin{itemize}
\item[(a)] If $r=0$, then $\ell_{A_0}(M)=c_M$.
\item[(b)] If $r=1$, then $\ell_{A_0}(M_n)=c_M$ for $n$ large enough.
\item[(c)] If $r>1$, then $\ell_{A_0}(M_n)=\dfrac{c_M}{(r-1)!}n^{r-1}+\rho_nn^{r-2}$, where $\rho_n$ has a limit when $n\to+\infty$.
\end{itemize}
\end{corollary}
\begin{proof}
In this case we have $P_M=Q_M(1-T)^{-r}$, so this follows from the formula (\ref{polynomial in Z[T,T^-1] value of Q(1) as limit}).
\end{proof}
For simplicity, we may endow $A$ with the usual graduation henceforth, so we have
\[P_M=Q_M(1-T)^{-r}.\]
\begin{example}[\textbf{Example of Poincar\'e series for graded modules}]
\mbox{}
\begin{itemize}
\item[(a)] Let $\begin{tikzcd}[column sep=12pt]0\ar[r]&M'\ar[r]&M\ar[r]&M''\ar[r]&0\end{tikzcd}$ be an exact sequence of graded $A$-modules and of homomorphisms of degree $0$ such that $M$ is finitely generated over $A$ and $M_n$ of finite length over $A_0$ for each $n$. Then by the additivity of legnth, we have
\[P_M=P_{M'}+P_{M''},\quad Q_M=Q_{M'}+Q_{M''},\quad c_M=c_{M'}+c_{M''}.\] 
\item[(b)] Let $M(p)$ be the modulu deduced from $M$ by shifting $p$ from the graduation of $M$. Since $M(p)_n=M_{p+n}$, we have
\[P_{M(p)}=T^{-p}P_M,\quad Q_{M(p)}=T^{-p}Q_M,\quad c_{M(p)}=c_M.\]
\item[(c)] Let $M$ be a free graded $A$-module generated by lineraly independent homogeneous elements of degrees $\delta_1,\dots,\delta_s$. As $M$ is isomorphic to $A(-\delta_1)\oplus\cdots\oplus A(-\delta_s)$, we have
\[P_M=\ell(A_0)\Big(\sum_{i=1}^{s}T^{\delta_s}\Big)(1-T)^{-r},\quad Q_M=\ell(A_0)\Big(\sum_{i=1}^{s}T^{\delta_s}\Big),\quad c_M=s\cdot\ell(A_0).\]
\item[(d)] Let $M$ be a finitely generated $A$-module and suppose that there exists a long exact sequence of graded $A$-modules and of homomorphisms of degree $0$
\[\begin{tikzcd}
0\ar[r]&L_n\ar[r]&L_{n-1}\ar[r]&\cdots\ar[r]&L_0\ar[r]&M\ar[r]&0
\end{tikzcd}\] 
such that for each $k$, the module $L_k$ is free and generated by linearly independent homogeneous elements of degrees $\delta_{k,1},\dots,\delta_{k,m_k}$. Then by (c),
\[Q_M=\ell(A_0)\sum_{k=0}^{n}\sum_{j=1}^{m_k}(-1)^kT^{\delta_{k,j}},\quad c_M=\ell(A_0)\sum_{k=0}^{n}(-1)^km_k.\]
\end{itemize}
\end{example}
\begin{proposition}\label{graded module generating set Poincare series inequality}
Let $M$ be a graded $A$-module. If $M$ is generated by $M_0$ and $M_0$ is an $A_0$-module with finite length. Then we have
\[P_M\leq(1-T)^{-r}\ell_{A_0}(M_0),\quad c_M\leq\ell_{A_0}(M_0).\]
Moreover, the following conditions are equivalent:
\begin{itemize}
\item[(\rmnum{1})] $c_M=\ell_{A_0}(M_0)$.
\item[(\rmnum{2})] $P_M=\ell_{A_0}(M_0)(1-T)^{-r}$.
\item[(\rmnum{3})] The canonical homomorphism $\varphi:A\otimes_{A_0}M_0\to M$ is bijective.
\end{itemize}
\end{proposition}
\begin{proof}
Let $R$ be the kernel of $\varphi$. Then since $\varphi$ is surjective, we have
\[P_M=P_{A\otimes_{A_0}M_0}-P_R=\ell_{A_0}(M_0)(1-T)^{-r}-P_R,\quad c_M=\ell_{A_0}(M_0)-c_R\]
which proves the first assertion. Now the conditions (\rmnum{1}), (\rmnum{2}), and (\rmnum{3}) are equivalent to $c_R=0$, $P_R=0$, and $R=0$, respectively, so we have (\rmnum{3})$\Rightarrow$(\rmnum{2})$\Rightarrow$(\rmnum{1}). It remains to show that $c_R=0$ implies $R=0$. Now assume that $R\neq 0$ and let
\[0=M^r\subset M^{r-1}\subset\cdots\subset M^0=M_0\]
be a Jordan-H\"older series of the $A_0$-module $M_0$. Let $R^k$ be the intersection of $R$ and the image of $A\otimes_{A_0}M^k$ in $A\otimes_{A_0}M_0$. Then there exists an integer $k$ such that $R^k\neq R^{k-1}$. Define $L=R^{k-1}/R^k$, we have $0\leq c_L\leq c_R$, and it suffices to prove that $c_L\neq 0$. Now, if $K$ is the quotient field of $A_0$ by the maximal ideal annihilating $M^{k-1}/M^k$, then $L$ is identified with a nonzero graded submodule of $K[(X_i)_{i\in I}]$. So $L$ contains a submodule isomorphic to a shifted module of $K[(X_i)_{i\in I}]$. As $c_{K[(X_i)_{i\in I}]}=1$, we have $c_L\geq 1$, which proves the claim.
\end{proof}
\begin{proposition}\label{graded module polynomial ring over field c_M is rank}
Suppose that $A_0$ is a field and $M$ is a finitely generated graded $A$-module. If $K$ is the field of fraction of $A$, then $c_M$ is equal to the rank of the $A$-module $M$, which is also the dimension of the $K$-vector space $M\otimes_AK$.
\end{proposition}
\begin{proof}
This is clear if $M=A$, since $c_A=1$. On the other hand, let $x\in A$, homogeneous of degree d$,$ and not zero. We have $(A/xA)\otimes_AK=0$. From the exact sequence
\[\begin{tikzcd}
0\ar[r]&A(-d)\ar[r,"h_x"]&A\ar[r]&A/xA\ar[r]&0
\end{tikzcd}\]
we get $c_{A/xA}=0$. The proposition is therefore verified when $M$ is generated by a homogeneous element. The general case follows, since any finite type graded $A$-module has a composition sequence whose quotients are of the previous form.
\end{proof}
\begin{example}
Let $A_0$ be a field, $P$ a homogeneous polynomial in $A$ of degree $s$, and $M$ the module $A/PA$. Since the rank of $M$ equals to the degree of $P$, by \cref{graded module polynomial ring over field c_M is rank} we have $c_M=s$. On the other hand, it is not hard to compute that
\[\ell_{A_0}(M_n)=\binom{n+r-1}{r-1}-\binom{n-s+r-1}{r-1}\]
and therefore
\[c_M=(r-1)!\lim_{n\to\infty}n^{1-r}\Big(\binom{n+r-1}{r-1}-\binom{n-s+r-1}{r-1}\Big)=s\]
verifying our claim.
\end{example}
\subsection{Hilbert-Samuel series of a good-filtered module}
Let $A$ be a Noetherian ring, $\mathfrak{I}$ an ideal of $A$, and $M$ a finitely generated $A$-module. Recall that an exhaustive filtration $\mathcal{F}=(M_n)_{n\in\Z}$ on $M$ consisting of submodules is called \textbf{$\mathfrak{I}$-good} if
\begin{itemize}
\item[(a)] $\mathfrak{I} M_n\sub M_{n+1}$ for all $n\in\Z$.
\item[(b)] There exists an integer $n_0$ such that $\mathfrak{I} M_n=M_{n+1}$ for all $n\geq n_0$.
\end{itemize}
Similar to the case of series, we say the filtration $\mathcal{F}$ is \textbf{bounded below} if there exist $n_1\in\Z$ such that $M_{n}=M$ for all $n\leq n_1$. Note that if the filtration $\mathcal{F}$ is exhaustive and $M$ is finitely generated, then $\mathcal{F}$ is bounded below since $M$ is then Noetherian.
\begin{lemma}\label{filtration good finite legnth quotient if}
Let $(M_n)$ be an $\mathfrak{I}$-good filtration on $M$. If $M/\mathfrak{I} M$ has finite length, then $M/M_{n+1}$ and $M_{n}/M_{n+1}$ have finite length for all $n\in\Z$.
\end{lemma}
\begin{proof}
Since $\mathcal{F}$ is bounded below, there exist $n_1$ such that $M_{n_1}=M$. Also, as $(M_n)$ is $\mathfrak{I}$-good, there exist $n_0$ such that $\mathfrak{I} M_n=M_{n+1}$ for all $n\geq n_0$. Then for all $n\in\Z$,
\begin{align}\label{filtration good bounded difference with adic filtration}
\mathfrak{I}^{n-n_1}M\sub M_n\sub\mathfrak{I}^{n-n_0}M
\end{align}
where we define $\mathfrak{I}^n=A$ for $n\leq 0$. Then $\ell(M/M_n)\leq\ell(M/\mathfrak{I}^{n-n_1}M)$ and it suffices to prove that $\mathfrak{I}^nM/\mathfrak{I}^{n+1}M$ has finite length for each $n$. So we are reduced to the case of $\mathfrak{I}$-adic filtration. Let $(x_1,\dots,x_r)$ be a finite generating system of the $A$-module $\mathfrak{I}$, and let $I$ be the finite set of monomials of total degree $n$ in $r$ variables $X_1,\dots,X_r$. The homomorphism from $(M/\mathfrak{I} M)^I$ to $\mathfrak{I}^nM/\mathfrak{I}^{n+1}M$ which maps $(u_m)_{m\in I}$ to $\sum_mm(x_1,\dots,x_r)u_m$ is surjective. Since $M/\mathfrak{I} M$ has finite length, so does $\mathfrak{I}^nM/\mathfrak{I}^{n+1}M$.
\end{proof}
Suppose henceforth that $M/\mathfrak{I}M$ has finite length. Let $\mathcal{F}=(M_n)$ be an $\mathfrak{I}$-good filtration on $M$. We define therefore the \textbf{Hilbert-Samuel series} $H_{M}$ of $M$ with respect to the $\mathfrak{I}$-good filtration $(M_n)$ by
\[H_{M,\mathscr{F}}=\sum_{n\in\Z}\ell_{A/\mathfrak{I}}(M_n/M_{n+1})\cdot T^n\]
and the map $n\mapsto\ell_{A/\mathfrak{I}}(M_n/M_{n+1})$ is called the \textbf{Hilbert-Samuel function} of $M$ with respect to $(M_n)$.\par
Since there exists an integer $n_1$ such that $M_n=M$ for $n\leq n_1$, we see $H_M$ is an element in $\Z((T))$. We denote by $H_{M,\mathfrak{I}}$ by the Hilbert-Samuel series of $M$ with respect to the $\mathfrak{I}$-adic filtration, that is,
\[H_{M,\mathfrak{I}}=\sum_{n\in\Z}\ell_{A/\mathfrak{I}}(\mathfrak{I}^nM/\mathfrak{I}^{n+1}M)\cdot T^n.\]
If $P\in\Z((T))$ and $r\in\N$, we may use $P^{(1)}$ to denote the series $(1-T)^{-r}P$. In other words, if $P=\sum_{n\in\Z}a_nT^n$, then
\[P^{(1)}=\sum_{n\in\Z}\Big(\sum_{i\leq n}a_i\Big)T^n.\]
\begin{proposition}\label{filtration good Hilbert-Samuel series product with (1-T) formula}
Let $\mathcal{F}$ be an $\mathfrak{I}$-good filtration on $M$. Then
\[H_{M,\mathcal{F}}^{(1)}=\sum_{n\in\Z}\ell_A(M/M_{n+1})\cdot T^n.\]
If $\mathcal{F}'$ is another $\mathfrak{I}$-good filtration on $M$, then there exists an integer $m$ such that $H_{M,\mathcal{F}'}^{(1)}\geq T^mH_{M,\mathcal{F}}^{(1)}$.
\end{proposition}
\begin{proof}
Since $\ell_{A/\mathfrak{I}}(M_n/M_{n+1})=\ell_A(M_n/M_{n+1})$, the first claim follows from the definition of $H_{M,\mathcal{F}}$ and the observation
\[\sum_{i\leq n}\ell_{A/\mathfrak{I}}(M_i/M_{i+1})=\sum_{i\leq n}\ell_{A}(M_i/M_{i+1})=\ell_{A}(M/M_{n+1}).\]
If $\mathcal{F}=(M_n')$ is another filtration, then there exist an integer $n_2$ such that $M_n'\sub\mathfrak{I}^{n-n_2}M$ for all $n$, whence $M_n'\sub M_{n-(n_2-n_1)}$, and the second claim therefore follows.
\end{proof}
\begin{theorem}\label{filtration good Hilbert-Samuel series expression as d-product}
Let $A$ be a Noetherian ring, $\mathfrak{I}$ an idela of $A$, $M$ a finitely generated $A$-module such that $M/\mathfrak{I} M$ is nonzero and has finite length. Let $\mathcal{F}$ be an $\mathfrak{I}$-good filtration on $M$, then there exists a positive integer $d$ and a uniquely determined element $R$ in $\Z[T,T^{-1}]$ such that
\[H_{M,\mathcal{F}}=(1-T)^{-d}R\And R(1)>0.\]
Moreover, the integers $d$ and $R(1)$ are independent of the filtration $\mathcal{F}$.
\end{theorem}
\begin{proof}
Consider the graded ring $\gr(A)$ and the graded module $\gr(M)$. Since $M_{n_1}=M$ and $\mathfrak{I} M_n=M_{n+1}$ for $n\geq n_0$, $\gr(M)$ is generated by $\bigoplus_{n_1\leq n\leq n_0}\gr_n(M)$, so is finitely generated. Furthermore, if $(x_1,\dots,x_r)$ is a finite generating system of the $A$-module $\mathfrak{I}$, then $\gr(A)$ is generated by $\gr_0(A)$ and the classes of $x_i$ modulo $\mathfrak{I}^2$, so is isomorphic to a graded ring quotient of $B=(A/\mathfrak{I})[X_1,\dots,X_r]$. According to \cref{graded module Poincare series Q_M is polynomial}, we have
\[(1-T)^{r}H_{M,\mathcal{F}}\in\Z[T,T^{-1}].\]
Since $H_{M,\mathcal{F}}\neq 0$, by \cref{series in Z((T)) product with (1-T) polynomial iff} there exist $d\in\N$ and $R\in\Z[T,T^{-1}]$ uniquely determined such that $R(1)>0$ and $H_{M,\mathcal{F}}=(1-T)^{-d}R$.\par
If $\mathcal{F}'$ is another $\mathfrak{I}$-good filtration on $M$, then similarly
\[H_{M,\mathcal{F}'}=(1-T)^{-d'}R'.\]
By \cref{filtration good Hilbert-Samuel series product with (1-T) formula}, there exists an integer $m$ such that
\[(1-T)^{-d'-1}R'\geq T^m(1-T)^{-d-1}R\]
and due to \cref{polynomial product by (1-T) order iff}, this implies $d'\geq d$, or $d'=d$ and $R'(1)\geq R(1)$. Exchange the role of $\mathcal{F}$ and $\mathcal{F}'$, we then get $d=d'$ and $R'(1)=R(1)$.
\end{proof}
\begin{remark}
In the notations of \cref{filtration good Hilbert-Samuel series expression as d-product}, if $R=\sum_{i\in\Z}a_iT^i$ and suppose that $d>0$, then the relation $H_{M,\mathcal{F}}=(1-T)^{-d}R$ implies
\[\ell_{A/\mathfrak{I}}(M_n/M_{n+1})=\sum_{i\in\Z}a_i\binom{n-i+d-1}{d-1}=\sum_{i\leq n}a_i\binom{n-i+d-1}{d-1}\]
Similarly, the relation $H_{M,\mathcal{F}}^{(1)}=(1-T)^{-d-1}R$ implies
\[\ell_{A}(M/M_{n+1})=\sum_{i\in\Z}a_i\binom{n-i+d}{d}=\sum_{i\leq n}a_i\binom{n-i+d}{d}\]
\end{remark}
Let $A$ be a Noetherian ring, $\mathfrak{I}$ an idela of $A$, $M$ a finitely generated $A$-module such that $M/\mathfrak{I} M$ has finite length. If $M\neq\mathfrak{I} M$, then by \cref{filtration good Hilbert-Samuel series expression as d-product} there exists integers $d_\mathfrak{I}(M)\geq 0$ and $e_\mathfrak{I}(M)>0$ such that, for any $\mathfrak{I}$-good filtration $\mathcal{F}$ on $M$,
\[H_{M,\mathcal{F}}=(1-T)^{-d_\mathfrak{I}(M)}R,\quad R(1)=e_\mathfrak{I}(M).\]
If $M=\mathfrak{I} M$, then we may set $d_\mathfrak{I}(M)=-\infty$ and $e_\mathfrak{I}(M)=0$.
\begin{corollary}\label{filtration good index and length formula}
Let $A$ be a Noetherian ring, $\mathfrak{I}$ an idela of $A$, $M$ a finitely generated $A$-module such that $M/\mathfrak{I} M$ is nonzero and has finite length. Let $\mathcal{F}$ be an $\mathfrak{I}$-good filtration on $M$.
\begin{itemize}
\item[(a)] For $d_\mathfrak{I}(M)\leq 0$, it is necessary and sufficient that the sequence $(\mathfrak{I}^nM)$ be stationary, or that the sequence $(M_n)$ is stationary. In this case, we have, for all $n$ large enough,
\[\ell_A(M/M_{n+1})=\ell_A(M/\mathfrak{I}^{n+1}M)=e_\mathfrak{I}(M).\] 
\item[(b)] Suppose that $d_\mathfrak{I}(M)>0$, then
\begin{align}\label{filtration good index and length formula-1}
\ell_{A/\mathfrak{I}}(M_n/M_{n+1})=e_\mathfrak{I}(M)\frac{n^{d_\mathfrak{I}(M)-1}}{(d_\mathfrak{I}(M)-1)!}+\rho_nn^{d_\mathfrak{I}(M)-2}
\end{align}
and
\begin{align}\label{filtration good index and length formula-2}
\ell_{A}(M/M_{n+1})=e_\mathfrak{I}(M)\frac{n^{d_\mathfrak{I}(M)}}{d_\mathfrak{I}(M)!}+\sigma_nn^{d_\mathfrak{I}(M)-1}.
\end{align}
where $\rho_n$ and $\sigma_n$ have limits when $n\to\infty$. 
\end{itemize}
\end{corollary}
\begin{proof}
This follows from formulas (\ref{polynomial product with (1-T)^-r coefficients}) and (\ref{polynomial in Z[T,T^-1] value of Q(1) as limit}), and the definitons above.
\end{proof}
\begin{example}\label{multiplicity of a hypersurface in A^n}
Let $k$ be an algebraically closed field and $F\in k[X_1,\dots,X_r]$ be an irreducible polynomial $F$ in $A$ contained in the maximal ideal $\m=(X_1,\dots,X_r)$, and let $m$ be the largest integer such that $F\in\m^m$. Set $A=(k[X_1,\dots,X_r]/(F))_\m$ and let $\mathfrak{I}=(\m/(F))_\m$ be the maximal ideal of $A$; we want to determine $d_\mathfrak{I}(A)$ and $e_\mathfrak{I}(A)$. For this, we first note that $A/\mathfrak{I}=k$ by the Nullstellensatz and
\[\begin{tikzcd}
0\ar[r]&\mathfrak{I}^{n}/\mathfrak{I}^{n+1}\ar[r]&A/\mathfrak{I}^{n+1}\ar[r]&A/\mathfrak{I}^{n}\ar[r]&0
\end{tikzcd}\]
Therefore we are reduced to compute $\dim_k(A/\mathfrak{I}^n)$. To this end, we first note that
\begin{align*}
A/\mathfrak{I}^n\cong(k[X_1,\dots,X_r]/(F))/(\m^n/(F))=k[X_1,\dots,X_n]/(\m^n+(F)).
\end{align*}
and we can use the following exact sequence to compute $\dim_k(k[X_1,\dots,X_r]/(\m^n+(F)))$:
\[\begin{tikzcd}[column sep=12pt]
0\ar[r]&k[X_1,\dots,X_r]/\m^{n-m}\ar[r,"\psi"]&k[X_1,\dots,X_r]/\m^n\ar[r]&k[X_1,\dots,X_r]/(\m^n+(F))\ar[r]&0
\end{tikzcd}\]
where $\psi$ is the homomorphism given by $\psi(\bar{G})=\widebar{FG}$. Since
\begin{align}\label{multiplicity of a hypersurface in A^n-1}
\dim_k(k[X_1,\dots,X_r]/\m^n)=\sum_{k=0}^{n-1}\binom{k+r-1}{r-1}
\end{align}
it follows that
\begin{align}\label{multiplicity of a hypersurface in A^n-2}
\dim_k(k[X_1,\dots,X_r]/(\m^n+(F)))=\sum_{k=n-m}^{n-1}\binom{k+r-1}{r-1}.
\end{align}
Since $\binom{n+r-1}{r-1}=\frac{n^{r-1}}{(r-1)!}+\cdots$, from (\ref{multiplicity of a hypersurface in A^n-2}) we deduce that
\[\dim_k(k[X_1,\dots,X_r]/(\m^n+(F)))=m\frac{n^{r-1}}{(r-1)!}+\cdots.\]
This shows $d_\mathfrak{I}(A)=r-1$ and $e_\mathfrak{I}(A)=m$. In the language of algebraic geometry, the ring $A$ is the coordinate ring of the variety $V=V(F)$, and the integer $m$ is called the \textbf{multiplicity} of the point $p=(0,\dots,0)$ in $V$.
\end{example}
\begin{example}
Suppose that $\mathfrak{I}$ is contained in the Jacobson radical of $A$. Then from the Nakayama lemma, the sequence $(\mathfrak{I}^nM)$ is stationary if and only if we have $\mathfrak{I}^nM=0$ for $n$ large enough. It then follows from part (a) of \cref{filtration good index and length formula} that we have $d_\mathfrak{I}(M)\leq 0$ if and only if $M$ is of finite length and in this case $e_\mathfrak{I}(M)=\ell_A(M)$.\par
Consider now the special case $A$ is local, $\mathfrak{I}=\m$ is the maximal ideal of $A$, and $M=A$. Then by \cref{Noe local is Artin iff power of maximal ideal}, we see $\m^n=0$ if and only if $A$ is Artinian. By \cref{filtration good index and length formula}, in this case we have $d_\m(A)=0$ and $e_\m(A)=\ell_A(A)$.\par
Furthermore, to provide a concrete example, let consider $A=k[X]/(X^n)$ where $k$ is an algebraically closed field. Then $\m=(X)/(X^n)$ is the maximal ideal of $A$ and $\m^n=0$, so $e_\m(A)=n$.
\end{example}
\begin{proposition}\label{filtration good index and length of generating set}
Let $A$ be a Noetherian ring, $x_1,\dots,x_r$ elements of $A$, $\mathfrak{I}$ the ideal they generate and $M$ a finitely generated $A$-module such that $M/\mathfrak{I}M$ is non-zero and of finite length.
\begin{itemize}
\item[(a)] We have $d_\mathfrak{I}(M)\leq r$, and if $d_\mathfrak{I}(M)=r$ then $e_\mathfrak{I}(M)\leq\ell_A(M/\mathfrak{I}M)$.
\item[(c)] If $(x_1,\dots,x_r)$ is completely secant for $M$, then $d_\mathfrak{I}(M)=r$ and $e_\mathfrak{I}(M)=\ell_A(M/\mathfrak{I}M)$. The converse is also true if $\mathfrak{I}$ belongs to the Jacobson radical of $A$. 
\end{itemize}
\end{proposition}
\begin{proof}
Let $R$ be the ring $(A/\mathfrak{I})[X_1,\dots,X_r]$, and give $N=\bigoplus_n\mathfrak{I}^nM/\mathfrak{I}^{n+1}M$ be the graded $R$-module structure for which $h_{X_i}$ is the multiplication by the class $x_i$ mod $\mathfrak{I}^2$. Then we have
\[P_N=H_{M,\mathfrak{I}}=(1-T)^{-d_\mathfrak{I}(M)}R=(1-T)^{-r}Q_N\]
where $R(1)=e_\mathfrak{I}(M)>0$ and $Q_N(1)=c_N$. By \cref{polynomial product by (1-T) order iff}, we therefore have either $d_\mathfrak{I}(M)<r$ and $c_N=0$, or $d_\mathfrak{I}(M)=r$ and $c_N=e_\mathfrak{I}(M)$. Furthermore, according to \cref{graded module generating set Poincare series inequality}, we have $c_N\leq\ell_A(M/\mathfrak{I}M)$, and the equality holds if and only if canonical homomorphism
\[\varphi:(A/\mathfrak{I})[X_1,\dots,X_r]\otimes_{A/\mathfrak{I}}(M/\mathfrak{I}M)\to \bigoplus_{n}\mathfrak{I}^nM/I^{n+1}M\]
is bijective. This then proves the claim, in view of \cref{Koszul complex completely secant annd regular relation}.
\end{proof}
\begin{proposition}\label{filtration good index and exact sequence}
Let $0\to M'\to M\to M''\to 0$ be an exact sequence of finitely generated modules over an Noetherian ring $A$ and $\mathfrak{I}$ be an ideal of $A$.
\begin{itemize}
\item[(a)] For $M/\mathfrak{I} M$ to have finite length, it is necessary and sufficient that $M'/\mathfrak{I} M'$ and $M''/\mathfrak{I} M''$ both have finite length.
\item[(b)] Suppose that $M/\mathfrak{I} M$ has finite length. Then there are exactly the following three cases:
\begin{itemize}
\item[(\rmnum{1})] $d_\mathfrak{I}(M)=d_\mathfrak{I}(M')>d_\mathfrak{I}(M'')$ and $e_\mathfrak{I}(M)=e_\mathfrak{I}(M')$.
\item[(\rmnum{2})] $d_\mathfrak{I}(M)=d_\mathfrak{I}(M'')>d_\mathfrak{I}(M')$ and $e_\mathfrak{I}(M)=e_\mathfrak{I}(M'')$.
\item[(\rmnum{3})] $d_\mathfrak{I}(M)=d_\mathfrak{I}(M')=d_\mathfrak{I}(M'')$ and $e_\mathfrak{I}(M)=e_\mathfrak{I}(M')+e_\mathfrak{I}(M'')$.
\end{itemize}  
\end{itemize}
\end{proposition}
\begin{proof}
By \cref{associated prime maximal iff finite length} and \cref{supp of module quotient by ideal product}, for the module $M/\mathfrak{I} M$ to have finite length, it is necessary and sufficient that the element in
\[\supp(M/\mathfrak{I} M)=\supp(M)\cap V(\mathfrak{I})\]
is maximal.  Since $\supp(M)=\supp(M')\cap\supp(M'')$, the assertion in (a) follows.\par
Now endow $M$ with an $\mathfrak{I}$-good filtration $\mathcal{F}$ (for example the $\mathfrak{I}$-adic filtration), $M'$ the submodule filtration $\mathcal{F}'$, and $M''$ the quotient filtration $\mathcal{F}''$. By \cref{filtration I-good submodule and quotient}, the filtrations $\mathcal{F}'$ and $\mathcal{F}''$ are $\mathfrak{I}$-good. Then we have for each $n$ an exact sequence of $A$-modules
\[\begin{tikzcd}
0\ar[r]&M_n'/M_{n+1}'\ar[r]&M_n/M_{n+1}\ar[r]&M_n''/M_{n+1}''\ar[r]&0
\end{tikzcd}\]
which implies $H_{M,\mathcal{F}}=H_{M',\mathcal{F}'}+H_{M'',\mathcal{F}''}$, and 
\[(1-T)^{-d_\q(M)}R=(1-T)^{-d_\q(M')}R'+(1-T)^{-d_\q(M'')}R''\]
with $R,R',R''\in\Z[T,T^{-1}]$. Assertion (b) therefore follows.
\end{proof}
\begin{theorem}\label{Noe local ring Hilbert-Samuel degree is dimension}
Let $A$ be a Noetherian local ring, $\mathfrak{I}$ a proper ideal of $A$ and $M$ a finitely generated $A$-module such that $M/\mathfrak{I} M$ is non-zero and of finite length. Then the integer $d_\mathfrak{I}(M)$ equals to the dimension of the $A$-module $M$.
\end{theorem}
\begin{proof}
We may suppose that $M\neq 0$. Let $(x_1,\dots,x_r)$ be a system of parameters for $M$ and $\x$ the ideal they generate. Then by \cref{filtration good index and length of generating set} we have $d_\x(M)\leq r$. Since $\x\sub\mathfrak{I}$, we have $H_{M,\mathfrak{I}}^{(1)}\leq H_{M,I}^{(1)}$ and therefore (\cref{polynomial product by (1-T) order iff})
\[d_\mathfrak{I}(M)\leq d_\x(M)\leq r=\dim_A(M).\]

We now prove the reverse inequality $\dim_A(M)\leq d_\mathfrak{I}(M)$ by induction on $\dim_A(M)$. The claim is clear when $\dim_A(M)=0$. Suppose then we have $\dim_A(M)>0$, and $\dim_A(N)\leq d_\mathfrak{I}(N)$ for any finitely generated $A$-module $N$ such that $\dim_A(N)<\dim_A(M)$. If 
\[0=M_0\subset M_1\subset\cdots\subset M_n=M\]
is a composition sequence of $M$, we have
\[\dim_A(M)=\sup(\dim_A(M_i/M_{i-1})),\quad d_\mathfrak{I}(M)=\sup(d_\mathfrak{I}(M_i/M_{i-1}))\]
by \cref{dimension of module prop} and \ref{filtration good index and exact sequence}. Due to \cref{associated prime of Noe chain of submodule}, we are now reduced to the case that $M$ is of the form $A/\p$, where $\p$ is a prime ideal of $A$ such that $\p\neq\m_A$ (otherwise $\dim(A/\p)=0$). Let $x\in\m_A-\p$, then the homothety $h_x$ with ratio $x$ is injective on $M$, and we have an exact sequence
\[\begin{tikzcd}
0\ar[r]&M\ar[r,"h_x"]&M\ar[r]&M/xM\ar[r]&0
\end{tikzcd}\]
By \cref{Noe local ring secant iff minimal prime of supp}, the element $x$ is secant for $M$, so we have $\dim_A(M/xM)=\dim_A(M)-1$. On the other hand, since $e_\mathfrak{I}(M/xM)>0$, from \cref{filtration good index and exact sequence} we conclude $d_\mathfrak{I}(M/xM)\leq d_\mathfrak{I}(M)-1$. Now by the induction hypothesis,
\[\dim_A(M)=\dim_A(M/xM)+1\leq d_\mathfrak{I}(M/xM)+1\leq d_\mathfrak{I}(M)\]
which finishes the induction process.
\end{proof}
\begin{corollary}\label{Noe ring Hilbert-Samuel degree multiplicity char}
Let $A$ be a Noetherian ring, $\mathfrak{I}$ an idela of $A$, $M$ a finitely generated $A$-module such that $M/\mathfrak{I} M$ is nonzero and has finite length. Then 
\begin{itemize}
\item[(a)] $d_\mathfrak{I}(M)$ is the supremum of the integers $\dim_{A_\m}(M_\m)$, where $\m$ runs through the set $\supp(M)\cap V(\mathfrak{I})$.
\item[(b)] $e_\mathfrak{I}(M)$ is the sum of the integers $e_{\mathfrak{I}_\m}(M_\m)$, where $\m$ runs through the elements of $\supp(M)\cap V(\mathfrak{I})$ such that $\dim_{A_\m}(M_\m)=d_\mathfrak{I}(M)$. 
\end{itemize}
\end{corollary}
\begin{proof}
By \cref{module of finite length equals sum of localization length}, the length of $M/\mathfrak{I}^nM$ is the sum of the $\ell_{A_\m}(M_\m/\mathfrak{I}_\m^nM_\m)$ where $\m$ runs through $\Ass(M/\mathfrak{I}^nM)$. Since $M/\mathfrak{I}^nM$ has finite length, the elements of $\Ass(M/\mathfrak{I}^nM)$ are maximal ideals and we have
\[\Ass(M/\mathfrak{I}^nM)=\supp(M/\mathfrak{I}^nM)=\supp(M)\cap V(\mathfrak{I}^n)=\supp(M)\cap V(\mathfrak{I})\]
(\cref{associated prime maximal iff finite length} and \cref{supp of module quotient by ideal product}), so we have
\[H_{M,\mathfrak{I}}=\sum_{\m}H_{M_\m,\mathfrak{I}_\m}\]
where $\m$ runs through prime ideals in $\supp(M)\cap V(\mathfrak{I})$. The claims then follow from this.
\end{proof}
\begin{corollary}\label{Noe ring Hilbert-Samuel degree and dimension of completion}
Let $A$ be a Noetherian ring, $\mathfrak{I}$ an ideal of $A$, and $M$ a finitely generated $A$-module such that $M/\mathfrak{I} M$ has finite length. Let $\widehat{A}$ and $\widehat{M}$ be the Hausdorff completions of $A$ and $M$ under the $\mathfrak{I}$-adic topology. Then $d_\mathfrak{I}(M)=\dim_{\widehat{A}}(\widehat{M})$. If $\mathfrak{I}$ is contained in the Jacobson radical of $A$ then $d_\mathfrak{I}(M)=\dim_A(M)$.
\end{corollary}
\begin{proof}
By \cref{Noe ring module dimension of completion} and \cref{Noe ring Hilbert-Samuel degree multiplicity char}, we have
\[\dim_{\widehat{A}}(\widehat{M})=\sup_{\m\in V(\mathfrak{I})}\dim_{A_\m}(M_\m)=\sup_{\m\in \supp(M)\cap V(\mathfrak{I})}\dim_{A_\m}(M_\m)=d_\mathfrak{I}(M).\]
If $\mathfrak{I}$ is contained in the Jacobson radical of $A$ then we have $\dim_A(M)=\dim_{\widehat{A}}(\widehat{M})$, whence $d_\mathfrak{I}(M)=\dim_A(M)$.
\end{proof}
\begin{lemma}\label{filtration inclusion Hilbert-Samuel series relation}
Let $A$ be a ring, $M$ an $A$-module, and $(P_n)$, $(Q_n)$ two filtrations on $M$ of submodules with $(Q_n)$ bounded below. Suppose that $Q_n\sub P_n$ and $\ell_A(P_n/Q_n)<+\infty$ for each $n\in\Z$. Then
\begin{equation}\label{filtration inclusion Hilbert-Samuel series relation-1}
\begin{aligned}
\sum_{n\in\Z}\ell_A((P_{n+1}\cap Q_n)/Q_{n+1})T^n&\leq\sum_{n\in\Z}\ell_A(P_{n+1}/Q_{n+1})T^n\\
&\leq(1-T)^{-1}\sum_{n\in\Z}\ell_A((P_{n+1}\cap Q_n)/Q_{n+1})T^n.
\end{aligned}
\end{equation}
\end{lemma}
\begin{proof}
It suffices to prove the following inequalities:
\begin{align}\label{filtration inclusion Hilbert-Samuel series relation-2}
\ell_A((P_{n+1}\cap Q_n)/Q_{n+1})\leq\ell_A(P_{n+1}/Q_{n+1})\leq\sum_{i\leq n}\ell_A((P_{n+1}\cap Q_n)/Q_{n+1}).
\end{align}
The first part is immediate. For the second part, since $(Q_n)$ is bounded below, for $i$ small enough we have $P_{n+1}\cap Q_i=P_{n+1}$, whence
\[\ell_A(P_{n+1}/Q_{n+1})\leq\sum_{i\leq n}\ell_A((P_{n+1}\cap Q_i)/(P_{n+1}\cap Q_{i+1})).\]
On the other hand, the module $(P_{n+1}\cap Q_i)/(P_{n+1}\cap Q_{i+1})$ is isomorphic to $(P_{n+1}\cap Q_i+Q_{i+1})/Q_{i+1}$, which is a submodule of $(P_{i+1}\cap Q_i)/Q_{i+1}$ when $i\leq n$. From these remarks we see (\ref{filtration inclusion Hilbert-Samuel series relation-2}) follows.
\end{proof}
\begin{proposition}\label{filtration and graded endomorphism Hilbert-Samuel series relation}
Let $A$ be a ring, $M$ an $A$-module and $\mathcal{F}$ a bounded below filtration on $M$ consisting of submodules such that $M_n/M_{n+1}$ has finite length for all $n\in\Z$. Let $\phi$ be an endomorphism of $M$ with kernel $M'$ and cokernel $M''$. Endow $M'$ with the submodule filtration $\mathcal{F}'$ and $M''$ the quotient filtration $\mathcal{F}''$. If there exist an integer $\delta$ such that $\phi(M_n)\sub M_{n+\delta}$ and $\gr(\phi)$ is the associated homomorphism on $\gr(M)$, then we have 
\begin{align}
H_{M',\mathcal{F}'}&\leq P_{\ker\gr(\phi)},\label{filtration and graded endomorphism Hilbert-Samuel series relation-1}\\
(1-T^\delta)H_{M,\mathcal{F}}^{(1)}+T^\delta P_{\ker\gr(\phi)}\leq &H_{M'',\mathcal{F}''}^{(1)}\leq(1-T^\delta)H_{M,\mathcal{F}}^{(1)}+T^\delta P_{\ker\gr(\phi)}^{(1)}.\label{filtration and graded endomorphism Hilbert-Samuel series relation-2}
\end{align}
\end{proposition}
\begin{proof}
The sequence $N_n=\phi^{-1}(M_{n+\delta})$ forms a filtration on $M$ and $M_n\sub N_n$ for all $n\in\Z$. By definition, we have $\ker(\gr(\phi)_n)=(N_{n+1}\cap M_n)/M_{n+1}$, and
\begin{align}\label{filtration and graded endomorphism Hilbert-Samuel series relation-3}
P_{\ker\gr(\phi)}=\sum_{n\in\Z}\ell_A((N_{n+1}\cap M_n)/M_{n+1})T^n.
\end{align}
For each $n$, the module $(M'\cap M_n)/(M'\cap M_{n+1})$ is identified with the submodule $(M'\cap M_n+M_{n+1})/M_{n+1}$ of $(N_{n+1}\cap M_n)/M_{n+1}$, so follows from (\ref{filtration and graded endomorphism Hilbert-Samuel series relation-3}) that $H_{\mathcal{M}',\mathcal{F}'}\leq P_{\ker\gr(\phi)}$. According to \cref{filtration inclusion Hilbert-Samuel series relation}, we also have
\begin{align}\label{filtration and graded endomorphism Hilbert-Samuel series relation-4}
P_{\ker\gr(\phi)}\leq\sum_{n\in\Z}\ell_A(N_{n+1}/M_{n+1})T^n\leq P_{\ker\gr(\phi)}^{(1)}.
\end{align}
For each $n\in\Z$, we have an exact sequence
\[\begin{tikzcd}
0\ar[r]&N_{n+1}/M_{n+1}\ar[r]&M/M_{n+1}\ar[r]&M/M_{n+\delta+1}\ar[r]&M''/M''_{n+\delta+1}\ar[r]&0
\end{tikzcd}\]
This then implies, by the additivity of length, that
\[\ell_A(M''/M''_{n+\delta+1})=\ell_A(M/M_{n+\delta+1})-\ell_A(M/M_{n+1})+\ell_A(N_{n+1}/M_{n+1}).\]
Multiply by $T^{n+\delta}$ on both sides and take summation on $n$, we then get
\begin{align}\label{filtration and graded endomorphism Hilbert-Samuel series relation-5}
H_{M'',\mathcal{F}''}^{(1)}=(1-T^\delta)H_{M,\mathcal{F}}^{(1)}+T^\delta\sum_{n\in\Z}\ell_A(N_{n+1}/M_{n+1})T^{n}.
\end{align}
The inequality (\ref{filtration and graded endomorphism Hilbert-Samuel series relation-2}) then follows from (\ref{filtration and graded endomorphism Hilbert-Samuel series relation-4}) and (\ref{filtration and graded endomorphism Hilbert-Samuel series relation-5}).
\end{proof}
\begin{corollary}\label{filtration and graded endomorphism injective iff series equal}
Under the conditions of \cref{filtration and graded endomorphism Hilbert-Samuel series relation}, we have $H_{M'',\mathcal{F}''}\geq(1-T^\delta)H_{M,\mathcal{F}}^{(1)}$, and the equality holds if and only if $\gr(\phi)$ is injective. In this case, we have $M'\sub\bigcap_nM_n$, and the following sequence is exact:
\[\begin{tikzcd}
0\ar[r]&\gr(M)\ar[r,"\gr(\phi)"]&\gr(M)\ar[r]&\gr(M'')\ar[r]&0
\end{tikzcd}\]
\end{corollary}
\begin{proof}
The first assertion follows from (\ref{filtration and graded endomorphism Hilbert-Samuel series relation-1}). Suppose now $\gr(\phi)$ is injective. By \cref{filtration gr(phi) injective surjective iff} we have $\ker\phi\sub\phi^{-1}(M_{n+\delta})=M_n$ for all $n$, whence $M'\sub\bigcap_nM_n$. Now consider the following exact sequence
\[\begin{tikzcd}
0\ar[r]&M/M'\ar[r,"\bar{\phi}"]&M\ar[r]&M/\phi(M)\ar[r]&0
\end{tikzcd}\]
Since $\gr(\phi)$ is injective, we have $\bar{\phi}^{-1}(M_n)=M_{n-\delta}/M'$. Hence the filtration on $M/M'$ deduced by the filtration $\mathcal{F}$ on $M$ is the filtration $(M_{n-\delta}/M')$. The associated graded module is $\gr(M)(-\delta)$ and we have an exact sequence of graded modules
\[\begin{tikzcd}
0\ar[r]&\gr(M)(-\delta)\ar[r]&\gr(M)\ar[r]&\gr(M'')\ar[r]&0
\end{tikzcd}\]
which proves our claim.
\end{proof}
\begin{proposition}\label{Noe ring graded module quotient by sequence series relation}
Let $A$ be a Noetherian ring, $M$ a finitely generated $A$-module, $\mathfrak{I}$ an ideal of $A$ such that $M/\mathfrak{I} M$ has finite length, and $\mathcal{F}$ an $\mathfrak{I}$-good filtration on $M$. Let $(x_1,\dots,x_s)$ be a sequence of elements of $A$, $(\delta_1,\dots,\delta_s)$ a sequence of positive integers such that $x_i\in\mathfrak{I}^{\delta_i}$ for each $i$, and let $\xi_i$ be the image of $x_i$ in $\gr_{\delta_i}(A)=\mathfrak{I}^{\delta_i}/\mathfrak{I}^{\delta_i+1}$.
\begin{itemize}
\item[(a)] Endow the $A$-module $\widebar{M}=M/\sum_ix_iM$ with the $\mathfrak{I}$-good filtration $\widebar{\mathcal{F}}$. Then we have
\begin{align}\label{Noe ring graded module quotient by sequence series relation-1}
H_{\widebar{M},\widebar{\mathcal{F}}}^{(s)}\geq\prod_{i=1}^{s}(1-T^{\delta_i})H_{M,\mathcal{F}}^{(s)}.
\end{align}
\item[(b)] For the equality in (a) holds, it is necessary and sufficient that the sequence $(\xi_1,\dots,\xi_s)$ of elements of $\gr(A)$ is completely secant for $\gr(M)$. In this case, the canonical homomorphism
\[\theta:\gr(M)/\sum_i\xi_i\cdot\gr(M)\to\gr(\widebar{M})\]
is bijective.
\item[(c)] Suppose the condition in (b) hold, and the modules $M_i=M/(x_1M+\cdots+x_iM)$ is Hausdorff in the $\mathfrak{I}$-adic topology for each $i$. Then the sequence $(x_1,\dots,x_s)$ is completely secant for $M$.
\end{itemize}
\end{proposition}
\begin{proof}
When $s=1$, we have $\bigcap_nM_n=\bigcap_n\mathfrak{I}^nM$ and the sequence $\{\xi_1\}$ is completely secant for $\gr(M)$ if and only if the homothety with ratio $\xi_1$ in $\gr(M)$ is injective. The claim then results immediately from \cref{filtration and graded endomorphism injective iff series equal} applied to the homothety $\phi=h_{x_1}$ in $M$.\par
Suppose then $s\geq 2$ and we prove by induction on $s$. By the induction hypothesis applied on $M_1=M/x_1M$ with the quoteint filtration $\mathcal{F}_1$ and the sequence $(x_2,\dots,x_s)$, we have
\begin{align}\label{Noe ring graded module quotient by sequence series relation-2}
H_{\widebar{M},\widebar{F}}^{(s-1)}\geq\prod_{i=2}^{s}(1-T^{\delta_i})H_{M_1,\mathcal{F}_1}^{(s-1)}
\end{align}
and the equality holds if and only if the sequence $(\xi_2,\dots,\xi_s)$ is completely secant for the $\gr(A)$-module $\gr(M_1)$. Since the element $(1-T^{\delta_1})/(1-T)$ is positive, the case $s=1$ already treated and the formula (\ref{Noe ring graded module quotient by sequence series relation-2}) provide the inequalities
\begin{align}\label{Noe ring graded module quotient by sequence series relation-3}
H_{\widebar{M},\widebar{\mathcal{F}}}^{(s)}\geq\prod_{i=2}^{s}(1-T^{\delta_i})H_{M_1,\mathcal{F}_1}^{(s)}\geq\prod_{i=1}^{s}(1-T^{\delta_i})H_{M_1,\mathcal{F}_1}^{(s)}
\end{align}
which proves (a).\par
We can have equality in (\ref{Noe ring graded module quotient by sequence series relation-1}) only if we have simultaneously equality in (\ref{Noe ring graded module quotient by sequence series relation-2}) and equality
\begin{align}\label{Noe ring graded module quotient by sequence series relation-4}
H_{M_1,\mathcal{F}_1}^{(1)}=(1-T^{\delta_1})H_{M,\mathcal{F}}^{(1)}.
\end{align}
This last relation means that $\{\xi_1\}$ is completely secant for $\gr(M)$ and implies that the canonical homomorphism from $\gr(M)/\xi_1\gr(M)$ to $\gr(M_1)$ is a isomorphism. In other words, we have equality in (\ref{Noe ring graded module quotient by sequence series relation-1}) if and only if $\{\xi_1\}$ is completely secant for $\gr(M)$ and $\{\xi_2,\dots,\xi_s\}$ is completely secant for $\gr(M)$. This means $\{\xi_1,\dots,\xi_s\}$ is completely secant for $\gr(M)$. We have thus demonstrated the equivalence of the two conditions of (b).\par
Now suppose that $\{\xi_1,\dots,\xi_s\}$ is completely secant for $\gr(M)$ and $M_i$ is Hausdorff for the $\mathfrak{I}$-adic topology for all $i$. According to the above argument and the induction hypothesis, the sequence $(x_2,\dots,x_s)$ is completely secant for $M_1$. As we have $M_1=M/x_1M$ and that $\{x_1\}$ is completely secant for $M$, the sequence $(x_1,x_2,\dots,x_s)$ is completely secant for $M$.
\end{proof}
\section{Regular local rings}
\subsection{Regularity of local rings}
Let $(A,\m)$ be a Noetherian local ring and $(x_i)_{i\in I}$ a family of elements of $\m$. In view of \cref{local ring module generating set of M/mM}, it amounts to the same thing to suppose that the family $(x_i)_{i\in I}$ generates the ideal $\m$ of $A$, or that the classes of $x_i$ modulo $\m^2$ generate the $\kappa_A$-vector space $\m/\m^2$. In this case, we have $\dim(A)\leq|I|$ by \cref{Noe local ring dimension char defining ideal}, and therefore
\[\dim(A)\leq[\m/\m^2:\kappa_A]\leq|I|\]
for every generating family $(x_i)_{i\in I}$ of the ideal $\m$.
\begin{definition}
A Noetherian local ring $(A,\m)$ is called \textbf{regular} if $\dim(A)=[\m/\m^2,\kappa_A]$. A family of elements of $\m$ whose calsses modulo $\m^2$ form a basis of the $\kappa_A$-space $\m/\m^2$ is called a \textbf{system of parameters} for $A$.
\end{definition}
A system of parameters in a regular local ring $A$ is therefore a finite family $(x_i)_{i\in I}$ generating the ideal $\m$ of $A$ and such that $|I|=\dim(A)$. Conversely, if the maximal ideal $\m$ of a Noetherian local ring $A$ is generated by $d$ elements with $d\leq\dim(A)$, then the ring $A$ is regular.
\begin{example}[\textbf{Examples of regular local rings}]
\mbox{}
\begin{itemize}
\item[(a)] The regular local rings of dimension $0$ (resp. $1$) are the fields (resp. the discrete valuation rings). In fact, the claim is clear for dimension $0$, and a discrete valuation ring is clearly regular (since it is a PID). Conversely, assume that $(A,\m)$ is a regular local ring of dimension one, with $t$ its generator of $\m$. For all $n\geq 0$ we have $\dim_{\kappa_A}(\m^n/\m^{n+1})=1$ because it is generated by $t^n$ (and it cannot be zero since otherwise $\m$ will be the nilradical of $A$, which implies $\dim(A)=0$, contradiction). In particular $\m^n=(t^n)$ and the graded ring $\gr_\m(A)$ is isomorphic to the polynomial ring $\kappa_A[T]$. For $x\in A\setminus\{0\}$, define $v(x)$ to be the largest integer $n$ such that $x\in\m^n$. Then we have $v(xy)=v(x)+v(y)$ for all $x,y\in A\setminus\{0\}$. We extend this to the field of fractions $K$ of $A$. Then it is clear that $A$ is the set of elements of $K$ which have positive valuation. Hence we see that $A$ is a discrete valuation ring. Note that if $(A,\m)$ is a discrete valuation ring, then an element $t$ of $\m$ is a uniformizer if and only if $\{t\}$ is a system of parameters of $A$.
\item[(b)] Let $k$ be a field and $n$ a positive integer. The ring $A=k\llbracket X_1,\dots,X_n\rrbracket$ is Noetherian and local with dimension $n$. Since its maximal ideal $\m$ is generated by $X_1,\dots,X_n$, we see $A$ is regular. Moreover, a sequence $(F_1,\dots,F_n)$ is a system of parameters if and only if their image in $\m/\m^2$ is lineraly independent, or equivalently the Jacobi matrix $(\partial F_i/\partial X_j)$ is nonsingular at the point $0$.\par
More generally, let $A$ be a regular local ring of dimension $r$. The ring $A\llbracket X_1,\dots,X_n\rrbracket$ is then a regular local ring of dimension $r+n$. If $(a_1,\dots,a_r)$ is a system of parameters for $A$, then $(a_1,\dots,a_r,X_1,\dots,X_n)$ is a system of parameters for $A\llbracket X_1,\dots,X_n\rrbracket$.
\item[(c)] Let $A$ be a regular local ring of dimension $r$. The ring $A\{X_1,\dots,X_n\}$ of restricted formal power series is a regular local ring of dimension $n+r$. If $(a_1,\dots,a_r)$ is a system of parameters for $A$, then $(a_1,\dots,a_r,X_1,\dots,X_n)$ is a system of parameters for $A\{X_1,\dots,X_n\}$.
\item[(d)] Let $k$ be a field, $A$ a finitely generated $k$-algebra that is an integral domain and $\m$ a maximal ideal of $A$. Then the Noetherian local ring $A_\m$ is regular if and only if we have $\dim(A)=[\m/\m^2:A/\m]$. In fact, we have $\dim(A_\m)=\dim(A)$ by \cref{algebra finite over field dimension prop} and the $A/\m$-vector space $\m A_\m/(\m A_\m)^2$ is isomorphic to $\m/\m^2$. In particular, if $k$ is algebraically closed, this condition is equivalent to $\dim(A)=[\m/\m^2:k]$ by the Nullstellensatz.
\item[(e)] Let $A$ be a regular local ring. We will see later that the local ring $A_\p$ is regular for any prime ideal $\p$ of $A$. 
\end{itemize}
\end{example}
\begin{proposition}\label{Noe local ring flat extension regular iff}
Let $\rho:A\to B$ be a local homomorphism of Noetherian local rings such that $B$ is flat over $A$. Suppose that $\m_B=\m_A^e$, then $\dim(B)=\dim(A)$ and $B$ is regular if and only if $A$ is regular.
\end{proposition}
\begin{proof}
The first assertion follows from \cref{Noe local ring extension dim of tensor}. Since $B$ is flat over $A$, we can identify $\m_B/\m_B^2$ with $B\otimes_A(\m_A/\m_A^2)$ or with $\kappa_B\otimes_{\kappa_A}(\m_A/\m_A^2)$. We then have
\[[\m_B/\m_B^2:\kappa_A]=[\m_A/\m_A^2:\kappa_A]\]
so the second claim follows.
\end{proof}
\begin{corollary}\label{Noe local ring regular iff completion is}
A Noetherian local ring $A$ is regular if and only if its completion $\widehat{A}$ is.
\end{corollary}
\begin{proof}
The completion $\widehat{A}$ is flat over $A$ and $\hat{\m}$ is generated by $\m$ in $\widehat{A}$, so we can apply \cref{Noe local ring flat extension regular iff}.
\end{proof}
\begin{theorem}\label{Noe local ring regular iff m_A generated by completely secant}
Let $A$ be a Noetherian local ring. The following conditions are equivalent:
\begin{itemize}
\item[(\rmnum{1})] $A$ is regular.
\item[(\rmnum{2})] The ideal $\m_A$ is generated by a subset of $\m_A$ secant for $A$.
\item[(\rmnum{3})] The ideal $\m_A$ is generated by a subset of $\m_A$ completely secant for $A$.
\item[(\rmnum{4})] The homomorphism $\gamma:\bm{S}_A(\m_A/\m_A^2)\to\gr(A)$ is bijective.
\item[(\rmnum{5})] There exists a positive integer $r$ such that $H_{A,\m_A}=(1-T)^{-r}$.
\item[(\rmnum{6})] We have $H_{A,\m_A}=(1-T)^{-d}$ where $d=\dim(A)$. 
\end{itemize}
If these conditions are fulfilled, every system of parameters for $A$ is a completely secant sequence for $A$.
\end{theorem}
\begin{proof}
Since every secant subset has cardinality smaller than $\dim(A)$, we see (\rmnum{3})$\Rightarrow$(\rmnum{2})$\Rightarrow$(\rmnum{1}). To see (\rmnum{4})$\Rightarrow$(\rmnum{3}), let $(x_1,\dots,x_r)$ be a sequence of elements of $\m_A$ whose classes modulo $\m_A^2$ form a basis $(\xi_1,\dots,\xi_r)$ for $\m_A/\m_A^2$. If property (\rmnum{4}) is satisfied, $\gr(A)$ is then the polynomial algebra $\kappa_A[\xi_1,\dots,\xi_r]$ and the sequence $(x_1,\dots,x_r)$ is completely secant by \cref{Koszul complex completely secant annd regular relation}. This also proves the last assertion of the theorem.\par
Let $r=[\m_A/\m_A^2:\kappa_A]$. Then the Poincar\'e series of the graded module $S=\bm{S}_A(\m_A/\m_A^2)$ over $\kappa_A$ is equal to
\[P_S=\sum_{n\in\N}[S_n:\kappa_A]T^n=(1-T)^{-r}.\]
Suppose that the homomorphism $\gamma$ is not bijective. As $\gamma$ is surjective, there exists a homogeneous element $u$ of $S$ of degree $d>0$, canceled by $\gamma$. Then
\[H_{A,\m_A}=P_S-P_{\ker\gamma}\leq P_S-P_{uS}=(1-T^d)(1-T)^{-r}=(1+T+\cdots+T^{d-1})(1-T)^{-(r-1)}.\]
By \cref{Noe local ring Hilbert-Samuel degree is dimension} and \cref{polynomial product by (1-T) order iff}, we get $\dim(A)<r$, so $A$ is not regular.\par
Finally, let us prove the equivalence of conditions (\rmnum{4}) and (\rmnum{6}). If we have $H_{A,\m_A}=(1-T)^{-r}$ for some $r$, then by \cref{Noe local ring Hilbert-Samuel degree is dimension} we must have $r=\dim(A)$. Also, $[\m_A/\m_A^2:\kappa_A]=r$ because $(1-T)^{-r}=1+rT+\cdots$. Therefore our claim follows.
\end{proof}
\begin{corollary}\label{regular local ring integrally closed}
A regular local ring is integrally closed, and in particular an integral domain.
\end{corollary}
\begin{proof}
If $A$ is a regular local ring, then $\gr(A)$ is isomorphic to a polynomial algebra over a field, so it is integrally closed. Since $A$ is a Zariski ring, by \cref{filtration gr(A) completely integrally closed then} we see $A$ is integrally closed.
\end{proof}
\begin{corollary}\label{Noe local ring regular homomorphism bijection iff residue field vector space bijection}
Let $\rho:A\to B$ be a local homomorphism of Noetherian local rings. Suppose that $A$ is complete and $B$ is regular. Then for $\rho$ to be bijective, it is necessary and sufficient that it induces bijections of $\kappa_A$ on $\kappa_B$ and of $\m_A/\m_A^2$ to $\m_B/\m_B^2$. 
\end{corollary}
\begin{proof}
These conditions are clearly necessary. Suppose now $\rho$ induces bijections of $\kappa_A$ on $\kappa_B$ and of $\m_A/\m_A^2$ to $\m_B/\m_B^2$. Since the ring $\gr(A)$ is generated by $\gr_0(A)$ and $\gr_1(A)$, it follows that $\gr(\rho)$ is bijective. Therefore $\rho$ is bijective by \cref{filtration gr(phi) injective surjective iff}. 
\end{proof}
\begin{corollary}\label{Noe local ring residue regular iff power series}
Let $k$ be a ring and $A$ a Noetherian local $k$-algebra with residue field equal to $k$. Then $A$ is regular if and only if the completion $\widehat{A}$ is isomorphic to a formal series ring $k\llbracket X_1,\dots,X_n\rrbracket$.
\end{corollary}
\begin{proof}
This follows from \cref{Noe local ring regular iff m_A generated by completely secant} and \cref{filtered complete ring generator of A and gr(A)}.
\end{proof}
\subsection{Quotients of regular local rings}
\begin{proposition}\label{Noe local ring quotient by sequence regular iff}
Let $A$ be a Noetherian local ring, $\bm{x}=(x_1,\dots,x_r)$ a sequence of elements of $\m_A$ and $\x$ the ideal it generates. The following conditions are equivalent.
\begin{itemize}
\item[(\rmnum{1})] $A$ is regular and $\bm{x}$ can be extended to a system of parameters for $A$. 
\item[(\rmnum{2})] $A/\x$ is regular and $\bm{x}$ is secant for $A$.
\item[(\rmnum{3})] $A/\x$ is regular and $\bm{x}$ is completely secant for $A$.
\end{itemize}
Moreover, if these conditions are satisfied, then $\x$ is a prime ideal of $A$.
\end{proposition}
\begin{proof}
Clearly (\rmnum{3}) implies (\rmnum{2}). Suppose now that $\bm{x}$ is secant for $A$ and the Noetherian local ring $A/\x$ is regular. Let $(x_{r+1},\dots,x_d)$ be a sequence of elements of $A$ whose class modulo $I$ form a system of parameters for $A/\x$. Then the sequence $(x_1,\dots,x_d)$ generates the maixmal ideal $\m_A$ of $A$, and we have
\[\dim(A)=r+\dim(A/\x)=r+(d-r)=d.\]
Therefore, $A$ is regular and $(x_1,\dots,x_d)$ is a system of parameters for $A$.\par
Finally, assume that (\rmnum{1}) is satisfied. Then $A$ is regular and $\bm{x}$ is completely secant for $A$, hence secant for $A$. Then we have
\[\dim(A/\x)=\dim(A)-r\]
and the classes $x_1,\dots,x_r$ modulo $\m_A^2$ is lineraly independent over $\kappa_A$, so
\[[\m_A/(\m_A^2+\x):\kappa_A]=[\m_A/\m_A^2:\kappa_A]-r.\]
Therefore we see $A/\x$ is regular, and in particular $\x$ is prime since $A/\x$ is integral.
\end{proof}
\begin{corollary}\label{Noe local ring quotient by element regular iff}
Let $A$ be a Noetherian local ring and $x$ and element of $\m_A$. The following conditions are equivalent.
\begin{itemize}
\item[(\rmnum{1})] $A$ is regular, and $x$ does not belong to $\m_A$.
\item[(\rmnum{2})] $A/xA$ is regular and $\dim(A/xA)=\dim(A)-1$.
\item[(\rmnum{3})] $A/xA$ is regular and $x$ is not a divisor of $0$ in $A$. 
\end{itemize}
\end{corollary}
\begin{proof}
This follows by applying \cref{Noe local ring quotient by sequence regular iff} on $\x=xA$.
\end{proof}
\begin{corollary}\label{regular local ring quotient by ideal regular iff}
Let $A$ be a regular local ring and $\mathfrak{I}$ an ideal of $A$. Then $A/\mathfrak{I}$ is regular if and only if $\mathfrak{I}$ is generated by a subset of a system of parameters for $A$. 
\end{corollary}
\begin{proof}
The condition is sufficient by \cref{Noe local ring quotient by sequence regular iff}. Suppose now $A/\mathfrak{I}$ is regular, and $\bm{x}=(x_1,\dots,x_r)$ a sequence of elements of $\mathfrak{I}$ whose classes module $\m_A^2$ form a basis for $(\mathfrak{I}+\m_A^2)/\m_A^2$ over $\kappa_A$. Let $\x$ be the ideal generated by $\bm{x}$, so we have $\x\sub\mathfrak{I}$ and $\bm{x}$ is part of a system of parameters for $A$, so the local Noetherian ring $A/\x$ is regular. Moreover, the vector spaces $\m_A/(\mathfrak{I}+\m_A^2)$ and $\m_A/(\x+\m_A^2)$ have the same dimension over $\kappa_A$. Consequently, the regular local rings $A/\mathfrak{I}$ and $A/\x$ have the same dimension. As the ideals $\mathfrak{I}$ and $\x$ are prime and we have $\x\sub\mathfrak{I}$, we finally have $\mathfrak{I}=\x$.
\end{proof}
\begin{example}
Let $k$ be a field. Then the ring $A=k\llbracket X_1,\dots,X_n\rrbracket$ of power series is local. Let $\mathfrak{I}$ be a proper ideal of $A$. By \cref{regular local ring quotient by ideal regular iff}, for $A/\mathfrak{I}$ to be regular, it is necessary and sufficient that we can find a positive integer $r$ and elements $F_1,\dots,F_r$ of $A$, generating $\mathfrak{I}$, and such that the matrix $(\partial F_i/\partial X_j(0))$ is of rank $r$ (Jacobian criterion). We then have $\dim(A/\mathfrak{I})=n-r$.
\end{example}
\subsection{Eisenstein polynomials}
Let $A$ be a ring, $\p$ a prime ideal of $A$, and $P$ a polynomial in $A[T]$. We say $P$ is an \textbf{Eisenstein polynomial} for $\p$ if it satisfies the following conditions:
\begin{enumerate}[leftmargin=30pt]
\item[(a)] $P$ is monic with degree $d\geq 2$;
\item[(b)] $P(T)\equiv T^d$ mod $\p A[T]$;
\item[(c)] $P(0)\notin\p^2$. 
\end{enumerate}
In other words, an Eisenstein polynomial for $\p$ is of the form $P(T)=T^d+\sum_{i=0}^{d-1}a_iT^i$, with $d\geq 1$ and $a_1,\dots,a_{d-1}$ belong to $\p$ with $a_0\in\p\setminus\p^2$.\par
We say $P$ is an Eisenstein polynomial for $\p A_\p$ if the canonical image of $P$ in the polynomial ring $A_\p[T]$ is an Eisenstein polynomial for the ideal $\p A_\p$. This means $P$ is an Eisenstein polynomial for $\p$ and that it also satisfies the following condition, stronger than (c):
\begin{enumerate}[leftmargin=30pt]
\item[(c')] any element $a$ in $A$ such that $aP(0)\in\p^2$ belongs to $\p$. 
\end{enumerate}
\begin{proposition}\label{Eisenstein polynomial irreducible}
Let $A$ be a ring, $\p$ a prime ideal of $A$ and $P\in A[T]$ an Eisenstein polynomial for $\p$.
\begin{itemize}
\item[(a)] $P$ has no decomposition of the form $P=P_1P_2$, where $P_1,P_2$ are two monic polynomials in $A[T]$ unequal to $1$.
\item[(b)] Suppose that $A$ is integrally closed with $K$ its fraction field. Then $P$ is irreducible in $K[T]$.
\end{itemize}
\end{proposition}
\begin{proof}
Let $\pi$ be the canonical map from $A$ to the residue field $\kappa(\p)$ of $\p$ and $\tilde{\pi}:A[T]\to\kappa(\p)[T]$ be the extension of $\pi$ such that $\tilde{\pi}(T)=T$. Suppose that $P=P_1P_2$ where $P_1,P_2$ are two monic polynomials unequal to $1$. Then we have $\tilde{\pi}(P)=T^d=\tilde{\pi}(P_1)\tilde{\pi}(P_2)$ in $\kappa(\p)[T]$, with $d$ the degree of $P$. If $d_i$ is the degree of $P_i$, we get $\tilde{\pi}(P_i)=T^{d_i}$, which means $P_i(T)\equiv T^{d_i}$ mod $\p A[T]$, in particular $P_i(0)\in\p$. BUt then $P(0)=P_1(0)P_2(0)\in\p^2$, which is a contradiction. The assertion (b) now follows from (a).
\end{proof}
\begin{proposition}\label{Noe local ring polynomial ring quotient by Eisenstein poly prop}
Let $A$ be a Noetherian local ring and $P_1,\dots,P_r$ be monic polynomials in $A[T]$ such that $P_i(T)\equiv T^{d_i}$ mod $\m_AA[T]$ and $\deg(P_i)\geq 2$ for each $i$. Let $\q$ be the ideal in $A[T_1,\dots,T_r]$ generated by $P_1(T_1),\dots,P_r(T_r)$ and $B$ the $A$-algebra $A[T_1,\dots,T_r]/\q$. For each $i$, we denote by $d_i$ the degree of $P_i$, $\xi_i$ the class of $T_i$ mod $\q$, and $\gamma_i$ the class of $c_i=P_i(0)$ mod $\m_A^2$. 
\begin{itemize}
\item[(a)] $B$ is a Noetherian local ring with maximal ideal
\[\m_B=B\m_A+\sum_{i=1}^{r}B\xi_i.\]
We have $\dim(A)=\dim(B)$ and $[\kappa_B:\kappa_A]=1$. Moreover, the monomials $\xi_1^{\alpha_1}\cdots\xi_r^{\alpha_r}$, with $0\leq\alpha_i<d$ for $1\leq i\leq r$, form a basis for the $A$-module $B$.
\item[(b)] Let $\lambda$ be the canonical homomorphism from $\m_A/\m_A^2$ to $\m_B/\m_B^2$. Then the kernel of $\lambda$ is the $\kappa_A$-vector space generated by $\gamma_1,\dots,\gamma_r$ and the classes $\xi_1,\dots,\xi_r$ form a basis over $\kappa_A$ for the cokernel of $\lambda$.
\item[(c)] For $B$ to be regular, it is necessary and sufficient that $A$ is regular and the $\gamma_1,\dots,\gamma_r$ is linearly independent in the $\kappa_A$-vector space $\m_A/\m_A^2$. 
\end{itemize}
\end{proposition}
\begin{proof}
The $A$-algebra $B$ is isomomorphic to the tensor product $B_1\otimes_A\otimes\cdots\otimes_AB_r$ where $B_i=A[T]/(P_i)$ for each $i$. It then follows that the monomials $\xi_1^{\alpha_1}\cdots\xi_r^{\alpha_r}$, with $0\leq\alpha_i<d$ for $1\leq i\leq r$, form a basis for the $A$-module $B$. In particular, $B$ is integral over $A$, so $A$ and $B$ has the same dimension.\par
By \cref{Noe semilocal ring finite extension is Noe semilocal}, $B$ is Noetherian, and any maximal ideal of $B$ contains $B\m_A$. COnversely, from the hypothesis on $P_1,\dots,P_r$ and the relations $P_i(\xi_i)=0$, we get $\xi_i^{d_i}\in B\m_A$ for each $i$. Therefore any maximal ideal of $B$ containing $\xi_1,\dots,\xi_r$ must contains the ideal $\n=B\m_A+B\xi_1+\cdots+B\xi_r$. But we have $\m_A=A\cap\n$ and $B=A+\n$, so $B/\n$ is isomomorphic to $A/\m_A$ and therefore is maixmal in $B$. This shows $B$ is local with maximal ideal $B\m_A+\sum_{i=1}^{r}B\xi_i$, and we have $[\kappa_B:\kappa_A]=1$. This proves (a).\par
Let $\varphi$ be the canonical homomorphism of $(A/\m_A^2)[T_1,\dots,T_r]$ to $B/\m_B^2$. By part (a), the kernel of $\varphi$ is the ideal generated by the classes $\bar{P}_i(T_i)$ of $P_i(T_i)$ modulo $\m_A^2A[T_1,\dots,T_r]$, the monomials $T_iT_j$, and $xT_i$ with $1\leq i,j\leq r$ and $x\in\m_A/\m_A^2$. By hypothesis, we have $P_i(T)\equiv T^{d_i}$ mod $\m_AA[T]$, so we can replace $\bar{P}_i(T_i)$ with $\gamma_i$ in the description above. As a result, if we set $\r=\m_A^2+\sum_{i=1}^{r}Ac_i$, the ring $B/\m_B^2$ is then isomomorphic to $(A/\r)[T_1,\dots,T_r]$ quotient the ideal generated by the monomials $T_iT_j$ and $xT_i$, with $1\leq i,j\leq r$ and $x\in\m_A/\r$. If we denote by $\tau_i$ the class of $\xi_i$ modulo $\m_B^2$, this means
\begin{align}\label{Noe local ring polynomial ring quotient by Eisenstein poly prop-1}
B/\m_B^2=(A/\r)\oplus\kappa_A\tau_1\oplus\cdots\oplus\kappa_A\tau_r.
\end{align}
and
\begin{align}\label{Noe local ring polynomial ring quotient by Eisenstein poly prop-2}
\m_B/\m_B^2=(\m_A/\r)\oplus\kappa_A\tau_1\oplus\cdots\oplus\kappa_A\tau_r.
\end{align}
Assertion (b) now follows from these.\par
According to formula (\ref{Noe local ring polynomial ring quotient by Eisenstein poly prop-2}) and the relation $[\kappa_B:\kappa_A]=1$, we get
\begin{align}\label{Noe local ring polynomial ring quotient by Eisenstein poly prop-3}
[\m_B/\m_B^2:\kappa_B]=[\m_A/\m_A^2:\kappa_A]+(r-[\r/\m_A^2:\kappa_A]).
\end{align}
But the $\kappa_A$-vector space $\r/\m_A^2$ is generated by $\gamma_1,\dots,\gamma_r$, and we have 
\begin{align}\label{Noe local ring polynomial ring quotient by Eisenstein poly prop-4}
\dim(B)=\dim(A)\leq [\m_A/\m_A^2:\kappa_A].
\end{align}
Combine formula (\ref{Noe local ring polynomial ring quotient by Eisenstein poly prop-3}) and formula (\ref{Noe local ring polynomial ring quotient by Eisenstein poly prop-4}), we see assertion (c) follows.
\end{proof}
\begin{corollary}\label{Noe local ring polynomial ring quotient by one Eisenstein poly}
Let $A$ be a regular Noetherian local ring and $P\in A[T]$ an Eisenstein polynomial for $\m_A$. Then $B=A[T]/(P)$ is a Noetherian local regular ring, with the same dimension as $A$, and $[\kappa_B:\kappa_A]=1$. In fact, we have $\m_B=B\m_A+B\xi$, where $\xi$ is the class of $T$ modulo $(P)$.
\end{corollary}
\begin{proof}
By definition we have $\deg(P)\geq 2$ and $P(T)\equiv T^d$ mod $\m_A$, so this is a special case of \cref{Noe local ring polynomial ring quotient by Eisenstein poly prop} with $r=1$.
\end{proof}
\begin{proposition}\label{Eisenstein polynomial field extension local ring prop}
Let $A$ be an integral domain, with fraction field $K$, and $L$ an algebraic extension of $K$. Let $B$ be the integral closure of $A$ in $L$ and $\p$ a prime ideal of $A$. Suppose that the local ring $A_\p$ is regular and Noetherian; let $\xi$ be a primitive element for $L$ over $K$ (i.e., $L=K(\xi)$) and suppose that the minimal polynomial of $\xi$ over $K$ is an Eisenstein polynomial $P\in A[T]$ for $\p A_\p$.
\begin{itemize}
\item[(a)] There exists in $B$ a unique prime ideal $\mathfrak{P}$ lying over $\p$.
\item[(b)] The local ring $B_{\mathfrak{P}}$ is Noetherian and regular, with the same dimension as $A_\p$. Moreover, $B_{\mathfrak{P}}=A_\p[\xi]$.
\item[(d)] The canonical homomorphism from $A/\p$ to $B/\mathfrak{P}$ induces an isomorphism on the fraction fields.
\end{itemize}
\end{proposition}
\begin{proof}
Set $C=A_\p[\xi]$ and let $d$ be the degree of $P$. By \cref{Noe local ring polynomial ring quotient by Eisenstein poly prop} applied to the ring $A_\p$, the Eisenstein polynomial $P$ is irreducible in $K[T]$ and $(1,\xi,\dots,\xi^{d-1})$ is a basis for $L$ over $K$, and for $C$ over $A_\p$. Since $P$ is monic, the kernel of the canonical homomorphism from $A_\p[T]$ to $C$ is equal to $(P)$; in other words, $C=A_\p[T]/(P)$, so we can apply \cref{Noe local ring polynomial ring quotient by one Eisenstein poly}: $C$ is a regular Noetherian local with dimension equal to that of $A_\p$, the maximal ideal $\m_C$ of $C$ is generated by $\p\cup\{\xi\}$ and the residue field $\kappa_C$ is a trivial extension of $\kappa(\p)$. To prove the proposition, it suffices to find a unique prime ideal $\mathfrak{P}$ of $B$ lying over $\p$, and such that $C=B_{\mathfrak{P}}$.\par
Let $S=A-\p$, then the integral closure of $A_\p$ in $L$ is equal to $S^{-1}B$. But $t$ is integral over $A_\p$, and $C=A_\p[\xi]$ is a regular local ring, hence integrally closed (\cref{regular local ring integrally closed}). It then follows that $C=S^{-1}B$, so $S^{-1}B$ is local. By \cref{integral ring lying over prime exist} and \cref{integral ring maximal ideal iff contraction is}, there then exists a unique prime ideal of $S^{-1}B$ lying over $\p A_\p$, which corresponds to a unique prime ideal $\mathfrak{P}$ of $B$ lying over $\p$, and we have $B_{\mathfrak{P}}=S^{-1}B=C$ (\cref{integral lift of prime lemma}). 
\end{proof}
\begin{corollary}\label{Eisenstein polynomial field extension DVR prop}
With the hypothesis in \cref{Eisenstein polynomial field extension local ring prop}, suppose that $A_\p$ is a DVR. Then $B_{\mathfrak{P}}$ is a DVR, $\xi$ is a uniformizer for $B_{\mathfrak{P}}$, and we have
\begin{align}\label{Eisenstein polynomial field extension DVR prop-1}
f(B_{\mathfrak{P}}/A_\p)=1,\quad e(B_{\mathfrak{P}}/A_\p)=[L:K].
\end{align}
\end{corollary}
\begin{proof}
In fact, DVRs are regular Noetherian local rings with dimension $1$; set $d=[L:K]$ and $P(T)=T^d+\sum_{i=0}^{d-1}a_iT^i$. Let $v_{\mathfrak{P}}$ be the valuation on $B_{\mathfrak{P}}$. Then since $P$ is an Eisenstein polynomial, $a_0\in\m_{A_\p}\setminus\m_{A_\p}^2$ and $a_i\in\m_{A_\p}$ for each $i$, whence
\[d=v_{\mathfrak{P}}(\xi^d)=v_{\mathfrak{P}}(-\sum_{i=0}^{d-1}a_i\xi^i)=\min_i\{v_{\mathfrak{P}}(a_i)+i\}=v_{\mathfrak{P}}(a_0).\]
This shows $d=e(B_{\mathfrak{P}}/A_\p)$. Since $[\kappa_B:\kappa_A]=1$, we also have $f(B_{\mathfrak{P}}/A_\p)=1$.
\end{proof}
\begin{example}[\textbf{Examples of extensions by Eisenstein polynomials}]
\mbox{}
\begin{itemize}
\item[(a)] Let $A=\Z$ and $L=\Q(p^{1/d})$, where $p$ is a prime number and $d\geq 2$. Let $B$ be the integral closure of $\Z$ in $L$. Then $P(T)=T^d-p$ is an Eisenstein polynomial in $\Z[T]$ for $p\Z_{(p)}$, and there exists a unique prime ideal $\mathfrak{P}$ of $B$ lying over $p\Z$. Since $Z_{(p)}$ is a DVR, there exists on $Q(p^{1/d})$ a unique discrete valuation $v_{\mathfrak{P}}$ such that $v(p)>0$. We have $[L:K]=v_{\mathfrak{P}}(p)=d$, and $B/\mathfrak{P}$ is a field with $p$ elements. The valuation ring $B_{\mathfrak{P}}$ of $v_{\mathfrak{P}}$ is equal to $\Z_{(p)}[p^{1/d}]$. 
\item[(b)] Let $A=\Z$ and $L=\Q(\zeta)$ where $\zeta=e^{2\pi i/p^f}$, with $p$ a prime and $f\geq 1$. Let $B$ be the integral closure of $\Z$ in $L$ and $P$ the polynomial in $\Z[T]$ defined by
\[P(T-1)=\frac{T^{p^f}-1}{T^{p^{f-1}}-1}.\]
Set $d=p^f-p^{f-1}$. We have $P(\zeta-1)=0$, $P(0)=p$, and
\[P(T-1)\equiv(T-1)^d\mod p\Z[T],\]
so $P(T)\equiv T^d$. Therefore $P$ is an Eisenstein polynomial for $p\Z_{(p)}$. There exists a unique prime ideal $\mathfrak{P}$ of $B$ lying over $\p\Z$, and we have $B_{\mathfrak{P}}=\Z_{(p)}[\zeta]$. Moreover, $\zeta-1$ is a uniformizer for $B_{\mathfrak{P}}$ and
\[[L:K]=d=p^f-p^{f-1}.\]
If $v_{\mathfrak{P}}$ is the unique normalized valuation on $\Q(\zeta)$ such that $v_{\mathfrak{P}}(p)>0$, we have $v_{\mathfrak{P}}(p)=d$. Moreover, the field $B/\mathfrak{P}$ has $p$ elements. We can prove that $B$ is equal to $\Z[\zeta]$.
\end{itemize}
\end{example}
\section{Dimension and graded rings}
Let $R$ be a graded ring of type $\Z$ and $(R_n)_{n\in\Z}$ its graduation; we assume that $R_n=0$ for $n<0$. For each $n\in\Z$, let $R_{\geq n}=\bigoplus_{i\geq n}R_i$. We have $R=R_{\geq 0}$ and each $R_{\geq n}$ is an homogeneous ideal of $R$. Denote by $S$ the multiplicative subset $1+R_{\geq n}$ consists of elements of $R$ with degree zero term equals to $1$, and consider the localization $S^{-1}R$. By identifying $R$ with a subring of $\widehat{R}=\prod_nR_n$, we see $S$ is invertible in the completion $\widehat{R}$, whence $S^{-1}R$ can be identified with a subring of $\widehat{R}$ containing $R$. For $s\in S$ and $a\in R_{\geq n}$, the element $s^{-1}a-a$ of $\widehat{R}$ belongs to $\prod_{i\geq n+1}R_i$; therefore we have $S^{-1}R_{\geq n}=(S^{-1}R)\cap\prod_{i\geq n}R_i$. From these we deduce the following results:
\begin{proposition}\label{dimension graded ring S^-1R prop}
Let $R=\bigoplus_{n\in\N}R_n$ be a graded ring and $S$ be defined as above.
\begin{itemize}
\item[(a)] The ideals $S^{-1}R_{\geq n}$ form an exhaustive and separated filtration of $S^{-1}R$.
\item[(b)] The homomorphism $i_S:R\to S^{-1}R$ induces an isomorphism $u_n$ of $R_n$ to $S^{-1}R_{\geq n}/S^{-1}R_{\geq n+1}$ for each $n$; the $u_n$ are homogeneous components of an isomorphism of $R$ onto the graded ring associated with $S^{-1}R$, filtered by $S^{-1}R_{\geq n}$.
\end{itemize}
\end{proposition}
\begin{remark}\label{dimension graded ring S^-1R is local if R_0 local}
Rn element $a/s$ of $S^{-1}R$ with $a\in R$, $s\in S$, is invertible if and only if the degree $0$ component of $a$ is invertible in $R_0$. Therefore, if $R_0$ is a local ring, so is $S^{-1}R$ and the canonical injection $R_0\to S^{-1}R$ induces an isomorphism on residue fields.
\end{remark}
\begin{remark}
Suppose that $R$ is generated by $R_0$ and $R_1$; then for each $n$ we have $R_{n+1}=R_1\cdot R_{n}$, so $R_{\geq n+1}=R_1\cdot R_{\geq n}$ and $S^{-1}R_{\geq n+1}=R_1\cdot S^{-1}R_{\geq n}$. As a result, the filtration $(S^{-1}R_{\geq n})$ of $S^{-1}R$ is $S^{-1}R_{\geq 1}$-adic.
\end{remark}
\begin{example}
\mbox{}
\begin{itemize}
\item[(a)] Let $\p$ be a homogeneous prime ideal of $\C[X_0,\dots,X_n]$ different to the ideal generated by the $X_i$'s (denoted by $\m$). Let $X\sub\mathbb{CP}^n$ be the projective variety defined by $\p$ and $C$ the algebraic variety in $\C^{n+1}$ defined by $\p$ (so that $C$ the \textbf{cone} of $X$). Let $R=\C[X_0,\dots,X_n]/\p$ be the coordinate ring of $C$ and consider the localization $S^{-1}R$. We claim that $S^{-1}R$ is the local ring at the origin, that is, $S^{-1}R=R_{\m_0}$. This follows from the fact that a prime ideal $\q$ is disjoint from $S$ if and only if every element of $\q$ has zero constant term, or equivalently is contained in the maximal ideal $\m$. (Therefore $S$ and $R\setminus\m/\p$ have the same saturation.) 
\item[(b)] Let $A$ be a local ring and $\a$ a proper ideal of $A$. Then $R=\bigoplus_n\a^n/\a^{n+1}$ is a graded ring and $R_0=A/\a$ is local; it is generated by $R_0$ and $R_1$. The ring $S^{-1}R$ is then local and the filtration $(S^{-1}R_{\geq 1})$ is the $S^{-1}R_{\geq 1}$-adic filtration. Note that in general the rings $A$ and $S^{-1}R$ are not isomorphic.
\end{itemize}
\end{example}
\subsection{Chains of homogeneous ideals}
In this part, we will use $\mathrm{dimgr}(R)$ to denote the supremum of the length of chains homogeneous prime ideals in $R$; similarly, if $\p$ is a homogeneous ideal of $R$, we use $\mathrm{htgr}(\p)$ to denote the supremum of the length of chains homogeneous prime ideals in $R$ for which $\p$ is the largest element. Note that if $\p$ is a homogeneous prime ideal of $R$, we have $\p\cap S=\emp$; in fact, if $\p$ contains an element with degree zero componenet $1$, then $1\in\p$. The map $\p\mapsto S^{-1}\p$ from the set of homogeneous prime ideals of $R$ to the set of prime ideals of $S^{-1}R$ is then injective and increasing. Concequently, we have
\begin{align}\label{dimension homogeneous ideal dimgr prop}
\mathrm{dimgr}(R)\leq\dim(S^{-1}R)\leq\dim(R),\quad \mathrm{htgr}(\p)\leq\height(S^{-1}\p)=\height(\p).
\end{align}
\begin{example}\label{polynomial ring prime ideal contration prop}
Let $A$ be an arbitrary ring and consider the polynomial ring $A[X]$. For a homogeneous prime ideal $\p$ of $A[X]$, we set $\p_0=\p\cap A$ to be the constant part of elements in $\p$. We distinguish two cases:
\begin{itemize}
\item[(a)] If $X\in\p$, then it is easy to see $\p=\p_0+X\cdot A[X]$ because any polynomial $f\in\p$ can be written as $f=f(0)+X\cdot g$. 
\item[(b)] If $X\notin\p$, then $\p=\p_0\cdot A[X]$. In fact, let $f\in\p$ and write
\[f(X)=a_0+a_1X+\cdots+a_nX^n\]
with $a_i\in A$. Then $a_0\in\p_0$ by definition, and we have $g=a_1X+\cdots+a_nX^n\in\p$. Note that $g$ is divisible by $X$, so $g/X\in\p$ since $\p$ is prime and $X\notin\p$. Replacing this process we eventually conclude that $a_i\in\p_0$ for all $i$, whence $f\in\p_0\cdot A[X]$.
\end{itemize}
We now compute $\mathrm{htgr}(\p)$. Let $\q_0\subset\q_1\subset\cdots\subset\q_n=\p_0$ be a saturated chain in $A$, so that $n=\height(\p_0)$. Then by extending them in $A[X]$, we get a chain 
\[\q_0\cdot A[X]\subset\q_1\cdot A[X]\subset\cdots\subset\q_n\cdot A[X]=\p_0\cdot A[X].\]
In case (b), we have $\p=\p_0\cdot A[X]$, and any homogeneous prime ideal $\p'\sub\p$ of $A[X]$ satisfies $\p'=\p'_0\cdot A[X]$ (using the claim in (b)). From this, it is easy to conclude that $\mathrm{htgr}(\p)=\height(\p)$ in this case.\par
On the other hand, in case (a) we claim that the chain $\p_0\cdot A[X]\sub\p=\p_0+X\cdot A[X]$ is saturated for homogeneous prime ideals of $A$, so that $\mathrm{htgr}(\p)=\height(\p)+1$. For this, let $\p'\sub\p$ be a homogeneous prime ideal containing $\p_0\cdot A[X]$ and choose $f\in\p'\setminus(\p_0\cdot A[X])$; write $f=a_0+a_1X+\cdots+a_nX^n$ as usual, and recall that $a_0\in\p_0$ with $a_i\notin\p_0$. Since $f\notin\p_0\cdot A[X]$, we may assume that $a_d\notin\p_0$ with $1\leq d\leq n$. As $\p'$ is homogeneous, each homogeneous componenet of $f$ belongs to $\p'$, so $a_dX^d\in\p'\sub\p$. But every element of $\p$ has constant term in $\p_0$ by definition and $\p'$ is prime, so $X^d\in\p'$, and therefore $X\in\p'$. This shows $\p'=\p$ and justifies our claim.\par
In summary, we have proved the following equality
\[\mathrm{htgr}(\p)=\begin{cases}
\height(\p_0)+1&\text{if $X\in\p$},\\
\height(\p_0)&\text{if $X\notin\p$}.
\end{cases}\]
Now by taking $\p$ to be the maximal ideals of $A[X]$, we deduce that $\mathrm{dimgr}(A[X])\leq\dim(A)+1$. However, it is easy to see this equality actually holds, for example we can take a saturated chain $\q_0\subset\q_1\subset\cdots\subset\q_n$ in $A$, extending it to a chain
\[\q_0\cdot A[X]\subset\q_1\cdot A[X]\subset\cdots\subset\q_n\cdot A[X]\subset\q_n\cdot A[X]+X\cdot A[X].\]
It is easy to see the ideals above are homogeneous and prime, which shows $\mathrm{dimgr}(A[X])\geq\dim(A)+1$.\par
Finally, recall that in \cref{polynomial ring dimension counterexample} we have shown that for any positive integers $n$ and $d$ with $n+1\leq d\leq 2n+1$, there exists a ring $A$ such that $\dim(A)=n$ $\dim(A[X])=d$. In view of our previous result, we see the inequality in (\ref{dimension homogeneous ideal dimgr prop}) can be strict. But also recall that for a Noetherian ring $A$, we always have $\dim(A[X])=\dim(A)+1$ (\cref{Noe ring dimension of polynomial ring}).
\end{example}
We now turn to our main theorem of this part, which asserts that if the graded ring $R$ is Noetherian then $\mathrm{dimgr}(R)=\dim(R)$. First, we recall that for an arbitrary ideal $\a$ of $R$, we denote by $\a^{h}$ the ideal generated by homogeneous elements in $\a$: in other words $\a^{h}=\sum_n(\a\cap R_n)$. We note that $\a^h$ is a homogeneous and contained in $\a$. Moreover, by \cref{graded ring associated prime ideal}, $\p^h$ is prime if $\p$ is prime.
\begin{lemma}\label{graded ring maximal iff maximal homogeneous}
Any maximal element of the set of homogeneous prime ideals of $R$ is a maximal ideal of $R$ containing $R_{\geq 1}$.
\end{lemma}
\begin{proof}
Let $\m$ be a proper homogeneous ideal of $R$. Then we have
\[\m\sub(\m\cap R_0)+R_{\geq 1}\neq R.\]
If $\m$ is maximal, then $\m=\m_0+R_{\geq 1}$, where $\m_0$ is a maximal ideal of $R_0$.
\end{proof}
\begin{lemma}\label{graded ring prime ideal chain equal homogenous}
Let $\p$ and $\q$ be distinct prime ideals of $R$ such that $\q\subset\p$. If $\q^h=\p^h$, then $\q$ is homogeneous, $\p$ is not homogeneous, and $\height(\p/\q)=1$.
\end{lemma}
\begin{proof}
By replacing $R$ with $R/\q^h$, we can assume that $\q^h=0$, so that $R$ is integral (since $\q^h$ is prime), $\p^h=0$. It suffices to prove that $\height(\p)\leq 1$, because $\height(\p)=0$ implies $\p=\{0\}$ when $R$ is integral. Now since $\p^h=\{0\}$, we have $\p\cap R_n=\{0\}$ for each $n$, so $\p$ is disjoint from the multiplicative subset $T=\bigcup_n(R_n\setminus\{0\})$. The ring $R_\p$ is then isomomorphic to a subring of $T^{-1}R$, and we get
\[\height(\p)=\dim(R_\p)\leq\dim(T^{-1}R).\]
However, by \cref{graded ring localization of homogeneous element is polynomial} the ring $T^{-1}R$ is isomomorphic to $K[X,X^{-1}]$, where $K$ is a field. Therefore $\dim(T^{-1}R)\leq 1$ and $\height(\p)\leq 1$, which proves the claim.
\end{proof}
\begin{proposition}\label{graded ring height of p^h prop}
Let $\p$ be a prime ideal of $R$. If $\p\neq\p^h$ then $\height(\p^h)=\height(\p)-1$.
\end{proposition}
\begin{proof}
By \cref{graded ring associated prime ideal}, $\p^h$ is a prime ideal contained in $\p$, so $\height(\p^h)\leq\height(\p)-1$. The proposition is trivial if $\height(\p^h)=+\infty$, so we may assume that $\height(\p^h)<+\infty$. We now demonstrate the inequality $\height(\p)\leq\height(\p^h)+1$ by proving that, for any prime ideal $\q$ contained in $\p$ and distinct from $\p$, we have $\height(\q)\leq\height(\p^h)$.\par
We now proceed by induction on $\height(\p)$. We distinguish two cases: if $\q^h\neq\p^h$, then we have $\height(\q^h)<\height(\p^h)$ since $\q^h\subset\p^h$; also, the equality
\[\height(\q)\leq\height(\q^h)+1\]
holds by the induction hypothesis if $\q\neq\q^h$ and is trivial if $\q=\q^h$. Concequently, we get $\height(\q)\leq\height(\q^h)+1\leq\height(\p^h)$, which proves our claim. On the other hand, if $\q^h=\p^h$, then $\q=\q^h$ by \cref{graded ring prime ideal chain equal homogenous}, so $\height(\q)=\height(\q^h)=\height(\p^h)$.
\end{proof}
\begin{theorem}\label{dimension graded ring Noetherian prop}
Suppose that the ring $R$ is Noetherian.
\begin{itemize}
\item[(a)] Any chain of homogeneous prime ideals of $R$, saturated as a chain of homogeneous prime ideals, is saturated as a chain of prime ideals.
\item[(b)] For any homogeneous prime ideal $\p$ of $R$, we have $\mathrm{htgr}(\p)=\height(S^{-1}\p)=\height(\p)$.
\item[(c)] We have $\mathrm{dimgr}(R)=\dim(S^{-1}R)=\dim(R)$.
\end{itemize}
\end{theorem}
\begin{proof}
To show (a), it suffices to prove that, if $\p$ and $\q$ are distinct homogeneous prime idelas of $R$ such that $\q\subset\p$ and there exists no homogeneous prime ideals between $\q$ and $\p$, then $\height(\p/\q)=1$. By quotienting by $\q$, we may assume that $\q=\{0\}$ (the quotient ring is still Noetherian). In other words, it suffices to show that, if $R$ is integral and Noetherian, and if $\p$ is an homogeneous prime ideal of $R$ minimal among nonzero homogeneous prime ideals, then $\height(\p)=1$. Now let $a$ be a nonzero homogeneous element of $\p$ and $\r$ a prime ideal of $R$ such that $a\in\r\subset\p$ and is minimal with this property. Then $\r^h$ is a nonzero prime ideal ($a\in\r^h$) contained in $\r$, so $\r^h=\r$, which means $\r$ is homogeneous. By the hypothesis on $\p$, this means $\p=\r$. Since $R$ is Noetherian and integral, $\p$ has height $1$ (\cref{Noe integral domain minimal prime of element height 1}), which proves (a).\par
Let $\p$ be a homogeneous prime ideal of $R$. We already have
\[\mathrm{htgr}(\p)\leq\height(S^{-1}\p)\leq\height(\p).\]
We now prove the converse inequality by induction on $\mathrm{htgr}(\p)$. If $\mathrm{htgr}(\p)=0$, $\p$ is minimal among the homogeneous prime ideals, so it is a minimal prime ideal of $R$ (\cref{graded ring minimal prime are graded}) and $\height(\p)=0$. Suppose now $\height(\p)>0$, we show the inequality $\height(\q)\leq\mathrm{htgr}(\p)-1$ for any prime ideal $\q$ contained in $\p$ and distinct from $\p$. We distinguish two cases: If $\q$ is homogeneous, this is true from the induction hypothesis:
\[\height(\q)\leq\mathrm{htgr}(\q)\leq\mathrm{htgr}(\p)-1.\]
If $\q$ is not homogeneous, we have $\q^h\subset\q\subset\p$, so $\height(\q^h)\leq\mathrm{htgr}(\q^h)$ by the induction hypothesis. Recall that $\height(\q)=\height(\q^h)+1$ by \cref{graded ring height of p^h prop}, so $\height(\q)\leq\mathrm{htgr}(\q^h)+1$. The proof will now be completed if we can prove the inequality $\mathrm{htgr}(\q^h)\leq\mathrm{htgr}(\p)-2$. Assume on the contrary that we have $\mathrm{htgr}(\q^h)=\mathrm{htgr}(\p)-1$; then the chain $\q^h\subset\p$ is saturated by part (a), which contradicts the fact $\q^h\subset\q\subset\p$.\par
Finally, we show that inequality $\dim(R)\leq\mathrm{dimgr}(R)$, or equivalently $\height(\p)\leq\mathrm{dimgr}(R)$ for any prime ideal $\p$ of $R$. If $\p$ is homogeneous, we have $\height(\p)=\mathrm{htgr}(\p)\leq\mathrm{dimgr}(R)$ by (b). If $\p$ is not homogeneous, we have $\height(\p)=\mathrm{htgr}(\p^h)+1$ by \cref{graded ring height of p^h prop}; let $\m$ be a homogeneous maximal ideal of $R$ containing $\p^h$ (note that the homogeneous ideal is not maximal among homogeneous prime ideals by \cref{graded ring maximal iff maximal homogeneous}, since we have $\p^h\subset\p$). Then by \cref{graded ring maximal iff maximal homogeneous}, $\m$ is maximal, so distinct from $\p^h$, and we have
\[\mathrm{htgr}(\p^h)+1\leq\mathrm{htgr}(\m)\leq\mathrm{dimgr}(R).\]
This shows $\height(\p)\leq\mathrm{dimgr}(R)$, which finishes the proof.
\end{proof}
\subsection{Dimension of graded modules}
Again $R$ denote a graded ring with positive degrees and $S=1+R_{\geq 1}$. We now let $M$ be a graded $R$-module of type $\Z$. Then $S^{-1}M$ is an $S^{-1}R$-module, and if we write $M_{\geq n}=\bigoplus_{i\geq n}M_i$, the submodules $S^{-1}M_{\geq n}$ form an exhaustive and separated filtration of $S^{-1}M$ and the canonical map $M\mapsto S^{-1}M$ induces an isomorphism of $M$ onto the graded module of $S^{-1}M$ associated to this filtration.
\begin{lemma}\label{dimension graded module filtration good if}
Suppose that $R$ is generated by $R_0\cup R_1$ and that $M$ is generated by $\bigoplus_{i\leq n_0}M_i$ for an integer $n_0$. Then the filtration $(S^{-1}M_{\geq n})$ of $S^{-1}M$ is good for the ideal $S^{-1}R_{\geq 1}$ of $S^{-1}R$.
\end{lemma}
\begin{proof}
For $n\geq n_0$, we have $M_{\geq n+1}=R_1\cdot M_{\geq n}$, so
\[S^{-1}M_{\geq n+1}=R_1\cdot S^{-1}M_{\geq n}=S^{-1}R_{\geq 1}\cdot S^{-1}M_{\geq n},\]
which shows the filtration $(S^{-1}M_{\geq n})$ is $(S^{-1}R_{\geq 1})$-good.
\end{proof}
\begin{proposition}\label{dimension graded module equal to localization at S}
Suppose that $R$ is Noetherian and $M$ is finitely generated. Then $\dim_R(M)=\dim_{S^{-1}R}(S^{-1}M)$.
\end{proposition}
\begin{proof}
Let $\a$ be the annihilator of the $R$-module $M$, which is a homogeneous ideal of $R$. Since $M$ is finitely generated, the annihilator of $S^{-1}M$ is the ideal $S^{-1}\a$. We have $\dim_R(M)=\dim(R/\a)$ and $\dim_{S^{-1}R}(S^{-1}M)=\dim(S^{-1}R/S^{-1}\a)$ by definition. Since $R/\a$ is Noetherian, an application of \cref{dimension graded ring Noetherian prop} proves the claim.
\end{proof}
\begin{proposition}\label{dimension graded module generating function prop}
Suppose that $R_0$ is an Artinian local ring, $R$ is generated by $R_0\cup R_1$, and $R_1$ is a finitely generated $R_0$-module. If $M$ is nonzero and finitely generated as an $R$-module, then $M_n$ is an $R_0$-module with finite length for each $n$, and there exists $Q(T)\in\Z[T,T^{-1}]$ such that $Q(1)>0$ and the following equality holds in the ring $\Z((T))$:
\[\sum_{n\in\Z}\ell_{R_0}(M_n)\cdot T^n=(1-T^{-d})Q(T),\]
where $d=\dim_R(M)$.
\end{proposition}
\begin{proof}
The ring $S^{-1}R$ is local and Noetherian by \cref{dimension graded ring S^-1R is local if R_0 local}, the $S^{-1}R$-module $S^{-1}M$ is finitely generated and nonzero, and $d=\dim_R(M)$ is nonzero. Also, by \cref{dimension graded module filtration good if} the filtration $(S^{-1}M_{\geq n})$ is $S^{-1}R_{\geq 1}$-good and $S^{-1}R_{\geq 1}$ is a defining ideal of $S^{-1}R$ (\cref{Noe local ring defining ideal iff}). Finally, we have $\ell_{S^{-1}R}(S^{-1}M_{\geq n}/S^{-1}M_{\geq n+1})=\ell_{R_0}(M_n)$ for each $n$, so it suffices to apply \cref{filtration good Hilbert-Samuel series expression as d-product}. 
\end{proof}
\begin{corollary}\label{Noe local ring gr dim for defining ideal}
If $R$ is a Noetherian local ring and $\a$ is a defining ideal of $R$, then $\dim(R)=\dim(\gr_\a(R))$.
\end{corollary}
\begin{proof}
Apply \cref{dimension graded module generating function prop} on the module $M=B=\gr_\a(R)$, we get
\[\sum_{n=0}^{\infty}\ell_{R/\a}(\a^n/\a^{n+1})\cdot T^n=(1-T)^{-d}Q(T)\]
where $d=\dim(\gr_\a(R))$ and $Q(1)\neq 0$. We have $d=\dim(R)$ by \cref{Noe local ring Hilbert-Samuel degree is dimension}, whence the corollary.
\end{proof}
\begin{corollary}\label{dimension graded module secant element}
Suppose that $M$ is a finitely generated $R$-module. Let $a$ be a homogeneous element of $R$ with positive degree, and not belonging to any minimal element $\p$ in $\supp(M)$ such that $\dim(R/\p)=\dim_R(M)$. Then $\dim_R(M/aM)=\dim_R(M)-1$.
\end{corollary}
\begin{proof}
Put $d=\dim_R(M)$; by the hypothesis on $a$, we have $\dim_R(M/aM)<d$. Also, by \cref{dimension graded module equal to localization at S}, we have
\[\dim_R(M/aM)=\dim_{S^{-1}M}(S^{-1}M/(b/1)\cdot S^{-1}M)\]
and the formula (\ref{dimension and secant sequence-3}) implies the following inequality:
\[\dim_{S^{-1}R}(S^{-1}M/(b/1)\cdot S^{-1}M)\geq\dim_{S^{-1}R}(S^{-1}M)-1.\]
Finally, $\dim_{S^{-1}R}(S^{-1}M)=\dim_R(M)=d$ by \cref{dimension graded module equal to localization at S}. We then conclude that $\dim_R(M/bM)\geq d-1$, which proves the claim.
\end{proof}
\begin{proposition}\label{dimension graded module maximal secant sequence}
Suppose that $R_0$ is an Artinian local ring, $R$ is an $R_0$-algebra of finite type and $M$ is a finitely generated $R$-module.
\begin{itemize}
\item[(a)] Let $a_1,\dots,a_n$ be homogeneous elements of $R$ with positive degrees and $\varphi$ the homomorphism from $R_0[X_1,\dots,X_n]$ to $R$ map $X_i$ to $a_i$ for each $i$. Then $S^{-1}M/\sum_{i=1}^{n}(a_i/1)\cdot S^{-1}M$ has finite length over $S^{-1}R$ if and only if $\varphi^*(M)$ is a finitely generated module over $R_0[X_1,\dots,X_n]$. 
\item[(b)] There exists a family $(a_1,\dots,a_d)$ of elements of $R$, homogeneous with the same positive degree, where $d=\dim_R(M)$, such that $(a_1/1,\dots,a_d/1)$ is a maximal secant sequence for the $S^{-1}R$-module $S^{-1}M$. If moreover $R$ is generated by $R_1$ as an $R_0$-algebra and the residue field of $R_0$ is infinite, each $a_i$ can be taken to have degree $1$.
\end{itemize}
\end{proposition}
\begin{proof}
Put $N=M/\sum_{i=1}^{n}a_iM$, we have $\dim_R(N)=\dim_{S^{-1}R}(S^{-1}N)$ by \cref{dimension graded module equal to localization at S}. Also, the $S^{-1}R$-module $S^{-1}N$ is of finite length if and only if the $R$-module $N$ has finite length, that is, $N$ is a finitely generated $R_0$-module. If $\varphi^*(M)$ is the pullback module over $R_0[X_1,\dots,X_n]$, we have $\varphi^*(N)=\varphi^*(M)/\sum_{i=1}^{n}X_i\cdot\varphi^*(M)$. Therefore $\varphi^*(M)$ is a finitely generated $R_0[X_1,\dots,X_n]$-module if and only if $N$ is a finitely generated $R_0$-module, which proves (a).\par
Let's assume the hypothesis in \cref{dimension graded module maximal secant sequence}(b). We may assume that $\dim_R(M)>0$. Note that any minimal element of $\supp(M)$ is homogeneous (\cref{graded ring minimal prime are graded}). According to \cref{graded ring prime avoidence}, there exists a homogeneous element $a$ of $R$ with positive degree, not belonging to any minimal element $\p$ in $\supp(M)$ such that $\dim(R/\p)=\dim_R(M)$. By \cref{dimension graded module secant element}, we have $\dim_R(M/aM)=\dim_R(M)-1$. Suppose further that $R$ is generated by $R_1$ as an $R_0$-algebra and the residue field $k$ of $R_0$ is infinite. For any minimal element $\p$ in $\supp(M)$ such that $\dim(R/\p)=\dim_R(M)$, consider the subspace $V_\p=(\p\cap R_1)\otimes_{R_0}k$ of the $k$-vector space $V=R_1\otimes_{R_0}k$. These subspaces are proper: if $V_\p=V$, then $\p\cap H_1=H_1$ by \cref{finite module M=IM+N then M=N}, so $H_1\sub\p$ and $\dim_R(M)=\dim(R/\p)\leq\dim(R/R_1)=0$, which contradicts to our assumption. Since $k$ is infinite, the union of $V_\p$'s is distinct from $V$. If $a\in R_1$ is such that $a\otimes 1$ is not contained in any of the $V_\p$, then $a$ is not contained in the minimal element $\p$ such that $\dim(R/\p)=\dim_R(M)$, whence $\dim_R(M/aM)=\dim_R(M)-1$.\par
By induction on the dimension $d=\dim_R(M)$, we can construct a sequence $(b_1,\dots,b_d)$ of elements in $R$, with $n_i=\deg(b_i)>0$, wuch that $M/\sum_{i=1}^{d}b_iM$ is a $R$-module of finite length. Moreover, if $R$ is generated by $R_1$ as an $R_0$-algebra and the residue field of $R_0$ is infinite, each $b_i$ can be taken to have degree $1$. Now by \cref{dimension graded module equal to localization at S}, $\dim_{S^{-1}R}(S^{-1}M)=d$ and
\[\dim_{S^{-1}R}(S^{-1}M/\sum_{i=1}^{d}(b_i/1)\cdot S^{-1}M)=0.\]
Therefore $(b_1/1,\dots,b_d/1)$ is a maximal secant sequence for $S^{-1}M$. Put $a_i=b_i^{(n_1\cdots n_d)/n_i}$, we see the $a_i$'s have the same degree, and $(a_1/1,\dots,a_d/1)$ is a maximal secant sequence for $S^{-1}M$ by \cref{Noe local ring sequence secant iff power secant}.
\end{proof}
\begin{corollary}\label{dimension graded ring finitely generated over polynomial ring}
Suppose that $R_0$ is a field and $R$ is an $R_0$-algebra of finite type. Put $n=\dim(R)$. Then there exists homogeneous elements $a_1,\dots,a_n$ of $R$ with the same positive degree such that the $R_0$-homomorphism $\varphi:R_0[X_1,\dots,X_n]\to R$, defined by $\varphi(X_i)=a_i$ for each $i$, is injective and makes $R$ a finitely generated $R_0[X_1,\dots,x_n]$-module. If $R$ is generated by $R_1$ as an $R_0$-algebra and $R_0$ is infinite, each $a_i$ can be taken to have degree $1$.
\end{corollary}
\begin{proof}
There exists by \cref{dimension graded module maximal secant sequence} an $R_0$-homomorphism $\varphi$ of the indicated form making $R$ a finitely generated $R_0[X_1,\dots,X_n]$-module. By \cref{integral extension dimension of module ideal prop}, we also have
\[\dim(R)=n=\dim(R_0[X_1,\dots,X_n]).\]
Since the ring $R_0[X_1,\dots,X_n]$ is integral and $R\cong R_0[X_1,\dots,X_n]/\ker\varphi$, we conclude that $\ker\varphi=0$.
\end{proof}
\begin{remark}
Assume the hypothesis in \cref{dimension graded ring finitely generated over polynomial ring}. Let $(h_1,\dots,h_r)$ be a basis for the $R_0$-vector space $R_1$. For $\lambda=(\lambda_1,\dots,\lambda_r)\in R_0^r$, put $h_\lambda=\lambda_1h_1+\cdots+\lambda_rh_r$. Then \cref{dimension graded module maximal secant sequence} and \cref{dimension graded ring finitely generated over polynomial ring} imply the following result: the set of elements $(\lambda^{(1)},\dots,\lambda^{(n)})$ of $(R_0^r)^n$ such that the elements $a_i=h_{\lambda_i}\in R_1$ satisfies the conclusion of \cref{dimension graded ring finitely generated over polynomial ring} contains the complement of the union of a finite number of vector subspaces of $(R_0^r)^n$ distinct from the entire space.
\end{remark}
\begin{corollary}\label{Noe local ring dimension geq n iff}
Let $A$ be a Noetherian local ring and $d$ a non-negative integer. For $\dim(A)\geq n$, it is necessary and sufficient that for any integer $r\geq 0$ we have
\[[\m_A^r/\m_A^{r+1}:\kappa_A]\geq\binom{n+r-1}{n-1}.\]
Moreover, the equality holds if and only if $A$ is regular of dimension $n$.
\end{corollary}
\begin{proof}
By \cref{Noe local ring Hilbert-Samuel degree is dimension} and \cref{Noe local ring regular iff m_A generated by completely secant}, this condition is sufficient. We now demonstrate the necessity, so assume that $\dim(A)\geq n$. Consider the graded ring $\gr(A)=\gr_{\m_A}(A)$; let $k$ be a extension of the residue field $\kappa_A$ such that $k$ is infinite, and put $R=k\otimes_{\kappa_A}\gr(A)$. The ring $R$ has dimension $\geq n$ by \cref{Noe local ring gr dim for defining ideal}. We deduce by \cref{dimension graded ring finitely generated over polynomial ring} the existence of an injective $k$-algebra homomorphism $\varphi:R_0[X_1,\dots,X_n]\to R$. Therefore for any integer $r\geq 0$,
\[[\m_A^r/\m_A^{r+1}:\kappa_A]=[R_r:R_0]\geq\binom{n+r-1}{n-1},\]
and the equality holds if and only if $\varphi$ is bijective, whence $A$ is regular (\cref{Noe local ring regular iff m_A generated by completely secant}). 
\end{proof}
\begin{remark}
Note that we have the equality
\[\binom{n+r}{n}=\sum_{i=0}^{r}\binom{n+i-1}{n-1},\quad \ell_A(A/\m_A^{r+1})=\sum_{i=0}^{r}[\gr_i(A):\kappa_A].\]
Therefore the condition in \cref{Noe local ring dimension geq n iff} is equivalent to $\ell_A(A/\m_A^{r+1})\geq\binom{n+r}{n}$.
\end{remark}
\subsection{Dimension and graded algebras}
\begin{lemma}\label{graded module finitely generated if radical quotient}
Let $A$ be a ring, $\r$ its Jacobson radical, $R=\bigoplus_{i\in\Z}R_i$ a graded $A$-algebra, and $M=\bigoplus_{i\in\Z}M_i$ a graded $R$-module. Suppose that each $M_i$ is a finitely generated $A$-module and $M/\r M$ is a finitely generated $R/\r R$ module. Then $M$ is a finitely generated $R$-module.
\end{lemma}
\begin{proof}
Let $m_1,\dots,m_n$ be homogeneous elements of $M$ whose images generate $M/\r M$. Let $N$ be the graded submodule of $M$ generated by $m_1,\dots,m_n$. We have $M_i=N_i+\r M_i$ for each $i$, whence $M_i=N_i$ (\cref{finite module M=IM+N then M=N}); this shows $M=N$.
\end{proof}
\begin{lemma}\label{ring finite type extension localization finite}
Let $\rho:A\to B$ be a ring homomorphism and $S$ be a multiplicative subset of $A$. Suppose that $B$ is an $A$-algebra of finite type and $S^{-1}B$ is a finitely generated $S^{-1}A$-module. Then there exists $f\in S$ such that $B_f$ is a finitely generated $A_f$-module.
\end{lemma}
\begin{proof}
Let $X$ be a generating set of the $A$-algebra $B$. For each $x\in X$, the image of $x$ in $S^{-1}B$ is integral over $S^{-1}A$, so there exist an integer $n(x)\geq 0$, elements $b_1(x),\dots,b_{n(x)}(x)$ of $A$, and an element $f(x)\in S$, such that
\[f(x)x^{n(x)}+b_1(x)x^{n(x)-1}+\cdots+b_{n(x)}=0.\]
Let $f=\prod_{x\in X}f(x)$; the image of $x\in X$ in $B_f$ is then integral over $A_f$, so $B_f$ is a finitely generated $A_f$-module.
\end{proof}
\begin{proposition}\label{dimension graded algebra upper semi-continuous}
Suppose that $R$ is an $R_0$-algebra of finite type. Then the function $\p\mapsto\dim(R\otimes_{R_0}\kappa(\p))$ is upper semi-continuous on $\Spec(A_0)$.
\end{proposition}
\begin{proof}
Since $R$ is an $R_0$-algebra of finite type, each $R_i$ is finitely generated as an $R_0$-module (\cref{graded ring finitely generated criterion}). Let $\p\in\Spec(R_0)$ and suppose $\dim(R\otimes_{R_0}\kappa(\p))=n\geq 0$. By \cref{dimension graded ring finitely generated over polynomial ring}, there exists $a_1,\dots,a_n$ in $R$, homogeneous of degree $d>0$, such that the $\kappa(\p)$-homomorphism $\bar{\varphi}:\kappa(\p)[X_1,\dots,X_n]\to R\otimes_{R_0}\kappa(\p)$ map $X_i$ to $a_i\otimes 1$ for each $i$, makes $R\otimes_{R_0}\kappa(\p)$ a finitely generated $\kappa(\p)[X_1,\dots,X_n]$-module. Let $\varphi$ be the $R_0$-homomorphism from $R_0[X_1,\dots,X_n]=H$ to $R$ map $X_i$ to $a_i$ for each $i$.\par
For $m\in\Z$, define
\[\tilde{R}_m=\bigoplus_{(m-1)d<i\leq md}R_i.\]
Then $(\tilde{R}_m)$ is a graduation of type $\Z$ on $R$ which is compatible with the $R_0[X_1,\dots,X_n]$-module structure induced by $\varphi$; moreover, each $\tilde{R}_m$ is finitely generated over $R_0$. Now note that the Jacobson radical of $(R_0)_\p$ is $\p(R_0)_\p$ and
\[R_\p/\p(R_0)_\p R_\p=R\otimes_{R_0}\kappa(\p),\quad H_\p/\p(R_0)_\p H_\p=\kappa(\p)[X_1,\dots,X_n].\]
By using \cref{graded module finitely generated if radical quotient}, we conclude that $R_\p$ is a finitely generated $H_\p$-module. Now apply \cref{ring finite type extension localization finite}, there then exists $f\in R_0\setminus\p$ such that $R_f$ is finitely generated $H_f$-module. For any $\q\in\Spec((R_0)_f)$, $R\otimes_{R_0}\kappa(\q)$ is then a finitely generated $\kappa(\q)[X_1,\dots,X_n]$-module, whence $\dim(R\otimes_{R_0}\kappa(\q))\leq n$ by \cref{integral extension dimension of module ideal prop}. This completes the proof.
\end{proof}
From now on, we assume that $R_0$ is a field and $R$ is an $R_0$-algebra of finite type. In this case, we set $R_+=R_{\geq 1}$, which is a maximal ideal of $R$. The ring $S^{-1}R$ can be then identified with the local ring $R_{R_+}$, whose maximal ideal is $(R_+)_{R_+}=S^{-1}R_+$ with residue field $A_0$.
\begin{proposition}\label{graded algebra over field regular iff}
Suppose that $R_0$ a field and $R$ is a finite type $R_0$-algebra.
\begin{itemize}
\item[(a)] We have $\dim(R)\leq[R_+/R_+^2:R_0]$.
\item[(b)] The following conditions are equivalent:
\begin{itemize}
\item[(\rmnum{1})] $\dim(R)=[R_+/R_+^2:R_0]$.
\item[(\rmnum{2})] The Noetherian local ring $S^{-1}R$ is regular.
\item[(\rmnum{3})] $R$ is generated as an $R_0$-algebra by a family of homogeneous elements with positive degrees, algebraically independent over $R_0$.
\end{itemize}
\item[(c)] Suppose that the conditions in (b) holds, and let $a_1,\dots,a_n\in R$ be homogeneous elements with positive degrees. Then the following conditions are equivalent:
\begin{itemize}
\item[(\rmnum{1}')] The images of $a_i$'s form a basis for the $R_0$-vector space $R_+/R_+^2$.
\item[(\rmnum{2}')] The images of $a_i$'s form a system of parameters in the Noetherian local ring $S^{-1}R$.
\item[(\rmnum{3}')] The $a_i$'s are algebraically independent over $R_0$ and generate $R$ as an $R_0$-algbera. 
\end{itemize}  
\end{itemize}
\end{proposition}
\begin{proof}
Since $R$ is Noetherian, we have $\dim(R)=\dim(S^{-1}R)$ by \cref{dimension graded ring Noetherian prop}; also,
\[[R_+/R_+^2:R_0]=[(S^{-1}R_+)/(S^{-1}R_+)^2:R_0]\]
by \cref{localization of sum and intersection}. These together prove (a) and the equivalences (\rmnum{1})$\Leftrightarrow$(\rmnum{2}), (\rmnum{1}')$\Leftrightarrow$(\rmnum{2}'). The implications (\rmnum{3})$\Rightarrow$(\rmnum{1}) and (\rmnum{3}')$\Rightarrow$(\rmnum{1}') are trivial. To establish (\rmnum{1})$\Rightarrow$(\rmnum{3}), suppose that $\dim(R)=[R_+/R_+^2:R_0]$ and let $a_1,\dots,a_n$ be homogeneous elements of $R$ with positive degrees whose classes in $R_+/R_+^2$ form a basis. Consider the homomorphism $\varphi:R_0[X_1,\dots,X_n]$ to $R$ map $X_i$ to $a_i$. The ideal $R_+$ of $R$ is generated by $a_i$, so the homomorphism $\varphi$ is surjective. Thus we have
\[\dim(R_0[X_1,\dots,X_n])=n=\dim(R)=\dim(R_0[X_1,\dots,X_n]/\ker\varphi)\]
which implies $\ker\varphi=0$ (recall that $R_0[X_1,\dots,X_n]$ is an integra ldomain). This implies (\rmnum{3}), and the implication (\rmnum{1}')$\Rightarrow$(\rmnum{3}') can be done similarly.
\end{proof}
If $R$ satisfies the hypothesis in (b), we say $R$ is an regular graded $R_0$-algebra, or an polynomial graded $R_0$-algebra. In this case, any family $(a_1,\dots,a_n)$ satisfying the hypothesis in (c) will be called a system of parameters of $R$.
\begin{remark}
In the notations of (c), let $d_i$ be the degree of $a_i$ for each $i$. Then the Poincar\'e series of $H$ is given by
\[P_{R}=\sum_{n\in\Z}[R_n:R_0]\cdot T^n=\prod_{i=1}^{n}(1-T^{d_i})^{-1}.\]
Conversely, if there exist integers $d_i>0$ such that $P_R=\prod_i(1-T^{d_i})^{-1}$, we may conclude that $R$ is a polynomial graded algebra. For example, if $R$ is generated by $X$ with degree $1$ and $Y$ with degree $2$ with the sole relation $X^2=0$, we have $P_R=(1-T)^{-1}$, so $R$ is regular.
\end{remark}
\section{Multiplicity}
\subsection{Multiplicity relative to an ideal}
Let $A$ be a Noetherian ring and $M$ be a finitely generated $A$-module. Let $\a$ be an ideal of $A$ contained in the Jacobson radical of $A$ and such that $M/\a M$ is of finite length. Suppose that $M$ is nonzero and put $d=\dim_A(M)$. Recall that (\cref{filtration good index and length formula}) there exists a unique integer $e_\a(M)>0$ such that, for $n\geq 1$,
\begin{align}\label{Noe ring multiplicity formula}
\ell_A(M/\a^{n+1}M)=e_\a(M)\frac{n^d}{d!}+\beta_nn^{d-1}
\end{align}
where $\beta_n$ has a limit when $n$ goes to infinity. The integer $e_\a(M)$ (or $e_\a^A(M)$ if we want to emphasize the ring $A$) is called the \textbf{multiplicity} of $M$ relative to $\a$. If $A$ is a local ring and $\m$ is the maximal ideal of $A$, we write $e(M)$ or $e^A(M)$ for $e_\m^A(M)$. Recall that equivalently, we have
\[\ell_A(\a^nM/\a^{n+1}M)=e_\q(M)\frac{n^{d-1}}{(d-1)!}+\alpha_nn^{d-1}\]
where $\alpha_n$ has a limit when $n$ goes to infinity.
\begin{remark}\label{Noe ring contained ideal multiplicity inequality}
Let $\b$ be an ideal of $A$ contained in the Jacobson radical of $A$ and contains $\a$. Then $M/\b^nM$ is a quotient of $M/\a^nM$, whence $e_\b(M)\leq e_\a(M)$ by the formula (\ref{Noe ring multiplicity formula}). Moreover, if the $\b$-adic filtration on $M$ is $\a$-good, then $e_\a(M)=e_\b(M)$ by \cref{filtration good Hilbert-Samuel series expression as d-product}.
\end{remark}
Recall that we can reduce the calculus of the multiplicities to the case where $A$ is local since, according to \cref{Noe ring Hilbert-Samuel degree multiplicity char}, we have
\[e_\a(M)=\sum_\m e_{\a_\m}(M_\m)\]
where $\m$ runs through the elements of $\supp(M)\cap V(\a)$ such that $\dim_{A_\m}(M_\m)=d_\a(M)$.
\begin{proposition}\label{Noe ring multiplicity and completion}
Let $\widehat{A}$ (resp. $\widehat{M}$) be the $\a$-adic completion of $A$ (resp. $M$), then $e_\a^A(M)=e_{\hat{\a}}^{\widehat{A}}(\widehat{M})$.
\end{proposition}
\begin{proof}
This follows from the fact that $\gr_{\a}(M)\cong\gr_{\hat{\a}}(\widehat{M})$.
\end{proof}
\begin{proposition}\label{Noe local ring multiplicity one}
If $A$ is regular, then $e(A)=1$.
\end{proposition}
\begin{proof}
If $A$ is regular then $H_{A,\m_A}=(1-T)^{-r}$ (\cref{Noe local ring regular iff m_A generated by completely secant}), whence $e(A)=1$.
\end{proof}
\begin{example}
By definition $e_{\a^r}(M)=r^de_\a(M)$ where $d=\dim_A(M)$. Concequently, if $A$ is a regular local ring, we have $e_{\m_A^r}(A)=r^d$. For example, if $A$ is a discrete valuation ring, then $e_{\m_A^r}(A)=r$, whence $e_{\a}(A)=\ell(A/\a)$ for any ideal $\a$.
\end{example}
For an ideal $\a$ of $A$ contained in the Jacobson radical of $A$, we let $\mathcal{C}(\a)$ denote the set of classes of finitely generated $A$-modules $M$ such that $M/\a M$ is of finite length. For each $d\in\N$, let $\mathcal{C}(\a)_{\leq d}$ be the subset of $\mathcal{C}(\a)$ consists of classes of $A$-modules of dimension $\leq d$. We have a map $e_{\a,d}:\mathcal{C}(\a)_{\leq d}\to\Z$ such that $e_{\a,d}(M)=e_{\a}(M)$ if $\dim_A(M)=d$ and $e_{\a,d}(M)=0$ otherwise. This map is additive by \cref{filtration good index and exact sequence}, so we get an induced homomorphism, still denoted by $e_{\a,d}$, from the Grothendieck group $K(\mathcal{C}(\a)_{\leq d})$ to $\Z$, which is zero on $K(\mathcal{C}(\a)_{\leq d-1})$.
\begin{proposition}\label{Noe ring multiplicity localization formula}
Let $M$ be a finitely generated $A$-module with dimension $d\geq 0$. Let $\a$ be an ideal of $A$ contained in the Jacobson radical of $A$ such that $M/\a M$ has finite length. Then
\[e_{\a}(M)=\sum_{\coht(\p)=d}\ell_{A_\p}(M_\p)\cdot e_{\a}(A/\p)\]
\end{proposition}
\begin{proof}
By \cref{associated prime of Noe prime ideals in composition}, $M$ has a composition series $(M_i)_{0\leq i\leq n}$ such that $M_i/M_{i+1}$ is isomomorphic to $A/\p_i$, where $\p_i$ is a prime ideal of $M$. Also, by \cref{Noe ring finite module length of M_p is number of p_i}, the number $\ell_{A_\p}(M_\p)$ is the times of $i$ such that $M_i/M_{i+1}\cong A/\p$. The proposition then follows from \cref{filtration good index and exact sequence}.
\end{proof}
\begin{corollary}\label{Noe ring semi-local multiplicity formula}
Suppose that $A$ is Noetherian semi-local and let $\a$ be a defining ideal for $A$.
\begin{itemize}
\item[(a)] We have $e_\a(A)=\sum_{\p}\ell_{A_\p}(A_\p)\cdot e_{\a}(A/\p)$, where $\p$ turns through minimal primes of $A$ such that $\coht(\p)=\dim(A)$.
\item[(b)] Suppose that $A$ is integral and $M$ is a finitely generated $A$-module such that $\dim_A(M)=\dim(A)$. Then $e_\a(M)=\rank(M)\cdot e_\a(A)$.
\end{itemize}
\end{corollary}
\begin{proof}
Part (a) follows directly from \cref{Noe ring multiplicity localization formula} by applying on $M=A$. Under the hypothesis of (b), the minimal prime ideal of $A$ is $(0)$, and $\ell_{A_{(0)}}(M_{(0)})=\rank(M)$, so the claim follows.
\end{proof}
\subsection{Extension of scalars}
\begin{proposition}\label{Noe local ring flat module tensor multiplicity formula}
Let $\rho:A\to B$ be a local homomorphism of Noetherian local rings. Let $N$ be a finitely generated $B$-module, flat over $A$, and such that $N\otimes_A\kappa_A$ is a $B$-module of finite length. If $M$ is a nonzero finitely generated $A$-module and $\a$ is a proper ideal of $A$ such that $M/\a M$ is of finite length, then $(M\otimes_AN)/\a^e(M\otimes_AN)$ is a $B$-module of finite length, and we have
\[e_{\a^e}^B(M\otimes_AN)=\ell_B(N\otimes_A\kappa_A)\cdot e_\a^A(M).\]
\end{proposition}
\begin{proof}
Let $L$ be an $A$-module of finite length $r$. Then by \cref{associated prime of Noe prime ideals in composition} and \cref{associated prime maximal iff finite length}, $L$ has a composition series of length $r$, with quotients $\kappa_A$. Since $N$ is flat over $A$, the $B$-module $L\otimes_AN$ possesses a composition series of length $r$, with quotients $N\otimes_A\kappa_A$. Therefore the length of $L\otimes_AN$ is $r\cdot\ell_B(N\otimes_A\kappa_A)$. The $B$-module $(M\otimes_AN)/(\a^e)^n(M\otimes_AN)$ is isomomorphic to $(M/\a^nM)\otimes_AN$ for each $n\in\N$, the proposition then follows from the definition of multiplicity.
\end{proof}
\begin{corollary}\label{Noe local ring flat extension multiplicity formula}
Suppose that $\rho:A\to B$ be a local homomorphism of Noetherian local rings, $B$ is flat over $A$ and $\m_A^e=\m_B$. Then
\[e_{\a^e}^B(M\otimes_AB)=e_{\a}^A(M).\] 
\end{corollary}
\begin{proof}
In this case we have $B\otimes_A\kappa_A=B/\m_A^e$, which has length $1$ over $B$, so the claim follows from \cref{Noe local ring flat module tensor multiplicity formula}.
\end{proof}
\begin{example}
Let $X$ be a complex algebraic variety, $\mathscr{O}_{X,x}$ the local ring of $X$ at a rational point $x$, and $X^{an}$ the analytic space associated with $X$. Let $\mathscr{O}_{X^{an},x}$ be the local ring of $X^{an}$ at $x$. Then $e(\mathscr{O}_{X^{an},x})=e(\mathscr{O}_{X,x})$. 
\end{example}
\begin{proposition}\label{Noe semi-local ring finite extension multiplicity prop}
Suppose that $A$ is Noetherian semi-local and let $\rho:A\to B$ be a ring homomorphism making $B$ a finitely generated $A$-module. Let $N$ be a nonzero finitely generated $B$-module and $\a$ an ideal of $A$ contained in the Jacobson radical of $A$ such that $N/\a N$ has finite length. Let $\m_1,\dots,\m_r$ be the maximal ideals of $B$ such that $\dim_{B_{\m_i}}(N_{\m_i})=\dim_B(N)$ (by \cref{Noe semilocal ring finite extension is Noe semilocal} $B$ is semi-local). Put $B_i=B_{\m_i}$ and $\a_i=\a B_i$ for each $i$. Then we have the equalities
\[\dim_A(N)=\dim_B(N),\quad e_{\a^e}^B(N)=\sum_{i=1}^{r}e_{\a_i}^{B_i}(N_{\m_i}),\quad e_{\a}^A(N)=\sum_{i=1}^{r}[B/\m_i:A/\m_i^c]\cdot e_{\a_i}^{B_i}(N_{\m_i}).\]
\end{proposition}
\begin{proof}
Since $B$ is integral over $A$, we have $\dim_A(N)=\dim_B(N)$. The second equality follows from \cref{Noe ring Hilbert-Samuel degree multiplicity char} (note that $\m_i$ is in $V(\a^e)$ for each $i$ since $\m_i^c\supset\a$ by \cref{integral ring maximal ideal iff contraction is}). Now let $E$ be a $B$-module of finite length; we have
\[\ell_A(E)=\sum_{\m}[B/\m:A/\m^c]\cdot\ell_{B_\m}(E_\m),\]
where $\m$ runs through maximal ideals of $B$: this is evident when $E$ is one of the $B/\m$, and the general case follows since $E$ has a composition sequence with quotients isomomorphic to $B/\m$, with $\m$ a maximal ideal of $B$. Apply this formula to the $B$-module $N/\a^{n+1}N$, we deduce the equality from the definition of multiplicity.
\end{proof}
\begin{corollary}\label{Noe semi-local ring finite extension multiplicity equal if}
If $[B/\m_i:A/\m_i^c]=1$ for each $i$, then $e_\a^A(N)=e_{\a^e}^B(N)$.
\end{corollary}
\begin{lemma}\label{ring extension localization at prime tensor bijective iff}
Let $\rho:A\to B$ be a ring homomorphism and $\p$ a prime ideal of $A$. Consider the following conditions:
\begin{itemize}
\item[(\rmnum{1})] The canonical homomorphism $\tilde{\rho}:A_\p\to A_\p\otimes_AB$ is bijective.
\item[(\rmnum{2})] There is only one prime ideal $\mathfrak{P}$ of $B$ lying over $\p$ and the canonical homomorphism $\rho_\p:A_\p\to B_{\mathfrak{P}}$ is bijective.
\end{itemize}
We have (\rmnum{1})$\Rightarrow$(\rmnum{2}). If $\p$ is minimal, or if $B$ is integral over $A$, then (\rmnum{1})$\Leftrightarrow$(\rmnum{2}).
\end{lemma}
\begin{proof}
The ring $A_\p\otimes_AB$ can be identified with the localization $S^{-1}B$ by the multiplicative subset $S=\rho(A-\p)$. The prime ideals of $S^{-1}B$ are of the form $S^{-1}\mathfrak{P}$, wherer $\mathfrak{P}$ is a prime ideal of $B$ such that $\mathfrak{P}^c\sub\p$. If $\mathfrak{P}$ is such an ideal, $(S^{-1}B)_{S^{-1}\mathfrak{P}}$ is identified with $B_{\mathfrak{P}}$.\par
If condition (\rmnum{1}) is satisfied, there exists (\cref{integral lift of prime lemma}) a unique prime ideal $\mathfrak{P}$ of $B$ such that $\mathfrak{P}^c=\p$. Moreover, $B_{\mathfrak{P}}$ is identified with the ring $(S^{-1}B)_{S^{-1}\mathfrak{P}}$, hence with $(A_\p)_{\q}$, where $\q$ is the inverse image of $S^{-1}\mathfrak{P}$ under the isomorphism $\tilde{\rho}:A_\p\to S^{-1}B$; then $\tilde{\rho}^{-1}(S^{-1}\mathfrak{P})=(A-\p)^{-1}\p=\p A_\p$, hence (\rmnum{2}).\par
Conversely, asssume (\rmnum{2}) and let $\mathfrak{P}$ be the unique prime ideal of $B$ lying over $\p$. Since $(S^{-1}B)_{S^{-1}\mathfrak{P}}$ is identified with $B_{\mathfrak{P}}$, it suffices to prove that $S^{-1}B$ is a local ring with maximal ideal $S^{-1}\mathfrak{P}$, which means any prime ideal $\mathfrak{Q}$ of $B$ such that $\mathfrak{Q}^c\sub\p$ is contained in $\mathfrak{P}$. If $\p$ is minimal, we have $\mathfrak{Q}^c=\p$, whence $\mathfrak{Q}=\mathfrak{P}$. If $B$ is integral over $A$, then by the going up theorem, there exists a prime ideal $\mathfrak{P}'$ of $B$ such that $\mathfrak{Q}\sub\mathfrak{P}'$ and $(\mathfrak{P}')^c=\p$. Then $\mathfrak{P}'=\mathfrak{P}$, whence $\mathfrak{Q}\sub\mathfrak{P}$.
\end{proof}
\begin{lemma}\label{Noe semi-local finite extension unique lying over prop}
Suppose that $A$ is Noetherian and semi-local. Let $\a$ be a defining ideal of $A$, and $\rho:A\to B$ a ring homomorphism making $B$ a finitely generated $A$-module. Suppose that, for any prime ideal $\p$ (necessarily minimal) of $A$ such that $\dim(A/\p)=\dim(A)$, there exists a unique prime ideal $\mathfrak{P}$ of $B$ lying over $\p$ such that the canonical homomorphism $\rho_\p:A_\p\to B_{\mathfrak{P}}$ is bijective. Then $\dim_A(B)=\dim(A)$ and $e_\a^A(B)=e_\a^A(A)$.
\end{lemma}
\begin{proof}
Let $\mathscr{P}_A$ (resp. $\mathscr{P}_B$) be the set of prime ideals of $A$ such that $\dim(A/\p)=\dim(A)$ (resp. $\dim_A(B/\p^e)=\dim_A(B)$); we have $\mathscr{P}_A\neq\emp$. Let $\p\in\mathscr{P}_A$; there exists by hypothesis a unique ideal of $B$ lying over $\p$. By \cref{prime ideal is contraction of a prime iff} we have $\p^{ec}=\p$, and \cref{integral extension dimension of module ideal prop} then shows that $\dim(A/\p)=\dim(B/\p^e)=\dim_A(B/\p^e)$, whence
\[\dim_A(B)\geq\dim_A(B/\p^e)=\dim(A/\p)=\dim(A)\geq\dim_A(B).\]
This implies $\mathscr{P}_A\sub\mathscr{P}_B$ and $\dim(A)=\dim_A(B)$. Conversely, if $\p\in\mathscr{P}_B$, we have the inequalities
\[\dim_A(B/\p^e)=\dim_A(B)=\dim(A)\geq\dim(A/\p)\geq\dim_A(B/\p^e)\]
so $\p\in\mathscr{P}_B$ and therefore $\mathscr{P}_A=\mathscr{P}_B$. Now by \cref{Noe ring multiplicity localization formula} and \cref{Noe ring semi-local multiplicity formula}, we have
\[e_\a^A(A)=\sum_{\p\in\mathscr{P}_A}\ell_{A_\p}(A_\p)\cdot e_\a^A(A/\p),\quad e_\a^A(B)=\sum_{\p\in\mathscr{P}_B}\ell_{A_\p}(A_\p\otimes_AB)\cdot e_\a^A(A/\p).\]
Also, by \cref{ring extension localization at prime tensor bijective iff}, under the hypothesis on $\rho$ we have $\ell_{A_\p}(A_\p\otimes_AB)=\ell_{A_\p}(A_\p)=1$ for every $\p\in\mathscr{P}_A$, so $e_\a^A(A)=e_\a^A(B)$.
\end{proof}
\begin{proposition}\label{Noe semi-local finite extension of total fraction prop}
Suppose that $A$ is a Noetherian semi-local ring and reduced; let $\a$ be a defining ideal of $A$; let $Q(A)$ be the total ring of fraction of $A$, and $B$ a sub $A$-algebra of $Q(A)$ that is finitely generated as an $A$-module. Then $B$ is semi-local and $\a^e$ is a defining ideal of $B$. Moreover, if for any maximal ideal $\m$ of $B$ such that $\dim(B_\m)=\dim(B)$ we have $[B/\m:A/\m^c]=1$, then $e_\a^A(A)=e_{\a^e}^B(B)$. 
\end{proposition}
\begin{proof}
By Corallary~\ref{Noe semilocal ring finite extension is Noe semilocal}, $B$ is semi-local with defining ideal $\a^e$. Then $e_{\a^e}^B(B)=e_{\a}^A(B)$ by \cref{Noe semi-local ring finite extension multiplicity equal if}. Recall that $Q(A)$ can be identified with the ring $\prod_\p A_\p$, where $\p$ runs through minimal prime ideals of $A$ (\cref{Noe ring minimal prime and Ass localization prop}), so the canonical map $A_\p\to A_\p\otimes_AB$ is bijective for any minimal prime $\p$ of $A$. The proposition them follows from \cref{ring extension localization at prime tensor bijective iff} and \cref{Noe semi-local finite extension unique lying over prop}.
\end{proof}
\begin{example}
Let $k$ be a field with $\char(k)\neq 2$ and let $A$ be the local ring $k\llbracket X,Y\rrbracket/(X^2+Y^2)$, with residue field $k$. It is easy to compute that $e_{\m_A}^A(A)=2$. Put $B=k\llbracket X,T\rrbracket/(T^2+1)$ where $T=Y/X$, we distinguish two cases:
\begin{itemize}
\item[(a)] If $-1$ is a squre of an element $i$ in $k$, then $B$ has two maximal ideals generated by $\{X,T+i\}$ and $\{X,T-i\}$, respectively, with residue field $k$. We have $e_{\m_AB}^B(B)=2$. 
\item[(b)] If $-1$ is not a squre in $k$, then $B$ has a unique maximal ideal $(X)$ with residue field $k[T]/(T^2+1)$, and we have $e_{\m_AB}^B(B)=1$.
\end{itemize}
\end{example}
\subsection{Multiplicity and secant sequences}
\begin{proposition}\label{Noe local ring secant sequence multiplicity inequality}
Suppose that $A$ is a Noetherian local ring and $s\geq 1$ be an integer. For $1\leq i\leq s$, let $\delta_i$ be a positive integer, $x_i$ an element of $\m_A^{\delta_i}$, and $\xi_i$ the its class in $\m_A^{\delta_i}/\m_A^{\delta_i+1}$. Suppose that $(x_1,\dots,x_s)$ is a secant sequence for $A$ and let $\x$ be the ideal of $A$ generated by $(x_1,\dots,x_s)$. Then $e(A/\x)\geq\delta_1\cdots\delta_s e(A)$ and the equality holds if $(\xi_1,\dots,\xi_s)$ is a completely secant sequence for $\gr(A)$.
\end{proposition}
\begin{proof}
Put $B=A/\x$, and consider the formal series
\[H_A=\sum_{n=0}^{\infty}\ell_{\kappa_A}(\m_A^n/\m_A^{n+1})\cdot T^n,\quad H_B=\sum_{n=0}^{\infty}\ell_{\kappa_B}(\m_B^n/\m_B^{n+1})\cdot T^n.\]
Recall that, by \cref{Noe ring graded module quotient by sequence series relation}, we have the following inequality in $\Z\llbracket T\rrbracket$:
\begin{align}\label{Noe local ring secant sequence multiplicity inequality-1}
H_B^{(r)}\geq\prod_{i=1}^{s}(1-T^{\delta_i})H_A^{(s)}
\end{align}
and the equality holds if and only if $(\xi_1,\dots,\xi_s)$ is completely secant for $\gr(A)$. Put $d=\dim(A)$, so that $\dim(B)=d-s$ since $(x_1,\dots,x_s)$ is secant for $A$. By \cref{filtration good Hilbert-Samuel series expression as d-product}, there then exists elements $R_A$ and $R_B$ in $\Z[T,T^{-1}]$ such that
\[H_A=(1-T)^{-d}R_A,\quad H_B=(1-T)^{-d+s}R_B\]
and we have $R_A(1)=e(A)$, $R_B(1)=e(A/\x)$. Plug this into (\ref{Noe local ring secant sequence multiplicity inequality-1}), we find that
\[(1-T)^{-d}R_B(T)=H_B^{(s)}\geq\prod_{i=1}^{s}(1-T^{\delta_i})H_A^{(s)}=(1-T)^{-d}R(T)R_A(T)\]
where $R(T)$ is the polynomial in $\Z[T]$ given by
\[R(T)=\prod_{i=1}^{s}\frac{1-T^{\delta_i}}{1-T}.\]
Since $R(1)=\delta_1\cdots\delta_s$, we conclude that $e(B)\geq \delta_1\cdots\delta_s e(A)$, and the equality holds if and only if $(\xi_1,\dots,\xi_s)$ is completely secant for $\gr(A)$.
\end{proof}
\begin{remark}
We can show conversely that, if $A$ is a regular local ring and $(x_1,\dots,x_s)$ is an arbitrary sequence of elements in $\m$, then $e(A/\x)=\delta_1\cdots\delta_s$ if and only if the sequence $(\xi_1,\dots,\xi_s)$ is completely secant for $\gr(A)$. The point is that in this case we do not need to assume that $(x_1,\dots,x_s)$ is secant for $A$.
\end{remark}
\begin{example}
Let $A$ be the formal series ring $k\llbracket X_1,\dots,X_n\rrbracket$ over a field $k$; let $F_1,\dots,F_s$ be elements of $A$, $\a$ the ideal they generate, and $B=A/\a$. Let $P_1,\dots,P_s\in k[X_1,\dots,X_n]$ be the initial forms of the series $F_1,\dots,F_s$ and $\delta_1,\dots,\delta_s$ be their degrees, respectively. By \cref{Noe local ring secant sequence multiplicity inequality}, if the sequence $(F_1,\dots,F_s)$ is secant for $A$, we have $e(B)\geq\delta_1,\dots,\delta_s$, and the equality holds if and only if if the sequence $(P_1,\dots,P_s)$ is completely secant for the ring $k[X_1,\dots,X_n]$.\par
Consider for example the ring $B=k\llbracket X,Y\rrbracket/\a$, where $\a$ is generated by $X^2+Y^3$ and $X^2+Y^4$; the preceding inequality is $e(B)\geq 4$, and this inequality is strict because $(X^2,X^2)$ is not completely secant. On the other hand, note that $\a$ is also generated by $X^2+Y^3$ and $Y^4-Y^3$, for which the sequence of initial forms, namely $(X^2,Y^3)$, is completely secant, so we obtain that $e(B)=6$.
\end{example}
From now on, let $A$ be a Noetherian ring, $\a$ an ideal of $A$ contained in the Jacobson radical of $A$, and $M$ a finitely generated $A$-module such that $M/\a M$ has finite length. Recall that in this case we can talk about the multiplicity $e_\a(M)$ of $M$ relative to $\a$.
\begin{proposition}\label{Noe ring module quotient by element multiplicity prop}
Let $\delta$ be a positive integer, $x$ an element of $\a^\delta$, $\xi$ its class in $\a^\delta/\a^{\delta+1}$, and $\varphi$ the multiplication by $\xi$ on $\gr(M)$.
\begin{itemize}
\item[(a)] The dimension of the $A$-module $M/xM$ is equal to $\dim_A(M)$ or $\dim_A(M)-1$. In the second case, we have $e_{\a}(M/xM)\geq\delta e_\a(M)$.
\item[(b)] Suppose that $\dim_A(M)\geq 1$ and the kernel of $\varphi$ has finite length over $A/\a$. Then we have $\dim_A(M/xM)=\dim_A(M)-1$. Moreover:
\begin{itemize}
\item[(\rmnum{1})] If $\dim_A(M)>1$, then $e_\a(M/xM)=\delta e_\a(M)$.
\item[(\rmnum{2})] If $\dim_A(M)=1$, then for any $n\geq 0$ we have
\[n\delta e_\a(M)\leq \ell_A(M/x^nM)\leq n\delta e_\a(M)+\ell_{A/\a}(\ker\varphi^n)\]
and $\delta e_\a(M)=e_{xA}(M)\leq\ell_A(M/xM)$. 
\end{itemize}
\end{itemize}
\end{proposition}
\begin{proof}
Put $M'=M/xM$; consider the Hilbert-Samuel series $H_M=H_{M,\a}$ and $H_{M'}=H_{M',\a}$, as well as the Poincare series $P(T)=\sum_{n=0}^{\infty}\ell_{A/\a}(\ker\varphi_n)\cdot T^n$ (where $\varphi_n$ is the restriction of $\varphi$ on $\gr_n(M)$). By \cref{filtration good Hilbert-Samuel series expression as d-product}, there then exists elements $R_M$ and $R_{M'}$ in $\Z[T,T^{-1}]$ such that
\[H_M=(1-T)^{-d}R_M,\quad H_{M'}=(1-T)^{-d'}R_{M'}\]
where $d=\dim_A(M)$, $d'=\dim_A(M')$, $R_M(1)=e_\a(M)$, and $R_{M'}=e_{\a}(M')$. By \cref{filtration and graded endomorphism Hilbert-Samuel series relation}, we have the inequality
\[(1-T^\delta)H_M^{(1)}\leq H_{M'}^{(1)}\leq (1-T^\delta)H_M^{(1)}+T^\delta P^{(1)}\]
in the ring $\Z((T))$. Posing $R(T)=(1-T^\delta)/(1-T)$, this in equivalent to
\begin{align}\label{Noe ring module quotient by element multiplicity prop-1}
(1-T)^{-d}R(T)R_M(T)\leq (1-T)^{-d'-1}R_{M'}(T)\leq (1-T)^{-d}R(T)R_M(T)+(1-T)^{-1}T^\delta P(T).
\end{align}
By \cref{polynomial product by (1-T) order iff}, the first inequality implies that, either $d<d'+1$, or $d=d'+1$ and $R(1)R_M(1)\leq R_{M'}(1)$, which means $e_\a(M')\geq \delta e_\a(M)$. Since we always have $d'\leq d$, this proves (a).\par
Under the hypothesis of (b), we have $P(T)\in\Z[T]$ and $P(1)=\ell_A(\ker\varphi)$. The second inequality of (\ref{Noe ring module quotient by element multiplicity prop-1}) shows
\[(1-T)^{-d'-1}R_{M'}(T)\leq (1-T)^{-d}[R(T)R_M(T)+T^\delta(1-T)^{-d-1}P(T)].\]
Suppose that $d>1$, again by \cref{polynomial product by (1-T) order iff}, either $d'+1<d$, or $d'+1=d$ and $R_{M'}(1)=R(1)R(1)$, which means $e_\a(M')=\delta e_\a(M)$. This demonstrate the case (\rmnum{1}) in (b), in view of part (a).\par
Suppose now $d=1$. Then by the same reasoning, we have $d'=0$ and
\[R_{M'}(1)\leq R(1)R_M(1)+P(1).\]
Recall that if $d'=0$ then $R_{M'}(1)=\ell_A(M')$; therefore we conclude that
\begin{align}\label{Noe ring module quotient by element multiplicity prop-2}
\delta e_\a(M)\leq\ell_A(M/xM)\leq\delta e_\a(M)+\ell_A(\ker\varphi).
\end{align}

Since $\ker\varphi$ has finite length, $\varphi_m$ is injective for $m$ large enough, so $(\varphi^n)_m$ is also injective for $m$ large enough for each integer $n\geq 1$, which means $\ker\varphi^n$ has finite length. By replacing $x$ with $x^n$ in (\ref{Noe ring module quotient by element multiplicity prop-2}), we then get
\begin{align}\label{Noe ring module quotient by element multiplicity prop-3}
n\delta e_\a(M)\leq\ell_A(M/x^nM)\leq n\delta e_\a(M)+\ell_A(\ker\varphi^n).
\end{align}
Now the submodules $\ker\varphi^n$ of the $\gr(A)$-Noetherian module $\gr(M)$ form an increasing sequence therefore stationary and that each of them is of finite length over $A/\a$. Dividing by $n$ in inequality (\ref{Noe ring module quotient by element multiplicity prop-3}) and letting $n$ tend to $+\infty$, we find that $e_{xA}(M)=\delta e_{\a}(M)$ by definition of $e_{xA}(M)$ (\cref{filtration good index and length formula}).
\end{proof}
\begin{lemma}\label{graded ring finite module length finite iff}
Let $R$ be a Noetherian graded ring with positive degrees, $E$ a finitely generated graded $R$-module such that $E_n$ has finite length over $R_0$ for each $n\in\Z$. The following conditions are equivalent:
\begin{itemize}
\item[(\rmnum{1})] $E$ is an $R$-module of finite length;
\item[(\rmnum{2})] there exists an integer $n_0$ such that $E_n=0$ for $n\geq n_0$;
\item[(\rmnum{3})] every associated prime of $E$ contains $R_+$. 
\end{itemize} 
\end{lemma}
\begin{proof}
The equivalence of (\rmnum{1}) and (\rmnum{2}) is clear. Let $\p$ be an associated prime of $E$. If (\rmnum{3}) is satisfied, we have $\p=\p_0+R_+$, where $\p_0$ is a prime ideal of $R_0$, and the $R$-module $R/\p$ is isomomorphic to $R_0/\p_0$. By \cref{graded module ass injection of A/p}, the $R_0$-module $R_0/\p_0$ is isomomorphic to a submodule of one of the $E_k$, hence has finite length. Concequently, $\p_0$ is a maximal ideal of $R_0$, and $\p$ is then maximal. Since $\p$ is arbitrary, it follows from \cref{associated prime maximal iff finite length} that $E$ has finite length.\par
Conversely, assume that $E$ has finite length and let $\p$ be an associated prime of $E$. Then $\p$ is homogeneous by \cref{graded ring ass is homogeneous} and maximal by \cref{associated prime maximal iff finite length}, hence contains $R_+$ (\cref{graded ring maximal iff maximal homogeneous}).
\end{proof}
\begin{proposition}\label{Noe ring gr module homothety injective for large iff}
Let $\p_1,\dots,\p_r$ be the prime ideals of $\gr(A)$ associated with the graded module $\gr(M)$ and not containing $\gr_1(A)$. Let $\delta>0$ be an integer, $\xi$ an element of $\gr_\delta(A)$, and $\varphi$ the homothety of $\gr(M)$ with ratio $\xi$. Then for $\varphi_n$ to be injective for all $n$ large enough, it is necessary and sufficient that $\xi$ does not belong to any of the $\p_i$.
\end{proposition}
\begin{proof}
In fact, the prime ideals associated with the $\gr(A)$-module $\ker\varphi$ is the primes associated with $\gr(M)$ and contain $\xi$. By \cref{graded ring finite module length finite iff}, $(\ker\varphi)_n$ is zero for $n$ large enough, if and only if these prime ideals contain $\gr_+(A)$ (or equivalently they contain $\gr_1(A)$), whence the proposition.
\end{proof}
An element $x$ of $A$ is said to be \textbf{superficial} for $M$ relative to $\a$ if it is contained in $\a$ and if, for $n$ large enough, the map $\a^nM/\a^{n+1}M\to\a^{n+1}M/\a^{n+2}M$ induced by the multiplication by $x$ is injective. If moreover $x\in\a^\delta$, then $x$ is said to be superficial of order $\delta$ for $M$. In the notation of \cref{Noe ring gr module homothety injective for large iff}, we see $x$ is superficial of order $\delta$ for $M$ if its class $\xi$ in $\gr(A)$ belongs to $\gr_\delta(A)$ and is not contained in the $\p_i$.\par
By \cref{graded ring prime avoidence}, there exists a homogeneous element of $\gr(A)$ with positive degree that does not belong to any $\p_i$. Concequently, there exists an integer $\delta>0$ and an element of $A$ that is superficial of order $\delta$ for $M$.
\begin{remark}
We note that, by \cref{graded ring finite module length finite iff}, an element $x$ of $A$ satisfies the hypothesis in \cref{Noe ring module quotient by element multiplicity prop} if and only if it is superficial for $M$ relative to $\a$.
\end{remark}
\begin{example}\label{Noe local ring residue infinite superficial element exist}
Suppose that $A$ is local with maximal ideal $\m$ and residue field $k$, and consider the canonical surjection $\pi:\a\to\a\otimes_Ak$. Since $\a\otimes_Ak=\a/\a\m$ and $\a\sub\m$, we see $\pi$ factors through the canonical map $\a\to\a/\a^2$:
\[\begin{tikzcd}
\a\ar[r]&\a/\a^2\ar[r,"\bar{\pi}"]&\a\otimes_Ak
\end{tikzcd}\]
By Nakayama lemma, the subspace $V_i=\bar{\pi}(\p_i\cap(\a/\a^2))$ of $\a\otimes_Ak$ is proper. If $\alpha\in\a\otimes_Ak$ is not contained in any of the $V_i$, then $\pi^{-1}(\alpha)$ consists of superficial elements for $M$ by \cref{Noe ring gr module homothety injective for large iff}. In particular, if $k$ is infinite, the union of $V_i$ is distinct from $\a\otimes_Ak$ so there exists a superficial element for $M$.
\end{example}
\begin{theorem}\label{Noe ring superficial sequence prop}
Let $A$ be a Noetherian ring, $\a$ an ideal of $A$ contained in the Jacobson radical of $A$ and $M$ a finitely generated $A$-module such that $M/\a M$ is of finite length. Let $x_1,\dots,x_r$ be elements of $\a$ and $\x$ the ideal they generate.
\begin{itemize}
\item[(a)] We have $\dim_A(M/\x M)\geq\dim_A(M)-r$.
\item[(b)] If $\dim_A(M/\x M)=\dim_A(M)-r$, then $e_\a(M/\x M)\geq e_\a(M)$.
\item[(c)] If $r<\dim_A(M)$ and for each $i$ the element $x_i$ is superficial for $M/(x_1M+\cdots+x_{i-1}M)$ relative to $\a$, then
\[\dim_A(M/\x M)=\dim_A(M)-1,\quad e_\a(M/\x M)=e_\a(M).\]
\item[(d)] If $r=\dim_A(M)$ and for each $i$ the element $x_i$ is superficial for $M/(x_1M+\cdots+x_{i-1}M)$ relative to $\a$, then
\[e_\a(M)=e_{\x}(M)\leq\ell_A(M/\x M)<+\infty.\] 
\end{itemize}
\end{theorem}
\begin{proof}
The assertions in (a), (b), and (c) for $r=1$ follows from \cref{Noe ring module quotient by element multiplicity prop}, and the general case follows by induction. Now assume the hypothesis in (d) and put $\x'=Ax_1+\cdots+Ax_{r-1}$, $M'=M/\x'M$; then $M/\x M$ is identified with $M'/x_rM'$. By part (c), we have $\dim_A(M')=1$ and $e_\a(M)=e_\a(M')$; \cref{Noe ring module quotient by element multiplicity prop} also implies that $M/\x M$ has finite length and
\[e_\a(M')=e_{x_rA}(M')\leq\ell_A(M'/\x_rM')=\ell_A(M/\x M).\]
On the other hand, since $x_r^nM'=\x^nM'$ for each $n$, we have $e_{x_rA}(M')=e_{\x}(M')$. By applying (b) to $x_1,\dots,x_{r-1}$, we also obtain that $e_{\x}(M')\geq e_\x(M)$, so
\[e_\x(M)\leq e_\x(M')=e_{x_rA}(M')=e_{\a}(M')=e_{\a}(M).\]
Since $\x$ is contained in $\a$, this implies $e_{\x}(M)=e_{\a}(M)$ and completes the proof.
\end{proof}
\begin{corollary}\label{Noe local ring residue infinite superficial sequence}
Suppose that $A$ is local, with residue field infinite, and put $d=\dim_A(M)$. Then there exists a sequence $(x_1,\dots,x_d)$ of elements in $\a$ such that, if $\x$ is the ideal they generated, then
\[e_\a(M)=e_{\x}(M)\leq\ell_A(M/\x M)<+\infty\]
\end{corollary}
\begin{proof}
This follows from \cref{Noe ring superficial sequence prop} and \cref{Noe local ring residue infinite superficial element exist}.
\end{proof}
\begin{remark}
In the situation of \cref{Noe local ring residue infinite superficial sequence}, we have
\[e_\a(M)=e_{\x}(M)\leq\ell_A(M/\x M),\quad \ell_A(M/\a M)\leq\ell_A(M/\x M).\]
The three cases
\[e_\a(M)<\ell_A(M/\a M),\quad e_\a(M)=\ell_A(M/\a M),\quad e_\a(M)>\ell_A(M/\a M)\]
are all possible. 
\end{remark}
\chapter{Complete Noetherian local rings}
In this section, all rings are assumed to be commutative with unit. We denote by $1_A$ for the unit of a ring $A$. If $\p$ a prime ideal of $A$, we denote by $\kappa(\p)$ the residue field of $A_\p$. If $A$ is local, $\m_A$ will denote its maximal ideal and $\kappa_A$ its residue field.\par
We say a ring homomorphism $\rho:A\to B$ is flat (resp. faithfully flat) if $B$ is a flat $A$-module (resp. faithfully flat $A$-module). Recall that by \cref{ring faithfully flat iff}, if $A$ and $B$ are local rings, $\rho$ is faithfully flat if and only if it is local and flat.
\section{Witt vectors and Witt rings}
\subsection{Witt polynomials}
In this part, we fix a prime number $p$. For a positive integer $n$, we define the \textbf{$\bm{n}$-th Witt polynomial} $\Phi_n$ to be
\begin{align}\label{Witt polynomial n-th def}
\Phi_n(X_0,\dots,X_n)=\sum_{i=0}^{n}p^iX_i^{p^{n-i}}=X_0^{p^n}+pX_1^{p^{n-1}}+\cdots+p^nX_n.
\end{align}
Clearly we have $\Phi_0=X_0$ and the following recursive formulae
\begin{align}
\Phi_{n+1}(X_0,\dots,X_{n+1})&=\Phi_n(X_0^p,\dots,X_n^p)+p^{n+1}X_{n+1},\label{Witt polynomial recursion formula-1}\\
\Phi_{n+1}(X_0,\dots,X_{n+1})&=X_0^{p^{n+1}}+p\Phi_n(X_1,\dots,X_{n+1}).\label{Witt polynomial recursion formula-2}
\end{align}
If we assign $X_i$ with weight $p^i$, the polynomial $\Phi_n$ is isobare with weight $p^n$.
\begin{lemma}\label{filtered ring p power lemma}
Let $A$ be a filtered ring with $(A_n)_{n\in\Z}$ its filtration. Let $x,y$ be elements of $A$ congruent mod $A_m$, then
\begin{align}\label{Witt polynomial filtered ring congruence prop-2}
x^{p^n}\equiv y^{p^n}\mod A_{m+n}.
\end{align}
\end{lemma}
\begin{proof}
Let $P=\sum_{i=0}^{p-1}X^iY^{p-1-i}$ in $\Z[X,Y]$. From the hypothesis on $x$ and $y$, we have
\[P(x,y)\equiv P(x,x)\equiv px^{p-1}\mod A_m.\]
But we have $A_m+pA\sub A_1$, from which $P(x,y)\in A_1$. Finally, $x^p-y^p=(x-y)P(x,y)$ is contained in $A_mA_1\sub A_{m+1}$, which complete the induction process.
\end{proof}
\begin{proposition}\label{Witt polynomial filtered ring congruence prop}
Let $A$ be a filtered ring with $(A_n)_{n\in\Z}$ its filtration. Suppose that $A_0=A$ and $p\cdot 1\in A_1$. Let $a_0,\dots,a_n$ and $b_0,\dots,b_n$ be elements of $A$, and $m\geq 1$ be an integer. If $a_i\equiv b_i$ mod $A_m$ for each $i$, then
\begin{align}\label{Witt polynomial filtered ring congruence prop-1}
\Phi_i(a_0,\dots,a_i)\equiv\Phi_i(b_0,\dots,b_i)\mod A_{m+i}\for 0\leq i\leq n.
\end{align}
Moreover, the converse holds if for each integer $k\geq 1$ and $x\in A$, the relation $px\in A_{k+1}$ implies $x\in A_k$.
\end{proposition}
\begin{proof}
We prove (\ref{Witt polynomial filtered ring congruence prop-1}) by induction on $n$. The case $n=0$ is immediate, so suppose $n\geq 1$. Apply (\ref{Witt polynomial filtered ring congruence prop-2}) on $a_i$ and $b_i$, we get
\begin{align}\label{Witt polynomial filtered ring congruence prop-3}
a_i^p\equiv b_i^p\mod A_{m+1}\for 0\leq i\leq n-1,
\end{align}
also, by the induction hypothesis on $(a_0,\dots,a_{n-1})$ and $(b_0,\dots,b_{n-1})$,
\begin{align}\label{Witt polynomial filtered ring congruence prop-4}
\Phi_{n-1}(a_0^p,\dots,a_{n-1}^p)\equiv\Phi_{n-1}(b_0^p,\dots,b_{n-1}^p)\mod A_{m+n}.
\end{align}
Therefore, from (\ref{Witt polynomial recursion formula-1}) we conclude that
\begin{align}\label{Witt polynomial filtered ring congruence prop-5}
\Phi_n(a_0,\dots,a_n)-p^na_n\equiv\Phi_n(b_0,\dots,b_n)-p^nb_n\mod A_{m+n}.
\end{align}
Now $a_n-b_n$ is contained in $A_m$, the element $p^na_n-p^nb_n$ then belongs to $A_{m+n}$ and this implies the congruence
\[\Phi_n(a_0,\dots,a_n)\equiv\Phi_n(b_0,\dots,b_n)\mod A_{m+n}.\]

Conversely, assume that for each integer $k\geq 1$ and $x\in A$, the relation $px\in A_{k+1}$ implies $x\in A_k$. We prove the inverse direction by induction on $n$. The case $n=0$ is clear, and suppose that $n\geq 1$. Since $a_i\equiv b_i$ mod $A_m$ for $0\leq i\leq n-1$, by the induction hypothesis, we deduce the congruence (\ref{Witt polynomial filtered ring congruence prop-3}), (\ref{Witt polynomial filtered ring congruence prop-4}), and (\ref{Witt polynomial filtered ring congruence prop-5}). But by hypothesis $\Phi_n(a_0,\dots,a_n)\equiv\Phi_n(b_0,\dots,b_n)$ mod $A_{m+n}$, so $p^n(a_n-b_n)\in A_{m+n}$. Since $px\in A_{k+1}$ implies $x\in A_k$, we conclude that $a_n\equiv b_n$ mod $A_m$, which completes the proof.
\end{proof}
Let $A$ be a ring, and give $A^{\N}$ the structure of product ring. We define two maps $f_A$ and $v_A$ by the following formulae
\[f_A(\bm{a})=(a_1,a_2,\dots),\quad v_A(\bm{a})=(0,pa_0,pa_1,\dots).\]
where $\bm{a}=(a_n)_{n\in\N}$. For any positive integer $m$, let $\Phi_m$ be the map on $A^\N$ defined by $\bm{a}=(a_n)_{n\in\N}\mapsto\Phi_m(a_0,\dots,a_m)$. We denote by $\Phi_A$ the map $\bm{a}\mapsto(\Phi_n(\bm{a}))_{n\in\N}$ on $A^\N$.
\begin{lemma}\label{Witt polynomial image lifting prop}
Suppose that $A$ is endowed with an endomorphism $\sigma$ satisfying $\sigma(a)=a^p$ mod $pA$ for any $a\in A$. Let $a_0,\dots,a_{n-1}$ be elemnts of $A$. Put $u_i=\Phi_i(a_0,\dots,a_i)$ for each $i$, and let $u_n$ be an element of $A$. Then the following conditions are equivalent:
\begin{itemize}
\item[(\rmnum{1})] There exists $a_n\in A$ such that $u_n=\Phi_n(a_0,\dots,a_n)$.
\item[(\rmnum{2})] We have $\sigma(u_{n-1})\equiv u_n$ mod $p^nA$. 
\end{itemize}
\end{lemma}
\begin{proof}
For $0\leq i\leq n-1$, we have $\sigma(a_i)\equiv a_i^p$ mod $pA$. Applying \cref{Witt polynomial filtered ring congruence prop} on the filtration $(p^nA)_{n\in\N}$ and $m=1$, we have a congruence
\[\sigma(u_{n-1})=\Phi_{n-1}(\sigma(a_0),\dots,\sigma(a_{n-1}))\equiv\Phi_{n-1}(a_0^p,\dots,a_0^p)\mod p^nA.\]
But, from formula (\ref{Witt polynomial recursion formula-1}), the relation $u_n=\Phi_n(a_0,\dots,a_n)$ is equivalent to
\[u_n=\Phi_{n-1}(a_0^p,\dots,a_{n-1}^p)+p^na_n.\]
Thus the lemma is proved.
\end{proof}
\begin{proposition}\label{Witt polynomial map Phi prop if p}
Let $A$ be a ring and $f_A$, $v_A$, and $\Phi_A$ be the maps defined above.
\begin{itemize}
\item[(a)] If $p$ is not a divisor of zero in $A$, the map $\Phi_A$ is injective.
\item[(b)] If $p$ is invertible in $A$, the map $\Phi_A$ is bijective.
\item[(c)] If $\sigma$ is an endomorphism of $A$ satisfying $\sigma(a)=a^p$ mod $pA$ for any $a\in A$, the image of $\Phi_A$ is a subring of $A^\N$ invariant under $f_A$ and $v_A$, which consists of elements $(u_n)_{n\in\A}$ such that $\sigma(u_n)\equiv u_{n+1}$ mod $p^{n+1}A$ for all $n\in\N$.  
\end{itemize}
\end{proposition}
\begin{proof}
If $\bm{a}=(a_n)_{n\in\N}$ and $\bm{u}=(u_n)_{n\in\N}$ are elements of $A^\N$, the relation $\Phi_A(\bm{a})=\bm{u}$ is equivalent, by formula (\ref{Witt polynomial recursion formula-2}), to
\begin{equation}\label{Witt polynomial map Phi image equation}
\begin{cases}
u_0=a_0,\\
u_n=\Phi_{n-1}(a_0^p,\dots,a_{n-1}^p)+p^na_n\for n\geq 1.
\end{cases}
\end{equation}
Let $\bm{u}=(u_n)_{n\in\N}$ be in $A^\N$. If $p$ is not a zero divisor of $A$ (resp. if $p$ is invertible in $A$), there then exists at most (resp. exactly) one sequence $(a_n)_{n\in\N}$ satisfying (\ref{Witt polynomial map Phi image equation}), whence (a) and (b).\par
Now by \cref{Witt polynomial image lifting prop}, the image of $\Phi_A$ consists of elements $\bm{u}=(u_n)_{n\in\N}$ of $A^\N$ such that $\sigma(u_n)=u_{n+1}$ mod $p^{n+1}A$ for all $n\in\N$. Since $\sigma$ is an endomorphism, it follows that this is a subring of $A^\N$ and invariant under $l_A$ and $v_A$.
\end{proof}
\begin{remark}
Let $\bm{a}=(a_n)_{n\in\N}$ and $\bm{u}=(u_n)_{n\in\N}$ be elements of $A^\N$ such that $\Phi_A(\bm{a})=\bm{u}$. Then by (\ref{Witt polynomial map Phi image equation}), we deduce the following assertions:
\begin{itemize}
\item[(a)] If the $u_n$, for $0\leq n\leq m$, is contained in a subring $B$ of $A$ and if, for any $x\in A$, the relation $px\in B$ implies $x\in B$, then $a_n$, for $0\leq n\leq m$, are contained in $B$.
\item[(b)] If $A$ is graded of type $\N$, $p$ is not a zero divisor of $A$, and $u_n$ is homogeneous of degree $dp^n$ for $0\leq n\leq m$ (where $d\in\N$), then $a_n$ is homogeneous of degree $dp^n$ for $0\leq n\leq m$.
\end{itemize}
\end{remark}
Now let $A$ be the polynomial ring $\Z[\bm{X},\bm{Y}]$ with indeterminates $\bm{X}=(X_n)_{n\in\N}$ and $\bm{Y}=(Y_n)_{n\in\N}$. Let $\theta$ be the endomorphism of $A$ defined by $\theta(X_n)=X_n^p$ and $\theta(Y_n)=Y_n^p$ for $n\in\N$. Then $p$ is not a divisor of zero in $A$ and the set of $a\in A$ such that $\theta(a)=a^p$ mod $pA$ is a subring of $A$ containing $X_n$ and $Y_n$, hence equal to $A$. By \cref{Witt polynomial map Phi prop if p}(a) and (c), there exist elements $\bm{S}=(S_n)_{n\in\N}$, $\bm{P}=(P_n)_{n\in\N}$, $\bm{I}=(I_n)_{n\in\N}$, and $\bm{F}=(F_n)_{n\in\N}$ in $A^\N$ characterized by the equalities
\begin{equation}\label{Witt polynomial map on Z[X,Y] SPIL}
\begin{cases}
\Phi_A(\bm{S})=\Phi_A(\bm{X})+\Phi_A(\bm{Y})\\
\Phi_A(\bm{P})=\Phi_A(\bm{X})\Phi_A(\bm{Y})\\
\Phi_A(\bm{I})=-\Phi_A(\bm{X})\\
\Phi_A(\bm{F})=f_A(\Phi_A(\bm{X})).
\end{cases}
\end{equation}
The elements $S_n$, $P_n$, $I_n$, and $F_n$ of $A$ are then characterized by the following formulae (where $n\in\N$):
\begin{align}
\Phi_n(S_0,\dots,S_n)&=\Phi_n(X_0,\dots,X_n)+\Phi_n(Y_0,\dots,Y_n),\label{Witt polynomial map on Z[X,Y] S_n}\\
\Phi_n(P_0,\dots,P_n)&=\Phi_n(X_0,\dots,X_n)\Phi_n(Y_0,\dots,Y_n),\label{Witt polynomial map on Z[X,Y] P_n}\\
\Phi_n(I_0,\dots,I_n)&=-\Phi_n(X_0,\dots,X_n),\label{Witt polynomial map on Z[X,Y] I_n}\\
\Phi_n(F_0,\dots,F_n)&=\Phi_{n+1}(X_0,\dots,X_{n+1}).\label{Witt polynomial map on Z[X,Y] F_n}
\end{align}
Formula (\ref{Witt polynomial recursion formula-2}) makes it possible in practice to determine the polynomials $S_n$, $P_n$, $I_n$ and $F_n$ step by step.
\begin{example}[\textbf{First terms of the polynomials $S_n$, $P_n$, $I_n$, and $F_n$}]\label{Witt ring SPIF first terms}
\mbox{}
\begin{itemize}
\item[(a)] We have
\[S_0=X_0+Y_0,\quad S_1=X_1+Y_1-\sum_{i=1}^{p-1}\frac{1}{p}\binom{p}{i}X_0Y_0^{p-i}.\]
Furthermore, $S_n-X_n-Y_n$ is in the ring $\Z[X_0,\dots,X_{n-1},Y_0,\dots,Y_{n-1}]$.
\item[(b)] We have
\[P_0=X_0Y_0,\quad P_1=pX_1Y_1+X_0^pY_1+X_1Y_0^p.\] 
\item[(c)] If $p\neq 2$, we have $I_n=-X_n$. If $p=2$, then
\[I_0=-X_0,\quad I_1=-(X_0^2+X_1),\quad I_2=-X_0^4-X_0^2X_1-X_1^2-X_2.\] 
\item[(d)] We have
\[F_0=X_0^p+pX_1,\quad F_1=X_1^p+pX_2-\sum_{i=0}^{p-1}\binom{p}{i}p^{p-i-1}X_0^{pi}X_1^{p-i}.\]
Note that $\Phi_n(F_0,\dots,F_n)\equiv\Phi_n(X_0^p,\dots,X_n^p)$ mod $p^{n+1}A$ for each $n\in\N$ since $F_n\equiv X_n^p$ mod $pA$ for each $n\in\N$. 
\end{itemize}
\end{example}
\begin{remark}\label{Witt vector for divisor-stable set}
A subset $J\sub\N_+$ is called \textbf{divisor-stable} provided that $J\neq\emp$, and if $n\in J$, then all proper divisors of $n$ are also in $J$. If $J$ is a divisor-stable set, we let $\p(J)$ denote the set of prime numbers contained in $J$. For any element $j$ in $J$, define the polynomial $\varphi_j$ of $\Z[(X_j)_{j\in J}]$ by the formula
\[\varphi_j=\sum_{d\mid j}dX_d^{j/d}.\]
Note that we have, for each integer $n\geq 0$, that
\[\varphi_{p^n}=\Phi_n(X_{p^0},\dots,X_{p^n}).\]
For any ring $A$ and any $n\in J$, we denote by $\varphi_n$ the map from $A^J$ to $A$ given by $(a_j)_{j\in J}\mapsto\varphi_n((a_j)_{j\in J})$; we denote by $\varphi_A$, or simply $\varphi$, the map $A^J$ to $A$ defined by $\bm{a}=(a_j)_{j\in J}\mapsto(\varphi_n(\bm{a}))_{n\in J}$. Let $\mathcal{A}=\Z[(X_j)_{j\in J},(Y_j)_{J\in J}]$ be the polynomial ring of indeterminates $\bm{X}=(X_j)_{j\in J}$ and $\bm{Y}=(Y_j)_{j\in J}$. We can show that there exists elements $\bm{s}=(s_j)_{j\in J}$, $\bm{p}=(p_j)_{j\in J}$ and $\bm{i}=(i_j)_{j\in J}$ characterized by the following equalities:
\begin{align*}
\varphi_{\mathcal{A}}(\bm{s})&=\varphi_{\mathcal{A}}(\bm{X})+\varphi_{\mathcal{A}}(\bm{Y})\\
\varphi_{\mathcal{A}}(\bm{p})&=\varphi_{\mathcal{A}}(\bm{X})\varphi_{\mathcal{A}}(\bm{Y})\\
\varphi_{\mathcal{A}}(\bm{i})&=-\varphi_{\mathcal{A}}(\bm{X}).
\end{align*}
In particular, this definition works, for example, if $J$ is the set of integers $j\geq 1$.
\end{remark}
\subsection{The ring of Witt vectors}
Let $A$ be a ring. If $\bm{a}=(a_n)_{n\in\N}$ and $\bm{b}=(b_n)_{n\in\N}$ are elements of $A^\N$, we denote by $S_A(\bm{a},\bm{b})$, or simply $S(\bm{a},\bm{b})$, the sequence $(S_n(a_0,\dots,a_n,b_0,\dots,b_n))_{n\in\N}$. Simialrly, we denote by $P_A(\bm{a},\bm{b})$ and $I_A(\bm{a})$ the resulting substitutions. By substituting $X_n$ into $a_n$, $Y_n$ into $b_n$, for each $n\in\N$, in the formulae (\ref{Witt polynomial map on Z[X,Y] S_n}), (\ref{Witt polynomial map on Z[X,Y] P_n}), and (\ref{Witt polynomial map on Z[X,Y] I_n}), we obtain the equalities
\begin{align}
\Phi_A(S_A(\bm{a},\bm{b}))&=\Phi_A(\bm{a})+\Phi_A(\bm{b})\label{Witt ring S_n}\\
\Phi_A(P_A(\bm{a},\bm{b}))&=\Phi_A(\bm{a})\Phi_A(\bm{b})\label{Witt ring P_n}\\
\Phi_A(I_A(\bm{a}))&=-\Phi_A(\bm{a}).\label{Witt ring I_n}
\end{align}
We will denote by $W(A)$ the set $A^{\N}$ endowed with the laws of composition $S_A$ and $P_A$.\par
Let $\rho:A\to B$ be a ring homomorphism. We denote $W(\rho)$ the map $\rho^\N:A^\N\to B^\N$ which sends an element $\bm{a}=(a_n)_{n\in\N}$ to $(\rho(a_n))_{n\in\N}$. From the definition, we have
\begin{align}
W(\rho)\circ S_A&=S_B\circ(W(\rho)\times W(\rho))\label{Witt ring induced map S_n}\\
W(\rho)\circ P_A&=P_B\circ(W(\rho)\times W(\rho))\label{Witt ring induced map P_n}\\
W(\rho)\circ I_A&=I_B\circ W(\rho)\label{Witt ring induced map I_n}\\
W(\rho)\circ\Phi_A&=\Phi_B\circ W(\rho)\label{Witt ring induced map Phi_n}
\end{align}
\begin{lemma}\label{Witt ring surjective lift}
Let $A$ be a ring. Then there exists a surjective homomorphism $\rho:B\to A$, where $B$ is a ring satisfying the following condition: $p$ is not a divisor of zero in $B$, and there exists an endomorphism $\sigma$ of $B$ such that $\sigma(b)\equiv b^p$ mod $pB$ for any $b\in B$.
\end{lemma}
\begin{proof}
It suffices to take $B=\Z[(X_a)_{a\in A}]$, and let $\sigma$ be the endomorphism on $B$ defined by $\sigma(X_a)=X_a^p$ for any $a\in A$. The homomorphism $\rho:B\to A$ is defined to send $X_a$ to $a$ for any $a\in A$.
\end{proof}
\begin{theorem}\label{Witt ring structure existence thm}
Let $A$ be a ring.
\begin{itemize}
\item[(a)] With the map $S_A$ and $P_A$, the set $W(A)$ is a commutative ring. The addition identity is the sequence $\bm{0}_A$ with zero terms, and the multiplication identity is the sequence $\mathbf{1}_A$ of which all the terms are zero except that of index $0$ which is equal to $1$. The additive inverse of an element $\bm{a}$ in $W(A)$ is $I_A(\bm{a})$. 
\item[(b)] Let $\rho:A\to B$ be a homomorphism of rings. Then the induced map $W(\rho):W(A)\to W(B)$ is a homomorphism of rings.
\item[(c)] The map $\Phi_A$ is a homomorphism from $W(A)$ to the product ring $A^\N$. In particular, for each $n\in\N$, the map $\Phi_n:\bm{a}\mapsto\Phi_n(a_0,\dots,a_n)$ is a homomorphism from $W(A)$ to $A$.
\end{itemize}
\end{theorem}
\begin{proof}
Concerning the formulae (\ref{Witt ring S_n}), (\ref{Witt ring P_n}), (\ref{Witt ring induced map S_n}), and (\ref{Witt ring induced map Phi_n}), it suffices to prove (a). Let $\rho:B\to A$ be a ring homomorphism satisfying the conditions in \cref{Witt ring surjective lift}. Let $B'$ be the image of $\Phi_B$ in $B^\N$, which by \cref{Witt polynomial map Phi prop if p} consists of elements $(b_n)_{n\in\N}$ such that $\sigma(b_n)\equiv b_{n+1}$ mod $p^{n+1}B$ for all $n\in\N$. By \cref{Witt polynomial map Phi prop if p}, $\Phi_B$ induces a bijection between $W(B)$ and $B'$. In view of the formulas (\ref{Witt ring S_n}) to (\ref{Witt ring I_n}) and the relations $\Phi_n(\bm{0}_B)=0$ and $\Phi_n(\mathbf{1}_B)=1$, we get a ring structure on $W(B)$ by transporting structure, with addition identity $\bm{0}_B$, multiplication identity $\mathbf{1}_B$, and additive inverse $I_B(\bm{b})$.\par
The map $W(\rho):W(B)\to W(A)$ is surjective (since $\rho$ is surjective). By the fomulae (\ref{Witt ring induced map S_n}) and (\ref{Witt ring induced map P_n}), the equivalence relation $R$ on $W(B)$ associated with the map $W(\phi)$ is compatible with the ring structure on $W(B)$. Thus $W(\rho)$ is a bijection from the quotient ring $W(B)/R$ to $W(A)$, compatible with addition and multiplication. Assertion (a) then follows by transporting structure.
\end{proof}
For a ring $A$, the ring $W(A)$ is called the \textbf{Witt ring} with coefficients in $A$. For $\bm{a}$ in $W(A)$ and $n\in\N$, the element $\Phi_n(\bm{a})=\Phi_n(a_0,\dots,a_n)$ is called the \textbf{phantom components} of order $n$ of $\bm{a}$.
\begin{remark}
Retain the notations in \cref{Witt vector for divisor-stable set}. Let $A$ be a ring. If $\bm{a}$ and $\bm{b}$ and elements of $A^J$ and $\bm{r}=(r_j)_{j\in J}$ belongs to $\mathcal{A}^J$, we denote by $\bm{r}_A(\bm{a},\bm{b})$ the element $(r_j(\bm{a},\bm{b}))_{j\in J}$ of $A^J$. Let $U(A)$ be the set $A^J$ endowed with the maps $\bm{s}_A$ and $\bm{p}_A$. WE can show that, with the addition $\bm{s}_A$ and multiplication $\bm{p}_A$, $U(A)$ is a commutative ring; we call it the \textbf{universal Witt ring} of $A$. The addition idnetity of $U(A)$ is the element with components all zero, and the multiplication identity is the element with components zero except $1_A$ at index $1$; the additive inverse of $\bm{a}$ in $U(A)$ is $\bm{i}_A(\bm{a})$. The map $\varphi_A$ is a homomorphism of $U(A)$ to the product ring $A^J$.
\end{remark}
From now on, we use $+$ and $\times$ to denote the addition and multiplication in the ring $W(A)$; for simplicity, we write $\bm{0}$ for $\bm{0}_A$ and $\mathbf{1}$ for $\mathbf{1}_A$. Let us introduce\footnote{The letter $F$ is the initial of the name of Frobenius, and the latter $V$ is that of the German word \textit{Verschiebung}, which means shift.} two maps $F_A$ and $V_A$ on $W(A)$ by the formulae
\begin{align}
F_A(\bm{a})&=(F_n(a_0,\dots,a_{n+1}))_{n\in\N},\label{Witt ring map F}\\
V_A(\bm{a})&=(0,a_0,a_1,\dots).\label{Witt ring map V}
\end{align}
where $\bm{a}=(a_n)_{n\in\N}$. The map $V_A$ is often called the \textbf{shift} operator. Note that by (\ref{Witt polynomial map on Z[X,Y] F_n}) we have the formula
\[\Phi_n(F_0(\bm{a}),\dots,F_n(\bm{a}))=\Phi_{n+1}(a_0,\dots,a_{n+1}).\]
We can also write it in the form
\begin{align}\label{Witt ring Phi circ F}
\Phi_A\circ F_A=f_A\circ\Phi_A.
\end{align}
From the recursive formula (\ref{Witt polynomial recursion formula-2}), we can also prove that
\begin{align}\label{Witt ring Phi circ V}
\Phi_A\circ V_A=v_A\circ\Phi_A.
\end{align}
Let $\rho:A\to B$ be a ring homomorphism. It is clear that we have
\begin{align}
W(\rho)\circ F_A&=F_B\circ W(\rho),\label{Witt ring F and induced map}\\
W(\rho)\circ V_A&=V_B\circ W(\rho).\label{Witt ring V and induced map}
\end{align}
\begin{proposition}\label{Witt ring F and V map prop}
Let $A$ be a ring and $W(A)$ be the Witt ring with coefficients $A$.
\begin{itemize}
\item[(a)] The map $F_A$ is an endomorphism of the ring $W(A)$, and $V_A$ is an endomorphism of the additive group of $W(A)$.
\item[(b)] For $\bm{a}\in W(A)$, we have $F_A(V_A(\bm{a}))=p\cdot\bm{a}$.
\item[(c)] Let $\bm{a}$ and $\bm{b}$ be elements in $W(A)$. Then we have
\begin{align}
V_A(\bm{a}\times F_A(\bm{b}))&=V_A(\bm{a})\times\bm{b},\label{Witt ring F and V map with prod-1}\\
V_A(\bm{a})\times V_A(\bm{b})&=p\cdot V_A(\bm{a}\times\bm{b}).\label{Witt ring F and V map with prod-2}
\end{align}
\item[(d)] Put $\bm{\mu}=V_A(\mathbf{1})=(0,1,0,\dots)$. Then for $\bm{b}$ in $W(A)$ we have
\begin{align}\label{Witt ring F and V map VF}
V_A(F_A(\bm{b}))=\bm{\mu}\times\bm{b}.
\end{align}
\item[(e)] For any element $\bm{a}$ in $W(A)$, denote by $\bm{a}^{p}$ the $p$-th power of $\bm{a}$ in $W(A)$. Then
\begin{align}\label{Witt ring F map Frobenius like}
F_A(\bm{a})\equiv\bm{a}^p \mod pW(A).
\end{align}
\end{itemize}
\end{proposition}
\begin{proof}
Let $\rho:B\to A$ be a ring homomorphism satisfying the conditions of \cref{Witt ring surjective lift}. Then $W(\rho):W(B)\to W(A)$ is a surjective ring homomorphism, and $\Phi_B:W(B)\to B^\N$ is an injective homomorphism of rings. Furthermore, $f_B:B^\N\to B^\N$ is a ring homomorphism. By the formulae (\ref{Witt ring Phi circ F}) and (\ref{Witt ring F and induced map}), we have
\[\Phi_B\circ F_B=f_B\circ\Phi_B,\quad W(\rho)\circ F_B=F_A\circ W(\rho),\]
which proves the first assertion of (a). The second one can be proved similarly, using (\ref{Witt ring Phi circ V}) and (\ref{Witt ring V and induced map}).\par
Let $\bm{a}$ be an element of $W(A)$, and choose a element $\bm{x}$ in $W(B)$ whose image under $W(\rho)$ is $\bm{a}$. Put $\xi=\Phi_B(\bm{x})$. By definition of $f_B$ and $v_B$, we have $f_B(v_B(\xi))=p\cdot\xi$. By formulae (\ref{Witt ring Phi circ F}) and (\ref{Witt ring Phi circ V}), the elements $F_B(V_B(\bm{x}))$ and $p\cdot\bm{x}$ of $W(B)$ then have the same image under the injective map $\Phi_B$, and therefore are equal. The formula $F_A(V_A(\bm{a}))=p\cdot\bm{a}$ follows then from (\ref{Witt ring F and induced map}) and (\ref{Witt ring V and induced map}). This proves (b).\par
Reasoning analogously, we reduce the demonstration of formula (\ref{Witt ring F and V map with prod-1}) to the relation
\[v_B(\xi f_B(\eta))=v_B(\xi)\eta\]
where $\xi,\eta$ are in $B^\N$. This follows from the equalities
\[\xi f_B(\eta)=(\xi_0\eta_1,\xi_1\eta_2,\dots),\quad v_B(\xi)\eta=(0,p\xi_0\eta_1,p\xi_1\eta_2,\dots).\]
Taking into account (a) and (b), the formula (\ref{Witt ring F and V map with prod-2}) results from the formula (\ref{Witt ring F and V map with prod-1}), where we replace $\bm{b}$ by $V_A(\bm{b})$. Formula (\ref{Witt ring F and V map VF}) is the particular case $\bm{a}=\mathbf{1}$ of formula (\ref{Witt ring F and V map with prod-1}).\par
Simialrly, we deduce the formula (\ref{Witt ring F map Frobenius like}) from the relation
\[f_B(\xi)\equiv \xi^p\mod p\Phi_B(B^\N)\]
where $\xi^p$ denote the product of $B^\N$. By \cref{Witt polynomial map Phi prop if p}(c), this is equivalent to the fact that for all $n\geq 0$, we have
\[\sigma(\xi_{n+1}-\xi_n^p)\equiv\xi_{n+2}-\xi_{n+1}^p\mod p^{n+2}B.\]
But, for $n\geq 0$, we have $\sigma(\xi_n)\equiv\xi_{n+1}$ mod $p^{n+1}B$, since $\xi=\Phi_B(\bm{x})$ (again by \cref{Witt polynomial map Phi prop if p}(c)); we then deduce from \cref{filtered ring p power lemma} that
\[\sigma(\xi_n)^p\equiv\xi_{n+1}^p\mod p^{n+2}B,\]
which completes the proof.
\end{proof}
\subsection{Filtration and topology of the Witt ring}
\begin{lemma}\label{Witt ring truncation lemma}
Let $A$ be a ring and $m\geq 1$ an integer. Then 
\[\bm{a}=(a_0,\dots,a_{m-1},0,\dots)+(\underbrace{0,\dots,0}_{\text{$m$-terms}},a_m,a_{m+1},\dots)\]
for any $\bm{a}$ in $W(A)$.
\end{lemma}
\begin{proof}
Let $\rho:B\to A$ be a ring homomorphism satisfying the conditions of \cref{Witt ring surjective lift}. Then $W(\rho):W(B)\to W(A)$ is a surjective ring homomorphism, and $\Phi_B:W(B)\to B^\N$ is an injective homomorphism of rings. It suffices to prove that
\begin{align}\label{Witt ring truncation lemma-1}
\Phi_n(\bm{b})=\Phi_n(b_0,\dots,b_{m-1},0,\dots)+\Phi_n(0,\dots,0,b_m,b_{m+1},\dots)
\end{align}
for any $\bm{b}$ in $W(B)$ and $m\geq 1$. But we have
\[\Phi_n(b_0,\dots,b_{m-1},0,\dots)=\begin{cases}
\Phi_n(b_0,\dots,b_n)&\text{if $0\leq n<m$}\\
\sum_{i=0}^{m-1}p^ib_i^{p^{n-i}}&\text{if $m\leq n$}
\end{cases}\]
and
\[\Phi_n(0,b_m,b_{m+1},\dots)=\begin{cases}
0&\text{if $0\leq n<m$}\\
\sum_{i=m}^{m}p^ib_i^{p^{n-i}}&\text{if $m\leq n$},
\end{cases}\]
whence the formula (\ref{Witt ring truncation lemma-1}).
\end{proof}
Let $A$ be a ring. For each integer $m\geq 0$, we denote by $V_m(A)$ the set of Witt vectors $\bm{a}=(a_n)_{n\in\N}$ such that $a_n=0$ for $0\leq n<m$, which is also the $m$-fold image of $V_A$. From \cref{Witt ring F and V map prop} and induction on $m$, we have the formulae
\begin{align}
V^m(\bm{a}+\bm{b})&=V^m(\bm{a})+V^m(\bm{b}),\label{Witt ring m-fold V prop-1}\\
V^m(\bm{a})\times\bm{b}&=V^m(\bm{a}\times F^m(\bm{b})).\label{Witt ring m-fold V prop-2}
\end{align}
We set $V_m(A)=W(A)$ if $m<0$. The sequence $(V_m(A))_{m\in\Z}$ is then a decreasing filtration of additive subgroups of $W(A)$, and is compatible with the ring structure of $W(A)$ if and only if the ring $A$ has characteristic $p$.\par
In the following, we denote by $\mathcal{T}$ the topology on $W(A)$ associated with the filtration $(V_m(A))_{m\in\Z}$. As $V_m(A)$ is an ideal of $W(A)$ for all $m\in\Z$, the topology $\mathcal{T}$ is compatible with the ring structure of $W(A)$ (\cref{topo ring given by filter of ideal}). Let $\bm{a}\in W(A)$; the sets $\bm{a}+V_m(A)$, for $m\in\N$, form a fundamental system of neighborhoods of $\bm{a}$ in $\mathcal{T}$. By \cref{Witt ring truncation lemma}, $\bm{a}+V_m(A)$ consists of Witt vectors $\bm{b}$ such that $a_i=b_i$ for $0\leq i<m$. Consequently, $\mathcal{T}$ is none other than the topology product on $A^\N$ of the discrete topology on each of the factors, and $W(A)$ is therefore a separated and complete topological ring.\par
Let $\tau_A$ be the map of $A$ to $W(A)$ that sends an element $a$ of $A$ to $(a,0,0,\dots)$. We have $\Phi_n(\tau(a))=a^{p^n}$ for each $n\in\N$. For any ring homomorphism $\rho:B\to A$, it is clear that $W(\rho)\circ\tau_B=\tau_A\circ\rho$.
\begin{proposition}\label{Witt ring map tau prop}
Let $a,b$ be elements of $A$ and $\bm{x}=(x_n)_{n\in\N}$ an element of $W(A)$.
\begin{itemize}
\item[(a)] We have the formulas
\begin{align}
&\tau(ab)=\tau(b)\times\tau(b),\label{Witt ring map tau prop-1}\\
&\tau(a)\times\bm{x}=(a^{p^n}x_n)_{n\in\N}.\label{Witt ring map tau prop-2}
\end{align}
\item[(b)] The series $\sum_nV^n(\tau(x_n))$ converges in $W(A)$ to $\bm{x}$.
\end{itemize}
\end{proposition}
\begin{proof}
Let $n$ be a positive integer. The polynomial $P_n(X_0,\dots,X_n,Y_0,\dots,Y_n)$ is isobare with weight $p^n$ is we assign $X_i$ with weight $p^i$. We then have
\[P_n(X_0,0,\dots,0,Y_0,\dots,Y_n)=X_0^{p^n}P_n(1,0,\dots,0,Y_0,\dots,Y_n).\]
Since $\mathbf{1}=(1,0,0,\dots)$ is a unit in the Witt ring with coefficients in $\Z[(X_n)_{n\in\N},(Y_n)_{n\in\N}]$, we have
\[P_n(1,0,\dots,0,Y_0,\dots,Y_n)=Y_n.\]
By Substituting $X_0$ with $a$ and $Y_i$ with $x_i$, we deduce that
\[P_n(a,0,\dots,0,x_0,\dots,x_n)=a^{p^n}x_n.\]
By the definition of product in $W(A)$, this proves (\ref{Witt ring map tau prop-2}), and (\ref{Witt ring map tau prop-1}) is a special case of (\ref{Witt ring map tau prop-2}).\par
We now prove (b). By definition, $V^n(\tau(x_n))$ is the sequence of which all the components are zero, except that of index $n$ which is equal to $x_n$. It follows from \cref{Witt ring truncation lemma}, by induction on $m$, which we have
\[\sum_{n=0}^{m}V^n(\tau(x_n))=(x_0,\dots,x_m,0,0,\dots)\]
for each $m\geq 0$; we then deduce (b) by passing to limit in the topology $\mathcal{T}$ of $W(A)$, which is the product of the discrete topology on $A$.
\end{proof}
Now for each integer $n\geq 1$, let $W_n(A)$ be the quotient ring $W(A)/V_n(A)$. For any elements $a_0,\dots,a_{n-1}$ of $A$, we denote by $[a_0,\dots,a_{n-1}]$ or $[a_i]_{0\leq i<n}$ the class modulo $V_n(A)$ of the element $(a_0,\dots,a_{n-1},0,0,\dots)$ of $W(A)$. By \cref{Witt ring truncation lemma}, the map $(a_0,\dots,a_{n-1})\mapsto[a_0,\dots,a_{n-1}]$ from $A^n$ to $W_n(A)$ is bijective. For this reasong, we say that the elements of $W_n(A)$ are the Witt vectors of length $n$; similarly, one sometimes qualifies as Witt vectors of infinite length the elements of $W(A)$.\par
We denote by $\pi_n$ the canonical homomorphism from $W(A)$ to $W_n(A)$. According to \cref{Witt ring truncation lemma}, for $\bm{a}=(a_n)_{n\in\N}$ in $W(A)$, we have
\[\pi_n(\bm{a})=[a_0,\dots,a_{n-1}].\]
By the definition of the ring structure of $W(A)$, we have the following description of operations in $W_n(A)$:
\begin{align*}
[a_0,\dots,a_{n-1}]+[b_0,\dots,b_{n-1}]&=[S_i(a_0,\dots,a_i;b_0,\dots,b_i)]_{0\leq i<n}\\
[a_0,\dots,a_{n-1}]\times[b_0,\dots,b_{n-1}]&=[S_i(a_0,\dots,a_i;b_0,\dots,b_i)]_{0\leq i<n}\\
-[a_0,\dots,a_{n-1}]&=[I_i(a_0,\dots,a_i)]_{0\leq i<n}.
\end{align*}
Moreover, the addition identity of $W_n(A)$ is $[0,\dots,0]$ and the multiplication identity of $W(A)$ is $[1,0,\dots,0]$.\par
Let $i$ be an integer such that $0\leq i\leq n$. By passing to quotient, the homomorphism $\Phi_i$ from $W(A)$ to $A$ defines a homomorphism from $W_n(A)$ to $A$, still denoted by $\Phi_i$. It associates an Witt vector the element $\Phi_i(a_0,\dots,a_i)$ (called the phantom component of index $i$ of $[a_0,\dots,a_{n-1}]$).\par
Let $\rho:A\to B$ be a ring homormorphism. By passing to quotient, the ring homomorphism $W(\rho)$ induces a homomorphism $W_n(\rho):W_n(B)\to W_n(A)$. We have the following formula
\begin{align}\label{Witt ring induced map on W_n}
W_n(\rho)[b_0,\dots,b_{n-1}]=[\rho(b_0),\dots,\rho(b_{n-1})]
\end{align}
for $[b_0,\dots,b_{n-1}]$ in $W_n(B)$.\par
Let $m$ and $n$ be integers such that $1\leq n\leq m$. We have $V_n(A)\supset V_m(A)$, whence a canonical homomorphism $W_m(A)=W(A)/V_m(A)$ to $W_n(A)=W(A)/V_n(A)$, denoted by $\pi_{n,m}$. More precisely,
\begin{align}\label{Witt ring transition map}
\pi_{n,m}[a_0,\dots,a_{m-1}]=[a_0,\dots,a_{n-1}]
\end{align}
for $[a_0,\dots,a_{m-1}]$ in $W_m(A)$. The family $(W_n(A),\pi_{n,m})$ is a inverse system of rings and the map $\pi:\bm{a}\mapsto(\pi_n(\bm{a}))_{n\geq 1}$ is a ring homomorphism from $W(A)$ to $\llim W_n(A)$. Since $W(A)$ is separated and complete for the filtration $(V_n(A))_{n\in\Z}$, the canonical homomorphism $\pi$ is an isomorphism of topological rings, where $W_n(A)$ has the discrete topology for each $n\geq 1$.\par
Henceforth, the homomorphisms $\pi_n$ and $\pi_{n,m}$ will be called the projection from $W(A)$ to $W_n(A)$ and $W_m(A)$ to $W_n(A)$, respectively.
\begin{example}
\mbox{}
\begin{itemize}
\item[(a)] The homomorphism $\Phi_0:W_1(A)\to A$ is an isomorphism.
\item[(b)] Explicitly, the operations on $W_2(A)$ are given by
\begin{align*}
[a_0,a_1]+[b_0,b_1]&=[a_0+b_0,a_1+b_1-\sum_{i=1}^{p-1}\frac{1}{p}\binom{p}{i}a_0^ib_0^{p-i}],\\
[a_0,a_1]\times[b_0,b_1]&=[a_0b_0,a_0^pb_1+a_1b_0^p+pa_1b_1].
\end{align*}
where $[a_0,a_1]$ and $[b_0,b_1]$ are in $W_2(A)$. The phantom componenet of $[a_0,a_1]$ is $a_0$ and $a_0^p+pa_1$.
\item[(c)] Let $n\geq 1$ be an integer. If $a_0,\dots,a_{n-1}$ and $b_0,\dots,b_{n-1}$ are integers such that $a_i\equiv b_i$ mod $p$ for each $i$, then (\cref{Witt polynomial filtered ring congruence prop})
\[\Phi_{n-1}(a_0,\dots,a_{n-1})\equiv\Phi_{n-1}(b_0,\dots,b_{n-1})\mod p^n.\]
Therefore, by passing to quotient, $\Phi_{n-1}$ defines a homomorphism $\varphi_n:W_n(\Z/p\Z)\to\Z/p^n\Z$. The image of $\varphi_n$ is a subgroup of $\Z/p^n\Z$ containing $1$, hence the whole $\Z/p^n\Z$. Since $\Z/p^n\Z$ and $W_n(\Z/p\Z)$ are both finite sets, $\varphi_n$ is an isomorphism.\par
Let $m$ and $n$ be integers such that $1\leq n\leq m$. There exists a ring homomorphism $\alpha_{n,m}:\Z/p^m\Z\to\Z/p^n\Z$; moreover, the diagram
\[\begin{tikzcd}
\Z/p^m\Z\ar[r,"\alpha_{n,m}"]\ar[d]&\Z/p^n\Z\ar[d]\\
W_m(\Z/p\Z)\ar[r,"\pi_{n,m}"]&W_n(\Z/p\Z)
\end{tikzcd}\]
is commutative. This then gives a map $\varphi=\llim\varphi_n$ and an isomorphism of topological rings $W(\Z/p\Z)=\llim W_n(\Z/p\Z)$ and $Z_\p=\llim\Z/p^n\Z$.
\end{itemize}
\end{example}
Let $m,n\geq 1$ be integers. By construction, we have an exact sequence of additive groups
\begin{equation}\label{Witt ring W_n exact sequence}
\begin{tikzcd}
0\ar[r]&W(A)\ar[r,"V^m"]&W(A)\ar[r,"\pi_m"]&W_m(A)\ar[r]&0
\end{tikzcd}
\end{equation}
By passing to quotient, the endomorphism $V^n$ of the additive group of $W(A)$ defines a homomorphism $V_m^n$ on the additive group $W_m(A)$ to $W_{m+n}(A)$. In other words, we have a commutative diagram
\[\begin{tikzcd}
W(A)\ar[d,"\pi_n"]\ar[r,"V^n"]&W(A)\ar[d,"\pi_{n+m}"]\\
W_m(A)\ar[r,"V_m^n"]&W_{n+m}(A)
\end{tikzcd}\] 
Moreover, from (\ref{Witt ring W_n exact sequence}) we deduce the following exact sequence
\begin{equation}\label{Witt ring V_m^n exact sequence}
\begin{tikzcd}
0\ar[r]&W_m(A)\ar[r,"V_m^n"]&W_{m+n}(A)\ar[r,"\pi_{n,n+m}"]&W_n(A)\ar[r]&0
\end{tikzcd}
\end{equation}
and for $[a_0,\dots,a_{m-1}]$ in $W_m(A)$,
\begin{align}\label{Witt ring V_m^n def}
V_m^n[a_0,\dots,a_{m-1}]=[\underbrace{0,\dots,0}_{\text{$n$ terms}},a_0,\dots,a_{m-1}]
\end{align}

From \cref{Witt ring F and V map VF}(c), we have $FV^{m+1}(\bm{a})=p\cdot V^m(\bm{a})$ for $\bm{a}$ in $W(A)$ and therefore $F(V_{m+1}(A))\sub V_m(A)$. By induction on $n$, we then deduce that $F^n$ maps $V_{n+m}(A)$ to $V_m(A)$, and therefore defines, by passing to quotient, a homomorphism $F_m^n:W_{n+m}(A)\to W_m(A)$. By construction, we have the following exact sequence
\[\begin{tikzcd}
W(A)\ar[r,"F^n"]\ar[d,"\pi_{n+m}"]&W(A)\ar[d,"\pi_m"]\\
W_{n+m}(A)\ar[r,"F_m^n"]&W_m(A)
\end{tikzcd}\]
Recall the polynomial $F_i$ in $\Z[X_0,\dots,X_{i+1}]$ for each $i\geq 0$. The homomorphism $F_m^1$ of $W_{m+1}(A)$ to $W_m(A)$ has the following explicit expression
\begin{align}\label{Witt ring F_m^n def}
F_m^1[a_0,\dots,a_m]=[F_i(a_0,\dots,a_{i+1})]_{0\leq i<m}.
\end{align}
Let $\bm{a},\bm{a}'\in W_m(A)$ and $\bm{b}\in W_{m+1}(A)$. The following formulae then follows from \cref{Witt ring F and V map VF}:
\begin{align*}
F_m^1(V_m^1(\bm{a}))&=p\cdot\bm{a}\\
V_m^1(\bm{a}\times F_m^1(\bm{b}))&=V_m^1(\bm{a})\times\bm{b}\\
V_m^1(\bm{a})\times V_m^1(\bm{a}')&=p\cdot V_m^1(\bm{a}\times\bm{a}')\\
V_m^1(F_m^1(\bm{b}))&=\bm{\mu}_{m+1}\times\bm{b}
\end{align*}
where $\bm{\mu}_{m+1}=[0,1,0\dots,0]$.
\subsection{The Witt ring with characteristic \texorpdfstring{$p$}{p}}
\begin{proposition}\label{Witt ring for char p prop}
Let $A$ be a ring with characteristic $p$. Then for $\bm{a},\bm{b}$ in $W(A)$ and positive integers $m,n$, we have
\begin{align}
F(\bm{a})&=(a_n^p)_{n\in\N}\label{Witt ring for char p prop-1}\\
p\cdot\bm{a}=VF(\bm{a})=F&V(\bm{a})=(0,a_0^p,a_1^p,\dots)\label{Witt ring for char p prop-2}\\
V^m(\bm{a})\times V^n(\bm{b})&=V^{m+n}(F^n(\bm{a})\times F^m(\bm{b})).\label{Witt ring for char p prop-3}
\end{align}
\end{proposition}
\begin{proof}
The formula (\ref{Witt ring for char p prop-1}) follows from \cref{Witt ring SPIF first terms}(d). We immediately deduce from this the equality
\[VF(\bm{a})=FV(\bm{a})=(0,a_0^p,a_1^p,\dots)\]
and the equality $p\cdot\bm{a}=FV(\bm{a})$ is already proved in \cref{Witt ring F and V map prop}(b), whence (\ref{Witt ring for char p prop-2}). Now recall that by (\ref{Witt ring m-fold V prop-2}), we have
\[V^m(\bm{a})\times V^n(\bm{b})=V^m(\bm{a}\times F^m(V^n(\bm{b}))).\]
By (\ref{Witt ring m-fold V prop-1}), we also deduce that
\[V^n(F^m(\bm{b}))\times\bm{a}=V^n(F^m(\bm{b})\times F^n(\bm{a})).\]
Fomula (\ref{Witt ring for char p prop-3}) then follows from these two equalities and the relation $F^m\circ V^n=V^n\circ F^m$, which comes from (\ref{Witt ring for char p prop-1}).
\end{proof}
\begin{corollary}\label{Witt ring char p V_m filtration compatible}
Let $m$ and $n$ be positive integers, then
\[V_m(A)\times V_n(A)\sub V_{m+n}(A).\]
\end{corollary}
\begin{proof}
This follows from (\ref{Witt ring for char p prop-3}), since $V_m(A)$ is the image of $V^m$.
\end{proof}
\begin{remark}
Let $A$ be a ring with characteristic $p$. By \cref{Witt ring for char p prop}, we have the formulae
\[F_m^n[a_0,\dots,a_{n+m-1}]=[a_0^{p^n},\dots,a_{m-1}^{p^n}],\quad p^n\cdot[a_0,\dots,a_{n+m-1}]=[\underbrace{0,\dots,0}_{\text{$n$ terms}},a_0^{p^n},\dots,a_{m-1}^{p^n}]\]
for any Witt vector $[a_0,\dots,a_{n+m-1}]$ of length $n+m$.
\end{remark}
\begin{proposition}\label{Witt ring V_1-adic and p-adic topo prop}
Let $A$ be a ring.
\begin{itemize}
\item[(a)] For each integer $k\geq 1$, we have $V_1(A)^k=p^{k-1}\cdot V_1(A)$. 
\item[(b)] Suppose that $A$ is a ring with characteristic $p$. Then the $V_1(A)$-adic topology and the $p$-adic topology coincide on $W(A)$, and is finer than the topology $\mathcal{T}$. The ring $W(A)$ is separated and complete for the $p$-adic topology.
\end{itemize}
\end{proposition}
\begin{proof}
We prove (a) by induction on $k$. The case $k=1$ is evident, so suppose $k\geq 2$. By the induction hypothesis, we have $V_1(A)^{k-1}=p^{k-2}\cdot V_1(A)$ and Concequently $V_1(A)^k=p^{k-2}\cdot V_1(A)^2$. But by \cref{Witt ring F and V map with prod-2}(d), $V_1(A)^2=p\cdot V_1(A)$, so (a) follows.\par
Suppose now that $A$ has characteristic $p$. Then by (\ref{Witt ring for char p prop-2}),
\[pW(A)=VF(W(A))\sub V_1(A)\]
combine this with (a), we deduce by induction that $p^k\cdot W(A)\sub V_1(A)^k\sub p^{k-1}\cdot W(A)$, and by \cref{Witt ring char p V_m filtration compatible} the inclusion $V_1(A)^k\sub V_k(A)$, for each $k\geq 1$. The first assertion in (b) then follow.\par
Let $k\geq 1$ be an integer. By (\ref{Witt ring for char p prop-2}), the ideal $p^k\cdot W(A)$ of $W(A)$ is the set of elements $\bm{a}=(a_n)_{n\in\N}$ of $W(A)$ such that $a_n=0$ for $n<k$ and $a_n\in A^{p^k}$ for $n\geq k$. It is therefore closed in the topology $\mathcal{T}$. Since $W(A)$ is separated and complete for the topology $\mathcal{T}$ and that the ideals $p^k\cdot W(A)$ of $W(A)$, for $k\geq 1$, form a base of neighborhoods of $\bm{0}$ in $W(A)$ for the $p$-adic topology, the ring $W(A)$ is separated and complete for the $p$-adic topology (\cref{topological group abelian two topo complete prop}).
\end{proof}
\begin{proposition}\label{Witt ring perfect p ring prop}
Let $A$ be a perfect ring of characteristic $p$.
\begin{itemize}
\item[(a)] For any element $\bm{a}=(a_n)_{n\in\N}$ in $W(A)$, the series $\sum_np^n\tau(a_n^{p^{-n}})$ converges in $W(A)$ to $\bm{a}$.
\item[(b)] In $W(A)$, the $V_1(A)$-adic topology, the $p$-adic topology, and the topology $\mathcal{T}$ coincide. Moreover precisely, we have $V_n(A)=p^nW(A)=V_1(A)^n$ for each integer $n\geq 0$. In particular, $\Phi_0$ defines an isomorphism of $W(A)/pW(A)$ to $A$.
\end{itemize}
\end{proposition}
\begin{proof}
By definition, the map $a\mapsto a^p$ is an automorphism of $A$. \cref{Witt ring for char p prop} then show that $F$ is an automorphism of $W(A)$, and for $n\in\N$, we have
\[p^n\cdot W(A)=V^nF^n(W(A))=V^n(W(A))=V_n(A).\]
In particular, $V_1(A)^n=(pW(A))^n=p^n\cdot W(A)$. Assertion (b) then follows. By \cref{Witt ring for char p prop}, we have
\[p^n\cdot\tau(a_n^{p^{-n}})=V^nF^n\tau(a_n^{p^{-n}})=V^n\tau(a_n)\]
so (a) follows from \cref{Witt ring map tau prop}.
\end{proof}
\begin{proposition}\label{Witt ring perfect field char p prop}
Let $A$ be a field with characteristic $p$. Then $W(A)$ is a local integral domain, separated and complete, with maximal ideal $V_1(A)$ and residue field isomorphic to $A$. If $A$ is perfect, the ring $W(A)$ is a DVR, and its maximal ideal is $pW(A)$.
\end{proposition}
\begin{proof}
The homomorphism $\Phi_0$ defines an isomomorphic of $W(A)/V_1(A)$ to $A$. The ideal $V_1(A)$ of $W(A)$ is therefore maximal. The ring $W(A)$ is separated and complete for the $V_1(A)$-topology (\cref{Witt ring V_1-adic and p-adic topo prop}), so is local with maximal ideal $V_1(A)$ by \cref{filtration m-adic completion is local}.\par
Let $\bm{a}$ and $\bm{b}$ be two nonzero elements in $W(A)$. There exists integers $n,m\geq 0$ and elements $\tilde{\bm{a}}=(\tilde{a}_n)_{n\in\N}$ and $\tilde{\bm{b}}=(\tilde{b}_n)_{n\in\N}$ of $W(A)$ such that $\bm{a}=V^m(\tilde{\bm{a}})$, $\bm{b}=V^n(\tilde{\bm{b}})$ and the eleemnts $\tilde{a}_0$, $\tilde{b}_0$ are nonzero. Then the $m+n$-th component of $\bm{a}\times\bm{b}$ is equal to the $0$-th component of $F^n(\tilde{\bm{a}})\times F^m(\tilde{\bm{b}})$ (formula (\ref{Witt ring for char p prop-3})), which is $\tilde{a}^{p^n}\tilde{b}^{p^m}$ (formula (\ref{Witt ring for char p prop-1}) and \cref{Witt ring SPIF first terms}). Therefore $\bm{a}\times\bm{b}$ is nonzero and $W(A)$ is integral.\par
If the field $A$ is perfect, the maximal ideal $V_1(a)$ of $W(A)$ is equal to $pW(A)$ by \cref{Witt ring perfect p ring prop}(b) and therefore $W(A)$ is a DVR by \cref{integral domain DVR iff}(c).
\end{proof}
\section{Cohen rings}
In this part, we fix a prime integer $p$.
\subsection{\texorpdfstring{$p$}{p}-rings}
Let $C$ be a ring. We say $C$ is a \textbf{$\bm{p}$-ring} if the ideal $pC$ of $C$ is maximal and if $C$ is separated and complete for the $pC$-adic topology.\par
If $p1_C$ is nilpotent in $C$ and the ideal $pC$ is maximal, then $C$ is a $p$-ring and the $pC$-adic topology is discrete. In particular, every field with characteristic $p$ is a $p$-ring.
\begin{proposition}\label{p-ring prop}
Let $C$ be a $p$-ring.
\begin{itemize}
\item[(a)] The ring $C$ is local with maximal ideal $pC$.
\item[(b)] Suppose that $p1_C$ is nilpotent and let $d$ be the smallest positive integer such that $p^d1_C=0$. Then the ideals of $C$ are of the form $p^iC$ and $p^iC\neq p^jC$ if $i,j$ are distinct integers such that $0\leq i,j\leq d$. The $C$-module $C$ has length $d$.
\item[(c)] Suppose that $p1_C$ is not nilpotent. Then $C$ is a DVR whose residue field is of characteristic $p$, and the fraction field of characteristic $0$. The ideals $p^nC$, where $n\in\N$, are distinct; they are the nonzero ideals of $C$. The $C$-module $C$ has infinite length.
\end{itemize}
\end{proposition}
\begin{proof}
Assertion (a) follows from \cref{filtration m-adic completion is local}. By hypothesis we have $\bigcap_{n\in\N}p^nC=\{0\}$. Let $x\neq 0$ be in $C$; there exists an integer $n\geq 0$ such that $x\in p^nC$, $x\notin p^{n+1}C$. Write $x=p^ny$ where $y\in C$, then $y\notin pC$, so it is invertible.\par
Suppose that $p1_C$ is not nilpotent. If $x_1$ and $x_1$ are two nonzero elements in $C$, there exist integers $n_1,n_2\geq 0$ and invertible elements $y_1,y_2$ of $C$ such that $x_1=p^{n_1}y_1$, $x_2=p^{n_2}y_2$. We then have $x_1x_2=p^{n_1+n_2}y_1y_2\neq 0$, so $C$ is integral. Since $C$ is a local ring, not a field, and the maximal ideal $\m_C=pC$ is principal, we conclude that $C$ is a DVR (\cref{integral domain DVR iff}). The nonzero ideals of $C$ are then of the form $p^nC$, and are all distinct. In particular, the ring $C$ is not Artinian, so the $C$-module $C$ has infinite length. The residue field $C/pC$ of $C$ clearly has characteristic $p$. Let $q$ be the characteristic of the fraction field of $C$. We have $p1_C\neq 0$, so $q\neq p$. But if $q\neq 0$, we have $q1_C\neq 0$ and $C/pC$ then has characteristic $q\neq p$, which is absurd. This proves (c).\par
Suppose that $p1_C$ is nilpotent and let $d$ be the smallest positive integer such that $p^d1_C=0$. We have a sequence of ideals
\begin{align}\label{p-ring nilpotent J-H sequence}
C\sups pC\sups\cdots\sups p^{d-1}C\sups p^dC=\{0\}.
\end{align}
If $i$ is an integer such that $0\leq i<d$ and $p^iC=p^{i+1}C$, then
\[p^{d-i-1}p^iC=p^{d-i-1}p^{i+1}C=p^dC=\{0\}\]
contradicting the hypothesis $p^{d-1}1_C\neq 0$. Therefore the ideals in the sequence above are all distinct. Let $\a$ be an ideal of $C$ and $i$ the smallest integer such that $\a\sups p^iC$. Let $x\neq 0$ be in $\a$; we can write $x=p^mu$ where $m\geq 0$ and $u$ is invertible. Then $p^m1_C\in\a$, whence $p^mC\sub\a$ and $m\geq i$. This shows $x\in p^iC$, so $\a=p^iC$. We now conclude that (\ref{p-ring nilpotent J-H sequence}) is a Jordan-H\"older sequence for the $C$-module $C$, so $C$ has length $d$.
\end{proof}
\begin{corollary}\label{p-ring integral is DVR}
If the $p$-ring $C$ is integral, it is a DVR, or a field with characteristic $p$. 
\end{corollary}
\begin{proof}
Suppose that $C$ is integral. If $p1_C$ is nilpotent, then $p1_C=0$, and $\{0\}$ is a maximal ideal of $C$, whence $C$ is a field with characteristic $p$. Otherwise, $p1_C$ is not nilpotent and $C$ is a DVR by \cref{p-ring prop}.
\end{proof}
\begin{corollary}\label{p-ring quotient}
Let $C$ be a $p$-ring and $\a$ a proper ideal of $C$. Then $C/\a$ is a $p$-ring.
\end{corollary}
\begin{proof}
We may suppose that $\a\neq\{0\}$. Then there exists an integer $i\geq 1$ such that $\a=p^iC$. The idela $pC/\a$ of $C/\a$ is maximal and we have $p^i1_{C/\a}=0$, so $C/\a$ is a $p$-ring.
\end{proof}
Let $C$ be a $p$-ring. The length of $C$, denoted by $l(C)$, is the supremum in $\widebar{\R}$, of the set of integers $n\geq 1$ such that $p^{n-1}1_C\neq 0$. By \cref{p-ring prop}, $l(C)$ is finite and equal to the length of the $C$-module $C$, or is $+\infty$, in which case the $C$-module $C$ has infinite length.
\begin{example}[\textbf{Example of $p$-rings}]
\mbox{}
\begin{itemize}
\item[(a)] For each integer $n\geq 1$, the ring $\Z/p^n\Z$ is a $p$-ring of length $n$. The ring $\Z_p$ of $p$-adic integers is a $p$-ring of length infinite.
\item[(b)] Let $K$ be a perfect field of characteristic $p$. By \cref{Witt ring perfect field char p prop}, the Witt ring $W(K)$ is a $p$-ring of infinite length. The map $(a_n)_{n\in\N}\mapsto a_0$ induces by passing to quotient an isomorphism of $W(K)/pW(K)$ to the field $K$. For each integer $n\geq 1$, the ring
\[W_n(K)=W(K)/p^nW(K)\]
is a $p$-ring of length $n$. Note that $W(\F_p)=\Z_p$, so this generalize the example in (a). 
\end{itemize}
\end{example}
\begin{proposition}\label{p-ring homomorphism prop}
Let $C_1$ and $C_2$ be $p$-rings and $\rho:C_1\to C_2$ a ring homomorphism. Let $\bar{\rho}$ be the homomorphism of $\kappa_{C_1}=C_1/pC_1$ to $\kappa_{C_2}=C_2/pC_2$ induced by $\rho$ on the residue field.
\begin{itemize}
\item[(a)] We have $l(C_1)\geq l(C_2)$, and $\rho$ is injective if and only if $l(C_1)=l(C_2)$.
\item[(b)] For $\rho$ to be surjective, it is necessary and sufficient that $\bar{\rho}$ being an isomorphism.
\item[(c)] For $\rho$ to be an isomorphism, it is necessary and sufficient that $\bar{\rho}$ being an isomorphism and we have $l(C_1)=l(C_2)$.
\end{itemize} 
\end{proposition}
\begin{proof}
Let $n\geq $ be an integer. We have $\rho(p^{n-1}1_{C_1})=p^{n-1}1_{C_2}$, so the relation $p^{n-1}1_{C_2}\neq 0$ implies $p^{n-1}1_{C_1}\neq 0$ and they are equivalent if $\rho$ is injective. We then have $l(C_2)\leq l(C_1)$ where the equality holds if $\rho$ is injective. If $\rho$ is not injective, there exists an integer $i<i(C_1)$ such that the kernel of $\rho$ is the ideal $p^iC_1$ of $C_1$; we have $p^i1_{C_2}=0$, so $l(C_2)\leq i$. This proves (a).
Since $\kappa_{C_1}$ and $\kappa_{C_2}$ are fields, the homomorphism $\bar{\rho}$ is injective. If $\rho$ is surjective, so is $\bar{\rho}$ and hence it is an isomorphism. Conversely, assume that $\bar{\rho}$ is surjective. Then for each integer $n\geq 0$, the map $\bar{\rho}_n:p^nC_1/p^{n+1}C_1\to p^nC_2/p^{n+1}C_2$ induced by $\rho$ is surjective. Since $C$ is complete for the $pC_1$-adic topology, $\rho$ is then surjective by \cref{filtration gr(phi) surjective imply phi surjective if}. This proves (b). Finally, (a) and (b) imply (c).
\end{proof}
\begin{proposition}\label{p-ring inverse limit surjective prop}
Let $(C_n,\pi_{n,m})$ be a inverse system of rings indexed by $\N$. Suppose that $C_n$ is a $p$-ring for each $n\in\N$ and the homomorphism $\pi_{n,m}$ are surjective. Then $C=\llim C_n$ is a $p$-ring, and for each $n\in\N$, the canonical homomorphism $\pi_n:C\to C_n$ is surjective and induces an isomorphism of $\kappa_C$ to $\kappa_{C_n}$.
\end{proposition}
\begin{proof}
Since the maps $\pi_{n,m}$ are surjective, $\pi_n$ is surjective (\cref{inverse limit derived exact sequence}). Let $d_n$ be the length of $C_n$. By \cref{p-ring homomorphism prop}(a), the sequence $d_n$ in $\N\cup\{+\infty\}$ is increasing; if it is stationary, there exists an integer $n_0$ such that $\pi_{n,m}$ is an isomorphism of $C_m$ to $C_n$ for $m\geq n\geq N_0$, hence $C$ is isomorphic to $C_{n_0}$, a $p$-ring.\par
It then suffices to consider the case where $d_n$ are all finite, and $(d_n)$ tends to infinity. Let us endow the ring $C$ with the trivial filtration. For each $n\in\N$, let $I_n$ be the kernel of $\pi_n$, and put $I_n=C$ for $n<0$. Endow $C$ with the filtration $(I_n)_{n\in\N}$. It is then separated and complete, for the topology $\mathcal{T}$ defined bby the filtration is the inverse limit topology of the discrete topologis over $C_n$. Let $k\geq 1$ be an integer. We have $p^kC\sub\llim_n(p^kC_n)$. Conversely, if $x=(x_n)_{n\in\N}\in\llim_n(p^kC_n)$ and if we set $X_n=\{y\in C:\pi_n(p^ky)=x_n\}$, the sequence $(X_n)_{n\in\N}$ is a decreasing sequence of nonempty closed affine subsets of $C$. Since $C/I_n$ is an Artinian $C$-module, the intersection of $X_n$ is nonempty (AC, $\S$2, n7, prop.7); for every $z\in\bigcap_{n\in\N}X_n$, we have $p^kz=x$. We have therefore proved that $p^kC=\llim_n(p^kC_n)$ for each integer $k\geq 1$. In particular, the ideal $p^kC$ of $C$ is closed for the topology $\mathcal{T}$. For $C$, the $p$-adic topology is finer than the topology $\mathcal{T}$ since we have $p^{d_n}C\sub I_n$. It then follows from \cref{topological group abelian two topo complete prop} that $C$ is separated and complete for the $pC$-adic topology. In addition we have $pC=\llim pC_n=\pi_0^{-1}(pC_0)$ and therefore the surjective homomorphism of $C/pC$ to $C_0/pC_0$ induced by $\pi_0$ is an isomorphism. This shows that $pC$ is maximal and therefore $C$ is a $p$-ring. The final assertion follows from \cref{p-ring homomorphism prop}(b).
\end{proof}
Now let $A$ be a separated and complete local ring, with residue field of characteristic $p$. A Cohen subring of $A$ is defined to be a subring $C$ of $A$ that is a $p$-ring and such that $A=\m_A+C$ (i.e., $A/\m_A=C/(\m_A\cap C)$). In this case, the ideal $\m_A\cap C$ is maximal in $C$, hence equal to $pC$. The canonical map $\kappa_C=C/pC$ to $\kappa_A=A/\m_A$ is then an isomorphism.
\begin{example}
Let $C$ be a $p$-ring. The formal series ring $A=C\llbracket X_1,\dots,X_n\rrbracket$ is then Noetherian, local, separated and complete, with maximal ideal generated by $(p,T_1,\dots,T_n)$. It is immediate that $C$ is a Cohen subring of $A$. This applies in particular if $C$ is equal to $\Z_p$, to $\Z/p^n\Z$, or a field with characteristic $p$.
\end{example}
\begin{theorem}\label{Cohen subring contain p-basis prop}
Let $A$ be a seaprated and complete local ring, with residue field $k$ of characteristic $p$. Let $\pi:A\to k$ be the canonical map, and $S$ a subset of $A$ such that $\pi$ induces a bijection of $S$ to a $p$-basis for $k$.
\begin{itemize}
\item[(a)] There exists a unique Cohen subring $C$ of $A$ containing $S$.
\item[(b)] The subring $C$ is closed in $A$, and the $pC$-adic topology of $C$ is induced by the $\m_A$-adic topology of $A$.
\item[(c)] Every closed subring $B$ of $A$, containing $S$, and such that $A=B+\m_A$, contains $C$.
\end{itemize}
\end{theorem}
\begin{proof}
We divide the proof into several parts. First, we assume that $\m_A$ is nilpotent. Let $n$ be a positive integer such that $\m_A^{n+1}=\{0\}$. If $\Phi_n$ is the $n$-th Witt polynomial, the map $\rho:[a_0,\dots,a_n]\mapsto\Phi_n(a_0,\dots,a_n)$ is a homomorphism from $W_{n+1}(A)$ to $A$. Let $B_n$ be the image of $\rho_n$ and $C_n$ the subring of $A$ generated by $B_n\cup S$. We note that $pA\sub\m_A$ and $B_n$ consists of elements of the form $a_0^{p^n}+pa_1^{p^{n-1}}+\cdots+p^na_n$, where $a_0,\dots,a_n\in A$. Concequently, we have $\pi(B_n)=k^{p^n}$, and $\pi(C_n)=k^{p^n}[\pi(S)]$. Since $\pi(S)$ is a $p$-basis for $k$, we have $k=k^{p^n}[\pi(S)]$ (A, \Rmnum{5}, p.96), whence $\pi(C_n)=k$, and $C_n+\m_A=A$.\par
Let $B$ be a subring of $A$ containing $S$. We claim that, for $B$ to contain $C_n$, it is necessary and sufficient that $A=B+\m_A$. First, if $B$ contains $C_n$, then
\[B+\m_A\sups C_n+\m_A=A\]
so $B+\m_A=A$. Conversely, suppose that $B+\m_A=A$. Let $a_0,\dots,a_n$ be elements of $A$; there exists by hypothesis elements $b_0,\dots,b_0$ of $B$ such that $a_i\equiv b_i$ mod $\m_A$ for each $i$. By \cref{Witt polynomial filtered ring congruence prop} and the hypothesis $\m_A^{n+1}=0$, we then have $\Phi_n(a_0,\dots,a_n)=\Phi_n(b_0,\dots,b_n)\in B$, whence $B_n\sub B$. Since $C_n$ is the ring generated by $B_n\cup S$, we have $C_n\sub B$.\par
Let $\mathcal{S}$ be the set of subrings $B$ of $A$ containing $S$ and such that $B+\m_A=A$. There exists by the above claim a smallest element $C$ in $\mathcal{S}$ (take the smallest integer $n$ such that $\m_A^{n+1}=\{0\}$, then $C_n$ is the desired subring), and we have $C_n=C$ for any integer $n\geq 0$ such that $\m_A^{n+1}=\{0\}$. We have $C+\m_A=A$ by construction and $p1_C$ is nilpotent; clearly $pC\sub C\cap\m_A$. We now prove that $C\cap\m_A\sub pC$. Choose an integer $n\geq 1$ such that $\m_A^n=\{0\}$, so $C=C_n=C_{n-1}$. Let $\Lambda$ be the subset of $\N^{\oplus S}$ formed by elements $(\alpha_s)_{s\in S}$ with finite support such that $0\leq\alpha_s<p^n$ for each $s\in S$. Since $B_n$ contains $s^{p^n}=\Phi_n(s,0,\dots,0)$ for each $s\in S$, the monomials $z_\alpha=\prod_{s\in S}s^{\alpha_s}$, for $\alpha\in\Lambda$, generate the $C_n$ as a $B_n$-module. Moreover, by the formula
\[\Phi_n(a_0,\dots,a_n)=a_0^{p^n}+p\Phi_{n-1}(a_1,\dots,a_n),\]
any element of $B_n$ is of the form $a^{p^n}+pb$ where $a\in A$ and $b\in B_{n-1}$. Therefore any element of $C=C_n$ is of the form
\begin{align}\label{Cohen subring contain p-basis prop-1}
x=\sum_{\alpha\in\Lambda}c_\alpha^{p^n}z_\alpha+py
\end{align}
where $c_\alpha\in A$ for $\alpha\in\Lambda$, and $y\in C_{n-1}=C$. If $x$ is in $C\cap\m_A$, we have $\pi(x)=0$ whence $\sum_{\alpha\in\Lambda}\pi(c_\alpha)^{p^n}\pi(z_\alpha)=0$. Since $\pi(S)$ is a $p$-basis for $k$, we have $\pi(c_\alpha)=0$ for all $\alpha\in\Lambda$ (A, \Rmnum{5}, p.96). Then $c_\alpha\in\m_A$, whence $c_\alpha^n=0$ and a fortiori $c_\alpha^{p^n}=0$. By (\ref{Cohen subring contain p-basis prop-1}), we then get $x=py$, whence $C\cap\m_A\sub pC$. With this, it is now clear that $C$ is a Cohen subring of $A$. We have $p^nC=\m_A^n=\{0\}$ for $n$ large enough and assertion (b) is then trivial. Assertion (c) follows from the construction of $C$.\par
We now turn to the general case. For each integer $n\geq 0$, let $A_n$ be the local ring $A/\m_A^{n+1}$, $\m_n=\m_A/\m_A{n+1}$, and $\pi_n:A\to A_n$ the canonical homomorphism. By the previous case, there exists a unique Cohen subring $C_n$ in $A_n$ containing $\pi_n(S)$. For $0\leq n\leq m$, we denote by $\pi_{n,m}$ the canonical map from $A_m$ to $A_n$. By \cref{p-ring quotient}, $\pi_{n,m}(C_m)$ is a $p$-ring; we have $\pi_{n,m}(C_m)+\m_n=A_n$, so $\pi_{n,m}(C_m)$ is equal to the Cohen subring $C_n$ of $A_n$. By \cref{p-ring inverse limit surjective prop}, the subring $\llim C_n$ of $\llim A_n$ is a $p$-ring. Put $C=\bigcap_{n\in\N}\pi_n^{-1}(C_n)$. Then $C$ is the preimage of $\llim C_n$ under the isomorphism $a\mapsto(\pi_n(a))_{n\in\N}$ of $A$ to $\llim A_n$, whence a closed subring of $A$, and a $p$-ring. We have $\pi_n(C)=C_n$ for each $n\in\N$ (\cref{p-ring inverse limit surjective prop}) and in particular $\pi_0(C)=A_0$, which means $\pi(C)=k$. Therefore $C$ is a Cohen subring of $A$.\par
For each integer $n\geq 0$, let $J_n=C\cap\m_A^n$. Since the local ring $A$ is separated, we have $\bigcap_{n\in\N}J_n=\{0\}$, and by the structure of ideals of $p$-rings, every ideal of $C$ is of the form $p^iC$ and contains some $J_n$. Conversely, $J_n$ contains $p^nC$. Therefore the $pC$-adic topology is induced by the $\m_A$-topology on $A$. This proves (b). Now let $B$ be a closed subring of $A$ containing $S$ and such that $B+\m_A=A$. Since $B$ is closed, we have $B=\bigcap_{n\in\N}\pi_n^{-1}(\pi_n(B))$. We have $\pi_n(B)\supset\pi_n(S)$ and $\pi_n(B)+\m_n=A_n$, whence $\pi_n(B)\sups C_n$ and therefore $B\sups C$. This proves (c) and completes the proof.
\end{proof}
\begin{remark}
Suppose that $p1_A$ is not nilpotent (in particular if $A$ is an integral domain with fraction field of characteristic $0$). Then $C$ is a DVR with fraction field of characteristic $0$.
\end{remark}
\subsection{Existence and uniqueness of \texorpdfstring{$p$}{p}-rings}
\begin{proposition}\label{p-ring residue field lifting prop}
Let $C_1$ and $C_2$ be $p$-rings such that $l(C)\geq l(C_2)$, $\pi_1$ (resp. $\pi_2$) the canonical homomorphism of $C_1$ (resp. $C_2$) to $\kappa_{C_1}$ (resp. $\kappa_{C_2}$). Let $(x_\lambda)_{\lambda\in\Lambda}$ (resp. $(y_\lambda)_{\lambda\in\Lambda}$) be a family of elements of $C_1$ (resp. $C_2$) whose image under $\pi_1$ (resp. $\pi_2$) is a $p$-basis of $\kappa_{C_1}$ (resp. $\kappa_{C_2}$). Let $\eta$ be an isomorphism of $\kappa_{C_1}$ to $\kappa_{C_2}$ such that $\eta(\pi_1(x_\lambda))=\pi_2(y_\lambda)$ for every $\lambda\in\Lambda$. There then exists a unique homomorphism $\rho:C_1\to C_2$ such that $\eta\circ\pi_1=\pi_2\circ\rho$ and $\rho(x_\lambda)=y_\lambda$ for every $\lambda\in\Lambda$. It is surjective, and an isomorphism if $l(C_1)=l(C_2)$.
\end{proposition}
\begin{proof}
Let $A$ be the subring of $C_1\times C_2$ formed by pairs $(x,y)$ such that $\eta(\pi_1(x))=\pi_2(y)$. The map $(x,y)\mapsto\pi_1(x)$ is a surjective homomorphism of $A$ to $\kappa_C$. Its kernel $\m$, equal to $pC_1\times pC_2$, is then a maximal ideal of $A$. The subspace $A$ of $C_1\times C_2$ is closed, hence complete, and the topology induced on $A$ from $C_1\times C_2$ is the $\m$-adic topology. Thus $A$ is a separated and complete local ring with maximal ideal $\m$. For each $\lambda\in\Lambda$, we have $(x_\lambda,y_\lambda)\in A$ by hypothesis; if $\xi_\lambda$ is the calss of $(x_\lambda,y_\lambda)$ modulo $\m$, the family $(\xi_\lambda)_{\lambda\in\Lambda}$ is a $p$-basis of $A/\m$. By \cref{Cohen subring contain p-basis prop}, there exists a unique Cohen subring $\tilde{C}$ of $A$ containing $(x_\lambda,y_\lambda)$ for each $\lambda\in\Lambda$. By definition, we have $l(\tilde{C})=l(C_1)\geq l(C_2)$. The restriction to $\tilde{C}$ of the projection of $C_1\times C_2$ to $C_1$ is a homomorphism $h_1:\tilde{C}\to C_1$ which induces an isomorphism of $\kappa_{\tilde{C}}$ to $\kappa_{C_1}$. By \cref{p-ring homomorphism prop}(c), $h_1$ is an isomorphism of $\tilde{C}$ to $C_1$. We also see that the restriction $h_2$ to $\tilde{C}$ of the projection from $C_1\times C_2$ to $C_2$ is a surjective homomorphism. Therefore, $\tilde{C}$ is the graph of a surjective homomorphism $\rho=h_2\circ h_1^{-1}$ of $C_1$ to $C_2$, and we have $\eta\circ\pi_1=\pi_2\circ\rho$, $\rho(x_\lambda)=y_\lambda$ for $\lambda\in\Lambda$. Moreover, if $l(C_1)=l(C_2)$, $\rho$ is an isomorphism.\par
Let $\rho_1$ be a homomorphism of $C_1$ to $C_2$ such that $\eta\circ\pi_1=\pi_2\circ\rho$, $\rho(x_\lambda)=y_\lambda$ for $\lambda\in\Lambda$, and let $\tilde{C}'$ be the graph of $\rho_1$. It is immediate that $\tilde{C}'$ is a Cohen subring of $A$, containing $(x_\lambda,y_\lambda)$ for $\lambda\in\Lambda$, whence $\tilde{C}'=\tilde{C}$ and $\rho_1=\rho$.
\end{proof}
\begin{proposition}\label{p-ring arbitrary length exist}
Let $k$ be a field of characteristic $p$, and let $n\geq 1$ an integer, or $+\infty$. Then there exists a $p$-ring of length $n$ with residue field isomorphic to $k$.
\end{proposition}
\begin{proof}
The ring $W(k)$ of Witt vectors with coefficients in $k$ is an integral seaprated and complete local ring, with residue field isomorphic to $k$ (\cref{Witt ring perfect field char p prop}), and we have $p\cdot 1_{W(k)}\neq 0$. Let $C$ be a Cohen subring of $W(k)$ (\cref{Cohen subring contain p-basis prop}). Then $C$ is a $p$-ring of length $+\infty$ with residue field isomorphic to $k$, and, if $n\geq 1$ is an integer, the quotient $C/p^nC$ is a $p$-ring of length $n$ with residue field isomomorphic to $k$.
\end{proof}
\begin{remark}
Let $n\geq 1$ be an integer and $S$ a $p$-basis of $k$. We can show that the subring $W_n(k)$ generated by $W_n(k^{p^n})$ and the elements $[\xi,0,\dots,0]$ (where $\xi\in S$), is a $p$-ring of length $n$ with residue field isomorphic to $k$.
\end{remark}
\begin{corollary}\label{p-ring finite length is quotient}
Let $C$ be a $p$-ring of finite length $n$. Then there exists a $p$-ring $\tilde{C}$ of infinite length such that $C$ is isomomorphic to $\tilde{C}/p^n\tilde{C}$.
\end{corollary}
\begin{proof}
By \cref{p-ring arbitrary length exist}, there exists a $p$-ring $\tilde{C}$ of infinite length such that $\kappa_{\tilde{C}}$ is isomorphic to $\kappa_C$. Then $\tilde{C}/p^n\tilde{C}=\tilde{C}_n$ is a $p$-ring of length $n$, and $\kappa_{\tilde{C}_n}$ is isomomorphic to $\kappa_{\tilde{C}}$, hence to $\kappa_C$. By \cref{p-ring residue field lifting prop}, the ring $C$ and $\tilde{C}_n$ is isomorphic. 
\end{proof}
\begin{proposition}\label{p-ring residue perfect isomorphic to Witt ring}
Let $C$ be a $p$-ring, with residue field $k$ prefect. Suppose that $C$ has finite length $n$ (resp. infinite length). There exists a unique isomorphism $\rho_C:W_n(k)\to C$ (resp. $\rho_C:W(k)\to C$) which induces by passing to quotient the identity map on $k$.
\end{proposition}
\begin{proof}
Since $W_n(k)$ (resp. $W(k)$) is a $p$-ring with residue field $k$ and length $n$ (infinite), and $\emp$ is a $p$-basis for $k$, \cref{p-ring residue field lifting prop} then implies the claim.
\end{proof}
\subsection{Multiplicative representatives}
\begin{theorem}\label{Witt ring unique map to local sep complete ring}
Let $A$ be a separated and complete local ring, $k$ the residue field, and $\pi:A\to k$ the canonical map. Suppose that $k$ is perfect with characteristic $p$.
\begin{itemize}
\item[(a)] There exists a unique homomorphism $\rho:W(k)\to A$ such that $\pi(\rho(\bm{a}))=a_0$ for each $\bm{a}=(a_n)_{n\in\N}$ in $W(k)$.
\item[(b)] The homomorphism $\rho$ is continuous if we endow $W(k)$ with the $pW(k)$-adic topology, and its image is the unique Cohen subring of $A$.
\end{itemize}
\end{theorem}
\begin{proof}
By \cref{Cohen subring contain p-basis prop}, there exists a unique Cohen subring $C$ of $A$. Let $\rho:W(k)\to A$ be a ring homomorphism such that $\pi(\rho(\bm{a}))=a_0$ for $\bm{a}=(a_n)_{n\in\N}$ in $W(k)$; is is immediate that the image of $\rho$ is a Cohen subring of $A$, hence equal to $C$. The existence and uniqueness of $\rho$ then follows from \cref{p-ring residue perfect isomorphic to Witt ring}. The $pC$-adic topology on $C$ is induced by the $\m_A$-adic topology (\cref{Cohen subring contain p-basis prop}(b)), so $\rho$ is continuous.
\end{proof}
\begin{proposition}\label{local sep complete ring multiplicative rep}
Retain the hypothesis and notations in Therorem~\ref{Witt ring unique map to local sep complete ring}. There exist a multiplicative subset $S$ of $A$ such that $\pi$ induces a bijection of $S$ to $k$. For an element $a\in A$ to belong to $S$, it is necessary and sufficient that for each $n\in\N$, there exists an element $a_n$ in $A$ such that $a=a_n^{p^n}$. The set $S$ consists of elements of the form $\rho(x,0,0,\dots,)$.
\end{proposition}
\begin{proof}
Let $S$ be a multiplicative subset of $A$ such that $\pi$ induces a bijection of $S$ to $k$. Let $T$ be the set of elements of $A$ which are $p^n$-th powers for every $n\in\N$. Let $a\in S$ and $n\in\N$; since $k$ is perfect, there exists $x_n$ in $k$ such that $x_n^{p^n}=\pi(\bm{a})$. But $\pi(S)=k$, so there exists $a_n\in S$ such that $x_n=\pi(a_n)$. Then we have $\pi(a_n^{p^n})=\pi(a)$ whence $a_n^{p^n}=a$ since $\pi$ is injective on $S$. This shows $S\sub T$. Conversely, we show that the restriction of $\pi$ on $T$ is injective, which then proves that $S=T$: let $a$ and $b$ be elements of $T$ such that $\pi(a)=\pi(b)$. Let $n\in\N$; there exists two elements $a_n$ and $b_n$ of $A$ such that $a=a_n^{p^n}$, $b=b_n^{p^n}$. Then $\pi(a_n)^{p^n}=\pi(b_n)^{p^n}$, hence $\pi(a_n)=\pi(b_n)$. This means $a_n\equiv b_n$ mod $\m_A$, so $a_n^{p^n}\equiv b_n^{p^n}$ mod $\m_A^{n+1}$ (\cref{filtered ring p power lemma}). But then $a\equiv b$ mod $\m_A^{n+1}$. Since $n$ is arbitrary and $A$ is separated, we conclude that $a=b$.\par
In the notations of Therorem~\ref{Witt ring unique map to local sep complete ring}, let $\varphi=\rho\circ\tau_k$, which means
\[\varphi(x)=\rho(x,0,0,\dots)\]
for $x\in k$. Since $\tau_k$ is multiplicative (\cref{Witt ring map tau prop}), so is $\varphi$ and it is clear that $\pi\circ\varphi$ is the identity map on $k$. The image of $\varphi$ then satisfies the requirements.
\end{proof}
The elements of $S$ are often called the \textbf{multiplicative (or Teichmüller) representatives} of $A$. Recall that by \cref{Witt ring perfect p ring prop}, we have
\[\bm{a}=\sum_{n=0}^{\infty}p^n\tau_k(a_n^{p^{-n}})\]
for $\bm{a}=(a_n)_{n\in\N}$ in $W(k)$. Therefore
\[\rho(\bm{a})=\sum_{n=0}^{\infty}p^n\varphi(a_n^{p^{-n}})\]
because $\rho$ is continuous. By this formula, the unique Cohen subring of $A$ consists of elements of the form $\sum_{n=0}^{\infty}p^ns_n$ where $s_n\in S$ for each $n\in\N$.
\begin{example}[\textbf{Example of multiplicative representatives}]
\mbox{}
\begin{itemize}
\item[(a)] Let $k$ be a perfect field of characteristic $p$. The multiplicative representatives of the ring $W(k)$ is the Witt vectors $\tau(x)=(x,0,0,\dots)$ for $x\in k$.
\item[(b)] Let $A$ be an integral separated and complete local ring. We suppose that the residue field $k$ of $A$ is finite, with $q=p^f$ elements, hence perfect with characteristic $p$. We have $x^q=x$ for each $x\in k$, hence $s^q=s$ for each multiplicative representative. It follows that the set of multiplicative representatives consists of $0$ and the $q-1$-th roots of the unity in the field of fractions of $A$. If the field of fractions of $A$ is locally compact, the existence of the multiplicative representatives follows also from (AC, \Rmnum{6}, $\S$9, n2, prop.3).\par
In particular, consider the case $A=\Z_p$. Then the multiplicative representatives are $0$ and the $(p-1)$-th roots of unity in the field $Q_p$.
\end{itemize}
\end{example}
\subsection{Structure of Noetherian complete local rings}
We now use the results in this part to give a structural theorem for Noetherian complete local rings (such rings are separated by \cref{Noe ring Krull intersection thm}). Let $A$ and $C$ be Noetherian complete local rings and $\rho:C\to A$ a ring homomorphism which induces an isomorphism residue fields. Let $(p_1,\dots,p_m)$ be a sequence generating the maximal ideal $\m_C$ of $C$, and let $(t_1,\dots,t_n)$ be a sequence of elements of $A$. Let $S$ denote the sequence $(\rho(p_1),\dots,\rho(p_m),t_1,\dots,t_n)$.
\begin{lemma}\label{Noe complete local power series ring lemma}
Let $B=C\llbracket T_1,\dots,T_n\rrbracket$ and $\eta:B\to A$ be the unique homomorphism extending $\rho$ that sends $T_i$ to $t_i$ for each $i$.
\begin{itemize}
\item[(a)] For $\eta$ to be surjective, it is necessary and suffcient that $S$ generates the ideal $\m_A$ of $A$, or its class modulo $\m_A^2$ generates $\m_A/\m_A^2$ as a vector space over $\kappa_A$.
\item[(b)] For $\eta$ to make $A$ a finite $B$-algebra, it is necessary and sufficient that $S$ generates a defining ideal of the $\m_A$-adic topology on $A$.
\end{itemize}
\end{lemma}
\begin{proof}
Let $\n$ be the ideal of $B$ generated by $T_1,\dots,T_n$. Any homomorphism $\eta$ of $B$ to $A$ extending $\rho$ and such that $\eta(T_i)=t_i$ maps $\n$ to $\m_A$, hence is continuous when $B$ is endowed with the $\n$-adic topology. The existence and uniqueness of $\eta$ then follows from (A, \Rmnum{4}, p.26, prop.4).\par
The ring $B=C\llbracket T_1,\dots,T_n\rrbracket$ is a Noetherian complete local ring, and $\m_B$ is generated by $p_1,\dots,p_m,T_1,\dots,T_n$. We then have $\eta(\m_B)\sub\m_A$ and $\eta$ is defines a homomorphism $\gr(\eta):\gr(B)\to\gr(A)$. But the ring $\gr(A)$ is generated by $A/\m_A=\kappa_A$ and $\m_A/\m_A$, $\gr(\eta)$ induces an isomorphism of $\kappa_B=\kappa_C$ to $\kappa_A$, and the classes modulo $\m_B^2$ of the elements $p_1,\dots,p_m,T_1,\dots,T_n$ generate $\m_B/\m_B^2$ as a $\kappa_B$-vector space; moreover $\eta$ is surjective if and only if $\gr(\eta)$ is surjective (\cref{filtration gr(phi) surjective imply phi surjective if}). These prove (a).\par
The ideal $\a$ of $A$ generated by $S$ is equal to $\eta(\m_B)A$. Since $\m_A$ contains $\eta(\m_B)$, $A$ is a Zariski ring for the $\a$-adic topology. The ring $A/\a$ is Artinian if and only if its length as an $A$-module is finite. But as any simple module of $A$ is killed by $\m_A$ and, by hypothesis, $A/\m_A$ and $B/\m_B$ are isomorphic, this happens exactly when the dimension of the vector space $A/\a$ over $B/\m_B$ is finite. By \cref{Zariski ring quotient Artinian iff semilocal}, we see $\a$ is a defining ideal of $A$ if and only if the dimension of $A/\a$ over $B/\m_B$ is finite. This is indeed the case if $A$ is a finite $B$-algebra. Conversely, suppose then $\a$ is a defining ideal of $A$. The $\m_B$-adic topology on the $B$-module $A$ then coincides with the $\m_A$-adic topology on $A$, hence is separated. Since $A/\a$ is a finitely generated $B/\m_B$-module, $A$ is a finitely generated $B$-module (\cref{filtration finiteness of gr(M) imply that of M}). This proves (c).
\end{proof}
\begin{lemma}\label{Noe complete local regular power series ring lemma}
Suppose that $C$ is regular and $(p_1,\dots,p_m)$ is a system of parameters for $C$.
\begin{itemize}
\item[(a)] If the sequence $S$ is secant for $A$, the homomorphism $\eta:B\to A$ is injective. 
\item[(b)] For $\eta$ to be inejctive and make $A$ a finite $B$-algebra, it is necessary and sufficient that $S$ is a maximal secant sequence for $A$. In this case $A$ has dimension $m+n$.
\end{itemize}
\end{lemma}
\begin{proof}
For the sequence $S$ to be maximal secant for $A$, it is necessary and sufficient that it generates a defining ideal of $A$, and in that case $A$ has dimension $m+n$ (\cref{Noe local ring secant sequence prop} and \cref{Noe local ring defining ideal iff}). By \cref{Noe complete local power series ring lemma}(b), this is equivalent to that $A$ is a finite $B$-algebra, and has dimension $m+n$. Now $C$ is a Noetherian integral domain with dimension $m$, so $B=C\llbracket T_1,\dots,T_n\rrbracket$ is a Noetherian integral domain with dimension $m+n$. If $A$ is a finite $B$-algebra, let $\a$ be the kernel of $\eta$, then $\dim(A)=\dim(B/\a)$. Since $B$ is a Noetherian integral domain with finite dimension, we have $\dim(B/\a)<\dim(B)$ if $\a\neq\{0\}$. Thus, if $A$ is a finite $B$ algebra, $\eta$ is injective if and only if $A$ has dimension $m+n$. This proves (b).\par
Suppose that $S$ is secant for $A$. We may add elements $t_{n+1},\dots,t_{n+r}$ of $\m_A$ to form a maximal secant sequence. By the argument above, there exists an injective homomorphism $\tilde{\eta}$ from $C\llbracket T_1,\dots,T_n,T_{n+1},\dots,T_{n+r}\rrbracket=B\llbracket T_{n+1},\dots,T_{n+r}\rrbracket$ which extends $\eta$ and sends $T_{n+j}$ to $t_{n+j}$ for $1\leq j\leq r$. Hence $\eta$ is injective. This proves (a).
\end{proof}
\begin{theorem}[\textbf{Cohen Structure Theorem}]
Let $A$ be a Noetherian complete local ring with residue field $k$ of characteristic $p$. Let $C$ be a $p$-ring of infinite length, whose residue field is isomorphic to $k$.
\begin{itemize}
\item[(a)] If $n$ is the dimension of $\m_A/(\m_A^2+pA)$ over $k$, there exists an ideal $\a$ of $C\llbracket T_1,\dots,T_n\rrbracket$ such that $A$ is isomorphic to $C\llbracket T_1,\dots,T_n\rrbracket/\a$.
\item[(b)] Let $d$ be the dimension of $A$. Suppose that $p1_A$ is not a divisor of zero in $A$. Then there exists a subring $B$ of $A$ isomorphic to $C\llbracket T_1,\dots,T_{d-1}\rrbracket$ such that $A$ is a finite $B$-algebra.
\end{itemize}
\end{theorem}
\begin{proof}
Let $\tilde{C}$ be a Cohen subring of $A$. Since $C$ has infnite length, there exists a homomorphism of $C$ to $\tilde{C}$. Concequently, there exists a local homomorphism $\rho:C\to A$. Choose elements $t_1,\dots,t_n$ of $\m_A$ whose classes form a basis for $\m_A/(\m_A^2+pA)$ over $k$. We have $\rho(p1_C)=p1_A$, and \cref{Noe complete local power series ring lemma}(b) proves the existence of a surjective homomorphism of $C\llbracket T_1,\dots,T_m\rrbracket$ to $A$, which sends $T_i$ to $t_i$ for each $i$. This proves (a).\par
Suppose now that $p1_A$ is not a divisor of zero in $A$, hence secant for $A$. There exists by \cref{Noe local ring quotient by sequence regular iff} elements $t_1,\dots,t_{d-1}$ of $\m_A$ such that $(p1_A,t_1,\dots,t_{d-1})$ is a maximal secant sequence for $A$. The Noetherian local ring $C$ is regular (it is a DVR), and $p1_C$ is a system of parameters of $C$. Assertion (b) now follows from \cref{Noe complete local regular power series ring lemma}(b).
\end{proof}
\subsection{Coefficient fields}
Let $A$ be a ring. Recall that the characteristic of $A$ is defined if $A$ contains a subfield. It is equal to $0$ if and only if $A$ contains a subfield isomorphic to $\Q$, and equal to a prime number $p$ if and only if $p1_A=0$. If the characteristic of $A$ is defined and $\rho:A\to B$ is a nonzero ring homomorphism, then the characteristic of $B$ is defined and equal to $A$.\par
Now let $A$ be a local ring with maximal ideal $\m$ and residue field $k$. We distinguish two cases.
\begin{itemize}
\item[(a)] Suppose that $k$ has characteristic $0$. Then $A$ contains a field and the characteristic of $A$ is $0$. In fact, the canonical homomorphism of $\Z$ to $A$ is injective since the canonical map of $\Z$ to $k$ factors through it, and for any nonzero integer $n$, $n1_A$ is invertible in $A$ since it is not contained in the maximal ideal $\m$.
\item[(b)] Suppose that $k$ has characteristic $p\neq 0$. Then $A$ contains a field if and only if $p1_A=0$ (since such a field must have characteristic $p$), and in this case $A$ has characteristic $p$.
\end{itemize}
Now assume that $A$ is an integral domain with fraction field $K$. We now the field $K$ also has a characteristic, which is equal to that of $A$ (if the characteristic of $A$ is defined). Then $A$ contains a subfield if and only if the characteristic of $K$ and $k$ are equal. In this case, we say $A$ is a local ring with \textbf{equal characteristic}. We also note that, if the characteristic of $k$ and $K$ are unequal, then $k$ must have characteristic $p\neq 0$ with $K$ characteristic $0$. In this case we say $A$ is a local ring with \textbf{unequal characteristic}.
\begin{proposition}\label{local sep complete residue field extension lift prop}
Let $k_0$ be a field, $A$ a $k_0$-algebra and a separated and complete local ring, $k$ a sub-$k_0$-extension of $\kappa_A$ which has a separable transcendental basis $(\xi)_{\lambda\in\Lambda}$ over $k_0$. For any $\lambda\in\Lambda$, let $x_\lambda$ be a representative of $\xi_\lambda$ in $A$. Then there exists a unique subfield $K$ of $A$, containing $k_0$ and the elements $x_\lambda$, such that the canonical homomorphism $\pi:A\to\kappa_A$ induces an isomorphism of $K$ to $k$.
\end{proposition}
\begin{proof}
Let $\varphi$ be the $k_0$-homomorphism of the polynomial ring $k_0[(X_\lambda)_{\lambda\in\Lambda}]$ to $A$ which sends $X_\lambda$ to $x_\lambda$ for each $\lambda$. Let $u$ be a nonzero element in $k_0[(X_\lambda)_{\lambda\in\Lambda}]$; we have $\pi(\varphi(u))\neq 0$ since the family $(\xi)_{\lambda}$ is algebraically independent over $k_0$ in $\kappa_A$, so $\varphi(u)$ is invertible in $A$. It follows that $\varphi$ extens to a homomorphism $\psi$ of the field $k_1=k_0((X_\lambda)_{\lambda\in\Lambda})$ to $A$. Then $A$ is a $k_1$- algebra, $\kappa_A$ is an extension of $k_1$ and $k$ a subextension of $\kappa_A$ which is algebraic and separable over $k_1$. It suffices to prove that there exists a unique subfield $K$ of $A$ containing $\psi(k_1)$ and such that $\pi(K)=k$.\par
Let $\mathcal{S}$ be the set of subfield $K$ of $A$, containing $\psi(k_1)$ and such that $\pi(K)\sub k$; this is inductive with respect to the inclusion relation. Let $K$ be a maximal element in $\mathcal{S}$; we consider $k$ as an extension (algebraic and separable) of $K$. Let $\xi\in k$ and $P\in K[X]$ its minimal polynomial over $k$. Since $\xi$ is a simple root of $P$, the Hensel lemma (\cref{Hensel ring root of restricted power series}) then shows the existence of an element $x\in A$ such that $\pi(x)=\xi$ and $P(x)=0$. The subfield $K(x)$ of $A$ then belongs to $\mathcal{S}$, so $x\in K$ by the maximality of $K$. This shows $\xi\in\pi(K)$, so $\pi(K)=k$.\par
We now show the uniqueness of $K$. Let $K$ and $K'$ be subfields of $A$ containing $\psi(k_1)$ and such that $\pi(K)=\pi(K')=k$. Let $\xi\in k$, and let $x\in K$, $x'\in K'$ be elements such that $\pi(x)=\pi(x')=\xi$. If $P\in k_1[X]$ is the minimal polynomial of $\xi$ over $k_1$, then $\xi$ is a simple root of $P$, and we have $P(x)=P(x')=0$. By Hensel lemma we then have $x=x'$, so $K=K'$.
\end{proof}
\begin{remark}
The previous proof applies more generally to the case where we only assume that A is a Henselian local ring. The uniqueness proof uses the assumption that the local ring $A$ is separate, but not that it is complete.
\end{remark}
Let $A$ be a local ring. A \textbf{coefficient field} of $A$ is defined to be a subfield $K$ of $A$ such that the canonical homomorphism of $A$ to $\kappa_A$ induces an isomorphism of $K$ to $\kappa_A$ (in other words, $A=K+\m_A$). It is clear that such a field exists only when the characteristic of $A$ is defined. This condition is also sufficient when $A$ is separated and complete. More precisely, we have the following theorem:
\begin{theorem}\label{local sep complete coefficient field exist}
Let $A$ be a seperated and complete local ring of equal characteristic $p$.
\begin{itemize}
\item[(a)] Suppose $p=0$ and let $(x_\lambda)_{\lambda\in\Lambda}$ be a family of elements of $A$ whose class modulo $\m_A$ forms a transcendental basis for $\kappa_A$ over $\Q$. Then there exists a unique coefficient field of $A$ containing the elements $x_\lambda$.
\item[(b)] Suppose $p\neq 0$ and let $(x_\lambda)_{\lambda\in\Lambda}$ be a family of elements of $A$ whose class modulo $\m_A$ forms a $p$-basis for $\kappa_A$. Then there exists a unique coefficient field of $A$ containing the elements $x_\lambda$, which is a Cohen subring of $A$.
\end{itemize}
\end{theorem}
\begin{proof}
Suppose that $p=0$, so $A$ is a $\Q$-algebra. Since any transcendental basis for $\kappa_A$ over $\Q$ is separable, the assertion in (a) follows from \cref{local sep complete residue field extension lift prop} applied to $k_0=\Q$, $k=\kappa_A$.\par
Now suppose that $p\neq 0$. Then we have $p1_A=0$, and any Cohen subring $C$ of $A$ satisfies $pC=0$. Therefore, the notations of coefficient field and Cohen subring coincide on $A$. Assertion (b) therefore follows from \cref{Cohen subring contain p-basis prop}.
\end{proof}
\begin{corollary}\label{local sep complete residue algebra over Q rep field unique}
Let $A$ be a separated and complete local ring, whose residue field is an algebraic extension of $\Q$. Then there exists a unique coefficient field of $A$.
\end{corollary}
\begin{proof}
In fact, the ring $A$ is of characteristic $0$, and the empty set is a transcendental basis for $\kappa_A$ over $\Q$.
\end{proof}
\begin{corollary}
Let $A$ be a separated and complete local ring, with residue field of characteristic $p\neq 0$. Suppose that the residue field $\kappa_A$ is perfect. Then there exists a unique coefficient field of $A$, which coincide with the multiplicative representatives.
\end{corollary}
\begin{proof}
This follows from \cref{local sep complete coefficient field exist}(b) and \cref{local sep complete ring multiplicative rep}.
\end{proof}
\begin{theorem}\label{Noe local complete coefficient field prop}
Let $A$ be a Noetherian complete local ring with dimension $d$. Let $K$ be a coefficient field of $A$, and $m$ the dimension of $\m_A/\m_A^2$ over $K$.
\begin{itemize}
\item[(a)] There exists an ideal $\a$ of $K\llbracket T_1,\dots,T_m\rrbracket$ such that $A$ is isomorphic to $K\llbracket T_1,\dots,T_m\rrbracket/\a$.
\item[(b)] There exists a sub-$K$-algebra $B$ of $A$, isomorphic to $K\llbracket T_1,\dots,T_{d}\rrbracket$, such that $A$ is a finite $B$-algebra.
\item[(c)] If $A$ is regular then there exists an isomorphism of $A$ to $K\llbracket T_1,\dots,T_d\rrbracket$.
\end{itemize}
\end{theorem}
\begin{proof}
Let $t_1,\dots,t_m$ be elements of $\m_A$ whose classes in $\m_A$ form a basis for $\m_A/\m_A^2$. By \cref{Noe complete local power series ring lemma}, there exists a surjective $K$-homomorphism of $K\llbracket T_1,\dots, T_m\rrbracket$ to $A$, sending $T_i$ to $t_i$. This proves (a). Similarly, (b) follows from \cref{Noe complete local regular power series ring lemma}, and the existence of a maximal secant sequence for $A$. Finally, assertion (c) follows from \cref{Noe local ring residue regular iff power series}.
\end{proof}
\section{Integral closure of complete local rings}
\subsection{Japanese rings}
Let $A$ be a Noetherian integral domain. It is called an \textbf{N-$1$ ring} if the integral closure of $A$ in its fraction field $K$ is a finite $A$-algebra. It is called an \textbf{N-$2$ ring} (or \textbf{Japanese}) if the integral closure of $A$ in any finite extension of $K$ is a finite $A$-algebra. In other words, $A$ is Japanese if it satisfies the following condition: any integral $A$-algebra $B$ integral over $A$, contained in a finite type extension of the field of fractions $K$ of $A$, is a finite $A$-algebra. Indeed, the field of fractions $L$ of $B$ is an algebraic extension of $K$, so is of finite degree over $K$. The $A$-algebra $B$ is contained in the integral closure of $A$ in $L$, and is therefore finite if the latter is finite.\par
We note that Japanese rings are stable under localization. That is, if $A$ is a Noetherian integral Japanese ring and $S$ a multiplicative subset of $A$ non containing $0$, then the fraction ring $S^{-1}A$ is Japanese. This follows from the observation that the integral closure of $S^{-1}A$ is the localization of that of $A$, and hence is finite over $A$ if $A$ is Japanese.
\begin{example}
We recall that, by \cref{algebra finite integral closure in finite algebraic extension is finite}, any integral $k$-algebra of finite type, where $k$ is a field, is a Japanese ring. Note that this is true for the coordinate ring of an algebraic variety.
\end{example}
\begin{proposition}\label{japanese iff finite radical extension}
Let $A$ be a Noetherian integral domain, with fraction field $K$. Suppose that for any finite purely inseparable extension $L$ of $K$, the integral closure of $A$ in $L$ is a finite $A$-algebra. Then $A$ is Japanese.
\end{proposition}
\begin{proof}
Let $E$ be a finite extension of $K$. Let $N$ be the normal closure of $E$ over $K$ and $L$ the invariant field of $\Aut_K(N)$. Then $L$ is the purely inseparable closure of $K$ in $N$ and the extension $L/N$ is separable (\cref{field ext normal inseparable closure is fixed field}). The integral closure $B$ of $A$ in $L$ is then by hypothesis a finite $A$-algebra; the integral closure $C$ of $B$ in $E$ is a finite $B$-algebra by \cref{integral closure in finite separable finite if Noe}, which contains the integral closure of $A$ in $E$. Since $A$ is Noetherian, this completes the proof. 
\end{proof}
\begin{corollary}\label{japanese iff integrally closed if perfect frac}
Suppose that $K$ is perfect (for example of characteristic $0$). Then $A$ is Japanese if and only if it is integrally closed.
\end{corollary}
\begin{proposition}\label{japanese stable under finite ring extension}
Let $B$ be a Noetherian integral domain and $A$ a Noetherian subring of $B$, such that $B$ is a finite $A$-algebra. For $A$ to be Japanese, it is necessary and sufficient that $B$ is Japanese.
\end{proposition}
\begin{proof}
Let $K$ (resp. $L$) be the fraction field of $A$ (resp. $B$). Suppose that $A$ is Japanese, and let $E$ be a finite extension of $L$. Let $C$ be the integral closure of $B$ in $E$. By \cref{integral element under homomorphism}, $C$ is also the integral closure of $A$ in $E$, hence is a finite $A$-algebra (since $E$ is a finite extension of $K$ and $A$ is Japanese). A fortiori, $C$ is a finite $B$-algebra, this proves $B$ is Japanese.\par
Conversely, suppose that $B$ is Japanese and let $N$ be a finite extension of $K$. Let $D$ be the integral closure of $A$ in $N$. Let $E$ be the composed extension of $L$ and $N$ (over $K$); since $B$ is Japanese, the integral closure $D'$ of $B$ in $E$ is a finite $B$-algebra, hence a finite $A$-algebra; the $A$-module $D$ is a sub-module of $D'$, so is finite over $A$. This shows $A$ is Japanese. 
\end{proof}
\begin{theorem}[\textbf{Tate}]\label{Noe integrally closed Japanese if quotient by a}
Let $A$ be an integrally closed Noetherian ring, $a$ an element of $A$. Suppose that the ideal $aA$ is prime, the ring $A/aA$ is Japanese and $A$ is complete for the $aA$-adic topology. Then the ring $A$ is Japanese.
\end{theorem}
\begin{proof}
Let $K$ be the fraction field of $A$. The assertion is trivial if $K$ has characteristic $0$. We may therefore assume that $K$ has characteristic $p>0$. Also, suppose that $a\neq 0$. Let $L$ be a finite purely inseparable extension of $K$ and $q=p^e$ a power of $p$ such that $L\sub K^{1/q}$. By extending our field, we may assume that $a=x^q$ for some $x\in L$. It then suffices to prove that the integral closure $B$ of $A$ is a finite $A$-algebra.\par
We claim that $xB$ is the unique ideal of $B$ lying over $aA$ (such an ideal exists by going up theorem). In fact, let $\mathfrak{P}$ be such an ideal. We have $x^q=a\in\mathfrak{P}$, hence $xB\sub\mathfrak{P}$ since $\mathfrak{P}$ is prime. Conversely, let $y$ be an element of $\mathfrak{P}$; the element $y^q$ of $K$ is integral over $A$, hence belongs to $A$ (recall that $A$ is integrally closed). Since $\mathfrak{P}\cap A=aA$, there exist $b\in A$ such that $y^q=ab=x^qb$. Then the lement $y/x$ of $L$ is integral over $A$, hence contained in $B$; this means $y\in xB$, hence $\mathfrak{P}=xB$.\par
Recall that the ring $B_{xB}$ is the integral closure of $A_{aA}$ in $L$. By \cref{integral domain DVR iff}, $A_{aA}$ is a DVR; we then deduce from the Krull-Akizuki theorem that $B_{xB}$ is Noetherian (the ring $\kappa(xB)$ is a finite extension of $\kappa(aA)$). The ring $B/xB$ is integral over the Japanese ring $A/aA$ and its fraction field is a finite extension of that of $A$. Concequently, $B/xB$ is a finitely generated $(A/aA)$-module. For each integer $i\geq 0$, this is isomorphic to the module $x^iB/x^{i+1}B$; so the $(A/aA)$-module $B/aB$ possesses a composition sequence of length $q$ with quotients finitely generated $(A/aA)$-modules, hence is also finitely generated over $(A/aA)$.\par
Endow the ring $A$ with the $aA$-adic topology and $B$ the $aB$-adic topology. Then $A$ is complete by hypothesis; since $B_{xB}$ is a Noetherian integral domain, the $aB_{xB}$-adic filtration on $B_{xB}$ is separated (Krull intersection theorem); therefore we have $\bigcap_na^nB\sub\bigcap_na^nB_{xB}=\{0\}$, and the $aB$-adic filtration on $B$ is separated. The $\gr_{aA}(A)$-module $\gr_{aB}(B)$ is generated by $\gr_0(B)$, hence is finitely generated. It then follows from \cref{filtration finiteness of gr(M) imply that of M} that $B$ is a finitely generated $A$-module, which completes the proof.
\end{proof}
\begin{corollary}
Let $R$ be a Noetherian integral domain and $n$ an integer. If $R$ is Japanese, so is the ring $R\llbracket T_1,\dots,T_n\rrbracket$.
\end{corollary}
\begin{proof}
Argue by induction, we may assume that $n=1$. Let $S$ be the integral closure of $R$. If $R$ is Japanese, $S$ is a finite $R$-algebra, hence a Japanese ring (\cref{japanese stable under finite ring extension}). The ring $S\llbracket T\rrbracket$ is Noetherian and integrally closed, so by applying \cref{Noe integrally closed Japanese if quotient by a} on $A=S\llbracket T\rrbracket $ and $a=T$, we deduce that $S\llbracket T\rrbracket$ is Japanese. Thus $R\llbracket T\rrbracket$ is Japanese (\cref{japanese stable under finite ring extension}).  
\end{proof}
\begin{theorem}[\textbf{Nagata}]\label{Noe integral local complete is Japanese}
Every Noetherian integral complete local ring is Japanese.
\end{theorem}
\begin{proof}
By \cref{local sep complete coefficient field exist}, there exist an integer $n\geq 0$, a ring $R$ that is a field of a DVR whose fraction field has characteristic $0$, and a subring $B$ of $A$, isomorphic to $R\llbracket T_1,\dots,T_n\rrbracket$, such that $A$ is a finite $B$-algebra. Since $R$ is Japanese, we see $B$ is Japanese, hence so is $A$.
\end{proof}
A Noetherian semi-local ring $A$ is called \textbf{analytically unramified} if its completion $\widehat{A}$ is reduced. A prime ideal $\p$ of $A$ is said to be \textbf{analytically unramified} if $\widehat{A}/\widehat{\p}=\widehat{A/\p}$ is reduced. \cref{Noe integral local complete is Japanese} then imply the following important result:
\begin{corollary}\label{Noe semilocal comlete integral clo in Q(A) finite}
Let $A$ be a semi-local Noetherian ring which is analytically unramified. Then the integral closure $\widebar{A}$ of $A$ in the total fraction ring $Q(A)$ is a finite $A$-algebra.
\end{corollary}
\begin{proof}
Suppose first that $A$ is local and complete, with $\p_1,\dots,\p_n$ the minimal prime ideals of $A$. For each $i=1,\dots,n$, let $K_i$ be the fraction field of $A/\p_i$ and $\widebar{A}_i$ the integral closure of $A/\p_i$ in $K_i$. Since $A$ is reduced, $Q(A)$ is the product of the $K_i$ and $\widebar{A}$ is the product of $\widebar{A}_i$ (\cref{Noe reduced integral closure in Q(A)}). Since the local ring $A/\p_i$ is integral and complete, it is Japanese (\cref{Noe integral local complete is Japanese}), so $\widebar{A}_i$ is a finite $A_i$-algebra, and therefore $\widebar{A}$ is finite over $A$. If $A$ is semi-local and complete, it is isomorphic to a finite product of complete local rings (\cref{filtration completion of semilocal ring with Jacobson radical}), and we conclude the claim by the previous case.\par
In the general case, note that the completion $\widehat{A}$ of $A$ is semi-local, complete, Noetherian and faithfully flat ($A$ is a Zariski ring). Let $S$ be the set of elements of $A$ that are not zero divisors; we have $Q(A)=S^{-1}A$. Since $\widehat{A}$ is flat over $A$, the elements of $S$ are not zero divisors in $\widehat{A}$, and $S^{-1}\widehat{A}$ is identified with a subring of $Q(\widehat{A})$. Moreover, the ring $\widebar{A}\otimes_A\widehat{A}$ is identified as a subring of $Q(A)\otimes_A\widehat{A}$, hence a subring $T$ integral over $\widehat{A}$. The preceding arguments then apply and show that $\widebar{A}\otimes_A\widehat{A}$ is a finitely generated $\widehat{A}$-module. Therefore, $\widebar{A}$ is a finitely generated $A$-module (\cref{module extension to faithfully flat ring finiteness iff}).
\end{proof}
Recall that an algebra $E$ over a field $K$ is called separable if the ring $L\otimes_KE$ is reduced for any extension $L$ of $K$. The following proposition generalizes \cref{Noe integral local complete is Japanese}.
\begin{proposition}\label{Noe semilocal integral tensor reduced then Japanese}
Let $A$ be a semi-local Noetherian integral ring, $K$ its fraction field. If the $K$-algebra $K\otimes_A\widehat{A}$ is separable, the ring $A$ is Japanese.
\end{proposition}
\begin{proof}
Let $L$ be a finite extension of $K$ and $B$ the integral closure of $A$ in $L$. Let $F$ be a finite subset of $B$ such that $L=K[F]$ (\cref{integral closure in finite separable contained in finite}); let $C$ be the $A$-algebra generated by $F$. Then $L$ is the fraction field of $C$, the ring $B$ is the integral closure of $C$, and it suffices to prove that $B$ is a finite $C$-algebra. But $C$ is a Noetherian semi-local ring, with completion identified with $C\otimes_A\widehat{A}$, which is a subring of the reduced ring $L\otimes_A\widehat{A}=L\otimes_K(K\otimes_A\widehat{A})$ and therefore reduced. The proposition then follows from \cref{Noe semilocal comlete integral clo in Q(A) finite}.
\end{proof}
Finally, before considering Nagata ring, we prove some lemmas which will be used latter.
\begin{lemma}\label{Noe semilocal finite algebra completion iso}
Let $A$ be a Noetherian semi-local ring and $B$ a finite $A$-algebra. Then $B$ is Noetherian and semi-local; let $\m_1,\dots,\m_n$ be its maximal ideals. The canonical homomorphism of $B$ to $\prod_{i=1}^{n}\widehat{B}_{\m_i}$ then extends to an isomorphism of $\widehat{A}\otimes_AB$ to $\prod_{i=1}^{n}\widehat{B}_{\m_i}$.
\end{lemma}
\begin{proof}
By \cref{Noe semilocal ring finite extension is Noe semilocal}, the ring $B$ is semi-local and $\m_AB$ is a defining ideal. By \cref{filtration Noe I-adic completion is tensor}(b), the ring $\widehat{A}\otimes_AB$ is the completion of $B$ for the topology defined by its Jacobson radical. Then we can apply \cref{filtration completion of semilocal ring with Jacobson radical}.
\end{proof}
\begin{lemma}\label{Noe module injective to prod of Ass localization}
Let $A$ be a Noetherian ring and $M$ an $A$-module. The canonical map of $M$ into $\prod_{\p\in\Ass(M)}M_\p$ is injective.
\end{lemma}
\begin{proof}
Let $m$ be a nonzero element in $M$; then $\Ann(m)$ is contained in a prime ideal $\p$ in $\Ass(M)$ (\cref{associated prime maximal element of Ann}), and the image of $m$ in $M_\p$ is thus nonzero. 
\end{proof}
\begin{lemma}\label{Noe local ring local DVR ana unramified then Ass}
Let $A$ be a Noetherian local ring and $\p$ a prime ideal of $A$. Assume that $A_\p$ is a DVR and $\p$ is analytically unramified, then for any associated prime $\hat{\q}$ of $\widehat{A}/\p\widehat{A}$, the local ring $\widehat{A}_{\hat{\q}}$ is a DVR.
\end{lemma}
\begin{proof}
Let $\pi$ be a uniformizer for $A_\p$. Since $\p$ is analytically unramified, the associated primes of $\widehat{A}/\p\widehat{A}$ are exactly the minimal primes of $\widehat{A}$ over $\p\widehat{A}$, and in particular there is no embedded associated primel. Also, since $\widehat{A}/\p\widehat{A}=\widehat{A/\p}$ is reduced, we have $\{0\}=\bigcap\hat{\q}/\p\widehat{A}$ in $\widehat{A}/\p\widehat{A}$, where $\hat{\q}$ runs through minimal primes over $\p\widehat{A}$. Therefore in $\widehat{A}$ we have
\[\p\widehat{A}=\bigcap_{\hat{\q}\in\Ass(\widehat{A}/\p\widehat{A})}\hat{\q}\]
which is then the unique primary decomposition of $\p\widehat{A}$ in $\widehat{A}$. By \cref{primary decomposition minimal part and saturation}, the $\hat{\q}$-primary component of this decomposition is given by the saturation of $\p\widehat{A}$ with respect to $\hat{\q}$ (which is equal to $\hat{\q}$), so in particular, we have
\[\hat{\q}\widehat{A}_{\hat{\q}}=\p\widehat{A}_{\hat{\q}}=\pi\widehat{A}_{\hat{\q}}.\]
Since $\pi$ is $\widehat{A}$-regular by the flatness of $\widehat{A}$ over $A$, the local ring $\widehat{A}_{\hat{\q}}$ is regular of dimension $1$, hence a DVR.
\end{proof}
\begin{lemma}\label{Noe ring ana unramified if no embedded and at point}
Let $A$ be a Noetherian semi-local integral domain with maximal ideal $\m$. Let $x$ be a nonzero element in $\m$ such that
\begin{itemize}
\item[(a)] $A/xA$ has no embedded primes;
\item[(b)] for each associated prime $\p$ of $A/xA$, the local ring $A_\p$ is regular and $\p$ is analytically unramified. 
\end{itemize}
Then $A$ is analytically unramified.
\end{lemma}
\begin{proof}
Let $\p_1,\dots,\p_r$ be the associated primes of $A/xA$. Since $A/xA$ has no embedded primes, we see that each $\p_i$ has height $1$, and is a minimal prime over $xA$. For each $i$, let $\hat{\q}_{i,1},\dots,\hat{\q}_{i,n_i}$ be the associated primeds of the $\widehat{A}$-module $\widehat{A}/\p_i\widehat{A}$. By \cref{Noe local ring local DVR ana unramified then Ass} we see that $\widehat{A}_{\hat{\q}_{i,j}}$ is a DVR for each $i,j$. Also, since $\widehat{A}$ is flat over $A$, by \cref{associated prime extension scalar M otimes N} we have
\[\Ass_{\widehat{A}}(\widehat{A}/x\widehat{A})=\bigcup_{\p\in\Ass_A(A/xA)}\Ass_{\widehat{A}}(\widehat{A}/\p\widehat{A})=\{\hat{\q}_{i,j}\}.\]
Let $y$ be a nonzero nilpotent in $\widehat{A}$. Then since $\widehat{A}_{\hat{\q}_{i,j}}$ is integral for each $i,j$, the element $y$ is mapped to zero in $\widehat{A}_{\hat{\q}_{i,j}}$. By \cref{Noe module injective to prod of Ass localization}, the canonical image of $y$ in $\widehat{A}/x\widehat{A}$ is then zero, which means $y=xy'$ for some $y'\in\widehat{A}$. But since the homothety with ratio $x$ is injective on $A$, it is also injective on $\widehat{A}$, which implies $y'\in\widehat{A}$ is also a nonzero nilpotent element. Repeating this process, we conclude that $y\in\bigcap_nx^n\widehat{A}=\{0\}$ (\cref{Noe ring Krull intersection thm}), which shows $\widehat{A}$ is reduced and $A$ is therefore analytically unramified.
\end{proof}
\subsection{Nagata rings}
A Noetherian ring is called Nagata if for any prime ideal $\p$ of $A$, the Noetherian integral domain $A/\p$ is Japanese. In this part, we give some of the properties of Nagata rings, and a criterion of Noetherian semi-local rings to be Nagata.
\begin{example}[\textbf{Example of Nagata rings}]
\mbox{}
\begin{itemize}
\item[(a)] Any algebra of finite type over a field is a Nagata ring.
\item[(b)] Any Noetherian complete local ring is a Nagata ring by \cref{Noe integral local complete is Japanese}.
\item[(c)] The ring $\Z$ is a Nagata ring since any quotient $\Z/p\Z$ is a perfect field (hence Japanese).
\item[(d)] We will see that any algebra of finite type over a Nagata ring is Nagata. 
\end{itemize}
\end{example}
\begin{proposition}\label{Nagata ring stable under finite ext and localization}
Let $A$ be a Nagata ring.
\begin{itemize}
\item[(a)] Any finite $A$-algebra is a Nagata ring.
\item[(b)] For any multiplicative subset $S$ of $A$, the ring $S^{-1}A$ is a Nagata ring. 
\end{itemize}
\end{proposition}
\begin{proof}
Let $B$ be a finite $A$-algebra and $\rho:A\to B$ the homomorphism. For each prime ideal $\mathfrak{P}$ of $B$, the ring $B/\mathfrak{P}$ is a finite algebra of the Japanese ring $A/\p^c$, hence is Japanese.\par
Let $S$ be a multiplicative subset of $A$ and $\q$ a prime ideal of $S^{-1}A$; then there exist a prime ideal $\p$ of $A$ such that $\q=S^{-1}\p$. The ring $S^{-1}A/\q$ is a fraction ring of the Japanese ring $A/\p$, hence is Japanese.
\end{proof}
\begin{theorem}[\textbf{Nagata}]\label{Noe local reduced Nagata is ana unramified}
Let $A$ be a Noetherian semi-local integral domain. If $A$ is a Nagata ring then it is analytically unramified.
\end{theorem}
\begin{proof}
We use induction on $\dim(A)$. Let $B$ be the integral closure of $A$ in its fraction field $K$. Then $B$ is finite over $A$, hence $\dim(A)=\dim(B)$ and for any $\mathfrak{P}\in\Spec(B)$, the domain $B/\mathfrak{P}$ is finite over $A/\p$ (where $\p=\mathfrak{P}\cap A$), which is assumed to be Japanese. Thus $B/\mathfrak{P}$ is Japanese and $B$ is therefore a Nagata ring. Moreover, the defining ideal of $B$ is given by the extension of that of $A$, which implies $\widehat{B}=B\otimes_A\widehat{A}$; since $\widehat{A}$ is a flat $A$-module, we can then identify $\widehat{A}$ as a subring of $\widehat{B}$. Thus we can replace $A$ by its integral closure and assume that $A$ is integrally closed. In this case, for any nonzero element $x\in A$, the $A$-module $A/xA$ has no embedded primes by \cref{Krull Noe domain divisorial iff prime ideal belong height 1}. If $\p\in\Ass(A/xA)$, then $A/\p$ is a Nagata domain and $\dim(A/\p)<\dim(A)$, hence $\p$ is analytically unramified by the induction hypothesis. Moreover, $A_\p$ is a DVR because $\height(\p)=1$ (\cref{Krull domain iff prime of height 1} and \cref{Krull Noe domain divisorial iff prime ideal belong height 1}). So the conditions of \cref{Noe ring ana unramified if no embedded and at point} are satisfied, and $A$ is analytically unramified.
\end{proof}
\begin{theorem}[\textbf{Zariski-Nagata}]\label{Nagata ring char by reduced}
Let $A$ be a Noetherian semi-local ring. The following conditions are equivalent:
\begin{itemize}
\item[(\rmnum{1})] $A$ is a Nagata ring;
\item[(\rmnum{2})] for any prime ideal $\p$ of $A$, the $\kappa(\p)$-algebra $\kappa(\p)\otimes_A\widehat{A}$ is separable; 
\item[(\rmnum{3})] for any reduced $A$-algebra $R$, the ring $R\otimes_A\widehat{A}$ is reduced.
\end{itemize}
\end{theorem}
\begin{proof}
We first demonstrate the equivalence of (\rmnum{2}) and (\rmnum{3}). The implication (\rmnum{3})$\Rightarrow$(\rmnum{2}) is trivial; suppose conversely that $A$ satisfies (\rmnum{2}). Then for any $A$-algebra $K$ that is a field, the ring $K\otimes_A\widehat{A}$ is reduced since it is isomorphic to $K\otimes_{\kappa(\p)}\kappa(\p)\otimes_A\widehat{A}$. Now let $C$ be a reduced $A$-algebra of finite type, then the ring $C$, being Noetherian, is isomorphic to a subring of a finite product $K_1\times\cdots\times K_n$ of fields (\cref{Noe ring minimal prime and Ass localization prop}). Since $\widehat{A}$ is flat over $A$, the ring $C\otimes_A\widehat{A}$ is isomorphic to a subring of the reduced ring $\prod_i(K_i\otimes_A\widehat{A})$, hence is reduced. Finally, since any reduced $A$-algebra $R$ is the union of a family of finite type $A$-algebras $(C_\alpha)$, we see $R\otimes_A\widehat{A}$ is reduced (it is the inductive limit of the family $(C_\alpha\otimes_A\widehat{A})$).\par
We now show that (\rmnum{2})$\Leftrightarrow$(\rmnum{1}). Let $\p$ be a prime ideal of $A$. The fraction field $K$ of $A/\p$ is then identified with $\kappa(\p)$, and the $K$-algebra $K\otimes_{A/\p}\widehat{A/\p}$ is identified with $\kappa(\p)\otimes_{A/\p}\widehat{A}/\p\widehat{A}$, hence to $\kappa(\p)\otimes_A\widehat{A}$. If $\kappa(\p)\otimes_A\widehat{A}$ is a separable $\kappa(\p)$-algebra, then $A/\p$ is Japanese by \cref{Noe semilocal integral tensor reduced then Japanese}.\par Conversely, we prove (\rmnum{1})$\Rightarrow$(\rmnum{2}). Let $A$ be a semi-local Noetherian Nagata ring, let $\p$ be an ideal of $A$, and $L$ a finite extension of $\kappa(\p)$. It suffices to show that the ring $L\otimes_A\widehat{A}$ is reduced. Let $B$ by the integral closure of $A/\p$ in $L$; since $A/\p$ is Japanese, $B$ is a finite $A$-algebra hence a semi-local Nagata ring (\cref{Nagata ring stable under finite ext and localization}). Let $\m_1,\dots,\m_r$ be the maximal ideal of $B$; the ring $L\otimes_A\widehat{A}$ is identified with a fraction ring of $B\otimes_A\widehat{A}$, and with that of a product of complete local rings $\widehat{B}_{\m_i}$ (\cref{Noe semilocal finite algebra completion iso}). It then suffices to prove that, for any maximal ideal $\m$ of $B$, the ring $\widehat{B}_\m$ is reduced (\cref{localization and nilradical}). But the ring $B_\m$ is a Noetherian local ring that is Nagata (\cref{Nagata ring stable under finite ext and localization}), so by \cref{Noe local reduced Nagata is ana unramified} it is analytically unramified, which is what we want.
\end{proof}
\begin{corollary}[\textbf{Chevally}]\label{finite reduced k-alg local ana unramified}
Let $A$ be a reduced algebra of finite type over a field, and $\p$ a prime ideal of $A$. Then the local ring $A_\p$ is analytically unramified.
\end{corollary}
\begin{proof}
Since $A$ is reduced, the local ring $A_\p$ is reduced. The ring $A$ is Nagata, hence $A_\p$ is also Nagata, and \cref{Noe local reduced Nagata is ana unramified} implies the claim.
\end{proof}
\begin{corollary}\label{Nagata iff ana unramified for local k-alg of char 0}
Let $k$ be a field of characteristic $0$ and $A$ a Noetherian local $k$-algebra. Then $A$ is a Nagata ring if and only if every prime ideal $\p$ of $A$ is analytically unramified.
\end{corollary}
\begin{proof}
In fact, since the residue field $\kappa(\p)$ has characteristic $0$, the $\kappa(\p)$-algebra $\kappa(\p)\otimes_A\widehat{A}=\kappa(\p)\otimes_{A/\p}\widehat{A/\p}$ is separable if and only if it is reduced, which shows that the stated condition is sufficient (\cref{Nagata ring char by reduced}, not that $\kappa(\p)\otimes_{A/\p}\widehat{A/\p}$ is a localization of $\widehat{A/\p}$). It is also necessary by take $R=A/\p$ in \cref{Nagata ring char by reduced}(\rmnum{3}).
\end{proof}
A Noetherian ring $A$ is called \textbf{universally Japanese} if any integral $A$-algebra $B$ of finite type over $A$ is Japanese. It is clear that such a ring is Nagata, since $A/\p$ is a finite type $A$-algebra for any prime ideal $\p$ of $A$. We now prove the reverse direction, which is not at all obvious. For this, we first establish some lemmas. 
\begin{lemma}\label{Noe domain Nor set open if localization normal}
Let $A$ be a Noetherian integral domain and put $X=\Spec(A)$. Suppoe that there exist nonzero element $f$ in $A$ such that $A_f$ is integrally closed, then the set
\[\Nor(X)=\{\p\in\Spec(A):\text{$A_\p$ is integrally closed}\}\]
is open in $X$.
\end{lemma}
\begin{proof}
If $f\notin\p$ for some $\p\in X$, then $A_\p$ is a localization of $A_f$, hence integrally closed. This shows $D(f)\sub\Nor(X)$. Now we put
\[E=\{\p\in\Ass_A(A/fA):\text{either $\height(\p)=1$ and $A_\p$ is not regular, or $\height(\p)>1$}\}.\]
Since $\Ass_A(A/fA)$ is finite, it is clear that $E$ is a finite subset of $X$. On the other hand, by Serre's criterion for normality, if $\p\notin\Nor(X)$ then there exist a prime $\q\sub\p$ such that either $\height(\q)\geq 2$ and $\depth(A_\q)<2$, or $\height(\q)=1$ and $A_\q$ is not regular. This in particular also means that $A_{\q}$ is not integrally closed, and hence $f\in\q$. In both cases we see $\q$ is an associated prime of $A/fA$. It then follows that $\Nor(X)=X\setminus\bigcup_{\p\in E}V(\p)$, so $\Nor(X)$ is open in $X$.
\end{proof}
\begin{lemma}\label{Noe domain N-1 iff maximal localization}
Let $A$ be a Noetherian domain with quotient field $K$. Then $A$ is N-$1$ if and only if the following conditions hold:
\begin{itemize}
\item[(a)] there exists a nonzero element $f\in A$ such that $A_f$ is integrally closed;
\item[(b)] $A_\m$ is N-$1$ for each maximal ideal $\m$ of $A$.
\end{itemize}
\end{lemma}
\begin{proof}
First assume that $A$ is N-$1$. Let $\widebar{A}$ be the integral closure of $A$ in its fraction field $K$. By assumption we can find $x_1,\dots,x_n$ in $\widebar{A}$ which generate it as an $A$-module. Since $\widebar{A}\sub K$, we may further assume that $x_i=a_i/b$ for $a_i,b\in A$. Then $A_f\cong\widebar{A}_{f}$, where the latter is the integral closure of the former. This shows $A_f$ is integrally closed, hence (a) holds. It is clear that (b) follos, since the integral closure of $A_\m$ is just $\widebar{A}_\m$.\par
Conversely, assume (a), (b) for $A$ and let $K$ be the fraction field of $A$. For a subring $B$ of $K$ that is finite over $A$, we set $X=\Spec(A)$, $Y=\Spec(B)$. Note that $A_f=B_f$ since $A_f$ is integrally closed, so by \cref{Noe domain Nor set open if localization normal} the set $\Nor(Y)$ is open in $Y$. Since the canonical map $\widebar{X}\to X$ is closed (\cref{Spec of ring map going up and going down}), the image of the complement of $\Nor(Y)$ is then closed in $X$, we denote it by $Z_B$ for such a ring $B$.\par
Pick a maximal ideal $\m$ of $A$ and let $A_\m\sub\widebar{A}_\m$ be the integral closure of the local ring in $K$. By assumption, this is a finite ring extension, so we can find finitely many elements $x_1,\dots,x_n$ in $\widebar{A}$ such that $\widebar{A}_\m$ is generated by $x_1,\dots,x_n$ over $A_\m$. Let $B(\m)=A[x_1,\dots,x_n]\sub K$, then $B(\m)$ is finite over $A$, hence is Noetherian. Let $\mathfrak{M}$ be any prime ideal of $B(\m)$ lying over $\m$, then
\[B(\m)_{\mathfrak{M}}\supset B(\m)_\m\supset B(\m),\]
and $B(\m)_\m=\widebar{A}_\m$ is integrally closed. Thus $B(\m)_{\mathfrak{M}}$ is a localization of the integrally closed ring $\widebar{A}_\m$, so is integrally closed. This shows $\m\notin Z_{B(\m)}$, so the intersection $\bigcap_{\m\in\Max(A)}Z_{B(\m)}$ contains no closed point, and is therefore empty (since it is a closed subset of $X$). As $X$ is quasi-compact, we can choose maximal ideals $\m_1,\dots,\m_r$ such that $\bigcap_i Z_{B(\m_i)}=\emp$. Put $B_i=B(\m_i)$ and let $C$ be the ring generated by all of these; it is finite over $A$. We claim that $Z_C=\emp$, in other words, $C_{\mathfrak{Q}}$ is integrally closed for any $\mathfrak{Q}\in\Spec(C)$. In fact, the contraction of any prime $\mathfrak{Q}$ of $C$ into $A$ is not contained in some $Z_{B_i}$, so it lies over a prime $\mathfrak{P}_i$ of $B_i$ such that $(B_i)_{\mathfrak{P}_i}$ is integrally closed. This implies $C_{\mathfrak{Q}}=(B_i)_{\mathfrak{P}_i}$ is integrally closed too, so $C$ is integrally closed (\cref{integral closed is local property}). In other words, $C$ is the integral closure of $A$ in $K$, so $A$ is N-$1$. 
\end{proof}
\begin{theorem}[\textbf{Nagata}]\label{Nagata iff universally japanese}
Let $A$ be a Noetherian ring. The following are equivalent:
\begin{itemize}
\item[(\rmnum{1})] $A$ is a Nagata ring;
\item[(\rmnum{2})] any finite type $A$-algebra $B$ is Nagata;
\item[(\rmnum{3})] $A$ is universally Japanese.
\end{itemize}
\end{theorem}
\begin{proof}
It is clear that a Noetherian universally Japanese ring is universally Nagata (i.e., condition (\rmnum{2}) holds). Let $A$ be a Noetherian Nagata ring. We will show that any finite type $A$-algebra $B$ is Nagata. Note that the canonical image of $A$ in $B$ is also a Nagata ring, so we may assume that $A$ is a subring of $B$. Then $B=A[x_1,\dots,x_n]$ with $x_i\in B$, and by induction it suffices to consider the case $B=A[x]$.\par
Let $\mathfrak{P}$ be a prime ideal of $B$ and $\p=\mathfrak{P}^c$. Then $B/\mathfrak{P}=(A/\p)[\bar{x}]$, where $A/\p$ is a Nagata domain (\cref{Nagata ring stable under finite ext and localization}), and we must show that $B/\mathfrak{P}$ is Japanese. We are then reduced to prove the following assertion:
\begin{itemize}[leftmargin=30pt]
\item[(A1)] If $A$ is a Noetherian Nagata domain and $B=A[x]$ is an integral domain generated by a single $x$ over $A$, then $B$ is Japanese.
\end{itemize}
Let $K$ be the fraction field of $A$ and $\widebar{A}$ the integral closure of $A$ in $K$. Let $\widebar{B}$ be the subring of the fraction field of $B$ generated by $\widebar{A}$ and $B$. As $\widebar{A}$ is finite over $A$ (by the Nagata property, since $A$ is an integral domain), $\widebar{B}$ is finite over $B$. Since $B$ is Noetherian, it then suffices to prove that $\widebar{B}$ is Japanese (\cref{japanese stable under finite ring extension}). Hence we can add the integrally closed assumption in (A1):
\begin{itemize}[leftmargin=30pt]
\item[(A2)] If $A$ is an integrally closed Noetherian Nagata domain and $B=A[x]$ is an integral domain generated by a single $x$ over $A$, then $B$ is Japanese.
\end{itemize}
In this case, let $L$ be the fraction field of $B$. We first note that, if $x$ is transcendental over $A$, then $B=A[x]$ is isomorphic to the polynomial ring $A[X]$, hence is integrally closed. Thus if $L$ has characteristic $0$, we are done (\cref{japanese iff integrally closed if perfect frac}). Suppose otherwise $L$ has characteristic $p>0$, and take a finite purely inseparable extension $E$ of $L=K(x)$. Let $q=p^e$ be a power of $p$ such that $E\sub L^{1/q}$. Then there exists a finite purely inseparable extension $K'$ of $K$ such that $E\sub K'(x^{1/q})$. If $\widebar{A}$ (resp. $\widebar{B}$) is the integral closure of $A$ in $K'$ (resp. of $B$ in $L$), then $\widebar{A}[x^{1/q}]$ is normal and we have $B=A[x]\sub\widebar{B}\sub\widebar{A}[x^{1/q}]$. Since $\widebar{A}[x^{1/q}]$ is finite over $B$, we conclude that $\widebar{B}$ is finite over $B$.\par
Now assume that $x$ is algebraic over $A$, so that $L/K$ is a finite extension. Let $E$ be a finite extension of $L$ (hence finite over $K$) and $\widebar{A}$ be the integral closure of $A$ in $E$. Then the integral closure $\widebar{B}$ of $B$ in $E$ is equal to the integral closure of $\widebar{A}[x]$ in $E$. Also the fraction field of $\widebar{A}$ is $E$ and $\widebar{A}$ is a finite $A$-algebra ($A$ is Japanese). This implies that $\widebar{A}$ is a Nagata ring (\cref{Nagata ring stable under finite ext and localization}). To show that $\widebar{B}$ is finite over $B$ is the same as showing that $\widebar{B}$ is finite over $\widebar{A}[x]$, so by replacing $A$ by $\widebar{A}$ and $B$ by $\widebar{B}[x]$, we are reduced to prove the following statement:
\begin{itemize}[leftmargin=30pt]
\item[(A3)] If $A$ is an integrally closed Noetherian Nagata domain with fraction field $K$ and $x$ is an element of $K$, the ring $B=A[x]$ is N-$1$.
\end{itemize}
We now proceed with proving (A3). Since $x\in K$, we can write $x=a/b$ where $a,b\in A$. Then $B_a=B[1/a]=A[1/a]$ is integrally closed since it is a localization of $A$. By \cref{Noe domain N-1 iff maximal localization}, it then suffices to prove that $B_{\mathfrak{M}}$ is N-$1$ for any maximal ideal $\mathfrak{M}$ of $B$. Now pick such a maximal ideal and set $\m=\mathfrak{M}\cap A$. The residue field extension $\kappa(\m)/\kappa(\mathfrak{M})$ is finite and generated by the image of $x$. Hence there exists a monic polynomial $f(X)\in A[X]$ with $f(x)\in\mathfrak{M}$. Let $E/K$ be a finite extension of fields such that $f(X)$ splits completely in $E[X]$. Let $\widebar{A}$ be the integral closure of $A$ in $E$, and $\widebar{B}\sub E$ be the subring generated by $\widebar{A}$ and $x$. As $A$ is a Nagata ring, we see $\widebar{A}$ is finite over $A$ and hence Nagata (\cref{Nagata ring stable under finite ext and localization}). Moreover, $\widebar{B}$ is finite over $B$. If for every maximal ideal $\widebar{\mathfrak{M}}$ of $\widebar{B}$ the local ring $\widebar{B}_{\widebar{\mathfrak{M}}}$ is N-$1$, then $\widebar{B}_{\widebar{\mathfrak{M}}}$ is N-$1$ by \cref{Noe domain N-1 iff maximal localization}, which in turn implies that $B_{\mathfrak{M}}$ is N-$1$ (by a proof similar to that of \cref{japanese stable under finite ring extension}). Thus, after replacing $A$ by $\widebar{A}$ and $B$ by $\widebar{B}$, and $\mathfrak{M}$ by any of the maximal ideals $\widebar{\mathfrak{M}}$ lying over $\mathfrak{M}$, we reach the situation where the polynomial $f(X)$ above split completely: $f(X)\prod_{i=1}^{d}(X-a_i)$ with $a_i\in A$. Since $f(x)\in\mathfrak{M}$ we see that $x-a_i\in\mathfrak{M}$ for some $i$. Finally, after replacing $x$ by $x-a_i$ we may assume that $x\in\mathfrak{M}$.\par
We will show that \cref{Noe ring ana unramified if no embedded and at point} applies to the local ring $B_{\mathfrak{M}}$ and the element $x$. This will imply that $B_{\mathfrak{M}}$ is analytically unramified, whence it is N-$1$ by \cref{Noe semilocal comlete integral clo in Q(A) finite}. Let $Q$ be the kenrel of the canonical map $A[X]\to B$, where $X$ is mapped to $x$. Then by \cref{integral closed domain ring A[x] prop}, the ideal $Q$ is generated by all linear forms $aX-b$, where $ax=b$ in $K$. Therefore, we see that
\[B/xB=A[X]/(x,\a)=A/I\]
where $I$ is the constant term of the elements in $Q$, or in other words $I=xA\cap A$. Now note that $A$ is a Noetherian integrally closed domain and $I$ is divisorial, so by \cref{Krull domain Noetherian primary decomposition}, if $\div(xA\cap A)=\sum_in_iP_i$ with $\p_i$ the corresponding prime of $P_i$, then
\[I=xA\cap A=\bigcap_i\p_i^{(n_i)}\]
is the unique reduced primary decomposition of $I$ and the $\p_i$ are isolated primes of $I$. This shows $A/I=B/xB$ has no embedded primes, so does $B_{\mathfrak{M}}$, in view of \cref{associated prime of localization}.\par
Now let $\mathcal{P}\in\Spec(B_{\mathfrak{M}})$ be an associated prime of $B_{\mathcal{P}}/xB_{\mathfrak{M}}$. Then $\p=\mathcal{P}\cap A$ is an associated prime of $A/(xB_{\mathfrak{M}}\cap A)=A/I$ and $\height(\p)=1$ by \cref{Krull Noe domain divisorial iff prime ideal belong height 1}, so $(B_{\mathfrak{M}})_{\mathcal{P}}=A_\p$ is a DVR (\cref{Krull domain iff prime of height 1}), hence regular. Finally, $B_{\mathfrak{M}}/\mathcal{P}$ is a localization of $B/(\mathcal{P}\cap B)$ (which is isomorphic to $A/\p$ since $x\in\p$), so $B_{\mathfrak{M}}/\mathcal{P}$ is a Noetherian Nagata local domain, hence analytically unramified. Thus the conditions in \cref{Noe ring ana unramified if no embedded and at point} are verified and our proof is completed.
\end{proof}
\chapter{Depth, regularity, and duality}
\section{Depth of modules}
\subsection{Homological definition of depth}
Let $A$ be a ring, $\mathfrak{I}$ an ideal of $A$, and $M$ an $A$-module. We define the \textbf{depth of $\bm{M}$ relative to $\mathfrak{I}$}, denoted by $\depth_A(\mathfrak{I},M)$ or $\depth(\mathfrak{I},M)$, to be the infimum in $\N\cup\{+\infty\}$ of the set of integers $i$ such that $\Ext_A^i(A/\mathfrak{I},M)$ is nonzero. If $A$ is a local ring, the depth of $M$ relative to the maximum ideal $\m_A$ of $A$ is simply called the \textbf{depth of $\bm{M}$} and denoted by $\depth_A(M)$ or $\depth(M)$; the \textbf{depth of the local ring $\bm{A}$} is defined to be the depth of the $A$-module $A$.
\begin{example}\label{depth of module infty iff M=IM}
Let $M$ be a fnitely generated $A$-module and such that $M=\mathfrak{I} M$, which means $\supp(M)\cap V(\mathfrak{I})=\emp$ (\cref{supp of module quotient by ideal product}). In this case, $\depth_A(\mathfrak{I},M)$ is equal to $+\infty$: in fact, the ideal $\Ann(M)+\mathfrak{I}$ is equal to $A$ (since $V(\Ann(M)+\mathfrak{I})=\emp$) and is contained in the annihilator of $\Ext_A^i(A/\mathfrak{I},M)$ for each $i$. Conversely, we shall see that (\cref{depth of module fg ideal bounded by generator number}) if $\mathfrak{I}$ is a finitely generated ideal, then $\depth_A(\mathfrak{I},M)=+\infty$ implies $M=\mathfrak{I} M$.
\end{example}
\begin{example}\label{depth of module zero iff annihilated by element}
For $\depth_A(\mathfrak{I},M)$ to be zero, it is necessary and sufficient that $\Hom_A(A/\mathfrak{I},M)$ is zero, which signifies that $M$ has a nonzero element annihilated by $\mathfrak{I}$. This in particular implies that $\Ass(M)\cap V(\mathfrak{I})\neq\emp$ (\cref{associated prime maximal element of Ann}). If $A$ is Noetherian and the $A$-module $M$ is finitely generated, the following conditions are equivalent (\cref{associated prime ideal finite if Noe finite}):
\begin{itemize}
\item[(\rmnum{1})] $\depth_A(\mathfrak{I},M)=0$;
\item[(\rmnum{2})] for any $x\in\mathfrak{I}$, the homothety $x_M$ is not injective; 
\item[(\rmnum{3})] $\Ass(M)\cap V(\mathfrak{I})\neq\emp$.
\end{itemize} 
In particular, for a local ring to have zero depth, it is necessary and sufficient that there exist a nonzero element $x$ of $A$ such tha t$\m_Ax=0$. If $A$ is not a field, such an element is then not invertible, hence belongs to $\m_A$ and satisfies $x^2=0$. Therefore, any reduced local ring of dimension $\geq 1$ has depth $\geq 1$.
\end{example}
\begin{example}\label{depth of module product is inf}
Let $\{M_i\}_{i\in I}$ be a family of $A$-module and $\mathfrak{I}$ be an ideal of $A$. Then in view of (A, \Rmnum{10} p.89, prop.7), we have $\depth_A(\mathfrak{I})=\inf_{i\in I}\depth_A(\mathfrak{I},M_i)$.
\end{example}
\begin{proposition}\label{depth of module exact sequence prop}
Let $A$ be a ring, $\mathfrak{I}$ an ideal of $A$, and let
\[\begin{tikzcd}
0\ar[r]&M'\ar[r]&M\ar[r]&M''\ar[r]&0
\end{tikzcd}\]
be an exact sequence of $A$-modules. Then there are exactly three cases:
\begin{itemize}
\item[(\rmnum{1})] $\depth(\mathfrak{I},M)=\depth(\mathfrak{I},M')\leq\depth(\mathfrak{I},M'')$;
\item[(\rmnum{2})] $\depth(\mathfrak{I},M)=\depth(\mathfrak{I},M'')\leq\depth(\mathfrak{I},M')$;
\item[(\rmnum{3})] $\depth(\mathfrak{I},M'')=\depth(\mathfrak{I},M')-1<\depth(\mathfrak{I},M'')$;
\end{itemize}
\end{proposition}
\begin{proof}
Consider the long exact sequence associated with this short exact sequence:
\[\begin{tikzcd}[column sep=4mm]
0\ar[r]&\Ext_A^0(A/\mathfrak{I},M')\ar[r]&\Ext_A^0(A/\mathfrak{I},M)\ar[r]&\Ext_A^0(A/\mathfrak{I},M'')\ar[r]&\Ext_A^1(A/\mathfrak{I},M')\ar[r]&\cdots
\end{tikzcd}\]
We may exclude the case $\depth(\mathfrak{I},M')=\depth(\mathfrak{I},M)=\depth(\mathfrak{I},M')=+\infty$. There then exists in this sequence a first non-zero module, and the next module is thus nonzero. The three cases then correspond to which module is the first to be nonzero.
\end{proof}
\begin{remark}\label{depth of module exact sequence annihilated by A/I prop}
Suppose that we have $\depth(\mathfrak{I},M')=\depth(\mathfrak{I},M)$ and the injection $u:M'\to M$ belongs to $\mathfrak{I}\Hom_A(M',M)$ (which is equivalent to that $1_{A/\mathfrak{I}}\otimes u$ is zero). We then have $\depth(\mathfrak{I},M'')=\depth(\mathfrak{I},M)-1$: in fact, the hypotheses implies that $\Ext_A^i(1_{A/\mathfrak{I}},u)=0$ for any integer $i$, which means the homomorphism $\Ext_A^i(A/\mathfrak{I},M'')\to\Ext_A^i(A/\mathfrak{I},M'')$ is surjective.
\end{remark}
\begin{proposition}\label{depth of module Ext with annhilated module}
Let $A$ be a ring, $\mathfrak{I}$ an ideal of $A$, $M$ an $A$-module and $N$ an $A$-module killed by a power of $\mathfrak{I}$. Then $\Ext_A^i(N,M)=0$ for any integer $i<\depth_A(\mathfrak{I},M)$.
\end{proposition}
\begin{proof}
We first assume that $\mathfrak{I}N=0$ and prove by induction on the integer $i<\depth_A(\mathfrak{I},M)$. The assertion is clear for $i<0$; now consider $N$ as a $(A/\mathfrak{I})$-module and choose an exact sequence of $(A/\mathfrak{I})$-modules
\[\begin{tikzcd}
0\ar[r]&K\ar[r]&(A/\mathfrak{I})^{\oplus I}\ar[r]&N\ar[r]&0
\end{tikzcd}\]
We then deduce a long exact sequence of extension modules
\[\begin{tikzcd}[column sep=5mm]
\cdots\ar[r]&\Ext_A^{i-1}(K,M)\ar[r]&\Ext_A^i(N,M)\ar[r]&\Ext_A^i((A/\mathfrak{I})^{\oplus I},M)\ar[r]&\Ext^i_A(K,M)\ar[r]&\cdots 
\end{tikzcd}\]
The $A$-module $\Ext_A^{i-1}(K,M)$ is zero by induction hypothesis, and $\Ext_A^i((A/\mathfrak{I})^{\oplus I},M)$ is isomorphic to $\Ext_A^i((A/\mathfrak{I}),M)^{\oplus I}$ (A, \Rmnum{10}, p.89, prop.7), which is zero by the definition of $\depth_A(\mathfrak{I},M)$. We therefore conclude that $\Ext_A^i(N,M)=0$.\par
Passing to the general case, we preceed by induction on the smallest number $m>0$ such that $\mathfrak{I}^mN=0$. We have proved the case $m=1$, so assume that $m>1$ and let $i<\depth_A(\mathfrak{I},M)$ be an integer. Consider the sequence
\[\begin{tikzcd}
\Ext_A^i(N/\mathfrak{I}N,M)\ar[r]&\Ext_A^i(N,M)\ar[r]&\Ext_A^i(\mathfrak{I}N,M)
\end{tikzcd}\]
The extremal modules are zero by induction hypothesis, since $N/\mathfrak{I}N$ and $\mathfrak{I}N$ are annihilated by $\mathfrak{I}^{m-1}$. We then deduce that $\Ext_A^i(N,M)=0$, which completes the proof.
\end{proof}
\begin{corollary}\label{depth of module ideal contained in power prop}
Let $m>0$ be an integer and $\mathfrak{I}'$ be an ideal of $A$ contained in $\mathfrak{I}^m$. Then for any $A$-module $M$ we have $\depth_A(\mathfrak{I},M)\leq\depth_A(\mathfrak{I}',M)$.
\end{corollary}
\begin{proof}
In fact, $\mathfrak{I}^m$ annihilates the $A$-module $A/\mathfrak{I}'$, so $\Ext_A^i(A/\mathfrak{I}',M)=0$ for $i<\depth_A(\mathfrak{I},M)$ by \cref{depth of module Ext with annhilated module}. 
\end{proof}
\begin{corollary}\label{depth of module fg ideal V(I) inequality}
Suppose that the ideal $\mathfrak{I}$ is finitely generated, and let $\mathfrak{I}'$ be an ideal of $A$ such that $V(\mathfrak{I}')\sub V(\mathfrak{I})$. Then we have $\depth_A(\mathfrak{I},M)\leq\depth_A(\mathfrak{I}',M)$. Moreover, if the ideal $\mathfrak{I}'$ is also finitely generated and $V(\mathfrak{I})=V(\mathfrak{I}')$, then  $\depth_A(\mathfrak{I},M)\leq\depth_A(\mathfrak{I}',M)$.
\end{corollary}
\begin{proof}
By \cref{Spec of ring closed subsets prop}, there exists an integer $m>0$ such that $\mathfrak{I}^m\sub\mathfrak{I}'$. The first assertion therefore follows from \cref{depth of module ideal contained in power prop}, and the second one follows also from this.
\end{proof}
We conclude this paragraph by a cohomological application of depth. This result can also be servered as our motivation of defining the depth via extension module.
\begin{proposition}\label{depth of module homology null if large depth}
Let $A$ be a ring, $C_\bullet$ be a right bounded complex of $A$-modules, and $n_0$ be an integer. Suppose that for any integers $m\geq n\geq n_0$, the depth of the $A$-module $C_m$ relative to the annihilator of $H_n(C_\bullet)$ is strictly bigger than $m-n$, then we have $H_n(C_\bullet)=0$ for $n\geq n_0$.
\end{proposition}
\begin{proof}
Since $C_\bullet$ is right bounded, we have $H_n(C_\bullet)=0$ for $n\gg 0$. If the conclusion is false, then there exists an integer $m\geq n_0$ such that $H_n(C_\bullet)=0$ for $n>m$ and $H_m(C_\bullet)\neq 0$. We denote by $\mathfrak{I}$ the annihilator of $H_m(C_\bullet)$, so that $\depth_A(\mathfrak{I},H_m(C_\bullet))=0$ by \cref{depth of module zero iff annihilated by element}. Moreover, since $Z_m(C_\bullet)$ is a submodule of $C_m$, and by hypothesis we have $\depth_A(\mathfrak{I},C_m)>m-m=0$, we conclude that $\depth_A(\mathfrak{I},Z_m(C_\bullet))>0$. From the exact sequence
\[\begin{tikzcd}
0\ar[r]&B_m(C_\bullet)\ar[r]&Z_m(C_\bullet)\ar[r]&H_m(C_\bullet)\ar[r]&0
\end{tikzcd}\]
we then deduce that $\depth_A(\mathfrak{I},B_m(C_\bullet))=1$ (\cref{depth of module exact sequence prop}). Now by the definition of $m$, we have $B_n(C_\bullet)=Z_n(C_\bullet)=0$ for $n>m$, so from the canonical exact sequences
\[\begin{tikzcd}
0\ar[r]&B_n(C_\bullet)\ar[r]&C_n\ar[r]&B_{n-1}(C_\bullet)\ar[r]&0
\end{tikzcd}\]
and the hypothesis $\depth_A(\mathfrak{I},C_n)>n-m$, we derive by induction that $\depth_A(\mathfrak{I},B_n(C_\bullet))=n-m+1$ for any $n\geq m$ (\cref{depth of module exact sequence prop}). But this is absurd since $B_n(C_\bullet)=0$ for $n\gg 0$.
\end{proof}
\begin{corollary}\label{depth of module homology null if annhilated H_n}
Let $A$ be a ring, $\mathfrak{I}$ be an ideal of $A$, $C_\bullet$ be a right bounded complex of $A$-modules, and $n_0$ be an integer. Suppose that $\mathfrak{I}H_n(C_\bullet)=0$ and $\depth_A(\mathfrak{I},C_n)>n-n_0$ for $n\geq n_0$. Then we have $H_n(C_\bullet)=0$ for $n\geq n_0$.
\end{corollary}
\begin{proof}
In fact, for $n\geq n_0$ the annihilator $\mathfrak{I}_n$ of $H_n(C_\bullet)$ contains $\mathfrak{I}$, so we have $\depth_A(\mathfrak{I}_n,C_m)\geq\depth_A(\mathfrak{I},C_m)$ for $m\geq n\geq n_0$ (\cref{depth of module ideal contained in power prop}), and \cref{depth of module homology null if large depth} then proves the corollary.
\end{proof}
\begin{corollary}\label{depth of module local ring homology null if finite length}
Let $A$ be a local ring with maximal $\m$, $C_\bullet$ be a right bounded complex of $A$-modules, and $n_0$ be an integer. Suppose that for $n\geq n_0$, $H_n(C_\bullet)$ is of finite length and $\depth_A(\m,C_n)>n-n_0$. Then we have $H_n(C_\bullet)=0$ for $n\geq n_0$.
\end{corollary}
\begin{proof}
The $A$-module $\bigoplus_{n\geq n_0}H_n(C_\bullet)$ is then of finite length. Let $\mathfrak{I}$ be its annihilator, then $A/\mathfrak{I}$ is Artinian, so $\mathfrak{I}$ is contained in a power of the maximal ideal $\m$ of $A$. We then have $\depth_A(\mathfrak{I},C_n)\geq\depth_A(\m,C_n)>n-n_0$ (\cref{depth of module ideal contained in power prop}), so we can apply \cref{depth of module homology null if large depth}.
\end{proof}
\subsection{Depth and Koszul complex}
Let $A$ be a ring, $M$ be an $A$-module, and $\bm{x}=(x_i)_{i\in I}$ be a family of elements of $A$. Let $K^\bullet(\bm{x},M)$ be the Koszul complex associated with $\bm{x}$ and $M$. By definition, we have $K^p(\bm{x},M)=0$ for $p<0$, and for $p\geq 0$ the $A$-module $K^p(\bm{x},M)=\Hom_A(\bigw^pA^{\oplus I},M)$ is canonically identified with the $A$-module $C^p(M)$ formed by alternating maps from $I^p$ to $M$, with differential $\partial^p:K^p(\bm{x},M)\to K^{p+1}(\bm{x},M)$ given by the formula
\[(\partial^pm)(\alpha_1,\dots,\alpha_{p+1})=\sum_{j=1}^{p+1}(-1)^{j+1}x_{\alpha_j}\cdot m(\alpha_1,\dots,\hat{\alpha}_i,\dots,\alpha_{p+1})\]
where $m\in K^p(\bm{x},M)$ and $(\alpha_1,\dots,\alpha_{p+1})\in I^{p+1}$. In particular, the complex $K^\bullet(\bm{x},M)$ only depends on the $\Z$-module structure on $M$ and the endomorphisms $(x_i)_M$.\par
We denote by $H^\bullet(\bm{x},M)$ the cohomology of the complex $K^\bullet(\bm{x},M)$. By definition, the $A$-module $H^0(\bm{x},M)$ is identified with $\Hom_A(A/\mathfrak{I},M)$, where $\mathfrak{I}$ is the ideal generated by the $x_i$.
\begin{theorem}\label{depth of module Koszul cohomology char}
Let $A$ be a ring, $\mathfrak{I}$ be an ideal of $A$, $\bm{x}=(x_i)_{i\in I}$ be a family of generator of $\mathfrak{I}$, and $M$ be an $A$-module. Then the depth of $M$ relative to $\mathfrak{I}$ is the infermum of integers $i$ such that $H^i(\bm{x},M)\neq 0$.
\end{theorem}
\begin{proof}
We put $p=\depth_A(\mathfrak{I},M)$, and consider the complex $K^\bullet(\bm{x},M)$. Its cohomology is annihilated by $\mathfrak{I}$ (\cref{Koszul complex annihilator prop}), and the depth of $K^i(\bm{x},M)$ relative to $\mathfrak{I}$ is equal to $p$ of $+\infty$ since $K^i(\bm{x},M)$ is either isomorphic to a product of $M$ or zero (\cref{depth of module product is inf}). By applying \cref{depth of module homology null if annhilated H_n} to the relabeled complex $K_{-i}(\bm{x},M)$ (defined by $K_{-i}(\bm{x},M)=K^i(\bm{x},M)$), we conclude that $H^i(\bm{x},M)=0$ for $i<p$. It then remains to prove that $H^p(\bm{x},M)$ is nonzero if $p<+\infty$.\par
The case $p=0$ is evident, so we suppose that $0<p<+\infty$ and $H^p(\bm{x},M)=0$. Let $L^\bullet$ be a free resolution of the $A$-module $A/\mathfrak{I}$, and denote by $C^\bullet$ the Hom complex $\Hom(L^\bullet,M)$. The $A$-module $H^i(C^\bullet)$ is then isomorphic to $\Ext_A^i(A/\mathfrak{I},M)$, and is hence zero for $i<p$. For each $i<p$, we then have a canonical exact sequence
\[\begin{tikzcd}
0\ar[r]&B^i(C^\bullet)\ar[r]&C^i\ar[r]&B^{i+1}(C^\bullet)\ar[r]&0
\end{tikzcd}\]
Since each $C^i$ is isomorphic to a product of $M$, by hypothesis we have $H^n(\bm{x},C^i)=0$ for $n\leq p$, and the connecting homomorphism $\partial^n:H^n(\bm{x},B^{i+1}(C^\bullet))\to H^{n+1}(\bm{x},B^i(C^\bullet))$ in the induced long exact sequence is then injective for $n\leq p$ and $i<p$. As $B^0(C^\bullet)=0$, it then follows that $H^{p-i}(\bm{x},B^{i+1}(C^\bullet))=0$ for $i<p$, and in particular $H^1(\bm{x},B^p(C^\bullet))=0$. Now from the exact sequence
\[\begin{tikzcd}
0\ar[r]&B^p(C^\bullet)\ar[r]&Z^p(C^\bullet)\ar[r]&H^p(C^\bullet)\ar[r]&0
\end{tikzcd}\]
we then obtain a surjection $H^0(\bm{x},Z^p(C^\bullet))\to H^0(\bm{x},H^p(C^\bullet))$. As the $A$-module $H^p(C^\bullet)$ is isomorphic to $\Ext_A^p(A/\mathfrak{I},M)$, which is nonzero and annihilated by $\mathfrak{I}$, we finally conclude that $H^0(\bm{x},H^p(C^\bullet))\neq 0$, whence $H^0(\bm{x},Z^p(C^\bullet))\neq 0$ and then $H^0(\bm{x},C^p)\neq 0$. But then $H^0(\bm{x},M)\neq 0$, which is a contradiction and therefore compeltes the proof.
\end{proof}
\begin{corollary}\label{depth of module fg ideal bounded by generator number}
Suppose that $\mathfrak{I}$ is finitely generated and $\mathfrak{I}M\neq M$. Then $\depth_A(\mathfrak{I},M)\leq|I|$, and for the equality holds, it is necessary and sufficient that the family $\bm{x}$ is complete secant for $M$.
\end{corollary}
\begin{proof}
Suppose that $I$ is finite and $n=|I|$. Then the $A$-module $H^n(\bm{x},M)$ is canonically isomorphic to $H_0(\bm{x},M)$, hence to $M/\mathfrak{I}M$. The inequality $\depth_A(\mathfrak{I},M)\leq n$ then follows from \cref{depth of module Koszul cohomology char}. For the equality holds, it is necesary and sufficient that the $A$-modules $H^i(\bm{x},M)=0$ for $i<n$, which means that $\bm{x}$ is complete secant for $M$.
\end{proof}
\begin{corollary}\label{depth of module local ring bounded by generator mod m_A}
Let $A$ be a local ring, $\mathfrak{I}$ be a finitely generated proper ideal of $A$, and $M$ be a finitely generated nonzero $A$-module. Put $r=[\mathfrak{I}/\m_A\mathfrak{I}:\kappa_A]$, then we have $\depth_A(\mathfrak{I},M)\leq r$, and the equality holds if and only if $\mathfrak{I}$ is generated by a family that is complete secant for $M$. In this case, for a generating family of $\mathfrak{I}$ to be complete secant for $M$, it is necesary and sufficient that it consists of $r$ elements.
\end{corollary}
\begin{proof}
By Nakayama's lemma, we have $\mathfrak{I}M\neq M$, and $r$ is the minimal number of generators of $\mathfrak{I}$. The assertion then follows from \cref{depth of module fg ideal bounded by generator number}.
\end{proof}
\begin{proposition}\label{depth of module localization char}
Let $A$ be a ring, $\mathfrak{I}$ be a finitely generated ideal of $A$, and $M$ be an $A$-module. Denote by $\Omega$ the set of maximal ideals of $V(\mathfrak{I})\cap\supp(M)$. Then we have
\[\depth_A(\mathfrak{I},M)=\inf_{\p\in\Spec(A)}\depth_{A_\p}(\mathfrak{I}_\p,M_\p)=\inf_{\m\in\Omega}\depth_{A_\m}(\mathfrak{I}_{\m},M_{\m}).\] 
\end{proposition}
\begin{proof}
Let $\bm{x}=(x_i)_{i\in I}$ be a finite generating family of $\mathfrak{I}$. Let $\p$ be a prime ideal of $A$, then the ideal $\mathfrak{I}_\p$ is generated by the image $\bm{x}_\p$ of the family $\bm{x}$ in $A_\p$. For any $p\geq 0$, the $A_{\p}$-module $(H^p(\bm{x},M))_\p$ is isomorphic to $H^p(\bm{x}_\p,M_{\p})$, so by \cref{depth of module Koszul cohomology char} we have
\[\depth_A(\mathfrak{I},M)\leq\inf_{\p\in\Spec(A)}\depth_{A_\p}(\mathfrak{I}_\p,M_\p)=\inf_{\m\in\Omega}\depth_{A_\m}(\mathfrak{I}_{\m},M_{\m}).\]

Let $p$ be an integer that is strictly smaller than $\depth_{A_\m}(\mathfrak{I}_{\m},M_{\m})$ for any $\m\in\Omega$. We then have $H^p(\bm{x}_\m,M_\m)=0$ for any maximal ideal $\m$ of $A$: this follows from \cref{depth of module Koszul cohomology char} if $\m\in\Omega$, and from that fact that $M_\m=0$ if $\m\notin\supp(M)$, and that $\mathfrak{I}A_\m=A_\m$ if $\m\notin V(\mathfrak{I})$ (which annihilates $H^p(\bm{x}_\m,M_\m)$). We then conclude that $(H^p(\bm{x},M_\m))_m=0$ for any maximal ideal $\m$ of $A$, whence $H^p(\bm{x},M)=0$. The proposition then follows from \cref{depth of module Koszul cohomology char}.
\end{proof}
We now turn to another characterization of depth, which utilize the concept of regular sequences. Let $A$ be a ring and $M$ be an $A$-module. Recall that a sequence $(x_1,\dots,x_r)$ of elements of $A$ is said to be \textbf{regular for $\bm{M}$} or \textbf{$\bm{M}$-regular} if, for any $1\leq i\leq r$, the homothety with ratio $x_i$ is injective on the $A$-module $M_i=M/(x_1M+\cdots+x_{i-1}M)$. If $(x_1,\dots,x_r)$ is an $M$-regular sequence, then for any flat $A$-module $N$, the sequence $(x_1,\dots,x_r)$ is $M\otimes N$-regular, and for any flat ring homomorphism $\rho:A\to B$, the image $(\rho(x_1),\dots,\rho(x_r))$ is regular for the $B$-module $B\otimes_AM$. In particular, for any prime ideal $\p$ of $A$, the image of $(x_1,\dots,x_r)$ in $A_\p$ is $M_\p$-regular.\par
The imporance of regular sequence is that the depth of $M$ relative to an ideal $\mathfrak{I}$ can also be characterized by the length of maximal regular $M$-sequences. Therefore, we can also define $\depth_A(\mathfrak{I},M)$ by this maximal length. The connection of regular sequences with Ext modules then allows us to apply homological methods to commutative algebras.
\begin{proposition}\label{depth of module quotient by regular sequence}
Let $A$ be a ring, $\mathfrak{I}$ be an ideal of $A$, $M$ be an $A$-module, and $(x_1,\dots,x_r)$ be an $M$-regular sequence of elements of $\mathfrak{I}$. Then we have
\[\depth_A(\mathfrak{I},M)=r+\depth_A(\mathfrak{I},M/(x_1M+\cdots+x_rM)),\]
and in particular $\depth_A(\mathfrak{I},M)\leq r$.
\end{proposition}
\begin{proof}
The case $n=1$ follows from \cref{depth of module exact sequence prop}, since $M\stackrel{\cdot x_1}{\to}M$ is injective by hypothesis. The general case can then be proved by induction on $n$.
\end{proof}
\begin{theorem}\label{depth of module is maximal length of regular sequence}
Let $A$ be a Noetherian ring, $\mathfrak{I}$ be an ideal of $A$, and $M$ be a finitely generated $A$-module.
\begin{itemize}
\item[(a)] Suppose that $\depth_A(\mathfrak{I},M)$ is finite, then any $M$-regular sequence of elements of $\mathfrak{I}$ can be completed into an $M$-regular sequence of length $\depth_A(\mathfrak{I},M)$ of elements of $\mathfrak{I}$.
\item[(b)] The depth of $M$ relative to $\mathfrak{I}$ is the maximal length of $M$-regular sequences of elements of $\mathfrak{I}$.
\item[(c)] For $\depth_A(\mathfrak{I},M)$ to be finite, it is necessary and sufficient that the support of $M$ meets $V(\mathfrak{I})$, or that we have $\mathfrak{I}M\neq M$.
\end{itemize}
\end{theorem}
\begin{proof}
Let $(x_1,\dots,x_r)$ be an $M$-regular sequence of elements of $\mathfrak{I}$. We have $n\leq\depth_A(\mathfrak{I},M)$ by \cref{depth of module quotient by regular sequence}; suppose that this inequality is strict, and denote by $N$ the $A$-module $M/(x_1M+\cdots+x_rM)$. We then have $\depth_A(\mathfrak{I},N)>0$, so there exists an element $x$ of $\mathfrak{I}$ such that the homothety $x_N$ is injective (\cref{depth of module zero iff annihilated by element}), which means $(x_1,\dots,x_r,x)$ is $M$-regular. It then follows by recurrence that for any integer $s$ such that $r\leq s\leq\depth_A(\mathfrak{I},M)$ the sequence $(x_1,\dots,x_r)$ can be extended into an $M$-regular sequence of length $s$, which proves the assertion (a) and (b). Assertion (c) then follows from \cref{depth of module zero iff annihilated by element} and \cref{depth of module fg ideal bounded by generator number}.
\end{proof}
\begin{corollary}\label{depth of module regular sequence maximal iff}
For an $M$-regular $(x_1,\dots,x_r)$ of elements of $\mathfrak{I}$, the following properties are equivalent:
\begin{itemize}
\item[(\rmnum{1})] $r=\depth_A(\mathfrak{I},M)$;
\item[(\rmnum{2})] the sequence $(x_1,\dots,x_r)$ is a maximal $M$-regular sequence of elements of $\mathfrak{I}$. 
\item[(\rmnum{3})] the $A$-module $M/(x_1M+\cdots+x_rM)$ possesses a nonzero element annihilated by $\mathfrak{I}$;
\item[(\rmnum{4})] $\Ass(M/(x_1M+\cdots+x_rM))\cap V(\mathfrak{I})\neq\emp$.
\end{itemize}
\end{corollary}
\begin{proof}
The equivalence of (\rmnum{1}) and (\rmnum{2}) follows from \cref{depth of module is maximal length of regular sequence}, and that of (\rmnum{2}), (\rmnum{3}) and (\rmnum{4}) follows from \cref{depth of module zero iff annihilated by element} applied to $M/(x_1M+\cdots+x_rM)$.
\end{proof}
\begin{corollary}\label{depth of module local Noe depth leq dim}
Let $A$ be a Noetherian local ring and $M$ be a finitely generated $A$-module. Then $\depth_A(M)\leq\dim_A(M)<+\infty$.
\end{corollary}
\begin{proof}
In fact, any $M$-regular sequence of elements of $\m_A$ is complete secant for $M$ (\cref{Koszul complex completely secant annd regular relation}), hence secant for $M$ (\cref{Noe local ring secant iff minimal prime of supp}).
\end{proof}
\begin{proposition}\label{depth of module local Noe inequality}
Let $A$ be a Noetherian local ring, $M$ be a finitely generated nonzero $A$-module, and $\mathfrak{I}$ be a proper ideal of $A$. Then we have the inequalities
\[\depth_A(\mathfrak{I},M)\leq\codim(\supp(M)\cap V(\mathfrak{I}),\supp(M))\leq\dim(M)-\dim(M/\mathfrak{I}M)\leq[\mathfrak{I}/\m_A\mathfrak{I}:\kappa_A].\]
\end{proposition}
\begin{proof}
For any element $\p$ of $\supp(M)\cap V(\mathfrak{I})$, $\depth_A(\mathfrak{I},M)$ is smaller than $\dim_{A_\p}(M_\p)$ (\cref{depth of module localization char} and \cref{depth of module local Noe depth leq dim}), which is equal to $\codim(V(\p),\supp(M))$. If $\p$ runs through $\supp(M)\cap V(\mathfrak{I})$, then $V(\p)$ runs through irreducible closed subsets of $\supp(M)\cap V(\mathfrak{I})$, whence the first inequality. The second one follows from \cref{topo space codim prop}. Moreover, we can always find a generating set of $\mathfrak{I}$ with cardinality $[\mathfrak{I}/\m_A\mathfrak{I}:\kappa_A]$, so the third inequality follows from (\ref{dimension and secant sequence-3}).
\end{proof}
\begin{remark}
Consider the inequalities in \cref{depth of module local Noe inequality}.
\begin{itemize}
\item[(a)] For that we have $\depth_A(\mathfrak{I},M)=[\mathfrak{I}/\m_A\mathfrak{I}:\kappa_A]$, it is necessary and sufficient that $\mathfrak{I}$ can be generated by a $M$-regular sequence (\cref{depth of module local ring bounded by generator mod m_A}). 
\item[(b)] The equality $\dim(M)-\dim(M/\mathfrak{I}M)=[\mathfrak{I}/\m_A\mathfrak{I}:\kappa_A]$ signifies that $\mathfrak{I}$ can be generated by a sequence that is secant for $M$ (\cref{Noe local ring secant sequence prop}).
\item[(c)] If $M$ is Cohen-Macaulay, then we have $\depth_A(\mathfrak{I},M)=\dim(M)-\dim(M/\mathfrak{I}M)$. 
\end{itemize}
\end{remark}
\begin{lemma}\label{Noe ring prime chain Ext lemma}
Let $A$ be a Noetherian ring, $\p\sub\p_1\sub\cdots\sub\p_{r-1}\sub\q$ be a saturated chain of length $r$ of prime ideals of $A$, $M$ be a finitely generated $A$-module, and $n$ be an integer. If the $A$-module $\Ext_{A_\p}^n(\kappa(\p),M_\p)$ is nonzero, then so is $\Ext_{A_\q}^{n+r}(\kappa(\q),M_\q)$.
\end{lemma}
\begin{proof}
It evidently suffices to prove the case $r=1$; by replacing $A$, $M$, $\p$ and $\q$ with $A_\q$, $M_\q$, $\p A_\q$ and $\q A_\q$ respectively, we can then assume that $A$ is local and $\q=\m_A$. Let $x$ be an element of $\m_A-\p$. The $A_\p$-module $\Ext_A^n(A/\p,M)\otimes_AA_\p$ is isomorphic to $\Ext_{A_\p}^n(\kappa(\p),M_\p)$ (A, \Rmnum{10}, p.111, prop.10(b)), so is nonzero by hypothesis; a fortiori $\Ext_A^n(A/\p,M)$ is nonzero. The exact sequence
\[\begin{tikzcd}
0\ar[r]&A/\p\ar[r,"x_{A/\p}"]&A/\p\ar[r]&A/(\p+xA)\ar[r]&0
\end{tikzcd}\]
induces an exact sequence
\[\begin{tikzcd}
\Ext_A^n(A/\p,M)\ar[r,"u"]&\Ext_A^n(A/\p,M)\ar[r]&\Ext_A^{n+1}(A/(\p+xA),M)
\end{tikzcd}
\]
where $u$ is the homothety with ratio $x$. By Nakayama's lemma, this homomorphism is not surjective, so the $A$-module $\Ext_A^{n+1}(A/(\p+xA),M)$ is nonzero. Now if $\Ext_A^{n+1}(\kappa_A,M)$ is zero, we then deduce, by induction on the length of $N$, that $\Ext_A^{n+1}(N,M)=0$ for any $A$-module $N$ of finite length. But since $r=1$, the unique prime ideal of $A$ containing $\p+xA$ is $\m_A$, so the $A$-module $A/(\p+xA)$ is of finite length. This contradicts the fact that $\Ext_A^{n+1}(A/(\p+xA),M)\neq 0$, which proves our claim.
\end{proof}
\begin{proposition}\label{depth of module prime ideal inclusion inequality}
Let $A$ be a Noetherian ring, $M$ be a finitely generated $A$-module, $\p\sub\q$ be prime ideals of $\supp(M)$. Then we have
\[\depth_{A_\q}(M_\q)\leq\depth_{A_\p}(M_\p)+\dim(A_\q/\p A_\q).\]
More precisely, for any saturated chain of prime ideals $\p\sub\p_1\sub\cdots\sub\p_{r-1}\sub\q$, we have
\[\depth_{A_\q}(M_\q)\leq\depth_{A_\p}(M_\p)+r.\]
\end{proposition}
\begin{proof}
Put $p=\depth_{A_\p}(M_\p)$, it suffices to prove the second inequality. This is evident if $p=+\infty$; in the contrary case we have $\Ext_{A_\p}^p(\kappa(\p),M_\p)\neq 0$, so $\Ext_{A_\q}^{p+r}(\kappa(\q),M_\q)\neq 0$ by \cref{Noe ring prime chain Ext lemma}, which implies $\depth_{A_\q}(M_\q)\leq p+r$. As $\dim(A_\q+\p A_\q)$ is the supremum of the length of saturated chains of prime ideals with endpoints $\p$ and $\q$, the first assertion then follows.
\end{proof}
\begin{corollary}\label{depth of module dim-depth increasing}
We have the inequality 
\[\dim(M_\q)-\depth_{A_\q}(M_\q)\geq\dim(M_\p)-\depth_{A_\p}(M_\p)\geq 0.\]
\end{corollary}
\begin{proof}
This follows from \cref{depth of module prime ideal inclusion inequality} and $\dim(M_\q)\geq\dim(M_\p)+\dim(A_\q/\p A_\q)$ (\cref{dimension of module prop}(a) and \cref{topo space codim prop}(b)).
\end{proof}
\begin{corollary}\label{depth of module inequality dim(A/p)}
Let $A$ be a Noetherian local ring and $M$ be a finitely generated $A$-module. Then we have the inequality
\[\depth_A(M)\leq\inf_{\p\in\Ass(M)}\dim(A/\p).\]
\end{corollary}
\begin{proof}
Let $\p$ be an associated prime of $M$; we have $\depth_{A_\p}(M_\p)=0$ by \cref{depth of module zero iff annihilated by element}. \cref{depth of module prime ideal inclusion inequality} applied to the ideals $\p\sub\m_A$ then implies the inequality $\depth_A(M)\leq\dim(A/\p)$, whence the corollary.
\end{proof}
\begin{remark}
We have $\sup_{\p\in\Ass(M)}\dim(A/\p)=\dim(M)$ in view of \cref{dimension of module sup of dim(A/p)}, so \cref{depth of module inequality dim(A/p)} implies that $\depth_A(M)\leq\dim(M)$ for $M\neq 0$, which is \cref{depth of module local Noe depth leq dim}.
\end{remark}
\subsection{Extension of scalars}
\begin{proposition}\label{depth of module base change pullback prop}
Let $\rho:A\to B$ be a ring homomorphism, $\mathfrak{I}$ be an ideal of $A$, and $N$ be a $B$-module. Then we have $\depth_A(\mathfrak{I},N)=\depth_B(\mathfrak{I}B,N)$.
\end{proposition}
\begin{proof}
Let $\bm{x}=(x_i)_{i\in I}$ be a generating family of $\mathfrak{I}$; the family $\rho(\bm{x})=(\rho(x_i))_{i\in I}$ then generates $\mathfrak{I}B$. By construction the complex $K^\bullet(\rho(\bm{x}),N)$ is equal to $K^\bullet(\bm{x},N)$, so the proposition follows from \cref{depth of module Koszul cohomology char}.
\end{proof}
\begin{proposition}\label{depth of module flat base change pushout prop}
Let $A$ be a ring, $\mathfrak{I}$ be a finitely generated ideal of $A$ and $M$ be an $A$-module. Let $\rho:A\to B$ be a flat ring homomorphism.
\begin{itemize}
\item[(a)] We have $\depth_A(\mathfrak{I},M)\leq\depth_B(\mathfrak{I}B,B\otimes_AM)$.
\item[(b)] Suppose moreover that any maximal ideal of $\supp(M)\cap V(\mathfrak{I})$ belongs to the image of the canonical map $\Spec(B)\to\Spec(A)$. Then we have $\depth_A(\mathfrak{I},M)=\depth_B(\mathfrak{I}B,B\otimes_AM)$. This is the case for example if the $A$-module $B$ is faithfully flat.
\end{itemize}
\end{proposition}
\begin{proof}
Let $\bm{x}=(x_i)_{i\in I}$ be a family of elements of $A$. For each integer $p\geq 0$, let $u^p:B\otimes_AC^p(M)\to C^p(B\otimes_AM)$ be the $B$-linear homomorphism give by
\[b\otimes m\mapsto ((\alpha_1,\dots,\alpha_p)\mapsto b\otimes m(\alpha_1,\dots,\alpha_p)).\]
The family $(u^p)$ then defines an isomorphism of complexes $u:B\otimes_AK^\bullet(\bm{x},M)\to K^\bullet(\bm{x},B\otimes_AM)$. Now consider the canonical homomorphism
\[\gamma^p(B,K^\bullet(\bm{x},M)):B\otimes_AH^p(\bm{x},M)\to H^p(B\otimes_AK^\bullet(\bm{x},M))\]
By composing with $H^p(u)$, we obtain a homomorphism $v^p:B\otimes_AH^p(\bm{x},M)\to H^p(\bm{x},B\otimes_AM)$. It is clear from \cref{module complex splitting homology exact tensor preserve homology} that $v^p$ is an isomorphism if $B$ is flat over $A$, so assertion (a) follows from \cref{depth of module Koszul cohomology char}.\par
Suppose that $p$ is an integer that is strictly smaller than $\depth_B(\mathfrak{I}B,B\otimes_AM)$, and let $\m$ be a maximal ideal of $A$ belonging to $\supp(M)\cap V(\mathfrak{I})$. Let $\bm{x}$ be a finite generating family of $\mathfrak{I}$. Under the hypothesis of (b), there exists a prime ideal $\mathfrak{P}$ of $B$ lying over $\m$, and we have a canonical isomorphism
\[B_{\mathfrak{P}}\otimes_{A_\m}(A_\m\otimes_AH^p(\bm{x},M))\to B_{\mathfrak{P}}\otimes_B(B\otimes_AH^p(\bm{x},M)).\]
Now $B\otimes_AH^p(\bm{x},M)$ is isomorphic to $H^p(\rho(\bm{x}),B\otimes_AM)$, hence is zero; moreover $B_{\mathfrak{P}}$ is faithfully flat over $A_\m$ (\cref{module flat iff localization at maximal} and \cref{ring faithfully flat iff}), so we conclude that $A_\m\otimes_AH^p(\bm{x},M)=0$, and therefore $p<\depth_{A_\m}(\mathfrak{I}_{\m},M_\m)$. The first assertion of (b) then follows from \cref{depth of module localization char}, and the second one follows from \cref{ring faithfully flat iff}.
\end{proof}
\begin{corollary}\label{depth of module local ring completion equality}
Let $A$ be a Noetherian ring, $\mathfrak{I}$ be an ideal of $A$, $M$ be a finitely generated $A$-module, $\widehat{A}$ and $\widehat{M}$ be the $\mathfrak{I}$-adic completion of $A$ and $M$. Then we have $\depth_A(\mathfrak{I},M)=\depth_{\widehat{A}}(\mathfrak{I}\widehat{A},\widehat{M})$.
\end{corollary}
\begin{proof}
In fact, the $A$-module $\widehat{A}$ is flat and $\widehat{M}$ is isomorphic to $\widehat{A}\otimes_AM$ (\cref{filtration on completion is product with completion ring}); moreover, any maximal ideal of $A$ containing $\mathfrak{I}$ belongs to the image of the map $\Spec(\widehat{A})\to\Spec(A)$ (\cref{filtration I-adic completion maximal ideal}).
\end{proof}
\begin{lemma}\label{Noe local ring extension N/yN flat iff}
Let $\rho:A\to B$ be a homomorphism of Noetherian local rings, $N$ be a finitely generated $B$-module, and $y$ be an element of $\m_B$. Then the following conditions are equivalent:
\begin{itemize}
\item[(\rmnum{1})] the $A$-module $N/yN$ is flat and the homothety $y_N$ is injective;
\item[(\rmnum{2})] the $A$-module $N$ is flat and the homothety $y_{\kappa_A\otimes N}$ is injective.
\end{itemize}
If these are satisfied, then the homothery $y_{M\otimes_AN}$ is injective for any $A$-module $M$.
\end{lemma}
\begin{proof}
Suppose that the hypotheses of (\rmnum{1}) are satisfied, let us prove (\rmnum{2}) as well as the last assertion. Let $M$ be an $A$-module; since the $A$-module $N/yN$ is flat, we deduce from the exact sequence $0\to N\stackrel{y_N}{\to} N\to N/yN\to 0$ the following exact sequences
\begin{equation*}
\begin{gathered}
\begin{tikzcd}
0\ar[r]&M\otimes_AN\ar[r,"u"]&M\otimes_AN\ar[r]&M\otimes_A(N/yN)\ar[r]&0
\end{tikzcd}\\
\begin{tikzcd}
0\ar[r]&\Tor_1^A(M,N)\ar[r,"v"]&\Tor_1^A(M,N)\ar[r]&0
\end{tikzcd}
\end{gathered}
\end{equation*}
where $u=1_M\otimes y_N$ and $v=\Tor_1^A(1_M,y_N)$. It then follows that the homothety with ratio $y$ is injective on $M\otimes_AN$, and bijective on $\Tor_1^A(M,N)$. Suppose moreover that $M$ is finitely generated, then the $B$-module $\Tor_1^A(M,N)$ is also finitely generated, and hence is zero by Nakayama's lemma. This implies the flatness of the $A$-module $N$.\par
Conversely, assume that the hypotheses in (\rmnum{2}) are satisfied. Consider the exact sequences of $B$-modules
\begin{equation}\label{Noe local ring extension N/yN flat iff-1}
\begin{tikzcd}
0\ar[r]&\ker y_N\ar[r]&N\ar[r,"p"]&\im y_N\ar[r]&0
\end{tikzcd}
\end{equation}
\vspace*{-3mm}
\begin{equation}\label{Noe local ring extension N/yN flat iff-2}
\begin{tikzcd}
0\ar[r]&\im y_N\ar[r,"i"]&N\ar[r]&N/yN\ar[r]&0
\end{tikzcd}
\end{equation}
where $p$ and $i$ are canonical homomorphisms. We then deduce that the homomorphism $1\otimes p:\kappa_A\otimes_AN\to\kappa_A\otimes\im y_N$ is surjective, and (since $N$ is flat) that the kernel of the homomorphism $1\otimes i:\kappa_A\otimes_A\im y_N\to\kappa_A\otimes_AN$ is isomorphic to $\Tor_1^A(\kappa_A,N/yN)$. But the map $(1\otimes i)\circ(1\otimes p)$, equal to $y_{\kappa_A\otimes_AN}$, is injective by hypothesis; we then deduce that $1\otimes p$ is bijective and $1\otimes i$ is injective, and therefore $\Tor_1^A(\kappa_A,N/yN)=0$. It then follows that the $A$-module $N/yN$ is flat (\cref{filtration module and flatness} and \cref{Noe ring radical extension finite module is ideally Hausdorff}).\par
Since $N$ and $N/yN$ are flat over $A$, so is $\im y_N$ (by exace sequence (\ref{Noe local ring extension N/yN flat iff-2})). We then deduce from (\ref{Noe local ring extension N/yN flat iff-1}) that $\kappa_A\otimes_A\ker y_N$ is isomorphic to the kernel of $1\otimes p$, which is then zero. The homothety $y_N$ is then injective by Nakayama's lemma.
\end{proof}
\begin{proposition}\label{depth of module regular sequence and N/yN flat}
Let $\rho:A\to B$ be a homomorphism of Noetherian local rings, $N$ be a finitely generated $B$-module, and $\bm{y}=(y_1,\dots,y_s)$ be a sequence of elements of $\m_B$. Denote by $\Y$ the ideal of $B$ generated by this sequence, then the following conditions are equivalent:
\begin{itemize}
\item[(\rmnum{1})] the $A$-module $N/\Y N$ is flat and the sequence $\bm{y}$ is $N$-regular.
\item[(\rmnum{2})] the $A$-module $N$ is flat and the sequence $\bm{y}$ is $(\kappa_A\otimes_AN)$-regular.
\end{itemize}
If these are satisfied, then for any $A$-module $M$, the sequence $\bm{y}$ is $(M\otimes_AN)$-regular.
\end{proposition}
\begin{proof}
We prove the equivalence by recurrence on $s$. The case $s=0$ is evident, so suppose that $s\geq 1$. Denote by $\tilde{\bm{y}}$ the sequence $(y_1,\dots,y_{s-1})$ and $\widetilde{\Y}$ the ideal of $B$ it generates. By \cref{Noe local ring extension N/yN flat iff} applied to the $B$-module $N/\widetilde{\Y}N$ and to the element $y_s$ of $B$, we see that (\rmnum{1}) is equivalent to the following condition:
\begin{itemize}
\item[(\rmnum{1}')] the $A$-module $N/\widetilde{\Y}N$ is flat, and sequence $\tilde{\bm{y}}$ is $N$-regular, and the homothery with ratio $y_s$ is injective on $\kappa_A\otimes_A(N/\widetilde{\Y}N)=(\kappa_A\otimes N)/\widetilde{\Y}(\kappa_A\otimes_AN)$.
\end{itemize}
This condition is equivalent to (\rmnum{2}) by the recurrence hypotheses. Finally, the last assertion follows similarly by recurrence on $s$ and utilize \cref{Noe local ring extension N/yN flat iff}.
\end{proof}
\begin{proposition}\label{depth of module flat tensor and regular sequence}
Let $\rho:A\to B$ be a homomorphism of Noetherian local rings, $M$ be a finitely generated $A$-module, and $N$ be a finitely generated $B$-module. Suppose that the $A$-module $N$ is flat.
\begin{itemize}
\item[(a)] Let $(x_1,\dots,x_r)$ be an $M$-regular sequence of elements of $\m_A$ and $(y_1,\dots,y_s)$ be a $(\kappa_A\otimes_AN)$-regular sequence of elements of $\m_B$. Then $(y_1,\dots,y_s,\rho(x_1),\dots,\rho(x_r))$ is an $(M\otimes_AN)$-regular sequence of elements of $\m_B$.
\item[(b)] We have the equality
\[\depth_B(M\otimes_AN)=\depth_A(M)+\depth_B(\kappa_A\otimes_AN).\] 
\end{itemize}
\end{proposition}
\begin{proof}
Denote by $\x$ the ideal of $A$ generated $\bm{x}$ and $\Y$ the ideal of $B$ generated by $\bm{y}$. By \cref{depth of module regular sequence and N/yN flat}, the sequence $\bm{y}$ is $M\otimes_AN$-regular and $N/\Y N$ is flat for $A$, so that the sequence $\rho(\bm{x})=(\rho(x_1),\dots,\rho(x_r))$ is regular for $M\otimes_A(N/\Y N)=(M\otimes_AN)/\Y(M\otimes_AN)$. This proves the assertion in (a).\par
To prove (b), we can suppose that $M$ and $N$ are nonzero. By Nakayama's lemma, $\kappa_A\otimes_AN$ is also nonzero, so that $\depth_A(M)$ and $\depth_B(\kappa_A\otimes N)$ are finite (\cref{depth of module local Noe depth leq dim}). Choose $\bm{x}$ and $\bm{y}$ to be maximal regular sequences, we then have $r=\depth_A(M)$, $s=\depth_B(\kappa_A\otimes N)$, and there exists an injective $A$-linear map $u:\kappa_A\to M/\x M$ and an injective $B$-linear map $v:\kappa_B\to\kappa_A\otimes_A(N/\Y N)$ (\cref{depth of module regular sequence maximal iff}). Since $N/\Y N$ is flat over $A$, the $B$-linear map
\[(u\otimes 1_{N/\Y N})\circ v:\kappa_B\to (M/\x M)\otimes_A(N/\Y N)=(M\otimes_AN)/(\rho(\x)+\Y)(M\otimes_AN)\]
is injective. This implies the equality $\depth_B(M\otimes_AN)=r+s$, in view of \cref{depth of module regular sequence maximal iff}.
\end{proof}
\begin{remark}
We note that under the hypotheses of \cref{depth of module flat tensor and regular sequence}, we have a similar equality for dimensions: (c.f. \cref{Noe local ring extension dim of tensor})
\[\dim_B(M\otimes_AN)=\dim_A(M)+\dim_B(\kappa_A\otimes_AN).\]
\end{remark}
\begin{corollary}
Let $\rho:A\to B$ be a flat homomorphism of Noetherian local rings. Then we have
\begin{align*}
\depth(B)&=\depth(A)+\depth(\kappa_A\otimes_AB),\\
\dim(B)&=\dim(A)+\dim(\kappa_A\otimes_AB).
\end{align*}
\end{corollary}
\begin{proof}
In fact, the depth (resp. dimension) of the $B$-module $\kappa_A\otimes_AB$ is equal to the depth (resp. dimension) of the ring $\kappa_A\otimes_AB$ by \cref{depth of module base change pullback prop}.
\end{proof}
\subsection{Depth along a closed subset}
We now introduce the concept of depth along a closed subset. Let $A$ be a Noetherian ring, $F$ be a closed subset of $\Spec(A)$, and $M$ be an $A$-module. By \cref{depth of module fg ideal V(I) inequality}, the integer $\depth_A(\mathfrak{I},M)$ does not depend on the ideal $\mathfrak{I}$ such that $F=V(\mathfrak{I})$, we then write $\depth_F(M)$ for this integer, and call it the \textbf{depth of $M$ along $F$}. By \cref{depth of module Ext with annhilated module} and \cref{supp of module intersection iff homothety is nilpotent}, we see that the inequality $\depth_F(M)\geq r$ is equivalent to the property that for any finitely generated $A$-module $N$ with support contained in $F$, we have $\Ext_A^i(N,M)=0$ for $i<r$. Also, if $M$ is finitely generated, then we have $\depth_F(M)=0$ if and only if $\Ass(M)\cap F=\emp$, and $\depth_F(M)<+\infty$ if and only if $\supp(M)\cap F\neq\emp$.
\begin{proposition}\label{depth of module along closed localization char}
Let $A$ be a Noetherian ring, $F$ be a closed subset of $\Spec(A)$, and $M$ be a finitely generated $A$-module. Then we have
\[\depth_F(M)=\inf_{\p\in F}\depth_{A_\p}(M_{\p})=\inf_{\p\in\supp(M)\cap F}\depth_{A_\p}(M_\p).\]
\end{proposition}
\begin{proof}
This is clear if $\depth_F(M)=+\infty$. If $\depth_F(M)=0$, there exists a prime ideal $\p\in\Ass(M)\cap F$; we have $\p A_\p\in\Ass(M_\p)$ (\cref{associated prime of localization}), so $\depth_{A_\p}(M_\p)=0$, whence the assertion in this case.\par
Suppose that $0<\depth_F(M)<+\infty$; let $\mathfrak{I}$ be an ideal of $A$ such that $V(\mathfrak{I})=F$, and $x$ be an element of $\mathfrak{I}$ such that the homothety $x_M$ is injective (\cref{depth of module zero iff annihilated by element}). For any prime ideal $\p$, the homothety $x_{M_\p}$ is injective, and by \cref{depth of module quotient by regular sequence} we have
\[\depth_F(M/xM)=\depth_F(M)-1,\quad \depth_{A_\p}((M/xM)_\p)=\depth_{A_\p}(M_\p)-1.\]
The conclusion then follows by induction on $\depth_F(M)$.
\end{proof}
\begin{remark}\label{depth of module along prime ideal inequality}
If $\q$ is a point of $\supp(M)$, we then have $\depth_A(\q,M)=\inf_{\p\sups\q}\depth_{A_\p}(M_\p)$. In particular, we have $\depth_A(\q,M)\leq\depth_{A_\q}(M_\q)$, and the equality holds if $\q$ is maximal. In the general case, we may have $\depth_A(\q,M)<\depth_{A_\q}(M_\q)$, or $\depth_A(\q,M)<\inf\depth_{A_\m}(M_\m)$ where $\m$ runs through maximal ideals of $A$ cotaining $\q$. For example, let $\p$ be a non maximal prime ideal of $A$, cotaining $\q$ and distinct from $\q$; put $M=A/\p$, then we have $\depth_A(\q,M)=0$, $\depth_{A_\q}(M_\q)=+\infty$ and $\depth_{A_\m}(M_\m)>0$ for any maximal ideal $\m$ of $A$.
\end{remark}
\begin{proposition}\label{depth of module Noe fg supp=V(I) char}
Let $A$ be a Noetherian ring, $M$ and $N$ be finitely generated $A$-modules, and $F$ be the support of $N$. Then $\depth_F(M)$ is the infermum of the integers $i$ such that $\Ext_A^i(N,M)\neq 0$.
\end{proposition}
\begin{proof}
By \cref{depth of module Ext with annhilated module} and \cref{supp of module intersection iff homothety is nilpotent}, we have $\Ext_A^i(N,M)=0$ for $i<\depth_F(M)$. It remains to prove that if $\depth_F(M)=n<=\infty$, then $\Ext_A^n(N,M)\neq 0$. Let $\mathfrak{I}$ be the annihilator of $N$; we then have $F=V(\mathfrak{I})$, so $\depth_F(M)=\depth_A(\mathfrak{I},M)$. By \cref{depth of module is maximal length of regular sequence}, there exists an $M$-regular sequence $(x_1,\dots,x_n)$ of length $n$ formed by elements of $\mathfrak{I}$, and the depth of $\widebar{M}=M/(x_1M+\cdots+x_nM)$ relative to $\mathfrak{I}$ is zero. By \cref{Koszul complex regular sequence extension class and Hom prop}, it then suffices to prove that $\Hom_A(N,\widebar{M})\neq 0$. Now by \cref{depth of module along closed localization char}, there exists $\p\in\supp(M)\cap\supp(N)$ such that $\depth_{A_\p}(\widebar{M}_\p)=0$, which means $\Hom_{A_\p}(\kappa(\p),\widebar{M}_\p)\neq 0$. Since $N_\p$ is nonzero, the $\kappa(\p)$-vector space $N_\p\otimes_{A_\p}\kappa(\p)$ is nonzero (Nakayama's lemma), so there exists a surjective $A_\p$-linear map from $N_\p$ to $\kappa(\p)$. It follows that $\Hom_{A_\p}(N_\p,\widebar{M}_\p)\neq 0$, so $\Hom_A(N,\widebar{M})\neq 0$ (\cref{localization and Hom set if finite presented}), which proves the assertion.
\end{proof}
\begin{remark}\label{Noe ring module grade def}
Let $A$ be a Noetherian ring and $N$ be a fintely generated $A$-module. The \textbf{grade} of $N$, denoted by $\mathrm{grade}(N)$, is defined to be the infermum of the integers $i$ such that $\Ext_A^i(N,A)\neq 0$. By \cref{depth of module Noe fg supp=V(I) char}, this is also the depth of $A$ along the support of $N$, and is the maximal length of $A$-regular sequences of elements of the annihilator of $N$. As for any prime ideal of $A$, the annihilator of $N_\p$ is equal to $\Ann(N)_\p$, we deduce from \cref{depth of module localization char} the equality
\[\mathrm{grade}(N)=\inf_{\p\in\Spec(A)}\mathrm{grade}(N_\p)=\inf_{\m\in\Omega}\mathrm{grade}(N_\m)\]
where $\Omega$ denote the set of maximal ideals of $A$.
\end{remark}
We recall that the localization operation is faithfully exact in the sense that a homomorphism $u:M\to N$ of $A$-modules is injective (resp. surjective) if and only if $u_\p:M_\p\to N_\p$ is injective (resp. surjective) for any prime ideal $\p$ of $A$. Using the notion of depth, we can show that under certain conditions, this injectivity (resp. surjectivity) of $u_\p$ only need to be checked on an open subset of $\Spec(A)$. 
\begin{lemma}\label{depth of module injective and bijective lemma}
Let $A$ be a Noetherian ring, $F$ be a closed subset of $\Spec(A)$, $U$ be its complement, and $u:M\to N$ be a homomorphism of finitely generated $A$-module.
\begin{itemize}
\item[(a)] Suppose that $u_\p:M_\p\to N_\p$ is injective for $\p\in U$ and that $\depth_F(M)\geq 1$, then $u$ is injective.
\item[(b)] Suppose that $u_\p:M_\p\to N_\p$ is bijective for $\p\in U$ and that $\depth_F(M)\geq 2$ and $\depth_F(N)\geq 1$, then $u$ is bijective.
\end{itemize}
\end{lemma}
\begin{proof}
For the first assertion, we note that the hypotheses of (a) imply that $\supp(\ker u)\sub F$, so that $\Hom_A(\ker u,M)=0$ by \cref{depth of module Ext with annhilated module}. We then conclude that $\ker u=0$, since $\ker u$ is a submodule of $M$. As for the assertions in (b), we note that in this case $u$ is injective by (a), and we have $\supp(\coker u)\sub F$. Again by \cref{depth of module Ext with annhilated module}, we have $\Hom_A(\coker u,N)=0$ and $\Ext_A^1(\coker u,M)=0$. From the exact sequence
\[\begin{tikzcd}
\Hom_A(\coker u,N)\ar[r]&\Hom_A(\coker u,\coker u)\ar[r]&\Ext_A^1(\coker u,M)
\end{tikzcd}\]
we deduce that $\Hom_A(\coker u,\coker u)=0$, whence $\coker u=0$.
\end{proof}
\begin{remark}\label{depth of ring geq 1 if}
Let $A$ be a Noetherian ring, $F$ be a closed subset of $\Spec(A)$, and $U$ be its complement. For that we have $\depth_F(A)\geq 1$, it is necessary and sufficient that $\Ass(A)\sub U$ (\cref{depth of module zero iff annihilated by element}). If this condition is satisfied, then any irreducible component of $\Spec(A)$ meets $U$, so that $U$ is dense in $\Spec(A)$.
\end{remark}
\begin{theorem}[\textbf{Hartshorne}]\label{depth of ring geq 2 connected theorem}
Let $A$ be a Noetherian ring, $F$ be a closed subset of $A$, and $U$ be its complement. Suppose that $\depth_F(A)\geq 2$, then for any connected component $Y$ of $\Spec(A)$, the set $Y\cap U$ is connected and dense in $Y$.
\end{theorem}
\begin{proof}
Suppose first that $\Spec(A)$ is connected. By \cref{depth of ring geq 1 if}, $U$ is dense in $\Spec(A)$ and it then suffices to prove that it is connected. To this end, suppose that there are two nonempty open disjoint subsets $U_0$ and $U_1$ of $\Spec(A)$ whose union is $U$. As the set $\Ass(A)$ is contained in $U$ by \cref{depth of ring geq 1 if}, it is the union of the disjoint subsets $\Ass(A)\cap U_0$ and $\Ass(A)\cap U_1$. By \cref{associated prime submodule with given subset}, there exists ideals $\mathfrak{I}_0$ and $\mathfrak{I}_1$ of $A$ such that
\[\Ass(\mathfrak{I}_i)=\Ass(A)\cap U_i,\quad\Ass(A/\mathfrak{I}_i)=\Ass(A)\cap U_{1-i}\quad (i=0,1).\]
The complement of $U_i$ in $\Spec(A)$ contains $\Ass(A/\mathfrak{I}_i)$ and $\Ass(\mathfrak{I}_j)$; as it is closed, it also contains $\supp(A/\mathfrak{I}_i)$ and $\supp(\mathfrak{I}_{1-i})$. For $\p\in U_i$, we then conclude that $(A/\mathfrak{I}_i)_{\p}=0$ and $(\mathfrak{I}_{1-i})_\p=0$, which also implies that $\mathfrak{I}_0$ and $\mathfrak{I}_1$ are proper ideals of $A$. Now let $B$ be the $A$-module $A/\mathfrak{I}_0\times A/\mathfrak{I}_1$ and $u:A\to B$ be the canonical homomorphism. By the preceding remarks, the homomorphism $u_\p$ is bijective for $\p\in U$; on the other hand, we have $\Ass(B)\sub U_0\cup U_1=U$ (\cref{associated prime and exact sequence}), so $\depth_F(B)\geq 1$ in view of \cref{depth of ring geq 1 if}. \cref{depth of module injective and bijective lemma} then implies that $u$ is bijective, which contradicts the connectedness of $\Spec(A)$.\par
In the general case, let $\mathfrak{I}$ be an ideal of $A$ such that $F=V(\mathfrak{I})$ and let $Y$ be a connected component of $\Spec(A)$. By \cref{Spec of ring connected iff}, there exists an element $f$ of $A$ such that $Y$ is identified with the subset $\Spec(A_f)$ of $\Spec(A)$ (for example, if $Y=V(e)$ for some idempotent $e$, then we can choose $f=1-e$). Then $Y\cap F$ is identified with $V(\mathfrak{I}_f)$, and we have $\depth_{A_f}(\mathfrak{I}_f,A_f)\geq\depth_A(\mathfrak{I},A)\geq 2$ in view of \cref{depth of module flat base change pushout prop}. It then follows from our previous arguments that $Y\cap U=Y-(Y\cap F)$ is connected and dense in $Y$.
\end{proof}
\begin{corollary}\label{depth of ring geq 2 connected component correspond}
The map which associates each connected component of $U$ with its closure in $\Spec(A)$ is a bijection from the set of connected components of $U$ onto the set of connected components of $\Spec(A)$.
\end{corollary}
\begin{corollary}\label{Noe local ring depth geq 2 punctured space connected}
For any Noetherian local ring $B$ with $\depth(B)\geq 2$, the space $\Spec(B)-\{\m_B\}$ is connected.
\end{corollary}
\begin{proof}
This follows from the observation that any local ring has no idempotents, so $\Spec(B)$ is connected and we can apply \cref{depth of ring geq 2 connected theorem} to $F=V(\m_B)=\{\m_B\}$.
\end{proof}
\begin{corollary}\label{depth of ring geq 2 locally integral normal char}
Under the hypotheses of \cref{depth of ring geq 2 connected theorem}, suppose that $\Spec(A_\p)$ is irreducible (resp. $A_\p$ is integral) for any $\p\in U$. Then $\Spec(A_\p)$ is irreducible (resp. $A_\p$ is integral) for any $\p\in\Spec(A)$.
\end{corollary}
\begin{proof}
Let $(Y_i)_{i\in I}$ be the (finite) family of irreducible components of $\Spec(A)$. Let $\p\in U$; as $\Spec(A_\p)$ is irreducible, $\p$ contains a unique minimal prime ideal of $A$, so is contained in a unique $Y_i$. The intersection $Y_i\cap U$ is then nonempty disjoint and closed in $U$, dense in $Y_i$, and irreducible by \cref{topo space open irre closed intersection}, so they form the connected components of $U$. The closures $Y_i$ are then the connected components of $\Spec(A)$ by \cref{depth of ring geq 2 connected component correspond}. This proves that the connected components of $\Spec(A)$ are irreducible, so that $\Spec(A_\p)$ is irreducible for any $\p\in\Spec(A_\p)$.\par
Now suppose that $A_\q$ is integral for each $\q\in U$. Let $\p\in\Spec(A)$, since $\Spec(A_\p)$ is irreducible, the nilradical of $A_\p$ is the unique prime ideal of $A_\p$, and it therefore belongs to $\Ass(A_\p)$ (\cref{associated prime and supp}), and equal to $\q A_\p$, where $\q$ is an associated prime of $A$ (\cref{associated prime of localization}). We have $\q\in U$ by \cref{depth of ring geq 2 connected theorem} and $\q A_\q\in\Ass(A_\q)$ (\cref{associated prime of localization}); since $A_\q$ is integral, $\q$ is then zero, so $A_\p$ is also integral.
\end{proof}
\begin{corollary}\label{Noe ring depth geq 2 for height>d connecting component}
Let $A$ be a Noetherian ring with $\Spec(A)$ connected. Suppose that there exists an integer $d\geq 1$ such that we have $\depth(A_\p)\geq 2$ for any prime ideal of $A$ with $\height(\p)>d$.
\begin{itemize}
\item[(a)] For any closed subset $Z$ of $\Spec(A)$ with $\codim(Z)>d$, the space $\Spec(A)-Z$ is connected.
\item[(b)] Let $Y$ and $Y'$ be irreducible components of $\Spec(A)$. Then there exists a sequence $(X_1,\dots,X_n)$ of irreducible components of $\Spec(A)$ such that $X_1=Y$, $X_n=Y'$, and for each $i=1,\dots,n-1$, we have $\codim(X_i\cap X_{i+1})\leq d$. 
\end{itemize}
\end{corollary}
\begin{proof}
Let $Z\sub\Spec(A)$ be a closed subset with $\codim(Z)>d$. For any $\p\in Z$, we have $\dim(A_\p)>d$, so $\depth(A_\p)\geq 2$, which implies that $\depth_Z(A)\geq 2$ (\cref{depth of module along closed localization char}). Then $\Spec(A)-Z$ is connected by \cref{depth of ring geq 2 connected theorem}.\par
To prove (b), denote by $Z$ the union of subsets $X'\cap X''$ where $(X',X'')$ runs through subsets of couples of irreducible components of $\Spec(A)$ such that $\codim(X'\cap X'')>d$. In view of (a), the subset $\Spec(A)-Z$ is then connected. All irreducible components of $\Spec(A)$ meet $\Spec(A)-Z$, and their trace over $\Spec(A)-Z$ are the irreducible components of $\Spec(A)-Z$ (\cref{topo space open irre closed intersection}). Since $\Spec(A)-Z$ is connected, there exists a sequence $(X_1,\dots,X_n)$ of irreducible components of $\Spec(A)$ such that $X_1-Z=Y-Z$, $X_n-Z=Y'-Z$, and $(X_i-Z)\cap(X_{i+1}-Z)\neq\emp$ for $1\leq i\leq n-1$. The construction of $Z$ then implies that $X_1=Y$, $X_n=Y'$, and $\codim(X_i\cap X_{i+1})\leq d$.
\end{proof}
\begin{example}
Let $k$ be a field and $S=k[T_1,T_2,T_3,T_4]$ be the polynomial ring. Recall that any maximal chain of prime ideals of $S$ has length $4$ (\cref{algebra finite over field dimension prop}). Let $\m$ be the maximal ideal of $S$ generated by the $T_i$, and for $i\leq i<j\leq 4$, let $\p_{ij}=(T_i,T_j)$. The ideals $\p_{ij}$ are then prime with height $2$, and their sum is the maximal ideal $\m$.
\begin{itemize}
\item[(a)] Let $\a$ be the ideal of $S$ defined by $\a=(T_1T_2,T_3T_4)$, and $A=S/\a$. Then we have $\a=\p_{13}\cap\p_{14}\cap\p_{23}\cap\p_{24}$, and the ring $A=S/\a$ is reduced. The irreducible components of $\Spec(A)$ are the subsets $X_{ij}=V(\p_{ij}/\a)$ for $i=1,2$, $j=3,4$, which are all of dimension $2$ and contain the closed point $\m/\a$. In particular, $\Spec(A)$ is connected of dimension $2$. The intersection of two distinct components $X_{ij}$ and $X_{kl}$ is reduced to $\{\m/\a\}$ if $\{i,j\}\cap\{k,l\}=\emp$, and is of dimension $1$ otherwise. It then follows that the conclusion of \cref{Noe ring depth geq 2 for height>d connecting component} is valid for $d=1$ (in fact $A$ is Cohen-Macaulay).
\item[(b)] Let $\b$ be the ideal of $S$ defined by $\b=(T_1T_2,T_1T_3,T_2T_4,T_3T_4)$, and $B=S/\b$. Then we have $\b=\p_{14}\p_{23}=\p_{14}\cap\p_{23}$, and the ring $B$ is reduced. The space $\Spec(B)$ is identified with the closed subset $X_{14}\cup X_{23}$ of $\Spec(A)$, it have two irreducible components (of dimension $2$), whose intersection reduces to $\{\m/\b\}$. The depth of $B$ along this closed point is strictly positive because $B$ is reduced, and smaller than $1$ in view of \cref{depth of ring geq 2 connected theorem} (since $\Spec(B)-\{\m/\b\}$ is not connected), so it is equal to $1$ (the ring $B$ is not Cohen-Macaulay). 
\end{itemize}
\end{example}
\subsection{Serre's criterion for normality}
Let $A$ be a Noetherian ring. We denote by $(Y_i)_{i\in I}$ the finite family of connected components of $\Spec(A)$. By \cref{Spec of ring connected iff}, for each $i$ there exists a unique idempotent $e_i$ of $A$ such that $Y_i=V(e_i)$, and the canonical homomorphism $A\to\prod_iA/Ae_i$ is bijective. The quotient rings $A/Ae_i$ are called the canonical components of $A$. Put $f_i=1-e_i$, then we have $\sum_if_i=1$ and $(f_i)_{i\in I}$ is an orthogonal family of nonzero idempotents of $A$. It then follows that the image of $f_i$ in $A/Ae_j$ is equal to $1$ if $j=i$, and to $0$ otherwise. The canonical homomorphism $A\to\prod_jA/Ae_j$ then induces a canonical isomorphism $A_{f_i}\to A/Ae_i$.\par
By \cref{Spec of ring irreducible subset iff}, we see that the following conditions are equivalent:
\begin{itemize}
\item[(\rmnum{1})] the connected components of $\Spec(A)$ are irreducible;
\item[(\rmnum{2})] each prime (resp. maximal) ideal of $A$ belongs to a unique irreducible component of $\Spec(A)$;
\item[(\rmnum{3})] each prime (resp. maximal) ideal of $A$ contains a unique minimal prime ideal; 
\item[(\rmnum{4})] for any prime (resp. maximal) ideal $\p$ of $A$, the topological space $\Spec(A_\p)$ is irreducible;
\item[(\rmnum{5})] for any canonical component $C$ of $A$, the topological space $\Spec(C)$ is irreducible.
\end{itemize}
Now note that if $A$ is reduced, then each ring $A_\p$ is reduced, and the converse is also true by \cref{localization and nilradical}. Applying \cref{Spec of ring irreducible iff reduced ring integral}, we then deduce that the following conditions are equivalent:
\begin{itemize}
\item[(\rmnum{1})] $A$ is reduced and the connected components of $\Spec(A)$ are irreducible;
\item[(\rmnum{2})] for any prime (resp. maximal) ideal of $A$, the ring $A_\p$ is integral;
\item[(\rmnum{3})] the canonical components of $A$ are integral. 
\end{itemize}
The Noetherian ring $A$ is called \textbf{locally integral} if it satisfies the above equivalent condition. Suppose that $A$ is locally integral; let $u:A\to\prod_{j\in J}A_j$ be an isomorphism from $A$ to a (finite) product of integral rings. Then there exists a bijection $\sigma:J\to I$ such that the map from $\Spec(\prod_{j\in J}A_j)$ to $\Spec(A)$ associated with $u$ defines a homeomorphism from $\Spec(A_j)$ to the connected component $Y_{\sigma(j)}$ of $\Spec(A)$. We then deduce that $u$ is an isomorphism from the canonical component $A/Ae_{\sigma(j)}$ to $A_j$.
\begin{proposition}\label{Noe ring locally integrally closed iff}
Let $A$ be a Noetherian ring. The following conditions are equivalent:
\begin{itemize}
\item[(\rmnum{1})] $A$ is reduced and integrally closed in its total fraction ring;
\item[(\rmnum{2})] $A$ is isomorphic to the product of a finite family of integrally closed rings;  
\item[(\rmnum{3})] the canonical components of $A$ are integrally closed;
\item[(\rmnum{4})] for any prime (resp. maximal) ideal $\p$ of $A$, the ring $A_\p$ is integrally closed.  
\end{itemize}
\end{proposition}
\begin{proof}
The equivalence of (\rmnum{1}) and (\rmnum{2}) follows from \cref{Noe reduced integral closure in Q(A)}, and that of (\rmnum{2}) and (\rmnum{3}) follows from the preceding remarks. Let $\p$ be a prime ideal of $A$, then there exists a unique canonical component $A'$ of $A$ such that $\p$ belongs to the closed subset $\Spec(A')$ of $\Spec(A)$ and we have a canonical isomorphism $A_\p\to A'_{\p A'}$. The equivalence of (\rmnum{3}) and (\rmnum{4}) then follows from \cref{integral closure and localization} and \cref{integral closed is local property}.
\end{proof}
A ring $A$ is called \textbf{normal} if it is Noetherian and it satisfies the equivalent conditions of \cref{Noe ring locally integrally closed iff}. With this notion, a Noetherian ring is integrally closed if and only if it is integral and normal. It is clear that a local normal ring is integrally closed.\par
We now prresent Serre's criterion for the normality of a Noetherian ring. To beging with, we give the following characterization for a Noetherian ring to be reduced. Recall that for any prime ideal $\p$ of $A$, we have $\depth(A_\p)\leq\height(\p)$ by \cref{depth of module local Noe depth leq dim}.
\begin{proposition}\label{Noe ring reduced iff R0S1}
Let $A$ be a Noetherian ring. Then $A$ is reduced if and only if it satisfies the following conditions:
\begin{itemize}
\item[(R0)] for any minimal prime ideal, $A_\p$ is a regular local ring;
\item[(S1)] for any prime ideal $\p$ of $A$ with $\height(\p)\geq 1$, we have $\depth(A_\p)\geq 1$.
\end{itemize}
\end{proposition}
\begin{proof}
Denote by $\n$ the nilradical of $A$. If $A$ is reduced, then each local ring $A_\p$ is reduced, so if $\height(\p)=0$, then $A_\p$ is a field, and if $\height(\p)\geq 1$, then $\depth(A_\p)\geq 1$ (\cref{depth of module zero iff annihilated by element}). Conversely, assume that $A$ satisfies the above conditions. Then for any minimal prime ideal $\p$ of $A$, we have $\n_\p=0$ by condition (R0), which means $\p\notin\supp(\n)$, and a fortiori $\p\notin\Ass(\n)$. For any $\p\in\Spec(A)$ with $\height(\p)\geq 1$, by condition (S1) and \cref{depth of module zero iff annihilated by element} we have $\p A_\p\notin\Ass_{A_\p}(A_\p)$ and a fortiori $\p A_\p\notin\Ass_{A_\p}(\n_\p)$ (\cref{associated prime and exact sequence}), so $\p\notin\Ass_A(\n)$ (\cref{associated prime of localization}). We then conclude that $\Ass(\n)=\emp$, so $\n=0$ and $A$ is reduced.
\end{proof}
\begin{proposition}\label{Noe integral closed depth of reflexive geq 2}
Let $A$ be an integrally closed Noetherian ring, $\mathfrak{I}$ be an ideal of $A$ with $\height(\mathfrak{I})\geq 2$, and $M$ be a finitely generated reflexive $A$-module. Then we have $\depth_A(\mathfrak{I},M)\geq 2$.
\end{proposition}
\begin{proof}
Since $M$ is reflexive, we can choose a finite dimensional vector space $V$ over the fraction field of $A$ such that $M$ is a lattice in $V$ (\cref{module reflexive is lattice of vector space}). The associated primes of $V/M$, being of height $1$ (\cref{lattice in vector space reflexive iff}), then do not belong to $V(\mathfrak{I})$. By \cref{depth of module zero iff annihilated by element}, we then conclude that $\depth_A(\mathfrak{I},V/M)\geq 1$. On the other hand, the $A$-module $V$ is divisible and torsion free, hence injective, and this implies $\depth_A(\mathfrak{I},V)=+\infty$ by \cref{depth of module infty iff M=IM}. The inequality $\depth_A(\mathfrak{I},M)\geq 2$ then follows from \cref{depth of module exact sequence prop}.
\end{proof}
\begin{corollary}\label{Noe integral closed depth geq 2 if dim geq 2}
An integrally closed Noetherian local ring of dimension $\geq 2$ has depth $\geq 2$.
\end{corollary}
\begin{proof}
This follows from the fact that an integrally closed Noetherian local ring $A$ is Krull (\cref{Krull domain iff completely integrally closed}), hence reflexive (\cref{Krull domain iff prime of height 1}), so we can apply \cref{Noe integral closed depth of reflexive geq 2} to $\mathfrak{I}=\m_A$.
\end{proof}
\begin{theorem}[\textbf{Serre's Normality Criterion}]\label{Noe ring normal iff R1S2}
Let $A$ be a Noetherian ring. Then $A$ is normal if and only if it satisfies the following properties:
\begin{itemize}
\item[(R1)] for any prime ideal $\p$ of $A$ with $\height(\p)\leq 1$, $A_\p$ is a regular local ring.
\item[(S2)] for any prime ideal $\p$ of $A$ with $\height(\p)\geq 2$, we have $\depth(A_\p)\geq 2$.
\end{itemize}
\end{theorem}
\begin{proof}
By definition, if $A$ is normal, then $A_\p$ is integrally closed for any prime ideal $\p$ of $A$, so condition (S2) follows from \cref{Noe integral closed depth geq 2 if dim geq 2}. Also, it is clear that an integrally closed Noetherian local ring of dimension $\leq 1$ must be regular. Conversely, assume that $A$ satisfies condition (R1) and (S2). We show that $A_\p$ is integrally closed by recurrence on $\height(\p)$. For $\height(\p)\leq 1$, this follows directly from condition (R1). Suppose then that $\height(\p)\geq 2$ and that $A_\q$ is integrally closed for any prime ideal $\q$ with $\height(\q)<\height(\p)$. By condition (S2), we have $\depth(A_\p)\geq 2$. By the recurrence hypotheses and \cref{depth of ring geq 2 locally integral normal char} applied to the ring $A_\p$ and the closed subset $\{\p A_\p\}$ of $\Spec(A_\p)$, we then conclude that $A_\p$ is integral. Let $K$ be the fraction field of $A_\p$ and $B$ be a subring of $K$ which is finite over $A_\p$. It then suffices to prove that $B=A_\p$, so let $i:A_\p\to B$ be the canonical injection. As $B$ is contained in $K$, it is a torsion-free $A_\p$-module, so $\depth_{A_\p}(B)\geq 1$. On the other hand, for any prime ideal $\q$ of $A_\p$ distinct from $\p A_p$, the homomorphism $i_\q:(A_\p)_\q\to B_\q$ is bijective since $A_\q$ is integrally closed by hypotheses. By \cref{depth of module injective and bijective lemma} applied to the closed subset $F=\{\p A_\p\}$ of $\Spec(A_\p)$, the homomorphism $i$ is then bijective, which proves our assertion.
\end{proof}
\begin{remark}
A convenient form of \cref{Noe ring normal iff R1S2} is the following: let $A$ be a Noetherian ring such that for any prime ideal $\p$ of $A$, the ring $A_\p$ is either integrally closed, or $\depth(A_\p)\geq 2$, then $A$ is normal. In fact, in this case if $\height(\p)\leq 1$ then $\depth(A_\p)\leq\height(\p)\leq 1$, so $A_\p$ is integrally closed. If $\height(\p)\geq 2$, then {Noe integral closed depth geq 2 if dim geq 2} implies in both cases that $\depth(A_\p)\geq 2$, so we can apply \cref{Noe ring normal iff R1S2}. Similarly, we see if for any prime ideal $\p$ of $A$, the ring $A_\p$ is either regular of $\depth(A_\p)\geq 1$, then $A$ is reduced. 
\end{remark}
\begin{corollary}\label{Noe ring depth along closed normal and reduced if}
Let $A$ be a Noetherian ring, $F$ be a closed subset of $\Spec(A)$, and $U$ be its complement. Suppose that $\depth_F(A)\geq 2$ (resp. $\depth_F(A)\geq 1$) and that for any $\p\in U$, the ring $A_\p$ is integrally closed (resp. reduced). Then $A$ is normal (resp. reduced).
\end{corollary}
\begin{proof}
For any $\p\in F$, we have $\depth(A_\p)\geq\depth_F(A)$ by \cref{depth of module along closed localization char}, so it suffices to apply the preceding remark.
\end{proof}
\begin{corollary}\label{Noe ring flat base change normality prop}
Let $\rho:A\to B$ be a flat homomorphism of rings.
\begin{itemize}
\item[(a)] If $B$ is normal and faithfully flat over $A$, then $A$ is normal.
\item[(b)] Suppose that $A$ is normal and the ring $\kappa(\p)\otimes_AB$ is normal (resp. reduced) for any minimal prime ideal $\p$ (resp. prime ideal $\p$ of height one) of $A$. Then the ring $B$ is normal.
\end{itemize}
\end{corollary}
\begin{corollary}\label{Noe ring normal flat base change normal if}
Let $\rho:A\to B$ be a homomorphism of Noetherian rings. Suppose that $B$ is a flat $A$-module, that $A$ is normal, and that $\kappa(\p)\otimes_AB$ is normal for any $\p\in\Spec(A)$. Then $B$ is normal.
\end{corollary}
\section{Cohen-Macaulay rings and modules}
\subsection{Cohen-Macaulay modules}
Let $A$ be a Noetherian ring, $M$ be a finitely generated $A$-module, and $\p$ be a prime ideal of $A$. If $\p\notin\supp(M)$, then we have $M_\p=0$, so $\depth_{A_\p}(M_\p)=+\infty$. If $\p\in\supp(M)$, then by \cref{depth of module inequality dim(A/p)},
\[0\leq\depth_{A_\p}(M_\p)\leq\dim_{A_\p}(M_\p)<+\infty.\]
The module $M$ is called \textbf{Cohen-Macaulay} if for any maximal ideal $\m\in\supp(M)$, we have $\depth_{A_\m}(M_\m)=\dim_{A_\m}(M_\m)$. If $A$ is local, then for a finitely generated nonzero $A$-module to be Cohen-Macaulay, it is necessary and sufficient that its depth is equal to its dimension.
\begin{example}\label{CM module if length finite}
Any $A$-module of finite length is Cohen-Macaulay. In fact, if $M$ is a module of finite length, then $M_\m$ is of finite length for any maximal ideal $\m$ of $A$ (\cref{associated prime of finite length localization iff}), and we have $\dim_{A_\m}(M_\m)=0$ by \cref{associated prime maximal iff finite length}, whence $\depth_{A_\m}(M_\m)=0$.
\end{example}
\begin{example}\label{CM module direct factor is CM}
Let $N$ be a direct factor of a finitely generated Cohen-Macaulay $A$-module $M$. Then $N$ is Cohen-Macaulay; in factm for any maximal ideal $\m$ of $A$, the $A_\m$-module $N_\m$ is a direct factor of $M_\m$, and we therefore have
\[\depth_{A_\m}(N_\m)\geq\depth_{A_\m}(M_\m)\geq\dim_{A_\m}(M_\m)\geq\dim_{A_\m}(N_\m),\]
where we use \cref{depth of module product is inf} and \cref{dimension of module prop}.
\end{example}
\begin{example}\label{CM module quotient regular sequence is CM}
Let $M$ be a finitely generated Cohen-Macaulay $A$-module and $(x_1,\dots,x_n)$ be an $M$-regular sequence for elements of $A$. Then the $A$-module $\widebar{M}=M/(x_1M+\cdots+x_nM)$ is Cohen-Macaulay. To see this, let $\m$ be a maximal ideal of $A$ belonging to the support of $\widebar{M}$; we have $x_i\in\m$ for each $i$ since $x_i$ annihilates $\widebar{M}$, and the canonical image of $x_i$ in $A_\m$ form an $M_\m$-regular sequence of elements of $\m A_\m$. We then have the equalities (\cref{depth of module quotient by regular sequence} and \cref{Noe local ring complete secant is secant})
\[\depth_{A_\m}(\widebar{M}_{\m})=\depth_{A_\m}(M_\m)-n,\quad \dim_{A_\m}(\widebar{M}_\m)=\dim_{A_\m}(M_\m)-n.\]
whence our assertion.
\end{example}
\begin{example}\label{CM module over quotient ring iff}
Let $M$ be a finitely generated $A$-module and $\a$ be an ideal of $A$ such that $\a M=0$. For that the $A$-module $M$ to be Cohen-Macaulay, it is necessary and sufficient that it is Cohen-Macaulay as an $(A/\a)$-module. In fact, put $B=A/\a$; let $\mathfrak{M}$ be the maximal ideal of $B$ and $\m$ be its inverse image in $A$. We then have $\dim_{A_\m}(M_\m)=\dim_{B_\mathfrak{M}}(M_\mathfrak{M})$ and $\depth_{A_\m}(M_\m)=\depth_{B_\mathfrak{M}}(M_\mathfrak{M})$ (\cref{depth of module base change pullback prop}).
\end{example}
\begin{proof}
This follows directly from \cref{depth of module dim-depth increasing} and {depth of module prime ideal inclusion inequality}.
\end{proof}
\begin{proposition}\label{CM module iff depth=dim for any prime}
Let $A$ be a Noetherian ring and $M$ be a finitely generated $A$-module. Then the following conditions are equivalent:
\begin{itemize}
\item[(\rmnum{1})] the $A$-module $M$ is Cohen-Macaulay;
\item[(\rmnum{2})] $\depth_{A_\p}(M_\p)=\dim_{A_\p}(M_\p)$ for any $\p\in\supp(M)$;
\item[(\rmnum{3})] $\depth_F(M)=\codim(\supp(M)\cap F,\supp(M))$ for any closed subset $F$ of $\Spec(A)$.
\item[(\rmnum{4})] $\depth_A(\p,M)=\dim_{A_\p}(M_\p)$ for any $\p\in\supp(M)$.
\end{itemize}
\end{proposition}
\begin{proof}
We first note that (\rmnum{1})$\Rightarrow$(\rmnum{2}) in view of \cref{depth of module dim-depth increasing} and {depth of module prime ideal inclusion inequality}. For (\rmnum{2})$\Rightarrow$(\rmnum{3}), note that by \cref{depth of module along closed localization char}, $\depth_F(M)$ is the infermum of the integers $\depth_{A_\p}(M_\p)$ for $\p\in\supp(M)\cap F$. If $M$ is Cohen-Macaulay, then for each such ideal $\p$ we have the equality
\[\depth_{A_\p}(M_\p)=\dim_{A_\p}(M_\p)=\codim(V(\p),\supp(M))\]
(\cref{dimension of module prop}), so (\rmnum{3}) follows from the definition of $\codim(\supp(M)\cap F,\supp(M))$. On the other hand, it is clear that (\rmnum{3})$\Rightarrow$(\rmnum{4}) by taking $F=V(\p)$ (\cref{dimension of module prop}), and (\rmnum{4})$\Rightarrow$(\rmnum{1}) since we have the inequality $\depth_A(\p,M)\leq\depth_{A_\p}(M_\p)\leq\dim(M_\p)$ for any $\p\in\supp(M)$ (\cref{depth of module along prime ideal inequality}).
\end{proof}
\begin{corollary}\label{CM module depth difference is dim difference}
Let $A$ be a Noetherian ring and $M$ be a finitely generated $A$-module. Then for any prime ideals $\p\sub\q$ of $\supp(M)$, we have
\[\depth_{A_\q}(M_\q)-\depth_{A_\p}(M_\p)=\dim(A_\q/\p A_\q)=\dim_{A_\q}(M_\q)-\dim_{A_\p}(M_\p).\]
\end{corollary}
\begin{proof}
This follows from \cref{depth of module prime ideal inclusion inequality} and $\dim(M_\q)\geq\dim(M_\p)+\dim(A_\q/\p A_\q)$ (\cref{dimension of module prop}(a) and \cref{topo space codim prop}(b))
\end{proof}
\begin{corollary}\label{CM module localization is CM}
Let $S$ be a multiplicative subset of $A$ and $M$ be a finitely generated Cohen-Macaulay $A$-module. Then $S^{-1}M$ is a Cohen-Macaulay $S^{-1}A$-module.
\end{corollary}
\begin{proof}
In fact, let $\q\in\Spec(S^{-1}A)$, $i_A^S:A\to S^{-1}A$ be the canonical homomorpism, and $\p=(i_A^S)^{-1}(\q)$. The ring $(S^{-1}A)_\q$ is then identified with $A_\p$, and the $A_\p$-module $(S^{-1}M)_\q$ is identified with the $A_\p$-module $M_\p$, whence the corolary.
\end{proof}
\begin{proposition}\label{CM module supp is catenary}
Let $A$ be a Noetherian ring and $M$ be a finitely generated Cohen-Macaulay $A$-module.
\begin{itemize}
\item[(a)] The $A$-module $M$ has no embedded associated prime ideals.
\item[(b)] Let $X$ be a closed irreducible subset of $\supp(M)$ and $Y$ be a closed subset of $X$, then we have
\[\codim(Y,X)+\codim(X,\supp(M))=\codim(Y,\supp(M)).\]
\item[(c)] The topological space $\supp(M)$ is catenary.
\item[(d)] Let $X_1$ and $X_2$ be irreducible components of $\supp(M)$ and $Y$ be a closed subset of $X_1\cap X_2$. Then we have $\codim(Y,X_1)=\codim(Y,X_2)$.
\end{itemize}
\end{proposition}
\begin{proof}
Let $\p\in\Ass(M)$; we have $\depth_A(\p,M)=0$ (\cref{depth of module zero iff annihilated by element}), so $\dim_{A_\p}(M_\p)=0$ by \cref{CM module iff depth=dim for any prime}. By \cref{dimension of module prop}, this implies that $\p$ is a minimal element of $\supp(M)$, so the assertion in (a) follows.\par
To prove the assertion in (b), it suffices to let $X$ be a closed irreducible subset of $\supp(M)$ and $Y$ be a closed irreducible subset of $X$. Let $\p$ and $\q$ be prime ideals of $\supp(M)$ such that $Y=V(\q)$ and $X=V(\p)$, then it follows from \cref{CM module depth difference is dim difference} that
\begin{align*}
\codim(Y,X)&=\dim(A_\q/\p A_\q)=\dim(M_\q)-\dim(M_\p)\\
&=\codim(Y,\supp(M))-\codim(X,\supp(M)).
\end{align*}
Now if $X,Y,Z$ are closed irreducible subsets of $\supp(M)$ such that $Z\sub Y\sub X$, then the codimensions of them in $\supp(M)$ are finite, and we deduce from (b) the equality
\[\codim(Z,Y)+\codim(Y,X)=\codim(Z,X)\]
so we conclude from \cref{topo space catenary iff} that $\supp(M)$ is catenary. Finally, if $X_1$ and $X_2$ are irreducible components of $\supp(M)$ and $Y$ be a closed subset of $X_1\cap X_2$, then
\begin{equation*}
\codim(Y,X_1)=\codim(Y,\supp(M))=\codim(Y,X_2).\qedhere
\end{equation*}
\end{proof}
In particular, if there exists a finitely generated $A$-module $M$ such that $\supp(M)=\Spec(A)$, then the ring $A$ is catenary and any fraction ring of quotient ring of $A$ is catenary.
\begin{remark}
Under the hypotheses of \cref{CM module supp is catenary}, it may happen that two components irreducible $X_1$ and $X_2$ of $\supp(M)$ have an intersection $Y$ reducing to a point and that $\dim(X_1)\neq\dim(X_2)$ and $\dim(X_2)\neq\codim(Y,X_2)$. However, this cannot happen if $A$ is local, as shown in the next corollary.
\end{remark}
\begin{corollary}\label{CM module over local Noe supp equi-dimension}
Let $A$ be a Noetherian local ring and $M$ be a finitely generated nonzero Cohen-Macaulay ring.
\begin{itemize}
\item[(a)] Any maximal chain of irreducible closed subsets of $\supp(M)$ has length equal to $\dim(M)$.
\item[(b)] For any closed subset $X$ of $\supp(M)$, we have
\[\codim(X,\supp(M))=\dim(\supp(M))-\dim(X).\]
\item[(c)] All irreducible components of $\supp(M)$ have the same dimension. 
\item[(d)] For any ideal $\mathfrak{I}$ of $A$, we have
\[\depth_A(\mathfrak{I},M)=\dim(M)-\dim(M/\mathfrak{I}M).\]
\end{itemize}
\end{corollary}
\begin{proof}
A maximal chain of irreducible closed subsets of $\supp(M)$ has smallest element $\{\m_A\}$ and largest element a irreducible component $X$ of $\supp(M)$. Its length is therefore equal to the codimension of $\{\m_A\}$ in $X$ (\cref{CM module supp is catenary}(c)); by \cref{CM module supp is catenary}(b), this is equal to $\codim(\{\m_A\},\supp(M))$, which is $\dim(M)$. Now (b) is a concequence of (a) if the subset $X$ is irreducible, and the general case follows from (\ref{topo space codim is inf of component}), and (c) is a concequence of (b). Finally, for any ideal $\mathfrak{I}$, we have $\depth_A(\mathfrak{I},M)=\codim(\supp(M)\cap V(\mathfrak{I}),\supp(M))$ by \cref{CM module iff depth=dim for any prime}, so to prove (d) it suffices to apply (b) to $X=\supp(M)\cap V(\mathfrak{I})=\supp(M/\mathfrak{I}M)$.
\end{proof}
We now consider Cohen-Macaulay modules over Noetherian local rings. In particular, we will consider the behaviour of these modules with regular sequence of $\m_A$.
\begin{proposition}\label{CM module over Noe local iff depth=dim}
Let $A$ be a Noetherian local ring, $M$ be a finitely generated nonzero $A$-module of dimension $d$. The following conditions are equivalent:
\begin{itemize}
\item[(\rmnum{1})] the $A$-module $M$ is Cohen-Macaulay;
\item[(\rmnum{2})] $\depth(M)=d$;
\item[(\rmnum{3})] $\Ext_A^i(\kappa_A,M)=0$ for any integer $i<d$;
\item[(\rmnum{4})] $\Ext_A^i(N,M)=0$ for any $A$-module $N$ of finite length and any integer $i<d$.
\item[(\rmnum{5})] $\Ext_A^i(N,M)=0$ for any finitely generated $A$-module $N$ and any integer $i<d-\dim(M\otimes_AN)$.
\item[(\rmnum{6})] there exists an $M$-regular sequence of elements of $\m_A$ of length $d$.
\end{itemize}
\end{proposition}
\begin{proof}
The equivalences of (\rmnum{1}), (\rmnum{2}), (\rmnum{3}), (\rmnum{4}) and (\rmnum{6}) are clear from our definition. As for (\rmnum{5}), note that it is clear that (\rmnum{5})$\Rightarrow$(\rmnum{4}), since if $N$ is of finite length, then $\dim(M\otimes_AN)=0$ for any finitely generated $A$-module $M$ (\cref{associated prime maximal iff finite length} and \cref{supp of module finite tensor}). Conversely, suppose that $M$ is Cohen-Macaulay and let $N$ be a finitely generated $A$-module. Put $F=\supp(N)$, then by \cref{supp of module finite tensor} we have $\supp(M)\cap F=\supp(M\otimes_AN)$, so that
\[\depth_F(M)=\codim(\supp(M)\cap F,\supp(M))=\dim(M)-\dim(M\otimes_AN)\]
(\cref{CM module iff depth=dim for any prime} and \cref{CM module over local Noe supp equi-dimension}). The implication (\rmnum{1})$\Rightarrow$(\rmnum{5}) then follows from \cref{depth of module Noe fg supp=V(I) char}.
\end{proof}
A finitely generated module $M$ over a Noetherian local ring is called \textbf{pure} if for any associated prime ideal $\p$ of $M$, we have $\dim(A/\p)=\dim(M)$. This signifies that $M$ has no embedded associated primes and that the irreducible components of $M$ have the same dimension. As we have seen (\cref{CM module over local Noe supp equi-dimension}), every Cohen-Macaulay module over a Noetherian local ring is pure.
\begin{lemma}\label{Noe local pure module secant element iff homothety injective}
Let $A$ be a Noetherian local ring, $M$ be a finitely generated pure $A$-module, and $x$ be an element of $\m_A$. Then the following conditions are equivalent:
\begin{itemize}
\item[(\rmnum{1})] $\dim(M/xM)=\dim(M)-1$; 
\item[(\rmnum{2})] the homothety $x_M$ is injective.
\end{itemize}
\end{lemma}
\begin{proof}
We can suppose that $M$ is nonzero. The assertion in (\rmnum{1}) is equivalent to that fact that $x$ does not belong to any minimal element $\p$ of $\supp(M)$ such that $\dim(A/\p)=\dim(M)$ (\cref{Noe local ring secant iff minimal prime of supp}), while the assertion in (\rmnum{2}) is equivalent to the fact that $x$ does not belong to any associated prime ideal of $M$ (\cref{associated prime and homothety injective}). Since $M$ is pure, we see these two conditions are equivalent.
\end{proof}
Let $A$ be a Noetherian local ring and $M$ be a finitely generated nonzero $A$-module. Recall that a sequence $(x_1,\dots,x_r)$ of elements of $\m_A$ is said to be \textit{secant} for $M$ if we have
\[\dim(M/(x_1M+\cdots+x_rM))=\dim(M)-r.\]
\begin{proposition}\label{CM over local Noe iff quotient by regular is CM}
Let $A$ be a Noetherian local ring, $M$ be a finitely generated nonzero $A$-module, and $(x_1,\dots,x_r)$ be a sequence of elements of $\m_A$ that is secant for $M$. Then the following conditions are equivalent:
\begin{itemize}
\item[(\rmnum{1})] the $A$-module $M$ is Cohen-Macaulay;
\item[(\rmnum{2})] the sequence $(x_1,\dots,x_r)$ is $M$-regular and the $A$-module $M/(x_1M+\cdots+x_rM)$ is Cohen-Macaulay.
\end{itemize}
\end{proposition}
\begin{proof}
Suppose that the sequence $(x_1,\dots,x_r)$ is $M$-regular. Then by \cref{depth of module quotient by regular sequence} we have
\begin{equation*}
\dim(M)=r+\dim(M/(x_1M+\cdots+x_rM)),\quad \depth(M)=r+\depth(M/(x_1M+\cdots+x_rM))
\end{equation*}
whence the implication (\rmnum{2})$\Rightarrow$(\rmnum{1}). Suppose now that the $A$-module $M$ is Cohen-Macaulay, we prove (\rmnum{2}) by recurrence on $r$. The assertion is evident if $r=0$, so assume that $r\geq 1$; the $A$-module $N=M/(x_1M+\cdots+x_{r-1}M)$ is then Cohen-Macaulay by hypotheses and we have $\dim(N/x_rN)=\dim(N)-1$ since the sequence $(x_1,\dots,x_r)$ is secant. The hypothety $(x_r)_N$ is then injective by \cref{Noe local pure module secant element iff homothety injective}, and $N/x_rN$ is Cohen-Macaulay by \cref{CM module quotient regular sequence is CM}, whence (\rmnum{2}).
\end{proof}
\begin{theorem}\label{CM over local Noe iff secant is regular}
Let $A$ be a Noetherian local ring, $M$ be a finitely generated nonzero $A$-module of dimension $d$, $\bm{x}=(x_1,\dots,x_r)$ be a sequence of elements of $\m_A$ that is secant for $M$, and $\mathfrak{I}$ be the ideal generated by $\bm{x}$. Then the following conditions are equivalent:
\begin{itemize}
\item[(\rmnum{1})] the $A$-module $M$ is Cohen-Macaulay;
\item[(\rmnum{2})] the sequence $\bm{x}$ is regular for $M$;
\item[(\rmnum{3})] the sequence $\bm{x}$ is completely secant for $M$;
\item[(\rmnum{4})] the multiplicity $e_{\mathfrak{I}}(M)$ of $M$ relative to $\mathfrak{I}$ is equal to the length of the $A$-module $M/\mathfrak{I}M$;
\item[(\rmnum{5})] for each integer $1\leq i\leq d$, the $A$-module $M/(x_1M+\cdots+x_{i-1}M)$ is pure.
\end{itemize}
\end{theorem}
\begin{proof}
The equivalence of (\rmnum{2}) and (\rmnum{3}) follows from \cref{Koszul complex completely secant annd regular relation}. The $A$-moduele $M.\mathfrak{I}M$ being of finite length (\cref{Noe local ring secant sequence prop}), the equivalence of (\rmnum{3}) and (\rmnum{4}) follows from \cref{filtration good index and length of generating set}.\par
It remains to see that equivalence of (\rmnum{1}), (\rmnum{2}) and (\rmnum{5}). If $M$ is Cohen-Macaulay, then each $M/(x_1M+\cdots+x_{i-1}M)$ is Cohen-Macaulay by \cref{CM module quotient regular sequence is CM}, hence pure. Conversely, by {Noe local pure module secant element iff homothety injective}, we see that (\rmnum{5})$\Rightarrow$(\rmnum{2}); the fact that (\rmnum{2})$\Rightarrow$(\rmnum{1}) follows from \cref{CM over local Noe iff quotient by regular is CM}, since $M/\mathfrak{I}M$ is of finite length, hence Cohen-Macaulay.
\end{proof}
\subsection{Strongly secant subsets}
Let $A$ be a Noetherian ring, $M$ be a finitely generated $A$-module, and $S$ be a subset of $A$. We denote by $SM$ the submodule $\sum_{s\in S}sM$ of $M$, and $\mathfrak{S}$ the ideal generated by $S$.
\begin{lemma}\label{CM module strongly secant lemma}
Let $\widebar{\mathfrak{S}}$ be the image of $\mathfrak{S}$ in $A/\Ann(M)$, then
\[\height(\widebar{\mathfrak{S}})=\codim(\supp(M/SM),\supp(M)).\]
If $M\neq SM$, then we have $\height(\widebar{\mathfrak{S}})\leq|S|$.
\end{lemma}
\begin{proof}
We denote by $\a$ the annihilator of $M$. By \cref{supp of module quotient by ideal product}, the support of the $A$-module $M/SM$ is equal to $V(\mathfrak{S}+\a)$, its codimension in $\supp(M)$ is therefore equal to the codimension of $V(\mathfrak{S}+\a)$ in $V(\a)$, which is also the codimension of $V((\mathfrak{S}+\a)/\a)$ in $\Spec(A/\a)$; this is exactly the height of $\widebar{\mathfrak{S}}$. If $M\neq SM$, then the inequality $\height(\widebar{\mathfrak{S}})\leq|S|$ is evident if $S$ is infinite, and follows from \cref{Krull principal ideal theorem} if $S$ is finite.
\end{proof}
Let $A$ be a Noetherian ring, $M$ be a finitely generated $A$-module, and $S$ be a subset of $A$. We say that the subset $S$ is \textbf{strongly secant} for $M$ if the reverse inequality of \cref{CM module strongly secant lemma} is true, in other words, if we have
\[|S|\leq\codim(\supp(M/SM),\supp(M)).\]
From this definition, we see that any finite subset $S$ of $A$ such that $SM=M$ is strongly secant for $M$, and if $SM\neq M$, then for a subset $S$ to be strongly secant for $M$, it is necessary and sufficient that
\[|S|=\height(\widebar{\mathfrak{S}})=\codim(\supp(M/SM),\supp(M)).\]
\begin{example}
If $A$ is a Noetherian local ring and $M$ is a nonzero finitely generated $A$-module, then any subset $S$ of $\m_A$ that is strongly secant for $M$ is secant for $M$. In fact, as the $A$-module $M/SM$ is nonzero by Nayakama's lemma, we have
\[|S|\leq\codim(\supp(M/SM),\supp(M))\leq\dim(M)-\dim(M/SM)\]
(\cref{topo space codim prop}), whence our assertion.
\end{example}
\begin{proposition}\label{Noe module strongly secant iff image in M_p secant}
Let $A$ be a Noetherian ring, $M$ be a finitely generated $A$-module, and $S$ be a finite subset of $A$. Then the following conditions are equivalent:
\begin{itemize}
\item[(\rmnum{1})] $S$ is strongly secant for $M$;
\item[(\rmnum{2})] for any prime ideal $\p$ of $\supp(M/SM)$, the canonical map $i_\p:A\to A_\p$ induces a bijection from $S$ onto a subset of $\p A_\p$ that is secant for $M_\p$.
\end{itemize}
\end{proposition}
\begin{proof}
Let $\p\in\supp(M/SM)$ and $\tilde{S}$ be the image of $S$ in $A_\p$. The set $\tilde{S}$ is contained in the maximal ideal $\p A_\p$, and we have (\cref{dimension of module prop})
\[\dim(M_\p/\tilde{S}M_\p)=\codim(V(\p),\supp(M/SM)).\]
If $S$ is strongly secant for $M$, the inequality $|S|\leq\codim(\supp(M/SM),\supp(M))$ and \cref{topo space codim prop}(b) implies the relations
\[|S|+\dim(M_\p/\tilde{S}M_\p)\leq\codim(V(\p),\supp(M))=\dim(M_\p).\]
As $M_\p$ is nonzero, we also have $\dim(M_\p)\leq|\tilde{S}|+\dim(M_\p/\tilde{S}M_\p)$ by formula (\ref{dimension and secant sequence-3}). Condition (\rmnum{2}) then follows from the inequality $|\tilde{S}|\leq|S|$.\par
Conversely, assume condition (\rmnum{2}); we may suppose that $M\neq SM$. For any prime ideal $\p\in\supp(M/SM)$, 
\[|S|=|\tilde{S}|\leq\dim(M_\p)=\codim(V(\p),\supp(M))\]
this implies (\rmnum{1}) by passing to infermum, in view of the definition of codimensions.
\end{proof}
\begin{corollary}\label{Noe module regular sequence is strongly secant}
Let $A$ be a Noetherian ring and $M$ be a finitely generated $A$-module. Then any $M$-regular sequence is strongly secant for $M$.
\end{corollary}
\begin{proof}
Let $\bm{x}$ be a $M$-regular sequence, and $\mathfrak{I}$ be the ideal it generates. For any prime ideal $\p\in\supp(M/\mathfrak{I}M)$, the image of $\bm{x}$ in $A_\p$ is an $M$-regular sequence of elements of $\p A_\p$, so it is secant for $M_\p$.
\end{proof}
\begin{proposition}\label{CM module quotient by strongly secant is CM}
Let $A$ be a Noetherian ring, $M$ be a finitely generated Cohen-Macaulay $A$-module, and $S$ be a finite subset of $A$ that is strongly secant for $M$. Then the $A$-module $M/SM$ is Cohen-Macaulay.
\end{proposition}
\begin{proof}
For any maximal ideal $\m\in\supp(M/SM)$, the image of $S$ in $A_\m$ is secant for $M_\m$ (\cref{Noe module strongly secant iff image in M_p secant}). Since $M_\m$ is a Cohen-Macaulay $A_\m$-module (\cref{CM module localization is CM}), so is $(M/SM)_\m$ by \cref{CM over local Noe iff quotient by regular is CM}, whence the proposition.
\end{proof}
\begin{theorem}[\textbf{Cohen-Macaulay}]\label{CM module iff M/IM no embedded associated prime}
Let $A$ be a Noetherian ring and $M$ be a finitely generated $A$-module. Then the following conditions are equivalent:
\begin{itemize}
\item[(\rmnum{1})] the $A$-module is Cohen-Macaulay;
\item[(\rmnum{2})] for any ideal $\mathfrak{I}$ of $A$ generated by an $M$-regular sequence of elements of $A$, the $A$-module $M/\mathfrak{I}M$ has no embedded associated prime ideals;
\item[(\rmnum{3})] for any finite subset $S$ of $A$ that is strongly secant for $M$, the $A$-module $M/SM$ has no embedded associated prime ideals;
\end{itemize}
\end{theorem}
\begin{proof}
If $M$ is Cohen-Macaulay and $S$ is a finite subset that is strongly secant for $M$, the $A$-module $M/SM$ is Cohen-Macaulay by \cref{CM module quotient by strongly secant is CM}, and in particular it has no embedded associated primes by \cref{CM module supp is catenary}. This proves (\rmnum{1})$\Rightarrow$(\rmnum{3}), and (\rmnum{3})$\Rightarrow$(\rmnum{2}) follows from \cref{Noe module regular sequence is strongly secant}.\par
Finally, to prove that (\rmnum{2})$\Rightarrow$(\rmnum{1}), let $\p\in\supp(M)$, we prove that the $A_\p$-module $M_\p$ is Cohen-Macaulay by recurrence on the integer $\dim(M_\p)$. If $\dim(M_\p)=0$, then $M_\p$ is of finite length, so is Cohen-Macaulay (\cref{CM module if length finite}). Suppose that $\dim(M_\p)>0$, which means that $\p$ is not a minimal element of $\supp(M)$. As $M$ has no embedded associated primes, each element in $\Ass(M)$ is minimal, so $V(\p)\cap\Ass(M)=\emp$ and there exists an element $x$ of $\p$ such that the homothety $x_M$ is injective (\cref{depth of module zero iff annihilated by element}). The homothety $x_{M_\p}$ is then injective and we have $\dim(M_\p/xM_\p)<\dim(M_\p)$ by \cref{Noe local ring secant iff minimal prime of supp}. By the recurrence hypotheses applied to the $A$-module $M/xM$ and the prime ideal $\p$ of $\supp(M/xM)$, the $A_\p$-module $M_\p/xM_\p$ is Cohen-Macaulay, which implies that the $A_\p$-module $M_\p$ is Cohen-Macaulay (\cref{CM over local Noe iff quotient by regular is CM}).
\end{proof}
\subsection{Cohen-Macaulay rings}
We say that a ring $A$ is \textbf{Cohen-Macaulay} if it is Noetherian and the $A$-module $A$ is Cohen-Macaulay. 
\begin{example}[\textbf{Examples of Cohen-Macaulay rings}]\label{CM ring example}
\mbox{}
\begin{itemize}
\item[(a)] Any Artinian ring is Cohen-Macaulay: this follows directly from \cref{CM module if length finite}.
\item[(b)] A Cohen-Macaulay ring has no embedded associated primes (\cref{CM module supp is catenary}). Conversely, let $A$ is Notherian ring of dimension $\leq 1$ which has no embedded associated primes; for any nonempty finite strongly secant subset $S$ of $A$, the $A$-module $A/SA$ has dimension $\leq 0$, hence is Cohen-Macaulay; therefore $A$ is Cohen-Macaulay by \cref{CM module iff M/IM no embedded associated prime}. In particular, any reduced Noetherian ring of dimension $\leq 1$ is Cohen-Macaulay (\cref{Noe module regular sequence is strongly secant}). 
\item[(c)] A Noetherian normal ring of dimension $\leq 2$ is Cohen-Macaulay by \cref{Noe ring normal iff R1S2}. Conversely, let $A$ be a Cohen-Macaulay ring whose local ring $A_\p$ is integrally closed for any prime ideal $\p$ with $\height(\p)\leq 1$; then $A$ is normal by \cref{Noe ring normal iff R1S2}.
\item[(d)] If $A$ is Cohen-Macaulay, so is the localizatin $S^{-1}A$ for any multiplicative subset $S$ of $A$ (\cref{CM module localization is CM}). Conversely, it follows from our definition that, if $A_\m$ is Cohen-Macaulay for any maximal ideal $\m$ of $A$, then $A$ is Cohen-Macaulay.
\item[(e)] Let $A$ be a Noetherian ring and $\mathfrak{I}$ be an ideal of $A$. For the ring $A/\mathfrak{I}$ to be Cohen-Macaulay, it is necessary and sufficient that it is a Cohen-Macaulay $A$-module (\cref{CM module over quotient ring iff}). On the other hand, if $A$ is local and $\mathfrak{I}$ is generated by an $A$-regular sequence, then $A/\mathfrak{I}$ is Cohen-Macaulay if and only if $A$ is (\cref{CM over local Noe iff quotient by regular is CM}).
\item[(f)] For a Noetherian local ring to be Cohen-Macaulay, it is necessary and sufficient that it has a defining ideal that is generated by an $A$-regular sequence: this follows from \cref{CM module over Noe local iff depth=dim} and the fact an $A$-regular sequence of $\m_A$ generates a defining ideal if and only if its length is equal to $\dim(A)$ (\cref{Noe local ring defining ideal iff}). In particular, any Noetherian regular local ring is Cohen-Macaulay (\cref{Noe local ring regular iff m_A generated by completely secant}). More generally, the quotient of a Noetherian regular local ring $A$ by an ideal generated by an $A$-regular sequence is Cohen-Macaulay (\cref{CM module quotient regular sequence is CM})
\end{itemize}
\end{example}
\begin{proposition}\label{CM ring iff unmixed}
For a Noetherian ring $A$, the following conditions are equivalent:
\begin{itemize}
\item[(\rmnum{1})] $A$ is a Cohen-Macaulay ring;
\item[(\rmnum{2})] for any closed subset $F$ of $\Spec(A)$, we have $\depth_F(A)=\codim(F)$;
\item[(\rmnum{3})] any ideal $\mathfrak{I}$ of $A$ contains an $A$-regular sequence of length $\height(\mathfrak{I})$;
\item[(\rmnum{3}')] any maximal ideal $\m$ of $A$ contains an $A$-regular sequence of length $\height(\m)$;
\item[(\rmnum{4})] for any ideal $\mathfrak{I}$ of $A$, we have $\Ext_A^i(A/\mathfrak{I},A)=0$ for $i<\height(\mathfrak{I})$.
\item[(\rmnum{4}')] for any maximal ideal $\m$ of $A$, we have $\Ext_A^i(A/\m,A)=0$ for $i<\height(\m)$;
\item[(\rmnum{5})] for any prime ideal $\p$ of $A$ and any ideal $\mathfrak{I}$ of $A_\p$ generated by a maximal secant sequence for $A_\p$, we have $e_{\mathfrak{I}}(A_\p)=\ell(A_\p/\mathfrak{I}A_\p)$.
\item[(\rmnum{5}')] for any maximal ideal $\m$ of $A$, there exists an ideal $\mathfrak{I}$ of $A_\m$, generated by a maximal secant sequence for $A_\m$, satisfying $e_{\mathfrak{I}}(A_\m)=\ell(A_\m/\mathfrak{I}A_\m)$.
\item[(\rmnum{6})] (\textbf{Cohen-Macaulay Criterion}) for any finite subset $S$ of $A$ such that the ideal $\mathfrak{S}$ generated by $S$ has height $|S|$, the $A$-module $A/\mathfrak{S}$ has no embedded associated primes.
\end{itemize}
\end{proposition}
\begin{proof}
The equivalence of (\rmnum{1}) and (\rmnum{2}) follows from \cref{CM module iff depth=dim for any prime}. On the other hand, in view of \cref{depth of module is maximal length of regular sequence} and the definition of depth, conditions (\rmnum{3}) and (\rmnum{4}) (resp. (\rmnum{3}') and (\rmnum{4}')) signifies that we have $\depth_A(\mathfrak{I},A)\geq\height(\mathfrak{I})$ for any ideal (resp. any maximal ideal) $\mathfrak{I}$ of $A$. We then have
\[\text{(\rmnum{1})$\Leftrightarrow$(\rmnum{2})$\Rightarrow$(\rmnum{3})$\Leftrightarrow$(\rmnum{4})$\Rightarrow$(\rmnum{3}')$\Leftrightarrow$(\rmnum{4}')}.\]
But (\rmnum{4}') implies, for any maximal ideal $\m$ of $A$, $\Ext_{A_\m}^i(\kappa(\m),A_\m)=0$ for $i<\dim(A_\m)$, whence $\depth(A_\m)=\dim(A_\m)$, and $A$ is then Cohen-Macaulay. Finally, the equivalence of (\rmnum{1}), (\rmnum{5}) and (\rmnum{5}') follows from \cref{CM over local Noe iff secant is regular}, and that of (\rmnum{1}) and (\rmnum{6}) from \cref{CM module iff M/IM no embedded associated prime}.
\end{proof}
\begin{remark}\label{module supp of fiber over prime char}
Let $\rho:A\to B$ be a ring homomorphism, and let $\p\in\Spec(A)$. Denote by $\widebar{B}$ the ring $\kappa(\p)\otimes_AB$, which is identified with $S^{-1}B/\p(S^{-1}B)$, where $S$ is the multiplicative subset $\rho(A-\p)$ of $B$. The prime ideals of $\widebar{B}$ are then of the form $\mathfrak{P}\widebar{B}$, where $\mathfrak{P}$ is a prime ideal of $B$ lying over $\p$. For such a prime ideal $\mathfrak{P}$, we have $S\sub B-\mathfrak{P}$, so the local ring of $\widebar{B}$ at $\mathfrak{P}\widebar{B}$ is identified with $B_\mathfrak{P}/\p B_\mathfrak{P}$, which is $\kappa(\p)\otimes_AB_\mathfrak{P}$.\par
Similarly, if $N$ is an $B$-module, the $\widebar{B}_{\mathfrak{P}\widebar{B}}$-module $(\kappa(\p)\otimes_AN)_{\mathfrak{P}\widebar{B}}$ is identified with $\kappa(\p)\otimes_AN_{\mathfrak{P}}$. Suppose moreover that the $B$-module $N$ is finitely generated, then by Nakayama's lemma, the condition $\kappa(\p)\otimes_AN_\mathfrak{P}=0$ is equivalent to $N_\mathfrak{P}=0$. Therefore the support of $\widebar{B}$-module $\kappa(\p)\otimes_AN$ is formed by the prime ideals $\mathfrak{P}\widebar{B}$, where $\mathfrak{P}$ runs through the prime ideals of $\supp_B(N)$ lying over $\p$. In particular, for the module $\kappa(\p)\otimes_AN$ to be nonzero, it is necessary and sufficient that there exists a prime ideal of $\supp_B(N)$ lying over $\p$.
\end{remark}
\begin{proposition}\label{CM module finite base change CM iff}
Let $\rho:A\to B$ be a homomorphism of Noetherian rings and $N$ be a finitely generated $B$-module that is also a finitely generated $A$-module. For the $A$-module $N$ to be Cohen-Macaulay, it is necessary and sufficient that the $B$-module $N$ is Cohen-Macaulay and, for any couple $(\mathfrak{M},\mathfrak{N})$ of maximal ideals of $\supp_B(N)$ such that $\mathfrak{M}^c=\mathfrak{N}^c$, we have $\dim_{B_\mathfrak{M}}(N_\mathfrak{M})=\dim_{B_{\mathfrak{N}}}(N_{\mathfrak{N}})$.
\end{proposition}
\begin{proof}
The $A$-module $B/\Ann_B(N)$ is isomorphic to a sub-$A$-module of the finitely generated $A$-module $\End_A(N)$, hence is finitely generated. By replacing $A$ by $A/\Ann_A(N)$ and $B$ by $B/\Ann_B(N)$, we may then reduce to the case where $\rho$ is injective and $B$ is a finite $A$-algebra, and where we have $\supp_A(N)=\Spec(A)$, $\supp_B(N)=\Spec(B)$. The map $f:\Spec(B)\to\Spec(A)$ induced by $\rho$ is then surjective and a prime ideal $\mathfrak{P}$ is maximal if and only if $f(\mathfrak{P})$ is maxima in $A$ (\cref{integral ring maximal ideal iff contraction is}).\par
Let $\m$ be a maximal ideal of $A$. By \cref{module supp of fiber over prime char}, the prime ideals of the ring $B_\m$ containing $\m B_\m$ are of the form $\mathfrak{P}B_\m$ where $\mathfrak{P}$ is an ideal of $B$ (necessarily maximal) such that $f(\mathfrak{P})=\m$. By \cref{depth of module base change pullback prop} and \cref{depth of module along closed localization char}, have
\[\depth_{A_\m}(N_\m)=\depth_{B_\m}(\m B_\m,N_\m)=\inf_{\mathfrak{P}\in f^{-1}(\m)}\depth_{B_\mathfrak{P}}(N_\mathfrak{P}),\]
while by \cref{integral extension dimension of module ideal prop} and \cref{dimension of module prop},
\[\dim_{A_\m}(N_\m)=\dim_{B_\m}(N_\m)=\sup_{\mathfrak{P}\in f^{-1}(\m)}\dim_{B_{\mathfrak{P}}}(N_\mathfrak{P}).\]
As we have $\depth_{B_\mathfrak{P}}(N_\mathfrak{P})\leq\dim_{B_\mathfrak{P}}(N_\mathfrak{P})$ for any $\mathfrak{P}\in f^{-1}(\m)$, the proposition then follows from these equalities.
\end{proof}
\begin{corollary}\label{CM module finite algebra is CM ring}
Let $\rho:A\to B$ be a homomorphism of Noetherian ring. If $B$ is a finite $A$-algebra and a Cohen-Macaulay $A$-module, then it is a Cohen-Macaulay ring. If $\rho$ is also injective, we have $\height(\a^e)=\height(\a)$ for any ideal $\a$ of $A$, and $\height(\b)=\height(\b^c)$ for any ideal $\b$ of $B$.
\end{corollary}
\begin{proof}
The first assertion follows from \cref{CM module finite base change CM iff}. Suppose that $\rho$ is injective, and let $\a$ be an ideal of $A$. We have $\height(\a)=\depth_A(\a,B)$ and $\height(\a^e)=\depth_B(\a^e,B)$ since the $A$-module $B$ is Cohen-Macaulay with support equal to $\Spec(A)$ (\cref{CM module iff depth=dim for any prime}), and $\depth_A(\a,B)=\depth_B(\a^e,B)$ by \cref{depth of module base change pullback prop}, whence $\height(\a^e)=\height(\a)$. Now if $\b$ is an ideal of $B$, then by the preceding arguments we have $\height(\b^c)=\height(\b^{ce})$. But $\b^{ce}$ is contained in $\b$, hence has height smaller than $\height(\b)$ and we have $\height(\b)\leq\height(\b^c)$ by \cref{integral extension dimension of module ideal prop}(b).
\end{proof}
\begin{corollary}\label{CM ring over Noe integrally closed is CM module}
Let $A$ be a Noetherian integrally closed ring and $B$ be a ring containing $A$. Suppose that $B$ is a finitely generated torsion-free $A$-module. If $B$ is a Cohen-Macaulay ring, the $A$-module $B$ is Cohen-Macaulay.
\end{corollary}
\begin{proof}
In fact, any prime ideals of $B$ lying over the same ideal of $A$ have the same height (\cref{integrally closed integral extension height prop}). We can then apply \cref{CM module finite base change CM iff} with $N=B$.
\end{proof}
\begin{corollary}\label{CM ring over Noe integrally closed finite field extension prop}
Let $A$ be an integrally closed ring, $K$ be its fraction field, $L$ be a finite $K$-algebra such that $[L:K]1_A$ is invertible in $A$, and $B$ be a sub-$A$-algebra of $L$ that is finite over $A$.
\begin{itemize}
\item[(a)] The sub-$A$-module $A1_B$ of $B$ is a direct factor.
\item[(b)] For any ideal $\mathfrak{I}$ of $A$, we have $\depth_A(\mathfrak{I},A)\geq\depth_{B}(\mathfrak{I}B,B)$.
\item[(c)] If $B$ is a Cohen-Macaulay ring, so is $A$.
\end{itemize}
\end{corollary}
\begin{proof}
The $K$-linear map $\tr_{L/K}:L\to K$ sends $B$ into $A$ (\cref{integral element in field extension norm and trace integral}), so defines an $A$-linear map $t:B\to A$. For any $x\in A$, we have $t(x1_B)=[L:K]x$, whence the assertion of (a). By \cref{depth of module base change pullback prop}, we have $\depth_A(\mathfrak{I},B)=\depth_B(\mathfrak{I}B,B)$; but by (a) and \cref{depth of module product is inf}, we have $\depth_A(\mathfrak{I},A)\geq\depth_A(\mathfrak{I},B)$, whence (b).\par
If the ring $B$ is Noetherian, so is $A$: in fact, by (a) we have $\a B\cap A=\a$ for any ideal $\a$ of $A$, so any increasing sequence $(\a_n)$ of ideals of $A$ is stationary if the sequence $(\a_nB)$ is stationary. Under the hypothesis of (c), the $A$-module $B$ is Cohen-Macaulay (\cref{CM ring over Noe integrally closed is CM module}), and so is the $A$-module $A$ (\cref{CM module direct factor is CM}). 
\end{proof}
\begin{example}
The result of \cref{CM ring over Noe integrally closed finite field extension prop} is applicable in the following situations:
\begin{itemize}
\item[(a)] Consider a Noetherian integrally closed ring $A$, a separable extension $L$ of its fraction field, of finite degree $n$ such that $n1_A$ is invertible in $A$, and let $B$ be the integral closure of $A$ in $L$ (\cref{integral closure in finite separable finite if Noe}).
\item[(b)] Consider a Noetherian integrally closed ring $B$ and a finite group $G$ acting on $B$, such that $|G|$ is invertible in $B$. Let $A$ be the ring of invariant elements of $B$ under the action of $G$. Then we are in a particular case of (a): the group $G$ acts on the fraction field $L$ of $B$, and the invariant subfield of $L$ for this action is the fraction field $K$ of $A$ (\cref{algebra action invariant and fraction field}). The extension $L/K$ is Galois, and a fortiori separable; its Galois group is isomorphic to $G$, so $[L:K]$ is equal to $|G|$. The inverse of $[L:K]1_B$ is invariant under $G$, so $[L:K]1_A$ is invertible in $A$. As $B$ is integrally closed, the ring $A$, being equal to $K\cap B$, is then integrally closed and $B$ is its integral closure in $L$ (\cref{algebra action integral over fix point}).
\end{itemize}
In particular, if the ring $B$ is Cohen-Macaulay, then so is the ring $A$.
\end{example}
\subsection{Flat base change}
\begin{proposition}\label{CM module tensor product with flat CM iff}
Let $\rho:A\to B$ be a homomorphism of Noetherian rings, $M$ be a finitely generated $A$-module and $N$ be a finitely generated $B$-module that is flat over $A$. Let $f:\Spec(B)\to\Spec(A)$ be the associated map. The following conditions are equivalent:
\begin{itemize}
\item[(\rmnum{1})] the $B$-module $M\otimes_AN$ is Cohen-Macaulay;
\item[(\rmnum{2})] the $(\kappa(\p)\otimes_AB)$-module $\kappa(\p)\otimes_AN$ is Cohen-Macaulay for any $\p\in\supp(M)$, and the $A_\p$-module $M_\p$ is Cohen-Macaulay for any $\p\in f(\supp_B(N))$;
\item[(\rmnum{3})] for any maximal ideal of $\supp_B(N)$ whose inverse image $\p$ in $A$ belongs to $\supp_A(M)$, the $A_\p$-module $M_\p$ and the $(\kappa(\p)\otimes_AB)$-module $\kappa(\p)\otimes_AN$ are Cohen-Macaulay.
\end{itemize}
If the $B$-module is faithfully flat, these conditions imply that the $A$-module $M$ is Cohen-Macaulay.
\end{proposition}
\begin{proof}
Let $\mathfrak{P}$ be a prime ideal belonging to the support of $M\otimes_AN$, and put $\p=\mathfrak{P}^c$. As the module $(M\otimes_AN)_{\mathfrak{P}}$ is identified with $M_\p\otimes_{A_\p}N_{\mathfrak{P}}$, the modules $M_\p$ and $N_\mathfrak{P}$ are necessarily nonzero, and so is the module $\kappa(\p)\otimes_AN_\mathfrak{P}$ (\cref{module supp of fiber over prime char}). The $A_\p$-module $N_\mathfrak{P}$, being isomorphic to a fraction module of $N_\p$, is flat and by \cref{depth of module flat tensor and regular sequence}(b) and \cref{Noe local ring extension dim of tensor} we have the equalities
\begin{gather*}
\depth_{B_{\mathfrak{P}}}((M\otimes_AN)_{\mathfrak{P}})=\depth_{A_\p}(M_\p)+\depth_{B_{\mathfrak{P}}}(\kappa(\p)\otimes_AN_{\mathfrak{P}}),\\
\dim_{B_{\mathfrak{P}}}((M\otimes_AN)_{\mathfrak{P}})=\dim_{A_\p}(M_\p)+\dim_{B_{\mathfrak{P}}}(\kappa(\p)\otimes_AN_{\mathfrak{P}}),
\end{gather*}
where each term are nonnegative integers. In view of the fact that the $B_{\mathfrak{P}}$-module $\kappa(\p)\otimes_AN_{\mathfrak{P}}$ is Cohen-Macaulay if and only if so is it as a $(\kappa(\p)\otimes_AB_{\mathfrak{P}})$-module (\cref{CM module over quotient ring iff}), we then duce the equivalence of the following conditions:
\begin{itemize}
\item[($\alpha$)] the $B_\mathfrak{P}$-module $(M\otimes_AN)_\mathfrak{P}$ is Cohen-Macaulay;
\item[($\beta$)] the $A_\p$-module $M_\p$ and the $(\kappa(\p)\otimes_AB_\mathfrak{P})$-module $\kappa(\p)\otimes_AN_\mathfrak{P}$ are Cohen-Macaulay.
\end{itemize}
We now prove that (\rmnum{3}) implies (\rmnum{1}). Let $\mathfrak{N}$ be a maximal ideal of $B$ belonging to the support of $M\otimes_AN$, and put $\p=\mathfrak{N}^c$. By the preceding remarks, we have $\p\in\supp_A(M)$, so (\rmnum{3}) and \cref{module supp of fiber over prime char} imply that the condition ($\beta$) above is satisfied for $\mathfrak{P}=\mathfrak{N}$, so the $B_\mathfrak{N}$-module $(M\otimes_AN)_{\mathfrak{N}}$ is Cohen-Macaulay, whence (\rmnum{1}).\par
The implication (\rmnum{2})$\Rightarrow$(\rmnum{3}) is clear, so it remains to prove that (\rmnum{1})$\Rightarrow$(\rmnum{2}). Suppose that the $B$-module $M\otimes_AN$ is Cohen-Macaualy, and let $\p$ be an element of $\supp_A(M)$. We can suppose that the $(\kappa(\p)\otimes_AB)$-module $\kappa(\p)\otimes_AN$ is nonzero, which means there exists a prime ideal $\mathfrak{P}$ of $\supp_B(N)$ lying over $\p$ (\cref{module supp of fiber over prime char}). The $B_\mathfrak{P}$-module $(M\otimes_AN)_\mathfrak{P}$ is Cohen-Macaulay by hypotheses, so it follows from the implication ($\alpha$)$\Rightarrow$($\beta$) and \cref{module supp of fiber over prime char} that the $A_\p$-module $M_\p$ and the $(\kappa(\p)\otimes_AB)$-module $(\kappa(\p)\otimes_AN)$ are Cohen-Macaulay, whence (\rmnum{2}).\par
If moreover $N$ is faithfully flat over $A$, we have $\kappa(\p)\otimes_AN\neq 0$ for any $p\in\Spec(A)$, so $f(\supp_B(N))=\Spec(A)$ by \cref{module supp of fiber over prime char}, and (\rmnum{2}) then implies that $M$ is Cohen-Macaulay.
\end{proof}
\begin{corollary}\label{Noe local ring local homomorphism tensor module CM iff}
Let $\rho:A\to B$ be a local homomorphism of Noetherian local rings, $M$ be a finitely generated nonzero $A$-module and $N$ be a finitely generated nonzero $B$-module which is flat over $A$. For the $B$-module $M\otimes_AN$ to be Cohen-Macaulay, it is necessary and sufficient that the $A$-module $M$ and the $(\kappa_A\otimes_AB)$-module $\kappa_A\otimes_AN$ are Cohen-Macaulay.
\end{corollary}
\begin{proof}
In fact, $N$ is a faithfully flat $A$-module since $\kappa_A\otimes_AN$ is nonzero (\ref{module faithfully flat iff}).
\end{proof}
\begin{corollary}\label{Noe ring flat finite base change of CM module}
Let $A$ be a Noetherian ring, $B$ be a finite and flat $A$-algebra, and $M$ be a finitely generated Cohen-Macaulay $A$-module. Then the $B$-module $M\otimes_AB$ is Cohen-Macaulay.
\end{corollary}
\begin{proof}
For any $\p\in\Spec(A)$ the ring $\kappa(\p)\otimes_AB$ is a finite $\kappa(\p)$-algebra, hence Cohen-Macaulay, and we can apply \cref{CM module tensor product with flat CM iff}.
\end{proof}
\begin{corollary}\label{Noe ring I-adic completion CM iff}
Let $A$ be a Noetherian ring, $\mathfrak{I}$ be an ideal of $A$ and $M$ be a finitely generated $A$-module. Let $\widehat{A}$ and $\widehat{M}$ be the $\mathfrak{I}$-adic completion of $A$ and $M$, and $S$ be the multiplicative subset $1+\mathfrak{I}$ of $A$. Consider the following conditions:
\begin{itemize}
\item[(\rmnum{1})] the $A$-module $M$ is Cohen-Macaulay;
\item[(\rmnum{2})] the $\widehat{A}$-module $\widehat{M}$ is Cohen-Macaulay;
\item[(\rmnum{3})] the $S^{-1}A$-module $S^{-1}M$ is Cohen-Macaulay;
\item[(\rmnum{4})] for any maximal ideal $\m\in\supp(M)\cap V(\mathfrak{I})$, the $A_\m$-module $M_\m$ is Cohen-Macaulay;
\item[(\rmnum{5})] for any prime ideal $\p\in\supp(M)$ not meeting $S$, the $A_\p$-module $M_\p$ is Cohen-Macaulay and the ring $\kappa(\p)\otimes_A\widehat{A}$ is Cohen-Macaulay. 
\end{itemize}
Then conditions (\rmnum{2}) to (\rmnum{5}) are equivalent, and are implied by (\rmnum{1}). If the ideal $\mathfrak{I}$ is contained in the Jacobson radical of $A$, then they are all equivalent.
\end{corollary}
\begin{proof}
We have seen that (\rmnum{1}) implies (\rmnum{3}) in \cref{CM ring example}(d), and (\rmnum{3}) is identical to (\rmnum{1}) if the ideal $\mathfrak{I}$ is contained in the Jacobson radical of $A$ (since the elements of $S$ are then invertible). By \cref{Noe semilocal ring completion is Noe semilocal}, the ring $\widehat{A}$ is Noetherian, and it is identified with the $S^{-1}\mathfrak{I}$-adic completion of $S^{-1}A$ (\cref{Zariski ring and localization}); similarly, $\widehat{M}$ is identified with the $S^{-1}\mathfrak{I}$-adic completion of $S^{-1}M$. To prove the equivalence of (\rmnum{2}) to (\rmnum{5}), we can then replace $A$ by $S^{-1}A$, $\mathfrak{I}$ by $S^{-1}\mathfrak{I}$, and $M$ by $S^{-1}M$. In other words, we can assume that $\mathfrak{I}$ is contained in the Jacobson radical of $A$. The $A$-module $\widehat{A}$ is then faithfully flat by \cref{Zariski ring def}.\par
Now it is clear that (\rmnum{5}) implies (\rmnum{4}) and (\rmnum{4}) implies (\rmnum{1}). Let $\m$ be a maximal ideal of $A$; $\m\widehat{A}$ is then a maximal ideal of $\widehat{A}$ lying over $\m$, and any maximal ideal of $\widehat{A}$ is obtained in this way (\cref{filtration I-adic completion maximal ideal}). The ring $\kappa(\m)\otimes_A\widehat{A}$ is the residue field of $\hat{\m}$, hence Cohen-Macaulay. If the $A$-module $M$ is Cohen-Macaulay, so is the $\widehat{A}$-module $\widehat{M}$ (\cref{CM module tensor product with flat CM iff}); this porves (\rmnum{1})$\Rightarrow$(\rmnum{2}).\par 
Finally, to see that (\rmnum{2})$\Rightarrow$(\rmnum{5}), note that $\widehat{A}$-module $\widehat{M}$ is isomorphic to $M\otimes_A\widehat{A}$ (\cref{filtration on completion is product with completion ring}); if it is Cohen-Macaulay, then by \cref{CM module tensor product with flat CM iff} we conclude that $\kappa(\p)\otimes_A\widehat{A}$ is a Cohen-Macaulay ring for any $\p\in\supp(M)$, and that the $A$-module $M$ is Cohen-Macaulay.
\end{proof}
\begin{proposition}\label{Noe ring flat homomorphism CM ring iff}
Let $\rho:A\to B$ be a flat homomorphism of Noetherian rings. Then the following conditions are equivalent:
\begin{itemize}
\item[(\rmnum{1})] $B$ is a Cohen-Macaulay ring;
\item[(\rmnum{2})] for any prime ideal $\mathfrak{P}$ of $B$ with $\p=\mathfrak{P}^c$, the rings $A_\p$ and $\kappa(\p)\otimes_AB$ are Cohen-Macaulay;
\item[(\rmnum{3})] for any maximal ideal $\mathfrak{N}$ of $B$ with with $\p=\mathfrak{N}^c$, the rings $A_\p$ and $\kappa(\p)\otimes_AB$ are Cohen-Macaulay;
\end{itemize}
If $B$ is faithfully flat over $A$, these conditions imply that $A$ is a Cohen-Macaulay ring.
\end{proposition}
\begin{proof}
This is a particular case of \cref{CM module tensor product with flat CM iff}, where $M=A$ and $N=B$.
\end{proof}
\begin{corollary}\label{CM ring finite flat extension is CM}
Any finite and flat algebra over a Cohen-Macaulay ring is Cohen-Macaulay.
\end{corollary}
\begin{corollary}\label{CM ring polynomial ring is CM}
Let $A$ be a Cohen-Macaulay ring and $n>0$ be a positive integer. Then the rings $A[X_1,\dots,X_n]$ and $A\llbracket X_1,\dots,X_n\rrbracket$ are Cohen-Macaulay.
\end{corollary}
\begin{proof}
It suffices to deal with the case where $n=1$. The ring $A[T]$ is Noetherian by Hilbert basis theorem, and for any field $k$, the ring $k[T]$ is Cohen-Macaulay by \cref{CM ring example}. The ring $A[T]$ is then Cohen-Macaulay by \cref{Noe ring flat homomorphism CM ring iff} and $A\llbracket T\rrbracket$ is Cohen-Macaulay by \cref{Noe ring I-adic completion CM iff}.
\end{proof}
\begin{corollary}\label{CM ring is universal catenary}
Any finitely generated algebra over a Cohen-Macaulay ring is catenary.
\end{corollary}
\begin{proof}
In fact, any such algebra is a quotient of a polynomial algebra over a Cohen-Macaulay ring, hence a quotient of Cohen-Macaulay ring, and therefore catenary (\cref{CM module supp is catenary}).
\end{proof}
\section{Depth and homological dimension}
\subsection{Projective dimension and injective dimension}
Let $A$ be a ring, $M$ be a $A$-module. Recall that the \textbf{projective dimension} of $M$, denoted by $\projdim_A(M)$, is defined to be the infermum of the lengths of projective resolutions of $M$. We have $\projdim_A(0)=-\infty$ and $\projdim_A(M)\geq 0$ if $M$ is nonzero. For the $A$-module $M$ to be projective, it is necessary and sufficient that $\projdim_A(M)\leq 0$.
\begin{example}\label{Koszul complex is free resolution for regular sequence}
Let $\mathfrak{I}$ be an ideal of $A$ generated by an $A$-regular sequence $\bm{x}=(x_1,\dots,x_r)$. We have $\projdim_A(A/\mathfrak{I})\leq r$: in fact, this is clear if $A=0$, and in the contrary case, the Koszul complex $K_\bullet(\bm{x},A)$ is a free resolution of $A/\mathfrak{I}$ of length $r$. Moreover, for any $A$-module $N$, the $A$-module $\Ext_A^r(A/\mathfrak{I},N)$ and $N/\mathfrak{I}N$ are isomorphic, so for that $\projdim_A(A/\mathfrak{I})=r$, it is necessary and sufficient that $\mathfrak{I}$ is a proper ideal of $A$ (A, \Rmnum{10}, p.134, prop.1).
\end{example}
Similarly, we define the \textbf{injective dimension} of $M$, denoted by $\injdim_A(M)$, to be the infermum of the lengths of injective resolutions of $M$. It is clear that we have $\injdim_A(0)=-\infty$, and $\injdim_A(M)\geq 0$ if $M\neq 0$. For the $A$-module $M$ to be injective, it is necessary and sufficient that $\injdim_A(M)\leq 0$.
\begin{proposition}\label{module id<n iff Ext zero}
Let $A$ be a ring, $M$ be an $A$-module and $n\geq 0$ be an integer. The following conditions are equivalent:
\begin{itemize}
\item[(\rmnum{1})] $\injdim_A(M)\leq n$;
\item[(\rmnum{2})] for any $A$-module $N$ and any integer $i>n$, we have $\Ext_A^i(N,M)=0$;
\item[(\rmnum{3})] for any ideal $\a$ of $A$, we have $\Ext_A^{n+1}(A/\a,M)=0$;
\item[(\rmnum{4})] for any exact sequence of $A$-modules
\[\begin{tikzcd}
0\ar[r]&M\ar[r]&I^0\ar[r]&\cdots\ar[r]&I^{n-1}\ar[r]&Q\ar[r]&0
\end{tikzcd}\]
where the $I^i$ are injective, the $A$-module $Q$ is injective.
\end{itemize}
\end{proposition}
\begin{proof}
The implications (\rmnum{1})$\Rightarrow$(\rmnum{2})$\Rightarrow$(\rmnum{3}) are clear from definition. In the situation of (\rmnum{4}), we have for any $A$-module $N$ an isomorphism $\Ext_A^1(N,Q)\cong\Ext_A^{n+1}(N,M)$, so the hypothesis of (\rmnum{3}) implies that $\Ext_A^1(A/\a,Q)=0$ for any ideal $\a$ of $A$, and $Q$ is therefore injective.\par
Finally, for (\rmnum{4})$\Rightarrow$(\rmnum{1}), consider the exact sequence
\[\begin{tikzcd}
0\ar[r]&M\ar[r]&I^0(M)\ar[r]&\cdots\ar[r]&I^{n-1}(M)\ar[r]&K^{n-1}(M)\ar[r]&0
\end{tikzcd}\]
where $I^i(M)$ are free $A$-modules; if condition (\rmnum{4}) is satisfied, then the $A$-module $K^{n-1}(M)$ is injective, whence $\injdim_A(M)\leq n$.
\end{proof}
Recall that the \textbf{homological dimension} (or \textbf{global dimension}) of the ring $A$, denoted by $\gldim(A)$, is the supremum of integers $n$ such that there exists two $A$-modules $M$ and $N$ such that $\Ext_A^n(M,N)$ is nonzero. This is then also the supremum of the projective (or injective) dimensions of $A$-modules.
\begin{proposition}\label{module Tor and Ext localization prop}
Let $A$ a ring, $M$ and $N$ be $A$-modules, $i$ an integer and $S$ be a multiplicative subset of $A$. Then we have a canonical isomorphism of $S^{-1}A$-modules
\[S^{-1}\Tor_i^a(M,N)\to\Tor_i^{S^{-1}A}(S^{-1}M,S^{-1}N).\]
If the ring $A$ is Noetherian and $M$ is finitely generated, we have a canonical isomorphism 
\[S^{-1}\Ext_A^i(M,N)\to\Ext_{S^{-1}A}^i(S^{-1}M,S^{-1}N).\]
\end{proposition}
\begin{proof}
As the $A$-module $S^{-1}A$ is flat, this follows from (A, \Rmnum{10}, p.110, prop.9) and (A, \Rmnum{10}, p.111, prop.10).
\end{proof}
\begin{corollary}\label{module Tor and Ext supp prop}
Let $A$ be a ring, $M$ and $N$ be $A$-modules, and $i$ be an integer.
\begin{itemize}
\item[(a)] The support of $\Tor_i^A(M,N)$ is contained in $\supp(M)\cap\supp(N)$, and so is the support of $\Ext_A^i(M,N)$ if $A$ is Noetherian and $M$ is finitely generated.
\item[(b)] Suppose that $A$ is Noetherian and the module $M$ and $N$ are finitely generated. If the $A$-module $M\otimes_AN$ is of finite length, so are the $A$-modules $\Tor_i^A(M,N)$ and $\Ext_A^i(M,N)$.
\end{itemize}
\end{corollary}
\begin{proof}
If $\p$ is a prime ideal of $A$ not belonging to $\supp(M)\cap\supp(N)$, then the modules $M_\p$ and $N_\p$ are nonzero, and this implies (a) in view of \cref{module Tor and Ext localization prop}. Now for a finitely generated module over a Noetherian ring to be of finite length, it is necessary and sufficient that its support consists of maximal ideals (\cref{associated prime maximal iff finite length}). Under the hypotheses of (b), the $A$-modules $\Tor_i^A(M,N)$ and $\Ext_A^i(M,N)$ are finitely generated, so the assertions of (b) follow from (a) (\cref{supp of module finite tensor}).
\end{proof}
\begin{proposition}\label{Noe ring pd id hd localization prop}
Let $A$ be a Noetherian ring, $M$ be a finitely generated $A$-module and $N$ be an $A$-module. Then we have
\begin{align}\label{Noe ring pd id hd localization prop-1}
\projdim_A(M)=\sup_{\p}\projdim_{A_\p}(M_\p),\quad \injdim_A(M)=\sup_{\p}\injdim_{A_\p}(N_\p)
\end{align}
where $\p$ runs through prime (resp. maximal) ideals of $A$. Moreover, the map $\p\mapsto\projdim_{A_\p}(M_\p)$ on $\Spec(A)$ is upper semi-continuous.
\end{proposition}
\begin{proof}
Let $n\geq 0$ be an integer, and suppose that we have $\projdim_A(M)<n$. For any prime ideal $\p$ of $A$ and any $A_\p$-module $Q$, the $A_\p$-module $\Ext_{A_\p}^n$ is isomorphic to $(\Ext_A^n(M,Q))_\p$ (\cref{module Tor and Ext localization prop}), hence is zero. In view of (A, \Rmnum{10}, p.134, prop.1), we then deduce the inequality $\projdim_{A_\p}(M_\p)\leq\projdim_A(M)$. Suppose conversely that we have $\projdim_{A_\m}(M_\m)<n$ for any maximal ideal $\m$ of $A$, and let $R$ be an $A$-module. We have $(\Ext_A^n(M,R))_\m=0$ for any $\m$ (\cref{module Tor and Ext localization prop}), so $\Ext_A^n(M,R)=0$, which implies $\projdim_A(M)<n$. The first inequality of (\ref{Noe ring pd id hd localization prop-1}) then follows, and the second one can be proved similarly. As $\gldim(A)$ is the suporemum of injective dimensions of $A$-modules, the third one then follows.\par
Now let $\p$ be a prime ideal of $A$ and $n=\projdim_{A_\p}(M_\p)$; we want to show that there exists an open neighborhood $U$ of $\p$ in $\Spec(A)$ such that $\projdim_{A_\q}(M_\q)\leq n$ for any $\q\in U$. This is clear if $n=+\infty$, and if $n=-\infty$, this follows from the fact that $\supp(M)$ is closed. Suppose now that $n$ is finite and choose an exact sequence of $A$-modules
\[\begin{tikzcd}
P_{n-1}\ar[r,"d_{n-1}"]&P_{n-2}\ar[r]&\cdots\ar[r]&P_0\ar[r,"d_0"]&M\ar[r]&0
\end{tikzcd}\]
where the $P_i$ are free of finite rank. Put $P=\ker d_{n-1}$, which is an $A$-module of finite presentation. The $A_\p$-module $P_\p$ is projective, so by \cref{projective module rank n iff}, there exists an element $f$ of $A-\p$ such that $A_f$-module $P_f$ is free; the $A_\q$-module $P_\q$ is then free for $\q\in D(f)$, which proves the second assertion.
\end{proof}
\begin{corollary}\label{Noe ring pd id hd localization inequality}
For any multiplicative subset $S$ of $A$, we have
\[\projdim_{S^{-1}A}(S^{-1}M)\leq\projdim_A(M),\quad \injdim_{S^{-1}A}(S^{-1}N)\leq\injdim_A(N).\]
\end{corollary}
\begin{corollary}
If $\projdim_{A_\m}(M_\m)$ is finite for any maximal ideal $\m$ of $\supp(M)$, then we have $\projdim_A(M)<+\infty$.
\end{corollary}
\begin{proof}
In fact, the subspace $X$ of $\supp(M)$ formed by maximal ideals is quasi-compact (since $\Spec(A)$ is Noetherian); the map $\m\mapsto\projdim_{A_\m}(M_\m)$ from $X$ to $\widebar{\R}$ is upper semi-continuous, hence bounded.
\end{proof}
\subsection{Homological dimension of Noetherian rings}
Let $A$ be a Noetherian local ring and $M$ be a finitely generated $A$-module. Recall that a resolution
\[\begin{tikzcd}
\cdots\ar[r]&L_n\ar[r,"d_n"]&L_{n-1}\ar[r]&\cdots\ar[r]&L_0\ar[r,"d_0"]&M\ar[r]&0
\end{tikzcd}\]
of $M$ is called a \textbf{minimal projective resolution} if each module $L_i$ is free of finite rank, and if the complex $L\otimes_A\kappa_A$ has zero differential. For any integer $i\geq 0$, we then have (A, \Rmnum{10} p103, example 3)
\begin{align}\label{Noe ring minimal projective resolution Ext Tor dimension}
[\Ext_A^i(M,\kappa_A):\kappa_A]=[\Tor_i^A(M,\kappa_A):\kappa_A]=\rank_A(L_i)
\end{align}
By (A, \Rmnum{10} p.56, prop.10), any finitely generated $A$-module admits a minimal projective resolution.
\begin{proposition}\label{Noe local ring pd<n iff}
Let $A$ be a Noetherian local ring, $M$ be a finitely generated $A$-module and $n\geq 0$ be an integer. The following conditions are equivalent:
\begin{itemize}
\item[(\rmnum{1})] $\projdim_A(M)<n$; 
\item[(\rmnum{2})] $\Tor_n^A(M,\kappa_A)=0$;
\item[(\rmnum{3})] $\Ext_A^n(M,\kappa_A)=0$;
\item[(\rmnum{4})] any minimal projective resolution of $M$ has length $<n$. 
\end{itemize}
\end{proposition}
\begin{proof}
The assertions (\rmnum{1})$\Rightarrow$(\rmnum{2}) and (\rmnum{1})$\Rightarrow$(\rmnum{3}) are immediate. Let $L$ be a minimal projective resolution of $M$; if (\rmnum{2}) or (\rmnum{3}) are satisfied, we have $L_n=0$ by \ref{Noe ring minimal projective resolution Ext Tor dimension}. As any minimal projective resolution of $M$ is isomorphic to $L$ (A, \Rmnum{10}, p.54, prop.8), we then deduce (\rmnum{4}). The implication (\rmnum{4})$\Rightarrow$(\rmnum{1}) is trivial.
\end{proof}
\begin{corollary}\label{Noe local ring hd<n iff}
Let $A$ be a Noetherian local ring and $n\geq 0$ be an integer. The following conditions are equivalent:
\begin{itemize}
\item[(\rmnum{1})] $\gldim(A)<n$; 
\item[(\rmnum{2})] $\Ext_A^i(M,N)=0$ and $\Tor_i^A(M,N)=0$ for any couple $(M,N)$ of $A$-modules and any integer $i\geq n$;
\item[(\rmnum{3})] $\Tor_n^A(\kappa_A,\kappa_A)=0$;
\item[(\rmnum{4})] $\Ext_A^n(\kappa_A,\kappa_A)=0$;
\item[(\rmnum{5})] $\projdim_A(\kappa_A)<n$.
\end{itemize}
\end{corollary}
\begin{proof}
It is clear that (\rmnum{1}) implies (\rmnum{2}) and (\rmnum{2}) implies (\rmnum{3}) and (\rmnum{4}). By \cref{Noe local ring pd<n iff} applied to the $A$-module $\kappa_A$, conditions (\rmnum{3}) and (\rmnum{4}) both implies (\rmnum{5}). To see (\rmnum{5})$\Rightarrow$(\rmnum{1}), we note that if $\projdim_A(\kappa_A)<n$, then $\Tor_n^A(M,\kappa_A)=0$ for any $A$-module $M$; then any finitely generated $A$-module has projective dimension $<n$ (\cref{Noe local ring pd<n iff}), so $\gldim(A)<n$.
\end{proof}
\begin{corollary}\label{Noe local ring hd is pd of kappa_A}
For a Noetherian local ring, we have $\gldim(A)=\projdim_A(\kappa_A)$.
\end{corollary}
\begin{example}\label{Noe local ring hd=0 iff field}
Let $A$ be a local ring. Then the $A$-module $\Tor_1^A(\kappa_A,\kappa_A)$ is isomorphic to $\m_A/\m_A^2$ (A, \Rmnum{10}, p.72, example), so if $A$ is Noetherian, for that $\Tor_1^A(\kappa_A,\kappa_A)=0$, it is necessary and sufficient that $\m_A=0$, which means $A$ is a field. \cref{Noe local ring hd<n iff} then implies that a Noetherian local ring with homological dimension being zero is a field.
\end{example}
\begin{example}\label{Noe local ring pd is Ext^i_A nonzero}
Let $A$ be a Noetherian local ring, $M$ be a finitely generated $A$-module with finite projective dimension $n$, and $N$ be a nonzero finitely generated $A$-module. The $A$-module $\Ext_A^n(M,N)$ is then nonzero: let $L$ be a minimal projective resolution of $M$, and $d$ be its differential. We then have an exact sequence
\[\begin{tikzcd}
\Hom_A(L_{n-1},N)\ar[r,"{\Hom(d_n,1)}"]&\Hom_A(L_n,N)\ar[r]&\Ext_A^n(M,N)\ar[r]&0
\end{tikzcd}\]
As $d_n\otimes 1_{\kappa_A}$ is zero, we then deduce that tensoring with $\kappa_A$ induces an isomorphism $\kappa_A\otimes_A\Hom_A(L_n,N)\to\kappa_A\otimes_A\Ext_A^n(M,N)$, whence in view of the formula (\ref{Noe ring minimal projective resolution Ext Tor dimension}),
\[[\kappa_A\otimes_A\Ext_A^n(M,N):\kappa_A]=[\Ext_A^n(M,\kappa_A):\kappa_A][\kappa_A\otimes_AN:\kappa_A];\]
which is nonzero by \cref{Noe local ring pd<n iff} and Nakayama's lemma. Therefore the projective dimension of $M$ is the largest integer $i$ such that $\Ext_A^i(M,N)$ is nonzero.
\end{example}
\begin{example}
Let $A$ be a Noetherian ring, $M$ be a finitely generated $A$-module with finite projective dimension, $N$ be a finitely generated module with support equal to $\Spec(A)$. By \cref{Noe local ring pd is Ext^i_A nonzero} and \cref{Noe ring pd id hd localization prop}, the projective dimension $n$ of $M$ is the largest integer $i$ such that $\Ext_A^i(M,N)\neq 0$, and the support of the $A$-module $\Ext_A^n(M,N)$ is the set of elements $\p\in\Spec(A)$ such that $\projdim_{A_\p}(M_\p)=n$.
\end{example}
\begin{proposition}\label{Noe ring pd<n iff}
Let $A$ be a Noetherian ring, $M$ be a finitely generated $A$-module and $n\geq 0$ be an integer. Then the following conditions are equivalent:
\begin{itemize}
\item[(\rmnum{1})] $\projdim_A(M)<n$;
\item[(\rmnum{2})] for any maximal ideal $\m$ of $A$, we have $\Ext_A^n(M,A/\m)=0$ (resp. $\Tor_n^{A}(M,A/\m)=0$);
\item[(\rmnum{3})] for any maximal ideal $\m$ of $A$, we have $\Ext_{A_\m}^n(M_\m,A/\m)=0$ (resp. $\Tor_n^{A_\m}(M_\m,A/\m)=0$).
\end{itemize}
\end{proposition}
\begin{proof}
It is clear that (\rmnum{1})$\Rightarrow$(\rmnum{2}), and (\rmnum{2})$\Rightarrow$(\rmnum{1}) by \cref{Noe ring pd id hd localization prop}. Finally, by \cref{Noe local ring pd<n iff}, condition (\rmnum{3}) implies the inequality $\projdim_{A_\m}(M_\m)<n$ for any maximal ideal $\m$ of $A$, whence $\projdim_A(M)<n$ by \cref{Noe ring pd id hd localization prop}.
\end{proof}
\begin{remark}\label{Noe ring hd<n iff}
Let $A$ be a Noetheiran ring and $n\geq 0$ be an integer. If $\m$ and $\n$ are two distinct maximal ideals of $A$, the $A$-modules $\Ext_A^n(A/\m,A/\n)$ and $\Tor_n^A(A/\m,A/\n)$ are then annihilated by $\m+\n$, whence is zero. By an argument similar to \cref{Noe local ring hd<n iff}, we deduce from \cref{Noe ring pd<n iff} the equivalence of the following conditions:
\begin{itemize}
\item[(\rmnum{1})] $\gldim_A(M)<n$;
\item[(\rmnum{2})] $\Ext_A^i(M,N)=0$ and $\Tor_i^A(M,N)=0$ for any couple $(M,N)$ of $A$-modules and any integer $i\geq n$;
\item[(\rmnum{3})] $\Tor_n^A(A/\m,A/\m)=0$ for any maximal ideal $\m$ of $A$;
\item[(\rmnum{4})] $\Ext_A^n(A/\m,A/\m)=0$ for any maximal ideal $\m$ of $A$;
\item[(\rmnum{5})] $\projdim_A(A/\m)<n$ for any maximal ideal $\m$ of $A$.  
\end{itemize}
In particular, we have $\gldim(A)=\sup_\m\projdim_A(A/\m)$, where $\m$ runs through maximal ideals of $A$.
\end{remark}
\begin{proposition}\label{Noe ring id<n iff}
Let $A$ be a Noetherian ring, $N$ be an $A$-module, $n\geq 0$ be an integer. Then the following conditions are equivalent:
\begin{itemize}
\item[(\rmnum{1})] $\injdim_A(N)<n$;
\item[(\rmnum{2})] for any prime ideal $\p$ of $A$, we have $\Ext_A^n(A/\p,N)=0$;
\item[(\rmnum{3})] for any prime ideal $\p$ of $A$, we have $\Ext_{A_\p}^n(\kappa(\p),N_\p)=0$.
\end{itemize}
If the $A$-module $N$ is finitely generated, these conditions are equivalent to:
\begin{itemize}
\item[(\rmnum{4})] for any maximal ideal $\m$ of $A$, we have $\Ext_A^i(A/\m,N)=0$ for $n\leq i\leq n+\height(\m)$.
\end{itemize}
\end{proposition}
\begin{proof}

\end{proof}
\begin{remark}
Let $N$ be a finitely generated $A$-module; the condition $\Ext_A^n(A/\m,N)=0$ for any maximal ideal $\m$ of $A$ does not imply necessarily $\injdim_A(N)<n$. For example, if $A$ is local and not Gorenstein, we then have $\Ext_A^n(A/\m,A)=0$ for $n<\depth(A)$, but $\injdim_A(A)=+\infty$.
\end{remark}
\begin{proposition}\label{Noe ring dim leq id}
Let $A$ be a Noetherian ring, $M$ be a finitely generated $A$-module. Then we have $\dim_A(M)\leq\injdim_A(M)$.
\end{proposition}
\begin{proof}
Let $r\leq\dim_A(M)$ be a positive integer. Then there exists a saturated chain of prime ideals $\p\sub\p_1\sub\cdots\sub\p_{r-1}\sub\q$ such that $\p$ is a minimal element of $\supp(M)$; the $A_\p$-module $M_\p$ is then of finite length, so we have $\Hom_{A_\p}(\kappa(\p),M_\p)\neq 0$ (\cref{depth of module zero iff annihilated by element}), whence $\Ext_{A_\q}^r(\kappa(\q),M_\q)\neq 0$ (\cref{Noe ring prime chain Ext lemma}); this implies $\injdim_A(M)\geq r$ by \cref{Noe ring id<n iff}. 
\end{proof}
\begin{proposition}\label{Noe local ring id=depth(A)}
Let $A$ be a Noetherian local ring and $M$ be a finitely generated nonzero $A$-module with finite injective dimension. Then we have $\injdim_A(M)=\depth(A)$.
\end{proposition}
\begin{proof}
Put $r=\injdim_A(M)$, we then have $\Ext_A^i(\kappa_A,M)=0$ for $i>r$, so $\Ext_A^r(\kappa_A,M)\neq 0$ by \cref{Noe ring id<n iff}(\rmnum{4}). Let $s=\depth(A)$ and $(x_1,\dots,x_s)$ be an $A$-regular sequence of element sof $\m_A$; put $N=A/(x_1A+\cdots+x_sA)$. By \cref{Koszul complex is free resolution for regular sequence}, we have $\projdim_A(N)=s$ and $\Ext_A^s(N,M)\neq 0$, so $s\leq\injdim_A(M)=r$. But $N$ is of zero depth (\cref{depth of module quotient by regular sequence}), so there exists an exact sequence of $A$-modules
\[\begin{tikzcd}
0\ar[r]&\kappa_A\ar[r]&N\ar[r]&N'\ar[r]&0
\end{tikzcd}\]
from which we deduce an exact sequence of extension modules
\[\begin{tikzcd}
\Ext_A^r(N,M)\ar[r]&\Ext_A^r(\kappa_A,M)\ar[r]&\Ext_A^{r+1}(N',M)
\end{tikzcd}\]
As we have $\Ext_A^{r+1}(N',M)=0$ and $\Ext_A^r(\kappa_A,M)\neq 0$, we obtain that $\Ext_A^r(N,M)\neq 0$, so $r\leq\projdim_A(N)=s$; we then conclude that $r=s$, whence the proposition.
\end{proof}
\subsection{Depth and projective dimension}
\begin{theorem}[\textbf{Auslander-Buchsbaum}]\label{Noe local ring Auslander-Buchsbaum formula}
Let $A$ be a Noetherian local ring and $M$ be a finitely generated $A$-module with finite projective dimension. Then we have
\[\projdim_A(M)+\depth_A(M)=\depth(A).\]
\end{theorem}
\begin{proof}
We prove by induction on $\projdim_A(M)$. If $\projdim_A(M)=0$, then $M$ is free of finite rank, so $\depth_A(M)=\depth(A)$ by \cref{depth of module product is inf}. Now suppose that $\depth_A(M)=1$ and choose a minimal projective resolution of $M$:
\[\begin{tikzcd}
0\ar[r]&L_1\ar[r,"d_1"]&L_0\ar[r,"d_0"]&M\ar[r]&0
\end{tikzcd}\] 
The $A$-modules $L_0$ and $L_1$ are free of finite rank and nonzero, so their depths are equal to $\depth(A)$ (\cref{depth of module product is inf}). The map $1_{\kappa_A}\otimes d_1:\kappa_A\otimes_AL_1\to\kappa_A\otimes_AL_0$ is zero, so $d_1$ belongs to $\m_A\Hom_A(L_1,L_0)$. By \cref{depth of module exact sequence annihilated by A/I prop}, we then have $\depth_A(M)=\depth(A)-1$.\par
Finally, suppose that $\projdim_A(M)>1$. We choose an exact sequence
\[\begin{tikzcd}
0\ar[r]&N\ar[r]&L\ar[r]&M\ar[r]&0
\end{tikzcd}\]
where $L$ is a free $A$-module of finite rank. By \cref{depth of module product is inf}, we then have $\depth_A(L)=\depth(A)$, and $\projdim_A(N)=\projdim_A(M)-1$ in view of (A, \Rmnum{10}, p.135, cor.2(c)); the induction hypotheses implies that $\depth_A(N)=\depth(A)-\projdim_A(N)$, and in particular $\depth_A(N)<\depth_A(L)$. Now \cref{depth of module exact sequence prop} implies that $\depth(M)=\depth_A(N)-1$, which proves the assertion.
\end{proof}
\begin{remark}
In view of \cref{Noe local ring hd is pd of kappa_A}, the fomula \cref{Noe local ring Auslander-Buchsbaum formula} applied to the $A$-module $\kappa_A$ implies that we are in exactly one of the following two cases:
\begin{itemize}
\item[(\rmnum{1})] $\projdim_A(\kappa_A)=\gldim(A)=+\infty$;
\item[(\rmnum{2})] $\projdim_A(\kappa_A)=\gldim(A)=\depth(A)<+\infty$.
\end{itemize}
Later we shall see that case (\rmnum{2}) characterizes regular local rings.
\end{remark}
\begin{corollary}\label{Noe local ring pd smaller than depth}
Retain the hypotheses of \ref{Noe local ring Auslander-Buchsbaum formula}.
\begin{itemize}
\item[(a)] We have $\projdim_A(M)\leq\depth(A)$, and for the equality holds, it is necessary and sufficient that $\m_A\in\Ass(M)$.
\item[(b)] We have $\depth_A(M)\leq\depth(A)$, and for the equality holds, it is necessary and sufficient that $M$ is free.
\end{itemize}
\end{corollary}
\begin{proof}
This follows from the fact that $\depth_A(M)=0$ if and only if $\m_A\in\Ass(A)$ by \cref{depth of module zero iff annihilated by element}, and $\projdim_A(M)=0$ if and only if $M$ is projective, hence free.
\end{proof}
\begin{corollary}\label{Noe local ring CM pd is dim-dim + dim-depth}
Retain the hypotheses of \ref{Noe local ring Auslander-Buchsbaum formula} and suppose that $A$ is a Cohen-Macaulay ring. Then $\projdim_A(M)$ is the sum of the positive integers $\dim(A)-\dim_A(M)$ and $\dim_A(M)-\depth(M)$. 
\end{corollary}
In particular, we have $\projdim_A(M)\geq\dim(A)-\dim_A(M)$, and the equality holds if and only if $M$ is Cohen-Macaulay.
\begin{corollary}\label{Noe ring depth along supp of Ext prop}
Let $A$ be a Noetherian ring, $M$ be a finitely generated $A$-module with finite projective dimension, $N$ be a finitely generated $A$-module, $i\geq 0$ be an integer, and $F$ be the support of $\Ext_A^i(M,N)$ (resp. $\Tor_i^A(M,N)$). Then we have $\depth_F(A)\geq i$.
\end{corollary}
\begin{proof}
For $\p\in F$ we have $\Ext^i_{A_\p}(M_\p,N_\p)\neq 0$ (resp. $\Tor_i^{A_\p}(M_\p,N_\p)\neq 0$) by \cref{module Tor and Ext localization prop}, so $i\leq\projdim_{A_\p}(M_\p)\leq\projdim_A(M)<+\infty$ in view of \cref{Noe ring pd id hd localization prop}. The formula \cref{Noe local ring Auslander-Buchsbaum formula} then implies that $\depth(A_\p)\geq i$, and therefore $\depth_F(A)=\inf_{\p\in F}\depth(A_\p)\geq i$ (\cref{depth of module along closed localization char}).
\end{proof}
With the terminologies of \cref{Noe ring module grade def}, the conclusion of \cref{Noe ring depth along supp of Ext prop} signifies that the modules $\Ext_A^i(M,N)$ and $\Tor_i^A(M,N)$ are of grade $\geq i$. This then implies that the codimension of their support in $\Spec(A)$ is $\geq i$ (\cref{depth of module local Noe inequality}).
\begin{corollary}\label{Noe CM ring dim-depth semi-continous on Spec(A)}
Let $A$ be a Noetherian Cohen-Macaulay ring and $M$ be a finitely generated $A$-module with finite projective dimension.
\begin{itemize}
\item[(a)] For $\p\in\Spec(A)$, denote by $\mathscr{C}(\p)$ the set of irreducible components of $\supp(M)$ containing $\p$. Then we have
\[\dim_{A_\p}(M_\p)-\depth_{A_\p}(M_\p)=\projdim_{A_\p}(M_\p)-\inf_{X\in\mathcal{C}(\p)}\codim(X,\Spec(A)).\]
\item[(b)] The function $\p\mapsto\dim_{A_\p}(M_\p)-\depth_{A_\p}(M_\p)$ on $\Spec(A)$ is upper semi-continuous.
\item[(c)] The set of prime ideals $\p$ of $A$ such that the $A_\p$-module $M_\p$ is Cohen-Macaulay is open and dense in $\Spec(A)$. Its intersection with $\supp(M)$ is dense in $\supp(M)$.
\end{itemize}
\end{corollary}
\subsection{Gorenstein rings}
We say that a ring $A$ is \textbf{Gorenstein} if it is Noetherian and the $A_\m$-module $A_\m$ is of finite injective dimension for any maximal ideal $\m$ of $A$. For a Noetherian local ring to be Gorenstein, it is then necessary and sufficient that $\injdim_A(A)<+\infty$.
\begin{proposition}\label{Gorenstein ring is CM and id=dim}
A Gorenstein ring $A$ is Cohen-Macaulay and we have $\injdim_A(A)=\dim(A)$.
\end{proposition}
\begin{proof}
For any maximal ideal $\m$ of $A$, we have (\cref{Noe ring dim leq id}, \cref{Noe local ring id=depth(A)} and \cref{depth of module local Noe depth leq dim})
\[\dim(A_\m)\leq\injdim_{A_\m}(A_\m),\quad \injdim_{A_\m}(A_\m)=\depth(A_\m),\quad \depth(A_\m)\leq\dim(A_\m).\]
It then follows that $A$ is Cohen-Macaulay, and we have $\injdim_A(A)=\dim(A)$ by passing to supremum (\cref{Noe ring pd id hd localization prop}).
\end{proof}
Thus the Noetherian rings $A$ such that $\injdim_A(A)$ is finite are finite-dimensional Gorenstein rings (\cref{Noe ring pd id hd localization prop}), and the Noetherian rings such that the $A$-module $A$ is injective are the Artinian Gorenstein rings.
\begin{example}\label{Gorenstein ring localization is Gorenstein}
For any multiplicative subset $S$ of a Gorenstein ring $A$, the fraction ring $S^{-1}A$ is Gorenstein: in fact, any maximal ideal of $S^{-1}A$ is of the form $S^{-1}\p$, where $\p$ is a prime ideal of $A$ not meeting $S$. Let $\m$ be a maximal ideal of $A$ containing $\p$, then the ring $B=(S^{-1}A)_{S^{-1}\p}$ is isomorphic to $A_\p$, hence is a fraction ring of $A_\m$, and therefore satisfies $\injdim_B(B)<+\infty$ (\cref{Noe ring pd id hd localization inequality}).
\end{example}
\begin{example}\label{Gorenstein ring quotient by regular sequence}
Let $A$ be a Gorenstein ring and $\mathfrak{I}$ be an ideal of $A$, generated by an $A$-regular sequence $\bm{x}$. The quotient ring $A/\mathfrak{I}$ is a Gorenstein ring: for any maximal ideal of $A$ containing $\mathfrak{I}$, the image of $\bm{x}$ in $A_\m$ is $A_\m$-regular and generates the ideal $\mathfrak{I}_\m$, so $A_\m/\mathfrak{I}_\m$ is Gorenstein by (CA, \Rmnum{10}, cor. de la prop.7 n4). On the other hand, if $A$ is a Noetherian local ring and $\mathfrak{I}$ is an ideal generated by an $A$-regular sequence of elements of $\m_A$, then $A$ is Gorenstein if $A/\mathfrak{I}$ is Gorenstein.
\end{example}
\begin{example}\label{regular local ring is Gorenstein}
Let $A$ be a regular local ring, then $A$ is Gorenstein. In fact, let $\bm{x}$ be a system of parameters of $A$. Then $\bm{x}$ is $A$-regular and generates the ideal $\m_A$ (\cref{Noe local ring regular iff m_A generated by completely secant}), so we can apply \cref{Gorenstein ring quotient by regular sequence}.
\end{example}
\section{Regular rings}
\subsection{Homological properties of regular local rings}
\begin{proposition}\label{Noe regular local ring hd is dim}
Let $A$ be a Noetherian regular local ring of dimension $n$. Then $\gldim(A)=n$, and for any integer $i\geq 0$,
\[[\Ext_A^i(\kappa_A,\kappa_A):\kappa_A]=[\Tor_i^A(\kappa_A,\kappa_A):\kappa_A]=\binom{n}{i}\]
\end{proposition}
\begin{proof}
Let $\bm{x}=(x_1,\dots,x_n)$ be a system of parameters of $A$. The sequence $\bm{x}$ generates $\m_A$ and is completely secant for $A$ (\cref{Noe local ring regular iff m_A generated by completely secant}). The Koszul complex $K_\bullet(\bm{x},A)$ is then a free resolution of $\kappa_A$, whose differentials are zero modulo $\m_A$. For any intger $i\geq 0$, we have (formula (\ref{Noe ring minimal projective resolution Ext Tor dimension}))
\[[\Ext_A^i(\kappa_A,\kappa_A):\kappa_A]=[\Tor_i^A(\kappa_A,\kappa_A):\kappa_A]=\rank_A(K_i(\bm{x},A))=\binom{n}{i}.\]
I then follows from \cref{Noe local ring hd<n iff} that we have $\projdim_A(\kappa_A)\geq n$, whence $\projdim_A(\kappa_A)=n$ and $\gldim(A)=n$ by \cref{Noe local ring hd is pd of kappa_A}.
\end{proof}
\begin{proposition}\label{Noe regular local is UFD}
A Noetherian regular local ring is factorial.
\end{proposition}
\begin{proof}
By \cref{Noe regular local ring hd is dim}, the minimal projective resolution of a finitely generated module over a Noetherian regular local ring has finite length. It then follows from \cref{Krull Noe finite free resolution then principal} that this ring is factorial.
\end{proof}
\begin{proposition}\label{Noe regular local fg module pd finite}
Let $A$ be a Noetherian regular local ring and $M$ be a finitely generated nonzero $A$-module. Then the projective dimension of $M$ is finite and we have
\[\projdim_A(M)+\depth_A(M)=\dim(A).\]
\end{proposition}
\begin{proof}
In fact, $M$ has finite projective dimension by \cref{Noe regular local ring hd is dim}, and we have $\depth(A)=\dim(A)$ since $A$ is Cohen-Macaulay (\cref{CM ring example}). We can therefore apply the formula \cref{Noe local ring Auslander-Buchsbaum formula}.
\end{proof}
\begin{corollary}\label{Noe regular local pd>dim-dim equality iff CM}
We have $\projdim_A(M)\geq\dim(A)-\dim(M)$, and the equality holds if and only if $M$ is Cohen-Macaulay.
\end{corollary}
\begin{corollary}\label{Noe regular local fg module free iff CM dim(A)}
For the $A$-module $M$ to be free, it is necessary and sufficient it is Cohen-Macaulay and of dimension $\dim(A)$, or that it has depth $\geq\dim(A)$.
\end{corollary}
\begin{corollary}\label{Noe regular local fg reflexive module free}
Any finitely generated reflexive module over a Noetherian regular local ring of dimension $2$ is free.
\end{corollary}
\begin{proof}
In fact, a Noetherian regular local ring is integrally closed (\cref{regular local ring integrally closed}). The corollary then follows from \cref{Noe regular local fg module free iff CM dim(A)} and \cref{Noe integral closed depth of reflexive geq 2}.
\end{proof}
\begin{corollary}\label{Noe local ring local finite base change CM iff pd=dim-dim}
Let $\rho:A\to B$ be a local homomorphism of Noetherian local rings. Suppose that $A$ is regular and that $B$ is a finitely generated $A$-module. Then we have $\projdim_A(B)\geq\dim(A)-\dim(B)$, and the equality holds if and only if $B$ is Cohen-Macaulay. In particular, for the ring $B$ to be Cohen-Macaulay with dimension equal to $\dim(A)$, it is necessary and sufficient that the $A$-module $B$ is free.
\end{corollary}
\begin{proof}
In fact, we have $\dim(B)=\dim_A(B)$ by \cref{integral extension dimension of module ideal prop}, so $B$ is Cohen-Macaulay if and only if it is a Cohen-Macaulay $A$-module (\cref{CM module finite base change CM iff}). It then suffices to apply \cref{Noe regular local pd>dim-dim equality iff CM} and \cref{Noe regular local fg module free iff CM dim(A)}.
\end{proof}
\subsection{Homological characterization for regularity}
\begin{theorem}[\textbf{Serre}]\label{Noe local regular iff finite gldim}
For a Noetherian local ring to be regular, it is necessary and sufficient that it has finite homological dimension.
\end{theorem}
\begin{proof}
We have seen that a Noetherian regular local ring has finite homological dimension. Conversely, let $A$ be a Noetherian local ring with finite homological dimension $n$. By \cref{Noe local ring hd is pd of kappa_A} and \cref{Noe local ring Auslander-Buchsbaum formula}, we have
\[n=\gldim(A)=\projdim_A(\kappa_A)=\depth(A).\]
If $n=0$, the $A$-module $\kappa_A$ is free, so $\m_A=0$ and $A$ is a field. Suppose that $n>0$ and we preceed by induction on $n$. Since $\depth(A)>0$, the ideal $\m_A$ is not assocaited with $A$ (\cref{depth of module zero iff annihilated by element}), hence is not contained in the union of $\m_A^2$ with the associated primes of $A$ (\cref{prime ideal contained in union}). By \cref{associated prime and homothety injective}, we can therefore choose an element $x$ of $\m_A-\m_A^2$ such that the homothety $x_A$ is injective. Let $B$ be the Noetherian local ring $A/xA$ and consider the exact sequence
\[\begin{tikzcd}
0\ar[r]&\kappa_A\ar[r,"i"]&\m_A/x\m_A\ar[r,"p"]&\m_B\ar[r]&0
\end{tikzcd}\]
where $i$ is the map induced by the homothety $x_A$ on $\m_A$ and $p$ is the canonical surjection. Since the class of $x$ in the $\kappa_A$-vector space $\m_A/\m_A^2$ is nonzero, there exists an $A$-linear map $\phi:\m_A\to\kappa_A$ with $\phi(x)=1$; by passing to quotient, we then from $\phi$ a retraction of $i$, such that the preceding sequence splits. By (CA, \Rmnum{10}, cor.2 de la prop.7 du $\S$3, n4) and (A, \Rmnum{10}, p.135, cor.1), this implies the relations
\[\projdim_B(\m_B)\leq\projdim_B(\m_A/x\m_A)=\projdim_A(\m_A)<+\infty.\]
(CA, \Rmnum{10}, cor.2 de la prop.7 du $\S$3, n4) applied to the exact sequence $0\to\m_B\to B\to\kappa_B\to 0$ of $B$-modules then implies that $\projdim_B(\kappa_B)<+\infty$. The ring $B$ is then of finite homological dimension (\cref{Noe local ring hd is pd of kappa_A}), and of depth $n-1$ (\cref{depth of module quotient by regular sequence}). It then follows from the induction hypotheses that $B$ is regular, so $A$ is regular by \cref{Noe local ring quotient by element regular iff}.
\end{proof}
Therefore, if $A$ is a Noetherian local ring, the following conditions are equivalent:
\begin{itemize}
\item[(\rmnum{1})] $A$ is regular;
\item[(\rmnum{2})] the $A$-module $\kappa_A$ has finite projective dimension;
\item[(\rmnum{3})] any finitely generated $A$-module has finite projective dimension.
\end{itemize}
Now we say that a ring $A$ is \textbf{regular} if it is Noetherian and the local ring $A_\m$ is regular for any maximal ideal $\m$ of $A$. From the above equivalence, it is not hard to derive the following characterization for regular rings.
\begin{proposition}\label{Noe ring regular iff pd finite}
Let $A$ be a Noetherian ring. The following conditions are equivalent:
\begin{itemize}
\item[(\rmnum{1})] $A$ is regular;
\item[(\rmnum{2})] any finitely generated $A$-module has finite projective dimension;
\item[(\rmnum{3})] for any maximal ideal $\m$ of $A$, the projective dimension of $A/\m$ is finite;
\item[(\rmnum{4})] for any prime ideal $\p$ of $A$, the local ring $A_\p$ is regular.
\end{itemize}
\end{proposition}
\begin{proof}
Let $\p$ be a prime ideal of $A$; if the $A$-module $A/\p$ has finite projective dimension, so does the $A_\p$-module $\kappa(\p)$ (\cref{Noe ring pd id hd localization inequality}), so the local ring $A_\p$ is regular by {Noe local regular iff finite gldim}. We then reduce that (\rmnum{2})$\Rightarrow$(\rmnum{4}) and (\rmnum{3})$\Rightarrow$(\rmnum{1}). The implications (\rmnum{4})$\Rightarrow$(\rmnum{1}) and (\rmnum{2})$\Rightarrow$(\rmnum{3}) are clear.\par
To see that (\rmnum{1})$\Rightarrow$(\rmnum{2}), let $M$ be a fintely generated $A$-module. Under the hypotheses of (\rmnum{1}), we have $\projdim_{A_\m}(M_\m)\leq\gldim(A_\m)<+\infty$ for any maximal ideal $\m$ of $A$ (\cref{Noe regular local ring hd is dim}), so $M$ has finite projective dimension (\cref{Noe ring pd id hd localization prop}), whence (\rmnum{2}).
\end{proof}
\begin{example}
If a ring $A$ is regular, then so is $S^{-1}A$ for any multiplicative subset $S$ of $A$: this follows for example by the characterization (\rmnum{3}) of \cref{Noe ring regular iff pd finite}.
\end{example}
\begin{example}
For a ring to be regular, it is necessary and sufficient that it is isomorphic to the product of a finite family of regular integral domains. This follows from the fact that any regular ring is locally integral, since regular local rings are integral.
\end{example}
\begin{corollary}\label{Noe ring gldim finite iff regular with finite dim}
Let $A$ be a Noetherian ring. Then the following conditions are equivalent:
\begin{itemize}
\item[(\rmnum{1})] $\gldim(A)<+\infty$;
\item[(\rmnum{2})] $A$ is regular and $\dim(A)<+\infty$.
\end{itemize}
If these conditions are satisfied, we have $\dim(A)=\gldim(A)$.
\end{corollary}
\begin{proof}
If $A$ is regular, for any maximal ideal $\m$ of $A$ we have $\dim(A_\m)=\gldim(A_\m)$ (\cref{Noe regular local ring hd is dim}), and then (\cref{dimension of ring prop} and \cref{Noe ring pd id hd localization prop})
\[\gldim(A)=\sup_\m\gldim(A_\m)=\sup_\m\dim(A_\m)=\dim(A).\]
On the other hand, if $\gldim(A)<+\infty$, the ring $A$ is regular by \cref{Noe ring regular iff pd finite}, so the corollary follows. 
\end{proof}
\begin{corollary}\label{Noe regular is Gorenstein normal}
A regular ring is normal, Gorenstein and Cohen-Macaulay.
\end{corollary}
\begin{proof}
In fact, a regular local ring is integrally closed (\cref{regular local ring integrally closed}), Gorenstein (\cref{regular local ring is Gorenstein}), and Cohen-Macaulay (\cref{CM ring example}(f)).
\end{proof}
\begin{corollary}\label{Noe ring I-adic completion regular iff}
Let $A$ be a Noetherian ring, $\mathfrak{I}$ be an ideal of $A$ and $\widehat{A}$ be the $\mathfrak{I}$-adic completion of $A$.
\begin{itemize}
\item[(a)] For the ring $\widehat{A}$ to be regular, it is necessary and sufficient that, for any maximal ideal $\m$ of $A$ containing $\mathfrak{I}$, the ring $A_\m$ is regular.
\item[(b)] If the ring $A$ is regular, then $\widehat{A}$ is regular. If the ring $\widehat{A}$ is regular and $\mathfrak{I}$ is contained in the Jacobson radical of $A$, then $A$ is regular.
\end{itemize}
\end{corollary}
\begin{proof}
By \cref{filtration I-adic completion maximal ideal}, for the ring $\widehat{A}$ to be regular, it is necessary and sufficient that $\widehat{A}_{\hat{\m}}$ is regular for any maximal ideal $\m$ of $A$ containing $\mathfrak{I}$. As the local rings $\widehat{A}_{\hat{\m}}$ and $A_\m$ are isomorphic, assertion of (a) then follows from \cref{Noe local ring regular iff completion is}, and (b) follows from (a).
\end{proof}
\begin{corollary}\label{Noe regular ring symmetric algebra of projective is regular}
Let $A$ be a regular ring and $P$ be a finitely generated projective $A$-module. Then the symmetric algebra $\bm{S}_A(P)$ is regular.
\end{corollary}
\begin{proof}
Let $\mathfrak{P}$ be a prime ideal of $\bm{S}_A(P)$ and $\p$ be its contraction in $A$. The local ring $\bm{S}_A(P)_{\mathfrak{P}}$ is then a fraction field of $\bm{S}_A(P)_\p$, which is isomorphic to $\bm{S}_{A_\p}(P)_\p$ by (A, \Rmnum{3}, p.72, prop.7); it then suffices to prove that the later is regular. This allow us to reduce to the case where $A$ is local, and then $P$ is free of finite rank. By \cref{Noe regular local ring hd is dim} and (A, \Rmnum{10}, p.143, cor.1), we have
\[\gldim(\bm{S}_A(P))=\gldim(A)+\rank_A(P)<+\infty\]
and $\bm{S}_A(P)$ is therefore regular by \cref{Noe ring regular iff pd finite}.
\end{proof}
\begin{corollary}\label{Noe regular ring symmetric algebra is regular}
Let $A$ be a regular ring and $(T_i)_{i\in I}$ be a finite family of indeterminates. Then rings $A[(T_i)_{i\in I}]$ and $A\llbracket(T_i)_{i\in I}\rrbracket$ are regular.
\end{corollary}
\begin{proof}
This follows from \cref{Noe regular ring symmetric algebra is regular} and \cref{Noe ring I-adic completion regular iff}(b).
\end{proof}
\begin{proposition}\label{Noe regular CM module lying over is projective}
Let $\rho:A\to B$ be a homomorphism of Noetherian rings and $N$ be a $B$-module. Suppose that
\begin{itemize}
\item[(a)] the ring $A$ is regular,
\item[(b)] $N$ is a finitely generated $A$-module,
\item[(c)] the $B$-module $N$ is Cohen-Macaulay,
\item[(d)] any minimal prime ideal of $\supp_B(N)$ is lying over a minimal prime ideal of $A$.
\end{itemize}
Then $N$ is a (finitely generated) projective $A$-module.
\end{proposition}
\begin{proof}
It suffices to prove that, for any maximal ideal $\m$ of $A$, the $A_\m$-module $N_\m$ is free (\cref{module finite projective iff}). The $A$-module $B/\Ann_B(N)$ is isomorphic to a sub-$A$-module of the finitely generated $A$-module $\End_A(N)$, hence is finitely generated. By replacing $B$ by $B/\Ann_B(N)$, we may then reduce to the case where $B$ is a finite $A$-algebra and where $\supp_B(N)=\Spec(B)$.\par
Let $\m$ be a maximal ideal of $\supp_A(N)$, and put $n=\dim(A_\m)$. By \cref{Noe regular local fg module free iff CM dim(A)}, it suffices to prove that $N_\m$ is a Cohen-Macaulay $A_\m$-module of dimension $n$. By \cref{integral ring maximal ideal iff contraction is}, any maximal ideal of $B_\m$ is of the form $\mathfrak{N}B_\m$, where $\mathfrak{N}$ is a maximal ideal of $B$ lying over $\m$. Let $\mathfrak{P}$ be a minimal prime ideal of $\supp_B(N)$ which is contained in $\mathfrak{N}$. The closed subset $V(\mathfrak{P}B_\mathfrak{N})$ of $\supp_{B_\mathfrak{N}}(N_\mathfrak{N})$ then has zero codimension, and since the $B_\mathfrak{N}$-module $N_\mathfrak{N}$ is Cohen-Macaulay, \cref{CM module over local Noe supp equi-dimension} implies that $\dim_{B_\mathfrak{N}}(N_\mathfrak{N})=\dim(B_\mathfrak{N}/\mathfrak{P}B_\mathfrak{N})$. But $\mathfrak{P}$ is lying over a minimal prime of $A$ which is contained in $\m$, so $\mathfrak{P}B_\mathfrak{N}$ is lying over a minimal prime ideal of $A_\m$, which is zero since the local ring $A_\m$ is regular (hence integral). The canonical map $A_\m\to B_\mathfrak{N}/\mathfrak{P}B_\mathfrak{N}$ is then injective and it follows from \cref{integral extension dimension of module ideal prop} that we have
\[\dim_{B_\mathfrak{N}}(N_\mathfrak{N})=\dim(B_\mathfrak{N}/\mathfrak{P}B_\mathfrak{N})=\dim(A_\m)=n.\]
In view of this, \cref{CM module finite base change CM iff} then implies that the $A_\m$-module $N_\m$ is Cohen-Macaulay, and the proposition then follows from the relations $\dim_{A_\m}(N_\m)=\dim_{B_\m}(N_\m)$ (\cref{integral extension dimension of module ideal prop}) and $\dim_{B_\m}(N_\m)\geq\dim_{B_\mathfrak{N}}(N_\mathfrak{N})=n$.
\end{proof}
\begin{corollary}\label{Noe integral CM over regular iff projective module}
Let $B$ be an integral Noetherian ring and $A$ be a subring of $B$ which is regular. Suppose that $B$ is a finie $A$-algebra, then for the ring $B$ to be Cohen-Macaulay, it is necessary and sufficient that the $A$-module $B$ is projective.
\end{corollary}
\begin{proof}
If the ring $B$ is Cohen-Macaulay, the $A$-module $B$ is projective by \cref{Noe regular CM module lying over is projective}. Conversely, assume that the $A$-module $B$ is projective. We know that the $A$-module $A$ is Cohen-Macaulay (\cref{Noe regular is Gorenstein normal}), and the $A$-module $B$ is a direct factor of a free $A$-module of finite rank, hence Cohen-Macaulay (\cref{CM module direct factor is CM}), so we can apply \cref{CM module finite base change CM iff}.
\end{proof}
\begin{example}\label{Noe integral algebra over field CM iff projective}
Let $B$ be an integral algebra of finite type over a field $k$. By the normalization lemma (\cref{algebra finite over field normalization}), there exists a subring $A$ of $B$ which is isomorphic to a polynomial algebra and such that $B$ is finite over $A$. By \cref{Noe integral CM over regular iff projective module}, we see that $B$ is a Cohen-Macaulay ring if and only if the $A$-module $B$ is projective (hence free). 
\end{example}
\subsection{Flat base change}
\begin{proposition}\label{Noe regular faithfully flat descent}
Let $\rho:A\to B$ be a homomorphism of Noetherian rings such that $B$ is a faithfully flat $A$-module.
\begin{itemize}
\item[(a)] For any finitely generated $A$-module $M$, we have $\projdim_A(M)=\projdim_B(B\otimes_AM)$.
\item[(b)] If the ring $B$ is regular, so is $A$.
\end{itemize}
\end{proposition}
\begin{proof}
The $B$-module $B\otimes_AM$ is finitely generated, and for it to be nonzero, it is necessary and sufficient that $M$ is nonzero (\cref{module faithfully flat iff}). For it to be projective, it is necessary and sufficient that $M$ is a projective $A$-module (\cref{module extension to faithfully flat finite projective iff}). This proves (a) if $\projdim_A(M)\leq 0$, so suppose that $\projdim_A(M)\geq 1$ (whence $\projdim_B(B\otimes_AM)\geq 1$) and we proceed by induction on $\projdim_A(M)$. Choose an exact sequence of $A$-modules
\[\begin{tikzcd}
0\ar[r]&N\ar[r]&L\ar[r]&M\ar[r]&0
\end{tikzcd}\]
where $L$ is a free $A$-module of finite rank. We have $\projdim_A(N)=\projdim_A(M)-1$ (A, \Rmnum{10} p.135, cor.2 de la prop.1). As $B$ is flat over $A$, the sequence
\[\begin{tikzcd}
0\ar[r]&B\otimes_AN\ar[r]&B\otimes_AL\ar[r]&B\otimes_AM\ar[r]&0
\end{tikzcd}\]
is exact and we have $\projdim_B(B\otimes_AN)=\projdim_B(B\otimes_AM)-1$. Assertion (a) then follows from the induction hypotheses applied to $N$, and (b) follows from (a) and \cref{Noe ring regular iff pd finite}.
\end{proof}
\begin{corollary}\label{Noe subring of regular domain regular iff}
Let $B$ be a regular domain and $A$ be a Noetherian subring of $B$ such that $B$ is finite over $A$. Then the following conditions are equivalent:
\begin{itemize}
\item[(\rmnum{1})] $A$ is regular;
\item[(\rmnum{2})] $B$ is a projective $A$-module;
\item[(\rmnum{3})] $B$ is a flat $A$-module;
\item[(\rmnum{4})] $B$ is a faithfully flat $A$-module.
\end{itemize}
\end{corollary}
\begin{proof}
The implication (\rmnum{1})$\Rightarrow$(\rmnum{2}) follows from \cref{Noe integral CM over regular iff projective module}, and (\rmnum{2})$\Rightarrow$(\rmnum{3}) is clear. For any prime ideal $\p$ of $A$, we have $\p^e\neq B$ (\cref{integral ring lying over prime exist}), so it suffices to apply (\cref{ring faithfully flat iff}) to see that (\rmnum{3})$\Rightarrow$(\rmnum{4}). Finally, (\rmnum{4}) implies (\rmnum{1}) by \cref{Noe regular faithfully flat descent}(b).
\end{proof}
For any Noetherian local ring $A$, we denote by $\delta(A)$ the integer
\[\delta(A)=[\m_A/\m_A^2:\kappa_A]-\dim(A).\]
Recall that $\delta(A)$ is always positive, and its vanishing characterize regular local rings. Now let $\rho:A\to B$ be a local homomorphism of Noetherian local rings; we then have a $\kappa_A$-linear homomorphism $\m_A/\m_A^2\to\m_B/\m_B^2$, whence a $\kappa_B$-linear homomorphism
\[d\rho:\kappa_B\otimes_{\kappa_A}(\m_A/\m_A^2)\to\m_B/\m_B^2.\]
\begin{lemma}\label{Noe local ring embdim local homomorphism prop}
Under the above hypotheses, we have
\[\delta(B)+[\ker d\rho:\kappa_B]=\delta_A+\delta(\kappa_A\otimes_AB)+(\dim(A)-\dim(B)+\dim(\kappa_A\otimes_AB)).\]
\end{lemma}
\begin{proof}
Let $\widebar{B}$ be the local ring $\kappa_A\otimes_AB$. Since $\m_{\widebar{B}}=\m_B/\m_AB$, we have the exact sequence of $B$-modules
\[\begin{tikzcd}
B\otimes_A\m_A\ar[r]&\m_B\ar[r]&\m_{\widebar{B}}\ar[r]&0
\end{tikzcd}\]
By tensoring with $\kappa_B$ over $B$, we then obtain an exact sequence of $\kappa_B$-vector spaces
\begin{equation}\label{Noe local ring embdim local homomorphism prop-1}
\begin{tikzcd}
\kappa_B\otimes_{\kappa_A}(\m_A/\m_A^2)\ar[r,"d\rho"]&\m_B/\m_B^2\ar[r]&\m_{\widebar{B}}/\m_{\widebar{B}}^2\ar[r]&0
\end{tikzcd}
\end{equation}
where we use the equality
\[\kappa_B\otimes_BB\otimes_A\m_A=\kappa_B\otimes_A\m_A=\kappa_B\otimes_{\kappa_A}\kappa_A\otimes_A\m_A=\kappa_B\otimes_{\kappa_A}(\m_A/\m_A^2).\]
Since $[\m_{\widebar{B}}/\m_{\widebar{B}}^2:\kappa_{B}]=[\m_{\widebar{B}}/\m_{\widebar{B}}^2:\kappa_{\widebar{B}}]$ and $[\kappa_B\otimes_{\kappa_A}(\m_A/\m_A^2):\kappa_B]=[\m_A/\m_A^2:\kappa_A]$, from (\ref{Noe local ring embdim local homomorphism prop-1}) we then deduce the equality
\[[\ker d\rho:\kappa_B]+[\m_B/\m_B^2:\kappa_B]=[\m_A/\m_A^2:\kappa_A]+[\m_{\widebar{B}}/\m_{\widebar{B}}^2:\kappa_{\widebar{B}}],\]
which proves the lemma.
\end{proof}
\begin{proposition}\label{Noe local ring local homomorphism submersion iff}
Let $\rho:A\to B$ be a local homomorphism of Noetherian local rings. The following conditions are equivalent:
\begin{itemize}
\item[(\rmnum{1})] $B$ is regular and the $\kappa_B$-linear map $d\rho:\kappa_B\otimes_{\kappa_A}(\m_A/\m_A^2)\to\m_B/\m_B^2$ is injective;
\item[(\rmnum{2})] $B$ and $\kappa_A\otimes_AB$ are regular and the $A$-module $B$ is flat; 
\item[(\rmnum{3})] $A$ and $\kappa_A\otimes_AB$ are regular and the $A$-module $B$ is flat.
\item[(\rmnum{4})] $A$ and $\kappa_A\otimes_AB$ are regular and we have $\dim(B)=\dim(A)+\dim(\kappa_A\otimes_AB)$.
\end{itemize}
\end{proposition}
\begin{proof}
We have $\dim(B)\leq\dim(A)+\dim(\kappa_A\otimes_AB)$ by \cref{Noe local ring homomorphism dim and secant prop}; the equivalence of (\rmnum{1}) and (\rmnum{4}) then follows from \cref{Noe local ring embdim local homomorphism prop}. Under the hypotheses of (\rmnum{2}), the $A$-module $B$ is faithfully flat since $\rho$ is a local homomorphism and this implies (\rmnum{3}) by \cref{Noe regular faithfully flat descent}. The implication (\rmnum{3})$\Rightarrow$(\rmnum{4}) follows from \cref{Noe local ring homomorphism dim and secant prop}.\par
Now it suffices to prove that if the equivalent conditions of (\rmnum{1}) and (\rmnum{4}) are satisfied, then the $A$-module $B$ is flat. To this end, let $\bm{x}$ be a system of parameters of $A$. Since $d\rho$ is injective, the image of $\bm{x}$ under $\rho$ can be extended into a system of parameters of $B$. The sequence $\bm{x}$ is then completely secant for $A$ and for $B$ (\cref{Noe local ring quotient by sequence regular iff}), and generates the ideal $\m_A$ of $A$. By \ref{Koszul complex sequence completely secant is free resolution}, the $A$-module $\Tor_1^A(\kappa_A,B)$ is then isomorphic to $H_1(\bm{x},B)$, hence is zero. It then follows that $B$ is flat over $A$ (\cref{filtration module and flatness} and \cref{Noe ring radical extension finite module is ideally Hausdorff}). 
\end{proof}
\begin{example}
Let $X$ and $Y$ be finite dimensional complex analytic varieties, $f:X\to Y$ be a morphis, and $x$ be a point of $X$. Consider the local homomorphism $\rho:\mathscr{O}_{Y,f(x)}\to\mathscr{O}_{X,x}$ induced by $f$. The map $d\rho$ is then the tangent map $d_xf:T_xX\to T_{f(x)}Y$. The equivalent conditions of \cref{Noe local ring local homomorphism submersion iff} are therefore equivalent in this case to the fact that $f$ is a submersion at $x$.
\end{example}
\begin{corollary}\label{Noe ring flat extension regular if fiber regular}
Let $\rho:A\to B$ be a flat homomorphism of Noetherian rings. If $A$ is regular and $\kappa(\mathfrak{N}^c)\otimes_AB$ is regular for any maximal ideal $\mathfrak{N}$ of $B$, then the ring $B$ is regular.
\end{corollary}
\begin{proof}
In fact, for any maximal ideal $\mathfrak{N}$ of $B$, the $A_{\mathfrak{N}^c}$-module $B_\mathfrak{N}$ is flat (\cref{module flat iff localization at maximal}), so the ring $B_\mathfrak{N}$ is regular by \cref{Noe local ring local homomorphism submersion iff}.
\end{proof}
\section{Complete intersections}
\subsection{Completely secant ideals}
Let $A$ be a ring and $\mathfrak{I}$ be an ideal of $A$. We say that the ideal $\mathfrak{I}$ is \textbf{completely secant at a point $\p$} of $V(\mathfrak{I})$ if the ideal $\mathfrak{I}_\p$ of $A_\p$ is generated by a completely secant sequence for $A_\p$. We say that $\mathfrak{I}$ is \textbf{completely secant} if it is completely secant at every point of $V(\mathfrak{I})$.\par
If the ideal $\mathfrak{I}$ is completely secant, so is the ideal $S^{-1}\mathfrak{I}$ of $S^{-1}A$ for any multiplicative subset $S$ of $A$. Any ideal generated by a completely secant sequence is completely secant. More precisely, we have the following:
\begin{proposition}\label{completely secant ideal generated by sequence iff}
Let $A$ be a ring, $\mathfrak{I}$ be an ideal of $A$ generated by a finite sequence $\bm{x}=(x_1,\dots,x_r)$ of elements of $A$. Then the following conditions are equivalent:
\begin{itemize}
\item[(\rmnum{1})] the sequence $\bm{x}$ is completely secant for $A$;
\item[(\rmnum{2})] for any prime (resp. maximal) ideal $\p\in V(\mathfrak{I})$, the image of $\bm{x}$ in $A_\p$ is completely secant for $A_\p$;
\item[(\rmnum{3})] for any prime (resp. maximal) ideal $\p\in V(\mathfrak{I})$, the ideal $\mathfrak{I}$ is completely secant at $\p$ and the image of $\bm{x}$ in the $\kappa(\p)$-vector space $\kappa(\p)\otimes_A\mathfrak{I}$ form a basis.
\end{itemize}
If $A$ is Noetherian, these conditions are equivalent to:
\begin{itemize}
\item[(\rmnum{4})] for any integer $n\geq 0$, the $A/\mathfrak{I}$-module $\mathfrak{I}^n/\mathfrak{I}^{n+1}$ is free and the image of the monomials $x_1^{\alpha_1}\cdots x_r^{\alpha_r}$ of total degree $n$ form a basis.
\end{itemize}
\end{proposition}
\begin{proof}
Let $\p$ be a prime ideal of $A$, and denote by $\bm{x}_\p$ the image of $\bm{x}$ in $A_\p$. For any integer $i\geq 0$, the $A_\p$-module $H_i(\bm{x}_\p,A_\p)$ is isomorphic to $(H_i(\bm{x},A))_\p$ (\cref{Koszul complex ring extension prop}(b)); it is then zero if $\p$ does not contain $\mathfrak{I}$, since we then have $\mathfrak{I}_\p=A_\p$ (\cref{Koszul complex annihilator prop}). The equivalence of (\rmnum{2}) and (\rmnum{2}) then folows from \cref{localization module zero iff}. On the other hand, the equivalence of (\rmnum{2}) and (\rmnum{3}) follows from \cref{depth of module local ring bounded by generator mod m_A}.\par
The implication (\rmnum{1})$\Rightarrow$(\rmnum{4}) is a concequence of \cref{Koszul complex completely secant annd regular relation}. Finally, for any $\p\in V(\mathfrak{I})$, condition (\rmnum{4}) implies by localization the analogous results for the $(A_\p/\mathfrak{I}_\p)$-module $\mathfrak{I}_\p^n/\mathfrak{I}^{n+1}_\p$, and this implies (\rmnum{1}) by \cref{Koszul sequence Zariski ring complete secant is regular}.
\end{proof}
\begin{remark}
If $A$ is Noetherian, we can replace "completely secant" by "regular" in condition (\rmnum{2}) of \cref{completely secant ideal generated by sequence iff}. However, this is not the case in condition (\rmnum{1}): a completely secant sequence for $A$ is not necessarily $A$-regular (exercise 1).
\end{remark}
\begin{remark}
Let $A$ be a ring, $\mathfrak{I}$ be a finitely generated ideal of $A$ and $\p$ be a prime ideal of $A$ containing $\mathfrak{I}$. By \cref{depth of module local ring bounded by generator mod m_A}, we have
\[\depth_{A_\p}(\mathfrak{I}_\p,A_\p)\leq[\kappa(\p)\otimes_A\mathfrak{I}:\kappa(\p)],\]
and the equality holds if and only if $\mathfrak{I}$ is completely secant at $\p$. Suppose that $\mathfrak{I}$ is proper and generated by a completely secant sequence $(x_1,\dots,x_r)$. We then have (\cref{depth of module base change pullback prop} and \cref{completely secant ideal generated by sequence iff})
\[\depth_A(\mathfrak{I},A)=\inf_{\p\in V(\mathfrak{I})}\depth_{A_\p}(\mathfrak{I}_\p,A_\p)=r.\]
If $A$ is also Noetherian, then by \cref{Noe ring codim of intersection with hypersurface} and \cref{depth of module local Noe inequality} we have
\[\depth_A(\mathfrak{I},A)\leq\codim(V(\mathfrak{I}),\Spec(A))\leq r,\]
whence
\[\depth_A(\mathfrak{I},A)=r=\codim(V(\mathfrak{I}),\Spec(A))=\height(\mathfrak{I}).\]
\end{remark}
For an ideal $\mathfrak{I}$ of a ring $A$, the graded $A$-module $\bigoplus_{n\in\N}\mathfrak{I}^n$ possesses a natural graded $A$-algebra structure, induced from the multiplication of the ring $A$. The identity map on $\mathfrak{I}$ then extends to a canonical surjective homomorphism of graded $A$-algebras
\[\alpha_\mathfrak{I}:\bm{S}_A(\mathfrak{I})\to\bigoplus_{n\in\N}\mathfrak{I}^n.\]
By extension of scalars to the ring $A/\mathfrak{I}$, we also deduce a canonical surjective homomorphism of graded $A/\mathfrak{I}$-algebras
\[\beta_\mathfrak{I}:\bm{S}_{A/\mathfrak{I}}(\mathfrak{I}/\mathfrak{I}^2)\to\gr_{\mathfrak{I}}(A).\]
As the case of \cref{Koszul complex completely secant annd regular relation}, the injectivity of the canonical homomorphisms $\alpha_\mathfrak{I}$ and $\beta_\mathfrak{I}$ characterizes completely secant ideals. More precisely, we have the following theorem:
\begin{theorem}\label{Noe ring completely secant ideal iff alpha and beta isomorphism}
Let $A$ be a Noetherian ring and $\mathfrak{I}$ be an ideal of $A$. The following conditions are equivalent:
\begin{itemize}
\item[(\rmnum{1})] the ideal $\mathfrak{I}$ is completely secant;
\item[(\rmnum{2})] the ideal $\mathfrak{I}$ is completely secant at every maximal ideal $\m\in V(\mathfrak{I})$; 
\item[(\rmnum{3})] the $A/\mathfrak{I}$-module $\mathfrak{I}/\mathfrak{I}^2$ is projective and the canonical homomorphism $\alpha_\mathfrak{I}:\bm{S}_A(\mathfrak{I})\to\bigoplus_{n\in\N}\mathfrak{I}^n$ is bijective;
\item[(\rmnum{4})] the $A/\mathfrak{I}$-module $\mathfrak{I}/\mathfrak{I}^2$ is projective and the canonical homomorphism $\beta_\mathfrak{I}:\bm{S}_{A/\mathfrak{I}}(\mathfrak{I}/\mathfrak{I}^2)\to\gr_{\mathfrak{I}}(A)$ is bijective.
\end{itemize}
\end{theorem}
\begin{proof}
It is clear that (\rmnum{1})$\Rightarrow$(\rmnum{2}) and (\rmnum{3})$\Rightarrow$(\rmnum{4}). Suppose that condition (\rmnum{2}) is satisfied; to see that (\rmnum{2})$\Rightarrow$(\rmnum{3}), it suffices to prove that for any maximal ideal $\m$ of $A$, the $A_\m/\mathfrak{I}_\m$-module $\mathfrak{I}_\m/\mathfrak{I}_\m^2$ is free and the homomorphism $\alpha_{\mathfrak{I}_\m}:S_{A_\m}(\mathfrak{I}_\m)\to\bigoplus_{n\in\N}\mathfrak{I}_\m^n$ is bijective (\cref{module finite projective iff} and \cref{localization homomorphism inj surj iff}). But this assertion is trivial if $\m$ does not belong to $V(\mathfrak{I})$ (since we then have $\mathfrak{I}_\m=A_\m$), and follows from \cref{Koszul complex completely secant annd regular relation} if $\m\in V(\mathfrak{I})$.\par
Finally, suppose that condition (\rmnum{4}) is satisfied, and let $\p$ be a prime ideal of $A$ containing $\mathfrak{I}$. Then the $A_\p/\mathfrak{I}_\p$-module $\mathfrak{I}_\p/\mathfrak{I}_\p^2$ is free. Let $\bm{x}=(x_1,\dots,x_r)$ be a sequence of elements of $\mathfrak{I}_\p$ whose image forms a basis for $\mathfrak{I}_\p/\mathfrak{I}_\p^2$. The sequence $\bm{x}$ generates $\mathfrak{I}_\p$ by Nakayama's lemma, and satisfies condition (\rmnum{4}) of \cref{completely secant ideal generated by sequence iff}. Therefore the ideal $\mathfrak{I}_\p$ of $A_\p$ is completely secant, and $\mathfrak{I}$ is then completely secant at $\p$.
\end{proof}
\begin{remark}
Suppose that the ideal $\mathfrak{I}$ is completely secant and let $\bm{x}=(x_1,\dots,x_r)$ be a sequence of elements of $\mathfrak{I}$ such that for any maximal ideal $\m\in V(\mathfrak{I})$, the image of $\bm{x}$ in $\mathfrak{I}/\m\mathfrak{I}$ forms a basis for the $A/\m$-vector space $\mathfrak{I}/\m\mathfrak{I}$. Then the $A/\mathfrak{I}$-module $\mathfrak{I}/\mathfrak{I}^2$ is free and the canonical image of $\bm{x}$ in $\mathfrak{I}/\mathfrak{I}^2$ forms a basis: in fact, it suffices to verify that for each $\m\in V(\mathfrak{I})$, the image of $\bm{x}$ forms a basis for the $A_\m/\mathfrak{I}_\m$-module $\mathfrak{I}_\m/\mathfrak{I}^2_\m$, and this follows from \cref{local ring module M/mM is free then M free} and \cref{local ring fp module free iff flat iff projective} since the $A_\m/\mathfrak{I}_\m$-module $\mathfrak{I}_\m/\mathfrak{I}^2_\m$ is projective by \cref{Noe ring completely secant ideal iff alpha and beta isomorphism}.
\end{remark}
\begin{corollary}\label{Noe ring flat extension completely secant iff}
Let $\rho:A\to B$ be a flat homomorphism of Noetherian rings, and $\mathfrak{I}$ be an ideal of $A$. 
\begin{itemize}
\item[(a)] If $\mathfrak{I}$ is completely secant, the ideal $\mathfrak{I}B$ is completely secant.
\item[(b)] Suppose that the ideal $\mathfrak{I}B$ is completely secant and that any maximal ideal $\m\in V(\mathfrak{I})$ is the inverse image of a maximal ideal of $B$. Then the ideal $\mathfrak{I}$ is completely secant. This is the case for example if $B$ is faithfully flat over $A$.
\end{itemize}
\end{corollary}
\begin{proof}
Since the $A$-module $B$ if flat, $\mathfrak{I}^n\otimes_AB$ is identified with $\mathfrak{I}^nB$ and $(\mathfrak{I}^n/\mathfrak{I}^{n+1})\otimes_{A/\mathfrak{I}}(B/\mathfrak{I}B)$ is identified with $\mathfrak{I}^nB/\mathfrak{I}^{n+1}B$ for each integer $n\geq 0$. Assertion (a) then follows from criterion (\rmnum{3}) of \cref{Noe ring completely secant ideal iff alpha and beta isomorphism}. Under the hypotheses of (b), the $A/\mathfrak{I}$-module $B/\mathfrak{I}B$ is faithfully flat (\cref{module flat extension is flat} and \cref{ring faithfully flat iff}) and $\mathfrak{I}$ is completely secant by criterion (\rmnum{4}) of \cref{Noe ring completely secant ideal iff alpha and beta isomorphism}. The last assertion follows from \cref{ring faithfully flat iff}.
\end{proof}
\begin{corollary}\label{Noe ring completely secant completion iff}
Let $A$ be a Noetherian ring, $\mathfrak{I}$ be an ideal of $A$, $\widehat{A}$ be the $\mathfrak{I}$-adic completion of $A$ and $\widehat{\mathfrak{I}}=\mathfrak{I}\widehat{A}$ be the completion of $\mathfrak{I}$. For the ideal $\widehat{\mathfrak{I}}$ of $\widehat{A}$ to be completely secant, it is necessary and sufficient that the ideal $\mathfrak{I}$ of $A$ is completely secant.
\end{corollary}
\begin{proof}
In fact, by \cref{filtration I-adic completion maximal ideal} the condition of \cref{Noe ring flat extension completely secant iff}(b) is satisfied.
\end{proof}
\begin{corollary}\label{Noe ring quotient by completely secant ideal CM Gorenstein}
Let $A$ be a Noetherian ring and $\mathfrak{I}$ be a completely secant ideal of $A$. If $A$ is a Cohen-Macaulay (resp. Gorenstein) ring, so is $A/\mathfrak{I}$.
\end{corollary}
\begin{proof}
Let $\m$ be a maximal ideal of $A$ containing $\mathfrak{I}$. The ideal $\mathfrak{I}_\m$ of $A_\m$ is generated by an $A_\m$-regular sequence, so $(A/\mathfrak{I})_\m$ is Cohen-Macaulay (resp. Corenstein) by \cref{CM ring example}(e) (resp. \cref{Gorenstein ring quotient by regular sequence}).
\end{proof}
\begin{remark}
A Noetherian ring $A$ is said to be \textbf{complete intersection} if for any maximal ideal $\m$ of $A$, the completion of the local ring $A_\m$ is isomorphic to a quotient of a complete Noetherian regular local ring by a completely secant ideal. It then follows from \cref{Noe ring quotient by completely secant ideal CM Gorenstein} and \cref{regular local ring is Gorenstein} that such a ring is Gorenstein.
\end{remark}
\begin{proposition}\label{Noe ring regular iff maximal ideal completely secant}
Let $A$ be a Noetherian ring. Then the following conditions are equivalent:
\begin{itemize}
\item[(\rmnum{1})] $A$ is regular;
\item[(\rmnum{2})] any maximal ideal of $A$ is completely secant;
\item[(\rmnum{3})] any ideal $\mathfrak{I}$ of $A$ such that $A/\mathfrak{I}$ is regular is completely secant.
\end{itemize}
\end{proposition}
\begin{proof}
Suppose that the ring $A$ is regular; let $\mathfrak{I}$ be an ideal of $A$ such that $A/\mathfrak{I}$ is regular and $\p$ be a prime idela of $A$ containing $\mathfrak{I}$. Then the local rings $A_\p$ and $A_\p/\mathfrak{I}_\p$ are regular, so $\mathfrak{I}_\p$ is generated by a completely secant sequence for $A_\p$ (\cref{Noe local ring quotient by sequence regular iff}), and this signifies that $\mathfrak{I}$ is completely secant at $\p$. This proves (\rmnum{1})$\Rightarrow$(\rmnum{3}), and (\rmnum{3}) $\Rightarrow$(\rmnum{2}) since a field is regular. Finally, under the hypotheses of (\rmnum{2}), let $\m$ be a maximal ideal of $A$. Then the maximal ideal $\m A_\m$ of $A_\m$ is generated by a completely secant sequence for $A_\m$, so $A_\m$ is regular (\cref{Noe local ring regular iff m_A generated by completely secant}), whence (\rmnum{1}).
\end{proof}
\begin{proposition}\label{Noe ring A/I regular completely secant iff}
Let $A$ be a Noetherian ring and $\mathfrak{I}$ be an ideal of $A$ such that $A/\mathfrak{I}$ is regular. Then the following conditions are equivalent:
\begin{itemize}
\item[(\rmnum{1})] the ideal $\mathfrak{I}$ is completely secant;
\item[(\rmnum{2})] for any prime (resp. maximal) ideal $\p$ of $A$ containing $\mathfrak{I}$, the ring $A_\p$ is regular;
\item[(\rmnum{3})] the $\mathfrak{I}$-adic completion of $A$ is regular.
\end{itemize}
\end{proposition}
\begin{proof}
By \cref{Noe ring completely secant ideal iff alpha and beta isomorphism}, condition (\rmnum{1}) signifies that for any prime (resp. maximal) ideal $\p$ of $A$ containing $\mathfrak{I}$, the ideal $\mathfrak{I}_\p$ of the local ring $A_\p$ is generated by a completely secant sequence for $A_\p$. Since $A_\p/\mathfrak{I}_\p$ is regular by hypothesis, this latter condition is equivalent to that $A_\p$ is regular (\cref{Noe local ring quotient by sequence regular iff}). This proves the equivalence of (\rmnum{1}) and (\rmnum{2}), and the equivalence of (\rmnum{2}) and (\rmnum{3}) follows from \cref{Noe ring I-adic completion regular iff}.
\end{proof}
\begin{proposition}\label{Noe regular ring A/I isomorphic to subring completely secant}
Let $A$ be a regular ring, $\mathfrak{I}$ be an ideal of $A$ and $A_0$ be a subring of $A$ such that the canonical homomorphism $A_0\to A/\mathfrak{I}$ is bijective.
\begin{itemize}
\item[(a)] The ideal $\mathfrak{I}$ is completely secant, the $A_0$-module $\mathfrak{I}/\mathfrak{I}^2$ is finitely generated and projective, and the ring $A_0$ is regular.
\item[(b)] Let $\varphi:\mathfrak{I}/\mathfrak{I}^2\to\mathfrak{I}$ be a $A_0$-linear section of the canonical surjection $\mathfrak{I}\to\mathfrak{I}/\mathfrak{I}^2$. Then the $A_0$-homomorphism $\bm{S}_{A_0}(\mathfrak{I}/\mathfrak{I}^2)\to A$ extending $\varphi$ then extends to an isomorphism from the completion of the graded ring $\bm{S}_{A_0}(\mathfrak{I}/\mathfrak{I}^2)$ to the $\mathfrak{I}$-adic completion of $A$.
\end{itemize}
\end{proposition}
\begin{proof}
Let $\p$ be a prime ideal of $A$ containing $\mathfrak{I}$. We have $\p=(\p\cap A_0)\oplus\mathfrak{I}$ and therefore $\p^2=(\p\cap A_0)^2\oplus\p\mathfrak{I}$, hence $\p^2\cap\mathfrak{I}=\p\mathfrak{I}$. Denote by $i:\mathfrak{I}\to\p$ be the canonical injection, then the map $i\otimes 1_{A/\p}:\mathfrak{I}\otimes_AA/\p\to\p\otimes_AA/\p$ is injective, and so is the map $i_\p\otimes 1_{A/\p}:\mathfrak{I}_\p\otimes_{A_\p}\kappa(\p)\to\p\otimes_{A_\p}\kappa(\p)$. Nakayama's lemma then implies that the ideal $\mathfrak{I}_\p$ of $A_\p$ is generated by a subset of a system of parameters of the regular local ring $A_\p$. By \cref{regular local ring quotient by ideal regular iff}, the ring $A_\p/\mathfrak{I}_\p$ is regular and the ideal $\mathfrak{I}_\p$ is completely secant. By \cref{Noe ring completely secant ideal iff alpha and beta isomorphism}, $\mathfrak{I}$ is then completely secant, the ring $A_0$, isomorphic to $A/\mathfrak{I}$, is regular, and the $A_0$-module $\mathfrak{I}/\mathfrak{I}^2$ is finitely generated and projective.\par
Now it follows from \cref{Noe ring completely secant ideal iff alpha and beta isomorphism} that the canoncal homomorphism $\beta_\mathfrak{I}:\bm{S}_{A_0}(\mathfrak{I}/\mathfrak{I}^2)$ is bijective. Let $f:\bm{S}_{A_0}(\mathfrak{I}/\mathfrak{I}^2)\to A$ be the $A_0$-homomorphism extending the $A_0$-linear map $\varphi:\mathfrak{I}/\mathfrak{I}^2\to\mathfrak{I}$. If we endow $A$ with the $\mathfrak{I}$-adic filtration and $\bm{S}_{A_0}(\mathfrak{I}/\mathfrak{I}^2)$ the graduation filtration, $\beta_\mathfrak{I}$ is identified with the homomorphism induced from $f$ by passing to graded algebras, and assertion (b) follows from \cref{filtration gr(phi) bijective imply phi bijective if exhaustive and domain complete}.
\end{proof}
\subsection{Graded regular rings}
Let $A_0$ be a ring and $P$ be a graded $A_0$-module of type $\N$. Denote by $A$ the ring $\bm{S}_{A_0}(P)$, then there exists over $A$ a unique graduation of type $\N$ for which $A_0$ is of degree $0$ and $P$ is a graded submodule of $A$. Let $A_+=\bigoplus_{n>0}A_n$ be the irrelevant ideal of $A$, then the canonical map $P\to A_+/A_+^2$ is an isomorphism of graded $A_0$-modules (A, \Rmnum{3}, p76, prop.10).\par 
If the graded $A_0$-module $P$ is free and $(x_i)_{i\in I}$ is a basis for $P$ formed by homogeneous elements, the graded $A_0$-algebra $\bm{S}_{A_0}(P)$ is then isomorphic to the polynomial algebra $A_0[(X_i)_{i\in I}]$, endowed with the graduation for which each $X_i$ is homogeneous of degree $\deg(x_i)$. Any graded $A_0$-algbera of type $\N$, isomorphic to the graded $A_0$-algebra of this form, is said to be \textbf{polynomial}.\par
If $A_0$ is regular and the $A_0$-module $P$ is projective and finitely generated, then the ring $\bm{S}_{A_0}(P)$ is regular by \cref{Noe regular ring symmetric algebra of projective is regular}. Conversely, we have the following theorem which asserting that any regular graded $A_0$-algebra is of this form.
\begin{theorem}\label{graded ring regular iff A_0 regular polynomial}
Let $A$ be a regular graded ring of type $\N$. Then the ring $A_0$ formed by elements of degree $0$ in $A$ is regular, and there exists a finitely generated projective graded $A_0$-module $P$ with positive degrees such that $A$ is isomorphic to the graded $A_0$-algebra $\bm{S}_{A_0}(P)$. 
\end{theorem}
\begin{proof}
Denote by $P$ the graded $A_0$-module $A_+/A_+^2$. By \cref{Noe regular ring A/I isomorphic to subring completely secant}, the ring $A_0$ is regular and the $A_0$-module $P$ is projective and finitely generated. The homogeneous components of $P$ are then projective and there exists an $A_0$-linear section $\varphi:P\to A_+$ of the canonical surjection $A_+\to P$, which is graded of degree $0$. Let $f:\bm{S}_{A_0}(P)\to A$ be the induced homomorphism of graded $A_0$-algebras. Then by \cref{Noe regular ring A/I isomorphic to subring completely secant}, $\hat{f}:\widehat{\bm{S}_{A_0}(P)}\to\widehat{A}$ is an isomorphism, so $f$ is injective and its image is dense in $A$ for the $A_+$-adic topology. But since the topologies induced over the homogeneous components of $A$ are diecrete and the image of $f$ is a graded submodule, this implies that $f$ is bijective.
\end{proof}
\begin{corollary}\label{graded ring regular finite extension free iff}
Let $B$ be a regular graded ring of positive degrees. Suppose that any finitely generated projective $B_0$-module is free.
\begin{itemize}
\item[(a)] The ring $B_0$ is integral and regular and the $B_0$-algebra $B$ is a polynomial graded $B_0$-algebra of finite type.
\item[(b)] Let $A$ be a graded subring of $B$ such that $A_0=B_0$ and that $B$ is a finitely generated $A$-module. Then the following conditions are equivalent:
\begin{itemize}
\item[(\rmnum{1})] the graded $A$-module $B$ is free; 
\item[(\rmnum{2})] the $A$-module $B$ is flat;
\item[(\rmnum{3})] $A$ is a polynomial graded $A_0$-algebra of finite type.
\end{itemize}
\end{itemize}
\end{corollary}
\begin{corollary}\label{graded polynomial k-algebra free over graded subring iff}
Let $k$ be a field, $B$ be a polynomial graded $k$-algebra of finite type, and $A$ be a graded subring of $B$. Then the following conditions are equivalent:
\begin{itemize}
\item[(\rmnum{1})] $B$ is a graded free $A$-module;
\item[(\rmnum{2})] $B$ is a flat $A$-module;
\item[(\rmnum{3})] $\Tor_1^A(k,B)=0$;
\item[(\rmnum{4})] the algebra $A$ is a polynomial graded $k$-algebra of finite type, and any algebraically free generating sequence of $A$ formed by homogeneous elements is $B$-regular.
\end{itemize}
\end{corollary}
\subsection{Extension of scalars}
\begin{proposition}\label{Noe ring completely secant ideal iff flat fiber}
Let $\rho:A\to B$ be a homomorphism of Noetherian rings and $\mathfrak{I}$ be an ideal of $B$. Then the following conditions are equivalent:
\begin{itemize}
\item[(\rmnum{1})] the $A$-module $B/\mathfrak{I}$ is flat and the ideal $\mathfrak{I}$ is completely secant;
\item[(\rmnum{2})] for any $\mathfrak{P}\in V(\mathfrak{I})$, the $A$-module $B_\mathfrak{P}$ is flat and, for any $A$-algebra $A'$ such that the ring $A'\otimes_AB$ is Noetherian, the ideal $\mathfrak{I}(A'\otimes_AB)$ is completely secant;
\item[(\rmnum{3})] for any maximal ideal $\mathfrak{N}$ of $B$ containing $\mathfrak{I}$, the $A$-module $B_\mathfrak{N}$ is flat and the ideal $\mathfrak{I}(\kappa(\mathfrak{N}^c)\otimes_AB_\mathfrak{N})$ of $\kappa(\mathfrak{N}^c)\otimes_AB_\mathfrak{N}$ is completely secant.
\end{itemize}
\end{proposition}
\section{Geometric regularity and normality}
\subsection{Algebras essentially of finite type}
Let $k$ be a ring, $A$ be a $k$-algebra, and $\bm{x}=(x_i)_{i\in I}$ be a family of element of $A$; denote by $A'$ the subalgebra of $A$ generated by the $x_i$. We say that $\bm{x}$ is an \textbf{essentially generating family} of the $k$-algebra $A$ if, for any element $a\in A$, there exists an element $s$ of $A'$, invertible in $A$, such that $sa\in A'$. Equivalently, this amounts to saying that, for any $a\in A$, there exist polynomials $P,Q$ of $k[(X_i)_{i\in I}]$ such that $Q(\bm{x})$ is invertible in $A$ and that $a=P(\bm{x})Q(\bm{x})^{-1}$.\par
A $k$-algebra $A$ is said to be \textbf{essentially of finite type} if it admits a finite essentially generating family. Equivalently, this means there exists a $k$-algebra $A'$ of finite type and a multiplicative subset $S$ of $A'$ such that $A$ is isomorphic to the $k$-algebra $S^{-1}A'$.
\begin{example}\label{field ext es.ft iff finitely generated}
To say that a field extension $L/K$ is essentially of finite type signifies that it is a finitely generated field extension (note that this is not equivalent to that $K$ is a \textit{$k$-algebra} of finite type). In fact, if $\bm{x}=(x_i)_{i\in I}$ is an essentially generating family for $L$, then we obtain a surjective homomorphism from $K((x_i)_{i\in I})\to L$. Also note that by Zariski's lemma (\ref{field ft algebra Zariski lemma}), $L$ is of finite type over $K$ if and only if it is a finite extension of $K$.
\end{example}  
\begin{example}\label{algebra local es.ft iff local ring of ft}
For a local $k$-algebra to be essentially of finite type, it is necessary and sufficient that it is isomorphic to a $k$-algebra of the form $A_\p$, where $A$ is a $k$-algebra of finite type and $\p$ is a prime ideal of $A$. In fact, if the local $k$-algebra is isomorphic to $S^{-1}A$, then by (\cref{localization at comparable prime}), it is isomorphic to $A_\p$, where $\p$ is the inverse image of the maximal ideal of $S^{-1}A$ in $A$.
\end{example}
\begin{proposition}\label{algebra es.ft over Noe is Noe}
If the ring $k$ is Noetherian, then any $k$-algebra essentially of finite type is a Noetherian ring.
\end{proposition}
\begin{proof}
This follows from Hilbert basis theorem and \cref{Noe module localization is Noe}.
\end{proof}
\begin{proposition}\label{algebra es.ft quotient fraction product base change}
Let $k$ be a ring.
\begin{itemize}
\item[(a)] Any quotient algebra of a $k$-algebra essentially of finite type is a $k$-algebra essentially of finite type.
\item[(b)] Any fraction ring of a $k$-algebra essentially of finite type is a $k$-algebra essentially of finite type.
\item[(c)] The product of a finite family of $k$-algebras essentially of finite type is a $k$-algebra essentially of finite type.
\item[(d)] Let $k\to k'$ be a ring homomorphism. For any $k$-algebra $A$ essentially of finite type, the $k'$-algebra $A_{(k')}=k'\otimes_kA$ is essentially of finite type.
\end{itemize}
\end{proposition}
\begin{proof}
Since localization commutes with quotients, it is clear that assertion (a) is true, and assertion (b) follows from definition. Similarly, since localization commutes with base change, it is easy to conclude (d), and assertion (c) follows from the definition of essentially generating families.
\end{proof}
\begin{corollary}\label{algebra es.ft base change to Noe is Noe}
Let $A$ be a $k$-algebra essentially of finite type and $B$ be a Noetherian $k$-algebra. Then the ring $A\otimes_kB$ is Noetherian.
\end{corollary}
\begin{proof}
In fact, this is a $B$-algebra essentially of finite type, so the conclusion follows from \cref{algebra es.ft over Noe is Noe}.
\end{proof}
\begin{proposition}\label{algebra es.ft transitive prop}
Let $k$ be a ring, $A$ be a $k$-algebra and $B$ be an $A$-algebra. If $A$ is essentially of finite type over $k$ and $B$ is essentially of finite type over $A$, then $B$ is essentially of finite type over $k$.
\end{proposition}
\begin{proof}
Denote by $\rho:A\to B$ the canonical map. Let $\bm{x}=(x_i)_{i\in I}$ be an essentially generating family of the $k$-algebra $A$ and $A'$ the subalgebra it generates; let $\bm{y}=(y_j)_{j\in J}$ be an essentially generating family of the $A$-algebra $B$, and $B'$ be the subalgebra generated by the $\rho(x_i)$ and $y_j$. Let $b\in B$, then by hypothesis there exist polynomials $P,Q$ of $A[(Y_j)_{j\in J}]$ such that $Q(\bm{y})$ is invertible in $B$ and that we have $Q(\bm{y})b=P(\bm{y})$. The nonzero coefficients of $P$ and $Q$ are finite in number, so there exists a polynomial $R\in k[(X_i)_{i\in I}]$ such that $R(\bm{x})$ is invertible in $A$ and that $R(\bm{x})P$ and $R(\bm{x})Q$ belongs to $A'[(Y_j)_{j\in J}]$. Then $\rho(R(\bm{x}))Q(\bm{y})$ is invertible in $B$, and we have
\[\rho(R(\bm{x}))Q(\bm{y})b=\rho(R(\bm{x}))P(\bm{y})\in B'.\]
In other words, the elements $\rho(x_i)$ and $y_j$ for an essentially generating family for the $k$-algebra $B$.
\end{proof}
\begin{corollary}\label{algebra es.ft tensor product prop}
The tensor product of two $k$-algebras essentially of finite type is a $k$-algebra essentially of finite type.
\end{corollary}
\begin{proof}
Let $A$ and $B$ be two $k$-algebras essentially of finite type. Then $A\otimes_kB$ is essentially of finite type over $A$ (\cref{algebra es.ft quotient fraction product base change}), and hence over $k$ (\cref{algebra es.ft transitive prop}).
\end{proof}
\subsection{Tensor product of Cohen-Macaulay algebras}
\subsection{Geometrically regular and geometrically normal algebras}
Recall that a separable algebra $A$ over a field $k$ is defined by the property that for any finite field extension $k'$ of $k$, the ring $A_{(k')}$ is reduced (in other words, the $k$-algebra $A$ is \textit{geometrically reduced}). In this paragraph, we develop similar notions for a $k$-algebra $A$, concering the regularity (resp. normality) for any base change $A_{(k')}$ into a field extension $k'$ of $k$. Such $k$-algebras are then called \textit{geometrically regular} (resp. \textit{geometrically normal}), and they will play an important role when we discuss smooth algebras.\par
Before giving the definition, we first recall if $k'$ is a separable extension of $k$, then the base change $A_{(k')}$ of any reduced $k$-algebra is still reduced. We now prove a similar result for regular and normal algebras. However, for this to work, we need to impose some finite condition on the $k$-algebras. The technique points of our consideration are contained in the following two lemmas.
\begin{lemma}\label{Noe ring exhaustive filtration regular and normal}
Let $A$ be a Noetherian ring with an exhaustive filtration $(A_\alpha)_{\alpha\in I}$ of Noetherian subrings.
\begin{itemize}
\item[(a)] If the rings $A_\alpha$ are regular and $A$ is a flat $A_\alpha$-module for each $\alpha\in I$, then $A$ is regular.
\item[(b)] If the rings $A_\alpha$ are normal, then $A$ is normal.
\end{itemize}
\end{lemma}
\begin{proof}

\end{proof}
\begin{lemma}\label{field ext tensor regular if separable ft}
Let $k$ be a field and $K,L$ be field extensions of $k$ such that one of them is separable over $k$. Suppose that $K$ is finitely generated over $k$, then the ring $K\otimes_kL$ is regular.
\end{lemma}
\begin{proof}

\end{proof}
\begin{proposition}\label{algebra separable base change regular normal prop}
Let $k$ be a field, $A$ be a $k$-algebra and $K$ be a extension of $k$. Suppose that $A$ is essentially of finite type or that the extension $K$ is finitely generated.
\begin{itemize}
\item[(a)] If the ring $A_{(K)}$ is regular (resp. normal), then $A$ is regular (resp. normal).
\item[(b)] If the ring $A$ is regular (resp. normal) and the extension $K/k$ is separable, then $A_{(K)}$ is regular (resp. normal). 
\end{itemize}
\end{proposition}
\begin{proof}
Being the base change of the free $k$-module $K$ by $A$, the $A$-module $A_{(K)}$ is free, hence faithfully flat. Assertion (a) then follows from \cref{ring faithfully flat Noe Artin descent} and \cref{Noe regular faithfully flat descent} (resp. \cref{field ext tensor regular if separable ft}). Under the hypothesis of (b), for any prime ideal $\p$ of $A$, the ring $\kappa(\p)\otimes_kK$ is regular by \cref{field ext tensor regular if separable ft}, and a fortiori normal (\cref{Noe regular is Gorenstein normal}). Assertion (b) then follows from \cref{Noe local ring local homomorphism submersion iff} (resp. \cref{Noe ring normal flat base change normal if}).
\end{proof}
Let $k$ be a field and $A$ be a $k$-algebra. We say that $A$ is \textbf{geometrically regular} (resp. \textbf{geometrically normal}) if the ring $A_{(k')}$ is regular (resp. normal) for any finite extension $k'$ of $k$. It is clear tha any geometrically regular (resp. geometrically normal) $k$-algebra is itself regular (resp. normal), as we can take $k'=k$ in the defintion. Also, in view of \cref{algebra separable base change regular normal prop}, to verify that a $k$-algebra is geometrically regular (resp. geometrically normal), it suffices to consider only finite purly separable extensions of $k$.
\begin{example}\label{algebra over perfect geomregular normal iff regular normal}
If the field $k$ is perfect, any regular (resp. normal) $k$-algebra is geometrically regular (resp. geometrically normal). This is an analogue for separable algebras, since in this case any reduced $k$-algebra is separable.
\end{example}
\begin{example}\label{algebra Artin georegular iff separable}
Let $A$ be an Artinian $k$-algebra. Then the following conditions are equivalent:
\begin{itemize}
\item $A$ is separable;
\item $A$ is geometrically regular;
\item $A$ is geometrically normal.
\end{itemize}
In fact, if $A$ is geometrically normal then it is separable, and if $A$ is separable, for any finite extension $k'$ of $k$, the ring $A_{(k')}$ is reduced and Artinian, hence isomorphic to a product of fields, and therefore regular (A, \Rmnum{8}, $\S$8, n1, prop.2). Note that if the $k$-algebra $A$ is finite dimensional, then these conditions are also equivalent to that $A$ is \'etale (A, \Rmnum{5}, p.34, th.4).
\end{example}
\begin{example}\label{algebra local regular separable residue is georegular}
Let $A$ be a regular local $k$-algebra. If the extension $\kappa_A$ of $k$ is separable, then $A$ is geometrically regular. In fact, let $k'$ be a finite extension of $k$. The $A$-module $A_{(k')}$ is then free and finitely generated, hence flat, and each maximal ideal of $A_{(k')}$ is lying over $\m_A$ (\cref{integral ring lying over prime exist}). The ring $\kappa_A\otimes_AA_{(k')}$, being isomorphic to $\kappa_A\otimes_kk'$, is regular by \cref{field ext tensor regular if separable ft}. \cref{Noe ring flat extension regular if fiber regular} then implies that the ring $A_{(k')}$ is regular, so the $k$-algebra $A$ is geometrically regular.
\end{example}
\begin{proposition}\label{algebra Noe georegular normal separable and localization}
Let $k$ be a field and $A$ be a Noetherian $k$-algebra.
\begin{itemize}
\item[(a)] If $A$ is geometrically regular (resp. geometrically normal, resp. separable), so is the ring $S^{-1}A$ for any multiplicative subset $S$ of $A$.
\item[(b)] If $A_\m$ is geometrically regular (resp. geometrically normal, resp. separable) for any maximal ideal $\m$ of $A$, then $A$ is geometrically regular (resp. geometrically normal, resp. separable).
\end{itemize}
\end{proposition}
\begin{proof}
Assertion (a) follows from the fact that $(S^{-1}A)_{(k')}$ is isomorphic to a fraction ring of $A_{(k')}$ for any field extension $k'$ of $k$. Now suppose that $A_\m$ is geometrically regular (resp. geometrically normal, resp. separable) for any maximal ideal $\m$ of $A$. Let $k'$ be a finite extension of $k$, and $\m'$ be a maximal ideal of $A_{(k')}$. It suffices to verify that the local ring $(A_{(k')})_{\m'}$ is regular (resp. normal, resp. reduced).\par
The canonical homomorphism $A\to A_{(k')}$ is finite, so the maximal ideal $\m'$ is lying over a maximal ideal $\m$ of $A$ (\cref{integral ring lying over prime exist}) and $(A_{(k')})_{\m'}$ is isomorphic to a fraction ring of the regular (resp. normal, resp. reduced) ring $(A_\m)_{(k')}$, hence is regular (resp. normal, resp. reduced).
\end{proof}
\begin{lemma}\label{field ext ft separable modification}
Let $k$ be a field and $K$ be a finitely generated extension of $k$. Then there exists an finite extension $L$ of $K$ and a sub-$k$-extension $k'$ of $L$ which is finite and purely inseparable over $k$, such that the extension $L/k'$ is separable.
\end{lemma}
\begin{proposition}\label{algebra es.ft georegular normal tensor with regular normal}
Let $k$ be a field, $A$ and $B$ be $k$-algebras such that one of them is essentially of finite type. Suppose that $A$ is geometrically regular (resp. geometrically normal). If the ring $B$ is regular (resp. normal), so is the ring $A\otimes_kB$.
\end{proposition}
\begin{proof}

\end{proof}
\begin{corollary}\label{algebra es.ft georegular normal tensor product}
Let $k$ be a field. Then the tensor product of two geometrically regular (resp. geometrically normal) $k$-algebras, one of which is essentially of finite type, is a geometrically regular (resp. geometrically normal) $k$-algebra.
\end{corollary}
\begin{proof}
Let $A$ and $B$ be $k$-algebras satisfying the hypotheses of the corollary, and $k'$ be a finite extension of $k$. The ring $B_{(k')}$ is regular (resp. normal), so $A\otimes_kB_{(k')}$ is regular (resp. normal) by \cref{algebra es.ft georegular normal tensor with regular normal}, which is isomorphic to $(A\otimes_kB)_{(k')}$.
\end{proof}
\begin{corollary}\label{algebra es.ft georegular normal arbitrary base change}
Let $k$ be a field, $A$ be a geometrically regular (resp. geometrically normal) $k$-algebra and $K$ be an extension of $k$. Suppose that $A$ is essentially of finite type or the extension $K/k$ is finitely generated.
\begin{itemize}
\item[(a)] The ring $A_{(K)}$ is regular (resp. normal).
\item[(b)] If the extension $K/k$ is separable, the $k$-algebra $A_{(K)}$ is geometrically regular (resp. geometrically normal).
\end{itemize}
\end{corollary}
\begin{proof}
Assertion (a) follows from \cref{algebra es.ft georegular normal tensor with regular normal}, since a field is regular. Assertion (b) follows from \cref{algebra es.ft georegular normal tensor product} and \cref{algebra Artin georegular iff separable}.
\end{proof}
\begin{corollary}\label{algebra es.ft georegular normal iff base change}
Let $k$ be a field, $A$ be a $k$-algebra and $K$ be an extension of $k$. Suppose that the $k$-algebra $A$ is essentially of finite type or the extension $K/k$ is finitely generated. Fot the $k$-algebra $A$ to be geometrically regular (resp. geometrically normal), it is necessary and sufficient that so is the $K$-algebra $A_{(K)}$.
\end{corollary}
\begin{proof}
Suppose that $A$ is geometrically regular (resp. geometrically normal) and let $K'$ be a finite extension of $K$. The ring $K'\otimes_KA_{(K)}$, isomorphic to $K'\otimes_kA$, is regular (resp. normal) by \cref{algebra es.ft georegular normal arbitrary base change}.\par
Suppose conversely that the $K$-algebra $A_{(K)}$ is geometrically regular (resp. geometrically normal) and let $k'$ be a finite extension of $k$. Let $L$ be the composition extension of $k'$ and $K$, then the ring $A_{(L)}$ is identified with $L\otimes_KA_{(K)}$, hence is regular (resp. normal); therefore the ring $A_{(k')}$ is regular (resp. normal) by \cref{algebra separable base change regular normal prop}.
\end{proof}
\begin{corollary}\label{algebra es.ft georegular normal iff base change to perfect}
Let $k$ be a field, $A$ be a $k$-algebra essentially of finite type and $K$ be a extension of $k$ which is a perfect field. For the $k$-algebra $A$ to be geometrically regular (resp. geometrically normal), it is necessary and sufficient that $A_{(K)}$ is regular (resp. normal).
\end{corollary}
\begin{proof}
This follows from \cref{algebra es.ft georegular normal iff base change} and \cref{algebra over perfect geomregular normal iff regular normal}.
\end{proof}
\subsection{Characterization of geometrically regular algebras}
\begin{proposition}\label{algebra es.ft georegular iff tensor regular}
Let $k$ be a field and $A$ be a $k$-algebra essentially of finite type. Let $\mathfrak{I}$ be the kernel of the multiplication homomorpism $\mu:A\otimes_kA\to A$. Then the following conditions are equivalent:
\begin{itemize}
\item[(\rmnum{1})] the $k$-algebra $A$ is geometrically regular;
\item[(\rmnum{2})] for any regular $k$-algebra $R$, the ring $A\otimes_kR$ is regular;
\item[(\rmnum{3})] the ring $A\otimes_kA$ is regular;
\item[(\rmnum{4})] the ideal $\mathfrak{I}$ of $A\otimes_kA$ is completely secant.
\end{itemize}
\end{proposition}
\begin{proof}
We denote by $B$ the ring $A\otimes_kA$, which is endowed with the $A$-algebra structure defined by the homomorphism $\rho:A\to A\otimes_kA$ such that $\rho(x)=x\otimes 1$. Then $\mu$ is a homomorphism of $A$-algebras and induces an isomorphism from $B/\mathfrak{I}$ to $A$. From \cref{algebra es.ft georegular normal tensor with regular normal}, it is clear that (\rmnum{1}) implies (\rmnum{2}), and (\rmnum{2}) implies (\rmnum{3}) by takeing $C=k$ and $C=A$. Also, to see that (\rmnum{3})$\Rightarrow$(\rmnum{4}), note that the $A$-module $B$ is faithfully flat as the base change of the free $k$-module $A$ by $A$. If the ring $B$ is regular, then $A$ is regular by \cref{Noe regular faithfully flat descent}, and the ideal $\mathfrak{I}$ is then completely secant by \cref{Noe ring regular iff maximal ideal completely secant}.\par
Suppose that the ideal $\mathfrak{I}$ is completely secant, we prove that $A$ is regular. Let $\m$ be a maximal ideal of $A$ and $\nu:(A/\m)\otimes_kA\to A/\m$ be the homomorphism induced by $\mu$. The maximal ideal $\n=\ker\nu$ is equal to $\mathfrak{I}((A/\m)\otimes_kA)$, and by applying \cref{Noe ring completely secant ideal iff flat fiber} to the ring homomorphism $B\to(A/\m)\otimes_kA$ and use \cref{depth of module regular sequence and N/yN flat}, we see that the ideal $\n$ of $(A/\m)\otimes_kA$ is completely secant. Therefore (\cref{Noe ring A/I regular completely secant iff}) the local ring $((A/\m)\otimes_kA)_\n$ is regular. Denote by $j:A\to(A/\m)\otimes_kA$ the homomorphism $x\mapsto 1\otimes x$; since $\nu\circ j$ is the canonical homomorphism $A\to A/\m$, we have $j^{-1}(\n)=\m$. Thus $j$ extends to a local homomorphism $j_\m:A_\m\to((A/\m)\otimes_kA)_\n$, which is faithfully flat since $j$ is flat (the homomorphism $k\to A/\m$ is free and $j$ is its base change to $A$). By \cref{Noe regular faithfully flat descent}, the ring $A_\m$ is then regular. This proves that the ring $A$ is regular.\par
Now let $k'$ be an extension of $k$. The kernel of the multiplication homomorphism $\mu':A_{(k')}\otimes_{k'}A_{(k')}\to A_{(k')}$ is none other that $\mathfrak{I}B_{(k')}$, so it is completely secant in $A_{(k')}$ (\cref{Noe ring completely secant ideal iff flat fiber}). The $k'$-algebra $A_{(k')}$ then satisfies condition (\rmnum{4}), and by our preceding arguments, it is therefore regular. This proves (\rmnum{1}) and completes the proof.
\end{proof}
Recall that (A, \Rmnum{3}, p.133-134) the quotient $\mathfrak{I}/\mathfrak{I}^2$ endowed with the $A$-module structure induced by $\rho$, is the differential module $\Omega_{A/k}$ of the $k$-algebra $A$. If the $k$-algebra $A$ is essentially of finite type, the ring $A\otimes_kA$ is Noetherian (\cref{algebra es.ft over Noe is Noe}), so $\Omega_{A/k}$ is a finitely generated $A$-module. We denote by $d:A\to\Omega_{A/k}$ the $k$-linear map which sends an element $x\in A$ to the class of $x\otimes 1-1\otimes x$ in $\Omega_{A/k}$. Then $d$ is a $k$-derivation, and for any $A$-module $M$ and any $k$-derivation $D:A\to M$, there exists a unique $A$-linear map $\delta:\Omega_{A/k}\to M$ such that $D=\delta\circ d$ (A, \Rmnum{3}, p.133-134 prop.18).\par
If $S$ is a multiplicative subset of $A$, the canonical $S^{-1}A$-linear map (A, \Rmnum{3}, p.136) $S^{-1}\Omega_{A/k}\to\Omega_{S^{-1}A/k}$ is bijective: in fact, it suffices to verify that, for any $S^{-1}M$, any $k$-derivation $D:A\to M$ extends uniquely to a $k$-derivation $D:S^{-1}A\to M$, and this follows from (A, \Rmnum{3}, p.123, prop.4).\par
Now let $k$ be a field and $A$ be a local $k$-algebra essentially of finite type. Then by \cref{algebra local es.ft iff local ring of ft}, $A$ is isomorphic to a local ring $B_\mathfrak{P}$, where $B$ is a finite type $k$-algebra and $\mathfrak{P}$ is an ideal of $B$. Then we have $\dim(A)=\height(\mathfrak{P})$, and by \cref{algebra finite over field dimension prop}(c), the dimension of the integral domain $B/\mathfrak{P}$ is given by
\[\dim(B/\mathfrak{P})=\tr.\deg_k(\kappa(\mathfrak{P}))=\tr.\deg_k(\kappa_A).\]
It then follows from \cref{algebra finite over field integral domain height formula} that $n:=\dim(A)+\tr.\deg_k(\kappa_A)$ is equal to $\dim(B)$.\par
We now characterize geometrically regular local $k$-algebras by the differential module $\Omega_{A/k}$. We shall see that this property is equivalent to the freeness of $\Omega_{A/k}$ with rank equal to $n$. For simplicity, we single out the following two lemmas which will be use in our proof.
\begin{lemma}\label{algebra connected base change canonical component dim}

\end{lemma}
\begin{lemma}\label{algebra ft Omega rank equal dim}
Let $k$ 
\end{lemma}
\begin{theorem}\label{algebra es.ft georegular iff Omega free}
Let $k$ be a field and $A$ be a local $k$-algebra essentially of finite type. Then we have
\[[\kappa_A\otimes_A\Omega_{A/k}:\kappa_A]\geq n=\dim(A)+\tr.\deg_k(\kappa_A),\]
and the following conditions are equivalent:
\begin{itemize}
\item[(\rmnum{1})] the $k$-algebra $A$ is geometrically regular;
\item[(\rmnum{2})] the $A$-module $\Omega_{A/k}$ is free of rank $n$;
\item[(\rmnum{3})] $[\kappa_A\otimes_A\Omega_{A/k}:\kappa_A]=n$.
\end{itemize}
\end{theorem}
\begin{proof}
It is clear that (\rmnum{2}) implies (\rmnum{3}), so to prove the theorem, it suffices to prove the implications (\rmnum{1})$\Rightarrow$(\rmnum{2}) and (\rmnum{3})$\Rightarrow$(\rmnum{1}). Choose a finite type $k$-algebra $B$ such that $A$ is isomorphic to $B_\mathfrak{P}$, where $\mathfrak{P}$ is a prime ideal of $B$. Then by the arguments above, we have $n=\dim(B)$, and by replacing $B$ by a ring $B_f$, where $f\in B-\mathfrak{P}$, we can assume that $B$ is connected of dimension $n$.\par
Suppose that the $k$-algebra $A$ is 
\end{proof}
\begin{example}\label{field ext ft separable iff Omega dim=trdeg}
Let $K$ be a finitely generated extension of $k$, \cref{algebra es.ft georegular iff Omega free} then proves that the extension $K/k$ is separable if and only if $\Omega_{K/k}$ has dimension $\tr.\deg_k(K)$, in view of \cref{algebra Artin georegular iff separable}.
\end{example}
\begin{corollary}\label{algebra es.ft georegular point open in Spec}
Let $k$ be a field and $A$ be $k$-algebra essentially of finite type. Then the set of elements $\p$ of $\Spec(A)$ such that the $k$-algebra $A_\p$ is geometrically regular is open in $\Spec(A)$.
\end{corollary}
\begin{proof}
We can suppose that the $k$-algebra $A$ is of finite type. The considered subset is then formed by the prime ideals $\p$ such that $[\kappa(\p)\otimes_k\Omega_{A/k}:\kappa(\p)]\leq\dim_\p(A)$. Now the function $\p\mapsto\dim_{\p}(A)$ is lower semi-continuous by definition, and the function $\p\mapsto[\kappa(\p)\otimes_k\Omega_{A/k}:\kappa(\p)]$ is upper semi-continuous (Nakayama's lemma and \cref{localization map lift to principal open}).
\end{proof}
\begin{corollary}\label{algebra es.ft georegular iff Omea projective minimal prime regular}
Let $k$ be a field and $A$ be a $k$-algebra essentially of finite type. For $A$ to be geometrically regular, it is necessary and sufficient that the $A$-module $\Omega_{A/k}$ is projective and for any minimal prime ideal $\p$ of $A$, the $k$-algebra $A_\p$ is separable.
\end{corollary}
\begin{proof}
Suppose that $A$ is geometrically regular. For any prime ideal of $A$, the $k$-algebra $A_\p$ is geometrically regular, so the $A_\p$-module $\Omega_{A_\p/k}$ is free (\cref{algebra es.ft georegular iff Omega free}). Moreover, if $\p$ is minimal, the local $k$-algebra $A_\p$ is also Artinian, hence is a separable field extension of $k$ (\cref{algebra Artin georegular iff separable}).\par
Conversely, suppose that the $A$-module $\Omega_{A/k}$ is projective and $A_\p$ is separable for any minimal prime $\p$ of $A$. Now let $\p$ be a prime ideal of $A$, and $\q$ be a minimal prime contained in $\p$. Since the $A_\p$-module $(\Omega_{A/k})_\p$ is free by \cref{module finite projective iff}, we have
\[[\kappa(\p)\otimes_A\Omega_{A/k}:\kappa(\p)]=[\kappa(\q)\otimes_A\Omega_{A/k}:\kappa(\q)].\]
The $k$-algebra $A_\q$ is Artinian and separable, hence geometrically regular (\cref{algebra Artin georegular iff separable}). \cref{algebra es.ft georegular iff Omega free} then implies that
\[[\kappa(\q)\otimes_A\Omega_{A/k}:\kappa(\q)]=\tr.\deg_k(\kappa(\q)).\]
It then follows from \cref{algebra es.ft georegular iff Omega free} that the $k$-algebra $A_\p$ is geometrically regular, whence the corollary.
\end{proof}
\begin{remark}\label{algebra es.ft georegular iff Omega projective Q(A) separable}
Suppose that the $k$-algebra $A$ essentially of finite type is geometrically regular. Then the total fraction ring $Q(A)$ of $A$ is identified with the product of $A_\p$, where $\p$ runs through minimal prime ideals of $A$ (\cref{Noe ring minimal prime and Ass localization prop}); it is then a separable $k$-algebra. Conversely, suppose that $Q(A)$ is a separable $k$-algebra; for any minimal prime ideal $\p$ of $A$, the $k$-algebra $A_\p$ is a fraction ring of $Q(A)$ (\cref{Noe ring minimal prime and Ass localization prop}), hence a separable $k$-algebra (\cref{algebra Noe georegular normal separable and localization}). If the $A$-module $\Omega_{A/k}$ is also projective, it then follows from \cref{algebra es.ft georegular iff Omea projective minimal prime regular} that the $k$-algebra $A$ is geometrically regular.
\end{remark}
\section{Smooth algebras}
\subsection{Infinitesimal liftings}
Let $k$ be a ring, $C$ be a $k$-algebra and $N$ be a square zero ideal of $C$. We denote by $\pi:C\to C/N$ the canonical homomorphism; since $N^2=0$, the $C$-module $N$ has a natrual $C/N$-module structure. Let $A$ be an $k$-algebra and $\varphi:A\to C/N$ be a homomorphism of $k$-algebra. Endow $N$ with the $A$-module structure induced by $\varphi$, we define a \textbf{lifting} of $\varphi$ (to $C$) to be a homomorphism $\tilde{\varphi}:A\to C$ of $k$-algebras such that $\pi\circ\tilde{\varphi}=\varphi$.
\[\begin{tikzcd}[row sep=12mm,column sep=12mm]
k\ar[r]\ar[d]&C\ar[d,"\pi"]\\
A\ar[r,"\varphi",pos=.6]\ar[ru,dashed,"\tilde{\varphi}"{anchor=south}]&C/N
\end{tikzcd}\]

\begin{proposition}\label{algebra derivation act on lifting}
If $\varphi$ admits a lefting, the map $(\delta,\tilde{\varphi})\mapsto\delta+\tilde{\varphi}$ defines a simply transitive action of the group $\Der_k(A,N)$ of $k$-derivations from $A$ to $N$ on the set of leftings of $\varphi$.
\end{proposition}
\begin{proof}
Let $\tilde{\varphi}_0:A\to C$ be a lifting of $\varphi$. The map $\delta\mapsto\delta+\tilde{\varphi}_0$ induces a bijection from the set of maps $A\to N$ to the set of maps $\tilde{\varphi}:A\to C$ such that $\pi\circ\tilde{\varphi}=\varphi$. Fix $\delta$, and put $\tilde{\varphi}=\delta+\tilde{\varphi}_0$. For $\tilde{\varphi}$ to be a homomorphism of $k$-algebras, it is necessary and sufficient that $\delta$ is a $k$-derivation: in fact, for $x,y\in A$ and $\lambda\in k$, we have the relations
\begin{align*}
\tilde{\varphi}(x+y)-\tilde{\varphi}(x)-\tilde{\varphi}(y)&=\delta(x+y)-\delta(x)-\delta(y)\\
\tilde{\varphi}(\lambda x)-\lambda\tilde{\varphi}(x)&=\delta(\lambda x)-\lambda\delta(x)\\
\tilde{\varphi}(xy)-\tilde{\varphi}(x)\tilde{\varphi}(y)&=\delta(xy)-\delta(x)\delta(y)-\delta(x)\tilde{\varphi}_0(y)-\tilde{\varphi}_0(x)\delta(y)\\
&=\delta(xy)-\varphi(x)\delta(y)-\varphi(y)\delta(x),
\end{align*}
the last equality resulting from the fact that $N$ has zero square.
\end{proof}
\begin{example}\label{algebra lifting of square zero extension by module}
Let $B$ be a $k$-algebra, $N$ be a $B$-module. In this section, we often endow the $k$-module $B\oplus N$ with the $k$-algebra structure defined by $(b,x)(d,y)=(bd,by+dx)$; $N$ is then a square zero ideal of $B\oplus N$. Let $\varphi:A\to B$ be a homomorphism of $k$-algebras, then the liftings of $\varphi$ to $B\oplus N$ are the maps $x\mapsto(\varphi(x),\delta(x))$, where $\delta$ is a $k$-derivation of $A$ to $N$.
\end{example}
Let $\Omega_{A/k}$ be the module of $k$-differentials of the ring $A$, and $d:A\to\Omega_{A/k}$ be the universal $k$-derivation. Recall that for any $A$-module $M$, the map $v\mapsto v\circ d$ is an $A$-linear isomorphism from $\Hom_A(\Omega_{A/k},M)$ to $\Der_k(A,M)$. Let $\mathfrak{I}$ be an ideal of $A$. By (A, \Rmnum{3}, p.137), we have an exact sequence of $A/\mathfrak{I}$-modules
\[\begin{tikzcd}
\mathfrak{I}/\mathfrak{I}^2\ar[r,"\bar{d}"]&(A/\mathfrak{I})\otimes_A\Omega_{A/k}\ar[r]&\Omega_{(A/\mathfrak{I})/k}\ar[r]&0
\end{tikzcd}\]
where $\bar{d}$ is the homomorphism induced by the restriction of $d$ to $\mathfrak{I}$.\par
Let $\rho:A\to A/\mathfrak{I}^2$ and $\pi:A/\mathfrak{I}^2\to A/\mathfrak{I}$ be the canonical homomorphisms. If
\[\alpha:(A/\mathfrak{I})\otimes_A\Omega_{A/k}\to\mathfrak{I}/\mathfrak{I}^2\]
is a $k$-linear map, we then have an associated $k$-linear map $h_\alpha:A\to A/\mathfrak{I}^2$ defined by
\begin{align*}
h_\alpha(x)=\rho(x)-\alpha(1\otimes dx).
\end{align*}
If $\alpha$ is a \textit{retraction} of $\bar{d}$ (that is, $\alpha\circ\bar{d}=\id_{\mathfrak{I}/\mathfrak{I}^2}$), then the map $h_\alpha$ factors into a $k$-linear map $h_\alpha:A/\mathfrak{I}\to A/\mathfrak{I}^2$. On the other hand, any $k$-linear map $h:A/\mathfrak{I}\to A/\mathfrak{I}^2$ induces a $k$-linear map $\psi_h:(A/\mathfrak{I})\oplus(\mathfrak{I}/\mathfrak{I}^2)\to A/\mathfrak{I}^2$ given by $(x,y)\mapsto h(x)+y$.
\begin{proposition}\label{algebra lifting of A to A/I char}
Endow $(A/\mathfrak{I})\oplus(\mathfrak{I}/\mathfrak{I}^2)$ with the $k$-algebra structure defined in \cref{algebra lifting of square zero extension by module}. Then the maps $\alpha\mapsto h_\alpha$ and $h\mapsto\psi_h$ induce bijections between the following sets:
\begin{itemize}
\item[(\rmnum{1})] the set of $A/\mathfrak{I}$-linear retractions $\alpha$ of $\bar{d}$;
\item[(\rmnum{2})] the set of $k$-algebra homomorphisms $h:A/\mathfrak{I}\to A/\mathfrak{I}^2$ such that $\pi\circ h=\id_{A/\mathfrak{I}}$.
\item[(\rmnum{3})] the set of $k$-algebra isomorphisms $\psi:(A/\mathfrak{I})\oplus(\mathfrak{I}/\mathfrak{I}^2)\to A/\mathfrak{I}^2$ such that $\pi\circ\psi=\mathrm{pr}_1$ and $\psi(0,z)=z$ for $z\in\mathfrak{I}/\mathfrak{I}^2$.
\end{itemize}
\end{proposition} 
\begin{proof}
We apply \cref{algebra derivation act on lifting} to $C=A/\mathfrak{I}^2$ and $N=\mathfrak{I}/\mathfrak{I}^2$. Let $\varphi:A\to A/\mathfrak{I}$ be the canonical surjection, then the homomorphism $\rho:A\to A/\mathfrak{I}^2$ is a lifting of $\varphi$ to $A/\mathfrak{I}^2$.
\[\begin{tikzcd}
k\ar[r]\ar[d]&A/\mathfrak{I}^2\ar[d,"\pi"]\\
A\ar[r,"\varphi",pos=.6]\ar[ru,dashed,"\rho"{anchor=south}]&A/\mathfrak{I}
\end{tikzcd}\]
The $A$-module $\Hom_{A/\mathfrak{I}}((A/\mathfrak{I})\otimes_A\Omega_{A/k},\mathfrak{I}/\mathfrak{I}^2)$ is identified with $\Hom_A(\Omega_{A/k},\mathfrak{I}/\mathfrak{I}^2)$, so the map $\alpha\mapsto h_\alpha$ is a bijection from this set to the set of liftings of $\varphi$ to $A/\mathfrak{I}^2$ (\cref{algebra derivation act on lifting}). Now for $x\in\mathfrak{I}$ we have $1\otimes dx=\bar{d}(\rho(x))$, so for the map $h_\alpha$ to factor through $A/\mathfrak{I}$, it is necessary and sufficient that $\alpha\circ\bar{d}$ is the identity on $\mathfrak{I}/\mathfrak{I}^2$, which means that $\alpha$ is a retraction of $\bar{d}$. This proves the correspondence of the sets in (\rmnum{1}) and (\rmnum{2}) under the map $\alpha\mapsto h_\alpha$.\par
The map $h\mapsto\psi_h$ is a bijection from the set of $k$-linear homomorphisms $A/\mathfrak{I}\to A/\mathfrak{I}^2$ to the set of $k$-linear homomorphisms $\psi:(A/\mathfrak{I})\oplus(\mathfrak{I}/\mathfrak{I}^2)\to A/\mathfrak{I}^2$ such that $\psi(0,z)=z$ for $z\in\mathfrak{I}/\mathfrak{I}^2$. For the equality $\pi\circ\psi_h=\mathrm{pr}_1$ to be true, it is necessary and sufficient that $\pi\circ h=\id_{A/\mathfrak{I}}$, which means $h(\pi(z))\equiv z$ mod $\mathfrak{I}/\mathfrak{I}^2$ for any $z\in A/\mathfrak{I}^2$. Suppose that this condition is satisfied, then for $h$ to be a ring homomorphism, it is necessary and sufficient that $\psi_h$ is a ring homomorphism. Moreover, in this case $\psi_h$ is bijective: the inverse map is given by $z\mapsto(\pi(z),z-h(\pi(z)))$, where $z\in A/\mathfrak{I}^2$.  This proves the correspondence of the sets in (\rmnum{2}) and (\rmnum{3}).
\end{proof}
\subsection{Formally smooth algebras}
Let $k$ be a ring and $A$ be a linearly topologized $k$-algebra. We say that $A$ is \textbf{formally smooth over $\bm{k}$}, or that a \textbf{formally smooth $\bm{k}$-algebra}, if it satisfies the following condition: for any $k$-algebra $C$ and a square zero ideal $N$ of $C$ ( endowed with the discrete topology), any continuous homomorphism from $A$ into the $k$-algebra $C/N$ can be lifted into a continuous homomorphism of $A$ into the $k$-algebra $C$. We say that a $k$-algebra $A$ is \textbf{formally smooth} if it is formally smooth endowed with the discrete topology, which is also the $(0)$-adic topology. In this case, it is then formally smooth for any $\mathfrak{I}$-adic topology.
\begin{remark}\label{algebra formally smooth adic ideal inclusion}
Let $k$ be a ring, $A$ be a $k$-algebra and $\mathfrak{I}$ be an ideal of $A$. Endow $A$ with the $\mathfrak{I}$-adic topology. Let $C$ be an $k$-algebra, $N$ be a square zero ideal of $C$, and endow $C$ and $C/N$ with the discrete topology. Let $\varphi:A\to C/N$ be a continuous homomorphism of $k$-algebras. Then any lifting $\tilde{\varphi}:A\to C$ of $\varphi$ is continuous: in fact, there exists an integer $n>0$ such that $\varphi(\mathfrak{I}^n)=0$, and we have $\tilde{\varphi}(\mathfrak{I}^n)\sub N$, whence $\tilde{\varphi}(\mathfrak{I}^{2n})\sub N^2=0$. From this, we conclude that if $A$ is formally smooth for the $\mathfrak{I}$-adic topology, it is also formally smooth for the $\mathfrak{I}'$-adic topology for any ideal $\mathfrak{I}'$ containing $\mathfrak{I}$ (note that a homomorphism from $A$ to a discrete $k$-algebra is continuous if and only if its kernel is open in $A$).
\end{remark}
\begin{example}\label{algebra formally smooth A/I conormal sequence split exact}
Let $k$ be a ring, $A$ be a $k$-algebra and $\mathfrak{I}$ be an ideal of $A$. If the $k$-algebra $A/\mathfrak{I}$ is formally smooth (for the discrete topology), then the identity map on $A/\mathfrak{I}$ admits a lifting to $A/\mathfrak{I}^2$, so the sets of \cref{algebra lifting of A to A/I char} are nonempty. In particular, the sequence
\[\begin{tikzcd}
0\to\mathfrak{I}/\mathfrak{I}^2\ar[r,"\bar{d}"]&(A/\mathfrak{I})\otimes_A\Omega_{A/k}\ar[r]&\Omega_{(A/\mathfrak{I})/k}\ar[r]&0
\end{tikzcd}\]
is then exact and split.
\end{example}
\begin{remark}\label{algebra formally smooth derivation k to M extension}
Let $k$ be a ring, $A$ be a linearly topologized formally smooth $k$-algebra, $M$ be an $A$-module whose annihilator is open in $A$. Then \textit{any derivation $\delta:k\to M$ extends to a derivation $\tilde{\delta}:A\to M$}. In fact, put $B=A/\Ann(M)$, the map $\lambda\mapsto(\lambda 1_B,\delta(\lambda))$ then defines an algebra homomorphism from $k$ to $B\oplus M$, which gives a $k$-algebra structure on $B\oplus M$. The canonical surjection $\varphi:A\to B$ is continuous, hence admits a lifting $\tilde{\varphi}:A\to B\oplus M$. By \cref{algebra lifting of square zero extension by module}, $\mathrm{pr}_2\circ\tilde{\varphi}$ is then a derivation from $A$ to $M$ which extends $\delta$.
\end{remark}
\begin{proposition}\label{algebra formally smooth transitive product completion prop}
Let $k$ be a ring.
\begin{itemize}
\item[(a)] Let $A$ and $B$ be linearly topologized $k$-algebras and $\rho:A\to B$ be a continuous homomorphism of $k$-algebras. If $A$ is formally smooth over $k$ and $B$ is formally smooth over $A$, then $B$ is formally smooth over $k$.
\item[(b)] The product of a finite family of linearly topologized $k$-algebras is formally smooth over $k$ if and only if each $k$-algebra is formally smooth over $k$.
\item[(c)] Let $A$ be a linearly topologized $k$-algebra and $\widehat{A}$ be the completion of $A$. For $A$ to be formally smooth over $k$, it is necessary and sufficient that $\widehat{A}$ is formally smooth over $k$.
\end{itemize}
\end{proposition}
\begin{proof}
Let $C$ be a $k$-algebra, $N$ be a square zero ideal of $C$, and $\pi:C\to C/N$ be the canonical surjection. Endow $C$ and $C/N$ the discrete topology. In the situation of (a), let $\psi:B\to C/N$ be a continuous homomorphism of $k$-algebras. Since $A$ is formally smooth over $k$, there exists a continuous homomorphism of $k$-algebras $\tilde{\varphi}:A\to C$ such that $\pi\circ\tilde{\varphi}=\psi\circ\rho$.
\[\begin{tikzcd}
&&C\ar[d,"\pi"]\\
A\ar[rru,"\tilde{\varphi}"{anchor=south}]\ar[r,swap,"\rho"]&B\ar[ru,swap,"\tilde{\psi}"{anchor=north},pos=.6]\ar[r,swap,"\psi"]&C/N
\end{tikzcd}\]
Consider $C$ and $C/N$ as $A$-algebras via the homomorphism $\tilde{\varphi}$, so that $\psi$ is a homomorphism of $A$-algebras. Since $B$ is formally smooth over $A$, there exsits a continuous homomorphism $\tilde{\psi}:B\to C$ of $A$-algebras such that $\pi\circ\tilde{\psi}=\psi$, which proves (a). As for the assertion of (b), it suffices to note that giving a continuous $k$-homomorphism from a product $A=\prod_iA_i$ to $C$ (resp. $C/N$) is equivalent to giving $n$ continuous $k$-homomorphisms $A_i\to C$ (resp. $A_i\to C/N$) and that any continuous $k$-homomorphism $A_i\to C$ (resp. $A_i\to C/N$) gives by composition a continuous $k$-homomorphism $A\to A_i\to C$ (resp. $A\to A_i\to C/N$).\par
Finally, let $i:A\to\widehat{A}$ be the canonical homomorphism. For any ring $D$, endowed with the discrete topology, the map which assocaites a continuous homomorphism $f:\widehat{A}\to D$ the continuous homomorphism $f\circ i:A\to D$ is bijective, so the assertion of (c) follows.
\end{proof}
\begin{example}
Assertion (c) of \cref{algebra formally smooth transitive product completion prop} is applicable in particular if the topology of $A$ is the $\mathfrak{I}$-adic topology, where $\mathfrak{I}$ is a finitely generated ideal; the closure $\widehat{\mathfrak{I}}$ of $\mathfrak{I}$ in $\widehat{A}$ is then equal to $\mathfrak{I}\widehat{A}$ and the topology of $\widehat{A}$ is the $\widehat{\mathfrak{I}}$-topology (\cref{filtration I-adic completion of finite ideal prop}). Therefore, it is equivalent to say that $A$ is formally smooth for the $\mathfrak{I}$-adic topology or that the completion $\widehat{A}$ is formally smooth for the $\widehat{\mathfrak{I}}$-adic topology.
\end{example}
\begin{proposition}\label{algebra formally smooth localization tensor prop}
Let $k$ be a ring, $A$ and $B$ be $k$-algebra, $\mathfrak{I}$ be an ideal of $A$ and $\mathfrak{K}$ be an ideal of $B$.
\begin{itemize}
\item[(a)] Let $S$ be a multiplicative subset of $A$ and $T$ be a multiplicative subset of $k$ whose image in $A$ is contained in $S$. If $A$ is formally smooth over $k$ for the $\mathfrak{I}$-adic topology, $S^{-1}A$ is formally smooth over $T^{-1}k$ for the $S^{-1}\mathfrak{I}$-adic topology.
\item[(b)] Let $k'$ be a $k$-algebra. If $A$ is formally smooth over $k$ for the $\mathfrak{I}$-adic topology, the $k'$-algebra $A_{(k')}$ is formally smooth for the $\mathfrak{I}A_{(k')}$-algebra.
\item[(c)] Let $\mathfrak{R}$ be the ideal of $A\otimes_kB$ generated by the image of $\mathfrak{I}\otimes_kB$ and $A\otimes_k\mathfrak{K}$. If $A$ and $B$ are formally smooth over $k$ for the $\mathfrak{I}$-adic topology and $\mathfrak{K}$-adic topology, respectively, then the $k$-algebra $A\otimes_kB$ is formally smooth for the $\mathfrak{R}$-adic topology.
\end{itemize}
\end{proposition}
\begin{proof}
Under the hypotheses of (a), let $C$ be a $T^{-1}k$-algebra, $N$ be a square zero ideal of $C$; endow $C$ and $C/N$ the discrete topology and let $\pi:C\to C/N$ be the canonical homomorphism. Let $\varphi:S^{-1}A\to C/N$ be a continuous homomorphism of $T^{-1}k$-algebras (for the $S^{-1}\mathfrak{I}$-topology). Let $i:A\to S^{-1}A$ be the canonical homomorphism, then $\varphi\circ i$ is a continuous homomorphism from $A$ to $C/N$ (for the $\mathfrak{I}$-adic topology), so it admits a lefting $\tilde{\varphi}_0:A\to C$. The elements of $\tilde{\varphi}_0(S)$ are invertible modulo $N$, hence are invertible ($N$ has zero square), therefore there exists a ring  homomorphism $\tilde{\varphi}:S^{-1}A\to C$ such that $\tilde{\varphi}\circ i=\tilde{\varphi}_0$, and the homomorphism $\tilde{\varphi}$ is $T^{-1}k$-linear. We have $\pi\circ\tilde{\varphi}\circ i =\varphi\circ i$, whence $\pi\circ\tilde{\varphi}=\varphi$, and $\tilde{\varphi}$ is then a lifting of $\varphi$.\par
We now turn to the hypotheses of (b). Let $C$ be a $k'$-algebra, $N$ be a square zero ideal of $C$; endow $C$ and $C/N$ the discrete topology. Let $\varphi:A_{(k')}\to C/N$ be a homomorphism of $k'$-algebras, continuous for the $\mathfrak{I}A_{(k')}$-adic topology. Let $i:A\to A_{(k')}$ be the canonical homomorphism. The map $\varphi\circ i$ is then a homomorphism of $k$-algebras from $A$ to $C/N$, continuous for the $\mathfrak{I}$-adic topology; if $A$ is formally smooth over $k$ for the $\mathfrak{I}$-adic toplogy, $\varphi\circ i$ admits a lifting $\tilde{\psi}:A\to C$. The homomorphism $\tilde{\varphi}:A_{(k')}\to C$ induced by $\tilde{\psi}$ is then a lifting of $\varphi$.\par
Finally, consider the hypotheses of (c). The $B$-algebra $A\otimes_kB$ is formally smooth for the $\mathfrak{I}(A\otimes_AB)$-adic topology by (b), hence for the $\mathfrak{R}$-adic topology; moreover the canonical homomorphism $B\to A\otimes_kB$ is continuous if we endow $B$ with the $\mathfrak{K}$-adic topology and $A\otimes_kB$ with the $\mathfrak{R}$-adic topology. Assertion (c) then follows from \cref{algebra formally smooth transitive product completion prop}(a).
\end{proof}
\begin{example}\label{algebra local ring formally smooth}
Let $A$ be a ring and $\p$ be a prime ideal of $A$. Then the local ring $A_\p$ is formally smooth over $A$. To see this, let $C$ be an $A$-algebra, $N$ be a square zero ideal of $C$, and $\pi:C\to C/N$ be the canonical homomorphism. If $\varphi:A_\p\to C/N$ is a homomorphism of $A$-algebras, then the image of elements of $\varphi(A-\p)$ are invertible in $C$ mod $N$, hence invertible. There then exists a unique lifting $\tilde{\varphi}:A_\p\to C$ of $\varphi$, so $A_\p$ is formally smooth over $A$ (it is in fact formally \'etale over $A$).
\end{example}
\begin{example}\label{algebra symmetric of projective is formally smooth}
Let $k$ be a ring, and $P$ be a projective $k$-module. The symmetric $k$-algebra $\bm{S}_k(P)$ is formally smooth for the discrete topology, and a fortiori for that defined by its graduation. In fact, for any $k$-algebra $C$ and any ideal $N$ of $C$, any algebra homomorphism from $\bm{S}_k(P)$ to $C$ (resp. resp. $C/N$) corresponds to $k$-linear maps from $P$ to $C$ (resp. $C/N$), and the canonical map $\Hom_k(P,C)\to\Hom_k(P,C/N)$ is surjective since $P$ is projective. In particular, by \cref{algebra formally smooth transitive product completion prop} the $k$-algebra $\widehat{\bm{S}}_k(P)=\prod_{n\in\N}\bm{S}_k^n(P)$ is formally smooth (for the product topology of the discrete topology over $\bm{S}_k^n(P)$).
\end{example}
\begin{example}\label{algebra polynomial ring formally smooth}
For any family $\bm{T}=(T_i)_{i\in I}$ of indeterminates, the $k$-algebra $k[\bm{T}]$ and $k\llbracket\bm{T}\rrbracket$, endowded with the canonical topology, are formally smooth over $k$: this follows from \cref{algebra symmetric of projective is formally smooth}. If $k$ is a field, the extension $k(\bm{T})$ is formally smooth over $k$ by \cref{algebra formally smooth localization tensor prop}.
\end{example}
\begin{example}\label{algebra k[T]/f formally smooth iff root lifting}
Let $f\in k[T]$ be a polynomial in one indeterminates. For the $k$-algebra $k[T]/(f)$ to be formally smooth over $k$, it is necessary and sufficient that the following property is satisfied: for any $k$-algebra $C$ and any square zero ideal $N$ of $C$, any root of $f$ in $C/N$ extends to a root of $f$ in $C$. This is the case if $f$ and $f'$ generate the unit ideal of $k[T]$; to see this, let $\alpha$ be a root of $f$ in $C/N$ and $a$ be an element of $C$ whose image in $C/N$ is equal to $\alpha$. Then $f(a)$ belongs to $N$ and therefore $f'(a)$ is invertible in $C$; the element $b=a-f'(a)^{-1}f(a)$ then has image $\alpha$ in $C$. Since $f'(a)^{-1}f(a)$ has zero square, we have
\begin{equation*}
f(b)=f(a)-f'(a)f'(a)^{-1}f(a)=0.
\end{equation*}
It is worth noting that in this case the lifting of any homomorphism $k[T]/(f)\to C$ is necessarily unique, which means $k[T]/(f)$ is always formally unramified.
\end{example}
\begin{theorem}[\textbf{Cohen}]\label{field ext formally smooth if separable}
Let $k$ be a field anf $K$ be a separable extension of $k$. Then $K$ is a formally smooth $k$-algebra.
\end{theorem}
\begin{proof}
Let $C$ be a $k$-algebra, $N$ be a square zero ideal of $C$, $\pi:C\to C/N$ be the canonical homomorphism, and $\varphi:K\to C/N$ be a homomorphism of $k$-algebras. In order to construct a lifting of $\varphi$, we distinguish into two cases.\par
Suppose first that $k$ has characteristic zero. Consider the couple $(K',\tilde{\varphi}')$, where $K'$ is a sub-extension of $K$ and $\tilde{\varphi}:K'\to C$ is a lifting of the restriction of $\varphi$ to $K'$. The set of such couples, endowed with the order defined by extension relation, is inductive, so by Zorn lemma there exists a maximal couple $(K',\tilde{\varphi}')$. We now prove that $K'=K$, so let $x\in K-K'$. If $x$ is transcendantal over $K'$, then the $K'$-algebra $K'(x)$ is formally smooth (\cref{algebra polynomial ring formally smooth}). On the other hand, if $x$ is algebraic over $K'$, its minimal polynomial $f\in K'[T]$ is separable, so it is coprime to its derivative (\cref{poloynomial separable iff f' coprime to f}) and the field $K'(x)$, isomorphic to $K'[T]/(f)$, is then formally smooth (\cref{algebra k[T]/f formally smooth iff root lifting}). In both cases, $K'(x)$ is formally smooth over $K'$, and there exists an extension of $\tilde{\varphi}'$ to $K'(x)$ which lifts the restriction of $\varphi$ to $K'(x)$; this contradicts the maximality of $(K',\tilde{\varphi}')$.\par
Now suppose that $k$ has characteristic $p>0$. Consider the Frobenius homomorphism $F:C\to C$ defined by $F(x)=x^p$. Since $F(x)=0$ for $x\in N$, so there exists a unique homomorphism $\lambda:C/N\to C$ such that $\lambda\circ\pi=F$. We have $\pi(\lambda(\pi(x)))=\pi(x^p)=\pi(x)^p$, and since $\pi$ is surjective, this implies $\pi(\lambda(z))=z^p$ for any element $z\in C/N$. Moreover, let $f:K\to K^p$ be the isomorphic $y\mapsto y^p$ and $f^{-1}:K^p\to K$ be the inverse isomorphism. Let $g:K^p\to C$ be the composition homomorphism
\[\begin{tikzcd}
K^p\ar[r,"f^{-1}"]&K\ar[r,"\varphi"]&C/N\ar[r,"\lambda"]&C
\end{tikzcd}\]
For any $x\in K$, we have $g(x^p)=\lambda(\varphi(x))$, and the map $g$ is $k$-linear because $\lambda(\alpha z)=\alpha^p\lambda(z)$ for $\alpha\in k$ and $z\in C/N$. Since the extension $K/k$ is separable, $k(K^p)$ is identified with $k\otimes_{k^p}K^p$ (A, \Rmnum{5}, p.119, remark), so there exists a unique $k$-homomorphism $\tilde{g}:k(K^p)\to C$ which extends $g$. Let $(a_i)_{i\in I}$ be a $p$-basis for $K$ over $k(K^p)$ (A, \Rmnum{5}, p.98, theorem 2); for any $i\in I$, choose an element $b_i\in C$ such that $\pi(b_i)=\varphi(a_i)$. We have
\[\tilde{g}(a_i^p)=g(a_i^p)=\lambda(\varphi(a_i))=\lambda(\pi(b_i))=b_i^p.\]
By (A, \Rmnum{5}, p.94, remark), there is then a unique homomorphism $\tilde{\varphi}:K\to C$ which extends $\tilde{g}$ and satisfies $\tilde{\varphi}(a_i)=b_i$ for each $i$. Note that $\pi(\tilde{\varphi}(a_i))=\pi(b_i)=\varphi(a_i)$ and for each $x\in K$ we have
\begin{gather*}
\pi(\tilde{\varphi}(x^p))=\pi(h(x^p))=\pi(g(x^p))=\pi(\lambda(\varphi(x)))=\varphi(x^p);
\end{gather*}
so $\pi\circ\tilde{\varphi}=\varphi$ and this completes the proof of the theorem.
\end{proof}
\begin{corollary}\label{algebra formally smooth over separable extension}
Let $k$ be a field, $K$ be a separable extension of $k$ and $A$ be a linearly topologized $K$-algebra. If $A$ is formally smooth over $K$, it is formally smooth over $k$.
\end{corollary}
\begin{proof}
This follows from \cref{field ext formally smooth if separable} and \cref{algebra formally smooth transitive product completion prop}.
\end{proof}
\begin{remark}
Let $k$ be a field, then any \'etale $k$-algebra is formally smooth (A, \Rmnum{5}, p.34, th.4(d)). Later we shall see that any field extension that is formally smooth is absolutely regular, hence separable.
\end{remark}
We conclude this paragraph by establishing a stronger lifting property for formally smooth algebras. For this, we let $k$ be a ring, $C$ be a $k$-algebra, $(C_n)_{n\in\Z}$ be a filtration of $C$ which is compatible with the $k$-algebra structure and such that $C_0=C$. Suppose that $C$ is separated and complete with respect to this filtration, so that the canonical homomorphism $C\to\llim C/C_n$ is a homeomorphism. Let $m>0$ be an integer and $\pi:C\to C/m$ be the canonical homomorphism.
\begin{proposition}\label{algebra formally smooth lifting for complete algebras}
Let $A$ be a formally smooth lineraly topologized $k$-algebra. Then any continuous $k$-homomorphism $\varphi:A\to C/C_m$ admits a continuous lifting to $C$. 
\end{proposition}
\begin{proof}
For any integer $n>m$, let $\pi_n:C/C_n\to C/C_{n-1}$ be the canonical homomorphism. Since $C$ is identified with the limit $\llim C/C_n$, giving a continuous lifting of $\varphi$ to $C$ is equivalent to giving a family $(\varphi_n)_{n>m}$ of continuous $k$-homomorphisms $\varphi_n:A\to C/C_n$ satisfying $\pi_n\circ\varphi_n=\varphi_{n-1}$. By induction on $m$, we are then reduced to proving the statement when $C_{m+1}=0$. The ideal $C_m$ then has square zero (since $2m\geq m+1$), and we can use the hypothesis that $A$ is formally smooth over $k$.
\end{proof}
\begin{remark}\label{algebra formally smooth lifting for nilpotent ideal}
The result of \cref{algebra formally smooth lifting for complete algebras} if the filtration is defined by a nilpotent ideal $N$ of $C$: that is, if $C_n=N^n$ and the ideal $N$ is nilpotent. If $A$ is a formally smooth lineraly topologized $k$-algebra, we then conclude that any $k$-homomorphism $\varphi:A\to C/N$ admits a lifting to $C$. This can also be considered as a slightly stronger (but in fact equaivalent) definition for formally smoothness, and justifies the name "infinitesimal lifting".
\end{remark}
\subsection{Criterion by associated graded algebras}
\begin{theorem}\label{algebra A/I formally smooth iff I completely secant}
Let $k$ be a ring, $A$ be a $k$-algebra and $\mathfrak{I}$ be an ideal of $A$ such that the $k$-algebra $A/\mathfrak{I}$ is formally smooth. Endow $A$ with the $\mathfrak{I}$-adic topology, the following conditions are equivalent:
\begin{itemize}
\item[(\rmnum{1})] the $k$-algebra $A$ is formally smooth;
\item[(\rmnum{2})] the $A/\mathfrak{I}$-module $\mathfrak{I}/\mathfrak{I}^2$ is projective and the canonical homomorphism
\[\beta_\mathfrak{I}:\bm{S}_{A/\mathfrak{I}}(\mathfrak{I}/\mathfrak{I}^2)\to\gr_\mathfrak{I}(A)\]
is bijective;
\item[(\rmnum{3})] the $A/\mathfrak{I}$-module $\mathfrak{I}/\mathfrak{I}^2$ is projective and the completion of $A$ is $k$-linearly homeomorphic to the completion of the graded algebra $\bm{S}_{A/\mathfrak{I}}(\mathfrak{I}/\mathfrak{I}^2)$.
\end{itemize}
If $A$ is Noetherian, the these conditions are equivalent to
\begin{itemize}
\item[(\rmnum{4})] the ideal $\mathfrak{I}$ is completely secant.
\end{itemize}
\end{theorem}
\begin{proof}
We first show that (\rmnum{3}) implies (\rmnum{1}): in fact, under the hypotheses of (\rmnum{3}), the algebra $\bm{S}_{A/\mathfrak{I}}(\mathfrak{I}/\mathfrak{I}^2)$, endowed with the topology given by its graduation, is formally smooth over $A/\mathfrak{I}$ (\cref{algebra symmetric of projective is formally smooth}), hence over $k$ (\cref{algebra formally smooth transitive product completion prop}(a)); assertion (\rmnum{1}) then follows from \cref{algebra formally smooth transitive product completion prop}(c). Also, if $A$ is Noetherian, then (\rmnum{2}) is equivalent to (\rmnum{4}) by \cref{Noe ring completely secant ideal iff alpha and beta isomorphism}.\par
Let $\widehat{A}$ be the completion of $A$ and $\widehat{\mathfrak{I}}$ be the completion of $\mathfrak{I}$. The canonical homomorphism $i:A\to\widehat{A}$ induces an isomorphism $A/\mathfrak{I}\to\widehat{A}/\widehat{\mathfrak{I}}$, which admits a lifting $\varphi:A/\mathfrak{I}\to\widehat{A}$ (\cref{algebra formally smooth lifting for complete algebras}). Let $\lambda:\widehat{\mathfrak{I}}\to\mathfrak{I}/\mathfrak{I}^2$ be the surjection induced by the canonical isomorphism $\mathfrak{I}/\mathfrak{I}^2\to\widehat{\mathfrak{I}}/\widehat{\mathfrak{I}}^2$. For $a\in A$ and $z\in\widehat{\mathfrak{I}}$, we then have $\varphi(\bar{a})\equiv i(a)$ mod $\widehat{\mathfrak{I}}$, whence $\varphi(\bar{a}z)\equiv i(a)z$ mod $\widehat{\mathfrak{I}}^2$ and
\[\lambda(\varphi(\bar{a})z)=\lambda(i(a)z)=\bar{a}\lambda(z).\]
In other words, $\lambda$ is $A/\mathfrak{I}$-linear if we endow $\widehat{\mathfrak{I}}$ the $A/\mathfrak{I}$-module structure induced by $\varphi$. Suppose that the homomorphism $\lambda$ admits a $A/\mathfrak{I}$-linear \textit{section} $\sigma:\mathfrak{I}/\mathfrak{I}^2\to\widehat{\mathfrak{I}}$. 
\[\begin{tikzcd}[row sep=4mm,column sep=6mm]
\widehat{\mathfrak{I}}\ar[rr,"\lambda"]\ar[rd]&&\mathfrak{I}/\mathfrak{I}^2\ar[ll,bend right=25pt,swap,"\sigma"{anchor=south}]\\
&\widehat{\mathfrak{I}}/\widehat{\mathfrak{I}}^2\ar[ru,"\sim"]
\end{tikzcd}\quad \begin{tikzcd}[row sep=6mm,column sep=6mm]
&\widehat{A}\ar[d]\\
A/\mathfrak{I}\ar[ru,"\varphi"]\ar[r]&\widehat{A}/\widehat{\mathfrak{I}}
\end{tikzcd}\]
Let $S$ be the symmetric algebra $\bm{S}_{A/\mathfrak{I}}(\mathfrak{I}/\mathfrak{I}^2)$ and $\widehat{S}$ be its completion, and $\theta:S\to\widehat{A}$ be the $k$-homomorphism defined by $\theta(x)=\varphi(x)$ for $x\in S^0=A/\mathfrak{I}$ and $\theta(x)=\sigma(x)$ for $x\in S^1=\mathfrak{I}/\mathfrak{I}^2$. Since $\theta$ sends $S^1$ into $\widehat{\mathfrak{I}}$, it sends $S^n$ into $\widehat{\mathfrak{I}}^n$ and therefore extends to a continuous homomorphism $\hat{\theta}:\widehat{S}\to\widehat{A}$. The map $\gr_1(\theta):\mathfrak{I}/\mathfrak{I}^2\to\widehat{\mathfrak{I}}/\widehat{\mathfrak{I}}^2$ is the composition of $\sigma$ with the surjection $\widehat{\mathfrak{I}}\to\widehat{\mathfrak{I}}/\widehat{\mathfrak{I}}$, and since $\sigma$ is a section of $\lambda$, $\gr_1(\theta)$ then coincides with the canonical isomorphism $\mathfrak{I}/\mathfrak{I}^2\to\widehat{\mathfrak{I}}/\widehat{\mathfrak{I}}$. The map $\gr(\theta):S\to\gr_{\widehat{\mathfrak{I}}}(\widehat{A})$ is then the composition of the canonical surjection $\beta$ with the canonical isomorphism $\gr_\mathfrak{I}(A)\to\gr_{\widehat{\mathfrak{I}}}(\widehat{A})$. Now under the hypothesis of (\rmnum{2}), the $A/\mathfrak{I}$-module $\mathfrak{I}/\mathfrak{I}^2$ is projective, so $\lambda$ always admits a $A/\mathfrak{I}$-linear section. The induced homomorphism $\hat{\theta}:\widehat{S}\to\widehat{A}$ is then an isomorphism by the preceding arguments, whence (\rmnum{3}).\par
Suppose that the $k$-algebra $A$ is formally smooth, we prove that (\rmnum{1})$\Rightarrow$(\rmnum{2}). We first show that $\mathfrak{I}/\mathfrak{I}^2$ is projective. Let $M$ be a $A/\mathfrak{I}$-module and $f:M\to\mathfrak{I}/\mathfrak{I}^2$ be a surjective $A/\mathfrak{I}$-linear map. It suffices to prove that $f$ admits a $A/\mathfrak{I}$-linear section. Let $\pi:A/\mathfrak{I}^2\to A/\mathfrak{I}$ be the canonical surjection. By \cref{algebra lifting of A to A/I char}, there exists an isomorphism $\psi:A/\mathfrak{I}\oplus\mathfrak{I}/\mathfrak{I}^2\to A/\mathfrak{I}^2$ such that $\pi\circ\psi=\mathrm{pr}_1$ and $\psi(0,z)=z$ for $z\in\mathfrak{I}/\mathfrak{I}^2$. Consider the $k$-algebra $(A/\mathfrak{I})\oplus M$ and the map $u:(A/\mathfrak{I})\oplus M\to A/\mathfrak{I}^2$ defined by $u(x,m)=\psi(x,f(m))$. This is a surjective homomorphism of $k$-algebras, whose kernel is the submodule $\ker f$ of $M$, hence has zero square. The canonical surjection $\rho:A\to A/\mathfrak{I}^2$ is continuous, and as the $k$-algebra $A$ is formally smooth, there exists a $k$-homomorphism $\tilde{\rho}:A\to(A/\mathfrak{I})\oplus M$ such that $u\circ\tilde{\rho}=\rho$. As $\mathrm{pr}_1=\pi\circ\psi=\pi\circ u$, we have
\[\mathrm{pr}_1\circ\tilde{\rho}=\pi\circ u\circ\tilde{\rho}=\pi\circ\rho,\]
so $\mathrm{pr}_1\circ\tilde{\rho}$ is the canonical surjection $A\to A/\mathfrak{I}$. We then have $\tilde{\mathfrak{I}}(\mathfrak{I})\sub M$ and therefore $\tilde{\rho}(\mathfrak{I}^2)=0$, so $\tilde{\rho}$ induces a $A/\mathfrak{I}$-linear map $s:\mathfrak{I}/\mathfrak{I}^2\to M$. We have $u\circ\tilde{\rho}=\rho$ and $\mathrm{pr}_2\circ\psi^{-1}\circ u(y,m)=f(m)$ for $y\in A/\mathfrak{I}$ and $m\in M$. For $x\in\mathfrak{I}$, if $\bar{x}$ is the class of $x$ in $\mathfrak{I}/\mathfrak{I}^2$, we have
\[f(s(\bar{x}))=f(\mathrm{pr}_2(\tilde{\rho}(x)))=\mathrm{pr}_2(\psi^{-1}(\rho(x)))=\bar{x},\]
so $s$ is a section of $f$.

\begin{figure*}[htbp]
\centering
\[\begin{tikzcd}[row sep=6mm,column sep=6mm]
A/\mathfrak{I}\ar[d,equal]\ar[r]&(A/\mathfrak{I})\oplus M\ar[d]\ar[dd,bend left=70pt,dashed,red,"u",pos=.37]\ar[r]&M\ar[d,"f"]\\
A/\mathfrak{I}\ar[r]&(A/\mathfrak{I})\oplus\mathfrak{I}/\mathfrak{I}^2\ar[d,"\psi"]\ar[r]&\mathfrak{I}/\mathfrak{I}^2\\
A\ar[rd]\ar[ruu,bend left=10pt,dashed,red,"\tilde{\rho}"]\ar[r,"\rho"]&A/\mathfrak{I}^2\ar[d,"\pi"]\\
&A/\mathfrak{I}
\end{tikzcd}\]
\vspace*{-5mm}
\caption{The illustration of the proof of (\rmnum{1})$\Rightarrow$(\rmnum{2}) in \cref{algebra A/I formally smooth iff I completely secant}.}
\vspace*{-2mm}
\end{figure*}

Finally, it remains to prove that the homomorphism $\beta_\mathfrak{I}$ is injective. Since the $A/\mathfrak{I}$-module $\mathfrak{I}/\mathfrak{I}$ is injective, $\lambda$ admits a $A/\mathfrak{I}$-linear section, and we have the induced homomorphism $\theta:S\to\widehat{A}$, the homomorphism $\gr(\theta)$ is identified with $\beta$ by our previous arguments. For each integer $m\geq 0$, let $S_m=\sum_{i>m}S^i$ and $\theta_m:S/S_m\to A/\mathfrak{I}^{m+1}$ be the induced homomorphism. The composition of $\theta_m$ with the canonical surjection $A/\mathfrak{I}^{m+1}\to A/\mathfrak{I}$ is then the canonical projection of $S/S_m$ fo $S^0=A/\mathfrak{I}$, so its kernel is nilpotent. By \cref{algebra formally smooth lifting for complete algebras}, there exists a lifting $\psi_m:A\to S/S_m$ of the canonical surjection $A\to A/\mathfrak{I}^{m+1}$. As the composition of $\psi_m$ with the projection $S/S_m\to A/\mathfrak{I}$ is the canonical surjection $A\to A/\mathfrak{I}$, $\psi_m(\mathfrak{I})$ is formed by elements of positive degrees. By passing to associated algebras, we then deduce a graded $k$-linear map $\gr(\psi_m):\gr_\mathfrak{I}(A)\to S/S_m$ such that $\gr_m(\theta)\circ\gr_m(\psi_m)=\id_{\mathfrak{I}^m/\mathfrak{I}^{m+1}}$. It then follows that $\gr_m(\theta)$, and hence $\beta_m$, is injective.
\end{proof}
\begin{corollary}\label{Noe local algebra kappa_A separable formally smooth iff regular}
Let $k$ be a field and $A$ be a Noetherian local $k$-algebra such that the extension $\kappa_A$ of $k$ is separable. Then the following conditions are equivalent:
\begin{itemize}
\item[(\rmnum{1})] the $k$-algebra $A$ is formally smooth for the $\m_A$-adic topology;
\item[(\rmnum{2})] the ring $A$ is regular;
\item[(\rmnum{3})] the $k$-algebra $A$ is geometrically regular;
\item[(\rmnum{4})] the $k$-algebra $\widehat{A}$ is isomorphic to $\kappa_A\llbracket T_1,\dots,T_n\rrbracket$, where $n=\dim(A)$.
\end{itemize}
\end{corollary}
\begin{proof}
The conditions (\rmnum{2}) and (\rmnum{3}) are equivalent by (example 3 du $\S$6, n4), and amounts to saying that the ideal $\m_A$ is completely secant (\cref{Noe local ring regular iff m_A generated by completely secant}). Moreover, any isomorphism from $\widehat{A}$ to $\kappa_A\llbracket T_1,\dots,T_n\rrbracket$ is bicontinuous since they are local rings. As the $k$-algebra $\kappa_A$ is formally smooth by \cref{field ext formally smooth if separable}, the corollary follows from \cref{algebra A/I formally smooth iff I completely secant} applied to $\mathfrak{I}=\m_A$.
\end{proof}
\begin{corollary}\label{Noe ring Zariski k-algebra formally smooth is geometrically regular}
Let $k$ be a field, $A$ be a Noetherian $k$-algebra and $\mathfrak{I}$ be an ideal contained in the Jacobson radical of $A$. Suppose that $A$ is formally smooth over $k$ for the $\mathfrak{I}$-adic topology, then it is geometrically regular.
\end{corollary}
\begin{proof}
Let $k'$ be a finite extension of $k$ and $A'=A_{(k')}$. Then it suffices to prove that, for any maximal ideal $\m'$ of $A'$, the Noetherian local ring $A'_{\m'}$ is regular. Now we have $\mathfrak{I}A'\sub\m'$: in fact, the inverse image of $\m'$ in $A$ is a maximal ideal of $A$ (\cref{integral ring maximal ideal iff contraction is}), hence contains $\mathfrak{I}$. The $k'$-algebra $A'$ is formally smooth for $\mathfrak{I}A'$-adic topology by \cref{algebra formally smooth localization tensor prop}(b), and the $k'$-algebra $A'_{\m'}$ is then formally smooth for the $\mathfrak{I}A'_{\m'}$-adic topology (\cref{algebra formally smooth localization tensor prop}(a)), hence also for the $\m'A_{\m'}$-adic topology (\cref{algebra formally smooth adic ideal inclusion}). Let $k_0$ be the prime subfield of $k'$, then $A'_{\m'}$ is formally smooth over $k_0$ for the $\m'A_{\m'}$-adic topology (\cref{algebra formally smooth over separable extension}); as $k_0$ is perfect, $\kappa(\m')$ is separable over $k_0$, so the ring $A'_{\m'}$ is regular (\cref{Noe local algebra kappa_A separable formally smooth iff regular}).
\end{proof}
\begin{corollary}\label{algebra formally smooth Omega projective}
Let $k$ be a ring and $A$ be a formally smooth $k$-algebra.
\begin{itemize}
\item[(a)] The $A$-module $\Omega_{A/k}$ is projective.
\item[(b)] Suppose that the ring $A\otimes_kA$ is Noetherian and let $\mathfrak{I}$ be the kernel of the multiplication map $A\otimes_kA\to A$. Then the ideal $\mathfrak{I}$ is completely secant.
\end{itemize}
\end{corollary}
\begin{proof}
The $k$-algebras $A$ and $A\otimes_kA$ are formally smooth (\cref{algebra formally smooth localization tensor prop}(c)), and $A$ is isomorphic to a quotient algebra of $A\otimes_kA$ by the kernel $\mathfrak{I}$. By definition we have $\Omega_{A/k}=\mathfrak{I}$, so the corollary follows from \cref{algebra A/I formally smooth iff I completely secant}.
\end{proof}
\subsection{Formally smoothness for local algebras}
\begin{proposition}\label{algebra A/m formally smooth iff A_m regular injective tangent}
Let $k_0$ be a ring, $k$ be a $k_0$-algebra, $A$ be a $k$-algebra, and $\m$ be a maximal ideal of $A$. Suppose that $k$ and $A/\m$ are formally smooth over $k_0$. For $A$ to be formally smooth over $k$ for the $\m$-adic topology, it is necessary and sufficient that the following two conditions are satisfied
\begin{itemize}
\item[(\rmnum{1})] the canonical homomorphism $\alpha_\m:\bm{S}_{A/\m}(\m/\m^2)\to\gr_\m(A)$ is bijective;
\item[(\rmnum{2})] the $A/\m$-linear map
\[\omega:A/\m\otimes_k\Omega_{k/k_0}\to A/\m\otimes_A\Omega_{A/k_0}\]
induced by the canonical map $k\to A$ is injective.
\end{itemize}
\end{proposition}
\begin{proof}
Let $d_k:k\to\Omega_{k/k_0}$ and $d_A:A\to\Omega_{A/k_0}$ be the universal $k_0$-derivations, and endow $A$ with the $\m$-adic topology. Suppose first that $A$ is formally smooth over $k$. Then $A$ is formally smooth over $k_0$ (\cref{algebra formally smooth transitive product completion prop}), which is equivalent to (\Rmnum{1}) in view of \cref{algebra A/I formally smooth iff I completely secant}. Moreover, the $k_0$-derivation $\lambda\mapsto 1\otimes d_k(\lambda)$ of $k$ into $A/\m\otimes_k\Omega_{k/k_0}$ can then be extended to a $k_0$-derivation from $A$ into $A/\m\otimes_k\Omega_{k/k_0}$ (\cref{algebra formally smooth derivation k to M extension}), so there exists an $A$-linear map $u:\Omega_{A/k_0}\to A/\m\otimes_k\Omega_{k/k_0}$ such that $u(d_A(\lambda 1_A))=1\otimes d_k(\lambda)$ for $\lambda\in k$. The $A/\m$-linear map $\bar{u}:A/\m\otimes_A\Omega_{A/k_0}\to A/\m\otimes_k\Omega_{k/k_0}$ induced by $u$ is then a retraction of $\omega$, which proves (\rmnum{2}).\par
Conversely, assume that conditions (\rmnum{1}) and (\rmnum{2}) are satisfied. Then $A$ is formally smooth over $k_0$ (\cref{algebra A/I formally smooth iff I completely secant}) and the $A$-module $\Omega_{A/k_0}$ is projective \cref{algebra formally smooth Omega projective}. Fix an integer $r\geq 0$ and consider the $A/\m^r$-linear map
\[\omega_r:A/\m^r\otimes\Omega_{k/k_0}\to A/\m^r\otimes_A\Omega_{A/k_0}\]
induced by the canonical homomorphism $k\to A$. Let $(\lambda_i)_{i\in I}$ be a family of elements of $k$ such that the $d_k(\lambda_i)$ form a basis for the $k$-vector space $\Omega_{k/k_0}$. By condition (\rmnum{2}), the elements $1\otimes d_A(\lambda_i1_A)$ are linearly indepedent in $A/\m\otimes_A\Omega_{A/k_0}$. By \cref{local ring module basis of direct factor iff}, the $1\otimes d_A(\lambda_i1_A)$ then form a basis for a direct summand of the $A/\m^r$-module $A/\m^r\otimes_A\Omega_{A/k_0}$. There then exists a $A/\m^r$-linear map
\[u_r:A/\m^r\otimes_A\Omega_{A/k_0}\to A/\m^r\otimes_k\Omega_{k/k_0}\]
such that $u_r(1\otimes d_A(\lambda_i1_A))=1\otimes d_k(\lambda_i)$ for each $i$, and hence $u_r\circ\omega_r=\id$.\par
We now verify that $A$ is formally smooth over $k$. Let $C$ be a $k$-algebra, $N$ be a square zero ideal of $C$, and $\pi:C\to C/N$ be the canonical homomorphism. Endow $C$ and $C/N$ the discrete topology. Let $\varphi:A\to C/N$ be a continuous homomorphism of $k$-algebras. Since $A$ is formally smooth over $k_0$, there exists a $k_0$-linear lifting $\tilde{\varphi}_0:A\to C$ of $\varphi$. By \cref{algebra derivation act on lifting}, the $k_0$-linear liftings $\tilde{\varphi}:A\to C$ of $\varphi$ are given by the maps $x\mapsto v(d_A(x))+\tilde{\varphi}_0(x)$, where $v\in\Hom_A(\Omega_{A/k_0},N)$, so we have to choose $v$ so that $\tilde{\varphi}_0$ is a homomorphism of $k$-algebras. Now the map $\lambda\mapsto\lambda 1_C-\tilde{\varphi}_0(\lambda\bm{1}_A)$ is a $k_0$-derivation from $k$ to $N$ (\cref{algebra derivation act on lifting}), so can be written as $h\circ d_k$ where $h\in\Hom_k(\Omega_{k/k_0},N)$. We choose an integer $r\geq 0$ such that the kernel of $\varphi$ contains $\m^r$ (note that $\ker\varphi$ is open in $A$). The $A$-module $N$ is then annihilated by $\m^r$ and it suffices to choose $v$ to be the composition of the homomorphisms
\[\begin{tikzcd}
\Omega_{A/k_0}\ar[r]&A/\m^r\otimes_A\Omega_{A/k_0}\ar[r,"u_r"]&A/\m^r\otimes_k\Omega_{k/k_0}\ar[r,"\tilde{h}"]&N
\end{tikzcd}\]
where $\tilde{h}$ is induced by $h$. In fact, in this case, for $\lambda\in k$ we have
\begin{equation*}
v(d_A(\lambda 1_A))=\tilde{h}(u_r(1\otimes d_A(\lambda 1_A)))=\tilde{h}(1\otimes d_k(\lambda 1_A))=h(d_k(\lambda))=\lambda 1_C-\tilde{\varphi}_0(\lambda 1_A).\qedhere
\end{equation*}
\end{proof}
\begin{remark}\label{algebra A/m k formally smooth remark}
If $A$ is a Noetherian ring, condition (\rmnum{1}) of \cref{algebra A/m formally smooth iff A_m regular injective tangent} then signifies that the local ring $A_\m$ is regular (\cref{Noe local ring regular iff m_A generated by completely secant}).
\end{remark}
\begin{proposition}\label{Noe local k-algebra formally smooth iff georegular}
Let $k$ be a field anf $A$ be a Noetherian local $k$-algebra. The following conditions are equivalent:
\begin{itemize}
\item[(\rmnum{1})] $A$ is formally smooth for the $\m_A$-adic topology;
\item[(\rmnum{2})] $A$ is regular and the $\kappa_A$-linear map
\[\omega:\kappa_A\otimes_k\Omega_k\to\kappa_A\otimes_A\Omega_A\]
induced by the canonical map $k\to A$ is injective;
\item[(\rmnum{3})] $A$ is geometrically regular.
\end{itemize}
\end{proposition}
\begin{proof}
To see that (\rmnum{1})$\Leftrightarrow$(\rmnum{2}), it suffices to apply \cref{algebra A/m formally smooth iff A_m regular injective tangent} and \cref{algebra A/m k formally smooth remark}, where we choose $k_0$ to be the prime subfield of $k$. In fact, $k$ and $\kappa_A$ are then formally smooth over $k_0$ (\cref{field ext formally smooth if separable}). Now (\rmnum{1})$\Rightarrow$(\rmnum{3}) follows from \cref{Noe ring Zariski k-algebra formally smooth is geometrically regular}.\par
If $k$ has characteristic zero, then it follows from \cref{Noe local algebra kappa_A separable formally smooth iff regular} that (\rmnum{3}) implies (\rmnum{1}), whence the proposition in this case. Suppose now that $k$ has characteristic $p>0$, and we prove that (\rmnum{3})$\Rightarrow$(\rmnum{2}). Let $k'/k$ be a finite purely inseparable extension of height $\leq 1$. If $A$ and $A_{(k')}$ are regular, the canonical map $\kappa_A\otimes_{k'^p}\Omega_{k'^p/k^p}\to\kappa_A\otimes_A\Omega_A$ is injective by \cref{Noe ring completely secant ideal iff flat fiber}. By (A, \Rmnum{5}, p.97, th.1(b)) applied to the extension $k/k^p$, the $k$-vector space $\Omega_k$, which coincides with $\Omega_{k/k^p}$, is the union of the subspaces $k\otimes_{k'^p}\Omega_{k'^p/k^p}$, where $k'$ runs through the set of finite purely inseparable extension of $k$ of height $\leq 1$ in a fixed algebraic closure of $k$. From this, we see that condition (\rmnum{2}) is then satisfied.
\end{proof}
\subsection{Jacobian criterion}
Let $k$ be a ring, $A$ be a $k$-algebra, $\mathfrak{I}$ be an ideal of $A$ and $\bar{d}:\mathfrak{I}/\mathfrak{I}^2\to A/\mathfrak{I}\otimes_A\Omega_{A/k}$ be the canonical map. For each $A/\mathfrak{I}$-algebra $R$, we denote by
\[\bar{d}_R:R\otimes_{A/\mathfrak{I}}\mathfrak{I}/\mathfrak{I}^2\to R\otimes_A\Omega_{A/k}\]
the $R$-linear map induced by $\bar{d}$. If the $k$-algebra $A/\mathfrak{I}$ is formally smooth, $\bar{d}$ then possesses an $A$-linear retraction (\cref{algebra formally smooth A/I conormal sequence split exact}) and $\bar{d}_R$ possesses an $R$-linear retraction for any $R$. More generally, we have the following lemma.
\begin{lemma}\label{algebra formally smooth retraction for conormal map}
Let $\mathfrak{K}$ be an ideal of $A$ containing $\mathfrak{I}$. Suppose that there exists an integer $m\geq 0$ such that $\mathfrak{I}\cap\mathfrak{K}^m$ is contained in $\mathfrak{I}\mathfrak{K}$ (this is satisfied if $A$ is Noetherian). If $A/\mathfrak{I}$ is formally smooth over $k$ for the $\mathfrak{K}/\mathfrak{I}$, the map $\bar{d}_{A/\mathfrak{K}}:A/\mathfrak{K}\otimes_{A/\mathfrak{I}}\mathfrak{I}/\mathfrak{I}^2\to A/\mathfrak{K}\otimes_A\Omega_{A/k}$ possesses an $A$-linear retraction.
\end{lemma}
\begin{lemma}\label{algebra formally smooth A/I formally smooth iff retraction}
Suppose that $A$ is formally smooth over $k$ for the $\mathfrak{I}$-adic topology. For $A/\mathfrak{I}$ to be formally smooth over $k$, it is necessary and sufficient that the canonical map $\bar{d}:\mathfrak{I}/\mathfrak{I}^2\to A/\mathfrak{I}\otimes_A\Omega_{A/k}$ possesses an $A$-linear retraction.
\end{lemma}
\begin{theorem}[\textbf{Jacobian Criterion}]\label{algebra A/I formally smooth Jacobian criterion}
Let $k$ be a ring, $A$ be a formally smooth $k$-algebra and $\mathfrak{I}$ be a finitely generated ideal of $A$; put $B=A/\mathfrak{I}$.
\begin{itemize}
\item[(a)] Let $\mathfrak{P}$ be a prime ideal of $B$ and $\p$ be the prime ideal of $A$ such that $\mathfrak{P}=\p/\mathfrak{I}$. Then the following conditions are equivalent:
\begin{itemize}
\item[(\rmnum{1})] the $k$-algebra $B_\mathfrak{P}$ is formally smooth;
\item[(\rmnum{2})] there exists $f\in B-\mathfrak{P}$ such that the $k$-algebra $B_f$ is formally smooth;
\item[(\rmnum{3})] the $\kappa(\mathfrak{P})$-linear map
\[\bar{d}_{\kappa(\mathfrak{P})}:\kappa(\mathfrak{P})\otimes_B\mathfrak{I}/\mathfrak{I}^2\to\kappa(\p)\otimes_A\Omega_{A/k}\]
is injective;
\item[(\rmnum{4})] there exist elements $f_1,\dots,f_m$ of $\mathfrak{I}$ whose images $(f_1)_\p,\dots,(f_m)_\p$ generate $\mathfrak{I}_\p$ and $k$-derivations $D_1,\dots,D_m$ of $A$ such that $\det(D_j(f_i))\notin\p$.
\end{itemize}
\item[(b)] The set of prime ideals $\mathfrak{P}$ of $B$ satisfying the equivalent conditions of (a) is open in $\Spec(B)$. For $B$ to be formally smooth over $k$, it is necessary and sufficient that any prime (resp. maximal) ideal of $B$ satisfies these conditions.
\item[(c)] If $A$ is Noetherian, the conditions of (a) are equivalent to the following:
\begin{itemize}
\item[(\rmnum{5})] the $k$-algebra $B_\mathfrak{P}$ is formally smooth for the $\mathfrak{P}B_\mathfrak{P}$-adic topology.
\end{itemize}
Moreover, under condition (\rmnum{4}), the ideal $\mathfrak{I}_\p$ is completely secant and $((f_1)_\p,\dots,(f_m)_\p)$ is a completely secant sequence for $A_\p$.
\end{itemize}
\end{theorem}
\begin{proof}
Put $M=\mathfrak{I}/\mathfrak{I}^2$ and $N=B\otimes_A\Omega_{A/k}$. The $B$-module $M$ is finitely generated, and the $B$-module $N$ is projective (\cref{algebra formally smooth Omega projective}).
\end{proof}
\begin{corollary}\label{algebra local es.ft formally smooth adic then for discrete}
Let $k_0$ be a ring, $k$ be a Noetherian formally smooth $k_0$-algebra, and $A$ be a local $k$-algebra essentially of finite type. If the $k_0$-algebra $A$ is formally smooth for the $\m_A$-adic topology, it is formally smooth.
\end{corollary}
\begin{proof}
By hypotheses, there exists an integer $n\geq 0$, a multiplicative subset $S$ of $k[T_1,\dots,T_n]$, and a surjective $k$-homomorphism $S^{-1}k[T_1,\dots,T_n]\to A$. The $k$-algebra $S^{-1}k[T_1,\dots,T_n]$ is Noetherian and formally smooth by \cref{algebra formally smooth localization tensor prop} and \cref{algebra polynomial ring formally smooth}, and hence over $k_0$ (\cref{algebra formally smooth transitive product completion prop}). The corollary then follows from \cref{algebra A/I formally smooth Jacobian criterion}(c).
\end{proof}
\begin{corollary}\label{algebra local formally smooth locus open}
Let $k_0$ be a ring, $k$ be a Noetherian formally smooth $k_0$-algebra, and $A$ be a local $k$-algebra essentially of finite type. The set $U$ of prime ideals $\p$ of $A$ such that the $k_0$-algebra $A_\p$ is fomally smooth (for the discrete topology or the $\p A_\p$-adic topology) is open in $\Spec(A)$ and the following conditions are equivalent:
\begin{itemize}
\item[(\rmnum{1})] $U=\Spec(A)$;
\item[(\rmnum{2})] $U$ contains every maximal ideal of $A$;
\item[(\rmnum{3})] the $k_0$-algebra $A$ is formally smooth.
\end{itemize}
\end{corollary}
\begin{proof}
This is already contained in \cref{algebra A/I formally smooth Jacobian criterion}, in view of \cref{algebra local es.ft formally smooth adic then for discrete}.
\end{proof}
\begin{remark}\label{algebra local formally smooth locus open if ft over separable}
The results of \cref{algebra local es.ft formally smooth adic then for discrete} and \cref{algebra local formally smooth locus open} are applicable if $k_0$ is a field and that we are in one of the following two cases:
\begin{itemize}
\item[(a)] $A$ is an algebra essentially of finite type over a separable extension of $k$.
\item[(b)] $A$ is a complete Noetherian local algebra whose residue field $\kappa_A$ is a separable extension of $k_0$ (in this case we choose $k$ to be a formal series algebra over $\kappa_A$ for which $A$ is a quotient).
\end{itemize}
In each case, it follows from \cref{algebra local formally smooth locus open}, in view of \cref{Noe local k-algebra formally smooth iff georegular} and \cref{algebra Noe georegular normal separable and localization}, that the $k_0$-algebra $A$ is formally smooth if and only if it is gemetrically regular.
\end{remark}
\begin{corollary}[\textbf{Zariski's Jacobian Criterion}]\label{algebra local regular es.ft over field Jacobian criterion}
Let $k$ be a field, $A$ be a regular local $k$-algebra, and $\mathfrak{I}$ be a proper ideal of $A$. Suppose that the $k$-algebra $A$ is essentially of finite type or complete. For the local ring $A/\mathfrak{I}$ to be regular, it is necessary and sufficient that there exists a generating family $f_1,\dots,f_m$ of $\mathfrak{I}$ and derivations $D_1,\dots,D_m$ of $A$ such that $\det(D_j(f_i))\notin\m_A$. In this case, the elements $(f_1,\dots,f_m)$ is a subset of a system of parameters of $A$ and the ideal $\mathfrak{I}$ is prime. 
\end{corollary}
\begin{proof}
Let $k_0$ be the prime subfield of $k$. The $k_0$-algebra $A$ is geometrically regular by \cref{algebra local regular separable residue is georegular}, hence formally smooth (\cref{algebra local formally smooth locus open if ft over separable}). By the same reasoning, the regularity of $A/\mathfrak{I}$ is equivalent to its formally smoothness over $k_0$. The first assertion then follows from \cref{algebra A/I formally smooth Jacobian criterion}, which also implies that the sequence $(f_1,\dots,f_m)$ is completely secant for $A$ in this case. We can then apply \cref{regular local ring quotient by ideal regular iff} to conclude the second assertion (note that $A/\mathfrak{I}$ is then an integral domain).
\end{proof}
\begin{remark}\label{algebra local regular es.ft over field Jacobian criterion variant}
Under the hypotheses of \cref{algebra local regular es.ft over field Jacobian criterion}, the $A$-module $\Omega_{A/k}$ is projective by \cref{algebra formally smooth Omega projective}, hence free. Any derivation of $A$ in $\kappa_A$ is therefore lifted into a derivation of $A$. The condition of the statement can therefore be expressed as follows: there exists a generating system $(f_1,\dots,f_m)$ of $\mathfrak{I}$ and derivations $D_1,\dots,D_m$ of $A$ in $\kappa_A$ such that $\det(D_j(f_i))\neq 0$.
\end{remark}
\begin{corollary}[\textbf{Zariski}]\label{algebra es.ft over field regular locus open}
Let $k$ be a field and $A$ be a $k$-algebra essentially of finite type or a complete Noetherian local $k$-algebra. The set of prime ideals $\p$ of $A$ such that $A_\p$ is regular is open in $\Spec(A)$.
\end{corollary}
\begin{proof}
It suffices to apply \cref{algebra local formally smooth locus open if ft over separable} and choose $k_0$ to be the prime field of $k$.
\end{proof}
\subsection{Smooth algebras}
\begin{lemma}\label{Noe local ring local homomorphism formally smooth iff flat and fiber georegular}
Let $\rho:A\to B$ be a local homomorphism of Noetherian local rings. Suppose that $B$ is essentially of finite type over $A$. For the $A$-algebra to be formally smooth, it is necessary and sufficient that the $A$-module $B$ is flat and the $\kappa_A$-algebra $\kappa_A\otimes_AB$ is geometrically regular.
\end{lemma}
\begin{proof}
By hypotheses, there exists an integer $n\geq 0$, a prime ideal $\q$ of $A[T_1,\dots,T_n]$ and a surjective homomorphism $h:A[T_1,\dots,T_n]_\q\to B$. Denote by $C$ be the local $A$-alebra $A[T_1,\dots,T_n]_\q$; it is a formally smooth (\cref{algebra polynomial ring formally smooth}) and flat over $A$, and we can identify $B$ with the $A$-algebra $C/\mathfrak{I}$, where $\mathfrak{I}=\ker h$.\par
Put $\widebar{C}=\kappa_A\otimes_AC$ and $\widebar{B}=\kappa_A\otimes_AB$. Suppose that $B$ is formally smooth over $A$. The $\kappa_A$-algebra $\widebar{C}$ is then formally smooth (\cref{algebra formally smooth localization tensor prop}), hence geometrically regular (\cref{Noe ring Zariski k-algebra formally smooth is geometrically regular}). Moreover, since $\widebar{C}/\mathfrak{I}\widebar{C}$ is identified with $\widebar{B}$ and that the $\kappa_A$-algebra $\widebar{C}$ is formally smooth, the ideal $\mathfrak{I}\widebar{C}$ of $\widebar{C}$ is completely secant (\cref{algebra A/I formally smooth iff I completely secant}). It then follows from \cref{Noe ring completely secant ideal iff flat fiber} that the $A$-module $B$ is flat.\par
Conversely, suppose that $B$ is flat over $A$ and the $\kappa_A$-algebra $\widebar{B}$ is geometrically regular. Then the local $\kappa_A$-algebra $\widebar{B}$ is formally smooth (\cref{Noe local k-algebra formally smooth iff georegular}). Put $\widebar{\mathfrak{I}}=\kappa_A\otimes_A\mathfrak{I}$; since $B$ is a flat $A$-module, the canonical map $\widebar{\mathfrak{I}}\to\mathfrak{I}\widebar{C}$ is bijective and $\widebar{B}$ is identified with $\widebar{C}/\widebar{\mathfrak{I}}$. It then follows from \cref{algebra formally smooth A/I conormal sequence split exact} that the canonical map
\[\widebar{\mathfrak{I}}/\widebar{\mathfrak{I}}^2\to\widebar{B}\otimes_{\widebar{C}}\Omega_{\widebar{C}/\kappa_A}\]
is injective and admits a retraction. Now $\widebar{\mathfrak{I}}/\widebar{\mathfrak{I}}^2$ is identified with $\kappa_A\otimes_A\mathfrak{I}/\mathfrak{I}^2$, hence with $\widebar{B}\otimes_B\mathfrak{I}/\mathfrak{I}^2$. On the other hand the $\widebar{C}$-module $\Omega_{\widebar{C}/\kappa_A}$ is canonically isomorphic to $\widebar{C}\otimes_C\Omega_{C/A}$ (A, \Rmnum{3}, p.136, prop.20), hence $\widebar{B}\otimes_{\widebar{C}}\Omega_{\widebar{C}/\kappa_A}$ is canonically isomorphic to $\widebar{B}\otimes_C\Omega_{C/A}$. Passing to quotient by the maximal ideal of $\widebar{B}$, we then obtain an injective homomorphism (by the existence of retraction)
\[\kappa_B\otimes_B\mathfrak{I}/\mathfrak{I}^2\to\kappa_B\otimes_C\Omega_{C/A}\]
which is none other than $\bar{d}_{\kappa_B}$, so $B$ is formally smooth over $A$ (\cref{algebra A/I formally smooth Jacobian criterion}).
\end{proof}
\begin{theorem}\label{Noe ring formally smooth iff flat and fiber georegular}
Let $A$ be a Noetherian ring and $B$ be an $A$-algebra essentially of finite type. Then the following conditions are equivalent:
\begin{itemize}
\item[(\rmnum{1})] the $A$-algebra $B$ is formally smooth;
\item[(\rmnum{2})] for any prime ideal $\mathfrak{P}\in\Spec(B)$, the $A$-algebra $B_\mathfrak{P}$ is formally smooth (resp. formally smooth for the $\mathfrak{P}B_\mathfrak{P}$-adic topology);
\item[(\rmnum{3})] the $A$-module $B$ is flat and for any $\p\in\Spec(A)$, the $\kappa(\p)$-algebra $\kappa(\p)\otimes_AB$ is geometrically regular;
\item[(\rmnum{4})] the $A$-module $B$ is flat and for any regular $A$-algebra $R$, the ring $R\otimes_AB$ is geometrically regular;
\item[(\rmnum{5})] the $A$-module $B$ is flat and the kernel of the multiplication homomorphism $\mu:B\otimes_AB\to B$ is completely secant.
\end{itemize}
\end{theorem}
\begin{proof}
The equivalence of (\rmnum{1}) and (\rmnum{2}) folows from \cref{algebra local formally smooth locus open}. To see that (\rmnum{1})$\Rightarrow$(\rmnum{5}), suppose that $B$ is formally smooth over $A$, let $\mathfrak{P}$ be a prime ideal of $B$ and $\p$ be its contraction into $A$. The $A_\p$-algebra $B_\mathfrak{P}$ is formally smooth (\cref{algebra formally smooth localization tensor prop}), hence flat (\cref{Noe local ring local homomorphism formally smooth iff flat and fiber georegular}), so the $A$-module $B$ is flat (\cref{module flat iff localization at maximal}). On the other hand, the ring $B\otimes_AB$ is Noetherian (\cref{algebra es.ft base change to Noe is Noe}), so the ideal $\mathfrak{I}$ is completely secant in view of \cref{algebra formally smooth Omega projective}.\par
Suppose that condition (\rmnum{5}) is satisfied and let $\p\in\Spec(A)$; we show that (\rmnum{5})$\Rightarrow$(\rmnum{3}). The map
\[1\otimes\mu:\kappa(\p)\otimes_A(B\otimes_AB)\to\kappa(\p)\otimes_AB\]
is identified with the multiplication map
\[\mu_\p:(\kappa(\p)\otimes_AB)\otimes_{\kappa(\p)}(\kappa(\p)\otimes_AB)\to\kappa(\p)\otimes_AB)\]
of the $\kappa(\p)$-algebra $\kappa(\p)\otimes_AB$, so the kernel $\ker\mu_\p$ is identified with $\mathfrak{I}(\kappa(\p)\otimes_A(B\otimes_AB))$. It is completely secant since the $A$-module $B$ is flat (\cref{Noe ring completely secant ideal iff flat fiber}). Condition (\rmnum{3}) then follows from \cref{algebra es.ft georegular iff tensor regular}.\par
Under the hypotheses of (\rmnum{3}), let $\mathfrak{P}$ be a prime ideal of $B$ and $\p$ be its contraction in $A$. The $A_\p$-module $B_\mathfrak{P}$ is flat, and the $\kappa(\p)$-algebra $\kappa(\p)\otimes_{A_\p}B_\mathfrak{P}$, identified with a fraction ring of $\kappa(\p)\otimes_AB$, is geometrically regular (\cref{algebra Noe georegular normal separable and localization}). It then follows from \cref{Noe local ring local homomorphism formally smooth iff flat and fiber georegular} that $B_\mathfrak{P}$ is formally smooth over $A_\p$, hence over $A$ (\cref{algebra local ring formally smooth}); this proves (\rmnum{3})$\Rightarrow$(\rmnum{2}).\par
It remains to prove the equivalence of (\rmnum{3}) and (\rmnum{4}). First consider the condition (\rmnum{3}). Let $R$ be a regular $A$-algebra. The $R$-module $R\otimes_AB$ is flat by \cref{module flat base change localization}. Let $\r$ be a prime ideal of $R$ and $\p$ be its contraction in $A$. The ring $\kappa(\r)\otimes_R(R\otimes_AB)$, identified with $\kappa(\r)\otimes_{\kappa(\p)}(\kappa(\p)\otimes_AB)$, is regular by \cref{algebra es.ft georegular normal arbitrary base change}, so $R\otimes_AB$ is regular by \cref{algebra Noe georegular normal separable and localization}. Conversely, if condition (\rmnum{4}) is satisfied, for any priem ideal $\p$ of $A$ and any extension $k$ of $\kappa(\p)$, the ring $k\otimes_{\kappa(\p)}(\kappa(\p)\otimes_AB)$, which is identified with $k\otimes_AB$, is regular, whence (\rmnum{3}).
\end{proof}
Let $A$ be a Noetherian ring. We say that an $A$-algebra $B$ is \textbf{smooth} if it is essentially of finite type are satisfies the equivalent conditions of \cref{Noe ring formally smooth iff flat and fiber georegular}.
\begin{proposition}\label{Noe ring smooth algebra transitive tensor product prop}
Let $A$ be a Noetherian ring.
\begin{itemize}
\item[(a)] Let $A'$ be a Noetherian $A$-algebra and $B$ be a smooth $A$-algebra. Then the $A'$-algebra $A'\otimes_AB$ is smooth.
\item[(b)] Let $B$ be a smooth $A$-algebra and $C$ be a smooth $B$-algebra. Then the $A$-algebra $C$ is smooth.
\item[(c)] Let $B$ and $C$ be two smooth $A$-algebras. Then the $A$-algebra $B\otimes_AC$ is smooth.
\end{itemize}
\end{proposition}
\begin{proof}
This follows from the analogues results for algebras essentially of finite type and formally smooth.
\end{proof}
\begin{example}\label{algebra smooth over field iff es.ft and georegular}
The smooth algebras over a field $k$ are the $k$-algebras essentially of finite type and geometrically regular. In particular, if $k$ is a perfect field, then these are exactly regular $k$-algebras essentially of finite type.
\end{example}
\begin{example}\label{algebra polynomial and series over field smooth}
Let $A$ be a Noetherian ring and $(T_i)_{i\in I}$ be a finite family of indeterminates. Then the polynomial algebra $A[(T_i)_{i\in I}]$ is smooth over $A$. More generally, let $F_1,\dots,F_m$ be elements of $A[(T_i)_{i\in I}]$ and $B=A[(T_i)_{i\in I}]/(F_1,\dots,F_m)$. If for any maximal ideal $\mathfrak{N}$ of $B$, the class of the matrix $(\partial F_j/\partial T_i)$ mod $\mathfrak{N}$ is of rank $m$, then the $A$-algebra $B$ is smooth (\cref{algebra A/I formally smooth Jacobian criterion}).
\end{example}
\section{Duality of finite length modules}
\subsection{Indecomposable injective modules}